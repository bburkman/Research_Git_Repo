%%%%%
%\section{Methods}

%We have written this section as a guide for other to replicate and adapt our work, so some details may seem pedantic.

\subsection{Metrics}\label{Methods_Metrics}

{\bf Precision} tells us, of the ambulances we sent, how many were needed.  {\bf Recall} tells us, of the ambulances that were needed, how many we sent.  Recall only looks at elements of the minority class (positive class, ``need ambulance''), so is independent of the class imbalance.  Precision is affected by class imbalance, but is still relevant to our decisions in its imbalanced form.  Because the number of elements of the positive class in the test set is constant across all of our models, recall is proportional to TP. 

The {\bf F1 score} is the harmonic mean of precision and recall. Why the harmonic mean instead of the arithmetic or geometric?  For two positive numbers $a$ and $b$ with $0 < a < b$, 
$$a < Harm(a,b) < Geo(a,b) < Arith(a,b) < b$$
so the F1 score emphasizes what the model does poorly.  We will use F1 as our primary indicator, while looking at precision and recall.  

The area under the curve ({\bf AUC}) of the receiver operating characteristic (ROC) is a measure of how well a model separates the samples of the positive and negative classes.  We will use it to show that the additional features in the ``hard/expensive'' and ``medium'' datasets are important for discriminating between the two classes.  

The $\Delta FP/\Delta TP$ curve is related to the ROC; $\Delta FP/\Delta TP$ is the reciprocal of the product of the slope of the ROC curve and a factor that corrects for class imbalance. 

%$$mROC = \frac{\Delta TPR}{\Delta FPR} = \frac{\frac{\Delta TP}{P}}{\frac{\Delta FP}{N}} = \frac{N}{P} \cdot \frac{\Delta TP}{\Delta FP} = \frac{1}{ \frac{P}{N} \cdot \frac{\Delta FP}{\Delta TP}}$$

$$
\frac{\Delta FP}{\Delta TP} = 
\frac{N}{P} \cdot \frac{\frac{\Delta FP}{N}}{\frac{\Delta TP}{P}}
= \frac{N}{P} \cdot \frac{\Delta FPR}{\Delta TPR}
= \frac{1}{\frac{P}{N} \cdot \frac{\Delta TPR}{\Delta FPR}}
= \frac{1}{\frac{P}{N} \cdot mROC}
$$
 We will use this curve to find the value of the discrimination threshold where $\Delta FP/\Delta TP = 2.0$



%%%
\subsection{ Incorporating the $\Delta FP/\Delta TP < \omega$ Ethical Threshold}

We incorporated this ethical threshold in two ways, as a class weight and as the decision threshold.  

\subsubsection{Class Weight}

To understand why class weights can encode this threshold, we will use the $\alpha$-weighted binary crossentropy model as an example.  

To move the model towards $\Delta FP/\Delta TP < \omega \to \Delta FP - \omega\Delta TP < 0$ is equivalent to minimizing $FP - \omega TP$.  The $\alpha$-weighted binary crossentropy loss function is 

$$\displaystyle Loss = - \left(\alpha \sum_{y=1} \log \left( h_\theta (x_i) \right) + (1-\alpha) \sum_{y=0} \log \left(1 - h_\theta(x_i)\right)  \right)$$

In this function, $y_i \in \{0,1\}$ is the ground truth for sample $i$, 0 if in the majority class (''no ambulance'') and 1 if in the minority class (``yes ambulance'').  The term $h_\theta(x_i)\in [0,1]$ is the probability that $x_i$ is in the minority class, as calculated by this $\theta$ iteration of the model.

The TN, FP, FN, and TP are discrete, and the $h_\theta(x_i)$ is continuous.  To see how they relate, let us discretize the loss function, with 

$$
\log\left(h_\theta(x_i)\right) \to 
\begin{cases}
	0 & \text{if } h_\theta(x_i) \le 0.5 \cr
	1 & \text{if } h_\theta(x_i) > 0.5 \cr
\end{cases}
\text{\quad and \quad}
\log\left(1 - h_\theta(x_i)\right) \to 
\begin{cases}
	0 & \text{if } 1 - h_\theta(x_i) \le 0.5 \cr
	1 & \text{if } 1 - h_\theta(x_i) > 0.5 \cr
\end{cases}
$$

\noindent which makes $\displaystyle \sum_{y=1} \log ( h_\theta (x_i) )$ into $TP$ and  $\displaystyle \sum_{y=0} \log(1 - h_\theta(x_i))$ into $TN$, making the discrete version of the loss function
$$Loss = -\left(\alpha TP + (1-\alpha) TN\right)$$

In the following manipulations, note that adding a constant, or multiplying by a positive constant, does not change the effect of the loss function, which the model algorithm uses to compare one iteration to the next.

\

\hfil \begin{tabular}{ >{$\displaystyle}r<{$} @{\hspace{3pt}}>{$\displaystyle}c<{$} @{\hspace{3pt}}>{$\displaystyle}l<{$} l <{\vrule width 0pt height 12pt depth 12pt} }
	Loss &=& -\left(\alpha TP + (1-\alpha) TN\right) &\cr
	Loss &=& -\left(\frac{\omega}{\omega +1} TP + \frac{1}{\omega +1} TN\right) &Let $\displaystyle\alpha = \frac{\omega}{\omega +1}$, making $\displaystyle1 - \alpha = \frac{1}{\omega +1}$ \cr
	Loss &=& -\left( \omega \cdot TP + TN \right) & Multiply by $\omega+1$ \cr
	Loss &=& -\left( \omega \cdot TP + TN \right) + TN + FP & Add constant $TN+FP$, the number of majority samples\cr
	Loss &=& FP - \omega \cdot TP & \cr
\end{tabular}

Thus, we can incorporate the ethical threshold $p$ into our loss function as the class weight.  For this study we have arbitrarily chosen $\omega = 2$, so we will use class weight $\alpha = \omega/(\omega+1) = 2/3$.

%%%
\subsubsection{Decision Threshold}

Once a supervised learning algorithm learns a model using the training set, it evaluates the model on the test set, returning for each sample in the test set a probability $p\in [0,1]$ that the sample belongs to the positive class (``need ambulance'').  By default we use a decision threshold $p=0.5$ to discriminate between predicted negative (PN) and predicted positive (PP), but for good cause we can choose a different threshold.  We will choose to make the decision threshold the value of $p$ that makes $\Delta FP/\Delta TP = 2$.  Because many tools are built around having the decision threshold at $p=0.5$, rather than change the decision threshold we will linearly transform the probabilities $p \in [0,1]$ to shift the desired threshold of $p$ where $\Delta FP/\Delta TP = 2$ to $p=0.5$.  

Consider these results from one of our models.  The histogram shows, for the negative (``no ambulance'') and positive (``ambulance'') classes, the percentage of the dataset in each range of $p$.  In an ideal model, the negative class would be clustered on the left and the positive on the right.  The ROC curve and the area under the curve (AUC) indicates how cleanly the model separates the two classes, with $AUC=1$ being perfect and $AUC=0.5$ being basically random classification.


\

%%%
\parbox{\linewidth}{
%{\bf Balanced Random Forest model, Hard features, No Tomek, $\alpha = 2/3$}

\noindent\hspace{-6pt}%% Creator: Matplotlib, PGF backend
%%
%% To include the figure in your LaTeX document, write
%%   \input{<filename>.pgf}
%%
%% Make sure the required packages are loaded in your preamble
%%   \usepackage{pgf}
%%
%% Also ensure that all the required font packages are loaded; for instance,
%% the lmodern package is sometimes necessary when using math font.
%%   \usepackage{lmodern}
%%
%% Figures using additional raster images can only be included by \input if
%% they are in the same directory as the main LaTeX file. For loading figures
%% from other directories you can use the `import` package
%%   \usepackage{import}
%%
%% and then include the figures with
%%   \import{<path to file>}{<filename>.pgf}
%%
%% Matplotlib used the following preamble
%%   
%%   \usepackage{fontspec}
%%   \makeatletter\@ifpackageloaded{underscore}{}{\usepackage[strings]{underscore}}\makeatother
%%
\begingroup%
\makeatletter%
\begin{pgfpicture}%
\pgfpathrectangle{\pgfpointorigin}{\pgfqpoint{4.617331in}{1.754444in}}%
\pgfusepath{use as bounding box, clip}%
\begin{pgfscope}%
\pgfsetbuttcap%
\pgfsetmiterjoin%
\definecolor{currentfill}{rgb}{1.000000,1.000000,1.000000}%
\pgfsetfillcolor{currentfill}%
\pgfsetlinewidth{0.000000pt}%
\definecolor{currentstroke}{rgb}{1.000000,1.000000,1.000000}%
\pgfsetstrokecolor{currentstroke}%
\pgfsetdash{}{0pt}%
\pgfpathmoveto{\pgfqpoint{0.000000in}{0.000000in}}%
\pgfpathlineto{\pgfqpoint{4.617331in}{0.000000in}}%
\pgfpathlineto{\pgfqpoint{4.617331in}{1.754444in}}%
\pgfpathlineto{\pgfqpoint{0.000000in}{1.754444in}}%
\pgfpathlineto{\pgfqpoint{0.000000in}{0.000000in}}%
\pgfpathclose%
\pgfusepath{fill}%
\end{pgfscope}%
\begin{pgfscope}%
\pgfsetbuttcap%
\pgfsetmiterjoin%
\definecolor{currentfill}{rgb}{1.000000,1.000000,1.000000}%
\pgfsetfillcolor{currentfill}%
\pgfsetlinewidth{0.000000pt}%
\definecolor{currentstroke}{rgb}{0.000000,0.000000,0.000000}%
\pgfsetstrokecolor{currentstroke}%
\pgfsetstrokeopacity{0.000000}%
\pgfsetdash{}{0pt}%
\pgfpathmoveto{\pgfqpoint{0.553581in}{0.499444in}}%
\pgfpathlineto{\pgfqpoint{4.428581in}{0.499444in}}%
\pgfpathlineto{\pgfqpoint{4.428581in}{1.654444in}}%
\pgfpathlineto{\pgfqpoint{0.553581in}{1.654444in}}%
\pgfpathlineto{\pgfqpoint{0.553581in}{0.499444in}}%
\pgfpathclose%
\pgfusepath{fill}%
\end{pgfscope}%
\begin{pgfscope}%
\pgfpathrectangle{\pgfqpoint{0.553581in}{0.499444in}}{\pgfqpoint{3.875000in}{1.155000in}}%
\pgfusepath{clip}%
\pgfsetbuttcap%
\pgfsetmiterjoin%
\pgfsetlinewidth{1.003750pt}%
\definecolor{currentstroke}{rgb}{0.000000,0.000000,0.000000}%
\pgfsetstrokecolor{currentstroke}%
\pgfsetdash{}{0pt}%
\pgfpathmoveto{\pgfqpoint{0.543581in}{0.499444in}}%
\pgfpathlineto{\pgfqpoint{0.591947in}{0.499444in}}%
\pgfpathlineto{\pgfqpoint{0.591947in}{0.780988in}}%
\pgfpathlineto{\pgfqpoint{0.543581in}{0.780988in}}%
\pgfusepath{stroke}%
\end{pgfscope}%
\begin{pgfscope}%
\pgfpathrectangle{\pgfqpoint{0.553581in}{0.499444in}}{\pgfqpoint{3.875000in}{1.155000in}}%
\pgfusepath{clip}%
\pgfsetbuttcap%
\pgfsetmiterjoin%
\pgfsetlinewidth{1.003750pt}%
\definecolor{currentstroke}{rgb}{0.000000,0.000000,0.000000}%
\pgfsetstrokecolor{currentstroke}%
\pgfsetdash{}{0pt}%
\pgfpathmoveto{\pgfqpoint{0.684026in}{0.499444in}}%
\pgfpathlineto{\pgfqpoint{0.745412in}{0.499444in}}%
\pgfpathlineto{\pgfqpoint{0.745412in}{1.084017in}}%
\pgfpathlineto{\pgfqpoint{0.684026in}{1.084017in}}%
\pgfpathlineto{\pgfqpoint{0.684026in}{0.499444in}}%
\pgfpathclose%
\pgfusepath{stroke}%
\end{pgfscope}%
\begin{pgfscope}%
\pgfpathrectangle{\pgfqpoint{0.553581in}{0.499444in}}{\pgfqpoint{3.875000in}{1.155000in}}%
\pgfusepath{clip}%
\pgfsetbuttcap%
\pgfsetmiterjoin%
\pgfsetlinewidth{1.003750pt}%
\definecolor{currentstroke}{rgb}{0.000000,0.000000,0.000000}%
\pgfsetstrokecolor{currentstroke}%
\pgfsetdash{}{0pt}%
\pgfpathmoveto{\pgfqpoint{0.837492in}{0.499444in}}%
\pgfpathlineto{\pgfqpoint{0.898878in}{0.499444in}}%
\pgfpathlineto{\pgfqpoint{0.898878in}{1.366649in}}%
\pgfpathlineto{\pgfqpoint{0.837492in}{1.366649in}}%
\pgfpathlineto{\pgfqpoint{0.837492in}{0.499444in}}%
\pgfpathclose%
\pgfusepath{stroke}%
\end{pgfscope}%
\begin{pgfscope}%
\pgfpathrectangle{\pgfqpoint{0.553581in}{0.499444in}}{\pgfqpoint{3.875000in}{1.155000in}}%
\pgfusepath{clip}%
\pgfsetbuttcap%
\pgfsetmiterjoin%
\pgfsetlinewidth{1.003750pt}%
\definecolor{currentstroke}{rgb}{0.000000,0.000000,0.000000}%
\pgfsetstrokecolor{currentstroke}%
\pgfsetdash{}{0pt}%
\pgfpathmoveto{\pgfqpoint{0.990957in}{0.499444in}}%
\pgfpathlineto{\pgfqpoint{1.052343in}{0.499444in}}%
\pgfpathlineto{\pgfqpoint{1.052343in}{1.551036in}}%
\pgfpathlineto{\pgfqpoint{0.990957in}{1.551036in}}%
\pgfpathlineto{\pgfqpoint{0.990957in}{0.499444in}}%
\pgfpathclose%
\pgfusepath{stroke}%
\end{pgfscope}%
\begin{pgfscope}%
\pgfpathrectangle{\pgfqpoint{0.553581in}{0.499444in}}{\pgfqpoint{3.875000in}{1.155000in}}%
\pgfusepath{clip}%
\pgfsetbuttcap%
\pgfsetmiterjoin%
\pgfsetlinewidth{1.003750pt}%
\definecolor{currentstroke}{rgb}{0.000000,0.000000,0.000000}%
\pgfsetstrokecolor{currentstroke}%
\pgfsetdash{}{0pt}%
\pgfpathmoveto{\pgfqpoint{1.144422in}{0.499444in}}%
\pgfpathlineto{\pgfqpoint{1.205808in}{0.499444in}}%
\pgfpathlineto{\pgfqpoint{1.205808in}{1.599444in}}%
\pgfpathlineto{\pgfqpoint{1.144422in}{1.599444in}}%
\pgfpathlineto{\pgfqpoint{1.144422in}{0.499444in}}%
\pgfpathclose%
\pgfusepath{stroke}%
\end{pgfscope}%
\begin{pgfscope}%
\pgfpathrectangle{\pgfqpoint{0.553581in}{0.499444in}}{\pgfqpoint{3.875000in}{1.155000in}}%
\pgfusepath{clip}%
\pgfsetbuttcap%
\pgfsetmiterjoin%
\pgfsetlinewidth{1.003750pt}%
\definecolor{currentstroke}{rgb}{0.000000,0.000000,0.000000}%
\pgfsetstrokecolor{currentstroke}%
\pgfsetdash{}{0pt}%
\pgfpathmoveto{\pgfqpoint{1.297888in}{0.499444in}}%
\pgfpathlineto{\pgfqpoint{1.359274in}{0.499444in}}%
\pgfpathlineto{\pgfqpoint{1.359274in}{1.588158in}}%
\pgfpathlineto{\pgfqpoint{1.297888in}{1.588158in}}%
\pgfpathlineto{\pgfqpoint{1.297888in}{0.499444in}}%
\pgfpathclose%
\pgfusepath{stroke}%
\end{pgfscope}%
\begin{pgfscope}%
\pgfpathrectangle{\pgfqpoint{0.553581in}{0.499444in}}{\pgfqpoint{3.875000in}{1.155000in}}%
\pgfusepath{clip}%
\pgfsetbuttcap%
\pgfsetmiterjoin%
\pgfsetlinewidth{1.003750pt}%
\definecolor{currentstroke}{rgb}{0.000000,0.000000,0.000000}%
\pgfsetstrokecolor{currentstroke}%
\pgfsetdash{}{0pt}%
\pgfpathmoveto{\pgfqpoint{1.451353in}{0.499444in}}%
\pgfpathlineto{\pgfqpoint{1.512739in}{0.499444in}}%
\pgfpathlineto{\pgfqpoint{1.512739in}{1.487602in}}%
\pgfpathlineto{\pgfqpoint{1.451353in}{1.487602in}}%
\pgfpathlineto{\pgfqpoint{1.451353in}{0.499444in}}%
\pgfpathclose%
\pgfusepath{stroke}%
\end{pgfscope}%
\begin{pgfscope}%
\pgfpathrectangle{\pgfqpoint{0.553581in}{0.499444in}}{\pgfqpoint{3.875000in}{1.155000in}}%
\pgfusepath{clip}%
\pgfsetbuttcap%
\pgfsetmiterjoin%
\pgfsetlinewidth{1.003750pt}%
\definecolor{currentstroke}{rgb}{0.000000,0.000000,0.000000}%
\pgfsetstrokecolor{currentstroke}%
\pgfsetdash{}{0pt}%
\pgfpathmoveto{\pgfqpoint{1.604818in}{0.499444in}}%
\pgfpathlineto{\pgfqpoint{1.666204in}{0.499444in}}%
\pgfpathlineto{\pgfqpoint{1.666204in}{1.363793in}}%
\pgfpathlineto{\pgfqpoint{1.604818in}{1.363793in}}%
\pgfpathlineto{\pgfqpoint{1.604818in}{0.499444in}}%
\pgfpathclose%
\pgfusepath{stroke}%
\end{pgfscope}%
\begin{pgfscope}%
\pgfpathrectangle{\pgfqpoint{0.553581in}{0.499444in}}{\pgfqpoint{3.875000in}{1.155000in}}%
\pgfusepath{clip}%
\pgfsetbuttcap%
\pgfsetmiterjoin%
\pgfsetlinewidth{1.003750pt}%
\definecolor{currentstroke}{rgb}{0.000000,0.000000,0.000000}%
\pgfsetstrokecolor{currentstroke}%
\pgfsetdash{}{0pt}%
\pgfpathmoveto{\pgfqpoint{1.758284in}{0.499444in}}%
\pgfpathlineto{\pgfqpoint{1.819670in}{0.499444in}}%
\pgfpathlineto{\pgfqpoint{1.819670in}{1.238013in}}%
\pgfpathlineto{\pgfqpoint{1.758284in}{1.238013in}}%
\pgfpathlineto{\pgfqpoint{1.758284in}{0.499444in}}%
\pgfpathclose%
\pgfusepath{stroke}%
\end{pgfscope}%
\begin{pgfscope}%
\pgfpathrectangle{\pgfqpoint{0.553581in}{0.499444in}}{\pgfqpoint{3.875000in}{1.155000in}}%
\pgfusepath{clip}%
\pgfsetbuttcap%
\pgfsetmiterjoin%
\pgfsetlinewidth{1.003750pt}%
\definecolor{currentstroke}{rgb}{0.000000,0.000000,0.000000}%
\pgfsetstrokecolor{currentstroke}%
\pgfsetdash{}{0pt}%
\pgfpathmoveto{\pgfqpoint{1.911749in}{0.499444in}}%
\pgfpathlineto{\pgfqpoint{1.973135in}{0.499444in}}%
\pgfpathlineto{\pgfqpoint{1.973135in}{1.097683in}}%
\pgfpathlineto{\pgfqpoint{1.911749in}{1.097683in}}%
\pgfpathlineto{\pgfqpoint{1.911749in}{0.499444in}}%
\pgfpathclose%
\pgfusepath{stroke}%
\end{pgfscope}%
\begin{pgfscope}%
\pgfpathrectangle{\pgfqpoint{0.553581in}{0.499444in}}{\pgfqpoint{3.875000in}{1.155000in}}%
\pgfusepath{clip}%
\pgfsetbuttcap%
\pgfsetmiterjoin%
\pgfsetlinewidth{1.003750pt}%
\definecolor{currentstroke}{rgb}{0.000000,0.000000,0.000000}%
\pgfsetstrokecolor{currentstroke}%
\pgfsetdash{}{0pt}%
\pgfpathmoveto{\pgfqpoint{2.065214in}{0.499444in}}%
\pgfpathlineto{\pgfqpoint{2.126600in}{0.499444in}}%
\pgfpathlineto{\pgfqpoint{2.126600in}{0.982033in}}%
\pgfpathlineto{\pgfqpoint{2.065214in}{0.982033in}}%
\pgfpathlineto{\pgfqpoint{2.065214in}{0.499444in}}%
\pgfpathclose%
\pgfusepath{stroke}%
\end{pgfscope}%
\begin{pgfscope}%
\pgfpathrectangle{\pgfqpoint{0.553581in}{0.499444in}}{\pgfqpoint{3.875000in}{1.155000in}}%
\pgfusepath{clip}%
\pgfsetbuttcap%
\pgfsetmiterjoin%
\pgfsetlinewidth{1.003750pt}%
\definecolor{currentstroke}{rgb}{0.000000,0.000000,0.000000}%
\pgfsetstrokecolor{currentstroke}%
\pgfsetdash{}{0pt}%
\pgfpathmoveto{\pgfqpoint{2.218680in}{0.499444in}}%
\pgfpathlineto{\pgfqpoint{2.280066in}{0.499444in}}%
\pgfpathlineto{\pgfqpoint{2.280066in}{0.889839in}}%
\pgfpathlineto{\pgfqpoint{2.218680in}{0.889839in}}%
\pgfpathlineto{\pgfqpoint{2.218680in}{0.499444in}}%
\pgfpathclose%
\pgfusepath{stroke}%
\end{pgfscope}%
\begin{pgfscope}%
\pgfpathrectangle{\pgfqpoint{0.553581in}{0.499444in}}{\pgfqpoint{3.875000in}{1.155000in}}%
\pgfusepath{clip}%
\pgfsetbuttcap%
\pgfsetmiterjoin%
\pgfsetlinewidth{1.003750pt}%
\definecolor{currentstroke}{rgb}{0.000000,0.000000,0.000000}%
\pgfsetstrokecolor{currentstroke}%
\pgfsetdash{}{0pt}%
\pgfpathmoveto{\pgfqpoint{2.372145in}{0.499444in}}%
\pgfpathlineto{\pgfqpoint{2.433531in}{0.499444in}}%
\pgfpathlineto{\pgfqpoint{2.433531in}{0.799005in}}%
\pgfpathlineto{\pgfqpoint{2.372145in}{0.799005in}}%
\pgfpathlineto{\pgfqpoint{2.372145in}{0.499444in}}%
\pgfpathclose%
\pgfusepath{stroke}%
\end{pgfscope}%
\begin{pgfscope}%
\pgfpathrectangle{\pgfqpoint{0.553581in}{0.499444in}}{\pgfqpoint{3.875000in}{1.155000in}}%
\pgfusepath{clip}%
\pgfsetbuttcap%
\pgfsetmiterjoin%
\pgfsetlinewidth{1.003750pt}%
\definecolor{currentstroke}{rgb}{0.000000,0.000000,0.000000}%
\pgfsetstrokecolor{currentstroke}%
\pgfsetdash{}{0pt}%
\pgfpathmoveto{\pgfqpoint{2.525610in}{0.499444in}}%
\pgfpathlineto{\pgfqpoint{2.586997in}{0.499444in}}%
\pgfpathlineto{\pgfqpoint{2.586997in}{0.727073in}}%
\pgfpathlineto{\pgfqpoint{2.525610in}{0.727073in}}%
\pgfpathlineto{\pgfqpoint{2.525610in}{0.499444in}}%
\pgfpathclose%
\pgfusepath{stroke}%
\end{pgfscope}%
\begin{pgfscope}%
\pgfpathrectangle{\pgfqpoint{0.553581in}{0.499444in}}{\pgfqpoint{3.875000in}{1.155000in}}%
\pgfusepath{clip}%
\pgfsetbuttcap%
\pgfsetmiterjoin%
\pgfsetlinewidth{1.003750pt}%
\definecolor{currentstroke}{rgb}{0.000000,0.000000,0.000000}%
\pgfsetstrokecolor{currentstroke}%
\pgfsetdash{}{0pt}%
\pgfpathmoveto{\pgfqpoint{2.679076in}{0.499444in}}%
\pgfpathlineto{\pgfqpoint{2.740462in}{0.499444in}}%
\pgfpathlineto{\pgfqpoint{2.740462in}{0.666698in}}%
\pgfpathlineto{\pgfqpoint{2.679076in}{0.666698in}}%
\pgfpathlineto{\pgfqpoint{2.679076in}{0.499444in}}%
\pgfpathclose%
\pgfusepath{stroke}%
\end{pgfscope}%
\begin{pgfscope}%
\pgfpathrectangle{\pgfqpoint{0.553581in}{0.499444in}}{\pgfqpoint{3.875000in}{1.155000in}}%
\pgfusepath{clip}%
\pgfsetbuttcap%
\pgfsetmiterjoin%
\pgfsetlinewidth{1.003750pt}%
\definecolor{currentstroke}{rgb}{0.000000,0.000000,0.000000}%
\pgfsetstrokecolor{currentstroke}%
\pgfsetdash{}{0pt}%
\pgfpathmoveto{\pgfqpoint{2.832541in}{0.499444in}}%
\pgfpathlineto{\pgfqpoint{2.893927in}{0.499444in}}%
\pgfpathlineto{\pgfqpoint{2.893927in}{0.630664in}}%
\pgfpathlineto{\pgfqpoint{2.832541in}{0.630664in}}%
\pgfpathlineto{\pgfqpoint{2.832541in}{0.499444in}}%
\pgfpathclose%
\pgfusepath{stroke}%
\end{pgfscope}%
\begin{pgfscope}%
\pgfpathrectangle{\pgfqpoint{0.553581in}{0.499444in}}{\pgfqpoint{3.875000in}{1.155000in}}%
\pgfusepath{clip}%
\pgfsetbuttcap%
\pgfsetmiterjoin%
\pgfsetlinewidth{1.003750pt}%
\definecolor{currentstroke}{rgb}{0.000000,0.000000,0.000000}%
\pgfsetstrokecolor{currentstroke}%
\pgfsetdash{}{0pt}%
\pgfpathmoveto{\pgfqpoint{2.986006in}{0.499444in}}%
\pgfpathlineto{\pgfqpoint{3.047393in}{0.499444in}}%
\pgfpathlineto{\pgfqpoint{3.047393in}{0.591638in}}%
\pgfpathlineto{\pgfqpoint{2.986006in}{0.591638in}}%
\pgfpathlineto{\pgfqpoint{2.986006in}{0.499444in}}%
\pgfpathclose%
\pgfusepath{stroke}%
\end{pgfscope}%
\begin{pgfscope}%
\pgfpathrectangle{\pgfqpoint{0.553581in}{0.499444in}}{\pgfqpoint{3.875000in}{1.155000in}}%
\pgfusepath{clip}%
\pgfsetbuttcap%
\pgfsetmiterjoin%
\pgfsetlinewidth{1.003750pt}%
\definecolor{currentstroke}{rgb}{0.000000,0.000000,0.000000}%
\pgfsetstrokecolor{currentstroke}%
\pgfsetdash{}{0pt}%
\pgfpathmoveto{\pgfqpoint{3.139472in}{0.499444in}}%
\pgfpathlineto{\pgfqpoint{3.200858in}{0.499444in}}%
\pgfpathlineto{\pgfqpoint{3.200858in}{0.641406in}}%
\pgfpathlineto{\pgfqpoint{3.139472in}{0.641406in}}%
\pgfpathlineto{\pgfqpoint{3.139472in}{0.499444in}}%
\pgfpathclose%
\pgfusepath{stroke}%
\end{pgfscope}%
\begin{pgfscope}%
\pgfpathrectangle{\pgfqpoint{0.553581in}{0.499444in}}{\pgfqpoint{3.875000in}{1.155000in}}%
\pgfusepath{clip}%
\pgfsetbuttcap%
\pgfsetmiterjoin%
\pgfsetlinewidth{1.003750pt}%
\definecolor{currentstroke}{rgb}{0.000000,0.000000,0.000000}%
\pgfsetstrokecolor{currentstroke}%
\pgfsetdash{}{0pt}%
\pgfpathmoveto{\pgfqpoint{3.292937in}{0.499444in}}%
\pgfpathlineto{\pgfqpoint{3.354323in}{0.499444in}}%
\pgfpathlineto{\pgfqpoint{3.354323in}{0.551048in}}%
\pgfpathlineto{\pgfqpoint{3.292937in}{0.551048in}}%
\pgfpathlineto{\pgfqpoint{3.292937in}{0.499444in}}%
\pgfpathclose%
\pgfusepath{stroke}%
\end{pgfscope}%
\begin{pgfscope}%
\pgfpathrectangle{\pgfqpoint{0.553581in}{0.499444in}}{\pgfqpoint{3.875000in}{1.155000in}}%
\pgfusepath{clip}%
\pgfsetbuttcap%
\pgfsetmiterjoin%
\pgfsetlinewidth{1.003750pt}%
\definecolor{currentstroke}{rgb}{0.000000,0.000000,0.000000}%
\pgfsetstrokecolor{currentstroke}%
\pgfsetdash{}{0pt}%
\pgfpathmoveto{\pgfqpoint{3.446402in}{0.499444in}}%
\pgfpathlineto{\pgfqpoint{3.507789in}{0.499444in}}%
\pgfpathlineto{\pgfqpoint{3.507789in}{0.535003in}}%
\pgfpathlineto{\pgfqpoint{3.446402in}{0.535003in}}%
\pgfpathlineto{\pgfqpoint{3.446402in}{0.499444in}}%
\pgfpathclose%
\pgfusepath{stroke}%
\end{pgfscope}%
\begin{pgfscope}%
\pgfpathrectangle{\pgfqpoint{0.553581in}{0.499444in}}{\pgfqpoint{3.875000in}{1.155000in}}%
\pgfusepath{clip}%
\pgfsetbuttcap%
\pgfsetmiterjoin%
\pgfsetlinewidth{1.003750pt}%
\definecolor{currentstroke}{rgb}{0.000000,0.000000,0.000000}%
\pgfsetstrokecolor{currentstroke}%
\pgfsetdash{}{0pt}%
\pgfpathmoveto{\pgfqpoint{3.599868in}{0.499444in}}%
\pgfpathlineto{\pgfqpoint{3.661254in}{0.499444in}}%
\pgfpathlineto{\pgfqpoint{3.661254in}{0.514606in}}%
\pgfpathlineto{\pgfqpoint{3.599868in}{0.514606in}}%
\pgfpathlineto{\pgfqpoint{3.599868in}{0.499444in}}%
\pgfpathclose%
\pgfusepath{stroke}%
\end{pgfscope}%
\begin{pgfscope}%
\pgfpathrectangle{\pgfqpoint{0.553581in}{0.499444in}}{\pgfqpoint{3.875000in}{1.155000in}}%
\pgfusepath{clip}%
\pgfsetbuttcap%
\pgfsetmiterjoin%
\pgfsetlinewidth{1.003750pt}%
\definecolor{currentstroke}{rgb}{0.000000,0.000000,0.000000}%
\pgfsetstrokecolor{currentstroke}%
\pgfsetdash{}{0pt}%
\pgfpathmoveto{\pgfqpoint{3.753333in}{0.499444in}}%
\pgfpathlineto{\pgfqpoint{3.814719in}{0.499444in}}%
\pgfpathlineto{\pgfqpoint{3.814719in}{0.499444in}}%
\pgfpathlineto{\pgfqpoint{3.753333in}{0.499444in}}%
\pgfpathlineto{\pgfqpoint{3.753333in}{0.499444in}}%
\pgfpathclose%
\pgfusepath{stroke}%
\end{pgfscope}%
\begin{pgfscope}%
\pgfpathrectangle{\pgfqpoint{0.553581in}{0.499444in}}{\pgfqpoint{3.875000in}{1.155000in}}%
\pgfusepath{clip}%
\pgfsetbuttcap%
\pgfsetmiterjoin%
\pgfsetlinewidth{1.003750pt}%
\definecolor{currentstroke}{rgb}{0.000000,0.000000,0.000000}%
\pgfsetstrokecolor{currentstroke}%
\pgfsetdash{}{0pt}%
\pgfpathmoveto{\pgfqpoint{3.906799in}{0.499444in}}%
\pgfpathlineto{\pgfqpoint{3.968185in}{0.499444in}}%
\pgfpathlineto{\pgfqpoint{3.968185in}{0.499444in}}%
\pgfpathlineto{\pgfqpoint{3.906799in}{0.499444in}}%
\pgfpathlineto{\pgfqpoint{3.906799in}{0.499444in}}%
\pgfpathclose%
\pgfusepath{stroke}%
\end{pgfscope}%
\begin{pgfscope}%
\pgfpathrectangle{\pgfqpoint{0.553581in}{0.499444in}}{\pgfqpoint{3.875000in}{1.155000in}}%
\pgfusepath{clip}%
\pgfsetbuttcap%
\pgfsetmiterjoin%
\pgfsetlinewidth{1.003750pt}%
\definecolor{currentstroke}{rgb}{0.000000,0.000000,0.000000}%
\pgfsetstrokecolor{currentstroke}%
\pgfsetdash{}{0pt}%
\pgfpathmoveto{\pgfqpoint{4.060264in}{0.499444in}}%
\pgfpathlineto{\pgfqpoint{4.121650in}{0.499444in}}%
\pgfpathlineto{\pgfqpoint{4.121650in}{0.499444in}}%
\pgfpathlineto{\pgfqpoint{4.060264in}{0.499444in}}%
\pgfpathlineto{\pgfqpoint{4.060264in}{0.499444in}}%
\pgfpathclose%
\pgfusepath{stroke}%
\end{pgfscope}%
\begin{pgfscope}%
\pgfpathrectangle{\pgfqpoint{0.553581in}{0.499444in}}{\pgfqpoint{3.875000in}{1.155000in}}%
\pgfusepath{clip}%
\pgfsetbuttcap%
\pgfsetmiterjoin%
\pgfsetlinewidth{1.003750pt}%
\definecolor{currentstroke}{rgb}{0.000000,0.000000,0.000000}%
\pgfsetstrokecolor{currentstroke}%
\pgfsetdash{}{0pt}%
\pgfpathmoveto{\pgfqpoint{4.213729in}{0.499444in}}%
\pgfpathlineto{\pgfqpoint{4.275115in}{0.499444in}}%
\pgfpathlineto{\pgfqpoint{4.275115in}{0.499444in}}%
\pgfpathlineto{\pgfqpoint{4.213729in}{0.499444in}}%
\pgfpathlineto{\pgfqpoint{4.213729in}{0.499444in}}%
\pgfpathclose%
\pgfusepath{stroke}%
\end{pgfscope}%
\begin{pgfscope}%
\pgfpathrectangle{\pgfqpoint{0.553581in}{0.499444in}}{\pgfqpoint{3.875000in}{1.155000in}}%
\pgfusepath{clip}%
\pgfsetbuttcap%
\pgfsetmiterjoin%
\definecolor{currentfill}{rgb}{0.000000,0.000000,0.000000}%
\pgfsetfillcolor{currentfill}%
\pgfsetlinewidth{0.000000pt}%
\definecolor{currentstroke}{rgb}{0.000000,0.000000,0.000000}%
\pgfsetstrokecolor{currentstroke}%
\pgfsetstrokeopacity{0.000000}%
\pgfsetdash{}{0pt}%
\pgfpathmoveto{\pgfqpoint{0.591947in}{0.499444in}}%
\pgfpathlineto{\pgfqpoint{0.653333in}{0.499444in}}%
\pgfpathlineto{\pgfqpoint{0.653333in}{0.509031in}}%
\pgfpathlineto{\pgfqpoint{0.591947in}{0.509031in}}%
\pgfpathlineto{\pgfqpoint{0.591947in}{0.499444in}}%
\pgfpathclose%
\pgfusepath{fill}%
\end{pgfscope}%
\begin{pgfscope}%
\pgfpathrectangle{\pgfqpoint{0.553581in}{0.499444in}}{\pgfqpoint{3.875000in}{1.155000in}}%
\pgfusepath{clip}%
\pgfsetbuttcap%
\pgfsetmiterjoin%
\definecolor{currentfill}{rgb}{0.000000,0.000000,0.000000}%
\pgfsetfillcolor{currentfill}%
\pgfsetlinewidth{0.000000pt}%
\definecolor{currentstroke}{rgb}{0.000000,0.000000,0.000000}%
\pgfsetstrokecolor{currentstroke}%
\pgfsetstrokeopacity{0.000000}%
\pgfsetdash{}{0pt}%
\pgfpathmoveto{\pgfqpoint{0.745412in}{0.499444in}}%
\pgfpathlineto{\pgfqpoint{0.806799in}{0.499444in}}%
\pgfpathlineto{\pgfqpoint{0.806799in}{0.511614in}}%
\pgfpathlineto{\pgfqpoint{0.745412in}{0.511614in}}%
\pgfpathlineto{\pgfqpoint{0.745412in}{0.499444in}}%
\pgfpathclose%
\pgfusepath{fill}%
\end{pgfscope}%
\begin{pgfscope}%
\pgfpathrectangle{\pgfqpoint{0.553581in}{0.499444in}}{\pgfqpoint{3.875000in}{1.155000in}}%
\pgfusepath{clip}%
\pgfsetbuttcap%
\pgfsetmiterjoin%
\definecolor{currentfill}{rgb}{0.000000,0.000000,0.000000}%
\pgfsetfillcolor{currentfill}%
\pgfsetlinewidth{0.000000pt}%
\definecolor{currentstroke}{rgb}{0.000000,0.000000,0.000000}%
\pgfsetstrokecolor{currentstroke}%
\pgfsetstrokeopacity{0.000000}%
\pgfsetdash{}{0pt}%
\pgfpathmoveto{\pgfqpoint{0.898878in}{0.499444in}}%
\pgfpathlineto{\pgfqpoint{0.960264in}{0.499444in}}%
\pgfpathlineto{\pgfqpoint{0.960264in}{0.517189in}}%
\pgfpathlineto{\pgfqpoint{0.898878in}{0.517189in}}%
\pgfpathlineto{\pgfqpoint{0.898878in}{0.499444in}}%
\pgfpathclose%
\pgfusepath{fill}%
\end{pgfscope}%
\begin{pgfscope}%
\pgfpathrectangle{\pgfqpoint{0.553581in}{0.499444in}}{\pgfqpoint{3.875000in}{1.155000in}}%
\pgfusepath{clip}%
\pgfsetbuttcap%
\pgfsetmiterjoin%
\definecolor{currentfill}{rgb}{0.000000,0.000000,0.000000}%
\pgfsetfillcolor{currentfill}%
\pgfsetlinewidth{0.000000pt}%
\definecolor{currentstroke}{rgb}{0.000000,0.000000,0.000000}%
\pgfsetstrokecolor{currentstroke}%
\pgfsetstrokeopacity{0.000000}%
\pgfsetdash{}{0pt}%
\pgfpathmoveto{\pgfqpoint{1.052343in}{0.499444in}}%
\pgfpathlineto{\pgfqpoint{1.113729in}{0.499444in}}%
\pgfpathlineto{\pgfqpoint{1.113729in}{0.522900in}}%
\pgfpathlineto{\pgfqpoint{1.052343in}{0.522900in}}%
\pgfpathlineto{\pgfqpoint{1.052343in}{0.499444in}}%
\pgfpathclose%
\pgfusepath{fill}%
\end{pgfscope}%
\begin{pgfscope}%
\pgfpathrectangle{\pgfqpoint{0.553581in}{0.499444in}}{\pgfqpoint{3.875000in}{1.155000in}}%
\pgfusepath{clip}%
\pgfsetbuttcap%
\pgfsetmiterjoin%
\definecolor{currentfill}{rgb}{0.000000,0.000000,0.000000}%
\pgfsetfillcolor{currentfill}%
\pgfsetlinewidth{0.000000pt}%
\definecolor{currentstroke}{rgb}{0.000000,0.000000,0.000000}%
\pgfsetstrokecolor{currentstroke}%
\pgfsetstrokeopacity{0.000000}%
\pgfsetdash{}{0pt}%
\pgfpathmoveto{\pgfqpoint{1.205808in}{0.499444in}}%
\pgfpathlineto{\pgfqpoint{1.267195in}{0.499444in}}%
\pgfpathlineto{\pgfqpoint{1.267195in}{0.544317in}}%
\pgfpathlineto{\pgfqpoint{1.205808in}{0.544317in}}%
\pgfpathlineto{\pgfqpoint{1.205808in}{0.499444in}}%
\pgfpathclose%
\pgfusepath{fill}%
\end{pgfscope}%
\begin{pgfscope}%
\pgfpathrectangle{\pgfqpoint{0.553581in}{0.499444in}}{\pgfqpoint{3.875000in}{1.155000in}}%
\pgfusepath{clip}%
\pgfsetbuttcap%
\pgfsetmiterjoin%
\definecolor{currentfill}{rgb}{0.000000,0.000000,0.000000}%
\pgfsetfillcolor{currentfill}%
\pgfsetlinewidth{0.000000pt}%
\definecolor{currentstroke}{rgb}{0.000000,0.000000,0.000000}%
\pgfsetstrokecolor{currentstroke}%
\pgfsetstrokeopacity{0.000000}%
\pgfsetdash{}{0pt}%
\pgfpathmoveto{\pgfqpoint{1.359274in}{0.499444in}}%
\pgfpathlineto{\pgfqpoint{1.420660in}{0.499444in}}%
\pgfpathlineto{\pgfqpoint{1.420660in}{0.536226in}}%
\pgfpathlineto{\pgfqpoint{1.359274in}{0.536226in}}%
\pgfpathlineto{\pgfqpoint{1.359274in}{0.499444in}}%
\pgfpathclose%
\pgfusepath{fill}%
\end{pgfscope}%
\begin{pgfscope}%
\pgfpathrectangle{\pgfqpoint{0.553581in}{0.499444in}}{\pgfqpoint{3.875000in}{1.155000in}}%
\pgfusepath{clip}%
\pgfsetbuttcap%
\pgfsetmiterjoin%
\definecolor{currentfill}{rgb}{0.000000,0.000000,0.000000}%
\pgfsetfillcolor{currentfill}%
\pgfsetlinewidth{0.000000pt}%
\definecolor{currentstroke}{rgb}{0.000000,0.000000,0.000000}%
\pgfsetstrokecolor{currentstroke}%
\pgfsetstrokeopacity{0.000000}%
\pgfsetdash{}{0pt}%
\pgfpathmoveto{\pgfqpoint{1.512739in}{0.499444in}}%
\pgfpathlineto{\pgfqpoint{1.574125in}{0.499444in}}%
\pgfpathlineto{\pgfqpoint{1.574125in}{0.551796in}}%
\pgfpathlineto{\pgfqpoint{1.512739in}{0.551796in}}%
\pgfpathlineto{\pgfqpoint{1.512739in}{0.499444in}}%
\pgfpathclose%
\pgfusepath{fill}%
\end{pgfscope}%
\begin{pgfscope}%
\pgfpathrectangle{\pgfqpoint{0.553581in}{0.499444in}}{\pgfqpoint{3.875000in}{1.155000in}}%
\pgfusepath{clip}%
\pgfsetbuttcap%
\pgfsetmiterjoin%
\definecolor{currentfill}{rgb}{0.000000,0.000000,0.000000}%
\pgfsetfillcolor{currentfill}%
\pgfsetlinewidth{0.000000pt}%
\definecolor{currentstroke}{rgb}{0.000000,0.000000,0.000000}%
\pgfsetstrokecolor{currentstroke}%
\pgfsetstrokeopacity{0.000000}%
\pgfsetdash{}{0pt}%
\pgfpathmoveto{\pgfqpoint{1.666204in}{0.499444in}}%
\pgfpathlineto{\pgfqpoint{1.727591in}{0.499444in}}%
\pgfpathlineto{\pgfqpoint{1.727591in}{0.567298in}}%
\pgfpathlineto{\pgfqpoint{1.666204in}{0.567298in}}%
\pgfpathlineto{\pgfqpoint{1.666204in}{0.499444in}}%
\pgfpathclose%
\pgfusepath{fill}%
\end{pgfscope}%
\begin{pgfscope}%
\pgfpathrectangle{\pgfqpoint{0.553581in}{0.499444in}}{\pgfqpoint{3.875000in}{1.155000in}}%
\pgfusepath{clip}%
\pgfsetbuttcap%
\pgfsetmiterjoin%
\definecolor{currentfill}{rgb}{0.000000,0.000000,0.000000}%
\pgfsetfillcolor{currentfill}%
\pgfsetlinewidth{0.000000pt}%
\definecolor{currentstroke}{rgb}{0.000000,0.000000,0.000000}%
\pgfsetstrokecolor{currentstroke}%
\pgfsetstrokeopacity{0.000000}%
\pgfsetdash{}{0pt}%
\pgfpathmoveto{\pgfqpoint{1.819670in}{0.499444in}}%
\pgfpathlineto{\pgfqpoint{1.881056in}{0.499444in}}%
\pgfpathlineto{\pgfqpoint{1.881056in}{0.582731in}}%
\pgfpathlineto{\pgfqpoint{1.819670in}{0.582731in}}%
\pgfpathlineto{\pgfqpoint{1.819670in}{0.499444in}}%
\pgfpathclose%
\pgfusepath{fill}%
\end{pgfscope}%
\begin{pgfscope}%
\pgfpathrectangle{\pgfqpoint{0.553581in}{0.499444in}}{\pgfqpoint{3.875000in}{1.155000in}}%
\pgfusepath{clip}%
\pgfsetbuttcap%
\pgfsetmiterjoin%
\definecolor{currentfill}{rgb}{0.000000,0.000000,0.000000}%
\pgfsetfillcolor{currentfill}%
\pgfsetlinewidth{0.000000pt}%
\definecolor{currentstroke}{rgb}{0.000000,0.000000,0.000000}%
\pgfsetstrokecolor{currentstroke}%
\pgfsetstrokeopacity{0.000000}%
\pgfsetdash{}{0pt}%
\pgfpathmoveto{\pgfqpoint{1.973135in}{0.499444in}}%
\pgfpathlineto{\pgfqpoint{2.034521in}{0.499444in}}%
\pgfpathlineto{\pgfqpoint{2.034521in}{0.602176in}}%
\pgfpathlineto{\pgfqpoint{1.973135in}{0.602176in}}%
\pgfpathlineto{\pgfqpoint{1.973135in}{0.499444in}}%
\pgfpathclose%
\pgfusepath{fill}%
\end{pgfscope}%
\begin{pgfscope}%
\pgfpathrectangle{\pgfqpoint{0.553581in}{0.499444in}}{\pgfqpoint{3.875000in}{1.155000in}}%
\pgfusepath{clip}%
\pgfsetbuttcap%
\pgfsetmiterjoin%
\definecolor{currentfill}{rgb}{0.000000,0.000000,0.000000}%
\pgfsetfillcolor{currentfill}%
\pgfsetlinewidth{0.000000pt}%
\definecolor{currentstroke}{rgb}{0.000000,0.000000,0.000000}%
\pgfsetstrokecolor{currentstroke}%
\pgfsetstrokeopacity{0.000000}%
\pgfsetdash{}{0pt}%
\pgfpathmoveto{\pgfqpoint{2.126600in}{0.499444in}}%
\pgfpathlineto{\pgfqpoint{2.187987in}{0.499444in}}%
\pgfpathlineto{\pgfqpoint{2.187987in}{0.624137in}}%
\pgfpathlineto{\pgfqpoint{2.126600in}{0.624137in}}%
\pgfpathlineto{\pgfqpoint{2.126600in}{0.499444in}}%
\pgfpathclose%
\pgfusepath{fill}%
\end{pgfscope}%
\begin{pgfscope}%
\pgfpathrectangle{\pgfqpoint{0.553581in}{0.499444in}}{\pgfqpoint{3.875000in}{1.155000in}}%
\pgfusepath{clip}%
\pgfsetbuttcap%
\pgfsetmiterjoin%
\definecolor{currentfill}{rgb}{0.000000,0.000000,0.000000}%
\pgfsetfillcolor{currentfill}%
\pgfsetlinewidth{0.000000pt}%
\definecolor{currentstroke}{rgb}{0.000000,0.000000,0.000000}%
\pgfsetstrokecolor{currentstroke}%
\pgfsetstrokeopacity{0.000000}%
\pgfsetdash{}{0pt}%
\pgfpathmoveto{\pgfqpoint{2.280066in}{0.499444in}}%
\pgfpathlineto{\pgfqpoint{2.341452in}{0.499444in}}%
\pgfpathlineto{\pgfqpoint{2.341452in}{0.646981in}}%
\pgfpathlineto{\pgfqpoint{2.280066in}{0.646981in}}%
\pgfpathlineto{\pgfqpoint{2.280066in}{0.499444in}}%
\pgfpathclose%
\pgfusepath{fill}%
\end{pgfscope}%
\begin{pgfscope}%
\pgfpathrectangle{\pgfqpoint{0.553581in}{0.499444in}}{\pgfqpoint{3.875000in}{1.155000in}}%
\pgfusepath{clip}%
\pgfsetbuttcap%
\pgfsetmiterjoin%
\definecolor{currentfill}{rgb}{0.000000,0.000000,0.000000}%
\pgfsetfillcolor{currentfill}%
\pgfsetlinewidth{0.000000pt}%
\definecolor{currentstroke}{rgb}{0.000000,0.000000,0.000000}%
\pgfsetstrokecolor{currentstroke}%
\pgfsetstrokeopacity{0.000000}%
\pgfsetdash{}{0pt}%
\pgfpathmoveto{\pgfqpoint{2.433531in}{0.499444in}}%
\pgfpathlineto{\pgfqpoint{2.494917in}{0.499444in}}%
\pgfpathlineto{\pgfqpoint{2.494917in}{0.665542in}}%
\pgfpathlineto{\pgfqpoint{2.433531in}{0.665542in}}%
\pgfpathlineto{\pgfqpoint{2.433531in}{0.499444in}}%
\pgfpathclose%
\pgfusepath{fill}%
\end{pgfscope}%
\begin{pgfscope}%
\pgfpathrectangle{\pgfqpoint{0.553581in}{0.499444in}}{\pgfqpoint{3.875000in}{1.155000in}}%
\pgfusepath{clip}%
\pgfsetbuttcap%
\pgfsetmiterjoin%
\definecolor{currentfill}{rgb}{0.000000,0.000000,0.000000}%
\pgfsetfillcolor{currentfill}%
\pgfsetlinewidth{0.000000pt}%
\definecolor{currentstroke}{rgb}{0.000000,0.000000,0.000000}%
\pgfsetstrokecolor{currentstroke}%
\pgfsetstrokeopacity{0.000000}%
\pgfsetdash{}{0pt}%
\pgfpathmoveto{\pgfqpoint{2.586997in}{0.499444in}}%
\pgfpathlineto{\pgfqpoint{2.648383in}{0.499444in}}%
\pgfpathlineto{\pgfqpoint{2.648383in}{0.688523in}}%
\pgfpathlineto{\pgfqpoint{2.586997in}{0.688523in}}%
\pgfpathlineto{\pgfqpoint{2.586997in}{0.499444in}}%
\pgfpathclose%
\pgfusepath{fill}%
\end{pgfscope}%
\begin{pgfscope}%
\pgfpathrectangle{\pgfqpoint{0.553581in}{0.499444in}}{\pgfqpoint{3.875000in}{1.155000in}}%
\pgfusepath{clip}%
\pgfsetbuttcap%
\pgfsetmiterjoin%
\definecolor{currentfill}{rgb}{0.000000,0.000000,0.000000}%
\pgfsetfillcolor{currentfill}%
\pgfsetlinewidth{0.000000pt}%
\definecolor{currentstroke}{rgb}{0.000000,0.000000,0.000000}%
\pgfsetstrokecolor{currentstroke}%
\pgfsetstrokeopacity{0.000000}%
\pgfsetdash{}{0pt}%
\pgfpathmoveto{\pgfqpoint{2.740462in}{0.499444in}}%
\pgfpathlineto{\pgfqpoint{2.801848in}{0.499444in}}%
\pgfpathlineto{\pgfqpoint{2.801848in}{0.692534in}}%
\pgfpathlineto{\pgfqpoint{2.740462in}{0.692534in}}%
\pgfpathlineto{\pgfqpoint{2.740462in}{0.499444in}}%
\pgfpathclose%
\pgfusepath{fill}%
\end{pgfscope}%
\begin{pgfscope}%
\pgfpathrectangle{\pgfqpoint{0.553581in}{0.499444in}}{\pgfqpoint{3.875000in}{1.155000in}}%
\pgfusepath{clip}%
\pgfsetbuttcap%
\pgfsetmiterjoin%
\definecolor{currentfill}{rgb}{0.000000,0.000000,0.000000}%
\pgfsetfillcolor{currentfill}%
\pgfsetlinewidth{0.000000pt}%
\definecolor{currentstroke}{rgb}{0.000000,0.000000,0.000000}%
\pgfsetstrokecolor{currentstroke}%
\pgfsetstrokeopacity{0.000000}%
\pgfsetdash{}{0pt}%
\pgfpathmoveto{\pgfqpoint{2.893927in}{0.499444in}}%
\pgfpathlineto{\pgfqpoint{2.955313in}{0.499444in}}%
\pgfpathlineto{\pgfqpoint{2.955313in}{0.684171in}}%
\pgfpathlineto{\pgfqpoint{2.893927in}{0.684171in}}%
\pgfpathlineto{\pgfqpoint{2.893927in}{0.499444in}}%
\pgfpathclose%
\pgfusepath{fill}%
\end{pgfscope}%
\begin{pgfscope}%
\pgfpathrectangle{\pgfqpoint{0.553581in}{0.499444in}}{\pgfqpoint{3.875000in}{1.155000in}}%
\pgfusepath{clip}%
\pgfsetbuttcap%
\pgfsetmiterjoin%
\definecolor{currentfill}{rgb}{0.000000,0.000000,0.000000}%
\pgfsetfillcolor{currentfill}%
\pgfsetlinewidth{0.000000pt}%
\definecolor{currentstroke}{rgb}{0.000000,0.000000,0.000000}%
\pgfsetstrokecolor{currentstroke}%
\pgfsetstrokeopacity{0.000000}%
\pgfsetdash{}{0pt}%
\pgfpathmoveto{\pgfqpoint{3.047393in}{0.499444in}}%
\pgfpathlineto{\pgfqpoint{3.108779in}{0.499444in}}%
\pgfpathlineto{\pgfqpoint{3.108779in}{0.657655in}}%
\pgfpathlineto{\pgfqpoint{3.047393in}{0.657655in}}%
\pgfpathlineto{\pgfqpoint{3.047393in}{0.499444in}}%
\pgfpathclose%
\pgfusepath{fill}%
\end{pgfscope}%
\begin{pgfscope}%
\pgfpathrectangle{\pgfqpoint{0.553581in}{0.499444in}}{\pgfqpoint{3.875000in}{1.155000in}}%
\pgfusepath{clip}%
\pgfsetbuttcap%
\pgfsetmiterjoin%
\definecolor{currentfill}{rgb}{0.000000,0.000000,0.000000}%
\pgfsetfillcolor{currentfill}%
\pgfsetlinewidth{0.000000pt}%
\definecolor{currentstroke}{rgb}{0.000000,0.000000,0.000000}%
\pgfsetstrokecolor{currentstroke}%
\pgfsetstrokeopacity{0.000000}%
\pgfsetdash{}{0pt}%
\pgfpathmoveto{\pgfqpoint{3.200858in}{0.499444in}}%
\pgfpathlineto{\pgfqpoint{3.262244in}{0.499444in}}%
\pgfpathlineto{\pgfqpoint{3.262244in}{0.613190in}}%
\pgfpathlineto{\pgfqpoint{3.200858in}{0.613190in}}%
\pgfpathlineto{\pgfqpoint{3.200858in}{0.499444in}}%
\pgfpathclose%
\pgfusepath{fill}%
\end{pgfscope}%
\begin{pgfscope}%
\pgfpathrectangle{\pgfqpoint{0.553581in}{0.499444in}}{\pgfqpoint{3.875000in}{1.155000in}}%
\pgfusepath{clip}%
\pgfsetbuttcap%
\pgfsetmiterjoin%
\definecolor{currentfill}{rgb}{0.000000,0.000000,0.000000}%
\pgfsetfillcolor{currentfill}%
\pgfsetlinewidth{0.000000pt}%
\definecolor{currentstroke}{rgb}{0.000000,0.000000,0.000000}%
\pgfsetstrokecolor{currentstroke}%
\pgfsetstrokeopacity{0.000000}%
\pgfsetdash{}{0pt}%
\pgfpathmoveto{\pgfqpoint{3.354323in}{0.499444in}}%
\pgfpathlineto{\pgfqpoint{3.415709in}{0.499444in}}%
\pgfpathlineto{\pgfqpoint{3.415709in}{0.557915in}}%
\pgfpathlineto{\pgfqpoint{3.354323in}{0.557915in}}%
\pgfpathlineto{\pgfqpoint{3.354323in}{0.499444in}}%
\pgfpathclose%
\pgfusepath{fill}%
\end{pgfscope}%
\begin{pgfscope}%
\pgfpathrectangle{\pgfqpoint{0.553581in}{0.499444in}}{\pgfqpoint{3.875000in}{1.155000in}}%
\pgfusepath{clip}%
\pgfsetbuttcap%
\pgfsetmiterjoin%
\definecolor{currentfill}{rgb}{0.000000,0.000000,0.000000}%
\pgfsetfillcolor{currentfill}%
\pgfsetlinewidth{0.000000pt}%
\definecolor{currentstroke}{rgb}{0.000000,0.000000,0.000000}%
\pgfsetstrokecolor{currentstroke}%
\pgfsetstrokeopacity{0.000000}%
\pgfsetdash{}{0pt}%
\pgfpathmoveto{\pgfqpoint{3.507789in}{0.499444in}}%
\pgfpathlineto{\pgfqpoint{3.569175in}{0.499444in}}%
\pgfpathlineto{\pgfqpoint{3.569175in}{0.514946in}}%
\pgfpathlineto{\pgfqpoint{3.507789in}{0.514946in}}%
\pgfpathlineto{\pgfqpoint{3.507789in}{0.499444in}}%
\pgfpathclose%
\pgfusepath{fill}%
\end{pgfscope}%
\begin{pgfscope}%
\pgfpathrectangle{\pgfqpoint{0.553581in}{0.499444in}}{\pgfqpoint{3.875000in}{1.155000in}}%
\pgfusepath{clip}%
\pgfsetbuttcap%
\pgfsetmiterjoin%
\definecolor{currentfill}{rgb}{0.000000,0.000000,0.000000}%
\pgfsetfillcolor{currentfill}%
\pgfsetlinewidth{0.000000pt}%
\definecolor{currentstroke}{rgb}{0.000000,0.000000,0.000000}%
\pgfsetstrokecolor{currentstroke}%
\pgfsetstrokeopacity{0.000000}%
\pgfsetdash{}{0pt}%
\pgfpathmoveto{\pgfqpoint{3.661254in}{0.499444in}}%
\pgfpathlineto{\pgfqpoint{3.722640in}{0.499444in}}%
\pgfpathlineto{\pgfqpoint{3.722640in}{0.499852in}}%
\pgfpathlineto{\pgfqpoint{3.661254in}{0.499852in}}%
\pgfpathlineto{\pgfqpoint{3.661254in}{0.499444in}}%
\pgfpathclose%
\pgfusepath{fill}%
\end{pgfscope}%
\begin{pgfscope}%
\pgfpathrectangle{\pgfqpoint{0.553581in}{0.499444in}}{\pgfqpoint{3.875000in}{1.155000in}}%
\pgfusepath{clip}%
\pgfsetbuttcap%
\pgfsetmiterjoin%
\definecolor{currentfill}{rgb}{0.000000,0.000000,0.000000}%
\pgfsetfillcolor{currentfill}%
\pgfsetlinewidth{0.000000pt}%
\definecolor{currentstroke}{rgb}{0.000000,0.000000,0.000000}%
\pgfsetstrokecolor{currentstroke}%
\pgfsetstrokeopacity{0.000000}%
\pgfsetdash{}{0pt}%
\pgfpathmoveto{\pgfqpoint{3.814719in}{0.499444in}}%
\pgfpathlineto{\pgfqpoint{3.876105in}{0.499444in}}%
\pgfpathlineto{\pgfqpoint{3.876105in}{0.499444in}}%
\pgfpathlineto{\pgfqpoint{3.814719in}{0.499444in}}%
\pgfpathlineto{\pgfqpoint{3.814719in}{0.499444in}}%
\pgfpathclose%
\pgfusepath{fill}%
\end{pgfscope}%
\begin{pgfscope}%
\pgfpathrectangle{\pgfqpoint{0.553581in}{0.499444in}}{\pgfqpoint{3.875000in}{1.155000in}}%
\pgfusepath{clip}%
\pgfsetbuttcap%
\pgfsetmiterjoin%
\definecolor{currentfill}{rgb}{0.000000,0.000000,0.000000}%
\pgfsetfillcolor{currentfill}%
\pgfsetlinewidth{0.000000pt}%
\definecolor{currentstroke}{rgb}{0.000000,0.000000,0.000000}%
\pgfsetstrokecolor{currentstroke}%
\pgfsetstrokeopacity{0.000000}%
\pgfsetdash{}{0pt}%
\pgfpathmoveto{\pgfqpoint{3.968185in}{0.499444in}}%
\pgfpathlineto{\pgfqpoint{4.029571in}{0.499444in}}%
\pgfpathlineto{\pgfqpoint{4.029571in}{0.499444in}}%
\pgfpathlineto{\pgfqpoint{3.968185in}{0.499444in}}%
\pgfpathlineto{\pgfqpoint{3.968185in}{0.499444in}}%
\pgfpathclose%
\pgfusepath{fill}%
\end{pgfscope}%
\begin{pgfscope}%
\pgfpathrectangle{\pgfqpoint{0.553581in}{0.499444in}}{\pgfqpoint{3.875000in}{1.155000in}}%
\pgfusepath{clip}%
\pgfsetbuttcap%
\pgfsetmiterjoin%
\definecolor{currentfill}{rgb}{0.000000,0.000000,0.000000}%
\pgfsetfillcolor{currentfill}%
\pgfsetlinewidth{0.000000pt}%
\definecolor{currentstroke}{rgb}{0.000000,0.000000,0.000000}%
\pgfsetstrokecolor{currentstroke}%
\pgfsetstrokeopacity{0.000000}%
\pgfsetdash{}{0pt}%
\pgfpathmoveto{\pgfqpoint{4.121650in}{0.499444in}}%
\pgfpathlineto{\pgfqpoint{4.183036in}{0.499444in}}%
\pgfpathlineto{\pgfqpoint{4.183036in}{0.499444in}}%
\pgfpathlineto{\pgfqpoint{4.121650in}{0.499444in}}%
\pgfpathlineto{\pgfqpoint{4.121650in}{0.499444in}}%
\pgfpathclose%
\pgfusepath{fill}%
\end{pgfscope}%
\begin{pgfscope}%
\pgfpathrectangle{\pgfqpoint{0.553581in}{0.499444in}}{\pgfqpoint{3.875000in}{1.155000in}}%
\pgfusepath{clip}%
\pgfsetbuttcap%
\pgfsetmiterjoin%
\definecolor{currentfill}{rgb}{0.000000,0.000000,0.000000}%
\pgfsetfillcolor{currentfill}%
\pgfsetlinewidth{0.000000pt}%
\definecolor{currentstroke}{rgb}{0.000000,0.000000,0.000000}%
\pgfsetstrokecolor{currentstroke}%
\pgfsetstrokeopacity{0.000000}%
\pgfsetdash{}{0pt}%
\pgfpathmoveto{\pgfqpoint{4.275115in}{0.499444in}}%
\pgfpathlineto{\pgfqpoint{4.336501in}{0.499444in}}%
\pgfpathlineto{\pgfqpoint{4.336501in}{0.499444in}}%
\pgfpathlineto{\pgfqpoint{4.275115in}{0.499444in}}%
\pgfpathlineto{\pgfqpoint{4.275115in}{0.499444in}}%
\pgfpathclose%
\pgfusepath{fill}%
\end{pgfscope}%
\begin{pgfscope}%
\pgfsetbuttcap%
\pgfsetroundjoin%
\definecolor{currentfill}{rgb}{0.000000,0.000000,0.000000}%
\pgfsetfillcolor{currentfill}%
\pgfsetlinewidth{0.803000pt}%
\definecolor{currentstroke}{rgb}{0.000000,0.000000,0.000000}%
\pgfsetstrokecolor{currentstroke}%
\pgfsetdash{}{0pt}%
\pgfsys@defobject{currentmarker}{\pgfqpoint{0.000000in}{-0.048611in}}{\pgfqpoint{0.000000in}{0.000000in}}{%
\pgfpathmoveto{\pgfqpoint{0.000000in}{0.000000in}}%
\pgfpathlineto{\pgfqpoint{0.000000in}{-0.048611in}}%
\pgfusepath{stroke,fill}%
}%
\begin{pgfscope}%
\pgfsys@transformshift{0.591947in}{0.499444in}%
\pgfsys@useobject{currentmarker}{}%
\end{pgfscope}%
\end{pgfscope}%
\begin{pgfscope}%
\definecolor{textcolor}{rgb}{0.000000,0.000000,0.000000}%
\pgfsetstrokecolor{textcolor}%
\pgfsetfillcolor{textcolor}%
\pgftext[x=0.591947in,y=0.402222in,,top]{\color{textcolor}\rmfamily\fontsize{10.000000}{12.000000}\selectfont 0.0}%
\end{pgfscope}%
\begin{pgfscope}%
\pgfsetbuttcap%
\pgfsetroundjoin%
\definecolor{currentfill}{rgb}{0.000000,0.000000,0.000000}%
\pgfsetfillcolor{currentfill}%
\pgfsetlinewidth{0.803000pt}%
\definecolor{currentstroke}{rgb}{0.000000,0.000000,0.000000}%
\pgfsetstrokecolor{currentstroke}%
\pgfsetdash{}{0pt}%
\pgfsys@defobject{currentmarker}{\pgfqpoint{0.000000in}{-0.048611in}}{\pgfqpoint{0.000000in}{0.000000in}}{%
\pgfpathmoveto{\pgfqpoint{0.000000in}{0.000000in}}%
\pgfpathlineto{\pgfqpoint{0.000000in}{-0.048611in}}%
\pgfusepath{stroke,fill}%
}%
\begin{pgfscope}%
\pgfsys@transformshift{0.975610in}{0.499444in}%
\pgfsys@useobject{currentmarker}{}%
\end{pgfscope}%
\end{pgfscope}%
\begin{pgfscope}%
\definecolor{textcolor}{rgb}{0.000000,0.000000,0.000000}%
\pgfsetstrokecolor{textcolor}%
\pgfsetfillcolor{textcolor}%
\pgftext[x=0.975610in,y=0.402222in,,top]{\color{textcolor}\rmfamily\fontsize{10.000000}{12.000000}\selectfont 0.1}%
\end{pgfscope}%
\begin{pgfscope}%
\pgfsetbuttcap%
\pgfsetroundjoin%
\definecolor{currentfill}{rgb}{0.000000,0.000000,0.000000}%
\pgfsetfillcolor{currentfill}%
\pgfsetlinewidth{0.803000pt}%
\definecolor{currentstroke}{rgb}{0.000000,0.000000,0.000000}%
\pgfsetstrokecolor{currentstroke}%
\pgfsetdash{}{0pt}%
\pgfsys@defobject{currentmarker}{\pgfqpoint{0.000000in}{-0.048611in}}{\pgfqpoint{0.000000in}{0.000000in}}{%
\pgfpathmoveto{\pgfqpoint{0.000000in}{0.000000in}}%
\pgfpathlineto{\pgfqpoint{0.000000in}{-0.048611in}}%
\pgfusepath{stroke,fill}%
}%
\begin{pgfscope}%
\pgfsys@transformshift{1.359274in}{0.499444in}%
\pgfsys@useobject{currentmarker}{}%
\end{pgfscope}%
\end{pgfscope}%
\begin{pgfscope}%
\definecolor{textcolor}{rgb}{0.000000,0.000000,0.000000}%
\pgfsetstrokecolor{textcolor}%
\pgfsetfillcolor{textcolor}%
\pgftext[x=1.359274in,y=0.402222in,,top]{\color{textcolor}\rmfamily\fontsize{10.000000}{12.000000}\selectfont 0.2}%
\end{pgfscope}%
\begin{pgfscope}%
\pgfsetbuttcap%
\pgfsetroundjoin%
\definecolor{currentfill}{rgb}{0.000000,0.000000,0.000000}%
\pgfsetfillcolor{currentfill}%
\pgfsetlinewidth{0.803000pt}%
\definecolor{currentstroke}{rgb}{0.000000,0.000000,0.000000}%
\pgfsetstrokecolor{currentstroke}%
\pgfsetdash{}{0pt}%
\pgfsys@defobject{currentmarker}{\pgfqpoint{0.000000in}{-0.048611in}}{\pgfqpoint{0.000000in}{0.000000in}}{%
\pgfpathmoveto{\pgfqpoint{0.000000in}{0.000000in}}%
\pgfpathlineto{\pgfqpoint{0.000000in}{-0.048611in}}%
\pgfusepath{stroke,fill}%
}%
\begin{pgfscope}%
\pgfsys@transformshift{1.742937in}{0.499444in}%
\pgfsys@useobject{currentmarker}{}%
\end{pgfscope}%
\end{pgfscope}%
\begin{pgfscope}%
\definecolor{textcolor}{rgb}{0.000000,0.000000,0.000000}%
\pgfsetstrokecolor{textcolor}%
\pgfsetfillcolor{textcolor}%
\pgftext[x=1.742937in,y=0.402222in,,top]{\color{textcolor}\rmfamily\fontsize{10.000000}{12.000000}\selectfont 0.3}%
\end{pgfscope}%
\begin{pgfscope}%
\pgfsetbuttcap%
\pgfsetroundjoin%
\definecolor{currentfill}{rgb}{0.000000,0.000000,0.000000}%
\pgfsetfillcolor{currentfill}%
\pgfsetlinewidth{0.803000pt}%
\definecolor{currentstroke}{rgb}{0.000000,0.000000,0.000000}%
\pgfsetstrokecolor{currentstroke}%
\pgfsetdash{}{0pt}%
\pgfsys@defobject{currentmarker}{\pgfqpoint{0.000000in}{-0.048611in}}{\pgfqpoint{0.000000in}{0.000000in}}{%
\pgfpathmoveto{\pgfqpoint{0.000000in}{0.000000in}}%
\pgfpathlineto{\pgfqpoint{0.000000in}{-0.048611in}}%
\pgfusepath{stroke,fill}%
}%
\begin{pgfscope}%
\pgfsys@transformshift{2.126600in}{0.499444in}%
\pgfsys@useobject{currentmarker}{}%
\end{pgfscope}%
\end{pgfscope}%
\begin{pgfscope}%
\definecolor{textcolor}{rgb}{0.000000,0.000000,0.000000}%
\pgfsetstrokecolor{textcolor}%
\pgfsetfillcolor{textcolor}%
\pgftext[x=2.126600in,y=0.402222in,,top]{\color{textcolor}\rmfamily\fontsize{10.000000}{12.000000}\selectfont 0.4}%
\end{pgfscope}%
\begin{pgfscope}%
\pgfsetbuttcap%
\pgfsetroundjoin%
\definecolor{currentfill}{rgb}{0.000000,0.000000,0.000000}%
\pgfsetfillcolor{currentfill}%
\pgfsetlinewidth{0.803000pt}%
\definecolor{currentstroke}{rgb}{0.000000,0.000000,0.000000}%
\pgfsetstrokecolor{currentstroke}%
\pgfsetdash{}{0pt}%
\pgfsys@defobject{currentmarker}{\pgfqpoint{0.000000in}{-0.048611in}}{\pgfqpoint{0.000000in}{0.000000in}}{%
\pgfpathmoveto{\pgfqpoint{0.000000in}{0.000000in}}%
\pgfpathlineto{\pgfqpoint{0.000000in}{-0.048611in}}%
\pgfusepath{stroke,fill}%
}%
\begin{pgfscope}%
\pgfsys@transformshift{2.510264in}{0.499444in}%
\pgfsys@useobject{currentmarker}{}%
\end{pgfscope}%
\end{pgfscope}%
\begin{pgfscope}%
\definecolor{textcolor}{rgb}{0.000000,0.000000,0.000000}%
\pgfsetstrokecolor{textcolor}%
\pgfsetfillcolor{textcolor}%
\pgftext[x=2.510264in,y=0.402222in,,top]{\color{textcolor}\rmfamily\fontsize{10.000000}{12.000000}\selectfont 0.5}%
\end{pgfscope}%
\begin{pgfscope}%
\pgfsetbuttcap%
\pgfsetroundjoin%
\definecolor{currentfill}{rgb}{0.000000,0.000000,0.000000}%
\pgfsetfillcolor{currentfill}%
\pgfsetlinewidth{0.803000pt}%
\definecolor{currentstroke}{rgb}{0.000000,0.000000,0.000000}%
\pgfsetstrokecolor{currentstroke}%
\pgfsetdash{}{0pt}%
\pgfsys@defobject{currentmarker}{\pgfqpoint{0.000000in}{-0.048611in}}{\pgfqpoint{0.000000in}{0.000000in}}{%
\pgfpathmoveto{\pgfqpoint{0.000000in}{0.000000in}}%
\pgfpathlineto{\pgfqpoint{0.000000in}{-0.048611in}}%
\pgfusepath{stroke,fill}%
}%
\begin{pgfscope}%
\pgfsys@transformshift{2.893927in}{0.499444in}%
\pgfsys@useobject{currentmarker}{}%
\end{pgfscope}%
\end{pgfscope}%
\begin{pgfscope}%
\definecolor{textcolor}{rgb}{0.000000,0.000000,0.000000}%
\pgfsetstrokecolor{textcolor}%
\pgfsetfillcolor{textcolor}%
\pgftext[x=2.893927in,y=0.402222in,,top]{\color{textcolor}\rmfamily\fontsize{10.000000}{12.000000}\selectfont 0.6}%
\end{pgfscope}%
\begin{pgfscope}%
\pgfsetbuttcap%
\pgfsetroundjoin%
\definecolor{currentfill}{rgb}{0.000000,0.000000,0.000000}%
\pgfsetfillcolor{currentfill}%
\pgfsetlinewidth{0.803000pt}%
\definecolor{currentstroke}{rgb}{0.000000,0.000000,0.000000}%
\pgfsetstrokecolor{currentstroke}%
\pgfsetdash{}{0pt}%
\pgfsys@defobject{currentmarker}{\pgfqpoint{0.000000in}{-0.048611in}}{\pgfqpoint{0.000000in}{0.000000in}}{%
\pgfpathmoveto{\pgfqpoint{0.000000in}{0.000000in}}%
\pgfpathlineto{\pgfqpoint{0.000000in}{-0.048611in}}%
\pgfusepath{stroke,fill}%
}%
\begin{pgfscope}%
\pgfsys@transformshift{3.277591in}{0.499444in}%
\pgfsys@useobject{currentmarker}{}%
\end{pgfscope}%
\end{pgfscope}%
\begin{pgfscope}%
\definecolor{textcolor}{rgb}{0.000000,0.000000,0.000000}%
\pgfsetstrokecolor{textcolor}%
\pgfsetfillcolor{textcolor}%
\pgftext[x=3.277591in,y=0.402222in,,top]{\color{textcolor}\rmfamily\fontsize{10.000000}{12.000000}\selectfont 0.7}%
\end{pgfscope}%
\begin{pgfscope}%
\pgfsetbuttcap%
\pgfsetroundjoin%
\definecolor{currentfill}{rgb}{0.000000,0.000000,0.000000}%
\pgfsetfillcolor{currentfill}%
\pgfsetlinewidth{0.803000pt}%
\definecolor{currentstroke}{rgb}{0.000000,0.000000,0.000000}%
\pgfsetstrokecolor{currentstroke}%
\pgfsetdash{}{0pt}%
\pgfsys@defobject{currentmarker}{\pgfqpoint{0.000000in}{-0.048611in}}{\pgfqpoint{0.000000in}{0.000000in}}{%
\pgfpathmoveto{\pgfqpoint{0.000000in}{0.000000in}}%
\pgfpathlineto{\pgfqpoint{0.000000in}{-0.048611in}}%
\pgfusepath{stroke,fill}%
}%
\begin{pgfscope}%
\pgfsys@transformshift{3.661254in}{0.499444in}%
\pgfsys@useobject{currentmarker}{}%
\end{pgfscope}%
\end{pgfscope}%
\begin{pgfscope}%
\definecolor{textcolor}{rgb}{0.000000,0.000000,0.000000}%
\pgfsetstrokecolor{textcolor}%
\pgfsetfillcolor{textcolor}%
\pgftext[x=3.661254in,y=0.402222in,,top]{\color{textcolor}\rmfamily\fontsize{10.000000}{12.000000}\selectfont 0.8}%
\end{pgfscope}%
\begin{pgfscope}%
\pgfsetbuttcap%
\pgfsetroundjoin%
\definecolor{currentfill}{rgb}{0.000000,0.000000,0.000000}%
\pgfsetfillcolor{currentfill}%
\pgfsetlinewidth{0.803000pt}%
\definecolor{currentstroke}{rgb}{0.000000,0.000000,0.000000}%
\pgfsetstrokecolor{currentstroke}%
\pgfsetdash{}{0pt}%
\pgfsys@defobject{currentmarker}{\pgfqpoint{0.000000in}{-0.048611in}}{\pgfqpoint{0.000000in}{0.000000in}}{%
\pgfpathmoveto{\pgfqpoint{0.000000in}{0.000000in}}%
\pgfpathlineto{\pgfqpoint{0.000000in}{-0.048611in}}%
\pgfusepath{stroke,fill}%
}%
\begin{pgfscope}%
\pgfsys@transformshift{4.044917in}{0.499444in}%
\pgfsys@useobject{currentmarker}{}%
\end{pgfscope}%
\end{pgfscope}%
\begin{pgfscope}%
\definecolor{textcolor}{rgb}{0.000000,0.000000,0.000000}%
\pgfsetstrokecolor{textcolor}%
\pgfsetfillcolor{textcolor}%
\pgftext[x=4.044917in,y=0.402222in,,top]{\color{textcolor}\rmfamily\fontsize{10.000000}{12.000000}\selectfont 0.9}%
\end{pgfscope}%
\begin{pgfscope}%
\pgfsetbuttcap%
\pgfsetroundjoin%
\definecolor{currentfill}{rgb}{0.000000,0.000000,0.000000}%
\pgfsetfillcolor{currentfill}%
\pgfsetlinewidth{0.803000pt}%
\definecolor{currentstroke}{rgb}{0.000000,0.000000,0.000000}%
\pgfsetstrokecolor{currentstroke}%
\pgfsetdash{}{0pt}%
\pgfsys@defobject{currentmarker}{\pgfqpoint{0.000000in}{-0.048611in}}{\pgfqpoint{0.000000in}{0.000000in}}{%
\pgfpathmoveto{\pgfqpoint{0.000000in}{0.000000in}}%
\pgfpathlineto{\pgfqpoint{0.000000in}{-0.048611in}}%
\pgfusepath{stroke,fill}%
}%
\begin{pgfscope}%
\pgfsys@transformshift{4.428581in}{0.499444in}%
\pgfsys@useobject{currentmarker}{}%
\end{pgfscope}%
\end{pgfscope}%
\begin{pgfscope}%
\definecolor{textcolor}{rgb}{0.000000,0.000000,0.000000}%
\pgfsetstrokecolor{textcolor}%
\pgfsetfillcolor{textcolor}%
\pgftext[x=4.428581in,y=0.402222in,,top]{\color{textcolor}\rmfamily\fontsize{10.000000}{12.000000}\selectfont 1.0}%
\end{pgfscope}%
\begin{pgfscope}%
\definecolor{textcolor}{rgb}{0.000000,0.000000,0.000000}%
\pgfsetstrokecolor{textcolor}%
\pgfsetfillcolor{textcolor}%
\pgftext[x=2.491081in,y=0.223333in,,top]{\color{textcolor}\rmfamily\fontsize{10.000000}{12.000000}\selectfont \(\displaystyle p\)}%
\end{pgfscope}%
\begin{pgfscope}%
\pgfsetbuttcap%
\pgfsetroundjoin%
\definecolor{currentfill}{rgb}{0.000000,0.000000,0.000000}%
\pgfsetfillcolor{currentfill}%
\pgfsetlinewidth{0.803000pt}%
\definecolor{currentstroke}{rgb}{0.000000,0.000000,0.000000}%
\pgfsetstrokecolor{currentstroke}%
\pgfsetdash{}{0pt}%
\pgfsys@defobject{currentmarker}{\pgfqpoint{-0.048611in}{0.000000in}}{\pgfqpoint{-0.000000in}{0.000000in}}{%
\pgfpathmoveto{\pgfqpoint{-0.000000in}{0.000000in}}%
\pgfpathlineto{\pgfqpoint{-0.048611in}{0.000000in}}%
\pgfusepath{stroke,fill}%
}%
\begin{pgfscope}%
\pgfsys@transformshift{0.553581in}{0.499444in}%
\pgfsys@useobject{currentmarker}{}%
\end{pgfscope}%
\end{pgfscope}%
\begin{pgfscope}%
\definecolor{textcolor}{rgb}{0.000000,0.000000,0.000000}%
\pgfsetstrokecolor{textcolor}%
\pgfsetfillcolor{textcolor}%
\pgftext[x=0.278889in, y=0.451250in, left, base]{\color{textcolor}\rmfamily\fontsize{10.000000}{12.000000}\selectfont \(\displaystyle {0.0}\)}%
\end{pgfscope}%
\begin{pgfscope}%
\pgfsetbuttcap%
\pgfsetroundjoin%
\definecolor{currentfill}{rgb}{0.000000,0.000000,0.000000}%
\pgfsetfillcolor{currentfill}%
\pgfsetlinewidth{0.803000pt}%
\definecolor{currentstroke}{rgb}{0.000000,0.000000,0.000000}%
\pgfsetstrokecolor{currentstroke}%
\pgfsetdash{}{0pt}%
\pgfsys@defobject{currentmarker}{\pgfqpoint{-0.048611in}{0.000000in}}{\pgfqpoint{-0.000000in}{0.000000in}}{%
\pgfpathmoveto{\pgfqpoint{-0.000000in}{0.000000in}}%
\pgfpathlineto{\pgfqpoint{-0.048611in}{0.000000in}}%
\pgfusepath{stroke,fill}%
}%
\begin{pgfscope}%
\pgfsys@transformshift{0.553581in}{0.800963in}%
\pgfsys@useobject{currentmarker}{}%
\end{pgfscope}%
\end{pgfscope}%
\begin{pgfscope}%
\definecolor{textcolor}{rgb}{0.000000,0.000000,0.000000}%
\pgfsetstrokecolor{textcolor}%
\pgfsetfillcolor{textcolor}%
\pgftext[x=0.278889in, y=0.752769in, left, base]{\color{textcolor}\rmfamily\fontsize{10.000000}{12.000000}\selectfont \(\displaystyle {2.5}\)}%
\end{pgfscope}%
\begin{pgfscope}%
\pgfsetbuttcap%
\pgfsetroundjoin%
\definecolor{currentfill}{rgb}{0.000000,0.000000,0.000000}%
\pgfsetfillcolor{currentfill}%
\pgfsetlinewidth{0.803000pt}%
\definecolor{currentstroke}{rgb}{0.000000,0.000000,0.000000}%
\pgfsetstrokecolor{currentstroke}%
\pgfsetdash{}{0pt}%
\pgfsys@defobject{currentmarker}{\pgfqpoint{-0.048611in}{0.000000in}}{\pgfqpoint{-0.000000in}{0.000000in}}{%
\pgfpathmoveto{\pgfqpoint{-0.000000in}{0.000000in}}%
\pgfpathlineto{\pgfqpoint{-0.048611in}{0.000000in}}%
\pgfusepath{stroke,fill}%
}%
\begin{pgfscope}%
\pgfsys@transformshift{0.553581in}{1.102483in}%
\pgfsys@useobject{currentmarker}{}%
\end{pgfscope}%
\end{pgfscope}%
\begin{pgfscope}%
\definecolor{textcolor}{rgb}{0.000000,0.000000,0.000000}%
\pgfsetstrokecolor{textcolor}%
\pgfsetfillcolor{textcolor}%
\pgftext[x=0.278889in, y=1.054288in, left, base]{\color{textcolor}\rmfamily\fontsize{10.000000}{12.000000}\selectfont \(\displaystyle {5.0}\)}%
\end{pgfscope}%
\begin{pgfscope}%
\pgfsetbuttcap%
\pgfsetroundjoin%
\definecolor{currentfill}{rgb}{0.000000,0.000000,0.000000}%
\pgfsetfillcolor{currentfill}%
\pgfsetlinewidth{0.803000pt}%
\definecolor{currentstroke}{rgb}{0.000000,0.000000,0.000000}%
\pgfsetstrokecolor{currentstroke}%
\pgfsetdash{}{0pt}%
\pgfsys@defobject{currentmarker}{\pgfqpoint{-0.048611in}{0.000000in}}{\pgfqpoint{-0.000000in}{0.000000in}}{%
\pgfpathmoveto{\pgfqpoint{-0.000000in}{0.000000in}}%
\pgfpathlineto{\pgfqpoint{-0.048611in}{0.000000in}}%
\pgfusepath{stroke,fill}%
}%
\begin{pgfscope}%
\pgfsys@transformshift{0.553581in}{1.404002in}%
\pgfsys@useobject{currentmarker}{}%
\end{pgfscope}%
\end{pgfscope}%
\begin{pgfscope}%
\definecolor{textcolor}{rgb}{0.000000,0.000000,0.000000}%
\pgfsetstrokecolor{textcolor}%
\pgfsetfillcolor{textcolor}%
\pgftext[x=0.278889in, y=1.355807in, left, base]{\color{textcolor}\rmfamily\fontsize{10.000000}{12.000000}\selectfont \(\displaystyle {7.5}\)}%
\end{pgfscope}%
\begin{pgfscope}%
\definecolor{textcolor}{rgb}{0.000000,0.000000,0.000000}%
\pgfsetstrokecolor{textcolor}%
\pgfsetfillcolor{textcolor}%
\pgftext[x=0.223333in,y=1.076944in,,bottom,rotate=90.000000]{\color{textcolor}\rmfamily\fontsize{10.000000}{12.000000}\selectfont Percent of Data Set}%
\end{pgfscope}%
\begin{pgfscope}%
\pgfsetrectcap%
\pgfsetmiterjoin%
\pgfsetlinewidth{0.803000pt}%
\definecolor{currentstroke}{rgb}{0.000000,0.000000,0.000000}%
\pgfsetstrokecolor{currentstroke}%
\pgfsetdash{}{0pt}%
\pgfpathmoveto{\pgfqpoint{0.553581in}{0.499444in}}%
\pgfpathlineto{\pgfqpoint{0.553581in}{1.654444in}}%
\pgfusepath{stroke}%
\end{pgfscope}%
\begin{pgfscope}%
\pgfsetrectcap%
\pgfsetmiterjoin%
\pgfsetlinewidth{0.803000pt}%
\definecolor{currentstroke}{rgb}{0.000000,0.000000,0.000000}%
\pgfsetstrokecolor{currentstroke}%
\pgfsetdash{}{0pt}%
\pgfpathmoveto{\pgfqpoint{4.428581in}{0.499444in}}%
\pgfpathlineto{\pgfqpoint{4.428581in}{1.654444in}}%
\pgfusepath{stroke}%
\end{pgfscope}%
\begin{pgfscope}%
\pgfsetrectcap%
\pgfsetmiterjoin%
\pgfsetlinewidth{0.803000pt}%
\definecolor{currentstroke}{rgb}{0.000000,0.000000,0.000000}%
\pgfsetstrokecolor{currentstroke}%
\pgfsetdash{}{0pt}%
\pgfpathmoveto{\pgfqpoint{0.553581in}{0.499444in}}%
\pgfpathlineto{\pgfqpoint{4.428581in}{0.499444in}}%
\pgfusepath{stroke}%
\end{pgfscope}%
\begin{pgfscope}%
\pgfsetrectcap%
\pgfsetmiterjoin%
\pgfsetlinewidth{0.803000pt}%
\definecolor{currentstroke}{rgb}{0.000000,0.000000,0.000000}%
\pgfsetstrokecolor{currentstroke}%
\pgfsetdash{}{0pt}%
\pgfpathmoveto{\pgfqpoint{0.553581in}{1.654444in}}%
\pgfpathlineto{\pgfqpoint{4.428581in}{1.654444in}}%
\pgfusepath{stroke}%
\end{pgfscope}%
\begin{pgfscope}%
\pgfsetbuttcap%
\pgfsetmiterjoin%
\definecolor{currentfill}{rgb}{1.000000,1.000000,1.000000}%
\pgfsetfillcolor{currentfill}%
\pgfsetfillopacity{0.800000}%
\pgfsetlinewidth{1.003750pt}%
\definecolor{currentstroke}{rgb}{0.800000,0.800000,0.800000}%
\pgfsetstrokecolor{currentstroke}%
\pgfsetstrokeopacity{0.800000}%
\pgfsetdash{}{0pt}%
\pgfpathmoveto{\pgfqpoint{3.651636in}{1.154445in}}%
\pgfpathlineto{\pgfqpoint{4.331358in}{1.154445in}}%
\pgfpathquadraticcurveto{\pgfqpoint{4.359136in}{1.154445in}}{\pgfqpoint{4.359136in}{1.182222in}}%
\pgfpathlineto{\pgfqpoint{4.359136in}{1.557222in}}%
\pgfpathquadraticcurveto{\pgfqpoint{4.359136in}{1.585000in}}{\pgfqpoint{4.331358in}{1.585000in}}%
\pgfpathlineto{\pgfqpoint{3.651636in}{1.585000in}}%
\pgfpathquadraticcurveto{\pgfqpoint{3.623858in}{1.585000in}}{\pgfqpoint{3.623858in}{1.557222in}}%
\pgfpathlineto{\pgfqpoint{3.623858in}{1.182222in}}%
\pgfpathquadraticcurveto{\pgfqpoint{3.623858in}{1.154445in}}{\pgfqpoint{3.651636in}{1.154445in}}%
\pgfpathlineto{\pgfqpoint{3.651636in}{1.154445in}}%
\pgfpathclose%
\pgfusepath{stroke,fill}%
\end{pgfscope}%
\begin{pgfscope}%
\pgfsetbuttcap%
\pgfsetmiterjoin%
\pgfsetlinewidth{1.003750pt}%
\definecolor{currentstroke}{rgb}{0.000000,0.000000,0.000000}%
\pgfsetstrokecolor{currentstroke}%
\pgfsetdash{}{0pt}%
\pgfpathmoveto{\pgfqpoint{3.679414in}{1.432222in}}%
\pgfpathlineto{\pgfqpoint{3.957192in}{1.432222in}}%
\pgfpathlineto{\pgfqpoint{3.957192in}{1.529444in}}%
\pgfpathlineto{\pgfqpoint{3.679414in}{1.529444in}}%
\pgfpathlineto{\pgfqpoint{3.679414in}{1.432222in}}%
\pgfpathclose%
\pgfusepath{stroke}%
\end{pgfscope}%
\begin{pgfscope}%
\definecolor{textcolor}{rgb}{0.000000,0.000000,0.000000}%
\pgfsetstrokecolor{textcolor}%
\pgfsetfillcolor{textcolor}%
\pgftext[x=4.068303in,y=1.432222in,left,base]{\color{textcolor}\rmfamily\fontsize{10.000000}{12.000000}\selectfont Neg}%
\end{pgfscope}%
\begin{pgfscope}%
\pgfsetbuttcap%
\pgfsetmiterjoin%
\definecolor{currentfill}{rgb}{0.000000,0.000000,0.000000}%
\pgfsetfillcolor{currentfill}%
\pgfsetlinewidth{0.000000pt}%
\definecolor{currentstroke}{rgb}{0.000000,0.000000,0.000000}%
\pgfsetstrokecolor{currentstroke}%
\pgfsetstrokeopacity{0.000000}%
\pgfsetdash{}{0pt}%
\pgfpathmoveto{\pgfqpoint{3.679414in}{1.236944in}}%
\pgfpathlineto{\pgfqpoint{3.957192in}{1.236944in}}%
\pgfpathlineto{\pgfqpoint{3.957192in}{1.334167in}}%
\pgfpathlineto{\pgfqpoint{3.679414in}{1.334167in}}%
\pgfpathlineto{\pgfqpoint{3.679414in}{1.236944in}}%
\pgfpathclose%
\pgfusepath{fill}%
\end{pgfscope}%
\begin{pgfscope}%
\definecolor{textcolor}{rgb}{0.000000,0.000000,0.000000}%
\pgfsetstrokecolor{textcolor}%
\pgfsetfillcolor{textcolor}%
\pgftext[x=4.068303in,y=1.236944in,left,base]{\color{textcolor}\rmfamily\fontsize{10.000000}{12.000000}\selectfont Pos}%
\end{pgfscope}%
\end{pgfpicture}%
\makeatother%
\endgroup%

	
\noindent\begin{tabular}{@{\hspace{-6pt}}p{2.3in} @{\hspace{-6pt}}p{2.0in} p{1.8in}}
	\vskip 0pt
	\hfil Raw Model Output
	
%	%% Creator: Matplotlib, PGF backend
%%
%% To include the figure in your LaTeX document, write
%%   \input{<filename>.pgf}
%%
%% Make sure the required packages are loaded in your preamble
%%   \usepackage{pgf}
%%
%% Also ensure that all the required font packages are loaded; for instance,
%% the lmodern package is sometimes necessary when using math font.
%%   \usepackage{lmodern}
%%
%% Figures using additional raster images can only be included by \input if
%% they are in the same directory as the main LaTeX file. For loading figures
%% from other directories you can use the `import` package
%%   \usepackage{import}
%%
%% and then include the figures with
%%   \import{<path to file>}{<filename>.pgf}
%%
%% Matplotlib used the following preamble
%%   
%%   \usepackage{fontspec}
%%   \makeatletter\@ifpackageloaded{underscore}{}{\usepackage[strings]{underscore}}\makeatother
%%
\begingroup%
\makeatletter%
\begin{pgfpicture}%
\pgfpathrectangle{\pgfpointorigin}{\pgfqpoint{2.253750in}{1.754444in}}%
\pgfusepath{use as bounding box, clip}%
\begin{pgfscope}%
\pgfsetbuttcap%
\pgfsetmiterjoin%
\definecolor{currentfill}{rgb}{1.000000,1.000000,1.000000}%
\pgfsetfillcolor{currentfill}%
\pgfsetlinewidth{0.000000pt}%
\definecolor{currentstroke}{rgb}{1.000000,1.000000,1.000000}%
\pgfsetstrokecolor{currentstroke}%
\pgfsetdash{}{0pt}%
\pgfpathmoveto{\pgfqpoint{0.000000in}{0.000000in}}%
\pgfpathlineto{\pgfqpoint{2.253750in}{0.000000in}}%
\pgfpathlineto{\pgfqpoint{2.253750in}{1.754444in}}%
\pgfpathlineto{\pgfqpoint{0.000000in}{1.754444in}}%
\pgfpathlineto{\pgfqpoint{0.000000in}{0.000000in}}%
\pgfpathclose%
\pgfusepath{fill}%
\end{pgfscope}%
\begin{pgfscope}%
\pgfsetbuttcap%
\pgfsetmiterjoin%
\definecolor{currentfill}{rgb}{1.000000,1.000000,1.000000}%
\pgfsetfillcolor{currentfill}%
\pgfsetlinewidth{0.000000pt}%
\definecolor{currentstroke}{rgb}{0.000000,0.000000,0.000000}%
\pgfsetstrokecolor{currentstroke}%
\pgfsetstrokeopacity{0.000000}%
\pgfsetdash{}{0pt}%
\pgfpathmoveto{\pgfqpoint{0.515000in}{0.499444in}}%
\pgfpathlineto{\pgfqpoint{2.065000in}{0.499444in}}%
\pgfpathlineto{\pgfqpoint{2.065000in}{1.654444in}}%
\pgfpathlineto{\pgfqpoint{0.515000in}{1.654444in}}%
\pgfpathlineto{\pgfqpoint{0.515000in}{0.499444in}}%
\pgfpathclose%
\pgfusepath{fill}%
\end{pgfscope}%
\begin{pgfscope}%
\pgfpathrectangle{\pgfqpoint{0.515000in}{0.499444in}}{\pgfqpoint{1.550000in}{1.155000in}}%
\pgfusepath{clip}%
\pgfsetbuttcap%
\pgfsetmiterjoin%
\pgfsetlinewidth{1.003750pt}%
\definecolor{currentstroke}{rgb}{0.000000,0.000000,0.000000}%
\pgfsetstrokecolor{currentstroke}%
\pgfsetdash{}{0pt}%
\pgfpathmoveto{\pgfqpoint{0.505000in}{0.499444in}}%
\pgfpathlineto{\pgfqpoint{0.552805in}{0.499444in}}%
\pgfpathlineto{\pgfqpoint{0.552805in}{1.599444in}}%
\pgfpathlineto{\pgfqpoint{0.505000in}{1.599444in}}%
\pgfusepath{stroke}%
\end{pgfscope}%
\begin{pgfscope}%
\pgfpathrectangle{\pgfqpoint{0.515000in}{0.499444in}}{\pgfqpoint{1.550000in}{1.155000in}}%
\pgfusepath{clip}%
\pgfsetbuttcap%
\pgfsetmiterjoin%
\pgfsetlinewidth{1.003750pt}%
\definecolor{currentstroke}{rgb}{0.000000,0.000000,0.000000}%
\pgfsetstrokecolor{currentstroke}%
\pgfsetdash{}{0pt}%
\pgfpathmoveto{\pgfqpoint{0.643537in}{0.499444in}}%
\pgfpathlineto{\pgfqpoint{0.704025in}{0.499444in}}%
\pgfpathlineto{\pgfqpoint{0.704025in}{1.586999in}}%
\pgfpathlineto{\pgfqpoint{0.643537in}{1.586999in}}%
\pgfpathlineto{\pgfqpoint{0.643537in}{0.499444in}}%
\pgfpathclose%
\pgfusepath{stroke}%
\end{pgfscope}%
\begin{pgfscope}%
\pgfpathrectangle{\pgfqpoint{0.515000in}{0.499444in}}{\pgfqpoint{1.550000in}{1.155000in}}%
\pgfusepath{clip}%
\pgfsetbuttcap%
\pgfsetmiterjoin%
\pgfsetlinewidth{1.003750pt}%
\definecolor{currentstroke}{rgb}{0.000000,0.000000,0.000000}%
\pgfsetstrokecolor{currentstroke}%
\pgfsetdash{}{0pt}%
\pgfpathmoveto{\pgfqpoint{0.794756in}{0.499444in}}%
\pgfpathlineto{\pgfqpoint{0.855244in}{0.499444in}}%
\pgfpathlineto{\pgfqpoint{0.855244in}{1.385585in}}%
\pgfpathlineto{\pgfqpoint{0.794756in}{1.385585in}}%
\pgfpathlineto{\pgfqpoint{0.794756in}{0.499444in}}%
\pgfpathclose%
\pgfusepath{stroke}%
\end{pgfscope}%
\begin{pgfscope}%
\pgfpathrectangle{\pgfqpoint{0.515000in}{0.499444in}}{\pgfqpoint{1.550000in}{1.155000in}}%
\pgfusepath{clip}%
\pgfsetbuttcap%
\pgfsetmiterjoin%
\pgfsetlinewidth{1.003750pt}%
\definecolor{currentstroke}{rgb}{0.000000,0.000000,0.000000}%
\pgfsetstrokecolor{currentstroke}%
\pgfsetdash{}{0pt}%
\pgfpathmoveto{\pgfqpoint{0.945976in}{0.499444in}}%
\pgfpathlineto{\pgfqpoint{1.006464in}{0.499444in}}%
\pgfpathlineto{\pgfqpoint{1.006464in}{1.228580in}}%
\pgfpathlineto{\pgfqpoint{0.945976in}{1.228580in}}%
\pgfpathlineto{\pgfqpoint{0.945976in}{0.499444in}}%
\pgfpathclose%
\pgfusepath{stroke}%
\end{pgfscope}%
\begin{pgfscope}%
\pgfpathrectangle{\pgfqpoint{0.515000in}{0.499444in}}{\pgfqpoint{1.550000in}{1.155000in}}%
\pgfusepath{clip}%
\pgfsetbuttcap%
\pgfsetmiterjoin%
\pgfsetlinewidth{1.003750pt}%
\definecolor{currentstroke}{rgb}{0.000000,0.000000,0.000000}%
\pgfsetstrokecolor{currentstroke}%
\pgfsetdash{}{0pt}%
\pgfpathmoveto{\pgfqpoint{1.097195in}{0.499444in}}%
\pgfpathlineto{\pgfqpoint{1.157683in}{0.499444in}}%
\pgfpathlineto{\pgfqpoint{1.157683in}{1.095872in}}%
\pgfpathlineto{\pgfqpoint{1.097195in}{1.095872in}}%
\pgfpathlineto{\pgfqpoint{1.097195in}{0.499444in}}%
\pgfpathclose%
\pgfusepath{stroke}%
\end{pgfscope}%
\begin{pgfscope}%
\pgfpathrectangle{\pgfqpoint{0.515000in}{0.499444in}}{\pgfqpoint{1.550000in}{1.155000in}}%
\pgfusepath{clip}%
\pgfsetbuttcap%
\pgfsetmiterjoin%
\pgfsetlinewidth{1.003750pt}%
\definecolor{currentstroke}{rgb}{0.000000,0.000000,0.000000}%
\pgfsetstrokecolor{currentstroke}%
\pgfsetdash{}{0pt}%
\pgfpathmoveto{\pgfqpoint{1.248415in}{0.499444in}}%
\pgfpathlineto{\pgfqpoint{1.308903in}{0.499444in}}%
\pgfpathlineto{\pgfqpoint{1.308903in}{0.960015in}}%
\pgfpathlineto{\pgfqpoint{1.248415in}{0.960015in}}%
\pgfpathlineto{\pgfqpoint{1.248415in}{0.499444in}}%
\pgfpathclose%
\pgfusepath{stroke}%
\end{pgfscope}%
\begin{pgfscope}%
\pgfpathrectangle{\pgfqpoint{0.515000in}{0.499444in}}{\pgfqpoint{1.550000in}{1.155000in}}%
\pgfusepath{clip}%
\pgfsetbuttcap%
\pgfsetmiterjoin%
\pgfsetlinewidth{1.003750pt}%
\definecolor{currentstroke}{rgb}{0.000000,0.000000,0.000000}%
\pgfsetstrokecolor{currentstroke}%
\pgfsetdash{}{0pt}%
\pgfpathmoveto{\pgfqpoint{1.399634in}{0.499444in}}%
\pgfpathlineto{\pgfqpoint{1.460122in}{0.499444in}}%
\pgfpathlineto{\pgfqpoint{1.460122in}{0.845678in}}%
\pgfpathlineto{\pgfqpoint{1.399634in}{0.845678in}}%
\pgfpathlineto{\pgfqpoint{1.399634in}{0.499444in}}%
\pgfpathclose%
\pgfusepath{stroke}%
\end{pgfscope}%
\begin{pgfscope}%
\pgfpathrectangle{\pgfqpoint{0.515000in}{0.499444in}}{\pgfqpoint{1.550000in}{1.155000in}}%
\pgfusepath{clip}%
\pgfsetbuttcap%
\pgfsetmiterjoin%
\pgfsetlinewidth{1.003750pt}%
\definecolor{currentstroke}{rgb}{0.000000,0.000000,0.000000}%
\pgfsetstrokecolor{currentstroke}%
\pgfsetdash{}{0pt}%
\pgfpathmoveto{\pgfqpoint{1.550854in}{0.499444in}}%
\pgfpathlineto{\pgfqpoint{1.611342in}{0.499444in}}%
\pgfpathlineto{\pgfqpoint{1.611342in}{0.726415in}}%
\pgfpathlineto{\pgfqpoint{1.550854in}{0.726415in}}%
\pgfpathlineto{\pgfqpoint{1.550854in}{0.499444in}}%
\pgfpathclose%
\pgfusepath{stroke}%
\end{pgfscope}%
\begin{pgfscope}%
\pgfpathrectangle{\pgfqpoint{0.515000in}{0.499444in}}{\pgfqpoint{1.550000in}{1.155000in}}%
\pgfusepath{clip}%
\pgfsetbuttcap%
\pgfsetmiterjoin%
\pgfsetlinewidth{1.003750pt}%
\definecolor{currentstroke}{rgb}{0.000000,0.000000,0.000000}%
\pgfsetstrokecolor{currentstroke}%
\pgfsetdash{}{0pt}%
\pgfpathmoveto{\pgfqpoint{1.702073in}{0.499444in}}%
\pgfpathlineto{\pgfqpoint{1.762561in}{0.499444in}}%
\pgfpathlineto{\pgfqpoint{1.762561in}{0.615411in}}%
\pgfpathlineto{\pgfqpoint{1.702073in}{0.615411in}}%
\pgfpathlineto{\pgfqpoint{1.702073in}{0.499444in}}%
\pgfpathclose%
\pgfusepath{stroke}%
\end{pgfscope}%
\begin{pgfscope}%
\pgfpathrectangle{\pgfqpoint{0.515000in}{0.499444in}}{\pgfqpoint{1.550000in}{1.155000in}}%
\pgfusepath{clip}%
\pgfsetbuttcap%
\pgfsetmiterjoin%
\pgfsetlinewidth{1.003750pt}%
\definecolor{currentstroke}{rgb}{0.000000,0.000000,0.000000}%
\pgfsetstrokecolor{currentstroke}%
\pgfsetdash{}{0pt}%
\pgfpathmoveto{\pgfqpoint{1.853293in}{0.499444in}}%
\pgfpathlineto{\pgfqpoint{1.913781in}{0.499444in}}%
\pgfpathlineto{\pgfqpoint{1.913781in}{0.534742in}}%
\pgfpathlineto{\pgfqpoint{1.853293in}{0.534742in}}%
\pgfpathlineto{\pgfqpoint{1.853293in}{0.499444in}}%
\pgfpathclose%
\pgfusepath{stroke}%
\end{pgfscope}%
\begin{pgfscope}%
\pgfpathrectangle{\pgfqpoint{0.515000in}{0.499444in}}{\pgfqpoint{1.550000in}{1.155000in}}%
\pgfusepath{clip}%
\pgfsetbuttcap%
\pgfsetmiterjoin%
\definecolor{currentfill}{rgb}{0.000000,0.000000,0.000000}%
\pgfsetfillcolor{currentfill}%
\pgfsetlinewidth{0.000000pt}%
\definecolor{currentstroke}{rgb}{0.000000,0.000000,0.000000}%
\pgfsetstrokecolor{currentstroke}%
\pgfsetstrokeopacity{0.000000}%
\pgfsetdash{}{0pt}%
\pgfpathmoveto{\pgfqpoint{0.552805in}{0.499444in}}%
\pgfpathlineto{\pgfqpoint{0.613293in}{0.499444in}}%
\pgfpathlineto{\pgfqpoint{0.613293in}{0.524964in}}%
\pgfpathlineto{\pgfqpoint{0.552805in}{0.524964in}}%
\pgfpathlineto{\pgfqpoint{0.552805in}{0.499444in}}%
\pgfpathclose%
\pgfusepath{fill}%
\end{pgfscope}%
\begin{pgfscope}%
\pgfpathrectangle{\pgfqpoint{0.515000in}{0.499444in}}{\pgfqpoint{1.550000in}{1.155000in}}%
\pgfusepath{clip}%
\pgfsetbuttcap%
\pgfsetmiterjoin%
\definecolor{currentfill}{rgb}{0.000000,0.000000,0.000000}%
\pgfsetfillcolor{currentfill}%
\pgfsetlinewidth{0.000000pt}%
\definecolor{currentstroke}{rgb}{0.000000,0.000000,0.000000}%
\pgfsetstrokecolor{currentstroke}%
\pgfsetstrokeopacity{0.000000}%
\pgfsetdash{}{0pt}%
\pgfpathmoveto{\pgfqpoint{0.704025in}{0.499444in}}%
\pgfpathlineto{\pgfqpoint{0.764512in}{0.499444in}}%
\pgfpathlineto{\pgfqpoint{0.764512in}{0.561002in}}%
\pgfpathlineto{\pgfqpoint{0.704025in}{0.561002in}}%
\pgfpathlineto{\pgfqpoint{0.704025in}{0.499444in}}%
\pgfpathclose%
\pgfusepath{fill}%
\end{pgfscope}%
\begin{pgfscope}%
\pgfpathrectangle{\pgfqpoint{0.515000in}{0.499444in}}{\pgfqpoint{1.550000in}{1.155000in}}%
\pgfusepath{clip}%
\pgfsetbuttcap%
\pgfsetmiterjoin%
\definecolor{currentfill}{rgb}{0.000000,0.000000,0.000000}%
\pgfsetfillcolor{currentfill}%
\pgfsetlinewidth{0.000000pt}%
\definecolor{currentstroke}{rgb}{0.000000,0.000000,0.000000}%
\pgfsetstrokecolor{currentstroke}%
\pgfsetstrokeopacity{0.000000}%
\pgfsetdash{}{0pt}%
\pgfpathmoveto{\pgfqpoint{0.855244in}{0.499444in}}%
\pgfpathlineto{\pgfqpoint{0.915732in}{0.499444in}}%
\pgfpathlineto{\pgfqpoint{0.915732in}{0.584003in}}%
\pgfpathlineto{\pgfqpoint{0.855244in}{0.584003in}}%
\pgfpathlineto{\pgfqpoint{0.855244in}{0.499444in}}%
\pgfpathclose%
\pgfusepath{fill}%
\end{pgfscope}%
\begin{pgfscope}%
\pgfpathrectangle{\pgfqpoint{0.515000in}{0.499444in}}{\pgfqpoint{1.550000in}{1.155000in}}%
\pgfusepath{clip}%
\pgfsetbuttcap%
\pgfsetmiterjoin%
\definecolor{currentfill}{rgb}{0.000000,0.000000,0.000000}%
\pgfsetfillcolor{currentfill}%
\pgfsetlinewidth{0.000000pt}%
\definecolor{currentstroke}{rgb}{0.000000,0.000000,0.000000}%
\pgfsetstrokecolor{currentstroke}%
\pgfsetstrokeopacity{0.000000}%
\pgfsetdash{}{0pt}%
\pgfpathmoveto{\pgfqpoint{1.006464in}{0.499444in}}%
\pgfpathlineto{\pgfqpoint{1.066951in}{0.499444in}}%
\pgfpathlineto{\pgfqpoint{1.066951in}{0.603262in}}%
\pgfpathlineto{\pgfqpoint{1.006464in}{0.603262in}}%
\pgfpathlineto{\pgfqpoint{1.006464in}{0.499444in}}%
\pgfpathclose%
\pgfusepath{fill}%
\end{pgfscope}%
\begin{pgfscope}%
\pgfpathrectangle{\pgfqpoint{0.515000in}{0.499444in}}{\pgfqpoint{1.550000in}{1.155000in}}%
\pgfusepath{clip}%
\pgfsetbuttcap%
\pgfsetmiterjoin%
\definecolor{currentfill}{rgb}{0.000000,0.000000,0.000000}%
\pgfsetfillcolor{currentfill}%
\pgfsetlinewidth{0.000000pt}%
\definecolor{currentstroke}{rgb}{0.000000,0.000000,0.000000}%
\pgfsetstrokecolor{currentstroke}%
\pgfsetstrokeopacity{0.000000}%
\pgfsetdash{}{0pt}%
\pgfpathmoveto{\pgfqpoint{1.157683in}{0.499444in}}%
\pgfpathlineto{\pgfqpoint{1.218171in}{0.499444in}}%
\pgfpathlineto{\pgfqpoint{1.218171in}{0.615559in}}%
\pgfpathlineto{\pgfqpoint{1.157683in}{0.615559in}}%
\pgfpathlineto{\pgfqpoint{1.157683in}{0.499444in}}%
\pgfpathclose%
\pgfusepath{fill}%
\end{pgfscope}%
\begin{pgfscope}%
\pgfpathrectangle{\pgfqpoint{0.515000in}{0.499444in}}{\pgfqpoint{1.550000in}{1.155000in}}%
\pgfusepath{clip}%
\pgfsetbuttcap%
\pgfsetmiterjoin%
\definecolor{currentfill}{rgb}{0.000000,0.000000,0.000000}%
\pgfsetfillcolor{currentfill}%
\pgfsetlinewidth{0.000000pt}%
\definecolor{currentstroke}{rgb}{0.000000,0.000000,0.000000}%
\pgfsetstrokecolor{currentstroke}%
\pgfsetstrokeopacity{0.000000}%
\pgfsetdash{}{0pt}%
\pgfpathmoveto{\pgfqpoint{1.308903in}{0.499444in}}%
\pgfpathlineto{\pgfqpoint{1.369391in}{0.499444in}}%
\pgfpathlineto{\pgfqpoint{1.369391in}{0.629856in}}%
\pgfpathlineto{\pgfqpoint{1.308903in}{0.629856in}}%
\pgfpathlineto{\pgfqpoint{1.308903in}{0.499444in}}%
\pgfpathclose%
\pgfusepath{fill}%
\end{pgfscope}%
\begin{pgfscope}%
\pgfpathrectangle{\pgfqpoint{0.515000in}{0.499444in}}{\pgfqpoint{1.550000in}{1.155000in}}%
\pgfusepath{clip}%
\pgfsetbuttcap%
\pgfsetmiterjoin%
\definecolor{currentfill}{rgb}{0.000000,0.000000,0.000000}%
\pgfsetfillcolor{currentfill}%
\pgfsetlinewidth{0.000000pt}%
\definecolor{currentstroke}{rgb}{0.000000,0.000000,0.000000}%
\pgfsetstrokecolor{currentstroke}%
\pgfsetstrokeopacity{0.000000}%
\pgfsetdash{}{0pt}%
\pgfpathmoveto{\pgfqpoint{1.460122in}{0.499444in}}%
\pgfpathlineto{\pgfqpoint{1.520610in}{0.499444in}}%
\pgfpathlineto{\pgfqpoint{1.520610in}{0.635634in}}%
\pgfpathlineto{\pgfqpoint{1.460122in}{0.635634in}}%
\pgfpathlineto{\pgfqpoint{1.460122in}{0.499444in}}%
\pgfpathclose%
\pgfusepath{fill}%
\end{pgfscope}%
\begin{pgfscope}%
\pgfpathrectangle{\pgfqpoint{0.515000in}{0.499444in}}{\pgfqpoint{1.550000in}{1.155000in}}%
\pgfusepath{clip}%
\pgfsetbuttcap%
\pgfsetmiterjoin%
\definecolor{currentfill}{rgb}{0.000000,0.000000,0.000000}%
\pgfsetfillcolor{currentfill}%
\pgfsetlinewidth{0.000000pt}%
\definecolor{currentstroke}{rgb}{0.000000,0.000000,0.000000}%
\pgfsetstrokecolor{currentstroke}%
\pgfsetstrokeopacity{0.000000}%
\pgfsetdash{}{0pt}%
\pgfpathmoveto{\pgfqpoint{1.611342in}{0.499444in}}%
\pgfpathlineto{\pgfqpoint{1.671830in}{0.499444in}}%
\pgfpathlineto{\pgfqpoint{1.671830in}{0.635745in}}%
\pgfpathlineto{\pgfqpoint{1.611342in}{0.635745in}}%
\pgfpathlineto{\pgfqpoint{1.611342in}{0.499444in}}%
\pgfpathclose%
\pgfusepath{fill}%
\end{pgfscope}%
\begin{pgfscope}%
\pgfpathrectangle{\pgfqpoint{0.515000in}{0.499444in}}{\pgfqpoint{1.550000in}{1.155000in}}%
\pgfusepath{clip}%
\pgfsetbuttcap%
\pgfsetmiterjoin%
\definecolor{currentfill}{rgb}{0.000000,0.000000,0.000000}%
\pgfsetfillcolor{currentfill}%
\pgfsetlinewidth{0.000000pt}%
\definecolor{currentstroke}{rgb}{0.000000,0.000000,0.000000}%
\pgfsetstrokecolor{currentstroke}%
\pgfsetstrokeopacity{0.000000}%
\pgfsetdash{}{0pt}%
\pgfpathmoveto{\pgfqpoint{1.762561in}{0.499444in}}%
\pgfpathlineto{\pgfqpoint{1.823049in}{0.499444in}}%
\pgfpathlineto{\pgfqpoint{1.823049in}{0.618856in}}%
\pgfpathlineto{\pgfqpoint{1.762561in}{0.618856in}}%
\pgfpathlineto{\pgfqpoint{1.762561in}{0.499444in}}%
\pgfpathclose%
\pgfusepath{fill}%
\end{pgfscope}%
\begin{pgfscope}%
\pgfpathrectangle{\pgfqpoint{0.515000in}{0.499444in}}{\pgfqpoint{1.550000in}{1.155000in}}%
\pgfusepath{clip}%
\pgfsetbuttcap%
\pgfsetmiterjoin%
\definecolor{currentfill}{rgb}{0.000000,0.000000,0.000000}%
\pgfsetfillcolor{currentfill}%
\pgfsetlinewidth{0.000000pt}%
\definecolor{currentstroke}{rgb}{0.000000,0.000000,0.000000}%
\pgfsetstrokecolor{currentstroke}%
\pgfsetstrokeopacity{0.000000}%
\pgfsetdash{}{0pt}%
\pgfpathmoveto{\pgfqpoint{1.913781in}{0.499444in}}%
\pgfpathlineto{\pgfqpoint{1.974269in}{0.499444in}}%
\pgfpathlineto{\pgfqpoint{1.974269in}{0.571558in}}%
\pgfpathlineto{\pgfqpoint{1.913781in}{0.571558in}}%
\pgfpathlineto{\pgfqpoint{1.913781in}{0.499444in}}%
\pgfpathclose%
\pgfusepath{fill}%
\end{pgfscope}%
\begin{pgfscope}%
\pgfsetbuttcap%
\pgfsetroundjoin%
\definecolor{currentfill}{rgb}{0.000000,0.000000,0.000000}%
\pgfsetfillcolor{currentfill}%
\pgfsetlinewidth{0.803000pt}%
\definecolor{currentstroke}{rgb}{0.000000,0.000000,0.000000}%
\pgfsetstrokecolor{currentstroke}%
\pgfsetdash{}{0pt}%
\pgfsys@defobject{currentmarker}{\pgfqpoint{0.000000in}{-0.048611in}}{\pgfqpoint{0.000000in}{0.000000in}}{%
\pgfpathmoveto{\pgfqpoint{0.000000in}{0.000000in}}%
\pgfpathlineto{\pgfqpoint{0.000000in}{-0.048611in}}%
\pgfusepath{stroke,fill}%
}%
\begin{pgfscope}%
\pgfsys@transformshift{0.552805in}{0.499444in}%
\pgfsys@useobject{currentmarker}{}%
\end{pgfscope}%
\end{pgfscope}%
\begin{pgfscope}%
\definecolor{textcolor}{rgb}{0.000000,0.000000,0.000000}%
\pgfsetstrokecolor{textcolor}%
\pgfsetfillcolor{textcolor}%
\pgftext[x=0.552805in,y=0.402222in,,top]{\color{textcolor}\rmfamily\fontsize{10.000000}{12.000000}\selectfont 0.0}%
\end{pgfscope}%
\begin{pgfscope}%
\pgfsetbuttcap%
\pgfsetroundjoin%
\definecolor{currentfill}{rgb}{0.000000,0.000000,0.000000}%
\pgfsetfillcolor{currentfill}%
\pgfsetlinewidth{0.803000pt}%
\definecolor{currentstroke}{rgb}{0.000000,0.000000,0.000000}%
\pgfsetstrokecolor{currentstroke}%
\pgfsetdash{}{0pt}%
\pgfsys@defobject{currentmarker}{\pgfqpoint{0.000000in}{-0.048611in}}{\pgfqpoint{0.000000in}{0.000000in}}{%
\pgfpathmoveto{\pgfqpoint{0.000000in}{0.000000in}}%
\pgfpathlineto{\pgfqpoint{0.000000in}{-0.048611in}}%
\pgfusepath{stroke,fill}%
}%
\begin{pgfscope}%
\pgfsys@transformshift{0.930854in}{0.499444in}%
\pgfsys@useobject{currentmarker}{}%
\end{pgfscope}%
\end{pgfscope}%
\begin{pgfscope}%
\definecolor{textcolor}{rgb}{0.000000,0.000000,0.000000}%
\pgfsetstrokecolor{textcolor}%
\pgfsetfillcolor{textcolor}%
\pgftext[x=0.930854in,y=0.402222in,,top]{\color{textcolor}\rmfamily\fontsize{10.000000}{12.000000}\selectfont 0.25}%
\end{pgfscope}%
\begin{pgfscope}%
\pgfsetbuttcap%
\pgfsetroundjoin%
\definecolor{currentfill}{rgb}{0.000000,0.000000,0.000000}%
\pgfsetfillcolor{currentfill}%
\pgfsetlinewidth{0.803000pt}%
\definecolor{currentstroke}{rgb}{0.000000,0.000000,0.000000}%
\pgfsetstrokecolor{currentstroke}%
\pgfsetdash{}{0pt}%
\pgfsys@defobject{currentmarker}{\pgfqpoint{0.000000in}{-0.048611in}}{\pgfqpoint{0.000000in}{0.000000in}}{%
\pgfpathmoveto{\pgfqpoint{0.000000in}{0.000000in}}%
\pgfpathlineto{\pgfqpoint{0.000000in}{-0.048611in}}%
\pgfusepath{stroke,fill}%
}%
\begin{pgfscope}%
\pgfsys@transformshift{1.308903in}{0.499444in}%
\pgfsys@useobject{currentmarker}{}%
\end{pgfscope}%
\end{pgfscope}%
\begin{pgfscope}%
\definecolor{textcolor}{rgb}{0.000000,0.000000,0.000000}%
\pgfsetstrokecolor{textcolor}%
\pgfsetfillcolor{textcolor}%
\pgftext[x=1.308903in,y=0.402222in,,top]{\color{textcolor}\rmfamily\fontsize{10.000000}{12.000000}\selectfont 0.5}%
\end{pgfscope}%
\begin{pgfscope}%
\pgfsetbuttcap%
\pgfsetroundjoin%
\definecolor{currentfill}{rgb}{0.000000,0.000000,0.000000}%
\pgfsetfillcolor{currentfill}%
\pgfsetlinewidth{0.803000pt}%
\definecolor{currentstroke}{rgb}{0.000000,0.000000,0.000000}%
\pgfsetstrokecolor{currentstroke}%
\pgfsetdash{}{0pt}%
\pgfsys@defobject{currentmarker}{\pgfqpoint{0.000000in}{-0.048611in}}{\pgfqpoint{0.000000in}{0.000000in}}{%
\pgfpathmoveto{\pgfqpoint{0.000000in}{0.000000in}}%
\pgfpathlineto{\pgfqpoint{0.000000in}{-0.048611in}}%
\pgfusepath{stroke,fill}%
}%
\begin{pgfscope}%
\pgfsys@transformshift{1.686951in}{0.499444in}%
\pgfsys@useobject{currentmarker}{}%
\end{pgfscope}%
\end{pgfscope}%
\begin{pgfscope}%
\definecolor{textcolor}{rgb}{0.000000,0.000000,0.000000}%
\pgfsetstrokecolor{textcolor}%
\pgfsetfillcolor{textcolor}%
\pgftext[x=1.686951in,y=0.402222in,,top]{\color{textcolor}\rmfamily\fontsize{10.000000}{12.000000}\selectfont 0.75}%
\end{pgfscope}%
\begin{pgfscope}%
\pgfsetbuttcap%
\pgfsetroundjoin%
\definecolor{currentfill}{rgb}{0.000000,0.000000,0.000000}%
\pgfsetfillcolor{currentfill}%
\pgfsetlinewidth{0.803000pt}%
\definecolor{currentstroke}{rgb}{0.000000,0.000000,0.000000}%
\pgfsetstrokecolor{currentstroke}%
\pgfsetdash{}{0pt}%
\pgfsys@defobject{currentmarker}{\pgfqpoint{0.000000in}{-0.048611in}}{\pgfqpoint{0.000000in}{0.000000in}}{%
\pgfpathmoveto{\pgfqpoint{0.000000in}{0.000000in}}%
\pgfpathlineto{\pgfqpoint{0.000000in}{-0.048611in}}%
\pgfusepath{stroke,fill}%
}%
\begin{pgfscope}%
\pgfsys@transformshift{2.065000in}{0.499444in}%
\pgfsys@useobject{currentmarker}{}%
\end{pgfscope}%
\end{pgfscope}%
\begin{pgfscope}%
\definecolor{textcolor}{rgb}{0.000000,0.000000,0.000000}%
\pgfsetstrokecolor{textcolor}%
\pgfsetfillcolor{textcolor}%
\pgftext[x=2.065000in,y=0.402222in,,top]{\color{textcolor}\rmfamily\fontsize{10.000000}{12.000000}\selectfont 1.0}%
\end{pgfscope}%
\begin{pgfscope}%
\definecolor{textcolor}{rgb}{0.000000,0.000000,0.000000}%
\pgfsetstrokecolor{textcolor}%
\pgfsetfillcolor{textcolor}%
\pgftext[x=1.290000in,y=0.223333in,,top]{\color{textcolor}\rmfamily\fontsize{10.000000}{12.000000}\selectfont \(\displaystyle p\)}%
\end{pgfscope}%
\begin{pgfscope}%
\pgfsetbuttcap%
\pgfsetroundjoin%
\definecolor{currentfill}{rgb}{0.000000,0.000000,0.000000}%
\pgfsetfillcolor{currentfill}%
\pgfsetlinewidth{0.803000pt}%
\definecolor{currentstroke}{rgb}{0.000000,0.000000,0.000000}%
\pgfsetstrokecolor{currentstroke}%
\pgfsetdash{}{0pt}%
\pgfsys@defobject{currentmarker}{\pgfqpoint{-0.048611in}{0.000000in}}{\pgfqpoint{-0.000000in}{0.000000in}}{%
\pgfpathmoveto{\pgfqpoint{-0.000000in}{0.000000in}}%
\pgfpathlineto{\pgfqpoint{-0.048611in}{0.000000in}}%
\pgfusepath{stroke,fill}%
}%
\begin{pgfscope}%
\pgfsys@transformshift{0.515000in}{0.499444in}%
\pgfsys@useobject{currentmarker}{}%
\end{pgfscope}%
\end{pgfscope}%
\begin{pgfscope}%
\definecolor{textcolor}{rgb}{0.000000,0.000000,0.000000}%
\pgfsetstrokecolor{textcolor}%
\pgfsetfillcolor{textcolor}%
\pgftext[x=0.348333in, y=0.451250in, left, base]{\color{textcolor}\rmfamily\fontsize{10.000000}{12.000000}\selectfont \(\displaystyle {0}\)}%
\end{pgfscope}%
\begin{pgfscope}%
\pgfsetbuttcap%
\pgfsetroundjoin%
\definecolor{currentfill}{rgb}{0.000000,0.000000,0.000000}%
\pgfsetfillcolor{currentfill}%
\pgfsetlinewidth{0.803000pt}%
\definecolor{currentstroke}{rgb}{0.000000,0.000000,0.000000}%
\pgfsetstrokecolor{currentstroke}%
\pgfsetdash{}{0pt}%
\pgfsys@defobject{currentmarker}{\pgfqpoint{-0.048611in}{0.000000in}}{\pgfqpoint{-0.000000in}{0.000000in}}{%
\pgfpathmoveto{\pgfqpoint{-0.000000in}{0.000000in}}%
\pgfpathlineto{\pgfqpoint{-0.048611in}{0.000000in}}%
\pgfusepath{stroke,fill}%
}%
\begin{pgfscope}%
\pgfsys@transformshift{0.515000in}{0.827959in}%
\pgfsys@useobject{currentmarker}{}%
\end{pgfscope}%
\end{pgfscope}%
\begin{pgfscope}%
\definecolor{textcolor}{rgb}{0.000000,0.000000,0.000000}%
\pgfsetstrokecolor{textcolor}%
\pgfsetfillcolor{textcolor}%
\pgftext[x=0.348333in, y=0.779764in, left, base]{\color{textcolor}\rmfamily\fontsize{10.000000}{12.000000}\selectfont \(\displaystyle {5}\)}%
\end{pgfscope}%
\begin{pgfscope}%
\pgfsetbuttcap%
\pgfsetroundjoin%
\definecolor{currentfill}{rgb}{0.000000,0.000000,0.000000}%
\pgfsetfillcolor{currentfill}%
\pgfsetlinewidth{0.803000pt}%
\definecolor{currentstroke}{rgb}{0.000000,0.000000,0.000000}%
\pgfsetstrokecolor{currentstroke}%
\pgfsetdash{}{0pt}%
\pgfsys@defobject{currentmarker}{\pgfqpoint{-0.048611in}{0.000000in}}{\pgfqpoint{-0.000000in}{0.000000in}}{%
\pgfpathmoveto{\pgfqpoint{-0.000000in}{0.000000in}}%
\pgfpathlineto{\pgfqpoint{-0.048611in}{0.000000in}}%
\pgfusepath{stroke,fill}%
}%
\begin{pgfscope}%
\pgfsys@transformshift{0.515000in}{1.156474in}%
\pgfsys@useobject{currentmarker}{}%
\end{pgfscope}%
\end{pgfscope}%
\begin{pgfscope}%
\definecolor{textcolor}{rgb}{0.000000,0.000000,0.000000}%
\pgfsetstrokecolor{textcolor}%
\pgfsetfillcolor{textcolor}%
\pgftext[x=0.278889in, y=1.108279in, left, base]{\color{textcolor}\rmfamily\fontsize{10.000000}{12.000000}\selectfont \(\displaystyle {10}\)}%
\end{pgfscope}%
\begin{pgfscope}%
\pgfsetbuttcap%
\pgfsetroundjoin%
\definecolor{currentfill}{rgb}{0.000000,0.000000,0.000000}%
\pgfsetfillcolor{currentfill}%
\pgfsetlinewidth{0.803000pt}%
\definecolor{currentstroke}{rgb}{0.000000,0.000000,0.000000}%
\pgfsetstrokecolor{currentstroke}%
\pgfsetdash{}{0pt}%
\pgfsys@defobject{currentmarker}{\pgfqpoint{-0.048611in}{0.000000in}}{\pgfqpoint{-0.000000in}{0.000000in}}{%
\pgfpathmoveto{\pgfqpoint{-0.000000in}{0.000000in}}%
\pgfpathlineto{\pgfqpoint{-0.048611in}{0.000000in}}%
\pgfusepath{stroke,fill}%
}%
\begin{pgfscope}%
\pgfsys@transformshift{0.515000in}{1.484988in}%
\pgfsys@useobject{currentmarker}{}%
\end{pgfscope}%
\end{pgfscope}%
\begin{pgfscope}%
\definecolor{textcolor}{rgb}{0.000000,0.000000,0.000000}%
\pgfsetstrokecolor{textcolor}%
\pgfsetfillcolor{textcolor}%
\pgftext[x=0.278889in, y=1.436794in, left, base]{\color{textcolor}\rmfamily\fontsize{10.000000}{12.000000}\selectfont \(\displaystyle {15}\)}%
\end{pgfscope}%
\begin{pgfscope}%
\definecolor{textcolor}{rgb}{0.000000,0.000000,0.000000}%
\pgfsetstrokecolor{textcolor}%
\pgfsetfillcolor{textcolor}%
\pgftext[x=0.223333in,y=1.076944in,,bottom,rotate=90.000000]{\color{textcolor}\rmfamily\fontsize{10.000000}{12.000000}\selectfont Percent of Data Set}%
\end{pgfscope}%
\begin{pgfscope}%
\pgfsetrectcap%
\pgfsetmiterjoin%
\pgfsetlinewidth{0.803000pt}%
\definecolor{currentstroke}{rgb}{0.000000,0.000000,0.000000}%
\pgfsetstrokecolor{currentstroke}%
\pgfsetdash{}{0pt}%
\pgfpathmoveto{\pgfqpoint{0.515000in}{0.499444in}}%
\pgfpathlineto{\pgfqpoint{0.515000in}{1.654444in}}%
\pgfusepath{stroke}%
\end{pgfscope}%
\begin{pgfscope}%
\pgfsetrectcap%
\pgfsetmiterjoin%
\pgfsetlinewidth{0.803000pt}%
\definecolor{currentstroke}{rgb}{0.000000,0.000000,0.000000}%
\pgfsetstrokecolor{currentstroke}%
\pgfsetdash{}{0pt}%
\pgfpathmoveto{\pgfqpoint{2.065000in}{0.499444in}}%
\pgfpathlineto{\pgfqpoint{2.065000in}{1.654444in}}%
\pgfusepath{stroke}%
\end{pgfscope}%
\begin{pgfscope}%
\pgfsetrectcap%
\pgfsetmiterjoin%
\pgfsetlinewidth{0.803000pt}%
\definecolor{currentstroke}{rgb}{0.000000,0.000000,0.000000}%
\pgfsetstrokecolor{currentstroke}%
\pgfsetdash{}{0pt}%
\pgfpathmoveto{\pgfqpoint{0.515000in}{0.499444in}}%
\pgfpathlineto{\pgfqpoint{2.065000in}{0.499444in}}%
\pgfusepath{stroke}%
\end{pgfscope}%
\begin{pgfscope}%
\pgfsetrectcap%
\pgfsetmiterjoin%
\pgfsetlinewidth{0.803000pt}%
\definecolor{currentstroke}{rgb}{0.000000,0.000000,0.000000}%
\pgfsetstrokecolor{currentstroke}%
\pgfsetdash{}{0pt}%
\pgfpathmoveto{\pgfqpoint{0.515000in}{1.654444in}}%
\pgfpathlineto{\pgfqpoint{2.065000in}{1.654444in}}%
\pgfusepath{stroke}%
\end{pgfscope}%
\begin{pgfscope}%
\pgfsetbuttcap%
\pgfsetmiterjoin%
\definecolor{currentfill}{rgb}{1.000000,1.000000,1.000000}%
\pgfsetfillcolor{currentfill}%
\pgfsetfillopacity{0.800000}%
\pgfsetlinewidth{1.003750pt}%
\definecolor{currentstroke}{rgb}{0.800000,0.800000,0.800000}%
\pgfsetstrokecolor{currentstroke}%
\pgfsetstrokeopacity{0.800000}%
\pgfsetdash{}{0pt}%
\pgfpathmoveto{\pgfqpoint{1.288056in}{1.154445in}}%
\pgfpathlineto{\pgfqpoint{1.967778in}{1.154445in}}%
\pgfpathquadraticcurveto{\pgfqpoint{1.995556in}{1.154445in}}{\pgfqpoint{1.995556in}{1.182222in}}%
\pgfpathlineto{\pgfqpoint{1.995556in}{1.557222in}}%
\pgfpathquadraticcurveto{\pgfqpoint{1.995556in}{1.585000in}}{\pgfqpoint{1.967778in}{1.585000in}}%
\pgfpathlineto{\pgfqpoint{1.288056in}{1.585000in}}%
\pgfpathquadraticcurveto{\pgfqpoint{1.260278in}{1.585000in}}{\pgfqpoint{1.260278in}{1.557222in}}%
\pgfpathlineto{\pgfqpoint{1.260278in}{1.182222in}}%
\pgfpathquadraticcurveto{\pgfqpoint{1.260278in}{1.154445in}}{\pgfqpoint{1.288056in}{1.154445in}}%
\pgfpathlineto{\pgfqpoint{1.288056in}{1.154445in}}%
\pgfpathclose%
\pgfusepath{stroke,fill}%
\end{pgfscope}%
\begin{pgfscope}%
\pgfsetbuttcap%
\pgfsetmiterjoin%
\pgfsetlinewidth{1.003750pt}%
\definecolor{currentstroke}{rgb}{0.000000,0.000000,0.000000}%
\pgfsetstrokecolor{currentstroke}%
\pgfsetdash{}{0pt}%
\pgfpathmoveto{\pgfqpoint{1.315834in}{1.432222in}}%
\pgfpathlineto{\pgfqpoint{1.593611in}{1.432222in}}%
\pgfpathlineto{\pgfqpoint{1.593611in}{1.529444in}}%
\pgfpathlineto{\pgfqpoint{1.315834in}{1.529444in}}%
\pgfpathlineto{\pgfqpoint{1.315834in}{1.432222in}}%
\pgfpathclose%
\pgfusepath{stroke}%
\end{pgfscope}%
\begin{pgfscope}%
\definecolor{textcolor}{rgb}{0.000000,0.000000,0.000000}%
\pgfsetstrokecolor{textcolor}%
\pgfsetfillcolor{textcolor}%
\pgftext[x=1.704722in,y=1.432222in,left,base]{\color{textcolor}\rmfamily\fontsize{10.000000}{12.000000}\selectfont Neg}%
\end{pgfscope}%
\begin{pgfscope}%
\pgfsetbuttcap%
\pgfsetmiterjoin%
\definecolor{currentfill}{rgb}{0.000000,0.000000,0.000000}%
\pgfsetfillcolor{currentfill}%
\pgfsetlinewidth{0.000000pt}%
\definecolor{currentstroke}{rgb}{0.000000,0.000000,0.000000}%
\pgfsetstrokecolor{currentstroke}%
\pgfsetstrokeopacity{0.000000}%
\pgfsetdash{}{0pt}%
\pgfpathmoveto{\pgfqpoint{1.315834in}{1.236944in}}%
\pgfpathlineto{\pgfqpoint{1.593611in}{1.236944in}}%
\pgfpathlineto{\pgfqpoint{1.593611in}{1.334167in}}%
\pgfpathlineto{\pgfqpoint{1.315834in}{1.334167in}}%
\pgfpathlineto{\pgfqpoint{1.315834in}{1.236944in}}%
\pgfpathclose%
\pgfusepath{fill}%
\end{pgfscope}%
\begin{pgfscope}%
\definecolor{textcolor}{rgb}{0.000000,0.000000,0.000000}%
\pgfsetstrokecolor{textcolor}%
\pgfsetfillcolor{textcolor}%
\pgftext[x=1.704722in,y=1.236944in,left,base]{\color{textcolor}\rmfamily\fontsize{10.000000}{12.000000}\selectfont Pos}%
\end{pgfscope}%
\end{pgfpicture}%
\makeatother%
\endgroup%
	
&
	\vskip 0pt
	\hfil ROC Curve
	
%	%% Creator: Matplotlib, PGF backend
%%
%% To include the figure in your LaTeX document, write
%%   \input{<filename>.pgf}
%%
%% Make sure the required packages are loaded in your preamble
%%   \usepackage{pgf}
%%
%% Also ensure that all the required font packages are loaded; for instance,
%% the lmodern package is sometimes necessary when using math font.
%%   \usepackage{lmodern}
%%
%% Figures using additional raster images can only be included by \input if
%% they are in the same directory as the main LaTeX file. For loading figures
%% from other directories you can use the `import` package
%%   \usepackage{import}
%%
%% and then include the figures with
%%   \import{<path to file>}{<filename>.pgf}
%%
%% Matplotlib used the following preamble
%%   
%%   \usepackage{fontspec}
%%   \makeatletter\@ifpackageloaded{underscore}{}{\usepackage[strings]{underscore}}\makeatother
%%
\begingroup%
\makeatletter%
\begin{pgfpicture}%
\pgfpathrectangle{\pgfpointorigin}{\pgfqpoint{2.221861in}{1.754444in}}%
\pgfusepath{use as bounding box, clip}%
\begin{pgfscope}%
\pgfsetbuttcap%
\pgfsetmiterjoin%
\definecolor{currentfill}{rgb}{1.000000,1.000000,1.000000}%
\pgfsetfillcolor{currentfill}%
\pgfsetlinewidth{0.000000pt}%
\definecolor{currentstroke}{rgb}{1.000000,1.000000,1.000000}%
\pgfsetstrokecolor{currentstroke}%
\pgfsetdash{}{0pt}%
\pgfpathmoveto{\pgfqpoint{0.000000in}{0.000000in}}%
\pgfpathlineto{\pgfqpoint{2.221861in}{0.000000in}}%
\pgfpathlineto{\pgfqpoint{2.221861in}{1.754444in}}%
\pgfpathlineto{\pgfqpoint{0.000000in}{1.754444in}}%
\pgfpathlineto{\pgfqpoint{0.000000in}{0.000000in}}%
\pgfpathclose%
\pgfusepath{fill}%
\end{pgfscope}%
\begin{pgfscope}%
\pgfsetbuttcap%
\pgfsetmiterjoin%
\definecolor{currentfill}{rgb}{1.000000,1.000000,1.000000}%
\pgfsetfillcolor{currentfill}%
\pgfsetlinewidth{0.000000pt}%
\definecolor{currentstroke}{rgb}{0.000000,0.000000,0.000000}%
\pgfsetstrokecolor{currentstroke}%
\pgfsetstrokeopacity{0.000000}%
\pgfsetdash{}{0pt}%
\pgfpathmoveto{\pgfqpoint{0.553581in}{0.499444in}}%
\pgfpathlineto{\pgfqpoint{2.103581in}{0.499444in}}%
\pgfpathlineto{\pgfqpoint{2.103581in}{1.654444in}}%
\pgfpathlineto{\pgfqpoint{0.553581in}{1.654444in}}%
\pgfpathlineto{\pgfqpoint{0.553581in}{0.499444in}}%
\pgfpathclose%
\pgfusepath{fill}%
\end{pgfscope}%
\begin{pgfscope}%
\pgfsetbuttcap%
\pgfsetroundjoin%
\definecolor{currentfill}{rgb}{0.000000,0.000000,0.000000}%
\pgfsetfillcolor{currentfill}%
\pgfsetlinewidth{0.803000pt}%
\definecolor{currentstroke}{rgb}{0.000000,0.000000,0.000000}%
\pgfsetstrokecolor{currentstroke}%
\pgfsetdash{}{0pt}%
\pgfsys@defobject{currentmarker}{\pgfqpoint{0.000000in}{-0.048611in}}{\pgfqpoint{0.000000in}{0.000000in}}{%
\pgfpathmoveto{\pgfqpoint{0.000000in}{0.000000in}}%
\pgfpathlineto{\pgfqpoint{0.000000in}{-0.048611in}}%
\pgfusepath{stroke,fill}%
}%
\begin{pgfscope}%
\pgfsys@transformshift{0.624035in}{0.499444in}%
\pgfsys@useobject{currentmarker}{}%
\end{pgfscope}%
\end{pgfscope}%
\begin{pgfscope}%
\definecolor{textcolor}{rgb}{0.000000,0.000000,0.000000}%
\pgfsetstrokecolor{textcolor}%
\pgfsetfillcolor{textcolor}%
\pgftext[x=0.624035in,y=0.402222in,,top]{\color{textcolor}\rmfamily\fontsize{10.000000}{12.000000}\selectfont \(\displaystyle {0.0}\)}%
\end{pgfscope}%
\begin{pgfscope}%
\pgfsetbuttcap%
\pgfsetroundjoin%
\definecolor{currentfill}{rgb}{0.000000,0.000000,0.000000}%
\pgfsetfillcolor{currentfill}%
\pgfsetlinewidth{0.803000pt}%
\definecolor{currentstroke}{rgb}{0.000000,0.000000,0.000000}%
\pgfsetstrokecolor{currentstroke}%
\pgfsetdash{}{0pt}%
\pgfsys@defobject{currentmarker}{\pgfqpoint{0.000000in}{-0.048611in}}{\pgfqpoint{0.000000in}{0.000000in}}{%
\pgfpathmoveto{\pgfqpoint{0.000000in}{0.000000in}}%
\pgfpathlineto{\pgfqpoint{0.000000in}{-0.048611in}}%
\pgfusepath{stroke,fill}%
}%
\begin{pgfscope}%
\pgfsys@transformshift{1.328581in}{0.499444in}%
\pgfsys@useobject{currentmarker}{}%
\end{pgfscope}%
\end{pgfscope}%
\begin{pgfscope}%
\definecolor{textcolor}{rgb}{0.000000,0.000000,0.000000}%
\pgfsetstrokecolor{textcolor}%
\pgfsetfillcolor{textcolor}%
\pgftext[x=1.328581in,y=0.402222in,,top]{\color{textcolor}\rmfamily\fontsize{10.000000}{12.000000}\selectfont \(\displaystyle {0.5}\)}%
\end{pgfscope}%
\begin{pgfscope}%
\pgfsetbuttcap%
\pgfsetroundjoin%
\definecolor{currentfill}{rgb}{0.000000,0.000000,0.000000}%
\pgfsetfillcolor{currentfill}%
\pgfsetlinewidth{0.803000pt}%
\definecolor{currentstroke}{rgb}{0.000000,0.000000,0.000000}%
\pgfsetstrokecolor{currentstroke}%
\pgfsetdash{}{0pt}%
\pgfsys@defobject{currentmarker}{\pgfqpoint{0.000000in}{-0.048611in}}{\pgfqpoint{0.000000in}{0.000000in}}{%
\pgfpathmoveto{\pgfqpoint{0.000000in}{0.000000in}}%
\pgfpathlineto{\pgfqpoint{0.000000in}{-0.048611in}}%
\pgfusepath{stroke,fill}%
}%
\begin{pgfscope}%
\pgfsys@transformshift{2.033126in}{0.499444in}%
\pgfsys@useobject{currentmarker}{}%
\end{pgfscope}%
\end{pgfscope}%
\begin{pgfscope}%
\definecolor{textcolor}{rgb}{0.000000,0.000000,0.000000}%
\pgfsetstrokecolor{textcolor}%
\pgfsetfillcolor{textcolor}%
\pgftext[x=2.033126in,y=0.402222in,,top]{\color{textcolor}\rmfamily\fontsize{10.000000}{12.000000}\selectfont \(\displaystyle {1.0}\)}%
\end{pgfscope}%
\begin{pgfscope}%
\definecolor{textcolor}{rgb}{0.000000,0.000000,0.000000}%
\pgfsetstrokecolor{textcolor}%
\pgfsetfillcolor{textcolor}%
\pgftext[x=1.328581in,y=0.223333in,,top]{\color{textcolor}\rmfamily\fontsize{10.000000}{12.000000}\selectfont False positive rate}%
\end{pgfscope}%
\begin{pgfscope}%
\pgfsetbuttcap%
\pgfsetroundjoin%
\definecolor{currentfill}{rgb}{0.000000,0.000000,0.000000}%
\pgfsetfillcolor{currentfill}%
\pgfsetlinewidth{0.803000pt}%
\definecolor{currentstroke}{rgb}{0.000000,0.000000,0.000000}%
\pgfsetstrokecolor{currentstroke}%
\pgfsetdash{}{0pt}%
\pgfsys@defobject{currentmarker}{\pgfqpoint{-0.048611in}{0.000000in}}{\pgfqpoint{-0.000000in}{0.000000in}}{%
\pgfpathmoveto{\pgfqpoint{-0.000000in}{0.000000in}}%
\pgfpathlineto{\pgfqpoint{-0.048611in}{0.000000in}}%
\pgfusepath{stroke,fill}%
}%
\begin{pgfscope}%
\pgfsys@transformshift{0.553581in}{0.551944in}%
\pgfsys@useobject{currentmarker}{}%
\end{pgfscope}%
\end{pgfscope}%
\begin{pgfscope}%
\definecolor{textcolor}{rgb}{0.000000,0.000000,0.000000}%
\pgfsetstrokecolor{textcolor}%
\pgfsetfillcolor{textcolor}%
\pgftext[x=0.278889in, y=0.503750in, left, base]{\color{textcolor}\rmfamily\fontsize{10.000000}{12.000000}\selectfont \(\displaystyle {0.0}\)}%
\end{pgfscope}%
\begin{pgfscope}%
\pgfsetbuttcap%
\pgfsetroundjoin%
\definecolor{currentfill}{rgb}{0.000000,0.000000,0.000000}%
\pgfsetfillcolor{currentfill}%
\pgfsetlinewidth{0.803000pt}%
\definecolor{currentstroke}{rgb}{0.000000,0.000000,0.000000}%
\pgfsetstrokecolor{currentstroke}%
\pgfsetdash{}{0pt}%
\pgfsys@defobject{currentmarker}{\pgfqpoint{-0.048611in}{0.000000in}}{\pgfqpoint{-0.000000in}{0.000000in}}{%
\pgfpathmoveto{\pgfqpoint{-0.000000in}{0.000000in}}%
\pgfpathlineto{\pgfqpoint{-0.048611in}{0.000000in}}%
\pgfusepath{stroke,fill}%
}%
\begin{pgfscope}%
\pgfsys@transformshift{0.553581in}{1.076944in}%
\pgfsys@useobject{currentmarker}{}%
\end{pgfscope}%
\end{pgfscope}%
\begin{pgfscope}%
\definecolor{textcolor}{rgb}{0.000000,0.000000,0.000000}%
\pgfsetstrokecolor{textcolor}%
\pgfsetfillcolor{textcolor}%
\pgftext[x=0.278889in, y=1.028750in, left, base]{\color{textcolor}\rmfamily\fontsize{10.000000}{12.000000}\selectfont \(\displaystyle {0.5}\)}%
\end{pgfscope}%
\begin{pgfscope}%
\pgfsetbuttcap%
\pgfsetroundjoin%
\definecolor{currentfill}{rgb}{0.000000,0.000000,0.000000}%
\pgfsetfillcolor{currentfill}%
\pgfsetlinewidth{0.803000pt}%
\definecolor{currentstroke}{rgb}{0.000000,0.000000,0.000000}%
\pgfsetstrokecolor{currentstroke}%
\pgfsetdash{}{0pt}%
\pgfsys@defobject{currentmarker}{\pgfqpoint{-0.048611in}{0.000000in}}{\pgfqpoint{-0.000000in}{0.000000in}}{%
\pgfpathmoveto{\pgfqpoint{-0.000000in}{0.000000in}}%
\pgfpathlineto{\pgfqpoint{-0.048611in}{0.000000in}}%
\pgfusepath{stroke,fill}%
}%
\begin{pgfscope}%
\pgfsys@transformshift{0.553581in}{1.601944in}%
\pgfsys@useobject{currentmarker}{}%
\end{pgfscope}%
\end{pgfscope}%
\begin{pgfscope}%
\definecolor{textcolor}{rgb}{0.000000,0.000000,0.000000}%
\pgfsetstrokecolor{textcolor}%
\pgfsetfillcolor{textcolor}%
\pgftext[x=0.278889in, y=1.553750in, left, base]{\color{textcolor}\rmfamily\fontsize{10.000000}{12.000000}\selectfont \(\displaystyle {1.0}\)}%
\end{pgfscope}%
\begin{pgfscope}%
\definecolor{textcolor}{rgb}{0.000000,0.000000,0.000000}%
\pgfsetstrokecolor{textcolor}%
\pgfsetfillcolor{textcolor}%
\pgftext[x=0.223333in,y=1.076944in,,bottom,rotate=90.000000]{\color{textcolor}\rmfamily\fontsize{10.000000}{12.000000}\selectfont True positive rate}%
\end{pgfscope}%
\begin{pgfscope}%
\pgfpathrectangle{\pgfqpoint{0.553581in}{0.499444in}}{\pgfqpoint{1.550000in}{1.155000in}}%
\pgfusepath{clip}%
\pgfsetbuttcap%
\pgfsetroundjoin%
\pgfsetlinewidth{1.505625pt}%
\definecolor{currentstroke}{rgb}{0.000000,0.000000,0.000000}%
\pgfsetstrokecolor{currentstroke}%
\pgfsetdash{{5.550000pt}{2.400000pt}}{0.000000pt}%
\pgfpathmoveto{\pgfqpoint{0.624035in}{0.551944in}}%
\pgfpathlineto{\pgfqpoint{2.033126in}{1.601944in}}%
\pgfusepath{stroke}%
\end{pgfscope}%
\begin{pgfscope}%
\pgfpathrectangle{\pgfqpoint{0.553581in}{0.499444in}}{\pgfqpoint{1.550000in}{1.155000in}}%
\pgfusepath{clip}%
\pgfsetrectcap%
\pgfsetroundjoin%
\pgfsetlinewidth{1.505625pt}%
\definecolor{currentstroke}{rgb}{0.000000,0.000000,0.000000}%
\pgfsetstrokecolor{currentstroke}%
\pgfsetdash{}{0pt}%
\pgfpathmoveto{\pgfqpoint{0.624035in}{0.551944in}}%
\pgfpathlineto{\pgfqpoint{0.625138in}{0.566577in}}%
\pgfpathlineto{\pgfqpoint{0.625316in}{0.567642in}}%
\pgfpathlineto{\pgfqpoint{0.626418in}{0.581053in}}%
\pgfpathlineto{\pgfqpoint{0.626512in}{0.582157in}}%
\pgfpathlineto{\pgfqpoint{0.627615in}{0.592097in}}%
\pgfpathlineto{\pgfqpoint{0.627764in}{0.593201in}}%
\pgfpathlineto{\pgfqpoint{0.628867in}{0.600419in}}%
\pgfpathlineto{\pgfqpoint{0.628998in}{0.601287in}}%
\pgfpathlineto{\pgfqpoint{0.630101in}{0.611345in}}%
\pgfpathlineto{\pgfqpoint{0.630278in}{0.612449in}}%
\pgfpathlineto{\pgfqpoint{0.631381in}{0.619588in}}%
\pgfpathlineto{\pgfqpoint{0.631540in}{0.620574in}}%
\pgfpathlineto{\pgfqpoint{0.632643in}{0.627082in}}%
\pgfpathlineto{\pgfqpoint{0.632848in}{0.628068in}}%
\pgfpathlineto{\pgfqpoint{0.633951in}{0.634734in}}%
\pgfpathlineto{\pgfqpoint{0.634101in}{0.635799in}}%
\pgfpathlineto{\pgfqpoint{0.635204in}{0.642780in}}%
\pgfpathlineto{\pgfqpoint{0.635484in}{0.643845in}}%
\pgfpathlineto{\pgfqpoint{0.635484in}{0.643885in}}%
\pgfpathlineto{\pgfqpoint{0.636587in}{0.651024in}}%
\pgfpathlineto{\pgfqpoint{0.636802in}{0.652128in}}%
\pgfpathlineto{\pgfqpoint{0.637905in}{0.658123in}}%
\pgfpathlineto{\pgfqpoint{0.638297in}{0.659228in}}%
\pgfpathlineto{\pgfqpoint{0.639400in}{0.664671in}}%
\pgfpathlineto{\pgfqpoint{0.639634in}{0.665775in}}%
\pgfpathlineto{\pgfqpoint{0.640736in}{0.670035in}}%
\pgfpathlineto{\pgfqpoint{0.640933in}{0.671140in}}%
\pgfpathlineto{\pgfqpoint{0.642007in}{0.675873in}}%
\pgfpathlineto{\pgfqpoint{0.642344in}{0.676819in}}%
\pgfpathlineto{\pgfqpoint{0.643437in}{0.682381in}}%
\pgfpathlineto{\pgfqpoint{0.643792in}{0.683485in}}%
\pgfpathlineto{\pgfqpoint{0.644895in}{0.688060in}}%
\pgfpathlineto{\pgfqpoint{0.645110in}{0.689086in}}%
\pgfpathlineto{\pgfqpoint{0.646213in}{0.694963in}}%
\pgfpathlineto{\pgfqpoint{0.646503in}{0.696067in}}%
\pgfpathlineto{\pgfqpoint{0.647606in}{0.701037in}}%
\pgfpathlineto{\pgfqpoint{0.647886in}{0.702102in}}%
\pgfpathlineto{\pgfqpoint{0.648989in}{0.706598in}}%
\pgfpathlineto{\pgfqpoint{0.649316in}{0.707663in}}%
\pgfpathlineto{\pgfqpoint{0.650419in}{0.713067in}}%
\pgfpathlineto{\pgfqpoint{0.650708in}{0.714171in}}%
\pgfpathlineto{\pgfqpoint{0.651811in}{0.717800in}}%
\pgfpathlineto{\pgfqpoint{0.652232in}{0.718904in}}%
\pgfpathlineto{\pgfqpoint{0.653335in}{0.724703in}}%
\pgfpathlineto{\pgfqpoint{0.653559in}{0.725728in}}%
\pgfpathlineto{\pgfqpoint{0.654652in}{0.729041in}}%
\pgfpathlineto{\pgfqpoint{0.654961in}{0.730146in}}%
\pgfpathlineto{\pgfqpoint{0.656045in}{0.733814in}}%
\pgfpathlineto{\pgfqpoint{0.656428in}{0.734879in}}%
\pgfpathlineto{\pgfqpoint{0.657522in}{0.738784in}}%
\pgfpathlineto{\pgfqpoint{0.657905in}{0.739848in}}%
\pgfpathlineto{\pgfqpoint{0.659008in}{0.744542in}}%
\pgfpathlineto{\pgfqpoint{0.659148in}{0.745528in}}%
\pgfpathlineto{\pgfqpoint{0.659148in}{0.745607in}}%
\pgfpathlineto{\pgfqpoint{0.660251in}{0.749354in}}%
\pgfpathlineto{\pgfqpoint{0.660438in}{0.750458in}}%
\pgfpathlineto{\pgfqpoint{0.661522in}{0.753614in}}%
\pgfpathlineto{\pgfqpoint{0.661989in}{0.754679in}}%
\pgfpathlineto{\pgfqpoint{0.663092in}{0.758268in}}%
\pgfpathlineto{\pgfqpoint{0.663512in}{0.759254in}}%
\pgfpathlineto{\pgfqpoint{0.664606in}{0.763908in}}%
\pgfpathlineto{\pgfqpoint{0.665139in}{0.765013in}}%
\pgfpathlineto{\pgfqpoint{0.666223in}{0.768918in}}%
\pgfpathlineto{\pgfqpoint{0.666550in}{0.769983in}}%
\pgfpathlineto{\pgfqpoint{0.667653in}{0.772191in}}%
\pgfpathlineto{\pgfqpoint{0.667989in}{0.773296in}}%
\pgfpathlineto{\pgfqpoint{0.669082in}{0.777753in}}%
\pgfpathlineto{\pgfqpoint{0.669288in}{0.778857in}}%
\pgfpathlineto{\pgfqpoint{0.670372in}{0.782762in}}%
\pgfpathlineto{\pgfqpoint{0.670802in}{0.783827in}}%
\pgfpathlineto{\pgfqpoint{0.671821in}{0.787219in}}%
\pgfpathlineto{\pgfqpoint{0.672363in}{0.788323in}}%
\pgfpathlineto{\pgfqpoint{0.673466in}{0.791282in}}%
\pgfpathlineto{\pgfqpoint{0.673952in}{0.792386in}}%
\pgfpathlineto{\pgfqpoint{0.675055in}{0.795660in}}%
\pgfpathlineto{\pgfqpoint{0.675419in}{0.796725in}}%
\pgfpathlineto{\pgfqpoint{0.676503in}{0.799762in}}%
\pgfpathlineto{\pgfqpoint{0.676914in}{0.800866in}}%
\pgfpathlineto{\pgfqpoint{0.677980in}{0.803509in}}%
\pgfpathlineto{\pgfqpoint{0.678541in}{0.804613in}}%
\pgfpathlineto{\pgfqpoint{0.679615in}{0.806901in}}%
\pgfpathlineto{\pgfqpoint{0.680120in}{0.808005in}}%
\pgfpathlineto{\pgfqpoint{0.681195in}{0.810806in}}%
\pgfpathlineto{\pgfqpoint{0.681690in}{0.811910in}}%
\pgfpathlineto{\pgfqpoint{0.682765in}{0.815381in}}%
\pgfpathlineto{\pgfqpoint{0.683270in}{0.816446in}}%
\pgfpathlineto{\pgfqpoint{0.684372in}{0.819759in}}%
\pgfpathlineto{\pgfqpoint{0.684671in}{0.820863in}}%
\pgfpathlineto{\pgfqpoint{0.685765in}{0.824019in}}%
\pgfpathlineto{\pgfqpoint{0.686157in}{0.825123in}}%
\pgfpathlineto{\pgfqpoint{0.687251in}{0.828081in}}%
\pgfpathlineto{\pgfqpoint{0.687821in}{0.829186in}}%
\pgfpathlineto{\pgfqpoint{0.688924in}{0.832538in}}%
\pgfpathlineto{\pgfqpoint{0.689372in}{0.833643in}}%
\pgfpathlineto{\pgfqpoint{0.690447in}{0.836207in}}%
\pgfpathlineto{\pgfqpoint{0.690943in}{0.837311in}}%
\pgfpathlineto{\pgfqpoint{0.691933in}{0.839086in}}%
\pgfpathlineto{\pgfqpoint{0.692363in}{0.840111in}}%
\pgfpathlineto{\pgfqpoint{0.693447in}{0.842596in}}%
\pgfpathlineto{\pgfqpoint{0.693933in}{0.843701in}}%
\pgfpathlineto{\pgfqpoint{0.695036in}{0.846067in}}%
\pgfpathlineto{\pgfqpoint{0.695578in}{0.847172in}}%
\pgfpathlineto{\pgfqpoint{0.696644in}{0.850209in}}%
\pgfpathlineto{\pgfqpoint{0.697148in}{0.851274in}}%
\pgfpathlineto{\pgfqpoint{0.698251in}{0.854114in}}%
\pgfpathlineto{\pgfqpoint{0.698690in}{0.855218in}}%
\pgfpathlineto{\pgfqpoint{0.699793in}{0.857624in}}%
\pgfpathlineto{\pgfqpoint{0.700242in}{0.858728in}}%
\pgfpathlineto{\pgfqpoint{0.701345in}{0.861174in}}%
\pgfpathlineto{\pgfqpoint{0.701812in}{0.862239in}}%
\pgfpathlineto{\pgfqpoint{0.702905in}{0.865157in}}%
\pgfpathlineto{\pgfqpoint{0.703690in}{0.866262in}}%
\pgfpathlineto{\pgfqpoint{0.704774in}{0.868826in}}%
\pgfpathlineto{\pgfqpoint{0.705513in}{0.869930in}}%
\pgfpathlineto{\pgfqpoint{0.706588in}{0.871902in}}%
\pgfpathlineto{\pgfqpoint{0.707335in}{0.872967in}}%
\pgfpathlineto{\pgfqpoint{0.708438in}{0.875570in}}%
\pgfpathlineto{\pgfqpoint{0.708765in}{0.876675in}}%
\pgfpathlineto{\pgfqpoint{0.709831in}{0.879278in}}%
\pgfpathlineto{\pgfqpoint{0.710326in}{0.880382in}}%
\pgfpathlineto{\pgfqpoint{0.711429in}{0.882630in}}%
\pgfpathlineto{\pgfqpoint{0.711812in}{0.883656in}}%
\pgfpathlineto{\pgfqpoint{0.712915in}{0.886299in}}%
\pgfpathlineto{\pgfqpoint{0.713279in}{0.887403in}}%
\pgfpathlineto{\pgfqpoint{0.714382in}{0.889533in}}%
\pgfpathlineto{\pgfqpoint{0.714943in}{0.890519in}}%
\pgfpathlineto{\pgfqpoint{0.716027in}{0.893241in}}%
\pgfpathlineto{\pgfqpoint{0.716616in}{0.894345in}}%
\pgfpathlineto{\pgfqpoint{0.719074in}{0.899985in}}%
\pgfpathlineto{\pgfqpoint{0.719494in}{0.901090in}}%
\pgfpathlineto{\pgfqpoint{0.720588in}{0.903219in}}%
\pgfpathlineto{\pgfqpoint{0.721663in}{0.904324in}}%
\pgfpathlineto{\pgfqpoint{0.722719in}{0.907322in}}%
\pgfpathlineto{\pgfqpoint{0.723373in}{0.908386in}}%
\pgfpathlineto{\pgfqpoint{0.724466in}{0.910595in}}%
\pgfpathlineto{\pgfqpoint{0.725046in}{0.911700in}}%
\pgfpathlineto{\pgfqpoint{0.726149in}{0.913632in}}%
\pgfpathlineto{\pgfqpoint{0.726784in}{0.914737in}}%
\pgfpathlineto{\pgfqpoint{0.727831in}{0.916827in}}%
\pgfpathlineto{\pgfqpoint{0.728392in}{0.917932in}}%
\pgfpathlineto{\pgfqpoint{0.729457in}{0.920298in}}%
\pgfpathlineto{\pgfqpoint{0.730307in}{0.921402in}}%
\pgfpathlineto{\pgfqpoint{0.731410in}{0.923256in}}%
\pgfpathlineto{\pgfqpoint{0.732055in}{0.924361in}}%
\pgfpathlineto{\pgfqpoint{0.733130in}{0.926609in}}%
\pgfpathlineto{\pgfqpoint{0.733700in}{0.927713in}}%
\pgfpathlineto{\pgfqpoint{0.734737in}{0.929607in}}%
\pgfpathlineto{\pgfqpoint{0.734756in}{0.929607in}}%
\pgfpathlineto{\pgfqpoint{0.735457in}{0.930711in}}%
\pgfpathlineto{\pgfqpoint{0.736513in}{0.932407in}}%
\pgfpathlineto{\pgfqpoint{0.737177in}{0.933511in}}%
\pgfpathlineto{\pgfqpoint{0.738270in}{0.935641in}}%
\pgfpathlineto{\pgfqpoint{0.738719in}{0.936667in}}%
\pgfpathlineto{\pgfqpoint{0.739822in}{0.938915in}}%
\pgfpathlineto{\pgfqpoint{0.740495in}{0.940019in}}%
\pgfpathlineto{\pgfqpoint{0.741597in}{0.942149in}}%
\pgfpathlineto{\pgfqpoint{0.742308in}{0.943254in}}%
\pgfpathlineto{\pgfqpoint{0.743410in}{0.944989in}}%
\pgfpathlineto{\pgfqpoint{0.744205in}{0.946094in}}%
\pgfpathlineto{\pgfqpoint{0.745298in}{0.948026in}}%
\pgfpathlineto{\pgfqpoint{0.746149in}{0.949131in}}%
\pgfpathlineto{\pgfqpoint{0.747139in}{0.950945in}}%
\pgfpathlineto{\pgfqpoint{0.747644in}{0.952010in}}%
\pgfpathlineto{\pgfqpoint{0.748728in}{0.954021in}}%
\pgfpathlineto{\pgfqpoint{0.749168in}{0.955126in}}%
\pgfpathlineto{\pgfqpoint{0.750242in}{0.956980in}}%
\pgfpathlineto{\pgfqpoint{0.751102in}{0.958084in}}%
\pgfpathlineto{\pgfqpoint{0.752158in}{0.959977in}}%
\pgfpathlineto{\pgfqpoint{0.752196in}{0.959977in}}%
\pgfpathlineto{\pgfqpoint{0.752971in}{0.961082in}}%
\pgfpathlineto{\pgfqpoint{0.754074in}{0.963290in}}%
\pgfpathlineto{\pgfqpoint{0.754840in}{0.964395in}}%
\pgfpathlineto{\pgfqpoint{0.755943in}{0.966643in}}%
\pgfpathlineto{\pgfqpoint{0.756710in}{0.967747in}}%
\pgfpathlineto{\pgfqpoint{0.757775in}{0.969877in}}%
\pgfpathlineto{\pgfqpoint{0.758383in}{0.970942in}}%
\pgfpathlineto{\pgfqpoint{0.759457in}{0.972717in}}%
\pgfpathlineto{\pgfqpoint{0.759476in}{0.972717in}}%
\pgfpathlineto{\pgfqpoint{0.760261in}{0.973822in}}%
\pgfpathlineto{\pgfqpoint{0.761364in}{0.975636in}}%
\pgfpathlineto{\pgfqpoint{0.761897in}{0.976740in}}%
\pgfpathlineto{\pgfqpoint{0.762999in}{0.978989in}}%
\pgfpathlineto{\pgfqpoint{0.763598in}{0.980093in}}%
\pgfpathlineto{\pgfqpoint{0.764691in}{0.981552in}}%
\pgfpathlineto{\pgfqpoint{0.765196in}{0.982578in}}%
\pgfpathlineto{\pgfqpoint{0.766299in}{0.984471in}}%
\pgfpathlineto{\pgfqpoint{0.766953in}{0.985576in}}%
\pgfpathlineto{\pgfqpoint{0.768018in}{0.987942in}}%
\pgfpathlineto{\pgfqpoint{0.769168in}{0.989046in}}%
\pgfpathlineto{\pgfqpoint{0.770271in}{0.990900in}}%
\pgfpathlineto{\pgfqpoint{0.770841in}{0.991926in}}%
\pgfpathlineto{\pgfqpoint{0.771915in}{0.993819in}}%
\pgfpathlineto{\pgfqpoint{0.773009in}{0.994923in}}%
\pgfpathlineto{\pgfqpoint{0.774112in}{0.996422in}}%
\pgfpathlineto{\pgfqpoint{0.774803in}{0.997487in}}%
\pgfpathlineto{\pgfqpoint{0.775878in}{0.999735in}}%
\pgfpathlineto{\pgfqpoint{0.776710in}{1.000840in}}%
\pgfpathlineto{\pgfqpoint{0.777803in}{1.002930in}}%
\pgfpathlineto{\pgfqpoint{0.778607in}{1.003995in}}%
\pgfpathlineto{\pgfqpoint{0.779691in}{1.005888in}}%
\pgfpathlineto{\pgfqpoint{0.780794in}{1.006993in}}%
\pgfpathlineto{\pgfqpoint{0.781850in}{1.009359in}}%
\pgfpathlineto{\pgfqpoint{0.781869in}{1.009359in}}%
\pgfpathlineto{\pgfqpoint{0.782504in}{1.010424in}}%
\pgfpathlineto{\pgfqpoint{0.783532in}{1.012475in}}%
\pgfpathlineto{\pgfqpoint{0.784588in}{1.013580in}}%
\pgfpathlineto{\pgfqpoint{0.785635in}{1.015394in}}%
\pgfpathlineto{\pgfqpoint{0.785691in}{1.015394in}}%
\pgfpathlineto{\pgfqpoint{0.786439in}{1.016420in}}%
\pgfpathlineto{\pgfqpoint{0.786439in}{1.016498in}}%
\pgfpathlineto{\pgfqpoint{0.788046in}{1.018510in}}%
\pgfpathlineto{\pgfqpoint{0.788850in}{1.019575in}}%
\pgfpathlineto{\pgfqpoint{0.789944in}{1.021823in}}%
\pgfpathlineto{\pgfqpoint{0.790832in}{1.022888in}}%
\pgfpathlineto{\pgfqpoint{0.791934in}{1.024781in}}%
\pgfpathlineto{\pgfqpoint{0.792514in}{1.025886in}}%
\pgfpathlineto{\pgfqpoint{0.793570in}{1.027503in}}%
\pgfpathlineto{\pgfqpoint{0.793617in}{1.027503in}}%
\pgfpathlineto{\pgfqpoint{0.794411in}{1.028607in}}%
\pgfpathlineto{\pgfqpoint{0.795504in}{1.030303in}}%
\pgfpathlineto{\pgfqpoint{0.796570in}{1.031408in}}%
\pgfpathlineto{\pgfqpoint{0.797673in}{1.033419in}}%
\pgfpathlineto{\pgfqpoint{0.798626in}{1.034524in}}%
\pgfpathlineto{\pgfqpoint{0.799710in}{1.036220in}}%
\pgfpathlineto{\pgfqpoint{0.800598in}{1.037324in}}%
\pgfpathlineto{\pgfqpoint{0.801663in}{1.039020in}}%
\pgfpathlineto{\pgfqpoint{0.802906in}{1.040125in}}%
\pgfpathlineto{\pgfqpoint{0.804000in}{1.041742in}}%
\pgfpathlineto{\pgfqpoint{0.804785in}{1.042846in}}%
\pgfpathlineto{\pgfqpoint{0.806579in}{1.046001in}}%
\pgfpathlineto{\pgfqpoint{0.807673in}{1.047027in}}%
\pgfpathlineto{\pgfqpoint{0.808757in}{1.048210in}}%
\pgfpathlineto{\pgfqpoint{0.810103in}{1.049315in}}%
\pgfpathlineto{\pgfqpoint{0.811168in}{1.050616in}}%
\pgfpathlineto{\pgfqpoint{0.812028in}{1.051721in}}%
\pgfpathlineto{\pgfqpoint{0.813131in}{1.053022in}}%
\pgfpathlineto{\pgfqpoint{0.813757in}{1.054127in}}%
\pgfpathlineto{\pgfqpoint{0.814851in}{1.055586in}}%
\pgfpathlineto{\pgfqpoint{0.815598in}{1.056690in}}%
\pgfpathlineto{\pgfqpoint{0.816682in}{1.058584in}}%
\pgfpathlineto{\pgfqpoint{0.817617in}{1.059688in}}%
\pgfpathlineto{\pgfqpoint{0.818701in}{1.061226in}}%
\pgfpathlineto{\pgfqpoint{0.819439in}{1.062331in}}%
\pgfpathlineto{\pgfqpoint{0.820542in}{1.063593in}}%
\pgfpathlineto{\pgfqpoint{0.821355in}{1.064697in}}%
\pgfpathlineto{\pgfqpoint{0.822458in}{1.066472in}}%
\pgfpathlineto{\pgfqpoint{0.823402in}{1.067577in}}%
\pgfpathlineto{\pgfqpoint{0.824430in}{1.069943in}}%
\pgfpathlineto{\pgfqpoint{0.825234in}{1.071048in}}%
\pgfpathlineto{\pgfqpoint{0.826804in}{1.073099in}}%
\pgfpathlineto{\pgfqpoint{0.827654in}{1.074163in}}%
\pgfpathlineto{\pgfqpoint{0.828757in}{1.075978in}}%
\pgfpathlineto{\pgfqpoint{0.829739in}{1.077082in}}%
\pgfpathlineto{\pgfqpoint{0.830841in}{1.078936in}}%
\pgfpathlineto{\pgfqpoint{0.831832in}{1.080001in}}%
\pgfpathlineto{\pgfqpoint{0.832879in}{1.081066in}}%
\pgfpathlineto{\pgfqpoint{0.833963in}{1.082091in}}%
\pgfpathlineto{\pgfqpoint{0.835000in}{1.084300in}}%
\pgfpathlineto{\pgfqpoint{0.836056in}{1.085405in}}%
\pgfpathlineto{\pgfqpoint{0.837150in}{1.087101in}}%
\pgfpathlineto{\pgfqpoint{0.837795in}{1.088166in}}%
\pgfpathlineto{\pgfqpoint{0.838879in}{1.089940in}}%
\pgfpathlineto{\pgfqpoint{0.838898in}{1.089940in}}%
\pgfpathlineto{\pgfqpoint{0.839655in}{1.091045in}}%
\pgfpathlineto{\pgfqpoint{0.840729in}{1.092268in}}%
\pgfpathlineto{\pgfqpoint{0.841636in}{1.093372in}}%
\pgfpathlineto{\pgfqpoint{0.842739in}{1.094319in}}%
\pgfpathlineto{\pgfqpoint{0.843570in}{1.095384in}}%
\pgfpathlineto{\pgfqpoint{0.844673in}{1.096961in}}%
\pgfpathlineto{\pgfqpoint{0.845253in}{1.098066in}}%
\pgfpathlineto{\pgfqpoint{0.846346in}{1.099525in}}%
\pgfpathlineto{\pgfqpoint{0.847384in}{1.100629in}}%
\pgfpathlineto{\pgfqpoint{0.848486in}{1.102325in}}%
\pgfpathlineto{\pgfqpoint{0.849898in}{1.103430in}}%
\pgfpathlineto{\pgfqpoint{0.851000in}{1.104376in}}%
\pgfpathlineto{\pgfqpoint{0.852047in}{1.105481in}}%
\pgfpathlineto{\pgfqpoint{0.853141in}{1.107137in}}%
\pgfpathlineto{\pgfqpoint{0.854057in}{1.108242in}}%
\pgfpathlineto{\pgfqpoint{0.855131in}{1.109622in}}%
\pgfpathlineto{\pgfqpoint{0.855150in}{1.109622in}}%
\pgfpathlineto{\pgfqpoint{0.855814in}{1.110727in}}%
\pgfpathlineto{\pgfqpoint{0.856870in}{1.111989in}}%
\pgfpathlineto{\pgfqpoint{0.856907in}{1.111989in}}%
\pgfpathlineto{\pgfqpoint{0.858113in}{1.113014in}}%
\pgfpathlineto{\pgfqpoint{0.859216in}{1.114237in}}%
\pgfpathlineto{\pgfqpoint{0.860337in}{1.115341in}}%
\pgfpathlineto{\pgfqpoint{0.861440in}{1.117274in}}%
\pgfpathlineto{\pgfqpoint{0.862673in}{1.118379in}}%
\pgfpathlineto{\pgfqpoint{0.863673in}{1.119562in}}%
\pgfpathlineto{\pgfqpoint{0.864617in}{1.120666in}}%
\pgfpathlineto{\pgfqpoint{0.865627in}{1.121455in}}%
\pgfpathlineto{\pgfqpoint{0.866365in}{1.122559in}}%
\pgfpathlineto{\pgfqpoint{0.867431in}{1.124216in}}%
\pgfpathlineto{\pgfqpoint{0.868552in}{1.125320in}}%
\pgfpathlineto{\pgfqpoint{0.869608in}{1.126346in}}%
\pgfpathlineto{\pgfqpoint{0.870786in}{1.127450in}}%
\pgfpathlineto{\pgfqpoint{0.871879in}{1.128634in}}%
\pgfpathlineto{\pgfqpoint{0.872842in}{1.129699in}}%
\pgfpathlineto{\pgfqpoint{0.873879in}{1.130921in}}%
\pgfpathlineto{\pgfqpoint{0.874991in}{1.132026in}}%
\pgfpathlineto{\pgfqpoint{0.876094in}{1.133485in}}%
\pgfpathlineto{\pgfqpoint{0.877776in}{1.135812in}}%
\pgfpathlineto{\pgfqpoint{0.878449in}{1.136917in}}%
\pgfpathlineto{\pgfqpoint{0.879515in}{1.138415in}}%
\pgfpathlineto{\pgfqpoint{0.880599in}{1.139520in}}%
\pgfpathlineto{\pgfqpoint{0.881674in}{1.140624in}}%
\pgfpathlineto{\pgfqpoint{0.881692in}{1.140624in}}%
\pgfpathlineto{\pgfqpoint{0.882300in}{1.141729in}}%
\pgfpathlineto{\pgfqpoint{0.883319in}{1.142754in}}%
\pgfpathlineto{\pgfqpoint{0.884365in}{1.143819in}}%
\pgfpathlineto{\pgfqpoint{0.885459in}{1.145121in}}%
\pgfpathlineto{\pgfqpoint{0.886758in}{1.146225in}}%
\pgfpathlineto{\pgfqpoint{0.887842in}{1.146935in}}%
\pgfpathlineto{\pgfqpoint{0.889160in}{1.148039in}}%
\pgfpathlineto{\pgfqpoint{0.890225in}{1.149420in}}%
\pgfpathlineto{\pgfqpoint{0.890253in}{1.149420in}}%
\pgfpathlineto{\pgfqpoint{0.891562in}{1.150524in}}%
\pgfpathlineto{\pgfqpoint{0.892664in}{1.151431in}}%
\pgfpathlineto{\pgfqpoint{0.893758in}{1.152536in}}%
\pgfpathlineto{\pgfqpoint{0.894749in}{1.153956in}}%
\pgfpathlineto{\pgfqpoint{0.895898in}{1.155060in}}%
\pgfpathlineto{\pgfqpoint{0.896992in}{1.156953in}}%
\pgfpathlineto{\pgfqpoint{0.897674in}{1.158018in}}%
\pgfpathlineto{\pgfqpoint{0.897674in}{1.158058in}}%
\pgfpathlineto{\pgfqpoint{0.898777in}{1.159202in}}%
\pgfpathlineto{\pgfqpoint{0.899618in}{1.160306in}}%
\pgfpathlineto{\pgfqpoint{0.900711in}{1.161765in}}%
\pgfpathlineto{\pgfqpoint{0.901711in}{1.162870in}}%
\pgfpathlineto{\pgfqpoint{0.902786in}{1.163856in}}%
\pgfpathlineto{\pgfqpoint{0.903889in}{1.164960in}}%
\pgfpathlineto{\pgfqpoint{0.904898in}{1.166262in}}%
\pgfpathlineto{\pgfqpoint{0.906010in}{1.167366in}}%
\pgfpathlineto{\pgfqpoint{0.907076in}{1.168392in}}%
\pgfpathlineto{\pgfqpoint{0.908356in}{1.169496in}}%
\pgfpathlineto{\pgfqpoint{0.909459in}{1.170837in}}%
\pgfpathlineto{\pgfqpoint{0.910328in}{1.171942in}}%
\pgfpathlineto{\pgfqpoint{0.911384in}{1.173283in}}%
\pgfpathlineto{\pgfqpoint{0.912702in}{1.174387in}}%
\pgfpathlineto{\pgfqpoint{0.913786in}{1.175767in}}%
\pgfpathlineto{\pgfqpoint{0.914954in}{1.176872in}}%
\pgfpathlineto{\pgfqpoint{0.916057in}{1.177897in}}%
\pgfpathlineto{\pgfqpoint{0.917263in}{1.179002in}}%
\pgfpathlineto{\pgfqpoint{0.918356in}{1.180303in}}%
\pgfpathlineto{\pgfqpoint{0.919412in}{1.181408in}}%
\pgfpathlineto{\pgfqpoint{0.920506in}{1.182473in}}%
\pgfpathlineto{\pgfqpoint{0.921955in}{1.183577in}}%
\pgfpathlineto{\pgfqpoint{0.923039in}{1.184958in}}%
\pgfpathlineto{\pgfqpoint{0.924263in}{1.186062in}}%
\pgfpathlineto{\pgfqpoint{0.925104in}{1.187166in}}%
\pgfpathlineto{\pgfqpoint{0.926618in}{1.188271in}}%
\pgfpathlineto{\pgfqpoint{0.927665in}{1.189099in}}%
\pgfpathlineto{\pgfqpoint{0.928880in}{1.190203in}}%
\pgfpathlineto{\pgfqpoint{0.929964in}{1.191584in}}%
\pgfpathlineto{\pgfqpoint{0.930917in}{1.192688in}}%
\pgfpathlineto{\pgfqpoint{0.932011in}{1.193714in}}%
\pgfpathlineto{\pgfqpoint{0.933095in}{1.194818in}}%
\pgfpathlineto{\pgfqpoint{0.934160in}{1.195883in}}%
\pgfpathlineto{\pgfqpoint{0.935385in}{1.196988in}}%
\pgfpathlineto{\pgfqpoint{0.936375in}{1.198013in}}%
\pgfpathlineto{\pgfqpoint{0.937815in}{1.199117in}}%
\pgfpathlineto{\pgfqpoint{0.938861in}{1.200261in}}%
\pgfpathlineto{\pgfqpoint{0.939684in}{1.201366in}}%
\pgfpathlineto{\pgfqpoint{0.940749in}{1.202391in}}%
\pgfpathlineto{\pgfqpoint{0.942039in}{1.203496in}}%
\pgfpathlineto{\pgfqpoint{0.943132in}{1.204442in}}%
\pgfpathlineto{\pgfqpoint{0.944572in}{1.205507in}}%
\pgfpathlineto{\pgfqpoint{0.945637in}{1.207282in}}%
\pgfpathlineto{\pgfqpoint{0.947188in}{1.208386in}}%
\pgfpathlineto{\pgfqpoint{0.948291in}{1.209294in}}%
\pgfpathlineto{\pgfqpoint{0.949703in}{1.210359in}}%
\pgfpathlineto{\pgfqpoint{0.950805in}{1.211345in}}%
\pgfpathlineto{\pgfqpoint{0.951843in}{1.212449in}}%
\pgfpathlineto{\pgfqpoint{0.952946in}{1.213435in}}%
\pgfpathlineto{\pgfqpoint{0.954198in}{1.214539in}}%
\pgfpathlineto{\pgfqpoint{0.955198in}{1.215407in}}%
\pgfpathlineto{\pgfqpoint{0.956749in}{1.216512in}}%
\pgfpathlineto{\pgfqpoint{0.957824in}{1.217300in}}%
\pgfpathlineto{\pgfqpoint{0.958656in}{1.218405in}}%
\pgfpathlineto{\pgfqpoint{0.959712in}{1.219430in}}%
\pgfpathlineto{\pgfqpoint{0.960936in}{1.220495in}}%
\pgfpathlineto{\pgfqpoint{0.961871in}{1.221245in}}%
\pgfpathlineto{\pgfqpoint{0.963581in}{1.222310in}}%
\pgfpathlineto{\pgfqpoint{0.964675in}{1.223138in}}%
\pgfpathlineto{\pgfqpoint{0.965918in}{1.224242in}}%
\pgfpathlineto{\pgfqpoint{0.966946in}{1.225505in}}%
\pgfpathlineto{\pgfqpoint{0.968161in}{1.226609in}}%
\pgfpathlineto{\pgfqpoint{0.969049in}{1.227319in}}%
\pgfpathlineto{\pgfqpoint{0.969245in}{1.227319in}}%
\pgfpathlineto{\pgfqpoint{0.970862in}{1.228423in}}%
\pgfpathlineto{\pgfqpoint{0.971936in}{1.229804in}}%
\pgfpathlineto{\pgfqpoint{0.973123in}{1.230908in}}%
\pgfpathlineto{\pgfqpoint{0.974207in}{1.232013in}}%
\pgfpathlineto{\pgfqpoint{0.975394in}{1.233077in}}%
\pgfpathlineto{\pgfqpoint{0.976497in}{1.233985in}}%
\pgfpathlineto{\pgfqpoint{0.977750in}{1.235089in}}%
\pgfpathlineto{\pgfqpoint{0.978852in}{1.235957in}}%
\pgfpathlineto{\pgfqpoint{0.980236in}{1.237061in}}%
\pgfpathlineto{\pgfqpoint{0.981310in}{1.237929in}}%
\pgfpathlineto{\pgfqpoint{0.982413in}{1.239033in}}%
\pgfpathlineto{\pgfqpoint{0.983497in}{1.240217in}}%
\pgfpathlineto{\pgfqpoint{0.984647in}{1.241321in}}%
\pgfpathlineto{\pgfqpoint{0.985750in}{1.242110in}}%
\pgfpathlineto{\pgfqpoint{0.987095in}{1.243214in}}%
\pgfpathlineto{\pgfqpoint{0.988198in}{1.244161in}}%
\pgfpathlineto{\pgfqpoint{0.989684in}{1.245265in}}%
\pgfpathlineto{\pgfqpoint{0.990778in}{1.246606in}}%
\pgfpathlineto{\pgfqpoint{0.992722in}{1.247711in}}%
\pgfpathlineto{\pgfqpoint{0.993759in}{1.248815in}}%
\pgfpathlineto{\pgfqpoint{0.994862in}{1.249919in}}%
\pgfpathlineto{\pgfqpoint{0.995937in}{1.250708in}}%
\pgfpathlineto{\pgfqpoint{0.995965in}{1.250708in}}%
\pgfpathlineto{\pgfqpoint{0.997217in}{1.251813in}}%
\pgfpathlineto{\pgfqpoint{0.998311in}{1.252641in}}%
\pgfpathlineto{\pgfqpoint{0.999759in}{1.253666in}}%
\pgfpathlineto{\pgfqpoint{1.000797in}{1.255008in}}%
\pgfpathlineto{\pgfqpoint{1.001918in}{1.256112in}}%
\pgfpathlineto{\pgfqpoint{1.003021in}{1.256901in}}%
\pgfpathlineto{\pgfqpoint{1.004021in}{1.258005in}}%
\pgfpathlineto{\pgfqpoint{1.005086in}{1.258636in}}%
\pgfpathlineto{\pgfqpoint{1.006348in}{1.259741in}}%
\pgfpathlineto{\pgfqpoint{1.007395in}{1.260529in}}%
\pgfpathlineto{\pgfqpoint{1.008759in}{1.261634in}}%
\pgfpathlineto{\pgfqpoint{1.009853in}{1.262581in}}%
\pgfpathlineto{\pgfqpoint{1.011498in}{1.263685in}}%
\pgfpathlineto{\pgfqpoint{1.012535in}{1.264316in}}%
\pgfpathlineto{\pgfqpoint{1.013993in}{1.265420in}}%
\pgfpathlineto{\pgfqpoint{1.015096in}{1.266446in}}%
\pgfpathlineto{\pgfqpoint{1.017208in}{1.267550in}}%
\pgfpathlineto{\pgfqpoint{1.018292in}{1.268970in}}%
\pgfpathlineto{\pgfqpoint{1.019862in}{1.270075in}}%
\pgfpathlineto{\pgfqpoint{1.020937in}{1.271021in}}%
\pgfpathlineto{\pgfqpoint{1.022573in}{1.272126in}}%
\pgfpathlineto{\pgfqpoint{1.024021in}{1.273427in}}%
\pgfpathlineto{\pgfqpoint{1.025694in}{1.274532in}}%
\pgfpathlineto{\pgfqpoint{1.026750in}{1.275675in}}%
\pgfpathlineto{\pgfqpoint{1.027582in}{1.276780in}}%
\pgfpathlineto{\pgfqpoint{1.028675in}{1.277569in}}%
\pgfpathlineto{\pgfqpoint{1.030750in}{1.278673in}}%
\pgfpathlineto{\pgfqpoint{1.031806in}{1.279383in}}%
\pgfpathlineto{\pgfqpoint{1.031844in}{1.279383in}}%
\pgfpathlineto{\pgfqpoint{1.033741in}{1.280487in}}%
\pgfpathlineto{\pgfqpoint{1.034769in}{1.281355in}}%
\pgfpathlineto{\pgfqpoint{1.034797in}{1.281355in}}%
\pgfpathlineto{\pgfqpoint{1.036554in}{1.282460in}}%
\pgfpathlineto{\pgfqpoint{1.037647in}{1.283406in}}%
\pgfpathlineto{\pgfqpoint{1.039320in}{1.284511in}}%
\pgfpathlineto{\pgfqpoint{1.040414in}{1.285536in}}%
\pgfpathlineto{\pgfqpoint{1.042208in}{1.286562in}}%
\pgfpathlineto{\pgfqpoint{1.042208in}{1.286601in}}%
\pgfpathlineto{\pgfqpoint{1.043264in}{1.287429in}}%
\pgfpathlineto{\pgfqpoint{1.044573in}{1.288534in}}%
\pgfpathlineto{\pgfqpoint{1.045563in}{1.289441in}}%
\pgfpathlineto{\pgfqpoint{1.047199in}{1.290545in}}%
\pgfpathlineto{\pgfqpoint{1.048199in}{1.291374in}}%
\pgfpathlineto{\pgfqpoint{1.049489in}{1.292478in}}%
\pgfpathlineto{\pgfqpoint{1.050554in}{1.293425in}}%
\pgfpathlineto{\pgfqpoint{1.052003in}{1.294529in}}%
\pgfpathlineto{\pgfqpoint{1.052993in}{1.295436in}}%
\pgfpathlineto{\pgfqpoint{1.054778in}{1.296541in}}%
\pgfpathlineto{\pgfqpoint{1.055835in}{1.297093in}}%
\pgfpathlineto{\pgfqpoint{1.057078in}{1.298197in}}%
\pgfpathlineto{\pgfqpoint{1.058003in}{1.299144in}}%
\pgfpathlineto{\pgfqpoint{1.058078in}{1.299144in}}%
\pgfpathlineto{\pgfqpoint{1.059610in}{1.300248in}}%
\pgfpathlineto{\pgfqpoint{1.060685in}{1.301037in}}%
\pgfpathlineto{\pgfqpoint{1.062442in}{1.302141in}}%
\pgfpathlineto{\pgfqpoint{1.063536in}{1.302812in}}%
\pgfpathlineto{\pgfqpoint{1.064909in}{1.303916in}}%
\pgfpathlineto{\pgfqpoint{1.066012in}{1.304508in}}%
\pgfpathlineto{\pgfqpoint{1.068424in}{1.305573in}}%
\pgfpathlineto{\pgfqpoint{1.069470in}{1.306243in}}%
\pgfpathlineto{\pgfqpoint{1.071966in}{1.307348in}}%
\pgfpathlineto{\pgfqpoint{1.073022in}{1.308176in}}%
\pgfpathlineto{\pgfqpoint{1.074592in}{1.309280in}}%
\pgfpathlineto{\pgfqpoint{1.075657in}{1.310069in}}%
\pgfpathlineto{\pgfqpoint{1.077181in}{1.311174in}}%
\pgfpathlineto{\pgfqpoint{1.078012in}{1.311844in}}%
\pgfpathlineto{\pgfqpoint{1.078255in}{1.311844in}}%
\pgfpathlineto{\pgfqpoint{1.079975in}{1.312949in}}%
\pgfpathlineto{\pgfqpoint{1.081040in}{1.313816in}}%
\pgfpathlineto{\pgfqpoint{1.082405in}{1.314921in}}%
\pgfpathlineto{\pgfqpoint{1.083470in}{1.315867in}}%
\pgfpathlineto{\pgfqpoint{1.085069in}{1.316932in}}%
\pgfpathlineto{\pgfqpoint{1.086153in}{1.318037in}}%
\pgfpathlineto{\pgfqpoint{1.087461in}{1.319102in}}%
\pgfpathlineto{\pgfqpoint{1.088536in}{1.320167in}}%
\pgfpathlineto{\pgfqpoint{1.090218in}{1.321271in}}%
\pgfpathlineto{\pgfqpoint{1.091293in}{1.321823in}}%
\pgfpathlineto{\pgfqpoint{1.093134in}{1.322928in}}%
\pgfpathlineto{\pgfqpoint{1.094237in}{1.324111in}}%
\pgfpathlineto{\pgfqpoint{1.096798in}{1.325215in}}%
\pgfpathlineto{\pgfqpoint{1.097891in}{1.325846in}}%
\pgfpathlineto{\pgfqpoint{1.099527in}{1.326951in}}%
\pgfpathlineto{\pgfqpoint{1.100564in}{1.327621in}}%
\pgfpathlineto{\pgfqpoint{1.100629in}{1.327621in}}%
\pgfpathlineto{\pgfqpoint{1.101788in}{1.328726in}}%
\pgfpathlineto{\pgfqpoint{1.102798in}{1.329672in}}%
\pgfpathlineto{\pgfqpoint{1.105396in}{1.330777in}}%
\pgfpathlineto{\pgfqpoint{1.106461in}{1.331881in}}%
\pgfpathlineto{\pgfqpoint{1.108658in}{1.332985in}}%
\pgfpathlineto{\pgfqpoint{1.109732in}{1.333932in}}%
\pgfpathlineto{\pgfqpoint{1.111564in}{1.335036in}}%
\pgfpathlineto{\pgfqpoint{1.112630in}{1.335983in}}%
\pgfpathlineto{\pgfqpoint{1.114340in}{1.337048in}}%
\pgfpathlineto{\pgfqpoint{1.115433in}{1.338074in}}%
\pgfpathlineto{\pgfqpoint{1.117218in}{1.339178in}}%
\pgfpathlineto{\pgfqpoint{1.118303in}{1.339770in}}%
\pgfpathlineto{\pgfqpoint{1.120461in}{1.340874in}}%
\pgfpathlineto{\pgfqpoint{1.121303in}{1.341584in}}%
\pgfpathlineto{\pgfqpoint{1.123022in}{1.342688in}}%
\pgfpathlineto{\pgfqpoint{1.124069in}{1.343359in}}%
\pgfpathlineto{\pgfqpoint{1.125546in}{1.344463in}}%
\pgfpathlineto{\pgfqpoint{1.126434in}{1.345134in}}%
\pgfpathlineto{\pgfqpoint{1.128555in}{1.346238in}}%
\pgfpathlineto{\pgfqpoint{1.129639in}{1.347106in}}%
\pgfpathlineto{\pgfqpoint{1.130966in}{1.348210in}}%
\pgfpathlineto{\pgfqpoint{1.131929in}{1.349275in}}%
\pgfpathlineto{\pgfqpoint{1.133658in}{1.350380in}}%
\pgfpathlineto{\pgfqpoint{1.134751in}{1.351011in}}%
\pgfpathlineto{\pgfqpoint{1.136434in}{1.352115in}}%
\pgfpathlineto{\pgfqpoint{1.137499in}{1.352746in}}%
\pgfpathlineto{\pgfqpoint{1.139050in}{1.353851in}}%
\pgfpathlineto{\pgfqpoint{1.140004in}{1.354561in}}%
\pgfpathlineto{\pgfqpoint{1.141583in}{1.355665in}}%
\pgfpathlineto{\pgfqpoint{1.142658in}{1.356178in}}%
\pgfpathlineto{\pgfqpoint{1.144527in}{1.357282in}}%
\pgfpathlineto{\pgfqpoint{1.145537in}{1.357795in}}%
\pgfpathlineto{\pgfqpoint{1.145583in}{1.357795in}}%
\pgfpathlineto{\pgfqpoint{1.147621in}{1.358899in}}%
\pgfpathlineto{\pgfqpoint{1.148714in}{1.359491in}}%
\pgfpathlineto{\pgfqpoint{1.150667in}{1.360595in}}%
\pgfpathlineto{\pgfqpoint{1.151770in}{1.361463in}}%
\pgfpathlineto{\pgfqpoint{1.153817in}{1.362567in}}%
\pgfpathlineto{\pgfqpoint{1.154798in}{1.363356in}}%
\pgfpathlineto{\pgfqpoint{1.156985in}{1.364461in}}%
\pgfpathlineto{\pgfqpoint{1.158023in}{1.365486in}}%
\pgfpathlineto{\pgfqpoint{1.159677in}{1.366590in}}%
\pgfpathlineto{\pgfqpoint{1.160780in}{1.367498in}}%
\pgfpathlineto{\pgfqpoint{1.162649in}{1.368602in}}%
\pgfpathlineto{\pgfqpoint{1.163602in}{1.369312in}}%
\pgfpathlineto{\pgfqpoint{1.165462in}{1.370416in}}%
\pgfpathlineto{\pgfqpoint{1.166518in}{1.371087in}}%
\pgfpathlineto{\pgfqpoint{1.168817in}{1.372191in}}%
\pgfpathlineto{\pgfqpoint{1.169901in}{1.372980in}}%
\pgfpathlineto{\pgfqpoint{1.171845in}{1.374085in}}%
\pgfpathlineto{\pgfqpoint{1.172948in}{1.374597in}}%
\pgfpathlineto{\pgfqpoint{1.175387in}{1.375702in}}%
\pgfpathlineto{\pgfqpoint{1.176471in}{1.376372in}}%
\pgfpathlineto{\pgfqpoint{1.179247in}{1.377477in}}%
\pgfpathlineto{\pgfqpoint{1.180322in}{1.378147in}}%
\pgfpathlineto{\pgfqpoint{1.180350in}{1.378147in}}%
\pgfpathlineto{\pgfqpoint{1.181612in}{1.378778in}}%
\pgfpathlineto{\pgfqpoint{1.183696in}{1.379883in}}%
\pgfpathlineto{\pgfqpoint{1.184649in}{1.380474in}}%
\pgfpathlineto{\pgfqpoint{1.184705in}{1.380474in}}%
\pgfpathlineto{\pgfqpoint{1.187135in}{1.381579in}}%
\pgfpathlineto{\pgfqpoint{1.188079in}{1.382328in}}%
\pgfpathlineto{\pgfqpoint{1.190425in}{1.383432in}}%
\pgfpathlineto{\pgfqpoint{1.191509in}{1.384024in}}%
\pgfpathlineto{\pgfqpoint{1.192593in}{1.385128in}}%
\pgfpathlineto{\pgfqpoint{1.193546in}{1.385641in}}%
\pgfpathlineto{\pgfqpoint{1.195995in}{1.386746in}}%
\pgfpathlineto{\pgfqpoint{1.197070in}{1.387574in}}%
\pgfpathlineto{\pgfqpoint{1.198668in}{1.388678in}}%
\pgfpathlineto{\pgfqpoint{1.199705in}{1.389349in}}%
\pgfpathlineto{\pgfqpoint{1.201425in}{1.390453in}}%
\pgfpathlineto{\pgfqpoint{1.202500in}{1.391124in}}%
\pgfpathlineto{\pgfqpoint{1.204266in}{1.392228in}}%
\pgfpathlineto{\pgfqpoint{1.205341in}{1.392741in}}%
\pgfpathlineto{\pgfqpoint{1.207341in}{1.393845in}}%
\pgfpathlineto{\pgfqpoint{1.208360in}{1.394595in}}%
\pgfpathlineto{\pgfqpoint{1.211388in}{1.395699in}}%
\pgfpathlineto{\pgfqpoint{1.212463in}{1.395936in}}%
\pgfpathlineto{\pgfqpoint{1.215182in}{1.397040in}}%
\pgfpathlineto{\pgfqpoint{1.216248in}{1.397829in}}%
\pgfpathlineto{\pgfqpoint{1.219042in}{1.398933in}}%
\pgfpathlineto{\pgfqpoint{1.220126in}{1.399643in}}%
\pgfpathlineto{\pgfqpoint{1.222005in}{1.400748in}}%
\pgfpathlineto{\pgfqpoint{1.223098in}{1.401103in}}%
\pgfpathlineto{\pgfqpoint{1.225351in}{1.402168in}}%
\pgfpathlineto{\pgfqpoint{1.226201in}{1.402720in}}%
\pgfpathlineto{\pgfqpoint{1.226435in}{1.402720in}}%
\pgfpathlineto{\pgfqpoint{1.228042in}{1.403824in}}%
\pgfpathlineto{\pgfqpoint{1.229145in}{1.404495in}}%
\pgfpathlineto{\pgfqpoint{1.231182in}{1.405599in}}%
\pgfpathlineto{\pgfqpoint{1.232276in}{1.406506in}}%
\pgfpathlineto{\pgfqpoint{1.234668in}{1.407611in}}%
\pgfpathlineto{\pgfqpoint{1.235444in}{1.407926in}}%
\pgfpathlineto{\pgfqpoint{1.237799in}{1.409031in}}%
\pgfpathlineto{\pgfqpoint{1.238827in}{1.409583in}}%
\pgfpathlineto{\pgfqpoint{1.241341in}{1.410687in}}%
\pgfpathlineto{\pgfqpoint{1.242426in}{1.411161in}}%
\pgfpathlineto{\pgfqpoint{1.244472in}{1.412265in}}%
\pgfpathlineto{\pgfqpoint{1.245528in}{1.412738in}}%
\pgfpathlineto{\pgfqpoint{1.247809in}{1.413843in}}%
\pgfpathlineto{\pgfqpoint{1.248912in}{1.414592in}}%
\pgfpathlineto{\pgfqpoint{1.250510in}{1.415657in}}%
\pgfpathlineto{\pgfqpoint{1.251519in}{1.416446in}}%
\pgfpathlineto{\pgfqpoint{1.253641in}{1.417550in}}%
\pgfpathlineto{\pgfqpoint{1.254575in}{1.418063in}}%
\pgfpathlineto{\pgfqpoint{1.254734in}{1.418063in}}%
\pgfpathlineto{\pgfqpoint{1.257594in}{1.419167in}}%
\pgfpathlineto{\pgfqpoint{1.258697in}{1.419404in}}%
\pgfpathlineto{\pgfqpoint{1.261472in}{1.420469in}}%
\pgfpathlineto{\pgfqpoint{1.262566in}{1.421218in}}%
\pgfpathlineto{\pgfqpoint{1.264725in}{1.422323in}}%
\pgfpathlineto{\pgfqpoint{1.265828in}{1.422914in}}%
\pgfpathlineto{\pgfqpoint{1.268192in}{1.423979in}}%
\pgfpathlineto{\pgfqpoint{1.269248in}{1.424532in}}%
\pgfpathlineto{\pgfqpoint{1.271772in}{1.425636in}}%
\pgfpathlineto{\pgfqpoint{1.272865in}{1.426149in}}%
\pgfpathlineto{\pgfqpoint{1.274547in}{1.427253in}}%
\pgfpathlineto{\pgfqpoint{1.275510in}{1.427805in}}%
\pgfpathlineto{\pgfqpoint{1.278099in}{1.428910in}}%
\pgfpathlineto{\pgfqpoint{1.279136in}{1.429659in}}%
\pgfpathlineto{\pgfqpoint{1.281248in}{1.430764in}}%
\pgfpathlineto{\pgfqpoint{1.282323in}{1.431395in}}%
\pgfpathlineto{\pgfqpoint{1.284977in}{1.432499in}}%
\pgfpathlineto{\pgfqpoint{1.285884in}{1.433051in}}%
\pgfpathlineto{\pgfqpoint{1.288781in}{1.434156in}}%
\pgfpathlineto{\pgfqpoint{1.289959in}{1.434826in}}%
\pgfpathlineto{\pgfqpoint{1.292576in}{1.435930in}}%
\pgfpathlineto{\pgfqpoint{1.293660in}{1.436562in}}%
\pgfpathlineto{\pgfqpoint{1.296510in}{1.437666in}}%
\pgfpathlineto{\pgfqpoint{1.297221in}{1.438258in}}%
\pgfpathlineto{\pgfqpoint{1.297576in}{1.438258in}}%
\pgfpathlineto{\pgfqpoint{1.301052in}{1.439362in}}%
\pgfpathlineto{\pgfqpoint{1.302099in}{1.439796in}}%
\pgfpathlineto{\pgfqpoint{1.302155in}{1.439796in}}%
\pgfpathlineto{\pgfqpoint{1.304230in}{1.440900in}}%
\pgfpathlineto{\pgfqpoint{1.305211in}{1.441452in}}%
\pgfpathlineto{\pgfqpoint{1.308118in}{1.442557in}}%
\pgfpathlineto{\pgfqpoint{1.309155in}{1.443227in}}%
\pgfpathlineto{\pgfqpoint{1.311744in}{1.444332in}}%
\pgfpathlineto{\pgfqpoint{1.312809in}{1.444766in}}%
\pgfpathlineto{\pgfqpoint{1.315230in}{1.445870in}}%
\pgfpathlineto{\pgfqpoint{1.316324in}{1.446304in}}%
\pgfpathlineto{\pgfqpoint{1.318959in}{1.447408in}}%
\pgfpathlineto{\pgfqpoint{1.320062in}{1.447882in}}%
\pgfpathlineto{\pgfqpoint{1.321922in}{1.448986in}}%
\pgfpathlineto{\pgfqpoint{1.322894in}{1.449459in}}%
\pgfpathlineto{\pgfqpoint{1.326071in}{1.450564in}}%
\pgfpathlineto{\pgfqpoint{1.327127in}{1.451155in}}%
\pgfpathlineto{\pgfqpoint{1.327165in}{1.451155in}}%
\pgfpathlineto{\pgfqpoint{1.329146in}{1.452260in}}%
\pgfpathlineto{\pgfqpoint{1.330230in}{1.452772in}}%
\pgfpathlineto{\pgfqpoint{1.332212in}{1.453837in}}%
\pgfpathlineto{\pgfqpoint{1.333314in}{1.454547in}}%
\pgfpathlineto{\pgfqpoint{1.335735in}{1.455652in}}%
\pgfpathlineto{\pgfqpoint{1.336828in}{1.456401in}}%
\pgfpathlineto{\pgfqpoint{1.339866in}{1.457506in}}%
\pgfpathlineto{\pgfqpoint{1.340931in}{1.458137in}}%
\pgfpathlineto{\pgfqpoint{1.343380in}{1.459241in}}%
\pgfpathlineto{\pgfqpoint{1.344258in}{1.459517in}}%
\pgfpathlineto{\pgfqpoint{1.347277in}{1.460582in}}%
\pgfpathlineto{\pgfqpoint{1.348277in}{1.461016in}}%
\pgfpathlineto{\pgfqpoint{1.350511in}{1.462120in}}%
\pgfpathlineto{\pgfqpoint{1.351604in}{1.462673in}}%
\pgfpathlineto{\pgfqpoint{1.354604in}{1.463777in}}%
\pgfpathlineto{\pgfqpoint{1.355688in}{1.464369in}}%
\pgfpathlineto{\pgfqpoint{1.358847in}{1.465473in}}%
\pgfpathlineto{\pgfqpoint{1.359875in}{1.465867in}}%
\pgfpathlineto{\pgfqpoint{1.362193in}{1.466972in}}%
\pgfpathlineto{\pgfqpoint{1.363277in}{1.467485in}}%
\pgfpathlineto{\pgfqpoint{1.365595in}{1.468589in}}%
\pgfpathlineto{\pgfqpoint{1.366670in}{1.469181in}}%
\pgfpathlineto{\pgfqpoint{1.369689in}{1.470285in}}%
\pgfpathlineto{\pgfqpoint{1.370651in}{1.470719in}}%
\pgfpathlineto{\pgfqpoint{1.370782in}{1.470719in}}%
\pgfpathlineto{\pgfqpoint{1.374502in}{1.471823in}}%
\pgfpathlineto{\pgfqpoint{1.375586in}{1.472218in}}%
\pgfpathlineto{\pgfqpoint{1.375595in}{1.472218in}}%
\pgfpathlineto{\pgfqpoint{1.379165in}{1.473322in}}%
\pgfpathlineto{\pgfqpoint{1.380203in}{1.473835in}}%
\pgfpathlineto{\pgfqpoint{1.383231in}{1.474939in}}%
\pgfpathlineto{\pgfqpoint{1.384296in}{1.475491in}}%
\pgfpathlineto{\pgfqpoint{1.386689in}{1.476596in}}%
\pgfpathlineto{\pgfqpoint{1.387792in}{1.477069in}}%
\pgfpathlineto{\pgfqpoint{1.391091in}{1.478173in}}%
\pgfpathlineto{\pgfqpoint{1.392081in}{1.478844in}}%
\pgfpathlineto{\pgfqpoint{1.395194in}{1.479948in}}%
\pgfpathlineto{\pgfqpoint{1.396268in}{1.480540in}}%
\pgfpathlineto{\pgfqpoint{1.399063in}{1.481644in}}%
\pgfpathlineto{\pgfqpoint{1.400156in}{1.481999in}}%
\pgfpathlineto{\pgfqpoint{1.403708in}{1.483104in}}%
\pgfpathlineto{\pgfqpoint{1.404792in}{1.483538in}}%
\pgfpathlineto{\pgfqpoint{1.407390in}{1.484642in}}%
\pgfpathlineto{\pgfqpoint{1.408558in}{1.485155in}}%
\pgfpathlineto{\pgfqpoint{1.411156in}{1.486259in}}%
\pgfpathlineto{\pgfqpoint{1.411998in}{1.486732in}}%
\pgfpathlineto{\pgfqpoint{1.412175in}{1.486732in}}%
\pgfpathlineto{\pgfqpoint{1.415717in}{1.487837in}}%
\pgfpathlineto{\pgfqpoint{1.416540in}{1.488152in}}%
\pgfpathlineto{\pgfqpoint{1.420297in}{1.489257in}}%
\pgfpathlineto{\pgfqpoint{1.421353in}{1.490006in}}%
\pgfpathlineto{\pgfqpoint{1.425372in}{1.491111in}}%
\pgfpathlineto{\pgfqpoint{1.426372in}{1.491387in}}%
\pgfpathlineto{\pgfqpoint{1.429325in}{1.492491in}}%
\pgfpathlineto{\pgfqpoint{1.430213in}{1.492925in}}%
\pgfpathlineto{\pgfqpoint{1.433876in}{1.494029in}}%
\pgfpathlineto{\pgfqpoint{1.434895in}{1.494345in}}%
\pgfpathlineto{\pgfqpoint{1.437895in}{1.495449in}}%
\pgfpathlineto{\pgfqpoint{1.438895in}{1.495883in}}%
\pgfpathlineto{\pgfqpoint{1.442138in}{1.496988in}}%
\pgfpathlineto{\pgfqpoint{1.443082in}{1.497303in}}%
\pgfpathlineto{\pgfqpoint{1.443110in}{1.497303in}}%
\pgfpathlineto{\pgfqpoint{1.447718in}{1.498407in}}%
\pgfpathlineto{\pgfqpoint{1.448774in}{1.498960in}}%
\pgfpathlineto{\pgfqpoint{1.452605in}{1.500064in}}%
\pgfpathlineto{\pgfqpoint{1.453699in}{1.500537in}}%
\pgfpathlineto{\pgfqpoint{1.456634in}{1.501642in}}%
\pgfpathlineto{\pgfqpoint{1.457662in}{1.501997in}}%
\pgfpathlineto{\pgfqpoint{1.460671in}{1.503101in}}%
\pgfpathlineto{\pgfqpoint{1.461774in}{1.503614in}}%
\pgfpathlineto{\pgfqpoint{1.465662in}{1.504718in}}%
\pgfpathlineto{\pgfqpoint{1.466708in}{1.505231in}}%
\pgfpathlineto{\pgfqpoint{1.470895in}{1.506335in}}%
\pgfpathlineto{\pgfqpoint{1.471905in}{1.506809in}}%
\pgfpathlineto{\pgfqpoint{1.476643in}{1.507913in}}%
\pgfpathlineto{\pgfqpoint{1.477727in}{1.508347in}}%
\pgfpathlineto{\pgfqpoint{1.481821in}{1.509451in}}%
\pgfpathlineto{\pgfqpoint{1.482877in}{1.509964in}}%
\pgfpathlineto{\pgfqpoint{1.485578in}{1.511069in}}%
\pgfpathlineto{\pgfqpoint{1.486662in}{1.511739in}}%
\pgfpathlineto{\pgfqpoint{1.489419in}{1.512843in}}%
\pgfpathlineto{\pgfqpoint{1.490410in}{1.513396in}}%
\pgfpathlineto{\pgfqpoint{1.494998in}{1.514461in}}%
\pgfpathlineto{\pgfqpoint{1.496083in}{1.514973in}}%
\pgfpathlineto{\pgfqpoint{1.498709in}{1.516078in}}%
\pgfpathlineto{\pgfqpoint{1.499802in}{1.516551in}}%
\pgfpathlineto{\pgfqpoint{1.503643in}{1.517655in}}%
\pgfpathlineto{\pgfqpoint{1.504662in}{1.517971in}}%
\pgfpathlineto{\pgfqpoint{1.509139in}{1.519075in}}%
\pgfpathlineto{\pgfqpoint{1.510083in}{1.519785in}}%
\pgfpathlineto{\pgfqpoint{1.510232in}{1.519785in}}%
\pgfpathlineto{\pgfqpoint{1.513737in}{1.520890in}}%
\pgfpathlineto{\pgfqpoint{1.514821in}{1.521205in}}%
\pgfpathlineto{\pgfqpoint{1.517831in}{1.522310in}}%
\pgfpathlineto{\pgfqpoint{1.518933in}{1.522862in}}%
\pgfpathlineto{\pgfqpoint{1.522102in}{1.523966in}}%
\pgfpathlineto{\pgfqpoint{1.523186in}{1.524361in}}%
\pgfpathlineto{\pgfqpoint{1.526541in}{1.525465in}}%
\pgfpathlineto{\pgfqpoint{1.527279in}{1.525623in}}%
\pgfpathlineto{\pgfqpoint{1.532849in}{1.526727in}}%
\pgfpathlineto{\pgfqpoint{1.533653in}{1.526846in}}%
\pgfpathlineto{\pgfqpoint{1.537532in}{1.527950in}}%
\pgfpathlineto{\pgfqpoint{1.538541in}{1.528384in}}%
\pgfpathlineto{\pgfqpoint{1.542672in}{1.529488in}}%
\pgfpathlineto{\pgfqpoint{1.543635in}{1.529764in}}%
\pgfpathlineto{\pgfqpoint{1.548691in}{1.530869in}}%
\pgfpathlineto{\pgfqpoint{1.549448in}{1.531105in}}%
\pgfpathlineto{\pgfqpoint{1.554700in}{1.532210in}}%
\pgfpathlineto{\pgfqpoint{1.555803in}{1.532525in}}%
\pgfpathlineto{\pgfqpoint{1.561420in}{1.533630in}}%
\pgfpathlineto{\pgfqpoint{1.562504in}{1.534064in}}%
\pgfpathlineto{\pgfqpoint{1.568738in}{1.535168in}}%
\pgfpathlineto{\pgfqpoint{1.569831in}{1.535444in}}%
\pgfpathlineto{\pgfqpoint{1.573738in}{1.536548in}}%
\pgfpathlineto{\pgfqpoint{1.574644in}{1.536746in}}%
\pgfpathlineto{\pgfqpoint{1.580794in}{1.537850in}}%
\pgfpathlineto{\pgfqpoint{1.581672in}{1.538244in}}%
\pgfpathlineto{\pgfqpoint{1.586411in}{1.539349in}}%
\pgfpathlineto{\pgfqpoint{1.587476in}{1.539862in}}%
\pgfpathlineto{\pgfqpoint{1.592383in}{1.540966in}}%
\pgfpathlineto{\pgfqpoint{1.593299in}{1.541360in}}%
\pgfpathlineto{\pgfqpoint{1.598953in}{1.542465in}}%
\pgfpathlineto{\pgfqpoint{1.599888in}{1.542701in}}%
\pgfpathlineto{\pgfqpoint{1.603392in}{1.543806in}}%
\pgfpathlineto{\pgfqpoint{1.604261in}{1.544161in}}%
\pgfpathlineto{\pgfqpoint{1.608196in}{1.545226in}}%
\pgfpathlineto{\pgfqpoint{1.609205in}{1.545462in}}%
\pgfpathlineto{\pgfqpoint{1.613701in}{1.546567in}}%
\pgfpathlineto{\pgfqpoint{1.614420in}{1.546803in}}%
\pgfpathlineto{\pgfqpoint{1.620252in}{1.547908in}}%
\pgfpathlineto{\pgfqpoint{1.621355in}{1.548145in}}%
\pgfpathlineto{\pgfqpoint{1.626654in}{1.549249in}}%
\pgfpathlineto{\pgfqpoint{1.627748in}{1.549564in}}%
\pgfpathlineto{\pgfqpoint{1.634206in}{1.550669in}}%
\pgfpathlineto{\pgfqpoint{1.635178in}{1.550945in}}%
\pgfpathlineto{\pgfqpoint{1.639580in}{1.552049in}}%
\pgfpathlineto{\pgfqpoint{1.640477in}{1.552365in}}%
\pgfpathlineto{\pgfqpoint{1.640664in}{1.552365in}}%
\pgfpathlineto{\pgfqpoint{1.645122in}{1.553469in}}%
\pgfpathlineto{\pgfqpoint{1.646047in}{1.553666in}}%
\pgfpathlineto{\pgfqpoint{1.650168in}{1.554771in}}%
\pgfpathlineto{\pgfqpoint{1.650785in}{1.554889in}}%
\pgfpathlineto{\pgfqpoint{1.651262in}{1.554889in}}%
\pgfpathlineto{\pgfqpoint{1.656094in}{1.555994in}}%
\pgfpathlineto{\pgfqpoint{1.657131in}{1.556151in}}%
\pgfpathlineto{\pgfqpoint{1.664645in}{1.557256in}}%
\pgfpathlineto{\pgfqpoint{1.665636in}{1.557690in}}%
\pgfpathlineto{\pgfqpoint{1.669916in}{1.558794in}}%
\pgfpathlineto{\pgfqpoint{1.670748in}{1.559031in}}%
\pgfpathlineto{\pgfqpoint{1.676468in}{1.560135in}}%
\pgfpathlineto{\pgfqpoint{1.676954in}{1.560253in}}%
\pgfpathlineto{\pgfqpoint{1.677384in}{1.560253in}}%
\pgfpathlineto{\pgfqpoint{1.682505in}{1.561358in}}%
\pgfpathlineto{\pgfqpoint{1.683552in}{1.561476in}}%
\pgfpathlineto{\pgfqpoint{1.689038in}{1.562581in}}%
\pgfpathlineto{\pgfqpoint{1.689889in}{1.562738in}}%
\pgfpathlineto{\pgfqpoint{1.697048in}{1.563843in}}%
\pgfpathlineto{\pgfqpoint{1.697991in}{1.564040in}}%
\pgfpathlineto{\pgfqpoint{1.703954in}{1.565144in}}%
\pgfpathlineto{\pgfqpoint{1.705010in}{1.565420in}}%
\pgfpathlineto{\pgfqpoint{1.705057in}{1.565420in}}%
\pgfpathlineto{\pgfqpoint{1.712655in}{1.566525in}}%
\pgfpathlineto{\pgfqpoint{1.713749in}{1.566722in}}%
\pgfpathlineto{\pgfqpoint{1.720852in}{1.567826in}}%
\pgfpathlineto{\pgfqpoint{1.721870in}{1.568024in}}%
\pgfpathlineto{\pgfqpoint{1.721945in}{1.568024in}}%
\pgfpathlineto{\pgfqpoint{1.726992in}{1.569128in}}%
\pgfpathlineto{\pgfqpoint{1.727870in}{1.569325in}}%
\pgfpathlineto{\pgfqpoint{1.734384in}{1.570430in}}%
\pgfpathlineto{\pgfqpoint{1.735328in}{1.570666in}}%
\pgfpathlineto{\pgfqpoint{1.735366in}{1.570666in}}%
\pgfpathlineto{\pgfqpoint{1.740814in}{1.571771in}}%
\pgfpathlineto{\pgfqpoint{1.741534in}{1.571889in}}%
\pgfpathlineto{\pgfqpoint{1.741908in}{1.571889in}}%
\pgfpathlineto{\pgfqpoint{1.747142in}{1.572993in}}%
\pgfpathlineto{\pgfqpoint{1.748244in}{1.573230in}}%
\pgfpathlineto{\pgfqpoint{1.752936in}{1.574334in}}%
\pgfpathlineto{\pgfqpoint{1.753590in}{1.574571in}}%
\pgfpathlineto{\pgfqpoint{1.759787in}{1.575675in}}%
\pgfpathlineto{\pgfqpoint{1.760394in}{1.575754in}}%
\pgfpathlineto{\pgfqpoint{1.766731in}{1.576859in}}%
\pgfpathlineto{\pgfqpoint{1.767357in}{1.577056in}}%
\pgfpathlineto{\pgfqpoint{1.767731in}{1.577056in}}%
\pgfpathlineto{\pgfqpoint{1.776301in}{1.578160in}}%
\pgfpathlineto{\pgfqpoint{1.777282in}{1.578397in}}%
\pgfpathlineto{\pgfqpoint{1.784506in}{1.579501in}}%
\pgfpathlineto{\pgfqpoint{1.785581in}{1.579777in}}%
\pgfpathlineto{\pgfqpoint{1.792955in}{1.580882in}}%
\pgfpathlineto{\pgfqpoint{1.793656in}{1.581000in}}%
\pgfpathlineto{\pgfqpoint{1.803768in}{1.582105in}}%
\pgfpathlineto{\pgfqpoint{1.804600in}{1.582381in}}%
\pgfpathlineto{\pgfqpoint{1.814021in}{1.583485in}}%
\pgfpathlineto{\pgfqpoint{1.814619in}{1.583603in}}%
\pgfpathlineto{\pgfqpoint{1.815114in}{1.583603in}}%
\pgfpathlineto{\pgfqpoint{1.823638in}{1.584708in}}%
\pgfpathlineto{\pgfqpoint{1.824442in}{1.584826in}}%
\pgfpathlineto{\pgfqpoint{1.832367in}{1.585930in}}%
\pgfpathlineto{\pgfqpoint{1.833199in}{1.586049in}}%
\pgfpathlineto{\pgfqpoint{1.843825in}{1.587153in}}%
\pgfpathlineto{\pgfqpoint{1.844274in}{1.587429in}}%
\pgfpathlineto{\pgfqpoint{1.844526in}{1.587429in}}%
\pgfpathlineto{\pgfqpoint{1.859302in}{1.588534in}}%
\pgfpathlineto{\pgfqpoint{1.860236in}{1.588731in}}%
\pgfpathlineto{\pgfqpoint{1.867451in}{1.589835in}}%
\pgfpathlineto{\pgfqpoint{1.868479in}{1.589954in}}%
\pgfpathlineto{\pgfqpoint{1.878330in}{1.591058in}}%
\pgfpathlineto{\pgfqpoint{1.879386in}{1.591255in}}%
\pgfpathlineto{\pgfqpoint{1.888498in}{1.592360in}}%
\pgfpathlineto{\pgfqpoint{1.889545in}{1.592517in}}%
\pgfpathlineto{\pgfqpoint{1.899265in}{1.593622in}}%
\pgfpathlineto{\pgfqpoint{1.900340in}{1.593780in}}%
\pgfpathlineto{\pgfqpoint{1.915340in}{1.594884in}}%
\pgfpathlineto{\pgfqpoint{1.915901in}{1.594963in}}%
\pgfpathlineto{\pgfqpoint{1.933125in}{1.596067in}}%
\pgfpathlineto{\pgfqpoint{1.934144in}{1.596343in}}%
\pgfpathlineto{\pgfqpoint{1.934228in}{1.596343in}}%
\pgfpathlineto{\pgfqpoint{1.946658in}{1.597448in}}%
\pgfpathlineto{\pgfqpoint{1.947508in}{1.597605in}}%
\pgfpathlineto{\pgfqpoint{1.962985in}{1.598710in}}%
\pgfpathlineto{\pgfqpoint{1.963892in}{1.598789in}}%
\pgfpathlineto{\pgfqpoint{1.983041in}{1.599893in}}%
\pgfpathlineto{\pgfqpoint{1.983499in}{1.599972in}}%
\pgfpathlineto{\pgfqpoint{1.984041in}{1.599972in}}%
\pgfpathlineto{\pgfqpoint{2.007079in}{1.601076in}}%
\pgfpathlineto{\pgfqpoint{2.007659in}{1.601155in}}%
\pgfpathlineto{\pgfqpoint{2.007892in}{1.601155in}}%
\pgfpathlineto{\pgfqpoint{2.033126in}{1.601944in}}%
\pgfpathlineto{\pgfqpoint{2.033126in}{1.601944in}}%
\pgfusepath{stroke}%
\end{pgfscope}%
\begin{pgfscope}%
\pgfsetrectcap%
\pgfsetmiterjoin%
\pgfsetlinewidth{0.803000pt}%
\definecolor{currentstroke}{rgb}{0.000000,0.000000,0.000000}%
\pgfsetstrokecolor{currentstroke}%
\pgfsetdash{}{0pt}%
\pgfpathmoveto{\pgfqpoint{0.553581in}{0.499444in}}%
\pgfpathlineto{\pgfqpoint{0.553581in}{1.654444in}}%
\pgfusepath{stroke}%
\end{pgfscope}%
\begin{pgfscope}%
\pgfsetrectcap%
\pgfsetmiterjoin%
\pgfsetlinewidth{0.803000pt}%
\definecolor{currentstroke}{rgb}{0.000000,0.000000,0.000000}%
\pgfsetstrokecolor{currentstroke}%
\pgfsetdash{}{0pt}%
\pgfpathmoveto{\pgfqpoint{2.103581in}{0.499444in}}%
\pgfpathlineto{\pgfqpoint{2.103581in}{1.654444in}}%
\pgfusepath{stroke}%
\end{pgfscope}%
\begin{pgfscope}%
\pgfsetrectcap%
\pgfsetmiterjoin%
\pgfsetlinewidth{0.803000pt}%
\definecolor{currentstroke}{rgb}{0.000000,0.000000,0.000000}%
\pgfsetstrokecolor{currentstroke}%
\pgfsetdash{}{0pt}%
\pgfpathmoveto{\pgfqpoint{0.553581in}{0.499444in}}%
\pgfpathlineto{\pgfqpoint{2.103581in}{0.499444in}}%
\pgfusepath{stroke}%
\end{pgfscope}%
\begin{pgfscope}%
\pgfsetrectcap%
\pgfsetmiterjoin%
\pgfsetlinewidth{0.803000pt}%
\definecolor{currentstroke}{rgb}{0.000000,0.000000,0.000000}%
\pgfsetstrokecolor{currentstroke}%
\pgfsetdash{}{0pt}%
\pgfpathmoveto{\pgfqpoint{0.553581in}{1.654444in}}%
\pgfpathlineto{\pgfqpoint{2.103581in}{1.654444in}}%
\pgfusepath{stroke}%
\end{pgfscope}%
\begin{pgfscope}%
\pgfsetbuttcap%
\pgfsetmiterjoin%
\definecolor{currentfill}{rgb}{1.000000,1.000000,1.000000}%
\pgfsetfillcolor{currentfill}%
\pgfsetfillopacity{0.800000}%
\pgfsetlinewidth{1.003750pt}%
\definecolor{currentstroke}{rgb}{0.800000,0.800000,0.800000}%
\pgfsetstrokecolor{currentstroke}%
\pgfsetstrokeopacity{0.800000}%
\pgfsetdash{}{0pt}%
\pgfpathmoveto{\pgfqpoint{0.840525in}{0.568889in}}%
\pgfpathlineto{\pgfqpoint{2.006358in}{0.568889in}}%
\pgfpathquadraticcurveto{\pgfqpoint{2.034136in}{0.568889in}}{\pgfqpoint{2.034136in}{0.596666in}}%
\pgfpathlineto{\pgfqpoint{2.034136in}{0.791111in}}%
\pgfpathquadraticcurveto{\pgfqpoint{2.034136in}{0.818888in}}{\pgfqpoint{2.006358in}{0.818888in}}%
\pgfpathlineto{\pgfqpoint{0.840525in}{0.818888in}}%
\pgfpathquadraticcurveto{\pgfqpoint{0.812747in}{0.818888in}}{\pgfqpoint{0.812747in}{0.791111in}}%
\pgfpathlineto{\pgfqpoint{0.812747in}{0.596666in}}%
\pgfpathquadraticcurveto{\pgfqpoint{0.812747in}{0.568889in}}{\pgfqpoint{0.840525in}{0.568889in}}%
\pgfpathlineto{\pgfqpoint{0.840525in}{0.568889in}}%
\pgfpathclose%
\pgfusepath{stroke,fill}%
\end{pgfscope}%
\begin{pgfscope}%
\pgfsetrectcap%
\pgfsetroundjoin%
\pgfsetlinewidth{1.505625pt}%
\definecolor{currentstroke}{rgb}{0.000000,0.000000,0.000000}%
\pgfsetstrokecolor{currentstroke}%
\pgfsetdash{}{0pt}%
\pgfpathmoveto{\pgfqpoint{0.868303in}{0.707777in}}%
\pgfpathlineto{\pgfqpoint{1.007192in}{0.707777in}}%
\pgfpathlineto{\pgfqpoint{1.146081in}{0.707777in}}%
\pgfusepath{stroke}%
\end{pgfscope}%
\begin{pgfscope}%
\definecolor{textcolor}{rgb}{0.000000,0.000000,0.000000}%
\pgfsetstrokecolor{textcolor}%
\pgfsetfillcolor{textcolor}%
\pgftext[x=1.257192in,y=0.659166in,left,base]{\color{textcolor}\rmfamily\fontsize{10.000000}{12.000000}\selectfont AUC 0.774)}%
\end{pgfscope}%
\end{pgfpicture}%
\makeatother%
\endgroup%

	
&
	\vskip 0pt
	\begin{tabular}{cc|c|c|}
	&\multicolumn{1}{c}{}& \multicolumn{2}{c}{Prediction} \cr
	&\multicolumn{1}{c}{} & \multicolumn{1}{c}{N} & \multicolumn{1}{c}{P} \cr\cline{3-4}
	\multirow{2}{*}{\rotatebox[origin=c]{90}{Actual}}&N &
118,776 & 31,995
	\vrule width 0pt height 10pt depth 2pt \cr\cline{3-4}
	&P & 
10,572 & 16,049
	\vrule width 0pt height 10pt depth 2pt \cr\cline{3-4}
	\end{tabular}

	\hfil\begin{tabular}{ll}
	\cr
0.334 & Precision \cr	0.603 & Recall \cr	0.430 & F1 \cr
\end{tabular}

\cr
\end{tabular}
} % End parbox

\

Mapping $\Delta FP/\Delta TP$ as a function of $p$ for this model, we see that it equals $\omega = 2$ when $p = 0.720$.  Using a linear transformation, we can map $0.720$ to $0.5$, keeping 0 at 0, to get transformed model output with the decision threshold where $\Delta FP/\Delta TP = 2$.  The ROC curve and its AUC are invariant under such a transformation.  

\

%%%
\parbox{\linewidth}{
%{\bf Balanced Random Forest model, Hard features, No Tomek, $\alpha = 2/3$}

\noindent\begin{tabular}{@{\hspace{-6pt}}p{2.3in} @{\hspace{-6pt}}p{2.0in} p{1.8in}}
	\vskip 0pt
	\qquad $\Delta FP/\Delta TP$ as a function of $p$
	
%	%% Creator: Matplotlib, PGF backend
%%
%% To include the figure in your LaTeX document, write
%%   \input{<filename>.pgf}
%%
%% Make sure the required packages are loaded in your preamble
%%   \usepackage{pgf}
%%
%% Also ensure that all the required font packages are loaded; for instance,
%% the lmodern package is sometimes necessary when using math font.
%%   \usepackage{lmodern}
%%
%% Figures using additional raster images can only be included by \input if
%% they are in the same directory as the main LaTeX file. For loading figures
%% from other directories you can use the `import` package
%%   \usepackage{import}
%%
%% and then include the figures with
%%   \import{<path to file>}{<filename>.pgf}
%%
%% Matplotlib used the following preamble
%%   
%%   \usepackage{fontspec}
%%   \makeatletter\@ifpackageloaded{underscore}{}{\usepackage[strings]{underscore}}\makeatother
%%
\begingroup%
\makeatletter%
\begin{pgfpicture}%
\pgfpathrectangle{\pgfpointorigin}{\pgfqpoint{2.247807in}{1.754444in}}%
\pgfusepath{use as bounding box, clip}%
\begin{pgfscope}%
\pgfsetbuttcap%
\pgfsetmiterjoin%
\definecolor{currentfill}{rgb}{1.000000,1.000000,1.000000}%
\pgfsetfillcolor{currentfill}%
\pgfsetlinewidth{0.000000pt}%
\definecolor{currentstroke}{rgb}{1.000000,1.000000,1.000000}%
\pgfsetstrokecolor{currentstroke}%
\pgfsetdash{}{0pt}%
\pgfpathmoveto{\pgfqpoint{0.000000in}{0.000000in}}%
\pgfpathlineto{\pgfqpoint{2.247807in}{0.000000in}}%
\pgfpathlineto{\pgfqpoint{2.247807in}{1.754444in}}%
\pgfpathlineto{\pgfqpoint{0.000000in}{1.754444in}}%
\pgfpathlineto{\pgfqpoint{0.000000in}{0.000000in}}%
\pgfpathclose%
\pgfusepath{fill}%
\end{pgfscope}%
\begin{pgfscope}%
\pgfsetbuttcap%
\pgfsetmiterjoin%
\definecolor{currentfill}{rgb}{1.000000,1.000000,1.000000}%
\pgfsetfillcolor{currentfill}%
\pgfsetlinewidth{0.000000pt}%
\definecolor{currentstroke}{rgb}{0.000000,0.000000,0.000000}%
\pgfsetstrokecolor{currentstroke}%
\pgfsetstrokeopacity{0.000000}%
\pgfsetdash{}{0pt}%
\pgfpathmoveto{\pgfqpoint{0.530556in}{0.499444in}}%
\pgfpathlineto{\pgfqpoint{2.080556in}{0.499444in}}%
\pgfpathlineto{\pgfqpoint{2.080556in}{1.654444in}}%
\pgfpathlineto{\pgfqpoint{0.530556in}{1.654444in}}%
\pgfpathlineto{\pgfqpoint{0.530556in}{0.499444in}}%
\pgfpathclose%
\pgfusepath{fill}%
\end{pgfscope}%
\begin{pgfscope}%
\pgfsetbuttcap%
\pgfsetroundjoin%
\definecolor{currentfill}{rgb}{0.000000,0.000000,0.000000}%
\pgfsetfillcolor{currentfill}%
\pgfsetlinewidth{0.803000pt}%
\definecolor{currentstroke}{rgb}{0.000000,0.000000,0.000000}%
\pgfsetstrokecolor{currentstroke}%
\pgfsetdash{}{0pt}%
\pgfsys@defobject{currentmarker}{\pgfqpoint{0.000000in}{-0.048611in}}{\pgfqpoint{0.000000in}{0.000000in}}{%
\pgfpathmoveto{\pgfqpoint{0.000000in}{0.000000in}}%
\pgfpathlineto{\pgfqpoint{0.000000in}{-0.048611in}}%
\pgfusepath{stroke,fill}%
}%
\begin{pgfscope}%
\pgfsys@transformshift{0.601010in}{0.499444in}%
\pgfsys@useobject{currentmarker}{}%
\end{pgfscope}%
\end{pgfscope}%
\begin{pgfscope}%
\definecolor{textcolor}{rgb}{0.000000,0.000000,0.000000}%
\pgfsetstrokecolor{textcolor}%
\pgfsetfillcolor{textcolor}%
\pgftext[x=0.601010in,y=0.402222in,,top]{\color{textcolor}\rmfamily\fontsize{10.000000}{12.000000}\selectfont 0.008}%
\end{pgfscope}%
\begin{pgfscope}%
\pgfsetbuttcap%
\pgfsetroundjoin%
\definecolor{currentfill}{rgb}{0.000000,0.000000,0.000000}%
\pgfsetfillcolor{currentfill}%
\pgfsetlinewidth{0.803000pt}%
\definecolor{currentstroke}{rgb}{0.000000,0.000000,0.000000}%
\pgfsetstrokecolor{currentstroke}%
\pgfsetdash{}{0pt}%
\pgfsys@defobject{currentmarker}{\pgfqpoint{0.000000in}{-0.048611in}}{\pgfqpoint{0.000000in}{0.000000in}}{%
\pgfpathmoveto{\pgfqpoint{0.000000in}{0.000000in}}%
\pgfpathlineto{\pgfqpoint{0.000000in}{-0.048611in}}%
\pgfusepath{stroke,fill}%
}%
\begin{pgfscope}%
\pgfsys@transformshift{2.024334in}{0.499444in}%
\pgfsys@useobject{currentmarker}{}%
\end{pgfscope}%
\end{pgfscope}%
\begin{pgfscope}%
\definecolor{textcolor}{rgb}{0.000000,0.000000,0.000000}%
\pgfsetstrokecolor{textcolor}%
\pgfsetfillcolor{textcolor}%
\pgftext[x=2.024334in,y=0.402222in,,top]{\color{textcolor}\rmfamily\fontsize{10.000000}{12.000000}\selectfont 0.97}%
\end{pgfscope}%
\begin{pgfscope}%
\definecolor{textcolor}{rgb}{0.000000,0.000000,0.000000}%
\pgfsetstrokecolor{textcolor}%
\pgfsetfillcolor{textcolor}%
\pgftext[x=1.305556in,y=0.223333in,,top]{\color{textcolor}\rmfamily\fontsize{10.000000}{12.000000}\selectfont \(\displaystyle p\)}%
\end{pgfscope}%
\begin{pgfscope}%
\pgfsetbuttcap%
\pgfsetroundjoin%
\definecolor{currentfill}{rgb}{0.000000,0.000000,0.000000}%
\pgfsetfillcolor{currentfill}%
\pgfsetlinewidth{0.803000pt}%
\definecolor{currentstroke}{rgb}{0.000000,0.000000,0.000000}%
\pgfsetstrokecolor{currentstroke}%
\pgfsetdash{}{0pt}%
\pgfsys@defobject{currentmarker}{\pgfqpoint{-0.048611in}{0.000000in}}{\pgfqpoint{-0.000000in}{0.000000in}}{%
\pgfpathmoveto{\pgfqpoint{-0.000000in}{0.000000in}}%
\pgfpathlineto{\pgfqpoint{-0.048611in}{0.000000in}}%
\pgfusepath{stroke,fill}%
}%
\begin{pgfscope}%
\pgfsys@transformshift{0.530556in}{0.540910in}%
\pgfsys@useobject{currentmarker}{}%
\end{pgfscope}%
\end{pgfscope}%
\begin{pgfscope}%
\definecolor{textcolor}{rgb}{0.000000,0.000000,0.000000}%
\pgfsetstrokecolor{textcolor}%
\pgfsetfillcolor{textcolor}%
\pgftext[x=0.363889in, y=0.492715in, left, base]{\color{textcolor}\rmfamily\fontsize{10.000000}{12.000000}\selectfont \(\displaystyle {0}\)}%
\end{pgfscope}%
\begin{pgfscope}%
\pgfsetbuttcap%
\pgfsetroundjoin%
\definecolor{currentfill}{rgb}{0.000000,0.000000,0.000000}%
\pgfsetfillcolor{currentfill}%
\pgfsetlinewidth{0.803000pt}%
\definecolor{currentstroke}{rgb}{0.000000,0.000000,0.000000}%
\pgfsetstrokecolor{currentstroke}%
\pgfsetdash{}{0pt}%
\pgfsys@defobject{currentmarker}{\pgfqpoint{-0.048611in}{0.000000in}}{\pgfqpoint{-0.000000in}{0.000000in}}{%
\pgfpathmoveto{\pgfqpoint{-0.000000in}{0.000000in}}%
\pgfpathlineto{\pgfqpoint{-0.048611in}{0.000000in}}%
\pgfusepath{stroke,fill}%
}%
\begin{pgfscope}%
\pgfsys@transformshift{0.530556in}{0.946322in}%
\pgfsys@useobject{currentmarker}{}%
\end{pgfscope}%
\end{pgfscope}%
\begin{pgfscope}%
\definecolor{textcolor}{rgb}{0.000000,0.000000,0.000000}%
\pgfsetstrokecolor{textcolor}%
\pgfsetfillcolor{textcolor}%
\pgftext[x=0.294444in, y=0.898128in, left, base]{\color{textcolor}\rmfamily\fontsize{10.000000}{12.000000}\selectfont \(\displaystyle {20}\)}%
\end{pgfscope}%
\begin{pgfscope}%
\pgfsetbuttcap%
\pgfsetroundjoin%
\definecolor{currentfill}{rgb}{0.000000,0.000000,0.000000}%
\pgfsetfillcolor{currentfill}%
\pgfsetlinewidth{0.803000pt}%
\definecolor{currentstroke}{rgb}{0.000000,0.000000,0.000000}%
\pgfsetstrokecolor{currentstroke}%
\pgfsetdash{}{0pt}%
\pgfsys@defobject{currentmarker}{\pgfqpoint{-0.048611in}{0.000000in}}{\pgfqpoint{-0.000000in}{0.000000in}}{%
\pgfpathmoveto{\pgfqpoint{-0.000000in}{0.000000in}}%
\pgfpathlineto{\pgfqpoint{-0.048611in}{0.000000in}}%
\pgfusepath{stroke,fill}%
}%
\begin{pgfscope}%
\pgfsys@transformshift{0.530556in}{1.351735in}%
\pgfsys@useobject{currentmarker}{}%
\end{pgfscope}%
\end{pgfscope}%
\begin{pgfscope}%
\definecolor{textcolor}{rgb}{0.000000,0.000000,0.000000}%
\pgfsetstrokecolor{textcolor}%
\pgfsetfillcolor{textcolor}%
\pgftext[x=0.294444in, y=1.303540in, left, base]{\color{textcolor}\rmfamily\fontsize{10.000000}{12.000000}\selectfont \(\displaystyle {40}\)}%
\end{pgfscope}%
\begin{pgfscope}%
\definecolor{textcolor}{rgb}{0.000000,0.000000,0.000000}%
\pgfsetstrokecolor{textcolor}%
\pgfsetfillcolor{textcolor}%
\pgftext[x=0.238889in,y=1.076944in,,bottom,rotate=90.000000]{\color{textcolor}\rmfamily\fontsize{10.000000}{12.000000}\selectfont \(\displaystyle \Delta\)FP/\(\displaystyle \Delta\)TP}%
\end{pgfscope}%
\begin{pgfscope}%
\pgfpathrectangle{\pgfqpoint{0.530556in}{0.499444in}}{\pgfqpoint{1.550000in}{1.155000in}}%
\pgfusepath{clip}%
\pgfsetrectcap%
\pgfsetroundjoin%
\pgfsetlinewidth{1.505625pt}%
\definecolor{currentstroke}{rgb}{0.000000,0.000000,0.000000}%
\pgfsetstrokecolor{currentstroke}%
\pgfsetdash{}{0pt}%
\pgfpathmoveto{\pgfqpoint{0.601010in}{1.601944in}}%
\pgfpathlineto{\pgfqpoint{0.615243in}{1.561939in}}%
\pgfpathlineto{\pgfqpoint{0.629477in}{1.519987in}}%
\pgfpathlineto{\pgfqpoint{0.643710in}{1.480900in}}%
\pgfpathlineto{\pgfqpoint{0.657943in}{1.443667in}}%
\pgfpathlineto{\pgfqpoint{0.672176in}{1.408434in}}%
\pgfpathlineto{\pgfqpoint{0.686410in}{1.330371in}}%
\pgfpathlineto{\pgfqpoint{0.700643in}{1.267651in}}%
\pgfpathlineto{\pgfqpoint{0.714876in}{1.207433in}}%
\pgfpathlineto{\pgfqpoint{0.729109in}{1.151946in}}%
\pgfpathlineto{\pgfqpoint{0.743343in}{1.099513in}}%
\pgfpathlineto{\pgfqpoint{0.757576in}{1.052935in}}%
\pgfpathlineto{\pgfqpoint{0.771809in}{1.014007in}}%
\pgfpathlineto{\pgfqpoint{0.786042in}{0.979069in}}%
\pgfpathlineto{\pgfqpoint{0.800276in}{0.948758in}}%
\pgfpathlineto{\pgfqpoint{0.814509in}{0.922183in}}%
\pgfpathlineto{\pgfqpoint{0.828742in}{0.898615in}}%
\pgfpathlineto{\pgfqpoint{0.842975in}{0.877569in}}%
\pgfpathlineto{\pgfqpoint{0.857209in}{0.857331in}}%
\pgfpathlineto{\pgfqpoint{0.871442in}{0.838338in}}%
\pgfpathlineto{\pgfqpoint{0.885675in}{0.822090in}}%
\pgfpathlineto{\pgfqpoint{0.899908in}{0.807805in}}%
\pgfpathlineto{\pgfqpoint{0.914142in}{0.794292in}}%
\pgfpathlineto{\pgfqpoint{0.928375in}{0.781980in}}%
\pgfpathlineto{\pgfqpoint{0.942608in}{0.770643in}}%
\pgfpathlineto{\pgfqpoint{0.956841in}{0.760249in}}%
\pgfpathlineto{\pgfqpoint{0.971075in}{0.750603in}}%
\pgfpathlineto{\pgfqpoint{0.985308in}{0.741532in}}%
\pgfpathlineto{\pgfqpoint{0.999541in}{0.734003in}}%
\pgfpathlineto{\pgfqpoint{1.013774in}{0.727075in}}%
\pgfpathlineto{\pgfqpoint{1.028007in}{0.720197in}}%
\pgfpathlineto{\pgfqpoint{1.042241in}{0.713595in}}%
\pgfpathlineto{\pgfqpoint{1.056474in}{0.707399in}}%
\pgfpathlineto{\pgfqpoint{1.070707in}{0.701829in}}%
\pgfpathlineto{\pgfqpoint{1.084940in}{0.696135in}}%
\pgfpathlineto{\pgfqpoint{1.099174in}{0.690924in}}%
\pgfpathlineto{\pgfqpoint{1.113407in}{0.685834in}}%
\pgfpathlineto{\pgfqpoint{1.127640in}{0.681070in}}%
\pgfpathlineto{\pgfqpoint{1.141873in}{0.676223in}}%
\pgfpathlineto{\pgfqpoint{1.156107in}{0.671763in}}%
\pgfpathlineto{\pgfqpoint{1.170340in}{0.667671in}}%
\pgfpathlineto{\pgfqpoint{1.184573in}{0.663708in}}%
\pgfpathlineto{\pgfqpoint{1.198806in}{0.659965in}}%
\pgfpathlineto{\pgfqpoint{1.213040in}{0.656061in}}%
\pgfpathlineto{\pgfqpoint{1.227273in}{0.652540in}}%
\pgfpathlineto{\pgfqpoint{1.241506in}{0.648888in}}%
\pgfpathlineto{\pgfqpoint{1.255739in}{0.645356in}}%
\pgfpathlineto{\pgfqpoint{1.269973in}{0.641959in}}%
\pgfpathlineto{\pgfqpoint{1.284206in}{0.638769in}}%
\pgfpathlineto{\pgfqpoint{1.298439in}{0.635617in}}%
\pgfpathlineto{\pgfqpoint{1.312672in}{0.632377in}}%
\pgfpathlineto{\pgfqpoint{1.326906in}{0.629137in}}%
\pgfpathlineto{\pgfqpoint{1.341139in}{0.625924in}}%
\pgfpathlineto{\pgfqpoint{1.355372in}{0.623014in}}%
\pgfpathlineto{\pgfqpoint{1.369605in}{0.620208in}}%
\pgfpathlineto{\pgfqpoint{1.383839in}{0.617626in}}%
\pgfpathlineto{\pgfqpoint{1.398072in}{0.615168in}}%
\pgfpathlineto{\pgfqpoint{1.412305in}{0.612747in}}%
\pgfpathlineto{\pgfqpoint{1.426538in}{0.610317in}}%
\pgfpathlineto{\pgfqpoint{1.440771in}{0.607976in}}%
\pgfpathlineto{\pgfqpoint{1.455005in}{0.605820in}}%
\pgfpathlineto{\pgfqpoint{1.469238in}{0.603844in}}%
\pgfpathlineto{\pgfqpoint{1.483471in}{0.602020in}}%
\pgfpathlineto{\pgfqpoint{1.497704in}{0.600118in}}%
\pgfpathlineto{\pgfqpoint{1.511938in}{0.598171in}}%
\pgfpathlineto{\pgfqpoint{1.526171in}{0.596238in}}%
\pgfpathlineto{\pgfqpoint{1.540404in}{0.594428in}}%
\pgfpathlineto{\pgfqpoint{1.554637in}{0.592637in}}%
\pgfpathlineto{\pgfqpoint{1.568871in}{0.590944in}}%
\pgfpathlineto{\pgfqpoint{1.583104in}{0.589243in}}%
\pgfpathlineto{\pgfqpoint{1.597337in}{0.587562in}}%
\pgfpathlineto{\pgfqpoint{1.611570in}{0.585881in}}%
\pgfpathlineto{\pgfqpoint{1.625804in}{0.584099in}}%
\pgfpathlineto{\pgfqpoint{1.640037in}{0.582384in}}%
\pgfpathlineto{\pgfqpoint{1.654270in}{0.580793in}}%
\pgfpathlineto{\pgfqpoint{1.668503in}{0.579207in}}%
\pgfpathlineto{\pgfqpoint{1.682737in}{0.577577in}}%
\pgfpathlineto{\pgfqpoint{1.696970in}{0.575975in}}%
\pgfpathlineto{\pgfqpoint{1.711203in}{0.574368in}}%
\pgfpathlineto{\pgfqpoint{1.725436in}{0.572829in}}%
\pgfpathlineto{\pgfqpoint{1.739670in}{0.571283in}}%
\pgfpathlineto{\pgfqpoint{1.753903in}{0.569752in}}%
\pgfpathlineto{\pgfqpoint{1.768136in}{0.568383in}}%
\pgfpathlineto{\pgfqpoint{1.782369in}{0.567039in}}%
\pgfpathlineto{\pgfqpoint{1.796603in}{0.565658in}}%
\pgfpathlineto{\pgfqpoint{1.810836in}{0.564371in}}%
\pgfpathlineto{\pgfqpoint{1.825069in}{0.563124in}}%
\pgfpathlineto{\pgfqpoint{1.839302in}{0.561940in}}%
\pgfpathlineto{\pgfqpoint{1.853535in}{0.560794in}}%
\pgfpathlineto{\pgfqpoint{1.867769in}{0.559649in}}%
\pgfpathlineto{\pgfqpoint{1.882002in}{0.558559in}}%
\pgfpathlineto{\pgfqpoint{1.896235in}{0.557501in}}%
\pgfpathlineto{\pgfqpoint{1.910468in}{0.556449in}}%
\pgfpathlineto{\pgfqpoint{1.924702in}{0.555475in}}%
\pgfpathlineto{\pgfqpoint{1.938935in}{0.554594in}}%
\pgfpathlineto{\pgfqpoint{1.953168in}{0.553719in}}%
\pgfpathlineto{\pgfqpoint{1.967401in}{0.553250in}}%
\pgfpathlineto{\pgfqpoint{1.981635in}{0.552777in}}%
\pgfpathlineto{\pgfqpoint{1.995868in}{0.552333in}}%
\pgfpathlineto{\pgfqpoint{2.010101in}{0.551944in}}%
\pgfusepath{stroke}%
\end{pgfscope}%
\begin{pgfscope}%
\pgfpathrectangle{\pgfqpoint{0.530556in}{0.499444in}}{\pgfqpoint{1.550000in}{1.155000in}}%
\pgfusepath{clip}%
\pgfsetbuttcap%
\pgfsetroundjoin%
\pgfsetlinewidth{1.505625pt}%
\definecolor{currentstroke}{rgb}{0.000000,0.000000,0.000000}%
\pgfsetstrokecolor{currentstroke}%
\pgfsetdash{{5.550000pt}{2.400000pt}}{0.000000pt}%
\pgfpathmoveto{\pgfqpoint{0.530556in}{0.581451in}}%
\pgfpathlineto{\pgfqpoint{2.080556in}{0.581451in}}%
\pgfusepath{stroke}%
\end{pgfscope}%
\begin{pgfscope}%
\pgfpathrectangle{\pgfqpoint{0.530556in}{0.499444in}}{\pgfqpoint{1.550000in}{1.155000in}}%
\pgfusepath{clip}%
\pgfsetrectcap%
\pgfsetroundjoin%
\pgfsetlinewidth{1.505625pt}%
\definecolor{currentstroke}{rgb}{0.121569,0.466667,0.705882}%
\pgfsetstrokecolor{currentstroke}%
\pgfsetdash{}{0pt}%
\pgfpathmoveto{\pgfqpoint{1.654270in}{0.581451in}}%
\pgfusepath{stroke}%
\end{pgfscope}%
\begin{pgfscope}%
\pgfpathrectangle{\pgfqpoint{0.530556in}{0.499444in}}{\pgfqpoint{1.550000in}{1.155000in}}%
\pgfusepath{clip}%
\pgfsetbuttcap%
\pgfsetroundjoin%
\definecolor{currentfill}{rgb}{0.000000,0.000000,0.000000}%
\pgfsetfillcolor{currentfill}%
\pgfsetlinewidth{1.003750pt}%
\definecolor{currentstroke}{rgb}{0.000000,0.000000,0.000000}%
\pgfsetstrokecolor{currentstroke}%
\pgfsetdash{}{0pt}%
\pgfsys@defobject{currentmarker}{\pgfqpoint{-0.041667in}{-0.041667in}}{\pgfqpoint{0.041667in}{0.041667in}}{%
\pgfpathmoveto{\pgfqpoint{0.000000in}{-0.041667in}}%
\pgfpathcurveto{\pgfqpoint{0.011050in}{-0.041667in}}{\pgfqpoint{0.021649in}{-0.037276in}}{\pgfqpoint{0.029463in}{-0.029463in}}%
\pgfpathcurveto{\pgfqpoint{0.037276in}{-0.021649in}}{\pgfqpoint{0.041667in}{-0.011050in}}{\pgfqpoint{0.041667in}{0.000000in}}%
\pgfpathcurveto{\pgfqpoint{0.041667in}{0.011050in}}{\pgfqpoint{0.037276in}{0.021649in}}{\pgfqpoint{0.029463in}{0.029463in}}%
\pgfpathcurveto{\pgfqpoint{0.021649in}{0.037276in}}{\pgfqpoint{0.011050in}{0.041667in}}{\pgfqpoint{0.000000in}{0.041667in}}%
\pgfpathcurveto{\pgfqpoint{-0.011050in}{0.041667in}}{\pgfqpoint{-0.021649in}{0.037276in}}{\pgfqpoint{-0.029463in}{0.029463in}}%
\pgfpathcurveto{\pgfqpoint{-0.037276in}{0.021649in}}{\pgfqpoint{-0.041667in}{0.011050in}}{\pgfqpoint{-0.041667in}{0.000000in}}%
\pgfpathcurveto{\pgfqpoint{-0.041667in}{-0.011050in}}{\pgfqpoint{-0.037276in}{-0.021649in}}{\pgfqpoint{-0.029463in}{-0.029463in}}%
\pgfpathcurveto{\pgfqpoint{-0.021649in}{-0.037276in}}{\pgfqpoint{-0.011050in}{-0.041667in}}{\pgfqpoint{0.000000in}{-0.041667in}}%
\pgfpathlineto{\pgfqpoint{0.000000in}{-0.041667in}}%
\pgfpathclose%
\pgfusepath{stroke,fill}%
}%
\begin{pgfscope}%
\pgfsys@transformshift{1.654270in}{0.581451in}%
\pgfsys@useobject{currentmarker}{}%
\end{pgfscope}%
\end{pgfscope}%
\begin{pgfscope}%
\pgfsetrectcap%
\pgfsetmiterjoin%
\pgfsetlinewidth{0.803000pt}%
\definecolor{currentstroke}{rgb}{0.000000,0.000000,0.000000}%
\pgfsetstrokecolor{currentstroke}%
\pgfsetdash{}{0pt}%
\pgfpathmoveto{\pgfqpoint{0.530556in}{0.499444in}}%
\pgfpathlineto{\pgfqpoint{0.530556in}{1.654444in}}%
\pgfusepath{stroke}%
\end{pgfscope}%
\begin{pgfscope}%
\pgfsetrectcap%
\pgfsetmiterjoin%
\pgfsetlinewidth{0.803000pt}%
\definecolor{currentstroke}{rgb}{0.000000,0.000000,0.000000}%
\pgfsetstrokecolor{currentstroke}%
\pgfsetdash{}{0pt}%
\pgfpathmoveto{\pgfqpoint{2.080556in}{0.499444in}}%
\pgfpathlineto{\pgfqpoint{2.080556in}{1.654444in}}%
\pgfusepath{stroke}%
\end{pgfscope}%
\begin{pgfscope}%
\pgfsetrectcap%
\pgfsetmiterjoin%
\pgfsetlinewidth{0.803000pt}%
\definecolor{currentstroke}{rgb}{0.000000,0.000000,0.000000}%
\pgfsetstrokecolor{currentstroke}%
\pgfsetdash{}{0pt}%
\pgfpathmoveto{\pgfqpoint{0.530556in}{0.499444in}}%
\pgfpathlineto{\pgfqpoint{2.080556in}{0.499444in}}%
\pgfusepath{stroke}%
\end{pgfscope}%
\begin{pgfscope}%
\pgfsetrectcap%
\pgfsetmiterjoin%
\pgfsetlinewidth{0.803000pt}%
\definecolor{currentstroke}{rgb}{0.000000,0.000000,0.000000}%
\pgfsetstrokecolor{currentstroke}%
\pgfsetdash{}{0pt}%
\pgfpathmoveto{\pgfqpoint{0.530556in}{1.654444in}}%
\pgfpathlineto{\pgfqpoint{2.080556in}{1.654444in}}%
\pgfusepath{stroke}%
\end{pgfscope}%
\begin{pgfscope}%
\pgfsetbuttcap%
\pgfsetmiterjoin%
\definecolor{currentfill}{rgb}{1.000000,1.000000,1.000000}%
\pgfsetfillcolor{currentfill}%
\pgfsetfillopacity{0.800000}%
\pgfsetlinewidth{1.003750pt}%
\definecolor{currentstroke}{rgb}{0.800000,0.800000,0.800000}%
\pgfsetstrokecolor{currentstroke}%
\pgfsetstrokeopacity{0.800000}%
\pgfsetdash{}{0pt}%
\pgfpathmoveto{\pgfqpoint{0.811987in}{1.126667in}}%
\pgfpathlineto{\pgfqpoint{1.983333in}{1.126667in}}%
\pgfpathquadraticcurveto{\pgfqpoint{2.011111in}{1.126667in}}{\pgfqpoint{2.011111in}{1.154444in}}%
\pgfpathlineto{\pgfqpoint{2.011111in}{1.557222in}}%
\pgfpathquadraticcurveto{\pgfqpoint{2.011111in}{1.585000in}}{\pgfqpoint{1.983333in}{1.585000in}}%
\pgfpathlineto{\pgfqpoint{0.811987in}{1.585000in}}%
\pgfpathquadraticcurveto{\pgfqpoint{0.784210in}{1.585000in}}{\pgfqpoint{0.784210in}{1.557222in}}%
\pgfpathlineto{\pgfqpoint{0.784210in}{1.154444in}}%
\pgfpathquadraticcurveto{\pgfqpoint{0.784210in}{1.126667in}}{\pgfqpoint{0.811987in}{1.126667in}}%
\pgfpathlineto{\pgfqpoint{0.811987in}{1.126667in}}%
\pgfpathclose%
\pgfusepath{stroke,fill}%
\end{pgfscope}%
\begin{pgfscope}%
\pgfsetrectcap%
\pgfsetroundjoin%
\pgfsetlinewidth{1.505625pt}%
\definecolor{currentstroke}{rgb}{0.000000,0.000000,0.000000}%
\pgfsetstrokecolor{currentstroke}%
\pgfsetdash{}{0pt}%
\pgfpathmoveto{\pgfqpoint{0.839765in}{1.473889in}}%
\pgfpathlineto{\pgfqpoint{0.978654in}{1.473889in}}%
\pgfpathlineto{\pgfqpoint{1.117543in}{1.473889in}}%
\pgfusepath{stroke}%
\end{pgfscope}%
\begin{pgfscope}%
\definecolor{textcolor}{rgb}{0.000000,0.000000,0.000000}%
\pgfsetstrokecolor{textcolor}%
\pgfsetfillcolor{textcolor}%
\pgftext[x=1.228654in,y=1.425277in,left,base]{\color{textcolor}\rmfamily\fontsize{10.000000}{12.000000}\selectfont \(\displaystyle \Delta FP/\Delta TP\)}%
\end{pgfscope}%
\begin{pgfscope}%
\pgfsetrectcap%
\pgfsetroundjoin%
\pgfsetlinewidth{1.505625pt}%
\definecolor{currentstroke}{rgb}{0.121569,0.466667,0.705882}%
\pgfsetstrokecolor{currentstroke}%
\pgfsetdash{}{0pt}%
\pgfpathmoveto{\pgfqpoint{0.839765in}{1.265555in}}%
\pgfpathlineto{\pgfqpoint{0.978654in}{1.265555in}}%
\pgfpathlineto{\pgfqpoint{1.117543in}{1.265555in}}%
\pgfusepath{stroke}%
\end{pgfscope}%
\begin{pgfscope}%
\pgfsetbuttcap%
\pgfsetroundjoin%
\definecolor{currentfill}{rgb}{0.000000,0.000000,0.000000}%
\pgfsetfillcolor{currentfill}%
\pgfsetlinewidth{1.003750pt}%
\definecolor{currentstroke}{rgb}{0.000000,0.000000,0.000000}%
\pgfsetstrokecolor{currentstroke}%
\pgfsetdash{}{0pt}%
\pgfsys@defobject{currentmarker}{\pgfqpoint{-0.041667in}{-0.041667in}}{\pgfqpoint{0.041667in}{0.041667in}}{%
\pgfpathmoveto{\pgfqpoint{0.000000in}{-0.041667in}}%
\pgfpathcurveto{\pgfqpoint{0.011050in}{-0.041667in}}{\pgfqpoint{0.021649in}{-0.037276in}}{\pgfqpoint{0.029463in}{-0.029463in}}%
\pgfpathcurveto{\pgfqpoint{0.037276in}{-0.021649in}}{\pgfqpoint{0.041667in}{-0.011050in}}{\pgfqpoint{0.041667in}{0.000000in}}%
\pgfpathcurveto{\pgfqpoint{0.041667in}{0.011050in}}{\pgfqpoint{0.037276in}{0.021649in}}{\pgfqpoint{0.029463in}{0.029463in}}%
\pgfpathcurveto{\pgfqpoint{0.021649in}{0.037276in}}{\pgfqpoint{0.011050in}{0.041667in}}{\pgfqpoint{0.000000in}{0.041667in}}%
\pgfpathcurveto{\pgfqpoint{-0.011050in}{0.041667in}}{\pgfqpoint{-0.021649in}{0.037276in}}{\pgfqpoint{-0.029463in}{0.029463in}}%
\pgfpathcurveto{\pgfqpoint{-0.037276in}{0.021649in}}{\pgfqpoint{-0.041667in}{0.011050in}}{\pgfqpoint{-0.041667in}{0.000000in}}%
\pgfpathcurveto{\pgfqpoint{-0.041667in}{-0.011050in}}{\pgfqpoint{-0.037276in}{-0.021649in}}{\pgfqpoint{-0.029463in}{-0.029463in}}%
\pgfpathcurveto{\pgfqpoint{-0.021649in}{-0.037276in}}{\pgfqpoint{-0.011050in}{-0.041667in}}{\pgfqpoint{0.000000in}{-0.041667in}}%
\pgfpathlineto{\pgfqpoint{0.000000in}{-0.041667in}}%
\pgfpathclose%
\pgfusepath{stroke,fill}%
}%
\begin{pgfscope}%
\pgfsys@transformshift{0.978654in}{1.265555in}%
\pgfsys@useobject{currentmarker}{}%
\end{pgfscope}%
\end{pgfscope}%
\begin{pgfscope}%
\definecolor{textcolor}{rgb}{0.000000,0.000000,0.000000}%
\pgfsetstrokecolor{textcolor}%
\pgfsetfillcolor{textcolor}%
\pgftext[x=1.228654in,y=1.216944in,left,base]{\color{textcolor}\rmfamily\fontsize{10.000000}{12.000000}\selectfont (0.720,2)}%
\end{pgfscope}%
\end{pgfpicture}%
\makeatother%
\endgroup%
	
&
	\vskip 0pt
	\hfill Transformed Model Output
	
%	%% Creator: Matplotlib, PGF backend
%%
%% To include the figure in your LaTeX document, write
%%   \input{<filename>.pgf}
%%
%% Make sure the required packages are loaded in your preamble
%%   \usepackage{pgf}
%%
%% Also ensure that all the required font packages are loaded; for instance,
%% the lmodern package is sometimes necessary when using math font.
%%   \usepackage{lmodern}
%%
%% Figures using additional raster images can only be included by \input if
%% they are in the same directory as the main LaTeX file. For loading figures
%% from other directories you can use the `import` package
%%   \usepackage{import}
%%
%% and then include the figures with
%%   \import{<path to file>}{<filename>.pgf}
%%
%% Matplotlib used the following preamble
%%   
%%   \usepackage{fontspec}
%%   \makeatletter\@ifpackageloaded{underscore}{}{\usepackage[strings]{underscore}}\makeatother
%%
\begingroup%
\makeatletter%
\begin{pgfpicture}%
\pgfpathrectangle{\pgfpointorigin}{\pgfqpoint{2.253750in}{1.754444in}}%
\pgfusepath{use as bounding box, clip}%
\begin{pgfscope}%
\pgfsetbuttcap%
\pgfsetmiterjoin%
\definecolor{currentfill}{rgb}{1.000000,1.000000,1.000000}%
\pgfsetfillcolor{currentfill}%
\pgfsetlinewidth{0.000000pt}%
\definecolor{currentstroke}{rgb}{1.000000,1.000000,1.000000}%
\pgfsetstrokecolor{currentstroke}%
\pgfsetdash{}{0pt}%
\pgfpathmoveto{\pgfqpoint{0.000000in}{0.000000in}}%
\pgfpathlineto{\pgfqpoint{2.253750in}{0.000000in}}%
\pgfpathlineto{\pgfqpoint{2.253750in}{1.754444in}}%
\pgfpathlineto{\pgfqpoint{0.000000in}{1.754444in}}%
\pgfpathlineto{\pgfqpoint{0.000000in}{0.000000in}}%
\pgfpathclose%
\pgfusepath{fill}%
\end{pgfscope}%
\begin{pgfscope}%
\pgfsetbuttcap%
\pgfsetmiterjoin%
\definecolor{currentfill}{rgb}{1.000000,1.000000,1.000000}%
\pgfsetfillcolor{currentfill}%
\pgfsetlinewidth{0.000000pt}%
\definecolor{currentstroke}{rgb}{0.000000,0.000000,0.000000}%
\pgfsetstrokecolor{currentstroke}%
\pgfsetstrokeopacity{0.000000}%
\pgfsetdash{}{0pt}%
\pgfpathmoveto{\pgfqpoint{0.515000in}{0.499444in}}%
\pgfpathlineto{\pgfqpoint{2.065000in}{0.499444in}}%
\pgfpathlineto{\pgfqpoint{2.065000in}{1.654444in}}%
\pgfpathlineto{\pgfqpoint{0.515000in}{1.654444in}}%
\pgfpathlineto{\pgfqpoint{0.515000in}{0.499444in}}%
\pgfpathclose%
\pgfusepath{fill}%
\end{pgfscope}%
\begin{pgfscope}%
\pgfpathrectangle{\pgfqpoint{0.515000in}{0.499444in}}{\pgfqpoint{1.550000in}{1.155000in}}%
\pgfusepath{clip}%
\pgfsetbuttcap%
\pgfsetmiterjoin%
\pgfsetlinewidth{1.003750pt}%
\definecolor{currentstroke}{rgb}{0.000000,0.000000,0.000000}%
\pgfsetstrokecolor{currentstroke}%
\pgfsetdash{}{0pt}%
\pgfpathmoveto{\pgfqpoint{0.505000in}{0.499444in}}%
\pgfpathlineto{\pgfqpoint{0.552805in}{0.499444in}}%
\pgfpathlineto{\pgfqpoint{0.552805in}{1.599444in}}%
\pgfpathlineto{\pgfqpoint{0.505000in}{1.599444in}}%
\pgfusepath{stroke}%
\end{pgfscope}%
\begin{pgfscope}%
\pgfpathrectangle{\pgfqpoint{0.515000in}{0.499444in}}{\pgfqpoint{1.550000in}{1.155000in}}%
\pgfusepath{clip}%
\pgfsetbuttcap%
\pgfsetmiterjoin%
\pgfsetlinewidth{1.003750pt}%
\definecolor{currentstroke}{rgb}{0.000000,0.000000,0.000000}%
\pgfsetstrokecolor{currentstroke}%
\pgfsetdash{}{0pt}%
\pgfpathmoveto{\pgfqpoint{0.643537in}{0.499444in}}%
\pgfpathlineto{\pgfqpoint{0.704025in}{0.499444in}}%
\pgfpathlineto{\pgfqpoint{0.704025in}{1.392582in}}%
\pgfpathlineto{\pgfqpoint{0.643537in}{1.392582in}}%
\pgfpathlineto{\pgfqpoint{0.643537in}{0.499444in}}%
\pgfpathclose%
\pgfusepath{stroke}%
\end{pgfscope}%
\begin{pgfscope}%
\pgfpathrectangle{\pgfqpoint{0.515000in}{0.499444in}}{\pgfqpoint{1.550000in}{1.155000in}}%
\pgfusepath{clip}%
\pgfsetbuttcap%
\pgfsetmiterjoin%
\pgfsetlinewidth{1.003750pt}%
\definecolor{currentstroke}{rgb}{0.000000,0.000000,0.000000}%
\pgfsetstrokecolor{currentstroke}%
\pgfsetdash{}{0pt}%
\pgfpathmoveto{\pgfqpoint{0.794756in}{0.499444in}}%
\pgfpathlineto{\pgfqpoint{0.855244in}{0.499444in}}%
\pgfpathlineto{\pgfqpoint{0.855244in}{1.173784in}}%
\pgfpathlineto{\pgfqpoint{0.794756in}{1.173784in}}%
\pgfpathlineto{\pgfqpoint{0.794756in}{0.499444in}}%
\pgfpathclose%
\pgfusepath{stroke}%
\end{pgfscope}%
\begin{pgfscope}%
\pgfpathrectangle{\pgfqpoint{0.515000in}{0.499444in}}{\pgfqpoint{1.550000in}{1.155000in}}%
\pgfusepath{clip}%
\pgfsetbuttcap%
\pgfsetmiterjoin%
\pgfsetlinewidth{1.003750pt}%
\definecolor{currentstroke}{rgb}{0.000000,0.000000,0.000000}%
\pgfsetstrokecolor{currentstroke}%
\pgfsetdash{}{0pt}%
\pgfpathmoveto{\pgfqpoint{0.945976in}{0.499444in}}%
\pgfpathlineto{\pgfqpoint{1.006464in}{0.499444in}}%
\pgfpathlineto{\pgfqpoint{1.006464in}{0.992551in}}%
\pgfpathlineto{\pgfqpoint{0.945976in}{0.992551in}}%
\pgfpathlineto{\pgfqpoint{0.945976in}{0.499444in}}%
\pgfpathclose%
\pgfusepath{stroke}%
\end{pgfscope}%
\begin{pgfscope}%
\pgfpathrectangle{\pgfqpoint{0.515000in}{0.499444in}}{\pgfqpoint{1.550000in}{1.155000in}}%
\pgfusepath{clip}%
\pgfsetbuttcap%
\pgfsetmiterjoin%
\pgfsetlinewidth{1.003750pt}%
\definecolor{currentstroke}{rgb}{0.000000,0.000000,0.000000}%
\pgfsetstrokecolor{currentstroke}%
\pgfsetdash{}{0pt}%
\pgfpathmoveto{\pgfqpoint{1.097195in}{0.499444in}}%
\pgfpathlineto{\pgfqpoint{1.157683in}{0.499444in}}%
\pgfpathlineto{\pgfqpoint{1.157683in}{0.827457in}}%
\pgfpathlineto{\pgfqpoint{1.097195in}{0.827457in}}%
\pgfpathlineto{\pgfqpoint{1.097195in}{0.499444in}}%
\pgfpathclose%
\pgfusepath{stroke}%
\end{pgfscope}%
\begin{pgfscope}%
\pgfpathrectangle{\pgfqpoint{0.515000in}{0.499444in}}{\pgfqpoint{1.550000in}{1.155000in}}%
\pgfusepath{clip}%
\pgfsetbuttcap%
\pgfsetmiterjoin%
\pgfsetlinewidth{1.003750pt}%
\definecolor{currentstroke}{rgb}{0.000000,0.000000,0.000000}%
\pgfsetstrokecolor{currentstroke}%
\pgfsetdash{}{0pt}%
\pgfpathmoveto{\pgfqpoint{1.248415in}{0.499444in}}%
\pgfpathlineto{\pgfqpoint{1.308903in}{0.499444in}}%
\pgfpathlineto{\pgfqpoint{1.308903in}{0.669135in}}%
\pgfpathlineto{\pgfqpoint{1.248415in}{0.669135in}}%
\pgfpathlineto{\pgfqpoint{1.248415in}{0.499444in}}%
\pgfpathclose%
\pgfusepath{stroke}%
\end{pgfscope}%
\begin{pgfscope}%
\pgfpathrectangle{\pgfqpoint{0.515000in}{0.499444in}}{\pgfqpoint{1.550000in}{1.155000in}}%
\pgfusepath{clip}%
\pgfsetbuttcap%
\pgfsetmiterjoin%
\pgfsetlinewidth{1.003750pt}%
\definecolor{currentstroke}{rgb}{0.000000,0.000000,0.000000}%
\pgfsetstrokecolor{currentstroke}%
\pgfsetdash{}{0pt}%
\pgfpathmoveto{\pgfqpoint{1.399634in}{0.499444in}}%
\pgfpathlineto{\pgfqpoint{1.460122in}{0.499444in}}%
\pgfpathlineto{\pgfqpoint{1.460122in}{0.545661in}}%
\pgfpathlineto{\pgfqpoint{1.399634in}{0.545661in}}%
\pgfpathlineto{\pgfqpoint{1.399634in}{0.499444in}}%
\pgfpathclose%
\pgfusepath{stroke}%
\end{pgfscope}%
\begin{pgfscope}%
\pgfpathrectangle{\pgfqpoint{0.515000in}{0.499444in}}{\pgfqpoint{1.550000in}{1.155000in}}%
\pgfusepath{clip}%
\pgfsetbuttcap%
\pgfsetmiterjoin%
\pgfsetlinewidth{1.003750pt}%
\definecolor{currentstroke}{rgb}{0.000000,0.000000,0.000000}%
\pgfsetstrokecolor{currentstroke}%
\pgfsetdash{}{0pt}%
\pgfpathmoveto{\pgfqpoint{1.550854in}{0.499444in}}%
\pgfpathlineto{\pgfqpoint{1.611342in}{0.499444in}}%
\pgfpathlineto{\pgfqpoint{1.611342in}{0.499444in}}%
\pgfpathlineto{\pgfqpoint{1.550854in}{0.499444in}}%
\pgfpathlineto{\pgfqpoint{1.550854in}{0.499444in}}%
\pgfpathclose%
\pgfusepath{stroke}%
\end{pgfscope}%
\begin{pgfscope}%
\pgfpathrectangle{\pgfqpoint{0.515000in}{0.499444in}}{\pgfqpoint{1.550000in}{1.155000in}}%
\pgfusepath{clip}%
\pgfsetbuttcap%
\pgfsetmiterjoin%
\pgfsetlinewidth{1.003750pt}%
\definecolor{currentstroke}{rgb}{0.000000,0.000000,0.000000}%
\pgfsetstrokecolor{currentstroke}%
\pgfsetdash{}{0pt}%
\pgfpathmoveto{\pgfqpoint{1.702073in}{0.499444in}}%
\pgfpathlineto{\pgfqpoint{1.762561in}{0.499444in}}%
\pgfpathlineto{\pgfqpoint{1.762561in}{0.499444in}}%
\pgfpathlineto{\pgfqpoint{1.702073in}{0.499444in}}%
\pgfpathlineto{\pgfqpoint{1.702073in}{0.499444in}}%
\pgfpathclose%
\pgfusepath{stroke}%
\end{pgfscope}%
\begin{pgfscope}%
\pgfpathrectangle{\pgfqpoint{0.515000in}{0.499444in}}{\pgfqpoint{1.550000in}{1.155000in}}%
\pgfusepath{clip}%
\pgfsetbuttcap%
\pgfsetmiterjoin%
\pgfsetlinewidth{1.003750pt}%
\definecolor{currentstroke}{rgb}{0.000000,0.000000,0.000000}%
\pgfsetstrokecolor{currentstroke}%
\pgfsetdash{}{0pt}%
\pgfpathmoveto{\pgfqpoint{1.853293in}{0.499444in}}%
\pgfpathlineto{\pgfqpoint{1.913781in}{0.499444in}}%
\pgfpathlineto{\pgfqpoint{1.913781in}{0.499444in}}%
\pgfpathlineto{\pgfqpoint{1.853293in}{0.499444in}}%
\pgfpathlineto{\pgfqpoint{1.853293in}{0.499444in}}%
\pgfpathclose%
\pgfusepath{stroke}%
\end{pgfscope}%
\begin{pgfscope}%
\pgfpathrectangle{\pgfqpoint{0.515000in}{0.499444in}}{\pgfqpoint{1.550000in}{1.155000in}}%
\pgfusepath{clip}%
\pgfsetbuttcap%
\pgfsetmiterjoin%
\definecolor{currentfill}{rgb}{0.000000,0.000000,0.000000}%
\pgfsetfillcolor{currentfill}%
\pgfsetlinewidth{0.000000pt}%
\definecolor{currentstroke}{rgb}{0.000000,0.000000,0.000000}%
\pgfsetstrokecolor{currentstroke}%
\pgfsetstrokeopacity{0.000000}%
\pgfsetdash{}{0pt}%
\pgfpathmoveto{\pgfqpoint{0.552805in}{0.499444in}}%
\pgfpathlineto{\pgfqpoint{0.613293in}{0.499444in}}%
\pgfpathlineto{\pgfqpoint{0.613293in}{0.534712in}}%
\pgfpathlineto{\pgfqpoint{0.552805in}{0.534712in}}%
\pgfpathlineto{\pgfqpoint{0.552805in}{0.499444in}}%
\pgfpathclose%
\pgfusepath{fill}%
\end{pgfscope}%
\begin{pgfscope}%
\pgfpathrectangle{\pgfqpoint{0.515000in}{0.499444in}}{\pgfqpoint{1.550000in}{1.155000in}}%
\pgfusepath{clip}%
\pgfsetbuttcap%
\pgfsetmiterjoin%
\definecolor{currentfill}{rgb}{0.000000,0.000000,0.000000}%
\pgfsetfillcolor{currentfill}%
\pgfsetlinewidth{0.000000pt}%
\definecolor{currentstroke}{rgb}{0.000000,0.000000,0.000000}%
\pgfsetstrokecolor{currentstroke}%
\pgfsetstrokeopacity{0.000000}%
\pgfsetdash{}{0pt}%
\pgfpathmoveto{\pgfqpoint{0.704025in}{0.499444in}}%
\pgfpathlineto{\pgfqpoint{0.764512in}{0.499444in}}%
\pgfpathlineto{\pgfqpoint{0.764512in}{0.574601in}}%
\pgfpathlineto{\pgfqpoint{0.704025in}{0.574601in}}%
\pgfpathlineto{\pgfqpoint{0.704025in}{0.499444in}}%
\pgfpathclose%
\pgfusepath{fill}%
\end{pgfscope}%
\begin{pgfscope}%
\pgfpathrectangle{\pgfqpoint{0.515000in}{0.499444in}}{\pgfqpoint{1.550000in}{1.155000in}}%
\pgfusepath{clip}%
\pgfsetbuttcap%
\pgfsetmiterjoin%
\definecolor{currentfill}{rgb}{0.000000,0.000000,0.000000}%
\pgfsetfillcolor{currentfill}%
\pgfsetlinewidth{0.000000pt}%
\definecolor{currentstroke}{rgb}{0.000000,0.000000,0.000000}%
\pgfsetstrokecolor{currentstroke}%
\pgfsetstrokeopacity{0.000000}%
\pgfsetdash{}{0pt}%
\pgfpathmoveto{\pgfqpoint{0.855244in}{0.499444in}}%
\pgfpathlineto{\pgfqpoint{0.915732in}{0.499444in}}%
\pgfpathlineto{\pgfqpoint{0.915732in}{0.598748in}}%
\pgfpathlineto{\pgfqpoint{0.855244in}{0.598748in}}%
\pgfpathlineto{\pgfqpoint{0.855244in}{0.499444in}}%
\pgfpathclose%
\pgfusepath{fill}%
\end{pgfscope}%
\begin{pgfscope}%
\pgfpathrectangle{\pgfqpoint{0.515000in}{0.499444in}}{\pgfqpoint{1.550000in}{1.155000in}}%
\pgfusepath{clip}%
\pgfsetbuttcap%
\pgfsetmiterjoin%
\definecolor{currentfill}{rgb}{0.000000,0.000000,0.000000}%
\pgfsetfillcolor{currentfill}%
\pgfsetlinewidth{0.000000pt}%
\definecolor{currentstroke}{rgb}{0.000000,0.000000,0.000000}%
\pgfsetstrokecolor{currentstroke}%
\pgfsetstrokeopacity{0.000000}%
\pgfsetdash{}{0pt}%
\pgfpathmoveto{\pgfqpoint{1.006464in}{0.499444in}}%
\pgfpathlineto{\pgfqpoint{1.066951in}{0.499444in}}%
\pgfpathlineto{\pgfqpoint{1.066951in}{0.618421in}}%
\pgfpathlineto{\pgfqpoint{1.006464in}{0.618421in}}%
\pgfpathlineto{\pgfqpoint{1.006464in}{0.499444in}}%
\pgfpathclose%
\pgfusepath{fill}%
\end{pgfscope}%
\begin{pgfscope}%
\pgfpathrectangle{\pgfqpoint{0.515000in}{0.499444in}}{\pgfqpoint{1.550000in}{1.155000in}}%
\pgfusepath{clip}%
\pgfsetbuttcap%
\pgfsetmiterjoin%
\definecolor{currentfill}{rgb}{0.000000,0.000000,0.000000}%
\pgfsetfillcolor{currentfill}%
\pgfsetlinewidth{0.000000pt}%
\definecolor{currentstroke}{rgb}{0.000000,0.000000,0.000000}%
\pgfsetstrokecolor{currentstroke}%
\pgfsetstrokeopacity{0.000000}%
\pgfsetdash{}{0pt}%
\pgfpathmoveto{\pgfqpoint{1.157683in}{0.499444in}}%
\pgfpathlineto{\pgfqpoint{1.218171in}{0.499444in}}%
\pgfpathlineto{\pgfqpoint{1.218171in}{0.628974in}}%
\pgfpathlineto{\pgfqpoint{1.157683in}{0.628974in}}%
\pgfpathlineto{\pgfqpoint{1.157683in}{0.499444in}}%
\pgfpathclose%
\pgfusepath{fill}%
\end{pgfscope}%
\begin{pgfscope}%
\pgfpathrectangle{\pgfqpoint{0.515000in}{0.499444in}}{\pgfqpoint{1.550000in}{1.155000in}}%
\pgfusepath{clip}%
\pgfsetbuttcap%
\pgfsetmiterjoin%
\definecolor{currentfill}{rgb}{0.000000,0.000000,0.000000}%
\pgfsetfillcolor{currentfill}%
\pgfsetlinewidth{0.000000pt}%
\definecolor{currentstroke}{rgb}{0.000000,0.000000,0.000000}%
\pgfsetstrokecolor{currentstroke}%
\pgfsetstrokeopacity{0.000000}%
\pgfsetdash{}{0pt}%
\pgfpathmoveto{\pgfqpoint{1.308903in}{0.499444in}}%
\pgfpathlineto{\pgfqpoint{1.369391in}{0.499444in}}%
\pgfpathlineto{\pgfqpoint{1.369391in}{0.623116in}}%
\pgfpathlineto{\pgfqpoint{1.308903in}{0.623116in}}%
\pgfpathlineto{\pgfqpoint{1.308903in}{0.499444in}}%
\pgfpathclose%
\pgfusepath{fill}%
\end{pgfscope}%
\begin{pgfscope}%
\pgfpathrectangle{\pgfqpoint{0.515000in}{0.499444in}}{\pgfqpoint{1.550000in}{1.155000in}}%
\pgfusepath{clip}%
\pgfsetbuttcap%
\pgfsetmiterjoin%
\definecolor{currentfill}{rgb}{0.000000,0.000000,0.000000}%
\pgfsetfillcolor{currentfill}%
\pgfsetlinewidth{0.000000pt}%
\definecolor{currentstroke}{rgb}{0.000000,0.000000,0.000000}%
\pgfsetstrokecolor{currentstroke}%
\pgfsetstrokeopacity{0.000000}%
\pgfsetdash{}{0pt}%
\pgfpathmoveto{\pgfqpoint{1.460122in}{0.499444in}}%
\pgfpathlineto{\pgfqpoint{1.520610in}{0.499444in}}%
\pgfpathlineto{\pgfqpoint{1.520610in}{0.575269in}}%
\pgfpathlineto{\pgfqpoint{1.460122in}{0.575269in}}%
\pgfpathlineto{\pgfqpoint{1.460122in}{0.499444in}}%
\pgfpathclose%
\pgfusepath{fill}%
\end{pgfscope}%
\begin{pgfscope}%
\pgfpathrectangle{\pgfqpoint{0.515000in}{0.499444in}}{\pgfqpoint{1.550000in}{1.155000in}}%
\pgfusepath{clip}%
\pgfsetbuttcap%
\pgfsetmiterjoin%
\definecolor{currentfill}{rgb}{0.000000,0.000000,0.000000}%
\pgfsetfillcolor{currentfill}%
\pgfsetlinewidth{0.000000pt}%
\definecolor{currentstroke}{rgb}{0.000000,0.000000,0.000000}%
\pgfsetstrokecolor{currentstroke}%
\pgfsetstrokeopacity{0.000000}%
\pgfsetdash{}{0pt}%
\pgfpathmoveto{\pgfqpoint{1.611342in}{0.499444in}}%
\pgfpathlineto{\pgfqpoint{1.671830in}{0.499444in}}%
\pgfpathlineto{\pgfqpoint{1.671830in}{0.499444in}}%
\pgfpathlineto{\pgfqpoint{1.611342in}{0.499444in}}%
\pgfpathlineto{\pgfqpoint{1.611342in}{0.499444in}}%
\pgfpathclose%
\pgfusepath{fill}%
\end{pgfscope}%
\begin{pgfscope}%
\pgfpathrectangle{\pgfqpoint{0.515000in}{0.499444in}}{\pgfqpoint{1.550000in}{1.155000in}}%
\pgfusepath{clip}%
\pgfsetbuttcap%
\pgfsetmiterjoin%
\definecolor{currentfill}{rgb}{0.000000,0.000000,0.000000}%
\pgfsetfillcolor{currentfill}%
\pgfsetlinewidth{0.000000pt}%
\definecolor{currentstroke}{rgb}{0.000000,0.000000,0.000000}%
\pgfsetstrokecolor{currentstroke}%
\pgfsetstrokeopacity{0.000000}%
\pgfsetdash{}{0pt}%
\pgfpathmoveto{\pgfqpoint{1.762561in}{0.499444in}}%
\pgfpathlineto{\pgfqpoint{1.823049in}{0.499444in}}%
\pgfpathlineto{\pgfqpoint{1.823049in}{0.499444in}}%
\pgfpathlineto{\pgfqpoint{1.762561in}{0.499444in}}%
\pgfpathlineto{\pgfqpoint{1.762561in}{0.499444in}}%
\pgfpathclose%
\pgfusepath{fill}%
\end{pgfscope}%
\begin{pgfscope}%
\pgfpathrectangle{\pgfqpoint{0.515000in}{0.499444in}}{\pgfqpoint{1.550000in}{1.155000in}}%
\pgfusepath{clip}%
\pgfsetbuttcap%
\pgfsetmiterjoin%
\definecolor{currentfill}{rgb}{0.000000,0.000000,0.000000}%
\pgfsetfillcolor{currentfill}%
\pgfsetlinewidth{0.000000pt}%
\definecolor{currentstroke}{rgb}{0.000000,0.000000,0.000000}%
\pgfsetstrokecolor{currentstroke}%
\pgfsetstrokeopacity{0.000000}%
\pgfsetdash{}{0pt}%
\pgfpathmoveto{\pgfqpoint{1.913781in}{0.499444in}}%
\pgfpathlineto{\pgfqpoint{1.974269in}{0.499444in}}%
\pgfpathlineto{\pgfqpoint{1.974269in}{0.499444in}}%
\pgfpathlineto{\pgfqpoint{1.913781in}{0.499444in}}%
\pgfpathlineto{\pgfqpoint{1.913781in}{0.499444in}}%
\pgfpathclose%
\pgfusepath{fill}%
\end{pgfscope}%
\begin{pgfscope}%
\pgfsetbuttcap%
\pgfsetroundjoin%
\definecolor{currentfill}{rgb}{0.000000,0.000000,0.000000}%
\pgfsetfillcolor{currentfill}%
\pgfsetlinewidth{0.803000pt}%
\definecolor{currentstroke}{rgb}{0.000000,0.000000,0.000000}%
\pgfsetstrokecolor{currentstroke}%
\pgfsetdash{}{0pt}%
\pgfsys@defobject{currentmarker}{\pgfqpoint{0.000000in}{-0.048611in}}{\pgfqpoint{0.000000in}{0.000000in}}{%
\pgfpathmoveto{\pgfqpoint{0.000000in}{0.000000in}}%
\pgfpathlineto{\pgfqpoint{0.000000in}{-0.048611in}}%
\pgfusepath{stroke,fill}%
}%
\begin{pgfscope}%
\pgfsys@transformshift{0.552805in}{0.499444in}%
\pgfsys@useobject{currentmarker}{}%
\end{pgfscope}%
\end{pgfscope}%
\begin{pgfscope}%
\definecolor{textcolor}{rgb}{0.000000,0.000000,0.000000}%
\pgfsetstrokecolor{textcolor}%
\pgfsetfillcolor{textcolor}%
\pgftext[x=0.552805in,y=0.402222in,,top]{\color{textcolor}\rmfamily\fontsize{10.000000}{12.000000}\selectfont 0.0}%
\end{pgfscope}%
\begin{pgfscope}%
\pgfsetbuttcap%
\pgfsetroundjoin%
\definecolor{currentfill}{rgb}{0.000000,0.000000,0.000000}%
\pgfsetfillcolor{currentfill}%
\pgfsetlinewidth{0.803000pt}%
\definecolor{currentstroke}{rgb}{0.000000,0.000000,0.000000}%
\pgfsetstrokecolor{currentstroke}%
\pgfsetdash{}{0pt}%
\pgfsys@defobject{currentmarker}{\pgfqpoint{0.000000in}{-0.048611in}}{\pgfqpoint{0.000000in}{0.000000in}}{%
\pgfpathmoveto{\pgfqpoint{0.000000in}{0.000000in}}%
\pgfpathlineto{\pgfqpoint{0.000000in}{-0.048611in}}%
\pgfusepath{stroke,fill}%
}%
\begin{pgfscope}%
\pgfsys@transformshift{0.930854in}{0.499444in}%
\pgfsys@useobject{currentmarker}{}%
\end{pgfscope}%
\end{pgfscope}%
\begin{pgfscope}%
\definecolor{textcolor}{rgb}{0.000000,0.000000,0.000000}%
\pgfsetstrokecolor{textcolor}%
\pgfsetfillcolor{textcolor}%
\pgftext[x=0.930854in,y=0.402222in,,top]{\color{textcolor}\rmfamily\fontsize{10.000000}{12.000000}\selectfont 0.25}%
\end{pgfscope}%
\begin{pgfscope}%
\pgfsetbuttcap%
\pgfsetroundjoin%
\definecolor{currentfill}{rgb}{0.000000,0.000000,0.000000}%
\pgfsetfillcolor{currentfill}%
\pgfsetlinewidth{0.803000pt}%
\definecolor{currentstroke}{rgb}{0.000000,0.000000,0.000000}%
\pgfsetstrokecolor{currentstroke}%
\pgfsetdash{}{0pt}%
\pgfsys@defobject{currentmarker}{\pgfqpoint{0.000000in}{-0.048611in}}{\pgfqpoint{0.000000in}{0.000000in}}{%
\pgfpathmoveto{\pgfqpoint{0.000000in}{0.000000in}}%
\pgfpathlineto{\pgfqpoint{0.000000in}{-0.048611in}}%
\pgfusepath{stroke,fill}%
}%
\begin{pgfscope}%
\pgfsys@transformshift{1.308903in}{0.499444in}%
\pgfsys@useobject{currentmarker}{}%
\end{pgfscope}%
\end{pgfscope}%
\begin{pgfscope}%
\definecolor{textcolor}{rgb}{0.000000,0.000000,0.000000}%
\pgfsetstrokecolor{textcolor}%
\pgfsetfillcolor{textcolor}%
\pgftext[x=1.308903in,y=0.402222in,,top]{\color{textcolor}\rmfamily\fontsize{10.000000}{12.000000}\selectfont 0.5}%
\end{pgfscope}%
\begin{pgfscope}%
\pgfsetbuttcap%
\pgfsetroundjoin%
\definecolor{currentfill}{rgb}{0.000000,0.000000,0.000000}%
\pgfsetfillcolor{currentfill}%
\pgfsetlinewidth{0.803000pt}%
\definecolor{currentstroke}{rgb}{0.000000,0.000000,0.000000}%
\pgfsetstrokecolor{currentstroke}%
\pgfsetdash{}{0pt}%
\pgfsys@defobject{currentmarker}{\pgfqpoint{0.000000in}{-0.048611in}}{\pgfqpoint{0.000000in}{0.000000in}}{%
\pgfpathmoveto{\pgfqpoint{0.000000in}{0.000000in}}%
\pgfpathlineto{\pgfqpoint{0.000000in}{-0.048611in}}%
\pgfusepath{stroke,fill}%
}%
\begin{pgfscope}%
\pgfsys@transformshift{1.686951in}{0.499444in}%
\pgfsys@useobject{currentmarker}{}%
\end{pgfscope}%
\end{pgfscope}%
\begin{pgfscope}%
\definecolor{textcolor}{rgb}{0.000000,0.000000,0.000000}%
\pgfsetstrokecolor{textcolor}%
\pgfsetfillcolor{textcolor}%
\pgftext[x=1.686951in,y=0.402222in,,top]{\color{textcolor}\rmfamily\fontsize{10.000000}{12.000000}\selectfont 0.75}%
\end{pgfscope}%
\begin{pgfscope}%
\pgfsetbuttcap%
\pgfsetroundjoin%
\definecolor{currentfill}{rgb}{0.000000,0.000000,0.000000}%
\pgfsetfillcolor{currentfill}%
\pgfsetlinewidth{0.803000pt}%
\definecolor{currentstroke}{rgb}{0.000000,0.000000,0.000000}%
\pgfsetstrokecolor{currentstroke}%
\pgfsetdash{}{0pt}%
\pgfsys@defobject{currentmarker}{\pgfqpoint{0.000000in}{-0.048611in}}{\pgfqpoint{0.000000in}{0.000000in}}{%
\pgfpathmoveto{\pgfqpoint{0.000000in}{0.000000in}}%
\pgfpathlineto{\pgfqpoint{0.000000in}{-0.048611in}}%
\pgfusepath{stroke,fill}%
}%
\begin{pgfscope}%
\pgfsys@transformshift{2.065000in}{0.499444in}%
\pgfsys@useobject{currentmarker}{}%
\end{pgfscope}%
\end{pgfscope}%
\begin{pgfscope}%
\definecolor{textcolor}{rgb}{0.000000,0.000000,0.000000}%
\pgfsetstrokecolor{textcolor}%
\pgfsetfillcolor{textcolor}%
\pgftext[x=2.065000in,y=0.402222in,,top]{\color{textcolor}\rmfamily\fontsize{10.000000}{12.000000}\selectfont 1.0}%
\end{pgfscope}%
\begin{pgfscope}%
\definecolor{textcolor}{rgb}{0.000000,0.000000,0.000000}%
\pgfsetstrokecolor{textcolor}%
\pgfsetfillcolor{textcolor}%
\pgftext[x=1.290000in,y=0.223333in,,top]{\color{textcolor}\rmfamily\fontsize{10.000000}{12.000000}\selectfont \(\displaystyle p\)}%
\end{pgfscope}%
\begin{pgfscope}%
\pgfsetbuttcap%
\pgfsetroundjoin%
\definecolor{currentfill}{rgb}{0.000000,0.000000,0.000000}%
\pgfsetfillcolor{currentfill}%
\pgfsetlinewidth{0.803000pt}%
\definecolor{currentstroke}{rgb}{0.000000,0.000000,0.000000}%
\pgfsetstrokecolor{currentstroke}%
\pgfsetdash{}{0pt}%
\pgfsys@defobject{currentmarker}{\pgfqpoint{-0.048611in}{0.000000in}}{\pgfqpoint{-0.000000in}{0.000000in}}{%
\pgfpathmoveto{\pgfqpoint{-0.000000in}{0.000000in}}%
\pgfpathlineto{\pgfqpoint{-0.048611in}{0.000000in}}%
\pgfusepath{stroke,fill}%
}%
\begin{pgfscope}%
\pgfsys@transformshift{0.515000in}{0.499444in}%
\pgfsys@useobject{currentmarker}{}%
\end{pgfscope}%
\end{pgfscope}%
\begin{pgfscope}%
\definecolor{textcolor}{rgb}{0.000000,0.000000,0.000000}%
\pgfsetstrokecolor{textcolor}%
\pgfsetfillcolor{textcolor}%
\pgftext[x=0.348333in, y=0.451250in, left, base]{\color{textcolor}\rmfamily\fontsize{10.000000}{12.000000}\selectfont \(\displaystyle {0}\)}%
\end{pgfscope}%
\begin{pgfscope}%
\pgfsetbuttcap%
\pgfsetroundjoin%
\definecolor{currentfill}{rgb}{0.000000,0.000000,0.000000}%
\pgfsetfillcolor{currentfill}%
\pgfsetlinewidth{0.803000pt}%
\definecolor{currentstroke}{rgb}{0.000000,0.000000,0.000000}%
\pgfsetstrokecolor{currentstroke}%
\pgfsetdash{}{0pt}%
\pgfsys@defobject{currentmarker}{\pgfqpoint{-0.048611in}{0.000000in}}{\pgfqpoint{-0.000000in}{0.000000in}}{%
\pgfpathmoveto{\pgfqpoint{-0.000000in}{0.000000in}}%
\pgfpathlineto{\pgfqpoint{-0.048611in}{0.000000in}}%
\pgfusepath{stroke,fill}%
}%
\begin{pgfscope}%
\pgfsys@transformshift{0.515000in}{0.937862in}%
\pgfsys@useobject{currentmarker}{}%
\end{pgfscope}%
\end{pgfscope}%
\begin{pgfscope}%
\definecolor{textcolor}{rgb}{0.000000,0.000000,0.000000}%
\pgfsetstrokecolor{textcolor}%
\pgfsetfillcolor{textcolor}%
\pgftext[x=0.278889in, y=0.889668in, left, base]{\color{textcolor}\rmfamily\fontsize{10.000000}{12.000000}\selectfont \(\displaystyle {10}\)}%
\end{pgfscope}%
\begin{pgfscope}%
\pgfsetbuttcap%
\pgfsetroundjoin%
\definecolor{currentfill}{rgb}{0.000000,0.000000,0.000000}%
\pgfsetfillcolor{currentfill}%
\pgfsetlinewidth{0.803000pt}%
\definecolor{currentstroke}{rgb}{0.000000,0.000000,0.000000}%
\pgfsetstrokecolor{currentstroke}%
\pgfsetdash{}{0pt}%
\pgfsys@defobject{currentmarker}{\pgfqpoint{-0.048611in}{0.000000in}}{\pgfqpoint{-0.000000in}{0.000000in}}{%
\pgfpathmoveto{\pgfqpoint{-0.000000in}{0.000000in}}%
\pgfpathlineto{\pgfqpoint{-0.048611in}{0.000000in}}%
\pgfusepath{stroke,fill}%
}%
\begin{pgfscope}%
\pgfsys@transformshift{0.515000in}{1.376281in}%
\pgfsys@useobject{currentmarker}{}%
\end{pgfscope}%
\end{pgfscope}%
\begin{pgfscope}%
\definecolor{textcolor}{rgb}{0.000000,0.000000,0.000000}%
\pgfsetstrokecolor{textcolor}%
\pgfsetfillcolor{textcolor}%
\pgftext[x=0.278889in, y=1.328086in, left, base]{\color{textcolor}\rmfamily\fontsize{10.000000}{12.000000}\selectfont \(\displaystyle {20}\)}%
\end{pgfscope}%
\begin{pgfscope}%
\definecolor{textcolor}{rgb}{0.000000,0.000000,0.000000}%
\pgfsetstrokecolor{textcolor}%
\pgfsetfillcolor{textcolor}%
\pgftext[x=0.223333in,y=1.076944in,,bottom,rotate=90.000000]{\color{textcolor}\rmfamily\fontsize{10.000000}{12.000000}\selectfont Percent of Data Set}%
\end{pgfscope}%
\begin{pgfscope}%
\pgfsetrectcap%
\pgfsetmiterjoin%
\pgfsetlinewidth{0.803000pt}%
\definecolor{currentstroke}{rgb}{0.000000,0.000000,0.000000}%
\pgfsetstrokecolor{currentstroke}%
\pgfsetdash{}{0pt}%
\pgfpathmoveto{\pgfqpoint{0.515000in}{0.499444in}}%
\pgfpathlineto{\pgfqpoint{0.515000in}{1.654444in}}%
\pgfusepath{stroke}%
\end{pgfscope}%
\begin{pgfscope}%
\pgfsetrectcap%
\pgfsetmiterjoin%
\pgfsetlinewidth{0.803000pt}%
\definecolor{currentstroke}{rgb}{0.000000,0.000000,0.000000}%
\pgfsetstrokecolor{currentstroke}%
\pgfsetdash{}{0pt}%
\pgfpathmoveto{\pgfqpoint{2.065000in}{0.499444in}}%
\pgfpathlineto{\pgfqpoint{2.065000in}{1.654444in}}%
\pgfusepath{stroke}%
\end{pgfscope}%
\begin{pgfscope}%
\pgfsetrectcap%
\pgfsetmiterjoin%
\pgfsetlinewidth{0.803000pt}%
\definecolor{currentstroke}{rgb}{0.000000,0.000000,0.000000}%
\pgfsetstrokecolor{currentstroke}%
\pgfsetdash{}{0pt}%
\pgfpathmoveto{\pgfqpoint{0.515000in}{0.499444in}}%
\pgfpathlineto{\pgfqpoint{2.065000in}{0.499444in}}%
\pgfusepath{stroke}%
\end{pgfscope}%
\begin{pgfscope}%
\pgfsetrectcap%
\pgfsetmiterjoin%
\pgfsetlinewidth{0.803000pt}%
\definecolor{currentstroke}{rgb}{0.000000,0.000000,0.000000}%
\pgfsetstrokecolor{currentstroke}%
\pgfsetdash{}{0pt}%
\pgfpathmoveto{\pgfqpoint{0.515000in}{1.654444in}}%
\pgfpathlineto{\pgfqpoint{2.065000in}{1.654444in}}%
\pgfusepath{stroke}%
\end{pgfscope}%
\begin{pgfscope}%
\pgfsetbuttcap%
\pgfsetmiterjoin%
\definecolor{currentfill}{rgb}{1.000000,1.000000,1.000000}%
\pgfsetfillcolor{currentfill}%
\pgfsetfillopacity{0.800000}%
\pgfsetlinewidth{1.003750pt}%
\definecolor{currentstroke}{rgb}{0.800000,0.800000,0.800000}%
\pgfsetstrokecolor{currentstroke}%
\pgfsetstrokeopacity{0.800000}%
\pgfsetdash{}{0pt}%
\pgfpathmoveto{\pgfqpoint{1.288056in}{1.154445in}}%
\pgfpathlineto{\pgfqpoint{1.967778in}{1.154445in}}%
\pgfpathquadraticcurveto{\pgfqpoint{1.995556in}{1.154445in}}{\pgfqpoint{1.995556in}{1.182222in}}%
\pgfpathlineto{\pgfqpoint{1.995556in}{1.557222in}}%
\pgfpathquadraticcurveto{\pgfqpoint{1.995556in}{1.585000in}}{\pgfqpoint{1.967778in}{1.585000in}}%
\pgfpathlineto{\pgfqpoint{1.288056in}{1.585000in}}%
\pgfpathquadraticcurveto{\pgfqpoint{1.260278in}{1.585000in}}{\pgfqpoint{1.260278in}{1.557222in}}%
\pgfpathlineto{\pgfqpoint{1.260278in}{1.182222in}}%
\pgfpathquadraticcurveto{\pgfqpoint{1.260278in}{1.154445in}}{\pgfqpoint{1.288056in}{1.154445in}}%
\pgfpathlineto{\pgfqpoint{1.288056in}{1.154445in}}%
\pgfpathclose%
\pgfusepath{stroke,fill}%
\end{pgfscope}%
\begin{pgfscope}%
\pgfsetbuttcap%
\pgfsetmiterjoin%
\pgfsetlinewidth{1.003750pt}%
\definecolor{currentstroke}{rgb}{0.000000,0.000000,0.000000}%
\pgfsetstrokecolor{currentstroke}%
\pgfsetdash{}{0pt}%
\pgfpathmoveto{\pgfqpoint{1.315834in}{1.432222in}}%
\pgfpathlineto{\pgfqpoint{1.593611in}{1.432222in}}%
\pgfpathlineto{\pgfqpoint{1.593611in}{1.529444in}}%
\pgfpathlineto{\pgfqpoint{1.315834in}{1.529444in}}%
\pgfpathlineto{\pgfqpoint{1.315834in}{1.432222in}}%
\pgfpathclose%
\pgfusepath{stroke}%
\end{pgfscope}%
\begin{pgfscope}%
\definecolor{textcolor}{rgb}{0.000000,0.000000,0.000000}%
\pgfsetstrokecolor{textcolor}%
\pgfsetfillcolor{textcolor}%
\pgftext[x=1.704722in,y=1.432222in,left,base]{\color{textcolor}\rmfamily\fontsize{10.000000}{12.000000}\selectfont Neg}%
\end{pgfscope}%
\begin{pgfscope}%
\pgfsetbuttcap%
\pgfsetmiterjoin%
\definecolor{currentfill}{rgb}{0.000000,0.000000,0.000000}%
\pgfsetfillcolor{currentfill}%
\pgfsetlinewidth{0.000000pt}%
\definecolor{currentstroke}{rgb}{0.000000,0.000000,0.000000}%
\pgfsetstrokecolor{currentstroke}%
\pgfsetstrokeopacity{0.000000}%
\pgfsetdash{}{0pt}%
\pgfpathmoveto{\pgfqpoint{1.315834in}{1.236944in}}%
\pgfpathlineto{\pgfqpoint{1.593611in}{1.236944in}}%
\pgfpathlineto{\pgfqpoint{1.593611in}{1.334167in}}%
\pgfpathlineto{\pgfqpoint{1.315834in}{1.334167in}}%
\pgfpathlineto{\pgfqpoint{1.315834in}{1.236944in}}%
\pgfpathclose%
\pgfusepath{fill}%
\end{pgfscope}%
\begin{pgfscope}%
\definecolor{textcolor}{rgb}{0.000000,0.000000,0.000000}%
\pgfsetstrokecolor{textcolor}%
\pgfsetfillcolor{textcolor}%
\pgftext[x=1.704722in,y=1.236944in,left,base]{\color{textcolor}\rmfamily\fontsize{10.000000}{12.000000}\selectfont Pos}%
\end{pgfscope}%
\end{pgfpicture}%
\makeatother%
\endgroup%

	
&
	\vskip 0pt
	\begin{tabular}{cc|c|c|}
	&\multicolumn{1}{c}{}& \multicolumn{2}{c}{Prediction} \cr
	&\multicolumn{1}{c}{} & \multicolumn{1}{c}{N} & \multicolumn{1}{c}{P} \cr\cline{3-4}
	\multirow{2}{*}{\rotatebox[origin=c]{90}{Actual}}&N &
142,035 & 8,736
	\vrule width 0pt height 10pt depth 2pt \cr\cline{3-4}
	&P & 
18,549 & 8,072
	\vrule width 0pt height 10pt depth 2pt \cr\cline{3-4}
	\end{tabular}

	\hfil\begin{tabular}{ll}
	\cr
0.480 & Precision \cr	0.303 & Recall \cr	0.372 & F1 \cr	0.774 & AUC \cr
\end{tabular}

\cr
\end{tabular}
} % End parbox

\

It is reasonably to ask, ``How is the transformed model ethically better?  We are only sending 8,072 needed ambulances instead of 16,049.  In the original model, $FP/TP = 31,995/16,049 = 1.994 < 2.0 = \omega$.  How is it better to send half as many needed ambulances?''  Because our ethical tradeoff was not for total number of FP and TP, but for marginal FP and TP.   Going from the original to the transformed model, we have $\Delta FP/\Delta TP = (31,995 - 8,736)/(16,049 - 8,072) = 23,259/7,977 = 2.91$, which is higher than our choice of ethical tradeoff $\omega = 2.0$.  For this model, it is at $p = 0.720$ that we reach our tradeoff point.  

We will use the model outputs transformed to have decision threshold at $\Delta FP/\Delta TP = 2.0$ to compare different models. 

%%%
\subsection{Preparing the Data}

The CRSS data is available \href{https://www.nhtsa.gov/crash-data-systems/crash-report-sampling-system}{online at this link}.  The three main files for each year are {\tt Accident}, {\tt Vehicle}, and {\tt Person}, and one uses the {\tt CASENUM} and \verb|VEH_NO| fields to merge them into one dataset.  

%Next we dropped fields that were just random noise, like \verb|MINUTE| and vehicle identification numbers (VIN), fields that had been collected in only some years, and fields that could not be known at the time of the crash, like the results of a driver drug test.  

\subsubsection{Order of Operations}

To prepare the data we needed to do two things, to bin (discretize) some features and to impute missing data.  We did not know which to do first, so we tested both ways using IVEware \citep{IVEware} for the imputation. The imputation is a stochastic process, and the difference between binning first and imputing first was as small as the difference between running twice with different random seeds.  Since IVEware can only handle up to about forty categories in each categorical field, we had had to bin some fields first either way, so we decided on binning first.  

\subsubsection{Binning}
To bin a field's many categories into fewer categories, sometimes the meaning of the categories was a sufficient guide.  In the \verb|HOSPITAL| field, which we used as our target variable, we were only interested in two values, whether or not the person went to the hospital.  The CRSS field has six values indicating how the person went to the hospital (ground ambulance, air ambulance, ...), and we merged those into one.  For fields where the binning was not so obvious, we looked at how each value in the field correlates to hospitalization.  We wanted to put \verb|AGE| into bands, and looked to divide  where the hospitalization rate changed.  Interestingly, ages 16, 17, and 18 have lower hospitalization rates than ages below or above, so we put them into their own band.  Around age 52 the hospitalization rate started to go up, so we split there.  We binned other fields in a similar way.  

The merging, dropping, and binning are all in the \verb|CRSS_04_Discretize| code.

\subsubsection{Imputing Missing Values}

About 47\% of the samples had unknown values in the thirty-eight fields we use for our analysis.   The CRSS authors imputed unknown values in ten of those fields, another seventeen had no unknown values, but eleven fields we want to use had missing values that were not imputed by CRSS.   The CRSS authors have a very helpful report on their imputation methods.  \citep{CRSS_Imputation}  The reasons why some fields get imputed include historical consistency going back to 1982.  

(See \verb|CRSS_04_5_Count_Missing_Values|)  

When the CRSS authors imputed unknown values for a field, they published two fields, one with the imputed values and one with the values signifying ``Unknown.''  We discarded the imputed fields and compared three methods for imputing missing values.  Impute to Mode assigns to all missing values in a feature the most common value in that feature.  IVEware: Imputation and Variance Estimation Software employs multivariate sequential regression, and is the method the CRSS authors used.  Round Robin Random Forest, like in MissForest, was consistently the most accurate.  We tested the methods by dropping all samples with missing values, randomly deleting (but keeping a copy of) fifteen percent of the known values, imputing, and comparing to the ground truth.  

(See \verb|CRSS_05_Impute_Random_Forest| for details.)

We did not address the question of incorrect data.  

%%%
\subsection{Selecting Features}

We selected three groups of features to see whether more information would improve the model.  

The first group of features held information that the police would already know before receiving a crash notification, like time of day, day of week, and urban/rural.  A crash on a Saturday night in a rural area is far more likely to need an ambulance than one in a city at rush hour, so if no information specific to the crash is available, how well can we predict whether an ambulance is needed?  We thought of this set of features as ``easy'' or ``baseline.''

The second group of features also included specific location and the age and sex of the primary user of the phone.  Is the vehicle in an intersection or in a parking lot?  Did the car end up off the roadway?  What is the speed limit on that road?  Getting that information from the latitude and longitude in the automated report would require instantaneous correlation with detailed maps.  Whether such information significantly improves the model will inform whether policymakers should invest the time and effort to have that information available.  We thought of this information as ``medium'' in cost.

The ``hard'' or ``expensive'' features would require regularly updated maps (work zones, lighting conditions), correlating records to guess which car the cell phone user is driving, and correlating multiple cell phone reports to count how many people are involved.  

We dropped all crashes with a pedestrian, because unlike a tree or other vehicle, hitting a pedestrian may not cause the sudden deceleration that a cell phone could distinguish from sudden braking, so the cell phone likely would not register it as a crash.  

(See \verb|CRSS_06_Build_Model| for details.)


%%%
\subsection{Handling Imbalanced Data}

In our dataset only about fifteen percent of the people needed an ambulance.   If a recommendation system never sent an ambulance, the model would have 85\% accuracy, but be useless.  Most algorithms for training models are designed for balanced data, with half of the samples in each of the negative and positive classes.  With an imbalanced data set we can address the imbalance in four levels:  Resampling the dataset, modifying the loss function, choosing metrics other than accuracy, and using learning methods that account for the imbalance.  

\subsubsection{Resampling the Dataset}

We can balance the dataset by undersampling the majority class (negative, ``No ambulance'') or oversampling the minority class (positive, ``Send Ambulance'').  To balance by undersampling would mean throwing out eighty percent of the majority class, losing valuable information.  A very popular method for oversampling is SMOTE (Synthetic Minority Oversampling TEchnique), which creates new minority samples between existing minority samples, but the ``between'' requires continuous data, and all of our data is discrete or categorical.  What is between a Buick and a Volvo?

Tomek Links is one of the few resampling methods that works for categorical data.  It is a selective undersampling method that removes majority samples that seem out of place.  A Tomek Link is a majority/minority pair that are each others' nearest neighbors, which was the case with about four percent of the majority samples.  We used the Tomek algorithm to remove the majority sample of each Tomek link, undersampling the majority class, and then running it again to remove more that had not been Tomek links in the first undersampling run.   


Consider this two-dimensional training dataset.  The six blue circles represent samples (elements) of the majority negative class (``no ambulance''), and the three red squares represent the minority positive class (``ambulance'').  Samples \#7 and \#1 are each others' nearest neighbors of different classes, so they are Tomek Links and the algorithm deletes \#1.  In a second Tomek run, once \#1 is gone, \#7 and \#2 are Tomek Links, so the method deletes \#2.  


\begin{center}
\begin{tikzpicture}[x = 3mm, y=3mm]
	\draw (-1,-1) rectangle (8,8);
	\tikzstyle{Square} = [
		draw = red, 
		very thick,
		rectangle,
		inner sep = 1mm,
		minimum size = 3 mm
	]
	\tikzstyle{SquareFill} = [
		draw = red, 
		fill = red,
		very thick,
		rectangle,
	]
	\tikzstyle{Circle} = [
		draw = blue, 
		circle,
		inner sep = 0.5 mm
	]
	\tikzstyle{CircleFill} = [
		draw = blue,
		fill = blue, 
		circle,
	]
	\node [Circle] (1) at (0,0) {1};
	\node [Circle] (2) at (2,4) {2};
	\node [Circle] (3) at (3,7) {3};
	\node [Circle] (4) at (7,2) {4};
	\node [Circle] (5) at (5,7) {5};
	\node [Circle] (6) at (7,6) {6};
	\node [Square] (7) at (0,2) {7};
	\node [Square] (8) at (3,0) {8};
	\node [Square] (9) at (5,0) {9};

	\node [Circle, cross out] (1) at (0,0) {1};
\end{tikzpicture}
\end{center}



Our original dataset has 619,027 samples.  We first removed the 27,723 crashes involving a pedestrian, leaving 591,304 samples.  Each sample had 82 features; we cut the number of features to 38 for our ``Hard'' features, then to 21 for ``Medium,'' and to 10 for ``Easy.''  We then split each of those three datasets 70/30 into a training set of 413,913 samples and a test set of 177,393 samples, preserving the proportions of negative and positive samples in both sets.  We did the train/test split twice with different random seeds (``Round 1'' and ``Round 2'') to gauge how much of the small differences in results were due to stochasticity instead of differences in the model algorithms or hyperparameters.  Tomek undersampling only applies to the training set, not to the test set.  

We then ran Imbalanced-Learn's  TomekLinks algorithm, then ran it again on the results to give our ``Tomek Once'' and ``Tomek Twice'' undersampled datasets.  

\

\hfil\begin{tabular}{lrrl}
\multicolumn{2}{l}{Hard Features, Round 1}   &  & \cr
 & Samples & \multicolumn{2}{c}{Change} \cr\hline
Original & 413,913 &  & \cr
Tomek Once & 399,515 & 14,398 & 3.48\%\cr
Tomek Twice & 396,511 & 3,004 & 0.75\%\cr\cline{3-4}
Total Change &  & 17,402 & 4.23\%\cr
\end{tabular}
\qquad\begin{tabular}{lrrl}
\multicolumn{2}{l}{Hard Features, Round 2} & \cr
 & Samples & \multicolumn{2}{c}{Change} \cr\hline
Original & 413,913 &  & \cr
Tomek Once & 399,714 & 14,199 & 3.43\%\cr
Tomek Twice & 396,718 & 2,996 & 0.75\%\cr\cline{3-4}
Total Change &  & 17,195 & 4.18\%\cr
\end{tabular}

\vskip 12pt

\hfil\begin{tabular}{lrrl}
\multicolumn{3}{l}{Medium Features, Round 1} & \cr
 & Samples & \multicolumn{2}{c}{Change} \cr\hline
Original & 413,913 &  & \cr
Tomek Once & 406,691 & 7,222 & 1.74\%\cr
Tomek Twice & 405,288 & 1,403 & 0.34\%\cr\cline{3-4}
Total Change &  & 8,625 & 2.08\%\cr
\end{tabular}
\qquad
\begin{tabular}{lrrl}
\multicolumn{3}{l}{Medium Features, Round 2} & \cr
 & Samples & \multicolumn{2}{c}{Change} \cr\hline
Original & 413,913 &  & \cr
Tomek Once & 406,781 & 7,132 & 1.72\%\cr
Tomek Twice & 405,368 & 1,413 & 0.35\%\cr\cline{3-4}
Total Change &  & 8,545 & 2.07\%\cr
\end{tabular}

\vskip 12pt

\hfil\begin{tabular}{lrrl}
\multicolumn{2}{l}{Easy Features, Round 1} & \cr
 & Samples & \multicolumn{2}{c}{Change} \cr\hline
Original & 413,913 &  & \cr
Tomek Once & 413,909 & 4 & 0.00097\%\cr
Tomek Twice & 413,908 & 1 & 0.00024\%\cr\cline{3-4}
Total Change &  & 5 & 0.00121\%\cr
\end{tabular}
\qquad
\begin{tabular}{lrrl}
\multicolumn{2}{l}{Easy Features, Round 2} & \cr
 & Samples & \multicolumn{2}{c}{Change} \cr\hline
Original & 413,913 &  & \cr
Tomek Once & 413,908 & 5 & 0.00121\%\cr
Tomek Twice & 413,907 & 1 & 0.00024\%\cr\cline{3-4}
Total Change &  & 6 & 0.00145\%\cr
\end{tabular}

\

We ran the models on the two rounds of Tomek undersampled training for the Hard-feature and Medium-feature sets, not for the Easy because the undersampling was so small.  

We were disappointed to not see a significant improvement in the model metrics from the undersampling; the difference between no undersampling, one runs of Tomek, and two runs turned out to be inconsequential, by which we mean that one approach was not consistently better when we ran the models with different random seeds.  


\subsubsection{Modifying the Loss Function}

A popular and well established way to modify the loss function for imbalanced data is with class weights, which can have the same effect as na{\"i}ve oversampling.  

Three of our seven models take class weights, and for those we tried three different class weights.  The Tomek undersampling changes the last weight slightly from $0.8499$ to as low as $0.8433$.

\

\hfil\begin{tabular}{c|l}
	$\alpha$ & Meaning \cr\hline
	1/2 & No class weight \cr
	2/3 & $\Delta FP/\Delta TP < 2.0$ goal \cr
	$0.85$ & Balanced classes \cr 
\end{tabular}

\


A related method is with focal loss, which has a modulating hyperparameter $\gamma$ that increases the penalty for low-confidence samples. \citep{lin2017focal}  We tried five values  of $\gamma$.

\

\hfil\begin{tabular}{c|c}
	$\gamma$  & Notes \cr\hline
	0.0 & Same as binary crossentropy \cr
	0.5 & Very light modulation \cr
	1.0 & Light modulation\cr
	2.0 & Recommended by Lin \cr
	5.0 & Heavy modulation \cr
\end{tabular}	

\

We did not see significant improvement using focal loss.  ({\bf Put in Label Reference}).

%%%
\subsubsection{Metrics for Imbalance}

In the \nameref{Methods_Metrics} subsection above we defined the metrics recall, precision, and f1.  The most common metric in machine learning, the one that most algorithms are designed to maximize, is accuracy, the proportion of samples correctly classified.  In that section's example of transformed model output, we had 150,107 out of 177,392 test samples correctly classified, giving 84.6\% accuracy.  Is that good?  The model below, the raw results of the Logistic Regression model of the easy features set, recommends sending no ambulances, and it is correct in 150,771 of 177,392 test samples, giving 84.99\% accuracy.  Is that better?



\

%%%
\parbox{\linewidth}{
%{\bf Balanced Random Forest model, Hard features, No Tomek, $\alpha = 2/3$}

\noindent\begin{tabular}{@{\hspace{-6pt}}p{2.3in} @{\hspace{-6pt}}p{2.0in} p{1.8in}}
	\vskip 0pt
	\qquad \qquad Raw Model Output
	
	%% Creator: Matplotlib, PGF backend
%%
%% To include the figure in your LaTeX document, write
%%   \input{<filename>.pgf}
%%
%% Make sure the required packages are loaded in your preamble
%%   \usepackage{pgf}
%%
%% Also ensure that all the required font packages are loaded; for instance,
%% the lmodern package is sometimes necessary when using math font.
%%   \usepackage{lmodern}
%%
%% Figures using additional raster images can only be included by \input if
%% they are in the same directory as the main LaTeX file. For loading figures
%% from other directories you can use the `import` package
%%   \usepackage{import}
%%
%% and then include the figures with
%%   \import{<path to file>}{<filename>.pgf}
%%
%% Matplotlib used the following preamble
%%   
%%   \usepackage{fontspec}
%%   \makeatletter\@ifpackageloaded{underscore}{}{\usepackage[strings]{underscore}}\makeatother
%%
\begingroup%
\makeatletter%
\begin{pgfpicture}%
\pgfpathrectangle{\pgfpointorigin}{\pgfqpoint{2.253750in}{1.754444in}}%
\pgfusepath{use as bounding box, clip}%
\begin{pgfscope}%
\pgfsetbuttcap%
\pgfsetmiterjoin%
\definecolor{currentfill}{rgb}{1.000000,1.000000,1.000000}%
\pgfsetfillcolor{currentfill}%
\pgfsetlinewidth{0.000000pt}%
\definecolor{currentstroke}{rgb}{1.000000,1.000000,1.000000}%
\pgfsetstrokecolor{currentstroke}%
\pgfsetdash{}{0pt}%
\pgfpathmoveto{\pgfqpoint{0.000000in}{0.000000in}}%
\pgfpathlineto{\pgfqpoint{2.253750in}{0.000000in}}%
\pgfpathlineto{\pgfqpoint{2.253750in}{1.754444in}}%
\pgfpathlineto{\pgfqpoint{0.000000in}{1.754444in}}%
\pgfpathlineto{\pgfqpoint{0.000000in}{0.000000in}}%
\pgfpathclose%
\pgfusepath{fill}%
\end{pgfscope}%
\begin{pgfscope}%
\pgfsetbuttcap%
\pgfsetmiterjoin%
\definecolor{currentfill}{rgb}{1.000000,1.000000,1.000000}%
\pgfsetfillcolor{currentfill}%
\pgfsetlinewidth{0.000000pt}%
\definecolor{currentstroke}{rgb}{0.000000,0.000000,0.000000}%
\pgfsetstrokecolor{currentstroke}%
\pgfsetstrokeopacity{0.000000}%
\pgfsetdash{}{0pt}%
\pgfpathmoveto{\pgfqpoint{0.515000in}{0.499444in}}%
\pgfpathlineto{\pgfqpoint{2.065000in}{0.499444in}}%
\pgfpathlineto{\pgfqpoint{2.065000in}{1.654444in}}%
\pgfpathlineto{\pgfqpoint{0.515000in}{1.654444in}}%
\pgfpathlineto{\pgfqpoint{0.515000in}{0.499444in}}%
\pgfpathclose%
\pgfusepath{fill}%
\end{pgfscope}%
\begin{pgfscope}%
\pgfpathrectangle{\pgfqpoint{0.515000in}{0.499444in}}{\pgfqpoint{1.550000in}{1.155000in}}%
\pgfusepath{clip}%
\pgfsetbuttcap%
\pgfsetmiterjoin%
\pgfsetlinewidth{1.003750pt}%
\definecolor{currentstroke}{rgb}{0.000000,0.000000,0.000000}%
\pgfsetstrokecolor{currentstroke}%
\pgfsetdash{}{0pt}%
\pgfpathmoveto{\pgfqpoint{0.505000in}{0.499444in}}%
\pgfpathlineto{\pgfqpoint{0.552805in}{0.499444in}}%
\pgfpathlineto{\pgfqpoint{0.552805in}{1.124643in}}%
\pgfpathlineto{\pgfqpoint{0.505000in}{1.124643in}}%
\pgfusepath{stroke}%
\end{pgfscope}%
\begin{pgfscope}%
\pgfpathrectangle{\pgfqpoint{0.515000in}{0.499444in}}{\pgfqpoint{1.550000in}{1.155000in}}%
\pgfusepath{clip}%
\pgfsetbuttcap%
\pgfsetmiterjoin%
\pgfsetlinewidth{1.003750pt}%
\definecolor{currentstroke}{rgb}{0.000000,0.000000,0.000000}%
\pgfsetstrokecolor{currentstroke}%
\pgfsetdash{}{0pt}%
\pgfpathmoveto{\pgfqpoint{0.643537in}{0.499444in}}%
\pgfpathlineto{\pgfqpoint{0.704025in}{0.499444in}}%
\pgfpathlineto{\pgfqpoint{0.704025in}{1.599444in}}%
\pgfpathlineto{\pgfqpoint{0.643537in}{1.599444in}}%
\pgfpathlineto{\pgfqpoint{0.643537in}{0.499444in}}%
\pgfpathclose%
\pgfusepath{stroke}%
\end{pgfscope}%
\begin{pgfscope}%
\pgfpathrectangle{\pgfqpoint{0.515000in}{0.499444in}}{\pgfqpoint{1.550000in}{1.155000in}}%
\pgfusepath{clip}%
\pgfsetbuttcap%
\pgfsetmiterjoin%
\pgfsetlinewidth{1.003750pt}%
\definecolor{currentstroke}{rgb}{0.000000,0.000000,0.000000}%
\pgfsetstrokecolor{currentstroke}%
\pgfsetdash{}{0pt}%
\pgfpathmoveto{\pgfqpoint{0.794756in}{0.499444in}}%
\pgfpathlineto{\pgfqpoint{0.855244in}{0.499444in}}%
\pgfpathlineto{\pgfqpoint{0.855244in}{0.882298in}}%
\pgfpathlineto{\pgfqpoint{0.794756in}{0.882298in}}%
\pgfpathlineto{\pgfqpoint{0.794756in}{0.499444in}}%
\pgfpathclose%
\pgfusepath{stroke}%
\end{pgfscope}%
\begin{pgfscope}%
\pgfpathrectangle{\pgfqpoint{0.515000in}{0.499444in}}{\pgfqpoint{1.550000in}{1.155000in}}%
\pgfusepath{clip}%
\pgfsetbuttcap%
\pgfsetmiterjoin%
\pgfsetlinewidth{1.003750pt}%
\definecolor{currentstroke}{rgb}{0.000000,0.000000,0.000000}%
\pgfsetstrokecolor{currentstroke}%
\pgfsetdash{}{0pt}%
\pgfpathmoveto{\pgfqpoint{0.945976in}{0.499444in}}%
\pgfpathlineto{\pgfqpoint{1.006464in}{0.499444in}}%
\pgfpathlineto{\pgfqpoint{1.006464in}{0.555351in}}%
\pgfpathlineto{\pgfqpoint{0.945976in}{0.555351in}}%
\pgfpathlineto{\pgfqpoint{0.945976in}{0.499444in}}%
\pgfpathclose%
\pgfusepath{stroke}%
\end{pgfscope}%
\begin{pgfscope}%
\pgfpathrectangle{\pgfqpoint{0.515000in}{0.499444in}}{\pgfqpoint{1.550000in}{1.155000in}}%
\pgfusepath{clip}%
\pgfsetbuttcap%
\pgfsetmiterjoin%
\pgfsetlinewidth{1.003750pt}%
\definecolor{currentstroke}{rgb}{0.000000,0.000000,0.000000}%
\pgfsetstrokecolor{currentstroke}%
\pgfsetdash{}{0pt}%
\pgfpathmoveto{\pgfqpoint{1.097195in}{0.499444in}}%
\pgfpathlineto{\pgfqpoint{1.157683in}{0.499444in}}%
\pgfpathlineto{\pgfqpoint{1.157683in}{0.502923in}}%
\pgfpathlineto{\pgfqpoint{1.097195in}{0.502923in}}%
\pgfpathlineto{\pgfqpoint{1.097195in}{0.499444in}}%
\pgfpathclose%
\pgfusepath{stroke}%
\end{pgfscope}%
\begin{pgfscope}%
\pgfpathrectangle{\pgfqpoint{0.515000in}{0.499444in}}{\pgfqpoint{1.550000in}{1.155000in}}%
\pgfusepath{clip}%
\pgfsetbuttcap%
\pgfsetmiterjoin%
\pgfsetlinewidth{1.003750pt}%
\definecolor{currentstroke}{rgb}{0.000000,0.000000,0.000000}%
\pgfsetstrokecolor{currentstroke}%
\pgfsetdash{}{0pt}%
\pgfpathmoveto{\pgfqpoint{1.248415in}{0.499444in}}%
\pgfpathlineto{\pgfqpoint{1.308903in}{0.499444in}}%
\pgfpathlineto{\pgfqpoint{1.308903in}{0.499444in}}%
\pgfpathlineto{\pgfqpoint{1.248415in}{0.499444in}}%
\pgfpathlineto{\pgfqpoint{1.248415in}{0.499444in}}%
\pgfpathclose%
\pgfusepath{stroke}%
\end{pgfscope}%
\begin{pgfscope}%
\pgfpathrectangle{\pgfqpoint{0.515000in}{0.499444in}}{\pgfqpoint{1.550000in}{1.155000in}}%
\pgfusepath{clip}%
\pgfsetbuttcap%
\pgfsetmiterjoin%
\pgfsetlinewidth{1.003750pt}%
\definecolor{currentstroke}{rgb}{0.000000,0.000000,0.000000}%
\pgfsetstrokecolor{currentstroke}%
\pgfsetdash{}{0pt}%
\pgfpathmoveto{\pgfqpoint{1.399634in}{0.499444in}}%
\pgfpathlineto{\pgfqpoint{1.460122in}{0.499444in}}%
\pgfpathlineto{\pgfqpoint{1.460122in}{0.499444in}}%
\pgfpathlineto{\pgfqpoint{1.399634in}{0.499444in}}%
\pgfpathlineto{\pgfqpoint{1.399634in}{0.499444in}}%
\pgfpathclose%
\pgfusepath{stroke}%
\end{pgfscope}%
\begin{pgfscope}%
\pgfpathrectangle{\pgfqpoint{0.515000in}{0.499444in}}{\pgfqpoint{1.550000in}{1.155000in}}%
\pgfusepath{clip}%
\pgfsetbuttcap%
\pgfsetmiterjoin%
\pgfsetlinewidth{1.003750pt}%
\definecolor{currentstroke}{rgb}{0.000000,0.000000,0.000000}%
\pgfsetstrokecolor{currentstroke}%
\pgfsetdash{}{0pt}%
\pgfpathmoveto{\pgfqpoint{1.550854in}{0.499444in}}%
\pgfpathlineto{\pgfqpoint{1.611342in}{0.499444in}}%
\pgfpathlineto{\pgfqpoint{1.611342in}{0.499444in}}%
\pgfpathlineto{\pgfqpoint{1.550854in}{0.499444in}}%
\pgfpathlineto{\pgfqpoint{1.550854in}{0.499444in}}%
\pgfpathclose%
\pgfusepath{stroke}%
\end{pgfscope}%
\begin{pgfscope}%
\pgfpathrectangle{\pgfqpoint{0.515000in}{0.499444in}}{\pgfqpoint{1.550000in}{1.155000in}}%
\pgfusepath{clip}%
\pgfsetbuttcap%
\pgfsetmiterjoin%
\pgfsetlinewidth{1.003750pt}%
\definecolor{currentstroke}{rgb}{0.000000,0.000000,0.000000}%
\pgfsetstrokecolor{currentstroke}%
\pgfsetdash{}{0pt}%
\pgfpathmoveto{\pgfqpoint{1.702073in}{0.499444in}}%
\pgfpathlineto{\pgfqpoint{1.762561in}{0.499444in}}%
\pgfpathlineto{\pgfqpoint{1.762561in}{0.499444in}}%
\pgfpathlineto{\pgfqpoint{1.702073in}{0.499444in}}%
\pgfpathlineto{\pgfqpoint{1.702073in}{0.499444in}}%
\pgfpathclose%
\pgfusepath{stroke}%
\end{pgfscope}%
\begin{pgfscope}%
\pgfpathrectangle{\pgfqpoint{0.515000in}{0.499444in}}{\pgfqpoint{1.550000in}{1.155000in}}%
\pgfusepath{clip}%
\pgfsetbuttcap%
\pgfsetmiterjoin%
\pgfsetlinewidth{1.003750pt}%
\definecolor{currentstroke}{rgb}{0.000000,0.000000,0.000000}%
\pgfsetstrokecolor{currentstroke}%
\pgfsetdash{}{0pt}%
\pgfpathmoveto{\pgfqpoint{1.853293in}{0.499444in}}%
\pgfpathlineto{\pgfqpoint{1.913781in}{0.499444in}}%
\pgfpathlineto{\pgfqpoint{1.913781in}{0.499444in}}%
\pgfpathlineto{\pgfqpoint{1.853293in}{0.499444in}}%
\pgfpathlineto{\pgfqpoint{1.853293in}{0.499444in}}%
\pgfpathclose%
\pgfusepath{stroke}%
\end{pgfscope}%
\begin{pgfscope}%
\pgfpathrectangle{\pgfqpoint{0.515000in}{0.499444in}}{\pgfqpoint{1.550000in}{1.155000in}}%
\pgfusepath{clip}%
\pgfsetbuttcap%
\pgfsetmiterjoin%
\definecolor{currentfill}{rgb}{0.000000,0.000000,0.000000}%
\pgfsetfillcolor{currentfill}%
\pgfsetlinewidth{0.000000pt}%
\definecolor{currentstroke}{rgb}{0.000000,0.000000,0.000000}%
\pgfsetstrokecolor{currentstroke}%
\pgfsetstrokeopacity{0.000000}%
\pgfsetdash{}{0pt}%
\pgfpathmoveto{\pgfqpoint{0.552805in}{0.499444in}}%
\pgfpathlineto{\pgfqpoint{0.613293in}{0.499444in}}%
\pgfpathlineto{\pgfqpoint{0.613293in}{0.545216in}}%
\pgfpathlineto{\pgfqpoint{0.552805in}{0.545216in}}%
\pgfpathlineto{\pgfqpoint{0.552805in}{0.499444in}}%
\pgfpathclose%
\pgfusepath{fill}%
\end{pgfscope}%
\begin{pgfscope}%
\pgfpathrectangle{\pgfqpoint{0.515000in}{0.499444in}}{\pgfqpoint{1.550000in}{1.155000in}}%
\pgfusepath{clip}%
\pgfsetbuttcap%
\pgfsetmiterjoin%
\definecolor{currentfill}{rgb}{0.000000,0.000000,0.000000}%
\pgfsetfillcolor{currentfill}%
\pgfsetlinewidth{0.000000pt}%
\definecolor{currentstroke}{rgb}{0.000000,0.000000,0.000000}%
\pgfsetstrokecolor{currentstroke}%
\pgfsetstrokeopacity{0.000000}%
\pgfsetdash{}{0pt}%
\pgfpathmoveto{\pgfqpoint{0.704025in}{0.499444in}}%
\pgfpathlineto{\pgfqpoint{0.764512in}{0.499444in}}%
\pgfpathlineto{\pgfqpoint{0.764512in}{0.682533in}}%
\pgfpathlineto{\pgfqpoint{0.704025in}{0.682533in}}%
\pgfpathlineto{\pgfqpoint{0.704025in}{0.499444in}}%
\pgfpathclose%
\pgfusepath{fill}%
\end{pgfscope}%
\begin{pgfscope}%
\pgfpathrectangle{\pgfqpoint{0.515000in}{0.499444in}}{\pgfqpoint{1.550000in}{1.155000in}}%
\pgfusepath{clip}%
\pgfsetbuttcap%
\pgfsetmiterjoin%
\definecolor{currentfill}{rgb}{0.000000,0.000000,0.000000}%
\pgfsetfillcolor{currentfill}%
\pgfsetlinewidth{0.000000pt}%
\definecolor{currentstroke}{rgb}{0.000000,0.000000,0.000000}%
\pgfsetstrokecolor{currentstroke}%
\pgfsetstrokeopacity{0.000000}%
\pgfsetdash{}{0pt}%
\pgfpathmoveto{\pgfqpoint{0.855244in}{0.499444in}}%
\pgfpathlineto{\pgfqpoint{0.915732in}{0.499444in}}%
\pgfpathlineto{\pgfqpoint{0.915732in}{0.625102in}}%
\pgfpathlineto{\pgfqpoint{0.855244in}{0.625102in}}%
\pgfpathlineto{\pgfqpoint{0.855244in}{0.499444in}}%
\pgfpathclose%
\pgfusepath{fill}%
\end{pgfscope}%
\begin{pgfscope}%
\pgfpathrectangle{\pgfqpoint{0.515000in}{0.499444in}}{\pgfqpoint{1.550000in}{1.155000in}}%
\pgfusepath{clip}%
\pgfsetbuttcap%
\pgfsetmiterjoin%
\definecolor{currentfill}{rgb}{0.000000,0.000000,0.000000}%
\pgfsetfillcolor{currentfill}%
\pgfsetlinewidth{0.000000pt}%
\definecolor{currentstroke}{rgb}{0.000000,0.000000,0.000000}%
\pgfsetstrokecolor{currentstroke}%
\pgfsetstrokeopacity{0.000000}%
\pgfsetdash{}{0pt}%
\pgfpathmoveto{\pgfqpoint{1.006464in}{0.499444in}}%
\pgfpathlineto{\pgfqpoint{1.066951in}{0.499444in}}%
\pgfpathlineto{\pgfqpoint{1.066951in}{0.525522in}}%
\pgfpathlineto{\pgfqpoint{1.006464in}{0.525522in}}%
\pgfpathlineto{\pgfqpoint{1.006464in}{0.499444in}}%
\pgfpathclose%
\pgfusepath{fill}%
\end{pgfscope}%
\begin{pgfscope}%
\pgfpathrectangle{\pgfqpoint{0.515000in}{0.499444in}}{\pgfqpoint{1.550000in}{1.155000in}}%
\pgfusepath{clip}%
\pgfsetbuttcap%
\pgfsetmiterjoin%
\definecolor{currentfill}{rgb}{0.000000,0.000000,0.000000}%
\pgfsetfillcolor{currentfill}%
\pgfsetlinewidth{0.000000pt}%
\definecolor{currentstroke}{rgb}{0.000000,0.000000,0.000000}%
\pgfsetstrokecolor{currentstroke}%
\pgfsetstrokeopacity{0.000000}%
\pgfsetdash{}{0pt}%
\pgfpathmoveto{\pgfqpoint{1.157683in}{0.499444in}}%
\pgfpathlineto{\pgfqpoint{1.218171in}{0.499444in}}%
\pgfpathlineto{\pgfqpoint{1.218171in}{0.501543in}}%
\pgfpathlineto{\pgfqpoint{1.157683in}{0.501543in}}%
\pgfpathlineto{\pgfqpoint{1.157683in}{0.499444in}}%
\pgfpathclose%
\pgfusepath{fill}%
\end{pgfscope}%
\begin{pgfscope}%
\pgfpathrectangle{\pgfqpoint{0.515000in}{0.499444in}}{\pgfqpoint{1.550000in}{1.155000in}}%
\pgfusepath{clip}%
\pgfsetbuttcap%
\pgfsetmiterjoin%
\definecolor{currentfill}{rgb}{0.000000,0.000000,0.000000}%
\pgfsetfillcolor{currentfill}%
\pgfsetlinewidth{0.000000pt}%
\definecolor{currentstroke}{rgb}{0.000000,0.000000,0.000000}%
\pgfsetstrokecolor{currentstroke}%
\pgfsetstrokeopacity{0.000000}%
\pgfsetdash{}{0pt}%
\pgfpathmoveto{\pgfqpoint{1.308903in}{0.499444in}}%
\pgfpathlineto{\pgfqpoint{1.369391in}{0.499444in}}%
\pgfpathlineto{\pgfqpoint{1.369391in}{0.499444in}}%
\pgfpathlineto{\pgfqpoint{1.308903in}{0.499444in}}%
\pgfpathlineto{\pgfqpoint{1.308903in}{0.499444in}}%
\pgfpathclose%
\pgfusepath{fill}%
\end{pgfscope}%
\begin{pgfscope}%
\pgfpathrectangle{\pgfqpoint{0.515000in}{0.499444in}}{\pgfqpoint{1.550000in}{1.155000in}}%
\pgfusepath{clip}%
\pgfsetbuttcap%
\pgfsetmiterjoin%
\definecolor{currentfill}{rgb}{0.000000,0.000000,0.000000}%
\pgfsetfillcolor{currentfill}%
\pgfsetlinewidth{0.000000pt}%
\definecolor{currentstroke}{rgb}{0.000000,0.000000,0.000000}%
\pgfsetstrokecolor{currentstroke}%
\pgfsetstrokeopacity{0.000000}%
\pgfsetdash{}{0pt}%
\pgfpathmoveto{\pgfqpoint{1.460122in}{0.499444in}}%
\pgfpathlineto{\pgfqpoint{1.520610in}{0.499444in}}%
\pgfpathlineto{\pgfqpoint{1.520610in}{0.499444in}}%
\pgfpathlineto{\pgfqpoint{1.460122in}{0.499444in}}%
\pgfpathlineto{\pgfqpoint{1.460122in}{0.499444in}}%
\pgfpathclose%
\pgfusepath{fill}%
\end{pgfscope}%
\begin{pgfscope}%
\pgfpathrectangle{\pgfqpoint{0.515000in}{0.499444in}}{\pgfqpoint{1.550000in}{1.155000in}}%
\pgfusepath{clip}%
\pgfsetbuttcap%
\pgfsetmiterjoin%
\definecolor{currentfill}{rgb}{0.000000,0.000000,0.000000}%
\pgfsetfillcolor{currentfill}%
\pgfsetlinewidth{0.000000pt}%
\definecolor{currentstroke}{rgb}{0.000000,0.000000,0.000000}%
\pgfsetstrokecolor{currentstroke}%
\pgfsetstrokeopacity{0.000000}%
\pgfsetdash{}{0pt}%
\pgfpathmoveto{\pgfqpoint{1.611342in}{0.499444in}}%
\pgfpathlineto{\pgfqpoint{1.671830in}{0.499444in}}%
\pgfpathlineto{\pgfqpoint{1.671830in}{0.499444in}}%
\pgfpathlineto{\pgfqpoint{1.611342in}{0.499444in}}%
\pgfpathlineto{\pgfqpoint{1.611342in}{0.499444in}}%
\pgfpathclose%
\pgfusepath{fill}%
\end{pgfscope}%
\begin{pgfscope}%
\pgfpathrectangle{\pgfqpoint{0.515000in}{0.499444in}}{\pgfqpoint{1.550000in}{1.155000in}}%
\pgfusepath{clip}%
\pgfsetbuttcap%
\pgfsetmiterjoin%
\definecolor{currentfill}{rgb}{0.000000,0.000000,0.000000}%
\pgfsetfillcolor{currentfill}%
\pgfsetlinewidth{0.000000pt}%
\definecolor{currentstroke}{rgb}{0.000000,0.000000,0.000000}%
\pgfsetstrokecolor{currentstroke}%
\pgfsetstrokeopacity{0.000000}%
\pgfsetdash{}{0pt}%
\pgfpathmoveto{\pgfqpoint{1.762561in}{0.499444in}}%
\pgfpathlineto{\pgfqpoint{1.823049in}{0.499444in}}%
\pgfpathlineto{\pgfqpoint{1.823049in}{0.499444in}}%
\pgfpathlineto{\pgfqpoint{1.762561in}{0.499444in}}%
\pgfpathlineto{\pgfqpoint{1.762561in}{0.499444in}}%
\pgfpathclose%
\pgfusepath{fill}%
\end{pgfscope}%
\begin{pgfscope}%
\pgfpathrectangle{\pgfqpoint{0.515000in}{0.499444in}}{\pgfqpoint{1.550000in}{1.155000in}}%
\pgfusepath{clip}%
\pgfsetbuttcap%
\pgfsetmiterjoin%
\definecolor{currentfill}{rgb}{0.000000,0.000000,0.000000}%
\pgfsetfillcolor{currentfill}%
\pgfsetlinewidth{0.000000pt}%
\definecolor{currentstroke}{rgb}{0.000000,0.000000,0.000000}%
\pgfsetstrokecolor{currentstroke}%
\pgfsetstrokeopacity{0.000000}%
\pgfsetdash{}{0pt}%
\pgfpathmoveto{\pgfqpoint{1.913781in}{0.499444in}}%
\pgfpathlineto{\pgfqpoint{1.974269in}{0.499444in}}%
\pgfpathlineto{\pgfqpoint{1.974269in}{0.499444in}}%
\pgfpathlineto{\pgfqpoint{1.913781in}{0.499444in}}%
\pgfpathlineto{\pgfqpoint{1.913781in}{0.499444in}}%
\pgfpathclose%
\pgfusepath{fill}%
\end{pgfscope}%
\begin{pgfscope}%
\pgfsetbuttcap%
\pgfsetroundjoin%
\definecolor{currentfill}{rgb}{0.000000,0.000000,0.000000}%
\pgfsetfillcolor{currentfill}%
\pgfsetlinewidth{0.803000pt}%
\definecolor{currentstroke}{rgb}{0.000000,0.000000,0.000000}%
\pgfsetstrokecolor{currentstroke}%
\pgfsetdash{}{0pt}%
\pgfsys@defobject{currentmarker}{\pgfqpoint{0.000000in}{-0.048611in}}{\pgfqpoint{0.000000in}{0.000000in}}{%
\pgfpathmoveto{\pgfqpoint{0.000000in}{0.000000in}}%
\pgfpathlineto{\pgfqpoint{0.000000in}{-0.048611in}}%
\pgfusepath{stroke,fill}%
}%
\begin{pgfscope}%
\pgfsys@transformshift{0.552805in}{0.499444in}%
\pgfsys@useobject{currentmarker}{}%
\end{pgfscope}%
\end{pgfscope}%
\begin{pgfscope}%
\definecolor{textcolor}{rgb}{0.000000,0.000000,0.000000}%
\pgfsetstrokecolor{textcolor}%
\pgfsetfillcolor{textcolor}%
\pgftext[x=0.552805in,y=0.402222in,,top]{\color{textcolor}\rmfamily\fontsize{10.000000}{12.000000}\selectfont 0.0}%
\end{pgfscope}%
\begin{pgfscope}%
\pgfsetbuttcap%
\pgfsetroundjoin%
\definecolor{currentfill}{rgb}{0.000000,0.000000,0.000000}%
\pgfsetfillcolor{currentfill}%
\pgfsetlinewidth{0.803000pt}%
\definecolor{currentstroke}{rgb}{0.000000,0.000000,0.000000}%
\pgfsetstrokecolor{currentstroke}%
\pgfsetdash{}{0pt}%
\pgfsys@defobject{currentmarker}{\pgfqpoint{0.000000in}{-0.048611in}}{\pgfqpoint{0.000000in}{0.000000in}}{%
\pgfpathmoveto{\pgfqpoint{0.000000in}{0.000000in}}%
\pgfpathlineto{\pgfqpoint{0.000000in}{-0.048611in}}%
\pgfusepath{stroke,fill}%
}%
\begin{pgfscope}%
\pgfsys@transformshift{0.930854in}{0.499444in}%
\pgfsys@useobject{currentmarker}{}%
\end{pgfscope}%
\end{pgfscope}%
\begin{pgfscope}%
\definecolor{textcolor}{rgb}{0.000000,0.000000,0.000000}%
\pgfsetstrokecolor{textcolor}%
\pgfsetfillcolor{textcolor}%
\pgftext[x=0.930854in,y=0.402222in,,top]{\color{textcolor}\rmfamily\fontsize{10.000000}{12.000000}\selectfont 0.25}%
\end{pgfscope}%
\begin{pgfscope}%
\pgfsetbuttcap%
\pgfsetroundjoin%
\definecolor{currentfill}{rgb}{0.000000,0.000000,0.000000}%
\pgfsetfillcolor{currentfill}%
\pgfsetlinewidth{0.803000pt}%
\definecolor{currentstroke}{rgb}{0.000000,0.000000,0.000000}%
\pgfsetstrokecolor{currentstroke}%
\pgfsetdash{}{0pt}%
\pgfsys@defobject{currentmarker}{\pgfqpoint{0.000000in}{-0.048611in}}{\pgfqpoint{0.000000in}{0.000000in}}{%
\pgfpathmoveto{\pgfqpoint{0.000000in}{0.000000in}}%
\pgfpathlineto{\pgfqpoint{0.000000in}{-0.048611in}}%
\pgfusepath{stroke,fill}%
}%
\begin{pgfscope}%
\pgfsys@transformshift{1.308903in}{0.499444in}%
\pgfsys@useobject{currentmarker}{}%
\end{pgfscope}%
\end{pgfscope}%
\begin{pgfscope}%
\definecolor{textcolor}{rgb}{0.000000,0.000000,0.000000}%
\pgfsetstrokecolor{textcolor}%
\pgfsetfillcolor{textcolor}%
\pgftext[x=1.308903in,y=0.402222in,,top]{\color{textcolor}\rmfamily\fontsize{10.000000}{12.000000}\selectfont 0.5}%
\end{pgfscope}%
\begin{pgfscope}%
\pgfsetbuttcap%
\pgfsetroundjoin%
\definecolor{currentfill}{rgb}{0.000000,0.000000,0.000000}%
\pgfsetfillcolor{currentfill}%
\pgfsetlinewidth{0.803000pt}%
\definecolor{currentstroke}{rgb}{0.000000,0.000000,0.000000}%
\pgfsetstrokecolor{currentstroke}%
\pgfsetdash{}{0pt}%
\pgfsys@defobject{currentmarker}{\pgfqpoint{0.000000in}{-0.048611in}}{\pgfqpoint{0.000000in}{0.000000in}}{%
\pgfpathmoveto{\pgfqpoint{0.000000in}{0.000000in}}%
\pgfpathlineto{\pgfqpoint{0.000000in}{-0.048611in}}%
\pgfusepath{stroke,fill}%
}%
\begin{pgfscope}%
\pgfsys@transformshift{1.686951in}{0.499444in}%
\pgfsys@useobject{currentmarker}{}%
\end{pgfscope}%
\end{pgfscope}%
\begin{pgfscope}%
\definecolor{textcolor}{rgb}{0.000000,0.000000,0.000000}%
\pgfsetstrokecolor{textcolor}%
\pgfsetfillcolor{textcolor}%
\pgftext[x=1.686951in,y=0.402222in,,top]{\color{textcolor}\rmfamily\fontsize{10.000000}{12.000000}\selectfont 0.75}%
\end{pgfscope}%
\begin{pgfscope}%
\pgfsetbuttcap%
\pgfsetroundjoin%
\definecolor{currentfill}{rgb}{0.000000,0.000000,0.000000}%
\pgfsetfillcolor{currentfill}%
\pgfsetlinewidth{0.803000pt}%
\definecolor{currentstroke}{rgb}{0.000000,0.000000,0.000000}%
\pgfsetstrokecolor{currentstroke}%
\pgfsetdash{}{0pt}%
\pgfsys@defobject{currentmarker}{\pgfqpoint{0.000000in}{-0.048611in}}{\pgfqpoint{0.000000in}{0.000000in}}{%
\pgfpathmoveto{\pgfqpoint{0.000000in}{0.000000in}}%
\pgfpathlineto{\pgfqpoint{0.000000in}{-0.048611in}}%
\pgfusepath{stroke,fill}%
}%
\begin{pgfscope}%
\pgfsys@transformshift{2.065000in}{0.499444in}%
\pgfsys@useobject{currentmarker}{}%
\end{pgfscope}%
\end{pgfscope}%
\begin{pgfscope}%
\definecolor{textcolor}{rgb}{0.000000,0.000000,0.000000}%
\pgfsetstrokecolor{textcolor}%
\pgfsetfillcolor{textcolor}%
\pgftext[x=2.065000in,y=0.402222in,,top]{\color{textcolor}\rmfamily\fontsize{10.000000}{12.000000}\selectfont 1.0}%
\end{pgfscope}%
\begin{pgfscope}%
\definecolor{textcolor}{rgb}{0.000000,0.000000,0.000000}%
\pgfsetstrokecolor{textcolor}%
\pgfsetfillcolor{textcolor}%
\pgftext[x=1.290000in,y=0.223333in,,top]{\color{textcolor}\rmfamily\fontsize{10.000000}{12.000000}\selectfont \(\displaystyle p\)}%
\end{pgfscope}%
\begin{pgfscope}%
\pgfsetbuttcap%
\pgfsetroundjoin%
\definecolor{currentfill}{rgb}{0.000000,0.000000,0.000000}%
\pgfsetfillcolor{currentfill}%
\pgfsetlinewidth{0.803000pt}%
\definecolor{currentstroke}{rgb}{0.000000,0.000000,0.000000}%
\pgfsetstrokecolor{currentstroke}%
\pgfsetdash{}{0pt}%
\pgfsys@defobject{currentmarker}{\pgfqpoint{-0.048611in}{0.000000in}}{\pgfqpoint{-0.000000in}{0.000000in}}{%
\pgfpathmoveto{\pgfqpoint{-0.000000in}{0.000000in}}%
\pgfpathlineto{\pgfqpoint{-0.048611in}{0.000000in}}%
\pgfusepath{stroke,fill}%
}%
\begin{pgfscope}%
\pgfsys@transformshift{0.515000in}{0.499444in}%
\pgfsys@useobject{currentmarker}{}%
\end{pgfscope}%
\end{pgfscope}%
\begin{pgfscope}%
\definecolor{textcolor}{rgb}{0.000000,0.000000,0.000000}%
\pgfsetstrokecolor{textcolor}%
\pgfsetfillcolor{textcolor}%
\pgftext[x=0.348333in, y=0.451250in, left, base]{\color{textcolor}\rmfamily\fontsize{10.000000}{12.000000}\selectfont \(\displaystyle {0}\)}%
\end{pgfscope}%
\begin{pgfscope}%
\pgfsetbuttcap%
\pgfsetroundjoin%
\definecolor{currentfill}{rgb}{0.000000,0.000000,0.000000}%
\pgfsetfillcolor{currentfill}%
\pgfsetlinewidth{0.803000pt}%
\definecolor{currentstroke}{rgb}{0.000000,0.000000,0.000000}%
\pgfsetstrokecolor{currentstroke}%
\pgfsetdash{}{0pt}%
\pgfsys@defobject{currentmarker}{\pgfqpoint{-0.048611in}{0.000000in}}{\pgfqpoint{-0.000000in}{0.000000in}}{%
\pgfpathmoveto{\pgfqpoint{-0.000000in}{0.000000in}}%
\pgfpathlineto{\pgfqpoint{-0.048611in}{0.000000in}}%
\pgfusepath{stroke,fill}%
}%
\begin{pgfscope}%
\pgfsys@transformshift{0.515000in}{1.009471in}%
\pgfsys@useobject{currentmarker}{}%
\end{pgfscope}%
\end{pgfscope}%
\begin{pgfscope}%
\definecolor{textcolor}{rgb}{0.000000,0.000000,0.000000}%
\pgfsetstrokecolor{textcolor}%
\pgfsetfillcolor{textcolor}%
\pgftext[x=0.278889in, y=0.961277in, left, base]{\color{textcolor}\rmfamily\fontsize{10.000000}{12.000000}\selectfont \(\displaystyle {20}\)}%
\end{pgfscope}%
\begin{pgfscope}%
\pgfsetbuttcap%
\pgfsetroundjoin%
\definecolor{currentfill}{rgb}{0.000000,0.000000,0.000000}%
\pgfsetfillcolor{currentfill}%
\pgfsetlinewidth{0.803000pt}%
\definecolor{currentstroke}{rgb}{0.000000,0.000000,0.000000}%
\pgfsetstrokecolor{currentstroke}%
\pgfsetdash{}{0pt}%
\pgfsys@defobject{currentmarker}{\pgfqpoint{-0.048611in}{0.000000in}}{\pgfqpoint{-0.000000in}{0.000000in}}{%
\pgfpathmoveto{\pgfqpoint{-0.000000in}{0.000000in}}%
\pgfpathlineto{\pgfqpoint{-0.048611in}{0.000000in}}%
\pgfusepath{stroke,fill}%
}%
\begin{pgfscope}%
\pgfsys@transformshift{0.515000in}{1.519498in}%
\pgfsys@useobject{currentmarker}{}%
\end{pgfscope}%
\end{pgfscope}%
\begin{pgfscope}%
\definecolor{textcolor}{rgb}{0.000000,0.000000,0.000000}%
\pgfsetstrokecolor{textcolor}%
\pgfsetfillcolor{textcolor}%
\pgftext[x=0.278889in, y=1.471304in, left, base]{\color{textcolor}\rmfamily\fontsize{10.000000}{12.000000}\selectfont \(\displaystyle {40}\)}%
\end{pgfscope}%
\begin{pgfscope}%
\definecolor{textcolor}{rgb}{0.000000,0.000000,0.000000}%
\pgfsetstrokecolor{textcolor}%
\pgfsetfillcolor{textcolor}%
\pgftext[x=0.223333in,y=1.076944in,,bottom,rotate=90.000000]{\color{textcolor}\rmfamily\fontsize{10.000000}{12.000000}\selectfont Percent of Data Set}%
\end{pgfscope}%
\begin{pgfscope}%
\pgfsetrectcap%
\pgfsetmiterjoin%
\pgfsetlinewidth{0.803000pt}%
\definecolor{currentstroke}{rgb}{0.000000,0.000000,0.000000}%
\pgfsetstrokecolor{currentstroke}%
\pgfsetdash{}{0pt}%
\pgfpathmoveto{\pgfqpoint{0.515000in}{0.499444in}}%
\pgfpathlineto{\pgfqpoint{0.515000in}{1.654444in}}%
\pgfusepath{stroke}%
\end{pgfscope}%
\begin{pgfscope}%
\pgfsetrectcap%
\pgfsetmiterjoin%
\pgfsetlinewidth{0.803000pt}%
\definecolor{currentstroke}{rgb}{0.000000,0.000000,0.000000}%
\pgfsetstrokecolor{currentstroke}%
\pgfsetdash{}{0pt}%
\pgfpathmoveto{\pgfqpoint{2.065000in}{0.499444in}}%
\pgfpathlineto{\pgfqpoint{2.065000in}{1.654444in}}%
\pgfusepath{stroke}%
\end{pgfscope}%
\begin{pgfscope}%
\pgfsetrectcap%
\pgfsetmiterjoin%
\pgfsetlinewidth{0.803000pt}%
\definecolor{currentstroke}{rgb}{0.000000,0.000000,0.000000}%
\pgfsetstrokecolor{currentstroke}%
\pgfsetdash{}{0pt}%
\pgfpathmoveto{\pgfqpoint{0.515000in}{0.499444in}}%
\pgfpathlineto{\pgfqpoint{2.065000in}{0.499444in}}%
\pgfusepath{stroke}%
\end{pgfscope}%
\begin{pgfscope}%
\pgfsetrectcap%
\pgfsetmiterjoin%
\pgfsetlinewidth{0.803000pt}%
\definecolor{currentstroke}{rgb}{0.000000,0.000000,0.000000}%
\pgfsetstrokecolor{currentstroke}%
\pgfsetdash{}{0pt}%
\pgfpathmoveto{\pgfqpoint{0.515000in}{1.654444in}}%
\pgfpathlineto{\pgfqpoint{2.065000in}{1.654444in}}%
\pgfusepath{stroke}%
\end{pgfscope}%
\begin{pgfscope}%
\pgfsetbuttcap%
\pgfsetmiterjoin%
\definecolor{currentfill}{rgb}{1.000000,1.000000,1.000000}%
\pgfsetfillcolor{currentfill}%
\pgfsetfillopacity{0.800000}%
\pgfsetlinewidth{1.003750pt}%
\definecolor{currentstroke}{rgb}{0.800000,0.800000,0.800000}%
\pgfsetstrokecolor{currentstroke}%
\pgfsetstrokeopacity{0.800000}%
\pgfsetdash{}{0pt}%
\pgfpathmoveto{\pgfqpoint{1.288056in}{1.154445in}}%
\pgfpathlineto{\pgfqpoint{1.967778in}{1.154445in}}%
\pgfpathquadraticcurveto{\pgfqpoint{1.995556in}{1.154445in}}{\pgfqpoint{1.995556in}{1.182222in}}%
\pgfpathlineto{\pgfqpoint{1.995556in}{1.557222in}}%
\pgfpathquadraticcurveto{\pgfqpoint{1.995556in}{1.585000in}}{\pgfqpoint{1.967778in}{1.585000in}}%
\pgfpathlineto{\pgfqpoint{1.288056in}{1.585000in}}%
\pgfpathquadraticcurveto{\pgfqpoint{1.260278in}{1.585000in}}{\pgfqpoint{1.260278in}{1.557222in}}%
\pgfpathlineto{\pgfqpoint{1.260278in}{1.182222in}}%
\pgfpathquadraticcurveto{\pgfqpoint{1.260278in}{1.154445in}}{\pgfqpoint{1.288056in}{1.154445in}}%
\pgfpathlineto{\pgfqpoint{1.288056in}{1.154445in}}%
\pgfpathclose%
\pgfusepath{stroke,fill}%
\end{pgfscope}%
\begin{pgfscope}%
\pgfsetbuttcap%
\pgfsetmiterjoin%
\pgfsetlinewidth{1.003750pt}%
\definecolor{currentstroke}{rgb}{0.000000,0.000000,0.000000}%
\pgfsetstrokecolor{currentstroke}%
\pgfsetdash{}{0pt}%
\pgfpathmoveto{\pgfqpoint{1.315834in}{1.432222in}}%
\pgfpathlineto{\pgfqpoint{1.593611in}{1.432222in}}%
\pgfpathlineto{\pgfqpoint{1.593611in}{1.529444in}}%
\pgfpathlineto{\pgfqpoint{1.315834in}{1.529444in}}%
\pgfpathlineto{\pgfqpoint{1.315834in}{1.432222in}}%
\pgfpathclose%
\pgfusepath{stroke}%
\end{pgfscope}%
\begin{pgfscope}%
\definecolor{textcolor}{rgb}{0.000000,0.000000,0.000000}%
\pgfsetstrokecolor{textcolor}%
\pgfsetfillcolor{textcolor}%
\pgftext[x=1.704722in,y=1.432222in,left,base]{\color{textcolor}\rmfamily\fontsize{10.000000}{12.000000}\selectfont Neg}%
\end{pgfscope}%
\begin{pgfscope}%
\pgfsetbuttcap%
\pgfsetmiterjoin%
\definecolor{currentfill}{rgb}{0.000000,0.000000,0.000000}%
\pgfsetfillcolor{currentfill}%
\pgfsetlinewidth{0.000000pt}%
\definecolor{currentstroke}{rgb}{0.000000,0.000000,0.000000}%
\pgfsetstrokecolor{currentstroke}%
\pgfsetstrokeopacity{0.000000}%
\pgfsetdash{}{0pt}%
\pgfpathmoveto{\pgfqpoint{1.315834in}{1.236944in}}%
\pgfpathlineto{\pgfqpoint{1.593611in}{1.236944in}}%
\pgfpathlineto{\pgfqpoint{1.593611in}{1.334167in}}%
\pgfpathlineto{\pgfqpoint{1.315834in}{1.334167in}}%
\pgfpathlineto{\pgfqpoint{1.315834in}{1.236944in}}%
\pgfpathclose%
\pgfusepath{fill}%
\end{pgfscope}%
\begin{pgfscope}%
\definecolor{textcolor}{rgb}{0.000000,0.000000,0.000000}%
\pgfsetstrokecolor{textcolor}%
\pgfsetfillcolor{textcolor}%
\pgftext[x=1.704722in,y=1.236944in,left,base]{\color{textcolor}\rmfamily\fontsize{10.000000}{12.000000}\selectfont Pos}%
\end{pgfscope}%
\end{pgfpicture}%
\makeatother%
\endgroup%

&
	\vskip 0pt
	\qquad \qquad ROC Curve
	
	%% Creator: Matplotlib, PGF backend
%%
%% To include the figure in your LaTeX document, write
%%   \input{<filename>.pgf}
%%
%% Make sure the required packages are loaded in your preamble
%%   \usepackage{pgf}
%%
%% Also ensure that all the required font packages are loaded; for instance,
%% the lmodern package is sometimes necessary when using math font.
%%   \usepackage{lmodern}
%%
%% Figures using additional raster images can only be included by \input if
%% they are in the same directory as the main LaTeX file. For loading figures
%% from other directories you can use the `import` package
%%   \usepackage{import}
%%
%% and then include the figures with
%%   \import{<path to file>}{<filename>.pgf}
%%
%% Matplotlib used the following preamble
%%   
%%   \usepackage{fontspec}
%%   \makeatletter\@ifpackageloaded{underscore}{}{\usepackage[strings]{underscore}}\makeatother
%%
\begingroup%
\makeatletter%
\begin{pgfpicture}%
\pgfpathrectangle{\pgfpointorigin}{\pgfqpoint{2.221861in}{1.754444in}}%
\pgfusepath{use as bounding box, clip}%
\begin{pgfscope}%
\pgfsetbuttcap%
\pgfsetmiterjoin%
\definecolor{currentfill}{rgb}{1.000000,1.000000,1.000000}%
\pgfsetfillcolor{currentfill}%
\pgfsetlinewidth{0.000000pt}%
\definecolor{currentstroke}{rgb}{1.000000,1.000000,1.000000}%
\pgfsetstrokecolor{currentstroke}%
\pgfsetdash{}{0pt}%
\pgfpathmoveto{\pgfqpoint{0.000000in}{0.000000in}}%
\pgfpathlineto{\pgfqpoint{2.221861in}{0.000000in}}%
\pgfpathlineto{\pgfqpoint{2.221861in}{1.754444in}}%
\pgfpathlineto{\pgfqpoint{0.000000in}{1.754444in}}%
\pgfpathlineto{\pgfqpoint{0.000000in}{0.000000in}}%
\pgfpathclose%
\pgfusepath{fill}%
\end{pgfscope}%
\begin{pgfscope}%
\pgfsetbuttcap%
\pgfsetmiterjoin%
\definecolor{currentfill}{rgb}{1.000000,1.000000,1.000000}%
\pgfsetfillcolor{currentfill}%
\pgfsetlinewidth{0.000000pt}%
\definecolor{currentstroke}{rgb}{0.000000,0.000000,0.000000}%
\pgfsetstrokecolor{currentstroke}%
\pgfsetstrokeopacity{0.000000}%
\pgfsetdash{}{0pt}%
\pgfpathmoveto{\pgfqpoint{0.553581in}{0.499444in}}%
\pgfpathlineto{\pgfqpoint{2.103581in}{0.499444in}}%
\pgfpathlineto{\pgfqpoint{2.103581in}{1.654444in}}%
\pgfpathlineto{\pgfqpoint{0.553581in}{1.654444in}}%
\pgfpathlineto{\pgfqpoint{0.553581in}{0.499444in}}%
\pgfpathclose%
\pgfusepath{fill}%
\end{pgfscope}%
\begin{pgfscope}%
\pgfsetbuttcap%
\pgfsetroundjoin%
\definecolor{currentfill}{rgb}{0.000000,0.000000,0.000000}%
\pgfsetfillcolor{currentfill}%
\pgfsetlinewidth{0.803000pt}%
\definecolor{currentstroke}{rgb}{0.000000,0.000000,0.000000}%
\pgfsetstrokecolor{currentstroke}%
\pgfsetdash{}{0pt}%
\pgfsys@defobject{currentmarker}{\pgfqpoint{0.000000in}{-0.048611in}}{\pgfqpoint{0.000000in}{0.000000in}}{%
\pgfpathmoveto{\pgfqpoint{0.000000in}{0.000000in}}%
\pgfpathlineto{\pgfqpoint{0.000000in}{-0.048611in}}%
\pgfusepath{stroke,fill}%
}%
\begin{pgfscope}%
\pgfsys@transformshift{0.624035in}{0.499444in}%
\pgfsys@useobject{currentmarker}{}%
\end{pgfscope}%
\end{pgfscope}%
\begin{pgfscope}%
\definecolor{textcolor}{rgb}{0.000000,0.000000,0.000000}%
\pgfsetstrokecolor{textcolor}%
\pgfsetfillcolor{textcolor}%
\pgftext[x=0.624035in,y=0.402222in,,top]{\color{textcolor}\rmfamily\fontsize{10.000000}{12.000000}\selectfont \(\displaystyle {0.0}\)}%
\end{pgfscope}%
\begin{pgfscope}%
\pgfsetbuttcap%
\pgfsetroundjoin%
\definecolor{currentfill}{rgb}{0.000000,0.000000,0.000000}%
\pgfsetfillcolor{currentfill}%
\pgfsetlinewidth{0.803000pt}%
\definecolor{currentstroke}{rgb}{0.000000,0.000000,0.000000}%
\pgfsetstrokecolor{currentstroke}%
\pgfsetdash{}{0pt}%
\pgfsys@defobject{currentmarker}{\pgfqpoint{0.000000in}{-0.048611in}}{\pgfqpoint{0.000000in}{0.000000in}}{%
\pgfpathmoveto{\pgfqpoint{0.000000in}{0.000000in}}%
\pgfpathlineto{\pgfqpoint{0.000000in}{-0.048611in}}%
\pgfusepath{stroke,fill}%
}%
\begin{pgfscope}%
\pgfsys@transformshift{1.328581in}{0.499444in}%
\pgfsys@useobject{currentmarker}{}%
\end{pgfscope}%
\end{pgfscope}%
\begin{pgfscope}%
\definecolor{textcolor}{rgb}{0.000000,0.000000,0.000000}%
\pgfsetstrokecolor{textcolor}%
\pgfsetfillcolor{textcolor}%
\pgftext[x=1.328581in,y=0.402222in,,top]{\color{textcolor}\rmfamily\fontsize{10.000000}{12.000000}\selectfont \(\displaystyle {0.5}\)}%
\end{pgfscope}%
\begin{pgfscope}%
\pgfsetbuttcap%
\pgfsetroundjoin%
\definecolor{currentfill}{rgb}{0.000000,0.000000,0.000000}%
\pgfsetfillcolor{currentfill}%
\pgfsetlinewidth{0.803000pt}%
\definecolor{currentstroke}{rgb}{0.000000,0.000000,0.000000}%
\pgfsetstrokecolor{currentstroke}%
\pgfsetdash{}{0pt}%
\pgfsys@defobject{currentmarker}{\pgfqpoint{0.000000in}{-0.048611in}}{\pgfqpoint{0.000000in}{0.000000in}}{%
\pgfpathmoveto{\pgfqpoint{0.000000in}{0.000000in}}%
\pgfpathlineto{\pgfqpoint{0.000000in}{-0.048611in}}%
\pgfusepath{stroke,fill}%
}%
\begin{pgfscope}%
\pgfsys@transformshift{2.033126in}{0.499444in}%
\pgfsys@useobject{currentmarker}{}%
\end{pgfscope}%
\end{pgfscope}%
\begin{pgfscope}%
\definecolor{textcolor}{rgb}{0.000000,0.000000,0.000000}%
\pgfsetstrokecolor{textcolor}%
\pgfsetfillcolor{textcolor}%
\pgftext[x=2.033126in,y=0.402222in,,top]{\color{textcolor}\rmfamily\fontsize{10.000000}{12.000000}\selectfont \(\displaystyle {1.0}\)}%
\end{pgfscope}%
\begin{pgfscope}%
\definecolor{textcolor}{rgb}{0.000000,0.000000,0.000000}%
\pgfsetstrokecolor{textcolor}%
\pgfsetfillcolor{textcolor}%
\pgftext[x=1.328581in,y=0.223333in,,top]{\color{textcolor}\rmfamily\fontsize{10.000000}{12.000000}\selectfont False positive rate}%
\end{pgfscope}%
\begin{pgfscope}%
\pgfsetbuttcap%
\pgfsetroundjoin%
\definecolor{currentfill}{rgb}{0.000000,0.000000,0.000000}%
\pgfsetfillcolor{currentfill}%
\pgfsetlinewidth{0.803000pt}%
\definecolor{currentstroke}{rgb}{0.000000,0.000000,0.000000}%
\pgfsetstrokecolor{currentstroke}%
\pgfsetdash{}{0pt}%
\pgfsys@defobject{currentmarker}{\pgfqpoint{-0.048611in}{0.000000in}}{\pgfqpoint{-0.000000in}{0.000000in}}{%
\pgfpathmoveto{\pgfqpoint{-0.000000in}{0.000000in}}%
\pgfpathlineto{\pgfqpoint{-0.048611in}{0.000000in}}%
\pgfusepath{stroke,fill}%
}%
\begin{pgfscope}%
\pgfsys@transformshift{0.553581in}{0.551944in}%
\pgfsys@useobject{currentmarker}{}%
\end{pgfscope}%
\end{pgfscope}%
\begin{pgfscope}%
\definecolor{textcolor}{rgb}{0.000000,0.000000,0.000000}%
\pgfsetstrokecolor{textcolor}%
\pgfsetfillcolor{textcolor}%
\pgftext[x=0.278889in, y=0.503750in, left, base]{\color{textcolor}\rmfamily\fontsize{10.000000}{12.000000}\selectfont \(\displaystyle {0.0}\)}%
\end{pgfscope}%
\begin{pgfscope}%
\pgfsetbuttcap%
\pgfsetroundjoin%
\definecolor{currentfill}{rgb}{0.000000,0.000000,0.000000}%
\pgfsetfillcolor{currentfill}%
\pgfsetlinewidth{0.803000pt}%
\definecolor{currentstroke}{rgb}{0.000000,0.000000,0.000000}%
\pgfsetstrokecolor{currentstroke}%
\pgfsetdash{}{0pt}%
\pgfsys@defobject{currentmarker}{\pgfqpoint{-0.048611in}{0.000000in}}{\pgfqpoint{-0.000000in}{0.000000in}}{%
\pgfpathmoveto{\pgfqpoint{-0.000000in}{0.000000in}}%
\pgfpathlineto{\pgfqpoint{-0.048611in}{0.000000in}}%
\pgfusepath{stroke,fill}%
}%
\begin{pgfscope}%
\pgfsys@transformshift{0.553581in}{1.076944in}%
\pgfsys@useobject{currentmarker}{}%
\end{pgfscope}%
\end{pgfscope}%
\begin{pgfscope}%
\definecolor{textcolor}{rgb}{0.000000,0.000000,0.000000}%
\pgfsetstrokecolor{textcolor}%
\pgfsetfillcolor{textcolor}%
\pgftext[x=0.278889in, y=1.028750in, left, base]{\color{textcolor}\rmfamily\fontsize{10.000000}{12.000000}\selectfont \(\displaystyle {0.5}\)}%
\end{pgfscope}%
\begin{pgfscope}%
\pgfsetbuttcap%
\pgfsetroundjoin%
\definecolor{currentfill}{rgb}{0.000000,0.000000,0.000000}%
\pgfsetfillcolor{currentfill}%
\pgfsetlinewidth{0.803000pt}%
\definecolor{currentstroke}{rgb}{0.000000,0.000000,0.000000}%
\pgfsetstrokecolor{currentstroke}%
\pgfsetdash{}{0pt}%
\pgfsys@defobject{currentmarker}{\pgfqpoint{-0.048611in}{0.000000in}}{\pgfqpoint{-0.000000in}{0.000000in}}{%
\pgfpathmoveto{\pgfqpoint{-0.000000in}{0.000000in}}%
\pgfpathlineto{\pgfqpoint{-0.048611in}{0.000000in}}%
\pgfusepath{stroke,fill}%
}%
\begin{pgfscope}%
\pgfsys@transformshift{0.553581in}{1.601944in}%
\pgfsys@useobject{currentmarker}{}%
\end{pgfscope}%
\end{pgfscope}%
\begin{pgfscope}%
\definecolor{textcolor}{rgb}{0.000000,0.000000,0.000000}%
\pgfsetstrokecolor{textcolor}%
\pgfsetfillcolor{textcolor}%
\pgftext[x=0.278889in, y=1.553750in, left, base]{\color{textcolor}\rmfamily\fontsize{10.000000}{12.000000}\selectfont \(\displaystyle {1.0}\)}%
\end{pgfscope}%
\begin{pgfscope}%
\definecolor{textcolor}{rgb}{0.000000,0.000000,0.000000}%
\pgfsetstrokecolor{textcolor}%
\pgfsetfillcolor{textcolor}%
\pgftext[x=0.223333in,y=1.076944in,,bottom,rotate=90.000000]{\color{textcolor}\rmfamily\fontsize{10.000000}{12.000000}\selectfont True positive rate}%
\end{pgfscope}%
\begin{pgfscope}%
\pgfpathrectangle{\pgfqpoint{0.553581in}{0.499444in}}{\pgfqpoint{1.550000in}{1.155000in}}%
\pgfusepath{clip}%
\pgfsetbuttcap%
\pgfsetroundjoin%
\pgfsetlinewidth{1.505625pt}%
\definecolor{currentstroke}{rgb}{0.000000,0.000000,0.000000}%
\pgfsetstrokecolor{currentstroke}%
\pgfsetdash{{5.550000pt}{2.400000pt}}{0.000000pt}%
\pgfpathmoveto{\pgfqpoint{0.624035in}{0.551944in}}%
\pgfpathlineto{\pgfqpoint{2.033126in}{1.601944in}}%
\pgfusepath{stroke}%
\end{pgfscope}%
\begin{pgfscope}%
\pgfpathrectangle{\pgfqpoint{0.553581in}{0.499444in}}{\pgfqpoint{1.550000in}{1.155000in}}%
\pgfusepath{clip}%
\pgfsetrectcap%
\pgfsetroundjoin%
\pgfsetlinewidth{1.505625pt}%
\definecolor{currentstroke}{rgb}{0.000000,0.000000,0.000000}%
\pgfsetstrokecolor{currentstroke}%
\pgfsetdash{}{0pt}%
\pgfpathmoveto{\pgfqpoint{0.624035in}{0.551944in}}%
\pgfpathlineto{\pgfqpoint{0.625119in}{0.554705in}}%
\pgfpathlineto{\pgfqpoint{0.627605in}{0.560779in}}%
\pgfpathlineto{\pgfqpoint{0.629175in}{0.564684in}}%
\pgfpathlineto{\pgfqpoint{0.629746in}{0.565749in}}%
\pgfpathlineto{\pgfqpoint{0.634867in}{0.576951in}}%
\pgfpathlineto{\pgfqpoint{0.638456in}{0.583893in}}%
\pgfpathlineto{\pgfqpoint{0.639549in}{0.586023in}}%
\pgfpathlineto{\pgfqpoint{0.640204in}{0.587127in}}%
\pgfpathlineto{\pgfqpoint{0.642063in}{0.590243in}}%
\pgfpathlineto{\pgfqpoint{0.642615in}{0.591347in}}%
\pgfpathlineto{\pgfqpoint{0.644316in}{0.594739in}}%
\pgfpathlineto{\pgfqpoint{0.645858in}{0.598407in}}%
\pgfpathlineto{\pgfqpoint{0.648578in}{0.602864in}}%
\pgfpathlineto{\pgfqpoint{0.650755in}{0.607243in}}%
\pgfpathlineto{\pgfqpoint{0.651400in}{0.608308in}}%
\pgfpathlineto{\pgfqpoint{0.663559in}{0.630750in}}%
\pgfpathlineto{\pgfqpoint{0.664970in}{0.633314in}}%
\pgfpathlineto{\pgfqpoint{0.669484in}{0.640256in}}%
\pgfpathlineto{\pgfqpoint{0.670120in}{0.641282in}}%
\pgfpathlineto{\pgfqpoint{0.671195in}{0.643530in}}%
\pgfpathlineto{\pgfqpoint{0.671213in}{0.643530in}}%
\pgfpathlineto{\pgfqpoint{0.672755in}{0.645896in}}%
\pgfpathlineto{\pgfqpoint{0.675503in}{0.650353in}}%
\pgfpathlineto{\pgfqpoint{0.676438in}{0.651379in}}%
\pgfpathlineto{\pgfqpoint{0.683036in}{0.661949in}}%
\pgfpathlineto{\pgfqpoint{0.683924in}{0.664040in}}%
\pgfpathlineto{\pgfqpoint{0.683952in}{0.664040in}}%
\pgfpathlineto{\pgfqpoint{0.688204in}{0.671652in}}%
\pgfpathlineto{\pgfqpoint{0.689634in}{0.674571in}}%
\pgfpathlineto{\pgfqpoint{0.702214in}{0.695121in}}%
\pgfpathlineto{\pgfqpoint{0.702774in}{0.695909in}}%
\pgfpathlineto{\pgfqpoint{0.750831in}{0.766433in}}%
\pgfpathlineto{\pgfqpoint{0.751354in}{0.767143in}}%
\pgfpathlineto{\pgfqpoint{0.756607in}{0.775071in}}%
\pgfpathlineto{\pgfqpoint{0.757383in}{0.776175in}}%
\pgfpathlineto{\pgfqpoint{0.758934in}{0.778068in}}%
\pgfpathlineto{\pgfqpoint{0.763345in}{0.784379in}}%
\pgfpathlineto{\pgfqpoint{0.767112in}{0.790650in}}%
\pgfpathlineto{\pgfqpoint{0.774925in}{0.801773in}}%
\pgfpathlineto{\pgfqpoint{0.778121in}{0.806704in}}%
\pgfpathlineto{\pgfqpoint{0.782317in}{0.813054in}}%
\pgfpathlineto{\pgfqpoint{0.796327in}{0.833012in}}%
\pgfpathlineto{\pgfqpoint{0.800196in}{0.838770in}}%
\pgfpathlineto{\pgfqpoint{0.803271in}{0.842241in}}%
\pgfpathlineto{\pgfqpoint{0.842683in}{0.892964in}}%
\pgfpathlineto{\pgfqpoint{0.843524in}{0.894069in}}%
\pgfpathlineto{\pgfqpoint{0.849346in}{0.902312in}}%
\pgfpathlineto{\pgfqpoint{0.855402in}{0.909964in}}%
\pgfpathlineto{\pgfqpoint{0.862262in}{0.917695in}}%
\pgfpathlineto{\pgfqpoint{0.866131in}{0.922507in}}%
\pgfpathlineto{\pgfqpoint{0.867160in}{0.923572in}}%
\pgfpathlineto{\pgfqpoint{0.868431in}{0.925583in}}%
\pgfpathlineto{\pgfqpoint{0.869374in}{0.926648in}}%
\pgfpathlineto{\pgfqpoint{0.871337in}{0.929173in}}%
\pgfpathlineto{\pgfqpoint{0.874487in}{0.932446in}}%
\pgfpathlineto{\pgfqpoint{0.875730in}{0.933511in}}%
\pgfpathlineto{\pgfqpoint{0.878833in}{0.936903in}}%
\pgfpathlineto{\pgfqpoint{0.882590in}{0.941676in}}%
\pgfpathlineto{\pgfqpoint{0.899375in}{0.960214in}}%
\pgfpathlineto{\pgfqpoint{0.900141in}{0.961279in}}%
\pgfpathlineto{\pgfqpoint{0.901506in}{0.963527in}}%
\pgfpathlineto{\pgfqpoint{0.935310in}{0.997645in}}%
\pgfpathlineto{\pgfqpoint{0.937282in}{0.999499in}}%
\pgfpathlineto{\pgfqpoint{0.938740in}{1.000840in}}%
\pgfpathlineto{\pgfqpoint{0.939581in}{1.001708in}}%
\pgfpathlineto{\pgfqpoint{0.944590in}{1.006125in}}%
\pgfpathlineto{\pgfqpoint{0.946123in}{1.007229in}}%
\pgfpathlineto{\pgfqpoint{0.947226in}{1.008413in}}%
\pgfpathlineto{\pgfqpoint{0.949291in}{1.010385in}}%
\pgfpathlineto{\pgfqpoint{0.951151in}{1.012396in}}%
\pgfpathlineto{\pgfqpoint{0.952889in}{1.014250in}}%
\pgfpathlineto{\pgfqpoint{0.953927in}{1.015355in}}%
\pgfpathlineto{\pgfqpoint{0.955123in}{1.016104in}}%
\pgfpathlineto{\pgfqpoint{0.968806in}{1.031447in}}%
\pgfpathlineto{\pgfqpoint{0.970441in}{1.032946in}}%
\pgfpathlineto{\pgfqpoint{0.972927in}{1.034997in}}%
\pgfpathlineto{\pgfqpoint{0.977338in}{1.038547in}}%
\pgfpathlineto{\pgfqpoint{1.011937in}{1.069549in}}%
\pgfpathlineto{\pgfqpoint{1.013974in}{1.071205in}}%
\pgfpathlineto{\pgfqpoint{1.017479in}{1.074282in}}%
\pgfpathlineto{\pgfqpoint{1.018965in}{1.075268in}}%
\pgfpathlineto{\pgfqpoint{1.023348in}{1.079212in}}%
\pgfpathlineto{\pgfqpoint{1.026432in}{1.081776in}}%
\pgfpathlineto{\pgfqpoint{1.027610in}{1.082841in}}%
\pgfpathlineto{\pgfqpoint{1.048218in}{1.101615in}}%
\pgfpathlineto{\pgfqpoint{1.049405in}{1.102483in}}%
\pgfpathlineto{\pgfqpoint{1.084555in}{1.133288in}}%
\pgfpathlineto{\pgfqpoint{1.086704in}{1.135260in}}%
\pgfpathlineto{\pgfqpoint{1.088302in}{1.136956in}}%
\pgfpathlineto{\pgfqpoint{1.093059in}{1.141374in}}%
\pgfpathlineto{\pgfqpoint{1.098396in}{1.145752in}}%
\pgfpathlineto{\pgfqpoint{1.107545in}{1.153443in}}%
\pgfpathlineto{\pgfqpoint{1.109873in}{1.154981in}}%
\pgfpathlineto{\pgfqpoint{1.111274in}{1.156165in}}%
\pgfpathlineto{\pgfqpoint{1.112648in}{1.157229in}}%
\pgfpathlineto{\pgfqpoint{1.112648in}{1.157269in}}%
\pgfpathlineto{\pgfqpoint{1.115181in}{1.158965in}}%
\pgfpathlineto{\pgfqpoint{1.117798in}{1.160582in}}%
\pgfpathlineto{\pgfqpoint{1.124789in}{1.166972in}}%
\pgfpathlineto{\pgfqpoint{1.126377in}{1.168076in}}%
\pgfpathlineto{\pgfqpoint{1.132452in}{1.172533in}}%
\pgfpathlineto{\pgfqpoint{1.133649in}{1.173085in}}%
\pgfpathlineto{\pgfqpoint{1.140499in}{1.177345in}}%
\pgfpathlineto{\pgfqpoint{1.142079in}{1.178410in}}%
\pgfpathlineto{\pgfqpoint{1.144228in}{1.180146in}}%
\pgfpathlineto{\pgfqpoint{1.145864in}{1.181053in}}%
\pgfpathlineto{\pgfqpoint{1.149686in}{1.184366in}}%
\pgfpathlineto{\pgfqpoint{1.151116in}{1.185589in}}%
\pgfpathlineto{\pgfqpoint{1.154574in}{1.187876in}}%
\pgfpathlineto{\pgfqpoint{1.155826in}{1.188744in}}%
\pgfpathlineto{\pgfqpoint{1.157153in}{1.189770in}}%
\pgfpathlineto{\pgfqpoint{1.164696in}{1.195134in}}%
\pgfpathlineto{\pgfqpoint{1.166724in}{1.196672in}}%
\pgfpathlineto{\pgfqpoint{1.172060in}{1.200695in}}%
\pgfpathlineto{\pgfqpoint{1.177873in}{1.205783in}}%
\pgfpathlineto{\pgfqpoint{1.211977in}{1.229449in}}%
\pgfpathlineto{\pgfqpoint{1.213584in}{1.230474in}}%
\pgfpathlineto{\pgfqpoint{1.213584in}{1.230514in}}%
\pgfpathlineto{\pgfqpoint{1.224715in}{1.237929in}}%
\pgfpathlineto{\pgfqpoint{1.225977in}{1.238718in}}%
\pgfpathlineto{\pgfqpoint{1.232631in}{1.243845in}}%
\pgfpathlineto{\pgfqpoint{1.234033in}{1.244950in}}%
\pgfpathlineto{\pgfqpoint{1.235397in}{1.246054in}}%
\pgfpathlineto{\pgfqpoint{1.239678in}{1.249288in}}%
\pgfpathlineto{\pgfqpoint{1.241482in}{1.250393in}}%
\pgfpathlineto{\pgfqpoint{1.242416in}{1.251024in}}%
\pgfpathlineto{\pgfqpoint{1.260454in}{1.262659in}}%
\pgfpathlineto{\pgfqpoint{1.272874in}{1.270785in}}%
\pgfpathlineto{\pgfqpoint{1.274977in}{1.272323in}}%
\pgfpathlineto{\pgfqpoint{1.284071in}{1.277371in}}%
\pgfpathlineto{\pgfqpoint{1.286632in}{1.278752in}}%
\pgfpathlineto{\pgfqpoint{1.288155in}{1.280132in}}%
\pgfpathlineto{\pgfqpoint{1.290333in}{1.281197in}}%
\pgfpathlineto{\pgfqpoint{1.298417in}{1.286877in}}%
\pgfpathlineto{\pgfqpoint{1.299314in}{1.287587in}}%
\pgfpathlineto{\pgfqpoint{1.304725in}{1.291531in}}%
\pgfpathlineto{\pgfqpoint{1.308249in}{1.294095in}}%
\pgfpathlineto{\pgfqpoint{1.348268in}{1.321665in}}%
\pgfpathlineto{\pgfqpoint{1.350539in}{1.322809in}}%
\pgfpathlineto{\pgfqpoint{1.356838in}{1.327148in}}%
\pgfpathlineto{\pgfqpoint{1.358277in}{1.328410in}}%
\pgfpathlineto{\pgfqpoint{1.360436in}{1.329396in}}%
\pgfpathlineto{\pgfqpoint{1.362062in}{1.330777in}}%
\pgfpathlineto{\pgfqpoint{1.369044in}{1.334879in}}%
\pgfpathlineto{\pgfqpoint{1.378539in}{1.341268in}}%
\pgfpathlineto{\pgfqpoint{1.379810in}{1.342294in}}%
\pgfpathlineto{\pgfqpoint{1.418063in}{1.367143in}}%
\pgfpathlineto{\pgfqpoint{1.420138in}{1.369549in}}%
\pgfpathlineto{\pgfqpoint{1.434960in}{1.378187in}}%
\pgfpathlineto{\pgfqpoint{1.436689in}{1.379212in}}%
\pgfpathlineto{\pgfqpoint{1.440512in}{1.381579in}}%
\pgfpathlineto{\pgfqpoint{1.443119in}{1.382604in}}%
\pgfpathlineto{\pgfqpoint{1.444727in}{1.383787in}}%
\pgfpathlineto{\pgfqpoint{1.553065in}{1.441176in}}%
\pgfpathlineto{\pgfqpoint{1.555887in}{1.442281in}}%
\pgfpathlineto{\pgfqpoint{1.556962in}{1.442715in}}%
\pgfpathlineto{\pgfqpoint{1.558906in}{1.443898in}}%
\pgfpathlineto{\pgfqpoint{1.560999in}{1.444963in}}%
\pgfpathlineto{\pgfqpoint{1.564448in}{1.447172in}}%
\pgfpathlineto{\pgfqpoint{1.567485in}{1.448276in}}%
\pgfpathlineto{\pgfqpoint{1.592626in}{1.461292in}}%
\pgfpathlineto{\pgfqpoint{1.595729in}{1.462475in}}%
\pgfpathlineto{\pgfqpoint{1.603701in}{1.466538in}}%
\pgfpathlineto{\pgfqpoint{1.606196in}{1.467642in}}%
\pgfpathlineto{\pgfqpoint{1.638421in}{1.482433in}}%
\pgfpathlineto{\pgfqpoint{1.639748in}{1.483222in}}%
\pgfpathlineto{\pgfqpoint{1.642430in}{1.484090in}}%
\pgfpathlineto{\pgfqpoint{1.645411in}{1.485194in}}%
\pgfpathlineto{\pgfqpoint{1.650103in}{1.486732in}}%
\pgfpathlineto{\pgfqpoint{1.651963in}{1.487955in}}%
\pgfpathlineto{\pgfqpoint{1.674664in}{1.497461in}}%
\pgfpathlineto{\pgfqpoint{1.677328in}{1.498920in}}%
\pgfpathlineto{\pgfqpoint{1.690010in}{1.504363in}}%
\pgfpathlineto{\pgfqpoint{1.692272in}{1.505468in}}%
\pgfpathlineto{\pgfqpoint{1.703973in}{1.510714in}}%
\pgfpathlineto{\pgfqpoint{1.706842in}{1.511779in}}%
\pgfpathlineto{\pgfqpoint{1.715020in}{1.515092in}}%
\pgfpathlineto{\pgfqpoint{1.718515in}{1.516196in}}%
\pgfpathlineto{\pgfqpoint{1.720151in}{1.517064in}}%
\pgfpathlineto{\pgfqpoint{1.724225in}{1.519036in}}%
\pgfpathlineto{\pgfqpoint{1.727319in}{1.520298in}}%
\pgfpathlineto{\pgfqpoint{1.738964in}{1.525031in}}%
\pgfpathlineto{\pgfqpoint{1.740216in}{1.525820in}}%
\pgfpathlineto{\pgfqpoint{1.744011in}{1.526924in}}%
\pgfpathlineto{\pgfqpoint{1.745590in}{1.527713in}}%
\pgfpathlineto{\pgfqpoint{1.749730in}{1.528818in}}%
\pgfpathlineto{\pgfqpoint{1.768170in}{1.534813in}}%
\pgfpathlineto{\pgfqpoint{1.769254in}{1.535207in}}%
\pgfpathlineto{\pgfqpoint{1.772114in}{1.536312in}}%
\pgfpathlineto{\pgfqpoint{1.892433in}{1.571495in}}%
\pgfpathlineto{\pgfqpoint{1.897162in}{1.572520in}}%
\pgfpathlineto{\pgfqpoint{1.898657in}{1.572993in}}%
\pgfpathlineto{\pgfqpoint{1.920461in}{1.578476in}}%
\pgfpathlineto{\pgfqpoint{1.926527in}{1.579580in}}%
\pgfpathlineto{\pgfqpoint{1.938004in}{1.582105in}}%
\pgfpathlineto{\pgfqpoint{1.946181in}{1.584077in}}%
\pgfpathlineto{\pgfqpoint{1.947331in}{1.584432in}}%
\pgfpathlineto{\pgfqpoint{1.953555in}{1.585536in}}%
\pgfpathlineto{\pgfqpoint{1.969527in}{1.589046in}}%
\pgfpathlineto{\pgfqpoint{1.978023in}{1.591058in}}%
\pgfpathlineto{\pgfqpoint{2.033126in}{1.601944in}}%
\pgfpathlineto{\pgfqpoint{2.033126in}{1.601944in}}%
\pgfusepath{stroke}%
\end{pgfscope}%
\begin{pgfscope}%
\pgfsetrectcap%
\pgfsetmiterjoin%
\pgfsetlinewidth{0.803000pt}%
\definecolor{currentstroke}{rgb}{0.000000,0.000000,0.000000}%
\pgfsetstrokecolor{currentstroke}%
\pgfsetdash{}{0pt}%
\pgfpathmoveto{\pgfqpoint{0.553581in}{0.499444in}}%
\pgfpathlineto{\pgfqpoint{0.553581in}{1.654444in}}%
\pgfusepath{stroke}%
\end{pgfscope}%
\begin{pgfscope}%
\pgfsetrectcap%
\pgfsetmiterjoin%
\pgfsetlinewidth{0.803000pt}%
\definecolor{currentstroke}{rgb}{0.000000,0.000000,0.000000}%
\pgfsetstrokecolor{currentstroke}%
\pgfsetdash{}{0pt}%
\pgfpathmoveto{\pgfqpoint{2.103581in}{0.499444in}}%
\pgfpathlineto{\pgfqpoint{2.103581in}{1.654444in}}%
\pgfusepath{stroke}%
\end{pgfscope}%
\begin{pgfscope}%
\pgfsetrectcap%
\pgfsetmiterjoin%
\pgfsetlinewidth{0.803000pt}%
\definecolor{currentstroke}{rgb}{0.000000,0.000000,0.000000}%
\pgfsetstrokecolor{currentstroke}%
\pgfsetdash{}{0pt}%
\pgfpathmoveto{\pgfqpoint{0.553581in}{0.499444in}}%
\pgfpathlineto{\pgfqpoint{2.103581in}{0.499444in}}%
\pgfusepath{stroke}%
\end{pgfscope}%
\begin{pgfscope}%
\pgfsetrectcap%
\pgfsetmiterjoin%
\pgfsetlinewidth{0.803000pt}%
\definecolor{currentstroke}{rgb}{0.000000,0.000000,0.000000}%
\pgfsetstrokecolor{currentstroke}%
\pgfsetdash{}{0pt}%
\pgfpathmoveto{\pgfqpoint{0.553581in}{1.654444in}}%
\pgfpathlineto{\pgfqpoint{2.103581in}{1.654444in}}%
\pgfusepath{stroke}%
\end{pgfscope}%
\begin{pgfscope}%
\pgfsetbuttcap%
\pgfsetmiterjoin%
\definecolor{currentfill}{rgb}{1.000000,1.000000,1.000000}%
\pgfsetfillcolor{currentfill}%
\pgfsetfillopacity{0.800000}%
\pgfsetlinewidth{1.003750pt}%
\definecolor{currentstroke}{rgb}{0.800000,0.800000,0.800000}%
\pgfsetstrokecolor{currentstroke}%
\pgfsetstrokeopacity{0.800000}%
\pgfsetdash{}{0pt}%
\pgfpathmoveto{\pgfqpoint{0.840525in}{0.568889in}}%
\pgfpathlineto{\pgfqpoint{2.006358in}{0.568889in}}%
\pgfpathquadraticcurveto{\pgfqpoint{2.034136in}{0.568889in}}{\pgfqpoint{2.034136in}{0.596666in}}%
\pgfpathlineto{\pgfqpoint{2.034136in}{0.791111in}}%
\pgfpathquadraticcurveto{\pgfqpoint{2.034136in}{0.818888in}}{\pgfqpoint{2.006358in}{0.818888in}}%
\pgfpathlineto{\pgfqpoint{0.840525in}{0.818888in}}%
\pgfpathquadraticcurveto{\pgfqpoint{0.812747in}{0.818888in}}{\pgfqpoint{0.812747in}{0.791111in}}%
\pgfpathlineto{\pgfqpoint{0.812747in}{0.596666in}}%
\pgfpathquadraticcurveto{\pgfqpoint{0.812747in}{0.568889in}}{\pgfqpoint{0.840525in}{0.568889in}}%
\pgfpathlineto{\pgfqpoint{0.840525in}{0.568889in}}%
\pgfpathclose%
\pgfusepath{stroke,fill}%
\end{pgfscope}%
\begin{pgfscope}%
\pgfsetrectcap%
\pgfsetroundjoin%
\pgfsetlinewidth{1.505625pt}%
\definecolor{currentstroke}{rgb}{0.000000,0.000000,0.000000}%
\pgfsetstrokecolor{currentstroke}%
\pgfsetdash{}{0pt}%
\pgfpathmoveto{\pgfqpoint{0.868303in}{0.707777in}}%
\pgfpathlineto{\pgfqpoint{1.007192in}{0.707777in}}%
\pgfpathlineto{\pgfqpoint{1.146081in}{0.707777in}}%
\pgfusepath{stroke}%
\end{pgfscope}%
\begin{pgfscope}%
\definecolor{textcolor}{rgb}{0.000000,0.000000,0.000000}%
\pgfsetstrokecolor{textcolor}%
\pgfsetfillcolor{textcolor}%
\pgftext[x=1.257192in,y=0.659166in,left,base]{\color{textcolor}\rmfamily\fontsize{10.000000}{12.000000}\selectfont AUC 0.659)}%
\end{pgfscope}%
\end{pgfpicture}%
\makeatother%
\endgroup%

	
&
	\vskip 0pt
	\begin{tabular}{cc|c|c|}
	&\multicolumn{1}{c}{}& \multicolumn{2}{c}{Prediction} \cr
	&\multicolumn{1}{c}{} & \multicolumn{1}{c}{N} & \multicolumn{1}{c}{P} \cr\cline{3-4}
	\multirow{2}{*}{\rotatebox[origin=c]{90}{Actual}}&N &
150,771 & 0
	\vrule width 0pt height 10pt depth 2pt \cr\cline{3-4}
	&P & 
26621 & 0
	\vrule width 0pt height 10pt depth 2pt \cr\cline{3-4}
	\end{tabular}

	\hfil\begin{tabular}{ll}
	\cr
	0.8499 & Accuracy\cr
und & Precision \cr	0.0 & Recall \cr	und & F1 \cr	0.659 & AUC \cr
\end{tabular}

\cr
\end{tabular}
} % End parbox

\

In this study, we  have arbitrarily decided that we are willing to trade off up to two false positives to get one more true positive.  Once we moved our decision thresholds to the ethical tradeoff point, the accuracy only varied from 0.836 to 0.854.  The difference in accuracy tells us how many more (or fewer) false positives than true positives we have, with them being equal at 0.8499, and we get the same information from precision being less than, more than, or equal to 0.5.    Therefore, we are not going to consider accuracy in evaluating our models. 

%%%
\subsubsection{ML Algorithms for Imbalanced Data}

{\bf [Expand this subsubsection]}

\begin{itemize}
	\item Random Undersampling Composite Models
	\item Bagging
	\item Boosting
\end{itemize}

%%%%%
\subsection{Models}

We used seven binary classification algorithms.  Three of them take class weights.

\

\hfil\begin{tabular}{llc}
&& Class \cr
Model & Source & Weights \cr\hline
KerasClassifier with the Binary Focal Crossentropy loss function & Keras & Yes \cr
Balanced Random Forest Classifier & Imbalanced-Learn & Yes \cr
Balanced Bagging Classifier & Imbalanced-Learn & No \cr
RUSBoost Classifier & Imbalanced-Learn & No \cr
Easy Ensemble Classifier with AdaBoost Estimator & Imbalanced-Learn & No \cr
Logistic Regression Classifier & Scikit-Learn & Yes \cr
AdaBoost  Classifier & Scikit-Learn & No \cr
\end{tabular}

\


For the focal loss function, we tried seven different combinations of the hyperparameters $\alpha$ for class weights and $\gamma$ for penalty on badly misclassified samples.  For the random forest and bagging models we tried three values of $\alpha$.  Altogether we had seventeen model/hyperparameter combinations.  We learned each of the seven models on datasets with the easy, medium, and hard features, and on the hard features we tested with Tomek undersampling 0, 1, and 2 times, for a total of five datasets, giving eighty-five model/hyperparameter/dataset combinations.    We learned each of those sixty-five with two different random seeds, for a total of one hundred seventy results.  

\

\hfil\noindent\begin{tabular}{ccc}
	\multicolumn{3}{c}{Seventeen Models} \cr
	Model & $\alpha$ & $\gamma$ \cr\hline
	Focal & 1/2 & 0.0 \cr
	Focal & 2/3 & 0.0 \cr
	Focal & 2/3 & 0.5 \cr
	Focal & 2/3 & 1.0 \cr
	Focal & 2/3 & 2.0 \cr
	Focal & 2/3 & 5.0 \cr
	Focal & 0.85 & 0.0 \cr
	Random Forest & 1/2 & \cr
	Random Forest & 2/3 & \cr
	Random Forest & 0.85 & \cr
	Bagging && \cr
	RUSBoost && \cr
	Easy Ens && \cr
	Log Reg & 1/2 & \cr
	Log Reg & 2/3 & \cr
	Log Reg & 0.85 & \cr
	AdaBoost && \cr
\end{tabular}
\quad
$\times$
\quad
\begin{tabular}{cc}
	\multicolumn{2}{c}{Seven Datasets} \cr
	Features & Tomek \cr\hline
	Hard & None \cr
	Hard & Once \cr
	Hard & Twice \cr
	Medium & None \cr
	Medium & Once \cr
	Medium & Twice \cr
	Easy & None \cr
\end{tabular}
\quad
$\times$
\quad
\begin{tabular}{cc}
	Run twice with \cr
	different \cr
	random seeds \cr\hline
	Random seed 1 \cr
	Random seed 2
\end{tabular}
\quad 
$=$ 
\quad 
\begin{tabular}{c}
	238  \cr Sets of  \cr
	Results \cr
\end{tabular}







