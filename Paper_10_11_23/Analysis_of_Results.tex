% Analysis of Results

Our ML algorithms assign to each sample (feature vector, crash person) a probability $p \in [0,1]$ that the person needs an ambulance.  The histogram below left shows the percentage of the dataset in each range of $p$, showing the percentages for the negative class (``Does not need an ambulance'') and the positive class (``Needs an ambulance'').  On the right, the Receiver Operating Characteristic (ROC) curve, and particularly the area under the curve (AUC), is a metric for how well the model separates the two classes, with $AUC=1.0$ being perfect and $AUC=0.5$ (the dashed line) being just random assignment with no insight.  

We would love to have results like in the graphs below, where the machine learning (ML) algorithm nearly perfectly separates the two classes.  There is some overlap between $p=0.6$ and $p=0.8$ with some samples the algorithm misclassifies, but the model clearly separates most samples.  Having an AUC of 0.996 would be amazing.  

[Put in \verb|BRFC_Hard_alpha_0_5_Train_Pred_Wide.pgf| once we have it.)

\begin{comment}

\noindent\begin{tabular}{@{\hspace{-6pt}}p{4.3in} @{\hspace{-6pt}}p{2.0in}}
	\vskip 0pt
	\hfil Raw Model Output
	
	\input{../Keras/Images/BRFC_Hard_Tomek_0_alpha_0_5_v1_Train_Pred_Wide.pgf}	
&
	\vskip 0pt
	\hfil ROC Curve
	
	\input{../Keras/Images/BRFC_Hard_Tomek_0_alpha_0_5_v1_Train_ROC.pgf}
	
\end{tabular}
\end{comment}

Unfortunately, our test results do not look quite that nice.  They do not separate the two classes as well.  Some distributions are clustered to one side or in the middle.  Some models give the results in $p \in [0,1]$ rounded to two decimal places so that we cannot hope for a level of detail beyond that, and one algorithm, Bagging, gives $p$ rounded to only one decimal place.  

Let us look at some examples.  In all of them, AUC is in the range $[0.7,0.8]$, so the various models separate the positive and negative classes about equally well overall, with none being dramatically better or worse.  We will later show how we investigated which models do a better job in the ranges of interest.  

\

%
\verb|BRFC_5_Fold_alpha_0_5_Hard_Test|

\

This model does not separate the negative and positive classes as well as the ideal, giving a much lower AUC (area under the ROC curve).  These results are actually from the same model as the ideal above, but the ideal are the results on the training set and below on the test set, showing overfitting.  

In these results, the 100 most frequent values comprised 93\% of the results, meaning that, while there is some noise making the distribution look continuous, it is mostly discrete to two decimal places, so we cannot hope for fine detail in tuning the decision threshold.  

\noindent\begin{tabular}{@{\hspace{-6pt}}p{4.3in} @{\hspace{-6pt}}p{2.0in}}
	\vskip 0pt
	\hfil Raw Model Output
	
	\input{../Keras/Images/BRFC_5_Fold_alpha_0_5_Hard_Test_Pred_Wide.pgf}	
&
	\vskip 0pt
	\hfil ROC Curve
	
	\input{../Keras/Images/BRFC_5_Fold_alpha_0_5_Hard_Test_ROC.pgf}
\end{tabular}

\

\

%
\verb|AdaBoost_5_Fold_Hard_Test|

\

In this model the values are clustered very tightly, but in that small range the 214,070 samples return 210,442 different values of $p$, so there is much diversity that we can't see in this representation.  


\

\verb|AdaBoost_5_Fold_Hard_Test|



\noindent\begin{tabular}{@{\hspace{-6pt}}p{4.3in} @{\hspace{-6pt}}p{2.0in}}
	\vskip 0pt
	\hfil Raw Model Output
	
	\input{../Keras/Images/AdaBoost_5_Fold_Hard_Test_Pred_Wide.pgf}	
&
	\vskip 0pt
	\hfil ROC Curve
	
	\input{../Keras/Images/AdaBoost_5_Fold_Hard_Test_ROC.pgf}
\end{tabular}


\

In this work we used two methods to give the results of different models similar distributions.  This case illustrates directly transforming the \verb|y_proba| values.  

To make a useful visualization of the results where we can see the interplay between the negative and positive classes, we can transform the data.  A transformation that preserves rank will have no effect on the ROC curve.  [Cite]  For the graph below, we mapped the smallest value in the set to 0 and the largest to 1.  

\

\verb|AdaBoost_5_Fold_Hard_Test_Transformed_100|

%
\noindent\begin{tabular}{@{\hspace{-6pt}}p{4.3in} @{\hspace{-6pt}}p{2.0in}}
	\vskip 0pt
	\hfil Raw Model Output
	
	\input{../Keras/Images/AdaBoost_5_Fold_Hard_Test_Transformed_100_Pred_Wide.pgf}	
&
	\vskip 0pt
	\hfil ROC Curve
	
	\input{../Keras/Images/AdaBoost_5_Fold_Hard_Test_Transformed_100_ROC.pgf}
\end{tabular}

The distribution has long tails, so we can make a more useful visualization by truncating the ends.  For this graph we mapped the 0.01 quantile to 0 and the 0.99 quantile to 1 leaving the center 98\% of the distribution and truncated the ends.  Our goal in clipping the tails is to make all of the models' results have approximately the same granularity when we choose the decision thresholds that give us the (politically) desired results.  


\

\verb|AdaBoost_5_Fold_Hard_Test_Transformed_98|

%
\noindent\begin{tabular}{@{\hspace{-6pt}}p{4.3in} @{\hspace{-6pt}}p{2.0in}}
	\vskip 0pt
	\hfil Raw Model Output
	
	\input{../Keras/Images/AdaBoost_5_Fold_Hard_Test_Transformed_98_Pred_Wide.pgf}	
&
	\vskip 0pt
	\hfil ROC Curve
	
	%% Creator: Matplotlib, PGF backend
%%
%% To include the figure in your LaTeX document, write
%%   \input{<filename>.pgf}
%%
%% Make sure the required packages are loaded in your preamble
%%   \usepackage{pgf}
%%
%% Also ensure that all the required font packages are loaded; for instance,
%% the lmodern package is sometimes necessary when using math font.
%%   \usepackage{lmodern}
%%
%% Figures using additional raster images can only be included by \input if
%% they are in the same directory as the main LaTeX file. For loading figures
%% from other directories you can use the `import` package
%%   \usepackage{import}
%%
%% and then include the figures with
%%   \import{<path to file>}{<filename>.pgf}
%%
%% Matplotlib used the following preamble
%%   
%%   \usepackage{fontspec}
%%   \makeatletter\@ifpackageloaded{underscore}{}{\usepackage[strings]{underscore}}\makeatother
%%
\begingroup%
\makeatletter%
\begin{pgfpicture}%
\pgfpathrectangle{\pgfpointorigin}{\pgfqpoint{2.221861in}{1.754444in}}%
\pgfusepath{use as bounding box, clip}%
\begin{pgfscope}%
\pgfsetbuttcap%
\pgfsetmiterjoin%
\definecolor{currentfill}{rgb}{1.000000,1.000000,1.000000}%
\pgfsetfillcolor{currentfill}%
\pgfsetlinewidth{0.000000pt}%
\definecolor{currentstroke}{rgb}{1.000000,1.000000,1.000000}%
\pgfsetstrokecolor{currentstroke}%
\pgfsetdash{}{0pt}%
\pgfpathmoveto{\pgfqpoint{0.000000in}{0.000000in}}%
\pgfpathlineto{\pgfqpoint{2.221861in}{0.000000in}}%
\pgfpathlineto{\pgfqpoint{2.221861in}{1.754444in}}%
\pgfpathlineto{\pgfqpoint{0.000000in}{1.754444in}}%
\pgfpathlineto{\pgfqpoint{0.000000in}{0.000000in}}%
\pgfpathclose%
\pgfusepath{fill}%
\end{pgfscope}%
\begin{pgfscope}%
\pgfsetbuttcap%
\pgfsetmiterjoin%
\definecolor{currentfill}{rgb}{1.000000,1.000000,1.000000}%
\pgfsetfillcolor{currentfill}%
\pgfsetlinewidth{0.000000pt}%
\definecolor{currentstroke}{rgb}{0.000000,0.000000,0.000000}%
\pgfsetstrokecolor{currentstroke}%
\pgfsetstrokeopacity{0.000000}%
\pgfsetdash{}{0pt}%
\pgfpathmoveto{\pgfqpoint{0.553581in}{0.499444in}}%
\pgfpathlineto{\pgfqpoint{2.103581in}{0.499444in}}%
\pgfpathlineto{\pgfqpoint{2.103581in}{1.654444in}}%
\pgfpathlineto{\pgfqpoint{0.553581in}{1.654444in}}%
\pgfpathlineto{\pgfqpoint{0.553581in}{0.499444in}}%
\pgfpathclose%
\pgfusepath{fill}%
\end{pgfscope}%
\begin{pgfscope}%
\pgfsetbuttcap%
\pgfsetroundjoin%
\definecolor{currentfill}{rgb}{0.000000,0.000000,0.000000}%
\pgfsetfillcolor{currentfill}%
\pgfsetlinewidth{0.803000pt}%
\definecolor{currentstroke}{rgb}{0.000000,0.000000,0.000000}%
\pgfsetstrokecolor{currentstroke}%
\pgfsetdash{}{0pt}%
\pgfsys@defobject{currentmarker}{\pgfqpoint{0.000000in}{-0.048611in}}{\pgfqpoint{0.000000in}{0.000000in}}{%
\pgfpathmoveto{\pgfqpoint{0.000000in}{0.000000in}}%
\pgfpathlineto{\pgfqpoint{0.000000in}{-0.048611in}}%
\pgfusepath{stroke,fill}%
}%
\begin{pgfscope}%
\pgfsys@transformshift{0.624035in}{0.499444in}%
\pgfsys@useobject{currentmarker}{}%
\end{pgfscope}%
\end{pgfscope}%
\begin{pgfscope}%
\definecolor{textcolor}{rgb}{0.000000,0.000000,0.000000}%
\pgfsetstrokecolor{textcolor}%
\pgfsetfillcolor{textcolor}%
\pgftext[x=0.624035in,y=0.402222in,,top]{\color{textcolor}\rmfamily\fontsize{10.000000}{12.000000}\selectfont \(\displaystyle {0.0}\)}%
\end{pgfscope}%
\begin{pgfscope}%
\pgfsetbuttcap%
\pgfsetroundjoin%
\definecolor{currentfill}{rgb}{0.000000,0.000000,0.000000}%
\pgfsetfillcolor{currentfill}%
\pgfsetlinewidth{0.803000pt}%
\definecolor{currentstroke}{rgb}{0.000000,0.000000,0.000000}%
\pgfsetstrokecolor{currentstroke}%
\pgfsetdash{}{0pt}%
\pgfsys@defobject{currentmarker}{\pgfqpoint{0.000000in}{-0.048611in}}{\pgfqpoint{0.000000in}{0.000000in}}{%
\pgfpathmoveto{\pgfqpoint{0.000000in}{0.000000in}}%
\pgfpathlineto{\pgfqpoint{0.000000in}{-0.048611in}}%
\pgfusepath{stroke,fill}%
}%
\begin{pgfscope}%
\pgfsys@transformshift{1.328581in}{0.499444in}%
\pgfsys@useobject{currentmarker}{}%
\end{pgfscope}%
\end{pgfscope}%
\begin{pgfscope}%
\definecolor{textcolor}{rgb}{0.000000,0.000000,0.000000}%
\pgfsetstrokecolor{textcolor}%
\pgfsetfillcolor{textcolor}%
\pgftext[x=1.328581in,y=0.402222in,,top]{\color{textcolor}\rmfamily\fontsize{10.000000}{12.000000}\selectfont \(\displaystyle {0.5}\)}%
\end{pgfscope}%
\begin{pgfscope}%
\pgfsetbuttcap%
\pgfsetroundjoin%
\definecolor{currentfill}{rgb}{0.000000,0.000000,0.000000}%
\pgfsetfillcolor{currentfill}%
\pgfsetlinewidth{0.803000pt}%
\definecolor{currentstroke}{rgb}{0.000000,0.000000,0.000000}%
\pgfsetstrokecolor{currentstroke}%
\pgfsetdash{}{0pt}%
\pgfsys@defobject{currentmarker}{\pgfqpoint{0.000000in}{-0.048611in}}{\pgfqpoint{0.000000in}{0.000000in}}{%
\pgfpathmoveto{\pgfqpoint{0.000000in}{0.000000in}}%
\pgfpathlineto{\pgfqpoint{0.000000in}{-0.048611in}}%
\pgfusepath{stroke,fill}%
}%
\begin{pgfscope}%
\pgfsys@transformshift{2.033126in}{0.499444in}%
\pgfsys@useobject{currentmarker}{}%
\end{pgfscope}%
\end{pgfscope}%
\begin{pgfscope}%
\definecolor{textcolor}{rgb}{0.000000,0.000000,0.000000}%
\pgfsetstrokecolor{textcolor}%
\pgfsetfillcolor{textcolor}%
\pgftext[x=2.033126in,y=0.402222in,,top]{\color{textcolor}\rmfamily\fontsize{10.000000}{12.000000}\selectfont \(\displaystyle {1.0}\)}%
\end{pgfscope}%
\begin{pgfscope}%
\definecolor{textcolor}{rgb}{0.000000,0.000000,0.000000}%
\pgfsetstrokecolor{textcolor}%
\pgfsetfillcolor{textcolor}%
\pgftext[x=1.328581in,y=0.223333in,,top]{\color{textcolor}\rmfamily\fontsize{10.000000}{12.000000}\selectfont False positive rate}%
\end{pgfscope}%
\begin{pgfscope}%
\pgfsetbuttcap%
\pgfsetroundjoin%
\definecolor{currentfill}{rgb}{0.000000,0.000000,0.000000}%
\pgfsetfillcolor{currentfill}%
\pgfsetlinewidth{0.803000pt}%
\definecolor{currentstroke}{rgb}{0.000000,0.000000,0.000000}%
\pgfsetstrokecolor{currentstroke}%
\pgfsetdash{}{0pt}%
\pgfsys@defobject{currentmarker}{\pgfqpoint{-0.048611in}{0.000000in}}{\pgfqpoint{-0.000000in}{0.000000in}}{%
\pgfpathmoveto{\pgfqpoint{-0.000000in}{0.000000in}}%
\pgfpathlineto{\pgfqpoint{-0.048611in}{0.000000in}}%
\pgfusepath{stroke,fill}%
}%
\begin{pgfscope}%
\pgfsys@transformshift{0.553581in}{0.551944in}%
\pgfsys@useobject{currentmarker}{}%
\end{pgfscope}%
\end{pgfscope}%
\begin{pgfscope}%
\definecolor{textcolor}{rgb}{0.000000,0.000000,0.000000}%
\pgfsetstrokecolor{textcolor}%
\pgfsetfillcolor{textcolor}%
\pgftext[x=0.278889in, y=0.503750in, left, base]{\color{textcolor}\rmfamily\fontsize{10.000000}{12.000000}\selectfont \(\displaystyle {0.0}\)}%
\end{pgfscope}%
\begin{pgfscope}%
\pgfsetbuttcap%
\pgfsetroundjoin%
\definecolor{currentfill}{rgb}{0.000000,0.000000,0.000000}%
\pgfsetfillcolor{currentfill}%
\pgfsetlinewidth{0.803000pt}%
\definecolor{currentstroke}{rgb}{0.000000,0.000000,0.000000}%
\pgfsetstrokecolor{currentstroke}%
\pgfsetdash{}{0pt}%
\pgfsys@defobject{currentmarker}{\pgfqpoint{-0.048611in}{0.000000in}}{\pgfqpoint{-0.000000in}{0.000000in}}{%
\pgfpathmoveto{\pgfqpoint{-0.000000in}{0.000000in}}%
\pgfpathlineto{\pgfqpoint{-0.048611in}{0.000000in}}%
\pgfusepath{stroke,fill}%
}%
\begin{pgfscope}%
\pgfsys@transformshift{0.553581in}{1.076944in}%
\pgfsys@useobject{currentmarker}{}%
\end{pgfscope}%
\end{pgfscope}%
\begin{pgfscope}%
\definecolor{textcolor}{rgb}{0.000000,0.000000,0.000000}%
\pgfsetstrokecolor{textcolor}%
\pgfsetfillcolor{textcolor}%
\pgftext[x=0.278889in, y=1.028750in, left, base]{\color{textcolor}\rmfamily\fontsize{10.000000}{12.000000}\selectfont \(\displaystyle {0.5}\)}%
\end{pgfscope}%
\begin{pgfscope}%
\pgfsetbuttcap%
\pgfsetroundjoin%
\definecolor{currentfill}{rgb}{0.000000,0.000000,0.000000}%
\pgfsetfillcolor{currentfill}%
\pgfsetlinewidth{0.803000pt}%
\definecolor{currentstroke}{rgb}{0.000000,0.000000,0.000000}%
\pgfsetstrokecolor{currentstroke}%
\pgfsetdash{}{0pt}%
\pgfsys@defobject{currentmarker}{\pgfqpoint{-0.048611in}{0.000000in}}{\pgfqpoint{-0.000000in}{0.000000in}}{%
\pgfpathmoveto{\pgfqpoint{-0.000000in}{0.000000in}}%
\pgfpathlineto{\pgfqpoint{-0.048611in}{0.000000in}}%
\pgfusepath{stroke,fill}%
}%
\begin{pgfscope}%
\pgfsys@transformshift{0.553581in}{1.601944in}%
\pgfsys@useobject{currentmarker}{}%
\end{pgfscope}%
\end{pgfscope}%
\begin{pgfscope}%
\definecolor{textcolor}{rgb}{0.000000,0.000000,0.000000}%
\pgfsetstrokecolor{textcolor}%
\pgfsetfillcolor{textcolor}%
\pgftext[x=0.278889in, y=1.553750in, left, base]{\color{textcolor}\rmfamily\fontsize{10.000000}{12.000000}\selectfont \(\displaystyle {1.0}\)}%
\end{pgfscope}%
\begin{pgfscope}%
\definecolor{textcolor}{rgb}{0.000000,0.000000,0.000000}%
\pgfsetstrokecolor{textcolor}%
\pgfsetfillcolor{textcolor}%
\pgftext[x=0.223333in,y=1.076944in,,bottom,rotate=90.000000]{\color{textcolor}\rmfamily\fontsize{10.000000}{12.000000}\selectfont True positive rate}%
\end{pgfscope}%
\begin{pgfscope}%
\pgfpathrectangle{\pgfqpoint{0.553581in}{0.499444in}}{\pgfqpoint{1.550000in}{1.155000in}}%
\pgfusepath{clip}%
\pgfsetbuttcap%
\pgfsetroundjoin%
\pgfsetlinewidth{1.505625pt}%
\definecolor{currentstroke}{rgb}{0.000000,0.000000,0.000000}%
\pgfsetstrokecolor{currentstroke}%
\pgfsetdash{{5.550000pt}{2.400000pt}}{0.000000pt}%
\pgfpathmoveto{\pgfqpoint{0.624035in}{0.551944in}}%
\pgfpathlineto{\pgfqpoint{2.033126in}{1.601944in}}%
\pgfusepath{stroke}%
\end{pgfscope}%
\begin{pgfscope}%
\pgfpathrectangle{\pgfqpoint{0.553581in}{0.499444in}}{\pgfqpoint{1.550000in}{1.155000in}}%
\pgfusepath{clip}%
\pgfsetrectcap%
\pgfsetroundjoin%
\pgfsetlinewidth{1.505625pt}%
\definecolor{currentstroke}{rgb}{0.000000,0.000000,0.000000}%
\pgfsetstrokecolor{currentstroke}%
\pgfsetdash{}{0pt}%
\pgfpathmoveto{\pgfqpoint{0.624035in}{0.551944in}}%
\pgfpathlineto{\pgfqpoint{0.628637in}{0.584231in}}%
\pgfpathlineto{\pgfqpoint{0.628843in}{0.585339in}}%
\pgfpathlineto{\pgfqpoint{0.629941in}{0.590703in}}%
\pgfpathlineto{\pgfqpoint{0.630194in}{0.591811in}}%
\pgfpathlineto{\pgfqpoint{0.631299in}{0.597846in}}%
\pgfpathlineto{\pgfqpoint{0.631477in}{0.598954in}}%
\pgfpathlineto{\pgfqpoint{0.632586in}{0.603629in}}%
\pgfpathlineto{\pgfqpoint{0.632823in}{0.604728in}}%
\pgfpathlineto{\pgfqpoint{0.633930in}{0.609757in}}%
\pgfpathlineto{\pgfqpoint{0.634275in}{0.610865in}}%
\pgfpathlineto{\pgfqpoint{0.635384in}{0.615502in}}%
\pgfpathlineto{\pgfqpoint{0.635652in}{0.616592in}}%
\pgfpathlineto{\pgfqpoint{0.636758in}{0.621099in}}%
\pgfpathlineto{\pgfqpoint{0.637059in}{0.622198in}}%
\pgfpathlineto{\pgfqpoint{0.638166in}{0.627115in}}%
\pgfpathlineto{\pgfqpoint{0.638424in}{0.628195in}}%
\pgfpathlineto{\pgfqpoint{0.639533in}{0.633075in}}%
\pgfpathlineto{\pgfqpoint{0.639852in}{0.634183in}}%
\pgfpathlineto{\pgfqpoint{0.640959in}{0.638765in}}%
\pgfpathlineto{\pgfqpoint{0.641341in}{0.639873in}}%
\pgfpathlineto{\pgfqpoint{0.642451in}{0.644679in}}%
\pgfpathlineto{\pgfqpoint{0.642694in}{0.645787in}}%
\pgfpathlineto{\pgfqpoint{0.643801in}{0.649707in}}%
\pgfpathlineto{\pgfqpoint{0.644174in}{0.650778in}}%
\pgfpathlineto{\pgfqpoint{0.645281in}{0.654820in}}%
\pgfpathlineto{\pgfqpoint{0.645586in}{0.655910in}}%
\pgfpathlineto{\pgfqpoint{0.646693in}{0.660398in}}%
\pgfpathlineto{\pgfqpoint{0.647014in}{0.661497in}}%
\pgfpathlineto{\pgfqpoint{0.648124in}{0.665259in}}%
\pgfpathlineto{\pgfqpoint{0.648485in}{0.666368in}}%
\pgfpathlineto{\pgfqpoint{0.649594in}{0.670130in}}%
\pgfpathlineto{\pgfqpoint{0.649909in}{0.671229in}}%
\pgfpathlineto{\pgfqpoint{0.651018in}{0.675429in}}%
\pgfpathlineto{\pgfqpoint{0.651339in}{0.676537in}}%
\pgfpathlineto{\pgfqpoint{0.652449in}{0.680541in}}%
\pgfpathlineto{\pgfqpoint{0.652742in}{0.681649in}}%
\pgfpathlineto{\pgfqpoint{0.653846in}{0.685365in}}%
\pgfpathlineto{\pgfqpoint{0.654156in}{0.686464in}}%
\pgfpathlineto{\pgfqpoint{0.655265in}{0.690496in}}%
\pgfpathlineto{\pgfqpoint{0.655617in}{0.691605in}}%
\pgfpathlineto{\pgfqpoint{0.656726in}{0.695460in}}%
\pgfpathlineto{\pgfqpoint{0.657167in}{0.696559in}}%
\pgfpathlineto{\pgfqpoint{0.658277in}{0.699744in}}%
\pgfpathlineto{\pgfqpoint{0.658598in}{0.700852in}}%
\pgfpathlineto{\pgfqpoint{0.659707in}{0.704568in}}%
\pgfpathlineto{\pgfqpoint{0.660014in}{0.705657in}}%
\pgfpathlineto{\pgfqpoint{0.661121in}{0.709094in}}%
\pgfpathlineto{\pgfqpoint{0.661384in}{0.710202in}}%
\pgfpathlineto{\pgfqpoint{0.662489in}{0.713880in}}%
\pgfpathlineto{\pgfqpoint{0.662493in}{0.713880in}}%
\pgfpathlineto{\pgfqpoint{0.662902in}{0.714988in}}%
\pgfpathlineto{\pgfqpoint{0.664011in}{0.718592in}}%
\pgfpathlineto{\pgfqpoint{0.664405in}{0.719691in}}%
\pgfpathlineto{\pgfqpoint{0.665512in}{0.723128in}}%
\pgfpathlineto{\pgfqpoint{0.665972in}{0.724236in}}%
\pgfpathlineto{\pgfqpoint{0.667081in}{0.727737in}}%
\pgfpathlineto{\pgfqpoint{0.667475in}{0.728845in}}%
\pgfpathlineto{\pgfqpoint{0.668584in}{0.732217in}}%
\pgfpathlineto{\pgfqpoint{0.668976in}{0.733325in}}%
\pgfpathlineto{\pgfqpoint{0.670083in}{0.736323in}}%
\pgfpathlineto{\pgfqpoint{0.670449in}{0.737432in}}%
\pgfpathlineto{\pgfqpoint{0.671549in}{0.740412in}}%
\pgfpathlineto{\pgfqpoint{0.671954in}{0.741520in}}%
\pgfpathlineto{\pgfqpoint{0.673050in}{0.744844in}}%
\pgfpathlineto{\pgfqpoint{0.673061in}{0.744844in}}%
\pgfpathlineto{\pgfqpoint{0.673516in}{0.745934in}}%
\pgfpathlineto{\pgfqpoint{0.674621in}{0.748961in}}%
\pgfpathlineto{\pgfqpoint{0.674961in}{0.750069in}}%
\pgfpathlineto{\pgfqpoint{0.676068in}{0.752658in}}%
\pgfpathlineto{\pgfqpoint{0.676464in}{0.753757in}}%
\pgfpathlineto{\pgfqpoint{0.677567in}{0.757137in}}%
\pgfpathlineto{\pgfqpoint{0.678022in}{0.758245in}}%
\pgfpathlineto{\pgfqpoint{0.679131in}{0.761095in}}%
\pgfpathlineto{\pgfqpoint{0.679546in}{0.762203in}}%
\pgfpathlineto{\pgfqpoint{0.683200in}{0.772177in}}%
\pgfpathlineto{\pgfqpoint{0.683704in}{0.773276in}}%
\pgfpathlineto{\pgfqpoint{0.684800in}{0.776116in}}%
\pgfpathlineto{\pgfqpoint{0.685304in}{0.777224in}}%
\pgfpathlineto{\pgfqpoint{0.686394in}{0.779776in}}%
\pgfpathlineto{\pgfqpoint{0.686896in}{0.780884in}}%
\pgfpathlineto{\pgfqpoint{0.687989in}{0.783631in}}%
\pgfpathlineto{\pgfqpoint{0.688416in}{0.784739in}}%
\pgfpathlineto{\pgfqpoint{0.689525in}{0.787459in}}%
\pgfpathlineto{\pgfqpoint{0.689971in}{0.788567in}}%
\pgfpathlineto{\pgfqpoint{0.691080in}{0.791119in}}%
\pgfpathlineto{\pgfqpoint{0.691556in}{0.792227in}}%
\pgfpathlineto{\pgfqpoint{0.692663in}{0.794937in}}%
\pgfpathlineto{\pgfqpoint{0.693207in}{0.796036in}}%
\pgfpathlineto{\pgfqpoint{0.694314in}{0.798764in}}%
\pgfpathlineto{\pgfqpoint{0.694737in}{0.799872in}}%
\pgfpathlineto{\pgfqpoint{0.695839in}{0.802480in}}%
\pgfpathlineto{\pgfqpoint{0.696317in}{0.803579in}}%
\pgfpathlineto{\pgfqpoint{0.697410in}{0.806298in}}%
\pgfpathlineto{\pgfqpoint{0.697961in}{0.807406in}}%
\pgfpathlineto{\pgfqpoint{0.699061in}{0.810098in}}%
\pgfpathlineto{\pgfqpoint{0.699577in}{0.811206in}}%
\pgfpathlineto{\pgfqpoint{0.700687in}{0.813795in}}%
\pgfpathlineto{\pgfqpoint{0.701158in}{0.814903in}}%
\pgfpathlineto{\pgfqpoint{0.702265in}{0.817268in}}%
\pgfpathlineto{\pgfqpoint{0.702734in}{0.818376in}}%
\pgfpathlineto{\pgfqpoint{0.703843in}{0.820472in}}%
\pgfpathlineto{\pgfqpoint{0.704411in}{0.821580in}}%
\pgfpathlineto{\pgfqpoint{0.705520in}{0.824197in}}%
\pgfpathlineto{\pgfqpoint{0.705973in}{0.825305in}}%
\pgfpathlineto{\pgfqpoint{0.707082in}{0.827708in}}%
\pgfpathlineto{\pgfqpoint{0.707467in}{0.828816in}}%
\pgfpathlineto{\pgfqpoint{0.708576in}{0.830976in}}%
\pgfpathlineto{\pgfqpoint{0.709043in}{0.832075in}}%
\pgfpathlineto{\pgfqpoint{0.710150in}{0.834673in}}%
\pgfpathlineto{\pgfqpoint{0.710657in}{0.835763in}}%
\pgfpathlineto{\pgfqpoint{0.711763in}{0.838343in}}%
\pgfpathlineto{\pgfqpoint{0.712202in}{0.839432in}}%
\pgfpathlineto{\pgfqpoint{0.713302in}{0.841788in}}%
\pgfpathlineto{\pgfqpoint{0.713830in}{0.842887in}}%
\pgfpathlineto{\pgfqpoint{0.714932in}{0.845457in}}%
\pgfpathlineto{\pgfqpoint{0.715481in}{0.846566in}}%
\pgfpathlineto{\pgfqpoint{0.716562in}{0.848558in}}%
\pgfpathlineto{\pgfqpoint{0.716569in}{0.848558in}}%
\pgfpathlineto{\pgfqpoint{0.717057in}{0.849657in}}%
\pgfpathlineto{\pgfqpoint{0.718161in}{0.852023in}}%
\pgfpathlineto{\pgfqpoint{0.718616in}{0.853122in}}%
\pgfpathlineto{\pgfqpoint{0.719723in}{0.855301in}}%
\pgfpathlineto{\pgfqpoint{0.720246in}{0.856390in}}%
\pgfpathlineto{\pgfqpoint{0.721346in}{0.858690in}}%
\pgfpathlineto{\pgfqpoint{0.721891in}{0.859789in}}%
\pgfpathlineto{\pgfqpoint{0.723000in}{0.862117in}}%
\pgfpathlineto{\pgfqpoint{0.723553in}{0.863207in}}%
\pgfpathlineto{\pgfqpoint{0.724658in}{0.865163in}}%
\pgfpathlineto{\pgfqpoint{0.725162in}{0.866262in}}%
\pgfpathlineto{\pgfqpoint{0.726269in}{0.868208in}}%
\pgfpathlineto{\pgfqpoint{0.726858in}{0.869307in}}%
\pgfpathlineto{\pgfqpoint{0.727960in}{0.871505in}}%
\pgfpathlineto{\pgfqpoint{0.728504in}{0.872603in}}%
\pgfpathlineto{\pgfqpoint{0.729602in}{0.874717in}}%
\pgfpathlineto{\pgfqpoint{0.730237in}{0.875826in}}%
\pgfpathlineto{\pgfqpoint{0.731340in}{0.878135in}}%
\pgfpathlineto{\pgfqpoint{0.731940in}{0.879243in}}%
\pgfpathlineto{\pgfqpoint{0.733049in}{0.881450in}}%
\pgfpathlineto{\pgfqpoint{0.733521in}{0.882549in}}%
\pgfpathlineto{\pgfqpoint{0.734630in}{0.884486in}}%
\pgfpathlineto{\pgfqpoint{0.735254in}{0.885594in}}%
\pgfpathlineto{\pgfqpoint{0.736361in}{0.887541in}}%
\pgfpathlineto{\pgfqpoint{0.736896in}{0.888621in}}%
\pgfpathlineto{\pgfqpoint{0.738005in}{0.890465in}}%
\pgfpathlineto{\pgfqpoint{0.738636in}{0.891573in}}%
\pgfpathlineto{\pgfqpoint{0.739738in}{0.893603in}}%
\pgfpathlineto{\pgfqpoint{0.740367in}{0.894711in}}%
\pgfpathlineto{\pgfqpoint{0.741467in}{0.896984in}}%
\pgfpathlineto{\pgfqpoint{0.742098in}{0.898092in}}%
\pgfpathlineto{\pgfqpoint{0.743202in}{0.900252in}}%
\pgfpathlineto{\pgfqpoint{0.743913in}{0.901351in}}%
\pgfpathlineto{\pgfqpoint{0.745022in}{0.903353in}}%
\pgfpathlineto{\pgfqpoint{0.745627in}{0.904462in}}%
\pgfpathlineto{\pgfqpoint{0.746706in}{0.906436in}}%
\pgfpathlineto{\pgfqpoint{0.746732in}{0.906436in}}%
\pgfpathlineto{\pgfqpoint{0.747410in}{0.907544in}}%
\pgfpathlineto{\pgfqpoint{0.748514in}{0.909546in}}%
\pgfpathlineto{\pgfqpoint{0.749098in}{0.910645in}}%
\pgfpathlineto{\pgfqpoint{0.750208in}{0.912610in}}%
\pgfpathlineto{\pgfqpoint{0.750829in}{0.913700in}}%
\pgfpathlineto{\pgfqpoint{0.751936in}{0.915627in}}%
\pgfpathlineto{\pgfqpoint{0.752628in}{0.916726in}}%
\pgfpathlineto{\pgfqpoint{0.753737in}{0.918281in}}%
\pgfpathlineto{\pgfqpoint{0.754382in}{0.919390in}}%
\pgfpathlineto{\pgfqpoint{0.755492in}{0.921429in}}%
\pgfpathlineto{\pgfqpoint{0.756176in}{0.922519in}}%
\pgfpathlineto{\pgfqpoint{0.757281in}{0.924391in}}%
\pgfpathlineto{\pgfqpoint{0.758001in}{0.925489in}}%
\pgfpathlineto{\pgfqpoint{0.759106in}{0.927743in}}%
\pgfpathlineto{\pgfqpoint{0.759868in}{0.928851in}}%
\pgfpathlineto{\pgfqpoint{0.760977in}{0.930947in}}%
\pgfpathlineto{\pgfqpoint{0.761657in}{0.932045in}}%
\pgfpathlineto{\pgfqpoint{0.762755in}{0.934113in}}%
\pgfpathlineto{\pgfqpoint{0.763294in}{0.935221in}}%
\pgfpathlineto{\pgfqpoint{0.764399in}{0.937056in}}%
\pgfpathlineto{\pgfqpoint{0.765039in}{0.938164in}}%
\pgfpathlineto{\pgfqpoint{0.766139in}{0.939803in}}%
\pgfpathlineto{\pgfqpoint{0.766149in}{0.939803in}}%
\pgfpathlineto{\pgfqpoint{0.766606in}{0.940911in}}%
\pgfpathlineto{\pgfqpoint{0.767715in}{0.942327in}}%
\pgfpathlineto{\pgfqpoint{0.768438in}{0.943435in}}%
\pgfpathlineto{\pgfqpoint{0.769545in}{0.945176in}}%
\pgfpathlineto{\pgfqpoint{0.770201in}{0.946266in}}%
\pgfpathlineto{\pgfqpoint{0.771308in}{0.948016in}}%
\pgfpathlineto{\pgfqpoint{0.771974in}{0.949115in}}%
\pgfpathlineto{\pgfqpoint{0.773084in}{0.950689in}}%
\pgfpathlineto{\pgfqpoint{0.773715in}{0.951788in}}%
\pgfpathlineto{\pgfqpoint{0.774824in}{0.953585in}}%
\pgfpathlineto{\pgfqpoint{0.775593in}{0.954675in}}%
\pgfpathlineto{\pgfqpoint{0.776695in}{0.956454in}}%
\pgfpathlineto{\pgfqpoint{0.777364in}{0.957553in}}%
\pgfpathlineto{\pgfqpoint{0.778473in}{0.959368in}}%
\pgfpathlineto{\pgfqpoint{0.779149in}{0.960477in}}%
\pgfpathlineto{\pgfqpoint{0.780256in}{0.962097in}}%
\pgfpathlineto{\pgfqpoint{0.780858in}{0.963205in}}%
\pgfpathlineto{\pgfqpoint{0.781951in}{0.964984in}}%
\pgfpathlineto{\pgfqpoint{0.782610in}{0.966092in}}%
\pgfpathlineto{\pgfqpoint{0.783720in}{0.967834in}}%
\pgfpathlineto{\pgfqpoint{0.784477in}{0.968942in}}%
\pgfpathlineto{\pgfqpoint{0.785584in}{0.970730in}}%
\pgfpathlineto{\pgfqpoint{0.786220in}{0.971838in}}%
\pgfpathlineto{\pgfqpoint{0.787329in}{0.973701in}}%
\pgfpathlineto{\pgfqpoint{0.788056in}{0.974809in}}%
\pgfpathlineto{\pgfqpoint{0.789165in}{0.976345in}}%
\pgfpathlineto{\pgfqpoint{0.789881in}{0.977453in}}%
\pgfpathlineto{\pgfqpoint{0.790985in}{0.978990in}}%
\pgfpathlineto{\pgfqpoint{0.791717in}{0.980089in}}%
\pgfpathlineto{\pgfqpoint{0.792826in}{0.981849in}}%
\pgfpathlineto{\pgfqpoint{0.793560in}{0.982957in}}%
\pgfpathlineto{\pgfqpoint{0.794665in}{0.984782in}}%
\pgfpathlineto{\pgfqpoint{0.795357in}{0.985881in}}%
\pgfpathlineto{\pgfqpoint{0.796464in}{0.987483in}}%
\pgfpathlineto{\pgfqpoint{0.797207in}{0.988591in}}%
\pgfpathlineto{\pgfqpoint{0.798312in}{0.990240in}}%
\pgfpathlineto{\pgfqpoint{0.799107in}{0.991348in}}%
\pgfpathlineto{\pgfqpoint{0.800216in}{0.993024in}}%
\pgfpathlineto{\pgfqpoint{0.800990in}{0.994132in}}%
\pgfpathlineto{\pgfqpoint{0.802093in}{0.995622in}}%
\pgfpathlineto{\pgfqpoint{0.802799in}{0.996703in}}%
\pgfpathlineto{\pgfqpoint{0.803899in}{0.998397in}}%
\pgfpathlineto{\pgfqpoint{0.804715in}{0.999506in}}%
\pgfpathlineto{\pgfqpoint{0.805817in}{1.001107in}}%
\pgfpathlineto{\pgfqpoint{0.806765in}{1.002216in}}%
\pgfpathlineto{\pgfqpoint{0.807874in}{1.003780in}}%
\pgfpathlineto{\pgfqpoint{0.808622in}{1.004879in}}%
\pgfpathlineto{\pgfqpoint{0.809731in}{1.006350in}}%
\pgfpathlineto{\pgfqpoint{0.810512in}{1.007459in}}%
\pgfpathlineto{\pgfqpoint{0.811622in}{1.008976in}}%
\pgfpathlineto{\pgfqpoint{0.812522in}{1.010085in}}%
\pgfpathlineto{\pgfqpoint{0.813632in}{1.011351in}}%
\pgfpathlineto{\pgfqpoint{0.814485in}{1.012459in}}%
\pgfpathlineto{\pgfqpoint{0.815581in}{1.014061in}}%
\pgfpathlineto{\pgfqpoint{0.816406in}{1.015169in}}%
\pgfpathlineto{\pgfqpoint{0.817499in}{1.016920in}}%
\pgfpathlineto{\pgfqpoint{0.818371in}{1.018028in}}%
\pgfpathlineto{\pgfqpoint{0.819478in}{1.019304in}}%
\pgfpathlineto{\pgfqpoint{0.820325in}{1.020412in}}%
\pgfpathlineto{\pgfqpoint{0.821434in}{1.022182in}}%
\pgfpathlineto{\pgfqpoint{0.822405in}{1.023290in}}%
\pgfpathlineto{\pgfqpoint{0.823515in}{1.024743in}}%
\pgfpathlineto{\pgfqpoint{0.824289in}{1.025851in}}%
\pgfpathlineto{\pgfqpoint{0.825396in}{1.027294in}}%
\pgfpathlineto{\pgfqpoint{0.826212in}{1.028402in}}%
\pgfpathlineto{\pgfqpoint{0.827319in}{1.029809in}}%
\pgfpathlineto{\pgfqpoint{0.828184in}{1.030917in}}%
\pgfpathlineto{\pgfqpoint{0.829282in}{1.032528in}}%
\pgfpathlineto{\pgfqpoint{0.830018in}{1.033627in}}%
\pgfpathlineto{\pgfqpoint{0.831125in}{1.035294in}}%
\pgfpathlineto{\pgfqpoint{0.831946in}{1.036393in}}%
\pgfpathlineto{\pgfqpoint{0.833046in}{1.037873in}}%
\pgfpathlineto{\pgfqpoint{0.834000in}{1.038982in}}%
\pgfpathlineto{\pgfqpoint{0.835105in}{1.040388in}}%
\pgfpathlineto{\pgfqpoint{0.835879in}{1.041468in}}%
\pgfpathlineto{\pgfqpoint{0.836986in}{1.042753in}}%
\pgfpathlineto{\pgfqpoint{0.837870in}{1.043852in}}%
\pgfpathlineto{\pgfqpoint{0.838970in}{1.045240in}}%
\pgfpathlineto{\pgfqpoint{0.839807in}{1.046348in}}%
\pgfpathlineto{\pgfqpoint{0.840917in}{1.047717in}}%
\pgfpathlineto{\pgfqpoint{0.841740in}{1.048825in}}%
\pgfpathlineto{\pgfqpoint{0.842845in}{1.049905in}}%
\pgfpathlineto{\pgfqpoint{0.843785in}{1.051004in}}%
\pgfpathlineto{\pgfqpoint{0.844883in}{1.052503in}}%
\pgfpathlineto{\pgfqpoint{0.845811in}{1.053612in}}%
\pgfpathlineto{\pgfqpoint{0.846904in}{1.055334in}}%
\pgfpathlineto{\pgfqpoint{0.847847in}{1.056433in}}%
\pgfpathlineto{\pgfqpoint{0.848952in}{1.057923in}}%
\pgfpathlineto{\pgfqpoint{0.849876in}{1.059031in}}%
\pgfpathlineto{\pgfqpoint{0.850980in}{1.060289in}}%
\pgfpathlineto{\pgfqpoint{0.851776in}{1.061397in}}%
\pgfpathlineto{\pgfqpoint{0.852873in}{1.062905in}}%
\pgfpathlineto{\pgfqpoint{0.853839in}{1.064004in}}%
\pgfpathlineto{\pgfqpoint{0.854949in}{1.065606in}}%
\pgfpathlineto{\pgfqpoint{0.855920in}{1.066714in}}%
\pgfpathlineto{\pgfqpoint{0.857015in}{1.068139in}}%
\pgfpathlineto{\pgfqpoint{0.857883in}{1.069238in}}%
\pgfpathlineto{\pgfqpoint{0.858980in}{1.070830in}}%
\pgfpathlineto{\pgfqpoint{0.859871in}{1.071939in}}%
\pgfpathlineto{\pgfqpoint{0.860981in}{1.073475in}}%
\pgfpathlineto{\pgfqpoint{0.862095in}{1.074574in}}%
\pgfpathlineto{\pgfqpoint{0.863195in}{1.075897in}}%
\pgfpathlineto{\pgfqpoint{0.864156in}{1.077005in}}%
\pgfpathlineto{\pgfqpoint{0.865266in}{1.078364in}}%
\pgfpathlineto{\pgfqpoint{0.866023in}{1.079473in}}%
\pgfpathlineto{\pgfqpoint{0.867133in}{1.080814in}}%
\pgfpathlineto{\pgfqpoint{0.868186in}{1.081922in}}%
\pgfpathlineto{\pgfqpoint{0.869288in}{1.083123in}}%
\pgfpathlineto{\pgfqpoint{0.870242in}{1.084222in}}%
\pgfpathlineto{\pgfqpoint{0.871349in}{1.085805in}}%
\pgfpathlineto{\pgfqpoint{0.872290in}{1.086895in}}%
\pgfpathlineto{\pgfqpoint{0.873397in}{1.088347in}}%
\pgfpathlineto{\pgfqpoint{0.874377in}{1.089456in}}%
\pgfpathlineto{\pgfqpoint{0.875435in}{1.090862in}}%
\pgfpathlineto{\pgfqpoint{0.876418in}{1.091970in}}%
\pgfpathlineto{\pgfqpoint{0.877499in}{1.093078in}}%
\pgfpathlineto{\pgfqpoint{0.878578in}{1.094186in}}%
\pgfpathlineto{\pgfqpoint{0.879687in}{1.095341in}}%
\pgfpathlineto{\pgfqpoint{0.880559in}{1.096449in}}%
\pgfpathlineto{\pgfqpoint{0.881659in}{1.097623in}}%
\pgfpathlineto{\pgfqpoint{0.881669in}{1.097623in}}%
\pgfpathlineto{\pgfqpoint{0.882680in}{1.098731in}}%
\pgfpathlineto{\pgfqpoint{0.883768in}{1.100016in}}%
\pgfpathlineto{\pgfqpoint{0.883784in}{1.100016in}}%
\pgfpathlineto{\pgfqpoint{0.884659in}{1.101115in}}%
\pgfpathlineto{\pgfqpoint{0.885745in}{1.102512in}}%
\pgfpathlineto{\pgfqpoint{0.886889in}{1.103620in}}%
\pgfpathlineto{\pgfqpoint{0.887987in}{1.104840in}}%
\pgfpathlineto{\pgfqpoint{0.889052in}{1.105948in}}%
\pgfpathlineto{\pgfqpoint{0.890147in}{1.106945in}}%
\pgfpathlineto{\pgfqpoint{0.891144in}{1.108043in}}%
\pgfpathlineto{\pgfqpoint{0.892239in}{1.109366in}}%
\pgfpathlineto{\pgfqpoint{0.893177in}{1.110474in}}%
\pgfpathlineto{\pgfqpoint{0.894270in}{1.111694in}}%
\pgfpathlineto{\pgfqpoint{0.895241in}{1.112802in}}%
\pgfpathlineto{\pgfqpoint{0.896350in}{1.114013in}}%
\pgfpathlineto{\pgfqpoint{0.897302in}{1.115121in}}%
\pgfpathlineto{\pgfqpoint{0.898409in}{1.116415in}}%
\pgfpathlineto{\pgfqpoint{0.899409in}{1.117514in}}%
\pgfpathlineto{\pgfqpoint{0.900513in}{1.118809in}}%
\pgfpathlineto{\pgfqpoint{0.901522in}{1.119908in}}%
\pgfpathlineto{\pgfqpoint{0.902612in}{1.121323in}}%
\pgfpathlineto{\pgfqpoint{0.903787in}{1.122431in}}%
\pgfpathlineto{\pgfqpoint{0.904875in}{1.123782in}}%
\pgfpathlineto{\pgfqpoint{0.906090in}{1.124890in}}%
\pgfpathlineto{\pgfqpoint{0.907179in}{1.126249in}}%
\pgfpathlineto{\pgfqpoint{0.908023in}{1.127348in}}%
\pgfpathlineto{\pgfqpoint{0.909120in}{1.128568in}}%
\pgfpathlineto{\pgfqpoint{0.909130in}{1.128568in}}%
\pgfpathlineto{\pgfqpoint{0.910298in}{1.129667in}}%
\pgfpathlineto{\pgfqpoint{0.911402in}{1.131055in}}%
\pgfpathlineto{\pgfqpoint{0.912502in}{1.132163in}}%
\pgfpathlineto{\pgfqpoint{0.913602in}{1.133355in}}%
\pgfpathlineto{\pgfqpoint{0.913612in}{1.133355in}}%
\pgfpathlineto{\pgfqpoint{0.914700in}{1.134463in}}%
\pgfpathlineto{\pgfqpoint{0.915807in}{1.135664in}}%
\pgfpathlineto{\pgfqpoint{0.916862in}{1.136773in}}%
\pgfpathlineto{\pgfqpoint{0.917967in}{1.137704in}}%
\pgfpathlineto{\pgfqpoint{0.918900in}{1.138812in}}%
\pgfpathlineto{\pgfqpoint{0.920010in}{1.140051in}}%
\pgfpathlineto{\pgfqpoint{0.921035in}{1.141159in}}%
\pgfpathlineto{\pgfqpoint{0.922144in}{1.142183in}}%
\pgfpathlineto{\pgfqpoint{0.923335in}{1.143291in}}%
\pgfpathlineto{\pgfqpoint{0.924445in}{1.144558in}}%
\pgfpathlineto{\pgfqpoint{0.925636in}{1.145666in}}%
\pgfpathlineto{\pgfqpoint{0.926743in}{1.146663in}}%
\pgfpathlineto{\pgfqpoint{0.927770in}{1.147771in}}%
\pgfpathlineto{\pgfqpoint{0.928873in}{1.148860in}}%
\pgfpathlineto{\pgfqpoint{0.929832in}{1.149969in}}%
\pgfpathlineto{\pgfqpoint{0.930918in}{1.151067in}}%
\pgfpathlineto{\pgfqpoint{0.930932in}{1.151067in}}%
\pgfpathlineto{\pgfqpoint{0.932069in}{1.152176in}}%
\pgfpathlineto{\pgfqpoint{0.933171in}{1.153368in}}%
\pgfpathlineto{\pgfqpoint{0.933179in}{1.153368in}}%
\pgfpathlineto{\pgfqpoint{0.934215in}{1.154457in}}%
\pgfpathlineto{\pgfqpoint{0.935313in}{1.155714in}}%
\pgfpathlineto{\pgfqpoint{0.936438in}{1.156823in}}%
\pgfpathlineto{\pgfqpoint{0.937545in}{1.157968in}}%
\pgfpathlineto{\pgfqpoint{0.938643in}{1.159058in}}%
\pgfpathlineto{\pgfqpoint{0.938643in}{1.159067in}}%
\pgfpathlineto{\pgfqpoint{0.939750in}{1.160333in}}%
\pgfpathlineto{\pgfqpoint{0.940754in}{1.161442in}}%
\pgfpathlineto{\pgfqpoint{0.941842in}{1.162503in}}%
\pgfpathlineto{\pgfqpoint{0.942914in}{1.163602in}}%
\pgfpathlineto{\pgfqpoint{0.944014in}{1.164701in}}%
\pgfpathlineto{\pgfqpoint{0.945449in}{1.165809in}}%
\pgfpathlineto{\pgfqpoint{0.946558in}{1.166703in}}%
\pgfpathlineto{\pgfqpoint{0.947783in}{1.167811in}}%
\pgfpathlineto{\pgfqpoint{0.948890in}{1.168854in}}%
\pgfpathlineto{\pgfqpoint{0.950004in}{1.169963in}}%
\pgfpathlineto{\pgfqpoint{0.951104in}{1.170903in}}%
\pgfpathlineto{\pgfqpoint{0.952368in}{1.171974in}}%
\pgfpathlineto{\pgfqpoint{0.953475in}{1.173036in}}%
\pgfpathlineto{\pgfqpoint{0.954633in}{1.174144in}}%
\pgfpathlineto{\pgfqpoint{0.955712in}{1.175224in}}%
\pgfpathlineto{\pgfqpoint{0.956887in}{1.176323in}}%
\pgfpathlineto{\pgfqpoint{0.957996in}{1.177655in}}%
\pgfpathlineto{\pgfqpoint{0.958986in}{1.178763in}}%
\pgfpathlineto{\pgfqpoint{0.960067in}{1.179862in}}%
\pgfpathlineto{\pgfqpoint{0.961310in}{1.180970in}}%
\pgfpathlineto{\pgfqpoint{0.962415in}{1.182106in}}%
\pgfpathlineto{\pgfqpoint{0.963503in}{1.183214in}}%
\pgfpathlineto{\pgfqpoint{0.964613in}{1.184434in}}%
\pgfpathlineto{\pgfqpoint{0.965818in}{1.185543in}}%
\pgfpathlineto{\pgfqpoint{0.966918in}{1.186632in}}%
\pgfpathlineto{\pgfqpoint{0.967955in}{1.187740in}}%
\pgfpathlineto{\pgfqpoint{0.969052in}{1.188737in}}%
\pgfpathlineto{\pgfqpoint{0.970276in}{1.189845in}}%
\pgfpathlineto{\pgfqpoint{0.971372in}{1.190767in}}%
\pgfpathlineto{\pgfqpoint{0.972720in}{1.191875in}}%
\pgfpathlineto{\pgfqpoint{0.973827in}{1.192955in}}%
\pgfpathlineto{\pgfqpoint{0.975105in}{1.194064in}}%
\pgfpathlineto{\pgfqpoint{0.976212in}{1.195293in}}%
\pgfpathlineto{\pgfqpoint{0.977720in}{1.196401in}}%
\pgfpathlineto{\pgfqpoint{0.979203in}{1.197779in}}%
\pgfpathlineto{\pgfqpoint{0.980589in}{1.198887in}}%
\pgfpathlineto{\pgfqpoint{0.981698in}{1.199996in}}%
\pgfpathlineto{\pgfqpoint{0.982857in}{1.201095in}}%
\pgfpathlineto{\pgfqpoint{0.983964in}{1.202287in}}%
\pgfpathlineto{\pgfqpoint{0.985319in}{1.203385in}}%
\pgfpathlineto{\pgfqpoint{0.986410in}{1.204242in}}%
\pgfpathlineto{\pgfqpoint{0.987730in}{1.205350in}}%
\pgfpathlineto{\pgfqpoint{0.988837in}{1.206300in}}%
\pgfpathlineto{\pgfqpoint{0.990017in}{1.207408in}}%
\pgfpathlineto{\pgfqpoint{0.991089in}{1.208144in}}%
\pgfpathlineto{\pgfqpoint{0.992522in}{1.209252in}}%
\pgfpathlineto{\pgfqpoint{0.993631in}{1.210277in}}%
\pgfpathlineto{\pgfqpoint{0.995038in}{1.211385in}}%
\pgfpathlineto{\pgfqpoint{0.996147in}{1.212512in}}%
\pgfpathlineto{\pgfqpoint{0.997440in}{1.213611in}}%
\pgfpathlineto{\pgfqpoint{0.998542in}{1.214607in}}%
\pgfpathlineto{\pgfqpoint{0.999853in}{1.215715in}}%
\pgfpathlineto{\pgfqpoint{1.000962in}{1.216646in}}%
\pgfpathlineto{\pgfqpoint{1.002161in}{1.217755in}}%
\pgfpathlineto{\pgfqpoint{1.003270in}{1.218472in}}%
\pgfpathlineto{\pgfqpoint{1.004337in}{1.219580in}}%
\pgfpathlineto{\pgfqpoint{1.005432in}{1.220688in}}%
\pgfpathlineto{\pgfqpoint{1.006406in}{1.221796in}}%
\pgfpathlineto{\pgfqpoint{1.007513in}{1.222858in}}%
\pgfpathlineto{\pgfqpoint{1.008721in}{1.223966in}}%
\pgfpathlineto{\pgfqpoint{1.009825in}{1.225000in}}%
\pgfpathlineto{\pgfqpoint{1.011289in}{1.226108in}}%
\pgfpathlineto{\pgfqpoint{1.012384in}{1.227170in}}%
\pgfpathlineto{\pgfqpoint{1.013920in}{1.228259in}}%
\pgfpathlineto{\pgfqpoint{1.015027in}{1.229470in}}%
\pgfpathlineto{\pgfqpoint{1.016392in}{1.230569in}}%
\pgfpathlineto{\pgfqpoint{1.017494in}{1.231584in}}%
\pgfpathlineto{\pgfqpoint{1.018883in}{1.232692in}}%
\pgfpathlineto{\pgfqpoint{1.019966in}{1.233558in}}%
\pgfpathlineto{\pgfqpoint{1.021383in}{1.234666in}}%
\pgfpathlineto{\pgfqpoint{1.022487in}{1.235728in}}%
\pgfpathlineto{\pgfqpoint{1.023784in}{1.236836in}}%
\pgfpathlineto{\pgfqpoint{1.024873in}{1.237488in}}%
\pgfpathlineto{\pgfqpoint{1.026188in}{1.238596in}}%
\pgfpathlineto{\pgfqpoint{1.027286in}{1.239378in}}%
\pgfpathlineto{\pgfqpoint{1.028834in}{1.240487in}}%
\pgfpathlineto{\pgfqpoint{1.029936in}{1.241446in}}%
\pgfpathlineto{\pgfqpoint{1.031313in}{1.242554in}}%
\pgfpathlineto{\pgfqpoint{1.032378in}{1.243383in}}%
\pgfpathlineto{\pgfqpoint{1.033904in}{1.244491in}}%
\pgfpathlineto{\pgfqpoint{1.035011in}{1.245487in}}%
\pgfpathlineto{\pgfqpoint{1.036322in}{1.246596in}}%
\pgfpathlineto{\pgfqpoint{1.037420in}{1.247518in}}%
\pgfpathlineto{\pgfqpoint{1.038954in}{1.248626in}}%
\pgfpathlineto{\pgfqpoint{1.040063in}{1.249687in}}%
\pgfpathlineto{\pgfqpoint{1.041280in}{1.250786in}}%
\pgfpathlineto{\pgfqpoint{1.042369in}{1.251745in}}%
\pgfpathlineto{\pgfqpoint{1.043682in}{1.252854in}}%
\pgfpathlineto{\pgfqpoint{1.044787in}{1.253766in}}%
\pgfpathlineto{\pgfqpoint{1.046128in}{1.254875in}}%
\pgfpathlineto{\pgfqpoint{1.047212in}{1.255703in}}%
\pgfpathlineto{\pgfqpoint{1.048642in}{1.256812in}}%
\pgfpathlineto{\pgfqpoint{1.049747in}{1.257706in}}%
\pgfpathlineto{\pgfqpoint{1.051006in}{1.258814in}}%
\pgfpathlineto{\pgfqpoint{1.052102in}{1.259708in}}%
\pgfpathlineto{\pgfqpoint{1.053657in}{1.260816in}}%
\pgfpathlineto{\pgfqpoint{1.054747in}{1.261915in}}%
\pgfpathlineto{\pgfqpoint{1.056464in}{1.263023in}}%
\pgfpathlineto{\pgfqpoint{1.057568in}{1.263936in}}%
\pgfpathlineto{\pgfqpoint{1.059039in}{1.265044in}}%
\pgfpathlineto{\pgfqpoint{1.060148in}{1.265863in}}%
\pgfpathlineto{\pgfqpoint{1.061731in}{1.266972in}}%
\pgfpathlineto{\pgfqpoint{1.062838in}{1.267856in}}%
\pgfpathlineto{\pgfqpoint{1.064351in}{1.268964in}}%
\pgfpathlineto{\pgfqpoint{1.065458in}{1.269803in}}%
\pgfpathlineto{\pgfqpoint{1.066771in}{1.270892in}}%
\pgfpathlineto{\pgfqpoint{1.067867in}{1.271758in}}%
\pgfpathlineto{\pgfqpoint{1.067881in}{1.271758in}}%
\pgfpathlineto{\pgfqpoint{1.069408in}{1.272866in}}%
\pgfpathlineto{\pgfqpoint{1.070505in}{1.273695in}}%
\pgfpathlineto{\pgfqpoint{1.071884in}{1.274785in}}%
\pgfpathlineto{\pgfqpoint{1.072991in}{1.275725in}}%
\pgfpathlineto{\pgfqpoint{1.074276in}{1.276834in}}%
\pgfpathlineto{\pgfqpoint{1.075379in}{1.277700in}}%
\pgfpathlineto{\pgfqpoint{1.076842in}{1.278808in}}%
\pgfpathlineto{\pgfqpoint{1.077951in}{1.279609in}}%
\pgfpathlineto{\pgfqpoint{1.079598in}{1.280717in}}%
\pgfpathlineto{\pgfqpoint{1.080707in}{1.281508in}}%
\pgfpathlineto{\pgfqpoint{1.081863in}{1.282617in}}%
\pgfpathlineto{\pgfqpoint{1.082956in}{1.283539in}}%
\pgfpathlineto{\pgfqpoint{1.084448in}{1.284647in}}%
\pgfpathlineto{\pgfqpoint{1.085522in}{1.285280in}}%
\pgfpathlineto{\pgfqpoint{1.086929in}{1.286388in}}%
\pgfpathlineto{\pgfqpoint{1.088039in}{1.287273in}}%
\pgfpathlineto{\pgfqpoint{1.089615in}{1.288381in}}%
\pgfpathlineto{\pgfqpoint{1.090710in}{1.289154in}}%
\pgfpathlineto{\pgfqpoint{1.092162in}{1.290262in}}%
\pgfpathlineto{\pgfqpoint{1.093240in}{1.291091in}}%
\pgfpathlineto{\pgfqpoint{1.093271in}{1.291091in}}%
\pgfpathlineto{\pgfqpoint{1.094906in}{1.292199in}}%
\pgfpathlineto{\pgfqpoint{1.096003in}{1.292972in}}%
\pgfpathlineto{\pgfqpoint{1.097697in}{1.294080in}}%
\pgfpathlineto{\pgfqpoint{1.098796in}{1.294751in}}%
\pgfpathlineto{\pgfqpoint{1.100328in}{1.295859in}}%
\pgfpathlineto{\pgfqpoint{1.101426in}{1.296595in}}%
\pgfpathlineto{\pgfqpoint{1.102908in}{1.297703in}}%
\pgfpathlineto{\pgfqpoint{1.104012in}{1.298476in}}%
\pgfpathlineto{\pgfqpoint{1.105596in}{1.299584in}}%
\pgfpathlineto{\pgfqpoint{1.106686in}{1.300301in}}%
\pgfpathlineto{\pgfqpoint{1.108147in}{1.301409in}}%
\pgfpathlineto{\pgfqpoint{1.109254in}{1.302136in}}%
\pgfpathlineto{\pgfqpoint{1.110999in}{1.303244in}}%
\pgfpathlineto{\pgfqpoint{1.112106in}{1.304110in}}%
\pgfpathlineto{\pgfqpoint{1.113574in}{1.305218in}}%
\pgfpathlineto{\pgfqpoint{1.114679in}{1.306094in}}%
\pgfpathlineto{\pgfqpoint{1.116255in}{1.307202in}}%
\pgfpathlineto{\pgfqpoint{1.117357in}{1.307919in}}%
\pgfpathlineto{\pgfqpoint{1.119184in}{1.309027in}}%
\pgfpathlineto{\pgfqpoint{1.120270in}{1.309912in}}%
\pgfpathlineto{\pgfqpoint{1.121938in}{1.311020in}}%
\pgfpathlineto{\pgfqpoint{1.123026in}{1.311746in}}%
\pgfpathlineto{\pgfqpoint{1.124822in}{1.312854in}}%
\pgfpathlineto{\pgfqpoint{1.125925in}{1.313655in}}%
\pgfpathlineto{\pgfqpoint{1.127533in}{1.314764in}}%
\pgfpathlineto{\pgfqpoint{1.128608in}{1.315453in}}%
\pgfpathlineto{\pgfqpoint{1.130195in}{1.316561in}}%
\pgfpathlineto{\pgfqpoint{1.131300in}{1.317259in}}%
\pgfpathlineto{\pgfqpoint{1.133207in}{1.318368in}}%
\pgfpathlineto{\pgfqpoint{1.134316in}{1.319038in}}%
\pgfpathlineto{\pgfqpoint{1.136131in}{1.320137in}}%
\pgfpathlineto{\pgfqpoint{1.137236in}{1.320938in}}%
\pgfpathlineto{\pgfqpoint{1.138983in}{1.322046in}}%
\pgfpathlineto{\pgfqpoint{1.140081in}{1.322893in}}%
\pgfpathlineto{\pgfqpoint{1.140088in}{1.322893in}}%
\pgfpathlineto{\pgfqpoint{1.141854in}{1.324002in}}%
\pgfpathlineto{\pgfqpoint{1.142930in}{1.324784in}}%
\pgfpathlineto{\pgfqpoint{1.144403in}{1.325892in}}%
\pgfpathlineto{\pgfqpoint{1.145503in}{1.326674in}}%
\pgfpathlineto{\pgfqpoint{1.145508in}{1.326674in}}%
\pgfpathlineto{\pgfqpoint{1.147304in}{1.327773in}}%
\pgfpathlineto{\pgfqpoint{1.148407in}{1.328481in}}%
\pgfpathlineto{\pgfqpoint{1.150140in}{1.329589in}}%
\pgfpathlineto{\pgfqpoint{1.151240in}{1.330483in}}%
\pgfpathlineto{\pgfqpoint{1.152935in}{1.331591in}}%
\pgfpathlineto{\pgfqpoint{1.153979in}{1.332411in}}%
\pgfpathlineto{\pgfqpoint{1.153993in}{1.332411in}}%
\pgfpathlineto{\pgfqpoint{1.155654in}{1.333519in}}%
\pgfpathlineto{\pgfqpoint{1.156754in}{1.334301in}}%
\pgfpathlineto{\pgfqpoint{1.158688in}{1.335410in}}%
\pgfpathlineto{\pgfqpoint{1.159774in}{1.336201in}}%
\pgfpathlineto{\pgfqpoint{1.161477in}{1.337300in}}%
\pgfpathlineto{\pgfqpoint{1.162572in}{1.337905in}}%
\pgfpathlineto{\pgfqpoint{1.164448in}{1.339013in}}%
\pgfpathlineto{\pgfqpoint{1.165558in}{1.339637in}}%
\pgfpathlineto{\pgfqpoint{1.167321in}{1.340746in}}%
\pgfpathlineto{\pgfqpoint{1.168424in}{1.341491in}}%
\pgfpathlineto{\pgfqpoint{1.170401in}{1.342599in}}%
\pgfpathlineto{\pgfqpoint{1.171501in}{1.343316in}}%
\pgfpathlineto{\pgfqpoint{1.173255in}{1.344424in}}%
\pgfpathlineto{\pgfqpoint{1.174355in}{1.345281in}}%
\pgfpathlineto{\pgfqpoint{1.176421in}{1.346389in}}%
\pgfpathlineto{\pgfqpoint{1.177528in}{1.347162in}}%
\pgfpathlineto{\pgfqpoint{1.178980in}{1.348270in}}%
\pgfpathlineto{\pgfqpoint{1.180080in}{1.348969in}}%
\pgfpathlineto{\pgfqpoint{1.181773in}{1.350077in}}%
\pgfpathlineto{\pgfqpoint{1.182840in}{1.350747in}}%
\pgfpathlineto{\pgfqpoint{1.184656in}{1.351855in}}%
\pgfpathlineto{\pgfqpoint{1.185751in}{1.352489in}}%
\pgfpathlineto{\pgfqpoint{1.187604in}{1.353597in}}%
\pgfpathlineto{\pgfqpoint{1.188659in}{1.354267in}}%
\pgfpathlineto{\pgfqpoint{1.190545in}{1.355376in}}%
\pgfpathlineto{\pgfqpoint{1.191647in}{1.355972in}}%
\pgfpathlineto{\pgfqpoint{1.193586in}{1.357080in}}%
\pgfpathlineto{\pgfqpoint{1.194658in}{1.357769in}}%
\pgfpathlineto{\pgfqpoint{1.196689in}{1.358877in}}%
\pgfpathlineto{\pgfqpoint{1.197759in}{1.359482in}}%
\pgfpathlineto{\pgfqpoint{1.199656in}{1.360591in}}%
\pgfpathlineto{\pgfqpoint{1.200763in}{1.361429in}}%
\pgfpathlineto{\pgfqpoint{1.203087in}{1.362537in}}%
\pgfpathlineto{\pgfqpoint{1.204190in}{1.363245in}}%
\pgfpathlineto{\pgfqpoint{1.204197in}{1.363245in}}%
\pgfpathlineto{\pgfqpoint{1.205822in}{1.364353in}}%
\pgfpathlineto{\pgfqpoint{1.206917in}{1.365023in}}%
\pgfpathlineto{\pgfqpoint{1.209009in}{1.366132in}}%
\pgfpathlineto{\pgfqpoint{1.210107in}{1.366895in}}%
\pgfpathlineto{\pgfqpoint{1.211842in}{1.367985in}}%
\pgfpathlineto{\pgfqpoint{1.212933in}{1.368618in}}%
\pgfpathlineto{\pgfqpoint{1.214856in}{1.369717in}}%
\pgfpathlineto{\pgfqpoint{1.215932in}{1.370341in}}%
\pgfpathlineto{\pgfqpoint{1.218034in}{1.371449in}}%
\pgfpathlineto{\pgfqpoint{1.219131in}{1.372101in}}%
\pgfpathlineto{\pgfqpoint{1.220778in}{1.373209in}}%
\pgfpathlineto{\pgfqpoint{1.221875in}{1.373852in}}%
\pgfpathlineto{\pgfqpoint{1.223970in}{1.374951in}}%
\pgfpathlineto{\pgfqpoint{1.225067in}{1.375621in}}%
\pgfpathlineto{\pgfqpoint{1.226707in}{1.376729in}}%
\pgfpathlineto{\pgfqpoint{1.227790in}{1.377307in}}%
\pgfpathlineto{\pgfqpoint{1.229697in}{1.378415in}}%
\pgfpathlineto{\pgfqpoint{1.230785in}{1.379039in}}%
\pgfpathlineto{\pgfqpoint{1.232805in}{1.380147in}}%
\pgfpathlineto{\pgfqpoint{1.233886in}{1.380948in}}%
\pgfpathlineto{\pgfqpoint{1.235778in}{1.382056in}}%
\pgfpathlineto{\pgfqpoint{1.236888in}{1.382643in}}%
\pgfpathlineto{\pgfqpoint{1.238891in}{1.383751in}}%
\pgfpathlineto{\pgfqpoint{1.239991in}{1.384412in}}%
\pgfpathlineto{\pgfqpoint{1.242104in}{1.385511in}}%
\pgfpathlineto{\pgfqpoint{1.243211in}{1.386144in}}%
\pgfpathlineto{\pgfqpoint{1.245361in}{1.387253in}}%
\pgfpathlineto{\pgfqpoint{1.246419in}{1.387802in}}%
\pgfpathlineto{\pgfqpoint{1.249041in}{1.388910in}}%
\pgfpathlineto{\pgfqpoint{1.250143in}{1.389627in}}%
\pgfpathlineto{\pgfqpoint{1.252414in}{1.390735in}}%
\pgfpathlineto{\pgfqpoint{1.253521in}{1.391425in}}%
\pgfpathlineto{\pgfqpoint{1.255481in}{1.392533in}}%
\pgfpathlineto{\pgfqpoint{1.256576in}{1.393129in}}%
\pgfpathlineto{\pgfqpoint{1.258790in}{1.394237in}}%
\pgfpathlineto{\pgfqpoint{1.259900in}{1.394945in}}%
\pgfpathlineto{\pgfqpoint{1.261910in}{1.396053in}}%
\pgfpathlineto{\pgfqpoint{1.263019in}{1.396695in}}%
\pgfpathlineto{\pgfqpoint{1.265184in}{1.397794in}}%
\pgfpathlineto{\pgfqpoint{1.266293in}{1.398362in}}%
\pgfpathlineto{\pgfqpoint{1.268732in}{1.399471in}}%
\pgfpathlineto{\pgfqpoint{1.269842in}{1.400216in}}%
\pgfpathlineto{\pgfqpoint{1.271936in}{1.401324in}}%
\pgfpathlineto{\pgfqpoint{1.273038in}{1.401957in}}%
\pgfpathlineto{\pgfqpoint{1.275527in}{1.403065in}}%
\pgfpathlineto{\pgfqpoint{1.276636in}{1.403736in}}%
\pgfpathlineto{\pgfqpoint{1.278897in}{1.404844in}}%
\pgfpathlineto{\pgfqpoint{1.280004in}{1.405580in}}%
\pgfpathlineto{\pgfqpoint{1.282218in}{1.406688in}}%
\pgfpathlineto{\pgfqpoint{1.283299in}{1.407247in}}%
\pgfpathlineto{\pgfqpoint{1.285757in}{1.408355in}}%
\pgfpathlineto{\pgfqpoint{1.286859in}{1.408867in}}%
\pgfpathlineto{\pgfqpoint{1.288951in}{1.409966in}}%
\pgfpathlineto{\pgfqpoint{1.290056in}{1.410636in}}%
\pgfpathlineto{\pgfqpoint{1.292422in}{1.411745in}}%
\pgfpathlineto{\pgfqpoint{1.293508in}{1.412313in}}%
\pgfpathlineto{\pgfqpoint{1.295954in}{1.413421in}}%
\pgfpathlineto{\pgfqpoint{1.297054in}{1.413970in}}%
\pgfpathlineto{\pgfqpoint{1.299343in}{1.415078in}}%
\pgfpathlineto{\pgfqpoint{1.300434in}{1.415591in}}%
\pgfpathlineto{\pgfqpoint{1.302957in}{1.416699in}}%
\pgfpathlineto{\pgfqpoint{1.304036in}{1.417360in}}%
\pgfpathlineto{\pgfqpoint{1.306123in}{1.418468in}}%
\pgfpathlineto{\pgfqpoint{1.307214in}{1.419018in}}%
\pgfpathlineto{\pgfqpoint{1.309524in}{1.420117in}}%
\pgfpathlineto{\pgfqpoint{1.310633in}{1.420666in}}%
\pgfpathlineto{\pgfqpoint{1.312571in}{1.421774in}}%
\pgfpathlineto{\pgfqpoint{1.313670in}{1.422389in}}%
\pgfpathlineto{\pgfqpoint{1.316018in}{1.423497in}}%
\pgfpathlineto{\pgfqpoint{1.317127in}{1.424186in}}%
\pgfpathlineto{\pgfqpoint{1.319287in}{1.425294in}}%
\pgfpathlineto{\pgfqpoint{1.320390in}{1.425853in}}%
\pgfpathlineto{\pgfqpoint{1.322151in}{1.426961in}}%
\pgfpathlineto{\pgfqpoint{1.323258in}{1.427408in}}%
\pgfpathlineto{\pgfqpoint{1.325207in}{1.428507in}}%
\pgfpathlineto{\pgfqpoint{1.326309in}{1.429029in}}%
\pgfpathlineto{\pgfqpoint{1.328901in}{1.430137in}}%
\pgfpathlineto{\pgfqpoint{1.330003in}{1.430770in}}%
\pgfpathlineto{\pgfqpoint{1.332386in}{1.431878in}}%
\pgfpathlineto{\pgfqpoint{1.333495in}{1.432567in}}%
\pgfpathlineto{\pgfqpoint{1.335658in}{1.433676in}}%
\pgfpathlineto{\pgfqpoint{1.336767in}{1.434123in}}%
\pgfpathlineto{\pgfqpoint{1.339047in}{1.435231in}}%
\pgfpathlineto{\pgfqpoint{1.340114in}{1.435901in}}%
\pgfpathlineto{\pgfqpoint{1.342907in}{1.437009in}}%
\pgfpathlineto{\pgfqpoint{1.343979in}{1.437587in}}%
\pgfpathlineto{\pgfqpoint{1.346106in}{1.438695in}}%
\pgfpathlineto{\pgfqpoint{1.347215in}{1.439217in}}%
\pgfpathlineto{\pgfqpoint{1.349305in}{1.440325in}}%
\pgfpathlineto{\pgfqpoint{1.350384in}{1.440865in}}%
\pgfpathlineto{\pgfqpoint{1.352380in}{1.441973in}}%
\pgfpathlineto{\pgfqpoint{1.353484in}{1.442448in}}%
\pgfpathlineto{\pgfqpoint{1.355635in}{1.443556in}}%
\pgfpathlineto{\pgfqpoint{1.356740in}{1.444050in}}%
\pgfpathlineto{\pgfqpoint{1.359495in}{1.445158in}}%
\pgfpathlineto{\pgfqpoint{1.360600in}{1.445596in}}%
\pgfpathlineto{\pgfqpoint{1.363191in}{1.446704in}}%
\pgfpathlineto{\pgfqpoint{1.364268in}{1.447300in}}%
\pgfpathlineto{\pgfqpoint{1.366620in}{1.448408in}}%
\pgfpathlineto{\pgfqpoint{1.367713in}{1.449041in}}%
\pgfpathlineto{\pgfqpoint{1.370729in}{1.450150in}}%
\pgfpathlineto{\pgfqpoint{1.371839in}{1.450578in}}%
\pgfpathlineto{\pgfqpoint{1.374343in}{1.451686in}}%
\pgfpathlineto{\pgfqpoint{1.375446in}{1.452245in}}%
\pgfpathlineto{\pgfqpoint{1.378072in}{1.453344in}}%
\pgfpathlineto{\pgfqpoint{1.379123in}{1.453875in}}%
\pgfpathlineto{\pgfqpoint{1.381635in}{1.454983in}}%
\pgfpathlineto{\pgfqpoint{1.382728in}{1.455541in}}%
\pgfpathlineto{\pgfqpoint{1.385725in}{1.456650in}}%
\pgfpathlineto{\pgfqpoint{1.386780in}{1.457022in}}%
\pgfpathlineto{\pgfqpoint{1.389323in}{1.458121in}}%
\pgfpathlineto{\pgfqpoint{1.390376in}{1.458447in}}%
\pgfpathlineto{\pgfqpoint{1.390411in}{1.458447in}}%
\pgfpathlineto{\pgfqpoint{1.392721in}{1.459555in}}%
\pgfpathlineto{\pgfqpoint{1.393830in}{1.460002in}}%
\pgfpathlineto{\pgfqpoint{1.396565in}{1.461110in}}%
\pgfpathlineto{\pgfqpoint{1.397649in}{1.461511in}}%
\pgfpathlineto{\pgfqpoint{1.400242in}{1.462619in}}%
\pgfpathlineto{\pgfqpoint{1.401328in}{1.463075in}}%
\pgfpathlineto{\pgfqpoint{1.401335in}{1.463075in}}%
\pgfpathlineto{\pgfqpoint{1.403906in}{1.464184in}}%
\pgfpathlineto{\pgfqpoint{1.404973in}{1.464742in}}%
\pgfpathlineto{\pgfqpoint{1.408153in}{1.465850in}}%
\pgfpathlineto{\pgfqpoint{1.409213in}{1.466223in}}%
\pgfpathlineto{\pgfqpoint{1.411866in}{1.467331in}}%
\pgfpathlineto{\pgfqpoint{1.412935in}{1.467806in}}%
\pgfpathlineto{\pgfqpoint{1.415982in}{1.468914in}}%
\pgfpathlineto{\pgfqpoint{1.417075in}{1.469277in}}%
\pgfpathlineto{\pgfqpoint{1.419561in}{1.470386in}}%
\pgfpathlineto{\pgfqpoint{1.420661in}{1.470786in}}%
\pgfpathlineto{\pgfqpoint{1.423496in}{1.471894in}}%
\pgfpathlineto{\pgfqpoint{1.424591in}{1.472360in}}%
\pgfpathlineto{\pgfqpoint{1.427256in}{1.473468in}}%
\pgfpathlineto{\pgfqpoint{1.428339in}{1.473906in}}%
\pgfpathlineto{\pgfqpoint{1.431693in}{1.475005in}}%
\pgfpathlineto{\pgfqpoint{1.432779in}{1.475480in}}%
\pgfpathlineto{\pgfqpoint{1.432800in}{1.475480in}}%
\pgfpathlineto{\pgfqpoint{1.435520in}{1.476569in}}%
\pgfpathlineto{\pgfqpoint{1.436588in}{1.476877in}}%
\pgfpathlineto{\pgfqpoint{1.436627in}{1.476877in}}%
\pgfpathlineto{\pgfqpoint{1.439716in}{1.477985in}}%
\pgfpathlineto{\pgfqpoint{1.440802in}{1.478432in}}%
\pgfpathlineto{\pgfqpoint{1.443551in}{1.479540in}}%
\pgfpathlineto{\pgfqpoint{1.444634in}{1.479903in}}%
\pgfpathlineto{\pgfqpoint{1.447730in}{1.481011in}}%
\pgfpathlineto{\pgfqpoint{1.448764in}{1.481365in}}%
\pgfpathlineto{\pgfqpoint{1.451513in}{1.482473in}}%
\pgfpathlineto{\pgfqpoint{1.452618in}{1.482874in}}%
\pgfpathlineto{\pgfqpoint{1.455298in}{1.483982in}}%
\pgfpathlineto{\pgfqpoint{1.456398in}{1.484364in}}%
\pgfpathlineto{\pgfqpoint{1.458392in}{1.485463in}}%
\pgfpathlineto{\pgfqpoint{1.459482in}{1.486012in}}%
\pgfpathlineto{\pgfqpoint{1.462278in}{1.487120in}}%
\pgfpathlineto{\pgfqpoint{1.463354in}{1.487484in}}%
\pgfpathlineto{\pgfqpoint{1.466959in}{1.488592in}}%
\pgfpathlineto{\pgfqpoint{1.468054in}{1.489039in}}%
\pgfpathlineto{\pgfqpoint{1.470866in}{1.490147in}}%
\pgfpathlineto{\pgfqpoint{1.471945in}{1.490557in}}%
\pgfpathlineto{\pgfqpoint{1.475247in}{1.491665in}}%
\pgfpathlineto{\pgfqpoint{1.476354in}{1.492149in}}%
\pgfpathlineto{\pgfqpoint{1.479495in}{1.493257in}}%
\pgfpathlineto{\pgfqpoint{1.480571in}{1.493658in}}%
\pgfpathlineto{\pgfqpoint{1.483878in}{1.494766in}}%
\pgfpathlineto{\pgfqpoint{1.484959in}{1.495232in}}%
\pgfpathlineto{\pgfqpoint{1.488627in}{1.496340in}}%
\pgfpathlineto{\pgfqpoint{1.489727in}{1.496833in}}%
\pgfpathlineto{\pgfqpoint{1.493271in}{1.497942in}}%
\pgfpathlineto{\pgfqpoint{1.494345in}{1.498361in}}%
\pgfpathlineto{\pgfqpoint{1.497603in}{1.499469in}}%
\pgfpathlineto{\pgfqpoint{1.498654in}{1.499832in}}%
\pgfpathlineto{\pgfqpoint{1.502465in}{1.500940in}}%
\pgfpathlineto{\pgfqpoint{1.503553in}{1.501331in}}%
\pgfpathlineto{\pgfqpoint{1.506623in}{1.502440in}}%
\pgfpathlineto{\pgfqpoint{1.507695in}{1.502812in}}%
\pgfpathlineto{\pgfqpoint{1.510779in}{1.503920in}}%
\pgfpathlineto{\pgfqpoint{1.511883in}{1.504311in}}%
\pgfpathlineto{\pgfqpoint{1.515828in}{1.505420in}}%
\pgfpathlineto{\pgfqpoint{1.516930in}{1.505801in}}%
\pgfpathlineto{\pgfqpoint{1.520019in}{1.506910in}}%
\pgfpathlineto{\pgfqpoint{1.521129in}{1.507263in}}%
\pgfpathlineto{\pgfqpoint{1.524281in}{1.508372in}}%
\pgfpathlineto{\pgfqpoint{1.525322in}{1.508809in}}%
\pgfpathlineto{\pgfqpoint{1.528582in}{1.509917in}}%
\pgfpathlineto{\pgfqpoint{1.529658in}{1.510374in}}%
\pgfpathlineto{\pgfqpoint{1.533561in}{1.511482in}}%
\pgfpathlineto{\pgfqpoint{1.534659in}{1.511780in}}%
\pgfpathlineto{\pgfqpoint{1.538847in}{1.512888in}}%
\pgfpathlineto{\pgfqpoint{1.539945in}{1.513354in}}%
\pgfpathlineto{\pgfqpoint{1.543383in}{1.514462in}}%
\pgfpathlineto{\pgfqpoint{1.544467in}{1.514937in}}%
\pgfpathlineto{\pgfqpoint{1.547933in}{1.516045in}}%
\pgfpathlineto{\pgfqpoint{1.549002in}{1.516399in}}%
\pgfpathlineto{\pgfqpoint{1.552931in}{1.517507in}}%
\pgfpathlineto{\pgfqpoint{1.554024in}{1.517842in}}%
\pgfpathlineto{\pgfqpoint{1.558370in}{1.518951in}}%
\pgfpathlineto{\pgfqpoint{1.559434in}{1.519407in}}%
\pgfpathlineto{\pgfqpoint{1.559479in}{1.519407in}}%
\pgfpathlineto{\pgfqpoint{1.563072in}{1.520506in}}%
\pgfpathlineto{\pgfqpoint{1.564172in}{1.520822in}}%
\pgfpathlineto{\pgfqpoint{1.568147in}{1.521931in}}%
\pgfpathlineto{\pgfqpoint{1.569238in}{1.522303in}}%
\pgfpathlineto{\pgfqpoint{1.572709in}{1.523411in}}%
\pgfpathlineto{\pgfqpoint{1.573816in}{1.523924in}}%
\pgfpathlineto{\pgfqpoint{1.578448in}{1.525022in}}%
\pgfpathlineto{\pgfqpoint{1.579543in}{1.525311in}}%
\pgfpathlineto{\pgfqpoint{1.583575in}{1.526419in}}%
\pgfpathlineto{\pgfqpoint{1.584658in}{1.526792in}}%
\pgfpathlineto{\pgfqpoint{1.588389in}{1.527900in}}%
\pgfpathlineto{\pgfqpoint{1.589492in}{1.528161in}}%
\pgfpathlineto{\pgfqpoint{1.593514in}{1.529269in}}%
\pgfpathlineto{\pgfqpoint{1.594621in}{1.529641in}}%
\pgfpathlineto{\pgfqpoint{1.598481in}{1.530750in}}%
\pgfpathlineto{\pgfqpoint{1.599548in}{1.531085in}}%
\pgfpathlineto{\pgfqpoint{1.604169in}{1.532193in}}%
\pgfpathlineto{\pgfqpoint{1.605224in}{1.532435in}}%
\pgfpathlineto{\pgfqpoint{1.605259in}{1.532435in}}%
\pgfpathlineto{\pgfqpoint{1.609980in}{1.533543in}}%
\pgfpathlineto{\pgfqpoint{1.611068in}{1.533935in}}%
\pgfpathlineto{\pgfqpoint{1.615328in}{1.535043in}}%
\pgfpathlineto{\pgfqpoint{1.616413in}{1.535387in}}%
\pgfpathlineto{\pgfqpoint{1.621081in}{1.536496in}}%
\pgfpathlineto{\pgfqpoint{1.622173in}{1.536784in}}%
\pgfpathlineto{\pgfqpoint{1.622190in}{1.536784in}}%
\pgfpathlineto{\pgfqpoint{1.626388in}{1.537892in}}%
\pgfpathlineto{\pgfqpoint{1.627464in}{1.538172in}}%
\pgfpathlineto{\pgfqpoint{1.627488in}{1.538172in}}%
\pgfpathlineto{\pgfqpoint{1.631663in}{1.539280in}}%
\pgfpathlineto{\pgfqpoint{1.632758in}{1.539597in}}%
\pgfpathlineto{\pgfqpoint{1.637064in}{1.540705in}}%
\pgfpathlineto{\pgfqpoint{1.638168in}{1.540975in}}%
\pgfpathlineto{\pgfqpoint{1.642327in}{1.542083in}}%
\pgfpathlineto{\pgfqpoint{1.643368in}{1.542400in}}%
\pgfpathlineto{\pgfqpoint{1.643403in}{1.542400in}}%
\pgfpathlineto{\pgfqpoint{1.647634in}{1.543508in}}%
\pgfpathlineto{\pgfqpoint{1.648725in}{1.543871in}}%
\pgfpathlineto{\pgfqpoint{1.648741in}{1.543871in}}%
\pgfpathlineto{\pgfqpoint{1.652768in}{1.544979in}}%
\pgfpathlineto{\pgfqpoint{1.653823in}{1.545231in}}%
\pgfpathlineto{\pgfqpoint{1.653847in}{1.545231in}}%
\pgfpathlineto{\pgfqpoint{1.658171in}{1.546339in}}%
\pgfpathlineto{\pgfqpoint{1.659180in}{1.546656in}}%
\pgfpathlineto{\pgfqpoint{1.663767in}{1.547764in}}%
\pgfpathlineto{\pgfqpoint{1.664870in}{1.548099in}}%
\pgfpathlineto{\pgfqpoint{1.669190in}{1.549207in}}%
\pgfpathlineto{\pgfqpoint{1.670212in}{1.549440in}}%
\pgfpathlineto{\pgfqpoint{1.670243in}{1.549440in}}%
\pgfpathlineto{\pgfqpoint{1.675738in}{1.550548in}}%
\pgfpathlineto{\pgfqpoint{1.676821in}{1.550828in}}%
\pgfpathlineto{\pgfqpoint{1.681601in}{1.551936in}}%
\pgfpathlineto{\pgfqpoint{1.682544in}{1.552159in}}%
\pgfpathlineto{\pgfqpoint{1.687049in}{1.553267in}}%
\pgfpathlineto{\pgfqpoint{1.688140in}{1.553565in}}%
\pgfpathlineto{\pgfqpoint{1.692793in}{1.554674in}}%
\pgfpathlineto{\pgfqpoint{1.693897in}{1.554897in}}%
\pgfpathlineto{\pgfqpoint{1.698403in}{1.556005in}}%
\pgfpathlineto{\pgfqpoint{1.699503in}{1.556285in}}%
\pgfpathlineto{\pgfqpoint{1.704275in}{1.557393in}}%
\pgfpathlineto{\pgfqpoint{1.705342in}{1.557533in}}%
\pgfpathlineto{\pgfqpoint{1.711018in}{1.558641in}}%
\pgfpathlineto{\pgfqpoint{1.712064in}{1.558948in}}%
\pgfpathlineto{\pgfqpoint{1.712118in}{1.558948in}}%
\pgfpathlineto{\pgfqpoint{1.717686in}{1.560056in}}%
\pgfpathlineto{\pgfqpoint{1.718786in}{1.560392in}}%
\pgfpathlineto{\pgfqpoint{1.724466in}{1.561500in}}%
\pgfpathlineto{\pgfqpoint{1.725571in}{1.561835in}}%
\pgfpathlineto{\pgfqpoint{1.731169in}{1.562943in}}%
\pgfpathlineto{\pgfqpoint{1.732079in}{1.563204in}}%
\pgfpathlineto{\pgfqpoint{1.732116in}{1.563204in}}%
\pgfpathlineto{\pgfqpoint{1.737951in}{1.564312in}}%
\pgfpathlineto{\pgfqpoint{1.739009in}{1.564489in}}%
\pgfpathlineto{\pgfqpoint{1.739016in}{1.564489in}}%
\pgfpathlineto{\pgfqpoint{1.744312in}{1.565597in}}%
\pgfpathlineto{\pgfqpoint{1.745414in}{1.565830in}}%
\pgfpathlineto{\pgfqpoint{1.750663in}{1.566938in}}%
\pgfpathlineto{\pgfqpoint{1.751683in}{1.567218in}}%
\pgfpathlineto{\pgfqpoint{1.751718in}{1.567218in}}%
\pgfpathlineto{\pgfqpoint{1.758496in}{1.568326in}}%
\pgfpathlineto{\pgfqpoint{1.759538in}{1.568624in}}%
\pgfpathlineto{\pgfqpoint{1.765668in}{1.569732in}}%
\pgfpathlineto{\pgfqpoint{1.766726in}{1.569974in}}%
\pgfpathlineto{\pgfqpoint{1.772953in}{1.571082in}}%
\pgfpathlineto{\pgfqpoint{1.774060in}{1.571315in}}%
\pgfpathlineto{\pgfqpoint{1.779531in}{1.572423in}}%
\pgfpathlineto{\pgfqpoint{1.780495in}{1.572638in}}%
\pgfpathlineto{\pgfqpoint{1.780638in}{1.572638in}}%
\pgfpathlineto{\pgfqpoint{1.787144in}{1.573746in}}%
\pgfpathlineto{\pgfqpoint{1.788007in}{1.573857in}}%
\pgfpathlineto{\pgfqpoint{1.788035in}{1.573857in}}%
\pgfpathlineto{\pgfqpoint{1.794342in}{1.574966in}}%
\pgfpathlineto{\pgfqpoint{1.795449in}{1.575096in}}%
\pgfpathlineto{\pgfqpoint{1.801071in}{1.576204in}}%
\pgfpathlineto{\pgfqpoint{1.802163in}{1.576428in}}%
\pgfpathlineto{\pgfqpoint{1.808852in}{1.577536in}}%
\pgfpathlineto{\pgfqpoint{1.809849in}{1.577759in}}%
\pgfpathlineto{\pgfqpoint{1.809922in}{1.577759in}}%
\pgfpathlineto{\pgfqpoint{1.816887in}{1.578868in}}%
\pgfpathlineto{\pgfqpoint{1.817872in}{1.579082in}}%
\pgfpathlineto{\pgfqpoint{1.825445in}{1.580190in}}%
\pgfpathlineto{\pgfqpoint{1.826484in}{1.580441in}}%
\pgfpathlineto{\pgfqpoint{1.833192in}{1.581550in}}%
\pgfpathlineto{\pgfqpoint{1.834278in}{1.581773in}}%
\pgfpathlineto{\pgfqpoint{1.841630in}{1.582881in}}%
\pgfpathlineto{\pgfqpoint{1.842613in}{1.583012in}}%
\pgfpathlineto{\pgfqpoint{1.849963in}{1.584120in}}%
\pgfpathlineto{\pgfqpoint{1.851039in}{1.584288in}}%
\pgfpathlineto{\pgfqpoint{1.858268in}{1.585396in}}%
\pgfpathlineto{\pgfqpoint{1.859335in}{1.585563in}}%
\pgfpathlineto{\pgfqpoint{1.866934in}{1.586672in}}%
\pgfpathlineto{\pgfqpoint{1.867958in}{1.586774in}}%
\pgfpathlineto{\pgfqpoint{1.867987in}{1.586774in}}%
\pgfpathlineto{\pgfqpoint{1.876423in}{1.587882in}}%
\pgfpathlineto{\pgfqpoint{1.877523in}{1.588050in}}%
\pgfpathlineto{\pgfqpoint{1.884816in}{1.589158in}}%
\pgfpathlineto{\pgfqpoint{1.885919in}{1.589363in}}%
\pgfpathlineto{\pgfqpoint{1.895839in}{1.590471in}}%
\pgfpathlineto{\pgfqpoint{1.896592in}{1.590555in}}%
\pgfpathlineto{\pgfqpoint{1.896780in}{1.590555in}}%
\pgfpathlineto{\pgfqpoint{1.906890in}{1.591663in}}%
\pgfpathlineto{\pgfqpoint{1.907901in}{1.591803in}}%
\pgfpathlineto{\pgfqpoint{1.917719in}{1.592911in}}%
\pgfpathlineto{\pgfqpoint{1.918765in}{1.593060in}}%
\pgfpathlineto{\pgfqpoint{1.929058in}{1.594168in}}%
\pgfpathlineto{\pgfqpoint{1.930130in}{1.594280in}}%
\pgfpathlineto{\pgfqpoint{1.930158in}{1.594280in}}%
\pgfpathlineto{\pgfqpoint{1.940895in}{1.595388in}}%
\pgfpathlineto{\pgfqpoint{1.941917in}{1.595509in}}%
\pgfpathlineto{\pgfqpoint{1.941974in}{1.595509in}}%
\pgfpathlineto{\pgfqpoint{1.953541in}{1.596617in}}%
\pgfpathlineto{\pgfqpoint{1.954561in}{1.596720in}}%
\pgfpathlineto{\pgfqpoint{1.954627in}{1.596720in}}%
\pgfpathlineto{\pgfqpoint{1.968248in}{1.597828in}}%
\pgfpathlineto{\pgfqpoint{1.969343in}{1.597930in}}%
\pgfpathlineto{\pgfqpoint{1.983413in}{1.599039in}}%
\pgfpathlineto{\pgfqpoint{1.984506in}{1.599150in}}%
\pgfpathlineto{\pgfqpoint{1.999921in}{1.600259in}}%
\pgfpathlineto{\pgfqpoint{2.001005in}{1.600361in}}%
\pgfpathlineto{\pgfqpoint{2.019033in}{1.601423in}}%
\pgfpathlineto{\pgfqpoint{2.033126in}{1.601944in}}%
\pgfpathlineto{\pgfqpoint{2.033126in}{1.601944in}}%
\pgfusepath{stroke}%
\end{pgfscope}%
\begin{pgfscope}%
\pgfsetrectcap%
\pgfsetmiterjoin%
\pgfsetlinewidth{0.803000pt}%
\definecolor{currentstroke}{rgb}{0.000000,0.000000,0.000000}%
\pgfsetstrokecolor{currentstroke}%
\pgfsetdash{}{0pt}%
\pgfpathmoveto{\pgfqpoint{0.553581in}{0.499444in}}%
\pgfpathlineto{\pgfqpoint{0.553581in}{1.654444in}}%
\pgfusepath{stroke}%
\end{pgfscope}%
\begin{pgfscope}%
\pgfsetrectcap%
\pgfsetmiterjoin%
\pgfsetlinewidth{0.803000pt}%
\definecolor{currentstroke}{rgb}{0.000000,0.000000,0.000000}%
\pgfsetstrokecolor{currentstroke}%
\pgfsetdash{}{0pt}%
\pgfpathmoveto{\pgfqpoint{2.103581in}{0.499444in}}%
\pgfpathlineto{\pgfqpoint{2.103581in}{1.654444in}}%
\pgfusepath{stroke}%
\end{pgfscope}%
\begin{pgfscope}%
\pgfsetrectcap%
\pgfsetmiterjoin%
\pgfsetlinewidth{0.803000pt}%
\definecolor{currentstroke}{rgb}{0.000000,0.000000,0.000000}%
\pgfsetstrokecolor{currentstroke}%
\pgfsetdash{}{0pt}%
\pgfpathmoveto{\pgfqpoint{0.553581in}{0.499444in}}%
\pgfpathlineto{\pgfqpoint{2.103581in}{0.499444in}}%
\pgfusepath{stroke}%
\end{pgfscope}%
\begin{pgfscope}%
\pgfsetrectcap%
\pgfsetmiterjoin%
\pgfsetlinewidth{0.803000pt}%
\definecolor{currentstroke}{rgb}{0.000000,0.000000,0.000000}%
\pgfsetstrokecolor{currentstroke}%
\pgfsetdash{}{0pt}%
\pgfpathmoveto{\pgfqpoint{0.553581in}{1.654444in}}%
\pgfpathlineto{\pgfqpoint{2.103581in}{1.654444in}}%
\pgfusepath{stroke}%
\end{pgfscope}%
\begin{pgfscope}%
\pgfsetbuttcap%
\pgfsetmiterjoin%
\definecolor{currentfill}{rgb}{1.000000,1.000000,1.000000}%
\pgfsetfillcolor{currentfill}%
\pgfsetfillopacity{0.800000}%
\pgfsetlinewidth{1.003750pt}%
\definecolor{currentstroke}{rgb}{0.800000,0.800000,0.800000}%
\pgfsetstrokecolor{currentstroke}%
\pgfsetstrokeopacity{0.800000}%
\pgfsetdash{}{0pt}%
\pgfpathmoveto{\pgfqpoint{0.832747in}{0.568889in}}%
\pgfpathlineto{\pgfqpoint{2.006358in}{0.568889in}}%
\pgfpathquadraticcurveto{\pgfqpoint{2.034136in}{0.568889in}}{\pgfqpoint{2.034136in}{0.596666in}}%
\pgfpathlineto{\pgfqpoint{2.034136in}{0.776388in}}%
\pgfpathquadraticcurveto{\pgfqpoint{2.034136in}{0.804166in}}{\pgfqpoint{2.006358in}{0.804166in}}%
\pgfpathlineto{\pgfqpoint{0.832747in}{0.804166in}}%
\pgfpathquadraticcurveto{\pgfqpoint{0.804970in}{0.804166in}}{\pgfqpoint{0.804970in}{0.776388in}}%
\pgfpathlineto{\pgfqpoint{0.804970in}{0.596666in}}%
\pgfpathquadraticcurveto{\pgfqpoint{0.804970in}{0.568889in}}{\pgfqpoint{0.832747in}{0.568889in}}%
\pgfpathlineto{\pgfqpoint{0.832747in}{0.568889in}}%
\pgfpathclose%
\pgfusepath{stroke,fill}%
\end{pgfscope}%
\begin{pgfscope}%
\pgfsetrectcap%
\pgfsetroundjoin%
\pgfsetlinewidth{1.505625pt}%
\definecolor{currentstroke}{rgb}{0.000000,0.000000,0.000000}%
\pgfsetstrokecolor{currentstroke}%
\pgfsetdash{}{0pt}%
\pgfpathmoveto{\pgfqpoint{0.860525in}{0.700000in}}%
\pgfpathlineto{\pgfqpoint{0.999414in}{0.700000in}}%
\pgfpathlineto{\pgfqpoint{1.138303in}{0.700000in}}%
\pgfusepath{stroke}%
\end{pgfscope}%
\begin{pgfscope}%
\definecolor{textcolor}{rgb}{0.000000,0.000000,0.000000}%
\pgfsetstrokecolor{textcolor}%
\pgfsetfillcolor{textcolor}%
\pgftext[x=1.249414in,y=0.651388in,left,base]{\color{textcolor}\rmfamily\fontsize{10.000000}{12.000000}\selectfont AUC=0.752}%
\end{pgfscope}%
\end{pgfpicture}%
\makeatother%
\endgroup%

\end{tabular}

\



The model below is as effective at separating the two classes ($\text{ROC}=0.778$), but the distribution is skewed to the left.  Its results were nearly continuous, with the 214,070 samples returning 210,157 unique values of $p$, so we can fine tune the decision threshold.  

\

%
\verb|KBFC_5_Fold_alpha_0_5_gamma_0_0_Hard_Test|

\

\noindent\begin{tabular}{@{\hspace{-6pt}}p{4.3in} @{\hspace{-6pt}}p{2.0in}}
	\vskip 0pt
	\hfil Raw Model Output
	
	\input{../Keras/Images/KBFC_5_Fold_alpha_0_5_gamma_0_0_Hard_Test_Pred_Wide.pgf}	
&
	\vskip 0pt
	\hfil ROC Curve
	
	\input{../Keras/Images/KBFC_5_Fold_alpha_0_5_gamma_0_0_Hard_Test_ROC.pgf}
\end{tabular}

\

The second method we will use to modify the model outputs' distribution is to employ class weights in the model building process.  Here we employed class weights proportional to the class imbalance.  The motivation behind class weights is to better separate the positive and negative classes, but note that the area under the ROC curve does not change.  We have not investigated whether the model using class weights does a better job at separating the classes in some intervals, but overall the effect is negligible.  One effect using class weights did have here is shifting the distribution.  


\

\verb|KBFC_5_Fold_alpha_balanced_gamma_0_0_Hard|



\noindent\begin{tabular}{@{\hspace{-6pt}}p{4.3in} @{\hspace{-6pt}}p{2.0in}}
	\vskip 0pt
	\hfil Raw Model Output
	
	\input{../Keras/Images/KBFC_5_Fold_alpha_balanced_gamma_0_0_Hard_Test_Pred_Wide.pgf}	
&
	\vskip 0pt
	\hfil ROC Curve
	
	\input{../Keras/Images/KBFC_5_Fold_alpha_balanced_gamma_0_0_Hard_Test_ROC.pgf}
\end{tabular}

\



%
\verb|Bagging_Hard_Tomek_0_v1_Test|

\

This model returned 217 different values, but most of them were rare.  Taking out the 5\% of the data set with the least frequent values, 95\% of the samples had only 10 values of $p$.  It may be a useful model, but we will not be able to fine tune the decision threshold.  

\noindent\begin{tabular}{@{\hspace{-6pt}}p{4.3in} @{\hspace{-6pt}}p{2.0in}}
	\vskip 0pt
	\hfil Raw Model Output
	
	\input{../Keras/Images/BalBag_5_Fold_Hard_Test_Pred_Wide.pgf}	
&
	\vskip 0pt
	\hfil ROC Curve
	
	\input{../Keras/Images/BalBag_5_Fold_Hard_Test_ROC.pgf}
\end{tabular}

\

\

Other stuff

\


%%%
\begin{comment}
If we set the discrimination threshold about $0.7$, the model would classify almost all of the samples, both positive and negative class, correctly, with about the same number of false positives (sending an ambulance when one is not needed, negative class samples with $p > 0.7$) and false negatives (not sending an ambulance when one is needed, positive class samples with $p < 0.7$).  If we (as a society) were willing to tolerate more false positives, we could set the discrimination threshold lower, and if budgets were tighter we could increase the $p$ threshold.  

The table below gives the number of true negatives (TN), false positives (FP), false negatives (FN), and true positives (TP) for the 499,496 samples in the test set, along with the precision and recall values, for different discrimination thresholds $p$.  The precision is the proportion of ambulances we sent that were needed, and the recall is the proportion of ambulances needed that we sent.  

$$\text{Precision} = \frac{TP}{FP+TP}, \qquad \text{Recall} = \frac{TP}{FN + TP}$$

\begin{center}
\begin{tabular}{rrrrrrrrrrrrrr}
\toprule
$p$ &   TN &       FP &      FN &      TP &  Precision &   Recall \\
\midrule
0.50 &  346,776 &   73,794 &       1 &  78,925 &  0.52 &  1.00       \\
0.60 &  390,335 &   30,235 &      89 &  78,837 &  0.72 &  1.00  \\
0.70 & 411,040 &    9,530 &   2,838 &  76,088 &  0.89 &  0.96 \\
0.80 & 418,739 &    1,831 &  19,174 &  59,752 &  0.97 &  0.76  \\
0.90 & 420,496 &       74 &  53,736 &  25,190 &  1.00 &  0.32 & \\
\bottomrule
\end{tabular}
\end{center}

\end{comment}
%%%



