%%%%% Introduction

\subsection{Introduction}

When a cell phone detects the deceleration profile of an automobile crash, the cell service provider can first call the phone, and if there is no response, automatically notify local emergency services that they suspect a crash at a certain location.  Before automated notifications, almost all crash notifications to emergency services came from an eyewitness who would report whether an ambulance was needed.  With an automated crash notification, the emergency dispatcher will send a police officer to investigate, but should the dispatcher also immediately send an ambulance?  

The political body setting the policy has three options.  

\begin{description}
	\item [Always] Erring on the side of caution, dispatch an ambulance  to every automated crash notification.  Using our assumption that the CRSS data is representative of such crashes, about 85\% will not need the ambulance, so this option has a False Positive Rate (FPR) of 85\%.
	\item [Never] Immediately dispatch police to investigate the crash, but wait for an eyewitness (perhaps the police) to say an ambulance is needed before sending one.  About 15\% of the crashes will need an ambulance, and this policy will delay sending the ambulance, so this option has a False Negative Rate of 15\%.
	\item [Sometimes]  With the information available to emergency dispatchers at the time of the notification, use a machine learning model to calculate the probability $p$ that the person whose phone sent an automated crash notification (``crash person'') needs an ambulance.  Considering the marginal and total costs and benefits, choose a decision threshold $\theta$ above which ($p > \theta$) we automatically dispatch an ambulance.  For each crash notification, if $p > \theta$, then dispatch an ambulance immediately. if $p<\theta$, wait for eyewitness confirmation before sending an ambulance.  This option does not replace the existing system of eyewitnesses calling in a crash, so ambulances will get to crash scenes at least as soon as before; the goal is to get some ambulances to the crash scenes where they are needed sooner. 
\end{description}

This paper will explore the ``Sometimes'' option.  

The machine learning model will not be perfect.  It will return some false positives (dispatching an ambulance that is not needed and would not have been sent using only eyewitness notifications) and false negatives (not immediately dispatching an ambulance that is needed).  False positives cost resources; false negatives cost lives.  

\subsubsection{Confusion Matrix}

The confusion matrix gives the totals based on the model and choice of decision treshold.

\begin{center}
\begin{tabular}{cccc}
	&& \multicolumn{2}{c}{Predicted} \cr
	&& PP & PN \cr\cline{3-4}
	\multirow{2}{*}{Actual} & P \vrule width 0pt height 14pt depth 6pt& \multicolumn{1}{|c|}{TP} &  \multicolumn{1}{c|}{FN} \cr \cline{3-4}
	& N \vrule width 0pt height 14pt depth 6pt& \multicolumn{1}{|c|}{FP} &  \multicolumn{1}{c|}{TN} \cr \cline{3-4}
\end{tabular}
\end{center}

\hfil\begin{tabular}{c>{\hangindent=2em}p{5in}}
	P & Number of crash persons who need an ambulance; also the number to whom we will {\it immediately} or {\it eventually} send an ambulance \cr
	N & Number of crash persons who do not need an ambulance\cr \cr
	PP & Number of crash persons to whom we immediately dispatch an ambulance\cr
	PN & Number of crash persons to whom we do not {\it immediately} dispatch an ambulance \cr \cr
	TP & Number of crash persons to whom we immediately dispatch a needed ambulance \cr
	FN & Number of crash persons who need an ambulance, but to whom we do not {\it immediately} send one, but we will send one later when an eyewitness calls\cr
	FP & Number of crash persons to whom we dispatch an ambulance that is not needed \cr
	TN & Number of crash persons who do not need an ambulance and to whom we do not immediately (or ever) send one \cr \cr
	$\displaystyle\frac{\text{TP}}{\text{FP} + \text{TP}} = \frac{ \text{TP}}{\text{PP}}$ & Precision, the proportion of immediately dispatched ambulances that are needed \\[1em]
	$\displaystyle\frac{\text{TP}}{\text{FN} + \text{TP}} = \frac{ \text{TP}}{\text{P}}$ & Accuracy, the proportion of crash persons needing an ambulance to whom we immediately dispatch one \\[1em]
$\displaystyle\frac{\text{FP}}{\text{FN} + \text{TP}} = \frac{ \text{FP}}{\text{P}}$ & Proportion of increase in number of ambulances sent because of immediate dispatch.  The P number of ambulances would have been dispatched anyway, but now the FP number of ambulances are also being sent.  \\[1em]
\end{tabular}


%%%
\subsection{Decision Threshold}

Here's something new from my last email.  

How to determine the decision threshold is a political decision, not a technical one.  We will consider three ways politicians might answer that question and how to implement each in our models and decision thresholds.  

\begin{enumerate}
	\item Our local fleet of ambulances now goes to $n$ crashes per year.  In the short term, without buying more ambulances and hiring more teams, we can increase the number of ambulance runs to crashes by some percentage, or in the longer term we are willing to increase the number of ambulances going to crashes by a larger, but still fixed, percentage.  
	
	\
	
	We will use 5\% as our example of how to implement this policy.  The increase does not include the true positives (TP), because those ambulances would go anyway; the increase is the allowable number of false positives (FP).  Set the decision threshold where the number of false positives is 5\% of the positive class ($\text{P} = \text{FN} + \text{TP}$).   

$$\frac{\text{FP}}{\text{P}} =  \frac{\text{FP}}{\text{FN} + \text{TP}} = 0.05  $$

Lots of ratios of TN, FP, FN, and TP have names, but I haven't found a name for FP/P or where other people have used it.  

\

	
	\item We are willing to send ambulances based on automated crash reports, but only up to the point where a certain proportion of the ambulances we immediately dispatch (FP+TP) are actually needed (TP), which is equivalent to saying that we are willing to immediately send a certain number of unneeded ambulances for each one we automatically send that is needed.  
	
	\
	
	This is what I was trying to get at with FP and TP, but I realized it's equivalent to Precision, which the readers will understand.  
	
	\
	
	Choose the decision threshold where the precision is the specified level.  We will use $1/3$ for our example, being willing to send two FP for each TP.  
	
	\
	
	\item By looking at the slope of the ROC curve we can (roughly) estimate the probability that a particular crash needs an ambulance.  Some of them almost definitely need an ambulance, and we should dispatch those immediately, but we will choose a minimum probability to which we will immediately dispatch an ambulance.  
	
	\
	
	I'm going to use 50\% as an example of the minimum probability.  The model assigns to each sample in the test set at value $p \in [0,1]$.  I had read that $p$ was the probability that the sample was in the positive class, but I don't think that's exactly true.  What is true is that $p$ generally increases with probability.  
	
	\
	
	In each sufficiently large* range of $p$ (like $p \in [0.60, 0.61)$) there are some number of elements of the negative and positive class.  For a given range of $p$, call the number of elements of the negative class ``Neg'' and the number of elements of the positive class ``Pos.''  For the samples in that range of $p$, the probability that they are in the positive class is 
	$$ \frac{\text{Pos}}{ \text{Pos} + \text{Neg}}$$
	
This expression is proportional to the slope of the ROC curve at that value of $p$. 
	
	\
	
	I want to call this ``marginal precision,'' but I haven't seen that term used that way in the ML literature.  I think ``marginal precision'' means something else in statistics.   See that 
	$$\text{Pos} = \frac{\Delta \text{TP}}{\Delta p}, \quad 
	 \text{Neg} = \frac{\Delta \text{FP}}{\Delta p}, \text{\quad and \quad} 
	 \text{Pos} + \text{Neg} = \frac{\Delta (\text{TP} + \text{FP})}{
\Delta p}, \text{\quad so \quad}
	\frac{\text{Pos}}{ \text{Pos} + \text{Neg}} = \frac{\Delta \text{TP}}{\Delta(\text{TP} + \text{FP})}
	$$

\end{enumerate}
	
	*We have two challenges with choosing $\Delta p$, the size of our range of $p$, which we would like to be really small.  One is that some of our ML algorithms return (almost all) values of $p$ rounded to two decimal places, so we can't get more precision than that.  One of my algorithms (Balanced Bagging) gives $p$ for each sample to only one decimal place; thus, ``sufficiently large'' depends on the algorithm.  
	
The other problem is that if you zoom in too close, the number of samples in each $\Delta p$ isn't large enough to compensate for the randomness, and you see that your ``curve'' is actually jagged.  Because I wanted more samples in my test set, I went from using a 70/30 train/test split to using 5-fold validation, where all of the samples are in a test set.  
	

\subsection{Simplifying Assumptions and Caveats}

\begin{itemize}
	\item We may use ``police'' to mean ``emergency services'' or ``emergency dispatcher.''  We use ``ambulance'' to signify both the vehicle and the associated team of emergency medical technicians.  
	\item The Google Pixel phone has had such a car crash detection feature since 2019.  Apple said in 2020 that it would introduce such an app, but did not release it until the iPhone 14 (2022).  It is also available on some Apple Watches.  
	\item We use the CRSS dataset as if it were representative of the kinds of crash persons whose phones would send an automated notification to emergency services.  Ideally, we would have a large ($>100,000$) set of detailed records of crashes that spawned an automated crash report, and Google and Apple may have such records, but to our knowledge no such data set is publicly available.  A paper using private data on the crashes that resulted in automated notifications would be very valuable to the discussion, but readers would not be able to replicate the results and adapt the methods.
	\item Not having seen actual data on automated notifications, we do not know what proportion of crashes result in an automated notification, nor do we know what proportion of automated notifications need an ambulance.  We are making the heroic assumption that the crashes in the CRSS data are a good proxy for crashes with automated notifications to emergency services.  
\end{itemize}






