%%%
\subsection{Model Evaluation:  Baselines for Comparison}

What do good results look like, what do bad results look like, how do we measure it, and when we compare two results, how much of the difference could be due to randomness?

In the supervised learning method we used here, for each of the $\approx 600,000$ samples (people) in the dataset, we know the answer (the {\it label} or {\it ground truth}) to the question, whether the person needed an ambulance, $y=0$ for ``no'' and $y=1$ for ``yes.''  We are trying use historical data to build a model to predict the label for new data (incoming automated crash notifications).

{\bf
Explain this better.  I confused the internal workings of the model building with the results.
}

We split the data 70/30 into a training set and a test set, making sure to keep the same proportion of positive and negative samples in both.  The binary classification models we used take the training data and training labels (\verb|X_train| and \verb|y_train|) and build a model, then apply the model to the test data (\verb|X_test|) and returns \verb|y_proba| that gives, for each sample, a continuous probability $p \in (0,1)$ that the sample belongs in the positive class.   If a sample has $p = 0.1$, the model is 90\% confident that this sample is in the negative class.   We then pick a threshold, usually but not necessarily $threshold = 0.5$, and make a binary prediction, that samples with $p > threshold$ need an ambulance, and those with $p < threshold$ do not.   

While building the model, the algorithm picks a starting point, measures how badly the model predicts the training data using the {\it loss function}, tweaks the model, measures again, and either keeps or rejects the candidate model based on the loss function.  The loss function used by the model is the sum not of how many binary predictions were incorrect, but how strongly incorrect the continuous predictions were.  If two negative samples ($y=0$) had $p=0.1$ and $p=0.4$, both correct classifications if $threshold=0.5$, then the $p=0.4$ sample would add much more to the loss value.  

A perfect model would not only predict each sample's label correctly, but would do it with perfect certainty.  In the real world, with interesting questions about real data, we will have false positives ($y=0$ and $p>threshold$) and false negatives ($y=1$ and $p<threshold$), but we hope those are few, and that the predictions are strongly correct, meaning the predictions are close to their labels.

When we get results for our models based on crash data, we need some frame of reference for what is ``good'' and ``bad,'' so we have created some sets of entirely artificial results using a gamma distribution for ideal results and a uniform distribution for awful results.  

The histogram below of the percent of samples with predictions $p$ in each range illustrates the best results we can hope for in the real world.  The positive class is small because the data is imbalanced. about 15\% of the dataset, as in our CRSS data.  There are some false positives and negatives, but the overwhelming majority of the predictions are correct, and most with strong confidence.  

The Receiver Operating Characteristic (ROC) is a parameterized curve following the probability threshold from $p=0$ to $p=1$, plotting the true positive rate (TPR) versus the false positive rate (FPR).  The Area Under the ROC curve (AUC) is often used to compare two models, with AUC of 1 indicating perfect prediction and AUC of 0.5 indicating no discernable pattern.  

We have added to the typical ROC curve labels for the medians of the probabilities in the negative and positive classes (0.338 and 0.655) and the default decision threshold $thr = 0.5$.

\begin{tabular}{p{0.5\textwidth} p{0.5\textwidth}}
  \vspace{0pt} \input{../Keras/Images/Ideal_Pred.pgf}
  &
  \vspace{0pt} %% Creator: Matplotlib, PGF backend
%%
%% To include the figure in your LaTeX document, write
%%   \input{<filename>.pgf}
%%
%% Make sure the required packages are loaded in your preamble
%%   \usepackage{pgf}
%%
%% Also ensure that all the required font packages are loaded; for instance,
%% the lmodern package is sometimes necessary when using math font.
%%   \usepackage{lmodern}
%%
%% Figures using additional raster images can only be included by \input if
%% they are in the same directory as the main LaTeX file. For loading figures
%% from other directories you can use the `import` package
%%   \usepackage{import}
%%
%% and then include the figures with
%%   \import{<path to file>}{<filename>.pgf}
%%
%% Matplotlib used the following preamble
%%   
%%   \usepackage{fontspec}
%%   \makeatletter\@ifpackageloaded{underscore}{}{\usepackage[strings]{underscore}}\makeatother
%%
\begingroup%
\makeatletter%
\begin{pgfpicture}%
\pgfpathrectangle{\pgfpointorigin}{\pgfqpoint{3.144311in}{2.646444in}}%
\pgfusepath{use as bounding box, clip}%
\begin{pgfscope}%
\pgfsetbuttcap%
\pgfsetmiterjoin%
\definecolor{currentfill}{rgb}{1.000000,1.000000,1.000000}%
\pgfsetfillcolor{currentfill}%
\pgfsetlinewidth{0.000000pt}%
\definecolor{currentstroke}{rgb}{1.000000,1.000000,1.000000}%
\pgfsetstrokecolor{currentstroke}%
\pgfsetdash{}{0pt}%
\pgfpathmoveto{\pgfqpoint{0.000000in}{0.000000in}}%
\pgfpathlineto{\pgfqpoint{3.144311in}{0.000000in}}%
\pgfpathlineto{\pgfqpoint{3.144311in}{2.646444in}}%
\pgfpathlineto{\pgfqpoint{0.000000in}{2.646444in}}%
\pgfpathlineto{\pgfqpoint{0.000000in}{0.000000in}}%
\pgfpathclose%
\pgfusepath{fill}%
\end{pgfscope}%
\begin{pgfscope}%
\pgfsetbuttcap%
\pgfsetmiterjoin%
\definecolor{currentfill}{rgb}{1.000000,1.000000,1.000000}%
\pgfsetfillcolor{currentfill}%
\pgfsetlinewidth{0.000000pt}%
\definecolor{currentstroke}{rgb}{0.000000,0.000000,0.000000}%
\pgfsetstrokecolor{currentstroke}%
\pgfsetstrokeopacity{0.000000}%
\pgfsetdash{}{0pt}%
\pgfpathmoveto{\pgfqpoint{0.553581in}{0.499444in}}%
\pgfpathlineto{\pgfqpoint{3.033581in}{0.499444in}}%
\pgfpathlineto{\pgfqpoint{3.033581in}{2.347444in}}%
\pgfpathlineto{\pgfqpoint{0.553581in}{2.347444in}}%
\pgfpathlineto{\pgfqpoint{0.553581in}{0.499444in}}%
\pgfpathclose%
\pgfusepath{fill}%
\end{pgfscope}%
\begin{pgfscope}%
\pgfsetbuttcap%
\pgfsetroundjoin%
\definecolor{currentfill}{rgb}{0.000000,0.000000,0.000000}%
\pgfsetfillcolor{currentfill}%
\pgfsetlinewidth{0.803000pt}%
\definecolor{currentstroke}{rgb}{0.000000,0.000000,0.000000}%
\pgfsetstrokecolor{currentstroke}%
\pgfsetdash{}{0pt}%
\pgfsys@defobject{currentmarker}{\pgfqpoint{0.000000in}{-0.048611in}}{\pgfqpoint{0.000000in}{0.000000in}}{%
\pgfpathmoveto{\pgfqpoint{0.000000in}{0.000000in}}%
\pgfpathlineto{\pgfqpoint{0.000000in}{-0.048611in}}%
\pgfusepath{stroke,fill}%
}%
\begin{pgfscope}%
\pgfsys@transformshift{0.666308in}{0.499444in}%
\pgfsys@useobject{currentmarker}{}%
\end{pgfscope}%
\end{pgfscope}%
\begin{pgfscope}%
\definecolor{textcolor}{rgb}{0.000000,0.000000,0.000000}%
\pgfsetstrokecolor{textcolor}%
\pgfsetfillcolor{textcolor}%
\pgftext[x=0.666308in,y=0.402222in,,top]{\color{textcolor}\rmfamily\fontsize{10.000000}{12.000000}\selectfont \(\displaystyle {0.00}\)}%
\end{pgfscope}%
\begin{pgfscope}%
\pgfsetbuttcap%
\pgfsetroundjoin%
\definecolor{currentfill}{rgb}{0.000000,0.000000,0.000000}%
\pgfsetfillcolor{currentfill}%
\pgfsetlinewidth{0.803000pt}%
\definecolor{currentstroke}{rgb}{0.000000,0.000000,0.000000}%
\pgfsetstrokecolor{currentstroke}%
\pgfsetdash{}{0pt}%
\pgfsys@defobject{currentmarker}{\pgfqpoint{0.000000in}{-0.048611in}}{\pgfqpoint{0.000000in}{0.000000in}}{%
\pgfpathmoveto{\pgfqpoint{0.000000in}{0.000000in}}%
\pgfpathlineto{\pgfqpoint{0.000000in}{-0.048611in}}%
\pgfusepath{stroke,fill}%
}%
\begin{pgfscope}%
\pgfsys@transformshift{1.229944in}{0.499444in}%
\pgfsys@useobject{currentmarker}{}%
\end{pgfscope}%
\end{pgfscope}%
\begin{pgfscope}%
\definecolor{textcolor}{rgb}{0.000000,0.000000,0.000000}%
\pgfsetstrokecolor{textcolor}%
\pgfsetfillcolor{textcolor}%
\pgftext[x=1.229944in,y=0.402222in,,top]{\color{textcolor}\rmfamily\fontsize{10.000000}{12.000000}\selectfont \(\displaystyle {0.25}\)}%
\end{pgfscope}%
\begin{pgfscope}%
\pgfsetbuttcap%
\pgfsetroundjoin%
\definecolor{currentfill}{rgb}{0.000000,0.000000,0.000000}%
\pgfsetfillcolor{currentfill}%
\pgfsetlinewidth{0.803000pt}%
\definecolor{currentstroke}{rgb}{0.000000,0.000000,0.000000}%
\pgfsetstrokecolor{currentstroke}%
\pgfsetdash{}{0pt}%
\pgfsys@defobject{currentmarker}{\pgfqpoint{0.000000in}{-0.048611in}}{\pgfqpoint{0.000000in}{0.000000in}}{%
\pgfpathmoveto{\pgfqpoint{0.000000in}{0.000000in}}%
\pgfpathlineto{\pgfqpoint{0.000000in}{-0.048611in}}%
\pgfusepath{stroke,fill}%
}%
\begin{pgfscope}%
\pgfsys@transformshift{1.793581in}{0.499444in}%
\pgfsys@useobject{currentmarker}{}%
\end{pgfscope}%
\end{pgfscope}%
\begin{pgfscope}%
\definecolor{textcolor}{rgb}{0.000000,0.000000,0.000000}%
\pgfsetstrokecolor{textcolor}%
\pgfsetfillcolor{textcolor}%
\pgftext[x=1.793581in,y=0.402222in,,top]{\color{textcolor}\rmfamily\fontsize{10.000000}{12.000000}\selectfont \(\displaystyle {0.50}\)}%
\end{pgfscope}%
\begin{pgfscope}%
\pgfsetbuttcap%
\pgfsetroundjoin%
\definecolor{currentfill}{rgb}{0.000000,0.000000,0.000000}%
\pgfsetfillcolor{currentfill}%
\pgfsetlinewidth{0.803000pt}%
\definecolor{currentstroke}{rgb}{0.000000,0.000000,0.000000}%
\pgfsetstrokecolor{currentstroke}%
\pgfsetdash{}{0pt}%
\pgfsys@defobject{currentmarker}{\pgfqpoint{0.000000in}{-0.048611in}}{\pgfqpoint{0.000000in}{0.000000in}}{%
\pgfpathmoveto{\pgfqpoint{0.000000in}{0.000000in}}%
\pgfpathlineto{\pgfqpoint{0.000000in}{-0.048611in}}%
\pgfusepath{stroke,fill}%
}%
\begin{pgfscope}%
\pgfsys@transformshift{2.357217in}{0.499444in}%
\pgfsys@useobject{currentmarker}{}%
\end{pgfscope}%
\end{pgfscope}%
\begin{pgfscope}%
\definecolor{textcolor}{rgb}{0.000000,0.000000,0.000000}%
\pgfsetstrokecolor{textcolor}%
\pgfsetfillcolor{textcolor}%
\pgftext[x=2.357217in,y=0.402222in,,top]{\color{textcolor}\rmfamily\fontsize{10.000000}{12.000000}\selectfont \(\displaystyle {0.75}\)}%
\end{pgfscope}%
\begin{pgfscope}%
\pgfsetbuttcap%
\pgfsetroundjoin%
\definecolor{currentfill}{rgb}{0.000000,0.000000,0.000000}%
\pgfsetfillcolor{currentfill}%
\pgfsetlinewidth{0.803000pt}%
\definecolor{currentstroke}{rgb}{0.000000,0.000000,0.000000}%
\pgfsetstrokecolor{currentstroke}%
\pgfsetdash{}{0pt}%
\pgfsys@defobject{currentmarker}{\pgfqpoint{0.000000in}{-0.048611in}}{\pgfqpoint{0.000000in}{0.000000in}}{%
\pgfpathmoveto{\pgfqpoint{0.000000in}{0.000000in}}%
\pgfpathlineto{\pgfqpoint{0.000000in}{-0.048611in}}%
\pgfusepath{stroke,fill}%
}%
\begin{pgfscope}%
\pgfsys@transformshift{2.920853in}{0.499444in}%
\pgfsys@useobject{currentmarker}{}%
\end{pgfscope}%
\end{pgfscope}%
\begin{pgfscope}%
\definecolor{textcolor}{rgb}{0.000000,0.000000,0.000000}%
\pgfsetstrokecolor{textcolor}%
\pgfsetfillcolor{textcolor}%
\pgftext[x=2.920853in,y=0.402222in,,top]{\color{textcolor}\rmfamily\fontsize{10.000000}{12.000000}\selectfont \(\displaystyle {1.00}\)}%
\end{pgfscope}%
\begin{pgfscope}%
\definecolor{textcolor}{rgb}{0.000000,0.000000,0.000000}%
\pgfsetstrokecolor{textcolor}%
\pgfsetfillcolor{textcolor}%
\pgftext[x=1.793581in,y=0.223333in,,top]{\color{textcolor}\rmfamily\fontsize{10.000000}{12.000000}\selectfont False positive rate}%
\end{pgfscope}%
\begin{pgfscope}%
\pgfsetbuttcap%
\pgfsetroundjoin%
\definecolor{currentfill}{rgb}{0.000000,0.000000,0.000000}%
\pgfsetfillcolor{currentfill}%
\pgfsetlinewidth{0.803000pt}%
\definecolor{currentstroke}{rgb}{0.000000,0.000000,0.000000}%
\pgfsetstrokecolor{currentstroke}%
\pgfsetdash{}{0pt}%
\pgfsys@defobject{currentmarker}{\pgfqpoint{-0.048611in}{0.000000in}}{\pgfqpoint{-0.000000in}{0.000000in}}{%
\pgfpathmoveto{\pgfqpoint{-0.000000in}{0.000000in}}%
\pgfpathlineto{\pgfqpoint{-0.048611in}{0.000000in}}%
\pgfusepath{stroke,fill}%
}%
\begin{pgfscope}%
\pgfsys@transformshift{0.553581in}{0.583444in}%
\pgfsys@useobject{currentmarker}{}%
\end{pgfscope}%
\end{pgfscope}%
\begin{pgfscope}%
\definecolor{textcolor}{rgb}{0.000000,0.000000,0.000000}%
\pgfsetstrokecolor{textcolor}%
\pgfsetfillcolor{textcolor}%
\pgftext[x=0.278889in, y=0.535250in, left, base]{\color{textcolor}\rmfamily\fontsize{10.000000}{12.000000}\selectfont \(\displaystyle {0.0}\)}%
\end{pgfscope}%
\begin{pgfscope}%
\pgfsetbuttcap%
\pgfsetroundjoin%
\definecolor{currentfill}{rgb}{0.000000,0.000000,0.000000}%
\pgfsetfillcolor{currentfill}%
\pgfsetlinewidth{0.803000pt}%
\definecolor{currentstroke}{rgb}{0.000000,0.000000,0.000000}%
\pgfsetstrokecolor{currentstroke}%
\pgfsetdash{}{0pt}%
\pgfsys@defobject{currentmarker}{\pgfqpoint{-0.048611in}{0.000000in}}{\pgfqpoint{-0.000000in}{0.000000in}}{%
\pgfpathmoveto{\pgfqpoint{-0.000000in}{0.000000in}}%
\pgfpathlineto{\pgfqpoint{-0.048611in}{0.000000in}}%
\pgfusepath{stroke,fill}%
}%
\begin{pgfscope}%
\pgfsys@transformshift{0.553581in}{0.919444in}%
\pgfsys@useobject{currentmarker}{}%
\end{pgfscope}%
\end{pgfscope}%
\begin{pgfscope}%
\definecolor{textcolor}{rgb}{0.000000,0.000000,0.000000}%
\pgfsetstrokecolor{textcolor}%
\pgfsetfillcolor{textcolor}%
\pgftext[x=0.278889in, y=0.871250in, left, base]{\color{textcolor}\rmfamily\fontsize{10.000000}{12.000000}\selectfont \(\displaystyle {0.2}\)}%
\end{pgfscope}%
\begin{pgfscope}%
\pgfsetbuttcap%
\pgfsetroundjoin%
\definecolor{currentfill}{rgb}{0.000000,0.000000,0.000000}%
\pgfsetfillcolor{currentfill}%
\pgfsetlinewidth{0.803000pt}%
\definecolor{currentstroke}{rgb}{0.000000,0.000000,0.000000}%
\pgfsetstrokecolor{currentstroke}%
\pgfsetdash{}{0pt}%
\pgfsys@defobject{currentmarker}{\pgfqpoint{-0.048611in}{0.000000in}}{\pgfqpoint{-0.000000in}{0.000000in}}{%
\pgfpathmoveto{\pgfqpoint{-0.000000in}{0.000000in}}%
\pgfpathlineto{\pgfqpoint{-0.048611in}{0.000000in}}%
\pgfusepath{stroke,fill}%
}%
\begin{pgfscope}%
\pgfsys@transformshift{0.553581in}{1.255444in}%
\pgfsys@useobject{currentmarker}{}%
\end{pgfscope}%
\end{pgfscope}%
\begin{pgfscope}%
\definecolor{textcolor}{rgb}{0.000000,0.000000,0.000000}%
\pgfsetstrokecolor{textcolor}%
\pgfsetfillcolor{textcolor}%
\pgftext[x=0.278889in, y=1.207250in, left, base]{\color{textcolor}\rmfamily\fontsize{10.000000}{12.000000}\selectfont \(\displaystyle {0.4}\)}%
\end{pgfscope}%
\begin{pgfscope}%
\pgfsetbuttcap%
\pgfsetroundjoin%
\definecolor{currentfill}{rgb}{0.000000,0.000000,0.000000}%
\pgfsetfillcolor{currentfill}%
\pgfsetlinewidth{0.803000pt}%
\definecolor{currentstroke}{rgb}{0.000000,0.000000,0.000000}%
\pgfsetstrokecolor{currentstroke}%
\pgfsetdash{}{0pt}%
\pgfsys@defobject{currentmarker}{\pgfqpoint{-0.048611in}{0.000000in}}{\pgfqpoint{-0.000000in}{0.000000in}}{%
\pgfpathmoveto{\pgfqpoint{-0.000000in}{0.000000in}}%
\pgfpathlineto{\pgfqpoint{-0.048611in}{0.000000in}}%
\pgfusepath{stroke,fill}%
}%
\begin{pgfscope}%
\pgfsys@transformshift{0.553581in}{1.591444in}%
\pgfsys@useobject{currentmarker}{}%
\end{pgfscope}%
\end{pgfscope}%
\begin{pgfscope}%
\definecolor{textcolor}{rgb}{0.000000,0.000000,0.000000}%
\pgfsetstrokecolor{textcolor}%
\pgfsetfillcolor{textcolor}%
\pgftext[x=0.278889in, y=1.543250in, left, base]{\color{textcolor}\rmfamily\fontsize{10.000000}{12.000000}\selectfont \(\displaystyle {0.6}\)}%
\end{pgfscope}%
\begin{pgfscope}%
\pgfsetbuttcap%
\pgfsetroundjoin%
\definecolor{currentfill}{rgb}{0.000000,0.000000,0.000000}%
\pgfsetfillcolor{currentfill}%
\pgfsetlinewidth{0.803000pt}%
\definecolor{currentstroke}{rgb}{0.000000,0.000000,0.000000}%
\pgfsetstrokecolor{currentstroke}%
\pgfsetdash{}{0pt}%
\pgfsys@defobject{currentmarker}{\pgfqpoint{-0.048611in}{0.000000in}}{\pgfqpoint{-0.000000in}{0.000000in}}{%
\pgfpathmoveto{\pgfqpoint{-0.000000in}{0.000000in}}%
\pgfpathlineto{\pgfqpoint{-0.048611in}{0.000000in}}%
\pgfusepath{stroke,fill}%
}%
\begin{pgfscope}%
\pgfsys@transformshift{0.553581in}{1.927444in}%
\pgfsys@useobject{currentmarker}{}%
\end{pgfscope}%
\end{pgfscope}%
\begin{pgfscope}%
\definecolor{textcolor}{rgb}{0.000000,0.000000,0.000000}%
\pgfsetstrokecolor{textcolor}%
\pgfsetfillcolor{textcolor}%
\pgftext[x=0.278889in, y=1.879250in, left, base]{\color{textcolor}\rmfamily\fontsize{10.000000}{12.000000}\selectfont \(\displaystyle {0.8}\)}%
\end{pgfscope}%
\begin{pgfscope}%
\pgfsetbuttcap%
\pgfsetroundjoin%
\definecolor{currentfill}{rgb}{0.000000,0.000000,0.000000}%
\pgfsetfillcolor{currentfill}%
\pgfsetlinewidth{0.803000pt}%
\definecolor{currentstroke}{rgb}{0.000000,0.000000,0.000000}%
\pgfsetstrokecolor{currentstroke}%
\pgfsetdash{}{0pt}%
\pgfsys@defobject{currentmarker}{\pgfqpoint{-0.048611in}{0.000000in}}{\pgfqpoint{-0.000000in}{0.000000in}}{%
\pgfpathmoveto{\pgfqpoint{-0.000000in}{0.000000in}}%
\pgfpathlineto{\pgfqpoint{-0.048611in}{0.000000in}}%
\pgfusepath{stroke,fill}%
}%
\begin{pgfscope}%
\pgfsys@transformshift{0.553581in}{2.263444in}%
\pgfsys@useobject{currentmarker}{}%
\end{pgfscope}%
\end{pgfscope}%
\begin{pgfscope}%
\definecolor{textcolor}{rgb}{0.000000,0.000000,0.000000}%
\pgfsetstrokecolor{textcolor}%
\pgfsetfillcolor{textcolor}%
\pgftext[x=0.278889in, y=2.215250in, left, base]{\color{textcolor}\rmfamily\fontsize{10.000000}{12.000000}\selectfont \(\displaystyle {1.0}\)}%
\end{pgfscope}%
\begin{pgfscope}%
\definecolor{textcolor}{rgb}{0.000000,0.000000,0.000000}%
\pgfsetstrokecolor{textcolor}%
\pgfsetfillcolor{textcolor}%
\pgftext[x=0.223333in,y=1.423444in,,bottom,rotate=90.000000]{\color{textcolor}\rmfamily\fontsize{10.000000}{12.000000}\selectfont True positive rate}%
\end{pgfscope}%
\begin{pgfscope}%
\pgfpathrectangle{\pgfqpoint{0.553581in}{0.499444in}}{\pgfqpoint{2.480000in}{1.848000in}}%
\pgfusepath{clip}%
\pgfsetbuttcap%
\pgfsetroundjoin%
\pgfsetlinewidth{1.505625pt}%
\definecolor{currentstroke}{rgb}{0.000000,0.000000,0.000000}%
\pgfsetstrokecolor{currentstroke}%
\pgfsetdash{{5.550000pt}{2.400000pt}}{0.000000pt}%
\pgfpathmoveto{\pgfqpoint{0.666308in}{0.583444in}}%
\pgfpathlineto{\pgfqpoint{2.920853in}{2.263444in}}%
\pgfusepath{stroke}%
\end{pgfscope}%
\begin{pgfscope}%
\pgfpathrectangle{\pgfqpoint{0.553581in}{0.499444in}}{\pgfqpoint{2.480000in}{1.848000in}}%
\pgfusepath{clip}%
\pgfsetrectcap%
\pgfsetroundjoin%
\pgfsetlinewidth{1.505625pt}%
\definecolor{currentstroke}{rgb}{0.121569,0.466667,0.705882}%
\pgfsetstrokecolor{currentstroke}%
\pgfsetdash{}{0pt}%
\pgfpathmoveto{\pgfqpoint{0.666308in}{0.583444in}}%
\pgfpathlineto{\pgfqpoint{0.667266in}{0.584536in}}%
\pgfpathlineto{\pgfqpoint{0.668365in}{0.593356in}}%
\pgfpathlineto{\pgfqpoint{0.668563in}{0.594448in}}%
\pgfpathlineto{\pgfqpoint{0.669662in}{0.606040in}}%
\pgfpathlineto{\pgfqpoint{0.669774in}{0.606796in}}%
\pgfpathlineto{\pgfqpoint{0.670873in}{0.633676in}}%
\pgfpathlineto{\pgfqpoint{0.670986in}{0.634600in}}%
\pgfpathlineto{\pgfqpoint{0.672085in}{0.655432in}}%
\pgfpathlineto{\pgfqpoint{0.672142in}{0.655936in}}%
\pgfpathlineto{\pgfqpoint{0.673241in}{0.682144in}}%
\pgfpathlineto{\pgfqpoint{0.673297in}{0.682816in}}%
\pgfpathlineto{\pgfqpoint{0.674396in}{0.710200in}}%
\pgfpathlineto{\pgfqpoint{0.674509in}{0.711040in}}%
\pgfpathlineto{\pgfqpoint{0.675580in}{0.745732in}}%
\pgfpathlineto{\pgfqpoint{0.675636in}{0.745732in}}%
\pgfpathlineto{\pgfqpoint{0.676735in}{0.778072in}}%
\pgfpathlineto{\pgfqpoint{0.676792in}{0.778828in}}%
\pgfpathlineto{\pgfqpoint{0.677891in}{0.807724in}}%
\pgfpathlineto{\pgfqpoint{0.677919in}{0.807724in}}%
\pgfpathlineto{\pgfqpoint{0.679018in}{0.844180in}}%
\pgfpathlineto{\pgfqpoint{0.679103in}{0.844768in}}%
\pgfpathlineto{\pgfqpoint{0.680202in}{0.875848in}}%
\pgfpathlineto{\pgfqpoint{0.680343in}{0.876688in}}%
\pgfpathlineto{\pgfqpoint{0.681442in}{0.910204in}}%
\pgfpathlineto{\pgfqpoint{0.681526in}{0.911296in}}%
\pgfpathlineto{\pgfqpoint{0.682625in}{0.947920in}}%
\pgfpathlineto{\pgfqpoint{0.682710in}{0.949012in}}%
\pgfpathlineto{\pgfqpoint{0.683809in}{0.974884in}}%
\pgfpathlineto{\pgfqpoint{0.683950in}{0.975556in}}%
\pgfpathlineto{\pgfqpoint{0.685049in}{0.998488in}}%
\pgfpathlineto{\pgfqpoint{0.685133in}{0.999244in}}%
\pgfpathlineto{\pgfqpoint{0.686233in}{1.017052in}}%
\pgfpathlineto{\pgfqpoint{0.686317in}{1.017976in}}%
\pgfpathlineto{\pgfqpoint{0.687360in}{1.042840in}}%
\pgfpathlineto{\pgfqpoint{0.687501in}{1.043848in}}%
\pgfpathlineto{\pgfqpoint{0.688600in}{1.064260in}}%
\pgfpathlineto{\pgfqpoint{0.688656in}{1.065184in}}%
\pgfpathlineto{\pgfqpoint{0.689755in}{1.085176in}}%
\pgfpathlineto{\pgfqpoint{0.689868in}{1.086268in}}%
\pgfpathlineto{\pgfqpoint{0.690967in}{1.109536in}}%
\pgfpathlineto{\pgfqpoint{0.691080in}{1.110544in}}%
\pgfpathlineto{\pgfqpoint{0.692179in}{1.129780in}}%
\pgfpathlineto{\pgfqpoint{0.692263in}{1.130368in}}%
\pgfpathlineto{\pgfqpoint{0.693306in}{1.149268in}}%
\pgfpathlineto{\pgfqpoint{0.693475in}{1.149688in}}%
\pgfpathlineto{\pgfqpoint{0.694574in}{1.170856in}}%
\pgfpathlineto{\pgfqpoint{0.694603in}{1.170856in}}%
\pgfpathlineto{\pgfqpoint{0.695702in}{1.190596in}}%
\pgfpathlineto{\pgfqpoint{0.695758in}{1.190596in}}%
\pgfpathlineto{\pgfqpoint{0.696857in}{1.210756in}}%
\pgfpathlineto{\pgfqpoint{0.696913in}{1.211596in}}%
\pgfpathlineto{\pgfqpoint{0.698013in}{1.235200in}}%
\pgfpathlineto{\pgfqpoint{0.698069in}{1.235956in}}%
\pgfpathlineto{\pgfqpoint{0.699168in}{1.252504in}}%
\pgfpathlineto{\pgfqpoint{0.699196in}{1.252504in}}%
\pgfpathlineto{\pgfqpoint{0.700295in}{1.271740in}}%
\pgfpathlineto{\pgfqpoint{0.700380in}{1.272748in}}%
\pgfpathlineto{\pgfqpoint{0.701451in}{1.288288in}}%
\pgfpathlineto{\pgfqpoint{0.701563in}{1.288960in}}%
\pgfpathlineto{\pgfqpoint{0.702634in}{1.304920in}}%
\pgfpathlineto{\pgfqpoint{0.702832in}{1.306012in}}%
\pgfpathlineto{\pgfqpoint{0.703931in}{1.321132in}}%
\pgfpathlineto{\pgfqpoint{0.704043in}{1.322140in}}%
\pgfpathlineto{\pgfqpoint{0.705143in}{1.337932in}}%
\pgfpathlineto{\pgfqpoint{0.705283in}{1.338856in}}%
\pgfpathlineto{\pgfqpoint{0.706383in}{1.354900in}}%
\pgfpathlineto{\pgfqpoint{0.706523in}{1.355992in}}%
\pgfpathlineto{\pgfqpoint{0.707594in}{1.371784in}}%
\pgfpathlineto{\pgfqpoint{0.707707in}{1.372372in}}%
\pgfpathlineto{\pgfqpoint{0.708806in}{1.385980in}}%
\pgfpathlineto{\pgfqpoint{0.708975in}{1.387072in}}%
\pgfpathlineto{\pgfqpoint{0.710046in}{1.397320in}}%
\pgfpathlineto{\pgfqpoint{0.710131in}{1.397572in}}%
\pgfpathlineto{\pgfqpoint{0.711230in}{1.411012in}}%
\pgfpathlineto{\pgfqpoint{0.711343in}{1.411432in}}%
\pgfpathlineto{\pgfqpoint{0.712385in}{1.423948in}}%
\pgfpathlineto{\pgfqpoint{0.712667in}{1.424956in}}%
\pgfpathlineto{\pgfqpoint{0.713738in}{1.434616in}}%
\pgfpathlineto{\pgfqpoint{0.713935in}{1.435456in}}%
\pgfpathlineto{\pgfqpoint{0.715034in}{1.447384in}}%
\pgfpathlineto{\pgfqpoint{0.715175in}{1.447552in}}%
\pgfpathlineto{\pgfqpoint{0.716274in}{1.459648in}}%
\pgfpathlineto{\pgfqpoint{0.716415in}{1.460740in}}%
\pgfpathlineto{\pgfqpoint{0.717514in}{1.472164in}}%
\pgfpathlineto{\pgfqpoint{0.717627in}{1.472584in}}%
\pgfpathlineto{\pgfqpoint{0.718726in}{1.483252in}}%
\pgfpathlineto{\pgfqpoint{0.718867in}{1.484008in}}%
\pgfpathlineto{\pgfqpoint{0.719966in}{1.494256in}}%
\pgfpathlineto{\pgfqpoint{0.720107in}{1.495348in}}%
\pgfpathlineto{\pgfqpoint{0.721206in}{1.505092in}}%
\pgfpathlineto{\pgfqpoint{0.721432in}{1.506184in}}%
\pgfpathlineto{\pgfqpoint{0.722531in}{1.518028in}}%
\pgfpathlineto{\pgfqpoint{0.722672in}{1.519120in}}%
\pgfpathlineto{\pgfqpoint{0.723771in}{1.527520in}}%
\pgfpathlineto{\pgfqpoint{0.723883in}{1.527940in}}%
\pgfpathlineto{\pgfqpoint{0.724983in}{1.539700in}}%
\pgfpathlineto{\pgfqpoint{0.725152in}{1.540540in}}%
\pgfpathlineto{\pgfqpoint{0.726251in}{1.549024in}}%
\pgfpathlineto{\pgfqpoint{0.726476in}{1.549864in}}%
\pgfpathlineto{\pgfqpoint{0.727519in}{1.560364in}}%
\pgfpathlineto{\pgfqpoint{0.727801in}{1.561204in}}%
\pgfpathlineto{\pgfqpoint{0.728815in}{1.568176in}}%
\pgfpathlineto{\pgfqpoint{0.729238in}{1.569268in}}%
\pgfpathlineto{\pgfqpoint{0.730337in}{1.580020in}}%
\pgfpathlineto{\pgfqpoint{0.730563in}{1.580692in}}%
\pgfpathlineto{\pgfqpoint{0.731633in}{1.588420in}}%
\pgfpathlineto{\pgfqpoint{0.731887in}{1.589512in}}%
\pgfpathlineto{\pgfqpoint{0.732986in}{1.597492in}}%
\pgfpathlineto{\pgfqpoint{0.733240in}{1.598584in}}%
\pgfpathlineto{\pgfqpoint{0.734339in}{1.606900in}}%
\pgfpathlineto{\pgfqpoint{0.734621in}{1.607992in}}%
\pgfpathlineto{\pgfqpoint{0.735692in}{1.614796in}}%
\pgfpathlineto{\pgfqpoint{0.736086in}{1.615888in}}%
\pgfpathlineto{\pgfqpoint{0.737185in}{1.624624in}}%
\pgfpathlineto{\pgfqpoint{0.737326in}{1.625464in}}%
\pgfpathlineto{\pgfqpoint{0.738425in}{1.631932in}}%
\pgfpathlineto{\pgfqpoint{0.738623in}{1.632688in}}%
\pgfpathlineto{\pgfqpoint{0.739665in}{1.638988in}}%
\pgfpathlineto{\pgfqpoint{0.740003in}{1.639828in}}%
\pgfpathlineto{\pgfqpoint{0.741103in}{1.646128in}}%
\pgfpathlineto{\pgfqpoint{0.741356in}{1.647220in}}%
\pgfpathlineto{\pgfqpoint{0.742455in}{1.654612in}}%
\pgfpathlineto{\pgfqpoint{0.742596in}{1.655452in}}%
\pgfpathlineto{\pgfqpoint{0.743667in}{1.661416in}}%
\pgfpathlineto{\pgfqpoint{0.743893in}{1.662508in}}%
\pgfpathlineto{\pgfqpoint{0.744992in}{1.667128in}}%
\pgfpathlineto{\pgfqpoint{0.745273in}{1.668220in}}%
\pgfpathlineto{\pgfqpoint{0.746373in}{1.675444in}}%
\pgfpathlineto{\pgfqpoint{0.746683in}{1.676116in}}%
\pgfpathlineto{\pgfqpoint{0.747782in}{1.680652in}}%
\pgfpathlineto{\pgfqpoint{0.748120in}{1.681744in}}%
\pgfpathlineto{\pgfqpoint{0.749191in}{1.689136in}}%
\pgfpathlineto{\pgfqpoint{0.749529in}{1.690228in}}%
\pgfpathlineto{\pgfqpoint{0.750515in}{1.695016in}}%
\pgfpathlineto{\pgfqpoint{0.750938in}{1.696108in}}%
\pgfpathlineto{\pgfqpoint{0.752037in}{1.702072in}}%
\pgfpathlineto{\pgfqpoint{0.752263in}{1.702996in}}%
\pgfpathlineto{\pgfqpoint{0.753305in}{1.707784in}}%
\pgfpathlineto{\pgfqpoint{0.753531in}{1.708792in}}%
\pgfpathlineto{\pgfqpoint{0.754630in}{1.713664in}}%
\pgfpathlineto{\pgfqpoint{0.755137in}{1.714672in}}%
\pgfpathlineto{\pgfqpoint{0.756236in}{1.718284in}}%
\pgfpathlineto{\pgfqpoint{0.756433in}{1.719292in}}%
\pgfpathlineto{\pgfqpoint{0.757533in}{1.723576in}}%
\pgfpathlineto{\pgfqpoint{0.757786in}{1.724500in}}%
\pgfpathlineto{\pgfqpoint{0.758857in}{1.729876in}}%
\pgfpathlineto{\pgfqpoint{0.759111in}{1.730968in}}%
\pgfpathlineto{\pgfqpoint{0.760210in}{1.737016in}}%
\pgfpathlineto{\pgfqpoint{0.760463in}{1.737940in}}%
\pgfpathlineto{\pgfqpoint{0.761534in}{1.741720in}}%
\pgfpathlineto{\pgfqpoint{0.762126in}{1.742728in}}%
\pgfpathlineto{\pgfqpoint{0.763225in}{1.747096in}}%
\pgfpathlineto{\pgfqpoint{0.763676in}{1.748188in}}%
\pgfpathlineto{\pgfqpoint{0.764775in}{1.752472in}}%
\pgfpathlineto{\pgfqpoint{0.765113in}{1.753564in}}%
\pgfpathlineto{\pgfqpoint{0.765536in}{1.756252in}}%
\pgfpathlineto{\pgfqpoint{0.771877in}{1.757260in}}%
\pgfpathlineto{\pgfqpoint{0.772920in}{1.760200in}}%
\pgfpathlineto{\pgfqpoint{0.773371in}{1.761124in}}%
\pgfpathlineto{\pgfqpoint{0.774470in}{1.765240in}}%
\pgfpathlineto{\pgfqpoint{0.774780in}{1.765996in}}%
\pgfpathlineto{\pgfqpoint{0.775879in}{1.770196in}}%
\pgfpathlineto{\pgfqpoint{0.776076in}{1.771204in}}%
\pgfpathlineto{\pgfqpoint{0.777175in}{1.776664in}}%
\pgfpathlineto{\pgfqpoint{0.777542in}{1.777672in}}%
\pgfpathlineto{\pgfqpoint{0.778641in}{1.781200in}}%
\pgfpathlineto{\pgfqpoint{0.778951in}{1.782208in}}%
\pgfpathlineto{\pgfqpoint{0.780050in}{1.787752in}}%
\pgfpathlineto{\pgfqpoint{0.780388in}{1.788676in}}%
\pgfpathlineto{\pgfqpoint{0.781459in}{1.792456in}}%
\pgfpathlineto{\pgfqpoint{0.781882in}{1.793548in}}%
\pgfpathlineto{\pgfqpoint{0.782896in}{1.797664in}}%
\pgfpathlineto{\pgfqpoint{0.783291in}{1.798756in}}%
\pgfpathlineto{\pgfqpoint{0.784390in}{1.802872in}}%
\pgfpathlineto{\pgfqpoint{0.784813in}{1.803880in}}%
\pgfpathlineto{\pgfqpoint{0.785855in}{1.806568in}}%
\pgfpathlineto{\pgfqpoint{0.786363in}{1.807576in}}%
\pgfpathlineto{\pgfqpoint{0.787462in}{1.811440in}}%
\pgfpathlineto{\pgfqpoint{0.787715in}{1.812532in}}%
\pgfpathlineto{\pgfqpoint{0.788814in}{1.816480in}}%
\pgfpathlineto{\pgfqpoint{0.789209in}{1.817572in}}%
\pgfpathlineto{\pgfqpoint{0.790308in}{1.820932in}}%
\pgfpathlineto{\pgfqpoint{0.790900in}{1.822024in}}%
\pgfpathlineto{\pgfqpoint{0.791971in}{1.826140in}}%
\pgfpathlineto{\pgfqpoint{0.792534in}{1.827232in}}%
\pgfpathlineto{\pgfqpoint{0.793605in}{1.832104in}}%
\pgfpathlineto{\pgfqpoint{0.794028in}{1.833196in}}%
\pgfpathlineto{\pgfqpoint{0.795099in}{1.835128in}}%
\pgfpathlineto{\pgfqpoint{0.795916in}{1.836220in}}%
\pgfpathlineto{\pgfqpoint{0.796987in}{1.840168in}}%
\pgfpathlineto{\pgfqpoint{0.797494in}{1.841260in}}%
\pgfpathlineto{\pgfqpoint{0.798593in}{1.843276in}}%
\pgfpathlineto{\pgfqpoint{0.799016in}{1.844200in}}%
\pgfpathlineto{\pgfqpoint{0.800031in}{1.847980in}}%
\pgfpathlineto{\pgfqpoint{0.800397in}{1.848988in}}%
\pgfpathlineto{\pgfqpoint{0.801496in}{1.853356in}}%
\pgfpathlineto{\pgfqpoint{0.802285in}{1.854280in}}%
\pgfpathlineto{\pgfqpoint{0.803356in}{1.857052in}}%
\pgfpathlineto{\pgfqpoint{0.803948in}{1.858144in}}%
\pgfpathlineto{\pgfqpoint{0.805047in}{1.861756in}}%
\pgfpathlineto{\pgfqpoint{0.805498in}{1.862764in}}%
\pgfpathlineto{\pgfqpoint{0.806569in}{1.866124in}}%
\pgfpathlineto{\pgfqpoint{0.807583in}{1.867132in}}%
\pgfpathlineto{\pgfqpoint{0.808683in}{1.870576in}}%
\pgfpathlineto{\pgfqpoint{0.809472in}{1.871668in}}%
\pgfpathlineto{\pgfqpoint{0.810571in}{1.873432in}}%
\pgfpathlineto{\pgfqpoint{0.811050in}{1.874440in}}%
\pgfpathlineto{\pgfqpoint{0.812121in}{1.878052in}}%
\pgfpathlineto{\pgfqpoint{0.812600in}{1.879144in}}%
\pgfpathlineto{\pgfqpoint{0.813699in}{1.883344in}}%
\pgfpathlineto{\pgfqpoint{0.814403in}{1.884436in}}%
\pgfpathlineto{\pgfqpoint{0.815418in}{1.887292in}}%
\pgfpathlineto{\pgfqpoint{0.815813in}{1.888132in}}%
\pgfpathlineto{\pgfqpoint{0.816771in}{1.890064in}}%
\pgfpathlineto{\pgfqpoint{0.817447in}{1.891072in}}%
\pgfpathlineto{\pgfqpoint{0.818490in}{1.893844in}}%
\pgfpathlineto{\pgfqpoint{0.818884in}{1.894768in}}%
\pgfpathlineto{\pgfqpoint{0.819983in}{1.896616in}}%
\pgfpathlineto{\pgfqpoint{0.821026in}{1.897708in}}%
\pgfpathlineto{\pgfqpoint{0.822125in}{1.900480in}}%
\pgfpathlineto{\pgfqpoint{0.822717in}{1.901572in}}%
\pgfpathlineto{\pgfqpoint{0.823816in}{1.904596in}}%
\pgfpathlineto{\pgfqpoint{0.824013in}{1.905520in}}%
\pgfpathlineto{\pgfqpoint{0.825056in}{1.908628in}}%
\pgfpathlineto{\pgfqpoint{0.826268in}{1.909636in}}%
\pgfpathlineto{\pgfqpoint{0.827311in}{1.911232in}}%
\pgfpathlineto{\pgfqpoint{0.828438in}{1.912324in}}%
\pgfpathlineto{\pgfqpoint{0.829537in}{1.914424in}}%
\pgfpathlineto{\pgfqpoint{0.830016in}{1.915516in}}%
\pgfpathlineto{\pgfqpoint{0.831087in}{1.917784in}}%
\pgfpathlineto{\pgfqpoint{0.831538in}{1.918708in}}%
\pgfpathlineto{\pgfqpoint{0.832524in}{1.920976in}}%
\pgfpathlineto{\pgfqpoint{0.833003in}{1.921984in}}%
\pgfpathlineto{\pgfqpoint{0.834046in}{1.924168in}}%
\pgfpathlineto{\pgfqpoint{0.834779in}{1.925176in}}%
\pgfpathlineto{\pgfqpoint{0.835878in}{1.928284in}}%
\pgfpathlineto{\pgfqpoint{0.836442in}{1.929376in}}%
\pgfpathlineto{\pgfqpoint{0.837541in}{1.931308in}}%
\pgfpathlineto{\pgfqpoint{0.838386in}{1.932400in}}%
\pgfpathlineto{\pgfqpoint{0.839485in}{1.934668in}}%
\pgfpathlineto{\pgfqpoint{0.840133in}{1.935676in}}%
\pgfpathlineto{\pgfqpoint{0.841120in}{1.938532in}}%
\pgfpathlineto{\pgfqpoint{0.842163in}{1.939624in}}%
\pgfpathlineto{\pgfqpoint{0.843262in}{1.941304in}}%
\pgfpathlineto{\pgfqpoint{0.844361in}{1.942396in}}%
\pgfpathlineto{\pgfqpoint{0.845460in}{1.945084in}}%
\pgfpathlineto{\pgfqpoint{0.846333in}{1.946176in}}%
\pgfpathlineto{\pgfqpoint{0.847433in}{1.949032in}}%
\pgfpathlineto{\pgfqpoint{0.848165in}{1.950040in}}%
\pgfpathlineto{\pgfqpoint{0.849236in}{1.951720in}}%
\pgfpathlineto{\pgfqpoint{0.849772in}{1.952728in}}%
\pgfpathlineto{\pgfqpoint{0.850843in}{1.954996in}}%
\pgfpathlineto{\pgfqpoint{0.851942in}{1.956088in}}%
\pgfpathlineto{\pgfqpoint{0.853013in}{1.958104in}}%
\pgfpathlineto{\pgfqpoint{0.853745in}{1.959196in}}%
\pgfpathlineto{\pgfqpoint{0.854816in}{1.960540in}}%
\pgfpathlineto{\pgfqpoint{0.855183in}{1.961380in}}%
\pgfpathlineto{\pgfqpoint{0.856282in}{1.963732in}}%
\pgfpathlineto{\pgfqpoint{0.857071in}{1.964824in}}%
\pgfpathlineto{\pgfqpoint{0.858113in}{1.966756in}}%
\pgfpathlineto{\pgfqpoint{0.858621in}{1.967848in}}%
\pgfpathlineto{\pgfqpoint{0.859663in}{1.969444in}}%
\pgfpathlineto{\pgfqpoint{0.860255in}{1.970536in}}%
\pgfpathlineto{\pgfqpoint{0.861298in}{1.972132in}}%
\pgfpathlineto{\pgfqpoint{0.862538in}{1.973224in}}%
\pgfpathlineto{\pgfqpoint{0.863609in}{1.974988in}}%
\pgfpathlineto{\pgfqpoint{0.864680in}{1.975912in}}%
\pgfpathlineto{\pgfqpoint{0.865751in}{1.978600in}}%
\pgfpathlineto{\pgfqpoint{0.866371in}{1.979692in}}%
\pgfpathlineto{\pgfqpoint{0.867470in}{1.981456in}}%
\pgfpathlineto{\pgfqpoint{0.868090in}{1.982464in}}%
\pgfpathlineto{\pgfqpoint{0.869189in}{1.983892in}}%
\pgfpathlineto{\pgfqpoint{0.870795in}{1.984900in}}%
\pgfpathlineto{\pgfqpoint{0.871782in}{1.985992in}}%
\pgfpathlineto{\pgfqpoint{0.872712in}{1.987084in}}%
\pgfpathlineto{\pgfqpoint{0.873416in}{1.987672in}}%
\pgfpathlineto{\pgfqpoint{0.875248in}{1.988764in}}%
\pgfpathlineto{\pgfqpoint{0.876347in}{1.990528in}}%
\pgfpathlineto{\pgfqpoint{0.877023in}{1.991620in}}%
\pgfpathlineto{\pgfqpoint{0.878094in}{1.993132in}}%
\pgfpathlineto{\pgfqpoint{0.878883in}{1.994224in}}%
\pgfpathlineto{\pgfqpoint{0.879983in}{1.995820in}}%
\pgfpathlineto{\pgfqpoint{0.880743in}{1.996912in}}%
\pgfpathlineto{\pgfqpoint{0.881730in}{1.997584in}}%
\pgfpathlineto{\pgfqpoint{0.882885in}{1.998592in}}%
\pgfpathlineto{\pgfqpoint{0.883984in}{1.999936in}}%
\pgfpathlineto{\pgfqpoint{0.885224in}{2.001028in}}%
\pgfpathlineto{\pgfqpoint{0.886098in}{2.001952in}}%
\pgfpathlineto{\pgfqpoint{0.887169in}{2.003044in}}%
\pgfpathlineto{\pgfqpoint{0.888268in}{2.005564in}}%
\pgfpathlineto{\pgfqpoint{0.889480in}{2.006656in}}%
\pgfpathlineto{\pgfqpoint{0.890579in}{2.008084in}}%
\pgfpathlineto{\pgfqpoint{0.891593in}{2.009176in}}%
\pgfpathlineto{\pgfqpoint{0.892664in}{2.010100in}}%
\pgfpathlineto{\pgfqpoint{0.893735in}{2.011192in}}%
\pgfpathlineto{\pgfqpoint{0.894834in}{2.012620in}}%
\pgfpathlineto{\pgfqpoint{0.895990in}{2.013712in}}%
\pgfpathlineto{\pgfqpoint{0.896892in}{2.015476in}}%
\pgfpathlineto{\pgfqpoint{0.898526in}{2.016568in}}%
\pgfpathlineto{\pgfqpoint{0.899597in}{2.017828in}}%
\pgfpathlineto{\pgfqpoint{0.900893in}{2.018920in}}%
\pgfpathlineto{\pgfqpoint{0.901908in}{2.020432in}}%
\pgfpathlineto{\pgfqpoint{0.902894in}{2.021524in}}%
\pgfpathlineto{\pgfqpoint{0.903909in}{2.022532in}}%
\pgfpathlineto{\pgfqpoint{0.904754in}{2.023624in}}%
\pgfpathlineto{\pgfqpoint{0.905515in}{2.024884in}}%
\pgfpathlineto{\pgfqpoint{0.907093in}{2.025976in}}%
\pgfpathlineto{\pgfqpoint{0.908136in}{2.026648in}}%
\pgfpathlineto{\pgfqpoint{0.909348in}{2.027740in}}%
\pgfpathlineto{\pgfqpoint{0.910419in}{2.028244in}}%
\pgfpathlineto{\pgfqpoint{0.911800in}{2.029336in}}%
\pgfpathlineto{\pgfqpoint{0.912730in}{2.031100in}}%
\pgfpathlineto{\pgfqpoint{0.914759in}{2.032192in}}%
\pgfpathlineto{\pgfqpoint{0.915858in}{2.032948in}}%
\pgfpathlineto{\pgfqpoint{0.917211in}{2.034040in}}%
\pgfpathlineto{\pgfqpoint{0.918141in}{2.034964in}}%
\pgfpathlineto{\pgfqpoint{0.919719in}{2.036056in}}%
\pgfpathlineto{\pgfqpoint{0.920536in}{2.037064in}}%
\pgfpathlineto{\pgfqpoint{0.922453in}{2.038072in}}%
\pgfpathlineto{\pgfqpoint{0.923326in}{2.038996in}}%
\pgfpathlineto{\pgfqpoint{0.925017in}{2.040088in}}%
\pgfpathlineto{\pgfqpoint{0.926003in}{2.041180in}}%
\pgfpathlineto{\pgfqpoint{0.926990in}{2.042272in}}%
\pgfpathlineto{\pgfqpoint{0.927610in}{2.043028in}}%
\pgfpathlineto{\pgfqpoint{0.929244in}{2.044120in}}%
\pgfpathlineto{\pgfqpoint{0.930174in}{2.044540in}}%
\pgfpathlineto{\pgfqpoint{0.932316in}{2.045632in}}%
\pgfpathlineto{\pgfqpoint{0.933387in}{2.046556in}}%
\pgfpathlineto{\pgfqpoint{0.934796in}{2.047648in}}%
\pgfpathlineto{\pgfqpoint{0.935839in}{2.048992in}}%
\pgfpathlineto{\pgfqpoint{0.938009in}{2.050084in}}%
\pgfpathlineto{\pgfqpoint{0.938770in}{2.050672in}}%
\pgfpathlineto{\pgfqpoint{0.940855in}{2.051764in}}%
\pgfpathlineto{\pgfqpoint{0.941870in}{2.052688in}}%
\pgfpathlineto{\pgfqpoint{0.943786in}{2.053780in}}%
\pgfpathlineto{\pgfqpoint{0.944829in}{2.054872in}}%
\pgfpathlineto{\pgfqpoint{0.946830in}{2.055964in}}%
\pgfpathlineto{\pgfqpoint{0.947901in}{2.057476in}}%
\pgfpathlineto{\pgfqpoint{0.949845in}{2.058568in}}%
\pgfpathlineto{\pgfqpoint{0.950888in}{2.059408in}}%
\pgfpathlineto{\pgfqpoint{0.952241in}{2.060500in}}%
\pgfpathlineto{\pgfqpoint{0.953312in}{2.061508in}}%
\pgfpathlineto{\pgfqpoint{0.955284in}{2.062600in}}%
\pgfpathlineto{\pgfqpoint{0.956327in}{2.063356in}}%
\pgfpathlineto{\pgfqpoint{0.957990in}{2.064364in}}%
\pgfpathlineto{\pgfqpoint{0.958948in}{2.066044in}}%
\pgfpathlineto{\pgfqpoint{0.962048in}{2.067136in}}%
\pgfpathlineto{\pgfqpoint{0.962893in}{2.068144in}}%
\pgfpathlineto{\pgfqpoint{0.965683in}{2.069236in}}%
\pgfpathlineto{\pgfqpoint{0.966783in}{2.070412in}}%
\pgfpathlineto{\pgfqpoint{0.968135in}{2.071504in}}%
\pgfpathlineto{\pgfqpoint{0.969178in}{2.072428in}}%
\pgfpathlineto{\pgfqpoint{0.970700in}{2.073520in}}%
\pgfpathlineto{\pgfqpoint{0.971771in}{2.075032in}}%
\pgfpathlineto{\pgfqpoint{0.973377in}{2.076040in}}%
\pgfpathlineto{\pgfqpoint{0.974476in}{2.077468in}}%
\pgfpathlineto{\pgfqpoint{0.976364in}{2.078560in}}%
\pgfpathlineto{\pgfqpoint{0.977097in}{2.079736in}}%
\pgfpathlineto{\pgfqpoint{0.979436in}{2.080828in}}%
\pgfpathlineto{\pgfqpoint{0.980479in}{2.082004in}}%
\pgfpathlineto{\pgfqpoint{0.982733in}{2.083096in}}%
\pgfpathlineto{\pgfqpoint{0.982818in}{2.083264in}}%
\pgfpathlineto{\pgfqpoint{0.986228in}{2.084356in}}%
\pgfpathlineto{\pgfqpoint{0.987130in}{2.085196in}}%
\pgfpathlineto{\pgfqpoint{0.988821in}{2.086288in}}%
\pgfpathlineto{\pgfqpoint{0.989863in}{2.086876in}}%
\pgfpathlineto{\pgfqpoint{0.992287in}{2.087968in}}%
\pgfpathlineto{\pgfqpoint{0.993273in}{2.088892in}}%
\pgfpathlineto{\pgfqpoint{0.994570in}{2.089984in}}%
\pgfpathlineto{\pgfqpoint{0.995669in}{2.090320in}}%
\pgfpathlineto{\pgfqpoint{0.997444in}{2.091328in}}%
\pgfpathlineto{\pgfqpoint{0.997923in}{2.091916in}}%
\pgfpathlineto{\pgfqpoint{0.999783in}{2.092924in}}%
\pgfpathlineto{\pgfqpoint{1.000883in}{2.093680in}}%
\pgfpathlineto{\pgfqpoint{1.002461in}{2.094688in}}%
\pgfpathlineto{\pgfqpoint{1.003475in}{2.095360in}}%
\pgfpathlineto{\pgfqpoint{1.004462in}{2.096452in}}%
\pgfpathlineto{\pgfqpoint{1.005251in}{2.096956in}}%
\pgfpathlineto{\pgfqpoint{1.007421in}{2.098048in}}%
\pgfpathlineto{\pgfqpoint{1.008435in}{2.098888in}}%
\pgfpathlineto{\pgfqpoint{1.010887in}{2.099980in}}%
\pgfpathlineto{\pgfqpoint{1.011958in}{2.100568in}}%
\pgfpathlineto{\pgfqpoint{1.014973in}{2.101660in}}%
\pgfpathlineto{\pgfqpoint{1.015791in}{2.102164in}}%
\pgfpathlineto{\pgfqpoint{1.018693in}{2.103256in}}%
\pgfpathlineto{\pgfqpoint{1.019623in}{2.104348in}}%
\pgfpathlineto{\pgfqpoint{1.023400in}{2.105440in}}%
\pgfpathlineto{\pgfqpoint{1.024358in}{2.105860in}}%
\pgfpathlineto{\pgfqpoint{1.027373in}{2.106952in}}%
\pgfpathlineto{\pgfqpoint{1.028219in}{2.107372in}}%
\pgfpathlineto{\pgfqpoint{1.031037in}{2.108464in}}%
\pgfpathlineto{\pgfqpoint{1.031939in}{2.109136in}}%
\pgfpathlineto{\pgfqpoint{1.034222in}{2.110228in}}%
\pgfpathlineto{\pgfqpoint{1.035236in}{2.110984in}}%
\pgfpathlineto{\pgfqpoint{1.038139in}{2.112076in}}%
\pgfpathlineto{\pgfqpoint{1.039238in}{2.113000in}}%
\pgfpathlineto{\pgfqpoint{1.041887in}{2.114008in}}%
\pgfpathlineto{\pgfqpoint{1.042845in}{2.115100in}}%
\pgfpathlineto{\pgfqpoint{1.045297in}{2.116192in}}%
\pgfpathlineto{\pgfqpoint{1.046002in}{2.116696in}}%
\pgfpathlineto{\pgfqpoint{1.050060in}{2.117788in}}%
\pgfpathlineto{\pgfqpoint{1.050990in}{2.118208in}}%
\pgfpathlineto{\pgfqpoint{1.052793in}{2.119300in}}%
\pgfpathlineto{\pgfqpoint{1.053554in}{2.119888in}}%
\pgfpathlineto{\pgfqpoint{1.056232in}{2.120980in}}%
\pgfpathlineto{\pgfqpoint{1.057077in}{2.121400in}}%
\pgfpathlineto{\pgfqpoint{1.059726in}{2.122492in}}%
\pgfpathlineto{\pgfqpoint{1.060628in}{2.123080in}}%
\pgfpathlineto{\pgfqpoint{1.062995in}{2.124172in}}%
\pgfpathlineto{\pgfqpoint{1.063953in}{2.125012in}}%
\pgfpathlineto{\pgfqpoint{1.067673in}{2.126104in}}%
\pgfpathlineto{\pgfqpoint{1.068575in}{2.126440in}}%
\pgfpathlineto{\pgfqpoint{1.070914in}{2.127448in}}%
\pgfpathlineto{\pgfqpoint{1.071478in}{2.127952in}}%
\pgfpathlineto{\pgfqpoint{1.074663in}{2.129044in}}%
\pgfpathlineto{\pgfqpoint{1.075311in}{2.129548in}}%
\pgfpathlineto{\pgfqpoint{1.079679in}{2.130556in}}%
\pgfpathlineto{\pgfqpoint{1.080778in}{2.131144in}}%
\pgfpathlineto{\pgfqpoint{1.082694in}{2.132236in}}%
\pgfpathlineto{\pgfqpoint{1.083793in}{2.132572in}}%
\pgfpathlineto{\pgfqpoint{1.086893in}{2.133664in}}%
\pgfpathlineto{\pgfqpoint{1.087823in}{2.134084in}}%
\pgfpathlineto{\pgfqpoint{1.091938in}{2.135092in}}%
\pgfpathlineto{\pgfqpoint{1.092981in}{2.135680in}}%
\pgfpathlineto{\pgfqpoint{1.097293in}{2.136772in}}%
\pgfpathlineto{\pgfqpoint{1.098335in}{2.137108in}}%
\pgfpathlineto{\pgfqpoint{1.102901in}{2.138200in}}%
\pgfpathlineto{\pgfqpoint{1.103972in}{2.139208in}}%
\pgfpathlineto{\pgfqpoint{1.108283in}{2.140300in}}%
\pgfpathlineto{\pgfqpoint{1.109298in}{2.140720in}}%
\pgfpathlineto{\pgfqpoint{1.112821in}{2.141812in}}%
\pgfpathlineto{\pgfqpoint{1.113723in}{2.142232in}}%
\pgfpathlineto{\pgfqpoint{1.116710in}{2.143324in}}%
\pgfpathlineto{\pgfqpoint{1.116879in}{2.143492in}}%
\pgfpathlineto{\pgfqpoint{1.122797in}{2.144584in}}%
\pgfpathlineto{\pgfqpoint{1.123586in}{2.144836in}}%
\pgfpathlineto{\pgfqpoint{1.129786in}{2.145928in}}%
\pgfpathlineto{\pgfqpoint{1.130519in}{2.146684in}}%
\pgfpathlineto{\pgfqpoint{1.138043in}{2.147776in}}%
\pgfpathlineto{\pgfqpoint{1.139030in}{2.148196in}}%
\pgfpathlineto{\pgfqpoint{1.141425in}{2.149288in}}%
\pgfpathlineto{\pgfqpoint{1.142496in}{2.149960in}}%
\pgfpathlineto{\pgfqpoint{1.146639in}{2.151052in}}%
\pgfpathlineto{\pgfqpoint{1.147428in}{2.151472in}}%
\pgfpathlineto{\pgfqpoint{1.152022in}{2.152564in}}%
\pgfpathlineto{\pgfqpoint{1.152529in}{2.152816in}}%
\pgfpathlineto{\pgfqpoint{1.156277in}{2.153824in}}%
\pgfpathlineto{\pgfqpoint{1.157263in}{2.154244in}}%
\pgfpathlineto{\pgfqpoint{1.159631in}{2.155336in}}%
\pgfpathlineto{\pgfqpoint{1.160702in}{2.155672in}}%
\pgfpathlineto{\pgfqpoint{1.166310in}{2.156764in}}%
\pgfpathlineto{\pgfqpoint{1.167353in}{2.157268in}}%
\pgfpathlineto{\pgfqpoint{1.172876in}{2.158360in}}%
\pgfpathlineto{\pgfqpoint{1.173383in}{2.158612in}}%
\pgfpathlineto{\pgfqpoint{1.178738in}{2.159704in}}%
\pgfpathlineto{\pgfqpoint{1.179837in}{2.160208in}}%
\pgfpathlineto{\pgfqpoint{1.183980in}{2.161300in}}%
\pgfpathlineto{\pgfqpoint{1.185023in}{2.161636in}}%
\pgfpathlineto{\pgfqpoint{1.190433in}{2.162728in}}%
\pgfpathlineto{\pgfqpoint{1.190772in}{2.162980in}}%
\pgfpathlineto{\pgfqpoint{1.196803in}{2.164072in}}%
\pgfpathlineto{\pgfqpoint{1.197873in}{2.164492in}}%
\pgfpathlineto{\pgfqpoint{1.203200in}{2.165584in}}%
\pgfpathlineto{\pgfqpoint{1.203707in}{2.165752in}}%
\pgfpathlineto{\pgfqpoint{1.208639in}{2.166844in}}%
\pgfpathlineto{\pgfqpoint{1.209682in}{2.167180in}}%
\pgfpathlineto{\pgfqpoint{1.215121in}{2.168272in}}%
\pgfpathlineto{\pgfqpoint{1.216023in}{2.168608in}}%
\pgfpathlineto{\pgfqpoint{1.221913in}{2.169700in}}%
\pgfpathlineto{\pgfqpoint{1.222673in}{2.170036in}}%
\pgfpathlineto{\pgfqpoint{1.228423in}{2.171128in}}%
\pgfpathlineto{\pgfqpoint{1.228423in}{2.171212in}}%
\pgfpathlineto{\pgfqpoint{1.234313in}{2.172304in}}%
\pgfpathlineto{\pgfqpoint{1.234820in}{2.172640in}}%
\pgfpathlineto{\pgfqpoint{1.242880in}{2.173732in}}%
\pgfpathlineto{\pgfqpoint{1.243162in}{2.173984in}}%
\pgfpathlineto{\pgfqpoint{1.251870in}{2.175076in}}%
\pgfpathlineto{\pgfqpoint{1.252743in}{2.175496in}}%
\pgfpathlineto{\pgfqpoint{1.257196in}{2.176588in}}%
\pgfpathlineto{\pgfqpoint{1.257957in}{2.176924in}}%
\pgfpathlineto{\pgfqpoint{1.271090in}{2.178016in}}%
\pgfpathlineto{\pgfqpoint{1.271851in}{2.178436in}}%
\pgfpathlineto{\pgfqpoint{1.280982in}{2.179528in}}%
\pgfpathlineto{\pgfqpoint{1.281320in}{2.179948in}}%
\pgfpathlineto{\pgfqpoint{1.288619in}{2.181040in}}%
\pgfpathlineto{\pgfqpoint{1.289464in}{2.181208in}}%
\pgfpathlineto{\pgfqpoint{1.294593in}{2.182300in}}%
\pgfpathlineto{\pgfqpoint{1.295552in}{2.182720in}}%
\pgfpathlineto{\pgfqpoint{1.304654in}{2.183812in}}%
\pgfpathlineto{\pgfqpoint{1.304654in}{2.183896in}}%
\pgfpathlineto{\pgfqpoint{1.311756in}{2.184988in}}%
\pgfpathlineto{\pgfqpoint{1.312545in}{2.185156in}}%
\pgfpathlineto{\pgfqpoint{1.320380in}{2.186248in}}%
\pgfpathlineto{\pgfqpoint{1.321394in}{2.186584in}}%
\pgfpathlineto{\pgfqpoint{1.328581in}{2.187676in}}%
\pgfpathlineto{\pgfqpoint{1.329116in}{2.187928in}}%
\pgfpathlineto{\pgfqpoint{1.337993in}{2.189020in}}%
\pgfpathlineto{\pgfqpoint{1.337993in}{2.189104in}}%
\pgfpathlineto{\pgfqpoint{1.354367in}{2.190196in}}%
\pgfpathlineto{\pgfqpoint{1.354367in}{2.190280in}}%
\pgfpathlineto{\pgfqpoint{1.370600in}{2.191372in}}%
\pgfpathlineto{\pgfqpoint{1.371614in}{2.191792in}}%
\pgfpathlineto{\pgfqpoint{1.379984in}{2.192884in}}%
\pgfpathlineto{\pgfqpoint{1.380633in}{2.193052in}}%
\pgfpathlineto{\pgfqpoint{1.388383in}{2.194144in}}%
\pgfpathlineto{\pgfqpoint{1.389031in}{2.194480in}}%
\pgfpathlineto{\pgfqpoint{1.396696in}{2.195572in}}%
\pgfpathlineto{\pgfqpoint{1.397232in}{2.195740in}}%
\pgfpathlineto{\pgfqpoint{1.407208in}{2.196832in}}%
\pgfpathlineto{\pgfqpoint{1.407377in}{2.197000in}}%
\pgfpathlineto{\pgfqpoint{1.415014in}{2.198092in}}%
\pgfpathlineto{\pgfqpoint{1.415353in}{2.198260in}}%
\pgfpathlineto{\pgfqpoint{1.427330in}{2.199352in}}%
\pgfpathlineto{\pgfqpoint{1.428288in}{2.199688in}}%
\pgfpathlineto{\pgfqpoint{1.441731in}{2.200780in}}%
\pgfpathlineto{\pgfqpoint{1.442238in}{2.201032in}}%
\pgfpathlineto{\pgfqpoint{1.456075in}{2.202124in}}%
\pgfpathlineto{\pgfqpoint{1.456075in}{2.202208in}}%
\pgfpathlineto{\pgfqpoint{1.468109in}{2.203300in}}%
\pgfpathlineto{\pgfqpoint{1.468109in}{2.203384in}}%
\pgfpathlineto{\pgfqpoint{1.481101in}{2.204476in}}%
\pgfpathlineto{\pgfqpoint{1.481101in}{2.204560in}}%
\pgfpathlineto{\pgfqpoint{1.491753in}{2.205652in}}%
\pgfpathlineto{\pgfqpoint{1.492261in}{2.205988in}}%
\pgfpathlineto{\pgfqpoint{1.509846in}{2.207080in}}%
\pgfpathlineto{\pgfqpoint{1.510184in}{2.207248in}}%
\pgfpathlineto{\pgfqpoint{1.524050in}{2.208340in}}%
\pgfpathlineto{\pgfqpoint{1.524670in}{2.208676in}}%
\pgfpathlineto{\pgfqpoint{1.537464in}{2.209768in}}%
\pgfpathlineto{\pgfqpoint{1.537464in}{2.209852in}}%
\pgfpathlineto{\pgfqpoint{1.549836in}{2.210944in}}%
\pgfpathlineto{\pgfqpoint{1.549836in}{2.211028in}}%
\pgfpathlineto{\pgfqpoint{1.563504in}{2.212120in}}%
\pgfpathlineto{\pgfqpoint{1.564463in}{2.212456in}}%
\pgfpathlineto{\pgfqpoint{1.578328in}{2.213548in}}%
\pgfpathlineto{\pgfqpoint{1.579343in}{2.213716in}}%
\pgfpathlineto{\pgfqpoint{1.588868in}{2.214808in}}%
\pgfpathlineto{\pgfqpoint{1.589347in}{2.214976in}}%
\pgfpathlineto{\pgfqpoint{1.610061in}{2.216068in}}%
\pgfpathlineto{\pgfqpoint{1.610765in}{2.216236in}}%
\pgfpathlineto{\pgfqpoint{1.624349in}{2.217328in}}%
\pgfpathlineto{\pgfqpoint{1.624349in}{2.217412in}}%
\pgfpathlineto{\pgfqpoint{1.633142in}{2.218504in}}%
\pgfpathlineto{\pgfqpoint{1.633142in}{2.218588in}}%
\pgfpathlineto{\pgfqpoint{1.656053in}{2.219680in}}%
\pgfpathlineto{\pgfqpoint{1.656730in}{2.219848in}}%
\pgfpathlineto{\pgfqpoint{1.668904in}{2.220940in}}%
\pgfpathlineto{\pgfqpoint{1.669806in}{2.221192in}}%
\pgfpathlineto{\pgfqpoint{1.685306in}{2.222284in}}%
\pgfpathlineto{\pgfqpoint{1.686152in}{2.222536in}}%
\pgfpathlineto{\pgfqpoint{1.710642in}{2.223628in}}%
\pgfpathlineto{\pgfqpoint{1.710867in}{2.223796in}}%
\pgfpathlineto{\pgfqpoint{1.736823in}{2.224888in}}%
\pgfpathlineto{\pgfqpoint{1.736823in}{2.224972in}}%
\pgfpathlineto{\pgfqpoint{1.769175in}{2.226064in}}%
\pgfpathlineto{\pgfqpoint{1.769175in}{2.226148in}}%
\pgfpathlineto{\pgfqpoint{1.795131in}{2.227240in}}%
\pgfpathlineto{\pgfqpoint{1.795131in}{2.227324in}}%
\pgfpathlineto{\pgfqpoint{1.815788in}{2.228416in}}%
\pgfpathlineto{\pgfqpoint{1.816295in}{2.228584in}}%
\pgfpathlineto{\pgfqpoint{1.833430in}{2.229676in}}%
\pgfpathlineto{\pgfqpoint{1.833430in}{2.229760in}}%
\pgfpathlineto{\pgfqpoint{1.855299in}{2.230852in}}%
\pgfpathlineto{\pgfqpoint{1.855299in}{2.230936in}}%
\pgfpathlineto{\pgfqpoint{1.884242in}{2.232028in}}%
\pgfpathlineto{\pgfqpoint{1.884242in}{2.232112in}}%
\pgfpathlineto{\pgfqpoint{1.914171in}{2.233204in}}%
\pgfpathlineto{\pgfqpoint{1.914171in}{2.233288in}}%
\pgfpathlineto{\pgfqpoint{1.943733in}{2.234380in}}%
\pgfpathlineto{\pgfqpoint{1.944635in}{2.234548in}}%
\pgfpathlineto{\pgfqpoint{1.970027in}{2.235640in}}%
\pgfpathlineto{\pgfqpoint{1.970027in}{2.235724in}}%
\pgfpathlineto{\pgfqpoint{1.990769in}{2.236816in}}%
\pgfpathlineto{\pgfqpoint{1.991333in}{2.236984in}}%
\pgfpathlineto{\pgfqpoint{2.016583in}{2.238076in}}%
\pgfpathlineto{\pgfqpoint{2.016668in}{2.238244in}}%
\pgfpathlineto{\pgfqpoint{2.033013in}{2.239336in}}%
\pgfpathlineto{\pgfqpoint{2.033943in}{2.239504in}}%
\pgfpathlineto{\pgfqpoint{2.060547in}{2.240596in}}%
\pgfpathlineto{\pgfqpoint{2.060547in}{2.240680in}}%
\pgfpathlineto{\pgfqpoint{2.084051in}{2.241772in}}%
\pgfpathlineto{\pgfqpoint{2.084502in}{2.241940in}}%
\pgfpathlineto{\pgfqpoint{2.112853in}{2.243032in}}%
\pgfpathlineto{\pgfqpoint{2.113191in}{2.243200in}}%
\pgfpathlineto{\pgfqpoint{2.139400in}{2.244292in}}%
\pgfpathlineto{\pgfqpoint{2.139851in}{2.244460in}}%
\pgfpathlineto{\pgfqpoint{2.179503in}{2.245552in}}%
\pgfpathlineto{\pgfqpoint{2.179503in}{2.245636in}}%
\pgfpathlineto{\pgfqpoint{2.216393in}{2.246728in}}%
\pgfpathlineto{\pgfqpoint{2.216393in}{2.246812in}}%
\pgfpathlineto{\pgfqpoint{2.273517in}{2.247904in}}%
\pgfpathlineto{\pgfqpoint{2.273517in}{2.247988in}}%
\pgfpathlineto{\pgfqpoint{2.336588in}{2.249080in}}%
\pgfpathlineto{\pgfqpoint{2.336588in}{2.249164in}}%
\pgfpathlineto{\pgfqpoint{2.380101in}{2.250256in}}%
\pgfpathlineto{\pgfqpoint{2.381059in}{2.250424in}}%
\pgfpathlineto{\pgfqpoint{2.450978in}{2.251516in}}%
\pgfpathlineto{\pgfqpoint{2.450978in}{2.251600in}}%
\pgfpathlineto{\pgfqpoint{2.504157in}{2.252692in}}%
\pgfpathlineto{\pgfqpoint{2.504157in}{2.252776in}}%
\pgfpathlineto{\pgfqpoint{2.564635in}{2.253868in}}%
\pgfpathlineto{\pgfqpoint{2.564635in}{2.253952in}}%
\pgfpathlineto{\pgfqpoint{2.638021in}{2.255044in}}%
\pgfpathlineto{\pgfqpoint{2.638021in}{2.255128in}}%
\pgfpathlineto{\pgfqpoint{2.675023in}{2.256220in}}%
\pgfpathlineto{\pgfqpoint{2.675023in}{2.256304in}}%
\pgfpathlineto{\pgfqpoint{2.741476in}{2.257396in}}%
\pgfpathlineto{\pgfqpoint{2.741476in}{2.257480in}}%
\pgfpathlineto{\pgfqpoint{2.791922in}{2.258572in}}%
\pgfpathlineto{\pgfqpoint{2.791922in}{2.258656in}}%
\pgfpathlineto{\pgfqpoint{2.842452in}{2.259748in}}%
\pgfpathlineto{\pgfqpoint{2.842452in}{2.259832in}}%
\pgfpathlineto{\pgfqpoint{2.899435in}{2.260924in}}%
\pgfpathlineto{\pgfqpoint{2.899435in}{2.261008in}}%
\pgfpathlineto{\pgfqpoint{2.917331in}{2.262100in}}%
\pgfpathlineto{\pgfqpoint{2.918035in}{2.262436in}}%
\pgfpathlineto{\pgfqpoint{2.920853in}{2.263192in}}%
\pgfpathlineto{\pgfqpoint{2.920853in}{2.263444in}}%
\pgfusepath{stroke}%
\end{pgfscope}%
\begin{pgfscope}%
\pgfsetrectcap%
\pgfsetmiterjoin%
\pgfsetlinewidth{0.803000pt}%
\definecolor{currentstroke}{rgb}{0.000000,0.000000,0.000000}%
\pgfsetstrokecolor{currentstroke}%
\pgfsetdash{}{0pt}%
\pgfpathmoveto{\pgfqpoint{0.553581in}{0.499444in}}%
\pgfpathlineto{\pgfqpoint{0.553581in}{2.347444in}}%
\pgfusepath{stroke}%
\end{pgfscope}%
\begin{pgfscope}%
\pgfsetrectcap%
\pgfsetmiterjoin%
\pgfsetlinewidth{0.803000pt}%
\definecolor{currentstroke}{rgb}{0.000000,0.000000,0.000000}%
\pgfsetstrokecolor{currentstroke}%
\pgfsetdash{}{0pt}%
\pgfpathmoveto{\pgfqpoint{3.033581in}{0.499444in}}%
\pgfpathlineto{\pgfqpoint{3.033581in}{2.347444in}}%
\pgfusepath{stroke}%
\end{pgfscope}%
\begin{pgfscope}%
\pgfsetrectcap%
\pgfsetmiterjoin%
\pgfsetlinewidth{0.803000pt}%
\definecolor{currentstroke}{rgb}{0.000000,0.000000,0.000000}%
\pgfsetstrokecolor{currentstroke}%
\pgfsetdash{}{0pt}%
\pgfpathmoveto{\pgfqpoint{0.553581in}{0.499444in}}%
\pgfpathlineto{\pgfqpoint{3.033581in}{0.499444in}}%
\pgfusepath{stroke}%
\end{pgfscope}%
\begin{pgfscope}%
\pgfsetrectcap%
\pgfsetmiterjoin%
\pgfsetlinewidth{0.803000pt}%
\definecolor{currentstroke}{rgb}{0.000000,0.000000,0.000000}%
\pgfsetstrokecolor{currentstroke}%
\pgfsetdash{}{0pt}%
\pgfpathmoveto{\pgfqpoint{0.553581in}{2.347444in}}%
\pgfpathlineto{\pgfqpoint{3.033581in}{2.347444in}}%
\pgfusepath{stroke}%
\end{pgfscope}%
\begin{pgfscope}%
\pgfsetbuttcap%
\pgfsetmiterjoin%
\definecolor{currentfill}{rgb}{1.000000,1.000000,1.000000}%
\pgfsetfillcolor{currentfill}%
\pgfsetlinewidth{1.003750pt}%
\definecolor{currentstroke}{rgb}{1.000000,1.000000,1.000000}%
\pgfsetstrokecolor{currentstroke}%
\pgfsetdash{}{0pt}%
\pgfpathmoveto{\pgfqpoint{1.738673in}{2.144572in}}%
\pgfpathlineto{\pgfqpoint{2.166173in}{2.144572in}}%
\pgfpathlineto{\pgfqpoint{2.166173in}{2.379017in}}%
\pgfpathlineto{\pgfqpoint{1.738673in}{2.379017in}}%
\pgfpathlineto{\pgfqpoint{1.738673in}{2.144572in}}%
\pgfpathclose%
\pgfusepath{stroke,fill}%
\end{pgfscope}%
\begin{pgfscope}%
\definecolor{textcolor}{rgb}{0.000000,0.000000,0.000000}%
\pgfsetstrokecolor{textcolor}%
\pgfsetfillcolor{textcolor}%
\pgftext[x=1.794229in,y=2.227072in,left,base]{\color{textcolor}\rmfamily\fontsize{10.000000}{12.000000}\selectfont 0.268}%
\end{pgfscope}%
\begin{pgfscope}%
\pgfsetbuttcap%
\pgfsetmiterjoin%
\definecolor{currentfill}{rgb}{1.000000,1.000000,1.000000}%
\pgfsetfillcolor{currentfill}%
\pgfsetlinewidth{1.003750pt}%
\definecolor{currentstroke}{rgb}{1.000000,1.000000,1.000000}%
\pgfsetstrokecolor{currentstroke}%
\pgfsetdash{}{0pt}%
\pgfpathmoveto{\pgfqpoint{0.656773in}{1.340860in}}%
\pgfpathlineto{\pgfqpoint{1.084273in}{1.340860in}}%
\pgfpathlineto{\pgfqpoint{1.084273in}{1.575305in}}%
\pgfpathlineto{\pgfqpoint{0.656773in}{1.575305in}}%
\pgfpathlineto{\pgfqpoint{0.656773in}{1.340860in}}%
\pgfpathclose%
\pgfusepath{stroke,fill}%
\end{pgfscope}%
\begin{pgfscope}%
\definecolor{textcolor}{rgb}{0.000000,0.000000,0.000000}%
\pgfsetstrokecolor{textcolor}%
\pgfsetfillcolor{textcolor}%
\pgftext[x=0.712329in,y=1.423360in,left,base]{\color{textcolor}\rmfamily\fontsize{10.000000}{12.000000}\selectfont 0.733}%
\end{pgfscope}%
\begin{pgfscope}%
\definecolor{textcolor}{rgb}{0.000000,0.000000,0.000000}%
\pgfsetstrokecolor{textcolor}%
\pgfsetfillcolor{textcolor}%
\pgftext[x=1.793581in,y=2.430778in,,base]{\color{textcolor}\rmfamily\fontsize{12.000000}{14.400000}\selectfont ROC Curve}%
\end{pgfscope}%
\begin{pgfscope}%
\pgfsetbuttcap%
\pgfsetmiterjoin%
\definecolor{currentfill}{rgb}{1.000000,1.000000,1.000000}%
\pgfsetfillcolor{currentfill}%
\pgfsetfillopacity{0.800000}%
\pgfsetlinewidth{1.003750pt}%
\definecolor{currentstroke}{rgb}{0.800000,0.800000,0.800000}%
\pgfsetstrokecolor{currentstroke}%
\pgfsetstrokeopacity{0.800000}%
\pgfsetdash{}{0pt}%
\pgfpathmoveto{\pgfqpoint{0.800942in}{0.568889in}}%
\pgfpathlineto{\pgfqpoint{2.936358in}{0.568889in}}%
\pgfpathquadraticcurveto{\pgfqpoint{2.964136in}{0.568889in}}{\pgfqpoint{2.964136in}{0.596666in}}%
\pgfpathlineto{\pgfqpoint{2.964136in}{0.791111in}}%
\pgfpathquadraticcurveto{\pgfqpoint{2.964136in}{0.818888in}}{\pgfqpoint{2.936358in}{0.818888in}}%
\pgfpathlineto{\pgfqpoint{0.800942in}{0.818888in}}%
\pgfpathquadraticcurveto{\pgfqpoint{0.773164in}{0.818888in}}{\pgfqpoint{0.773164in}{0.791111in}}%
\pgfpathlineto{\pgfqpoint{0.773164in}{0.596666in}}%
\pgfpathquadraticcurveto{\pgfqpoint{0.773164in}{0.568889in}}{\pgfqpoint{0.800942in}{0.568889in}}%
\pgfpathlineto{\pgfqpoint{0.800942in}{0.568889in}}%
\pgfpathclose%
\pgfusepath{stroke,fill}%
\end{pgfscope}%
\begin{pgfscope}%
\pgfsetrectcap%
\pgfsetroundjoin%
\pgfsetlinewidth{1.505625pt}%
\definecolor{currentstroke}{rgb}{0.121569,0.466667,0.705882}%
\pgfsetstrokecolor{currentstroke}%
\pgfsetdash{}{0pt}%
\pgfpathmoveto{\pgfqpoint{0.828720in}{0.707777in}}%
\pgfpathlineto{\pgfqpoint{0.967608in}{0.707777in}}%
\pgfpathlineto{\pgfqpoint{1.106497in}{0.707777in}}%
\pgfusepath{stroke}%
\end{pgfscope}%
\begin{pgfscope}%
\definecolor{textcolor}{rgb}{0.000000,0.000000,0.000000}%
\pgfsetstrokecolor{textcolor}%
\pgfsetfillcolor{textcolor}%
\pgftext[x=1.217608in,y=0.659166in,left,base]{\color{textcolor}\rmfamily\fontsize{10.000000}{12.000000}\selectfont Area Under Curve = 0.937)}%
\end{pgfscope}%
\end{pgfpicture}%
\makeatother%
\endgroup%

\end{tabular}

{\bf Incorporate $thr=0.5$ into the discussion.}

The {\it confusion matrix} for this ideal data set, here given as percentages of the entire dataset, shows few false positives and false negatives.  The metrics below are the ones we will watch when evaluating models.  Each of them tells a different story about what the model does well.

\vskip 6pt

$\displaystyle Precision = \frac{TP}{PP} = \frac{TP}{TP+FP}$ tells what proportion of the ambulances we sent were needed.  

\vskip 6pt

$\displaystyle Recall = \frac{TP}{P} = \frac{TP}{TP + FN}$ tells what proportion of ambulances we needed were sent.

\vskip 6pt

$\displaystyle F1 = \frac{2}{ \frac{1}{Precision} + \frac{1}{Recall}}$ is the harmonic mean of precision and recall. 

\vskip 6pt

Why the harmonic mean, not the arithmetic or geometric mean?  If $a$ and $b$ are positive numbers with $a<b$, then $a < \text{harmonic} < \text{geometric} < \text{arithmetic} < b$.  The harmonic, while being influenced by the larger number, is closest to the smaller, so the harmonic mean emphasizes what the model does poorly.  



\begin{center}
\begin{tabular}{cc}
\begin{tabular}{cc|c|c|c}
	&\multicolumn{1}{c}{}& \multicolumn{2}{c}{Prediction} \cr
	&\multicolumn{1}{c}{} & \multicolumn{1}{c}{Neg} & \multicolumn{1}{c}{Pos} & \multicolumn{1}{c}{Total} \cr\cline{3-4}
	\multirow{2}{*}{Actual}&Neg &TN = 67.0\% & FP = 18.7\% & N = 85.7\% \vrule width 0pt height 10pt depth 2pt \cr\cline{3-4}
	&Pos & FN = 3.2\% & TP = 11.1\% & P = 14.3\% \vrule width 0pt height 10pt depth 2pt \cr\cline{3-4}
	\multicolumn{2}{r}{Total} & \multicolumn{1}{c}{PN = 70.2\%} & \multicolumn{1}{c}{PP = 29.8\%} \vrule width 0pt height 14pt depth 2pt \cr
\end{tabular}
&
\begin{tabular}{ll}
0.372 & Precision \cr 
0.774 & Recall \cr 
0.502 & F1 \cr 
\end{tabular}
\end{tabular}
\end{center}

%%%%%
If we do not address the data imbalance, the model building algorithm will maximize accuracy by classifying most (or all) of the samples as ``No Ambulance'' with $p < 0.5$  We built the artificial results below by multiplying the probabilities in the above results by $0.5$.  Note that the Area Under the Curve (AUC) did not change.  

\begin{center}
\begin{tabular}{p{0.5\textwidth} p{0.5\textwidth}}
  \vspace{0pt} \input{../Keras/Images/Ideal_50_Pred.pgf}
  &
  \vspace{0pt} %% Creator: Matplotlib, PGF backend
%%
%% To include the figure in your LaTeX document, write
%%   \input{<filename>.pgf}
%%
%% Make sure the required packages are loaded in your preamble
%%   \usepackage{pgf}
%%
%% Also ensure that all the required font packages are loaded; for instance,
%% the lmodern package is sometimes necessary when using math font.
%%   \usepackage{lmodern}
%%
%% Figures using additional raster images can only be included by \input if
%% they are in the same directory as the main LaTeX file. For loading figures
%% from other directories you can use the `import` package
%%   \usepackage{import}
%%
%% and then include the figures with
%%   \import{<path to file>}{<filename>.pgf}
%%
%% Matplotlib used the following preamble
%%   
%%   \usepackage{fontspec}
%%   \makeatletter\@ifpackageloaded{underscore}{}{\usepackage[strings]{underscore}}\makeatother
%%
\begingroup%
\makeatletter%
\begin{pgfpicture}%
\pgfpathrectangle{\pgfpointorigin}{\pgfqpoint{2.221861in}{1.953444in}}%
\pgfusepath{use as bounding box, clip}%
\begin{pgfscope}%
\pgfsetbuttcap%
\pgfsetmiterjoin%
\definecolor{currentfill}{rgb}{1.000000,1.000000,1.000000}%
\pgfsetfillcolor{currentfill}%
\pgfsetlinewidth{0.000000pt}%
\definecolor{currentstroke}{rgb}{1.000000,1.000000,1.000000}%
\pgfsetstrokecolor{currentstroke}%
\pgfsetdash{}{0pt}%
\pgfpathmoveto{\pgfqpoint{0.000000in}{0.000000in}}%
\pgfpathlineto{\pgfqpoint{2.221861in}{0.000000in}}%
\pgfpathlineto{\pgfqpoint{2.221861in}{1.953444in}}%
\pgfpathlineto{\pgfqpoint{0.000000in}{1.953444in}}%
\pgfpathlineto{\pgfqpoint{0.000000in}{0.000000in}}%
\pgfpathclose%
\pgfusepath{fill}%
\end{pgfscope}%
\begin{pgfscope}%
\pgfsetbuttcap%
\pgfsetmiterjoin%
\definecolor{currentfill}{rgb}{1.000000,1.000000,1.000000}%
\pgfsetfillcolor{currentfill}%
\pgfsetlinewidth{0.000000pt}%
\definecolor{currentstroke}{rgb}{0.000000,0.000000,0.000000}%
\pgfsetstrokecolor{currentstroke}%
\pgfsetstrokeopacity{0.000000}%
\pgfsetdash{}{0pt}%
\pgfpathmoveto{\pgfqpoint{0.553581in}{0.499444in}}%
\pgfpathlineto{\pgfqpoint{2.103581in}{0.499444in}}%
\pgfpathlineto{\pgfqpoint{2.103581in}{1.654444in}}%
\pgfpathlineto{\pgfqpoint{0.553581in}{1.654444in}}%
\pgfpathlineto{\pgfqpoint{0.553581in}{0.499444in}}%
\pgfpathclose%
\pgfusepath{fill}%
\end{pgfscope}%
\begin{pgfscope}%
\pgfsetbuttcap%
\pgfsetroundjoin%
\definecolor{currentfill}{rgb}{0.000000,0.000000,0.000000}%
\pgfsetfillcolor{currentfill}%
\pgfsetlinewidth{0.803000pt}%
\definecolor{currentstroke}{rgb}{0.000000,0.000000,0.000000}%
\pgfsetstrokecolor{currentstroke}%
\pgfsetdash{}{0pt}%
\pgfsys@defobject{currentmarker}{\pgfqpoint{0.000000in}{-0.048611in}}{\pgfqpoint{0.000000in}{0.000000in}}{%
\pgfpathmoveto{\pgfqpoint{0.000000in}{0.000000in}}%
\pgfpathlineto{\pgfqpoint{0.000000in}{-0.048611in}}%
\pgfusepath{stroke,fill}%
}%
\begin{pgfscope}%
\pgfsys@transformshift{0.624035in}{0.499444in}%
\pgfsys@useobject{currentmarker}{}%
\end{pgfscope}%
\end{pgfscope}%
\begin{pgfscope}%
\definecolor{textcolor}{rgb}{0.000000,0.000000,0.000000}%
\pgfsetstrokecolor{textcolor}%
\pgfsetfillcolor{textcolor}%
\pgftext[x=0.624035in,y=0.402222in,,top]{\color{textcolor}\rmfamily\fontsize{10.000000}{12.000000}\selectfont \(\displaystyle {0.0}\)}%
\end{pgfscope}%
\begin{pgfscope}%
\pgfsetbuttcap%
\pgfsetroundjoin%
\definecolor{currentfill}{rgb}{0.000000,0.000000,0.000000}%
\pgfsetfillcolor{currentfill}%
\pgfsetlinewidth{0.803000pt}%
\definecolor{currentstroke}{rgb}{0.000000,0.000000,0.000000}%
\pgfsetstrokecolor{currentstroke}%
\pgfsetdash{}{0pt}%
\pgfsys@defobject{currentmarker}{\pgfqpoint{0.000000in}{-0.048611in}}{\pgfqpoint{0.000000in}{0.000000in}}{%
\pgfpathmoveto{\pgfqpoint{0.000000in}{0.000000in}}%
\pgfpathlineto{\pgfqpoint{0.000000in}{-0.048611in}}%
\pgfusepath{stroke,fill}%
}%
\begin{pgfscope}%
\pgfsys@transformshift{1.328581in}{0.499444in}%
\pgfsys@useobject{currentmarker}{}%
\end{pgfscope}%
\end{pgfscope}%
\begin{pgfscope}%
\definecolor{textcolor}{rgb}{0.000000,0.000000,0.000000}%
\pgfsetstrokecolor{textcolor}%
\pgfsetfillcolor{textcolor}%
\pgftext[x=1.328581in,y=0.402222in,,top]{\color{textcolor}\rmfamily\fontsize{10.000000}{12.000000}\selectfont \(\displaystyle {0.5}\)}%
\end{pgfscope}%
\begin{pgfscope}%
\pgfsetbuttcap%
\pgfsetroundjoin%
\definecolor{currentfill}{rgb}{0.000000,0.000000,0.000000}%
\pgfsetfillcolor{currentfill}%
\pgfsetlinewidth{0.803000pt}%
\definecolor{currentstroke}{rgb}{0.000000,0.000000,0.000000}%
\pgfsetstrokecolor{currentstroke}%
\pgfsetdash{}{0pt}%
\pgfsys@defobject{currentmarker}{\pgfqpoint{0.000000in}{-0.048611in}}{\pgfqpoint{0.000000in}{0.000000in}}{%
\pgfpathmoveto{\pgfqpoint{0.000000in}{0.000000in}}%
\pgfpathlineto{\pgfqpoint{0.000000in}{-0.048611in}}%
\pgfusepath{stroke,fill}%
}%
\begin{pgfscope}%
\pgfsys@transformshift{2.033126in}{0.499444in}%
\pgfsys@useobject{currentmarker}{}%
\end{pgfscope}%
\end{pgfscope}%
\begin{pgfscope}%
\definecolor{textcolor}{rgb}{0.000000,0.000000,0.000000}%
\pgfsetstrokecolor{textcolor}%
\pgfsetfillcolor{textcolor}%
\pgftext[x=2.033126in,y=0.402222in,,top]{\color{textcolor}\rmfamily\fontsize{10.000000}{12.000000}\selectfont \(\displaystyle {1.0}\)}%
\end{pgfscope}%
\begin{pgfscope}%
\definecolor{textcolor}{rgb}{0.000000,0.000000,0.000000}%
\pgfsetstrokecolor{textcolor}%
\pgfsetfillcolor{textcolor}%
\pgftext[x=1.328581in,y=0.223333in,,top]{\color{textcolor}\rmfamily\fontsize{10.000000}{12.000000}\selectfont False positive rate}%
\end{pgfscope}%
\begin{pgfscope}%
\pgfsetbuttcap%
\pgfsetroundjoin%
\definecolor{currentfill}{rgb}{0.000000,0.000000,0.000000}%
\pgfsetfillcolor{currentfill}%
\pgfsetlinewidth{0.803000pt}%
\definecolor{currentstroke}{rgb}{0.000000,0.000000,0.000000}%
\pgfsetstrokecolor{currentstroke}%
\pgfsetdash{}{0pt}%
\pgfsys@defobject{currentmarker}{\pgfqpoint{-0.048611in}{0.000000in}}{\pgfqpoint{-0.000000in}{0.000000in}}{%
\pgfpathmoveto{\pgfqpoint{-0.000000in}{0.000000in}}%
\pgfpathlineto{\pgfqpoint{-0.048611in}{0.000000in}}%
\pgfusepath{stroke,fill}%
}%
\begin{pgfscope}%
\pgfsys@transformshift{0.553581in}{0.551944in}%
\pgfsys@useobject{currentmarker}{}%
\end{pgfscope}%
\end{pgfscope}%
\begin{pgfscope}%
\definecolor{textcolor}{rgb}{0.000000,0.000000,0.000000}%
\pgfsetstrokecolor{textcolor}%
\pgfsetfillcolor{textcolor}%
\pgftext[x=0.278889in, y=0.503750in, left, base]{\color{textcolor}\rmfamily\fontsize{10.000000}{12.000000}\selectfont \(\displaystyle {0.0}\)}%
\end{pgfscope}%
\begin{pgfscope}%
\pgfsetbuttcap%
\pgfsetroundjoin%
\definecolor{currentfill}{rgb}{0.000000,0.000000,0.000000}%
\pgfsetfillcolor{currentfill}%
\pgfsetlinewidth{0.803000pt}%
\definecolor{currentstroke}{rgb}{0.000000,0.000000,0.000000}%
\pgfsetstrokecolor{currentstroke}%
\pgfsetdash{}{0pt}%
\pgfsys@defobject{currentmarker}{\pgfqpoint{-0.048611in}{0.000000in}}{\pgfqpoint{-0.000000in}{0.000000in}}{%
\pgfpathmoveto{\pgfqpoint{-0.000000in}{0.000000in}}%
\pgfpathlineto{\pgfqpoint{-0.048611in}{0.000000in}}%
\pgfusepath{stroke,fill}%
}%
\begin{pgfscope}%
\pgfsys@transformshift{0.553581in}{1.076944in}%
\pgfsys@useobject{currentmarker}{}%
\end{pgfscope}%
\end{pgfscope}%
\begin{pgfscope}%
\definecolor{textcolor}{rgb}{0.000000,0.000000,0.000000}%
\pgfsetstrokecolor{textcolor}%
\pgfsetfillcolor{textcolor}%
\pgftext[x=0.278889in, y=1.028750in, left, base]{\color{textcolor}\rmfamily\fontsize{10.000000}{12.000000}\selectfont \(\displaystyle {0.5}\)}%
\end{pgfscope}%
\begin{pgfscope}%
\pgfsetbuttcap%
\pgfsetroundjoin%
\definecolor{currentfill}{rgb}{0.000000,0.000000,0.000000}%
\pgfsetfillcolor{currentfill}%
\pgfsetlinewidth{0.803000pt}%
\definecolor{currentstroke}{rgb}{0.000000,0.000000,0.000000}%
\pgfsetstrokecolor{currentstroke}%
\pgfsetdash{}{0pt}%
\pgfsys@defobject{currentmarker}{\pgfqpoint{-0.048611in}{0.000000in}}{\pgfqpoint{-0.000000in}{0.000000in}}{%
\pgfpathmoveto{\pgfqpoint{-0.000000in}{0.000000in}}%
\pgfpathlineto{\pgfqpoint{-0.048611in}{0.000000in}}%
\pgfusepath{stroke,fill}%
}%
\begin{pgfscope}%
\pgfsys@transformshift{0.553581in}{1.601944in}%
\pgfsys@useobject{currentmarker}{}%
\end{pgfscope}%
\end{pgfscope}%
\begin{pgfscope}%
\definecolor{textcolor}{rgb}{0.000000,0.000000,0.000000}%
\pgfsetstrokecolor{textcolor}%
\pgfsetfillcolor{textcolor}%
\pgftext[x=0.278889in, y=1.553750in, left, base]{\color{textcolor}\rmfamily\fontsize{10.000000}{12.000000}\selectfont \(\displaystyle {1.0}\)}%
\end{pgfscope}%
\begin{pgfscope}%
\definecolor{textcolor}{rgb}{0.000000,0.000000,0.000000}%
\pgfsetstrokecolor{textcolor}%
\pgfsetfillcolor{textcolor}%
\pgftext[x=0.223333in,y=1.076944in,,bottom,rotate=90.000000]{\color{textcolor}\rmfamily\fontsize{10.000000}{12.000000}\selectfont True positive rate}%
\end{pgfscope}%
\begin{pgfscope}%
\pgfpathrectangle{\pgfqpoint{0.553581in}{0.499444in}}{\pgfqpoint{1.550000in}{1.155000in}}%
\pgfusepath{clip}%
\pgfsetbuttcap%
\pgfsetroundjoin%
\pgfsetlinewidth{1.505625pt}%
\definecolor{currentstroke}{rgb}{0.000000,0.000000,0.000000}%
\pgfsetstrokecolor{currentstroke}%
\pgfsetdash{{5.550000pt}{2.400000pt}}{0.000000pt}%
\pgfpathmoveto{\pgfqpoint{0.624035in}{0.551944in}}%
\pgfpathlineto{\pgfqpoint{2.033126in}{1.601944in}}%
\pgfusepath{stroke}%
\end{pgfscope}%
\begin{pgfscope}%
\pgfpathrectangle{\pgfqpoint{0.553581in}{0.499444in}}{\pgfqpoint{1.550000in}{1.155000in}}%
\pgfusepath{clip}%
\pgfsetrectcap%
\pgfsetroundjoin%
\pgfsetlinewidth{1.505625pt}%
\definecolor{currentstroke}{rgb}{0.000000,0.000000,0.000000}%
\pgfsetstrokecolor{currentstroke}%
\pgfsetdash{}{0pt}%
\pgfpathmoveto{\pgfqpoint{0.624035in}{0.551944in}}%
\pgfpathlineto{\pgfqpoint{0.626125in}{0.552364in}}%
\pgfpathlineto{\pgfqpoint{0.627217in}{0.562339in}}%
\pgfpathlineto{\pgfqpoint{0.627957in}{0.563442in}}%
\pgfpathlineto{\pgfqpoint{0.629002in}{0.565332in}}%
\pgfpathlineto{\pgfqpoint{0.629766in}{0.566434in}}%
\pgfpathlineto{\pgfqpoint{0.630858in}{0.569584in}}%
\pgfpathlineto{\pgfqpoint{0.631092in}{0.570477in}}%
\pgfpathlineto{\pgfqpoint{0.632184in}{0.574047in}}%
\pgfpathlineto{\pgfqpoint{0.632560in}{0.575149in}}%
\pgfpathlineto{\pgfqpoint{0.633664in}{0.579822in}}%
\pgfpathlineto{\pgfqpoint{0.633958in}{0.580924in}}%
\pgfpathlineto{\pgfqpoint{0.635061in}{0.585439in}}%
\pgfpathlineto{\pgfqpoint{0.635261in}{0.586332in}}%
\pgfpathlineto{\pgfqpoint{0.636341in}{0.592107in}}%
\pgfpathlineto{\pgfqpoint{0.636564in}{0.593209in}}%
\pgfpathlineto{\pgfqpoint{0.637656in}{0.600349in}}%
\pgfpathlineto{\pgfqpoint{0.637844in}{0.601347in}}%
\pgfpathlineto{\pgfqpoint{0.638948in}{0.607174in}}%
\pgfpathlineto{\pgfqpoint{0.639066in}{0.608277in}}%
\pgfpathlineto{\pgfqpoint{0.640134in}{0.614472in}}%
\pgfpathlineto{\pgfqpoint{0.640510in}{0.615574in}}%
\pgfpathlineto{\pgfqpoint{0.641590in}{0.622609in}}%
\pgfpathlineto{\pgfqpoint{0.641790in}{0.623712in}}%
\pgfpathlineto{\pgfqpoint{0.642847in}{0.631639in}}%
\pgfpathlineto{\pgfqpoint{0.643105in}{0.632637in}}%
\pgfpathlineto{\pgfqpoint{0.644209in}{0.641457in}}%
\pgfpathlineto{\pgfqpoint{0.644373in}{0.642559in}}%
\pgfpathlineto{\pgfqpoint{0.645477in}{0.650067in}}%
\pgfpathlineto{\pgfqpoint{0.645559in}{0.650644in}}%
\pgfpathlineto{\pgfqpoint{0.646651in}{0.658414in}}%
\pgfpathlineto{\pgfqpoint{0.646839in}{0.659464in}}%
\pgfpathlineto{\pgfqpoint{0.647943in}{0.665764in}}%
\pgfpathlineto{\pgfqpoint{0.648119in}{0.666814in}}%
\pgfpathlineto{\pgfqpoint{0.649223in}{0.674322in}}%
\pgfpathlineto{\pgfqpoint{0.649399in}{0.675424in}}%
\pgfpathlineto{\pgfqpoint{0.650503in}{0.684244in}}%
\pgfpathlineto{\pgfqpoint{0.650667in}{0.685189in}}%
\pgfpathlineto{\pgfqpoint{0.651736in}{0.692802in}}%
\pgfpathlineto{\pgfqpoint{0.651982in}{0.693747in}}%
\pgfpathlineto{\pgfqpoint{0.653086in}{0.702882in}}%
\pgfpathlineto{\pgfqpoint{0.653333in}{0.703932in}}%
\pgfpathlineto{\pgfqpoint{0.654425in}{0.711807in}}%
\pgfpathlineto{\pgfqpoint{0.654519in}{0.712332in}}%
\pgfpathlineto{\pgfqpoint{0.655622in}{0.721204in}}%
\pgfpathlineto{\pgfqpoint{0.655763in}{0.722202in}}%
\pgfpathlineto{\pgfqpoint{0.656855in}{0.729184in}}%
\pgfpathlineto{\pgfqpoint{0.657043in}{0.730234in}}%
\pgfpathlineto{\pgfqpoint{0.658135in}{0.738424in}}%
\pgfpathlineto{\pgfqpoint{0.658194in}{0.739369in}}%
\pgfpathlineto{\pgfqpoint{0.659298in}{0.747874in}}%
\pgfpathlineto{\pgfqpoint{0.659474in}{0.748819in}}%
\pgfpathlineto{\pgfqpoint{0.660578in}{0.757954in}}%
\pgfpathlineto{\pgfqpoint{0.660754in}{0.759057in}}%
\pgfpathlineto{\pgfqpoint{0.661846in}{0.765357in}}%
\pgfpathlineto{\pgfqpoint{0.662104in}{0.766302in}}%
\pgfpathlineto{\pgfqpoint{0.663208in}{0.774807in}}%
\pgfpathlineto{\pgfqpoint{0.663349in}{0.775909in}}%
\pgfpathlineto{\pgfqpoint{0.664453in}{0.782367in}}%
\pgfpathlineto{\pgfqpoint{0.664758in}{0.783417in}}%
\pgfpathlineto{\pgfqpoint{0.665862in}{0.791292in}}%
\pgfpathlineto{\pgfqpoint{0.666179in}{0.792237in}}%
\pgfpathlineto{\pgfqpoint{0.667283in}{0.800059in}}%
\pgfpathlineto{\pgfqpoint{0.667506in}{0.801162in}}%
\pgfpathlineto{\pgfqpoint{0.668609in}{0.808039in}}%
\pgfpathlineto{\pgfqpoint{0.668868in}{0.809089in}}%
\pgfpathlineto{\pgfqpoint{0.669972in}{0.815337in}}%
\pgfpathlineto{\pgfqpoint{0.670066in}{0.816387in}}%
\pgfpathlineto{\pgfqpoint{0.671169in}{0.823107in}}%
\pgfpathlineto{\pgfqpoint{0.671439in}{0.824209in}}%
\pgfpathlineto{\pgfqpoint{0.672543in}{0.833134in}}%
\pgfpathlineto{\pgfqpoint{0.672731in}{0.834237in}}%
\pgfpathlineto{\pgfqpoint{0.673823in}{0.840747in}}%
\pgfpathlineto{\pgfqpoint{0.674011in}{0.841797in}}%
\pgfpathlineto{\pgfqpoint{0.675115in}{0.846942in}}%
\pgfpathlineto{\pgfqpoint{0.675338in}{0.847992in}}%
\pgfpathlineto{\pgfqpoint{0.676442in}{0.853872in}}%
\pgfpathlineto{\pgfqpoint{0.676583in}{0.854974in}}%
\pgfpathlineto{\pgfqpoint{0.677663in}{0.861012in}}%
\pgfpathlineto{\pgfqpoint{0.677992in}{0.862009in}}%
\pgfpathlineto{\pgfqpoint{0.679072in}{0.869307in}}%
\pgfpathlineto{\pgfqpoint{0.679389in}{0.870409in}}%
\pgfpathlineto{\pgfqpoint{0.680493in}{0.877077in}}%
\pgfpathlineto{\pgfqpoint{0.680704in}{0.878127in}}%
\pgfpathlineto{\pgfqpoint{0.681773in}{0.882852in}}%
\pgfpathlineto{\pgfqpoint{0.681996in}{0.883954in}}%
\pgfpathlineto{\pgfqpoint{0.683100in}{0.890622in}}%
\pgfpathlineto{\pgfqpoint{0.683511in}{0.891672in}}%
\pgfpathlineto{\pgfqpoint{0.684603in}{0.896607in}}%
\pgfpathlineto{\pgfqpoint{0.684779in}{0.897657in}}%
\pgfpathlineto{\pgfqpoint{0.685836in}{0.903379in}}%
\pgfpathlineto{\pgfqpoint{0.686153in}{0.904482in}}%
\pgfpathlineto{\pgfqpoint{0.687256in}{0.909837in}}%
\pgfpathlineto{\pgfqpoint{0.687573in}{0.910939in}}%
\pgfpathlineto{\pgfqpoint{0.688572in}{0.915874in}}%
\pgfpathlineto{\pgfqpoint{0.688889in}{0.916977in}}%
\pgfpathlineto{\pgfqpoint{0.689992in}{0.922647in}}%
\pgfpathlineto{\pgfqpoint{0.690157in}{0.923749in}}%
\pgfpathlineto{\pgfqpoint{0.691261in}{0.930259in}}%
\pgfpathlineto{\pgfqpoint{0.691589in}{0.931362in}}%
\pgfpathlineto{\pgfqpoint{0.692670in}{0.936717in}}%
\pgfpathlineto{\pgfqpoint{0.692916in}{0.937819in}}%
\pgfpathlineto{\pgfqpoint{0.694020in}{0.942859in}}%
\pgfpathlineto{\pgfqpoint{0.694443in}{0.943962in}}%
\pgfpathlineto{\pgfqpoint{0.695500in}{0.948477in}}%
\pgfpathlineto{\pgfqpoint{0.695758in}{0.949579in}}%
\pgfpathlineto{\pgfqpoint{0.696862in}{0.954567in}}%
\pgfpathlineto{\pgfqpoint{0.697202in}{0.955617in}}%
\pgfpathlineto{\pgfqpoint{0.698306in}{0.961182in}}%
\pgfpathlineto{\pgfqpoint{0.698752in}{0.962232in}}%
\pgfpathlineto{\pgfqpoint{0.699856in}{0.966799in}}%
\pgfpathlineto{\pgfqpoint{0.700114in}{0.967902in}}%
\pgfpathlineto{\pgfqpoint{0.701218in}{0.972889in}}%
\pgfpathlineto{\pgfqpoint{0.701441in}{0.973939in}}%
\pgfpathlineto{\pgfqpoint{0.702545in}{0.978087in}}%
\pgfpathlineto{\pgfqpoint{0.702921in}{0.979189in}}%
\pgfpathlineto{\pgfqpoint{0.703989in}{0.983652in}}%
\pgfpathlineto{\pgfqpoint{0.704342in}{0.984754in}}%
\pgfpathlineto{\pgfqpoint{0.705434in}{0.989322in}}%
\pgfpathlineto{\pgfqpoint{0.705680in}{0.990424in}}%
\pgfpathlineto{\pgfqpoint{0.706772in}{0.995674in}}%
\pgfpathlineto{\pgfqpoint{0.707101in}{0.996724in}}%
\pgfpathlineto{\pgfqpoint{0.708205in}{1.001502in}}%
\pgfpathlineto{\pgfqpoint{0.708545in}{1.002604in}}%
\pgfpathlineto{\pgfqpoint{0.709649in}{1.006594in}}%
\pgfpathlineto{\pgfqpoint{0.709978in}{1.007592in}}%
\pgfpathlineto{\pgfqpoint{0.711023in}{1.010637in}}%
\pgfpathlineto{\pgfqpoint{0.711434in}{1.011739in}}%
\pgfpathlineto{\pgfqpoint{0.712503in}{1.016149in}}%
\pgfpathlineto{\pgfqpoint{0.712867in}{1.017252in}}%
\pgfpathlineto{\pgfqpoint{0.713970in}{1.020822in}}%
\pgfpathlineto{\pgfqpoint{0.714323in}{1.021819in}}%
\pgfpathlineto{\pgfqpoint{0.715427in}{1.025127in}}%
\pgfpathlineto{\pgfqpoint{0.715884in}{1.026124in}}%
\pgfpathlineto{\pgfqpoint{0.716977in}{1.029957in}}%
\pgfpathlineto{\pgfqpoint{0.717247in}{1.030954in}}%
\pgfpathlineto{\pgfqpoint{0.718339in}{1.036624in}}%
\pgfpathlineto{\pgfqpoint{0.718573in}{1.037674in}}%
\pgfpathlineto{\pgfqpoint{0.719677in}{1.041664in}}%
\pgfpathlineto{\pgfqpoint{0.720147in}{1.042767in}}%
\pgfpathlineto{\pgfqpoint{0.721227in}{1.045707in}}%
\pgfpathlineto{\pgfqpoint{0.721580in}{1.046757in}}%
\pgfpathlineto{\pgfqpoint{0.722660in}{1.050222in}}%
\pgfpathlineto{\pgfqpoint{0.723047in}{1.051324in}}%
\pgfpathlineto{\pgfqpoint{0.724139in}{1.054789in}}%
\pgfpathlineto{\pgfqpoint{0.724644in}{1.055892in}}%
\pgfpathlineto{\pgfqpoint{0.725736in}{1.059777in}}%
\pgfpathlineto{\pgfqpoint{0.726159in}{1.060827in}}%
\pgfpathlineto{\pgfqpoint{0.727263in}{1.063924in}}%
\pgfpathlineto{\pgfqpoint{0.727603in}{1.064974in}}%
\pgfpathlineto{\pgfqpoint{0.728672in}{1.068124in}}%
\pgfpathlineto{\pgfqpoint{0.729200in}{1.069227in}}%
\pgfpathlineto{\pgfqpoint{0.730304in}{1.072167in}}%
\pgfpathlineto{\pgfqpoint{0.730551in}{1.073269in}}%
\pgfpathlineto{\pgfqpoint{0.731631in}{1.076157in}}%
\pgfpathlineto{\pgfqpoint{0.732054in}{1.077259in}}%
\pgfpathlineto{\pgfqpoint{0.733122in}{1.080567in}}%
\pgfpathlineto{\pgfqpoint{0.733756in}{1.081669in}}%
\pgfpathlineto{\pgfqpoint{0.734848in}{1.084662in}}%
\pgfpathlineto{\pgfqpoint{0.735166in}{1.085764in}}%
\pgfpathlineto{\pgfqpoint{0.736258in}{1.089229in}}%
\pgfpathlineto{\pgfqpoint{0.736739in}{1.090332in}}%
\pgfpathlineto{\pgfqpoint{0.737808in}{1.093272in}}%
\pgfpathlineto{\pgfqpoint{0.738442in}{1.094374in}}%
\pgfpathlineto{\pgfqpoint{0.739522in}{1.098049in}}%
\pgfpathlineto{\pgfqpoint{0.739921in}{1.099152in}}%
\pgfpathlineto{\pgfqpoint{0.741025in}{1.101987in}}%
\pgfpathlineto{\pgfqpoint{0.741636in}{1.103089in}}%
\pgfpathlineto{\pgfqpoint{0.742739in}{1.105872in}}%
\pgfpathlineto{\pgfqpoint{0.743268in}{1.106974in}}%
\pgfpathlineto{\pgfqpoint{0.744360in}{1.109704in}}%
\pgfpathlineto{\pgfqpoint{0.744712in}{1.110754in}}%
\pgfpathlineto{\pgfqpoint{0.744747in}{1.110912in}}%
\pgfpathlineto{\pgfqpoint{0.744759in}{1.110912in}}%
\pgfpathlineto{\pgfqpoint{0.755363in}{1.112014in}}%
\pgfpathlineto{\pgfqpoint{0.756466in}{1.114377in}}%
\pgfpathlineto{\pgfqpoint{0.756772in}{1.115479in}}%
\pgfpathlineto{\pgfqpoint{0.757875in}{1.119627in}}%
\pgfpathlineto{\pgfqpoint{0.758592in}{1.120729in}}%
\pgfpathlineto{\pgfqpoint{0.759672in}{1.122882in}}%
\pgfpathlineto{\pgfqpoint{0.760153in}{1.123984in}}%
\pgfpathlineto{\pgfqpoint{0.761234in}{1.126819in}}%
\pgfpathlineto{\pgfqpoint{0.761727in}{1.127922in}}%
\pgfpathlineto{\pgfqpoint{0.762831in}{1.131387in}}%
\pgfpathlineto{\pgfqpoint{0.763312in}{1.132437in}}%
\pgfpathlineto{\pgfqpoint{0.764416in}{1.135219in}}%
\pgfpathlineto{\pgfqpoint{0.764897in}{1.136217in}}%
\pgfpathlineto{\pgfqpoint{0.765989in}{1.139104in}}%
\pgfpathlineto{\pgfqpoint{0.766389in}{1.140154in}}%
\pgfpathlineto{\pgfqpoint{0.767492in}{1.143199in}}%
\pgfpathlineto{\pgfqpoint{0.767997in}{1.144302in}}%
\pgfpathlineto{\pgfqpoint{0.769019in}{1.146349in}}%
\pgfpathlineto{\pgfqpoint{0.769547in}{1.147452in}}%
\pgfpathlineto{\pgfqpoint{0.770639in}{1.150182in}}%
\pgfpathlineto{\pgfqpoint{0.771273in}{1.151284in}}%
\pgfpathlineto{\pgfqpoint{0.772330in}{1.154014in}}%
\pgfpathlineto{\pgfqpoint{0.773129in}{1.155117in}}%
\pgfpathlineto{\pgfqpoint{0.774174in}{1.158004in}}%
\pgfpathlineto{\pgfqpoint{0.774855in}{1.159054in}}%
\pgfpathlineto{\pgfqpoint{0.775959in}{1.161364in}}%
\pgfpathlineto{\pgfqpoint{0.776170in}{1.162467in}}%
\pgfpathlineto{\pgfqpoint{0.777274in}{1.165197in}}%
\pgfpathlineto{\pgfqpoint{0.777779in}{1.166299in}}%
\pgfpathlineto{\pgfqpoint{0.778883in}{1.168819in}}%
\pgfpathlineto{\pgfqpoint{0.779481in}{1.169922in}}%
\pgfpathlineto{\pgfqpoint{0.780585in}{1.171339in}}%
\pgfpathlineto{\pgfqpoint{0.781008in}{1.172389in}}%
\pgfpathlineto{\pgfqpoint{0.782065in}{1.175119in}}%
\pgfpathlineto{\pgfqpoint{0.782581in}{1.176169in}}%
\pgfpathlineto{\pgfqpoint{0.783673in}{1.177797in}}%
\pgfpathlineto{\pgfqpoint{0.784261in}{1.178899in}}%
\pgfpathlineto{\pgfqpoint{0.785247in}{1.180684in}}%
\pgfpathlineto{\pgfqpoint{0.785822in}{1.181787in}}%
\pgfpathlineto{\pgfqpoint{0.786903in}{1.183887in}}%
\pgfpathlineto{\pgfqpoint{0.787830in}{1.184937in}}%
\pgfpathlineto{\pgfqpoint{0.788817in}{1.186879in}}%
\pgfpathlineto{\pgfqpoint{0.789556in}{1.187982in}}%
\pgfpathlineto{\pgfqpoint{0.790519in}{1.190029in}}%
\pgfpathlineto{\pgfqpoint{0.791353in}{1.191132in}}%
\pgfpathlineto{\pgfqpoint{0.792445in}{1.193442in}}%
\pgfpathlineto{\pgfqpoint{0.792997in}{1.194544in}}%
\pgfpathlineto{\pgfqpoint{0.794054in}{1.196067in}}%
\pgfpathlineto{\pgfqpoint{0.794876in}{1.197169in}}%
\pgfpathlineto{\pgfqpoint{0.795886in}{1.198482in}}%
\pgfpathlineto{\pgfqpoint{0.797201in}{1.199584in}}%
\pgfpathlineto{\pgfqpoint{0.798258in}{1.201579in}}%
\pgfpathlineto{\pgfqpoint{0.798739in}{1.202682in}}%
\pgfpathlineto{\pgfqpoint{0.799843in}{1.204572in}}%
\pgfpathlineto{\pgfqpoint{0.800195in}{1.205674in}}%
\pgfpathlineto{\pgfqpoint{0.801287in}{1.207564in}}%
\pgfpathlineto{\pgfqpoint{0.801874in}{1.208667in}}%
\pgfpathlineto{\pgfqpoint{0.802966in}{1.211239in}}%
\pgfpathlineto{\pgfqpoint{0.803882in}{1.212342in}}%
\pgfpathlineto{\pgfqpoint{0.804986in}{1.214337in}}%
\pgfpathlineto{\pgfqpoint{0.805714in}{1.215439in}}%
\pgfpathlineto{\pgfqpoint{0.806759in}{1.217277in}}%
\pgfpathlineto{\pgfqpoint{0.807111in}{1.218274in}}%
\pgfpathlineto{\pgfqpoint{0.808180in}{1.221109in}}%
\pgfpathlineto{\pgfqpoint{0.809084in}{1.222212in}}%
\pgfpathlineto{\pgfqpoint{0.810176in}{1.224049in}}%
\pgfpathlineto{\pgfqpoint{0.810892in}{1.225152in}}%
\pgfpathlineto{\pgfqpoint{0.811938in}{1.226517in}}%
\pgfpathlineto{\pgfqpoint{0.812478in}{1.227567in}}%
\pgfpathlineto{\pgfqpoint{0.813570in}{1.229614in}}%
\pgfpathlineto{\pgfqpoint{0.814122in}{1.230717in}}%
\pgfpathlineto{\pgfqpoint{0.815225in}{1.233027in}}%
\pgfpathlineto{\pgfqpoint{0.815942in}{1.234129in}}%
\pgfpathlineto{\pgfqpoint{0.816940in}{1.235809in}}%
\pgfpathlineto{\pgfqpoint{0.817820in}{1.236859in}}%
\pgfpathlineto{\pgfqpoint{0.818889in}{1.238697in}}%
\pgfpathlineto{\pgfqpoint{0.819429in}{1.239799in}}%
\pgfpathlineto{\pgfqpoint{0.820486in}{1.241689in}}%
\pgfpathlineto{\pgfqpoint{0.820897in}{1.242792in}}%
\pgfpathlineto{\pgfqpoint{0.821977in}{1.244367in}}%
\pgfpathlineto{\pgfqpoint{0.822881in}{1.245469in}}%
\pgfpathlineto{\pgfqpoint{0.823985in}{1.247464in}}%
\pgfpathlineto{\pgfqpoint{0.824655in}{1.248567in}}%
\pgfpathlineto{\pgfqpoint{0.825711in}{1.249827in}}%
\pgfpathlineto{\pgfqpoint{0.826369in}{1.250929in}}%
\pgfpathlineto{\pgfqpoint{0.827438in}{1.252767in}}%
\pgfpathlineto{\pgfqpoint{0.828001in}{1.253869in}}%
\pgfpathlineto{\pgfqpoint{0.829105in}{1.255812in}}%
\pgfpathlineto{\pgfqpoint{0.829845in}{1.256914in}}%
\pgfpathlineto{\pgfqpoint{0.830948in}{1.258804in}}%
\pgfpathlineto{\pgfqpoint{0.831712in}{1.259907in}}%
\pgfpathlineto{\pgfqpoint{0.832816in}{1.261797in}}%
\pgfpathlineto{\pgfqpoint{0.833861in}{1.262899in}}%
\pgfpathlineto{\pgfqpoint{0.834953in}{1.264422in}}%
\pgfpathlineto{\pgfqpoint{0.835739in}{1.265472in}}%
\pgfpathlineto{\pgfqpoint{0.836784in}{1.267257in}}%
\pgfpathlineto{\pgfqpoint{0.837477in}{1.268359in}}%
\pgfpathlineto{\pgfqpoint{0.838558in}{1.269724in}}%
\pgfpathlineto{\pgfqpoint{0.839920in}{1.270774in}}%
\pgfpathlineto{\pgfqpoint{0.841023in}{1.272664in}}%
\pgfpathlineto{\pgfqpoint{0.841975in}{1.273767in}}%
\pgfpathlineto{\pgfqpoint{0.843055in}{1.274869in}}%
\pgfpathlineto{\pgfqpoint{0.844217in}{1.275972in}}%
\pgfpathlineto{\pgfqpoint{0.845298in}{1.277809in}}%
\pgfpathlineto{\pgfqpoint{0.846002in}{1.278912in}}%
\pgfpathlineto{\pgfqpoint{0.847094in}{1.280172in}}%
\pgfpathlineto{\pgfqpoint{0.847963in}{1.281274in}}%
\pgfpathlineto{\pgfqpoint{0.849067in}{1.282902in}}%
\pgfpathlineto{\pgfqpoint{0.850124in}{1.284004in}}%
\pgfpathlineto{\pgfqpoint{0.851228in}{1.285474in}}%
\pgfpathlineto{\pgfqpoint{0.852472in}{1.286577in}}%
\pgfpathlineto{\pgfqpoint{0.853576in}{1.288624in}}%
\pgfpathlineto{\pgfqpoint{0.854645in}{1.289727in}}%
\pgfpathlineto{\pgfqpoint{0.855690in}{1.291197in}}%
\pgfpathlineto{\pgfqpoint{0.856934in}{1.292299in}}%
\pgfpathlineto{\pgfqpoint{0.857874in}{1.293297in}}%
\pgfpathlineto{\pgfqpoint{0.858766in}{1.294399in}}%
\pgfpathlineto{\pgfqpoint{0.859858in}{1.295817in}}%
\pgfpathlineto{\pgfqpoint{0.860868in}{1.296919in}}%
\pgfpathlineto{\pgfqpoint{0.861960in}{1.298179in}}%
\pgfpathlineto{\pgfqpoint{0.863205in}{1.299282in}}%
\pgfpathlineto{\pgfqpoint{0.864285in}{1.300962in}}%
\pgfpathlineto{\pgfqpoint{0.864837in}{1.302064in}}%
\pgfpathlineto{\pgfqpoint{0.865894in}{1.303482in}}%
\pgfpathlineto{\pgfqpoint{0.867009in}{1.304584in}}%
\pgfpathlineto{\pgfqpoint{0.868113in}{1.305582in}}%
\pgfpathlineto{\pgfqpoint{0.869205in}{1.306684in}}%
\pgfpathlineto{\pgfqpoint{0.870286in}{1.307682in}}%
\pgfpathlineto{\pgfqpoint{0.871131in}{1.308784in}}%
\pgfpathlineto{\pgfqpoint{0.872211in}{1.310464in}}%
\pgfpathlineto{\pgfqpoint{0.873092in}{1.311567in}}%
\pgfpathlineto{\pgfqpoint{0.874196in}{1.312932in}}%
\pgfpathlineto{\pgfqpoint{0.875523in}{1.314034in}}%
\pgfpathlineto{\pgfqpoint{0.876568in}{1.315189in}}%
\pgfpathlineto{\pgfqpoint{0.878141in}{1.316292in}}%
\pgfpathlineto{\pgfqpoint{0.879245in}{1.317919in}}%
\pgfpathlineto{\pgfqpoint{0.880513in}{1.319022in}}%
\pgfpathlineto{\pgfqpoint{0.881570in}{1.319967in}}%
\pgfpathlineto{\pgfqpoint{0.882521in}{1.321069in}}%
\pgfpathlineto{\pgfqpoint{0.883566in}{1.321962in}}%
\pgfpathlineto{\pgfqpoint{0.884564in}{1.323064in}}%
\pgfpathlineto{\pgfqpoint{0.885656in}{1.324639in}}%
\pgfpathlineto{\pgfqpoint{0.886455in}{1.325742in}}%
\pgfpathlineto{\pgfqpoint{0.887559in}{1.327054in}}%
\pgfpathlineto{\pgfqpoint{0.889250in}{1.328157in}}%
\pgfpathlineto{\pgfqpoint{0.890342in}{1.328944in}}%
\pgfpathlineto{\pgfqpoint{0.891316in}{1.330047in}}%
\pgfpathlineto{\pgfqpoint{0.892408in}{1.331307in}}%
\pgfpathlineto{\pgfqpoint{0.893864in}{1.332409in}}%
\pgfpathlineto{\pgfqpoint{0.894956in}{1.333774in}}%
\pgfpathlineto{\pgfqpoint{0.896166in}{1.334877in}}%
\pgfpathlineto{\pgfqpoint{0.897246in}{1.336032in}}%
\pgfpathlineto{\pgfqpoint{0.898667in}{1.337134in}}%
\pgfpathlineto{\pgfqpoint{0.899712in}{1.338237in}}%
\pgfpathlineto{\pgfqpoint{0.901520in}{1.339339in}}%
\pgfpathlineto{\pgfqpoint{0.902577in}{1.340284in}}%
\pgfpathlineto{\pgfqpoint{0.903446in}{1.341334in}}%
\pgfpathlineto{\pgfqpoint{0.904538in}{1.342594in}}%
\pgfpathlineto{\pgfqpoint{0.906065in}{1.343697in}}%
\pgfpathlineto{\pgfqpoint{0.907133in}{1.344484in}}%
\pgfpathlineto{\pgfqpoint{0.908625in}{1.345534in}}%
\pgfpathlineto{\pgfqpoint{0.909576in}{1.346322in}}%
\pgfpathlineto{\pgfqpoint{0.911314in}{1.347372in}}%
\pgfpathlineto{\pgfqpoint{0.912406in}{1.348894in}}%
\pgfpathlineto{\pgfqpoint{0.913674in}{1.349997in}}%
\pgfpathlineto{\pgfqpoint{0.914742in}{1.351047in}}%
\pgfpathlineto{\pgfqpoint{0.916281in}{1.352149in}}%
\pgfpathlineto{\pgfqpoint{0.917338in}{1.353042in}}%
\pgfpathlineto{\pgfqpoint{0.918993in}{1.354144in}}%
\pgfpathlineto{\pgfqpoint{0.920073in}{1.355037in}}%
\pgfpathlineto{\pgfqpoint{0.921330in}{1.356139in}}%
\pgfpathlineto{\pgfqpoint{0.922434in}{1.357662in}}%
\pgfpathlineto{\pgfqpoint{0.924042in}{1.358764in}}%
\pgfpathlineto{\pgfqpoint{0.925123in}{1.359814in}}%
\pgfpathlineto{\pgfqpoint{0.927354in}{1.360917in}}%
\pgfpathlineto{\pgfqpoint{0.928422in}{1.361967in}}%
\pgfpathlineto{\pgfqpoint{0.930242in}{1.363069in}}%
\pgfpathlineto{\pgfqpoint{0.931334in}{1.364277in}}%
\pgfpathlineto{\pgfqpoint{0.932673in}{1.365327in}}%
\pgfpathlineto{\pgfqpoint{0.933718in}{1.366534in}}%
\pgfpathlineto{\pgfqpoint{0.935397in}{1.367637in}}%
\pgfpathlineto{\pgfqpoint{0.936454in}{1.368477in}}%
\pgfpathlineto{\pgfqpoint{0.937898in}{1.369579in}}%
\pgfpathlineto{\pgfqpoint{0.938908in}{1.370262in}}%
\pgfpathlineto{\pgfqpoint{0.940822in}{1.371364in}}%
\pgfpathlineto{\pgfqpoint{0.941691in}{1.371994in}}%
\pgfpathlineto{\pgfqpoint{0.943417in}{1.373097in}}%
\pgfpathlineto{\pgfqpoint{0.944427in}{1.373937in}}%
\pgfpathlineto{\pgfqpoint{0.945754in}{1.375039in}}%
\pgfpathlineto{\pgfqpoint{0.946799in}{1.375617in}}%
\pgfpathlineto{\pgfqpoint{0.948408in}{1.376719in}}%
\pgfpathlineto{\pgfqpoint{0.949465in}{1.377717in}}%
\pgfpathlineto{\pgfqpoint{0.951203in}{1.378819in}}%
\pgfpathlineto{\pgfqpoint{0.952271in}{1.379554in}}%
\pgfpathlineto{\pgfqpoint{0.953504in}{1.380604in}}%
\pgfpathlineto{\pgfqpoint{0.954573in}{1.381182in}}%
\pgfpathlineto{\pgfqpoint{0.955970in}{1.382179in}}%
\pgfpathlineto{\pgfqpoint{0.957003in}{1.383124in}}%
\pgfpathlineto{\pgfqpoint{0.959105in}{1.384227in}}%
\pgfpathlineto{\pgfqpoint{0.960162in}{1.384909in}}%
\pgfpathlineto{\pgfqpoint{0.962217in}{1.386012in}}%
\pgfpathlineto{\pgfqpoint{0.963297in}{1.386694in}}%
\pgfpathlineto{\pgfqpoint{0.965575in}{1.387797in}}%
\pgfpathlineto{\pgfqpoint{0.966609in}{1.388374in}}%
\pgfpathlineto{\pgfqpoint{0.968030in}{1.389372in}}%
\pgfpathlineto{\pgfqpoint{0.969122in}{1.390422in}}%
\pgfpathlineto{\pgfqpoint{0.971423in}{1.391524in}}%
\pgfpathlineto{\pgfqpoint{0.972527in}{1.392522in}}%
\pgfpathlineto{\pgfqpoint{0.974652in}{1.393624in}}%
\pgfpathlineto{\pgfqpoint{0.975697in}{1.394307in}}%
\pgfpathlineto{\pgfqpoint{0.978069in}{1.395409in}}%
\pgfpathlineto{\pgfqpoint{0.979150in}{1.396092in}}%
\pgfpathlineto{\pgfqpoint{0.980782in}{1.397194in}}%
\pgfpathlineto{\pgfqpoint{0.981862in}{1.397824in}}%
\pgfpathlineto{\pgfqpoint{0.984093in}{1.398822in}}%
\pgfpathlineto{\pgfqpoint{0.985173in}{1.399767in}}%
\pgfpathlineto{\pgfqpoint{0.987581in}{1.400869in}}%
\pgfpathlineto{\pgfqpoint{0.988649in}{1.401342in}}%
\pgfpathlineto{\pgfqpoint{0.991139in}{1.402444in}}%
\pgfpathlineto{\pgfqpoint{0.992242in}{1.402864in}}%
\pgfpathlineto{\pgfqpoint{0.993816in}{1.403914in}}%
\pgfpathlineto{\pgfqpoint{0.994697in}{1.404807in}}%
\pgfpathlineto{\pgfqpoint{0.996951in}{1.405909in}}%
\pgfpathlineto{\pgfqpoint{0.998055in}{1.406749in}}%
\pgfpathlineto{\pgfqpoint{1.000216in}{1.407852in}}%
\pgfpathlineto{\pgfqpoint{1.001284in}{1.408377in}}%
\pgfpathlineto{\pgfqpoint{1.003069in}{1.409427in}}%
\pgfpathlineto{\pgfqpoint{1.004044in}{1.410214in}}%
\pgfpathlineto{\pgfqpoint{1.005746in}{1.411264in}}%
\pgfpathlineto{\pgfqpoint{1.006838in}{1.412262in}}%
\pgfpathlineto{\pgfqpoint{1.008858in}{1.413364in}}%
\pgfpathlineto{\pgfqpoint{1.009962in}{1.413837in}}%
\pgfpathlineto{\pgfqpoint{1.012909in}{1.414939in}}%
\pgfpathlineto{\pgfqpoint{1.013790in}{1.415307in}}%
\pgfpathlineto{\pgfqpoint{1.015786in}{1.416357in}}%
\pgfpathlineto{\pgfqpoint{1.016855in}{1.417092in}}%
\pgfpathlineto{\pgfqpoint{1.018604in}{1.418194in}}%
\pgfpathlineto{\pgfqpoint{1.019708in}{1.419087in}}%
\pgfpathlineto{\pgfqpoint{1.021857in}{1.420137in}}%
\pgfpathlineto{\pgfqpoint{1.022937in}{1.420767in}}%
\pgfpathlineto{\pgfqpoint{1.025673in}{1.421869in}}%
\pgfpathlineto{\pgfqpoint{1.026683in}{1.422657in}}%
\pgfpathlineto{\pgfqpoint{1.029219in}{1.423759in}}%
\pgfpathlineto{\pgfqpoint{1.030123in}{1.424599in}}%
\pgfpathlineto{\pgfqpoint{1.033071in}{1.425702in}}%
\pgfpathlineto{\pgfqpoint{1.034092in}{1.426647in}}%
\pgfpathlineto{\pgfqpoint{1.037333in}{1.427749in}}%
\pgfpathlineto{\pgfqpoint{1.038414in}{1.428484in}}%
\pgfpathlineto{\pgfqpoint{1.039917in}{1.429587in}}%
\pgfpathlineto{\pgfqpoint{1.040962in}{1.430217in}}%
\pgfpathlineto{\pgfqpoint{1.043486in}{1.431319in}}%
\pgfpathlineto{\pgfqpoint{1.044461in}{1.431897in}}%
\pgfpathlineto{\pgfqpoint{1.046622in}{1.432999in}}%
\pgfpathlineto{\pgfqpoint{1.047479in}{1.433629in}}%
\pgfpathlineto{\pgfqpoint{1.049498in}{1.434732in}}%
\pgfpathlineto{\pgfqpoint{1.050532in}{1.435204in}}%
\pgfpathlineto{\pgfqpoint{1.052035in}{1.436307in}}%
\pgfpathlineto{\pgfqpoint{1.052786in}{1.436989in}}%
\pgfpathlineto{\pgfqpoint{1.055922in}{1.438092in}}%
\pgfpathlineto{\pgfqpoint{1.056802in}{1.438354in}}%
\pgfpathlineto{\pgfqpoint{1.059573in}{1.439457in}}%
\pgfpathlineto{\pgfqpoint{1.060548in}{1.440139in}}%
\pgfpathlineto{\pgfqpoint{1.063178in}{1.441242in}}%
\pgfpathlineto{\pgfqpoint{1.064118in}{1.441872in}}%
\pgfpathlineto{\pgfqpoint{1.067370in}{1.442974in}}%
\pgfpathlineto{\pgfqpoint{1.068416in}{1.443499in}}%
\pgfpathlineto{\pgfqpoint{1.071116in}{1.444602in}}%
\pgfpathlineto{\pgfqpoint{1.072197in}{1.445284in}}%
\pgfpathlineto{\pgfqpoint{1.074498in}{1.446387in}}%
\pgfpathlineto{\pgfqpoint{1.075555in}{1.447122in}}%
\pgfpathlineto{\pgfqpoint{1.078573in}{1.448224in}}%
\pgfpathlineto{\pgfqpoint{1.079677in}{1.449064in}}%
\pgfpathlineto{\pgfqpoint{1.083575in}{1.450167in}}%
\pgfpathlineto{\pgfqpoint{1.084362in}{1.450482in}}%
\pgfpathlineto{\pgfqpoint{1.087485in}{1.451584in}}%
\pgfpathlineto{\pgfqpoint{1.088589in}{1.452267in}}%
\pgfpathlineto{\pgfqpoint{1.091442in}{1.453369in}}%
\pgfpathlineto{\pgfqpoint{1.092452in}{1.453842in}}%
\pgfpathlineto{\pgfqpoint{1.096339in}{1.454944in}}%
\pgfpathlineto{\pgfqpoint{1.097196in}{1.455312in}}%
\pgfpathlineto{\pgfqpoint{1.099721in}{1.456414in}}%
\pgfpathlineto{\pgfqpoint{1.100766in}{1.456939in}}%
\pgfpathlineto{\pgfqpoint{1.103936in}{1.458042in}}%
\pgfpathlineto{\pgfqpoint{1.104735in}{1.458252in}}%
\pgfpathlineto{\pgfqpoint{1.108716in}{1.459354in}}%
\pgfpathlineto{\pgfqpoint{1.109702in}{1.459774in}}%
\pgfpathlineto{\pgfqpoint{1.113553in}{1.460877in}}%
\pgfpathlineto{\pgfqpoint{1.114434in}{1.461297in}}%
\pgfpathlineto{\pgfqpoint{1.117346in}{1.462399in}}%
\pgfpathlineto{\pgfqpoint{1.118156in}{1.462819in}}%
\pgfpathlineto{\pgfqpoint{1.120834in}{1.463922in}}%
\pgfpathlineto{\pgfqpoint{1.121914in}{1.464499in}}%
\pgfpathlineto{\pgfqpoint{1.125437in}{1.465602in}}%
\pgfpathlineto{\pgfqpoint{1.126400in}{1.465917in}}%
\pgfpathlineto{\pgfqpoint{1.130545in}{1.467019in}}%
\pgfpathlineto{\pgfqpoint{1.131425in}{1.467334in}}%
\pgfpathlineto{\pgfqpoint{1.134044in}{1.468384in}}%
\pgfpathlineto{\pgfqpoint{1.134925in}{1.468804in}}%
\pgfpathlineto{\pgfqpoint{1.137438in}{1.469907in}}%
\pgfpathlineto{\pgfqpoint{1.138541in}{1.470642in}}%
\pgfpathlineto{\pgfqpoint{1.142557in}{1.471744in}}%
\pgfpathlineto{\pgfqpoint{1.143156in}{1.472007in}}%
\pgfpathlineto{\pgfqpoint{1.147923in}{1.473109in}}%
\pgfpathlineto{\pgfqpoint{1.148745in}{1.473529in}}%
\pgfpathlineto{\pgfqpoint{1.151740in}{1.474632in}}%
\pgfpathlineto{\pgfqpoint{1.152820in}{1.475052in}}%
\pgfpathlineto{\pgfqpoint{1.155944in}{1.476154in}}%
\pgfpathlineto{\pgfqpoint{1.157012in}{1.476627in}}%
\pgfpathlineto{\pgfqpoint{1.161263in}{1.477729in}}%
\pgfpathlineto{\pgfqpoint{1.162144in}{1.478254in}}%
\pgfpathlineto{\pgfqpoint{1.166430in}{1.479357in}}%
\pgfpathlineto{\pgfqpoint{1.167439in}{1.479934in}}%
\pgfpathlineto{\pgfqpoint{1.170845in}{1.481037in}}%
\pgfpathlineto{\pgfqpoint{1.171725in}{1.481352in}}%
\pgfpathlineto{\pgfqpoint{1.176340in}{1.482454in}}%
\pgfpathlineto{\pgfqpoint{1.177432in}{1.482769in}}%
\pgfpathlineto{\pgfqpoint{1.182305in}{1.483872in}}%
\pgfpathlineto{\pgfqpoint{1.183386in}{1.484397in}}%
\pgfpathlineto{\pgfqpoint{1.188047in}{1.485499in}}%
\pgfpathlineto{\pgfqpoint{1.189034in}{1.486077in}}%
\pgfpathlineto{\pgfqpoint{1.194095in}{1.487179in}}%
\pgfpathlineto{\pgfqpoint{1.194964in}{1.487494in}}%
\pgfpathlineto{\pgfqpoint{1.199484in}{1.488597in}}%
\pgfpathlineto{\pgfqpoint{1.200518in}{1.489227in}}%
\pgfpathlineto{\pgfqpoint{1.203418in}{1.490329in}}%
\pgfpathlineto{\pgfqpoint{1.204252in}{1.490697in}}%
\pgfpathlineto{\pgfqpoint{1.207211in}{1.491799in}}%
\pgfpathlineto{\pgfqpoint{1.207211in}{1.491852in}}%
\pgfpathlineto{\pgfqpoint{1.211791in}{1.492954in}}%
\pgfpathlineto{\pgfqpoint{1.212789in}{1.493322in}}%
\pgfpathlineto{\pgfqpoint{1.217192in}{1.494424in}}%
\pgfpathlineto{\pgfqpoint{1.217897in}{1.494634in}}%
\pgfpathlineto{\pgfqpoint{1.223005in}{1.495737in}}%
\pgfpathlineto{\pgfqpoint{1.223897in}{1.495842in}}%
\pgfpathlineto{\pgfqpoint{1.226621in}{1.496944in}}%
\pgfpathlineto{\pgfqpoint{1.227643in}{1.497469in}}%
\pgfpathlineto{\pgfqpoint{1.235322in}{1.498572in}}%
\pgfpathlineto{\pgfqpoint{1.236367in}{1.498729in}}%
\pgfpathlineto{\pgfqpoint{1.241757in}{1.499832in}}%
\pgfpathlineto{\pgfqpoint{1.242133in}{1.499989in}}%
\pgfpathlineto{\pgfqpoint{1.248063in}{1.501092in}}%
\pgfpathlineto{\pgfqpoint{1.248991in}{1.501354in}}%
\pgfpathlineto{\pgfqpoint{1.254028in}{1.502457in}}%
\pgfpathlineto{\pgfqpoint{1.255108in}{1.502614in}}%
\pgfpathlineto{\pgfqpoint{1.259946in}{1.503717in}}%
\pgfpathlineto{\pgfqpoint{1.260933in}{1.503927in}}%
\pgfpathlineto{\pgfqpoint{1.267696in}{1.505029in}}%
\pgfpathlineto{\pgfqpoint{1.268765in}{1.505344in}}%
\pgfpathlineto{\pgfqpoint{1.274248in}{1.506447in}}%
\pgfpathlineto{\pgfqpoint{1.274918in}{1.506552in}}%
\pgfpathlineto{\pgfqpoint{1.282128in}{1.507654in}}%
\pgfpathlineto{\pgfqpoint{1.283008in}{1.508127in}}%
\pgfpathlineto{\pgfqpoint{1.289420in}{1.509229in}}%
\pgfpathlineto{\pgfqpoint{1.290465in}{1.509649in}}%
\pgfpathlineto{\pgfqpoint{1.296723in}{1.510752in}}%
\pgfpathlineto{\pgfqpoint{1.297722in}{1.511014in}}%
\pgfpathlineto{\pgfqpoint{1.303393in}{1.512117in}}%
\pgfpathlineto{\pgfqpoint{1.304485in}{1.512379in}}%
\pgfpathlineto{\pgfqpoint{1.311084in}{1.513429in}}%
\pgfpathlineto{\pgfqpoint{1.311460in}{1.513587in}}%
\pgfpathlineto{\pgfqpoint{1.319445in}{1.514689in}}%
\pgfpathlineto{\pgfqpoint{1.320220in}{1.515109in}}%
\pgfpathlineto{\pgfqpoint{1.326338in}{1.516212in}}%
\pgfpathlineto{\pgfqpoint{1.327301in}{1.516474in}}%
\pgfpathlineto{\pgfqpoint{1.334041in}{1.517577in}}%
\pgfpathlineto{\pgfqpoint{1.334886in}{1.517839in}}%
\pgfpathlineto{\pgfqpoint{1.340664in}{1.518942in}}%
\pgfpathlineto{\pgfqpoint{1.340664in}{1.518994in}}%
\pgfpathlineto{\pgfqpoint{1.350304in}{1.520097in}}%
\pgfpathlineto{\pgfqpoint{1.351279in}{1.520464in}}%
\pgfpathlineto{\pgfqpoint{1.358007in}{1.521567in}}%
\pgfpathlineto{\pgfqpoint{1.358888in}{1.521724in}}%
\pgfpathlineto{\pgfqpoint{1.365264in}{1.522827in}}%
\pgfpathlineto{\pgfqpoint{1.366098in}{1.523037in}}%
\pgfpathlineto{\pgfqpoint{1.372697in}{1.524139in}}%
\pgfpathlineto{\pgfqpoint{1.373413in}{1.524349in}}%
\pgfpathlineto{\pgfqpoint{1.380964in}{1.525452in}}%
\pgfpathlineto{\pgfqpoint{1.382044in}{1.525977in}}%
\pgfpathlineto{\pgfqpoint{1.387222in}{1.527079in}}%
\pgfpathlineto{\pgfqpoint{1.387974in}{1.527237in}}%
\pgfpathlineto{\pgfqpoint{1.394338in}{1.528339in}}%
\pgfpathlineto{\pgfqpoint{1.395442in}{1.528654in}}%
\pgfpathlineto{\pgfqpoint{1.400174in}{1.529757in}}%
\pgfpathlineto{\pgfqpoint{1.400527in}{1.529914in}}%
\pgfpathlineto{\pgfqpoint{1.408605in}{1.531017in}}%
\pgfpathlineto{\pgfqpoint{1.409615in}{1.531227in}}%
\pgfpathlineto{\pgfqpoint{1.417201in}{1.532329in}}%
\pgfpathlineto{\pgfqpoint{1.418258in}{1.532539in}}%
\pgfpathlineto{\pgfqpoint{1.423964in}{1.533642in}}%
\pgfpathlineto{\pgfqpoint{1.424822in}{1.533852in}}%
\pgfpathlineto{\pgfqpoint{1.432008in}{1.534954in}}%
\pgfpathlineto{\pgfqpoint{1.432266in}{1.535112in}}%
\pgfpathlineto{\pgfqpoint{1.442353in}{1.536214in}}%
\pgfpathlineto{\pgfqpoint{1.442564in}{1.536319in}}%
\pgfpathlineto{\pgfqpoint{1.449504in}{1.537422in}}%
\pgfpathlineto{\pgfqpoint{1.450267in}{1.537842in}}%
\pgfpathlineto{\pgfqpoint{1.458534in}{1.538944in}}%
\pgfpathlineto{\pgfqpoint{1.459356in}{1.539102in}}%
\pgfpathlineto{\pgfqpoint{1.473083in}{1.540204in}}%
\pgfpathlineto{\pgfqpoint{1.473306in}{1.540362in}}%
\pgfpathlineto{\pgfqpoint{1.482829in}{1.541464in}}%
\pgfpathlineto{\pgfqpoint{1.483710in}{1.541569in}}%
\pgfpathlineto{\pgfqpoint{1.495405in}{1.542672in}}%
\pgfpathlineto{\pgfqpoint{1.496157in}{1.542934in}}%
\pgfpathlineto{\pgfqpoint{1.504470in}{1.544037in}}%
\pgfpathlineto{\pgfqpoint{1.505410in}{1.544299in}}%
\pgfpathlineto{\pgfqpoint{1.515121in}{1.545402in}}%
\pgfpathlineto{\pgfqpoint{1.516213in}{1.545559in}}%
\pgfpathlineto{\pgfqpoint{1.523775in}{1.546662in}}%
\pgfpathlineto{\pgfqpoint{1.524515in}{1.546819in}}%
\pgfpathlineto{\pgfqpoint{1.532006in}{1.547922in}}%
\pgfpathlineto{\pgfqpoint{1.532958in}{1.548079in}}%
\pgfpathlineto{\pgfqpoint{1.541835in}{1.549182in}}%
\pgfpathlineto{\pgfqpoint{1.542492in}{1.549287in}}%
\pgfpathlineto{\pgfqpoint{1.548258in}{1.550389in}}%
\pgfpathlineto{\pgfqpoint{1.549021in}{1.550652in}}%
\pgfpathlineto{\pgfqpoint{1.558955in}{1.551754in}}%
\pgfpathlineto{\pgfqpoint{1.559859in}{1.552069in}}%
\pgfpathlineto{\pgfqpoint{1.571132in}{1.553172in}}%
\pgfpathlineto{\pgfqpoint{1.571895in}{1.553329in}}%
\pgfpathlineto{\pgfqpoint{1.581783in}{1.554432in}}%
\pgfpathlineto{\pgfqpoint{1.582652in}{1.554589in}}%
\pgfpathlineto{\pgfqpoint{1.594253in}{1.555692in}}%
\pgfpathlineto{\pgfqpoint{1.595322in}{1.555902in}}%
\pgfpathlineto{\pgfqpoint{1.605314in}{1.557004in}}%
\pgfpathlineto{\pgfqpoint{1.605890in}{1.557162in}}%
\pgfpathlineto{\pgfqpoint{1.619534in}{1.558264in}}%
\pgfpathlineto{\pgfqpoint{1.620333in}{1.558474in}}%
\pgfpathlineto{\pgfqpoint{1.630772in}{1.559577in}}%
\pgfpathlineto{\pgfqpoint{1.630819in}{1.559682in}}%
\pgfpathlineto{\pgfqpoint{1.642738in}{1.560784in}}%
\pgfpathlineto{\pgfqpoint{1.643677in}{1.561047in}}%
\pgfpathlineto{\pgfqpoint{1.656793in}{1.562097in}}%
\pgfpathlineto{\pgfqpoint{1.657592in}{1.562307in}}%
\pgfpathlineto{\pgfqpoint{1.669041in}{1.563409in}}%
\pgfpathlineto{\pgfqpoint{1.669698in}{1.563619in}}%
\pgfpathlineto{\pgfqpoint{1.683989in}{1.564722in}}%
\pgfpathlineto{\pgfqpoint{1.685010in}{1.564879in}}%
\pgfpathlineto{\pgfqpoint{1.700311in}{1.565982in}}%
\pgfpathlineto{\pgfqpoint{1.700311in}{1.566034in}}%
\pgfpathlineto{\pgfqpoint{1.717666in}{1.567137in}}%
\pgfpathlineto{\pgfqpoint{1.718006in}{1.567294in}}%
\pgfpathlineto{\pgfqpoint{1.730125in}{1.568397in}}%
\pgfpathlineto{\pgfqpoint{1.730747in}{1.568712in}}%
\pgfpathlineto{\pgfqpoint{1.744298in}{1.569814in}}%
\pgfpathlineto{\pgfqpoint{1.745155in}{1.569972in}}%
\pgfpathlineto{\pgfqpoint{1.760138in}{1.571074in}}%
\pgfpathlineto{\pgfqpoint{1.760913in}{1.571389in}}%
\pgfpathlineto{\pgfqpoint{1.773677in}{1.572492in}}%
\pgfpathlineto{\pgfqpoint{1.774734in}{1.572754in}}%
\pgfpathlineto{\pgfqpoint{1.792406in}{1.573857in}}%
\pgfpathlineto{\pgfqpoint{1.793240in}{1.574067in}}%
\pgfpathlineto{\pgfqpoint{1.804043in}{1.575169in}}%
\pgfpathlineto{\pgfqpoint{1.804278in}{1.575274in}}%
\pgfpathlineto{\pgfqpoint{1.819355in}{1.576377in}}%
\pgfpathlineto{\pgfqpoint{1.819813in}{1.576534in}}%
\pgfpathlineto{\pgfqpoint{1.832014in}{1.577637in}}%
\pgfpathlineto{\pgfqpoint{1.833094in}{1.577794in}}%
\pgfpathlineto{\pgfqpoint{1.850508in}{1.578897in}}%
\pgfpathlineto{\pgfqpoint{1.850966in}{1.579002in}}%
\pgfpathlineto{\pgfqpoint{1.865315in}{1.580104in}}%
\pgfpathlineto{\pgfqpoint{1.866219in}{1.580367in}}%
\pgfpathlineto{\pgfqpoint{1.878279in}{1.581469in}}%
\pgfpathlineto{\pgfqpoint{1.878948in}{1.581574in}}%
\pgfpathlineto{\pgfqpoint{1.891454in}{1.582677in}}%
\pgfpathlineto{\pgfqpoint{1.892147in}{1.582887in}}%
\pgfpathlineto{\pgfqpoint{1.905075in}{1.583989in}}%
\pgfpathlineto{\pgfqpoint{1.906155in}{1.584199in}}%
\pgfpathlineto{\pgfqpoint{1.919694in}{1.585302in}}%
\pgfpathlineto{\pgfqpoint{1.920704in}{1.585512in}}%
\pgfpathlineto{\pgfqpoint{1.935558in}{1.586614in}}%
\pgfpathlineto{\pgfqpoint{1.936392in}{1.586772in}}%
\pgfpathlineto{\pgfqpoint{1.948369in}{1.587874in}}%
\pgfpathlineto{\pgfqpoint{1.949379in}{1.588242in}}%
\pgfpathlineto{\pgfqpoint{1.964691in}{1.589344in}}%
\pgfpathlineto{\pgfqpoint{1.965008in}{1.589554in}}%
\pgfpathlineto{\pgfqpoint{1.978606in}{1.590657in}}%
\pgfpathlineto{\pgfqpoint{1.979675in}{1.590867in}}%
\pgfpathlineto{\pgfqpoint{1.990513in}{1.591969in}}%
\pgfpathlineto{\pgfqpoint{1.990513in}{1.592022in}}%
\pgfpathlineto{\pgfqpoint{2.000118in}{1.593124in}}%
\pgfpathlineto{\pgfqpoint{2.001210in}{1.593387in}}%
\pgfpathlineto{\pgfqpoint{2.010581in}{1.594489in}}%
\pgfpathlineto{\pgfqpoint{2.011320in}{1.594699in}}%
\pgfpathlineto{\pgfqpoint{2.021783in}{1.595802in}}%
\pgfpathlineto{\pgfqpoint{2.022441in}{1.596169in}}%
\pgfpathlineto{\pgfqpoint{2.027243in}{1.597272in}}%
\pgfpathlineto{\pgfqpoint{2.028053in}{1.597902in}}%
\pgfpathlineto{\pgfqpoint{2.031001in}{1.599004in}}%
\pgfpathlineto{\pgfqpoint{2.031834in}{1.599424in}}%
\pgfpathlineto{\pgfqpoint{2.033044in}{1.600527in}}%
\pgfpathlineto{\pgfqpoint{2.033126in}{1.601944in}}%
\pgfpathlineto{\pgfqpoint{2.033126in}{1.601944in}}%
\pgfusepath{stroke}%
\end{pgfscope}%
\begin{pgfscope}%
\pgfsetrectcap%
\pgfsetmiterjoin%
\pgfsetlinewidth{0.803000pt}%
\definecolor{currentstroke}{rgb}{0.000000,0.000000,0.000000}%
\pgfsetstrokecolor{currentstroke}%
\pgfsetdash{}{0pt}%
\pgfpathmoveto{\pgfqpoint{0.553581in}{0.499444in}}%
\pgfpathlineto{\pgfqpoint{0.553581in}{1.654444in}}%
\pgfusepath{stroke}%
\end{pgfscope}%
\begin{pgfscope}%
\pgfsetrectcap%
\pgfsetmiterjoin%
\pgfsetlinewidth{0.803000pt}%
\definecolor{currentstroke}{rgb}{0.000000,0.000000,0.000000}%
\pgfsetstrokecolor{currentstroke}%
\pgfsetdash{}{0pt}%
\pgfpathmoveto{\pgfqpoint{2.103581in}{0.499444in}}%
\pgfpathlineto{\pgfqpoint{2.103581in}{1.654444in}}%
\pgfusepath{stroke}%
\end{pgfscope}%
\begin{pgfscope}%
\pgfsetrectcap%
\pgfsetmiterjoin%
\pgfsetlinewidth{0.803000pt}%
\definecolor{currentstroke}{rgb}{0.000000,0.000000,0.000000}%
\pgfsetstrokecolor{currentstroke}%
\pgfsetdash{}{0pt}%
\pgfpathmoveto{\pgfqpoint{0.553581in}{0.499444in}}%
\pgfpathlineto{\pgfqpoint{2.103581in}{0.499444in}}%
\pgfusepath{stroke}%
\end{pgfscope}%
\begin{pgfscope}%
\pgfsetrectcap%
\pgfsetmiterjoin%
\pgfsetlinewidth{0.803000pt}%
\definecolor{currentstroke}{rgb}{0.000000,0.000000,0.000000}%
\pgfsetstrokecolor{currentstroke}%
\pgfsetdash{}{0pt}%
\pgfpathmoveto{\pgfqpoint{0.553581in}{1.654444in}}%
\pgfpathlineto{\pgfqpoint{2.103581in}{1.654444in}}%
\pgfusepath{stroke}%
\end{pgfscope}%
\begin{pgfscope}%
\pgfsetbuttcap%
\pgfsetmiterjoin%
\definecolor{currentfill}{rgb}{1.000000,1.000000,1.000000}%
\pgfsetfillcolor{currentfill}%
\pgfsetlinewidth{0.000000pt}%
\definecolor{currentstroke}{rgb}{0.000000,0.000000,0.000000}%
\pgfsetstrokecolor{currentstroke}%
\pgfsetstrokeopacity{0.000000}%
\pgfsetdash{}{0pt}%
\pgfpathmoveto{\pgfqpoint{0.675275in}{0.961188in}}%
\pgfpathlineto{\pgfqpoint{1.074997in}{0.961188in}}%
\pgfpathlineto{\pgfqpoint{1.074997in}{1.167855in}}%
\pgfpathlineto{\pgfqpoint{0.675275in}{1.167855in}}%
\pgfpathlineto{\pgfqpoint{0.675275in}{0.961188in}}%
\pgfpathclose%
\pgfusepath{fill}%
\end{pgfscope}%
\begin{pgfscope}%
\definecolor{textcolor}{rgb}{0.000000,0.000000,0.000000}%
\pgfsetstrokecolor{textcolor}%
\pgfsetfillcolor{textcolor}%
\pgftext[x=0.716941in,y=1.029799in,left,base]{\color{textcolor}\rmfamily\fontsize{10.000000}{12.000000}\selectfont 0.338}%
\end{pgfscope}%
\begin{pgfscope}%
\pgfsetbuttcap%
\pgfsetmiterjoin%
\definecolor{currentfill}{rgb}{1.000000,1.000000,1.000000}%
\pgfsetfillcolor{currentfill}%
\pgfsetlinewidth{0.000000pt}%
\definecolor{currentstroke}{rgb}{0.000000,0.000000,0.000000}%
\pgfsetstrokecolor{currentstroke}%
\pgfsetstrokeopacity{0.000000}%
\pgfsetdash{}{0pt}%
\pgfpathmoveto{\pgfqpoint{0.582380in}{0.483333in}}%
\pgfpathlineto{\pgfqpoint{0.843214in}{0.483333in}}%
\pgfpathlineto{\pgfqpoint{0.843214in}{0.690000in}}%
\pgfpathlineto{\pgfqpoint{0.582380in}{0.690000in}}%
\pgfpathlineto{\pgfqpoint{0.582380in}{0.483333in}}%
\pgfpathclose%
\pgfusepath{fill}%
\end{pgfscope}%
\begin{pgfscope}%
\definecolor{textcolor}{rgb}{0.000000,0.000000,0.000000}%
\pgfsetstrokecolor{textcolor}%
\pgfsetfillcolor{textcolor}%
\pgftext[x=0.624047in,y=0.551944in,left,base]{\color{textcolor}\rmfamily\fontsize{10.000000}{12.000000}\selectfont 0.5}%
\end{pgfscope}%
\begin{pgfscope}%
\pgfsetbuttcap%
\pgfsetmiterjoin%
\definecolor{currentfill}{rgb}{1.000000,1.000000,1.000000}%
\pgfsetfillcolor{currentfill}%
\pgfsetlinewidth{0.000000pt}%
\definecolor{currentstroke}{rgb}{0.000000,0.000000,0.000000}%
\pgfsetstrokecolor{currentstroke}%
\pgfsetstrokeopacity{0.000000}%
\pgfsetdash{}{0pt}%
\pgfpathmoveto{\pgfqpoint{0.582380in}{0.483333in}}%
\pgfpathlineto{\pgfqpoint{0.982103in}{0.483333in}}%
\pgfpathlineto{\pgfqpoint{0.982103in}{0.690000in}}%
\pgfpathlineto{\pgfqpoint{0.582380in}{0.690000in}}%
\pgfpathlineto{\pgfqpoint{0.582380in}{0.483333in}}%
\pgfpathclose%
\pgfusepath{fill}%
\end{pgfscope}%
\begin{pgfscope}%
\definecolor{textcolor}{rgb}{0.000000,0.000000,0.000000}%
\pgfsetstrokecolor{textcolor}%
\pgfsetfillcolor{textcolor}%
\pgftext[x=0.624047in,y=0.551944in,left,base]{\color{textcolor}\rmfamily\fontsize{10.000000}{12.000000}\selectfont 0.655}%
\end{pgfscope}%
\begin{pgfscope}%
\definecolor{textcolor}{rgb}{0.000000,0.000000,0.000000}%
\pgfsetstrokecolor{textcolor}%
\pgfsetfillcolor{textcolor}%
\pgftext[x=1.328581in,y=1.737778in,,base]{\color{textcolor}\rmfamily\fontsize{12.000000}{14.400000}\selectfont ROC Curve}%
\end{pgfscope}%
\begin{pgfscope}%
\pgfsetbuttcap%
\pgfsetmiterjoin%
\definecolor{currentfill}{rgb}{1.000000,1.000000,1.000000}%
\pgfsetfillcolor{currentfill}%
\pgfsetfillopacity{0.800000}%
\pgfsetlinewidth{1.003750pt}%
\definecolor{currentstroke}{rgb}{0.800000,0.800000,0.800000}%
\pgfsetstrokecolor{currentstroke}%
\pgfsetstrokeopacity{0.800000}%
\pgfsetdash{}{0pt}%
\pgfpathmoveto{\pgfqpoint{0.840525in}{0.568889in}}%
\pgfpathlineto{\pgfqpoint{2.006358in}{0.568889in}}%
\pgfpathquadraticcurveto{\pgfqpoint{2.034136in}{0.568889in}}{\pgfqpoint{2.034136in}{0.596666in}}%
\pgfpathlineto{\pgfqpoint{2.034136in}{0.791111in}}%
\pgfpathquadraticcurveto{\pgfqpoint{2.034136in}{0.818888in}}{\pgfqpoint{2.006358in}{0.818888in}}%
\pgfpathlineto{\pgfqpoint{0.840525in}{0.818888in}}%
\pgfpathquadraticcurveto{\pgfqpoint{0.812747in}{0.818888in}}{\pgfqpoint{0.812747in}{0.791111in}}%
\pgfpathlineto{\pgfqpoint{0.812747in}{0.596666in}}%
\pgfpathquadraticcurveto{\pgfqpoint{0.812747in}{0.568889in}}{\pgfqpoint{0.840525in}{0.568889in}}%
\pgfpathlineto{\pgfqpoint{0.840525in}{0.568889in}}%
\pgfpathclose%
\pgfusepath{stroke,fill}%
\end{pgfscope}%
\begin{pgfscope}%
\pgfsetrectcap%
\pgfsetroundjoin%
\pgfsetlinewidth{1.505625pt}%
\definecolor{currentstroke}{rgb}{0.000000,0.000000,0.000000}%
\pgfsetstrokecolor{currentstroke}%
\pgfsetdash{}{0pt}%
\pgfpathmoveto{\pgfqpoint{0.868303in}{0.707777in}}%
\pgfpathlineto{\pgfqpoint{1.007192in}{0.707777in}}%
\pgfpathlineto{\pgfqpoint{1.146081in}{0.707777in}}%
\pgfusepath{stroke}%
\end{pgfscope}%
\begin{pgfscope}%
\definecolor{textcolor}{rgb}{0.000000,0.000000,0.000000}%
\pgfsetstrokecolor{textcolor}%
\pgfsetfillcolor{textcolor}%
\pgftext[x=1.257192in,y=0.659166in,left,base]{\color{textcolor}\rmfamily\fontsize{10.000000}{12.000000}\selectfont AUC 0.838)}%
\end{pgfscope}%
\end{pgfpicture}%
\makeatother%
\endgroup%

\end{tabular}
\end{center}

\begin{center}
\begin{tabular}{cc}
\begin{tabular}{cc|c|c|}
	&\multicolumn{1}{c}{}& \multicolumn{2}{c}{Prediction} \cr
	&\multicolumn{1}{c}{} & \multicolumn{1}{c}{N} & \multicolumn{1}{c}{P} \cr\cline{3-4}
	\multirow{2}{*}{Actual}&N & 85.7\% & 0.0\% \vrule width 0pt height 10pt depth 2pt \cr\cline{3-4}
	&P & 14.3\% & 0.0\% \vrule width 0pt height 10pt depth 2pt \cr\cline{3-4}
\end{tabular}
&
\begin{tabular}{ll}
0.857 & Accuracy \cr 
0.500 & Balanced Accuracy \cr 
0.000 & Precision \cr 
0.000 & Balanced Precision \cr 
0.000 & Recall \cr 
0.000 & F1 \cr 
0.000 & Balanced F1 \cr 
0.000 & Gmean \cr 
	\end{tabular}
\end{tabular}
\end{center}

%%%%%
Such a recommendation system (``Never send an ambulance'') would be useless, but note that the distribution still separates the negative and positive classes, just not at $p=0.5$.  We can fix that in two ways; the first is to shift the distribution to be centered at $p=0.5$.  By ``centered,'' we mean that the average of the medians of the negative and positive classes (the 0.107 and 0.293 on the ROC curve above) will now be 0.5.  Further research can explore whether centering the distribution at the $p=0.5$ threshold or another value of $p$ is most useful.  

\begin{center}
\begin{tabular}{p{0.5\textwidth} p{0.5\textwidth}}
  \vspace{0pt} \input{Ideal_50_Shifted_Pred.pgf}
  &
  \vspace{0pt} \input{Ideal_50_Shifted_ROC.pgf}
\end{tabular}
\end{center}

\begin{center}
\begin{tabular}{cc}
\begin{tabular}{cc|c|c|}
	&\multicolumn{1}{c}{}& \multicolumn{2}{c}{Prediction} \cr
	&\multicolumn{1}{c}{} & \multicolumn{1}{c}{N} & \multicolumn{1}{c}{P} \cr\cline{3-4}
	\multirow{2}{*}{Actual}&N & 67.2\% & 18.5\% \vrule width 0pt height 10pt depth 2pt \cr\cline{3-4}
	&P & 3.06\% & 11.22\% \vrule width 0pt height 10pt depth 2pt \cr\cline{3-4}
\end{tabular}
&
\begin{tabular}{ll}
0.784 & Accuracy \cr 
0.785 & Balanced Accuracy \cr 
0.377 & Precision \cr 
0.784 & Balanced Precision \cr 
0.786 & Recall \cr 
0.510 & F1 \cr 
0.785 & Balanced F1 \cr 
0.543 & Gmean \cr 	\end{tabular}
\end{tabular}
\end{center}

Another way is to linearly transform the probabilities.   Whether the distribution was clustered to the left or right, or clustered at the center, is not necessarily relevant, so we want to see it spread out.  We have arbitrarily chosen a transformation to put next the original models in our results to see if it will make a better model; tuning the transformation is an avenue for future work.  We have chosen to take the 0.05 quantile of the negative class and map it to $p=0.05$, and the 0.95 quantile of the positive class and map it to $p=0.95$.  This linear transformation gives the same metrics as the shift, and the ROC curve is the same except for the two labeled medians, now at 0.305 and 0.688.

\begin{center}
\begin{tabular}{p{0.5\textwidth} p{0.5\textwidth}}
  \vspace{0pt} \input{Ideal_50_Linear_Transform_Pred.pgf}
  &
  \vspace{0pt} %% Creator: Matplotlib, PGF backend
%%
%% To include the figure in your LaTeX document, write
%%   \input{<filename>.pgf}
%%
%% Make sure the required packages are loaded in your preamble
%%   \usepackage{pgf}
%%
%% Also ensure that all the required font packages are loaded; for instance,
%% the lmodern package is sometimes necessary when using math font.
%%   \usepackage{lmodern}
%%
%% Figures using additional raster images can only be included by \input if
%% they are in the same directory as the main LaTeX file. For loading figures
%% from other directories you can use the `import` package
%%   \usepackage{import}
%%
%% and then include the figures with
%%   \import{<path to file>}{<filename>.pgf}
%%
%% Matplotlib used the following preamble
%%   
%%   \usepackage{fontspec}
%%   \makeatletter\@ifpackageloaded{underscore}{}{\usepackage[strings]{underscore}}\makeatother
%%
\begingroup%
\makeatletter%
\begin{pgfpicture}%
\pgfpathrectangle{\pgfpointorigin}{\pgfqpoint{2.517770in}{2.184444in}}%
\pgfusepath{use as bounding box, clip}%
\begin{pgfscope}%
\pgfsetbuttcap%
\pgfsetmiterjoin%
\definecolor{currentfill}{rgb}{1.000000,1.000000,1.000000}%
\pgfsetfillcolor{currentfill}%
\pgfsetlinewidth{0.000000pt}%
\definecolor{currentstroke}{rgb}{1.000000,1.000000,1.000000}%
\pgfsetstrokecolor{currentstroke}%
\pgfsetdash{}{0pt}%
\pgfpathmoveto{\pgfqpoint{0.000000in}{0.000000in}}%
\pgfpathlineto{\pgfqpoint{2.517770in}{0.000000in}}%
\pgfpathlineto{\pgfqpoint{2.517770in}{2.184444in}}%
\pgfpathlineto{\pgfqpoint{0.000000in}{2.184444in}}%
\pgfpathlineto{\pgfqpoint{0.000000in}{0.000000in}}%
\pgfpathclose%
\pgfusepath{fill}%
\end{pgfscope}%
\begin{pgfscope}%
\pgfsetbuttcap%
\pgfsetmiterjoin%
\definecolor{currentfill}{rgb}{1.000000,1.000000,1.000000}%
\pgfsetfillcolor{currentfill}%
\pgfsetlinewidth{0.000000pt}%
\definecolor{currentstroke}{rgb}{0.000000,0.000000,0.000000}%
\pgfsetstrokecolor{currentstroke}%
\pgfsetstrokeopacity{0.000000}%
\pgfsetdash{}{0pt}%
\pgfpathmoveto{\pgfqpoint{0.553581in}{0.499444in}}%
\pgfpathlineto{\pgfqpoint{2.413581in}{0.499444in}}%
\pgfpathlineto{\pgfqpoint{2.413581in}{1.885444in}}%
\pgfpathlineto{\pgfqpoint{0.553581in}{1.885444in}}%
\pgfpathlineto{\pgfqpoint{0.553581in}{0.499444in}}%
\pgfpathclose%
\pgfusepath{fill}%
\end{pgfscope}%
\begin{pgfscope}%
\pgfsetbuttcap%
\pgfsetroundjoin%
\definecolor{currentfill}{rgb}{0.000000,0.000000,0.000000}%
\pgfsetfillcolor{currentfill}%
\pgfsetlinewidth{0.803000pt}%
\definecolor{currentstroke}{rgb}{0.000000,0.000000,0.000000}%
\pgfsetstrokecolor{currentstroke}%
\pgfsetdash{}{0pt}%
\pgfsys@defobject{currentmarker}{\pgfqpoint{0.000000in}{-0.048611in}}{\pgfqpoint{0.000000in}{0.000000in}}{%
\pgfpathmoveto{\pgfqpoint{0.000000in}{0.000000in}}%
\pgfpathlineto{\pgfqpoint{0.000000in}{-0.048611in}}%
\pgfusepath{stroke,fill}%
}%
\begin{pgfscope}%
\pgfsys@transformshift{0.638126in}{0.499444in}%
\pgfsys@useobject{currentmarker}{}%
\end{pgfscope}%
\end{pgfscope}%
\begin{pgfscope}%
\definecolor{textcolor}{rgb}{0.000000,0.000000,0.000000}%
\pgfsetstrokecolor{textcolor}%
\pgfsetfillcolor{textcolor}%
\pgftext[x=0.638126in,y=0.402222in,,top]{\color{textcolor}\rmfamily\fontsize{10.000000}{12.000000}\selectfont \(\displaystyle {0.0}\)}%
\end{pgfscope}%
\begin{pgfscope}%
\pgfsetbuttcap%
\pgfsetroundjoin%
\definecolor{currentfill}{rgb}{0.000000,0.000000,0.000000}%
\pgfsetfillcolor{currentfill}%
\pgfsetlinewidth{0.803000pt}%
\definecolor{currentstroke}{rgb}{0.000000,0.000000,0.000000}%
\pgfsetstrokecolor{currentstroke}%
\pgfsetdash{}{0pt}%
\pgfsys@defobject{currentmarker}{\pgfqpoint{0.000000in}{-0.048611in}}{\pgfqpoint{0.000000in}{0.000000in}}{%
\pgfpathmoveto{\pgfqpoint{0.000000in}{0.000000in}}%
\pgfpathlineto{\pgfqpoint{0.000000in}{-0.048611in}}%
\pgfusepath{stroke,fill}%
}%
\begin{pgfscope}%
\pgfsys@transformshift{1.483581in}{0.499444in}%
\pgfsys@useobject{currentmarker}{}%
\end{pgfscope}%
\end{pgfscope}%
\begin{pgfscope}%
\definecolor{textcolor}{rgb}{0.000000,0.000000,0.000000}%
\pgfsetstrokecolor{textcolor}%
\pgfsetfillcolor{textcolor}%
\pgftext[x=1.483581in,y=0.402222in,,top]{\color{textcolor}\rmfamily\fontsize{10.000000}{12.000000}\selectfont \(\displaystyle {0.5}\)}%
\end{pgfscope}%
\begin{pgfscope}%
\pgfsetbuttcap%
\pgfsetroundjoin%
\definecolor{currentfill}{rgb}{0.000000,0.000000,0.000000}%
\pgfsetfillcolor{currentfill}%
\pgfsetlinewidth{0.803000pt}%
\definecolor{currentstroke}{rgb}{0.000000,0.000000,0.000000}%
\pgfsetstrokecolor{currentstroke}%
\pgfsetdash{}{0pt}%
\pgfsys@defobject{currentmarker}{\pgfqpoint{0.000000in}{-0.048611in}}{\pgfqpoint{0.000000in}{0.000000in}}{%
\pgfpathmoveto{\pgfqpoint{0.000000in}{0.000000in}}%
\pgfpathlineto{\pgfqpoint{0.000000in}{-0.048611in}}%
\pgfusepath{stroke,fill}%
}%
\begin{pgfscope}%
\pgfsys@transformshift{2.329035in}{0.499444in}%
\pgfsys@useobject{currentmarker}{}%
\end{pgfscope}%
\end{pgfscope}%
\begin{pgfscope}%
\definecolor{textcolor}{rgb}{0.000000,0.000000,0.000000}%
\pgfsetstrokecolor{textcolor}%
\pgfsetfillcolor{textcolor}%
\pgftext[x=2.329035in,y=0.402222in,,top]{\color{textcolor}\rmfamily\fontsize{10.000000}{12.000000}\selectfont \(\displaystyle {1.0}\)}%
\end{pgfscope}%
\begin{pgfscope}%
\definecolor{textcolor}{rgb}{0.000000,0.000000,0.000000}%
\pgfsetstrokecolor{textcolor}%
\pgfsetfillcolor{textcolor}%
\pgftext[x=1.483581in,y=0.223333in,,top]{\color{textcolor}\rmfamily\fontsize{10.000000}{12.000000}\selectfont False positive rate}%
\end{pgfscope}%
\begin{pgfscope}%
\pgfsetbuttcap%
\pgfsetroundjoin%
\definecolor{currentfill}{rgb}{0.000000,0.000000,0.000000}%
\pgfsetfillcolor{currentfill}%
\pgfsetlinewidth{0.803000pt}%
\definecolor{currentstroke}{rgb}{0.000000,0.000000,0.000000}%
\pgfsetstrokecolor{currentstroke}%
\pgfsetdash{}{0pt}%
\pgfsys@defobject{currentmarker}{\pgfqpoint{-0.048611in}{0.000000in}}{\pgfqpoint{-0.000000in}{0.000000in}}{%
\pgfpathmoveto{\pgfqpoint{-0.000000in}{0.000000in}}%
\pgfpathlineto{\pgfqpoint{-0.048611in}{0.000000in}}%
\pgfusepath{stroke,fill}%
}%
\begin{pgfscope}%
\pgfsys@transformshift{0.553581in}{0.562444in}%
\pgfsys@useobject{currentmarker}{}%
\end{pgfscope}%
\end{pgfscope}%
\begin{pgfscope}%
\definecolor{textcolor}{rgb}{0.000000,0.000000,0.000000}%
\pgfsetstrokecolor{textcolor}%
\pgfsetfillcolor{textcolor}%
\pgftext[x=0.278889in, y=0.514250in, left, base]{\color{textcolor}\rmfamily\fontsize{10.000000}{12.000000}\selectfont \(\displaystyle {0.0}\)}%
\end{pgfscope}%
\begin{pgfscope}%
\pgfsetbuttcap%
\pgfsetroundjoin%
\definecolor{currentfill}{rgb}{0.000000,0.000000,0.000000}%
\pgfsetfillcolor{currentfill}%
\pgfsetlinewidth{0.803000pt}%
\definecolor{currentstroke}{rgb}{0.000000,0.000000,0.000000}%
\pgfsetstrokecolor{currentstroke}%
\pgfsetdash{}{0pt}%
\pgfsys@defobject{currentmarker}{\pgfqpoint{-0.048611in}{0.000000in}}{\pgfqpoint{-0.000000in}{0.000000in}}{%
\pgfpathmoveto{\pgfqpoint{-0.000000in}{0.000000in}}%
\pgfpathlineto{\pgfqpoint{-0.048611in}{0.000000in}}%
\pgfusepath{stroke,fill}%
}%
\begin{pgfscope}%
\pgfsys@transformshift{0.553581in}{1.192444in}%
\pgfsys@useobject{currentmarker}{}%
\end{pgfscope}%
\end{pgfscope}%
\begin{pgfscope}%
\definecolor{textcolor}{rgb}{0.000000,0.000000,0.000000}%
\pgfsetstrokecolor{textcolor}%
\pgfsetfillcolor{textcolor}%
\pgftext[x=0.278889in, y=1.144250in, left, base]{\color{textcolor}\rmfamily\fontsize{10.000000}{12.000000}\selectfont \(\displaystyle {0.5}\)}%
\end{pgfscope}%
\begin{pgfscope}%
\pgfsetbuttcap%
\pgfsetroundjoin%
\definecolor{currentfill}{rgb}{0.000000,0.000000,0.000000}%
\pgfsetfillcolor{currentfill}%
\pgfsetlinewidth{0.803000pt}%
\definecolor{currentstroke}{rgb}{0.000000,0.000000,0.000000}%
\pgfsetstrokecolor{currentstroke}%
\pgfsetdash{}{0pt}%
\pgfsys@defobject{currentmarker}{\pgfqpoint{-0.048611in}{0.000000in}}{\pgfqpoint{-0.000000in}{0.000000in}}{%
\pgfpathmoveto{\pgfqpoint{-0.000000in}{0.000000in}}%
\pgfpathlineto{\pgfqpoint{-0.048611in}{0.000000in}}%
\pgfusepath{stroke,fill}%
}%
\begin{pgfscope}%
\pgfsys@transformshift{0.553581in}{1.822444in}%
\pgfsys@useobject{currentmarker}{}%
\end{pgfscope}%
\end{pgfscope}%
\begin{pgfscope}%
\definecolor{textcolor}{rgb}{0.000000,0.000000,0.000000}%
\pgfsetstrokecolor{textcolor}%
\pgfsetfillcolor{textcolor}%
\pgftext[x=0.278889in, y=1.774250in, left, base]{\color{textcolor}\rmfamily\fontsize{10.000000}{12.000000}\selectfont \(\displaystyle {1.0}\)}%
\end{pgfscope}%
\begin{pgfscope}%
\definecolor{textcolor}{rgb}{0.000000,0.000000,0.000000}%
\pgfsetstrokecolor{textcolor}%
\pgfsetfillcolor{textcolor}%
\pgftext[x=0.223333in,y=1.192444in,,bottom,rotate=90.000000]{\color{textcolor}\rmfamily\fontsize{10.000000}{12.000000}\selectfont True positive rate}%
\end{pgfscope}%
\begin{pgfscope}%
\pgfpathrectangle{\pgfqpoint{0.553581in}{0.499444in}}{\pgfqpoint{1.860000in}{1.386000in}}%
\pgfusepath{clip}%
\pgfsetbuttcap%
\pgfsetroundjoin%
\pgfsetlinewidth{1.505625pt}%
\definecolor{currentstroke}{rgb}{0.000000,0.000000,0.000000}%
\pgfsetstrokecolor{currentstroke}%
\pgfsetdash{{5.550000pt}{2.400000pt}}{0.000000pt}%
\pgfpathmoveto{\pgfqpoint{0.638126in}{0.562444in}}%
\pgfpathlineto{\pgfqpoint{2.329035in}{1.822444in}}%
\pgfusepath{stroke}%
\end{pgfscope}%
\begin{pgfscope}%
\pgfpathrectangle{\pgfqpoint{0.553581in}{0.499444in}}{\pgfqpoint{1.860000in}{1.386000in}}%
\pgfusepath{clip}%
\pgfsetrectcap%
\pgfsetroundjoin%
\pgfsetlinewidth{1.505625pt}%
\definecolor{currentstroke}{rgb}{0.121569,0.466667,0.705882}%
\pgfsetstrokecolor{currentstroke}%
\pgfsetdash{}{0pt}%
\pgfpathmoveto{\pgfqpoint{0.638126in}{0.562444in}}%
\pgfpathlineto{\pgfqpoint{0.649173in}{0.595960in}}%
\pgfpathlineto{\pgfqpoint{0.650258in}{0.601567in}}%
\pgfpathlineto{\pgfqpoint{0.650498in}{0.602512in}}%
\pgfpathlineto{\pgfqpoint{0.651597in}{0.607867in}}%
\pgfpathlineto{\pgfqpoint{0.651752in}{0.608938in}}%
\pgfpathlineto{\pgfqpoint{0.652837in}{0.616309in}}%
\pgfpathlineto{\pgfqpoint{0.653091in}{0.617380in}}%
\pgfpathlineto{\pgfqpoint{0.654176in}{0.623365in}}%
\pgfpathlineto{\pgfqpoint{0.654373in}{0.624373in}}%
\pgfpathlineto{\pgfqpoint{0.655472in}{0.631996in}}%
\pgfpathlineto{\pgfqpoint{0.655712in}{0.632878in}}%
\pgfpathlineto{\pgfqpoint{0.656783in}{0.639745in}}%
\pgfpathlineto{\pgfqpoint{0.656966in}{0.640816in}}%
\pgfpathlineto{\pgfqpoint{0.658065in}{0.648376in}}%
\pgfpathlineto{\pgfqpoint{0.658276in}{0.649258in}}%
\pgfpathlineto{\pgfqpoint{0.659375in}{0.657448in}}%
\pgfpathlineto{\pgfqpoint{0.659544in}{0.658141in}}%
\pgfpathlineto{\pgfqpoint{0.660643in}{0.667024in}}%
\pgfpathlineto{\pgfqpoint{0.660798in}{0.667969in}}%
\pgfpathlineto{\pgfqpoint{0.661898in}{0.676915in}}%
\pgfpathlineto{\pgfqpoint{0.662053in}{0.677797in}}%
\pgfpathlineto{\pgfqpoint{0.663138in}{0.688129in}}%
\pgfpathlineto{\pgfqpoint{0.663405in}{0.689074in}}%
\pgfpathlineto{\pgfqpoint{0.664490in}{0.696697in}}%
\pgfpathlineto{\pgfqpoint{0.664800in}{0.697642in}}%
\pgfpathlineto{\pgfqpoint{0.665899in}{0.706903in}}%
\pgfpathlineto{\pgfqpoint{0.666097in}{0.707974in}}%
\pgfpathlineto{\pgfqpoint{0.667196in}{0.714085in}}%
\pgfpathlineto{\pgfqpoint{0.667351in}{0.715156in}}%
\pgfpathlineto{\pgfqpoint{0.668422in}{0.724354in}}%
\pgfpathlineto{\pgfqpoint{0.668563in}{0.725047in}}%
\pgfpathlineto{\pgfqpoint{0.669662in}{0.734560in}}%
\pgfpathlineto{\pgfqpoint{0.669774in}{0.735379in}}%
\pgfpathlineto{\pgfqpoint{0.670845in}{0.743380in}}%
\pgfpathlineto{\pgfqpoint{0.670972in}{0.743947in}}%
\pgfpathlineto{\pgfqpoint{0.672057in}{0.751948in}}%
\pgfpathlineto{\pgfqpoint{0.672170in}{0.752641in}}%
\pgfpathlineto{\pgfqpoint{0.673269in}{0.760264in}}%
\pgfpathlineto{\pgfqpoint{0.673368in}{0.761335in}}%
\pgfpathlineto{\pgfqpoint{0.674467in}{0.769336in}}%
\pgfpathlineto{\pgfqpoint{0.674692in}{0.770344in}}%
\pgfpathlineto{\pgfqpoint{0.675735in}{0.779416in}}%
\pgfpathlineto{\pgfqpoint{0.675890in}{0.780487in}}%
\pgfpathlineto{\pgfqpoint{0.676975in}{0.789874in}}%
\pgfpathlineto{\pgfqpoint{0.677200in}{0.790756in}}%
\pgfpathlineto{\pgfqpoint{0.678299in}{0.800458in}}%
\pgfpathlineto{\pgfqpoint{0.678511in}{0.801466in}}%
\pgfpathlineto{\pgfqpoint{0.679610in}{0.812869in}}%
\pgfpathlineto{\pgfqpoint{0.679849in}{0.813940in}}%
\pgfpathlineto{\pgfqpoint{0.680948in}{0.822445in}}%
\pgfpathlineto{\pgfqpoint{0.681075in}{0.823516in}}%
\pgfpathlineto{\pgfqpoint{0.682174in}{0.829816in}}%
\pgfpathlineto{\pgfqpoint{0.682287in}{0.830761in}}%
\pgfpathlineto{\pgfqpoint{0.683386in}{0.837943in}}%
\pgfpathlineto{\pgfqpoint{0.683612in}{0.839014in}}%
\pgfpathlineto{\pgfqpoint{0.684711in}{0.848338in}}%
\pgfpathlineto{\pgfqpoint{0.684894in}{0.849346in}}%
\pgfpathlineto{\pgfqpoint{0.685979in}{0.856591in}}%
\pgfpathlineto{\pgfqpoint{0.686218in}{0.857662in}}%
\pgfpathlineto{\pgfqpoint{0.687318in}{0.862765in}}%
\pgfpathlineto{\pgfqpoint{0.687473in}{0.863836in}}%
\pgfpathlineto{\pgfqpoint{0.688572in}{0.870892in}}%
\pgfpathlineto{\pgfqpoint{0.688727in}{0.871837in}}%
\pgfpathlineto{\pgfqpoint{0.689783in}{0.879838in}}%
\pgfpathlineto{\pgfqpoint{0.689967in}{0.880909in}}%
\pgfpathlineto{\pgfqpoint{0.691052in}{0.888091in}}%
\pgfpathlineto{\pgfqpoint{0.691348in}{0.889162in}}%
\pgfpathlineto{\pgfqpoint{0.692447in}{0.897100in}}%
\pgfpathlineto{\pgfqpoint{0.692672in}{0.898108in}}%
\pgfpathlineto{\pgfqpoint{0.693771in}{0.905983in}}%
\pgfpathlineto{\pgfqpoint{0.693997in}{0.906865in}}%
\pgfpathlineto{\pgfqpoint{0.695096in}{0.913417in}}%
\pgfpathlineto{\pgfqpoint{0.695237in}{0.914425in}}%
\pgfpathlineto{\pgfqpoint{0.696336in}{0.921607in}}%
\pgfpathlineto{\pgfqpoint{0.696561in}{0.922615in}}%
\pgfpathlineto{\pgfqpoint{0.697660in}{0.930238in}}%
\pgfpathlineto{\pgfqpoint{0.697956in}{0.931309in}}%
\pgfpathlineto{\pgfqpoint{0.699055in}{0.938806in}}%
\pgfpathlineto{\pgfqpoint{0.699196in}{0.939814in}}%
\pgfpathlineto{\pgfqpoint{0.700281in}{0.944476in}}%
\pgfpathlineto{\pgfqpoint{0.700493in}{0.945484in}}%
\pgfpathlineto{\pgfqpoint{0.701563in}{0.952162in}}%
\pgfpathlineto{\pgfqpoint{0.701761in}{0.953233in}}%
\pgfpathlineto{\pgfqpoint{0.702818in}{0.958399in}}%
\pgfpathlineto{\pgfqpoint{0.703113in}{0.959407in}}%
\pgfpathlineto{\pgfqpoint{0.704213in}{0.965581in}}%
\pgfpathlineto{\pgfqpoint{0.704438in}{0.966652in}}%
\pgfpathlineto{\pgfqpoint{0.705537in}{0.972259in}}%
\pgfpathlineto{\pgfqpoint{0.705636in}{0.973204in}}%
\pgfpathlineto{\pgfqpoint{0.706735in}{0.978118in}}%
\pgfpathlineto{\pgfqpoint{0.706890in}{0.979126in}}%
\pgfpathlineto{\pgfqpoint{0.707989in}{0.985426in}}%
\pgfpathlineto{\pgfqpoint{0.708102in}{0.986245in}}%
\pgfpathlineto{\pgfqpoint{0.709201in}{0.993112in}}%
\pgfpathlineto{\pgfqpoint{0.709468in}{0.994183in}}%
\pgfpathlineto{\pgfqpoint{0.710553in}{0.999916in}}%
\pgfpathlineto{\pgfqpoint{0.710878in}{1.000924in}}%
\pgfpathlineto{\pgfqpoint{0.711977in}{1.008736in}}%
\pgfpathlineto{\pgfqpoint{0.712202in}{1.009618in}}%
\pgfpathlineto{\pgfqpoint{0.713287in}{1.015351in}}%
\pgfpathlineto{\pgfqpoint{0.713597in}{1.016422in}}%
\pgfpathlineto{\pgfqpoint{0.714696in}{1.020580in}}%
\pgfpathlineto{\pgfqpoint{0.714922in}{1.021651in}}%
\pgfpathlineto{\pgfqpoint{0.716007in}{1.026691in}}%
\pgfpathlineto{\pgfqpoint{0.716345in}{1.027762in}}%
\pgfpathlineto{\pgfqpoint{0.717444in}{1.033558in}}%
\pgfpathlineto{\pgfqpoint{0.717627in}{1.034503in}}%
\pgfpathlineto{\pgfqpoint{0.718726in}{1.039921in}}%
\pgfpathlineto{\pgfqpoint{0.718867in}{1.040803in}}%
\pgfpathlineto{\pgfqpoint{0.719952in}{1.046284in}}%
\pgfpathlineto{\pgfqpoint{0.720121in}{1.047355in}}%
\pgfpathlineto{\pgfqpoint{0.721206in}{1.051639in}}%
\pgfpathlineto{\pgfqpoint{0.721389in}{1.052647in}}%
\pgfpathlineto{\pgfqpoint{0.722488in}{1.058317in}}%
\pgfpathlineto{\pgfqpoint{0.722897in}{1.059388in}}%
\pgfpathlineto{\pgfqpoint{0.723954in}{1.064302in}}%
\pgfpathlineto{\pgfqpoint{0.724179in}{1.065310in}}%
\pgfpathlineto{\pgfqpoint{0.725236in}{1.070539in}}%
\pgfpathlineto{\pgfqpoint{0.725729in}{1.071610in}}%
\pgfpathlineto{\pgfqpoint{0.726814in}{1.076776in}}%
\pgfpathlineto{\pgfqpoint{0.727138in}{1.077784in}}%
\pgfpathlineto{\pgfqpoint{0.728195in}{1.082887in}}%
\pgfpathlineto{\pgfqpoint{0.728533in}{1.083958in}}%
\pgfpathlineto{\pgfqpoint{0.729633in}{1.088116in}}%
\pgfpathlineto{\pgfqpoint{0.729872in}{1.089187in}}%
\pgfpathlineto{\pgfqpoint{0.730957in}{1.094731in}}%
\pgfpathlineto{\pgfqpoint{0.731281in}{1.095802in}}%
\pgfpathlineto{\pgfqpoint{0.732352in}{1.100905in}}%
\pgfpathlineto{\pgfqpoint{0.732718in}{1.101976in}}%
\pgfpathlineto{\pgfqpoint{0.733818in}{1.105315in}}%
\pgfpathlineto{\pgfqpoint{0.734142in}{1.106323in}}%
\pgfpathlineto{\pgfqpoint{0.735198in}{1.110229in}}%
\pgfpathlineto{\pgfqpoint{0.735494in}{1.111300in}}%
\pgfpathlineto{\pgfqpoint{0.736565in}{1.115962in}}%
\pgfpathlineto{\pgfqpoint{0.736819in}{1.116970in}}%
\pgfpathlineto{\pgfqpoint{0.737918in}{1.121065in}}%
\pgfpathlineto{\pgfqpoint{0.738172in}{1.121947in}}%
\pgfpathlineto{\pgfqpoint{0.739243in}{1.126420in}}%
\pgfpathlineto{\pgfqpoint{0.739538in}{1.127491in}}%
\pgfpathlineto{\pgfqpoint{0.740623in}{1.131712in}}%
\pgfpathlineto{\pgfqpoint{0.741018in}{1.132783in}}%
\pgfpathlineto{\pgfqpoint{0.742089in}{1.136059in}}%
\pgfpathlineto{\pgfqpoint{0.742286in}{1.136878in}}%
\pgfpathlineto{\pgfqpoint{0.743385in}{1.142674in}}%
\pgfpathlineto{\pgfqpoint{0.743568in}{1.143556in}}%
\pgfpathlineto{\pgfqpoint{0.744668in}{1.148218in}}%
\pgfpathlineto{\pgfqpoint{0.745062in}{1.149289in}}%
\pgfpathlineto{\pgfqpoint{0.746147in}{1.152187in}}%
\pgfpathlineto{\pgfqpoint{0.746513in}{1.153258in}}%
\pgfpathlineto{\pgfqpoint{0.747570in}{1.156534in}}%
\pgfpathlineto{\pgfqpoint{0.747894in}{1.157605in}}%
\pgfpathlineto{\pgfqpoint{0.748979in}{1.160944in}}%
\pgfpathlineto{\pgfqpoint{0.749628in}{1.162015in}}%
\pgfpathlineto{\pgfqpoint{0.750713in}{1.166299in}}%
\pgfpathlineto{\pgfqpoint{0.751079in}{1.167370in}}%
\pgfpathlineto{\pgfqpoint{0.752122in}{1.171402in}}%
\pgfpathlineto{\pgfqpoint{0.752347in}{1.172410in}}%
\pgfpathlineto{\pgfqpoint{0.753446in}{1.175371in}}%
\pgfpathlineto{\pgfqpoint{0.753968in}{1.176379in}}%
\pgfpathlineto{\pgfqpoint{0.755053in}{1.180222in}}%
\pgfpathlineto{\pgfqpoint{0.755518in}{1.181167in}}%
\pgfpathlineto{\pgfqpoint{0.756617in}{1.184758in}}%
\pgfpathlineto{\pgfqpoint{0.756955in}{1.185829in}}%
\pgfpathlineto{\pgfqpoint{0.758040in}{1.190176in}}%
\pgfpathlineto{\pgfqpoint{0.758589in}{1.191247in}}%
\pgfpathlineto{\pgfqpoint{0.759688in}{1.194271in}}%
\pgfpathlineto{\pgfqpoint{0.759942in}{1.195216in}}%
\pgfpathlineto{\pgfqpoint{0.761013in}{1.198429in}}%
\pgfpathlineto{\pgfqpoint{0.761267in}{1.199437in}}%
\pgfpathlineto{\pgfqpoint{0.762352in}{1.202776in}}%
\pgfpathlineto{\pgfqpoint{0.762788in}{1.203847in}}%
\pgfpathlineto{\pgfqpoint{0.763888in}{1.207312in}}%
\pgfpathlineto{\pgfqpoint{0.764353in}{1.208383in}}%
\pgfpathlineto{\pgfqpoint{0.765423in}{1.211533in}}%
\pgfpathlineto{\pgfqpoint{0.765903in}{1.212478in}}%
\pgfpathlineto{\pgfqpoint{0.766959in}{1.216384in}}%
\pgfpathlineto{\pgfqpoint{0.767579in}{1.217392in}}%
\pgfpathlineto{\pgfqpoint{0.768664in}{1.221424in}}%
\pgfpathlineto{\pgfqpoint{0.769017in}{1.222495in}}%
\pgfpathlineto{\pgfqpoint{0.770088in}{1.224889in}}%
\pgfpathlineto{\pgfqpoint{0.770454in}{1.225960in}}%
\pgfpathlineto{\pgfqpoint{0.771539in}{1.229362in}}%
\pgfpathlineto{\pgfqpoint{0.771976in}{1.230370in}}%
\pgfpathlineto{\pgfqpoint{0.773047in}{1.234024in}}%
\pgfpathlineto{\pgfqpoint{0.773540in}{1.235032in}}%
\pgfpathlineto{\pgfqpoint{0.774639in}{1.237678in}}%
\pgfpathlineto{\pgfqpoint{0.774921in}{1.238749in}}%
\pgfpathlineto{\pgfqpoint{0.776020in}{1.242718in}}%
\pgfpathlineto{\pgfqpoint{0.776443in}{1.243789in}}%
\pgfpathlineto{\pgfqpoint{0.777499in}{1.246372in}}%
\pgfpathlineto{\pgfqpoint{0.778063in}{1.247443in}}%
\pgfpathlineto{\pgfqpoint{0.779162in}{1.249648in}}%
\pgfpathlineto{\pgfqpoint{0.779416in}{1.250719in}}%
\pgfpathlineto{\pgfqpoint{0.780515in}{1.254436in}}%
\pgfpathlineto{\pgfqpoint{0.780797in}{1.255381in}}%
\pgfpathlineto{\pgfqpoint{0.781882in}{1.257901in}}%
\pgfpathlineto{\pgfqpoint{0.782685in}{1.258972in}}%
\pgfpathlineto{\pgfqpoint{0.782995in}{1.260043in}}%
\pgfpathlineto{\pgfqpoint{0.795592in}{1.261051in}}%
\pgfpathlineto{\pgfqpoint{0.796691in}{1.264453in}}%
\pgfpathlineto{\pgfqpoint{0.797100in}{1.265524in}}%
\pgfpathlineto{\pgfqpoint{0.798185in}{1.269115in}}%
\pgfpathlineto{\pgfqpoint{0.798565in}{1.270060in}}%
\pgfpathlineto{\pgfqpoint{0.799664in}{1.272328in}}%
\pgfpathlineto{\pgfqpoint{0.800242in}{1.273336in}}%
\pgfpathlineto{\pgfqpoint{0.801327in}{1.276108in}}%
\pgfpathlineto{\pgfqpoint{0.801722in}{1.277179in}}%
\pgfpathlineto{\pgfqpoint{0.802821in}{1.280203in}}%
\pgfpathlineto{\pgfqpoint{0.803243in}{1.281274in}}%
\pgfpathlineto{\pgfqpoint{0.804286in}{1.283794in}}%
\pgfpathlineto{\pgfqpoint{0.804836in}{1.284802in}}%
\pgfpathlineto{\pgfqpoint{0.805935in}{1.287763in}}%
\pgfpathlineto{\pgfqpoint{0.806583in}{1.288834in}}%
\pgfpathlineto{\pgfqpoint{0.807654in}{1.290976in}}%
\pgfpathlineto{\pgfqpoint{0.808133in}{1.292047in}}%
\pgfpathlineto{\pgfqpoint{0.809232in}{1.295701in}}%
\pgfpathlineto{\pgfqpoint{0.809725in}{1.296772in}}%
\pgfpathlineto{\pgfqpoint{0.810824in}{1.299418in}}%
\pgfpathlineto{\pgfqpoint{0.811458in}{1.300489in}}%
\pgfpathlineto{\pgfqpoint{0.812543in}{1.302631in}}%
\pgfpathlineto{\pgfqpoint{0.813375in}{1.303702in}}%
\pgfpathlineto{\pgfqpoint{0.814460in}{1.306285in}}%
\pgfpathlineto{\pgfqpoint{0.814840in}{1.307356in}}%
\pgfpathlineto{\pgfqpoint{0.815925in}{1.310191in}}%
\pgfpathlineto{\pgfqpoint{0.816588in}{1.311262in}}%
\pgfpathlineto{\pgfqpoint{0.817616in}{1.313089in}}%
\pgfpathlineto{\pgfqpoint{0.818433in}{1.314097in}}%
\pgfpathlineto{\pgfqpoint{0.819533in}{1.317373in}}%
\pgfpathlineto{\pgfqpoint{0.820124in}{1.318444in}}%
\pgfpathlineto{\pgfqpoint{0.821209in}{1.320523in}}%
\pgfpathlineto{\pgfqpoint{0.821759in}{1.321342in}}%
\pgfpathlineto{\pgfqpoint{0.822816in}{1.324303in}}%
\pgfpathlineto{\pgfqpoint{0.823379in}{1.325374in}}%
\pgfpathlineto{\pgfqpoint{0.824408in}{1.328083in}}%
\pgfpathlineto{\pgfqpoint{0.824986in}{1.329154in}}%
\pgfpathlineto{\pgfqpoint{0.826085in}{1.331044in}}%
\pgfpathlineto{\pgfqpoint{0.826536in}{1.332115in}}%
\pgfpathlineto{\pgfqpoint{0.827480in}{1.333879in}}%
\pgfpathlineto{\pgfqpoint{0.828213in}{1.334950in}}%
\pgfpathlineto{\pgfqpoint{0.829283in}{1.336336in}}%
\pgfpathlineto{\pgfqpoint{0.829861in}{1.337407in}}%
\pgfpathlineto{\pgfqpoint{0.830946in}{1.338919in}}%
\pgfpathlineto{\pgfqpoint{0.831735in}{1.339990in}}%
\pgfpathlineto{\pgfqpoint{0.832834in}{1.341691in}}%
\pgfpathlineto{\pgfqpoint{0.833609in}{1.342762in}}%
\pgfpathlineto{\pgfqpoint{0.834694in}{1.345156in}}%
\pgfpathlineto{\pgfqpoint{0.835328in}{1.346227in}}%
\pgfpathlineto{\pgfqpoint{0.836399in}{1.348243in}}%
\pgfpathlineto{\pgfqpoint{0.837104in}{1.349314in}}%
\pgfpathlineto{\pgfqpoint{0.838203in}{1.351708in}}%
\pgfpathlineto{\pgfqpoint{0.838851in}{1.352779in}}%
\pgfpathlineto{\pgfqpoint{0.839936in}{1.355110in}}%
\pgfpathlineto{\pgfqpoint{0.840598in}{1.356181in}}%
\pgfpathlineto{\pgfqpoint{0.841669in}{1.358008in}}%
\pgfpathlineto{\pgfqpoint{0.842444in}{1.359079in}}%
\pgfpathlineto{\pgfqpoint{0.843473in}{1.360969in}}%
\pgfpathlineto{\pgfqpoint{0.843952in}{1.361977in}}%
\pgfpathlineto{\pgfqpoint{0.845023in}{1.364434in}}%
\pgfpathlineto{\pgfqpoint{0.845953in}{1.365505in}}%
\pgfpathlineto{\pgfqpoint{0.847024in}{1.367836in}}%
\pgfpathlineto{\pgfqpoint{0.847644in}{1.368844in}}%
\pgfpathlineto{\pgfqpoint{0.848673in}{1.370482in}}%
\pgfpathlineto{\pgfqpoint{0.849307in}{1.371490in}}%
\pgfpathlineto{\pgfqpoint{0.850378in}{1.373506in}}%
\pgfpathlineto{\pgfqpoint{0.850913in}{1.374577in}}%
\pgfpathlineto{\pgfqpoint{0.851998in}{1.377034in}}%
\pgfpathlineto{\pgfqpoint{0.852717in}{1.378105in}}%
\pgfpathlineto{\pgfqpoint{0.853788in}{1.379932in}}%
\pgfpathlineto{\pgfqpoint{0.854534in}{1.381003in}}%
\pgfpathlineto{\pgfqpoint{0.855633in}{1.382893in}}%
\pgfpathlineto{\pgfqpoint{0.856493in}{1.383901in}}%
\pgfpathlineto{\pgfqpoint{0.857550in}{1.386043in}}%
\pgfpathlineto{\pgfqpoint{0.858057in}{1.387114in}}%
\pgfpathlineto{\pgfqpoint{0.859114in}{1.389004in}}%
\pgfpathlineto{\pgfqpoint{0.859720in}{1.390075in}}%
\pgfpathlineto{\pgfqpoint{0.860819in}{1.392280in}}%
\pgfpathlineto{\pgfqpoint{0.861523in}{1.393225in}}%
\pgfpathlineto{\pgfqpoint{0.862608in}{1.395115in}}%
\pgfpathlineto{\pgfqpoint{0.863355in}{1.396186in}}%
\pgfpathlineto{\pgfqpoint{0.864342in}{1.397824in}}%
\pgfpathlineto{\pgfqpoint{0.865413in}{1.398895in}}%
\pgfpathlineto{\pgfqpoint{0.866483in}{1.401730in}}%
\pgfpathlineto{\pgfqpoint{0.867188in}{1.402801in}}%
\pgfpathlineto{\pgfqpoint{0.868245in}{1.405447in}}%
\pgfpathlineto{\pgfqpoint{0.869133in}{1.406455in}}%
\pgfpathlineto{\pgfqpoint{0.870189in}{1.408534in}}%
\pgfpathlineto{\pgfqpoint{0.870852in}{1.409605in}}%
\pgfpathlineto{\pgfqpoint{0.871937in}{1.411495in}}%
\pgfpathlineto{\pgfqpoint{0.872585in}{1.412440in}}%
\pgfpathlineto{\pgfqpoint{0.873684in}{1.414645in}}%
\pgfpathlineto{\pgfqpoint{0.874107in}{1.415527in}}%
\pgfpathlineto{\pgfqpoint{0.875051in}{1.417165in}}%
\pgfpathlineto{\pgfqpoint{0.875868in}{1.418236in}}%
\pgfpathlineto{\pgfqpoint{0.876953in}{1.420063in}}%
\pgfpathlineto{\pgfqpoint{0.877474in}{1.421134in}}%
\pgfpathlineto{\pgfqpoint{0.878559in}{1.422772in}}%
\pgfpathlineto{\pgfqpoint{0.879701in}{1.423843in}}%
\pgfpathlineto{\pgfqpoint{0.880786in}{1.425796in}}%
\pgfpathlineto{\pgfqpoint{0.881490in}{1.426867in}}%
\pgfpathlineto{\pgfqpoint{0.882491in}{1.427938in}}%
\pgfpathlineto{\pgfqpoint{0.883477in}{1.429009in}}%
\pgfpathlineto{\pgfqpoint{0.884520in}{1.430458in}}%
\pgfpathlineto{\pgfqpoint{0.885210in}{1.431529in}}%
\pgfpathlineto{\pgfqpoint{0.886267in}{1.433923in}}%
\pgfpathlineto{\pgfqpoint{0.887113in}{1.434931in}}%
\pgfpathlineto{\pgfqpoint{0.888212in}{1.436758in}}%
\pgfpathlineto{\pgfqpoint{0.888930in}{1.437829in}}%
\pgfpathlineto{\pgfqpoint{0.890015in}{1.439530in}}%
\pgfpathlineto{\pgfqpoint{0.890804in}{1.440475in}}%
\pgfpathlineto{\pgfqpoint{0.891903in}{1.441924in}}%
\pgfpathlineto{\pgfqpoint{0.893172in}{1.442995in}}%
\pgfpathlineto{\pgfqpoint{0.894214in}{1.444507in}}%
\pgfpathlineto{\pgfqpoint{0.895018in}{1.445578in}}%
\pgfpathlineto{\pgfqpoint{0.896117in}{1.447216in}}%
\pgfpathlineto{\pgfqpoint{0.896849in}{1.448161in}}%
\pgfpathlineto{\pgfqpoint{0.897934in}{1.448917in}}%
\pgfpathlineto{\pgfqpoint{0.898864in}{1.449988in}}%
\pgfpathlineto{\pgfqpoint{0.899963in}{1.451185in}}%
\pgfpathlineto{\pgfqpoint{0.900795in}{1.452256in}}%
\pgfpathlineto{\pgfqpoint{0.901838in}{1.454020in}}%
\pgfpathlineto{\pgfqpoint{0.902655in}{1.455091in}}%
\pgfpathlineto{\pgfqpoint{0.903683in}{1.456162in}}%
\pgfpathlineto{\pgfqpoint{0.904839in}{1.457233in}}%
\pgfpathlineto{\pgfqpoint{0.905924in}{1.459186in}}%
\pgfpathlineto{\pgfqpoint{0.907248in}{1.460257in}}%
\pgfpathlineto{\pgfqpoint{0.908277in}{1.462273in}}%
\pgfpathlineto{\pgfqpoint{0.909559in}{1.463344in}}%
\pgfpathlineto{\pgfqpoint{0.910560in}{1.464604in}}%
\pgfpathlineto{\pgfqpoint{0.911786in}{1.465675in}}%
\pgfpathlineto{\pgfqpoint{0.912800in}{1.466746in}}%
\pgfpathlineto{\pgfqpoint{0.913744in}{1.467817in}}%
\pgfpathlineto{\pgfqpoint{0.914801in}{1.469140in}}%
\pgfpathlineto{\pgfqpoint{0.915689in}{1.470211in}}%
\pgfpathlineto{\pgfqpoint{0.916788in}{1.471786in}}%
\pgfpathlineto{\pgfqpoint{0.917478in}{1.472857in}}%
\pgfpathlineto{\pgfqpoint{0.918493in}{1.474495in}}%
\pgfpathlineto{\pgfqpoint{0.920001in}{1.475566in}}%
\pgfpathlineto{\pgfqpoint{0.921029in}{1.476763in}}%
\pgfpathlineto{\pgfqpoint{0.922382in}{1.477771in}}%
\pgfpathlineto{\pgfqpoint{0.923467in}{1.478590in}}%
\pgfpathlineto{\pgfqpoint{0.924017in}{1.479661in}}%
\pgfpathlineto{\pgfqpoint{0.925059in}{1.480669in}}%
\pgfpathlineto{\pgfqpoint{0.926356in}{1.481740in}}%
\pgfpathlineto{\pgfqpoint{0.927455in}{1.483378in}}%
\pgfpathlineto{\pgfqpoint{0.928216in}{1.484449in}}%
\pgfpathlineto{\pgfqpoint{0.929315in}{1.485520in}}%
\pgfpathlineto{\pgfqpoint{0.930033in}{1.486591in}}%
\pgfpathlineto{\pgfqpoint{0.931133in}{1.487977in}}%
\pgfpathlineto{\pgfqpoint{0.931950in}{1.489048in}}%
\pgfpathlineto{\pgfqpoint{0.933049in}{1.490686in}}%
\pgfpathlineto{\pgfqpoint{0.933993in}{1.491757in}}%
\pgfpathlineto{\pgfqpoint{0.935050in}{1.492765in}}%
\pgfpathlineto{\pgfqpoint{0.936205in}{1.493836in}}%
\pgfpathlineto{\pgfqpoint{0.937206in}{1.495348in}}%
\pgfpathlineto{\pgfqpoint{0.938403in}{1.496419in}}%
\pgfpathlineto{\pgfqpoint{0.939488in}{1.497742in}}%
\pgfpathlineto{\pgfqpoint{0.940616in}{1.498813in}}%
\pgfpathlineto{\pgfqpoint{0.941588in}{1.500073in}}%
\pgfpathlineto{\pgfqpoint{0.942758in}{1.501144in}}%
\pgfpathlineto{\pgfqpoint{0.943814in}{1.501963in}}%
\pgfpathlineto{\pgfqpoint{0.944970in}{1.502971in}}%
\pgfpathlineto{\pgfqpoint{0.945843in}{1.504105in}}%
\pgfpathlineto{\pgfqpoint{0.947379in}{1.505176in}}%
\pgfpathlineto{\pgfqpoint{0.948408in}{1.506058in}}%
\pgfpathlineto{\pgfqpoint{0.949930in}{1.507129in}}%
\pgfpathlineto{\pgfqpoint{0.950958in}{1.508389in}}%
\pgfpathlineto{\pgfqpoint{0.952058in}{1.509460in}}%
\pgfpathlineto{\pgfqpoint{0.953143in}{1.510909in}}%
\pgfpathlineto{\pgfqpoint{0.954284in}{1.511854in}}%
\pgfpathlineto{\pgfqpoint{0.955270in}{1.512484in}}%
\pgfpathlineto{\pgfqpoint{0.956778in}{1.513555in}}%
\pgfpathlineto{\pgfqpoint{0.957835in}{1.514437in}}%
\pgfpathlineto{\pgfqpoint{0.959314in}{1.515508in}}%
\pgfpathlineto{\pgfqpoint{0.960385in}{1.516453in}}%
\pgfpathlineto{\pgfqpoint{0.961428in}{1.517524in}}%
\pgfpathlineto{\pgfqpoint{0.962358in}{1.518973in}}%
\pgfpathlineto{\pgfqpoint{0.963978in}{1.520044in}}%
\pgfpathlineto{\pgfqpoint{0.965078in}{1.521052in}}%
\pgfpathlineto{\pgfqpoint{0.965867in}{1.522123in}}%
\pgfpathlineto{\pgfqpoint{0.966895in}{1.523005in}}%
\pgfpathlineto{\pgfqpoint{0.968361in}{1.524076in}}%
\pgfpathlineto{\pgfqpoint{0.969375in}{1.524958in}}%
\pgfpathlineto{\pgfqpoint{0.970756in}{1.526029in}}%
\pgfpathlineto{\pgfqpoint{0.971855in}{1.527352in}}%
\pgfpathlineto{\pgfqpoint{0.973138in}{1.528423in}}%
\pgfpathlineto{\pgfqpoint{0.974138in}{1.529494in}}%
\pgfpathlineto{\pgfqpoint{0.975420in}{1.530565in}}%
\pgfpathlineto{\pgfqpoint{0.976294in}{1.531321in}}%
\pgfpathlineto{\pgfqpoint{0.978083in}{1.532392in}}%
\pgfpathlineto{\pgfqpoint{0.979098in}{1.533463in}}%
\pgfpathlineto{\pgfqpoint{0.981085in}{1.534534in}}%
\pgfpathlineto{\pgfqpoint{0.982184in}{1.535668in}}%
\pgfpathlineto{\pgfqpoint{0.983241in}{1.536739in}}%
\pgfpathlineto{\pgfqpoint{0.984298in}{1.537747in}}%
\pgfpathlineto{\pgfqpoint{0.985566in}{1.538818in}}%
\pgfpathlineto{\pgfqpoint{0.986637in}{1.540015in}}%
\pgfpathlineto{\pgfqpoint{0.988257in}{1.541086in}}%
\pgfpathlineto{\pgfqpoint{0.989243in}{1.541716in}}%
\pgfpathlineto{\pgfqpoint{0.990653in}{1.542787in}}%
\pgfpathlineto{\pgfqpoint{0.991399in}{1.543480in}}%
\pgfpathlineto{\pgfqpoint{0.993189in}{1.544551in}}%
\pgfpathlineto{\pgfqpoint{0.994288in}{1.545433in}}%
\pgfpathlineto{\pgfqpoint{0.995598in}{1.546504in}}%
\pgfpathlineto{\pgfqpoint{0.996641in}{1.547575in}}%
\pgfpathlineto{\pgfqpoint{0.998445in}{1.548646in}}%
\pgfpathlineto{\pgfqpoint{0.999375in}{1.549402in}}%
\pgfpathlineto{\pgfqpoint{1.000854in}{1.550473in}}%
\pgfpathlineto{\pgfqpoint{1.001672in}{1.551103in}}%
\pgfpathlineto{\pgfqpoint{1.003363in}{1.552174in}}%
\pgfpathlineto{\pgfqpoint{1.004448in}{1.552930in}}%
\pgfpathlineto{\pgfqpoint{1.005730in}{1.554001in}}%
\pgfpathlineto{\pgfqpoint{1.006702in}{1.555009in}}%
\pgfpathlineto{\pgfqpoint{1.008196in}{1.556080in}}%
\pgfpathlineto{\pgfqpoint{1.009267in}{1.556962in}}%
\pgfpathlineto{\pgfqpoint{1.011000in}{1.558033in}}%
\pgfpathlineto{\pgfqpoint{1.011930in}{1.558852in}}%
\pgfpathlineto{\pgfqpoint{1.013607in}{1.559797in}}%
\pgfpathlineto{\pgfqpoint{1.014565in}{1.560490in}}%
\pgfpathlineto{\pgfqpoint{1.016453in}{1.561561in}}%
\pgfpathlineto{\pgfqpoint{1.017496in}{1.562254in}}%
\pgfpathlineto{\pgfqpoint{1.019060in}{1.563262in}}%
\pgfpathlineto{\pgfqpoint{1.019990in}{1.563955in}}%
\pgfpathlineto{\pgfqpoint{1.021836in}{1.564963in}}%
\pgfpathlineto{\pgfqpoint{1.022893in}{1.565908in}}%
\pgfpathlineto{\pgfqpoint{1.023780in}{1.566979in}}%
\pgfpathlineto{\pgfqpoint{1.024696in}{1.568050in}}%
\pgfpathlineto{\pgfqpoint{1.027162in}{1.569121in}}%
\pgfpathlineto{\pgfqpoint{1.028191in}{1.570192in}}%
\pgfpathlineto{\pgfqpoint{1.030023in}{1.571263in}}%
\pgfpathlineto{\pgfqpoint{1.031037in}{1.572082in}}%
\pgfpathlineto{\pgfqpoint{1.033207in}{1.573153in}}%
\pgfpathlineto{\pgfqpoint{1.034123in}{1.573720in}}%
\pgfpathlineto{\pgfqpoint{1.035800in}{1.574791in}}%
\pgfpathlineto{\pgfqpoint{1.036885in}{1.575799in}}%
\pgfpathlineto{\pgfqpoint{1.039449in}{1.576870in}}%
\pgfpathlineto{\pgfqpoint{1.040520in}{1.577941in}}%
\pgfpathlineto{\pgfqpoint{1.042718in}{1.579012in}}%
\pgfpathlineto{\pgfqpoint{1.043818in}{1.579516in}}%
\pgfpathlineto{\pgfqpoint{1.046128in}{1.580587in}}%
\pgfpathlineto{\pgfqpoint{1.047199in}{1.581406in}}%
\pgfpathlineto{\pgfqpoint{1.050384in}{1.582477in}}%
\pgfpathlineto{\pgfqpoint{1.051328in}{1.583044in}}%
\pgfpathlineto{\pgfqpoint{1.053061in}{1.584115in}}%
\pgfpathlineto{\pgfqpoint{1.054118in}{1.585186in}}%
\pgfpathlineto{\pgfqpoint{1.056218in}{1.586194in}}%
\pgfpathlineto{\pgfqpoint{1.057274in}{1.587013in}}%
\pgfpathlineto{\pgfqpoint{1.059247in}{1.588084in}}%
\pgfpathlineto{\pgfqpoint{1.060304in}{1.588651in}}%
\pgfpathlineto{\pgfqpoint{1.062051in}{1.589722in}}%
\pgfpathlineto{\pgfqpoint{1.062826in}{1.590163in}}%
\pgfpathlineto{\pgfqpoint{1.064489in}{1.591234in}}%
\pgfpathlineto{\pgfqpoint{1.065489in}{1.592305in}}%
\pgfpathlineto{\pgfqpoint{1.066828in}{1.593376in}}%
\pgfpathlineto{\pgfqpoint{1.067800in}{1.594195in}}%
\pgfpathlineto{\pgfqpoint{1.069815in}{1.595203in}}%
\pgfpathlineto{\pgfqpoint{1.070816in}{1.595896in}}%
\pgfpathlineto{\pgfqpoint{1.073662in}{1.596967in}}%
\pgfpathlineto{\pgfqpoint{1.074550in}{1.597471in}}%
\pgfpathlineto{\pgfqpoint{1.076593in}{1.598542in}}%
\pgfpathlineto{\pgfqpoint{1.077213in}{1.599487in}}%
\pgfpathlineto{\pgfqpoint{1.079383in}{1.600558in}}%
\pgfpathlineto{\pgfqpoint{1.080454in}{1.601377in}}%
\pgfpathlineto{\pgfqpoint{1.082074in}{1.602448in}}%
\pgfpathlineto{\pgfqpoint{1.082990in}{1.603141in}}%
\pgfpathlineto{\pgfqpoint{1.085259in}{1.604212in}}%
\pgfpathlineto{\pgfqpoint{1.086358in}{1.604716in}}%
\pgfpathlineto{\pgfqpoint{1.088232in}{1.605787in}}%
\pgfpathlineto{\pgfqpoint{1.089233in}{1.606291in}}%
\pgfpathlineto{\pgfqpoint{1.091938in}{1.607362in}}%
\pgfpathlineto{\pgfqpoint{1.093023in}{1.608433in}}%
\pgfpathlineto{\pgfqpoint{1.095503in}{1.609504in}}%
\pgfpathlineto{\pgfqpoint{1.096123in}{1.609945in}}%
\pgfpathlineto{\pgfqpoint{1.098053in}{1.610953in}}%
\pgfpathlineto{\pgfqpoint{1.099026in}{1.611583in}}%
\pgfpathlineto{\pgfqpoint{1.101675in}{1.612654in}}%
\pgfpathlineto{\pgfqpoint{1.102760in}{1.613284in}}%
\pgfpathlineto{\pgfqpoint{1.104338in}{1.614355in}}%
\pgfpathlineto{\pgfqpoint{1.105310in}{1.615048in}}%
\pgfpathlineto{\pgfqpoint{1.107790in}{1.616119in}}%
\pgfpathlineto{\pgfqpoint{1.108847in}{1.616623in}}%
\pgfpathlineto{\pgfqpoint{1.111257in}{1.617694in}}%
\pgfpathlineto{\pgfqpoint{1.112229in}{1.618387in}}%
\pgfpathlineto{\pgfqpoint{1.114469in}{1.619458in}}%
\pgfpathlineto{\pgfqpoint{1.115470in}{1.620214in}}%
\pgfpathlineto{\pgfqpoint{1.117316in}{1.621285in}}%
\pgfpathlineto{\pgfqpoint{1.118330in}{1.622293in}}%
\pgfpathlineto{\pgfqpoint{1.119514in}{1.623364in}}%
\pgfpathlineto{\pgfqpoint{1.120500in}{1.623742in}}%
\pgfpathlineto{\pgfqpoint{1.122318in}{1.624813in}}%
\pgfpathlineto{\pgfqpoint{1.123276in}{1.625128in}}%
\pgfpathlineto{\pgfqpoint{1.125601in}{1.626199in}}%
\pgfpathlineto{\pgfqpoint{1.126686in}{1.626892in}}%
\pgfpathlineto{\pgfqpoint{1.128955in}{1.627963in}}%
\pgfpathlineto{\pgfqpoint{1.130026in}{1.628593in}}%
\pgfpathlineto{\pgfqpoint{1.131688in}{1.629664in}}%
\pgfpathlineto{\pgfqpoint{1.132590in}{1.629916in}}%
\pgfpathlineto{\pgfqpoint{1.135916in}{1.630987in}}%
\pgfpathlineto{\pgfqpoint{1.137015in}{1.631743in}}%
\pgfpathlineto{\pgfqpoint{1.139171in}{1.632814in}}%
\pgfpathlineto{\pgfqpoint{1.140228in}{1.633570in}}%
\pgfpathlineto{\pgfqpoint{1.141918in}{1.634578in}}%
\pgfpathlineto{\pgfqpoint{1.142975in}{1.635271in}}%
\pgfpathlineto{\pgfqpoint{1.145089in}{1.636342in}}%
\pgfpathlineto{\pgfqpoint{1.146061in}{1.636909in}}%
\pgfpathlineto{\pgfqpoint{1.148513in}{1.637980in}}%
\pgfpathlineto{\pgfqpoint{1.149584in}{1.638421in}}%
\pgfpathlineto{\pgfqpoint{1.153135in}{1.639492in}}%
\pgfpathlineto{\pgfqpoint{1.154093in}{1.639996in}}%
\pgfpathlineto{\pgfqpoint{1.157193in}{1.641067in}}%
\pgfpathlineto{\pgfqpoint{1.158109in}{1.641697in}}%
\pgfpathlineto{\pgfqpoint{1.161505in}{1.642705in}}%
\pgfpathlineto{\pgfqpoint{1.162548in}{1.643461in}}%
\pgfpathlineto{\pgfqpoint{1.165732in}{1.644532in}}%
\pgfpathlineto{\pgfqpoint{1.166451in}{1.644784in}}%
\pgfpathlineto{\pgfqpoint{1.169973in}{1.645855in}}%
\pgfpathlineto{\pgfqpoint{1.171073in}{1.646485in}}%
\pgfpathlineto{\pgfqpoint{1.173750in}{1.647556in}}%
\pgfpathlineto{\pgfqpoint{1.174595in}{1.647871in}}%
\pgfpathlineto{\pgfqpoint{1.178174in}{1.648942in}}%
\pgfpathlineto{\pgfqpoint{1.179273in}{1.649257in}}%
\pgfpathlineto{\pgfqpoint{1.182712in}{1.650328in}}%
\pgfpathlineto{\pgfqpoint{1.183628in}{1.650706in}}%
\pgfpathlineto{\pgfqpoint{1.187348in}{1.651714in}}%
\pgfpathlineto{\pgfqpoint{1.188362in}{1.652092in}}%
\pgfpathlineto{\pgfqpoint{1.191645in}{1.653163in}}%
\pgfpathlineto{\pgfqpoint{1.192589in}{1.653793in}}%
\pgfpathlineto{\pgfqpoint{1.195154in}{1.654864in}}%
\pgfpathlineto{\pgfqpoint{1.196197in}{1.655620in}}%
\pgfpathlineto{\pgfqpoint{1.198338in}{1.656691in}}%
\pgfpathlineto{\pgfqpoint{1.199113in}{1.656943in}}%
\pgfpathlineto{\pgfqpoint{1.203242in}{1.658014in}}%
\pgfpathlineto{\pgfqpoint{1.204299in}{1.658329in}}%
\pgfpathlineto{\pgfqpoint{1.207822in}{1.659337in}}%
\pgfpathlineto{\pgfqpoint{1.208808in}{1.659778in}}%
\pgfpathlineto{\pgfqpoint{1.211175in}{1.660849in}}%
\pgfpathlineto{\pgfqpoint{1.212246in}{1.661290in}}%
\pgfpathlineto{\pgfqpoint{1.214881in}{1.662361in}}%
\pgfpathlineto{\pgfqpoint{1.215896in}{1.662865in}}%
\pgfpathlineto{\pgfqpoint{1.217798in}{1.663936in}}%
\pgfpathlineto{\pgfqpoint{1.218474in}{1.664440in}}%
\pgfpathlineto{\pgfqpoint{1.223054in}{1.665511in}}%
\pgfpathlineto{\pgfqpoint{1.223787in}{1.665826in}}%
\pgfpathlineto{\pgfqpoint{1.227746in}{1.666897in}}%
\pgfpathlineto{\pgfqpoint{1.228662in}{1.667338in}}%
\pgfpathlineto{\pgfqpoint{1.231297in}{1.668409in}}%
\pgfpathlineto{\pgfqpoint{1.232340in}{1.668850in}}%
\pgfpathlineto{\pgfqpoint{1.235609in}{1.669921in}}%
\pgfpathlineto{\pgfqpoint{1.236694in}{1.670803in}}%
\pgfpathlineto{\pgfqpoint{1.240470in}{1.671874in}}%
\pgfpathlineto{\pgfqpoint{1.241259in}{1.672252in}}%
\pgfpathlineto{\pgfqpoint{1.244740in}{1.673323in}}%
\pgfpathlineto{\pgfqpoint{1.245811in}{1.673827in}}%
\pgfpathlineto{\pgfqpoint{1.248361in}{1.674898in}}%
\pgfpathlineto{\pgfqpoint{1.249207in}{1.675654in}}%
\pgfpathlineto{\pgfqpoint{1.253363in}{1.676725in}}%
\pgfpathlineto{\pgfqpoint{1.254308in}{1.677103in}}%
\pgfpathlineto{\pgfqpoint{1.259169in}{1.678174in}}%
\pgfpathlineto{\pgfqpoint{1.259986in}{1.678552in}}%
\pgfpathlineto{\pgfqpoint{1.265059in}{1.679623in}}%
\pgfpathlineto{\pgfqpoint{1.266017in}{1.680316in}}%
\pgfpathlineto{\pgfqpoint{1.270188in}{1.681387in}}%
\pgfpathlineto{\pgfqpoint{1.270836in}{1.681513in}}%
\pgfpathlineto{\pgfqpoint{1.274739in}{1.682584in}}%
\pgfpathlineto{\pgfqpoint{1.275486in}{1.683025in}}%
\pgfpathlineto{\pgfqpoint{1.280488in}{1.684096in}}%
\pgfpathlineto{\pgfqpoint{1.281362in}{1.684348in}}%
\pgfpathlineto{\pgfqpoint{1.285322in}{1.685419in}}%
\pgfpathlineto{\pgfqpoint{1.286364in}{1.686112in}}%
\pgfpathlineto{\pgfqpoint{1.289788in}{1.687183in}}%
\pgfpathlineto{\pgfqpoint{1.290845in}{1.687687in}}%
\pgfpathlineto{\pgfqpoint{1.293973in}{1.688758in}}%
\pgfpathlineto{\pgfqpoint{1.295016in}{1.689262in}}%
\pgfpathlineto{\pgfqpoint{1.299229in}{1.690333in}}%
\pgfpathlineto{\pgfqpoint{1.299694in}{1.690585in}}%
\pgfpathlineto{\pgfqpoint{1.303781in}{1.691656in}}%
\pgfpathlineto{\pgfqpoint{1.304697in}{1.691908in}}%
\pgfpathlineto{\pgfqpoint{1.309304in}{1.692916in}}%
\pgfpathlineto{\pgfqpoint{1.310248in}{1.693294in}}%
\pgfpathlineto{\pgfqpoint{1.315462in}{1.694365in}}%
\pgfpathlineto{\pgfqpoint{1.316463in}{1.695184in}}%
\pgfpathlineto{\pgfqpoint{1.320154in}{1.696255in}}%
\pgfpathlineto{\pgfqpoint{1.321113in}{1.696633in}}%
\pgfpathlineto{\pgfqpoint{1.329356in}{1.697704in}}%
\pgfpathlineto{\pgfqpoint{1.330441in}{1.698208in}}%
\pgfpathlineto{\pgfqpoint{1.335852in}{1.699279in}}%
\pgfpathlineto{\pgfqpoint{1.336880in}{1.699531in}}%
\pgfpathlineto{\pgfqpoint{1.341418in}{1.700602in}}%
\pgfpathlineto{\pgfqpoint{1.342094in}{1.700791in}}%
\pgfpathlineto{\pgfqpoint{1.346843in}{1.701862in}}%
\pgfpathlineto{\pgfqpoint{1.347336in}{1.702240in}}%
\pgfpathlineto{\pgfqpoint{1.352493in}{1.703311in}}%
\pgfpathlineto{\pgfqpoint{1.353000in}{1.703626in}}%
\pgfpathlineto{\pgfqpoint{1.361441in}{1.704697in}}%
\pgfpathlineto{\pgfqpoint{1.362512in}{1.705264in}}%
\pgfpathlineto{\pgfqpoint{1.367373in}{1.706335in}}%
\pgfpathlineto{\pgfqpoint{1.368035in}{1.706587in}}%
\pgfpathlineto{\pgfqpoint{1.376588in}{1.707658in}}%
\pgfpathlineto{\pgfqpoint{1.377518in}{1.707973in}}%
\pgfpathlineto{\pgfqpoint{1.383380in}{1.709044in}}%
\pgfpathlineto{\pgfqpoint{1.384423in}{1.709485in}}%
\pgfpathlineto{\pgfqpoint{1.389355in}{1.710556in}}%
\pgfpathlineto{\pgfqpoint{1.390299in}{1.710808in}}%
\pgfpathlineto{\pgfqpoint{1.397020in}{1.711879in}}%
\pgfpathlineto{\pgfqpoint{1.398119in}{1.712194in}}%
\pgfpathlineto{\pgfqpoint{1.404531in}{1.713265in}}%
\pgfpathlineto{\pgfqpoint{1.405559in}{1.713832in}}%
\pgfpathlineto{\pgfqpoint{1.411492in}{1.714903in}}%
\pgfpathlineto{\pgfqpoint{1.412379in}{1.715218in}}%
\pgfpathlineto{\pgfqpoint{1.419763in}{1.716289in}}%
\pgfpathlineto{\pgfqpoint{1.420848in}{1.716541in}}%
\pgfpathlineto{\pgfqpoint{1.425512in}{1.717612in}}%
\pgfpathlineto{\pgfqpoint{1.426555in}{1.717864in}}%
\pgfpathlineto{\pgfqpoint{1.431458in}{1.718935in}}%
\pgfpathlineto{\pgfqpoint{1.432163in}{1.719061in}}%
\pgfpathlineto{\pgfqpoint{1.438927in}{1.720132in}}%
\pgfpathlineto{\pgfqpoint{1.439688in}{1.720447in}}%
\pgfpathlineto{\pgfqpoint{1.446057in}{1.721518in}}%
\pgfpathlineto{\pgfqpoint{1.446803in}{1.722022in}}%
\pgfpathlineto{\pgfqpoint{1.456357in}{1.723093in}}%
\pgfpathlineto{\pgfqpoint{1.456456in}{1.723219in}}%
\pgfpathlineto{\pgfqpoint{1.462754in}{1.724290in}}%
\pgfpathlineto{\pgfqpoint{1.463543in}{1.724542in}}%
\pgfpathlineto{\pgfqpoint{1.471364in}{1.725613in}}%
\pgfpathlineto{\pgfqpoint{1.472125in}{1.725802in}}%
\pgfpathlineto{\pgfqpoint{1.477803in}{1.726873in}}%
\pgfpathlineto{\pgfqpoint{1.478184in}{1.727062in}}%
\pgfpathlineto{\pgfqpoint{1.484736in}{1.728133in}}%
\pgfpathlineto{\pgfqpoint{1.485737in}{1.728385in}}%
\pgfpathlineto{\pgfqpoint{1.492937in}{1.729456in}}%
\pgfpathlineto{\pgfqpoint{1.494036in}{1.729897in}}%
\pgfpathlineto{\pgfqpoint{1.498898in}{1.730968in}}%
\pgfpathlineto{\pgfqpoint{1.499870in}{1.731220in}}%
\pgfpathlineto{\pgfqpoint{1.506056in}{1.732291in}}%
\pgfpathlineto{\pgfqpoint{1.507042in}{1.732606in}}%
\pgfpathlineto{\pgfqpoint{1.511495in}{1.733677in}}%
\pgfpathlineto{\pgfqpoint{1.512326in}{1.734307in}}%
\pgfpathlineto{\pgfqpoint{1.520443in}{1.735378in}}%
\pgfpathlineto{\pgfqpoint{1.521387in}{1.735504in}}%
\pgfpathlineto{\pgfqpoint{1.527939in}{1.736575in}}%
\pgfpathlineto{\pgfqpoint{1.528784in}{1.736890in}}%
\pgfpathlineto{\pgfqpoint{1.536619in}{1.737961in}}%
\pgfpathlineto{\pgfqpoint{1.537493in}{1.738213in}}%
\pgfpathlineto{\pgfqpoint{1.543946in}{1.739284in}}%
\pgfpathlineto{\pgfqpoint{1.544904in}{1.739662in}}%
\pgfpathlineto{\pgfqpoint{1.550583in}{1.740733in}}%
\pgfpathlineto{\pgfqpoint{1.550823in}{1.740859in}}%
\pgfpathlineto{\pgfqpoint{1.557586in}{1.741930in}}%
\pgfpathlineto{\pgfqpoint{1.557657in}{1.742119in}}%
\pgfpathlineto{\pgfqpoint{1.570057in}{1.743190in}}%
\pgfpathlineto{\pgfqpoint{1.570691in}{1.743505in}}%
\pgfpathlineto{\pgfqpoint{1.575200in}{1.744576in}}%
\pgfpathlineto{\pgfqpoint{1.576144in}{1.744891in}}%
\pgfpathlineto{\pgfqpoint{1.585500in}{1.745962in}}%
\pgfpathlineto{\pgfqpoint{1.586360in}{1.746151in}}%
\pgfpathlineto{\pgfqpoint{1.597520in}{1.747222in}}%
\pgfpathlineto{\pgfqpoint{1.597520in}{1.747285in}}%
\pgfpathlineto{\pgfqpoint{1.610258in}{1.748356in}}%
\pgfpathlineto{\pgfqpoint{1.610638in}{1.748608in}}%
\pgfpathlineto{\pgfqpoint{1.622348in}{1.749679in}}%
\pgfpathlineto{\pgfqpoint{1.623053in}{1.749868in}}%
\pgfpathlineto{\pgfqpoint{1.630619in}{1.750939in}}%
\pgfpathlineto{\pgfqpoint{1.630915in}{1.751065in}}%
\pgfpathlineto{\pgfqpoint{1.638496in}{1.752136in}}%
\pgfpathlineto{\pgfqpoint{1.639440in}{1.752451in}}%
\pgfpathlineto{\pgfqpoint{1.650121in}{1.753522in}}%
\pgfpathlineto{\pgfqpoint{1.650995in}{1.753774in}}%
\pgfpathlineto{\pgfqpoint{1.659083in}{1.754845in}}%
\pgfpathlineto{\pgfqpoint{1.659816in}{1.755097in}}%
\pgfpathlineto{\pgfqpoint{1.668186in}{1.756168in}}%
\pgfpathlineto{\pgfqpoint{1.668918in}{1.756294in}}%
\pgfpathlineto{\pgfqpoint{1.676288in}{1.757365in}}%
\pgfpathlineto{\pgfqpoint{1.677162in}{1.757554in}}%
\pgfpathlineto{\pgfqpoint{1.686617in}{1.758625in}}%
\pgfpathlineto{\pgfqpoint{1.687589in}{1.758877in}}%
\pgfpathlineto{\pgfqpoint{1.692493in}{1.759948in}}%
\pgfpathlineto{\pgfqpoint{1.693521in}{1.760200in}}%
\pgfpathlineto{\pgfqpoint{1.703272in}{1.761271in}}%
\pgfpathlineto{\pgfqpoint{1.703272in}{1.761334in}}%
\pgfpathlineto{\pgfqpoint{1.714742in}{1.762405in}}%
\pgfpathlineto{\pgfqpoint{1.715743in}{1.762657in}}%
\pgfpathlineto{\pgfqpoint{1.727297in}{1.763728in}}%
\pgfpathlineto{\pgfqpoint{1.727847in}{1.763854in}}%
\pgfpathlineto{\pgfqpoint{1.734850in}{1.764925in}}%
\pgfpathlineto{\pgfqpoint{1.735470in}{1.765051in}}%
\pgfpathlineto{\pgfqpoint{1.747081in}{1.766122in}}%
\pgfpathlineto{\pgfqpoint{1.747081in}{1.766185in}}%
\pgfpathlineto{\pgfqpoint{1.756310in}{1.767256in}}%
\pgfpathlineto{\pgfqpoint{1.756522in}{1.767445in}}%
\pgfpathlineto{\pgfqpoint{1.768738in}{1.768516in}}%
\pgfpathlineto{\pgfqpoint{1.769584in}{1.768642in}}%
\pgfpathlineto{\pgfqpoint{1.780702in}{1.769713in}}%
\pgfpathlineto{\pgfqpoint{1.780969in}{1.769902in}}%
\pgfpathlineto{\pgfqpoint{1.792327in}{1.770973in}}%
\pgfpathlineto{\pgfqpoint{1.792327in}{1.771036in}}%
\pgfpathlineto{\pgfqpoint{1.802416in}{1.772107in}}%
\pgfpathlineto{\pgfqpoint{1.803501in}{1.772296in}}%
\pgfpathlineto{\pgfqpoint{1.816352in}{1.773367in}}%
\pgfpathlineto{\pgfqpoint{1.817169in}{1.773493in}}%
\pgfpathlineto{\pgfqpoint{1.827413in}{1.774564in}}%
\pgfpathlineto{\pgfqpoint{1.828512in}{1.774690in}}%
\pgfpathlineto{\pgfqpoint{1.836995in}{1.775761in}}%
\pgfpathlineto{\pgfqpoint{1.837530in}{1.775887in}}%
\pgfpathlineto{\pgfqpoint{1.854820in}{1.776958in}}%
\pgfpathlineto{\pgfqpoint{1.855130in}{1.777084in}}%
\pgfpathlineto{\pgfqpoint{1.873547in}{1.778155in}}%
\pgfpathlineto{\pgfqpoint{1.874153in}{1.778281in}}%
\pgfpathlineto{\pgfqpoint{1.893711in}{1.779352in}}%
\pgfpathlineto{\pgfqpoint{1.894810in}{1.779667in}}%
\pgfpathlineto{\pgfqpoint{1.905279in}{1.780738in}}%
\pgfpathlineto{\pgfqpoint{1.905350in}{1.780864in}}%
\pgfpathlineto{\pgfqpoint{1.914481in}{1.781935in}}%
\pgfpathlineto{\pgfqpoint{1.915115in}{1.782061in}}%
\pgfpathlineto{\pgfqpoint{1.928952in}{1.783132in}}%
\pgfpathlineto{\pgfqpoint{1.929938in}{1.783258in}}%
\pgfpathlineto{\pgfqpoint{1.944931in}{1.784329in}}%
\pgfpathlineto{\pgfqpoint{1.945636in}{1.784455in}}%
\pgfpathlineto{\pgfqpoint{1.957402in}{1.785526in}}%
\pgfpathlineto{\pgfqpoint{1.957712in}{1.785652in}}%
\pgfpathlineto{\pgfqpoint{1.969830in}{1.786723in}}%
\pgfpathlineto{\pgfqpoint{1.970492in}{1.786849in}}%
\pgfpathlineto{\pgfqpoint{1.988162in}{1.787920in}}%
\pgfpathlineto{\pgfqpoint{1.988669in}{1.788109in}}%
\pgfpathlineto{\pgfqpoint{1.999364in}{1.789180in}}%
\pgfpathlineto{\pgfqpoint{1.999590in}{1.789306in}}%
\pgfpathlineto{\pgfqpoint{2.017936in}{1.790377in}}%
\pgfpathlineto{\pgfqpoint{2.018810in}{1.790503in}}%
\pgfpathlineto{\pgfqpoint{2.030491in}{1.791574in}}%
\pgfpathlineto{\pgfqpoint{2.031238in}{1.791889in}}%
\pgfpathlineto{\pgfqpoint{2.046188in}{1.792960in}}%
\pgfpathlineto{\pgfqpoint{2.046188in}{1.793023in}}%
\pgfpathlineto{\pgfqpoint{2.063999in}{1.794094in}}%
\pgfpathlineto{\pgfqpoint{2.064309in}{1.794220in}}%
\pgfpathlineto{\pgfqpoint{2.080979in}{1.795291in}}%
\pgfpathlineto{\pgfqpoint{2.081909in}{1.795606in}}%
\pgfpathlineto{\pgfqpoint{2.094337in}{1.796677in}}%
\pgfpathlineto{\pgfqpoint{2.094337in}{1.796740in}}%
\pgfpathlineto{\pgfqpoint{2.108019in}{1.797811in}}%
\pgfpathlineto{\pgfqpoint{2.108470in}{1.797937in}}%
\pgfpathlineto{\pgfqpoint{2.122265in}{1.799008in}}%
\pgfpathlineto{\pgfqpoint{2.122279in}{1.799134in}}%
\pgfpathlineto{\pgfqpoint{2.132707in}{1.800205in}}%
\pgfpathlineto{\pgfqpoint{2.132707in}{1.800268in}}%
\pgfpathlineto{\pgfqpoint{2.143810in}{1.801339in}}%
\pgfpathlineto{\pgfqpoint{2.144050in}{1.801465in}}%
\pgfpathlineto{\pgfqpoint{2.160607in}{1.802536in}}%
\pgfpathlineto{\pgfqpoint{2.161198in}{1.802662in}}%
\pgfpathlineto{\pgfqpoint{2.178446in}{1.803733in}}%
\pgfpathlineto{\pgfqpoint{2.178446in}{1.803796in}}%
\pgfpathlineto{\pgfqpoint{2.191311in}{1.804867in}}%
\pgfpathlineto{\pgfqpoint{2.191311in}{1.804930in}}%
\pgfpathlineto{\pgfqpoint{2.206938in}{1.806001in}}%
\pgfpathlineto{\pgfqpoint{2.207783in}{1.806127in}}%
\pgfpathlineto{\pgfqpoint{2.221916in}{1.807198in}}%
\pgfpathlineto{\pgfqpoint{2.221916in}{1.807261in}}%
\pgfpathlineto{\pgfqpoint{2.238797in}{1.808332in}}%
\pgfpathlineto{\pgfqpoint{2.239783in}{1.808521in}}%
\pgfpathlineto{\pgfqpoint{2.252451in}{1.809592in}}%
\pgfpathlineto{\pgfqpoint{2.252451in}{1.809655in}}%
\pgfpathlineto{\pgfqpoint{2.265344in}{1.810726in}}%
\pgfpathlineto{\pgfqpoint{2.266063in}{1.811041in}}%
\pgfpathlineto{\pgfqpoint{2.279534in}{1.812112in}}%
\pgfpathlineto{\pgfqpoint{2.280154in}{1.812301in}}%
\pgfpathlineto{\pgfqpoint{2.292836in}{1.813372in}}%
\pgfpathlineto{\pgfqpoint{2.293893in}{1.813624in}}%
\pgfpathlineto{\pgfqpoint{2.305602in}{1.814695in}}%
\pgfpathlineto{\pgfqpoint{2.305602in}{1.814758in}}%
\pgfpathlineto{\pgfqpoint{2.306039in}{1.814758in}}%
\pgfpathlineto{\pgfqpoint{2.329035in}{1.822444in}}%
\pgfpathlineto{\pgfqpoint{2.329035in}{1.822444in}}%
\pgfusepath{stroke}%
\end{pgfscope}%
\begin{pgfscope}%
\pgfsetrectcap%
\pgfsetmiterjoin%
\pgfsetlinewidth{0.803000pt}%
\definecolor{currentstroke}{rgb}{0.000000,0.000000,0.000000}%
\pgfsetstrokecolor{currentstroke}%
\pgfsetdash{}{0pt}%
\pgfpathmoveto{\pgfqpoint{0.553581in}{0.499444in}}%
\pgfpathlineto{\pgfqpoint{0.553581in}{1.885444in}}%
\pgfusepath{stroke}%
\end{pgfscope}%
\begin{pgfscope}%
\pgfsetrectcap%
\pgfsetmiterjoin%
\pgfsetlinewidth{0.803000pt}%
\definecolor{currentstroke}{rgb}{0.000000,0.000000,0.000000}%
\pgfsetstrokecolor{currentstroke}%
\pgfsetdash{}{0pt}%
\pgfpathmoveto{\pgfqpoint{2.413581in}{0.499444in}}%
\pgfpathlineto{\pgfqpoint{2.413581in}{1.885444in}}%
\pgfusepath{stroke}%
\end{pgfscope}%
\begin{pgfscope}%
\pgfsetrectcap%
\pgfsetmiterjoin%
\pgfsetlinewidth{0.803000pt}%
\definecolor{currentstroke}{rgb}{0.000000,0.000000,0.000000}%
\pgfsetstrokecolor{currentstroke}%
\pgfsetdash{}{0pt}%
\pgfpathmoveto{\pgfqpoint{0.553581in}{0.499444in}}%
\pgfpathlineto{\pgfqpoint{2.413581in}{0.499444in}}%
\pgfusepath{stroke}%
\end{pgfscope}%
\begin{pgfscope}%
\pgfsetrectcap%
\pgfsetmiterjoin%
\pgfsetlinewidth{0.803000pt}%
\definecolor{currentstroke}{rgb}{0.000000,0.000000,0.000000}%
\pgfsetstrokecolor{currentstroke}%
\pgfsetdash{}{0pt}%
\pgfpathmoveto{\pgfqpoint{0.553581in}{1.885444in}}%
\pgfpathlineto{\pgfqpoint{2.413581in}{1.885444in}}%
\pgfusepath{stroke}%
\end{pgfscope}%
\begin{pgfscope}%
\pgfsetbuttcap%
\pgfsetmiterjoin%
\definecolor{currentfill}{rgb}{1.000000,1.000000,1.000000}%
\pgfsetfillcolor{currentfill}%
\pgfsetlinewidth{1.003750pt}%
\definecolor{currentstroke}{rgb}{1.000000,1.000000,1.000000}%
\pgfsetstrokecolor{currentstroke}%
\pgfsetdash{}{0pt}%
\pgfpathmoveto{\pgfqpoint{1.428321in}{1.645381in}}%
\pgfpathlineto{\pgfqpoint{1.855821in}{1.645381in}}%
\pgfpathlineto{\pgfqpoint{1.855821in}{1.879826in}}%
\pgfpathlineto{\pgfqpoint{1.428321in}{1.879826in}}%
\pgfpathlineto{\pgfqpoint{1.428321in}{1.645381in}}%
\pgfpathclose%
\pgfusepath{stroke,fill}%
\end{pgfscope}%
\begin{pgfscope}%
\definecolor{textcolor}{rgb}{0.000000,0.000000,0.000000}%
\pgfsetstrokecolor{textcolor}%
\pgfsetfillcolor{textcolor}%
\pgftext[x=1.483877in,y=1.727881in,left,base]{\color{textcolor}\rmfamily\fontsize{10.000000}{12.000000}\selectfont 0.305}%
\end{pgfscope}%
\begin{pgfscope}%
\pgfsetbuttcap%
\pgfsetmiterjoin%
\definecolor{currentfill}{rgb}{1.000000,1.000000,1.000000}%
\pgfsetfillcolor{currentfill}%
\pgfsetlinewidth{1.003750pt}%
\definecolor{currentstroke}{rgb}{1.000000,1.000000,1.000000}%
\pgfsetstrokecolor{currentstroke}%
\pgfsetdash{}{0pt}%
\pgfpathmoveto{\pgfqpoint{0.703442in}{1.109944in}}%
\pgfpathlineto{\pgfqpoint{1.130942in}{1.109944in}}%
\pgfpathlineto{\pgfqpoint{1.130942in}{1.344389in}}%
\pgfpathlineto{\pgfqpoint{0.703442in}{1.344389in}}%
\pgfpathlineto{\pgfqpoint{0.703442in}{1.109944in}}%
\pgfpathclose%
\pgfusepath{stroke,fill}%
\end{pgfscope}%
\begin{pgfscope}%
\definecolor{textcolor}{rgb}{0.000000,0.000000,0.000000}%
\pgfsetstrokecolor{textcolor}%
\pgfsetfillcolor{textcolor}%
\pgftext[x=0.758998in,y=1.192444in,left,base]{\color{textcolor}\rmfamily\fontsize{10.000000}{12.000000}\selectfont 0.688}%
\end{pgfscope}%
\begin{pgfscope}%
\definecolor{textcolor}{rgb}{0.000000,0.000000,0.000000}%
\pgfsetstrokecolor{textcolor}%
\pgfsetfillcolor{textcolor}%
\pgftext[x=1.483581in,y=1.968778in,,base]{\color{textcolor}\rmfamily\fontsize{12.000000}{14.400000}\selectfont ROC Curve}%
\end{pgfscope}%
\begin{pgfscope}%
\pgfsetbuttcap%
\pgfsetmiterjoin%
\definecolor{currentfill}{rgb}{1.000000,1.000000,1.000000}%
\pgfsetfillcolor{currentfill}%
\pgfsetfillopacity{0.800000}%
\pgfsetlinewidth{1.003750pt}%
\definecolor{currentstroke}{rgb}{0.800000,0.800000,0.800000}%
\pgfsetstrokecolor{currentstroke}%
\pgfsetstrokeopacity{0.800000}%
\pgfsetdash{}{0pt}%
\pgfpathmoveto{\pgfqpoint{1.150525in}{0.568889in}}%
\pgfpathlineto{\pgfqpoint{2.316358in}{0.568889in}}%
\pgfpathquadraticcurveto{\pgfqpoint{2.344136in}{0.568889in}}{\pgfqpoint{2.344136in}{0.596666in}}%
\pgfpathlineto{\pgfqpoint{2.344136in}{0.791111in}}%
\pgfpathquadraticcurveto{\pgfqpoint{2.344136in}{0.818888in}}{\pgfqpoint{2.316358in}{0.818888in}}%
\pgfpathlineto{\pgfqpoint{1.150525in}{0.818888in}}%
\pgfpathquadraticcurveto{\pgfqpoint{1.122747in}{0.818888in}}{\pgfqpoint{1.122747in}{0.791111in}}%
\pgfpathlineto{\pgfqpoint{1.122747in}{0.596666in}}%
\pgfpathquadraticcurveto{\pgfqpoint{1.122747in}{0.568889in}}{\pgfqpoint{1.150525in}{0.568889in}}%
\pgfpathlineto{\pgfqpoint{1.150525in}{0.568889in}}%
\pgfpathclose%
\pgfusepath{stroke,fill}%
\end{pgfscope}%
\begin{pgfscope}%
\pgfsetrectcap%
\pgfsetroundjoin%
\pgfsetlinewidth{1.505625pt}%
\definecolor{currentstroke}{rgb}{0.121569,0.466667,0.705882}%
\pgfsetstrokecolor{currentstroke}%
\pgfsetdash{}{0pt}%
\pgfpathmoveto{\pgfqpoint{1.178303in}{0.707777in}}%
\pgfpathlineto{\pgfqpoint{1.317192in}{0.707777in}}%
\pgfpathlineto{\pgfqpoint{1.456081in}{0.707777in}}%
\pgfusepath{stroke}%
\end{pgfscope}%
\begin{pgfscope}%
\definecolor{textcolor}{rgb}{0.000000,0.000000,0.000000}%
\pgfsetstrokecolor{textcolor}%
\pgfsetfillcolor{textcolor}%
\pgftext[x=1.567192in,y=0.659166in,left,base]{\color{textcolor}\rmfamily\fontsize{10.000000}{12.000000}\selectfont AUC 0.847)}%
\end{pgfscope}%
\end{pgfpicture}%
\makeatother%
\endgroup%

\end{tabular}
\end{center}

\begin{center}
\begin{tabular}{cc}
\begin{tabular}{cc|c|c|}
	&\multicolumn{1}{c}{}& \multicolumn{2}{c}{Prediction} \cr
	&\multicolumn{1}{c}{} & \multicolumn{1}{c}{N} & \multicolumn{1}{c}{P} \cr\cline{3-4}
	\multirow{2}{*}{Actual}&N & 67.5\% & 18.2\% \vrule width 0pt height 10pt depth 2pt \cr\cline{3-4}
	&P & 3.11\% & 11.2\% \vrule width 0pt height 10pt depth 2pt \cr\cline{3-4}
\end{tabular}
&
\begin{tabular}{ll}
0.787 & Accuracy \cr 
0.785 & Balanced Accuracy \cr 
0.380 & Precision \cr 
0.786 & Balanced Precision \cr 
0.782 & Recall \cr 
0.512 & F1 \cr 
0.784 & Balanced F1 \cr 
0.547 & Gmean \cr 
	\end{tabular}
\end{tabular}
\end{center}




%%%%%
In the ideal results above, the algorithm learned a useful model from the patterns in the data.  The results below illustrate the worst case scenario, where the algorithm does not learn a good model, usually because the data does not have a pattern that predicts the target variable.  In the ROC curve, the median values of the probabilities for the two classes are so close that the labels are on top of each other.  

\begin{center}
\begin{tabular}{p{0.5\textwidth} p{0.5\textwidth}}
  \vspace{0pt} %% Creator: Matplotlib, PGF backend
%%
%% To include the figure in your LaTeX document, write
%%   \input{<filename>.pgf}
%%
%% Make sure the required packages are loaded in your preamble
%%   \usepackage{pgf}
%%
%% Also ensure that all the required font packages are loaded; for instance,
%% the lmodern package is sometimes necessary when using math font.
%%   \usepackage{lmodern}
%%
%% Figures using additional raster images can only be included by \input if
%% they are in the same directory as the main LaTeX file. For loading figures
%% from other directories you can use the `import` package
%%   \usepackage{import}
%%
%% and then include the figures with
%%   \import{<path to file>}{<filename>.pgf}
%%
%% Matplotlib used the following preamble
%%   
%%   \usepackage{fontspec}
%%   \makeatletter\@ifpackageloaded{underscore}{}{\usepackage[strings]{underscore}}\makeatother
%%
\begingroup%
\makeatletter%
\begin{pgfpicture}%
\pgfpathrectangle{\pgfpointorigin}{\pgfqpoint{3.025556in}{3.044944in}}%
\pgfusepath{use as bounding box, clip}%
\begin{pgfscope}%
\pgfsetbuttcap%
\pgfsetmiterjoin%
\definecolor{currentfill}{rgb}{1.000000,1.000000,1.000000}%
\pgfsetfillcolor{currentfill}%
\pgfsetlinewidth{0.000000pt}%
\definecolor{currentstroke}{rgb}{1.000000,1.000000,1.000000}%
\pgfsetstrokecolor{currentstroke}%
\pgfsetdash{}{0pt}%
\pgfpathmoveto{\pgfqpoint{0.000000in}{0.000000in}}%
\pgfpathlineto{\pgfqpoint{3.025556in}{0.000000in}}%
\pgfpathlineto{\pgfqpoint{3.025556in}{3.044944in}}%
\pgfpathlineto{\pgfqpoint{0.000000in}{3.044944in}}%
\pgfpathlineto{\pgfqpoint{0.000000in}{0.000000in}}%
\pgfpathclose%
\pgfusepath{fill}%
\end{pgfscope}%
\begin{pgfscope}%
\pgfsetbuttcap%
\pgfsetmiterjoin%
\definecolor{currentfill}{rgb}{1.000000,1.000000,1.000000}%
\pgfsetfillcolor{currentfill}%
\pgfsetlinewidth{0.000000pt}%
\definecolor{currentstroke}{rgb}{0.000000,0.000000,0.000000}%
\pgfsetstrokecolor{currentstroke}%
\pgfsetstrokeopacity{0.000000}%
\pgfsetdash{}{0pt}%
\pgfpathmoveto{\pgfqpoint{0.445556in}{1.096944in}}%
\pgfpathlineto{\pgfqpoint{2.925556in}{1.096944in}}%
\pgfpathlineto{\pgfqpoint{2.925556in}{2.944944in}}%
\pgfpathlineto{\pgfqpoint{0.445556in}{2.944944in}}%
\pgfpathlineto{\pgfqpoint{0.445556in}{1.096944in}}%
\pgfpathclose%
\pgfusepath{fill}%
\end{pgfscope}%
\begin{pgfscope}%
\pgfpathrectangle{\pgfqpoint{0.445556in}{1.096944in}}{\pgfqpoint{2.480000in}{1.848000in}}%
\pgfusepath{clip}%
\pgfsetbuttcap%
\pgfsetmiterjoin%
\pgfsetlinewidth{1.003750pt}%
\definecolor{currentstroke}{rgb}{0.000000,0.000000,0.000000}%
\pgfsetstrokecolor{currentstroke}%
\pgfsetdash{}{0pt}%
\pgfpathmoveto{\pgfqpoint{0.435556in}{1.096944in}}%
\pgfpathlineto{\pgfqpoint{0.508182in}{1.096944in}}%
\pgfpathlineto{\pgfqpoint{0.508182in}{2.688434in}}%
\pgfpathlineto{\pgfqpoint{0.435556in}{2.688434in}}%
\pgfusepath{stroke}%
\end{pgfscope}%
\begin{pgfscope}%
\pgfpathrectangle{\pgfqpoint{0.445556in}{1.096944in}}{\pgfqpoint{2.480000in}{1.848000in}}%
\pgfusepath{clip}%
\pgfsetbuttcap%
\pgfsetmiterjoin%
\pgfsetlinewidth{1.003750pt}%
\definecolor{currentstroke}{rgb}{0.000000,0.000000,0.000000}%
\pgfsetstrokecolor{currentstroke}%
\pgfsetdash{}{0pt}%
\pgfpathmoveto{\pgfqpoint{0.658485in}{1.096944in}}%
\pgfpathlineto{\pgfqpoint{0.758687in}{1.096944in}}%
\pgfpathlineto{\pgfqpoint{0.758687in}{2.856944in}}%
\pgfpathlineto{\pgfqpoint{0.658485in}{2.856944in}}%
\pgfpathlineto{\pgfqpoint{0.658485in}{1.096944in}}%
\pgfpathclose%
\pgfusepath{stroke}%
\end{pgfscope}%
\begin{pgfscope}%
\pgfpathrectangle{\pgfqpoint{0.445556in}{1.096944in}}{\pgfqpoint{2.480000in}{1.848000in}}%
\pgfusepath{clip}%
\pgfsetbuttcap%
\pgfsetmiterjoin%
\pgfsetlinewidth{1.003750pt}%
\definecolor{currentstroke}{rgb}{0.000000,0.000000,0.000000}%
\pgfsetstrokecolor{currentstroke}%
\pgfsetdash{}{0pt}%
\pgfpathmoveto{\pgfqpoint{0.908990in}{1.096944in}}%
\pgfpathlineto{\pgfqpoint{1.009192in}{1.096944in}}%
\pgfpathlineto{\pgfqpoint{1.009192in}{2.370136in}}%
\pgfpathlineto{\pgfqpoint{0.908990in}{2.370136in}}%
\pgfpathlineto{\pgfqpoint{0.908990in}{1.096944in}}%
\pgfpathclose%
\pgfusepath{stroke}%
\end{pgfscope}%
\begin{pgfscope}%
\pgfpathrectangle{\pgfqpoint{0.445556in}{1.096944in}}{\pgfqpoint{2.480000in}{1.848000in}}%
\pgfusepath{clip}%
\pgfsetbuttcap%
\pgfsetmiterjoin%
\pgfsetlinewidth{1.003750pt}%
\definecolor{currentstroke}{rgb}{0.000000,0.000000,0.000000}%
\pgfsetstrokecolor{currentstroke}%
\pgfsetdash{}{0pt}%
\pgfpathmoveto{\pgfqpoint{1.159495in}{1.096944in}}%
\pgfpathlineto{\pgfqpoint{1.259697in}{1.096944in}}%
\pgfpathlineto{\pgfqpoint{1.259697in}{2.707157in}}%
\pgfpathlineto{\pgfqpoint{1.159495in}{2.707157in}}%
\pgfpathlineto{\pgfqpoint{1.159495in}{1.096944in}}%
\pgfpathclose%
\pgfusepath{stroke}%
\end{pgfscope}%
\begin{pgfscope}%
\pgfpathrectangle{\pgfqpoint{0.445556in}{1.096944in}}{\pgfqpoint{2.480000in}{1.848000in}}%
\pgfusepath{clip}%
\pgfsetbuttcap%
\pgfsetmiterjoin%
\pgfsetlinewidth{1.003750pt}%
\definecolor{currentstroke}{rgb}{0.000000,0.000000,0.000000}%
\pgfsetstrokecolor{currentstroke}%
\pgfsetdash{}{0pt}%
\pgfpathmoveto{\pgfqpoint{1.410000in}{1.096944in}}%
\pgfpathlineto{\pgfqpoint{1.510202in}{1.096944in}}%
\pgfpathlineto{\pgfqpoint{1.510202in}{2.201625in}}%
\pgfpathlineto{\pgfqpoint{1.410000in}{2.201625in}}%
\pgfpathlineto{\pgfqpoint{1.410000in}{1.096944in}}%
\pgfpathclose%
\pgfusepath{stroke}%
\end{pgfscope}%
\begin{pgfscope}%
\pgfpathrectangle{\pgfqpoint{0.445556in}{1.096944in}}{\pgfqpoint{2.480000in}{1.848000in}}%
\pgfusepath{clip}%
\pgfsetbuttcap%
\pgfsetmiterjoin%
\pgfsetlinewidth{1.003750pt}%
\definecolor{currentstroke}{rgb}{0.000000,0.000000,0.000000}%
\pgfsetstrokecolor{currentstroke}%
\pgfsetdash{}{0pt}%
\pgfpathmoveto{\pgfqpoint{1.660505in}{1.096944in}}%
\pgfpathlineto{\pgfqpoint{1.760707in}{1.096944in}}%
\pgfpathlineto{\pgfqpoint{1.760707in}{2.725880in}}%
\pgfpathlineto{\pgfqpoint{1.660505in}{2.725880in}}%
\pgfpathlineto{\pgfqpoint{1.660505in}{1.096944in}}%
\pgfpathclose%
\pgfusepath{stroke}%
\end{pgfscope}%
\begin{pgfscope}%
\pgfpathrectangle{\pgfqpoint{0.445556in}{1.096944in}}{\pgfqpoint{2.480000in}{1.848000in}}%
\pgfusepath{clip}%
\pgfsetbuttcap%
\pgfsetmiterjoin%
\pgfsetlinewidth{1.003750pt}%
\definecolor{currentstroke}{rgb}{0.000000,0.000000,0.000000}%
\pgfsetstrokecolor{currentstroke}%
\pgfsetdash{}{0pt}%
\pgfpathmoveto{\pgfqpoint{1.911010in}{1.096944in}}%
\pgfpathlineto{\pgfqpoint{2.011212in}{1.096944in}}%
\pgfpathlineto{\pgfqpoint{2.011212in}{2.632263in}}%
\pgfpathlineto{\pgfqpoint{1.911010in}{2.632263in}}%
\pgfpathlineto{\pgfqpoint{1.911010in}{1.096944in}}%
\pgfpathclose%
\pgfusepath{stroke}%
\end{pgfscope}%
\begin{pgfscope}%
\pgfpathrectangle{\pgfqpoint{0.445556in}{1.096944in}}{\pgfqpoint{2.480000in}{1.848000in}}%
\pgfusepath{clip}%
\pgfsetbuttcap%
\pgfsetmiterjoin%
\pgfsetlinewidth{1.003750pt}%
\definecolor{currentstroke}{rgb}{0.000000,0.000000,0.000000}%
\pgfsetstrokecolor{currentstroke}%
\pgfsetdash{}{0pt}%
\pgfpathmoveto{\pgfqpoint{2.161515in}{1.096944in}}%
\pgfpathlineto{\pgfqpoint{2.261717in}{1.096944in}}%
\pgfpathlineto{\pgfqpoint{2.261717in}{2.463753in}}%
\pgfpathlineto{\pgfqpoint{2.161515in}{2.463753in}}%
\pgfpathlineto{\pgfqpoint{2.161515in}{1.096944in}}%
\pgfpathclose%
\pgfusepath{stroke}%
\end{pgfscope}%
\begin{pgfscope}%
\pgfpathrectangle{\pgfqpoint{0.445556in}{1.096944in}}{\pgfqpoint{2.480000in}{1.848000in}}%
\pgfusepath{clip}%
\pgfsetbuttcap%
\pgfsetmiterjoin%
\pgfsetlinewidth{1.003750pt}%
\definecolor{currentstroke}{rgb}{0.000000,0.000000,0.000000}%
\pgfsetstrokecolor{currentstroke}%
\pgfsetdash{}{0pt}%
\pgfpathmoveto{\pgfqpoint{2.412020in}{1.096944in}}%
\pgfpathlineto{\pgfqpoint{2.512222in}{1.096944in}}%
\pgfpathlineto{\pgfqpoint{2.512222in}{2.650987in}}%
\pgfpathlineto{\pgfqpoint{2.412020in}{2.650987in}}%
\pgfpathlineto{\pgfqpoint{2.412020in}{1.096944in}}%
\pgfpathclose%
\pgfusepath{stroke}%
\end{pgfscope}%
\begin{pgfscope}%
\pgfpathrectangle{\pgfqpoint{0.445556in}{1.096944in}}{\pgfqpoint{2.480000in}{1.848000in}}%
\pgfusepath{clip}%
\pgfsetbuttcap%
\pgfsetmiterjoin%
\pgfsetlinewidth{1.003750pt}%
\definecolor{currentstroke}{rgb}{0.000000,0.000000,0.000000}%
\pgfsetstrokecolor{currentstroke}%
\pgfsetdash{}{0pt}%
\pgfpathmoveto{\pgfqpoint{2.662525in}{1.096944in}}%
\pgfpathlineto{\pgfqpoint{2.762727in}{1.096944in}}%
\pgfpathlineto{\pgfqpoint{2.762727in}{2.650987in}}%
\pgfpathlineto{\pgfqpoint{2.662525in}{2.650987in}}%
\pgfpathlineto{\pgfqpoint{2.662525in}{1.096944in}}%
\pgfpathclose%
\pgfusepath{stroke}%
\end{pgfscope}%
\begin{pgfscope}%
\pgfpathrectangle{\pgfqpoint{0.445556in}{1.096944in}}{\pgfqpoint{2.480000in}{1.848000in}}%
\pgfusepath{clip}%
\pgfsetbuttcap%
\pgfsetmiterjoin%
\definecolor{currentfill}{rgb}{0.000000,0.000000,0.000000}%
\pgfsetfillcolor{currentfill}%
\pgfsetlinewidth{0.000000pt}%
\definecolor{currentstroke}{rgb}{0.000000,0.000000,0.000000}%
\pgfsetstrokecolor{currentstroke}%
\pgfsetstrokeopacity{0.000000}%
\pgfsetdash{}{0pt}%
\pgfpathmoveto{\pgfqpoint{0.508182in}{1.096944in}}%
\pgfpathlineto{\pgfqpoint{0.608384in}{1.096944in}}%
\pgfpathlineto{\pgfqpoint{0.608384in}{1.415242in}}%
\pgfpathlineto{\pgfqpoint{0.508182in}{1.415242in}}%
\pgfpathlineto{\pgfqpoint{0.508182in}{1.096944in}}%
\pgfpathclose%
\pgfusepath{fill}%
\end{pgfscope}%
\begin{pgfscope}%
\pgfpathrectangle{\pgfqpoint{0.445556in}{1.096944in}}{\pgfqpoint{2.480000in}{1.848000in}}%
\pgfusepath{clip}%
\pgfsetbuttcap%
\pgfsetmiterjoin%
\definecolor{currentfill}{rgb}{0.000000,0.000000,0.000000}%
\pgfsetfillcolor{currentfill}%
\pgfsetlinewidth{0.000000pt}%
\definecolor{currentstroke}{rgb}{0.000000,0.000000,0.000000}%
\pgfsetstrokecolor{currentstroke}%
\pgfsetstrokeopacity{0.000000}%
\pgfsetdash{}{0pt}%
\pgfpathmoveto{\pgfqpoint{0.758687in}{1.096944in}}%
\pgfpathlineto{\pgfqpoint{0.858889in}{1.096944in}}%
\pgfpathlineto{\pgfqpoint{0.858889in}{1.396519in}}%
\pgfpathlineto{\pgfqpoint{0.758687in}{1.396519in}}%
\pgfpathlineto{\pgfqpoint{0.758687in}{1.096944in}}%
\pgfpathclose%
\pgfusepath{fill}%
\end{pgfscope}%
\begin{pgfscope}%
\pgfpathrectangle{\pgfqpoint{0.445556in}{1.096944in}}{\pgfqpoint{2.480000in}{1.848000in}}%
\pgfusepath{clip}%
\pgfsetbuttcap%
\pgfsetmiterjoin%
\definecolor{currentfill}{rgb}{0.000000,0.000000,0.000000}%
\pgfsetfillcolor{currentfill}%
\pgfsetlinewidth{0.000000pt}%
\definecolor{currentstroke}{rgb}{0.000000,0.000000,0.000000}%
\pgfsetstrokecolor{currentstroke}%
\pgfsetstrokeopacity{0.000000}%
\pgfsetdash{}{0pt}%
\pgfpathmoveto{\pgfqpoint{1.009192in}{1.096944in}}%
\pgfpathlineto{\pgfqpoint{1.109394in}{1.096944in}}%
\pgfpathlineto{\pgfqpoint{1.109394in}{1.396519in}}%
\pgfpathlineto{\pgfqpoint{1.009192in}{1.396519in}}%
\pgfpathlineto{\pgfqpoint{1.009192in}{1.096944in}}%
\pgfpathclose%
\pgfusepath{fill}%
\end{pgfscope}%
\begin{pgfscope}%
\pgfpathrectangle{\pgfqpoint{0.445556in}{1.096944in}}{\pgfqpoint{2.480000in}{1.848000in}}%
\pgfusepath{clip}%
\pgfsetbuttcap%
\pgfsetmiterjoin%
\definecolor{currentfill}{rgb}{0.000000,0.000000,0.000000}%
\pgfsetfillcolor{currentfill}%
\pgfsetlinewidth{0.000000pt}%
\definecolor{currentstroke}{rgb}{0.000000,0.000000,0.000000}%
\pgfsetstrokecolor{currentstroke}%
\pgfsetstrokeopacity{0.000000}%
\pgfsetdash{}{0pt}%
\pgfpathmoveto{\pgfqpoint{1.259697in}{1.096944in}}%
\pgfpathlineto{\pgfqpoint{1.359899in}{1.096944in}}%
\pgfpathlineto{\pgfqpoint{1.359899in}{1.415242in}}%
\pgfpathlineto{\pgfqpoint{1.259697in}{1.415242in}}%
\pgfpathlineto{\pgfqpoint{1.259697in}{1.096944in}}%
\pgfpathclose%
\pgfusepath{fill}%
\end{pgfscope}%
\begin{pgfscope}%
\pgfpathrectangle{\pgfqpoint{0.445556in}{1.096944in}}{\pgfqpoint{2.480000in}{1.848000in}}%
\pgfusepath{clip}%
\pgfsetbuttcap%
\pgfsetmiterjoin%
\definecolor{currentfill}{rgb}{0.000000,0.000000,0.000000}%
\pgfsetfillcolor{currentfill}%
\pgfsetlinewidth{0.000000pt}%
\definecolor{currentstroke}{rgb}{0.000000,0.000000,0.000000}%
\pgfsetstrokecolor{currentstroke}%
\pgfsetstrokeopacity{0.000000}%
\pgfsetdash{}{0pt}%
\pgfpathmoveto{\pgfqpoint{1.510202in}{1.096944in}}%
\pgfpathlineto{\pgfqpoint{1.610404in}{1.096944in}}%
\pgfpathlineto{\pgfqpoint{1.610404in}{1.527582in}}%
\pgfpathlineto{\pgfqpoint{1.510202in}{1.527582in}}%
\pgfpathlineto{\pgfqpoint{1.510202in}{1.096944in}}%
\pgfpathclose%
\pgfusepath{fill}%
\end{pgfscope}%
\begin{pgfscope}%
\pgfpathrectangle{\pgfqpoint{0.445556in}{1.096944in}}{\pgfqpoint{2.480000in}{1.848000in}}%
\pgfusepath{clip}%
\pgfsetbuttcap%
\pgfsetmiterjoin%
\definecolor{currentfill}{rgb}{0.000000,0.000000,0.000000}%
\pgfsetfillcolor{currentfill}%
\pgfsetlinewidth{0.000000pt}%
\definecolor{currentstroke}{rgb}{0.000000,0.000000,0.000000}%
\pgfsetstrokecolor{currentstroke}%
\pgfsetstrokeopacity{0.000000}%
\pgfsetdash{}{0pt}%
\pgfpathmoveto{\pgfqpoint{1.760707in}{1.096944in}}%
\pgfpathlineto{\pgfqpoint{1.860909in}{1.096944in}}%
\pgfpathlineto{\pgfqpoint{1.860909in}{1.527582in}}%
\pgfpathlineto{\pgfqpoint{1.760707in}{1.527582in}}%
\pgfpathlineto{\pgfqpoint{1.760707in}{1.096944in}}%
\pgfpathclose%
\pgfusepath{fill}%
\end{pgfscope}%
\begin{pgfscope}%
\pgfpathrectangle{\pgfqpoint{0.445556in}{1.096944in}}{\pgfqpoint{2.480000in}{1.848000in}}%
\pgfusepath{clip}%
\pgfsetbuttcap%
\pgfsetmiterjoin%
\definecolor{currentfill}{rgb}{0.000000,0.000000,0.000000}%
\pgfsetfillcolor{currentfill}%
\pgfsetlinewidth{0.000000pt}%
\definecolor{currentstroke}{rgb}{0.000000,0.000000,0.000000}%
\pgfsetstrokecolor{currentstroke}%
\pgfsetstrokeopacity{0.000000}%
\pgfsetdash{}{0pt}%
\pgfpathmoveto{\pgfqpoint{2.011212in}{1.096944in}}%
\pgfpathlineto{\pgfqpoint{2.111414in}{1.096944in}}%
\pgfpathlineto{\pgfqpoint{2.111414in}{1.396519in}}%
\pgfpathlineto{\pgfqpoint{2.011212in}{1.396519in}}%
\pgfpathlineto{\pgfqpoint{2.011212in}{1.096944in}}%
\pgfpathclose%
\pgfusepath{fill}%
\end{pgfscope}%
\begin{pgfscope}%
\pgfpathrectangle{\pgfqpoint{0.445556in}{1.096944in}}{\pgfqpoint{2.480000in}{1.848000in}}%
\pgfusepath{clip}%
\pgfsetbuttcap%
\pgfsetmiterjoin%
\definecolor{currentfill}{rgb}{0.000000,0.000000,0.000000}%
\pgfsetfillcolor{currentfill}%
\pgfsetlinewidth{0.000000pt}%
\definecolor{currentstroke}{rgb}{0.000000,0.000000,0.000000}%
\pgfsetstrokecolor{currentstroke}%
\pgfsetstrokeopacity{0.000000}%
\pgfsetdash{}{0pt}%
\pgfpathmoveto{\pgfqpoint{2.261717in}{1.096944in}}%
\pgfpathlineto{\pgfqpoint{2.361919in}{1.096944in}}%
\pgfpathlineto{\pgfqpoint{2.361919in}{1.583753in}}%
\pgfpathlineto{\pgfqpoint{2.261717in}{1.583753in}}%
\pgfpathlineto{\pgfqpoint{2.261717in}{1.096944in}}%
\pgfpathclose%
\pgfusepath{fill}%
\end{pgfscope}%
\begin{pgfscope}%
\pgfpathrectangle{\pgfqpoint{0.445556in}{1.096944in}}{\pgfqpoint{2.480000in}{1.848000in}}%
\pgfusepath{clip}%
\pgfsetbuttcap%
\pgfsetmiterjoin%
\definecolor{currentfill}{rgb}{0.000000,0.000000,0.000000}%
\pgfsetfillcolor{currentfill}%
\pgfsetlinewidth{0.000000pt}%
\definecolor{currentstroke}{rgb}{0.000000,0.000000,0.000000}%
\pgfsetstrokecolor{currentstroke}%
\pgfsetstrokeopacity{0.000000}%
\pgfsetdash{}{0pt}%
\pgfpathmoveto{\pgfqpoint{2.512222in}{1.096944in}}%
\pgfpathlineto{\pgfqpoint{2.612424in}{1.096944in}}%
\pgfpathlineto{\pgfqpoint{2.612424in}{1.527582in}}%
\pgfpathlineto{\pgfqpoint{2.512222in}{1.527582in}}%
\pgfpathlineto{\pgfqpoint{2.512222in}{1.096944in}}%
\pgfpathclose%
\pgfusepath{fill}%
\end{pgfscope}%
\begin{pgfscope}%
\pgfpathrectangle{\pgfqpoint{0.445556in}{1.096944in}}{\pgfqpoint{2.480000in}{1.848000in}}%
\pgfusepath{clip}%
\pgfsetbuttcap%
\pgfsetmiterjoin%
\definecolor{currentfill}{rgb}{0.000000,0.000000,0.000000}%
\pgfsetfillcolor{currentfill}%
\pgfsetlinewidth{0.000000pt}%
\definecolor{currentstroke}{rgb}{0.000000,0.000000,0.000000}%
\pgfsetstrokecolor{currentstroke}%
\pgfsetstrokeopacity{0.000000}%
\pgfsetdash{}{0pt}%
\pgfpathmoveto{\pgfqpoint{2.762727in}{1.096944in}}%
\pgfpathlineto{\pgfqpoint{2.862929in}{1.096944in}}%
\pgfpathlineto{\pgfqpoint{2.862929in}{1.527582in}}%
\pgfpathlineto{\pgfqpoint{2.762727in}{1.527582in}}%
\pgfpathlineto{\pgfqpoint{2.762727in}{1.096944in}}%
\pgfpathclose%
\pgfusepath{fill}%
\end{pgfscope}%
\begin{pgfscope}%
\pgfsetbuttcap%
\pgfsetroundjoin%
\definecolor{currentfill}{rgb}{0.000000,0.000000,0.000000}%
\pgfsetfillcolor{currentfill}%
\pgfsetlinewidth{0.803000pt}%
\definecolor{currentstroke}{rgb}{0.000000,0.000000,0.000000}%
\pgfsetstrokecolor{currentstroke}%
\pgfsetdash{}{0pt}%
\pgfsys@defobject{currentmarker}{\pgfqpoint{0.000000in}{-0.048611in}}{\pgfqpoint{0.000000in}{0.000000in}}{%
\pgfpathmoveto{\pgfqpoint{0.000000in}{0.000000in}}%
\pgfpathlineto{\pgfqpoint{0.000000in}{-0.048611in}}%
\pgfusepath{stroke,fill}%
}%
\begin{pgfscope}%
\pgfsys@transformshift{0.508182in}{1.096944in}%
\pgfsys@useobject{currentmarker}{}%
\end{pgfscope}%
\end{pgfscope}%
\begin{pgfscope}%
\definecolor{textcolor}{rgb}{0.000000,0.000000,0.000000}%
\pgfsetstrokecolor{textcolor}%
\pgfsetfillcolor{textcolor}%
\pgftext[x=0.542904in, y=0.282083in, left, base,rotate=90.000000]{\color{textcolor}\rmfamily\fontsize{10.000000}{12.000000}\selectfont (-0.001, 0.1]}%
\end{pgfscope}%
\begin{pgfscope}%
\pgfsetbuttcap%
\pgfsetroundjoin%
\definecolor{currentfill}{rgb}{0.000000,0.000000,0.000000}%
\pgfsetfillcolor{currentfill}%
\pgfsetlinewidth{0.803000pt}%
\definecolor{currentstroke}{rgb}{0.000000,0.000000,0.000000}%
\pgfsetstrokecolor{currentstroke}%
\pgfsetdash{}{0pt}%
\pgfsys@defobject{currentmarker}{\pgfqpoint{0.000000in}{-0.048611in}}{\pgfqpoint{0.000000in}{0.000000in}}{%
\pgfpathmoveto{\pgfqpoint{0.000000in}{0.000000in}}%
\pgfpathlineto{\pgfqpoint{0.000000in}{-0.048611in}}%
\pgfusepath{stroke,fill}%
}%
\begin{pgfscope}%
\pgfsys@transformshift{0.758687in}{1.096944in}%
\pgfsys@useobject{currentmarker}{}%
\end{pgfscope}%
\end{pgfscope}%
\begin{pgfscope}%
\definecolor{textcolor}{rgb}{0.000000,0.000000,0.000000}%
\pgfsetstrokecolor{textcolor}%
\pgfsetfillcolor{textcolor}%
\pgftext[x=0.793409in, y=0.467222in, left, base,rotate=90.000000]{\color{textcolor}\rmfamily\fontsize{10.000000}{12.000000}\selectfont (0.1, 0.2]}%
\end{pgfscope}%
\begin{pgfscope}%
\pgfsetbuttcap%
\pgfsetroundjoin%
\definecolor{currentfill}{rgb}{0.000000,0.000000,0.000000}%
\pgfsetfillcolor{currentfill}%
\pgfsetlinewidth{0.803000pt}%
\definecolor{currentstroke}{rgb}{0.000000,0.000000,0.000000}%
\pgfsetstrokecolor{currentstroke}%
\pgfsetdash{}{0pt}%
\pgfsys@defobject{currentmarker}{\pgfqpoint{0.000000in}{-0.048611in}}{\pgfqpoint{0.000000in}{0.000000in}}{%
\pgfpathmoveto{\pgfqpoint{0.000000in}{0.000000in}}%
\pgfpathlineto{\pgfqpoint{0.000000in}{-0.048611in}}%
\pgfusepath{stroke,fill}%
}%
\begin{pgfscope}%
\pgfsys@transformshift{1.009192in}{1.096944in}%
\pgfsys@useobject{currentmarker}{}%
\end{pgfscope}%
\end{pgfscope}%
\begin{pgfscope}%
\definecolor{textcolor}{rgb}{0.000000,0.000000,0.000000}%
\pgfsetstrokecolor{textcolor}%
\pgfsetfillcolor{textcolor}%
\pgftext[x=1.043914in, y=0.467222in, left, base,rotate=90.000000]{\color{textcolor}\rmfamily\fontsize{10.000000}{12.000000}\selectfont (0.2, 0.3]}%
\end{pgfscope}%
\begin{pgfscope}%
\pgfsetbuttcap%
\pgfsetroundjoin%
\definecolor{currentfill}{rgb}{0.000000,0.000000,0.000000}%
\pgfsetfillcolor{currentfill}%
\pgfsetlinewidth{0.803000pt}%
\definecolor{currentstroke}{rgb}{0.000000,0.000000,0.000000}%
\pgfsetstrokecolor{currentstroke}%
\pgfsetdash{}{0pt}%
\pgfsys@defobject{currentmarker}{\pgfqpoint{0.000000in}{-0.048611in}}{\pgfqpoint{0.000000in}{0.000000in}}{%
\pgfpathmoveto{\pgfqpoint{0.000000in}{0.000000in}}%
\pgfpathlineto{\pgfqpoint{0.000000in}{-0.048611in}}%
\pgfusepath{stroke,fill}%
}%
\begin{pgfscope}%
\pgfsys@transformshift{1.259697in}{1.096944in}%
\pgfsys@useobject{currentmarker}{}%
\end{pgfscope}%
\end{pgfscope}%
\begin{pgfscope}%
\definecolor{textcolor}{rgb}{0.000000,0.000000,0.000000}%
\pgfsetstrokecolor{textcolor}%
\pgfsetfillcolor{textcolor}%
\pgftext[x=1.294419in, y=0.467222in, left, base,rotate=90.000000]{\color{textcolor}\rmfamily\fontsize{10.000000}{12.000000}\selectfont (0.3, 0.4]}%
\end{pgfscope}%
\begin{pgfscope}%
\pgfsetbuttcap%
\pgfsetroundjoin%
\definecolor{currentfill}{rgb}{0.000000,0.000000,0.000000}%
\pgfsetfillcolor{currentfill}%
\pgfsetlinewidth{0.803000pt}%
\definecolor{currentstroke}{rgb}{0.000000,0.000000,0.000000}%
\pgfsetstrokecolor{currentstroke}%
\pgfsetdash{}{0pt}%
\pgfsys@defobject{currentmarker}{\pgfqpoint{0.000000in}{-0.048611in}}{\pgfqpoint{0.000000in}{0.000000in}}{%
\pgfpathmoveto{\pgfqpoint{0.000000in}{0.000000in}}%
\pgfpathlineto{\pgfqpoint{0.000000in}{-0.048611in}}%
\pgfusepath{stroke,fill}%
}%
\begin{pgfscope}%
\pgfsys@transformshift{1.510202in}{1.096944in}%
\pgfsys@useobject{currentmarker}{}%
\end{pgfscope}%
\end{pgfscope}%
\begin{pgfscope}%
\definecolor{textcolor}{rgb}{0.000000,0.000000,0.000000}%
\pgfsetstrokecolor{textcolor}%
\pgfsetfillcolor{textcolor}%
\pgftext[x=1.544925in, y=0.467222in, left, base,rotate=90.000000]{\color{textcolor}\rmfamily\fontsize{10.000000}{12.000000}\selectfont (0.4, 0.5]}%
\end{pgfscope}%
\begin{pgfscope}%
\pgfsetbuttcap%
\pgfsetroundjoin%
\definecolor{currentfill}{rgb}{0.000000,0.000000,0.000000}%
\pgfsetfillcolor{currentfill}%
\pgfsetlinewidth{0.803000pt}%
\definecolor{currentstroke}{rgb}{0.000000,0.000000,0.000000}%
\pgfsetstrokecolor{currentstroke}%
\pgfsetdash{}{0pt}%
\pgfsys@defobject{currentmarker}{\pgfqpoint{0.000000in}{-0.048611in}}{\pgfqpoint{0.000000in}{0.000000in}}{%
\pgfpathmoveto{\pgfqpoint{0.000000in}{0.000000in}}%
\pgfpathlineto{\pgfqpoint{0.000000in}{-0.048611in}}%
\pgfusepath{stroke,fill}%
}%
\begin{pgfscope}%
\pgfsys@transformshift{1.760707in}{1.096944in}%
\pgfsys@useobject{currentmarker}{}%
\end{pgfscope}%
\end{pgfscope}%
\begin{pgfscope}%
\definecolor{textcolor}{rgb}{0.000000,0.000000,0.000000}%
\pgfsetstrokecolor{textcolor}%
\pgfsetfillcolor{textcolor}%
\pgftext[x=1.795430in, y=0.467222in, left, base,rotate=90.000000]{\color{textcolor}\rmfamily\fontsize{10.000000}{12.000000}\selectfont (0.5, 0.6]}%
\end{pgfscope}%
\begin{pgfscope}%
\pgfsetbuttcap%
\pgfsetroundjoin%
\definecolor{currentfill}{rgb}{0.000000,0.000000,0.000000}%
\pgfsetfillcolor{currentfill}%
\pgfsetlinewidth{0.803000pt}%
\definecolor{currentstroke}{rgb}{0.000000,0.000000,0.000000}%
\pgfsetstrokecolor{currentstroke}%
\pgfsetdash{}{0pt}%
\pgfsys@defobject{currentmarker}{\pgfqpoint{0.000000in}{-0.048611in}}{\pgfqpoint{0.000000in}{0.000000in}}{%
\pgfpathmoveto{\pgfqpoint{0.000000in}{0.000000in}}%
\pgfpathlineto{\pgfqpoint{0.000000in}{-0.048611in}}%
\pgfusepath{stroke,fill}%
}%
\begin{pgfscope}%
\pgfsys@transformshift{2.011212in}{1.096944in}%
\pgfsys@useobject{currentmarker}{}%
\end{pgfscope}%
\end{pgfscope}%
\begin{pgfscope}%
\definecolor{textcolor}{rgb}{0.000000,0.000000,0.000000}%
\pgfsetstrokecolor{textcolor}%
\pgfsetfillcolor{textcolor}%
\pgftext[x=2.045935in, y=0.467222in, left, base,rotate=90.000000]{\color{textcolor}\rmfamily\fontsize{10.000000}{12.000000}\selectfont (0.6, 0.7]}%
\end{pgfscope}%
\begin{pgfscope}%
\pgfsetbuttcap%
\pgfsetroundjoin%
\definecolor{currentfill}{rgb}{0.000000,0.000000,0.000000}%
\pgfsetfillcolor{currentfill}%
\pgfsetlinewidth{0.803000pt}%
\definecolor{currentstroke}{rgb}{0.000000,0.000000,0.000000}%
\pgfsetstrokecolor{currentstroke}%
\pgfsetdash{}{0pt}%
\pgfsys@defobject{currentmarker}{\pgfqpoint{0.000000in}{-0.048611in}}{\pgfqpoint{0.000000in}{0.000000in}}{%
\pgfpathmoveto{\pgfqpoint{0.000000in}{0.000000in}}%
\pgfpathlineto{\pgfqpoint{0.000000in}{-0.048611in}}%
\pgfusepath{stroke,fill}%
}%
\begin{pgfscope}%
\pgfsys@transformshift{2.261717in}{1.096944in}%
\pgfsys@useobject{currentmarker}{}%
\end{pgfscope}%
\end{pgfscope}%
\begin{pgfscope}%
\definecolor{textcolor}{rgb}{0.000000,0.000000,0.000000}%
\pgfsetstrokecolor{textcolor}%
\pgfsetfillcolor{textcolor}%
\pgftext[x=2.296440in, y=0.467222in, left, base,rotate=90.000000]{\color{textcolor}\rmfamily\fontsize{10.000000}{12.000000}\selectfont (0.7, 0.8]}%
\end{pgfscope}%
\begin{pgfscope}%
\pgfsetbuttcap%
\pgfsetroundjoin%
\definecolor{currentfill}{rgb}{0.000000,0.000000,0.000000}%
\pgfsetfillcolor{currentfill}%
\pgfsetlinewidth{0.803000pt}%
\definecolor{currentstroke}{rgb}{0.000000,0.000000,0.000000}%
\pgfsetstrokecolor{currentstroke}%
\pgfsetdash{}{0pt}%
\pgfsys@defobject{currentmarker}{\pgfqpoint{0.000000in}{-0.048611in}}{\pgfqpoint{0.000000in}{0.000000in}}{%
\pgfpathmoveto{\pgfqpoint{0.000000in}{0.000000in}}%
\pgfpathlineto{\pgfqpoint{0.000000in}{-0.048611in}}%
\pgfusepath{stroke,fill}%
}%
\begin{pgfscope}%
\pgfsys@transformshift{2.512222in}{1.096944in}%
\pgfsys@useobject{currentmarker}{}%
\end{pgfscope}%
\end{pgfscope}%
\begin{pgfscope}%
\definecolor{textcolor}{rgb}{0.000000,0.000000,0.000000}%
\pgfsetstrokecolor{textcolor}%
\pgfsetfillcolor{textcolor}%
\pgftext[x=2.546945in, y=0.467222in, left, base,rotate=90.000000]{\color{textcolor}\rmfamily\fontsize{10.000000}{12.000000}\selectfont (0.8, 0.9]}%
\end{pgfscope}%
\begin{pgfscope}%
\pgfsetbuttcap%
\pgfsetroundjoin%
\definecolor{currentfill}{rgb}{0.000000,0.000000,0.000000}%
\pgfsetfillcolor{currentfill}%
\pgfsetlinewidth{0.803000pt}%
\definecolor{currentstroke}{rgb}{0.000000,0.000000,0.000000}%
\pgfsetstrokecolor{currentstroke}%
\pgfsetdash{}{0pt}%
\pgfsys@defobject{currentmarker}{\pgfqpoint{0.000000in}{-0.048611in}}{\pgfqpoint{0.000000in}{0.000000in}}{%
\pgfpathmoveto{\pgfqpoint{0.000000in}{0.000000in}}%
\pgfpathlineto{\pgfqpoint{0.000000in}{-0.048611in}}%
\pgfusepath{stroke,fill}%
}%
\begin{pgfscope}%
\pgfsys@transformshift{2.762727in}{1.096944in}%
\pgfsys@useobject{currentmarker}{}%
\end{pgfscope}%
\end{pgfscope}%
\begin{pgfscope}%
\definecolor{textcolor}{rgb}{0.000000,0.000000,0.000000}%
\pgfsetstrokecolor{textcolor}%
\pgfsetfillcolor{textcolor}%
\pgftext[x=2.797450in, y=0.467222in, left, base,rotate=90.000000]{\color{textcolor}\rmfamily\fontsize{10.000000}{12.000000}\selectfont (0.9, 1.0]}%
\end{pgfscope}%
\begin{pgfscope}%
\definecolor{textcolor}{rgb}{0.000000,0.000000,0.000000}%
\pgfsetstrokecolor{textcolor}%
\pgfsetfillcolor{textcolor}%
\pgftext[x=1.685556in,y=0.226527in,,top]{\color{textcolor}\rmfamily\fontsize{10.000000}{12.000000}\selectfont Range of Prediction}%
\end{pgfscope}%
\begin{pgfscope}%
\pgfsetbuttcap%
\pgfsetroundjoin%
\definecolor{currentfill}{rgb}{0.000000,0.000000,0.000000}%
\pgfsetfillcolor{currentfill}%
\pgfsetlinewidth{0.803000pt}%
\definecolor{currentstroke}{rgb}{0.000000,0.000000,0.000000}%
\pgfsetstrokecolor{currentstroke}%
\pgfsetdash{}{0pt}%
\pgfsys@defobject{currentmarker}{\pgfqpoint{-0.048611in}{0.000000in}}{\pgfqpoint{-0.000000in}{0.000000in}}{%
\pgfpathmoveto{\pgfqpoint{-0.000000in}{0.000000in}}%
\pgfpathlineto{\pgfqpoint{-0.048611in}{0.000000in}}%
\pgfusepath{stroke,fill}%
}%
\begin{pgfscope}%
\pgfsys@transformshift{0.445556in}{1.096944in}%
\pgfsys@useobject{currentmarker}{}%
\end{pgfscope}%
\end{pgfscope}%
\begin{pgfscope}%
\definecolor{textcolor}{rgb}{0.000000,0.000000,0.000000}%
\pgfsetstrokecolor{textcolor}%
\pgfsetfillcolor{textcolor}%
\pgftext[x=0.278889in, y=1.048750in, left, base]{\color{textcolor}\rmfamily\fontsize{10.000000}{12.000000}\selectfont \(\displaystyle {0}\)}%
\end{pgfscope}%
\begin{pgfscope}%
\pgfsetbuttcap%
\pgfsetroundjoin%
\definecolor{currentfill}{rgb}{0.000000,0.000000,0.000000}%
\pgfsetfillcolor{currentfill}%
\pgfsetlinewidth{0.803000pt}%
\definecolor{currentstroke}{rgb}{0.000000,0.000000,0.000000}%
\pgfsetstrokecolor{currentstroke}%
\pgfsetdash{}{0pt}%
\pgfsys@defobject{currentmarker}{\pgfqpoint{-0.048611in}{0.000000in}}{\pgfqpoint{-0.000000in}{0.000000in}}{%
\pgfpathmoveto{\pgfqpoint{-0.000000in}{0.000000in}}%
\pgfpathlineto{\pgfqpoint{-0.048611in}{0.000000in}}%
\pgfusepath{stroke,fill}%
}%
\begin{pgfscope}%
\pgfsys@transformshift{0.445556in}{1.471412in}%
\pgfsys@useobject{currentmarker}{}%
\end{pgfscope}%
\end{pgfscope}%
\begin{pgfscope}%
\definecolor{textcolor}{rgb}{0.000000,0.000000,0.000000}%
\pgfsetstrokecolor{textcolor}%
\pgfsetfillcolor{textcolor}%
\pgftext[x=0.278889in, y=1.423218in, left, base]{\color{textcolor}\rmfamily\fontsize{10.000000}{12.000000}\selectfont \(\displaystyle {2}\)}%
\end{pgfscope}%
\begin{pgfscope}%
\pgfsetbuttcap%
\pgfsetroundjoin%
\definecolor{currentfill}{rgb}{0.000000,0.000000,0.000000}%
\pgfsetfillcolor{currentfill}%
\pgfsetlinewidth{0.803000pt}%
\definecolor{currentstroke}{rgb}{0.000000,0.000000,0.000000}%
\pgfsetstrokecolor{currentstroke}%
\pgfsetdash{}{0pt}%
\pgfsys@defobject{currentmarker}{\pgfqpoint{-0.048611in}{0.000000in}}{\pgfqpoint{-0.000000in}{0.000000in}}{%
\pgfpathmoveto{\pgfqpoint{-0.000000in}{0.000000in}}%
\pgfpathlineto{\pgfqpoint{-0.048611in}{0.000000in}}%
\pgfusepath{stroke,fill}%
}%
\begin{pgfscope}%
\pgfsys@transformshift{0.445556in}{1.845880in}%
\pgfsys@useobject{currentmarker}{}%
\end{pgfscope}%
\end{pgfscope}%
\begin{pgfscope}%
\definecolor{textcolor}{rgb}{0.000000,0.000000,0.000000}%
\pgfsetstrokecolor{textcolor}%
\pgfsetfillcolor{textcolor}%
\pgftext[x=0.278889in, y=1.797686in, left, base]{\color{textcolor}\rmfamily\fontsize{10.000000}{12.000000}\selectfont \(\displaystyle {4}\)}%
\end{pgfscope}%
\begin{pgfscope}%
\pgfsetbuttcap%
\pgfsetroundjoin%
\definecolor{currentfill}{rgb}{0.000000,0.000000,0.000000}%
\pgfsetfillcolor{currentfill}%
\pgfsetlinewidth{0.803000pt}%
\definecolor{currentstroke}{rgb}{0.000000,0.000000,0.000000}%
\pgfsetstrokecolor{currentstroke}%
\pgfsetdash{}{0pt}%
\pgfsys@defobject{currentmarker}{\pgfqpoint{-0.048611in}{0.000000in}}{\pgfqpoint{-0.000000in}{0.000000in}}{%
\pgfpathmoveto{\pgfqpoint{-0.000000in}{0.000000in}}%
\pgfpathlineto{\pgfqpoint{-0.048611in}{0.000000in}}%
\pgfusepath{stroke,fill}%
}%
\begin{pgfscope}%
\pgfsys@transformshift{0.445556in}{2.220348in}%
\pgfsys@useobject{currentmarker}{}%
\end{pgfscope}%
\end{pgfscope}%
\begin{pgfscope}%
\definecolor{textcolor}{rgb}{0.000000,0.000000,0.000000}%
\pgfsetstrokecolor{textcolor}%
\pgfsetfillcolor{textcolor}%
\pgftext[x=0.278889in, y=2.172154in, left, base]{\color{textcolor}\rmfamily\fontsize{10.000000}{12.000000}\selectfont \(\displaystyle {6}\)}%
\end{pgfscope}%
\begin{pgfscope}%
\pgfsetbuttcap%
\pgfsetroundjoin%
\definecolor{currentfill}{rgb}{0.000000,0.000000,0.000000}%
\pgfsetfillcolor{currentfill}%
\pgfsetlinewidth{0.803000pt}%
\definecolor{currentstroke}{rgb}{0.000000,0.000000,0.000000}%
\pgfsetstrokecolor{currentstroke}%
\pgfsetdash{}{0pt}%
\pgfsys@defobject{currentmarker}{\pgfqpoint{-0.048611in}{0.000000in}}{\pgfqpoint{-0.000000in}{0.000000in}}{%
\pgfpathmoveto{\pgfqpoint{-0.000000in}{0.000000in}}%
\pgfpathlineto{\pgfqpoint{-0.048611in}{0.000000in}}%
\pgfusepath{stroke,fill}%
}%
\begin{pgfscope}%
\pgfsys@transformshift{0.445556in}{2.594817in}%
\pgfsys@useobject{currentmarker}{}%
\end{pgfscope}%
\end{pgfscope}%
\begin{pgfscope}%
\definecolor{textcolor}{rgb}{0.000000,0.000000,0.000000}%
\pgfsetstrokecolor{textcolor}%
\pgfsetfillcolor{textcolor}%
\pgftext[x=0.278889in, y=2.546622in, left, base]{\color{textcolor}\rmfamily\fontsize{10.000000}{12.000000}\selectfont \(\displaystyle {8}\)}%
\end{pgfscope}%
\begin{pgfscope}%
\definecolor{textcolor}{rgb}{0.000000,0.000000,0.000000}%
\pgfsetstrokecolor{textcolor}%
\pgfsetfillcolor{textcolor}%
\pgftext[x=0.223333in,y=2.020944in,,bottom,rotate=90.000000]{\color{textcolor}\rmfamily\fontsize{10.000000}{12.000000}\selectfont Percent of Data Set}%
\end{pgfscope}%
\begin{pgfscope}%
\pgfsetrectcap%
\pgfsetmiterjoin%
\pgfsetlinewidth{0.803000pt}%
\definecolor{currentstroke}{rgb}{0.000000,0.000000,0.000000}%
\pgfsetstrokecolor{currentstroke}%
\pgfsetdash{}{0pt}%
\pgfpathmoveto{\pgfqpoint{0.445556in}{1.096944in}}%
\pgfpathlineto{\pgfqpoint{0.445556in}{2.944944in}}%
\pgfusepath{stroke}%
\end{pgfscope}%
\begin{pgfscope}%
\pgfsetrectcap%
\pgfsetmiterjoin%
\pgfsetlinewidth{0.803000pt}%
\definecolor{currentstroke}{rgb}{0.000000,0.000000,0.000000}%
\pgfsetstrokecolor{currentstroke}%
\pgfsetdash{}{0pt}%
\pgfpathmoveto{\pgfqpoint{2.925556in}{1.096944in}}%
\pgfpathlineto{\pgfqpoint{2.925556in}{2.944944in}}%
\pgfusepath{stroke}%
\end{pgfscope}%
\begin{pgfscope}%
\pgfsetrectcap%
\pgfsetmiterjoin%
\pgfsetlinewidth{0.803000pt}%
\definecolor{currentstroke}{rgb}{0.000000,0.000000,0.000000}%
\pgfsetstrokecolor{currentstroke}%
\pgfsetdash{}{0pt}%
\pgfpathmoveto{\pgfqpoint{0.445556in}{1.096944in}}%
\pgfpathlineto{\pgfqpoint{2.925556in}{1.096944in}}%
\pgfusepath{stroke}%
\end{pgfscope}%
\begin{pgfscope}%
\pgfsetrectcap%
\pgfsetmiterjoin%
\pgfsetlinewidth{0.803000pt}%
\definecolor{currentstroke}{rgb}{0.000000,0.000000,0.000000}%
\pgfsetstrokecolor{currentstroke}%
\pgfsetdash{}{0pt}%
\pgfpathmoveto{\pgfqpoint{0.445556in}{2.944944in}}%
\pgfpathlineto{\pgfqpoint{2.925556in}{2.944944in}}%
\pgfusepath{stroke}%
\end{pgfscope}%
\begin{pgfscope}%
\pgfsetbuttcap%
\pgfsetmiterjoin%
\definecolor{currentfill}{rgb}{1.000000,1.000000,1.000000}%
\pgfsetfillcolor{currentfill}%
\pgfsetfillopacity{0.800000}%
\pgfsetlinewidth{1.003750pt}%
\definecolor{currentstroke}{rgb}{0.800000,0.800000,0.800000}%
\pgfsetstrokecolor{currentstroke}%
\pgfsetstrokeopacity{0.800000}%
\pgfsetdash{}{0pt}%
\pgfpathmoveto{\pgfqpoint{0.542778in}{2.444250in}}%
\pgfpathlineto{\pgfqpoint{1.879722in}{2.444250in}}%
\pgfpathquadraticcurveto{\pgfqpoint{1.907500in}{2.444250in}}{\pgfqpoint{1.907500in}{2.472028in}}%
\pgfpathlineto{\pgfqpoint{1.907500in}{2.847722in}}%
\pgfpathquadraticcurveto{\pgfqpoint{1.907500in}{2.875500in}}{\pgfqpoint{1.879722in}{2.875500in}}%
\pgfpathlineto{\pgfqpoint{0.542778in}{2.875500in}}%
\pgfpathquadraticcurveto{\pgfqpoint{0.515000in}{2.875500in}}{\pgfqpoint{0.515000in}{2.847722in}}%
\pgfpathlineto{\pgfqpoint{0.515000in}{2.472028in}}%
\pgfpathquadraticcurveto{\pgfqpoint{0.515000in}{2.444250in}}{\pgfqpoint{0.542778in}{2.444250in}}%
\pgfpathlineto{\pgfqpoint{0.542778in}{2.444250in}}%
\pgfpathclose%
\pgfusepath{stroke,fill}%
\end{pgfscope}%
\begin{pgfscope}%
\pgfsetbuttcap%
\pgfsetmiterjoin%
\pgfsetlinewidth{1.003750pt}%
\definecolor{currentstroke}{rgb}{0.000000,0.000000,0.000000}%
\pgfsetstrokecolor{currentstroke}%
\pgfsetdash{}{0pt}%
\pgfpathmoveto{\pgfqpoint{0.570556in}{2.722027in}}%
\pgfpathlineto{\pgfqpoint{0.848333in}{2.722027in}}%
\pgfpathlineto{\pgfqpoint{0.848333in}{2.819250in}}%
\pgfpathlineto{\pgfqpoint{0.570556in}{2.819250in}}%
\pgfpathlineto{\pgfqpoint{0.570556in}{2.722027in}}%
\pgfpathclose%
\pgfusepath{stroke}%
\end{pgfscope}%
\begin{pgfscope}%
\definecolor{textcolor}{rgb}{0.000000,0.000000,0.000000}%
\pgfsetstrokecolor{textcolor}%
\pgfsetfillcolor{textcolor}%
\pgftext[x=0.959445in,y=2.722027in,left,base]{\color{textcolor}\rmfamily\fontsize{10.000000}{12.000000}\selectfont Negative Class}%
\end{pgfscope}%
\begin{pgfscope}%
\pgfsetbuttcap%
\pgfsetmiterjoin%
\definecolor{currentfill}{rgb}{0.000000,0.000000,0.000000}%
\pgfsetfillcolor{currentfill}%
\pgfsetlinewidth{0.000000pt}%
\definecolor{currentstroke}{rgb}{0.000000,0.000000,0.000000}%
\pgfsetstrokecolor{currentstroke}%
\pgfsetstrokeopacity{0.000000}%
\pgfsetdash{}{0pt}%
\pgfpathmoveto{\pgfqpoint{0.570556in}{2.526750in}}%
\pgfpathlineto{\pgfqpoint{0.848333in}{2.526750in}}%
\pgfpathlineto{\pgfqpoint{0.848333in}{2.623972in}}%
\pgfpathlineto{\pgfqpoint{0.570556in}{2.623972in}}%
\pgfpathlineto{\pgfqpoint{0.570556in}{2.526750in}}%
\pgfpathclose%
\pgfusepath{fill}%
\end{pgfscope}%
\begin{pgfscope}%
\definecolor{textcolor}{rgb}{0.000000,0.000000,0.000000}%
\pgfsetstrokecolor{textcolor}%
\pgfsetfillcolor{textcolor}%
\pgftext[x=0.959445in,y=2.526750in,left,base]{\color{textcolor}\rmfamily\fontsize{10.000000}{12.000000}\selectfont Positive Class}%
\end{pgfscope}%
\end{pgfpicture}%
\makeatother%
\endgroup%

  &
  \vspace{0pt} %% Creator: Matplotlib, PGF backend
%%
%% To include the figure in your LaTeX document, write
%%   \input{<filename>.pgf}
%%
%% Make sure the required packages are loaded in your preamble
%%   \usepackage{pgf}
%%
%% Also ensure that all the required font packages are loaded; for instance,
%% the lmodern package is sometimes necessary when using math font.
%%   \usepackage{lmodern}
%%
%% Figures using additional raster images can only be included by \input if
%% they are in the same directory as the main LaTeX file. For loading figures
%% from other directories you can use the `import` package
%%   \usepackage{import}
%%
%% and then include the figures with
%%   \import{<path to file>}{<filename>.pgf}
%%
%% Matplotlib used the following preamble
%%   
%%   \usepackage{fontspec}
%%   \makeatletter\@ifpackageloaded{underscore}{}{\usepackage[strings]{underscore}}\makeatother
%%
\begingroup%
\makeatletter%
\begin{pgfpicture}%
\pgfpathrectangle{\pgfpointorigin}{\pgfqpoint{2.517770in}{2.184444in}}%
\pgfusepath{use as bounding box, clip}%
\begin{pgfscope}%
\pgfsetbuttcap%
\pgfsetmiterjoin%
\definecolor{currentfill}{rgb}{1.000000,1.000000,1.000000}%
\pgfsetfillcolor{currentfill}%
\pgfsetlinewidth{0.000000pt}%
\definecolor{currentstroke}{rgb}{1.000000,1.000000,1.000000}%
\pgfsetstrokecolor{currentstroke}%
\pgfsetdash{}{0pt}%
\pgfpathmoveto{\pgfqpoint{0.000000in}{0.000000in}}%
\pgfpathlineto{\pgfqpoint{2.517770in}{0.000000in}}%
\pgfpathlineto{\pgfqpoint{2.517770in}{2.184444in}}%
\pgfpathlineto{\pgfqpoint{0.000000in}{2.184444in}}%
\pgfpathlineto{\pgfqpoint{0.000000in}{0.000000in}}%
\pgfpathclose%
\pgfusepath{fill}%
\end{pgfscope}%
\begin{pgfscope}%
\pgfsetbuttcap%
\pgfsetmiterjoin%
\definecolor{currentfill}{rgb}{1.000000,1.000000,1.000000}%
\pgfsetfillcolor{currentfill}%
\pgfsetlinewidth{0.000000pt}%
\definecolor{currentstroke}{rgb}{0.000000,0.000000,0.000000}%
\pgfsetstrokecolor{currentstroke}%
\pgfsetstrokeopacity{0.000000}%
\pgfsetdash{}{0pt}%
\pgfpathmoveto{\pgfqpoint{0.553581in}{0.499444in}}%
\pgfpathlineto{\pgfqpoint{2.413581in}{0.499444in}}%
\pgfpathlineto{\pgfqpoint{2.413581in}{1.885444in}}%
\pgfpathlineto{\pgfqpoint{0.553581in}{1.885444in}}%
\pgfpathlineto{\pgfqpoint{0.553581in}{0.499444in}}%
\pgfpathclose%
\pgfusepath{fill}%
\end{pgfscope}%
\begin{pgfscope}%
\pgfsetbuttcap%
\pgfsetroundjoin%
\definecolor{currentfill}{rgb}{0.000000,0.000000,0.000000}%
\pgfsetfillcolor{currentfill}%
\pgfsetlinewidth{0.803000pt}%
\definecolor{currentstroke}{rgb}{0.000000,0.000000,0.000000}%
\pgfsetstrokecolor{currentstroke}%
\pgfsetdash{}{0pt}%
\pgfsys@defobject{currentmarker}{\pgfqpoint{0.000000in}{-0.048611in}}{\pgfqpoint{0.000000in}{0.000000in}}{%
\pgfpathmoveto{\pgfqpoint{0.000000in}{0.000000in}}%
\pgfpathlineto{\pgfqpoint{0.000000in}{-0.048611in}}%
\pgfusepath{stroke,fill}%
}%
\begin{pgfscope}%
\pgfsys@transformshift{0.638126in}{0.499444in}%
\pgfsys@useobject{currentmarker}{}%
\end{pgfscope}%
\end{pgfscope}%
\begin{pgfscope}%
\definecolor{textcolor}{rgb}{0.000000,0.000000,0.000000}%
\pgfsetstrokecolor{textcolor}%
\pgfsetfillcolor{textcolor}%
\pgftext[x=0.638126in,y=0.402222in,,top]{\color{textcolor}\rmfamily\fontsize{10.000000}{12.000000}\selectfont \(\displaystyle {0.0}\)}%
\end{pgfscope}%
\begin{pgfscope}%
\pgfsetbuttcap%
\pgfsetroundjoin%
\definecolor{currentfill}{rgb}{0.000000,0.000000,0.000000}%
\pgfsetfillcolor{currentfill}%
\pgfsetlinewidth{0.803000pt}%
\definecolor{currentstroke}{rgb}{0.000000,0.000000,0.000000}%
\pgfsetstrokecolor{currentstroke}%
\pgfsetdash{}{0pt}%
\pgfsys@defobject{currentmarker}{\pgfqpoint{0.000000in}{-0.048611in}}{\pgfqpoint{0.000000in}{0.000000in}}{%
\pgfpathmoveto{\pgfqpoint{0.000000in}{0.000000in}}%
\pgfpathlineto{\pgfqpoint{0.000000in}{-0.048611in}}%
\pgfusepath{stroke,fill}%
}%
\begin{pgfscope}%
\pgfsys@transformshift{1.483581in}{0.499444in}%
\pgfsys@useobject{currentmarker}{}%
\end{pgfscope}%
\end{pgfscope}%
\begin{pgfscope}%
\definecolor{textcolor}{rgb}{0.000000,0.000000,0.000000}%
\pgfsetstrokecolor{textcolor}%
\pgfsetfillcolor{textcolor}%
\pgftext[x=1.483581in,y=0.402222in,,top]{\color{textcolor}\rmfamily\fontsize{10.000000}{12.000000}\selectfont \(\displaystyle {0.5}\)}%
\end{pgfscope}%
\begin{pgfscope}%
\pgfsetbuttcap%
\pgfsetroundjoin%
\definecolor{currentfill}{rgb}{0.000000,0.000000,0.000000}%
\pgfsetfillcolor{currentfill}%
\pgfsetlinewidth{0.803000pt}%
\definecolor{currentstroke}{rgb}{0.000000,0.000000,0.000000}%
\pgfsetstrokecolor{currentstroke}%
\pgfsetdash{}{0pt}%
\pgfsys@defobject{currentmarker}{\pgfqpoint{0.000000in}{-0.048611in}}{\pgfqpoint{0.000000in}{0.000000in}}{%
\pgfpathmoveto{\pgfqpoint{0.000000in}{0.000000in}}%
\pgfpathlineto{\pgfqpoint{0.000000in}{-0.048611in}}%
\pgfusepath{stroke,fill}%
}%
\begin{pgfscope}%
\pgfsys@transformshift{2.329035in}{0.499444in}%
\pgfsys@useobject{currentmarker}{}%
\end{pgfscope}%
\end{pgfscope}%
\begin{pgfscope}%
\definecolor{textcolor}{rgb}{0.000000,0.000000,0.000000}%
\pgfsetstrokecolor{textcolor}%
\pgfsetfillcolor{textcolor}%
\pgftext[x=2.329035in,y=0.402222in,,top]{\color{textcolor}\rmfamily\fontsize{10.000000}{12.000000}\selectfont \(\displaystyle {1.0}\)}%
\end{pgfscope}%
\begin{pgfscope}%
\definecolor{textcolor}{rgb}{0.000000,0.000000,0.000000}%
\pgfsetstrokecolor{textcolor}%
\pgfsetfillcolor{textcolor}%
\pgftext[x=1.483581in,y=0.223333in,,top]{\color{textcolor}\rmfamily\fontsize{10.000000}{12.000000}\selectfont False positive rate}%
\end{pgfscope}%
\begin{pgfscope}%
\pgfsetbuttcap%
\pgfsetroundjoin%
\definecolor{currentfill}{rgb}{0.000000,0.000000,0.000000}%
\pgfsetfillcolor{currentfill}%
\pgfsetlinewidth{0.803000pt}%
\definecolor{currentstroke}{rgb}{0.000000,0.000000,0.000000}%
\pgfsetstrokecolor{currentstroke}%
\pgfsetdash{}{0pt}%
\pgfsys@defobject{currentmarker}{\pgfqpoint{-0.048611in}{0.000000in}}{\pgfqpoint{-0.000000in}{0.000000in}}{%
\pgfpathmoveto{\pgfqpoint{-0.000000in}{0.000000in}}%
\pgfpathlineto{\pgfqpoint{-0.048611in}{0.000000in}}%
\pgfusepath{stroke,fill}%
}%
\begin{pgfscope}%
\pgfsys@transformshift{0.553581in}{0.562444in}%
\pgfsys@useobject{currentmarker}{}%
\end{pgfscope}%
\end{pgfscope}%
\begin{pgfscope}%
\definecolor{textcolor}{rgb}{0.000000,0.000000,0.000000}%
\pgfsetstrokecolor{textcolor}%
\pgfsetfillcolor{textcolor}%
\pgftext[x=0.278889in, y=0.514250in, left, base]{\color{textcolor}\rmfamily\fontsize{10.000000}{12.000000}\selectfont \(\displaystyle {0.0}\)}%
\end{pgfscope}%
\begin{pgfscope}%
\pgfsetbuttcap%
\pgfsetroundjoin%
\definecolor{currentfill}{rgb}{0.000000,0.000000,0.000000}%
\pgfsetfillcolor{currentfill}%
\pgfsetlinewidth{0.803000pt}%
\definecolor{currentstroke}{rgb}{0.000000,0.000000,0.000000}%
\pgfsetstrokecolor{currentstroke}%
\pgfsetdash{}{0pt}%
\pgfsys@defobject{currentmarker}{\pgfqpoint{-0.048611in}{0.000000in}}{\pgfqpoint{-0.000000in}{0.000000in}}{%
\pgfpathmoveto{\pgfqpoint{-0.000000in}{0.000000in}}%
\pgfpathlineto{\pgfqpoint{-0.048611in}{0.000000in}}%
\pgfusepath{stroke,fill}%
}%
\begin{pgfscope}%
\pgfsys@transformshift{0.553581in}{1.192444in}%
\pgfsys@useobject{currentmarker}{}%
\end{pgfscope}%
\end{pgfscope}%
\begin{pgfscope}%
\definecolor{textcolor}{rgb}{0.000000,0.000000,0.000000}%
\pgfsetstrokecolor{textcolor}%
\pgfsetfillcolor{textcolor}%
\pgftext[x=0.278889in, y=1.144250in, left, base]{\color{textcolor}\rmfamily\fontsize{10.000000}{12.000000}\selectfont \(\displaystyle {0.5}\)}%
\end{pgfscope}%
\begin{pgfscope}%
\pgfsetbuttcap%
\pgfsetroundjoin%
\definecolor{currentfill}{rgb}{0.000000,0.000000,0.000000}%
\pgfsetfillcolor{currentfill}%
\pgfsetlinewidth{0.803000pt}%
\definecolor{currentstroke}{rgb}{0.000000,0.000000,0.000000}%
\pgfsetstrokecolor{currentstroke}%
\pgfsetdash{}{0pt}%
\pgfsys@defobject{currentmarker}{\pgfqpoint{-0.048611in}{0.000000in}}{\pgfqpoint{-0.000000in}{0.000000in}}{%
\pgfpathmoveto{\pgfqpoint{-0.000000in}{0.000000in}}%
\pgfpathlineto{\pgfqpoint{-0.048611in}{0.000000in}}%
\pgfusepath{stroke,fill}%
}%
\begin{pgfscope}%
\pgfsys@transformshift{0.553581in}{1.822444in}%
\pgfsys@useobject{currentmarker}{}%
\end{pgfscope}%
\end{pgfscope}%
\begin{pgfscope}%
\definecolor{textcolor}{rgb}{0.000000,0.000000,0.000000}%
\pgfsetstrokecolor{textcolor}%
\pgfsetfillcolor{textcolor}%
\pgftext[x=0.278889in, y=1.774250in, left, base]{\color{textcolor}\rmfamily\fontsize{10.000000}{12.000000}\selectfont \(\displaystyle {1.0}\)}%
\end{pgfscope}%
\begin{pgfscope}%
\definecolor{textcolor}{rgb}{0.000000,0.000000,0.000000}%
\pgfsetstrokecolor{textcolor}%
\pgfsetfillcolor{textcolor}%
\pgftext[x=0.223333in,y=1.192444in,,bottom,rotate=90.000000]{\color{textcolor}\rmfamily\fontsize{10.000000}{12.000000}\selectfont True positive rate}%
\end{pgfscope}%
\begin{pgfscope}%
\pgfpathrectangle{\pgfqpoint{0.553581in}{0.499444in}}{\pgfqpoint{1.860000in}{1.386000in}}%
\pgfusepath{clip}%
\pgfsetbuttcap%
\pgfsetroundjoin%
\pgfsetlinewidth{1.505625pt}%
\definecolor{currentstroke}{rgb}{0.000000,0.000000,0.000000}%
\pgfsetstrokecolor{currentstroke}%
\pgfsetdash{{5.550000pt}{2.400000pt}}{0.000000pt}%
\pgfpathmoveto{\pgfqpoint{0.638126in}{0.562444in}}%
\pgfpathlineto{\pgfqpoint{2.329035in}{1.822444in}}%
\pgfusepath{stroke}%
\end{pgfscope}%
\begin{pgfscope}%
\pgfpathrectangle{\pgfqpoint{0.553581in}{0.499444in}}{\pgfqpoint{1.860000in}{1.386000in}}%
\pgfusepath{clip}%
\pgfsetrectcap%
\pgfsetroundjoin%
\pgfsetlinewidth{1.505625pt}%
\definecolor{currentstroke}{rgb}{0.121569,0.466667,0.705882}%
\pgfsetstrokecolor{currentstroke}%
\pgfsetdash{}{0pt}%
\pgfpathmoveto{\pgfqpoint{0.638126in}{0.562444in}}%
\pgfpathlineto{\pgfqpoint{0.671944in}{0.562444in}}%
\pgfpathlineto{\pgfqpoint{0.671944in}{0.575044in}}%
\pgfpathlineto{\pgfqpoint{0.705763in}{0.575044in}}%
\pgfpathlineto{\pgfqpoint{0.705763in}{0.587644in}}%
\pgfpathlineto{\pgfqpoint{0.719853in}{0.587644in}}%
\pgfpathlineto{\pgfqpoint{0.719853in}{0.600244in}}%
\pgfpathlineto{\pgfqpoint{0.767763in}{0.600244in}}%
\pgfpathlineto{\pgfqpoint{0.767763in}{0.612844in}}%
\pgfpathlineto{\pgfqpoint{0.793126in}{0.612844in}}%
\pgfpathlineto{\pgfqpoint{0.793126in}{0.638044in}}%
\pgfpathlineto{\pgfqpoint{0.798763in}{0.638044in}}%
\pgfpathlineto{\pgfqpoint{0.798763in}{0.663244in}}%
\pgfpathlineto{\pgfqpoint{0.832581in}{0.663244in}}%
\pgfpathlineto{\pgfqpoint{0.832581in}{0.675844in}}%
\pgfpathlineto{\pgfqpoint{0.849490in}{0.675844in}}%
\pgfpathlineto{\pgfqpoint{0.849490in}{0.688444in}}%
\pgfpathlineto{\pgfqpoint{0.863581in}{0.688444in}}%
\pgfpathlineto{\pgfqpoint{0.863581in}{0.701044in}}%
\pgfpathlineto{\pgfqpoint{0.872035in}{0.701044in}}%
\pgfpathlineto{\pgfqpoint{0.872035in}{0.738844in}}%
\pgfpathlineto{\pgfqpoint{0.883308in}{0.738844in}}%
\pgfpathlineto{\pgfqpoint{0.883308in}{0.751444in}}%
\pgfpathlineto{\pgfqpoint{0.931217in}{0.751444in}}%
\pgfpathlineto{\pgfqpoint{0.931217in}{0.776644in}}%
\pgfpathlineto{\pgfqpoint{0.936853in}{0.776644in}}%
\pgfpathlineto{\pgfqpoint{0.936853in}{0.789244in}}%
\pgfpathlineto{\pgfqpoint{0.948126in}{0.789244in}}%
\pgfpathlineto{\pgfqpoint{0.948126in}{0.814444in}}%
\pgfpathlineto{\pgfqpoint{0.950944in}{0.814444in}}%
\pgfpathlineto{\pgfqpoint{0.950944in}{0.827044in}}%
\pgfpathlineto{\pgfqpoint{0.953763in}{0.827044in}}%
\pgfpathlineto{\pgfqpoint{0.953763in}{0.839644in}}%
\pgfpathlineto{\pgfqpoint{0.967853in}{0.839644in}}%
\pgfpathlineto{\pgfqpoint{0.967853in}{0.852244in}}%
\pgfpathlineto{\pgfqpoint{0.976308in}{0.852244in}}%
\pgfpathlineto{\pgfqpoint{0.976308in}{0.877444in}}%
\pgfpathlineto{\pgfqpoint{1.021399in}{0.877444in}}%
\pgfpathlineto{\pgfqpoint{1.021399in}{0.890044in}}%
\pgfpathlineto{\pgfqpoint{1.024217in}{0.890044in}}%
\pgfpathlineto{\pgfqpoint{1.024217in}{0.902644in}}%
\pgfpathlineto{\pgfqpoint{1.027035in}{0.902644in}}%
\pgfpathlineto{\pgfqpoint{1.027035in}{0.915244in}}%
\pgfpathlineto{\pgfqpoint{1.060853in}{0.915244in}}%
\pgfpathlineto{\pgfqpoint{1.060853in}{0.927844in}}%
\pgfpathlineto{\pgfqpoint{1.072126in}{0.927844in}}%
\pgfpathlineto{\pgfqpoint{1.072126in}{0.953044in}}%
\pgfpathlineto{\pgfqpoint{1.080581in}{0.953044in}}%
\pgfpathlineto{\pgfqpoint{1.080581in}{0.965644in}}%
\pgfpathlineto{\pgfqpoint{1.091853in}{0.965644in}}%
\pgfpathlineto{\pgfqpoint{1.091853in}{0.978244in}}%
\pgfpathlineto{\pgfqpoint{1.097490in}{0.978244in}}%
\pgfpathlineto{\pgfqpoint{1.097490in}{0.990844in}}%
\pgfpathlineto{\pgfqpoint{1.108763in}{0.990844in}}%
\pgfpathlineto{\pgfqpoint{1.108763in}{1.003444in}}%
\pgfpathlineto{\pgfqpoint{1.125672in}{1.003444in}}%
\pgfpathlineto{\pgfqpoint{1.125672in}{1.016044in}}%
\pgfpathlineto{\pgfqpoint{1.128490in}{1.016044in}}%
\pgfpathlineto{\pgfqpoint{1.128490in}{1.066444in}}%
\pgfpathlineto{\pgfqpoint{1.136944in}{1.066444in}}%
\pgfpathlineto{\pgfqpoint{1.136944in}{1.079044in}}%
\pgfpathlineto{\pgfqpoint{1.193308in}{1.079044in}}%
\pgfpathlineto{\pgfqpoint{1.193308in}{1.091644in}}%
\pgfpathlineto{\pgfqpoint{1.196126in}{1.091644in}}%
\pgfpathlineto{\pgfqpoint{1.196126in}{1.104244in}}%
\pgfpathlineto{\pgfqpoint{1.198944in}{1.104244in}}%
\pgfpathlineto{\pgfqpoint{1.198944in}{1.116844in}}%
\pgfpathlineto{\pgfqpoint{1.244035in}{1.116844in}}%
\pgfpathlineto{\pgfqpoint{1.244035in}{1.129444in}}%
\pgfpathlineto{\pgfqpoint{1.260944in}{1.129444in}}%
\pgfpathlineto{\pgfqpoint{1.260944in}{1.142044in}}%
\pgfpathlineto{\pgfqpoint{1.263763in}{1.142044in}}%
\pgfpathlineto{\pgfqpoint{1.263763in}{1.154644in}}%
\pgfpathlineto{\pgfqpoint{1.286308in}{1.154644in}}%
\pgfpathlineto{\pgfqpoint{1.286308in}{1.179844in}}%
\pgfpathlineto{\pgfqpoint{1.303217in}{1.179844in}}%
\pgfpathlineto{\pgfqpoint{1.303217in}{1.192444in}}%
\pgfpathlineto{\pgfqpoint{1.342672in}{1.192444in}}%
\pgfpathlineto{\pgfqpoint{1.342672in}{1.205044in}}%
\pgfpathlineto{\pgfqpoint{1.365217in}{1.205044in}}%
\pgfpathlineto{\pgfqpoint{1.365217in}{1.217644in}}%
\pgfpathlineto{\pgfqpoint{1.384944in}{1.217644in}}%
\pgfpathlineto{\pgfqpoint{1.384944in}{1.230244in}}%
\pgfpathlineto{\pgfqpoint{1.390581in}{1.230244in}}%
\pgfpathlineto{\pgfqpoint{1.390581in}{1.242844in}}%
\pgfpathlineto{\pgfqpoint{1.415944in}{1.242844in}}%
\pgfpathlineto{\pgfqpoint{1.415944in}{1.255444in}}%
\pgfpathlineto{\pgfqpoint{1.444126in}{1.255444in}}%
\pgfpathlineto{\pgfqpoint{1.444126in}{1.280644in}}%
\pgfpathlineto{\pgfqpoint{1.477944in}{1.280644in}}%
\pgfpathlineto{\pgfqpoint{1.477944in}{1.293244in}}%
\pgfpathlineto{\pgfqpoint{1.486399in}{1.293244in}}%
\pgfpathlineto{\pgfqpoint{1.486399in}{1.305844in}}%
\pgfpathlineto{\pgfqpoint{1.514581in}{1.305844in}}%
\pgfpathlineto{\pgfqpoint{1.514581in}{1.318444in}}%
\pgfpathlineto{\pgfqpoint{1.520217in}{1.318444in}}%
\pgfpathlineto{\pgfqpoint{1.520217in}{1.331044in}}%
\pgfpathlineto{\pgfqpoint{1.537126in}{1.331044in}}%
\pgfpathlineto{\pgfqpoint{1.537126in}{1.343644in}}%
\pgfpathlineto{\pgfqpoint{1.545581in}{1.343644in}}%
\pgfpathlineto{\pgfqpoint{1.545581in}{1.368844in}}%
\pgfpathlineto{\pgfqpoint{1.562490in}{1.368844in}}%
\pgfpathlineto{\pgfqpoint{1.562490in}{1.394044in}}%
\pgfpathlineto{\pgfqpoint{1.568126in}{1.394044in}}%
\pgfpathlineto{\pgfqpoint{1.568126in}{1.406644in}}%
\pgfpathlineto{\pgfqpoint{1.587853in}{1.406644in}}%
\pgfpathlineto{\pgfqpoint{1.587853in}{1.419244in}}%
\pgfpathlineto{\pgfqpoint{1.604763in}{1.419244in}}%
\pgfpathlineto{\pgfqpoint{1.604763in}{1.431844in}}%
\pgfpathlineto{\pgfqpoint{1.607581in}{1.431844in}}%
\pgfpathlineto{\pgfqpoint{1.607581in}{1.444444in}}%
\pgfpathlineto{\pgfqpoint{1.621672in}{1.444444in}}%
\pgfpathlineto{\pgfqpoint{1.621672in}{1.457044in}}%
\pgfpathlineto{\pgfqpoint{1.630126in}{1.457044in}}%
\pgfpathlineto{\pgfqpoint{1.630126in}{1.469644in}}%
\pgfpathlineto{\pgfqpoint{1.638581in}{1.469644in}}%
\pgfpathlineto{\pgfqpoint{1.638581in}{1.482244in}}%
\pgfpathlineto{\pgfqpoint{1.647035in}{1.482244in}}%
\pgfpathlineto{\pgfqpoint{1.647035in}{1.494844in}}%
\pgfpathlineto{\pgfqpoint{1.652672in}{1.494844in}}%
\pgfpathlineto{\pgfqpoint{1.652672in}{1.507444in}}%
\pgfpathlineto{\pgfqpoint{1.669581in}{1.507444in}}%
\pgfpathlineto{\pgfqpoint{1.669581in}{1.520044in}}%
\pgfpathlineto{\pgfqpoint{1.692126in}{1.520044in}}%
\pgfpathlineto{\pgfqpoint{1.692126in}{1.532644in}}%
\pgfpathlineto{\pgfqpoint{1.728763in}{1.532644in}}%
\pgfpathlineto{\pgfqpoint{1.728763in}{1.545244in}}%
\pgfpathlineto{\pgfqpoint{1.759763in}{1.545244in}}%
\pgfpathlineto{\pgfqpoint{1.759763in}{1.557844in}}%
\pgfpathlineto{\pgfqpoint{1.782308in}{1.557844in}}%
\pgfpathlineto{\pgfqpoint{1.782308in}{1.570444in}}%
\pgfpathlineto{\pgfqpoint{1.807672in}{1.570444in}}%
\pgfpathlineto{\pgfqpoint{1.807672in}{1.583044in}}%
\pgfpathlineto{\pgfqpoint{1.835853in}{1.583044in}}%
\pgfpathlineto{\pgfqpoint{1.835853in}{1.595644in}}%
\pgfpathlineto{\pgfqpoint{1.886581in}{1.595644in}}%
\pgfpathlineto{\pgfqpoint{1.886581in}{1.620844in}}%
\pgfpathlineto{\pgfqpoint{1.889399in}{1.620844in}}%
\pgfpathlineto{\pgfqpoint{1.889399in}{1.633444in}}%
\pgfpathlineto{\pgfqpoint{1.920399in}{1.633444in}}%
\pgfpathlineto{\pgfqpoint{1.920399in}{1.646044in}}%
\pgfpathlineto{\pgfqpoint{1.948581in}{1.646044in}}%
\pgfpathlineto{\pgfqpoint{1.948581in}{1.671244in}}%
\pgfpathlineto{\pgfqpoint{1.962672in}{1.671244in}}%
\pgfpathlineto{\pgfqpoint{1.962672in}{1.683844in}}%
\pgfpathlineto{\pgfqpoint{1.979581in}{1.683844in}}%
\pgfpathlineto{\pgfqpoint{1.979581in}{1.696444in}}%
\pgfpathlineto{\pgfqpoint{2.061308in}{1.696444in}}%
\pgfpathlineto{\pgfqpoint{2.061308in}{1.709044in}}%
\pgfpathlineto{\pgfqpoint{2.075399in}{1.709044in}}%
\pgfpathlineto{\pgfqpoint{2.075399in}{1.721644in}}%
\pgfpathlineto{\pgfqpoint{2.112035in}{1.721644in}}%
\pgfpathlineto{\pgfqpoint{2.112035in}{1.734244in}}%
\pgfpathlineto{\pgfqpoint{2.114853in}{1.734244in}}%
\pgfpathlineto{\pgfqpoint{2.114853in}{1.759444in}}%
\pgfpathlineto{\pgfqpoint{2.174035in}{1.759444in}}%
\pgfpathlineto{\pgfqpoint{2.174035in}{1.772044in}}%
\pgfpathlineto{\pgfqpoint{2.199399in}{1.772044in}}%
\pgfpathlineto{\pgfqpoint{2.199399in}{1.784644in}}%
\pgfpathlineto{\pgfqpoint{2.269853in}{1.784644in}}%
\pgfpathlineto{\pgfqpoint{2.269853in}{1.797244in}}%
\pgfpathlineto{\pgfqpoint{2.298035in}{1.797244in}}%
\pgfpathlineto{\pgfqpoint{2.298035in}{1.809844in}}%
\pgfpathlineto{\pgfqpoint{2.309308in}{1.809844in}}%
\pgfpathlineto{\pgfqpoint{2.309308in}{1.822444in}}%
\pgfpathlineto{\pgfqpoint{2.329035in}{1.822444in}}%
\pgfpathlineto{\pgfqpoint{2.329035in}{1.822444in}}%
\pgfusepath{stroke}%
\end{pgfscope}%
\begin{pgfscope}%
\pgfsetrectcap%
\pgfsetmiterjoin%
\pgfsetlinewidth{0.803000pt}%
\definecolor{currentstroke}{rgb}{0.000000,0.000000,0.000000}%
\pgfsetstrokecolor{currentstroke}%
\pgfsetdash{}{0pt}%
\pgfpathmoveto{\pgfqpoint{0.553581in}{0.499444in}}%
\pgfpathlineto{\pgfqpoint{0.553581in}{1.885444in}}%
\pgfusepath{stroke}%
\end{pgfscope}%
\begin{pgfscope}%
\pgfsetrectcap%
\pgfsetmiterjoin%
\pgfsetlinewidth{0.803000pt}%
\definecolor{currentstroke}{rgb}{0.000000,0.000000,0.000000}%
\pgfsetstrokecolor{currentstroke}%
\pgfsetdash{}{0pt}%
\pgfpathmoveto{\pgfqpoint{2.413581in}{0.499444in}}%
\pgfpathlineto{\pgfqpoint{2.413581in}{1.885444in}}%
\pgfusepath{stroke}%
\end{pgfscope}%
\begin{pgfscope}%
\pgfsetrectcap%
\pgfsetmiterjoin%
\pgfsetlinewidth{0.803000pt}%
\definecolor{currentstroke}{rgb}{0.000000,0.000000,0.000000}%
\pgfsetstrokecolor{currentstroke}%
\pgfsetdash{}{0pt}%
\pgfpathmoveto{\pgfqpoint{0.553581in}{0.499444in}}%
\pgfpathlineto{\pgfqpoint{2.413581in}{0.499444in}}%
\pgfusepath{stroke}%
\end{pgfscope}%
\begin{pgfscope}%
\pgfsetrectcap%
\pgfsetmiterjoin%
\pgfsetlinewidth{0.803000pt}%
\definecolor{currentstroke}{rgb}{0.000000,0.000000,0.000000}%
\pgfsetstrokecolor{currentstroke}%
\pgfsetdash{}{0pt}%
\pgfpathmoveto{\pgfqpoint{0.553581in}{1.885444in}}%
\pgfpathlineto{\pgfqpoint{2.413581in}{1.885444in}}%
\pgfusepath{stroke}%
\end{pgfscope}%
\begin{pgfscope}%
\pgfsetbuttcap%
\pgfsetmiterjoin%
\definecolor{currentfill}{rgb}{1.000000,1.000000,1.000000}%
\pgfsetfillcolor{currentfill}%
\pgfsetlinewidth{1.003750pt}%
\definecolor{currentstroke}{rgb}{1.000000,1.000000,1.000000}%
\pgfsetstrokecolor{currentstroke}%
\pgfsetdash{}{0pt}%
\pgfpathmoveto{\pgfqpoint{1.430843in}{1.210744in}}%
\pgfpathlineto{\pgfqpoint{1.858343in}{1.210744in}}%
\pgfpathlineto{\pgfqpoint{1.858343in}{1.445189in}}%
\pgfpathlineto{\pgfqpoint{1.430843in}{1.445189in}}%
\pgfpathlineto{\pgfqpoint{1.430843in}{1.210744in}}%
\pgfpathclose%
\pgfusepath{stroke,fill}%
\end{pgfscope}%
\begin{pgfscope}%
\definecolor{textcolor}{rgb}{0.000000,0.000000,0.000000}%
\pgfsetstrokecolor{textcolor}%
\pgfsetfillcolor{textcolor}%
\pgftext[x=1.486399in,y=1.293244in,left,base]{\color{textcolor}\rmfamily\fontsize{10.000000}{12.000000}\selectfont 0.517}%
\end{pgfscope}%
\begin{pgfscope}%
\pgfsetbuttcap%
\pgfsetmiterjoin%
\definecolor{currentfill}{rgb}{1.000000,1.000000,1.000000}%
\pgfsetfillcolor{currentfill}%
\pgfsetlinewidth{1.003750pt}%
\definecolor{currentstroke}{rgb}{1.000000,1.000000,1.000000}%
\pgfsetstrokecolor{currentstroke}%
\pgfsetdash{}{0pt}%
\pgfpathmoveto{\pgfqpoint{1.287116in}{1.109944in}}%
\pgfpathlineto{\pgfqpoint{1.714616in}{1.109944in}}%
\pgfpathlineto{\pgfqpoint{1.714616in}{1.344389in}}%
\pgfpathlineto{\pgfqpoint{1.287116in}{1.344389in}}%
\pgfpathlineto{\pgfqpoint{1.287116in}{1.109944in}}%
\pgfpathclose%
\pgfusepath{stroke,fill}%
\end{pgfscope}%
\begin{pgfscope}%
\definecolor{textcolor}{rgb}{0.000000,0.000000,0.000000}%
\pgfsetstrokecolor{textcolor}%
\pgfsetfillcolor{textcolor}%
\pgftext[x=1.342672in,y=1.192444in,left,base]{\color{textcolor}\rmfamily\fontsize{10.000000}{12.000000}\selectfont 0.612}%
\end{pgfscope}%
\begin{pgfscope}%
\definecolor{textcolor}{rgb}{0.000000,0.000000,0.000000}%
\pgfsetstrokecolor{textcolor}%
\pgfsetfillcolor{textcolor}%
\pgftext[x=1.483581in,y=1.968778in,,base]{\color{textcolor}\rmfamily\fontsize{12.000000}{14.400000}\selectfont ROC Curve}%
\end{pgfscope}%
\begin{pgfscope}%
\pgfsetbuttcap%
\pgfsetmiterjoin%
\definecolor{currentfill}{rgb}{1.000000,1.000000,1.000000}%
\pgfsetfillcolor{currentfill}%
\pgfsetfillopacity{0.800000}%
\pgfsetlinewidth{1.003750pt}%
\definecolor{currentstroke}{rgb}{0.800000,0.800000,0.800000}%
\pgfsetstrokecolor{currentstroke}%
\pgfsetstrokeopacity{0.800000}%
\pgfsetdash{}{0pt}%
\pgfpathmoveto{\pgfqpoint{1.150525in}{0.568889in}}%
\pgfpathlineto{\pgfqpoint{2.316358in}{0.568889in}}%
\pgfpathquadraticcurveto{\pgfqpoint{2.344136in}{0.568889in}}{\pgfqpoint{2.344136in}{0.596666in}}%
\pgfpathlineto{\pgfqpoint{2.344136in}{0.791111in}}%
\pgfpathquadraticcurveto{\pgfqpoint{2.344136in}{0.818888in}}{\pgfqpoint{2.316358in}{0.818888in}}%
\pgfpathlineto{\pgfqpoint{1.150525in}{0.818888in}}%
\pgfpathquadraticcurveto{\pgfqpoint{1.122747in}{0.818888in}}{\pgfqpoint{1.122747in}{0.791111in}}%
\pgfpathlineto{\pgfqpoint{1.122747in}{0.596666in}}%
\pgfpathquadraticcurveto{\pgfqpoint{1.122747in}{0.568889in}}{\pgfqpoint{1.150525in}{0.568889in}}%
\pgfpathlineto{\pgfqpoint{1.150525in}{0.568889in}}%
\pgfpathclose%
\pgfusepath{stroke,fill}%
\end{pgfscope}%
\begin{pgfscope}%
\pgfsetrectcap%
\pgfsetroundjoin%
\pgfsetlinewidth{1.505625pt}%
\definecolor{currentstroke}{rgb}{0.121569,0.466667,0.705882}%
\pgfsetstrokecolor{currentstroke}%
\pgfsetdash{}{0pt}%
\pgfpathmoveto{\pgfqpoint{1.178303in}{0.707777in}}%
\pgfpathlineto{\pgfqpoint{1.317192in}{0.707777in}}%
\pgfpathlineto{\pgfqpoint{1.456081in}{0.707777in}}%
\pgfusepath{stroke}%
\end{pgfscope}%
\begin{pgfscope}%
\definecolor{textcolor}{rgb}{0.000000,0.000000,0.000000}%
\pgfsetstrokecolor{textcolor}%
\pgfsetfillcolor{textcolor}%
\pgftext[x=1.567192in,y=0.659166in,left,base]{\color{textcolor}\rmfamily\fontsize{10.000000}{12.000000}\selectfont AUC 0.562)}%
\end{pgfscope}%
\end{pgfpicture}%
\makeatother%
\endgroup%

\end{tabular}
\end{center}

\begin{center}
\begin{tabular}{cc}
\begin{tabular}{cc|c|c|}
	&\multicolumn{1}{c}{}& \multicolumn{2}{c}{Prediction} \cr
	&\multicolumn{1}{c}{} & \multicolumn{1}{c}{N} & \multicolumn{1}{c}{P} \cr\cline{3-4}
	\multirow{2}{*}{Actual}&N & 41.1\% & 44.6\% \vrule width 0pt height 10pt depth 2pt \cr\cline{3-4}
	&P & 5.71\% & 8.57\% \vrule width 0pt height 10pt depth 2pt \cr\cline{3-4}
\end{tabular}
&
\begin{tabular}{ll}
0.497 & Accuracy \cr 
0.540 & Balanced Accuracy \cr 
0.161 & Precision \cr 
0.536 & Balanced Precision \cr 
0.600 & Recall \cr 
0.254 & F1 \cr 
0.566 & Balanced F1 \cr 
0.278 & Gmean \cr 	\end{tabular}
\end{tabular}
\end{center}






