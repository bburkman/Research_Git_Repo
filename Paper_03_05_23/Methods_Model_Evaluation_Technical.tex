%%%
\subsection{Model Evaluation:  Baselines for Comparison}

What do good results look like, what do bad results look like, how do we measure it, and when we compare two results, how much of the difference could be due to randomness?

In the supervised learning method we used here, for each of the $\approx 600,000$ samples (people) in the dataset, we know the answer (the {\it label} or {\it ground truth}) to the question, whether the person needed an ambulance, $y=0$ for ``no'' and $y=1$ for ``yes.''  We are trying use historical data to build a model to predict the label for new data (incoming automated crash notifications).

{\bf
Explain this better.  I confused the internal workings of the model building with the results.
}

We split the data 70/30 into a training set and a test set, making sure to keep the same proportion of positive and negative samples in both.  The binary classification models we used take the training data and training labels (\verb|X_train| and \verb|y_train|) and build a model, then apply the model to the test data (\verb|X_test|) and returns \verb|y_proba| that gives, for each sample, a continuous probability $p \in (0,1)$ that the sample belongs in the positive class.   If a sample has $p = 0.1$, the model is 90\% confident that this sample is in the negative class.   We then pick a threshold, usually but not necessarily $threshold = 0.5$, and make a binary prediction, that samples with $p > threshold$ need an ambulance, and those with $p < threshold$ do not.   

While building the model, the algorithm picks a starting point, measures how badly the model predicts the training data using the {\it loss function}, tweaks the model, measures again, and either keeps or rejects the candidate model based on the loss function.  The loss function used by the model is the sum not of how many binary predictions were incorrect, but how strongly incorrect the continuous predictions were.  If two negative samples ($y=0$) had $p=0.1$ and $p=0.4$, both correct classifications if $threshold=0.5$, then the $p=0.4$ sample would add much more to the loss value.  

A perfect model would not only predict each sample's label correctly, but would do it with perfect certainty.  In the real world, with interesting questions about real data, we will have false positives ($y=0$ and $p>threshold$) and false negatives ($y=1$ and $p<threshold$), but we hope those are few, and that the predictions are strongly correct, meaning the predictions are close to their labels.

When we get results for our models based on crash data, we need some frame of reference for what is ``good'' and ``bad,'' so we have created some sets of entirely artificial results using a gamma distribution for ideal results and a uniform distribution for awful results.  

The histogram below of the percent of samples with predictions $p$ in each range illustrates the best results we can hope for in the real world.  The positive class is small because the data is imbalanced. about 15\% of the dataset, as in our CRSS data.  There are some false positives and negatives, but the overwhelming majority of the predictions are correct, and most with strong confidence.  

The Receiver Operating Characteristic (ROC) is a parameterized curve following the probability threshold from $p=0$ to $p=1$, plotting the true positive rate (TPR) versus the false positive rate (FPR).  The Area Under the ROC curve (AUC) is often used to compare two models, with AUC of 1 indicating perfect prediction and AUC of 0.5 indicating no discernable pattern.  

We have added to the typical ROC curve labels for the medians of the probabilities in the negative and positive classes (0.338 and 0.655) and the default decision threshold $thr = 0.5$.

\begin{tabular}{p{0.5\textwidth} p{0.5\textwidth}}
  \vspace{0pt} %% Creator: Matplotlib, PGF backend
%%
%% To include the figure in your LaTeX document, write
%%   \input{<filename>.pgf}
%%
%% Make sure the required packages are loaded in your preamble
%%   \usepackage{pgf}
%%
%% Also ensure that all the required font packages are loaded; for instance,
%% the lmodern package is sometimes necessary when using math font.
%%   \usepackage{lmodern}
%%
%% Figures using additional raster images can only be included by \input if
%% they are in the same directory as the main LaTeX file. For loading figures
%% from other directories you can use the `import` package
%%   \usepackage{import}
%%
%% and then include the figures with
%%   \import{<path to file>}{<filename>.pgf}
%%
%% Matplotlib used the following preamble
%%   
%%   \usepackage{fontspec}
%%   \makeatletter\@ifpackageloaded{underscore}{}{\usepackage[strings]{underscore}}\makeatother
%%
\begingroup%
\makeatletter%
\begin{pgfpicture}%
\pgfpathrectangle{\pgfpointorigin}{\pgfqpoint{3.095000in}{3.044944in}}%
\pgfusepath{use as bounding box, clip}%
\begin{pgfscope}%
\pgfsetbuttcap%
\pgfsetmiterjoin%
\definecolor{currentfill}{rgb}{1.000000,1.000000,1.000000}%
\pgfsetfillcolor{currentfill}%
\pgfsetlinewidth{0.000000pt}%
\definecolor{currentstroke}{rgb}{1.000000,1.000000,1.000000}%
\pgfsetstrokecolor{currentstroke}%
\pgfsetdash{}{0pt}%
\pgfpathmoveto{\pgfqpoint{0.000000in}{0.000000in}}%
\pgfpathlineto{\pgfqpoint{3.095000in}{0.000000in}}%
\pgfpathlineto{\pgfqpoint{3.095000in}{3.044944in}}%
\pgfpathlineto{\pgfqpoint{0.000000in}{3.044944in}}%
\pgfpathlineto{\pgfqpoint{0.000000in}{0.000000in}}%
\pgfpathclose%
\pgfusepath{fill}%
\end{pgfscope}%
\begin{pgfscope}%
\pgfsetbuttcap%
\pgfsetmiterjoin%
\definecolor{currentfill}{rgb}{1.000000,1.000000,1.000000}%
\pgfsetfillcolor{currentfill}%
\pgfsetlinewidth{0.000000pt}%
\definecolor{currentstroke}{rgb}{0.000000,0.000000,0.000000}%
\pgfsetstrokecolor{currentstroke}%
\pgfsetstrokeopacity{0.000000}%
\pgfsetdash{}{0pt}%
\pgfpathmoveto{\pgfqpoint{0.515000in}{1.096944in}}%
\pgfpathlineto{\pgfqpoint{2.995000in}{1.096944in}}%
\pgfpathlineto{\pgfqpoint{2.995000in}{2.944944in}}%
\pgfpathlineto{\pgfqpoint{0.515000in}{2.944944in}}%
\pgfpathlineto{\pgfqpoint{0.515000in}{1.096944in}}%
\pgfpathclose%
\pgfusepath{fill}%
\end{pgfscope}%
\begin{pgfscope}%
\pgfpathrectangle{\pgfqpoint{0.515000in}{1.096944in}}{\pgfqpoint{2.480000in}{1.848000in}}%
\pgfusepath{clip}%
\pgfsetbuttcap%
\pgfsetmiterjoin%
\pgfsetlinewidth{1.003750pt}%
\definecolor{currentstroke}{rgb}{0.000000,0.000000,0.000000}%
\pgfsetstrokecolor{currentstroke}%
\pgfsetdash{}{0pt}%
\pgfpathmoveto{\pgfqpoint{0.505000in}{1.096944in}}%
\pgfpathlineto{\pgfqpoint{0.577627in}{1.096944in}}%
\pgfpathlineto{\pgfqpoint{0.577627in}{1.643930in}}%
\pgfpathlineto{\pgfqpoint{0.505000in}{1.643930in}}%
\pgfusepath{stroke}%
\end{pgfscope}%
\begin{pgfscope}%
\pgfpathrectangle{\pgfqpoint{0.515000in}{1.096944in}}{\pgfqpoint{2.480000in}{1.848000in}}%
\pgfusepath{clip}%
\pgfsetbuttcap%
\pgfsetmiterjoin%
\pgfsetlinewidth{1.003750pt}%
\definecolor{currentstroke}{rgb}{0.000000,0.000000,0.000000}%
\pgfsetstrokecolor{currentstroke}%
\pgfsetdash{}{0pt}%
\pgfpathmoveto{\pgfqpoint{0.727930in}{1.096944in}}%
\pgfpathlineto{\pgfqpoint{0.828132in}{1.096944in}}%
\pgfpathlineto{\pgfqpoint{0.828132in}{2.793426in}}%
\pgfpathlineto{\pgfqpoint{0.727930in}{2.793426in}}%
\pgfpathlineto{\pgfqpoint{0.727930in}{1.096944in}}%
\pgfpathclose%
\pgfusepath{stroke}%
\end{pgfscope}%
\begin{pgfscope}%
\pgfpathrectangle{\pgfqpoint{0.515000in}{1.096944in}}{\pgfqpoint{2.480000in}{1.848000in}}%
\pgfusepath{clip}%
\pgfsetbuttcap%
\pgfsetmiterjoin%
\pgfsetlinewidth{1.003750pt}%
\definecolor{currentstroke}{rgb}{0.000000,0.000000,0.000000}%
\pgfsetstrokecolor{currentstroke}%
\pgfsetdash{}{0pt}%
\pgfpathmoveto{\pgfqpoint{0.978435in}{1.096944in}}%
\pgfpathlineto{\pgfqpoint{1.078637in}{1.096944in}}%
\pgfpathlineto{\pgfqpoint{1.078637in}{2.856944in}}%
\pgfpathlineto{\pgfqpoint{0.978435in}{2.856944in}}%
\pgfpathlineto{\pgfqpoint{0.978435in}{1.096944in}}%
\pgfpathclose%
\pgfusepath{stroke}%
\end{pgfscope}%
\begin{pgfscope}%
\pgfpathrectangle{\pgfqpoint{0.515000in}{1.096944in}}{\pgfqpoint{2.480000in}{1.848000in}}%
\pgfusepath{clip}%
\pgfsetbuttcap%
\pgfsetmiterjoin%
\pgfsetlinewidth{1.003750pt}%
\definecolor{currentstroke}{rgb}{0.000000,0.000000,0.000000}%
\pgfsetstrokecolor{currentstroke}%
\pgfsetdash{}{0pt}%
\pgfpathmoveto{\pgfqpoint{1.228940in}{1.096944in}}%
\pgfpathlineto{\pgfqpoint{1.329142in}{1.096944in}}%
\pgfpathlineto{\pgfqpoint{1.329142in}{2.393983in}}%
\pgfpathlineto{\pgfqpoint{1.228940in}{2.393983in}}%
\pgfpathlineto{\pgfqpoint{1.228940in}{1.096944in}}%
\pgfpathclose%
\pgfusepath{stroke}%
\end{pgfscope}%
\begin{pgfscope}%
\pgfpathrectangle{\pgfqpoint{0.515000in}{1.096944in}}{\pgfqpoint{2.480000in}{1.848000in}}%
\pgfusepath{clip}%
\pgfsetbuttcap%
\pgfsetmiterjoin%
\pgfsetlinewidth{1.003750pt}%
\definecolor{currentstroke}{rgb}{0.000000,0.000000,0.000000}%
\pgfsetstrokecolor{currentstroke}%
\pgfsetdash{}{0pt}%
\pgfpathmoveto{\pgfqpoint{1.479445in}{1.096944in}}%
\pgfpathlineto{\pgfqpoint{1.579647in}{1.096944in}}%
\pgfpathlineto{\pgfqpoint{1.579647in}{1.874976in}}%
\pgfpathlineto{\pgfqpoint{1.479445in}{1.874976in}}%
\pgfpathlineto{\pgfqpoint{1.479445in}{1.096944in}}%
\pgfpathclose%
\pgfusepath{stroke}%
\end{pgfscope}%
\begin{pgfscope}%
\pgfpathrectangle{\pgfqpoint{0.515000in}{1.096944in}}{\pgfqpoint{2.480000in}{1.848000in}}%
\pgfusepath{clip}%
\pgfsetbuttcap%
\pgfsetmiterjoin%
\pgfsetlinewidth{1.003750pt}%
\definecolor{currentstroke}{rgb}{0.000000,0.000000,0.000000}%
\pgfsetstrokecolor{currentstroke}%
\pgfsetdash{}{0pt}%
\pgfpathmoveto{\pgfqpoint{1.729950in}{1.096944in}}%
\pgfpathlineto{\pgfqpoint{1.830152in}{1.096944in}}%
\pgfpathlineto{\pgfqpoint{1.830152in}{1.536097in}}%
\pgfpathlineto{\pgfqpoint{1.729950in}{1.536097in}}%
\pgfpathlineto{\pgfqpoint{1.729950in}{1.096944in}}%
\pgfpathclose%
\pgfusepath{stroke}%
\end{pgfscope}%
\begin{pgfscope}%
\pgfpathrectangle{\pgfqpoint{0.515000in}{1.096944in}}{\pgfqpoint{2.480000in}{1.848000in}}%
\pgfusepath{clip}%
\pgfsetbuttcap%
\pgfsetmiterjoin%
\pgfsetlinewidth{1.003750pt}%
\definecolor{currentstroke}{rgb}{0.000000,0.000000,0.000000}%
\pgfsetstrokecolor{currentstroke}%
\pgfsetdash{}{0pt}%
\pgfpathmoveto{\pgfqpoint{1.980455in}{1.096944in}}%
\pgfpathlineto{\pgfqpoint{2.080657in}{1.096944in}}%
\pgfpathlineto{\pgfqpoint{2.080657in}{1.343370in}}%
\pgfpathlineto{\pgfqpoint{1.980455in}{1.343370in}}%
\pgfpathlineto{\pgfqpoint{1.980455in}{1.096944in}}%
\pgfpathclose%
\pgfusepath{stroke}%
\end{pgfscope}%
\begin{pgfscope}%
\pgfpathrectangle{\pgfqpoint{0.515000in}{1.096944in}}{\pgfqpoint{2.480000in}{1.848000in}}%
\pgfusepath{clip}%
\pgfsetbuttcap%
\pgfsetmiterjoin%
\pgfsetlinewidth{1.003750pt}%
\definecolor{currentstroke}{rgb}{0.000000,0.000000,0.000000}%
\pgfsetstrokecolor{currentstroke}%
\pgfsetdash{}{0pt}%
\pgfpathmoveto{\pgfqpoint{2.230960in}{1.096944in}}%
\pgfpathlineto{\pgfqpoint{2.331162in}{1.096944in}}%
\pgfpathlineto{\pgfqpoint{2.331162in}{1.205472in}}%
\pgfpathlineto{\pgfqpoint{2.230960in}{1.205472in}}%
\pgfpathlineto{\pgfqpoint{2.230960in}{1.096944in}}%
\pgfpathclose%
\pgfusepath{stroke}%
\end{pgfscope}%
\begin{pgfscope}%
\pgfpathrectangle{\pgfqpoint{0.515000in}{1.096944in}}{\pgfqpoint{2.480000in}{1.848000in}}%
\pgfusepath{clip}%
\pgfsetbuttcap%
\pgfsetmiterjoin%
\pgfsetlinewidth{1.003750pt}%
\definecolor{currentstroke}{rgb}{0.000000,0.000000,0.000000}%
\pgfsetstrokecolor{currentstroke}%
\pgfsetdash{}{0pt}%
\pgfpathmoveto{\pgfqpoint{2.481465in}{1.096944in}}%
\pgfpathlineto{\pgfqpoint{2.581667in}{1.096944in}}%
\pgfpathlineto{\pgfqpoint{2.581667in}{1.149427in}}%
\pgfpathlineto{\pgfqpoint{2.481465in}{1.149427in}}%
\pgfpathlineto{\pgfqpoint{2.481465in}{1.096944in}}%
\pgfpathclose%
\pgfusepath{stroke}%
\end{pgfscope}%
\begin{pgfscope}%
\pgfpathrectangle{\pgfqpoint{0.515000in}{1.096944in}}{\pgfqpoint{2.480000in}{1.848000in}}%
\pgfusepath{clip}%
\pgfsetbuttcap%
\pgfsetmiterjoin%
\pgfsetlinewidth{1.003750pt}%
\definecolor{currentstroke}{rgb}{0.000000,0.000000,0.000000}%
\pgfsetstrokecolor{currentstroke}%
\pgfsetdash{}{0pt}%
\pgfpathmoveto{\pgfqpoint{2.731970in}{1.096944in}}%
\pgfpathlineto{\pgfqpoint{2.832172in}{1.096944in}}%
\pgfpathlineto{\pgfqpoint{2.832172in}{1.123186in}}%
\pgfpathlineto{\pgfqpoint{2.731970in}{1.123186in}}%
\pgfpathlineto{\pgfqpoint{2.731970in}{1.096944in}}%
\pgfpathclose%
\pgfusepath{stroke}%
\end{pgfscope}%
\begin{pgfscope}%
\pgfpathrectangle{\pgfqpoint{0.515000in}{1.096944in}}{\pgfqpoint{2.480000in}{1.848000in}}%
\pgfusepath{clip}%
\pgfsetbuttcap%
\pgfsetmiterjoin%
\definecolor{currentfill}{rgb}{0.000000,0.000000,0.000000}%
\pgfsetfillcolor{currentfill}%
\pgfsetlinewidth{0.000000pt}%
\definecolor{currentstroke}{rgb}{0.000000,0.000000,0.000000}%
\pgfsetstrokecolor{currentstroke}%
\pgfsetstrokeopacity{0.000000}%
\pgfsetdash{}{0pt}%
\pgfpathmoveto{\pgfqpoint{0.577627in}{1.096944in}}%
\pgfpathlineto{\pgfqpoint{0.677829in}{1.096944in}}%
\pgfpathlineto{\pgfqpoint{0.677829in}{1.103114in}}%
\pgfpathlineto{\pgfqpoint{0.577627in}{1.103114in}}%
\pgfpathlineto{\pgfqpoint{0.577627in}{1.096944in}}%
\pgfpathclose%
\pgfusepath{fill}%
\end{pgfscope}%
\begin{pgfscope}%
\pgfpathrectangle{\pgfqpoint{0.515000in}{1.096944in}}{\pgfqpoint{2.480000in}{1.848000in}}%
\pgfusepath{clip}%
\pgfsetbuttcap%
\pgfsetmiterjoin%
\definecolor{currentfill}{rgb}{0.000000,0.000000,0.000000}%
\pgfsetfillcolor{currentfill}%
\pgfsetlinewidth{0.000000pt}%
\definecolor{currentstroke}{rgb}{0.000000,0.000000,0.000000}%
\pgfsetstrokecolor{currentstroke}%
\pgfsetstrokeopacity{0.000000}%
\pgfsetdash{}{0pt}%
\pgfpathmoveto{\pgfqpoint{0.828132in}{1.096944in}}%
\pgfpathlineto{\pgfqpoint{0.928334in}{1.096944in}}%
\pgfpathlineto{\pgfqpoint{0.928334in}{1.108761in}}%
\pgfpathlineto{\pgfqpoint{0.828132in}{1.108761in}}%
\pgfpathlineto{\pgfqpoint{0.828132in}{1.096944in}}%
\pgfpathclose%
\pgfusepath{fill}%
\end{pgfscope}%
\begin{pgfscope}%
\pgfpathrectangle{\pgfqpoint{0.515000in}{1.096944in}}{\pgfqpoint{2.480000in}{1.848000in}}%
\pgfusepath{clip}%
\pgfsetbuttcap%
\pgfsetmiterjoin%
\definecolor{currentfill}{rgb}{0.000000,0.000000,0.000000}%
\pgfsetfillcolor{currentfill}%
\pgfsetlinewidth{0.000000pt}%
\definecolor{currentstroke}{rgb}{0.000000,0.000000,0.000000}%
\pgfsetstrokecolor{currentstroke}%
\pgfsetstrokeopacity{0.000000}%
\pgfsetdash{}{0pt}%
\pgfpathmoveto{\pgfqpoint{1.078637in}{1.096944in}}%
\pgfpathlineto{\pgfqpoint{1.178839in}{1.096944in}}%
\pgfpathlineto{\pgfqpoint{1.178839in}{1.126835in}}%
\pgfpathlineto{\pgfqpoint{1.078637in}{1.126835in}}%
\pgfpathlineto{\pgfqpoint{1.078637in}{1.096944in}}%
\pgfpathclose%
\pgfusepath{fill}%
\end{pgfscope}%
\begin{pgfscope}%
\pgfpathrectangle{\pgfqpoint{0.515000in}{1.096944in}}{\pgfqpoint{2.480000in}{1.848000in}}%
\pgfusepath{clip}%
\pgfsetbuttcap%
\pgfsetmiterjoin%
\definecolor{currentfill}{rgb}{0.000000,0.000000,0.000000}%
\pgfsetfillcolor{currentfill}%
\pgfsetlinewidth{0.000000pt}%
\definecolor{currentstroke}{rgb}{0.000000,0.000000,0.000000}%
\pgfsetstrokecolor{currentstroke}%
\pgfsetstrokeopacity{0.000000}%
\pgfsetdash{}{0pt}%
\pgfpathmoveto{\pgfqpoint{1.329142in}{1.096944in}}%
\pgfpathlineto{\pgfqpoint{1.429344in}{1.096944in}}%
\pgfpathlineto{\pgfqpoint{1.429344in}{1.150557in}}%
\pgfpathlineto{\pgfqpoint{1.329142in}{1.150557in}}%
\pgfpathlineto{\pgfqpoint{1.329142in}{1.096944in}}%
\pgfpathclose%
\pgfusepath{fill}%
\end{pgfscope}%
\begin{pgfscope}%
\pgfpathrectangle{\pgfqpoint{0.515000in}{1.096944in}}{\pgfqpoint{2.480000in}{1.848000in}}%
\pgfusepath{clip}%
\pgfsetbuttcap%
\pgfsetmiterjoin%
\definecolor{currentfill}{rgb}{0.000000,0.000000,0.000000}%
\pgfsetfillcolor{currentfill}%
\pgfsetlinewidth{0.000000pt}%
\definecolor{currentstroke}{rgb}{0.000000,0.000000,0.000000}%
\pgfsetstrokecolor{currentstroke}%
\pgfsetstrokeopacity{0.000000}%
\pgfsetdash{}{0pt}%
\pgfpathmoveto{\pgfqpoint{1.579647in}{1.096944in}}%
\pgfpathlineto{\pgfqpoint{1.679849in}{1.096944in}}%
\pgfpathlineto{\pgfqpoint{1.679849in}{1.207645in}}%
\pgfpathlineto{\pgfqpoint{1.579647in}{1.207645in}}%
\pgfpathlineto{\pgfqpoint{1.579647in}{1.096944in}}%
\pgfpathclose%
\pgfusepath{fill}%
\end{pgfscope}%
\begin{pgfscope}%
\pgfpathrectangle{\pgfqpoint{0.515000in}{1.096944in}}{\pgfqpoint{2.480000in}{1.848000in}}%
\pgfusepath{clip}%
\pgfsetbuttcap%
\pgfsetmiterjoin%
\definecolor{currentfill}{rgb}{0.000000,0.000000,0.000000}%
\pgfsetfillcolor{currentfill}%
\pgfsetlinewidth{0.000000pt}%
\definecolor{currentstroke}{rgb}{0.000000,0.000000,0.000000}%
\pgfsetstrokecolor{currentstroke}%
\pgfsetstrokeopacity{0.000000}%
\pgfsetdash{}{0pt}%
\pgfpathmoveto{\pgfqpoint{1.830152in}{1.096944in}}%
\pgfpathlineto{\pgfqpoint{1.930354in}{1.096944in}}%
\pgfpathlineto{\pgfqpoint{1.930354in}{1.295493in}}%
\pgfpathlineto{\pgfqpoint{1.830152in}{1.295493in}}%
\pgfpathlineto{\pgfqpoint{1.830152in}{1.096944in}}%
\pgfpathclose%
\pgfusepath{fill}%
\end{pgfscope}%
\begin{pgfscope}%
\pgfpathrectangle{\pgfqpoint{0.515000in}{1.096944in}}{\pgfqpoint{2.480000in}{1.848000in}}%
\pgfusepath{clip}%
\pgfsetbuttcap%
\pgfsetmiterjoin%
\definecolor{currentfill}{rgb}{0.000000,0.000000,0.000000}%
\pgfsetfillcolor{currentfill}%
\pgfsetlinewidth{0.000000pt}%
\definecolor{currentstroke}{rgb}{0.000000,0.000000,0.000000}%
\pgfsetstrokecolor{currentstroke}%
\pgfsetstrokeopacity{0.000000}%
\pgfsetdash{}{0pt}%
\pgfpathmoveto{\pgfqpoint{2.080657in}{1.096944in}}%
\pgfpathlineto{\pgfqpoint{2.180859in}{1.096944in}}%
\pgfpathlineto{\pgfqpoint{2.180859in}{1.417750in}}%
\pgfpathlineto{\pgfqpoint{2.080657in}{1.417750in}}%
\pgfpathlineto{\pgfqpoint{2.080657in}{1.096944in}}%
\pgfpathclose%
\pgfusepath{fill}%
\end{pgfscope}%
\begin{pgfscope}%
\pgfpathrectangle{\pgfqpoint{0.515000in}{1.096944in}}{\pgfqpoint{2.480000in}{1.848000in}}%
\pgfusepath{clip}%
\pgfsetbuttcap%
\pgfsetmiterjoin%
\definecolor{currentfill}{rgb}{0.000000,0.000000,0.000000}%
\pgfsetfillcolor{currentfill}%
\pgfsetlinewidth{0.000000pt}%
\definecolor{currentstroke}{rgb}{0.000000,0.000000,0.000000}%
\pgfsetstrokecolor{currentstroke}%
\pgfsetstrokeopacity{0.000000}%
\pgfsetdash{}{0pt}%
\pgfpathmoveto{\pgfqpoint{2.331162in}{1.096944in}}%
\pgfpathlineto{\pgfqpoint{2.431364in}{1.096944in}}%
\pgfpathlineto{\pgfqpoint{2.431364in}{1.542440in}}%
\pgfpathlineto{\pgfqpoint{2.331162in}{1.542440in}}%
\pgfpathlineto{\pgfqpoint{2.331162in}{1.096944in}}%
\pgfpathclose%
\pgfusepath{fill}%
\end{pgfscope}%
\begin{pgfscope}%
\pgfpathrectangle{\pgfqpoint{0.515000in}{1.096944in}}{\pgfqpoint{2.480000in}{1.848000in}}%
\pgfusepath{clip}%
\pgfsetbuttcap%
\pgfsetmiterjoin%
\definecolor{currentfill}{rgb}{0.000000,0.000000,0.000000}%
\pgfsetfillcolor{currentfill}%
\pgfsetlinewidth{0.000000pt}%
\definecolor{currentstroke}{rgb}{0.000000,0.000000,0.000000}%
\pgfsetstrokecolor{currentstroke}%
\pgfsetstrokeopacity{0.000000}%
\pgfsetdash{}{0pt}%
\pgfpathmoveto{\pgfqpoint{2.581667in}{1.096944in}}%
\pgfpathlineto{\pgfqpoint{2.681869in}{1.096944in}}%
\pgfpathlineto{\pgfqpoint{2.681869in}{1.515504in}}%
\pgfpathlineto{\pgfqpoint{2.581667in}{1.515504in}}%
\pgfpathlineto{\pgfqpoint{2.581667in}{1.096944in}}%
\pgfpathclose%
\pgfusepath{fill}%
\end{pgfscope}%
\begin{pgfscope}%
\pgfpathrectangle{\pgfqpoint{0.515000in}{1.096944in}}{\pgfqpoint{2.480000in}{1.848000in}}%
\pgfusepath{clip}%
\pgfsetbuttcap%
\pgfsetmiterjoin%
\definecolor{currentfill}{rgb}{0.000000,0.000000,0.000000}%
\pgfsetfillcolor{currentfill}%
\pgfsetlinewidth{0.000000pt}%
\definecolor{currentstroke}{rgb}{0.000000,0.000000,0.000000}%
\pgfsetstrokecolor{currentstroke}%
\pgfsetstrokeopacity{0.000000}%
\pgfsetdash{}{0pt}%
\pgfpathmoveto{\pgfqpoint{2.832172in}{1.096944in}}%
\pgfpathlineto{\pgfqpoint{2.932374in}{1.096944in}}%
\pgfpathlineto{\pgfqpoint{2.932374in}{1.239187in}}%
\pgfpathlineto{\pgfqpoint{2.832172in}{1.239187in}}%
\pgfpathlineto{\pgfqpoint{2.832172in}{1.096944in}}%
\pgfpathclose%
\pgfusepath{fill}%
\end{pgfscope}%
\begin{pgfscope}%
\pgfsetbuttcap%
\pgfsetroundjoin%
\definecolor{currentfill}{rgb}{0.000000,0.000000,0.000000}%
\pgfsetfillcolor{currentfill}%
\pgfsetlinewidth{0.803000pt}%
\definecolor{currentstroke}{rgb}{0.000000,0.000000,0.000000}%
\pgfsetstrokecolor{currentstroke}%
\pgfsetdash{}{0pt}%
\pgfsys@defobject{currentmarker}{\pgfqpoint{0.000000in}{-0.048611in}}{\pgfqpoint{0.000000in}{0.000000in}}{%
\pgfpathmoveto{\pgfqpoint{0.000000in}{0.000000in}}%
\pgfpathlineto{\pgfqpoint{0.000000in}{-0.048611in}}%
\pgfusepath{stroke,fill}%
}%
\begin{pgfscope}%
\pgfsys@transformshift{0.577627in}{1.096944in}%
\pgfsys@useobject{currentmarker}{}%
\end{pgfscope}%
\end{pgfscope}%
\begin{pgfscope}%
\definecolor{textcolor}{rgb}{0.000000,0.000000,0.000000}%
\pgfsetstrokecolor{textcolor}%
\pgfsetfillcolor{textcolor}%
\pgftext[x=0.612349in, y=0.282083in, left, base,rotate=90.000000]{\color{textcolor}\rmfamily\fontsize{10.000000}{12.000000}\selectfont (-0.001, 0.1]}%
\end{pgfscope}%
\begin{pgfscope}%
\pgfsetbuttcap%
\pgfsetroundjoin%
\definecolor{currentfill}{rgb}{0.000000,0.000000,0.000000}%
\pgfsetfillcolor{currentfill}%
\pgfsetlinewidth{0.803000pt}%
\definecolor{currentstroke}{rgb}{0.000000,0.000000,0.000000}%
\pgfsetstrokecolor{currentstroke}%
\pgfsetdash{}{0pt}%
\pgfsys@defobject{currentmarker}{\pgfqpoint{0.000000in}{-0.048611in}}{\pgfqpoint{0.000000in}{0.000000in}}{%
\pgfpathmoveto{\pgfqpoint{0.000000in}{0.000000in}}%
\pgfpathlineto{\pgfqpoint{0.000000in}{-0.048611in}}%
\pgfusepath{stroke,fill}%
}%
\begin{pgfscope}%
\pgfsys@transformshift{0.828132in}{1.096944in}%
\pgfsys@useobject{currentmarker}{}%
\end{pgfscope}%
\end{pgfscope}%
\begin{pgfscope}%
\definecolor{textcolor}{rgb}{0.000000,0.000000,0.000000}%
\pgfsetstrokecolor{textcolor}%
\pgfsetfillcolor{textcolor}%
\pgftext[x=0.862854in, y=0.467222in, left, base,rotate=90.000000]{\color{textcolor}\rmfamily\fontsize{10.000000}{12.000000}\selectfont (0.1, 0.2]}%
\end{pgfscope}%
\begin{pgfscope}%
\pgfsetbuttcap%
\pgfsetroundjoin%
\definecolor{currentfill}{rgb}{0.000000,0.000000,0.000000}%
\pgfsetfillcolor{currentfill}%
\pgfsetlinewidth{0.803000pt}%
\definecolor{currentstroke}{rgb}{0.000000,0.000000,0.000000}%
\pgfsetstrokecolor{currentstroke}%
\pgfsetdash{}{0pt}%
\pgfsys@defobject{currentmarker}{\pgfqpoint{0.000000in}{-0.048611in}}{\pgfqpoint{0.000000in}{0.000000in}}{%
\pgfpathmoveto{\pgfqpoint{0.000000in}{0.000000in}}%
\pgfpathlineto{\pgfqpoint{0.000000in}{-0.048611in}}%
\pgfusepath{stroke,fill}%
}%
\begin{pgfscope}%
\pgfsys@transformshift{1.078637in}{1.096944in}%
\pgfsys@useobject{currentmarker}{}%
\end{pgfscope}%
\end{pgfscope}%
\begin{pgfscope}%
\definecolor{textcolor}{rgb}{0.000000,0.000000,0.000000}%
\pgfsetstrokecolor{textcolor}%
\pgfsetfillcolor{textcolor}%
\pgftext[x=1.113359in, y=0.467222in, left, base,rotate=90.000000]{\color{textcolor}\rmfamily\fontsize{10.000000}{12.000000}\selectfont (0.2, 0.3]}%
\end{pgfscope}%
\begin{pgfscope}%
\pgfsetbuttcap%
\pgfsetroundjoin%
\definecolor{currentfill}{rgb}{0.000000,0.000000,0.000000}%
\pgfsetfillcolor{currentfill}%
\pgfsetlinewidth{0.803000pt}%
\definecolor{currentstroke}{rgb}{0.000000,0.000000,0.000000}%
\pgfsetstrokecolor{currentstroke}%
\pgfsetdash{}{0pt}%
\pgfsys@defobject{currentmarker}{\pgfqpoint{0.000000in}{-0.048611in}}{\pgfqpoint{0.000000in}{0.000000in}}{%
\pgfpathmoveto{\pgfqpoint{0.000000in}{0.000000in}}%
\pgfpathlineto{\pgfqpoint{0.000000in}{-0.048611in}}%
\pgfusepath{stroke,fill}%
}%
\begin{pgfscope}%
\pgfsys@transformshift{1.329142in}{1.096944in}%
\pgfsys@useobject{currentmarker}{}%
\end{pgfscope}%
\end{pgfscope}%
\begin{pgfscope}%
\definecolor{textcolor}{rgb}{0.000000,0.000000,0.000000}%
\pgfsetstrokecolor{textcolor}%
\pgfsetfillcolor{textcolor}%
\pgftext[x=1.363864in, y=0.467222in, left, base,rotate=90.000000]{\color{textcolor}\rmfamily\fontsize{10.000000}{12.000000}\selectfont (0.3, 0.4]}%
\end{pgfscope}%
\begin{pgfscope}%
\pgfsetbuttcap%
\pgfsetroundjoin%
\definecolor{currentfill}{rgb}{0.000000,0.000000,0.000000}%
\pgfsetfillcolor{currentfill}%
\pgfsetlinewidth{0.803000pt}%
\definecolor{currentstroke}{rgb}{0.000000,0.000000,0.000000}%
\pgfsetstrokecolor{currentstroke}%
\pgfsetdash{}{0pt}%
\pgfsys@defobject{currentmarker}{\pgfqpoint{0.000000in}{-0.048611in}}{\pgfqpoint{0.000000in}{0.000000in}}{%
\pgfpathmoveto{\pgfqpoint{0.000000in}{0.000000in}}%
\pgfpathlineto{\pgfqpoint{0.000000in}{-0.048611in}}%
\pgfusepath{stroke,fill}%
}%
\begin{pgfscope}%
\pgfsys@transformshift{1.579647in}{1.096944in}%
\pgfsys@useobject{currentmarker}{}%
\end{pgfscope}%
\end{pgfscope}%
\begin{pgfscope}%
\definecolor{textcolor}{rgb}{0.000000,0.000000,0.000000}%
\pgfsetstrokecolor{textcolor}%
\pgfsetfillcolor{textcolor}%
\pgftext[x=1.614369in, y=0.467222in, left, base,rotate=90.000000]{\color{textcolor}\rmfamily\fontsize{10.000000}{12.000000}\selectfont (0.4, 0.5]}%
\end{pgfscope}%
\begin{pgfscope}%
\pgfsetbuttcap%
\pgfsetroundjoin%
\definecolor{currentfill}{rgb}{0.000000,0.000000,0.000000}%
\pgfsetfillcolor{currentfill}%
\pgfsetlinewidth{0.803000pt}%
\definecolor{currentstroke}{rgb}{0.000000,0.000000,0.000000}%
\pgfsetstrokecolor{currentstroke}%
\pgfsetdash{}{0pt}%
\pgfsys@defobject{currentmarker}{\pgfqpoint{0.000000in}{-0.048611in}}{\pgfqpoint{0.000000in}{0.000000in}}{%
\pgfpathmoveto{\pgfqpoint{0.000000in}{0.000000in}}%
\pgfpathlineto{\pgfqpoint{0.000000in}{-0.048611in}}%
\pgfusepath{stroke,fill}%
}%
\begin{pgfscope}%
\pgfsys@transformshift{1.830152in}{1.096944in}%
\pgfsys@useobject{currentmarker}{}%
\end{pgfscope}%
\end{pgfscope}%
\begin{pgfscope}%
\definecolor{textcolor}{rgb}{0.000000,0.000000,0.000000}%
\pgfsetstrokecolor{textcolor}%
\pgfsetfillcolor{textcolor}%
\pgftext[x=1.864874in, y=0.467222in, left, base,rotate=90.000000]{\color{textcolor}\rmfamily\fontsize{10.000000}{12.000000}\selectfont (0.5, 0.6]}%
\end{pgfscope}%
\begin{pgfscope}%
\pgfsetbuttcap%
\pgfsetroundjoin%
\definecolor{currentfill}{rgb}{0.000000,0.000000,0.000000}%
\pgfsetfillcolor{currentfill}%
\pgfsetlinewidth{0.803000pt}%
\definecolor{currentstroke}{rgb}{0.000000,0.000000,0.000000}%
\pgfsetstrokecolor{currentstroke}%
\pgfsetdash{}{0pt}%
\pgfsys@defobject{currentmarker}{\pgfqpoint{0.000000in}{-0.048611in}}{\pgfqpoint{0.000000in}{0.000000in}}{%
\pgfpathmoveto{\pgfqpoint{0.000000in}{0.000000in}}%
\pgfpathlineto{\pgfqpoint{0.000000in}{-0.048611in}}%
\pgfusepath{stroke,fill}%
}%
\begin{pgfscope}%
\pgfsys@transformshift{2.080657in}{1.096944in}%
\pgfsys@useobject{currentmarker}{}%
\end{pgfscope}%
\end{pgfscope}%
\begin{pgfscope}%
\definecolor{textcolor}{rgb}{0.000000,0.000000,0.000000}%
\pgfsetstrokecolor{textcolor}%
\pgfsetfillcolor{textcolor}%
\pgftext[x=2.115379in, y=0.467222in, left, base,rotate=90.000000]{\color{textcolor}\rmfamily\fontsize{10.000000}{12.000000}\selectfont (0.6, 0.7]}%
\end{pgfscope}%
\begin{pgfscope}%
\pgfsetbuttcap%
\pgfsetroundjoin%
\definecolor{currentfill}{rgb}{0.000000,0.000000,0.000000}%
\pgfsetfillcolor{currentfill}%
\pgfsetlinewidth{0.803000pt}%
\definecolor{currentstroke}{rgb}{0.000000,0.000000,0.000000}%
\pgfsetstrokecolor{currentstroke}%
\pgfsetdash{}{0pt}%
\pgfsys@defobject{currentmarker}{\pgfqpoint{0.000000in}{-0.048611in}}{\pgfqpoint{0.000000in}{0.000000in}}{%
\pgfpathmoveto{\pgfqpoint{0.000000in}{0.000000in}}%
\pgfpathlineto{\pgfqpoint{0.000000in}{-0.048611in}}%
\pgfusepath{stroke,fill}%
}%
\begin{pgfscope}%
\pgfsys@transformshift{2.331162in}{1.096944in}%
\pgfsys@useobject{currentmarker}{}%
\end{pgfscope}%
\end{pgfscope}%
\begin{pgfscope}%
\definecolor{textcolor}{rgb}{0.000000,0.000000,0.000000}%
\pgfsetstrokecolor{textcolor}%
\pgfsetfillcolor{textcolor}%
\pgftext[x=2.365884in, y=0.467222in, left, base,rotate=90.000000]{\color{textcolor}\rmfamily\fontsize{10.000000}{12.000000}\selectfont (0.7, 0.8]}%
\end{pgfscope}%
\begin{pgfscope}%
\pgfsetbuttcap%
\pgfsetroundjoin%
\definecolor{currentfill}{rgb}{0.000000,0.000000,0.000000}%
\pgfsetfillcolor{currentfill}%
\pgfsetlinewidth{0.803000pt}%
\definecolor{currentstroke}{rgb}{0.000000,0.000000,0.000000}%
\pgfsetstrokecolor{currentstroke}%
\pgfsetdash{}{0pt}%
\pgfsys@defobject{currentmarker}{\pgfqpoint{0.000000in}{-0.048611in}}{\pgfqpoint{0.000000in}{0.000000in}}{%
\pgfpathmoveto{\pgfqpoint{0.000000in}{0.000000in}}%
\pgfpathlineto{\pgfqpoint{0.000000in}{-0.048611in}}%
\pgfusepath{stroke,fill}%
}%
\begin{pgfscope}%
\pgfsys@transformshift{2.581667in}{1.096944in}%
\pgfsys@useobject{currentmarker}{}%
\end{pgfscope}%
\end{pgfscope}%
\begin{pgfscope}%
\definecolor{textcolor}{rgb}{0.000000,0.000000,0.000000}%
\pgfsetstrokecolor{textcolor}%
\pgfsetfillcolor{textcolor}%
\pgftext[x=2.616389in, y=0.467222in, left, base,rotate=90.000000]{\color{textcolor}\rmfamily\fontsize{10.000000}{12.000000}\selectfont (0.8, 0.9]}%
\end{pgfscope}%
\begin{pgfscope}%
\pgfsetbuttcap%
\pgfsetroundjoin%
\definecolor{currentfill}{rgb}{0.000000,0.000000,0.000000}%
\pgfsetfillcolor{currentfill}%
\pgfsetlinewidth{0.803000pt}%
\definecolor{currentstroke}{rgb}{0.000000,0.000000,0.000000}%
\pgfsetstrokecolor{currentstroke}%
\pgfsetdash{}{0pt}%
\pgfsys@defobject{currentmarker}{\pgfqpoint{0.000000in}{-0.048611in}}{\pgfqpoint{0.000000in}{0.000000in}}{%
\pgfpathmoveto{\pgfqpoint{0.000000in}{0.000000in}}%
\pgfpathlineto{\pgfqpoint{0.000000in}{-0.048611in}}%
\pgfusepath{stroke,fill}%
}%
\begin{pgfscope}%
\pgfsys@transformshift{2.832172in}{1.096944in}%
\pgfsys@useobject{currentmarker}{}%
\end{pgfscope}%
\end{pgfscope}%
\begin{pgfscope}%
\definecolor{textcolor}{rgb}{0.000000,0.000000,0.000000}%
\pgfsetstrokecolor{textcolor}%
\pgfsetfillcolor{textcolor}%
\pgftext[x=2.866894in, y=0.467222in, left, base,rotate=90.000000]{\color{textcolor}\rmfamily\fontsize{10.000000}{12.000000}\selectfont (0.9, 1.0]}%
\end{pgfscope}%
\begin{pgfscope}%
\definecolor{textcolor}{rgb}{0.000000,0.000000,0.000000}%
\pgfsetstrokecolor{textcolor}%
\pgfsetfillcolor{textcolor}%
\pgftext[x=1.755000in,y=0.226527in,,top]{\color{textcolor}\rmfamily\fontsize{10.000000}{12.000000}\selectfont Range of Prediction}%
\end{pgfscope}%
\begin{pgfscope}%
\pgfsetbuttcap%
\pgfsetroundjoin%
\definecolor{currentfill}{rgb}{0.000000,0.000000,0.000000}%
\pgfsetfillcolor{currentfill}%
\pgfsetlinewidth{0.803000pt}%
\definecolor{currentstroke}{rgb}{0.000000,0.000000,0.000000}%
\pgfsetstrokecolor{currentstroke}%
\pgfsetdash{}{0pt}%
\pgfsys@defobject{currentmarker}{\pgfqpoint{-0.048611in}{0.000000in}}{\pgfqpoint{-0.000000in}{0.000000in}}{%
\pgfpathmoveto{\pgfqpoint{-0.000000in}{0.000000in}}%
\pgfpathlineto{\pgfqpoint{-0.048611in}{0.000000in}}%
\pgfusepath{stroke,fill}%
}%
\begin{pgfscope}%
\pgfsys@transformshift{0.515000in}{1.096944in}%
\pgfsys@useobject{currentmarker}{}%
\end{pgfscope}%
\end{pgfscope}%
\begin{pgfscope}%
\definecolor{textcolor}{rgb}{0.000000,0.000000,0.000000}%
\pgfsetstrokecolor{textcolor}%
\pgfsetfillcolor{textcolor}%
\pgftext[x=0.348333in, y=1.048750in, left, base]{\color{textcolor}\rmfamily\fontsize{10.000000}{12.000000}\selectfont \(\displaystyle {0}\)}%
\end{pgfscope}%
\begin{pgfscope}%
\pgfsetbuttcap%
\pgfsetroundjoin%
\definecolor{currentfill}{rgb}{0.000000,0.000000,0.000000}%
\pgfsetfillcolor{currentfill}%
\pgfsetlinewidth{0.803000pt}%
\definecolor{currentstroke}{rgb}{0.000000,0.000000,0.000000}%
\pgfsetstrokecolor{currentstroke}%
\pgfsetdash{}{0pt}%
\pgfsys@defobject{currentmarker}{\pgfqpoint{-0.048611in}{0.000000in}}{\pgfqpoint{-0.000000in}{0.000000in}}{%
\pgfpathmoveto{\pgfqpoint{-0.000000in}{0.000000in}}%
\pgfpathlineto{\pgfqpoint{-0.048611in}{0.000000in}}%
\pgfusepath{stroke,fill}%
}%
\begin{pgfscope}%
\pgfsys@transformshift{0.515000in}{1.531405in}%
\pgfsys@useobject{currentmarker}{}%
\end{pgfscope}%
\end{pgfscope}%
\begin{pgfscope}%
\definecolor{textcolor}{rgb}{0.000000,0.000000,0.000000}%
\pgfsetstrokecolor{textcolor}%
\pgfsetfillcolor{textcolor}%
\pgftext[x=0.348333in, y=1.483210in, left, base]{\color{textcolor}\rmfamily\fontsize{10.000000}{12.000000}\selectfont \(\displaystyle {5}\)}%
\end{pgfscope}%
\begin{pgfscope}%
\pgfsetbuttcap%
\pgfsetroundjoin%
\definecolor{currentfill}{rgb}{0.000000,0.000000,0.000000}%
\pgfsetfillcolor{currentfill}%
\pgfsetlinewidth{0.803000pt}%
\definecolor{currentstroke}{rgb}{0.000000,0.000000,0.000000}%
\pgfsetstrokecolor{currentstroke}%
\pgfsetdash{}{0pt}%
\pgfsys@defobject{currentmarker}{\pgfqpoint{-0.048611in}{0.000000in}}{\pgfqpoint{-0.000000in}{0.000000in}}{%
\pgfpathmoveto{\pgfqpoint{-0.000000in}{0.000000in}}%
\pgfpathlineto{\pgfqpoint{-0.048611in}{0.000000in}}%
\pgfusepath{stroke,fill}%
}%
\begin{pgfscope}%
\pgfsys@transformshift{0.515000in}{1.965865in}%
\pgfsys@useobject{currentmarker}{}%
\end{pgfscope}%
\end{pgfscope}%
\begin{pgfscope}%
\definecolor{textcolor}{rgb}{0.000000,0.000000,0.000000}%
\pgfsetstrokecolor{textcolor}%
\pgfsetfillcolor{textcolor}%
\pgftext[x=0.278889in, y=1.917671in, left, base]{\color{textcolor}\rmfamily\fontsize{10.000000}{12.000000}\selectfont \(\displaystyle {10}\)}%
\end{pgfscope}%
\begin{pgfscope}%
\pgfsetbuttcap%
\pgfsetroundjoin%
\definecolor{currentfill}{rgb}{0.000000,0.000000,0.000000}%
\pgfsetfillcolor{currentfill}%
\pgfsetlinewidth{0.803000pt}%
\definecolor{currentstroke}{rgb}{0.000000,0.000000,0.000000}%
\pgfsetstrokecolor{currentstroke}%
\pgfsetdash{}{0pt}%
\pgfsys@defobject{currentmarker}{\pgfqpoint{-0.048611in}{0.000000in}}{\pgfqpoint{-0.000000in}{0.000000in}}{%
\pgfpathmoveto{\pgfqpoint{-0.000000in}{0.000000in}}%
\pgfpathlineto{\pgfqpoint{-0.048611in}{0.000000in}}%
\pgfusepath{stroke,fill}%
}%
\begin{pgfscope}%
\pgfsys@transformshift{0.515000in}{2.400326in}%
\pgfsys@useobject{currentmarker}{}%
\end{pgfscope}%
\end{pgfscope}%
\begin{pgfscope}%
\definecolor{textcolor}{rgb}{0.000000,0.000000,0.000000}%
\pgfsetstrokecolor{textcolor}%
\pgfsetfillcolor{textcolor}%
\pgftext[x=0.278889in, y=2.352132in, left, base]{\color{textcolor}\rmfamily\fontsize{10.000000}{12.000000}\selectfont \(\displaystyle {15}\)}%
\end{pgfscope}%
\begin{pgfscope}%
\pgfsetbuttcap%
\pgfsetroundjoin%
\definecolor{currentfill}{rgb}{0.000000,0.000000,0.000000}%
\pgfsetfillcolor{currentfill}%
\pgfsetlinewidth{0.803000pt}%
\definecolor{currentstroke}{rgb}{0.000000,0.000000,0.000000}%
\pgfsetstrokecolor{currentstroke}%
\pgfsetdash{}{0pt}%
\pgfsys@defobject{currentmarker}{\pgfqpoint{-0.048611in}{0.000000in}}{\pgfqpoint{-0.000000in}{0.000000in}}{%
\pgfpathmoveto{\pgfqpoint{-0.000000in}{0.000000in}}%
\pgfpathlineto{\pgfqpoint{-0.048611in}{0.000000in}}%
\pgfusepath{stroke,fill}%
}%
\begin{pgfscope}%
\pgfsys@transformshift{0.515000in}{2.834787in}%
\pgfsys@useobject{currentmarker}{}%
\end{pgfscope}%
\end{pgfscope}%
\begin{pgfscope}%
\definecolor{textcolor}{rgb}{0.000000,0.000000,0.000000}%
\pgfsetstrokecolor{textcolor}%
\pgfsetfillcolor{textcolor}%
\pgftext[x=0.278889in, y=2.786592in, left, base]{\color{textcolor}\rmfamily\fontsize{10.000000}{12.000000}\selectfont \(\displaystyle {20}\)}%
\end{pgfscope}%
\begin{pgfscope}%
\definecolor{textcolor}{rgb}{0.000000,0.000000,0.000000}%
\pgfsetstrokecolor{textcolor}%
\pgfsetfillcolor{textcolor}%
\pgftext[x=0.223333in,y=2.020944in,,bottom,rotate=90.000000]{\color{textcolor}\rmfamily\fontsize{10.000000}{12.000000}\selectfont Percent of Data Set}%
\end{pgfscope}%
\begin{pgfscope}%
\pgfsetrectcap%
\pgfsetmiterjoin%
\pgfsetlinewidth{0.803000pt}%
\definecolor{currentstroke}{rgb}{0.000000,0.000000,0.000000}%
\pgfsetstrokecolor{currentstroke}%
\pgfsetdash{}{0pt}%
\pgfpathmoveto{\pgfqpoint{0.515000in}{1.096944in}}%
\pgfpathlineto{\pgfqpoint{0.515000in}{2.944944in}}%
\pgfusepath{stroke}%
\end{pgfscope}%
\begin{pgfscope}%
\pgfsetrectcap%
\pgfsetmiterjoin%
\pgfsetlinewidth{0.803000pt}%
\definecolor{currentstroke}{rgb}{0.000000,0.000000,0.000000}%
\pgfsetstrokecolor{currentstroke}%
\pgfsetdash{}{0pt}%
\pgfpathmoveto{\pgfqpoint{2.995000in}{1.096944in}}%
\pgfpathlineto{\pgfqpoint{2.995000in}{2.944944in}}%
\pgfusepath{stroke}%
\end{pgfscope}%
\begin{pgfscope}%
\pgfsetrectcap%
\pgfsetmiterjoin%
\pgfsetlinewidth{0.803000pt}%
\definecolor{currentstroke}{rgb}{0.000000,0.000000,0.000000}%
\pgfsetstrokecolor{currentstroke}%
\pgfsetdash{}{0pt}%
\pgfpathmoveto{\pgfqpoint{0.515000in}{1.096944in}}%
\pgfpathlineto{\pgfqpoint{2.995000in}{1.096944in}}%
\pgfusepath{stroke}%
\end{pgfscope}%
\begin{pgfscope}%
\pgfsetrectcap%
\pgfsetmiterjoin%
\pgfsetlinewidth{0.803000pt}%
\definecolor{currentstroke}{rgb}{0.000000,0.000000,0.000000}%
\pgfsetstrokecolor{currentstroke}%
\pgfsetdash{}{0pt}%
\pgfpathmoveto{\pgfqpoint{0.515000in}{2.944944in}}%
\pgfpathlineto{\pgfqpoint{2.995000in}{2.944944in}}%
\pgfusepath{stroke}%
\end{pgfscope}%
\begin{pgfscope}%
\pgfsetbuttcap%
\pgfsetmiterjoin%
\definecolor{currentfill}{rgb}{1.000000,1.000000,1.000000}%
\pgfsetfillcolor{currentfill}%
\pgfsetfillopacity{0.800000}%
\pgfsetlinewidth{1.003750pt}%
\definecolor{currentstroke}{rgb}{0.800000,0.800000,0.800000}%
\pgfsetstrokecolor{currentstroke}%
\pgfsetstrokeopacity{0.800000}%
\pgfsetdash{}{0pt}%
\pgfpathmoveto{\pgfqpoint{1.560833in}{2.444250in}}%
\pgfpathlineto{\pgfqpoint{2.897778in}{2.444250in}}%
\pgfpathquadraticcurveto{\pgfqpoint{2.925556in}{2.444250in}}{\pgfqpoint{2.925556in}{2.472028in}}%
\pgfpathlineto{\pgfqpoint{2.925556in}{2.847722in}}%
\pgfpathquadraticcurveto{\pgfqpoint{2.925556in}{2.875500in}}{\pgfqpoint{2.897778in}{2.875500in}}%
\pgfpathlineto{\pgfqpoint{1.560833in}{2.875500in}}%
\pgfpathquadraticcurveto{\pgfqpoint{1.533056in}{2.875500in}}{\pgfqpoint{1.533056in}{2.847722in}}%
\pgfpathlineto{\pgfqpoint{1.533056in}{2.472028in}}%
\pgfpathquadraticcurveto{\pgfqpoint{1.533056in}{2.444250in}}{\pgfqpoint{1.560833in}{2.444250in}}%
\pgfpathlineto{\pgfqpoint{1.560833in}{2.444250in}}%
\pgfpathclose%
\pgfusepath{stroke,fill}%
\end{pgfscope}%
\begin{pgfscope}%
\pgfsetbuttcap%
\pgfsetmiterjoin%
\pgfsetlinewidth{1.003750pt}%
\definecolor{currentstroke}{rgb}{0.000000,0.000000,0.000000}%
\pgfsetstrokecolor{currentstroke}%
\pgfsetdash{}{0pt}%
\pgfpathmoveto{\pgfqpoint{1.588611in}{2.722027in}}%
\pgfpathlineto{\pgfqpoint{1.866389in}{2.722027in}}%
\pgfpathlineto{\pgfqpoint{1.866389in}{2.819250in}}%
\pgfpathlineto{\pgfqpoint{1.588611in}{2.819250in}}%
\pgfpathlineto{\pgfqpoint{1.588611in}{2.722027in}}%
\pgfpathclose%
\pgfusepath{stroke}%
\end{pgfscope}%
\begin{pgfscope}%
\definecolor{textcolor}{rgb}{0.000000,0.000000,0.000000}%
\pgfsetstrokecolor{textcolor}%
\pgfsetfillcolor{textcolor}%
\pgftext[x=1.977500in,y=2.722027in,left,base]{\color{textcolor}\rmfamily\fontsize{10.000000}{12.000000}\selectfont Negative Class}%
\end{pgfscope}%
\begin{pgfscope}%
\pgfsetbuttcap%
\pgfsetmiterjoin%
\definecolor{currentfill}{rgb}{0.000000,0.000000,0.000000}%
\pgfsetfillcolor{currentfill}%
\pgfsetlinewidth{0.000000pt}%
\definecolor{currentstroke}{rgb}{0.000000,0.000000,0.000000}%
\pgfsetstrokecolor{currentstroke}%
\pgfsetstrokeopacity{0.000000}%
\pgfsetdash{}{0pt}%
\pgfpathmoveto{\pgfqpoint{1.588611in}{2.526750in}}%
\pgfpathlineto{\pgfqpoint{1.866389in}{2.526750in}}%
\pgfpathlineto{\pgfqpoint{1.866389in}{2.623972in}}%
\pgfpathlineto{\pgfqpoint{1.588611in}{2.623972in}}%
\pgfpathlineto{\pgfqpoint{1.588611in}{2.526750in}}%
\pgfpathclose%
\pgfusepath{fill}%
\end{pgfscope}%
\begin{pgfscope}%
\definecolor{textcolor}{rgb}{0.000000,0.000000,0.000000}%
\pgfsetstrokecolor{textcolor}%
\pgfsetfillcolor{textcolor}%
\pgftext[x=1.977500in,y=2.526750in,left,base]{\color{textcolor}\rmfamily\fontsize{10.000000}{12.000000}\selectfont Positive Class}%
\end{pgfscope}%
\end{pgfpicture}%
\makeatother%
\endgroup%

  &
  \vspace{0pt} %% Creator: Matplotlib, PGF backend
%%
%% To include the figure in your LaTeX document, write
%%   \input{<filename>.pgf}
%%
%% Make sure the required packages are loaded in your preamble
%%   \usepackage{pgf}
%%
%% Also ensure that all the required font packages are loaded; for instance,
%% the lmodern package is sometimes necessary when using math font.
%%   \usepackage{lmodern}
%%
%% Figures using additional raster images can only be included by \input if
%% they are in the same directory as the main LaTeX file. For loading figures
%% from other directories you can use the `import` package
%%   \usepackage{import}
%%
%% and then include the figures with
%%   \import{<path to file>}{<filename>.pgf}
%%
%% Matplotlib used the following preamble
%%   
%%   \usepackage{fontspec}
%%   \makeatletter\@ifpackageloaded{underscore}{}{\usepackage[strings]{underscore}}\makeatother
%%
\begingroup%
\makeatletter%
\begin{pgfpicture}%
\pgfpathrectangle{\pgfpointorigin}{\pgfqpoint{2.221861in}{1.754444in}}%
\pgfusepath{use as bounding box, clip}%
\begin{pgfscope}%
\pgfsetbuttcap%
\pgfsetmiterjoin%
\definecolor{currentfill}{rgb}{1.000000,1.000000,1.000000}%
\pgfsetfillcolor{currentfill}%
\pgfsetlinewidth{0.000000pt}%
\definecolor{currentstroke}{rgb}{1.000000,1.000000,1.000000}%
\pgfsetstrokecolor{currentstroke}%
\pgfsetdash{}{0pt}%
\pgfpathmoveto{\pgfqpoint{0.000000in}{0.000000in}}%
\pgfpathlineto{\pgfqpoint{2.221861in}{0.000000in}}%
\pgfpathlineto{\pgfqpoint{2.221861in}{1.754444in}}%
\pgfpathlineto{\pgfqpoint{0.000000in}{1.754444in}}%
\pgfpathlineto{\pgfqpoint{0.000000in}{0.000000in}}%
\pgfpathclose%
\pgfusepath{fill}%
\end{pgfscope}%
\begin{pgfscope}%
\pgfsetbuttcap%
\pgfsetmiterjoin%
\definecolor{currentfill}{rgb}{1.000000,1.000000,1.000000}%
\pgfsetfillcolor{currentfill}%
\pgfsetlinewidth{0.000000pt}%
\definecolor{currentstroke}{rgb}{0.000000,0.000000,0.000000}%
\pgfsetstrokecolor{currentstroke}%
\pgfsetstrokeopacity{0.000000}%
\pgfsetdash{}{0pt}%
\pgfpathmoveto{\pgfqpoint{0.553581in}{0.499444in}}%
\pgfpathlineto{\pgfqpoint{2.103581in}{0.499444in}}%
\pgfpathlineto{\pgfqpoint{2.103581in}{1.654444in}}%
\pgfpathlineto{\pgfqpoint{0.553581in}{1.654444in}}%
\pgfpathlineto{\pgfqpoint{0.553581in}{0.499444in}}%
\pgfpathclose%
\pgfusepath{fill}%
\end{pgfscope}%
\begin{pgfscope}%
\pgfsetbuttcap%
\pgfsetroundjoin%
\definecolor{currentfill}{rgb}{0.000000,0.000000,0.000000}%
\pgfsetfillcolor{currentfill}%
\pgfsetlinewidth{0.803000pt}%
\definecolor{currentstroke}{rgb}{0.000000,0.000000,0.000000}%
\pgfsetstrokecolor{currentstroke}%
\pgfsetdash{}{0pt}%
\pgfsys@defobject{currentmarker}{\pgfqpoint{0.000000in}{-0.048611in}}{\pgfqpoint{0.000000in}{0.000000in}}{%
\pgfpathmoveto{\pgfqpoint{0.000000in}{0.000000in}}%
\pgfpathlineto{\pgfqpoint{0.000000in}{-0.048611in}}%
\pgfusepath{stroke,fill}%
}%
\begin{pgfscope}%
\pgfsys@transformshift{0.624035in}{0.499444in}%
\pgfsys@useobject{currentmarker}{}%
\end{pgfscope}%
\end{pgfscope}%
\begin{pgfscope}%
\definecolor{textcolor}{rgb}{0.000000,0.000000,0.000000}%
\pgfsetstrokecolor{textcolor}%
\pgfsetfillcolor{textcolor}%
\pgftext[x=0.624035in,y=0.402222in,,top]{\color{textcolor}\rmfamily\fontsize{10.000000}{12.000000}\selectfont \(\displaystyle {0.0}\)}%
\end{pgfscope}%
\begin{pgfscope}%
\pgfsetbuttcap%
\pgfsetroundjoin%
\definecolor{currentfill}{rgb}{0.000000,0.000000,0.000000}%
\pgfsetfillcolor{currentfill}%
\pgfsetlinewidth{0.803000pt}%
\definecolor{currentstroke}{rgb}{0.000000,0.000000,0.000000}%
\pgfsetstrokecolor{currentstroke}%
\pgfsetdash{}{0pt}%
\pgfsys@defobject{currentmarker}{\pgfqpoint{0.000000in}{-0.048611in}}{\pgfqpoint{0.000000in}{0.000000in}}{%
\pgfpathmoveto{\pgfqpoint{0.000000in}{0.000000in}}%
\pgfpathlineto{\pgfqpoint{0.000000in}{-0.048611in}}%
\pgfusepath{stroke,fill}%
}%
\begin{pgfscope}%
\pgfsys@transformshift{1.328581in}{0.499444in}%
\pgfsys@useobject{currentmarker}{}%
\end{pgfscope}%
\end{pgfscope}%
\begin{pgfscope}%
\definecolor{textcolor}{rgb}{0.000000,0.000000,0.000000}%
\pgfsetstrokecolor{textcolor}%
\pgfsetfillcolor{textcolor}%
\pgftext[x=1.328581in,y=0.402222in,,top]{\color{textcolor}\rmfamily\fontsize{10.000000}{12.000000}\selectfont \(\displaystyle {0.5}\)}%
\end{pgfscope}%
\begin{pgfscope}%
\pgfsetbuttcap%
\pgfsetroundjoin%
\definecolor{currentfill}{rgb}{0.000000,0.000000,0.000000}%
\pgfsetfillcolor{currentfill}%
\pgfsetlinewidth{0.803000pt}%
\definecolor{currentstroke}{rgb}{0.000000,0.000000,0.000000}%
\pgfsetstrokecolor{currentstroke}%
\pgfsetdash{}{0pt}%
\pgfsys@defobject{currentmarker}{\pgfqpoint{0.000000in}{-0.048611in}}{\pgfqpoint{0.000000in}{0.000000in}}{%
\pgfpathmoveto{\pgfqpoint{0.000000in}{0.000000in}}%
\pgfpathlineto{\pgfqpoint{0.000000in}{-0.048611in}}%
\pgfusepath{stroke,fill}%
}%
\begin{pgfscope}%
\pgfsys@transformshift{2.033126in}{0.499444in}%
\pgfsys@useobject{currentmarker}{}%
\end{pgfscope}%
\end{pgfscope}%
\begin{pgfscope}%
\definecolor{textcolor}{rgb}{0.000000,0.000000,0.000000}%
\pgfsetstrokecolor{textcolor}%
\pgfsetfillcolor{textcolor}%
\pgftext[x=2.033126in,y=0.402222in,,top]{\color{textcolor}\rmfamily\fontsize{10.000000}{12.000000}\selectfont \(\displaystyle {1.0}\)}%
\end{pgfscope}%
\begin{pgfscope}%
\definecolor{textcolor}{rgb}{0.000000,0.000000,0.000000}%
\pgfsetstrokecolor{textcolor}%
\pgfsetfillcolor{textcolor}%
\pgftext[x=1.328581in,y=0.223333in,,top]{\color{textcolor}\rmfamily\fontsize{10.000000}{12.000000}\selectfont False positive rate}%
\end{pgfscope}%
\begin{pgfscope}%
\pgfsetbuttcap%
\pgfsetroundjoin%
\definecolor{currentfill}{rgb}{0.000000,0.000000,0.000000}%
\pgfsetfillcolor{currentfill}%
\pgfsetlinewidth{0.803000pt}%
\definecolor{currentstroke}{rgb}{0.000000,0.000000,0.000000}%
\pgfsetstrokecolor{currentstroke}%
\pgfsetdash{}{0pt}%
\pgfsys@defobject{currentmarker}{\pgfqpoint{-0.048611in}{0.000000in}}{\pgfqpoint{-0.000000in}{0.000000in}}{%
\pgfpathmoveto{\pgfqpoint{-0.000000in}{0.000000in}}%
\pgfpathlineto{\pgfqpoint{-0.048611in}{0.000000in}}%
\pgfusepath{stroke,fill}%
}%
\begin{pgfscope}%
\pgfsys@transformshift{0.553581in}{0.551944in}%
\pgfsys@useobject{currentmarker}{}%
\end{pgfscope}%
\end{pgfscope}%
\begin{pgfscope}%
\definecolor{textcolor}{rgb}{0.000000,0.000000,0.000000}%
\pgfsetstrokecolor{textcolor}%
\pgfsetfillcolor{textcolor}%
\pgftext[x=0.278889in, y=0.503750in, left, base]{\color{textcolor}\rmfamily\fontsize{10.000000}{12.000000}\selectfont \(\displaystyle {0.0}\)}%
\end{pgfscope}%
\begin{pgfscope}%
\pgfsetbuttcap%
\pgfsetroundjoin%
\definecolor{currentfill}{rgb}{0.000000,0.000000,0.000000}%
\pgfsetfillcolor{currentfill}%
\pgfsetlinewidth{0.803000pt}%
\definecolor{currentstroke}{rgb}{0.000000,0.000000,0.000000}%
\pgfsetstrokecolor{currentstroke}%
\pgfsetdash{}{0pt}%
\pgfsys@defobject{currentmarker}{\pgfqpoint{-0.048611in}{0.000000in}}{\pgfqpoint{-0.000000in}{0.000000in}}{%
\pgfpathmoveto{\pgfqpoint{-0.000000in}{0.000000in}}%
\pgfpathlineto{\pgfqpoint{-0.048611in}{0.000000in}}%
\pgfusepath{stroke,fill}%
}%
\begin{pgfscope}%
\pgfsys@transformshift{0.553581in}{1.076944in}%
\pgfsys@useobject{currentmarker}{}%
\end{pgfscope}%
\end{pgfscope}%
\begin{pgfscope}%
\definecolor{textcolor}{rgb}{0.000000,0.000000,0.000000}%
\pgfsetstrokecolor{textcolor}%
\pgfsetfillcolor{textcolor}%
\pgftext[x=0.278889in, y=1.028750in, left, base]{\color{textcolor}\rmfamily\fontsize{10.000000}{12.000000}\selectfont \(\displaystyle {0.5}\)}%
\end{pgfscope}%
\begin{pgfscope}%
\pgfsetbuttcap%
\pgfsetroundjoin%
\definecolor{currentfill}{rgb}{0.000000,0.000000,0.000000}%
\pgfsetfillcolor{currentfill}%
\pgfsetlinewidth{0.803000pt}%
\definecolor{currentstroke}{rgb}{0.000000,0.000000,0.000000}%
\pgfsetstrokecolor{currentstroke}%
\pgfsetdash{}{0pt}%
\pgfsys@defobject{currentmarker}{\pgfqpoint{-0.048611in}{0.000000in}}{\pgfqpoint{-0.000000in}{0.000000in}}{%
\pgfpathmoveto{\pgfqpoint{-0.000000in}{0.000000in}}%
\pgfpathlineto{\pgfqpoint{-0.048611in}{0.000000in}}%
\pgfusepath{stroke,fill}%
}%
\begin{pgfscope}%
\pgfsys@transformshift{0.553581in}{1.601944in}%
\pgfsys@useobject{currentmarker}{}%
\end{pgfscope}%
\end{pgfscope}%
\begin{pgfscope}%
\definecolor{textcolor}{rgb}{0.000000,0.000000,0.000000}%
\pgfsetstrokecolor{textcolor}%
\pgfsetfillcolor{textcolor}%
\pgftext[x=0.278889in, y=1.553750in, left, base]{\color{textcolor}\rmfamily\fontsize{10.000000}{12.000000}\selectfont \(\displaystyle {1.0}\)}%
\end{pgfscope}%
\begin{pgfscope}%
\definecolor{textcolor}{rgb}{0.000000,0.000000,0.000000}%
\pgfsetstrokecolor{textcolor}%
\pgfsetfillcolor{textcolor}%
\pgftext[x=0.223333in,y=1.076944in,,bottom,rotate=90.000000]{\color{textcolor}\rmfamily\fontsize{10.000000}{12.000000}\selectfont True positive rate}%
\end{pgfscope}%
\begin{pgfscope}%
\pgfpathrectangle{\pgfqpoint{0.553581in}{0.499444in}}{\pgfqpoint{1.550000in}{1.155000in}}%
\pgfusepath{clip}%
\pgfsetbuttcap%
\pgfsetroundjoin%
\pgfsetlinewidth{1.505625pt}%
\definecolor{currentstroke}{rgb}{0.000000,0.000000,0.000000}%
\pgfsetstrokecolor{currentstroke}%
\pgfsetdash{{5.550000pt}{2.400000pt}}{0.000000pt}%
\pgfpathmoveto{\pgfqpoint{0.624035in}{0.551944in}}%
\pgfpathlineto{\pgfqpoint{2.033126in}{1.601944in}}%
\pgfusepath{stroke}%
\end{pgfscope}%
\begin{pgfscope}%
\pgfpathrectangle{\pgfqpoint{0.553581in}{0.499444in}}{\pgfqpoint{1.550000in}{1.155000in}}%
\pgfusepath{clip}%
\pgfsetrectcap%
\pgfsetroundjoin%
\pgfsetlinewidth{1.505625pt}%
\definecolor{currentstroke}{rgb}{0.000000,0.000000,0.000000}%
\pgfsetstrokecolor{currentstroke}%
\pgfsetdash{}{0pt}%
\pgfpathmoveto{\pgfqpoint{0.624035in}{0.551944in}}%
\pgfpathlineto{\pgfqpoint{0.626119in}{0.552260in}}%
\pgfpathlineto{\pgfqpoint{0.627222in}{0.560898in}}%
\pgfpathlineto{\pgfqpoint{0.627858in}{0.562002in}}%
\pgfpathlineto{\pgfqpoint{0.628904in}{0.564132in}}%
\pgfpathlineto{\pgfqpoint{0.629465in}{0.565236in}}%
\pgfpathlineto{\pgfqpoint{0.630521in}{0.567918in}}%
\pgfpathlineto{\pgfqpoint{0.631138in}{0.569023in}}%
\pgfpathlineto{\pgfqpoint{0.632241in}{0.573085in}}%
\pgfpathlineto{\pgfqpoint{0.632652in}{0.574190in}}%
\pgfpathlineto{\pgfqpoint{0.633755in}{0.580264in}}%
\pgfpathlineto{\pgfqpoint{0.634166in}{0.581171in}}%
\pgfpathlineto{\pgfqpoint{0.635269in}{0.586378in}}%
\pgfpathlineto{\pgfqpoint{0.635447in}{0.587482in}}%
\pgfpathlineto{\pgfqpoint{0.636531in}{0.593359in}}%
\pgfpathlineto{\pgfqpoint{0.636662in}{0.594384in}}%
\pgfpathlineto{\pgfqpoint{0.637764in}{0.600222in}}%
\pgfpathlineto{\pgfqpoint{0.637979in}{0.601287in}}%
\pgfpathlineto{\pgfqpoint{0.639082in}{0.607992in}}%
\pgfpathlineto{\pgfqpoint{0.639278in}{0.608978in}}%
\pgfpathlineto{\pgfqpoint{0.640372in}{0.614027in}}%
\pgfpathlineto{\pgfqpoint{0.640587in}{0.615131in}}%
\pgfpathlineto{\pgfqpoint{0.641690in}{0.622783in}}%
\pgfpathlineto{\pgfqpoint{0.641867in}{0.623887in}}%
\pgfpathlineto{\pgfqpoint{0.642970in}{0.631381in}}%
\pgfpathlineto{\pgfqpoint{0.643092in}{0.632446in}}%
\pgfpathlineto{\pgfqpoint{0.644194in}{0.641794in}}%
\pgfpathlineto{\pgfqpoint{0.644316in}{0.642741in}}%
\pgfpathlineto{\pgfqpoint{0.645419in}{0.649841in}}%
\pgfpathlineto{\pgfqpoint{0.645559in}{0.650866in}}%
\pgfpathlineto{\pgfqpoint{0.646662in}{0.658557in}}%
\pgfpathlineto{\pgfqpoint{0.646886in}{0.659662in}}%
\pgfpathlineto{\pgfqpoint{0.647979in}{0.665736in}}%
\pgfpathlineto{\pgfqpoint{0.648232in}{0.666840in}}%
\pgfpathlineto{\pgfqpoint{0.649335in}{0.674216in}}%
\pgfpathlineto{\pgfqpoint{0.649465in}{0.675281in}}%
\pgfpathlineto{\pgfqpoint{0.650568in}{0.683091in}}%
\pgfpathlineto{\pgfqpoint{0.650736in}{0.684116in}}%
\pgfpathlineto{\pgfqpoint{0.651830in}{0.692754in}}%
\pgfpathlineto{\pgfqpoint{0.652008in}{0.693819in}}%
\pgfpathlineto{\pgfqpoint{0.653073in}{0.701471in}}%
\pgfpathlineto{\pgfqpoint{0.653260in}{0.702496in}}%
\pgfpathlineto{\pgfqpoint{0.654353in}{0.710267in}}%
\pgfpathlineto{\pgfqpoint{0.654578in}{0.711331in}}%
\pgfpathlineto{\pgfqpoint{0.655671in}{0.720877in}}%
\pgfpathlineto{\pgfqpoint{0.655793in}{0.721863in}}%
\pgfpathlineto{\pgfqpoint{0.656895in}{0.730422in}}%
\pgfpathlineto{\pgfqpoint{0.657166in}{0.731526in}}%
\pgfpathlineto{\pgfqpoint{0.658269in}{0.740677in}}%
\pgfpathlineto{\pgfqpoint{0.658372in}{0.741663in}}%
\pgfpathlineto{\pgfqpoint{0.659475in}{0.749472in}}%
\pgfpathlineto{\pgfqpoint{0.659615in}{0.750577in}}%
\pgfpathlineto{\pgfqpoint{0.660671in}{0.759136in}}%
\pgfpathlineto{\pgfqpoint{0.660914in}{0.760082in}}%
\pgfpathlineto{\pgfqpoint{0.662017in}{0.767419in}}%
\pgfpathlineto{\pgfqpoint{0.662185in}{0.768365in}}%
\pgfpathlineto{\pgfqpoint{0.663279in}{0.776806in}}%
\pgfpathlineto{\pgfqpoint{0.663466in}{0.777832in}}%
\pgfpathlineto{\pgfqpoint{0.664559in}{0.785365in}}%
\pgfpathlineto{\pgfqpoint{0.664886in}{0.786430in}}%
\pgfpathlineto{\pgfqpoint{0.665989in}{0.793530in}}%
\pgfpathlineto{\pgfqpoint{0.666204in}{0.794634in}}%
\pgfpathlineto{\pgfqpoint{0.667307in}{0.802720in}}%
\pgfpathlineto{\pgfqpoint{0.667522in}{0.803785in}}%
\pgfpathlineto{\pgfqpoint{0.668625in}{0.809110in}}%
\pgfpathlineto{\pgfqpoint{0.668802in}{0.810175in}}%
\pgfpathlineto{\pgfqpoint{0.669905in}{0.815618in}}%
\pgfpathlineto{\pgfqpoint{0.670120in}{0.816683in}}%
\pgfpathlineto{\pgfqpoint{0.671223in}{0.824847in}}%
\pgfpathlineto{\pgfqpoint{0.671475in}{0.825873in}}%
\pgfpathlineto{\pgfqpoint{0.672559in}{0.834274in}}%
\pgfpathlineto{\pgfqpoint{0.672727in}{0.835378in}}%
\pgfpathlineto{\pgfqpoint{0.673830in}{0.842439in}}%
\pgfpathlineto{\pgfqpoint{0.674036in}{0.843425in}}%
\pgfpathlineto{\pgfqpoint{0.675129in}{0.849735in}}%
\pgfpathlineto{\pgfqpoint{0.675419in}{0.850840in}}%
\pgfpathlineto{\pgfqpoint{0.676503in}{0.856520in}}%
\pgfpathlineto{\pgfqpoint{0.676681in}{0.857584in}}%
\pgfpathlineto{\pgfqpoint{0.677784in}{0.863225in}}%
\pgfpathlineto{\pgfqpoint{0.678017in}{0.864329in}}%
\pgfpathlineto{\pgfqpoint{0.679120in}{0.871508in}}%
\pgfpathlineto{\pgfqpoint{0.679372in}{0.872612in}}%
\pgfpathlineto{\pgfqpoint{0.680475in}{0.878450in}}%
\pgfpathlineto{\pgfqpoint{0.680681in}{0.879554in}}%
\pgfpathlineto{\pgfqpoint{0.681784in}{0.884839in}}%
\pgfpathlineto{\pgfqpoint{0.681914in}{0.885865in}}%
\pgfpathlineto{\pgfqpoint{0.683017in}{0.892531in}}%
\pgfpathlineto{\pgfqpoint{0.683270in}{0.893595in}}%
\pgfpathlineto{\pgfqpoint{0.684344in}{0.898171in}}%
\pgfpathlineto{\pgfqpoint{0.684615in}{0.899157in}}%
\pgfpathlineto{\pgfqpoint{0.685718in}{0.905113in}}%
\pgfpathlineto{\pgfqpoint{0.686036in}{0.906217in}}%
\pgfpathlineto{\pgfqpoint{0.687129in}{0.911660in}}%
\pgfpathlineto{\pgfqpoint{0.687428in}{0.912765in}}%
\pgfpathlineto{\pgfqpoint{0.688531in}{0.917419in}}%
\pgfpathlineto{\pgfqpoint{0.688718in}{0.918523in}}%
\pgfpathlineto{\pgfqpoint{0.689821in}{0.923572in}}%
\pgfpathlineto{\pgfqpoint{0.690017in}{0.924676in}}%
\pgfpathlineto{\pgfqpoint{0.691120in}{0.930632in}}%
\pgfpathlineto{\pgfqpoint{0.691391in}{0.931736in}}%
\pgfpathlineto{\pgfqpoint{0.692494in}{0.937692in}}%
\pgfpathlineto{\pgfqpoint{0.692709in}{0.938797in}}%
\pgfpathlineto{\pgfqpoint{0.693812in}{0.943924in}}%
\pgfpathlineto{\pgfqpoint{0.694045in}{0.944910in}}%
\pgfpathlineto{\pgfqpoint{0.695148in}{0.949288in}}%
\pgfpathlineto{\pgfqpoint{0.695401in}{0.950393in}}%
\pgfpathlineto{\pgfqpoint{0.696494in}{0.955441in}}%
\pgfpathlineto{\pgfqpoint{0.696849in}{0.956467in}}%
\pgfpathlineto{\pgfqpoint{0.697924in}{0.961516in}}%
\pgfpathlineto{\pgfqpoint{0.698260in}{0.962581in}}%
\pgfpathlineto{\pgfqpoint{0.699345in}{0.967314in}}%
\pgfpathlineto{\pgfqpoint{0.699653in}{0.968379in}}%
\pgfpathlineto{\pgfqpoint{0.700746in}{0.972204in}}%
\pgfpathlineto{\pgfqpoint{0.701008in}{0.973112in}}%
\pgfpathlineto{\pgfqpoint{0.702111in}{0.978791in}}%
\pgfpathlineto{\pgfqpoint{0.702410in}{0.979896in}}%
\pgfpathlineto{\pgfqpoint{0.703503in}{0.984944in}}%
\pgfpathlineto{\pgfqpoint{0.703831in}{0.986049in}}%
\pgfpathlineto{\pgfqpoint{0.704924in}{0.990664in}}%
\pgfpathlineto{\pgfqpoint{0.705223in}{0.991610in}}%
\pgfpathlineto{\pgfqpoint{0.706326in}{0.996541in}}%
\pgfpathlineto{\pgfqpoint{0.706690in}{0.997527in}}%
\pgfpathlineto{\pgfqpoint{0.707765in}{1.002970in}}%
\pgfpathlineto{\pgfqpoint{0.708036in}{1.004074in}}%
\pgfpathlineto{\pgfqpoint{0.709130in}{1.008373in}}%
\pgfpathlineto{\pgfqpoint{0.709391in}{1.009478in}}%
\pgfpathlineto{\pgfqpoint{0.710485in}{1.012988in}}%
\pgfpathlineto{\pgfqpoint{0.710868in}{1.014092in}}%
\pgfpathlineto{\pgfqpoint{0.711961in}{1.018747in}}%
\pgfpathlineto{\pgfqpoint{0.712242in}{1.019851in}}%
\pgfpathlineto{\pgfqpoint{0.713289in}{1.023677in}}%
\pgfpathlineto{\pgfqpoint{0.713793in}{1.024742in}}%
\pgfpathlineto{\pgfqpoint{0.714877in}{1.028568in}}%
\pgfpathlineto{\pgfqpoint{0.715186in}{1.029593in}}%
\pgfpathlineto{\pgfqpoint{0.716251in}{1.033577in}}%
\pgfpathlineto{\pgfqpoint{0.716522in}{1.034681in}}%
\pgfpathlineto{\pgfqpoint{0.717606in}{1.039493in}}%
\pgfpathlineto{\pgfqpoint{0.717924in}{1.040480in}}%
\pgfpathlineto{\pgfqpoint{0.719018in}{1.043990in}}%
\pgfpathlineto{\pgfqpoint{0.719373in}{1.045055in}}%
\pgfpathlineto{\pgfqpoint{0.720457in}{1.048881in}}%
\pgfpathlineto{\pgfqpoint{0.720803in}{1.049946in}}%
\pgfpathlineto{\pgfqpoint{0.721887in}{1.053141in}}%
\pgfpathlineto{\pgfqpoint{0.722139in}{1.054245in}}%
\pgfpathlineto{\pgfqpoint{0.723233in}{1.058268in}}%
\pgfpathlineto{\pgfqpoint{0.723635in}{1.059373in}}%
\pgfpathlineto{\pgfqpoint{0.724709in}{1.063001in}}%
\pgfpathlineto{\pgfqpoint{0.725177in}{1.064106in}}%
\pgfpathlineto{\pgfqpoint{0.726186in}{1.067892in}}%
\pgfpathlineto{\pgfqpoint{0.726541in}{1.068996in}}%
\pgfpathlineto{\pgfqpoint{0.727644in}{1.072231in}}%
\pgfpathlineto{\pgfqpoint{0.728036in}{1.073335in}}%
\pgfpathlineto{\pgfqpoint{0.729139in}{1.075899in}}%
\pgfpathlineto{\pgfqpoint{0.729532in}{1.076964in}}%
\pgfpathlineto{\pgfqpoint{0.730607in}{1.080632in}}%
\pgfpathlineto{\pgfqpoint{0.731027in}{1.081618in}}%
\pgfpathlineto{\pgfqpoint{0.732111in}{1.084655in}}%
\pgfpathlineto{\pgfqpoint{0.732364in}{1.085681in}}%
\pgfpathlineto{\pgfqpoint{0.733429in}{1.089270in}}%
\pgfpathlineto{\pgfqpoint{0.733850in}{1.090374in}}%
\pgfpathlineto{\pgfqpoint{0.734924in}{1.093372in}}%
\pgfpathlineto{\pgfqpoint{0.735326in}{1.094476in}}%
\pgfpathlineto{\pgfqpoint{0.736429in}{1.097435in}}%
\pgfpathlineto{\pgfqpoint{0.736934in}{1.098500in}}%
\pgfpathlineto{\pgfqpoint{0.738009in}{1.101615in}}%
\pgfpathlineto{\pgfqpoint{0.738513in}{1.102720in}}%
\pgfpathlineto{\pgfqpoint{0.739569in}{1.105836in}}%
\pgfpathlineto{\pgfqpoint{0.739990in}{1.106901in}}%
\pgfpathlineto{\pgfqpoint{0.741093in}{1.109938in}}%
\pgfpathlineto{\pgfqpoint{0.741766in}{1.111003in}}%
\pgfpathlineto{\pgfqpoint{0.742831in}{1.114158in}}%
\pgfpathlineto{\pgfqpoint{0.743336in}{1.115263in}}%
\pgfpathlineto{\pgfqpoint{0.744382in}{1.118181in}}%
\pgfpathlineto{\pgfqpoint{0.755205in}{1.119286in}}%
\pgfpathlineto{\pgfqpoint{0.756298in}{1.121771in}}%
\pgfpathlineto{\pgfqpoint{0.756775in}{1.122875in}}%
\pgfpathlineto{\pgfqpoint{0.757878in}{1.125912in}}%
\pgfpathlineto{\pgfqpoint{0.758373in}{1.127016in}}%
\pgfpathlineto{\pgfqpoint{0.759457in}{1.129699in}}%
\pgfpathlineto{\pgfqpoint{0.759766in}{1.130803in}}%
\pgfpathlineto{\pgfqpoint{0.760841in}{1.134116in}}%
\pgfpathlineto{\pgfqpoint{0.761289in}{1.135063in}}%
\pgfpathlineto{\pgfqpoint{0.762392in}{1.137508in}}%
\pgfpathlineto{\pgfqpoint{0.762756in}{1.138613in}}%
\pgfpathlineto{\pgfqpoint{0.763859in}{1.141926in}}%
\pgfpathlineto{\pgfqpoint{0.764299in}{1.143030in}}%
\pgfpathlineto{\pgfqpoint{0.765401in}{1.145160in}}%
\pgfpathlineto{\pgfqpoint{0.765990in}{1.146264in}}%
\pgfpathlineto{\pgfqpoint{0.767046in}{1.149144in}}%
\pgfpathlineto{\pgfqpoint{0.767514in}{1.150209in}}%
\pgfpathlineto{\pgfqpoint{0.768616in}{1.152694in}}%
\pgfpathlineto{\pgfqpoint{0.769168in}{1.153798in}}%
\pgfpathlineto{\pgfqpoint{0.770243in}{1.156914in}}%
\pgfpathlineto{\pgfqpoint{0.770887in}{1.158018in}}%
\pgfpathlineto{\pgfqpoint{0.771972in}{1.160622in}}%
\pgfpathlineto{\pgfqpoint{0.772785in}{1.161686in}}%
\pgfpathlineto{\pgfqpoint{0.773869in}{1.163777in}}%
\pgfpathlineto{\pgfqpoint{0.774327in}{1.164881in}}%
\pgfpathlineto{\pgfqpoint{0.775401in}{1.167918in}}%
\pgfpathlineto{\pgfqpoint{0.775869in}{1.169023in}}%
\pgfpathlineto{\pgfqpoint{0.776906in}{1.171823in}}%
\pgfpathlineto{\pgfqpoint{0.777504in}{1.172928in}}%
\pgfpathlineto{\pgfqpoint{0.778579in}{1.175767in}}%
\pgfpathlineto{\pgfqpoint{0.779037in}{1.176872in}}%
\pgfpathlineto{\pgfqpoint{0.780130in}{1.178489in}}%
\pgfpathlineto{\pgfqpoint{0.780663in}{1.179515in}}%
\pgfpathlineto{\pgfqpoint{0.781766in}{1.182275in}}%
\pgfpathlineto{\pgfqpoint{0.782243in}{1.183380in}}%
\pgfpathlineto{\pgfqpoint{0.783317in}{1.185707in}}%
\pgfpathlineto{\pgfqpoint{0.783747in}{1.186732in}}%
\pgfpathlineto{\pgfqpoint{0.784831in}{1.188902in}}%
\pgfpathlineto{\pgfqpoint{0.785626in}{1.190006in}}%
\pgfpathlineto{\pgfqpoint{0.786691in}{1.192097in}}%
\pgfpathlineto{\pgfqpoint{0.787289in}{1.193201in}}%
\pgfpathlineto{\pgfqpoint{0.788364in}{1.194897in}}%
\pgfpathlineto{\pgfqpoint{0.789196in}{1.196001in}}%
\pgfpathlineto{\pgfqpoint{0.790299in}{1.198053in}}%
\pgfpathlineto{\pgfqpoint{0.791046in}{1.199117in}}%
\pgfpathlineto{\pgfqpoint{0.792131in}{1.201326in}}%
\pgfpathlineto{\pgfqpoint{0.792645in}{1.202391in}}%
\pgfpathlineto{\pgfqpoint{0.793738in}{1.204048in}}%
\pgfpathlineto{\pgfqpoint{0.794561in}{1.205152in}}%
\pgfpathlineto{\pgfqpoint{0.795654in}{1.206967in}}%
\pgfpathlineto{\pgfqpoint{0.796476in}{1.208071in}}%
\pgfpathlineto{\pgfqpoint{0.797579in}{1.210201in}}%
\pgfpathlineto{\pgfqpoint{0.798065in}{1.211305in}}%
\pgfpathlineto{\pgfqpoint{0.799131in}{1.213080in}}%
\pgfpathlineto{\pgfqpoint{0.800000in}{1.214106in}}%
\pgfpathlineto{\pgfqpoint{0.801084in}{1.216275in}}%
\pgfpathlineto{\pgfqpoint{0.801570in}{1.217340in}}%
\pgfpathlineto{\pgfqpoint{0.802645in}{1.219115in}}%
\pgfpathlineto{\pgfqpoint{0.803346in}{1.220219in}}%
\pgfpathlineto{\pgfqpoint{0.804383in}{1.222112in}}%
\pgfpathlineto{\pgfqpoint{0.804953in}{1.223099in}}%
\pgfpathlineto{\pgfqpoint{0.806047in}{1.225978in}}%
\pgfpathlineto{\pgfqpoint{0.806551in}{1.227082in}}%
\pgfpathlineto{\pgfqpoint{0.807589in}{1.229094in}}%
\pgfpathlineto{\pgfqpoint{0.808327in}{1.230198in}}%
\pgfpathlineto{\pgfqpoint{0.809430in}{1.232486in}}%
\pgfpathlineto{\pgfqpoint{0.810318in}{1.233590in}}%
\pgfpathlineto{\pgfqpoint{0.811411in}{1.235286in}}%
\pgfpathlineto{\pgfqpoint{0.812103in}{1.236391in}}%
\pgfpathlineto{\pgfqpoint{0.813084in}{1.237850in}}%
\pgfpathlineto{\pgfqpoint{0.813935in}{1.238954in}}%
\pgfpathlineto{\pgfqpoint{0.815009in}{1.240808in}}%
\pgfpathlineto{\pgfqpoint{0.815888in}{1.241873in}}%
\pgfpathlineto{\pgfqpoint{0.816991in}{1.243766in}}%
\pgfpathlineto{\pgfqpoint{0.817542in}{1.244871in}}%
\pgfpathlineto{\pgfqpoint{0.818645in}{1.246646in}}%
\pgfpathlineto{\pgfqpoint{0.819383in}{1.247750in}}%
\pgfpathlineto{\pgfqpoint{0.820477in}{1.249288in}}%
\pgfpathlineto{\pgfqpoint{0.821355in}{1.250393in}}%
\pgfpathlineto{\pgfqpoint{0.822458in}{1.252957in}}%
\pgfpathlineto{\pgfqpoint{0.823252in}{1.254061in}}%
\pgfpathlineto{\pgfqpoint{0.824337in}{1.255599in}}%
\pgfpathlineto{\pgfqpoint{0.825168in}{1.256704in}}%
\pgfpathlineto{\pgfqpoint{0.826271in}{1.258123in}}%
\pgfpathlineto{\pgfqpoint{0.826897in}{1.259188in}}%
\pgfpathlineto{\pgfqpoint{0.827982in}{1.260845in}}%
\pgfpathlineto{\pgfqpoint{0.829150in}{1.261949in}}%
\pgfpathlineto{\pgfqpoint{0.830196in}{1.262935in}}%
\pgfpathlineto{\pgfqpoint{0.831028in}{1.263922in}}%
\pgfpathlineto{\pgfqpoint{0.832131in}{1.265854in}}%
\pgfpathlineto{\pgfqpoint{0.832776in}{1.266959in}}%
\pgfpathlineto{\pgfqpoint{0.833860in}{1.268418in}}%
\pgfpathlineto{\pgfqpoint{0.834776in}{1.269522in}}%
\pgfpathlineto{\pgfqpoint{0.835804in}{1.270903in}}%
\pgfpathlineto{\pgfqpoint{0.836748in}{1.271928in}}%
\pgfpathlineto{\pgfqpoint{0.837804in}{1.273269in}}%
\pgfpathlineto{\pgfqpoint{0.838617in}{1.274374in}}%
\pgfpathlineto{\pgfqpoint{0.839692in}{1.276030in}}%
\pgfpathlineto{\pgfqpoint{0.840692in}{1.277135in}}%
\pgfpathlineto{\pgfqpoint{0.841655in}{1.278239in}}%
\pgfpathlineto{\pgfqpoint{0.842570in}{1.279344in}}%
\pgfpathlineto{\pgfqpoint{0.843673in}{1.280566in}}%
\pgfpathlineto{\pgfqpoint{0.844589in}{1.281671in}}%
\pgfpathlineto{\pgfqpoint{0.845692in}{1.283288in}}%
\pgfpathlineto{\pgfqpoint{0.846785in}{1.284392in}}%
\pgfpathlineto{\pgfqpoint{0.847851in}{1.285378in}}%
\pgfpathlineto{\pgfqpoint{0.848795in}{1.286483in}}%
\pgfpathlineto{\pgfqpoint{0.849888in}{1.287390in}}%
\pgfpathlineto{\pgfqpoint{0.850757in}{1.288494in}}%
\pgfpathlineto{\pgfqpoint{0.851814in}{1.290111in}}%
\pgfpathlineto{\pgfqpoint{0.852860in}{1.291176in}}%
\pgfpathlineto{\pgfqpoint{0.853832in}{1.292754in}}%
\pgfpathlineto{\pgfqpoint{0.854458in}{1.293858in}}%
\pgfpathlineto{\pgfqpoint{0.855524in}{1.295239in}}%
\pgfpathlineto{\pgfqpoint{0.856739in}{1.296343in}}%
\pgfpathlineto{\pgfqpoint{0.857804in}{1.297763in}}%
\pgfpathlineto{\pgfqpoint{0.858730in}{1.298789in}}%
\pgfpathlineto{\pgfqpoint{0.859823in}{1.299972in}}%
\pgfpathlineto{\pgfqpoint{0.860402in}{1.301076in}}%
\pgfpathlineto{\pgfqpoint{0.861430in}{1.302417in}}%
\pgfpathlineto{\pgfqpoint{0.862169in}{1.303482in}}%
\pgfpathlineto{\pgfqpoint{0.863272in}{1.304468in}}%
\pgfpathlineto{\pgfqpoint{0.864169in}{1.305573in}}%
\pgfpathlineto{\pgfqpoint{0.865272in}{1.306993in}}%
\pgfpathlineto{\pgfqpoint{0.866337in}{1.308058in}}%
\pgfpathlineto{\pgfqpoint{0.867431in}{1.309833in}}%
\pgfpathlineto{\pgfqpoint{0.868459in}{1.310898in}}%
\pgfpathlineto{\pgfqpoint{0.869561in}{1.312436in}}%
\pgfpathlineto{\pgfqpoint{0.870487in}{1.313540in}}%
\pgfpathlineto{\pgfqpoint{0.871589in}{1.314763in}}%
\pgfpathlineto{\pgfqpoint{0.872571in}{1.315867in}}%
\pgfpathlineto{\pgfqpoint{0.873636in}{1.316696in}}%
\pgfpathlineto{\pgfqpoint{0.874328in}{1.317761in}}%
\pgfpathlineto{\pgfqpoint{0.875347in}{1.318983in}}%
\pgfpathlineto{\pgfqpoint{0.876814in}{1.320088in}}%
\pgfpathlineto{\pgfqpoint{0.877907in}{1.321113in}}%
\pgfpathlineto{\pgfqpoint{0.879104in}{1.322218in}}%
\pgfpathlineto{\pgfqpoint{0.880206in}{1.323361in}}%
\pgfpathlineto{\pgfqpoint{0.881281in}{1.324387in}}%
\pgfpathlineto{\pgfqpoint{0.882347in}{1.325570in}}%
\pgfpathlineto{\pgfqpoint{0.883664in}{1.326675in}}%
\pgfpathlineto{\pgfqpoint{0.884758in}{1.328252in}}%
\pgfpathlineto{\pgfqpoint{0.886440in}{1.329357in}}%
\pgfpathlineto{\pgfqpoint{0.887496in}{1.330382in}}%
\pgfpathlineto{\pgfqpoint{0.888805in}{1.331487in}}%
\pgfpathlineto{\pgfqpoint{0.889879in}{1.332788in}}%
\pgfpathlineto{\pgfqpoint{0.890898in}{1.333893in}}%
\pgfpathlineto{\pgfqpoint{0.891954in}{1.334681in}}%
\pgfpathlineto{\pgfqpoint{0.893337in}{1.335786in}}%
\pgfpathlineto{\pgfqpoint{0.894384in}{1.337009in}}%
\pgfpathlineto{\pgfqpoint{0.895515in}{1.338074in}}%
\pgfpathlineto{\pgfqpoint{0.896562in}{1.339178in}}%
\pgfpathlineto{\pgfqpoint{0.897562in}{1.340282in}}%
\pgfpathlineto{\pgfqpoint{0.898665in}{1.341189in}}%
\pgfpathlineto{\pgfqpoint{0.899945in}{1.342294in}}%
\pgfpathlineto{\pgfqpoint{0.900992in}{1.343162in}}%
\pgfpathlineto{\pgfqpoint{0.902225in}{1.344266in}}%
\pgfpathlineto{\pgfqpoint{0.903328in}{1.345568in}}%
\pgfpathlineto{\pgfqpoint{0.904319in}{1.346672in}}%
\pgfpathlineto{\pgfqpoint{0.905272in}{1.347776in}}%
\pgfpathlineto{\pgfqpoint{0.906375in}{1.348881in}}%
\pgfpathlineto{\pgfqpoint{0.907403in}{1.349946in}}%
\pgfpathlineto{\pgfqpoint{0.909300in}{1.351050in}}%
\pgfpathlineto{\pgfqpoint{0.910347in}{1.351839in}}%
\pgfpathlineto{\pgfqpoint{0.911637in}{1.352943in}}%
\pgfpathlineto{\pgfqpoint{0.912674in}{1.353969in}}%
\pgfpathlineto{\pgfqpoint{0.914160in}{1.355073in}}%
\pgfpathlineto{\pgfqpoint{0.915235in}{1.355862in}}%
\pgfpathlineto{\pgfqpoint{0.916730in}{1.356967in}}%
\pgfpathlineto{\pgfqpoint{0.917646in}{1.357558in}}%
\pgfpathlineto{\pgfqpoint{0.919235in}{1.358623in}}%
\pgfpathlineto{\pgfqpoint{0.920216in}{1.359885in}}%
\pgfpathlineto{\pgfqpoint{0.921226in}{1.360950in}}%
\pgfpathlineto{\pgfqpoint{0.922244in}{1.361976in}}%
\pgfpathlineto{\pgfqpoint{0.924048in}{1.363080in}}%
\pgfpathlineto{\pgfqpoint{0.925085in}{1.363751in}}%
\pgfpathlineto{\pgfqpoint{0.926366in}{1.364855in}}%
\pgfpathlineto{\pgfqpoint{0.927431in}{1.365407in}}%
\pgfpathlineto{\pgfqpoint{0.929095in}{1.366512in}}%
\pgfpathlineto{\pgfqpoint{0.930114in}{1.367300in}}%
\pgfpathlineto{\pgfqpoint{0.931095in}{1.368405in}}%
\pgfpathlineto{\pgfqpoint{0.932179in}{1.369351in}}%
\pgfpathlineto{\pgfqpoint{0.933151in}{1.370456in}}%
\pgfpathlineto{\pgfqpoint{0.934160in}{1.371205in}}%
\pgfpathlineto{\pgfqpoint{0.936646in}{1.372310in}}%
\pgfpathlineto{\pgfqpoint{0.937749in}{1.373020in}}%
\pgfpathlineto{\pgfqpoint{0.939497in}{1.374124in}}%
\pgfpathlineto{\pgfqpoint{0.940431in}{1.374992in}}%
\pgfpathlineto{\pgfqpoint{0.942104in}{1.376096in}}%
\pgfpathlineto{\pgfqpoint{0.943114in}{1.376924in}}%
\pgfpathlineto{\pgfqpoint{0.945030in}{1.377989in}}%
\pgfpathlineto{\pgfqpoint{0.946123in}{1.378936in}}%
\pgfpathlineto{\pgfqpoint{0.947880in}{1.380040in}}%
\pgfpathlineto{\pgfqpoint{0.948964in}{1.381145in}}%
\pgfpathlineto{\pgfqpoint{0.950693in}{1.382249in}}%
\pgfpathlineto{\pgfqpoint{0.951740in}{1.383275in}}%
\pgfpathlineto{\pgfqpoint{0.953469in}{1.384379in}}%
\pgfpathlineto{\pgfqpoint{0.954562in}{1.385207in}}%
\pgfpathlineto{\pgfqpoint{0.956347in}{1.386312in}}%
\pgfpathlineto{\pgfqpoint{0.957404in}{1.387022in}}%
\pgfpathlineto{\pgfqpoint{0.958712in}{1.388126in}}%
\pgfpathlineto{\pgfqpoint{0.959805in}{1.388678in}}%
\pgfpathlineto{\pgfqpoint{0.961890in}{1.389783in}}%
\pgfpathlineto{\pgfqpoint{0.962964in}{1.390453in}}%
\pgfpathlineto{\pgfqpoint{0.964450in}{1.391518in}}%
\pgfpathlineto{\pgfqpoint{0.965357in}{1.391952in}}%
\pgfpathlineto{\pgfqpoint{0.967450in}{1.393056in}}%
\pgfpathlineto{\pgfqpoint{0.968535in}{1.393806in}}%
\pgfpathlineto{\pgfqpoint{0.970264in}{1.394910in}}%
\pgfpathlineto{\pgfqpoint{0.971348in}{1.395699in}}%
\pgfpathlineto{\pgfqpoint{0.972964in}{1.396764in}}%
\pgfpathlineto{\pgfqpoint{0.974011in}{1.397711in}}%
\pgfpathlineto{\pgfqpoint{0.976413in}{1.398815in}}%
\pgfpathlineto{\pgfqpoint{0.977441in}{1.399564in}}%
\pgfpathlineto{\pgfqpoint{0.979544in}{1.400669in}}%
\pgfpathlineto{\pgfqpoint{0.980628in}{1.401339in}}%
\pgfpathlineto{\pgfqpoint{0.982264in}{1.402444in}}%
\pgfpathlineto{\pgfqpoint{0.983320in}{1.403075in}}%
\pgfpathlineto{\pgfqpoint{0.985254in}{1.404179in}}%
\pgfpathlineto{\pgfqpoint{0.986357in}{1.404731in}}%
\pgfpathlineto{\pgfqpoint{0.988282in}{1.405836in}}%
\pgfpathlineto{\pgfqpoint{0.989385in}{1.406546in}}%
\pgfpathlineto{\pgfqpoint{0.992666in}{1.407650in}}%
\pgfpathlineto{\pgfqpoint{0.993759in}{1.408676in}}%
\pgfpathlineto{\pgfqpoint{0.996703in}{1.409741in}}%
\pgfpathlineto{\pgfqpoint{0.997647in}{1.410096in}}%
\pgfpathlineto{\pgfqpoint{1.000152in}{1.411161in}}%
\pgfpathlineto{\pgfqpoint{1.001255in}{1.411397in}}%
\pgfpathlineto{\pgfqpoint{1.003451in}{1.412502in}}%
\pgfpathlineto{\pgfqpoint{1.004554in}{1.413251in}}%
\pgfpathlineto{\pgfqpoint{1.007245in}{1.414355in}}%
\pgfpathlineto{\pgfqpoint{1.008283in}{1.415065in}}%
\pgfpathlineto{\pgfqpoint{1.010227in}{1.416170in}}%
\pgfpathlineto{\pgfqpoint{1.011217in}{1.416919in}}%
\pgfpathlineto{\pgfqpoint{1.013600in}{1.418024in}}%
\pgfpathlineto{\pgfqpoint{1.014638in}{1.418694in}}%
\pgfpathlineto{\pgfqpoint{1.017703in}{1.419798in}}%
\pgfpathlineto{\pgfqpoint{1.018713in}{1.420430in}}%
\pgfpathlineto{\pgfqpoint{1.021395in}{1.421495in}}%
\pgfpathlineto{\pgfqpoint{1.022367in}{1.422559in}}%
\pgfpathlineto{\pgfqpoint{1.024255in}{1.423664in}}%
\pgfpathlineto{\pgfqpoint{1.025180in}{1.423979in}}%
\pgfpathlineto{\pgfqpoint{1.027479in}{1.425084in}}%
\pgfpathlineto{\pgfqpoint{1.028554in}{1.425557in}}%
\pgfpathlineto{\pgfqpoint{1.030292in}{1.426661in}}%
\pgfpathlineto{\pgfqpoint{1.031274in}{1.427174in}}%
\pgfpathlineto{\pgfqpoint{1.033489in}{1.428279in}}%
\pgfpathlineto{\pgfqpoint{1.034573in}{1.428713in}}%
\pgfpathlineto{\pgfqpoint{1.036750in}{1.429817in}}%
\pgfpathlineto{\pgfqpoint{1.037704in}{1.430330in}}%
\pgfpathlineto{\pgfqpoint{1.040133in}{1.431434in}}%
\pgfpathlineto{\pgfqpoint{1.041218in}{1.431947in}}%
\pgfpathlineto{\pgfqpoint{1.043274in}{1.433051in}}%
\pgfpathlineto{\pgfqpoint{1.044246in}{1.433801in}}%
\pgfpathlineto{\pgfqpoint{1.047274in}{1.434905in}}%
\pgfpathlineto{\pgfqpoint{1.048040in}{1.435260in}}%
\pgfpathlineto{\pgfqpoint{1.051040in}{1.436364in}}%
\pgfpathlineto{\pgfqpoint{1.052115in}{1.436838in}}%
\pgfpathlineto{\pgfqpoint{1.054582in}{1.437942in}}%
\pgfpathlineto{\pgfqpoint{1.055648in}{1.438613in}}%
\pgfpathlineto{\pgfqpoint{1.058386in}{1.439717in}}%
\pgfpathlineto{\pgfqpoint{1.059302in}{1.440072in}}%
\pgfpathlineto{\pgfqpoint{1.061891in}{1.441176in}}%
\pgfpathlineto{\pgfqpoint{1.062975in}{1.441452in}}%
\pgfpathlineto{\pgfqpoint{1.065180in}{1.442557in}}%
\pgfpathlineto{\pgfqpoint{1.065825in}{1.442754in}}%
\pgfpathlineto{\pgfqpoint{1.069302in}{1.443858in}}%
\pgfpathlineto{\pgfqpoint{1.070358in}{1.444371in}}%
\pgfpathlineto{\pgfqpoint{1.073723in}{1.445436in}}%
\pgfpathlineto{\pgfqpoint{1.074760in}{1.445949in}}%
\pgfpathlineto{\pgfqpoint{1.078068in}{1.447053in}}%
\pgfpathlineto{\pgfqpoint{1.079087in}{1.447408in}}%
\pgfpathlineto{\pgfqpoint{1.082293in}{1.448513in}}%
\pgfpathlineto{\pgfqpoint{1.083377in}{1.448947in}}%
\pgfpathlineto{\pgfqpoint{1.085779in}{1.450051in}}%
\pgfpathlineto{\pgfqpoint{1.086835in}{1.450248in}}%
\pgfpathlineto{\pgfqpoint{1.089928in}{1.451313in}}%
\pgfpathlineto{\pgfqpoint{1.090798in}{1.451865in}}%
\pgfpathlineto{\pgfqpoint{1.093826in}{1.452970in}}%
\pgfpathlineto{\pgfqpoint{1.094872in}{1.453246in}}%
\pgfpathlineto{\pgfqpoint{1.097900in}{1.454350in}}%
\pgfpathlineto{\pgfqpoint{1.098938in}{1.454784in}}%
\pgfpathlineto{\pgfqpoint{1.101564in}{1.455888in}}%
\pgfpathlineto{\pgfqpoint{1.102630in}{1.456165in}}%
\pgfpathlineto{\pgfqpoint{1.105592in}{1.457269in}}%
\pgfpathlineto{\pgfqpoint{1.106686in}{1.457900in}}%
\pgfpathlineto{\pgfqpoint{1.109041in}{1.459004in}}%
\pgfpathlineto{\pgfqpoint{1.110144in}{1.459754in}}%
\pgfpathlineto{\pgfqpoint{1.112620in}{1.460858in}}%
\pgfpathlineto{\pgfqpoint{1.113714in}{1.461292in}}%
\pgfpathlineto{\pgfqpoint{1.116714in}{1.462396in}}%
\pgfpathlineto{\pgfqpoint{1.117536in}{1.462673in}}%
\pgfpathlineto{\pgfqpoint{1.122190in}{1.463777in}}%
\pgfpathlineto{\pgfqpoint{1.123153in}{1.464290in}}%
\pgfpathlineto{\pgfqpoint{1.127443in}{1.465394in}}%
\pgfpathlineto{\pgfqpoint{1.128508in}{1.465946in}}%
\pgfpathlineto{\pgfqpoint{1.132331in}{1.467051in}}%
\pgfpathlineto{\pgfqpoint{1.133303in}{1.467603in}}%
\pgfpathlineto{\pgfqpoint{1.135864in}{1.468707in}}%
\pgfpathlineto{\pgfqpoint{1.136807in}{1.468944in}}%
\pgfpathlineto{\pgfqpoint{1.139452in}{1.470048in}}%
\pgfpathlineto{\pgfqpoint{1.140443in}{1.470443in}}%
\pgfpathlineto{\pgfqpoint{1.143368in}{1.471547in}}%
\pgfpathlineto{\pgfqpoint{1.144116in}{1.471981in}}%
\pgfpathlineto{\pgfqpoint{1.148462in}{1.473085in}}%
\pgfpathlineto{\pgfqpoint{1.148957in}{1.473283in}}%
\pgfpathlineto{\pgfqpoint{1.153602in}{1.474387in}}%
\pgfpathlineto{\pgfqpoint{1.154462in}{1.474781in}}%
\pgfpathlineto{\pgfqpoint{1.158312in}{1.475886in}}%
\pgfpathlineto{\pgfqpoint{1.159406in}{1.476044in}}%
\pgfpathlineto{\pgfqpoint{1.163322in}{1.477109in}}%
\pgfpathlineto{\pgfqpoint{1.164322in}{1.477542in}}%
\pgfpathlineto{\pgfqpoint{1.167387in}{1.478647in}}%
\pgfpathlineto{\pgfqpoint{1.168490in}{1.479041in}}%
\pgfpathlineto{\pgfqpoint{1.171584in}{1.480146in}}%
\pgfpathlineto{\pgfqpoint{1.172640in}{1.480619in}}%
\pgfpathlineto{\pgfqpoint{1.176098in}{1.481723in}}%
\pgfpathlineto{\pgfqpoint{1.177116in}{1.482157in}}%
\pgfpathlineto{\pgfqpoint{1.180855in}{1.483262in}}%
\pgfpathlineto{\pgfqpoint{1.181864in}{1.483893in}}%
\pgfpathlineto{\pgfqpoint{1.185406in}{1.484997in}}%
\pgfpathlineto{\pgfqpoint{1.186462in}{1.485313in}}%
\pgfpathlineto{\pgfqpoint{1.191331in}{1.486417in}}%
\pgfpathlineto{\pgfqpoint{1.192201in}{1.486614in}}%
\pgfpathlineto{\pgfqpoint{1.195995in}{1.487719in}}%
\pgfpathlineto{\pgfqpoint{1.196911in}{1.488034in}}%
\pgfpathlineto{\pgfqpoint{1.200958in}{1.489138in}}%
\pgfpathlineto{\pgfqpoint{1.201986in}{1.489454in}}%
\pgfpathlineto{\pgfqpoint{1.205696in}{1.490558in}}%
\pgfpathlineto{\pgfqpoint{1.206734in}{1.490992in}}%
\pgfpathlineto{\pgfqpoint{1.212603in}{1.492097in}}%
\pgfpathlineto{\pgfqpoint{1.213706in}{1.492333in}}%
\pgfpathlineto{\pgfqpoint{1.218182in}{1.493438in}}%
\pgfpathlineto{\pgfqpoint{1.219051in}{1.493872in}}%
\pgfpathlineto{\pgfqpoint{1.223724in}{1.494976in}}%
\pgfpathlineto{\pgfqpoint{1.224771in}{1.495252in}}%
\pgfpathlineto{\pgfqpoint{1.230238in}{1.496356in}}%
\pgfpathlineto{\pgfqpoint{1.231341in}{1.496711in}}%
\pgfpathlineto{\pgfqpoint{1.236248in}{1.497816in}}%
\pgfpathlineto{\pgfqpoint{1.237257in}{1.498131in}}%
\pgfpathlineto{\pgfqpoint{1.243388in}{1.499236in}}%
\pgfpathlineto{\pgfqpoint{1.243930in}{1.499433in}}%
\pgfpathlineto{\pgfqpoint{1.249575in}{1.500537in}}%
\pgfpathlineto{\pgfqpoint{1.250304in}{1.500774in}}%
\pgfpathlineto{\pgfqpoint{1.255613in}{1.501878in}}%
\pgfpathlineto{\pgfqpoint{1.256491in}{1.502312in}}%
\pgfpathlineto{\pgfqpoint{1.261108in}{1.503417in}}%
\pgfpathlineto{\pgfqpoint{1.262080in}{1.503614in}}%
\pgfpathlineto{\pgfqpoint{1.267089in}{1.504718in}}%
\pgfpathlineto{\pgfqpoint{1.268052in}{1.504876in}}%
\pgfpathlineto{\pgfqpoint{1.274145in}{1.505980in}}%
\pgfpathlineto{\pgfqpoint{1.274912in}{1.506178in}}%
\pgfpathlineto{\pgfqpoint{1.279286in}{1.507282in}}%
\pgfpathlineto{\pgfqpoint{1.280267in}{1.507913in}}%
\pgfpathlineto{\pgfqpoint{1.285155in}{1.509018in}}%
\pgfpathlineto{\pgfqpoint{1.285856in}{1.509333in}}%
\pgfpathlineto{\pgfqpoint{1.293463in}{1.510437in}}%
\pgfpathlineto{\pgfqpoint{1.294351in}{1.510674in}}%
\pgfpathlineto{\pgfqpoint{1.298950in}{1.511779in}}%
\pgfpathlineto{\pgfqpoint{1.299744in}{1.512212in}}%
\pgfpathlineto{\pgfqpoint{1.307221in}{1.513317in}}%
\pgfpathlineto{\pgfqpoint{1.308108in}{1.513632in}}%
\pgfpathlineto{\pgfqpoint{1.314015in}{1.514737in}}%
\pgfpathlineto{\pgfqpoint{1.315109in}{1.515092in}}%
\pgfpathlineto{\pgfqpoint{1.321744in}{1.516196in}}%
\pgfpathlineto{\pgfqpoint{1.322763in}{1.516472in}}%
\pgfpathlineto{\pgfqpoint{1.328632in}{1.517577in}}%
\pgfpathlineto{\pgfqpoint{1.329669in}{1.518010in}}%
\pgfpathlineto{\pgfqpoint{1.336539in}{1.519115in}}%
\pgfpathlineto{\pgfqpoint{1.337623in}{1.519312in}}%
\pgfpathlineto{\pgfqpoint{1.343193in}{1.520416in}}%
\pgfpathlineto{\pgfqpoint{1.343885in}{1.520574in}}%
\pgfpathlineto{\pgfqpoint{1.349866in}{1.521679in}}%
\pgfpathlineto{\pgfqpoint{1.350913in}{1.521876in}}%
\pgfpathlineto{\pgfqpoint{1.358118in}{1.522980in}}%
\pgfpathlineto{\pgfqpoint{1.359203in}{1.523335in}}%
\pgfpathlineto{\pgfqpoint{1.364698in}{1.524440in}}%
\pgfpathlineto{\pgfqpoint{1.365782in}{1.524755in}}%
\pgfpathlineto{\pgfqpoint{1.370651in}{1.525859in}}%
\pgfpathlineto{\pgfqpoint{1.371530in}{1.526017in}}%
\pgfpathlineto{\pgfqpoint{1.378352in}{1.527122in}}%
\pgfpathlineto{\pgfqpoint{1.378969in}{1.527240in}}%
\pgfpathlineto{\pgfqpoint{1.385390in}{1.528344in}}%
\pgfpathlineto{\pgfqpoint{1.386212in}{1.528423in}}%
\pgfpathlineto{\pgfqpoint{1.392324in}{1.529488in}}%
\pgfpathlineto{\pgfqpoint{1.392558in}{1.529646in}}%
\pgfpathlineto{\pgfqpoint{1.400007in}{1.530750in}}%
\pgfpathlineto{\pgfqpoint{1.401044in}{1.530948in}}%
\pgfpathlineto{\pgfqpoint{1.409652in}{1.532052in}}%
\pgfpathlineto{\pgfqpoint{1.410362in}{1.532249in}}%
\pgfpathlineto{\pgfqpoint{1.417558in}{1.533354in}}%
\pgfpathlineto{\pgfqpoint{1.418353in}{1.533432in}}%
\pgfpathlineto{\pgfqpoint{1.424914in}{1.534537in}}%
\pgfpathlineto{\pgfqpoint{1.425970in}{1.534734in}}%
\pgfpathlineto{\pgfqpoint{1.434400in}{1.535838in}}%
\pgfpathlineto{\pgfqpoint{1.434400in}{1.535878in}}%
\pgfpathlineto{\pgfqpoint{1.441512in}{1.536982in}}%
\pgfpathlineto{\pgfqpoint{1.442596in}{1.537219in}}%
\pgfpathlineto{\pgfqpoint{1.451652in}{1.538323in}}%
\pgfpathlineto{\pgfqpoint{1.452549in}{1.538521in}}%
\pgfpathlineto{\pgfqpoint{1.462886in}{1.539586in}}%
\pgfpathlineto{\pgfqpoint{1.463793in}{1.539940in}}%
\pgfpathlineto{\pgfqpoint{1.472381in}{1.541045in}}%
\pgfpathlineto{\pgfqpoint{1.473381in}{1.541242in}}%
\pgfpathlineto{\pgfqpoint{1.481082in}{1.542346in}}%
\pgfpathlineto{\pgfqpoint{1.482110in}{1.542544in}}%
\pgfpathlineto{\pgfqpoint{1.492662in}{1.543648in}}%
\pgfpathlineto{\pgfqpoint{1.493709in}{1.543924in}}%
\pgfpathlineto{\pgfqpoint{1.502485in}{1.545029in}}%
\pgfpathlineto{\pgfqpoint{1.503354in}{1.545384in}}%
\pgfpathlineto{\pgfqpoint{1.511728in}{1.546488in}}%
\pgfpathlineto{\pgfqpoint{1.512587in}{1.546882in}}%
\pgfpathlineto{\pgfqpoint{1.522560in}{1.547987in}}%
\pgfpathlineto{\pgfqpoint{1.523410in}{1.548145in}}%
\pgfpathlineto{\pgfqpoint{1.534401in}{1.549209in}}%
\pgfpathlineto{\pgfqpoint{1.535391in}{1.549446in}}%
\pgfpathlineto{\pgfqpoint{1.543242in}{1.550551in}}%
\pgfpathlineto{\pgfqpoint{1.544046in}{1.550629in}}%
\pgfpathlineto{\pgfqpoint{1.553990in}{1.551734in}}%
\pgfpathlineto{\pgfqpoint{1.553990in}{1.551773in}}%
\pgfpathlineto{\pgfqpoint{1.564840in}{1.552878in}}%
\pgfpathlineto{\pgfqpoint{1.565784in}{1.553035in}}%
\pgfpathlineto{\pgfqpoint{1.578261in}{1.554140in}}%
\pgfpathlineto{\pgfqpoint{1.579037in}{1.554298in}}%
\pgfpathlineto{\pgfqpoint{1.593691in}{1.555402in}}%
\pgfpathlineto{\pgfqpoint{1.594373in}{1.555520in}}%
\pgfpathlineto{\pgfqpoint{1.608056in}{1.556625in}}%
\pgfpathlineto{\pgfqpoint{1.609028in}{1.556861in}}%
\pgfpathlineto{\pgfqpoint{1.617019in}{1.557966in}}%
\pgfpathlineto{\pgfqpoint{1.617776in}{1.558123in}}%
\pgfpathlineto{\pgfqpoint{1.628374in}{1.559228in}}%
\pgfpathlineto{\pgfqpoint{1.629421in}{1.559425in}}%
\pgfpathlineto{\pgfqpoint{1.638552in}{1.560529in}}%
\pgfpathlineto{\pgfqpoint{1.639252in}{1.560687in}}%
\pgfpathlineto{\pgfqpoint{1.649982in}{1.561792in}}%
\pgfpathlineto{\pgfqpoint{1.650626in}{1.561949in}}%
\pgfpathlineto{\pgfqpoint{1.663028in}{1.563054in}}%
\pgfpathlineto{\pgfqpoint{1.664103in}{1.563290in}}%
\pgfpathlineto{\pgfqpoint{1.675029in}{1.564395in}}%
\pgfpathlineto{\pgfqpoint{1.675029in}{1.564434in}}%
\pgfpathlineto{\pgfqpoint{1.689001in}{1.565539in}}%
\pgfpathlineto{\pgfqpoint{1.690038in}{1.565736in}}%
\pgfpathlineto{\pgfqpoint{1.704721in}{1.566840in}}%
\pgfpathlineto{\pgfqpoint{1.704721in}{1.566919in}}%
\pgfpathlineto{\pgfqpoint{1.717954in}{1.568024in}}%
\pgfpathlineto{\pgfqpoint{1.718235in}{1.568142in}}%
\pgfpathlineto{\pgfqpoint{1.732338in}{1.569246in}}%
\pgfpathlineto{\pgfqpoint{1.733095in}{1.569365in}}%
\pgfpathlineto{\pgfqpoint{1.744366in}{1.570469in}}%
\pgfpathlineto{\pgfqpoint{1.745114in}{1.570548in}}%
\pgfpathlineto{\pgfqpoint{1.758712in}{1.571652in}}%
\pgfpathlineto{\pgfqpoint{1.759618in}{1.571810in}}%
\pgfpathlineto{\pgfqpoint{1.773217in}{1.572914in}}%
\pgfpathlineto{\pgfqpoint{1.774002in}{1.572993in}}%
\pgfpathlineto{\pgfqpoint{1.787946in}{1.574098in}}%
\pgfpathlineto{\pgfqpoint{1.788974in}{1.574216in}}%
\pgfpathlineto{\pgfqpoint{1.799525in}{1.575320in}}%
\pgfpathlineto{\pgfqpoint{1.800226in}{1.575478in}}%
\pgfpathlineto{\pgfqpoint{1.813862in}{1.576583in}}%
\pgfpathlineto{\pgfqpoint{1.813862in}{1.576622in}}%
\pgfpathlineto{\pgfqpoint{1.827591in}{1.577726in}}%
\pgfpathlineto{\pgfqpoint{1.828619in}{1.577963in}}%
\pgfpathlineto{\pgfqpoint{1.841788in}{1.579067in}}%
\pgfpathlineto{\pgfqpoint{1.842890in}{1.579186in}}%
\pgfpathlineto{\pgfqpoint{1.855199in}{1.580290in}}%
\pgfpathlineto{\pgfqpoint{1.855741in}{1.580369in}}%
\pgfpathlineto{\pgfqpoint{1.869517in}{1.581473in}}%
\pgfpathlineto{\pgfqpoint{1.870246in}{1.581552in}}%
\pgfpathlineto{\pgfqpoint{1.888003in}{1.582657in}}%
\pgfpathlineto{\pgfqpoint{1.888769in}{1.582815in}}%
\pgfpathlineto{\pgfqpoint{1.904377in}{1.583919in}}%
\pgfpathlineto{\pgfqpoint{1.905433in}{1.584116in}}%
\pgfpathlineto{\pgfqpoint{1.920760in}{1.585221in}}%
\pgfpathlineto{\pgfqpoint{1.921714in}{1.585418in}}%
\pgfpathlineto{\pgfqpoint{1.932667in}{1.586522in}}%
\pgfpathlineto{\pgfqpoint{1.933321in}{1.586601in}}%
\pgfpathlineto{\pgfqpoint{1.946106in}{1.587705in}}%
\pgfpathlineto{\pgfqpoint{1.947116in}{1.587784in}}%
\pgfpathlineto{\pgfqpoint{1.961172in}{1.588889in}}%
\pgfpathlineto{\pgfqpoint{1.962266in}{1.589046in}}%
\pgfpathlineto{\pgfqpoint{1.974518in}{1.590151in}}%
\pgfpathlineto{\pgfqpoint{1.975499in}{1.590388in}}%
\pgfpathlineto{\pgfqpoint{1.986079in}{1.591492in}}%
\pgfpathlineto{\pgfqpoint{1.986845in}{1.591650in}}%
\pgfpathlineto{\pgfqpoint{1.997425in}{1.592754in}}%
\pgfpathlineto{\pgfqpoint{1.998528in}{1.592991in}}%
\pgfpathlineto{\pgfqpoint{2.006621in}{1.594095in}}%
\pgfpathlineto{\pgfqpoint{2.007294in}{1.594292in}}%
\pgfpathlineto{\pgfqpoint{2.018219in}{1.595397in}}%
\pgfpathlineto{\pgfqpoint{2.018948in}{1.595633in}}%
\pgfpathlineto{\pgfqpoint{2.025556in}{1.596738in}}%
\pgfpathlineto{\pgfqpoint{2.025948in}{1.596974in}}%
\pgfpathlineto{\pgfqpoint{2.029715in}{1.598079in}}%
\pgfpathlineto{\pgfqpoint{2.030771in}{1.598592in}}%
\pgfpathlineto{\pgfqpoint{2.032687in}{1.599696in}}%
\pgfpathlineto{\pgfqpoint{2.033126in}{1.601944in}}%
\pgfpathlineto{\pgfqpoint{2.033126in}{1.601944in}}%
\pgfusepath{stroke}%
\end{pgfscope}%
\begin{pgfscope}%
\pgfsetrectcap%
\pgfsetmiterjoin%
\pgfsetlinewidth{0.803000pt}%
\definecolor{currentstroke}{rgb}{0.000000,0.000000,0.000000}%
\pgfsetstrokecolor{currentstroke}%
\pgfsetdash{}{0pt}%
\pgfpathmoveto{\pgfqpoint{0.553581in}{0.499444in}}%
\pgfpathlineto{\pgfqpoint{0.553581in}{1.654444in}}%
\pgfusepath{stroke}%
\end{pgfscope}%
\begin{pgfscope}%
\pgfsetrectcap%
\pgfsetmiterjoin%
\pgfsetlinewidth{0.803000pt}%
\definecolor{currentstroke}{rgb}{0.000000,0.000000,0.000000}%
\pgfsetstrokecolor{currentstroke}%
\pgfsetdash{}{0pt}%
\pgfpathmoveto{\pgfqpoint{2.103581in}{0.499444in}}%
\pgfpathlineto{\pgfqpoint{2.103581in}{1.654444in}}%
\pgfusepath{stroke}%
\end{pgfscope}%
\begin{pgfscope}%
\pgfsetrectcap%
\pgfsetmiterjoin%
\pgfsetlinewidth{0.803000pt}%
\definecolor{currentstroke}{rgb}{0.000000,0.000000,0.000000}%
\pgfsetstrokecolor{currentstroke}%
\pgfsetdash{}{0pt}%
\pgfpathmoveto{\pgfqpoint{0.553581in}{0.499444in}}%
\pgfpathlineto{\pgfqpoint{2.103581in}{0.499444in}}%
\pgfusepath{stroke}%
\end{pgfscope}%
\begin{pgfscope}%
\pgfsetrectcap%
\pgfsetmiterjoin%
\pgfsetlinewidth{0.803000pt}%
\definecolor{currentstroke}{rgb}{0.000000,0.000000,0.000000}%
\pgfsetstrokecolor{currentstroke}%
\pgfsetdash{}{0pt}%
\pgfpathmoveto{\pgfqpoint{0.553581in}{1.654444in}}%
\pgfpathlineto{\pgfqpoint{2.103581in}{1.654444in}}%
\pgfusepath{stroke}%
\end{pgfscope}%
\begin{pgfscope}%
\pgfsetbuttcap%
\pgfsetmiterjoin%
\definecolor{currentfill}{rgb}{1.000000,1.000000,1.000000}%
\pgfsetfillcolor{currentfill}%
\pgfsetlinewidth{0.000000pt}%
\definecolor{currentstroke}{rgb}{0.000000,0.000000,0.000000}%
\pgfsetstrokecolor{currentstroke}%
\pgfsetstrokeopacity{0.000000}%
\pgfsetdash{}{0pt}%
\pgfpathmoveto{\pgfqpoint{1.286928in}{1.448926in}}%
\pgfpathlineto{\pgfqpoint{1.686650in}{1.448926in}}%
\pgfpathlineto{\pgfqpoint{1.686650in}{1.655593in}}%
\pgfpathlineto{\pgfqpoint{1.286928in}{1.655593in}}%
\pgfpathlineto{\pgfqpoint{1.286928in}{1.448926in}}%
\pgfpathclose%
\pgfusepath{fill}%
\end{pgfscope}%
\begin{pgfscope}%
\definecolor{textcolor}{rgb}{0.000000,0.000000,0.000000}%
\pgfsetstrokecolor{textcolor}%
\pgfsetfillcolor{textcolor}%
\pgftext[x=1.328595in,y=1.517537in,left,base]{\color{textcolor}\rmfamily\fontsize{10.000000}{12.000000}\selectfont 0.338}%
\end{pgfscope}%
\begin{pgfscope}%
\pgfsetbuttcap%
\pgfsetmiterjoin%
\definecolor{currentfill}{rgb}{1.000000,1.000000,1.000000}%
\pgfsetfillcolor{currentfill}%
\pgfsetlinewidth{0.000000pt}%
\definecolor{currentstroke}{rgb}{0.000000,0.000000,0.000000}%
\pgfsetstrokecolor{currentstroke}%
\pgfsetstrokeopacity{0.000000}%
\pgfsetdash{}{0pt}%
\pgfpathmoveto{\pgfqpoint{0.687846in}{1.008353in}}%
\pgfpathlineto{\pgfqpoint{1.087569in}{1.008353in}}%
\pgfpathlineto{\pgfqpoint{1.087569in}{1.215019in}}%
\pgfpathlineto{\pgfqpoint{0.687846in}{1.215019in}}%
\pgfpathlineto{\pgfqpoint{0.687846in}{1.008353in}}%
\pgfpathclose%
\pgfusepath{fill}%
\end{pgfscope}%
\begin{pgfscope}%
\definecolor{textcolor}{rgb}{0.000000,0.000000,0.000000}%
\pgfsetstrokecolor{textcolor}%
\pgfsetfillcolor{textcolor}%
\pgftext[x=0.729513in,y=1.076964in,left,base]{\color{textcolor}\rmfamily\fontsize{10.000000}{12.000000}\selectfont 0.658}%
\end{pgfscope}%
\begin{pgfscope}%
\pgfsetbuttcap%
\pgfsetmiterjoin%
\definecolor{currentfill}{rgb}{1.000000,1.000000,1.000000}%
\pgfsetfillcolor{currentfill}%
\pgfsetfillopacity{0.800000}%
\pgfsetlinewidth{1.003750pt}%
\definecolor{currentstroke}{rgb}{0.800000,0.800000,0.800000}%
\pgfsetstrokecolor{currentstroke}%
\pgfsetstrokeopacity{0.800000}%
\pgfsetdash{}{0pt}%
\pgfpathmoveto{\pgfqpoint{0.840525in}{0.568889in}}%
\pgfpathlineto{\pgfqpoint{2.006358in}{0.568889in}}%
\pgfpathquadraticcurveto{\pgfqpoint{2.034136in}{0.568889in}}{\pgfqpoint{2.034136in}{0.596666in}}%
\pgfpathlineto{\pgfqpoint{2.034136in}{0.791111in}}%
\pgfpathquadraticcurveto{\pgfqpoint{2.034136in}{0.818888in}}{\pgfqpoint{2.006358in}{0.818888in}}%
\pgfpathlineto{\pgfqpoint{0.840525in}{0.818888in}}%
\pgfpathquadraticcurveto{\pgfqpoint{0.812747in}{0.818888in}}{\pgfqpoint{0.812747in}{0.791111in}}%
\pgfpathlineto{\pgfqpoint{0.812747in}{0.596666in}}%
\pgfpathquadraticcurveto{\pgfqpoint{0.812747in}{0.568889in}}{\pgfqpoint{0.840525in}{0.568889in}}%
\pgfpathlineto{\pgfqpoint{0.840525in}{0.568889in}}%
\pgfpathclose%
\pgfusepath{stroke,fill}%
\end{pgfscope}%
\begin{pgfscope}%
\pgfsetrectcap%
\pgfsetroundjoin%
\pgfsetlinewidth{1.505625pt}%
\definecolor{currentstroke}{rgb}{0.000000,0.000000,0.000000}%
\pgfsetstrokecolor{currentstroke}%
\pgfsetdash{}{0pt}%
\pgfpathmoveto{\pgfqpoint{0.868303in}{0.707777in}}%
\pgfpathlineto{\pgfqpoint{1.007192in}{0.707777in}}%
\pgfpathlineto{\pgfqpoint{1.146081in}{0.707777in}}%
\pgfusepath{stroke}%
\end{pgfscope}%
\begin{pgfscope}%
\definecolor{textcolor}{rgb}{0.000000,0.000000,0.000000}%
\pgfsetstrokecolor{textcolor}%
\pgfsetfillcolor{textcolor}%
\pgftext[x=1.257192in,y=0.659166in,left,base]{\color{textcolor}\rmfamily\fontsize{10.000000}{12.000000}\selectfont AUC 0.840)}%
\end{pgfscope}%
\end{pgfpicture}%
\makeatother%
\endgroup%

\end{tabular}

{\bf Incorporate $thr=0.5$ into the discussion.}

The {\it confusion matrix} for this ideal data set, here given as percentages of the entire dataset, shows few false positives and false negatives.  The metrics below are the ones we will watch when evaluating models.  Each of them tells a different story about what the model does well.

\vskip 6pt

$\displaystyle Precision = \frac{TP}{PP} = \frac{TP}{TP+FP}$ tells what proportion of the ambulances we sent were needed.  

\vskip 6pt

$\displaystyle Recall = \frac{TP}{P} = \frac{TP}{TP + FN}$ tells what proportion of ambulances we needed were sent.

\vskip 6pt

$\displaystyle F1 = \frac{2}{ \frac{1}{Precision} + \frac{1}{Recall}}$ is the harmonic mean of precision and recall. 

\vskip 6pt

Why the harmonic mean, not the arithmetic or geometric mean?  If $a$ and $b$ are positive numbers with $a<b$, then $a < \text{harmonic} < \text{geometric} < \text{arithmetic} < b$.  The harmonic, while being influenced by the larger number, is closest to the smaller, so the harmonic mean emphasizes what the model does poorly.  



\begin{center}
\begin{tabular}{cc}
\begin{tabular}{cc|c|c|c}
	&\multicolumn{1}{c}{}& \multicolumn{2}{c}{Prediction} \cr
	&\multicolumn{1}{c}{} & \multicolumn{1}{c}{Neg} & \multicolumn{1}{c}{Pos} & \multicolumn{1}{c}{Total} \cr\cline{3-4}
	\multirow{2}{*}{Actual}&Neg &TN = 67.0\% & FP = 18.7\% & N = 85.7\% \vrule width 0pt height 10pt depth 2pt \cr\cline{3-4}
	&Pos & FN = 3.2\% & TP = 11.1\% & P = 14.3\% \vrule width 0pt height 10pt depth 2pt \cr\cline{3-4}
	\multicolumn{2}{r}{Total} & \multicolumn{1}{c}{PN = 70.2\%} & \multicolumn{1}{c}{PP = 29.8\%} \vrule width 0pt height 14pt depth 2pt \cr
\end{tabular}
&
\begin{tabular}{ll}
0.372 & Precision \cr 
0.774 & Recall \cr 
0.502 & F1 \cr 
\end{tabular}
\end{tabular}
\end{center}

%%%%%
If we do not address the data imbalance, the model building algorithm will maximize accuracy by classifying most (or all) of the samples as ``No Ambulance'' with $p < 0.5$  We built the artificial results below by multiplying the probabilities in the above results by $0.5$.  Note that the Area Under the Curve (AUC) did not change.  

\begin{center}
\begin{tabular}{p{0.5\textwidth} p{0.5\textwidth}}
  \vspace{0pt} %% Creator: Matplotlib, PGF backend
%%
%% To include the figure in your LaTeX document, write
%%   \input{<filename>.pgf}
%%
%% Make sure the required packages are loaded in your preamble
%%   \usepackage{pgf}
%%
%% Also ensure that all the required font packages are loaded; for instance,
%% the lmodern package is sometimes necessary when using math font.
%%   \usepackage{lmodern}
%%
%% Figures using additional raster images can only be included by \input if
%% they are in the same directory as the main LaTeX file. For loading figures
%% from other directories you can use the `import` package
%%   \usepackage{import}
%%
%% and then include the figures with
%%   \import{<path to file>}{<filename>.pgf}
%%
%% Matplotlib used the following preamble
%%   
%%   \usepackage{fontspec}
%%   \makeatletter\@ifpackageloaded{underscore}{}{\usepackage[strings]{underscore}}\makeatother
%%
\begingroup%
\makeatletter%
\begin{pgfpicture}%
\pgfpathrectangle{\pgfpointorigin}{\pgfqpoint{3.095000in}{3.044944in}}%
\pgfusepath{use as bounding box, clip}%
\begin{pgfscope}%
\pgfsetbuttcap%
\pgfsetmiterjoin%
\definecolor{currentfill}{rgb}{1.000000,1.000000,1.000000}%
\pgfsetfillcolor{currentfill}%
\pgfsetlinewidth{0.000000pt}%
\definecolor{currentstroke}{rgb}{1.000000,1.000000,1.000000}%
\pgfsetstrokecolor{currentstroke}%
\pgfsetdash{}{0pt}%
\pgfpathmoveto{\pgfqpoint{0.000000in}{0.000000in}}%
\pgfpathlineto{\pgfqpoint{3.095000in}{0.000000in}}%
\pgfpathlineto{\pgfqpoint{3.095000in}{3.044944in}}%
\pgfpathlineto{\pgfqpoint{0.000000in}{3.044944in}}%
\pgfpathlineto{\pgfqpoint{0.000000in}{0.000000in}}%
\pgfpathclose%
\pgfusepath{fill}%
\end{pgfscope}%
\begin{pgfscope}%
\pgfsetbuttcap%
\pgfsetmiterjoin%
\definecolor{currentfill}{rgb}{1.000000,1.000000,1.000000}%
\pgfsetfillcolor{currentfill}%
\pgfsetlinewidth{0.000000pt}%
\definecolor{currentstroke}{rgb}{0.000000,0.000000,0.000000}%
\pgfsetstrokecolor{currentstroke}%
\pgfsetstrokeopacity{0.000000}%
\pgfsetdash{}{0pt}%
\pgfpathmoveto{\pgfqpoint{0.515000in}{1.096944in}}%
\pgfpathlineto{\pgfqpoint{2.995000in}{1.096944in}}%
\pgfpathlineto{\pgfqpoint{2.995000in}{2.944944in}}%
\pgfpathlineto{\pgfqpoint{0.515000in}{2.944944in}}%
\pgfpathlineto{\pgfqpoint{0.515000in}{1.096944in}}%
\pgfpathclose%
\pgfusepath{fill}%
\end{pgfscope}%
\begin{pgfscope}%
\pgfpathrectangle{\pgfqpoint{0.515000in}{1.096944in}}{\pgfqpoint{2.480000in}{1.848000in}}%
\pgfusepath{clip}%
\pgfsetbuttcap%
\pgfsetmiterjoin%
\pgfsetlinewidth{1.003750pt}%
\definecolor{currentstroke}{rgb}{0.000000,0.000000,0.000000}%
\pgfsetstrokecolor{currentstroke}%
\pgfsetdash{}{0pt}%
\pgfpathmoveto{\pgfqpoint{0.505000in}{1.096944in}}%
\pgfpathlineto{\pgfqpoint{0.577627in}{1.096944in}}%
\pgfpathlineto{\pgfqpoint{0.577627in}{2.856944in}}%
\pgfpathlineto{\pgfqpoint{0.505000in}{2.856944in}}%
\pgfusepath{stroke}%
\end{pgfscope}%
\begin{pgfscope}%
\pgfpathrectangle{\pgfqpoint{0.515000in}{1.096944in}}{\pgfqpoint{2.480000in}{1.848000in}}%
\pgfusepath{clip}%
\pgfsetbuttcap%
\pgfsetmiterjoin%
\pgfsetlinewidth{1.003750pt}%
\definecolor{currentstroke}{rgb}{0.000000,0.000000,0.000000}%
\pgfsetstrokecolor{currentstroke}%
\pgfsetdash{}{0pt}%
\pgfpathmoveto{\pgfqpoint{0.727930in}{1.096944in}}%
\pgfpathlineto{\pgfqpoint{0.828132in}{1.096944in}}%
\pgfpathlineto{\pgfqpoint{0.828132in}{2.725341in}}%
\pgfpathlineto{\pgfqpoint{0.727930in}{2.725341in}}%
\pgfpathlineto{\pgfqpoint{0.727930in}{1.096944in}}%
\pgfpathclose%
\pgfusepath{stroke}%
\end{pgfscope}%
\begin{pgfscope}%
\pgfpathrectangle{\pgfqpoint{0.515000in}{1.096944in}}{\pgfqpoint{2.480000in}{1.848000in}}%
\pgfusepath{clip}%
\pgfsetbuttcap%
\pgfsetmiterjoin%
\pgfsetlinewidth{1.003750pt}%
\definecolor{currentstroke}{rgb}{0.000000,0.000000,0.000000}%
\pgfsetstrokecolor{currentstroke}%
\pgfsetdash{}{0pt}%
\pgfpathmoveto{\pgfqpoint{0.978435in}{1.096944in}}%
\pgfpathlineto{\pgfqpoint{1.078637in}{1.096944in}}%
\pgfpathlineto{\pgfqpoint{1.078637in}{1.515051in}}%
\pgfpathlineto{\pgfqpoint{0.978435in}{1.515051in}}%
\pgfpathlineto{\pgfqpoint{0.978435in}{1.096944in}}%
\pgfpathclose%
\pgfusepath{stroke}%
\end{pgfscope}%
\begin{pgfscope}%
\pgfpathrectangle{\pgfqpoint{0.515000in}{1.096944in}}{\pgfqpoint{2.480000in}{1.848000in}}%
\pgfusepath{clip}%
\pgfsetbuttcap%
\pgfsetmiterjoin%
\pgfsetlinewidth{1.003750pt}%
\definecolor{currentstroke}{rgb}{0.000000,0.000000,0.000000}%
\pgfsetstrokecolor{currentstroke}%
\pgfsetdash{}{0pt}%
\pgfpathmoveto{\pgfqpoint{1.228940in}{1.096944in}}%
\pgfpathlineto{\pgfqpoint{1.329142in}{1.096944in}}%
\pgfpathlineto{\pgfqpoint{1.329142in}{1.165385in}}%
\pgfpathlineto{\pgfqpoint{1.228940in}{1.165385in}}%
\pgfpathlineto{\pgfqpoint{1.228940in}{1.096944in}}%
\pgfpathclose%
\pgfusepath{stroke}%
\end{pgfscope}%
\begin{pgfscope}%
\pgfpathrectangle{\pgfqpoint{0.515000in}{1.096944in}}{\pgfqpoint{2.480000in}{1.848000in}}%
\pgfusepath{clip}%
\pgfsetbuttcap%
\pgfsetmiterjoin%
\pgfsetlinewidth{1.003750pt}%
\definecolor{currentstroke}{rgb}{0.000000,0.000000,0.000000}%
\pgfsetstrokecolor{currentstroke}%
\pgfsetdash{}{0pt}%
\pgfpathmoveto{\pgfqpoint{1.479445in}{1.096944in}}%
\pgfpathlineto{\pgfqpoint{1.579647in}{1.096944in}}%
\pgfpathlineto{\pgfqpoint{1.579647in}{1.096944in}}%
\pgfpathlineto{\pgfqpoint{1.479445in}{1.096944in}}%
\pgfpathlineto{\pgfqpoint{1.479445in}{1.096944in}}%
\pgfpathclose%
\pgfusepath{stroke}%
\end{pgfscope}%
\begin{pgfscope}%
\pgfpathrectangle{\pgfqpoint{0.515000in}{1.096944in}}{\pgfqpoint{2.480000in}{1.848000in}}%
\pgfusepath{clip}%
\pgfsetbuttcap%
\pgfsetmiterjoin%
\pgfsetlinewidth{1.003750pt}%
\definecolor{currentstroke}{rgb}{0.000000,0.000000,0.000000}%
\pgfsetstrokecolor{currentstroke}%
\pgfsetdash{}{0pt}%
\pgfpathmoveto{\pgfqpoint{1.729950in}{1.096944in}}%
\pgfpathlineto{\pgfqpoint{1.830152in}{1.096944in}}%
\pgfpathlineto{\pgfqpoint{1.830152in}{1.096944in}}%
\pgfpathlineto{\pgfqpoint{1.729950in}{1.096944in}}%
\pgfpathlineto{\pgfqpoint{1.729950in}{1.096944in}}%
\pgfpathclose%
\pgfusepath{stroke}%
\end{pgfscope}%
\begin{pgfscope}%
\pgfpathrectangle{\pgfqpoint{0.515000in}{1.096944in}}{\pgfqpoint{2.480000in}{1.848000in}}%
\pgfusepath{clip}%
\pgfsetbuttcap%
\pgfsetmiterjoin%
\pgfsetlinewidth{1.003750pt}%
\definecolor{currentstroke}{rgb}{0.000000,0.000000,0.000000}%
\pgfsetstrokecolor{currentstroke}%
\pgfsetdash{}{0pt}%
\pgfpathmoveto{\pgfqpoint{1.980455in}{1.096944in}}%
\pgfpathlineto{\pgfqpoint{2.080657in}{1.096944in}}%
\pgfpathlineto{\pgfqpoint{2.080657in}{1.096944in}}%
\pgfpathlineto{\pgfqpoint{1.980455in}{1.096944in}}%
\pgfpathlineto{\pgfqpoint{1.980455in}{1.096944in}}%
\pgfpathclose%
\pgfusepath{stroke}%
\end{pgfscope}%
\begin{pgfscope}%
\pgfpathrectangle{\pgfqpoint{0.515000in}{1.096944in}}{\pgfqpoint{2.480000in}{1.848000in}}%
\pgfusepath{clip}%
\pgfsetbuttcap%
\pgfsetmiterjoin%
\pgfsetlinewidth{1.003750pt}%
\definecolor{currentstroke}{rgb}{0.000000,0.000000,0.000000}%
\pgfsetstrokecolor{currentstroke}%
\pgfsetdash{}{0pt}%
\pgfpathmoveto{\pgfqpoint{2.230960in}{1.096944in}}%
\pgfpathlineto{\pgfqpoint{2.331162in}{1.096944in}}%
\pgfpathlineto{\pgfqpoint{2.331162in}{1.096944in}}%
\pgfpathlineto{\pgfqpoint{2.230960in}{1.096944in}}%
\pgfpathlineto{\pgfqpoint{2.230960in}{1.096944in}}%
\pgfpathclose%
\pgfusepath{stroke}%
\end{pgfscope}%
\begin{pgfscope}%
\pgfpathrectangle{\pgfqpoint{0.515000in}{1.096944in}}{\pgfqpoint{2.480000in}{1.848000in}}%
\pgfusepath{clip}%
\pgfsetbuttcap%
\pgfsetmiterjoin%
\pgfsetlinewidth{1.003750pt}%
\definecolor{currentstroke}{rgb}{0.000000,0.000000,0.000000}%
\pgfsetstrokecolor{currentstroke}%
\pgfsetdash{}{0pt}%
\pgfpathmoveto{\pgfqpoint{2.481465in}{1.096944in}}%
\pgfpathlineto{\pgfqpoint{2.581667in}{1.096944in}}%
\pgfpathlineto{\pgfqpoint{2.581667in}{1.096944in}}%
\pgfpathlineto{\pgfqpoint{2.481465in}{1.096944in}}%
\pgfpathlineto{\pgfqpoint{2.481465in}{1.096944in}}%
\pgfpathclose%
\pgfusepath{stroke}%
\end{pgfscope}%
\begin{pgfscope}%
\pgfpathrectangle{\pgfqpoint{0.515000in}{1.096944in}}{\pgfqpoint{2.480000in}{1.848000in}}%
\pgfusepath{clip}%
\pgfsetbuttcap%
\pgfsetmiterjoin%
\pgfsetlinewidth{1.003750pt}%
\definecolor{currentstroke}{rgb}{0.000000,0.000000,0.000000}%
\pgfsetstrokecolor{currentstroke}%
\pgfsetdash{}{0pt}%
\pgfpathmoveto{\pgfqpoint{2.731970in}{1.096944in}}%
\pgfpathlineto{\pgfqpoint{2.832172in}{1.096944in}}%
\pgfpathlineto{\pgfqpoint{2.832172in}{1.096944in}}%
\pgfpathlineto{\pgfqpoint{2.731970in}{1.096944in}}%
\pgfpathlineto{\pgfqpoint{2.731970in}{1.096944in}}%
\pgfpathclose%
\pgfusepath{stroke}%
\end{pgfscope}%
\begin{pgfscope}%
\pgfpathrectangle{\pgfqpoint{0.515000in}{1.096944in}}{\pgfqpoint{2.480000in}{1.848000in}}%
\pgfusepath{clip}%
\pgfsetbuttcap%
\pgfsetmiterjoin%
\definecolor{currentfill}{rgb}{0.000000,0.000000,0.000000}%
\pgfsetfillcolor{currentfill}%
\pgfsetlinewidth{0.000000pt}%
\definecolor{currentstroke}{rgb}{0.000000,0.000000,0.000000}%
\pgfsetstrokecolor{currentstroke}%
\pgfsetstrokeopacity{0.000000}%
\pgfsetdash{}{0pt}%
\pgfpathmoveto{\pgfqpoint{0.577627in}{1.096944in}}%
\pgfpathlineto{\pgfqpoint{0.677829in}{1.096944in}}%
\pgfpathlineto{\pgfqpoint{0.677829in}{1.114817in}}%
\pgfpathlineto{\pgfqpoint{0.577627in}{1.114817in}}%
\pgfpathlineto{\pgfqpoint{0.577627in}{1.096944in}}%
\pgfpathclose%
\pgfusepath{fill}%
\end{pgfscope}%
\begin{pgfscope}%
\pgfpathrectangle{\pgfqpoint{0.515000in}{1.096944in}}{\pgfqpoint{2.480000in}{1.848000in}}%
\pgfusepath{clip}%
\pgfsetbuttcap%
\pgfsetmiterjoin%
\definecolor{currentfill}{rgb}{0.000000,0.000000,0.000000}%
\pgfsetfillcolor{currentfill}%
\pgfsetlinewidth{0.000000pt}%
\definecolor{currentstroke}{rgb}{0.000000,0.000000,0.000000}%
\pgfsetstrokecolor{currentstroke}%
\pgfsetstrokeopacity{0.000000}%
\pgfsetdash{}{0pt}%
\pgfpathmoveto{\pgfqpoint{0.828132in}{1.096944in}}%
\pgfpathlineto{\pgfqpoint{0.928334in}{1.096944in}}%
\pgfpathlineto{\pgfqpoint{0.928334in}{1.197354in}}%
\pgfpathlineto{\pgfqpoint{0.828132in}{1.197354in}}%
\pgfpathlineto{\pgfqpoint{0.828132in}{1.096944in}}%
\pgfpathclose%
\pgfusepath{fill}%
\end{pgfscope}%
\begin{pgfscope}%
\pgfpathrectangle{\pgfqpoint{0.515000in}{1.096944in}}{\pgfqpoint{2.480000in}{1.848000in}}%
\pgfusepath{clip}%
\pgfsetbuttcap%
\pgfsetmiterjoin%
\definecolor{currentfill}{rgb}{0.000000,0.000000,0.000000}%
\pgfsetfillcolor{currentfill}%
\pgfsetlinewidth{0.000000pt}%
\definecolor{currentstroke}{rgb}{0.000000,0.000000,0.000000}%
\pgfsetstrokecolor{currentstroke}%
\pgfsetstrokeopacity{0.000000}%
\pgfsetdash{}{0pt}%
\pgfpathmoveto{\pgfqpoint{1.078637in}{1.096944in}}%
\pgfpathlineto{\pgfqpoint{1.178839in}{1.096944in}}%
\pgfpathlineto{\pgfqpoint{1.178839in}{1.504831in}}%
\pgfpathlineto{\pgfqpoint{1.078637in}{1.504831in}}%
\pgfpathlineto{\pgfqpoint{1.078637in}{1.096944in}}%
\pgfpathclose%
\pgfusepath{fill}%
\end{pgfscope}%
\begin{pgfscope}%
\pgfpathrectangle{\pgfqpoint{0.515000in}{1.096944in}}{\pgfqpoint{2.480000in}{1.848000in}}%
\pgfusepath{clip}%
\pgfsetbuttcap%
\pgfsetmiterjoin%
\definecolor{currentfill}{rgb}{0.000000,0.000000,0.000000}%
\pgfsetfillcolor{currentfill}%
\pgfsetlinewidth{0.000000pt}%
\definecolor{currentstroke}{rgb}{0.000000,0.000000,0.000000}%
\pgfsetstrokecolor{currentstroke}%
\pgfsetstrokeopacity{0.000000}%
\pgfsetdash{}{0pt}%
\pgfpathmoveto{\pgfqpoint{1.329142in}{1.096944in}}%
\pgfpathlineto{\pgfqpoint{1.429344in}{1.096944in}}%
\pgfpathlineto{\pgfqpoint{1.429344in}{1.539511in}}%
\pgfpathlineto{\pgfqpoint{1.329142in}{1.539511in}}%
\pgfpathlineto{\pgfqpoint{1.329142in}{1.096944in}}%
\pgfpathclose%
\pgfusepath{fill}%
\end{pgfscope}%
\begin{pgfscope}%
\pgfpathrectangle{\pgfqpoint{0.515000in}{1.096944in}}{\pgfqpoint{2.480000in}{1.848000in}}%
\pgfusepath{clip}%
\pgfsetbuttcap%
\pgfsetmiterjoin%
\definecolor{currentfill}{rgb}{0.000000,0.000000,0.000000}%
\pgfsetfillcolor{currentfill}%
\pgfsetlinewidth{0.000000pt}%
\definecolor{currentstroke}{rgb}{0.000000,0.000000,0.000000}%
\pgfsetstrokecolor{currentstroke}%
\pgfsetstrokeopacity{0.000000}%
\pgfsetdash{}{0pt}%
\pgfpathmoveto{\pgfqpoint{1.579647in}{1.096944in}}%
\pgfpathlineto{\pgfqpoint{1.679849in}{1.096944in}}%
\pgfpathlineto{\pgfqpoint{1.679849in}{1.096944in}}%
\pgfpathlineto{\pgfqpoint{1.579647in}{1.096944in}}%
\pgfpathlineto{\pgfqpoint{1.579647in}{1.096944in}}%
\pgfpathclose%
\pgfusepath{fill}%
\end{pgfscope}%
\begin{pgfscope}%
\pgfpathrectangle{\pgfqpoint{0.515000in}{1.096944in}}{\pgfqpoint{2.480000in}{1.848000in}}%
\pgfusepath{clip}%
\pgfsetbuttcap%
\pgfsetmiterjoin%
\definecolor{currentfill}{rgb}{0.000000,0.000000,0.000000}%
\pgfsetfillcolor{currentfill}%
\pgfsetlinewidth{0.000000pt}%
\definecolor{currentstroke}{rgb}{0.000000,0.000000,0.000000}%
\pgfsetstrokecolor{currentstroke}%
\pgfsetstrokeopacity{0.000000}%
\pgfsetdash{}{0pt}%
\pgfpathmoveto{\pgfqpoint{1.830152in}{1.096944in}}%
\pgfpathlineto{\pgfqpoint{1.930354in}{1.096944in}}%
\pgfpathlineto{\pgfqpoint{1.930354in}{1.096944in}}%
\pgfpathlineto{\pgfqpoint{1.830152in}{1.096944in}}%
\pgfpathlineto{\pgfqpoint{1.830152in}{1.096944in}}%
\pgfpathclose%
\pgfusepath{fill}%
\end{pgfscope}%
\begin{pgfscope}%
\pgfpathrectangle{\pgfqpoint{0.515000in}{1.096944in}}{\pgfqpoint{2.480000in}{1.848000in}}%
\pgfusepath{clip}%
\pgfsetbuttcap%
\pgfsetmiterjoin%
\definecolor{currentfill}{rgb}{0.000000,0.000000,0.000000}%
\pgfsetfillcolor{currentfill}%
\pgfsetlinewidth{0.000000pt}%
\definecolor{currentstroke}{rgb}{0.000000,0.000000,0.000000}%
\pgfsetstrokecolor{currentstroke}%
\pgfsetstrokeopacity{0.000000}%
\pgfsetdash{}{0pt}%
\pgfpathmoveto{\pgfqpoint{2.080657in}{1.096944in}}%
\pgfpathlineto{\pgfqpoint{2.180859in}{1.096944in}}%
\pgfpathlineto{\pgfqpoint{2.180859in}{1.096944in}}%
\pgfpathlineto{\pgfqpoint{2.080657in}{1.096944in}}%
\pgfpathlineto{\pgfqpoint{2.080657in}{1.096944in}}%
\pgfpathclose%
\pgfusepath{fill}%
\end{pgfscope}%
\begin{pgfscope}%
\pgfpathrectangle{\pgfqpoint{0.515000in}{1.096944in}}{\pgfqpoint{2.480000in}{1.848000in}}%
\pgfusepath{clip}%
\pgfsetbuttcap%
\pgfsetmiterjoin%
\definecolor{currentfill}{rgb}{0.000000,0.000000,0.000000}%
\pgfsetfillcolor{currentfill}%
\pgfsetlinewidth{0.000000pt}%
\definecolor{currentstroke}{rgb}{0.000000,0.000000,0.000000}%
\pgfsetstrokecolor{currentstroke}%
\pgfsetstrokeopacity{0.000000}%
\pgfsetdash{}{0pt}%
\pgfpathmoveto{\pgfqpoint{2.331162in}{1.096944in}}%
\pgfpathlineto{\pgfqpoint{2.431364in}{1.096944in}}%
\pgfpathlineto{\pgfqpoint{2.431364in}{1.096944in}}%
\pgfpathlineto{\pgfqpoint{2.331162in}{1.096944in}}%
\pgfpathlineto{\pgfqpoint{2.331162in}{1.096944in}}%
\pgfpathclose%
\pgfusepath{fill}%
\end{pgfscope}%
\begin{pgfscope}%
\pgfpathrectangle{\pgfqpoint{0.515000in}{1.096944in}}{\pgfqpoint{2.480000in}{1.848000in}}%
\pgfusepath{clip}%
\pgfsetbuttcap%
\pgfsetmiterjoin%
\definecolor{currentfill}{rgb}{0.000000,0.000000,0.000000}%
\pgfsetfillcolor{currentfill}%
\pgfsetlinewidth{0.000000pt}%
\definecolor{currentstroke}{rgb}{0.000000,0.000000,0.000000}%
\pgfsetstrokecolor{currentstroke}%
\pgfsetstrokeopacity{0.000000}%
\pgfsetdash{}{0pt}%
\pgfpathmoveto{\pgfqpoint{2.581667in}{1.096944in}}%
\pgfpathlineto{\pgfqpoint{2.681869in}{1.096944in}}%
\pgfpathlineto{\pgfqpoint{2.681869in}{1.096944in}}%
\pgfpathlineto{\pgfqpoint{2.581667in}{1.096944in}}%
\pgfpathlineto{\pgfqpoint{2.581667in}{1.096944in}}%
\pgfpathclose%
\pgfusepath{fill}%
\end{pgfscope}%
\begin{pgfscope}%
\pgfpathrectangle{\pgfqpoint{0.515000in}{1.096944in}}{\pgfqpoint{2.480000in}{1.848000in}}%
\pgfusepath{clip}%
\pgfsetbuttcap%
\pgfsetmiterjoin%
\definecolor{currentfill}{rgb}{0.000000,0.000000,0.000000}%
\pgfsetfillcolor{currentfill}%
\pgfsetlinewidth{0.000000pt}%
\definecolor{currentstroke}{rgb}{0.000000,0.000000,0.000000}%
\pgfsetstrokecolor{currentstroke}%
\pgfsetstrokeopacity{0.000000}%
\pgfsetdash{}{0pt}%
\pgfpathmoveto{\pgfqpoint{2.832172in}{1.096944in}}%
\pgfpathlineto{\pgfqpoint{2.932374in}{1.096944in}}%
\pgfpathlineto{\pgfqpoint{2.932374in}{1.096944in}}%
\pgfpathlineto{\pgfqpoint{2.832172in}{1.096944in}}%
\pgfpathlineto{\pgfqpoint{2.832172in}{1.096944in}}%
\pgfpathclose%
\pgfusepath{fill}%
\end{pgfscope}%
\begin{pgfscope}%
\pgfsetbuttcap%
\pgfsetroundjoin%
\definecolor{currentfill}{rgb}{0.000000,0.000000,0.000000}%
\pgfsetfillcolor{currentfill}%
\pgfsetlinewidth{0.803000pt}%
\definecolor{currentstroke}{rgb}{0.000000,0.000000,0.000000}%
\pgfsetstrokecolor{currentstroke}%
\pgfsetdash{}{0pt}%
\pgfsys@defobject{currentmarker}{\pgfqpoint{0.000000in}{-0.048611in}}{\pgfqpoint{0.000000in}{0.000000in}}{%
\pgfpathmoveto{\pgfqpoint{0.000000in}{0.000000in}}%
\pgfpathlineto{\pgfqpoint{0.000000in}{-0.048611in}}%
\pgfusepath{stroke,fill}%
}%
\begin{pgfscope}%
\pgfsys@transformshift{0.577627in}{1.096944in}%
\pgfsys@useobject{currentmarker}{}%
\end{pgfscope}%
\end{pgfscope}%
\begin{pgfscope}%
\definecolor{textcolor}{rgb}{0.000000,0.000000,0.000000}%
\pgfsetstrokecolor{textcolor}%
\pgfsetfillcolor{textcolor}%
\pgftext[x=0.612349in, y=0.282083in, left, base,rotate=90.000000]{\color{textcolor}\rmfamily\fontsize{10.000000}{12.000000}\selectfont (-0.001, 0.1]}%
\end{pgfscope}%
\begin{pgfscope}%
\pgfsetbuttcap%
\pgfsetroundjoin%
\definecolor{currentfill}{rgb}{0.000000,0.000000,0.000000}%
\pgfsetfillcolor{currentfill}%
\pgfsetlinewidth{0.803000pt}%
\definecolor{currentstroke}{rgb}{0.000000,0.000000,0.000000}%
\pgfsetstrokecolor{currentstroke}%
\pgfsetdash{}{0pt}%
\pgfsys@defobject{currentmarker}{\pgfqpoint{0.000000in}{-0.048611in}}{\pgfqpoint{0.000000in}{0.000000in}}{%
\pgfpathmoveto{\pgfqpoint{0.000000in}{0.000000in}}%
\pgfpathlineto{\pgfqpoint{0.000000in}{-0.048611in}}%
\pgfusepath{stroke,fill}%
}%
\begin{pgfscope}%
\pgfsys@transformshift{0.828132in}{1.096944in}%
\pgfsys@useobject{currentmarker}{}%
\end{pgfscope}%
\end{pgfscope}%
\begin{pgfscope}%
\definecolor{textcolor}{rgb}{0.000000,0.000000,0.000000}%
\pgfsetstrokecolor{textcolor}%
\pgfsetfillcolor{textcolor}%
\pgftext[x=0.862854in, y=0.467222in, left, base,rotate=90.000000]{\color{textcolor}\rmfamily\fontsize{10.000000}{12.000000}\selectfont (0.1, 0.2]}%
\end{pgfscope}%
\begin{pgfscope}%
\pgfsetbuttcap%
\pgfsetroundjoin%
\definecolor{currentfill}{rgb}{0.000000,0.000000,0.000000}%
\pgfsetfillcolor{currentfill}%
\pgfsetlinewidth{0.803000pt}%
\definecolor{currentstroke}{rgb}{0.000000,0.000000,0.000000}%
\pgfsetstrokecolor{currentstroke}%
\pgfsetdash{}{0pt}%
\pgfsys@defobject{currentmarker}{\pgfqpoint{0.000000in}{-0.048611in}}{\pgfqpoint{0.000000in}{0.000000in}}{%
\pgfpathmoveto{\pgfqpoint{0.000000in}{0.000000in}}%
\pgfpathlineto{\pgfqpoint{0.000000in}{-0.048611in}}%
\pgfusepath{stroke,fill}%
}%
\begin{pgfscope}%
\pgfsys@transformshift{1.078637in}{1.096944in}%
\pgfsys@useobject{currentmarker}{}%
\end{pgfscope}%
\end{pgfscope}%
\begin{pgfscope}%
\definecolor{textcolor}{rgb}{0.000000,0.000000,0.000000}%
\pgfsetstrokecolor{textcolor}%
\pgfsetfillcolor{textcolor}%
\pgftext[x=1.113359in, y=0.467222in, left, base,rotate=90.000000]{\color{textcolor}\rmfamily\fontsize{10.000000}{12.000000}\selectfont (0.2, 0.3]}%
\end{pgfscope}%
\begin{pgfscope}%
\pgfsetbuttcap%
\pgfsetroundjoin%
\definecolor{currentfill}{rgb}{0.000000,0.000000,0.000000}%
\pgfsetfillcolor{currentfill}%
\pgfsetlinewidth{0.803000pt}%
\definecolor{currentstroke}{rgb}{0.000000,0.000000,0.000000}%
\pgfsetstrokecolor{currentstroke}%
\pgfsetdash{}{0pt}%
\pgfsys@defobject{currentmarker}{\pgfqpoint{0.000000in}{-0.048611in}}{\pgfqpoint{0.000000in}{0.000000in}}{%
\pgfpathmoveto{\pgfqpoint{0.000000in}{0.000000in}}%
\pgfpathlineto{\pgfqpoint{0.000000in}{-0.048611in}}%
\pgfusepath{stroke,fill}%
}%
\begin{pgfscope}%
\pgfsys@transformshift{1.329142in}{1.096944in}%
\pgfsys@useobject{currentmarker}{}%
\end{pgfscope}%
\end{pgfscope}%
\begin{pgfscope}%
\definecolor{textcolor}{rgb}{0.000000,0.000000,0.000000}%
\pgfsetstrokecolor{textcolor}%
\pgfsetfillcolor{textcolor}%
\pgftext[x=1.363864in, y=0.467222in, left, base,rotate=90.000000]{\color{textcolor}\rmfamily\fontsize{10.000000}{12.000000}\selectfont (0.3, 0.4]}%
\end{pgfscope}%
\begin{pgfscope}%
\pgfsetbuttcap%
\pgfsetroundjoin%
\definecolor{currentfill}{rgb}{0.000000,0.000000,0.000000}%
\pgfsetfillcolor{currentfill}%
\pgfsetlinewidth{0.803000pt}%
\definecolor{currentstroke}{rgb}{0.000000,0.000000,0.000000}%
\pgfsetstrokecolor{currentstroke}%
\pgfsetdash{}{0pt}%
\pgfsys@defobject{currentmarker}{\pgfqpoint{0.000000in}{-0.048611in}}{\pgfqpoint{0.000000in}{0.000000in}}{%
\pgfpathmoveto{\pgfqpoint{0.000000in}{0.000000in}}%
\pgfpathlineto{\pgfqpoint{0.000000in}{-0.048611in}}%
\pgfusepath{stroke,fill}%
}%
\begin{pgfscope}%
\pgfsys@transformshift{1.579647in}{1.096944in}%
\pgfsys@useobject{currentmarker}{}%
\end{pgfscope}%
\end{pgfscope}%
\begin{pgfscope}%
\definecolor{textcolor}{rgb}{0.000000,0.000000,0.000000}%
\pgfsetstrokecolor{textcolor}%
\pgfsetfillcolor{textcolor}%
\pgftext[x=1.614369in, y=0.467222in, left, base,rotate=90.000000]{\color{textcolor}\rmfamily\fontsize{10.000000}{12.000000}\selectfont (0.4, 0.5]}%
\end{pgfscope}%
\begin{pgfscope}%
\pgfsetbuttcap%
\pgfsetroundjoin%
\definecolor{currentfill}{rgb}{0.000000,0.000000,0.000000}%
\pgfsetfillcolor{currentfill}%
\pgfsetlinewidth{0.803000pt}%
\definecolor{currentstroke}{rgb}{0.000000,0.000000,0.000000}%
\pgfsetstrokecolor{currentstroke}%
\pgfsetdash{}{0pt}%
\pgfsys@defobject{currentmarker}{\pgfqpoint{0.000000in}{-0.048611in}}{\pgfqpoint{0.000000in}{0.000000in}}{%
\pgfpathmoveto{\pgfqpoint{0.000000in}{0.000000in}}%
\pgfpathlineto{\pgfqpoint{0.000000in}{-0.048611in}}%
\pgfusepath{stroke,fill}%
}%
\begin{pgfscope}%
\pgfsys@transformshift{1.830152in}{1.096944in}%
\pgfsys@useobject{currentmarker}{}%
\end{pgfscope}%
\end{pgfscope}%
\begin{pgfscope}%
\definecolor{textcolor}{rgb}{0.000000,0.000000,0.000000}%
\pgfsetstrokecolor{textcolor}%
\pgfsetfillcolor{textcolor}%
\pgftext[x=1.864874in, y=0.467222in, left, base,rotate=90.000000]{\color{textcolor}\rmfamily\fontsize{10.000000}{12.000000}\selectfont (0.5, 0.6]}%
\end{pgfscope}%
\begin{pgfscope}%
\pgfsetbuttcap%
\pgfsetroundjoin%
\definecolor{currentfill}{rgb}{0.000000,0.000000,0.000000}%
\pgfsetfillcolor{currentfill}%
\pgfsetlinewidth{0.803000pt}%
\definecolor{currentstroke}{rgb}{0.000000,0.000000,0.000000}%
\pgfsetstrokecolor{currentstroke}%
\pgfsetdash{}{0pt}%
\pgfsys@defobject{currentmarker}{\pgfqpoint{0.000000in}{-0.048611in}}{\pgfqpoint{0.000000in}{0.000000in}}{%
\pgfpathmoveto{\pgfqpoint{0.000000in}{0.000000in}}%
\pgfpathlineto{\pgfqpoint{0.000000in}{-0.048611in}}%
\pgfusepath{stroke,fill}%
}%
\begin{pgfscope}%
\pgfsys@transformshift{2.080657in}{1.096944in}%
\pgfsys@useobject{currentmarker}{}%
\end{pgfscope}%
\end{pgfscope}%
\begin{pgfscope}%
\definecolor{textcolor}{rgb}{0.000000,0.000000,0.000000}%
\pgfsetstrokecolor{textcolor}%
\pgfsetfillcolor{textcolor}%
\pgftext[x=2.115379in, y=0.467222in, left, base,rotate=90.000000]{\color{textcolor}\rmfamily\fontsize{10.000000}{12.000000}\selectfont (0.6, 0.7]}%
\end{pgfscope}%
\begin{pgfscope}%
\pgfsetbuttcap%
\pgfsetroundjoin%
\definecolor{currentfill}{rgb}{0.000000,0.000000,0.000000}%
\pgfsetfillcolor{currentfill}%
\pgfsetlinewidth{0.803000pt}%
\definecolor{currentstroke}{rgb}{0.000000,0.000000,0.000000}%
\pgfsetstrokecolor{currentstroke}%
\pgfsetdash{}{0pt}%
\pgfsys@defobject{currentmarker}{\pgfqpoint{0.000000in}{-0.048611in}}{\pgfqpoint{0.000000in}{0.000000in}}{%
\pgfpathmoveto{\pgfqpoint{0.000000in}{0.000000in}}%
\pgfpathlineto{\pgfqpoint{0.000000in}{-0.048611in}}%
\pgfusepath{stroke,fill}%
}%
\begin{pgfscope}%
\pgfsys@transformshift{2.331162in}{1.096944in}%
\pgfsys@useobject{currentmarker}{}%
\end{pgfscope}%
\end{pgfscope}%
\begin{pgfscope}%
\definecolor{textcolor}{rgb}{0.000000,0.000000,0.000000}%
\pgfsetstrokecolor{textcolor}%
\pgfsetfillcolor{textcolor}%
\pgftext[x=2.365884in, y=0.467222in, left, base,rotate=90.000000]{\color{textcolor}\rmfamily\fontsize{10.000000}{12.000000}\selectfont (0.7, 0.8]}%
\end{pgfscope}%
\begin{pgfscope}%
\pgfsetbuttcap%
\pgfsetroundjoin%
\definecolor{currentfill}{rgb}{0.000000,0.000000,0.000000}%
\pgfsetfillcolor{currentfill}%
\pgfsetlinewidth{0.803000pt}%
\definecolor{currentstroke}{rgb}{0.000000,0.000000,0.000000}%
\pgfsetstrokecolor{currentstroke}%
\pgfsetdash{}{0pt}%
\pgfsys@defobject{currentmarker}{\pgfqpoint{0.000000in}{-0.048611in}}{\pgfqpoint{0.000000in}{0.000000in}}{%
\pgfpathmoveto{\pgfqpoint{0.000000in}{0.000000in}}%
\pgfpathlineto{\pgfqpoint{0.000000in}{-0.048611in}}%
\pgfusepath{stroke,fill}%
}%
\begin{pgfscope}%
\pgfsys@transformshift{2.581667in}{1.096944in}%
\pgfsys@useobject{currentmarker}{}%
\end{pgfscope}%
\end{pgfscope}%
\begin{pgfscope}%
\definecolor{textcolor}{rgb}{0.000000,0.000000,0.000000}%
\pgfsetstrokecolor{textcolor}%
\pgfsetfillcolor{textcolor}%
\pgftext[x=2.616389in, y=0.467222in, left, base,rotate=90.000000]{\color{textcolor}\rmfamily\fontsize{10.000000}{12.000000}\selectfont (0.8, 0.9]}%
\end{pgfscope}%
\begin{pgfscope}%
\pgfsetbuttcap%
\pgfsetroundjoin%
\definecolor{currentfill}{rgb}{0.000000,0.000000,0.000000}%
\pgfsetfillcolor{currentfill}%
\pgfsetlinewidth{0.803000pt}%
\definecolor{currentstroke}{rgb}{0.000000,0.000000,0.000000}%
\pgfsetstrokecolor{currentstroke}%
\pgfsetdash{}{0pt}%
\pgfsys@defobject{currentmarker}{\pgfqpoint{0.000000in}{-0.048611in}}{\pgfqpoint{0.000000in}{0.000000in}}{%
\pgfpathmoveto{\pgfqpoint{0.000000in}{0.000000in}}%
\pgfpathlineto{\pgfqpoint{0.000000in}{-0.048611in}}%
\pgfusepath{stroke,fill}%
}%
\begin{pgfscope}%
\pgfsys@transformshift{2.832172in}{1.096944in}%
\pgfsys@useobject{currentmarker}{}%
\end{pgfscope}%
\end{pgfscope}%
\begin{pgfscope}%
\definecolor{textcolor}{rgb}{0.000000,0.000000,0.000000}%
\pgfsetstrokecolor{textcolor}%
\pgfsetfillcolor{textcolor}%
\pgftext[x=2.866894in, y=0.467222in, left, base,rotate=90.000000]{\color{textcolor}\rmfamily\fontsize{10.000000}{12.000000}\selectfont (0.9, 1.0]}%
\end{pgfscope}%
\begin{pgfscope}%
\definecolor{textcolor}{rgb}{0.000000,0.000000,0.000000}%
\pgfsetstrokecolor{textcolor}%
\pgfsetfillcolor{textcolor}%
\pgftext[x=1.755000in,y=0.226527in,,top]{\color{textcolor}\rmfamily\fontsize{10.000000}{12.000000}\selectfont Range of Prediction}%
\end{pgfscope}%
\begin{pgfscope}%
\pgfsetbuttcap%
\pgfsetroundjoin%
\definecolor{currentfill}{rgb}{0.000000,0.000000,0.000000}%
\pgfsetfillcolor{currentfill}%
\pgfsetlinewidth{0.803000pt}%
\definecolor{currentstroke}{rgb}{0.000000,0.000000,0.000000}%
\pgfsetstrokecolor{currentstroke}%
\pgfsetdash{}{0pt}%
\pgfsys@defobject{currentmarker}{\pgfqpoint{-0.048611in}{0.000000in}}{\pgfqpoint{-0.000000in}{0.000000in}}{%
\pgfpathmoveto{\pgfqpoint{-0.000000in}{0.000000in}}%
\pgfpathlineto{\pgfqpoint{-0.048611in}{0.000000in}}%
\pgfusepath{stroke,fill}%
}%
\begin{pgfscope}%
\pgfsys@transformshift{0.515000in}{1.096944in}%
\pgfsys@useobject{currentmarker}{}%
\end{pgfscope}%
\end{pgfscope}%
\begin{pgfscope}%
\definecolor{textcolor}{rgb}{0.000000,0.000000,0.000000}%
\pgfsetstrokecolor{textcolor}%
\pgfsetfillcolor{textcolor}%
\pgftext[x=0.348333in, y=1.048750in, left, base]{\color{textcolor}\rmfamily\fontsize{10.000000}{12.000000}\selectfont \(\displaystyle {0}\)}%
\end{pgfscope}%
\begin{pgfscope}%
\pgfsetbuttcap%
\pgfsetroundjoin%
\definecolor{currentfill}{rgb}{0.000000,0.000000,0.000000}%
\pgfsetfillcolor{currentfill}%
\pgfsetlinewidth{0.803000pt}%
\definecolor{currentstroke}{rgb}{0.000000,0.000000,0.000000}%
\pgfsetstrokecolor{currentstroke}%
\pgfsetdash{}{0pt}%
\pgfsys@defobject{currentmarker}{\pgfqpoint{-0.048611in}{0.000000in}}{\pgfqpoint{-0.000000in}{0.000000in}}{%
\pgfpathmoveto{\pgfqpoint{-0.000000in}{0.000000in}}%
\pgfpathlineto{\pgfqpoint{-0.048611in}{0.000000in}}%
\pgfusepath{stroke,fill}%
}%
\begin{pgfscope}%
\pgfsys@transformshift{0.515000in}{1.581312in}%
\pgfsys@useobject{currentmarker}{}%
\end{pgfscope}%
\end{pgfscope}%
\begin{pgfscope}%
\definecolor{textcolor}{rgb}{0.000000,0.000000,0.000000}%
\pgfsetstrokecolor{textcolor}%
\pgfsetfillcolor{textcolor}%
\pgftext[x=0.278889in, y=1.533118in, left, base]{\color{textcolor}\rmfamily\fontsize{10.000000}{12.000000}\selectfont \(\displaystyle {10}\)}%
\end{pgfscope}%
\begin{pgfscope}%
\pgfsetbuttcap%
\pgfsetroundjoin%
\definecolor{currentfill}{rgb}{0.000000,0.000000,0.000000}%
\pgfsetfillcolor{currentfill}%
\pgfsetlinewidth{0.803000pt}%
\definecolor{currentstroke}{rgb}{0.000000,0.000000,0.000000}%
\pgfsetstrokecolor{currentstroke}%
\pgfsetdash{}{0pt}%
\pgfsys@defobject{currentmarker}{\pgfqpoint{-0.048611in}{0.000000in}}{\pgfqpoint{-0.000000in}{0.000000in}}{%
\pgfpathmoveto{\pgfqpoint{-0.000000in}{0.000000in}}%
\pgfpathlineto{\pgfqpoint{-0.048611in}{0.000000in}}%
\pgfusepath{stroke,fill}%
}%
\begin{pgfscope}%
\pgfsys@transformshift{0.515000in}{2.065680in}%
\pgfsys@useobject{currentmarker}{}%
\end{pgfscope}%
\end{pgfscope}%
\begin{pgfscope}%
\definecolor{textcolor}{rgb}{0.000000,0.000000,0.000000}%
\pgfsetstrokecolor{textcolor}%
\pgfsetfillcolor{textcolor}%
\pgftext[x=0.278889in, y=2.017486in, left, base]{\color{textcolor}\rmfamily\fontsize{10.000000}{12.000000}\selectfont \(\displaystyle {20}\)}%
\end{pgfscope}%
\begin{pgfscope}%
\pgfsetbuttcap%
\pgfsetroundjoin%
\definecolor{currentfill}{rgb}{0.000000,0.000000,0.000000}%
\pgfsetfillcolor{currentfill}%
\pgfsetlinewidth{0.803000pt}%
\definecolor{currentstroke}{rgb}{0.000000,0.000000,0.000000}%
\pgfsetstrokecolor{currentstroke}%
\pgfsetdash{}{0pt}%
\pgfsys@defobject{currentmarker}{\pgfqpoint{-0.048611in}{0.000000in}}{\pgfqpoint{-0.000000in}{0.000000in}}{%
\pgfpathmoveto{\pgfqpoint{-0.000000in}{0.000000in}}%
\pgfpathlineto{\pgfqpoint{-0.048611in}{0.000000in}}%
\pgfusepath{stroke,fill}%
}%
\begin{pgfscope}%
\pgfsys@transformshift{0.515000in}{2.550049in}%
\pgfsys@useobject{currentmarker}{}%
\end{pgfscope}%
\end{pgfscope}%
\begin{pgfscope}%
\definecolor{textcolor}{rgb}{0.000000,0.000000,0.000000}%
\pgfsetstrokecolor{textcolor}%
\pgfsetfillcolor{textcolor}%
\pgftext[x=0.278889in, y=2.501854in, left, base]{\color{textcolor}\rmfamily\fontsize{10.000000}{12.000000}\selectfont \(\displaystyle {30}\)}%
\end{pgfscope}%
\begin{pgfscope}%
\definecolor{textcolor}{rgb}{0.000000,0.000000,0.000000}%
\pgfsetstrokecolor{textcolor}%
\pgfsetfillcolor{textcolor}%
\pgftext[x=0.223333in,y=2.020944in,,bottom,rotate=90.000000]{\color{textcolor}\rmfamily\fontsize{10.000000}{12.000000}\selectfont Percent of Data Set}%
\end{pgfscope}%
\begin{pgfscope}%
\pgfsetrectcap%
\pgfsetmiterjoin%
\pgfsetlinewidth{0.803000pt}%
\definecolor{currentstroke}{rgb}{0.000000,0.000000,0.000000}%
\pgfsetstrokecolor{currentstroke}%
\pgfsetdash{}{0pt}%
\pgfpathmoveto{\pgfqpoint{0.515000in}{1.096944in}}%
\pgfpathlineto{\pgfqpoint{0.515000in}{2.944944in}}%
\pgfusepath{stroke}%
\end{pgfscope}%
\begin{pgfscope}%
\pgfsetrectcap%
\pgfsetmiterjoin%
\pgfsetlinewidth{0.803000pt}%
\definecolor{currentstroke}{rgb}{0.000000,0.000000,0.000000}%
\pgfsetstrokecolor{currentstroke}%
\pgfsetdash{}{0pt}%
\pgfpathmoveto{\pgfqpoint{2.995000in}{1.096944in}}%
\pgfpathlineto{\pgfqpoint{2.995000in}{2.944944in}}%
\pgfusepath{stroke}%
\end{pgfscope}%
\begin{pgfscope}%
\pgfsetrectcap%
\pgfsetmiterjoin%
\pgfsetlinewidth{0.803000pt}%
\definecolor{currentstroke}{rgb}{0.000000,0.000000,0.000000}%
\pgfsetstrokecolor{currentstroke}%
\pgfsetdash{}{0pt}%
\pgfpathmoveto{\pgfqpoint{0.515000in}{1.096944in}}%
\pgfpathlineto{\pgfqpoint{2.995000in}{1.096944in}}%
\pgfusepath{stroke}%
\end{pgfscope}%
\begin{pgfscope}%
\pgfsetrectcap%
\pgfsetmiterjoin%
\pgfsetlinewidth{0.803000pt}%
\definecolor{currentstroke}{rgb}{0.000000,0.000000,0.000000}%
\pgfsetstrokecolor{currentstroke}%
\pgfsetdash{}{0pt}%
\pgfpathmoveto{\pgfqpoint{0.515000in}{2.944944in}}%
\pgfpathlineto{\pgfqpoint{2.995000in}{2.944944in}}%
\pgfusepath{stroke}%
\end{pgfscope}%
\begin{pgfscope}%
\pgfsetbuttcap%
\pgfsetmiterjoin%
\definecolor{currentfill}{rgb}{1.000000,1.000000,1.000000}%
\pgfsetfillcolor{currentfill}%
\pgfsetfillopacity{0.800000}%
\pgfsetlinewidth{1.003750pt}%
\definecolor{currentstroke}{rgb}{0.800000,0.800000,0.800000}%
\pgfsetstrokecolor{currentstroke}%
\pgfsetstrokeopacity{0.800000}%
\pgfsetdash{}{0pt}%
\pgfpathmoveto{\pgfqpoint{1.560833in}{2.444250in}}%
\pgfpathlineto{\pgfqpoint{2.897778in}{2.444250in}}%
\pgfpathquadraticcurveto{\pgfqpoint{2.925556in}{2.444250in}}{\pgfqpoint{2.925556in}{2.472028in}}%
\pgfpathlineto{\pgfqpoint{2.925556in}{2.847722in}}%
\pgfpathquadraticcurveto{\pgfqpoint{2.925556in}{2.875500in}}{\pgfqpoint{2.897778in}{2.875500in}}%
\pgfpathlineto{\pgfqpoint{1.560833in}{2.875500in}}%
\pgfpathquadraticcurveto{\pgfqpoint{1.533056in}{2.875500in}}{\pgfqpoint{1.533056in}{2.847722in}}%
\pgfpathlineto{\pgfqpoint{1.533056in}{2.472028in}}%
\pgfpathquadraticcurveto{\pgfqpoint{1.533056in}{2.444250in}}{\pgfqpoint{1.560833in}{2.444250in}}%
\pgfpathlineto{\pgfqpoint{1.560833in}{2.444250in}}%
\pgfpathclose%
\pgfusepath{stroke,fill}%
\end{pgfscope}%
\begin{pgfscope}%
\pgfsetbuttcap%
\pgfsetmiterjoin%
\pgfsetlinewidth{1.003750pt}%
\definecolor{currentstroke}{rgb}{0.000000,0.000000,0.000000}%
\pgfsetstrokecolor{currentstroke}%
\pgfsetdash{}{0pt}%
\pgfpathmoveto{\pgfqpoint{1.588611in}{2.722027in}}%
\pgfpathlineto{\pgfqpoint{1.866389in}{2.722027in}}%
\pgfpathlineto{\pgfqpoint{1.866389in}{2.819250in}}%
\pgfpathlineto{\pgfqpoint{1.588611in}{2.819250in}}%
\pgfpathlineto{\pgfqpoint{1.588611in}{2.722027in}}%
\pgfpathclose%
\pgfusepath{stroke}%
\end{pgfscope}%
\begin{pgfscope}%
\definecolor{textcolor}{rgb}{0.000000,0.000000,0.000000}%
\pgfsetstrokecolor{textcolor}%
\pgfsetfillcolor{textcolor}%
\pgftext[x=1.977500in,y=2.722027in,left,base]{\color{textcolor}\rmfamily\fontsize{10.000000}{12.000000}\selectfont Negative Class}%
\end{pgfscope}%
\begin{pgfscope}%
\pgfsetbuttcap%
\pgfsetmiterjoin%
\definecolor{currentfill}{rgb}{0.000000,0.000000,0.000000}%
\pgfsetfillcolor{currentfill}%
\pgfsetlinewidth{0.000000pt}%
\definecolor{currentstroke}{rgb}{0.000000,0.000000,0.000000}%
\pgfsetstrokecolor{currentstroke}%
\pgfsetstrokeopacity{0.000000}%
\pgfsetdash{}{0pt}%
\pgfpathmoveto{\pgfqpoint{1.588611in}{2.526750in}}%
\pgfpathlineto{\pgfqpoint{1.866389in}{2.526750in}}%
\pgfpathlineto{\pgfqpoint{1.866389in}{2.623972in}}%
\pgfpathlineto{\pgfqpoint{1.588611in}{2.623972in}}%
\pgfpathlineto{\pgfqpoint{1.588611in}{2.526750in}}%
\pgfpathclose%
\pgfusepath{fill}%
\end{pgfscope}%
\begin{pgfscope}%
\definecolor{textcolor}{rgb}{0.000000,0.000000,0.000000}%
\pgfsetstrokecolor{textcolor}%
\pgfsetfillcolor{textcolor}%
\pgftext[x=1.977500in,y=2.526750in,left,base]{\color{textcolor}\rmfamily\fontsize{10.000000}{12.000000}\selectfont Positive Class}%
\end{pgfscope}%
\end{pgfpicture}%
\makeatother%
\endgroup%

  &
  \vspace{0pt} %% Creator: Matplotlib, PGF backend
%%
%% To include the figure in your LaTeX document, write
%%   \input{<filename>.pgf}
%%
%% Make sure the required packages are loaded in your preamble
%%   \usepackage{pgf}
%%
%% Also ensure that all the required font packages are loaded; for instance,
%% the lmodern package is sometimes necessary when using math font.
%%   \usepackage{lmodern}
%%
%% Figures using additional raster images can only be included by \input if
%% they are in the same directory as the main LaTeX file. For loading figures
%% from other directories you can use the `import` package
%%   \usepackage{import}
%%
%% and then include the figures with
%%   \import{<path to file>}{<filename>.pgf}
%%
%% Matplotlib used the following preamble
%%   
%%   \usepackage{fontspec}
%%   \makeatletter\@ifpackageloaded{underscore}{}{\usepackage[strings]{underscore}}\makeatother
%%
\begingroup%
\makeatletter%
\begin{pgfpicture}%
\pgfpathrectangle{\pgfpointorigin}{\pgfqpoint{2.221861in}{1.953444in}}%
\pgfusepath{use as bounding box, clip}%
\begin{pgfscope}%
\pgfsetbuttcap%
\pgfsetmiterjoin%
\definecolor{currentfill}{rgb}{1.000000,1.000000,1.000000}%
\pgfsetfillcolor{currentfill}%
\pgfsetlinewidth{0.000000pt}%
\definecolor{currentstroke}{rgb}{1.000000,1.000000,1.000000}%
\pgfsetstrokecolor{currentstroke}%
\pgfsetdash{}{0pt}%
\pgfpathmoveto{\pgfqpoint{0.000000in}{0.000000in}}%
\pgfpathlineto{\pgfqpoint{2.221861in}{0.000000in}}%
\pgfpathlineto{\pgfqpoint{2.221861in}{1.953444in}}%
\pgfpathlineto{\pgfqpoint{0.000000in}{1.953444in}}%
\pgfpathlineto{\pgfqpoint{0.000000in}{0.000000in}}%
\pgfpathclose%
\pgfusepath{fill}%
\end{pgfscope}%
\begin{pgfscope}%
\pgfsetbuttcap%
\pgfsetmiterjoin%
\definecolor{currentfill}{rgb}{1.000000,1.000000,1.000000}%
\pgfsetfillcolor{currentfill}%
\pgfsetlinewidth{0.000000pt}%
\definecolor{currentstroke}{rgb}{0.000000,0.000000,0.000000}%
\pgfsetstrokecolor{currentstroke}%
\pgfsetstrokeopacity{0.000000}%
\pgfsetdash{}{0pt}%
\pgfpathmoveto{\pgfqpoint{0.553581in}{0.499444in}}%
\pgfpathlineto{\pgfqpoint{2.103581in}{0.499444in}}%
\pgfpathlineto{\pgfqpoint{2.103581in}{1.654444in}}%
\pgfpathlineto{\pgfqpoint{0.553581in}{1.654444in}}%
\pgfpathlineto{\pgfqpoint{0.553581in}{0.499444in}}%
\pgfpathclose%
\pgfusepath{fill}%
\end{pgfscope}%
\begin{pgfscope}%
\pgfsetbuttcap%
\pgfsetroundjoin%
\definecolor{currentfill}{rgb}{0.000000,0.000000,0.000000}%
\pgfsetfillcolor{currentfill}%
\pgfsetlinewidth{0.803000pt}%
\definecolor{currentstroke}{rgb}{0.000000,0.000000,0.000000}%
\pgfsetstrokecolor{currentstroke}%
\pgfsetdash{}{0pt}%
\pgfsys@defobject{currentmarker}{\pgfqpoint{0.000000in}{-0.048611in}}{\pgfqpoint{0.000000in}{0.000000in}}{%
\pgfpathmoveto{\pgfqpoint{0.000000in}{0.000000in}}%
\pgfpathlineto{\pgfqpoint{0.000000in}{-0.048611in}}%
\pgfusepath{stroke,fill}%
}%
\begin{pgfscope}%
\pgfsys@transformshift{0.624035in}{0.499444in}%
\pgfsys@useobject{currentmarker}{}%
\end{pgfscope}%
\end{pgfscope}%
\begin{pgfscope}%
\definecolor{textcolor}{rgb}{0.000000,0.000000,0.000000}%
\pgfsetstrokecolor{textcolor}%
\pgfsetfillcolor{textcolor}%
\pgftext[x=0.624035in,y=0.402222in,,top]{\color{textcolor}\rmfamily\fontsize{10.000000}{12.000000}\selectfont \(\displaystyle {0.0}\)}%
\end{pgfscope}%
\begin{pgfscope}%
\pgfsetbuttcap%
\pgfsetroundjoin%
\definecolor{currentfill}{rgb}{0.000000,0.000000,0.000000}%
\pgfsetfillcolor{currentfill}%
\pgfsetlinewidth{0.803000pt}%
\definecolor{currentstroke}{rgb}{0.000000,0.000000,0.000000}%
\pgfsetstrokecolor{currentstroke}%
\pgfsetdash{}{0pt}%
\pgfsys@defobject{currentmarker}{\pgfqpoint{0.000000in}{-0.048611in}}{\pgfqpoint{0.000000in}{0.000000in}}{%
\pgfpathmoveto{\pgfqpoint{0.000000in}{0.000000in}}%
\pgfpathlineto{\pgfqpoint{0.000000in}{-0.048611in}}%
\pgfusepath{stroke,fill}%
}%
\begin{pgfscope}%
\pgfsys@transformshift{1.328581in}{0.499444in}%
\pgfsys@useobject{currentmarker}{}%
\end{pgfscope}%
\end{pgfscope}%
\begin{pgfscope}%
\definecolor{textcolor}{rgb}{0.000000,0.000000,0.000000}%
\pgfsetstrokecolor{textcolor}%
\pgfsetfillcolor{textcolor}%
\pgftext[x=1.328581in,y=0.402222in,,top]{\color{textcolor}\rmfamily\fontsize{10.000000}{12.000000}\selectfont \(\displaystyle {0.5}\)}%
\end{pgfscope}%
\begin{pgfscope}%
\pgfsetbuttcap%
\pgfsetroundjoin%
\definecolor{currentfill}{rgb}{0.000000,0.000000,0.000000}%
\pgfsetfillcolor{currentfill}%
\pgfsetlinewidth{0.803000pt}%
\definecolor{currentstroke}{rgb}{0.000000,0.000000,0.000000}%
\pgfsetstrokecolor{currentstroke}%
\pgfsetdash{}{0pt}%
\pgfsys@defobject{currentmarker}{\pgfqpoint{0.000000in}{-0.048611in}}{\pgfqpoint{0.000000in}{0.000000in}}{%
\pgfpathmoveto{\pgfqpoint{0.000000in}{0.000000in}}%
\pgfpathlineto{\pgfqpoint{0.000000in}{-0.048611in}}%
\pgfusepath{stroke,fill}%
}%
\begin{pgfscope}%
\pgfsys@transformshift{2.033126in}{0.499444in}%
\pgfsys@useobject{currentmarker}{}%
\end{pgfscope}%
\end{pgfscope}%
\begin{pgfscope}%
\definecolor{textcolor}{rgb}{0.000000,0.000000,0.000000}%
\pgfsetstrokecolor{textcolor}%
\pgfsetfillcolor{textcolor}%
\pgftext[x=2.033126in,y=0.402222in,,top]{\color{textcolor}\rmfamily\fontsize{10.000000}{12.000000}\selectfont \(\displaystyle {1.0}\)}%
\end{pgfscope}%
\begin{pgfscope}%
\definecolor{textcolor}{rgb}{0.000000,0.000000,0.000000}%
\pgfsetstrokecolor{textcolor}%
\pgfsetfillcolor{textcolor}%
\pgftext[x=1.328581in,y=0.223333in,,top]{\color{textcolor}\rmfamily\fontsize{10.000000}{12.000000}\selectfont False positive rate}%
\end{pgfscope}%
\begin{pgfscope}%
\pgfsetbuttcap%
\pgfsetroundjoin%
\definecolor{currentfill}{rgb}{0.000000,0.000000,0.000000}%
\pgfsetfillcolor{currentfill}%
\pgfsetlinewidth{0.803000pt}%
\definecolor{currentstroke}{rgb}{0.000000,0.000000,0.000000}%
\pgfsetstrokecolor{currentstroke}%
\pgfsetdash{}{0pt}%
\pgfsys@defobject{currentmarker}{\pgfqpoint{-0.048611in}{0.000000in}}{\pgfqpoint{-0.000000in}{0.000000in}}{%
\pgfpathmoveto{\pgfqpoint{-0.000000in}{0.000000in}}%
\pgfpathlineto{\pgfqpoint{-0.048611in}{0.000000in}}%
\pgfusepath{stroke,fill}%
}%
\begin{pgfscope}%
\pgfsys@transformshift{0.553581in}{0.551944in}%
\pgfsys@useobject{currentmarker}{}%
\end{pgfscope}%
\end{pgfscope}%
\begin{pgfscope}%
\definecolor{textcolor}{rgb}{0.000000,0.000000,0.000000}%
\pgfsetstrokecolor{textcolor}%
\pgfsetfillcolor{textcolor}%
\pgftext[x=0.278889in, y=0.503750in, left, base]{\color{textcolor}\rmfamily\fontsize{10.000000}{12.000000}\selectfont \(\displaystyle {0.0}\)}%
\end{pgfscope}%
\begin{pgfscope}%
\pgfsetbuttcap%
\pgfsetroundjoin%
\definecolor{currentfill}{rgb}{0.000000,0.000000,0.000000}%
\pgfsetfillcolor{currentfill}%
\pgfsetlinewidth{0.803000pt}%
\definecolor{currentstroke}{rgb}{0.000000,0.000000,0.000000}%
\pgfsetstrokecolor{currentstroke}%
\pgfsetdash{}{0pt}%
\pgfsys@defobject{currentmarker}{\pgfqpoint{-0.048611in}{0.000000in}}{\pgfqpoint{-0.000000in}{0.000000in}}{%
\pgfpathmoveto{\pgfqpoint{-0.000000in}{0.000000in}}%
\pgfpathlineto{\pgfqpoint{-0.048611in}{0.000000in}}%
\pgfusepath{stroke,fill}%
}%
\begin{pgfscope}%
\pgfsys@transformshift{0.553581in}{1.076944in}%
\pgfsys@useobject{currentmarker}{}%
\end{pgfscope}%
\end{pgfscope}%
\begin{pgfscope}%
\definecolor{textcolor}{rgb}{0.000000,0.000000,0.000000}%
\pgfsetstrokecolor{textcolor}%
\pgfsetfillcolor{textcolor}%
\pgftext[x=0.278889in, y=1.028750in, left, base]{\color{textcolor}\rmfamily\fontsize{10.000000}{12.000000}\selectfont \(\displaystyle {0.5}\)}%
\end{pgfscope}%
\begin{pgfscope}%
\pgfsetbuttcap%
\pgfsetroundjoin%
\definecolor{currentfill}{rgb}{0.000000,0.000000,0.000000}%
\pgfsetfillcolor{currentfill}%
\pgfsetlinewidth{0.803000pt}%
\definecolor{currentstroke}{rgb}{0.000000,0.000000,0.000000}%
\pgfsetstrokecolor{currentstroke}%
\pgfsetdash{}{0pt}%
\pgfsys@defobject{currentmarker}{\pgfqpoint{-0.048611in}{0.000000in}}{\pgfqpoint{-0.000000in}{0.000000in}}{%
\pgfpathmoveto{\pgfqpoint{-0.000000in}{0.000000in}}%
\pgfpathlineto{\pgfqpoint{-0.048611in}{0.000000in}}%
\pgfusepath{stroke,fill}%
}%
\begin{pgfscope}%
\pgfsys@transformshift{0.553581in}{1.601944in}%
\pgfsys@useobject{currentmarker}{}%
\end{pgfscope}%
\end{pgfscope}%
\begin{pgfscope}%
\definecolor{textcolor}{rgb}{0.000000,0.000000,0.000000}%
\pgfsetstrokecolor{textcolor}%
\pgfsetfillcolor{textcolor}%
\pgftext[x=0.278889in, y=1.553750in, left, base]{\color{textcolor}\rmfamily\fontsize{10.000000}{12.000000}\selectfont \(\displaystyle {1.0}\)}%
\end{pgfscope}%
\begin{pgfscope}%
\definecolor{textcolor}{rgb}{0.000000,0.000000,0.000000}%
\pgfsetstrokecolor{textcolor}%
\pgfsetfillcolor{textcolor}%
\pgftext[x=0.223333in,y=1.076944in,,bottom,rotate=90.000000]{\color{textcolor}\rmfamily\fontsize{10.000000}{12.000000}\selectfont True positive rate}%
\end{pgfscope}%
\begin{pgfscope}%
\pgfpathrectangle{\pgfqpoint{0.553581in}{0.499444in}}{\pgfqpoint{1.550000in}{1.155000in}}%
\pgfusepath{clip}%
\pgfsetbuttcap%
\pgfsetroundjoin%
\pgfsetlinewidth{1.505625pt}%
\definecolor{currentstroke}{rgb}{0.000000,0.000000,0.000000}%
\pgfsetstrokecolor{currentstroke}%
\pgfsetdash{{5.550000pt}{2.400000pt}}{0.000000pt}%
\pgfpathmoveto{\pgfqpoint{0.624035in}{0.551944in}}%
\pgfpathlineto{\pgfqpoint{2.033126in}{1.601944in}}%
\pgfusepath{stroke}%
\end{pgfscope}%
\begin{pgfscope}%
\pgfpathrectangle{\pgfqpoint{0.553581in}{0.499444in}}{\pgfqpoint{1.550000in}{1.155000in}}%
\pgfusepath{clip}%
\pgfsetrectcap%
\pgfsetroundjoin%
\pgfsetlinewidth{1.505625pt}%
\definecolor{currentstroke}{rgb}{0.000000,0.000000,0.000000}%
\pgfsetstrokecolor{currentstroke}%
\pgfsetdash{}{0pt}%
\pgfpathmoveto{\pgfqpoint{0.624035in}{0.551944in}}%
\pgfpathlineto{\pgfqpoint{0.626125in}{0.552364in}}%
\pgfpathlineto{\pgfqpoint{0.627217in}{0.562339in}}%
\pgfpathlineto{\pgfqpoint{0.627957in}{0.563442in}}%
\pgfpathlineto{\pgfqpoint{0.629002in}{0.565332in}}%
\pgfpathlineto{\pgfqpoint{0.629766in}{0.566434in}}%
\pgfpathlineto{\pgfqpoint{0.630858in}{0.569584in}}%
\pgfpathlineto{\pgfqpoint{0.631092in}{0.570477in}}%
\pgfpathlineto{\pgfqpoint{0.632184in}{0.574047in}}%
\pgfpathlineto{\pgfqpoint{0.632560in}{0.575149in}}%
\pgfpathlineto{\pgfqpoint{0.633664in}{0.579822in}}%
\pgfpathlineto{\pgfqpoint{0.633958in}{0.580924in}}%
\pgfpathlineto{\pgfqpoint{0.635061in}{0.585439in}}%
\pgfpathlineto{\pgfqpoint{0.635261in}{0.586332in}}%
\pgfpathlineto{\pgfqpoint{0.636341in}{0.592107in}}%
\pgfpathlineto{\pgfqpoint{0.636564in}{0.593209in}}%
\pgfpathlineto{\pgfqpoint{0.637656in}{0.600349in}}%
\pgfpathlineto{\pgfqpoint{0.637844in}{0.601347in}}%
\pgfpathlineto{\pgfqpoint{0.638948in}{0.607174in}}%
\pgfpathlineto{\pgfqpoint{0.639066in}{0.608277in}}%
\pgfpathlineto{\pgfqpoint{0.640134in}{0.614472in}}%
\pgfpathlineto{\pgfqpoint{0.640510in}{0.615574in}}%
\pgfpathlineto{\pgfqpoint{0.641590in}{0.622609in}}%
\pgfpathlineto{\pgfqpoint{0.641790in}{0.623712in}}%
\pgfpathlineto{\pgfqpoint{0.642847in}{0.631639in}}%
\pgfpathlineto{\pgfqpoint{0.643105in}{0.632637in}}%
\pgfpathlineto{\pgfqpoint{0.644209in}{0.641457in}}%
\pgfpathlineto{\pgfqpoint{0.644373in}{0.642559in}}%
\pgfpathlineto{\pgfqpoint{0.645477in}{0.650067in}}%
\pgfpathlineto{\pgfqpoint{0.645559in}{0.650644in}}%
\pgfpathlineto{\pgfqpoint{0.646651in}{0.658414in}}%
\pgfpathlineto{\pgfqpoint{0.646839in}{0.659464in}}%
\pgfpathlineto{\pgfqpoint{0.647943in}{0.665764in}}%
\pgfpathlineto{\pgfqpoint{0.648119in}{0.666814in}}%
\pgfpathlineto{\pgfqpoint{0.649223in}{0.674322in}}%
\pgfpathlineto{\pgfqpoint{0.649399in}{0.675424in}}%
\pgfpathlineto{\pgfqpoint{0.650503in}{0.684244in}}%
\pgfpathlineto{\pgfqpoint{0.650667in}{0.685189in}}%
\pgfpathlineto{\pgfqpoint{0.651736in}{0.692802in}}%
\pgfpathlineto{\pgfqpoint{0.651982in}{0.693747in}}%
\pgfpathlineto{\pgfqpoint{0.653086in}{0.702882in}}%
\pgfpathlineto{\pgfqpoint{0.653333in}{0.703932in}}%
\pgfpathlineto{\pgfqpoint{0.654425in}{0.711807in}}%
\pgfpathlineto{\pgfqpoint{0.654519in}{0.712332in}}%
\pgfpathlineto{\pgfqpoint{0.655622in}{0.721204in}}%
\pgfpathlineto{\pgfqpoint{0.655763in}{0.722202in}}%
\pgfpathlineto{\pgfqpoint{0.656855in}{0.729184in}}%
\pgfpathlineto{\pgfqpoint{0.657043in}{0.730234in}}%
\pgfpathlineto{\pgfqpoint{0.658135in}{0.738424in}}%
\pgfpathlineto{\pgfqpoint{0.658194in}{0.739369in}}%
\pgfpathlineto{\pgfqpoint{0.659298in}{0.747874in}}%
\pgfpathlineto{\pgfqpoint{0.659474in}{0.748819in}}%
\pgfpathlineto{\pgfqpoint{0.660578in}{0.757954in}}%
\pgfpathlineto{\pgfqpoint{0.660754in}{0.759057in}}%
\pgfpathlineto{\pgfqpoint{0.661846in}{0.765357in}}%
\pgfpathlineto{\pgfqpoint{0.662104in}{0.766302in}}%
\pgfpathlineto{\pgfqpoint{0.663208in}{0.774807in}}%
\pgfpathlineto{\pgfqpoint{0.663349in}{0.775909in}}%
\pgfpathlineto{\pgfqpoint{0.664453in}{0.782367in}}%
\pgfpathlineto{\pgfqpoint{0.664758in}{0.783417in}}%
\pgfpathlineto{\pgfqpoint{0.665862in}{0.791292in}}%
\pgfpathlineto{\pgfqpoint{0.666179in}{0.792237in}}%
\pgfpathlineto{\pgfqpoint{0.667283in}{0.800059in}}%
\pgfpathlineto{\pgfqpoint{0.667506in}{0.801162in}}%
\pgfpathlineto{\pgfqpoint{0.668609in}{0.808039in}}%
\pgfpathlineto{\pgfqpoint{0.668868in}{0.809089in}}%
\pgfpathlineto{\pgfqpoint{0.669972in}{0.815337in}}%
\pgfpathlineto{\pgfqpoint{0.670066in}{0.816387in}}%
\pgfpathlineto{\pgfqpoint{0.671169in}{0.823107in}}%
\pgfpathlineto{\pgfqpoint{0.671439in}{0.824209in}}%
\pgfpathlineto{\pgfqpoint{0.672543in}{0.833134in}}%
\pgfpathlineto{\pgfqpoint{0.672731in}{0.834237in}}%
\pgfpathlineto{\pgfqpoint{0.673823in}{0.840747in}}%
\pgfpathlineto{\pgfqpoint{0.674011in}{0.841797in}}%
\pgfpathlineto{\pgfqpoint{0.675115in}{0.846942in}}%
\pgfpathlineto{\pgfqpoint{0.675338in}{0.847992in}}%
\pgfpathlineto{\pgfqpoint{0.676442in}{0.853872in}}%
\pgfpathlineto{\pgfqpoint{0.676583in}{0.854974in}}%
\pgfpathlineto{\pgfqpoint{0.677663in}{0.861012in}}%
\pgfpathlineto{\pgfqpoint{0.677992in}{0.862009in}}%
\pgfpathlineto{\pgfqpoint{0.679072in}{0.869307in}}%
\pgfpathlineto{\pgfqpoint{0.679389in}{0.870409in}}%
\pgfpathlineto{\pgfqpoint{0.680493in}{0.877077in}}%
\pgfpathlineto{\pgfqpoint{0.680704in}{0.878127in}}%
\pgfpathlineto{\pgfqpoint{0.681773in}{0.882852in}}%
\pgfpathlineto{\pgfqpoint{0.681996in}{0.883954in}}%
\pgfpathlineto{\pgfqpoint{0.683100in}{0.890622in}}%
\pgfpathlineto{\pgfqpoint{0.683511in}{0.891672in}}%
\pgfpathlineto{\pgfqpoint{0.684603in}{0.896607in}}%
\pgfpathlineto{\pgfqpoint{0.684779in}{0.897657in}}%
\pgfpathlineto{\pgfqpoint{0.685836in}{0.903379in}}%
\pgfpathlineto{\pgfqpoint{0.686153in}{0.904482in}}%
\pgfpathlineto{\pgfqpoint{0.687256in}{0.909837in}}%
\pgfpathlineto{\pgfqpoint{0.687573in}{0.910939in}}%
\pgfpathlineto{\pgfqpoint{0.688572in}{0.915874in}}%
\pgfpathlineto{\pgfqpoint{0.688889in}{0.916977in}}%
\pgfpathlineto{\pgfqpoint{0.689992in}{0.922647in}}%
\pgfpathlineto{\pgfqpoint{0.690157in}{0.923749in}}%
\pgfpathlineto{\pgfqpoint{0.691261in}{0.930259in}}%
\pgfpathlineto{\pgfqpoint{0.691589in}{0.931362in}}%
\pgfpathlineto{\pgfqpoint{0.692670in}{0.936717in}}%
\pgfpathlineto{\pgfqpoint{0.692916in}{0.937819in}}%
\pgfpathlineto{\pgfqpoint{0.694020in}{0.942859in}}%
\pgfpathlineto{\pgfqpoint{0.694443in}{0.943962in}}%
\pgfpathlineto{\pgfqpoint{0.695500in}{0.948477in}}%
\pgfpathlineto{\pgfqpoint{0.695758in}{0.949579in}}%
\pgfpathlineto{\pgfqpoint{0.696862in}{0.954567in}}%
\pgfpathlineto{\pgfqpoint{0.697202in}{0.955617in}}%
\pgfpathlineto{\pgfqpoint{0.698306in}{0.961182in}}%
\pgfpathlineto{\pgfqpoint{0.698752in}{0.962232in}}%
\pgfpathlineto{\pgfqpoint{0.699856in}{0.966799in}}%
\pgfpathlineto{\pgfqpoint{0.700114in}{0.967902in}}%
\pgfpathlineto{\pgfqpoint{0.701218in}{0.972889in}}%
\pgfpathlineto{\pgfqpoint{0.701441in}{0.973939in}}%
\pgfpathlineto{\pgfqpoint{0.702545in}{0.978087in}}%
\pgfpathlineto{\pgfqpoint{0.702921in}{0.979189in}}%
\pgfpathlineto{\pgfqpoint{0.703989in}{0.983652in}}%
\pgfpathlineto{\pgfqpoint{0.704342in}{0.984754in}}%
\pgfpathlineto{\pgfqpoint{0.705434in}{0.989322in}}%
\pgfpathlineto{\pgfqpoint{0.705680in}{0.990424in}}%
\pgfpathlineto{\pgfqpoint{0.706772in}{0.995674in}}%
\pgfpathlineto{\pgfqpoint{0.707101in}{0.996724in}}%
\pgfpathlineto{\pgfqpoint{0.708205in}{1.001502in}}%
\pgfpathlineto{\pgfqpoint{0.708545in}{1.002604in}}%
\pgfpathlineto{\pgfqpoint{0.709649in}{1.006594in}}%
\pgfpathlineto{\pgfqpoint{0.709978in}{1.007592in}}%
\pgfpathlineto{\pgfqpoint{0.711023in}{1.010637in}}%
\pgfpathlineto{\pgfqpoint{0.711434in}{1.011739in}}%
\pgfpathlineto{\pgfqpoint{0.712503in}{1.016149in}}%
\pgfpathlineto{\pgfqpoint{0.712867in}{1.017252in}}%
\pgfpathlineto{\pgfqpoint{0.713970in}{1.020822in}}%
\pgfpathlineto{\pgfqpoint{0.714323in}{1.021819in}}%
\pgfpathlineto{\pgfqpoint{0.715427in}{1.025127in}}%
\pgfpathlineto{\pgfqpoint{0.715884in}{1.026124in}}%
\pgfpathlineto{\pgfqpoint{0.716977in}{1.029957in}}%
\pgfpathlineto{\pgfqpoint{0.717247in}{1.030954in}}%
\pgfpathlineto{\pgfqpoint{0.718339in}{1.036624in}}%
\pgfpathlineto{\pgfqpoint{0.718573in}{1.037674in}}%
\pgfpathlineto{\pgfqpoint{0.719677in}{1.041664in}}%
\pgfpathlineto{\pgfqpoint{0.720147in}{1.042767in}}%
\pgfpathlineto{\pgfqpoint{0.721227in}{1.045707in}}%
\pgfpathlineto{\pgfqpoint{0.721580in}{1.046757in}}%
\pgfpathlineto{\pgfqpoint{0.722660in}{1.050222in}}%
\pgfpathlineto{\pgfqpoint{0.723047in}{1.051324in}}%
\pgfpathlineto{\pgfqpoint{0.724139in}{1.054789in}}%
\pgfpathlineto{\pgfqpoint{0.724644in}{1.055892in}}%
\pgfpathlineto{\pgfqpoint{0.725736in}{1.059777in}}%
\pgfpathlineto{\pgfqpoint{0.726159in}{1.060827in}}%
\pgfpathlineto{\pgfqpoint{0.727263in}{1.063924in}}%
\pgfpathlineto{\pgfqpoint{0.727603in}{1.064974in}}%
\pgfpathlineto{\pgfqpoint{0.728672in}{1.068124in}}%
\pgfpathlineto{\pgfqpoint{0.729200in}{1.069227in}}%
\pgfpathlineto{\pgfqpoint{0.730304in}{1.072167in}}%
\pgfpathlineto{\pgfqpoint{0.730551in}{1.073269in}}%
\pgfpathlineto{\pgfqpoint{0.731631in}{1.076157in}}%
\pgfpathlineto{\pgfqpoint{0.732054in}{1.077259in}}%
\pgfpathlineto{\pgfqpoint{0.733122in}{1.080567in}}%
\pgfpathlineto{\pgfqpoint{0.733756in}{1.081669in}}%
\pgfpathlineto{\pgfqpoint{0.734848in}{1.084662in}}%
\pgfpathlineto{\pgfqpoint{0.735166in}{1.085764in}}%
\pgfpathlineto{\pgfqpoint{0.736258in}{1.089229in}}%
\pgfpathlineto{\pgfqpoint{0.736739in}{1.090332in}}%
\pgfpathlineto{\pgfqpoint{0.737808in}{1.093272in}}%
\pgfpathlineto{\pgfqpoint{0.738442in}{1.094374in}}%
\pgfpathlineto{\pgfqpoint{0.739522in}{1.098049in}}%
\pgfpathlineto{\pgfqpoint{0.739921in}{1.099152in}}%
\pgfpathlineto{\pgfqpoint{0.741025in}{1.101987in}}%
\pgfpathlineto{\pgfqpoint{0.741636in}{1.103089in}}%
\pgfpathlineto{\pgfqpoint{0.742739in}{1.105872in}}%
\pgfpathlineto{\pgfqpoint{0.743268in}{1.106974in}}%
\pgfpathlineto{\pgfqpoint{0.744360in}{1.109704in}}%
\pgfpathlineto{\pgfqpoint{0.744712in}{1.110754in}}%
\pgfpathlineto{\pgfqpoint{0.744747in}{1.110912in}}%
\pgfpathlineto{\pgfqpoint{0.744759in}{1.110912in}}%
\pgfpathlineto{\pgfqpoint{0.755363in}{1.112014in}}%
\pgfpathlineto{\pgfqpoint{0.756466in}{1.114377in}}%
\pgfpathlineto{\pgfqpoint{0.756772in}{1.115479in}}%
\pgfpathlineto{\pgfqpoint{0.757875in}{1.119627in}}%
\pgfpathlineto{\pgfqpoint{0.758592in}{1.120729in}}%
\pgfpathlineto{\pgfqpoint{0.759672in}{1.122882in}}%
\pgfpathlineto{\pgfqpoint{0.760153in}{1.123984in}}%
\pgfpathlineto{\pgfqpoint{0.761234in}{1.126819in}}%
\pgfpathlineto{\pgfqpoint{0.761727in}{1.127922in}}%
\pgfpathlineto{\pgfqpoint{0.762831in}{1.131387in}}%
\pgfpathlineto{\pgfqpoint{0.763312in}{1.132437in}}%
\pgfpathlineto{\pgfqpoint{0.764416in}{1.135219in}}%
\pgfpathlineto{\pgfqpoint{0.764897in}{1.136217in}}%
\pgfpathlineto{\pgfqpoint{0.765989in}{1.139104in}}%
\pgfpathlineto{\pgfqpoint{0.766389in}{1.140154in}}%
\pgfpathlineto{\pgfqpoint{0.767492in}{1.143199in}}%
\pgfpathlineto{\pgfqpoint{0.767997in}{1.144302in}}%
\pgfpathlineto{\pgfqpoint{0.769019in}{1.146349in}}%
\pgfpathlineto{\pgfqpoint{0.769547in}{1.147452in}}%
\pgfpathlineto{\pgfqpoint{0.770639in}{1.150182in}}%
\pgfpathlineto{\pgfqpoint{0.771273in}{1.151284in}}%
\pgfpathlineto{\pgfqpoint{0.772330in}{1.154014in}}%
\pgfpathlineto{\pgfqpoint{0.773129in}{1.155117in}}%
\pgfpathlineto{\pgfqpoint{0.774174in}{1.158004in}}%
\pgfpathlineto{\pgfqpoint{0.774855in}{1.159054in}}%
\pgfpathlineto{\pgfqpoint{0.775959in}{1.161364in}}%
\pgfpathlineto{\pgfqpoint{0.776170in}{1.162467in}}%
\pgfpathlineto{\pgfqpoint{0.777274in}{1.165197in}}%
\pgfpathlineto{\pgfqpoint{0.777779in}{1.166299in}}%
\pgfpathlineto{\pgfqpoint{0.778883in}{1.168819in}}%
\pgfpathlineto{\pgfqpoint{0.779481in}{1.169922in}}%
\pgfpathlineto{\pgfqpoint{0.780585in}{1.171339in}}%
\pgfpathlineto{\pgfqpoint{0.781008in}{1.172389in}}%
\pgfpathlineto{\pgfqpoint{0.782065in}{1.175119in}}%
\pgfpathlineto{\pgfqpoint{0.782581in}{1.176169in}}%
\pgfpathlineto{\pgfqpoint{0.783673in}{1.177797in}}%
\pgfpathlineto{\pgfqpoint{0.784261in}{1.178899in}}%
\pgfpathlineto{\pgfqpoint{0.785247in}{1.180684in}}%
\pgfpathlineto{\pgfqpoint{0.785822in}{1.181787in}}%
\pgfpathlineto{\pgfqpoint{0.786903in}{1.183887in}}%
\pgfpathlineto{\pgfqpoint{0.787830in}{1.184937in}}%
\pgfpathlineto{\pgfqpoint{0.788817in}{1.186879in}}%
\pgfpathlineto{\pgfqpoint{0.789556in}{1.187982in}}%
\pgfpathlineto{\pgfqpoint{0.790519in}{1.190029in}}%
\pgfpathlineto{\pgfqpoint{0.791353in}{1.191132in}}%
\pgfpathlineto{\pgfqpoint{0.792445in}{1.193442in}}%
\pgfpathlineto{\pgfqpoint{0.792997in}{1.194544in}}%
\pgfpathlineto{\pgfqpoint{0.794054in}{1.196067in}}%
\pgfpathlineto{\pgfqpoint{0.794876in}{1.197169in}}%
\pgfpathlineto{\pgfqpoint{0.795886in}{1.198482in}}%
\pgfpathlineto{\pgfqpoint{0.797201in}{1.199584in}}%
\pgfpathlineto{\pgfqpoint{0.798258in}{1.201579in}}%
\pgfpathlineto{\pgfqpoint{0.798739in}{1.202682in}}%
\pgfpathlineto{\pgfqpoint{0.799843in}{1.204572in}}%
\pgfpathlineto{\pgfqpoint{0.800195in}{1.205674in}}%
\pgfpathlineto{\pgfqpoint{0.801287in}{1.207564in}}%
\pgfpathlineto{\pgfqpoint{0.801874in}{1.208667in}}%
\pgfpathlineto{\pgfqpoint{0.802966in}{1.211239in}}%
\pgfpathlineto{\pgfqpoint{0.803882in}{1.212342in}}%
\pgfpathlineto{\pgfqpoint{0.804986in}{1.214337in}}%
\pgfpathlineto{\pgfqpoint{0.805714in}{1.215439in}}%
\pgfpathlineto{\pgfqpoint{0.806759in}{1.217277in}}%
\pgfpathlineto{\pgfqpoint{0.807111in}{1.218274in}}%
\pgfpathlineto{\pgfqpoint{0.808180in}{1.221109in}}%
\pgfpathlineto{\pgfqpoint{0.809084in}{1.222212in}}%
\pgfpathlineto{\pgfqpoint{0.810176in}{1.224049in}}%
\pgfpathlineto{\pgfqpoint{0.810892in}{1.225152in}}%
\pgfpathlineto{\pgfqpoint{0.811938in}{1.226517in}}%
\pgfpathlineto{\pgfqpoint{0.812478in}{1.227567in}}%
\pgfpathlineto{\pgfqpoint{0.813570in}{1.229614in}}%
\pgfpathlineto{\pgfqpoint{0.814122in}{1.230717in}}%
\pgfpathlineto{\pgfqpoint{0.815225in}{1.233027in}}%
\pgfpathlineto{\pgfqpoint{0.815942in}{1.234129in}}%
\pgfpathlineto{\pgfqpoint{0.816940in}{1.235809in}}%
\pgfpathlineto{\pgfqpoint{0.817820in}{1.236859in}}%
\pgfpathlineto{\pgfqpoint{0.818889in}{1.238697in}}%
\pgfpathlineto{\pgfqpoint{0.819429in}{1.239799in}}%
\pgfpathlineto{\pgfqpoint{0.820486in}{1.241689in}}%
\pgfpathlineto{\pgfqpoint{0.820897in}{1.242792in}}%
\pgfpathlineto{\pgfqpoint{0.821977in}{1.244367in}}%
\pgfpathlineto{\pgfqpoint{0.822881in}{1.245469in}}%
\pgfpathlineto{\pgfqpoint{0.823985in}{1.247464in}}%
\pgfpathlineto{\pgfqpoint{0.824655in}{1.248567in}}%
\pgfpathlineto{\pgfqpoint{0.825711in}{1.249827in}}%
\pgfpathlineto{\pgfqpoint{0.826369in}{1.250929in}}%
\pgfpathlineto{\pgfqpoint{0.827438in}{1.252767in}}%
\pgfpathlineto{\pgfqpoint{0.828001in}{1.253869in}}%
\pgfpathlineto{\pgfqpoint{0.829105in}{1.255812in}}%
\pgfpathlineto{\pgfqpoint{0.829845in}{1.256914in}}%
\pgfpathlineto{\pgfqpoint{0.830948in}{1.258804in}}%
\pgfpathlineto{\pgfqpoint{0.831712in}{1.259907in}}%
\pgfpathlineto{\pgfqpoint{0.832816in}{1.261797in}}%
\pgfpathlineto{\pgfqpoint{0.833861in}{1.262899in}}%
\pgfpathlineto{\pgfqpoint{0.834953in}{1.264422in}}%
\pgfpathlineto{\pgfqpoint{0.835739in}{1.265472in}}%
\pgfpathlineto{\pgfqpoint{0.836784in}{1.267257in}}%
\pgfpathlineto{\pgfqpoint{0.837477in}{1.268359in}}%
\pgfpathlineto{\pgfqpoint{0.838558in}{1.269724in}}%
\pgfpathlineto{\pgfqpoint{0.839920in}{1.270774in}}%
\pgfpathlineto{\pgfqpoint{0.841023in}{1.272664in}}%
\pgfpathlineto{\pgfqpoint{0.841975in}{1.273767in}}%
\pgfpathlineto{\pgfqpoint{0.843055in}{1.274869in}}%
\pgfpathlineto{\pgfqpoint{0.844217in}{1.275972in}}%
\pgfpathlineto{\pgfqpoint{0.845298in}{1.277809in}}%
\pgfpathlineto{\pgfqpoint{0.846002in}{1.278912in}}%
\pgfpathlineto{\pgfqpoint{0.847094in}{1.280172in}}%
\pgfpathlineto{\pgfqpoint{0.847963in}{1.281274in}}%
\pgfpathlineto{\pgfqpoint{0.849067in}{1.282902in}}%
\pgfpathlineto{\pgfqpoint{0.850124in}{1.284004in}}%
\pgfpathlineto{\pgfqpoint{0.851228in}{1.285474in}}%
\pgfpathlineto{\pgfqpoint{0.852472in}{1.286577in}}%
\pgfpathlineto{\pgfqpoint{0.853576in}{1.288624in}}%
\pgfpathlineto{\pgfqpoint{0.854645in}{1.289727in}}%
\pgfpathlineto{\pgfqpoint{0.855690in}{1.291197in}}%
\pgfpathlineto{\pgfqpoint{0.856934in}{1.292299in}}%
\pgfpathlineto{\pgfqpoint{0.857874in}{1.293297in}}%
\pgfpathlineto{\pgfqpoint{0.858766in}{1.294399in}}%
\pgfpathlineto{\pgfqpoint{0.859858in}{1.295817in}}%
\pgfpathlineto{\pgfqpoint{0.860868in}{1.296919in}}%
\pgfpathlineto{\pgfqpoint{0.861960in}{1.298179in}}%
\pgfpathlineto{\pgfqpoint{0.863205in}{1.299282in}}%
\pgfpathlineto{\pgfqpoint{0.864285in}{1.300962in}}%
\pgfpathlineto{\pgfqpoint{0.864837in}{1.302064in}}%
\pgfpathlineto{\pgfqpoint{0.865894in}{1.303482in}}%
\pgfpathlineto{\pgfqpoint{0.867009in}{1.304584in}}%
\pgfpathlineto{\pgfqpoint{0.868113in}{1.305582in}}%
\pgfpathlineto{\pgfqpoint{0.869205in}{1.306684in}}%
\pgfpathlineto{\pgfqpoint{0.870286in}{1.307682in}}%
\pgfpathlineto{\pgfqpoint{0.871131in}{1.308784in}}%
\pgfpathlineto{\pgfqpoint{0.872211in}{1.310464in}}%
\pgfpathlineto{\pgfqpoint{0.873092in}{1.311567in}}%
\pgfpathlineto{\pgfqpoint{0.874196in}{1.312932in}}%
\pgfpathlineto{\pgfqpoint{0.875523in}{1.314034in}}%
\pgfpathlineto{\pgfqpoint{0.876568in}{1.315189in}}%
\pgfpathlineto{\pgfqpoint{0.878141in}{1.316292in}}%
\pgfpathlineto{\pgfqpoint{0.879245in}{1.317919in}}%
\pgfpathlineto{\pgfqpoint{0.880513in}{1.319022in}}%
\pgfpathlineto{\pgfqpoint{0.881570in}{1.319967in}}%
\pgfpathlineto{\pgfqpoint{0.882521in}{1.321069in}}%
\pgfpathlineto{\pgfqpoint{0.883566in}{1.321962in}}%
\pgfpathlineto{\pgfqpoint{0.884564in}{1.323064in}}%
\pgfpathlineto{\pgfqpoint{0.885656in}{1.324639in}}%
\pgfpathlineto{\pgfqpoint{0.886455in}{1.325742in}}%
\pgfpathlineto{\pgfqpoint{0.887559in}{1.327054in}}%
\pgfpathlineto{\pgfqpoint{0.889250in}{1.328157in}}%
\pgfpathlineto{\pgfqpoint{0.890342in}{1.328944in}}%
\pgfpathlineto{\pgfqpoint{0.891316in}{1.330047in}}%
\pgfpathlineto{\pgfqpoint{0.892408in}{1.331307in}}%
\pgfpathlineto{\pgfqpoint{0.893864in}{1.332409in}}%
\pgfpathlineto{\pgfqpoint{0.894956in}{1.333774in}}%
\pgfpathlineto{\pgfqpoint{0.896166in}{1.334877in}}%
\pgfpathlineto{\pgfqpoint{0.897246in}{1.336032in}}%
\pgfpathlineto{\pgfqpoint{0.898667in}{1.337134in}}%
\pgfpathlineto{\pgfqpoint{0.899712in}{1.338237in}}%
\pgfpathlineto{\pgfqpoint{0.901520in}{1.339339in}}%
\pgfpathlineto{\pgfqpoint{0.902577in}{1.340284in}}%
\pgfpathlineto{\pgfqpoint{0.903446in}{1.341334in}}%
\pgfpathlineto{\pgfqpoint{0.904538in}{1.342594in}}%
\pgfpathlineto{\pgfqpoint{0.906065in}{1.343697in}}%
\pgfpathlineto{\pgfqpoint{0.907133in}{1.344484in}}%
\pgfpathlineto{\pgfqpoint{0.908625in}{1.345534in}}%
\pgfpathlineto{\pgfqpoint{0.909576in}{1.346322in}}%
\pgfpathlineto{\pgfqpoint{0.911314in}{1.347372in}}%
\pgfpathlineto{\pgfqpoint{0.912406in}{1.348894in}}%
\pgfpathlineto{\pgfqpoint{0.913674in}{1.349997in}}%
\pgfpathlineto{\pgfqpoint{0.914742in}{1.351047in}}%
\pgfpathlineto{\pgfqpoint{0.916281in}{1.352149in}}%
\pgfpathlineto{\pgfqpoint{0.917338in}{1.353042in}}%
\pgfpathlineto{\pgfqpoint{0.918993in}{1.354144in}}%
\pgfpathlineto{\pgfqpoint{0.920073in}{1.355037in}}%
\pgfpathlineto{\pgfqpoint{0.921330in}{1.356139in}}%
\pgfpathlineto{\pgfqpoint{0.922434in}{1.357662in}}%
\pgfpathlineto{\pgfqpoint{0.924042in}{1.358764in}}%
\pgfpathlineto{\pgfqpoint{0.925123in}{1.359814in}}%
\pgfpathlineto{\pgfqpoint{0.927354in}{1.360917in}}%
\pgfpathlineto{\pgfqpoint{0.928422in}{1.361967in}}%
\pgfpathlineto{\pgfqpoint{0.930242in}{1.363069in}}%
\pgfpathlineto{\pgfqpoint{0.931334in}{1.364277in}}%
\pgfpathlineto{\pgfqpoint{0.932673in}{1.365327in}}%
\pgfpathlineto{\pgfqpoint{0.933718in}{1.366534in}}%
\pgfpathlineto{\pgfqpoint{0.935397in}{1.367637in}}%
\pgfpathlineto{\pgfqpoint{0.936454in}{1.368477in}}%
\pgfpathlineto{\pgfqpoint{0.937898in}{1.369579in}}%
\pgfpathlineto{\pgfqpoint{0.938908in}{1.370262in}}%
\pgfpathlineto{\pgfqpoint{0.940822in}{1.371364in}}%
\pgfpathlineto{\pgfqpoint{0.941691in}{1.371994in}}%
\pgfpathlineto{\pgfqpoint{0.943417in}{1.373097in}}%
\pgfpathlineto{\pgfqpoint{0.944427in}{1.373937in}}%
\pgfpathlineto{\pgfqpoint{0.945754in}{1.375039in}}%
\pgfpathlineto{\pgfqpoint{0.946799in}{1.375617in}}%
\pgfpathlineto{\pgfqpoint{0.948408in}{1.376719in}}%
\pgfpathlineto{\pgfqpoint{0.949465in}{1.377717in}}%
\pgfpathlineto{\pgfqpoint{0.951203in}{1.378819in}}%
\pgfpathlineto{\pgfqpoint{0.952271in}{1.379554in}}%
\pgfpathlineto{\pgfqpoint{0.953504in}{1.380604in}}%
\pgfpathlineto{\pgfqpoint{0.954573in}{1.381182in}}%
\pgfpathlineto{\pgfqpoint{0.955970in}{1.382179in}}%
\pgfpathlineto{\pgfqpoint{0.957003in}{1.383124in}}%
\pgfpathlineto{\pgfqpoint{0.959105in}{1.384227in}}%
\pgfpathlineto{\pgfqpoint{0.960162in}{1.384909in}}%
\pgfpathlineto{\pgfqpoint{0.962217in}{1.386012in}}%
\pgfpathlineto{\pgfqpoint{0.963297in}{1.386694in}}%
\pgfpathlineto{\pgfqpoint{0.965575in}{1.387797in}}%
\pgfpathlineto{\pgfqpoint{0.966609in}{1.388374in}}%
\pgfpathlineto{\pgfqpoint{0.968030in}{1.389372in}}%
\pgfpathlineto{\pgfqpoint{0.969122in}{1.390422in}}%
\pgfpathlineto{\pgfqpoint{0.971423in}{1.391524in}}%
\pgfpathlineto{\pgfqpoint{0.972527in}{1.392522in}}%
\pgfpathlineto{\pgfqpoint{0.974652in}{1.393624in}}%
\pgfpathlineto{\pgfqpoint{0.975697in}{1.394307in}}%
\pgfpathlineto{\pgfqpoint{0.978069in}{1.395409in}}%
\pgfpathlineto{\pgfqpoint{0.979150in}{1.396092in}}%
\pgfpathlineto{\pgfqpoint{0.980782in}{1.397194in}}%
\pgfpathlineto{\pgfqpoint{0.981862in}{1.397824in}}%
\pgfpathlineto{\pgfqpoint{0.984093in}{1.398822in}}%
\pgfpathlineto{\pgfqpoint{0.985173in}{1.399767in}}%
\pgfpathlineto{\pgfqpoint{0.987581in}{1.400869in}}%
\pgfpathlineto{\pgfqpoint{0.988649in}{1.401342in}}%
\pgfpathlineto{\pgfqpoint{0.991139in}{1.402444in}}%
\pgfpathlineto{\pgfqpoint{0.992242in}{1.402864in}}%
\pgfpathlineto{\pgfqpoint{0.993816in}{1.403914in}}%
\pgfpathlineto{\pgfqpoint{0.994697in}{1.404807in}}%
\pgfpathlineto{\pgfqpoint{0.996951in}{1.405909in}}%
\pgfpathlineto{\pgfqpoint{0.998055in}{1.406749in}}%
\pgfpathlineto{\pgfqpoint{1.000216in}{1.407852in}}%
\pgfpathlineto{\pgfqpoint{1.001284in}{1.408377in}}%
\pgfpathlineto{\pgfqpoint{1.003069in}{1.409427in}}%
\pgfpathlineto{\pgfqpoint{1.004044in}{1.410214in}}%
\pgfpathlineto{\pgfqpoint{1.005746in}{1.411264in}}%
\pgfpathlineto{\pgfqpoint{1.006838in}{1.412262in}}%
\pgfpathlineto{\pgfqpoint{1.008858in}{1.413364in}}%
\pgfpathlineto{\pgfqpoint{1.009962in}{1.413837in}}%
\pgfpathlineto{\pgfqpoint{1.012909in}{1.414939in}}%
\pgfpathlineto{\pgfqpoint{1.013790in}{1.415307in}}%
\pgfpathlineto{\pgfqpoint{1.015786in}{1.416357in}}%
\pgfpathlineto{\pgfqpoint{1.016855in}{1.417092in}}%
\pgfpathlineto{\pgfqpoint{1.018604in}{1.418194in}}%
\pgfpathlineto{\pgfqpoint{1.019708in}{1.419087in}}%
\pgfpathlineto{\pgfqpoint{1.021857in}{1.420137in}}%
\pgfpathlineto{\pgfqpoint{1.022937in}{1.420767in}}%
\pgfpathlineto{\pgfqpoint{1.025673in}{1.421869in}}%
\pgfpathlineto{\pgfqpoint{1.026683in}{1.422657in}}%
\pgfpathlineto{\pgfqpoint{1.029219in}{1.423759in}}%
\pgfpathlineto{\pgfqpoint{1.030123in}{1.424599in}}%
\pgfpathlineto{\pgfqpoint{1.033071in}{1.425702in}}%
\pgfpathlineto{\pgfqpoint{1.034092in}{1.426647in}}%
\pgfpathlineto{\pgfqpoint{1.037333in}{1.427749in}}%
\pgfpathlineto{\pgfqpoint{1.038414in}{1.428484in}}%
\pgfpathlineto{\pgfqpoint{1.039917in}{1.429587in}}%
\pgfpathlineto{\pgfqpoint{1.040962in}{1.430217in}}%
\pgfpathlineto{\pgfqpoint{1.043486in}{1.431319in}}%
\pgfpathlineto{\pgfqpoint{1.044461in}{1.431897in}}%
\pgfpathlineto{\pgfqpoint{1.046622in}{1.432999in}}%
\pgfpathlineto{\pgfqpoint{1.047479in}{1.433629in}}%
\pgfpathlineto{\pgfqpoint{1.049498in}{1.434732in}}%
\pgfpathlineto{\pgfqpoint{1.050532in}{1.435204in}}%
\pgfpathlineto{\pgfqpoint{1.052035in}{1.436307in}}%
\pgfpathlineto{\pgfqpoint{1.052786in}{1.436989in}}%
\pgfpathlineto{\pgfqpoint{1.055922in}{1.438092in}}%
\pgfpathlineto{\pgfqpoint{1.056802in}{1.438354in}}%
\pgfpathlineto{\pgfqpoint{1.059573in}{1.439457in}}%
\pgfpathlineto{\pgfqpoint{1.060548in}{1.440139in}}%
\pgfpathlineto{\pgfqpoint{1.063178in}{1.441242in}}%
\pgfpathlineto{\pgfqpoint{1.064118in}{1.441872in}}%
\pgfpathlineto{\pgfqpoint{1.067370in}{1.442974in}}%
\pgfpathlineto{\pgfqpoint{1.068416in}{1.443499in}}%
\pgfpathlineto{\pgfqpoint{1.071116in}{1.444602in}}%
\pgfpathlineto{\pgfqpoint{1.072197in}{1.445284in}}%
\pgfpathlineto{\pgfqpoint{1.074498in}{1.446387in}}%
\pgfpathlineto{\pgfqpoint{1.075555in}{1.447122in}}%
\pgfpathlineto{\pgfqpoint{1.078573in}{1.448224in}}%
\pgfpathlineto{\pgfqpoint{1.079677in}{1.449064in}}%
\pgfpathlineto{\pgfqpoint{1.083575in}{1.450167in}}%
\pgfpathlineto{\pgfqpoint{1.084362in}{1.450482in}}%
\pgfpathlineto{\pgfqpoint{1.087485in}{1.451584in}}%
\pgfpathlineto{\pgfqpoint{1.088589in}{1.452267in}}%
\pgfpathlineto{\pgfqpoint{1.091442in}{1.453369in}}%
\pgfpathlineto{\pgfqpoint{1.092452in}{1.453842in}}%
\pgfpathlineto{\pgfqpoint{1.096339in}{1.454944in}}%
\pgfpathlineto{\pgfqpoint{1.097196in}{1.455312in}}%
\pgfpathlineto{\pgfqpoint{1.099721in}{1.456414in}}%
\pgfpathlineto{\pgfqpoint{1.100766in}{1.456939in}}%
\pgfpathlineto{\pgfqpoint{1.103936in}{1.458042in}}%
\pgfpathlineto{\pgfqpoint{1.104735in}{1.458252in}}%
\pgfpathlineto{\pgfqpoint{1.108716in}{1.459354in}}%
\pgfpathlineto{\pgfqpoint{1.109702in}{1.459774in}}%
\pgfpathlineto{\pgfqpoint{1.113553in}{1.460877in}}%
\pgfpathlineto{\pgfqpoint{1.114434in}{1.461297in}}%
\pgfpathlineto{\pgfqpoint{1.117346in}{1.462399in}}%
\pgfpathlineto{\pgfqpoint{1.118156in}{1.462819in}}%
\pgfpathlineto{\pgfqpoint{1.120834in}{1.463922in}}%
\pgfpathlineto{\pgfqpoint{1.121914in}{1.464499in}}%
\pgfpathlineto{\pgfqpoint{1.125437in}{1.465602in}}%
\pgfpathlineto{\pgfqpoint{1.126400in}{1.465917in}}%
\pgfpathlineto{\pgfqpoint{1.130545in}{1.467019in}}%
\pgfpathlineto{\pgfqpoint{1.131425in}{1.467334in}}%
\pgfpathlineto{\pgfqpoint{1.134044in}{1.468384in}}%
\pgfpathlineto{\pgfqpoint{1.134925in}{1.468804in}}%
\pgfpathlineto{\pgfqpoint{1.137438in}{1.469907in}}%
\pgfpathlineto{\pgfqpoint{1.138541in}{1.470642in}}%
\pgfpathlineto{\pgfqpoint{1.142557in}{1.471744in}}%
\pgfpathlineto{\pgfqpoint{1.143156in}{1.472007in}}%
\pgfpathlineto{\pgfqpoint{1.147923in}{1.473109in}}%
\pgfpathlineto{\pgfqpoint{1.148745in}{1.473529in}}%
\pgfpathlineto{\pgfqpoint{1.151740in}{1.474632in}}%
\pgfpathlineto{\pgfqpoint{1.152820in}{1.475052in}}%
\pgfpathlineto{\pgfqpoint{1.155944in}{1.476154in}}%
\pgfpathlineto{\pgfqpoint{1.157012in}{1.476627in}}%
\pgfpathlineto{\pgfqpoint{1.161263in}{1.477729in}}%
\pgfpathlineto{\pgfqpoint{1.162144in}{1.478254in}}%
\pgfpathlineto{\pgfqpoint{1.166430in}{1.479357in}}%
\pgfpathlineto{\pgfqpoint{1.167439in}{1.479934in}}%
\pgfpathlineto{\pgfqpoint{1.170845in}{1.481037in}}%
\pgfpathlineto{\pgfqpoint{1.171725in}{1.481352in}}%
\pgfpathlineto{\pgfqpoint{1.176340in}{1.482454in}}%
\pgfpathlineto{\pgfqpoint{1.177432in}{1.482769in}}%
\pgfpathlineto{\pgfqpoint{1.182305in}{1.483872in}}%
\pgfpathlineto{\pgfqpoint{1.183386in}{1.484397in}}%
\pgfpathlineto{\pgfqpoint{1.188047in}{1.485499in}}%
\pgfpathlineto{\pgfqpoint{1.189034in}{1.486077in}}%
\pgfpathlineto{\pgfqpoint{1.194095in}{1.487179in}}%
\pgfpathlineto{\pgfqpoint{1.194964in}{1.487494in}}%
\pgfpathlineto{\pgfqpoint{1.199484in}{1.488597in}}%
\pgfpathlineto{\pgfqpoint{1.200518in}{1.489227in}}%
\pgfpathlineto{\pgfqpoint{1.203418in}{1.490329in}}%
\pgfpathlineto{\pgfqpoint{1.204252in}{1.490697in}}%
\pgfpathlineto{\pgfqpoint{1.207211in}{1.491799in}}%
\pgfpathlineto{\pgfqpoint{1.207211in}{1.491852in}}%
\pgfpathlineto{\pgfqpoint{1.211791in}{1.492954in}}%
\pgfpathlineto{\pgfqpoint{1.212789in}{1.493322in}}%
\pgfpathlineto{\pgfqpoint{1.217192in}{1.494424in}}%
\pgfpathlineto{\pgfqpoint{1.217897in}{1.494634in}}%
\pgfpathlineto{\pgfqpoint{1.223005in}{1.495737in}}%
\pgfpathlineto{\pgfqpoint{1.223897in}{1.495842in}}%
\pgfpathlineto{\pgfqpoint{1.226621in}{1.496944in}}%
\pgfpathlineto{\pgfqpoint{1.227643in}{1.497469in}}%
\pgfpathlineto{\pgfqpoint{1.235322in}{1.498572in}}%
\pgfpathlineto{\pgfqpoint{1.236367in}{1.498729in}}%
\pgfpathlineto{\pgfqpoint{1.241757in}{1.499832in}}%
\pgfpathlineto{\pgfqpoint{1.242133in}{1.499989in}}%
\pgfpathlineto{\pgfqpoint{1.248063in}{1.501092in}}%
\pgfpathlineto{\pgfqpoint{1.248991in}{1.501354in}}%
\pgfpathlineto{\pgfqpoint{1.254028in}{1.502457in}}%
\pgfpathlineto{\pgfqpoint{1.255108in}{1.502614in}}%
\pgfpathlineto{\pgfqpoint{1.259946in}{1.503717in}}%
\pgfpathlineto{\pgfqpoint{1.260933in}{1.503927in}}%
\pgfpathlineto{\pgfqpoint{1.267696in}{1.505029in}}%
\pgfpathlineto{\pgfqpoint{1.268765in}{1.505344in}}%
\pgfpathlineto{\pgfqpoint{1.274248in}{1.506447in}}%
\pgfpathlineto{\pgfqpoint{1.274918in}{1.506552in}}%
\pgfpathlineto{\pgfqpoint{1.282128in}{1.507654in}}%
\pgfpathlineto{\pgfqpoint{1.283008in}{1.508127in}}%
\pgfpathlineto{\pgfqpoint{1.289420in}{1.509229in}}%
\pgfpathlineto{\pgfqpoint{1.290465in}{1.509649in}}%
\pgfpathlineto{\pgfqpoint{1.296723in}{1.510752in}}%
\pgfpathlineto{\pgfqpoint{1.297722in}{1.511014in}}%
\pgfpathlineto{\pgfqpoint{1.303393in}{1.512117in}}%
\pgfpathlineto{\pgfqpoint{1.304485in}{1.512379in}}%
\pgfpathlineto{\pgfqpoint{1.311084in}{1.513429in}}%
\pgfpathlineto{\pgfqpoint{1.311460in}{1.513587in}}%
\pgfpathlineto{\pgfqpoint{1.319445in}{1.514689in}}%
\pgfpathlineto{\pgfqpoint{1.320220in}{1.515109in}}%
\pgfpathlineto{\pgfqpoint{1.326338in}{1.516212in}}%
\pgfpathlineto{\pgfqpoint{1.327301in}{1.516474in}}%
\pgfpathlineto{\pgfqpoint{1.334041in}{1.517577in}}%
\pgfpathlineto{\pgfqpoint{1.334886in}{1.517839in}}%
\pgfpathlineto{\pgfqpoint{1.340664in}{1.518942in}}%
\pgfpathlineto{\pgfqpoint{1.340664in}{1.518994in}}%
\pgfpathlineto{\pgfqpoint{1.350304in}{1.520097in}}%
\pgfpathlineto{\pgfqpoint{1.351279in}{1.520464in}}%
\pgfpathlineto{\pgfqpoint{1.358007in}{1.521567in}}%
\pgfpathlineto{\pgfqpoint{1.358888in}{1.521724in}}%
\pgfpathlineto{\pgfqpoint{1.365264in}{1.522827in}}%
\pgfpathlineto{\pgfqpoint{1.366098in}{1.523037in}}%
\pgfpathlineto{\pgfqpoint{1.372697in}{1.524139in}}%
\pgfpathlineto{\pgfqpoint{1.373413in}{1.524349in}}%
\pgfpathlineto{\pgfqpoint{1.380964in}{1.525452in}}%
\pgfpathlineto{\pgfqpoint{1.382044in}{1.525977in}}%
\pgfpathlineto{\pgfqpoint{1.387222in}{1.527079in}}%
\pgfpathlineto{\pgfqpoint{1.387974in}{1.527237in}}%
\pgfpathlineto{\pgfqpoint{1.394338in}{1.528339in}}%
\pgfpathlineto{\pgfqpoint{1.395442in}{1.528654in}}%
\pgfpathlineto{\pgfqpoint{1.400174in}{1.529757in}}%
\pgfpathlineto{\pgfqpoint{1.400527in}{1.529914in}}%
\pgfpathlineto{\pgfqpoint{1.408605in}{1.531017in}}%
\pgfpathlineto{\pgfqpoint{1.409615in}{1.531227in}}%
\pgfpathlineto{\pgfqpoint{1.417201in}{1.532329in}}%
\pgfpathlineto{\pgfqpoint{1.418258in}{1.532539in}}%
\pgfpathlineto{\pgfqpoint{1.423964in}{1.533642in}}%
\pgfpathlineto{\pgfqpoint{1.424822in}{1.533852in}}%
\pgfpathlineto{\pgfqpoint{1.432008in}{1.534954in}}%
\pgfpathlineto{\pgfqpoint{1.432266in}{1.535112in}}%
\pgfpathlineto{\pgfqpoint{1.442353in}{1.536214in}}%
\pgfpathlineto{\pgfqpoint{1.442564in}{1.536319in}}%
\pgfpathlineto{\pgfqpoint{1.449504in}{1.537422in}}%
\pgfpathlineto{\pgfqpoint{1.450267in}{1.537842in}}%
\pgfpathlineto{\pgfqpoint{1.458534in}{1.538944in}}%
\pgfpathlineto{\pgfqpoint{1.459356in}{1.539102in}}%
\pgfpathlineto{\pgfqpoint{1.473083in}{1.540204in}}%
\pgfpathlineto{\pgfqpoint{1.473306in}{1.540362in}}%
\pgfpathlineto{\pgfqpoint{1.482829in}{1.541464in}}%
\pgfpathlineto{\pgfqpoint{1.483710in}{1.541569in}}%
\pgfpathlineto{\pgfqpoint{1.495405in}{1.542672in}}%
\pgfpathlineto{\pgfqpoint{1.496157in}{1.542934in}}%
\pgfpathlineto{\pgfqpoint{1.504470in}{1.544037in}}%
\pgfpathlineto{\pgfqpoint{1.505410in}{1.544299in}}%
\pgfpathlineto{\pgfqpoint{1.515121in}{1.545402in}}%
\pgfpathlineto{\pgfqpoint{1.516213in}{1.545559in}}%
\pgfpathlineto{\pgfqpoint{1.523775in}{1.546662in}}%
\pgfpathlineto{\pgfqpoint{1.524515in}{1.546819in}}%
\pgfpathlineto{\pgfqpoint{1.532006in}{1.547922in}}%
\pgfpathlineto{\pgfqpoint{1.532958in}{1.548079in}}%
\pgfpathlineto{\pgfqpoint{1.541835in}{1.549182in}}%
\pgfpathlineto{\pgfqpoint{1.542492in}{1.549287in}}%
\pgfpathlineto{\pgfqpoint{1.548258in}{1.550389in}}%
\pgfpathlineto{\pgfqpoint{1.549021in}{1.550652in}}%
\pgfpathlineto{\pgfqpoint{1.558955in}{1.551754in}}%
\pgfpathlineto{\pgfqpoint{1.559859in}{1.552069in}}%
\pgfpathlineto{\pgfqpoint{1.571132in}{1.553172in}}%
\pgfpathlineto{\pgfqpoint{1.571895in}{1.553329in}}%
\pgfpathlineto{\pgfqpoint{1.581783in}{1.554432in}}%
\pgfpathlineto{\pgfqpoint{1.582652in}{1.554589in}}%
\pgfpathlineto{\pgfqpoint{1.594253in}{1.555692in}}%
\pgfpathlineto{\pgfqpoint{1.595322in}{1.555902in}}%
\pgfpathlineto{\pgfqpoint{1.605314in}{1.557004in}}%
\pgfpathlineto{\pgfqpoint{1.605890in}{1.557162in}}%
\pgfpathlineto{\pgfqpoint{1.619534in}{1.558264in}}%
\pgfpathlineto{\pgfqpoint{1.620333in}{1.558474in}}%
\pgfpathlineto{\pgfqpoint{1.630772in}{1.559577in}}%
\pgfpathlineto{\pgfqpoint{1.630819in}{1.559682in}}%
\pgfpathlineto{\pgfqpoint{1.642738in}{1.560784in}}%
\pgfpathlineto{\pgfqpoint{1.643677in}{1.561047in}}%
\pgfpathlineto{\pgfqpoint{1.656793in}{1.562097in}}%
\pgfpathlineto{\pgfqpoint{1.657592in}{1.562307in}}%
\pgfpathlineto{\pgfqpoint{1.669041in}{1.563409in}}%
\pgfpathlineto{\pgfqpoint{1.669698in}{1.563619in}}%
\pgfpathlineto{\pgfqpoint{1.683989in}{1.564722in}}%
\pgfpathlineto{\pgfqpoint{1.685010in}{1.564879in}}%
\pgfpathlineto{\pgfqpoint{1.700311in}{1.565982in}}%
\pgfpathlineto{\pgfqpoint{1.700311in}{1.566034in}}%
\pgfpathlineto{\pgfqpoint{1.717666in}{1.567137in}}%
\pgfpathlineto{\pgfqpoint{1.718006in}{1.567294in}}%
\pgfpathlineto{\pgfqpoint{1.730125in}{1.568397in}}%
\pgfpathlineto{\pgfqpoint{1.730747in}{1.568712in}}%
\pgfpathlineto{\pgfqpoint{1.744298in}{1.569814in}}%
\pgfpathlineto{\pgfqpoint{1.745155in}{1.569972in}}%
\pgfpathlineto{\pgfqpoint{1.760138in}{1.571074in}}%
\pgfpathlineto{\pgfqpoint{1.760913in}{1.571389in}}%
\pgfpathlineto{\pgfqpoint{1.773677in}{1.572492in}}%
\pgfpathlineto{\pgfqpoint{1.774734in}{1.572754in}}%
\pgfpathlineto{\pgfqpoint{1.792406in}{1.573857in}}%
\pgfpathlineto{\pgfqpoint{1.793240in}{1.574067in}}%
\pgfpathlineto{\pgfqpoint{1.804043in}{1.575169in}}%
\pgfpathlineto{\pgfqpoint{1.804278in}{1.575274in}}%
\pgfpathlineto{\pgfqpoint{1.819355in}{1.576377in}}%
\pgfpathlineto{\pgfqpoint{1.819813in}{1.576534in}}%
\pgfpathlineto{\pgfqpoint{1.832014in}{1.577637in}}%
\pgfpathlineto{\pgfqpoint{1.833094in}{1.577794in}}%
\pgfpathlineto{\pgfqpoint{1.850508in}{1.578897in}}%
\pgfpathlineto{\pgfqpoint{1.850966in}{1.579002in}}%
\pgfpathlineto{\pgfqpoint{1.865315in}{1.580104in}}%
\pgfpathlineto{\pgfqpoint{1.866219in}{1.580367in}}%
\pgfpathlineto{\pgfqpoint{1.878279in}{1.581469in}}%
\pgfpathlineto{\pgfqpoint{1.878948in}{1.581574in}}%
\pgfpathlineto{\pgfqpoint{1.891454in}{1.582677in}}%
\pgfpathlineto{\pgfqpoint{1.892147in}{1.582887in}}%
\pgfpathlineto{\pgfqpoint{1.905075in}{1.583989in}}%
\pgfpathlineto{\pgfqpoint{1.906155in}{1.584199in}}%
\pgfpathlineto{\pgfqpoint{1.919694in}{1.585302in}}%
\pgfpathlineto{\pgfqpoint{1.920704in}{1.585512in}}%
\pgfpathlineto{\pgfqpoint{1.935558in}{1.586614in}}%
\pgfpathlineto{\pgfqpoint{1.936392in}{1.586772in}}%
\pgfpathlineto{\pgfqpoint{1.948369in}{1.587874in}}%
\pgfpathlineto{\pgfqpoint{1.949379in}{1.588242in}}%
\pgfpathlineto{\pgfqpoint{1.964691in}{1.589344in}}%
\pgfpathlineto{\pgfqpoint{1.965008in}{1.589554in}}%
\pgfpathlineto{\pgfqpoint{1.978606in}{1.590657in}}%
\pgfpathlineto{\pgfqpoint{1.979675in}{1.590867in}}%
\pgfpathlineto{\pgfqpoint{1.990513in}{1.591969in}}%
\pgfpathlineto{\pgfqpoint{1.990513in}{1.592022in}}%
\pgfpathlineto{\pgfqpoint{2.000118in}{1.593124in}}%
\pgfpathlineto{\pgfqpoint{2.001210in}{1.593387in}}%
\pgfpathlineto{\pgfqpoint{2.010581in}{1.594489in}}%
\pgfpathlineto{\pgfqpoint{2.011320in}{1.594699in}}%
\pgfpathlineto{\pgfqpoint{2.021783in}{1.595802in}}%
\pgfpathlineto{\pgfqpoint{2.022441in}{1.596169in}}%
\pgfpathlineto{\pgfqpoint{2.027243in}{1.597272in}}%
\pgfpathlineto{\pgfqpoint{2.028053in}{1.597902in}}%
\pgfpathlineto{\pgfqpoint{2.031001in}{1.599004in}}%
\pgfpathlineto{\pgfqpoint{2.031834in}{1.599424in}}%
\pgfpathlineto{\pgfqpoint{2.033044in}{1.600527in}}%
\pgfpathlineto{\pgfqpoint{2.033126in}{1.601944in}}%
\pgfpathlineto{\pgfqpoint{2.033126in}{1.601944in}}%
\pgfusepath{stroke}%
\end{pgfscope}%
\begin{pgfscope}%
\pgfsetrectcap%
\pgfsetmiterjoin%
\pgfsetlinewidth{0.803000pt}%
\definecolor{currentstroke}{rgb}{0.000000,0.000000,0.000000}%
\pgfsetstrokecolor{currentstroke}%
\pgfsetdash{}{0pt}%
\pgfpathmoveto{\pgfqpoint{0.553581in}{0.499444in}}%
\pgfpathlineto{\pgfqpoint{0.553581in}{1.654444in}}%
\pgfusepath{stroke}%
\end{pgfscope}%
\begin{pgfscope}%
\pgfsetrectcap%
\pgfsetmiterjoin%
\pgfsetlinewidth{0.803000pt}%
\definecolor{currentstroke}{rgb}{0.000000,0.000000,0.000000}%
\pgfsetstrokecolor{currentstroke}%
\pgfsetdash{}{0pt}%
\pgfpathmoveto{\pgfqpoint{2.103581in}{0.499444in}}%
\pgfpathlineto{\pgfqpoint{2.103581in}{1.654444in}}%
\pgfusepath{stroke}%
\end{pgfscope}%
\begin{pgfscope}%
\pgfsetrectcap%
\pgfsetmiterjoin%
\pgfsetlinewidth{0.803000pt}%
\definecolor{currentstroke}{rgb}{0.000000,0.000000,0.000000}%
\pgfsetstrokecolor{currentstroke}%
\pgfsetdash{}{0pt}%
\pgfpathmoveto{\pgfqpoint{0.553581in}{0.499444in}}%
\pgfpathlineto{\pgfqpoint{2.103581in}{0.499444in}}%
\pgfusepath{stroke}%
\end{pgfscope}%
\begin{pgfscope}%
\pgfsetrectcap%
\pgfsetmiterjoin%
\pgfsetlinewidth{0.803000pt}%
\definecolor{currentstroke}{rgb}{0.000000,0.000000,0.000000}%
\pgfsetstrokecolor{currentstroke}%
\pgfsetdash{}{0pt}%
\pgfpathmoveto{\pgfqpoint{0.553581in}{1.654444in}}%
\pgfpathlineto{\pgfqpoint{2.103581in}{1.654444in}}%
\pgfusepath{stroke}%
\end{pgfscope}%
\begin{pgfscope}%
\pgfsetbuttcap%
\pgfsetmiterjoin%
\definecolor{currentfill}{rgb}{1.000000,1.000000,1.000000}%
\pgfsetfillcolor{currentfill}%
\pgfsetlinewidth{0.000000pt}%
\definecolor{currentstroke}{rgb}{0.000000,0.000000,0.000000}%
\pgfsetstrokecolor{currentstroke}%
\pgfsetstrokeopacity{0.000000}%
\pgfsetdash{}{0pt}%
\pgfpathmoveto{\pgfqpoint{0.675275in}{0.961188in}}%
\pgfpathlineto{\pgfqpoint{1.074997in}{0.961188in}}%
\pgfpathlineto{\pgfqpoint{1.074997in}{1.167855in}}%
\pgfpathlineto{\pgfqpoint{0.675275in}{1.167855in}}%
\pgfpathlineto{\pgfqpoint{0.675275in}{0.961188in}}%
\pgfpathclose%
\pgfusepath{fill}%
\end{pgfscope}%
\begin{pgfscope}%
\definecolor{textcolor}{rgb}{0.000000,0.000000,0.000000}%
\pgfsetstrokecolor{textcolor}%
\pgfsetfillcolor{textcolor}%
\pgftext[x=0.716941in,y=1.029799in,left,base]{\color{textcolor}\rmfamily\fontsize{10.000000}{12.000000}\selectfont 0.338}%
\end{pgfscope}%
\begin{pgfscope}%
\pgfsetbuttcap%
\pgfsetmiterjoin%
\definecolor{currentfill}{rgb}{1.000000,1.000000,1.000000}%
\pgfsetfillcolor{currentfill}%
\pgfsetlinewidth{0.000000pt}%
\definecolor{currentstroke}{rgb}{0.000000,0.000000,0.000000}%
\pgfsetstrokecolor{currentstroke}%
\pgfsetstrokeopacity{0.000000}%
\pgfsetdash{}{0pt}%
\pgfpathmoveto{\pgfqpoint{0.582380in}{0.483333in}}%
\pgfpathlineto{\pgfqpoint{0.843214in}{0.483333in}}%
\pgfpathlineto{\pgfqpoint{0.843214in}{0.690000in}}%
\pgfpathlineto{\pgfqpoint{0.582380in}{0.690000in}}%
\pgfpathlineto{\pgfqpoint{0.582380in}{0.483333in}}%
\pgfpathclose%
\pgfusepath{fill}%
\end{pgfscope}%
\begin{pgfscope}%
\definecolor{textcolor}{rgb}{0.000000,0.000000,0.000000}%
\pgfsetstrokecolor{textcolor}%
\pgfsetfillcolor{textcolor}%
\pgftext[x=0.624047in,y=0.551944in,left,base]{\color{textcolor}\rmfamily\fontsize{10.000000}{12.000000}\selectfont 0.5}%
\end{pgfscope}%
\begin{pgfscope}%
\pgfsetbuttcap%
\pgfsetmiterjoin%
\definecolor{currentfill}{rgb}{1.000000,1.000000,1.000000}%
\pgfsetfillcolor{currentfill}%
\pgfsetlinewidth{0.000000pt}%
\definecolor{currentstroke}{rgb}{0.000000,0.000000,0.000000}%
\pgfsetstrokecolor{currentstroke}%
\pgfsetstrokeopacity{0.000000}%
\pgfsetdash{}{0pt}%
\pgfpathmoveto{\pgfqpoint{0.582380in}{0.483333in}}%
\pgfpathlineto{\pgfqpoint{0.982103in}{0.483333in}}%
\pgfpathlineto{\pgfqpoint{0.982103in}{0.690000in}}%
\pgfpathlineto{\pgfqpoint{0.582380in}{0.690000in}}%
\pgfpathlineto{\pgfqpoint{0.582380in}{0.483333in}}%
\pgfpathclose%
\pgfusepath{fill}%
\end{pgfscope}%
\begin{pgfscope}%
\definecolor{textcolor}{rgb}{0.000000,0.000000,0.000000}%
\pgfsetstrokecolor{textcolor}%
\pgfsetfillcolor{textcolor}%
\pgftext[x=0.624047in,y=0.551944in,left,base]{\color{textcolor}\rmfamily\fontsize{10.000000}{12.000000}\selectfont 0.655}%
\end{pgfscope}%
\begin{pgfscope}%
\definecolor{textcolor}{rgb}{0.000000,0.000000,0.000000}%
\pgfsetstrokecolor{textcolor}%
\pgfsetfillcolor{textcolor}%
\pgftext[x=1.328581in,y=1.737778in,,base]{\color{textcolor}\rmfamily\fontsize{12.000000}{14.400000}\selectfont ROC Curve}%
\end{pgfscope}%
\begin{pgfscope}%
\pgfsetbuttcap%
\pgfsetmiterjoin%
\definecolor{currentfill}{rgb}{1.000000,1.000000,1.000000}%
\pgfsetfillcolor{currentfill}%
\pgfsetfillopacity{0.800000}%
\pgfsetlinewidth{1.003750pt}%
\definecolor{currentstroke}{rgb}{0.800000,0.800000,0.800000}%
\pgfsetstrokecolor{currentstroke}%
\pgfsetstrokeopacity{0.800000}%
\pgfsetdash{}{0pt}%
\pgfpathmoveto{\pgfqpoint{0.840525in}{0.568889in}}%
\pgfpathlineto{\pgfqpoint{2.006358in}{0.568889in}}%
\pgfpathquadraticcurveto{\pgfqpoint{2.034136in}{0.568889in}}{\pgfqpoint{2.034136in}{0.596666in}}%
\pgfpathlineto{\pgfqpoint{2.034136in}{0.791111in}}%
\pgfpathquadraticcurveto{\pgfqpoint{2.034136in}{0.818888in}}{\pgfqpoint{2.006358in}{0.818888in}}%
\pgfpathlineto{\pgfqpoint{0.840525in}{0.818888in}}%
\pgfpathquadraticcurveto{\pgfqpoint{0.812747in}{0.818888in}}{\pgfqpoint{0.812747in}{0.791111in}}%
\pgfpathlineto{\pgfqpoint{0.812747in}{0.596666in}}%
\pgfpathquadraticcurveto{\pgfqpoint{0.812747in}{0.568889in}}{\pgfqpoint{0.840525in}{0.568889in}}%
\pgfpathlineto{\pgfqpoint{0.840525in}{0.568889in}}%
\pgfpathclose%
\pgfusepath{stroke,fill}%
\end{pgfscope}%
\begin{pgfscope}%
\pgfsetrectcap%
\pgfsetroundjoin%
\pgfsetlinewidth{1.505625pt}%
\definecolor{currentstroke}{rgb}{0.000000,0.000000,0.000000}%
\pgfsetstrokecolor{currentstroke}%
\pgfsetdash{}{0pt}%
\pgfpathmoveto{\pgfqpoint{0.868303in}{0.707777in}}%
\pgfpathlineto{\pgfqpoint{1.007192in}{0.707777in}}%
\pgfpathlineto{\pgfqpoint{1.146081in}{0.707777in}}%
\pgfusepath{stroke}%
\end{pgfscope}%
\begin{pgfscope}%
\definecolor{textcolor}{rgb}{0.000000,0.000000,0.000000}%
\pgfsetstrokecolor{textcolor}%
\pgfsetfillcolor{textcolor}%
\pgftext[x=1.257192in,y=0.659166in,left,base]{\color{textcolor}\rmfamily\fontsize{10.000000}{12.000000}\selectfont AUC 0.838)}%
\end{pgfscope}%
\end{pgfpicture}%
\makeatother%
\endgroup%

\end{tabular}
\end{center}

\begin{center}
\begin{tabular}{cc}
\begin{tabular}{cc|c|c|}
	&\multicolumn{1}{c}{}& \multicolumn{2}{c}{Prediction} \cr
	&\multicolumn{1}{c}{} & \multicolumn{1}{c}{N} & \multicolumn{1}{c}{P} \cr\cline{3-4}
	\multirow{2}{*}{Actual}&N & 85.7\% & 0.0\% \vrule width 0pt height 10pt depth 2pt \cr\cline{3-4}
	&P & 14.3\% & 0.0\% \vrule width 0pt height 10pt depth 2pt \cr\cline{3-4}
\end{tabular}
&
\begin{tabular}{ll}
0.857 & Accuracy \cr 
0.500 & Balanced Accuracy \cr 
0.000 & Precision \cr 
0.000 & Balanced Precision \cr 
0.000 & Recall \cr 
0.000 & F1 \cr 
0.000 & Balanced F1 \cr 
0.000 & Gmean \cr 
	\end{tabular}
\end{tabular}
\end{center}

%%%%%
Such a recommendation system (``Never send an ambulance'') would be useless, but note that the distribution still separates the negative and positive classes, just not at $p=0.5$.  We can fix that in two ways; the first is to shift the distribution to be centered at $p=0.5$.  By ``centered,'' we mean that the average of the medians of the negative and positive classes (the 0.107 and 0.293 on the ROC curve above) will now be 0.5.  Further research can explore whether centering the distribution at the $p=0.5$ threshold or another value of $p$ is most useful.  

\begin{center}
\begin{tabular}{p{0.5\textwidth} p{0.5\textwidth}}
  \vspace{0pt} %% Creator: Matplotlib, PGF backend
%%
%% To include the figure in your LaTeX document, write
%%   \input{<filename>.pgf}
%%
%% Make sure the required packages are loaded in your preamble
%%   \usepackage{pgf}
%%
%% Also ensure that all the required font packages are loaded; for instance,
%% the lmodern package is sometimes necessary when using math font.
%%   \usepackage{lmodern}
%%
%% Figures using additional raster images can only be included by \input if
%% they are in the same directory as the main LaTeX file. For loading figures
%% from other directories you can use the `import` package
%%   \usepackage{import}
%%
%% and then include the figures with
%%   \import{<path to file>}{<filename>.pgf}
%%
%% Matplotlib used the following preamble
%%   
%%   \usepackage{fontspec}
%%   \makeatletter\@ifpackageloaded{underscore}{}{\usepackage[strings]{underscore}}\makeatother
%%
\begingroup%
\makeatletter%
\begin{pgfpicture}%
\pgfpathrectangle{\pgfpointorigin}{\pgfqpoint{2.253750in}{1.953444in}}%
\pgfusepath{use as bounding box, clip}%
\begin{pgfscope}%
\pgfsetbuttcap%
\pgfsetmiterjoin%
\definecolor{currentfill}{rgb}{1.000000,1.000000,1.000000}%
\pgfsetfillcolor{currentfill}%
\pgfsetlinewidth{0.000000pt}%
\definecolor{currentstroke}{rgb}{1.000000,1.000000,1.000000}%
\pgfsetstrokecolor{currentstroke}%
\pgfsetdash{}{0pt}%
\pgfpathmoveto{\pgfqpoint{0.000000in}{0.000000in}}%
\pgfpathlineto{\pgfqpoint{2.253750in}{0.000000in}}%
\pgfpathlineto{\pgfqpoint{2.253750in}{1.953444in}}%
\pgfpathlineto{\pgfqpoint{0.000000in}{1.953444in}}%
\pgfpathlineto{\pgfqpoint{0.000000in}{0.000000in}}%
\pgfpathclose%
\pgfusepath{fill}%
\end{pgfscope}%
\begin{pgfscope}%
\pgfsetbuttcap%
\pgfsetmiterjoin%
\definecolor{currentfill}{rgb}{1.000000,1.000000,1.000000}%
\pgfsetfillcolor{currentfill}%
\pgfsetlinewidth{0.000000pt}%
\definecolor{currentstroke}{rgb}{0.000000,0.000000,0.000000}%
\pgfsetstrokecolor{currentstroke}%
\pgfsetstrokeopacity{0.000000}%
\pgfsetdash{}{0pt}%
\pgfpathmoveto{\pgfqpoint{0.515000in}{0.499444in}}%
\pgfpathlineto{\pgfqpoint{2.065000in}{0.499444in}}%
\pgfpathlineto{\pgfqpoint{2.065000in}{1.654444in}}%
\pgfpathlineto{\pgfqpoint{0.515000in}{1.654444in}}%
\pgfpathlineto{\pgfqpoint{0.515000in}{0.499444in}}%
\pgfpathclose%
\pgfusepath{fill}%
\end{pgfscope}%
\begin{pgfscope}%
\pgfpathrectangle{\pgfqpoint{0.515000in}{0.499444in}}{\pgfqpoint{1.550000in}{1.155000in}}%
\pgfusepath{clip}%
\pgfsetbuttcap%
\pgfsetmiterjoin%
\pgfsetlinewidth{1.003750pt}%
\definecolor{currentstroke}{rgb}{0.000000,0.000000,0.000000}%
\pgfsetstrokecolor{currentstroke}%
\pgfsetdash{}{0pt}%
\pgfpathmoveto{\pgfqpoint{0.505000in}{0.499444in}}%
\pgfpathlineto{\pgfqpoint{0.552805in}{0.499444in}}%
\pgfpathlineto{\pgfqpoint{0.552805in}{0.499444in}}%
\pgfpathlineto{\pgfqpoint{0.505000in}{0.499444in}}%
\pgfusepath{stroke}%
\end{pgfscope}%
\begin{pgfscope}%
\pgfpathrectangle{\pgfqpoint{0.515000in}{0.499444in}}{\pgfqpoint{1.550000in}{1.155000in}}%
\pgfusepath{clip}%
\pgfsetbuttcap%
\pgfsetmiterjoin%
\pgfsetlinewidth{1.003750pt}%
\definecolor{currentstroke}{rgb}{0.000000,0.000000,0.000000}%
\pgfsetstrokecolor{currentstroke}%
\pgfsetdash{}{0pt}%
\pgfpathmoveto{\pgfqpoint{0.643537in}{0.499444in}}%
\pgfpathlineto{\pgfqpoint{0.704025in}{0.499444in}}%
\pgfpathlineto{\pgfqpoint{0.704025in}{0.499444in}}%
\pgfpathlineto{\pgfqpoint{0.643537in}{0.499444in}}%
\pgfpathlineto{\pgfqpoint{0.643537in}{0.499444in}}%
\pgfpathclose%
\pgfusepath{stroke}%
\end{pgfscope}%
\begin{pgfscope}%
\pgfpathrectangle{\pgfqpoint{0.515000in}{0.499444in}}{\pgfqpoint{1.550000in}{1.155000in}}%
\pgfusepath{clip}%
\pgfsetbuttcap%
\pgfsetmiterjoin%
\pgfsetlinewidth{1.003750pt}%
\definecolor{currentstroke}{rgb}{0.000000,0.000000,0.000000}%
\pgfsetstrokecolor{currentstroke}%
\pgfsetdash{}{0pt}%
\pgfpathmoveto{\pgfqpoint{0.794756in}{0.499444in}}%
\pgfpathlineto{\pgfqpoint{0.855244in}{0.499444in}}%
\pgfpathlineto{\pgfqpoint{0.855244in}{0.576765in}}%
\pgfpathlineto{\pgfqpoint{0.794756in}{0.576765in}}%
\pgfpathlineto{\pgfqpoint{0.794756in}{0.499444in}}%
\pgfpathclose%
\pgfusepath{stroke}%
\end{pgfscope}%
\begin{pgfscope}%
\pgfpathrectangle{\pgfqpoint{0.515000in}{0.499444in}}{\pgfqpoint{1.550000in}{1.155000in}}%
\pgfusepath{clip}%
\pgfsetbuttcap%
\pgfsetmiterjoin%
\pgfsetlinewidth{1.003750pt}%
\definecolor{currentstroke}{rgb}{0.000000,0.000000,0.000000}%
\pgfsetstrokecolor{currentstroke}%
\pgfsetdash{}{0pt}%
\pgfpathmoveto{\pgfqpoint{0.945976in}{0.499444in}}%
\pgfpathlineto{\pgfqpoint{1.006464in}{0.499444in}}%
\pgfpathlineto{\pgfqpoint{1.006464in}{1.599444in}}%
\pgfpathlineto{\pgfqpoint{0.945976in}{1.599444in}}%
\pgfpathlineto{\pgfqpoint{0.945976in}{0.499444in}}%
\pgfpathclose%
\pgfusepath{stroke}%
\end{pgfscope}%
\begin{pgfscope}%
\pgfpathrectangle{\pgfqpoint{0.515000in}{0.499444in}}{\pgfqpoint{1.550000in}{1.155000in}}%
\pgfusepath{clip}%
\pgfsetbuttcap%
\pgfsetmiterjoin%
\pgfsetlinewidth{1.003750pt}%
\definecolor{currentstroke}{rgb}{0.000000,0.000000,0.000000}%
\pgfsetstrokecolor{currentstroke}%
\pgfsetdash{}{0pt}%
\pgfpathmoveto{\pgfqpoint{1.097195in}{0.499444in}}%
\pgfpathlineto{\pgfqpoint{1.157683in}{0.499444in}}%
\pgfpathlineto{\pgfqpoint{1.157683in}{1.578796in}}%
\pgfpathlineto{\pgfqpoint{1.097195in}{1.578796in}}%
\pgfpathlineto{\pgfqpoint{1.097195in}{0.499444in}}%
\pgfpathclose%
\pgfusepath{stroke}%
\end{pgfscope}%
\begin{pgfscope}%
\pgfpathrectangle{\pgfqpoint{0.515000in}{0.499444in}}{\pgfqpoint{1.550000in}{1.155000in}}%
\pgfusepath{clip}%
\pgfsetbuttcap%
\pgfsetmiterjoin%
\pgfsetlinewidth{1.003750pt}%
\definecolor{currentstroke}{rgb}{0.000000,0.000000,0.000000}%
\pgfsetstrokecolor{currentstroke}%
\pgfsetdash{}{0pt}%
\pgfpathmoveto{\pgfqpoint{1.248415in}{0.499444in}}%
\pgfpathlineto{\pgfqpoint{1.308903in}{0.499444in}}%
\pgfpathlineto{\pgfqpoint{1.308903in}{0.979473in}}%
\pgfpathlineto{\pgfqpoint{1.248415in}{0.979473in}}%
\pgfpathlineto{\pgfqpoint{1.248415in}{0.499444in}}%
\pgfpathclose%
\pgfusepath{stroke}%
\end{pgfscope}%
\begin{pgfscope}%
\pgfpathrectangle{\pgfqpoint{0.515000in}{0.499444in}}{\pgfqpoint{1.550000in}{1.155000in}}%
\pgfusepath{clip}%
\pgfsetbuttcap%
\pgfsetmiterjoin%
\pgfsetlinewidth{1.003750pt}%
\definecolor{currentstroke}{rgb}{0.000000,0.000000,0.000000}%
\pgfsetstrokecolor{currentstroke}%
\pgfsetdash{}{0pt}%
\pgfpathmoveto{\pgfqpoint{1.399634in}{0.499444in}}%
\pgfpathlineto{\pgfqpoint{1.460122in}{0.499444in}}%
\pgfpathlineto{\pgfqpoint{1.460122in}{0.638999in}}%
\pgfpathlineto{\pgfqpoint{1.399634in}{0.638999in}}%
\pgfpathlineto{\pgfqpoint{1.399634in}{0.499444in}}%
\pgfpathclose%
\pgfusepath{stroke}%
\end{pgfscope}%
\begin{pgfscope}%
\pgfpathrectangle{\pgfqpoint{0.515000in}{0.499444in}}{\pgfqpoint{1.550000in}{1.155000in}}%
\pgfusepath{clip}%
\pgfsetbuttcap%
\pgfsetmiterjoin%
\pgfsetlinewidth{1.003750pt}%
\definecolor{currentstroke}{rgb}{0.000000,0.000000,0.000000}%
\pgfsetstrokecolor{currentstroke}%
\pgfsetdash{}{0pt}%
\pgfpathmoveto{\pgfqpoint{1.550854in}{0.499444in}}%
\pgfpathlineto{\pgfqpoint{1.611342in}{0.499444in}}%
\pgfpathlineto{\pgfqpoint{1.611342in}{0.524541in}}%
\pgfpathlineto{\pgfqpoint{1.550854in}{0.524541in}}%
\pgfpathlineto{\pgfqpoint{1.550854in}{0.499444in}}%
\pgfpathclose%
\pgfusepath{stroke}%
\end{pgfscope}%
\begin{pgfscope}%
\pgfpathrectangle{\pgfqpoint{0.515000in}{0.499444in}}{\pgfqpoint{1.550000in}{1.155000in}}%
\pgfusepath{clip}%
\pgfsetbuttcap%
\pgfsetmiterjoin%
\pgfsetlinewidth{1.003750pt}%
\definecolor{currentstroke}{rgb}{0.000000,0.000000,0.000000}%
\pgfsetstrokecolor{currentstroke}%
\pgfsetdash{}{0pt}%
\pgfpathmoveto{\pgfqpoint{1.702073in}{0.499444in}}%
\pgfpathlineto{\pgfqpoint{1.762561in}{0.499444in}}%
\pgfpathlineto{\pgfqpoint{1.762561in}{0.499444in}}%
\pgfpathlineto{\pgfqpoint{1.702073in}{0.499444in}}%
\pgfpathlineto{\pgfqpoint{1.702073in}{0.499444in}}%
\pgfpathclose%
\pgfusepath{stroke}%
\end{pgfscope}%
\begin{pgfscope}%
\pgfpathrectangle{\pgfqpoint{0.515000in}{0.499444in}}{\pgfqpoint{1.550000in}{1.155000in}}%
\pgfusepath{clip}%
\pgfsetbuttcap%
\pgfsetmiterjoin%
\pgfsetlinewidth{1.003750pt}%
\definecolor{currentstroke}{rgb}{0.000000,0.000000,0.000000}%
\pgfsetstrokecolor{currentstroke}%
\pgfsetdash{}{0pt}%
\pgfpathmoveto{\pgfqpoint{1.853293in}{0.499444in}}%
\pgfpathlineto{\pgfqpoint{1.913781in}{0.499444in}}%
\pgfpathlineto{\pgfqpoint{1.913781in}{0.499444in}}%
\pgfpathlineto{\pgfqpoint{1.853293in}{0.499444in}}%
\pgfpathlineto{\pgfqpoint{1.853293in}{0.499444in}}%
\pgfpathclose%
\pgfusepath{stroke}%
\end{pgfscope}%
\begin{pgfscope}%
\pgfpathrectangle{\pgfqpoint{0.515000in}{0.499444in}}{\pgfqpoint{1.550000in}{1.155000in}}%
\pgfusepath{clip}%
\pgfsetbuttcap%
\pgfsetmiterjoin%
\definecolor{currentfill}{rgb}{0.000000,0.000000,0.000000}%
\pgfsetfillcolor{currentfill}%
\pgfsetlinewidth{0.000000pt}%
\definecolor{currentstroke}{rgb}{0.000000,0.000000,0.000000}%
\pgfsetstrokecolor{currentstroke}%
\pgfsetstrokeopacity{0.000000}%
\pgfsetdash{}{0pt}%
\pgfpathmoveto{\pgfqpoint{0.552805in}{0.499444in}}%
\pgfpathlineto{\pgfqpoint{0.613293in}{0.499444in}}%
\pgfpathlineto{\pgfqpoint{0.613293in}{0.499444in}}%
\pgfpathlineto{\pgfqpoint{0.552805in}{0.499444in}}%
\pgfpathlineto{\pgfqpoint{0.552805in}{0.499444in}}%
\pgfpathclose%
\pgfusepath{fill}%
\end{pgfscope}%
\begin{pgfscope}%
\pgfpathrectangle{\pgfqpoint{0.515000in}{0.499444in}}{\pgfqpoint{1.550000in}{1.155000in}}%
\pgfusepath{clip}%
\pgfsetbuttcap%
\pgfsetmiterjoin%
\definecolor{currentfill}{rgb}{0.000000,0.000000,0.000000}%
\pgfsetfillcolor{currentfill}%
\pgfsetlinewidth{0.000000pt}%
\definecolor{currentstroke}{rgb}{0.000000,0.000000,0.000000}%
\pgfsetstrokecolor{currentstroke}%
\pgfsetstrokeopacity{0.000000}%
\pgfsetdash{}{0pt}%
\pgfpathmoveto{\pgfqpoint{0.704025in}{0.499444in}}%
\pgfpathlineto{\pgfqpoint{0.764512in}{0.499444in}}%
\pgfpathlineto{\pgfqpoint{0.764512in}{0.499444in}}%
\pgfpathlineto{\pgfqpoint{0.704025in}{0.499444in}}%
\pgfpathlineto{\pgfqpoint{0.704025in}{0.499444in}}%
\pgfpathclose%
\pgfusepath{fill}%
\end{pgfscope}%
\begin{pgfscope}%
\pgfpathrectangle{\pgfqpoint{0.515000in}{0.499444in}}{\pgfqpoint{1.550000in}{1.155000in}}%
\pgfusepath{clip}%
\pgfsetbuttcap%
\pgfsetmiterjoin%
\definecolor{currentfill}{rgb}{0.000000,0.000000,0.000000}%
\pgfsetfillcolor{currentfill}%
\pgfsetlinewidth{0.000000pt}%
\definecolor{currentstroke}{rgb}{0.000000,0.000000,0.000000}%
\pgfsetstrokecolor{currentstroke}%
\pgfsetstrokeopacity{0.000000}%
\pgfsetdash{}{0pt}%
\pgfpathmoveto{\pgfqpoint{0.855244in}{0.499444in}}%
\pgfpathlineto{\pgfqpoint{0.915732in}{0.499444in}}%
\pgfpathlineto{\pgfqpoint{0.915732in}{0.503748in}}%
\pgfpathlineto{\pgfqpoint{0.855244in}{0.503748in}}%
\pgfpathlineto{\pgfqpoint{0.855244in}{0.499444in}}%
\pgfpathclose%
\pgfusepath{fill}%
\end{pgfscope}%
\begin{pgfscope}%
\pgfpathrectangle{\pgfqpoint{0.515000in}{0.499444in}}{\pgfqpoint{1.550000in}{1.155000in}}%
\pgfusepath{clip}%
\pgfsetbuttcap%
\pgfsetmiterjoin%
\definecolor{currentfill}{rgb}{0.000000,0.000000,0.000000}%
\pgfsetfillcolor{currentfill}%
\pgfsetlinewidth{0.000000pt}%
\definecolor{currentstroke}{rgb}{0.000000,0.000000,0.000000}%
\pgfsetstrokecolor{currentstroke}%
\pgfsetstrokeopacity{0.000000}%
\pgfsetdash{}{0pt}%
\pgfpathmoveto{\pgfqpoint{1.006464in}{0.499444in}}%
\pgfpathlineto{\pgfqpoint{1.066951in}{0.499444in}}%
\pgfpathlineto{\pgfqpoint{1.066951in}{0.524009in}}%
\pgfpathlineto{\pgfqpoint{1.006464in}{0.524009in}}%
\pgfpathlineto{\pgfqpoint{1.006464in}{0.499444in}}%
\pgfpathclose%
\pgfusepath{fill}%
\end{pgfscope}%
\begin{pgfscope}%
\pgfpathrectangle{\pgfqpoint{0.515000in}{0.499444in}}{\pgfqpoint{1.550000in}{1.155000in}}%
\pgfusepath{clip}%
\pgfsetbuttcap%
\pgfsetmiterjoin%
\definecolor{currentfill}{rgb}{0.000000,0.000000,0.000000}%
\pgfsetfillcolor{currentfill}%
\pgfsetlinewidth{0.000000pt}%
\definecolor{currentstroke}{rgb}{0.000000,0.000000,0.000000}%
\pgfsetstrokecolor{currentstroke}%
\pgfsetstrokeopacity{0.000000}%
\pgfsetdash{}{0pt}%
\pgfpathmoveto{\pgfqpoint{1.157683in}{0.499444in}}%
\pgfpathlineto{\pgfqpoint{1.218171in}{0.499444in}}%
\pgfpathlineto{\pgfqpoint{1.218171in}{0.577902in}}%
\pgfpathlineto{\pgfqpoint{1.157683in}{0.577902in}}%
\pgfpathlineto{\pgfqpoint{1.157683in}{0.499444in}}%
\pgfpathclose%
\pgfusepath{fill}%
\end{pgfscope}%
\begin{pgfscope}%
\pgfpathrectangle{\pgfqpoint{0.515000in}{0.499444in}}{\pgfqpoint{1.550000in}{1.155000in}}%
\pgfusepath{clip}%
\pgfsetbuttcap%
\pgfsetmiterjoin%
\definecolor{currentfill}{rgb}{0.000000,0.000000,0.000000}%
\pgfsetfillcolor{currentfill}%
\pgfsetlinewidth{0.000000pt}%
\definecolor{currentstroke}{rgb}{0.000000,0.000000,0.000000}%
\pgfsetstrokecolor{currentstroke}%
\pgfsetstrokeopacity{0.000000}%
\pgfsetdash{}{0pt}%
\pgfpathmoveto{\pgfqpoint{1.308903in}{0.499444in}}%
\pgfpathlineto{\pgfqpoint{1.369391in}{0.499444in}}%
\pgfpathlineto{\pgfqpoint{1.369391in}{0.676862in}}%
\pgfpathlineto{\pgfqpoint{1.308903in}{0.676862in}}%
\pgfpathlineto{\pgfqpoint{1.308903in}{0.499444in}}%
\pgfpathclose%
\pgfusepath{fill}%
\end{pgfscope}%
\begin{pgfscope}%
\pgfpathrectangle{\pgfqpoint{0.515000in}{0.499444in}}{\pgfqpoint{1.550000in}{1.155000in}}%
\pgfusepath{clip}%
\pgfsetbuttcap%
\pgfsetmiterjoin%
\definecolor{currentfill}{rgb}{0.000000,0.000000,0.000000}%
\pgfsetfillcolor{currentfill}%
\pgfsetlinewidth{0.000000pt}%
\definecolor{currentstroke}{rgb}{0.000000,0.000000,0.000000}%
\pgfsetstrokecolor{currentstroke}%
\pgfsetstrokeopacity{0.000000}%
\pgfsetdash{}{0pt}%
\pgfpathmoveto{\pgfqpoint{1.460122in}{0.499444in}}%
\pgfpathlineto{\pgfqpoint{1.520610in}{0.499444in}}%
\pgfpathlineto{\pgfqpoint{1.520610in}{0.680174in}}%
\pgfpathlineto{\pgfqpoint{1.460122in}{0.680174in}}%
\pgfpathlineto{\pgfqpoint{1.460122in}{0.499444in}}%
\pgfpathclose%
\pgfusepath{fill}%
\end{pgfscope}%
\begin{pgfscope}%
\pgfpathrectangle{\pgfqpoint{0.515000in}{0.499444in}}{\pgfqpoint{1.550000in}{1.155000in}}%
\pgfusepath{clip}%
\pgfsetbuttcap%
\pgfsetmiterjoin%
\definecolor{currentfill}{rgb}{0.000000,0.000000,0.000000}%
\pgfsetfillcolor{currentfill}%
\pgfsetlinewidth{0.000000pt}%
\definecolor{currentstroke}{rgb}{0.000000,0.000000,0.000000}%
\pgfsetstrokecolor{currentstroke}%
\pgfsetstrokeopacity{0.000000}%
\pgfsetdash{}{0pt}%
\pgfpathmoveto{\pgfqpoint{1.611342in}{0.499444in}}%
\pgfpathlineto{\pgfqpoint{1.671830in}{0.499444in}}%
\pgfpathlineto{\pgfqpoint{1.671830in}{0.517529in}}%
\pgfpathlineto{\pgfqpoint{1.611342in}{0.517529in}}%
\pgfpathlineto{\pgfqpoint{1.611342in}{0.499444in}}%
\pgfpathclose%
\pgfusepath{fill}%
\end{pgfscope}%
\begin{pgfscope}%
\pgfpathrectangle{\pgfqpoint{0.515000in}{0.499444in}}{\pgfqpoint{1.550000in}{1.155000in}}%
\pgfusepath{clip}%
\pgfsetbuttcap%
\pgfsetmiterjoin%
\definecolor{currentfill}{rgb}{0.000000,0.000000,0.000000}%
\pgfsetfillcolor{currentfill}%
\pgfsetlinewidth{0.000000pt}%
\definecolor{currentstroke}{rgb}{0.000000,0.000000,0.000000}%
\pgfsetstrokecolor{currentstroke}%
\pgfsetstrokeopacity{0.000000}%
\pgfsetdash{}{0pt}%
\pgfpathmoveto{\pgfqpoint{1.762561in}{0.499444in}}%
\pgfpathlineto{\pgfqpoint{1.823049in}{0.499444in}}%
\pgfpathlineto{\pgfqpoint{1.823049in}{0.499444in}}%
\pgfpathlineto{\pgfqpoint{1.762561in}{0.499444in}}%
\pgfpathlineto{\pgfqpoint{1.762561in}{0.499444in}}%
\pgfpathclose%
\pgfusepath{fill}%
\end{pgfscope}%
\begin{pgfscope}%
\pgfpathrectangle{\pgfqpoint{0.515000in}{0.499444in}}{\pgfqpoint{1.550000in}{1.155000in}}%
\pgfusepath{clip}%
\pgfsetbuttcap%
\pgfsetmiterjoin%
\definecolor{currentfill}{rgb}{0.000000,0.000000,0.000000}%
\pgfsetfillcolor{currentfill}%
\pgfsetlinewidth{0.000000pt}%
\definecolor{currentstroke}{rgb}{0.000000,0.000000,0.000000}%
\pgfsetstrokecolor{currentstroke}%
\pgfsetstrokeopacity{0.000000}%
\pgfsetdash{}{0pt}%
\pgfpathmoveto{\pgfqpoint{1.913781in}{0.499444in}}%
\pgfpathlineto{\pgfqpoint{1.974269in}{0.499444in}}%
\pgfpathlineto{\pgfqpoint{1.974269in}{0.499444in}}%
\pgfpathlineto{\pgfqpoint{1.913781in}{0.499444in}}%
\pgfpathlineto{\pgfqpoint{1.913781in}{0.499444in}}%
\pgfpathclose%
\pgfusepath{fill}%
\end{pgfscope}%
\begin{pgfscope}%
\pgfsetbuttcap%
\pgfsetroundjoin%
\definecolor{currentfill}{rgb}{0.000000,0.000000,0.000000}%
\pgfsetfillcolor{currentfill}%
\pgfsetlinewidth{0.803000pt}%
\definecolor{currentstroke}{rgb}{0.000000,0.000000,0.000000}%
\pgfsetstrokecolor{currentstroke}%
\pgfsetdash{}{0pt}%
\pgfsys@defobject{currentmarker}{\pgfqpoint{0.000000in}{-0.048611in}}{\pgfqpoint{0.000000in}{0.000000in}}{%
\pgfpathmoveto{\pgfqpoint{0.000000in}{0.000000in}}%
\pgfpathlineto{\pgfqpoint{0.000000in}{-0.048611in}}%
\pgfusepath{stroke,fill}%
}%
\begin{pgfscope}%
\pgfsys@transformshift{0.552805in}{0.499444in}%
\pgfsys@useobject{currentmarker}{}%
\end{pgfscope}%
\end{pgfscope}%
\begin{pgfscope}%
\definecolor{textcolor}{rgb}{0.000000,0.000000,0.000000}%
\pgfsetstrokecolor{textcolor}%
\pgfsetfillcolor{textcolor}%
\pgftext[x=0.552805in,y=0.402222in,,top]{\color{textcolor}\rmfamily\fontsize{10.000000}{12.000000}\selectfont 0.0}%
\end{pgfscope}%
\begin{pgfscope}%
\pgfsetbuttcap%
\pgfsetroundjoin%
\definecolor{currentfill}{rgb}{0.000000,0.000000,0.000000}%
\pgfsetfillcolor{currentfill}%
\pgfsetlinewidth{0.803000pt}%
\definecolor{currentstroke}{rgb}{0.000000,0.000000,0.000000}%
\pgfsetstrokecolor{currentstroke}%
\pgfsetdash{}{0pt}%
\pgfsys@defobject{currentmarker}{\pgfqpoint{0.000000in}{-0.048611in}}{\pgfqpoint{0.000000in}{0.000000in}}{%
\pgfpathmoveto{\pgfqpoint{0.000000in}{0.000000in}}%
\pgfpathlineto{\pgfqpoint{0.000000in}{-0.048611in}}%
\pgfusepath{stroke,fill}%
}%
\begin{pgfscope}%
\pgfsys@transformshift{0.930854in}{0.499444in}%
\pgfsys@useobject{currentmarker}{}%
\end{pgfscope}%
\end{pgfscope}%
\begin{pgfscope}%
\definecolor{textcolor}{rgb}{0.000000,0.000000,0.000000}%
\pgfsetstrokecolor{textcolor}%
\pgfsetfillcolor{textcolor}%
\pgftext[x=0.930854in,y=0.402222in,,top]{\color{textcolor}\rmfamily\fontsize{10.000000}{12.000000}\selectfont 0.25}%
\end{pgfscope}%
\begin{pgfscope}%
\pgfsetbuttcap%
\pgfsetroundjoin%
\definecolor{currentfill}{rgb}{0.000000,0.000000,0.000000}%
\pgfsetfillcolor{currentfill}%
\pgfsetlinewidth{0.803000pt}%
\definecolor{currentstroke}{rgb}{0.000000,0.000000,0.000000}%
\pgfsetstrokecolor{currentstroke}%
\pgfsetdash{}{0pt}%
\pgfsys@defobject{currentmarker}{\pgfqpoint{0.000000in}{-0.048611in}}{\pgfqpoint{0.000000in}{0.000000in}}{%
\pgfpathmoveto{\pgfqpoint{0.000000in}{0.000000in}}%
\pgfpathlineto{\pgfqpoint{0.000000in}{-0.048611in}}%
\pgfusepath{stroke,fill}%
}%
\begin{pgfscope}%
\pgfsys@transformshift{1.308903in}{0.499444in}%
\pgfsys@useobject{currentmarker}{}%
\end{pgfscope}%
\end{pgfscope}%
\begin{pgfscope}%
\definecolor{textcolor}{rgb}{0.000000,0.000000,0.000000}%
\pgfsetstrokecolor{textcolor}%
\pgfsetfillcolor{textcolor}%
\pgftext[x=1.308903in,y=0.402222in,,top]{\color{textcolor}\rmfamily\fontsize{10.000000}{12.000000}\selectfont 0.5}%
\end{pgfscope}%
\begin{pgfscope}%
\pgfsetbuttcap%
\pgfsetroundjoin%
\definecolor{currentfill}{rgb}{0.000000,0.000000,0.000000}%
\pgfsetfillcolor{currentfill}%
\pgfsetlinewidth{0.803000pt}%
\definecolor{currentstroke}{rgb}{0.000000,0.000000,0.000000}%
\pgfsetstrokecolor{currentstroke}%
\pgfsetdash{}{0pt}%
\pgfsys@defobject{currentmarker}{\pgfqpoint{0.000000in}{-0.048611in}}{\pgfqpoint{0.000000in}{0.000000in}}{%
\pgfpathmoveto{\pgfqpoint{0.000000in}{0.000000in}}%
\pgfpathlineto{\pgfqpoint{0.000000in}{-0.048611in}}%
\pgfusepath{stroke,fill}%
}%
\begin{pgfscope}%
\pgfsys@transformshift{1.686951in}{0.499444in}%
\pgfsys@useobject{currentmarker}{}%
\end{pgfscope}%
\end{pgfscope}%
\begin{pgfscope}%
\definecolor{textcolor}{rgb}{0.000000,0.000000,0.000000}%
\pgfsetstrokecolor{textcolor}%
\pgfsetfillcolor{textcolor}%
\pgftext[x=1.686951in,y=0.402222in,,top]{\color{textcolor}\rmfamily\fontsize{10.000000}{12.000000}\selectfont 0.75}%
\end{pgfscope}%
\begin{pgfscope}%
\pgfsetbuttcap%
\pgfsetroundjoin%
\definecolor{currentfill}{rgb}{0.000000,0.000000,0.000000}%
\pgfsetfillcolor{currentfill}%
\pgfsetlinewidth{0.803000pt}%
\definecolor{currentstroke}{rgb}{0.000000,0.000000,0.000000}%
\pgfsetstrokecolor{currentstroke}%
\pgfsetdash{}{0pt}%
\pgfsys@defobject{currentmarker}{\pgfqpoint{0.000000in}{-0.048611in}}{\pgfqpoint{0.000000in}{0.000000in}}{%
\pgfpathmoveto{\pgfqpoint{0.000000in}{0.000000in}}%
\pgfpathlineto{\pgfqpoint{0.000000in}{-0.048611in}}%
\pgfusepath{stroke,fill}%
}%
\begin{pgfscope}%
\pgfsys@transformshift{2.065000in}{0.499444in}%
\pgfsys@useobject{currentmarker}{}%
\end{pgfscope}%
\end{pgfscope}%
\begin{pgfscope}%
\definecolor{textcolor}{rgb}{0.000000,0.000000,0.000000}%
\pgfsetstrokecolor{textcolor}%
\pgfsetfillcolor{textcolor}%
\pgftext[x=2.065000in,y=0.402222in,,top]{\color{textcolor}\rmfamily\fontsize{10.000000}{12.000000}\selectfont 1.0}%
\end{pgfscope}%
\begin{pgfscope}%
\definecolor{textcolor}{rgb}{0.000000,0.000000,0.000000}%
\pgfsetstrokecolor{textcolor}%
\pgfsetfillcolor{textcolor}%
\pgftext[x=1.290000in,y=0.223333in,,top]{\color{textcolor}\rmfamily\fontsize{10.000000}{12.000000}\selectfont \(\displaystyle p\)}%
\end{pgfscope}%
\begin{pgfscope}%
\pgfsetbuttcap%
\pgfsetroundjoin%
\definecolor{currentfill}{rgb}{0.000000,0.000000,0.000000}%
\pgfsetfillcolor{currentfill}%
\pgfsetlinewidth{0.803000pt}%
\definecolor{currentstroke}{rgb}{0.000000,0.000000,0.000000}%
\pgfsetstrokecolor{currentstroke}%
\pgfsetdash{}{0pt}%
\pgfsys@defobject{currentmarker}{\pgfqpoint{-0.048611in}{0.000000in}}{\pgfqpoint{-0.000000in}{0.000000in}}{%
\pgfpathmoveto{\pgfqpoint{-0.000000in}{0.000000in}}%
\pgfpathlineto{\pgfqpoint{-0.048611in}{0.000000in}}%
\pgfusepath{stroke,fill}%
}%
\begin{pgfscope}%
\pgfsys@transformshift{0.515000in}{0.499444in}%
\pgfsys@useobject{currentmarker}{}%
\end{pgfscope}%
\end{pgfscope}%
\begin{pgfscope}%
\definecolor{textcolor}{rgb}{0.000000,0.000000,0.000000}%
\pgfsetstrokecolor{textcolor}%
\pgfsetfillcolor{textcolor}%
\pgftext[x=0.348333in, y=0.451250in, left, base]{\color{textcolor}\rmfamily\fontsize{10.000000}{12.000000}\selectfont \(\displaystyle {0}\)}%
\end{pgfscope}%
\begin{pgfscope}%
\pgfsetbuttcap%
\pgfsetroundjoin%
\definecolor{currentfill}{rgb}{0.000000,0.000000,0.000000}%
\pgfsetfillcolor{currentfill}%
\pgfsetlinewidth{0.803000pt}%
\definecolor{currentstroke}{rgb}{0.000000,0.000000,0.000000}%
\pgfsetstrokecolor{currentstroke}%
\pgfsetdash{}{0pt}%
\pgfsys@defobject{currentmarker}{\pgfqpoint{-0.048611in}{0.000000in}}{\pgfqpoint{-0.000000in}{0.000000in}}{%
\pgfpathmoveto{\pgfqpoint{-0.000000in}{0.000000in}}%
\pgfpathlineto{\pgfqpoint{-0.048611in}{0.000000in}}%
\pgfusepath{stroke,fill}%
}%
\begin{pgfscope}%
\pgfsys@transformshift{0.515000in}{0.837935in}%
\pgfsys@useobject{currentmarker}{}%
\end{pgfscope}%
\end{pgfscope}%
\begin{pgfscope}%
\definecolor{textcolor}{rgb}{0.000000,0.000000,0.000000}%
\pgfsetstrokecolor{textcolor}%
\pgfsetfillcolor{textcolor}%
\pgftext[x=0.278889in, y=0.789741in, left, base]{\color{textcolor}\rmfamily\fontsize{10.000000}{12.000000}\selectfont \(\displaystyle {10}\)}%
\end{pgfscope}%
\begin{pgfscope}%
\pgfsetbuttcap%
\pgfsetroundjoin%
\definecolor{currentfill}{rgb}{0.000000,0.000000,0.000000}%
\pgfsetfillcolor{currentfill}%
\pgfsetlinewidth{0.803000pt}%
\definecolor{currentstroke}{rgb}{0.000000,0.000000,0.000000}%
\pgfsetstrokecolor{currentstroke}%
\pgfsetdash{}{0pt}%
\pgfsys@defobject{currentmarker}{\pgfqpoint{-0.048611in}{0.000000in}}{\pgfqpoint{-0.000000in}{0.000000in}}{%
\pgfpathmoveto{\pgfqpoint{-0.000000in}{0.000000in}}%
\pgfpathlineto{\pgfqpoint{-0.048611in}{0.000000in}}%
\pgfusepath{stroke,fill}%
}%
\begin{pgfscope}%
\pgfsys@transformshift{0.515000in}{1.176427in}%
\pgfsys@useobject{currentmarker}{}%
\end{pgfscope}%
\end{pgfscope}%
\begin{pgfscope}%
\definecolor{textcolor}{rgb}{0.000000,0.000000,0.000000}%
\pgfsetstrokecolor{textcolor}%
\pgfsetfillcolor{textcolor}%
\pgftext[x=0.278889in, y=1.128232in, left, base]{\color{textcolor}\rmfamily\fontsize{10.000000}{12.000000}\selectfont \(\displaystyle {20}\)}%
\end{pgfscope}%
\begin{pgfscope}%
\pgfsetbuttcap%
\pgfsetroundjoin%
\definecolor{currentfill}{rgb}{0.000000,0.000000,0.000000}%
\pgfsetfillcolor{currentfill}%
\pgfsetlinewidth{0.803000pt}%
\definecolor{currentstroke}{rgb}{0.000000,0.000000,0.000000}%
\pgfsetstrokecolor{currentstroke}%
\pgfsetdash{}{0pt}%
\pgfsys@defobject{currentmarker}{\pgfqpoint{-0.048611in}{0.000000in}}{\pgfqpoint{-0.000000in}{0.000000in}}{%
\pgfpathmoveto{\pgfqpoint{-0.000000in}{0.000000in}}%
\pgfpathlineto{\pgfqpoint{-0.048611in}{0.000000in}}%
\pgfusepath{stroke,fill}%
}%
\begin{pgfscope}%
\pgfsys@transformshift{0.515000in}{1.514918in}%
\pgfsys@useobject{currentmarker}{}%
\end{pgfscope}%
\end{pgfscope}%
\begin{pgfscope}%
\definecolor{textcolor}{rgb}{0.000000,0.000000,0.000000}%
\pgfsetstrokecolor{textcolor}%
\pgfsetfillcolor{textcolor}%
\pgftext[x=0.278889in, y=1.466724in, left, base]{\color{textcolor}\rmfamily\fontsize{10.000000}{12.000000}\selectfont \(\displaystyle {30}\)}%
\end{pgfscope}%
\begin{pgfscope}%
\definecolor{textcolor}{rgb}{0.000000,0.000000,0.000000}%
\pgfsetstrokecolor{textcolor}%
\pgfsetfillcolor{textcolor}%
\pgftext[x=0.223333in,y=1.076944in,,bottom,rotate=90.000000]{\color{textcolor}\rmfamily\fontsize{10.000000}{12.000000}\selectfont Percent of Data Set}%
\end{pgfscope}%
\begin{pgfscope}%
\pgfsetrectcap%
\pgfsetmiterjoin%
\pgfsetlinewidth{0.803000pt}%
\definecolor{currentstroke}{rgb}{0.000000,0.000000,0.000000}%
\pgfsetstrokecolor{currentstroke}%
\pgfsetdash{}{0pt}%
\pgfpathmoveto{\pgfqpoint{0.515000in}{0.499444in}}%
\pgfpathlineto{\pgfqpoint{0.515000in}{1.654444in}}%
\pgfusepath{stroke}%
\end{pgfscope}%
\begin{pgfscope}%
\pgfsetrectcap%
\pgfsetmiterjoin%
\pgfsetlinewidth{0.803000pt}%
\definecolor{currentstroke}{rgb}{0.000000,0.000000,0.000000}%
\pgfsetstrokecolor{currentstroke}%
\pgfsetdash{}{0pt}%
\pgfpathmoveto{\pgfqpoint{2.065000in}{0.499444in}}%
\pgfpathlineto{\pgfqpoint{2.065000in}{1.654444in}}%
\pgfusepath{stroke}%
\end{pgfscope}%
\begin{pgfscope}%
\pgfsetrectcap%
\pgfsetmiterjoin%
\pgfsetlinewidth{0.803000pt}%
\definecolor{currentstroke}{rgb}{0.000000,0.000000,0.000000}%
\pgfsetstrokecolor{currentstroke}%
\pgfsetdash{}{0pt}%
\pgfpathmoveto{\pgfqpoint{0.515000in}{0.499444in}}%
\pgfpathlineto{\pgfqpoint{2.065000in}{0.499444in}}%
\pgfusepath{stroke}%
\end{pgfscope}%
\begin{pgfscope}%
\pgfsetrectcap%
\pgfsetmiterjoin%
\pgfsetlinewidth{0.803000pt}%
\definecolor{currentstroke}{rgb}{0.000000,0.000000,0.000000}%
\pgfsetstrokecolor{currentstroke}%
\pgfsetdash{}{0pt}%
\pgfpathmoveto{\pgfqpoint{0.515000in}{1.654444in}}%
\pgfpathlineto{\pgfqpoint{2.065000in}{1.654444in}}%
\pgfusepath{stroke}%
\end{pgfscope}%
\begin{pgfscope}%
\definecolor{textcolor}{rgb}{0.000000,0.000000,0.000000}%
\pgfsetstrokecolor{textcolor}%
\pgfsetfillcolor{textcolor}%
\pgftext[x=1.290000in,y=1.737778in,,base]{\color{textcolor}\rmfamily\fontsize{12.000000}{14.400000}\selectfont Probability Distribution}%
\end{pgfscope}%
\begin{pgfscope}%
\pgfsetbuttcap%
\pgfsetmiterjoin%
\definecolor{currentfill}{rgb}{1.000000,1.000000,1.000000}%
\pgfsetfillcolor{currentfill}%
\pgfsetfillopacity{0.800000}%
\pgfsetlinewidth{1.003750pt}%
\definecolor{currentstroke}{rgb}{0.800000,0.800000,0.800000}%
\pgfsetstrokecolor{currentstroke}%
\pgfsetstrokeopacity{0.800000}%
\pgfsetdash{}{0pt}%
\pgfpathmoveto{\pgfqpoint{1.288056in}{1.154445in}}%
\pgfpathlineto{\pgfqpoint{1.967778in}{1.154445in}}%
\pgfpathquadraticcurveto{\pgfqpoint{1.995556in}{1.154445in}}{\pgfqpoint{1.995556in}{1.182222in}}%
\pgfpathlineto{\pgfqpoint{1.995556in}{1.557222in}}%
\pgfpathquadraticcurveto{\pgfqpoint{1.995556in}{1.585000in}}{\pgfqpoint{1.967778in}{1.585000in}}%
\pgfpathlineto{\pgfqpoint{1.288056in}{1.585000in}}%
\pgfpathquadraticcurveto{\pgfqpoint{1.260278in}{1.585000in}}{\pgfqpoint{1.260278in}{1.557222in}}%
\pgfpathlineto{\pgfqpoint{1.260278in}{1.182222in}}%
\pgfpathquadraticcurveto{\pgfqpoint{1.260278in}{1.154445in}}{\pgfqpoint{1.288056in}{1.154445in}}%
\pgfpathlineto{\pgfqpoint{1.288056in}{1.154445in}}%
\pgfpathclose%
\pgfusepath{stroke,fill}%
\end{pgfscope}%
\begin{pgfscope}%
\pgfsetbuttcap%
\pgfsetmiterjoin%
\pgfsetlinewidth{1.003750pt}%
\definecolor{currentstroke}{rgb}{0.000000,0.000000,0.000000}%
\pgfsetstrokecolor{currentstroke}%
\pgfsetdash{}{0pt}%
\pgfpathmoveto{\pgfqpoint{1.315834in}{1.432222in}}%
\pgfpathlineto{\pgfqpoint{1.593611in}{1.432222in}}%
\pgfpathlineto{\pgfqpoint{1.593611in}{1.529444in}}%
\pgfpathlineto{\pgfqpoint{1.315834in}{1.529444in}}%
\pgfpathlineto{\pgfqpoint{1.315834in}{1.432222in}}%
\pgfpathclose%
\pgfusepath{stroke}%
\end{pgfscope}%
\begin{pgfscope}%
\definecolor{textcolor}{rgb}{0.000000,0.000000,0.000000}%
\pgfsetstrokecolor{textcolor}%
\pgfsetfillcolor{textcolor}%
\pgftext[x=1.704722in,y=1.432222in,left,base]{\color{textcolor}\rmfamily\fontsize{10.000000}{12.000000}\selectfont Neg}%
\end{pgfscope}%
\begin{pgfscope}%
\pgfsetbuttcap%
\pgfsetmiterjoin%
\definecolor{currentfill}{rgb}{0.000000,0.000000,0.000000}%
\pgfsetfillcolor{currentfill}%
\pgfsetlinewidth{0.000000pt}%
\definecolor{currentstroke}{rgb}{0.000000,0.000000,0.000000}%
\pgfsetstrokecolor{currentstroke}%
\pgfsetstrokeopacity{0.000000}%
\pgfsetdash{}{0pt}%
\pgfpathmoveto{\pgfqpoint{1.315834in}{1.236944in}}%
\pgfpathlineto{\pgfqpoint{1.593611in}{1.236944in}}%
\pgfpathlineto{\pgfqpoint{1.593611in}{1.334167in}}%
\pgfpathlineto{\pgfqpoint{1.315834in}{1.334167in}}%
\pgfpathlineto{\pgfqpoint{1.315834in}{1.236944in}}%
\pgfpathclose%
\pgfusepath{fill}%
\end{pgfscope}%
\begin{pgfscope}%
\definecolor{textcolor}{rgb}{0.000000,0.000000,0.000000}%
\pgfsetstrokecolor{textcolor}%
\pgfsetfillcolor{textcolor}%
\pgftext[x=1.704722in,y=1.236944in,left,base]{\color{textcolor}\rmfamily\fontsize{10.000000}{12.000000}\selectfont Pos}%
\end{pgfscope}%
\end{pgfpicture}%
\makeatother%
\endgroup%

  &
  \vspace{0pt} %% Creator: Matplotlib, PGF backend
%%
%% To include the figure in your LaTeX document, write
%%   \input{<filename>.pgf}
%%
%% Make sure the required packages are loaded in your preamble
%%   \usepackage{pgf}
%%
%% Also ensure that all the required font packages are loaded; for instance,
%% the lmodern package is sometimes necessary when using math font.
%%   \usepackage{lmodern}
%%
%% Figures using additional raster images can only be included by \input if
%% they are in the same directory as the main LaTeX file. For loading figures
%% from other directories you can use the `import` package
%%   \usepackage{import}
%%
%% and then include the figures with
%%   \import{<path to file>}{<filename>.pgf}
%%
%% Matplotlib used the following preamble
%%   
%%   \usepackage{fontspec}
%%   \makeatletter\@ifpackageloaded{underscore}{}{\usepackage[strings]{underscore}}\makeatother
%%
\begingroup%
\makeatletter%
\begin{pgfpicture}%
\pgfpathrectangle{\pgfpointorigin}{\pgfqpoint{2.221861in}{1.953444in}}%
\pgfusepath{use as bounding box, clip}%
\begin{pgfscope}%
\pgfsetbuttcap%
\pgfsetmiterjoin%
\definecolor{currentfill}{rgb}{1.000000,1.000000,1.000000}%
\pgfsetfillcolor{currentfill}%
\pgfsetlinewidth{0.000000pt}%
\definecolor{currentstroke}{rgb}{1.000000,1.000000,1.000000}%
\pgfsetstrokecolor{currentstroke}%
\pgfsetdash{}{0pt}%
\pgfpathmoveto{\pgfqpoint{0.000000in}{0.000000in}}%
\pgfpathlineto{\pgfqpoint{2.221861in}{0.000000in}}%
\pgfpathlineto{\pgfqpoint{2.221861in}{1.953444in}}%
\pgfpathlineto{\pgfqpoint{0.000000in}{1.953444in}}%
\pgfpathlineto{\pgfqpoint{0.000000in}{0.000000in}}%
\pgfpathclose%
\pgfusepath{fill}%
\end{pgfscope}%
\begin{pgfscope}%
\pgfsetbuttcap%
\pgfsetmiterjoin%
\definecolor{currentfill}{rgb}{1.000000,1.000000,1.000000}%
\pgfsetfillcolor{currentfill}%
\pgfsetlinewidth{0.000000pt}%
\definecolor{currentstroke}{rgb}{0.000000,0.000000,0.000000}%
\pgfsetstrokecolor{currentstroke}%
\pgfsetstrokeopacity{0.000000}%
\pgfsetdash{}{0pt}%
\pgfpathmoveto{\pgfqpoint{0.553581in}{0.499444in}}%
\pgfpathlineto{\pgfqpoint{2.103581in}{0.499444in}}%
\pgfpathlineto{\pgfqpoint{2.103581in}{1.654444in}}%
\pgfpathlineto{\pgfqpoint{0.553581in}{1.654444in}}%
\pgfpathlineto{\pgfqpoint{0.553581in}{0.499444in}}%
\pgfpathclose%
\pgfusepath{fill}%
\end{pgfscope}%
\begin{pgfscope}%
\pgfsetbuttcap%
\pgfsetroundjoin%
\definecolor{currentfill}{rgb}{0.000000,0.000000,0.000000}%
\pgfsetfillcolor{currentfill}%
\pgfsetlinewidth{0.803000pt}%
\definecolor{currentstroke}{rgb}{0.000000,0.000000,0.000000}%
\pgfsetstrokecolor{currentstroke}%
\pgfsetdash{}{0pt}%
\pgfsys@defobject{currentmarker}{\pgfqpoint{0.000000in}{-0.048611in}}{\pgfqpoint{0.000000in}{0.000000in}}{%
\pgfpathmoveto{\pgfqpoint{0.000000in}{0.000000in}}%
\pgfpathlineto{\pgfqpoint{0.000000in}{-0.048611in}}%
\pgfusepath{stroke,fill}%
}%
\begin{pgfscope}%
\pgfsys@transformshift{0.624035in}{0.499444in}%
\pgfsys@useobject{currentmarker}{}%
\end{pgfscope}%
\end{pgfscope}%
\begin{pgfscope}%
\definecolor{textcolor}{rgb}{0.000000,0.000000,0.000000}%
\pgfsetstrokecolor{textcolor}%
\pgfsetfillcolor{textcolor}%
\pgftext[x=0.624035in,y=0.402222in,,top]{\color{textcolor}\rmfamily\fontsize{10.000000}{12.000000}\selectfont \(\displaystyle {0.0}\)}%
\end{pgfscope}%
\begin{pgfscope}%
\pgfsetbuttcap%
\pgfsetroundjoin%
\definecolor{currentfill}{rgb}{0.000000,0.000000,0.000000}%
\pgfsetfillcolor{currentfill}%
\pgfsetlinewidth{0.803000pt}%
\definecolor{currentstroke}{rgb}{0.000000,0.000000,0.000000}%
\pgfsetstrokecolor{currentstroke}%
\pgfsetdash{}{0pt}%
\pgfsys@defobject{currentmarker}{\pgfqpoint{0.000000in}{-0.048611in}}{\pgfqpoint{0.000000in}{0.000000in}}{%
\pgfpathmoveto{\pgfqpoint{0.000000in}{0.000000in}}%
\pgfpathlineto{\pgfqpoint{0.000000in}{-0.048611in}}%
\pgfusepath{stroke,fill}%
}%
\begin{pgfscope}%
\pgfsys@transformshift{1.328581in}{0.499444in}%
\pgfsys@useobject{currentmarker}{}%
\end{pgfscope}%
\end{pgfscope}%
\begin{pgfscope}%
\definecolor{textcolor}{rgb}{0.000000,0.000000,0.000000}%
\pgfsetstrokecolor{textcolor}%
\pgfsetfillcolor{textcolor}%
\pgftext[x=1.328581in,y=0.402222in,,top]{\color{textcolor}\rmfamily\fontsize{10.000000}{12.000000}\selectfont \(\displaystyle {0.5}\)}%
\end{pgfscope}%
\begin{pgfscope}%
\pgfsetbuttcap%
\pgfsetroundjoin%
\definecolor{currentfill}{rgb}{0.000000,0.000000,0.000000}%
\pgfsetfillcolor{currentfill}%
\pgfsetlinewidth{0.803000pt}%
\definecolor{currentstroke}{rgb}{0.000000,0.000000,0.000000}%
\pgfsetstrokecolor{currentstroke}%
\pgfsetdash{}{0pt}%
\pgfsys@defobject{currentmarker}{\pgfqpoint{0.000000in}{-0.048611in}}{\pgfqpoint{0.000000in}{0.000000in}}{%
\pgfpathmoveto{\pgfqpoint{0.000000in}{0.000000in}}%
\pgfpathlineto{\pgfqpoint{0.000000in}{-0.048611in}}%
\pgfusepath{stroke,fill}%
}%
\begin{pgfscope}%
\pgfsys@transformshift{2.033126in}{0.499444in}%
\pgfsys@useobject{currentmarker}{}%
\end{pgfscope}%
\end{pgfscope}%
\begin{pgfscope}%
\definecolor{textcolor}{rgb}{0.000000,0.000000,0.000000}%
\pgfsetstrokecolor{textcolor}%
\pgfsetfillcolor{textcolor}%
\pgftext[x=2.033126in,y=0.402222in,,top]{\color{textcolor}\rmfamily\fontsize{10.000000}{12.000000}\selectfont \(\displaystyle {1.0}\)}%
\end{pgfscope}%
\begin{pgfscope}%
\definecolor{textcolor}{rgb}{0.000000,0.000000,0.000000}%
\pgfsetstrokecolor{textcolor}%
\pgfsetfillcolor{textcolor}%
\pgftext[x=1.328581in,y=0.223333in,,top]{\color{textcolor}\rmfamily\fontsize{10.000000}{12.000000}\selectfont False positive rate}%
\end{pgfscope}%
\begin{pgfscope}%
\pgfsetbuttcap%
\pgfsetroundjoin%
\definecolor{currentfill}{rgb}{0.000000,0.000000,0.000000}%
\pgfsetfillcolor{currentfill}%
\pgfsetlinewidth{0.803000pt}%
\definecolor{currentstroke}{rgb}{0.000000,0.000000,0.000000}%
\pgfsetstrokecolor{currentstroke}%
\pgfsetdash{}{0pt}%
\pgfsys@defobject{currentmarker}{\pgfqpoint{-0.048611in}{0.000000in}}{\pgfqpoint{-0.000000in}{0.000000in}}{%
\pgfpathmoveto{\pgfqpoint{-0.000000in}{0.000000in}}%
\pgfpathlineto{\pgfqpoint{-0.048611in}{0.000000in}}%
\pgfusepath{stroke,fill}%
}%
\begin{pgfscope}%
\pgfsys@transformshift{0.553581in}{0.551944in}%
\pgfsys@useobject{currentmarker}{}%
\end{pgfscope}%
\end{pgfscope}%
\begin{pgfscope}%
\definecolor{textcolor}{rgb}{0.000000,0.000000,0.000000}%
\pgfsetstrokecolor{textcolor}%
\pgfsetfillcolor{textcolor}%
\pgftext[x=0.278889in, y=0.503750in, left, base]{\color{textcolor}\rmfamily\fontsize{10.000000}{12.000000}\selectfont \(\displaystyle {0.0}\)}%
\end{pgfscope}%
\begin{pgfscope}%
\pgfsetbuttcap%
\pgfsetroundjoin%
\definecolor{currentfill}{rgb}{0.000000,0.000000,0.000000}%
\pgfsetfillcolor{currentfill}%
\pgfsetlinewidth{0.803000pt}%
\definecolor{currentstroke}{rgb}{0.000000,0.000000,0.000000}%
\pgfsetstrokecolor{currentstroke}%
\pgfsetdash{}{0pt}%
\pgfsys@defobject{currentmarker}{\pgfqpoint{-0.048611in}{0.000000in}}{\pgfqpoint{-0.000000in}{0.000000in}}{%
\pgfpathmoveto{\pgfqpoint{-0.000000in}{0.000000in}}%
\pgfpathlineto{\pgfqpoint{-0.048611in}{0.000000in}}%
\pgfusepath{stroke,fill}%
}%
\begin{pgfscope}%
\pgfsys@transformshift{0.553581in}{1.076944in}%
\pgfsys@useobject{currentmarker}{}%
\end{pgfscope}%
\end{pgfscope}%
\begin{pgfscope}%
\definecolor{textcolor}{rgb}{0.000000,0.000000,0.000000}%
\pgfsetstrokecolor{textcolor}%
\pgfsetfillcolor{textcolor}%
\pgftext[x=0.278889in, y=1.028750in, left, base]{\color{textcolor}\rmfamily\fontsize{10.000000}{12.000000}\selectfont \(\displaystyle {0.5}\)}%
\end{pgfscope}%
\begin{pgfscope}%
\pgfsetbuttcap%
\pgfsetroundjoin%
\definecolor{currentfill}{rgb}{0.000000,0.000000,0.000000}%
\pgfsetfillcolor{currentfill}%
\pgfsetlinewidth{0.803000pt}%
\definecolor{currentstroke}{rgb}{0.000000,0.000000,0.000000}%
\pgfsetstrokecolor{currentstroke}%
\pgfsetdash{}{0pt}%
\pgfsys@defobject{currentmarker}{\pgfqpoint{-0.048611in}{0.000000in}}{\pgfqpoint{-0.000000in}{0.000000in}}{%
\pgfpathmoveto{\pgfqpoint{-0.000000in}{0.000000in}}%
\pgfpathlineto{\pgfqpoint{-0.048611in}{0.000000in}}%
\pgfusepath{stroke,fill}%
}%
\begin{pgfscope}%
\pgfsys@transformshift{0.553581in}{1.601944in}%
\pgfsys@useobject{currentmarker}{}%
\end{pgfscope}%
\end{pgfscope}%
\begin{pgfscope}%
\definecolor{textcolor}{rgb}{0.000000,0.000000,0.000000}%
\pgfsetstrokecolor{textcolor}%
\pgfsetfillcolor{textcolor}%
\pgftext[x=0.278889in, y=1.553750in, left, base]{\color{textcolor}\rmfamily\fontsize{10.000000}{12.000000}\selectfont \(\displaystyle {1.0}\)}%
\end{pgfscope}%
\begin{pgfscope}%
\definecolor{textcolor}{rgb}{0.000000,0.000000,0.000000}%
\pgfsetstrokecolor{textcolor}%
\pgfsetfillcolor{textcolor}%
\pgftext[x=0.223333in,y=1.076944in,,bottom,rotate=90.000000]{\color{textcolor}\rmfamily\fontsize{10.000000}{12.000000}\selectfont True positive rate}%
\end{pgfscope}%
\begin{pgfscope}%
\pgfpathrectangle{\pgfqpoint{0.553581in}{0.499444in}}{\pgfqpoint{1.550000in}{1.155000in}}%
\pgfusepath{clip}%
\pgfsetbuttcap%
\pgfsetroundjoin%
\pgfsetlinewidth{1.505625pt}%
\definecolor{currentstroke}{rgb}{0.000000,0.000000,0.000000}%
\pgfsetstrokecolor{currentstroke}%
\pgfsetdash{{5.550000pt}{2.400000pt}}{0.000000pt}%
\pgfpathmoveto{\pgfqpoint{0.624035in}{0.551944in}}%
\pgfpathlineto{\pgfqpoint{2.033126in}{1.601944in}}%
\pgfusepath{stroke}%
\end{pgfscope}%
\begin{pgfscope}%
\pgfpathrectangle{\pgfqpoint{0.553581in}{0.499444in}}{\pgfqpoint{1.550000in}{1.155000in}}%
\pgfusepath{clip}%
\pgfsetrectcap%
\pgfsetroundjoin%
\pgfsetlinewidth{1.505625pt}%
\definecolor{currentstroke}{rgb}{0.000000,0.000000,0.000000}%
\pgfsetstrokecolor{currentstroke}%
\pgfsetdash{}{0pt}%
\pgfpathmoveto{\pgfqpoint{0.624035in}{0.551944in}}%
\pgfpathlineto{\pgfqpoint{0.626125in}{0.552364in}}%
\pgfpathlineto{\pgfqpoint{0.627217in}{0.562339in}}%
\pgfpathlineto{\pgfqpoint{0.627957in}{0.563442in}}%
\pgfpathlineto{\pgfqpoint{0.629002in}{0.565332in}}%
\pgfpathlineto{\pgfqpoint{0.629766in}{0.566434in}}%
\pgfpathlineto{\pgfqpoint{0.630858in}{0.569584in}}%
\pgfpathlineto{\pgfqpoint{0.631092in}{0.570477in}}%
\pgfpathlineto{\pgfqpoint{0.632184in}{0.574047in}}%
\pgfpathlineto{\pgfqpoint{0.632560in}{0.575149in}}%
\pgfpathlineto{\pgfqpoint{0.633664in}{0.579822in}}%
\pgfpathlineto{\pgfqpoint{0.633958in}{0.580924in}}%
\pgfpathlineto{\pgfqpoint{0.635061in}{0.585439in}}%
\pgfpathlineto{\pgfqpoint{0.635261in}{0.586332in}}%
\pgfpathlineto{\pgfqpoint{0.636341in}{0.592107in}}%
\pgfpathlineto{\pgfqpoint{0.636564in}{0.593209in}}%
\pgfpathlineto{\pgfqpoint{0.637656in}{0.600349in}}%
\pgfpathlineto{\pgfqpoint{0.637844in}{0.601347in}}%
\pgfpathlineto{\pgfqpoint{0.638948in}{0.607174in}}%
\pgfpathlineto{\pgfqpoint{0.639066in}{0.608277in}}%
\pgfpathlineto{\pgfqpoint{0.640134in}{0.614472in}}%
\pgfpathlineto{\pgfqpoint{0.640510in}{0.615574in}}%
\pgfpathlineto{\pgfqpoint{0.641590in}{0.622609in}}%
\pgfpathlineto{\pgfqpoint{0.641790in}{0.623712in}}%
\pgfpathlineto{\pgfqpoint{0.642847in}{0.631639in}}%
\pgfpathlineto{\pgfqpoint{0.643105in}{0.632637in}}%
\pgfpathlineto{\pgfqpoint{0.644209in}{0.641457in}}%
\pgfpathlineto{\pgfqpoint{0.644373in}{0.642559in}}%
\pgfpathlineto{\pgfqpoint{0.645477in}{0.650067in}}%
\pgfpathlineto{\pgfqpoint{0.645559in}{0.650644in}}%
\pgfpathlineto{\pgfqpoint{0.646651in}{0.658414in}}%
\pgfpathlineto{\pgfqpoint{0.646839in}{0.659464in}}%
\pgfpathlineto{\pgfqpoint{0.647943in}{0.665764in}}%
\pgfpathlineto{\pgfqpoint{0.648119in}{0.666814in}}%
\pgfpathlineto{\pgfqpoint{0.649223in}{0.674322in}}%
\pgfpathlineto{\pgfqpoint{0.649399in}{0.675424in}}%
\pgfpathlineto{\pgfqpoint{0.650503in}{0.684244in}}%
\pgfpathlineto{\pgfqpoint{0.650667in}{0.685189in}}%
\pgfpathlineto{\pgfqpoint{0.651736in}{0.692802in}}%
\pgfpathlineto{\pgfqpoint{0.651982in}{0.693747in}}%
\pgfpathlineto{\pgfqpoint{0.653086in}{0.702882in}}%
\pgfpathlineto{\pgfqpoint{0.653333in}{0.703932in}}%
\pgfpathlineto{\pgfqpoint{0.654425in}{0.711807in}}%
\pgfpathlineto{\pgfqpoint{0.654519in}{0.712332in}}%
\pgfpathlineto{\pgfqpoint{0.655622in}{0.721204in}}%
\pgfpathlineto{\pgfqpoint{0.655763in}{0.722202in}}%
\pgfpathlineto{\pgfqpoint{0.656855in}{0.729184in}}%
\pgfpathlineto{\pgfqpoint{0.657043in}{0.730234in}}%
\pgfpathlineto{\pgfqpoint{0.658135in}{0.738424in}}%
\pgfpathlineto{\pgfqpoint{0.658194in}{0.739369in}}%
\pgfpathlineto{\pgfqpoint{0.659298in}{0.747874in}}%
\pgfpathlineto{\pgfqpoint{0.659474in}{0.748819in}}%
\pgfpathlineto{\pgfqpoint{0.660578in}{0.757954in}}%
\pgfpathlineto{\pgfqpoint{0.660754in}{0.759057in}}%
\pgfpathlineto{\pgfqpoint{0.661846in}{0.765357in}}%
\pgfpathlineto{\pgfqpoint{0.662104in}{0.766302in}}%
\pgfpathlineto{\pgfqpoint{0.663208in}{0.774807in}}%
\pgfpathlineto{\pgfqpoint{0.663349in}{0.775909in}}%
\pgfpathlineto{\pgfqpoint{0.664453in}{0.782367in}}%
\pgfpathlineto{\pgfqpoint{0.664758in}{0.783417in}}%
\pgfpathlineto{\pgfqpoint{0.665862in}{0.791292in}}%
\pgfpathlineto{\pgfqpoint{0.666179in}{0.792237in}}%
\pgfpathlineto{\pgfqpoint{0.667283in}{0.800059in}}%
\pgfpathlineto{\pgfqpoint{0.667506in}{0.801162in}}%
\pgfpathlineto{\pgfqpoint{0.668609in}{0.808039in}}%
\pgfpathlineto{\pgfqpoint{0.668868in}{0.809089in}}%
\pgfpathlineto{\pgfqpoint{0.669972in}{0.815337in}}%
\pgfpathlineto{\pgfqpoint{0.670066in}{0.816387in}}%
\pgfpathlineto{\pgfqpoint{0.671169in}{0.823107in}}%
\pgfpathlineto{\pgfqpoint{0.671439in}{0.824209in}}%
\pgfpathlineto{\pgfqpoint{0.672543in}{0.833134in}}%
\pgfpathlineto{\pgfqpoint{0.672731in}{0.834237in}}%
\pgfpathlineto{\pgfqpoint{0.673823in}{0.840747in}}%
\pgfpathlineto{\pgfqpoint{0.674011in}{0.841797in}}%
\pgfpathlineto{\pgfqpoint{0.675115in}{0.846942in}}%
\pgfpathlineto{\pgfqpoint{0.675338in}{0.847992in}}%
\pgfpathlineto{\pgfqpoint{0.676442in}{0.853872in}}%
\pgfpathlineto{\pgfqpoint{0.676583in}{0.854974in}}%
\pgfpathlineto{\pgfqpoint{0.677663in}{0.861012in}}%
\pgfpathlineto{\pgfqpoint{0.677992in}{0.862009in}}%
\pgfpathlineto{\pgfqpoint{0.679072in}{0.869307in}}%
\pgfpathlineto{\pgfqpoint{0.679389in}{0.870409in}}%
\pgfpathlineto{\pgfqpoint{0.680493in}{0.877077in}}%
\pgfpathlineto{\pgfqpoint{0.680704in}{0.878127in}}%
\pgfpathlineto{\pgfqpoint{0.681773in}{0.882852in}}%
\pgfpathlineto{\pgfqpoint{0.681996in}{0.883954in}}%
\pgfpathlineto{\pgfqpoint{0.683100in}{0.890622in}}%
\pgfpathlineto{\pgfqpoint{0.683511in}{0.891672in}}%
\pgfpathlineto{\pgfqpoint{0.684603in}{0.896607in}}%
\pgfpathlineto{\pgfqpoint{0.684779in}{0.897657in}}%
\pgfpathlineto{\pgfqpoint{0.685836in}{0.903379in}}%
\pgfpathlineto{\pgfqpoint{0.686153in}{0.904482in}}%
\pgfpathlineto{\pgfqpoint{0.687256in}{0.909837in}}%
\pgfpathlineto{\pgfqpoint{0.687573in}{0.910939in}}%
\pgfpathlineto{\pgfqpoint{0.688572in}{0.915874in}}%
\pgfpathlineto{\pgfqpoint{0.688889in}{0.916977in}}%
\pgfpathlineto{\pgfqpoint{0.689992in}{0.922647in}}%
\pgfpathlineto{\pgfqpoint{0.690157in}{0.923749in}}%
\pgfpathlineto{\pgfqpoint{0.691261in}{0.930259in}}%
\pgfpathlineto{\pgfqpoint{0.691589in}{0.931362in}}%
\pgfpathlineto{\pgfqpoint{0.692670in}{0.936717in}}%
\pgfpathlineto{\pgfqpoint{0.692916in}{0.937819in}}%
\pgfpathlineto{\pgfqpoint{0.694020in}{0.942859in}}%
\pgfpathlineto{\pgfqpoint{0.694443in}{0.943962in}}%
\pgfpathlineto{\pgfqpoint{0.695500in}{0.948477in}}%
\pgfpathlineto{\pgfqpoint{0.695758in}{0.949579in}}%
\pgfpathlineto{\pgfqpoint{0.696862in}{0.954567in}}%
\pgfpathlineto{\pgfqpoint{0.697202in}{0.955617in}}%
\pgfpathlineto{\pgfqpoint{0.698306in}{0.961182in}}%
\pgfpathlineto{\pgfqpoint{0.698752in}{0.962232in}}%
\pgfpathlineto{\pgfqpoint{0.699856in}{0.966799in}}%
\pgfpathlineto{\pgfqpoint{0.700114in}{0.967902in}}%
\pgfpathlineto{\pgfqpoint{0.701218in}{0.972889in}}%
\pgfpathlineto{\pgfqpoint{0.701441in}{0.973939in}}%
\pgfpathlineto{\pgfqpoint{0.702545in}{0.978087in}}%
\pgfpathlineto{\pgfqpoint{0.702921in}{0.979189in}}%
\pgfpathlineto{\pgfqpoint{0.703989in}{0.983652in}}%
\pgfpathlineto{\pgfqpoint{0.704342in}{0.984754in}}%
\pgfpathlineto{\pgfqpoint{0.705434in}{0.989322in}}%
\pgfpathlineto{\pgfqpoint{0.705680in}{0.990424in}}%
\pgfpathlineto{\pgfqpoint{0.706772in}{0.995674in}}%
\pgfpathlineto{\pgfqpoint{0.707101in}{0.996724in}}%
\pgfpathlineto{\pgfqpoint{0.708205in}{1.001502in}}%
\pgfpathlineto{\pgfqpoint{0.708545in}{1.002604in}}%
\pgfpathlineto{\pgfqpoint{0.709649in}{1.006594in}}%
\pgfpathlineto{\pgfqpoint{0.709978in}{1.007592in}}%
\pgfpathlineto{\pgfqpoint{0.711023in}{1.010637in}}%
\pgfpathlineto{\pgfqpoint{0.711434in}{1.011739in}}%
\pgfpathlineto{\pgfqpoint{0.712503in}{1.016149in}}%
\pgfpathlineto{\pgfqpoint{0.712867in}{1.017252in}}%
\pgfpathlineto{\pgfqpoint{0.713970in}{1.020822in}}%
\pgfpathlineto{\pgfqpoint{0.714323in}{1.021819in}}%
\pgfpathlineto{\pgfqpoint{0.715427in}{1.025127in}}%
\pgfpathlineto{\pgfqpoint{0.715884in}{1.026124in}}%
\pgfpathlineto{\pgfqpoint{0.716977in}{1.029957in}}%
\pgfpathlineto{\pgfqpoint{0.717247in}{1.030954in}}%
\pgfpathlineto{\pgfqpoint{0.718339in}{1.036624in}}%
\pgfpathlineto{\pgfqpoint{0.718573in}{1.037674in}}%
\pgfpathlineto{\pgfqpoint{0.719677in}{1.041664in}}%
\pgfpathlineto{\pgfqpoint{0.720147in}{1.042767in}}%
\pgfpathlineto{\pgfqpoint{0.721227in}{1.045707in}}%
\pgfpathlineto{\pgfqpoint{0.721580in}{1.046757in}}%
\pgfpathlineto{\pgfqpoint{0.722660in}{1.050222in}}%
\pgfpathlineto{\pgfqpoint{0.723047in}{1.051324in}}%
\pgfpathlineto{\pgfqpoint{0.724139in}{1.054789in}}%
\pgfpathlineto{\pgfqpoint{0.724644in}{1.055892in}}%
\pgfpathlineto{\pgfqpoint{0.725736in}{1.059777in}}%
\pgfpathlineto{\pgfqpoint{0.726159in}{1.060827in}}%
\pgfpathlineto{\pgfqpoint{0.727263in}{1.063924in}}%
\pgfpathlineto{\pgfqpoint{0.727603in}{1.064974in}}%
\pgfpathlineto{\pgfqpoint{0.728672in}{1.068124in}}%
\pgfpathlineto{\pgfqpoint{0.729200in}{1.069227in}}%
\pgfpathlineto{\pgfqpoint{0.730304in}{1.072167in}}%
\pgfpathlineto{\pgfqpoint{0.730551in}{1.073269in}}%
\pgfpathlineto{\pgfqpoint{0.731631in}{1.076157in}}%
\pgfpathlineto{\pgfqpoint{0.732054in}{1.077259in}}%
\pgfpathlineto{\pgfqpoint{0.733122in}{1.080567in}}%
\pgfpathlineto{\pgfqpoint{0.733756in}{1.081669in}}%
\pgfpathlineto{\pgfqpoint{0.734848in}{1.084662in}}%
\pgfpathlineto{\pgfqpoint{0.735166in}{1.085764in}}%
\pgfpathlineto{\pgfqpoint{0.736258in}{1.089229in}}%
\pgfpathlineto{\pgfqpoint{0.736739in}{1.090332in}}%
\pgfpathlineto{\pgfqpoint{0.737808in}{1.093272in}}%
\pgfpathlineto{\pgfqpoint{0.738442in}{1.094374in}}%
\pgfpathlineto{\pgfqpoint{0.739522in}{1.098049in}}%
\pgfpathlineto{\pgfqpoint{0.739921in}{1.099152in}}%
\pgfpathlineto{\pgfqpoint{0.741025in}{1.101987in}}%
\pgfpathlineto{\pgfqpoint{0.741636in}{1.103089in}}%
\pgfpathlineto{\pgfqpoint{0.742739in}{1.105872in}}%
\pgfpathlineto{\pgfqpoint{0.743268in}{1.106974in}}%
\pgfpathlineto{\pgfqpoint{0.744360in}{1.109704in}}%
\pgfpathlineto{\pgfqpoint{0.744712in}{1.110754in}}%
\pgfpathlineto{\pgfqpoint{0.744747in}{1.110912in}}%
\pgfpathlineto{\pgfqpoint{0.744759in}{1.110912in}}%
\pgfpathlineto{\pgfqpoint{0.755363in}{1.112014in}}%
\pgfpathlineto{\pgfqpoint{0.756466in}{1.114377in}}%
\pgfpathlineto{\pgfqpoint{0.756772in}{1.115479in}}%
\pgfpathlineto{\pgfqpoint{0.757875in}{1.119627in}}%
\pgfpathlineto{\pgfqpoint{0.758592in}{1.120729in}}%
\pgfpathlineto{\pgfqpoint{0.759672in}{1.122882in}}%
\pgfpathlineto{\pgfqpoint{0.760153in}{1.123984in}}%
\pgfpathlineto{\pgfqpoint{0.761234in}{1.126819in}}%
\pgfpathlineto{\pgfqpoint{0.761727in}{1.127922in}}%
\pgfpathlineto{\pgfqpoint{0.762831in}{1.131387in}}%
\pgfpathlineto{\pgfqpoint{0.763312in}{1.132437in}}%
\pgfpathlineto{\pgfqpoint{0.764416in}{1.135219in}}%
\pgfpathlineto{\pgfqpoint{0.764897in}{1.136217in}}%
\pgfpathlineto{\pgfqpoint{0.765989in}{1.139104in}}%
\pgfpathlineto{\pgfqpoint{0.766389in}{1.140154in}}%
\pgfpathlineto{\pgfqpoint{0.767492in}{1.143199in}}%
\pgfpathlineto{\pgfqpoint{0.767997in}{1.144302in}}%
\pgfpathlineto{\pgfqpoint{0.769019in}{1.146349in}}%
\pgfpathlineto{\pgfqpoint{0.769547in}{1.147452in}}%
\pgfpathlineto{\pgfqpoint{0.770639in}{1.150182in}}%
\pgfpathlineto{\pgfqpoint{0.771273in}{1.151284in}}%
\pgfpathlineto{\pgfqpoint{0.772330in}{1.154014in}}%
\pgfpathlineto{\pgfqpoint{0.773129in}{1.155117in}}%
\pgfpathlineto{\pgfqpoint{0.774174in}{1.158004in}}%
\pgfpathlineto{\pgfqpoint{0.774855in}{1.159054in}}%
\pgfpathlineto{\pgfqpoint{0.775959in}{1.161364in}}%
\pgfpathlineto{\pgfqpoint{0.776170in}{1.162467in}}%
\pgfpathlineto{\pgfqpoint{0.777274in}{1.165197in}}%
\pgfpathlineto{\pgfqpoint{0.777779in}{1.166299in}}%
\pgfpathlineto{\pgfqpoint{0.778883in}{1.168819in}}%
\pgfpathlineto{\pgfqpoint{0.779481in}{1.169922in}}%
\pgfpathlineto{\pgfqpoint{0.780585in}{1.171339in}}%
\pgfpathlineto{\pgfqpoint{0.781008in}{1.172389in}}%
\pgfpathlineto{\pgfqpoint{0.782065in}{1.175119in}}%
\pgfpathlineto{\pgfqpoint{0.782581in}{1.176169in}}%
\pgfpathlineto{\pgfqpoint{0.783673in}{1.177797in}}%
\pgfpathlineto{\pgfqpoint{0.784261in}{1.178899in}}%
\pgfpathlineto{\pgfqpoint{0.785247in}{1.180684in}}%
\pgfpathlineto{\pgfqpoint{0.785822in}{1.181787in}}%
\pgfpathlineto{\pgfqpoint{0.786903in}{1.183887in}}%
\pgfpathlineto{\pgfqpoint{0.787830in}{1.184937in}}%
\pgfpathlineto{\pgfqpoint{0.788817in}{1.186879in}}%
\pgfpathlineto{\pgfqpoint{0.789556in}{1.187982in}}%
\pgfpathlineto{\pgfqpoint{0.790519in}{1.190029in}}%
\pgfpathlineto{\pgfqpoint{0.791353in}{1.191132in}}%
\pgfpathlineto{\pgfqpoint{0.792445in}{1.193442in}}%
\pgfpathlineto{\pgfqpoint{0.792997in}{1.194544in}}%
\pgfpathlineto{\pgfqpoint{0.794054in}{1.196067in}}%
\pgfpathlineto{\pgfqpoint{0.794876in}{1.197169in}}%
\pgfpathlineto{\pgfqpoint{0.795886in}{1.198482in}}%
\pgfpathlineto{\pgfqpoint{0.797201in}{1.199584in}}%
\pgfpathlineto{\pgfqpoint{0.798258in}{1.201579in}}%
\pgfpathlineto{\pgfqpoint{0.798739in}{1.202682in}}%
\pgfpathlineto{\pgfqpoint{0.799843in}{1.204572in}}%
\pgfpathlineto{\pgfqpoint{0.800195in}{1.205674in}}%
\pgfpathlineto{\pgfqpoint{0.801287in}{1.207564in}}%
\pgfpathlineto{\pgfqpoint{0.801874in}{1.208667in}}%
\pgfpathlineto{\pgfqpoint{0.802966in}{1.211239in}}%
\pgfpathlineto{\pgfqpoint{0.803882in}{1.212342in}}%
\pgfpathlineto{\pgfqpoint{0.804986in}{1.214337in}}%
\pgfpathlineto{\pgfqpoint{0.805714in}{1.215439in}}%
\pgfpathlineto{\pgfqpoint{0.806759in}{1.217277in}}%
\pgfpathlineto{\pgfqpoint{0.807111in}{1.218274in}}%
\pgfpathlineto{\pgfqpoint{0.808180in}{1.221109in}}%
\pgfpathlineto{\pgfqpoint{0.809084in}{1.222212in}}%
\pgfpathlineto{\pgfqpoint{0.810176in}{1.224049in}}%
\pgfpathlineto{\pgfqpoint{0.810892in}{1.225152in}}%
\pgfpathlineto{\pgfqpoint{0.811938in}{1.226517in}}%
\pgfpathlineto{\pgfqpoint{0.812478in}{1.227567in}}%
\pgfpathlineto{\pgfqpoint{0.813570in}{1.229614in}}%
\pgfpathlineto{\pgfqpoint{0.814122in}{1.230717in}}%
\pgfpathlineto{\pgfqpoint{0.815225in}{1.233027in}}%
\pgfpathlineto{\pgfqpoint{0.815942in}{1.234129in}}%
\pgfpathlineto{\pgfqpoint{0.816940in}{1.235809in}}%
\pgfpathlineto{\pgfqpoint{0.817820in}{1.236859in}}%
\pgfpathlineto{\pgfqpoint{0.818889in}{1.238697in}}%
\pgfpathlineto{\pgfqpoint{0.819429in}{1.239799in}}%
\pgfpathlineto{\pgfqpoint{0.820486in}{1.241689in}}%
\pgfpathlineto{\pgfqpoint{0.820897in}{1.242792in}}%
\pgfpathlineto{\pgfqpoint{0.821977in}{1.244367in}}%
\pgfpathlineto{\pgfqpoint{0.822881in}{1.245469in}}%
\pgfpathlineto{\pgfqpoint{0.823985in}{1.247464in}}%
\pgfpathlineto{\pgfqpoint{0.824655in}{1.248567in}}%
\pgfpathlineto{\pgfqpoint{0.825711in}{1.249827in}}%
\pgfpathlineto{\pgfqpoint{0.826369in}{1.250929in}}%
\pgfpathlineto{\pgfqpoint{0.827438in}{1.252767in}}%
\pgfpathlineto{\pgfqpoint{0.828001in}{1.253869in}}%
\pgfpathlineto{\pgfqpoint{0.829105in}{1.255812in}}%
\pgfpathlineto{\pgfqpoint{0.829845in}{1.256914in}}%
\pgfpathlineto{\pgfqpoint{0.830948in}{1.258804in}}%
\pgfpathlineto{\pgfqpoint{0.831712in}{1.259907in}}%
\pgfpathlineto{\pgfqpoint{0.832816in}{1.261797in}}%
\pgfpathlineto{\pgfqpoint{0.833861in}{1.262899in}}%
\pgfpathlineto{\pgfqpoint{0.834953in}{1.264422in}}%
\pgfpathlineto{\pgfqpoint{0.835739in}{1.265472in}}%
\pgfpathlineto{\pgfqpoint{0.836784in}{1.267257in}}%
\pgfpathlineto{\pgfqpoint{0.837477in}{1.268359in}}%
\pgfpathlineto{\pgfqpoint{0.838558in}{1.269724in}}%
\pgfpathlineto{\pgfqpoint{0.839920in}{1.270774in}}%
\pgfpathlineto{\pgfqpoint{0.841023in}{1.272664in}}%
\pgfpathlineto{\pgfqpoint{0.841975in}{1.273767in}}%
\pgfpathlineto{\pgfqpoint{0.843055in}{1.274869in}}%
\pgfpathlineto{\pgfqpoint{0.844217in}{1.275972in}}%
\pgfpathlineto{\pgfqpoint{0.845298in}{1.277809in}}%
\pgfpathlineto{\pgfqpoint{0.846002in}{1.278912in}}%
\pgfpathlineto{\pgfqpoint{0.847094in}{1.280172in}}%
\pgfpathlineto{\pgfqpoint{0.847963in}{1.281274in}}%
\pgfpathlineto{\pgfqpoint{0.849067in}{1.282902in}}%
\pgfpathlineto{\pgfqpoint{0.850124in}{1.284004in}}%
\pgfpathlineto{\pgfqpoint{0.851228in}{1.285474in}}%
\pgfpathlineto{\pgfqpoint{0.852472in}{1.286577in}}%
\pgfpathlineto{\pgfqpoint{0.853576in}{1.288624in}}%
\pgfpathlineto{\pgfqpoint{0.854645in}{1.289727in}}%
\pgfpathlineto{\pgfqpoint{0.855690in}{1.291197in}}%
\pgfpathlineto{\pgfqpoint{0.856934in}{1.292299in}}%
\pgfpathlineto{\pgfqpoint{0.857874in}{1.293297in}}%
\pgfpathlineto{\pgfqpoint{0.858766in}{1.294399in}}%
\pgfpathlineto{\pgfqpoint{0.859858in}{1.295817in}}%
\pgfpathlineto{\pgfqpoint{0.860868in}{1.296919in}}%
\pgfpathlineto{\pgfqpoint{0.861960in}{1.298179in}}%
\pgfpathlineto{\pgfqpoint{0.863205in}{1.299282in}}%
\pgfpathlineto{\pgfqpoint{0.864285in}{1.300962in}}%
\pgfpathlineto{\pgfqpoint{0.864837in}{1.302064in}}%
\pgfpathlineto{\pgfqpoint{0.865894in}{1.303482in}}%
\pgfpathlineto{\pgfqpoint{0.867009in}{1.304584in}}%
\pgfpathlineto{\pgfqpoint{0.868113in}{1.305582in}}%
\pgfpathlineto{\pgfqpoint{0.869205in}{1.306684in}}%
\pgfpathlineto{\pgfqpoint{0.870286in}{1.307682in}}%
\pgfpathlineto{\pgfqpoint{0.871131in}{1.308784in}}%
\pgfpathlineto{\pgfqpoint{0.872211in}{1.310464in}}%
\pgfpathlineto{\pgfqpoint{0.873092in}{1.311567in}}%
\pgfpathlineto{\pgfqpoint{0.874196in}{1.312932in}}%
\pgfpathlineto{\pgfqpoint{0.875523in}{1.314034in}}%
\pgfpathlineto{\pgfqpoint{0.876568in}{1.315189in}}%
\pgfpathlineto{\pgfqpoint{0.878141in}{1.316292in}}%
\pgfpathlineto{\pgfqpoint{0.879245in}{1.317919in}}%
\pgfpathlineto{\pgfqpoint{0.880513in}{1.319022in}}%
\pgfpathlineto{\pgfqpoint{0.881570in}{1.319967in}}%
\pgfpathlineto{\pgfqpoint{0.882521in}{1.321069in}}%
\pgfpathlineto{\pgfqpoint{0.883566in}{1.321962in}}%
\pgfpathlineto{\pgfqpoint{0.884564in}{1.323064in}}%
\pgfpathlineto{\pgfqpoint{0.885656in}{1.324639in}}%
\pgfpathlineto{\pgfqpoint{0.886455in}{1.325742in}}%
\pgfpathlineto{\pgfqpoint{0.887559in}{1.327054in}}%
\pgfpathlineto{\pgfqpoint{0.889250in}{1.328157in}}%
\pgfpathlineto{\pgfqpoint{0.890342in}{1.328944in}}%
\pgfpathlineto{\pgfqpoint{0.891316in}{1.330047in}}%
\pgfpathlineto{\pgfqpoint{0.892408in}{1.331307in}}%
\pgfpathlineto{\pgfqpoint{0.893864in}{1.332409in}}%
\pgfpathlineto{\pgfqpoint{0.894956in}{1.333774in}}%
\pgfpathlineto{\pgfqpoint{0.896166in}{1.334877in}}%
\pgfpathlineto{\pgfqpoint{0.897246in}{1.336032in}}%
\pgfpathlineto{\pgfqpoint{0.898667in}{1.337134in}}%
\pgfpathlineto{\pgfqpoint{0.899712in}{1.338237in}}%
\pgfpathlineto{\pgfqpoint{0.901520in}{1.339339in}}%
\pgfpathlineto{\pgfqpoint{0.902577in}{1.340284in}}%
\pgfpathlineto{\pgfqpoint{0.903446in}{1.341334in}}%
\pgfpathlineto{\pgfqpoint{0.904538in}{1.342594in}}%
\pgfpathlineto{\pgfqpoint{0.906065in}{1.343697in}}%
\pgfpathlineto{\pgfqpoint{0.907133in}{1.344484in}}%
\pgfpathlineto{\pgfqpoint{0.908625in}{1.345534in}}%
\pgfpathlineto{\pgfqpoint{0.909576in}{1.346322in}}%
\pgfpathlineto{\pgfqpoint{0.911314in}{1.347372in}}%
\pgfpathlineto{\pgfqpoint{0.912406in}{1.348894in}}%
\pgfpathlineto{\pgfqpoint{0.913674in}{1.349997in}}%
\pgfpathlineto{\pgfqpoint{0.914742in}{1.351047in}}%
\pgfpathlineto{\pgfqpoint{0.916281in}{1.352149in}}%
\pgfpathlineto{\pgfqpoint{0.917338in}{1.353042in}}%
\pgfpathlineto{\pgfqpoint{0.918993in}{1.354144in}}%
\pgfpathlineto{\pgfqpoint{0.920073in}{1.355037in}}%
\pgfpathlineto{\pgfqpoint{0.921330in}{1.356139in}}%
\pgfpathlineto{\pgfqpoint{0.922434in}{1.357662in}}%
\pgfpathlineto{\pgfqpoint{0.924042in}{1.358764in}}%
\pgfpathlineto{\pgfqpoint{0.925123in}{1.359814in}}%
\pgfpathlineto{\pgfqpoint{0.927354in}{1.360917in}}%
\pgfpathlineto{\pgfqpoint{0.928422in}{1.361967in}}%
\pgfpathlineto{\pgfqpoint{0.930242in}{1.363069in}}%
\pgfpathlineto{\pgfqpoint{0.931334in}{1.364277in}}%
\pgfpathlineto{\pgfqpoint{0.932673in}{1.365327in}}%
\pgfpathlineto{\pgfqpoint{0.933718in}{1.366534in}}%
\pgfpathlineto{\pgfqpoint{0.935397in}{1.367637in}}%
\pgfpathlineto{\pgfqpoint{0.936454in}{1.368477in}}%
\pgfpathlineto{\pgfqpoint{0.937898in}{1.369579in}}%
\pgfpathlineto{\pgfqpoint{0.938908in}{1.370262in}}%
\pgfpathlineto{\pgfqpoint{0.940822in}{1.371364in}}%
\pgfpathlineto{\pgfqpoint{0.941691in}{1.371994in}}%
\pgfpathlineto{\pgfqpoint{0.943417in}{1.373097in}}%
\pgfpathlineto{\pgfqpoint{0.944427in}{1.373937in}}%
\pgfpathlineto{\pgfqpoint{0.945754in}{1.375039in}}%
\pgfpathlineto{\pgfqpoint{0.946799in}{1.375617in}}%
\pgfpathlineto{\pgfqpoint{0.948408in}{1.376719in}}%
\pgfpathlineto{\pgfqpoint{0.949465in}{1.377717in}}%
\pgfpathlineto{\pgfqpoint{0.951203in}{1.378819in}}%
\pgfpathlineto{\pgfqpoint{0.952271in}{1.379554in}}%
\pgfpathlineto{\pgfqpoint{0.953504in}{1.380604in}}%
\pgfpathlineto{\pgfqpoint{0.954573in}{1.381182in}}%
\pgfpathlineto{\pgfqpoint{0.955970in}{1.382179in}}%
\pgfpathlineto{\pgfqpoint{0.957003in}{1.383124in}}%
\pgfpathlineto{\pgfqpoint{0.959105in}{1.384227in}}%
\pgfpathlineto{\pgfqpoint{0.960162in}{1.384909in}}%
\pgfpathlineto{\pgfqpoint{0.962217in}{1.386012in}}%
\pgfpathlineto{\pgfqpoint{0.963297in}{1.386694in}}%
\pgfpathlineto{\pgfqpoint{0.965575in}{1.387797in}}%
\pgfpathlineto{\pgfqpoint{0.966609in}{1.388374in}}%
\pgfpathlineto{\pgfqpoint{0.968030in}{1.389372in}}%
\pgfpathlineto{\pgfqpoint{0.969122in}{1.390422in}}%
\pgfpathlineto{\pgfqpoint{0.971423in}{1.391524in}}%
\pgfpathlineto{\pgfqpoint{0.972527in}{1.392522in}}%
\pgfpathlineto{\pgfqpoint{0.974652in}{1.393624in}}%
\pgfpathlineto{\pgfqpoint{0.975697in}{1.394307in}}%
\pgfpathlineto{\pgfqpoint{0.978069in}{1.395409in}}%
\pgfpathlineto{\pgfqpoint{0.979150in}{1.396092in}}%
\pgfpathlineto{\pgfqpoint{0.980782in}{1.397194in}}%
\pgfpathlineto{\pgfqpoint{0.981862in}{1.397824in}}%
\pgfpathlineto{\pgfqpoint{0.984093in}{1.398822in}}%
\pgfpathlineto{\pgfqpoint{0.985173in}{1.399767in}}%
\pgfpathlineto{\pgfqpoint{0.987581in}{1.400869in}}%
\pgfpathlineto{\pgfqpoint{0.988649in}{1.401342in}}%
\pgfpathlineto{\pgfqpoint{0.991139in}{1.402444in}}%
\pgfpathlineto{\pgfqpoint{0.992242in}{1.402864in}}%
\pgfpathlineto{\pgfqpoint{0.993816in}{1.403914in}}%
\pgfpathlineto{\pgfqpoint{0.994697in}{1.404807in}}%
\pgfpathlineto{\pgfqpoint{0.996951in}{1.405909in}}%
\pgfpathlineto{\pgfqpoint{0.998055in}{1.406749in}}%
\pgfpathlineto{\pgfqpoint{1.000216in}{1.407852in}}%
\pgfpathlineto{\pgfqpoint{1.001284in}{1.408377in}}%
\pgfpathlineto{\pgfqpoint{1.003069in}{1.409427in}}%
\pgfpathlineto{\pgfqpoint{1.004044in}{1.410214in}}%
\pgfpathlineto{\pgfqpoint{1.005746in}{1.411264in}}%
\pgfpathlineto{\pgfqpoint{1.006838in}{1.412262in}}%
\pgfpathlineto{\pgfqpoint{1.008858in}{1.413364in}}%
\pgfpathlineto{\pgfqpoint{1.009962in}{1.413837in}}%
\pgfpathlineto{\pgfqpoint{1.012909in}{1.414939in}}%
\pgfpathlineto{\pgfqpoint{1.013790in}{1.415307in}}%
\pgfpathlineto{\pgfqpoint{1.015786in}{1.416357in}}%
\pgfpathlineto{\pgfqpoint{1.016855in}{1.417092in}}%
\pgfpathlineto{\pgfqpoint{1.018604in}{1.418194in}}%
\pgfpathlineto{\pgfqpoint{1.019708in}{1.419087in}}%
\pgfpathlineto{\pgfqpoint{1.021857in}{1.420137in}}%
\pgfpathlineto{\pgfqpoint{1.022937in}{1.420767in}}%
\pgfpathlineto{\pgfqpoint{1.025673in}{1.421869in}}%
\pgfpathlineto{\pgfqpoint{1.026683in}{1.422657in}}%
\pgfpathlineto{\pgfqpoint{1.029219in}{1.423759in}}%
\pgfpathlineto{\pgfqpoint{1.030123in}{1.424599in}}%
\pgfpathlineto{\pgfqpoint{1.033071in}{1.425702in}}%
\pgfpathlineto{\pgfqpoint{1.034092in}{1.426647in}}%
\pgfpathlineto{\pgfqpoint{1.037333in}{1.427749in}}%
\pgfpathlineto{\pgfqpoint{1.038414in}{1.428484in}}%
\pgfpathlineto{\pgfqpoint{1.039917in}{1.429587in}}%
\pgfpathlineto{\pgfqpoint{1.040962in}{1.430217in}}%
\pgfpathlineto{\pgfqpoint{1.043486in}{1.431319in}}%
\pgfpathlineto{\pgfqpoint{1.044461in}{1.431897in}}%
\pgfpathlineto{\pgfqpoint{1.046622in}{1.432999in}}%
\pgfpathlineto{\pgfqpoint{1.047479in}{1.433629in}}%
\pgfpathlineto{\pgfqpoint{1.049498in}{1.434732in}}%
\pgfpathlineto{\pgfqpoint{1.050532in}{1.435204in}}%
\pgfpathlineto{\pgfqpoint{1.052035in}{1.436307in}}%
\pgfpathlineto{\pgfqpoint{1.052786in}{1.436989in}}%
\pgfpathlineto{\pgfqpoint{1.055922in}{1.438092in}}%
\pgfpathlineto{\pgfqpoint{1.056802in}{1.438354in}}%
\pgfpathlineto{\pgfqpoint{1.059573in}{1.439457in}}%
\pgfpathlineto{\pgfqpoint{1.060548in}{1.440139in}}%
\pgfpathlineto{\pgfqpoint{1.063178in}{1.441242in}}%
\pgfpathlineto{\pgfqpoint{1.064118in}{1.441872in}}%
\pgfpathlineto{\pgfqpoint{1.067370in}{1.442974in}}%
\pgfpathlineto{\pgfqpoint{1.068416in}{1.443499in}}%
\pgfpathlineto{\pgfqpoint{1.071116in}{1.444602in}}%
\pgfpathlineto{\pgfqpoint{1.072197in}{1.445284in}}%
\pgfpathlineto{\pgfqpoint{1.074498in}{1.446387in}}%
\pgfpathlineto{\pgfqpoint{1.075555in}{1.447122in}}%
\pgfpathlineto{\pgfqpoint{1.078573in}{1.448224in}}%
\pgfpathlineto{\pgfqpoint{1.079677in}{1.449064in}}%
\pgfpathlineto{\pgfqpoint{1.083575in}{1.450167in}}%
\pgfpathlineto{\pgfqpoint{1.084362in}{1.450482in}}%
\pgfpathlineto{\pgfqpoint{1.087485in}{1.451584in}}%
\pgfpathlineto{\pgfqpoint{1.088589in}{1.452267in}}%
\pgfpathlineto{\pgfqpoint{1.091442in}{1.453369in}}%
\pgfpathlineto{\pgfqpoint{1.092452in}{1.453842in}}%
\pgfpathlineto{\pgfqpoint{1.096339in}{1.454944in}}%
\pgfpathlineto{\pgfqpoint{1.097196in}{1.455312in}}%
\pgfpathlineto{\pgfqpoint{1.099721in}{1.456414in}}%
\pgfpathlineto{\pgfqpoint{1.100766in}{1.456939in}}%
\pgfpathlineto{\pgfqpoint{1.103936in}{1.458042in}}%
\pgfpathlineto{\pgfqpoint{1.104735in}{1.458252in}}%
\pgfpathlineto{\pgfqpoint{1.108716in}{1.459354in}}%
\pgfpathlineto{\pgfqpoint{1.109702in}{1.459774in}}%
\pgfpathlineto{\pgfqpoint{1.113553in}{1.460877in}}%
\pgfpathlineto{\pgfqpoint{1.114434in}{1.461297in}}%
\pgfpathlineto{\pgfqpoint{1.117346in}{1.462399in}}%
\pgfpathlineto{\pgfqpoint{1.118156in}{1.462819in}}%
\pgfpathlineto{\pgfqpoint{1.120834in}{1.463922in}}%
\pgfpathlineto{\pgfqpoint{1.121914in}{1.464499in}}%
\pgfpathlineto{\pgfqpoint{1.125437in}{1.465602in}}%
\pgfpathlineto{\pgfqpoint{1.126400in}{1.465917in}}%
\pgfpathlineto{\pgfqpoint{1.130545in}{1.467019in}}%
\pgfpathlineto{\pgfqpoint{1.131425in}{1.467334in}}%
\pgfpathlineto{\pgfqpoint{1.134044in}{1.468384in}}%
\pgfpathlineto{\pgfqpoint{1.134925in}{1.468804in}}%
\pgfpathlineto{\pgfqpoint{1.137438in}{1.469907in}}%
\pgfpathlineto{\pgfqpoint{1.138541in}{1.470642in}}%
\pgfpathlineto{\pgfqpoint{1.142557in}{1.471744in}}%
\pgfpathlineto{\pgfqpoint{1.143156in}{1.472007in}}%
\pgfpathlineto{\pgfqpoint{1.147923in}{1.473109in}}%
\pgfpathlineto{\pgfqpoint{1.148745in}{1.473529in}}%
\pgfpathlineto{\pgfqpoint{1.151740in}{1.474632in}}%
\pgfpathlineto{\pgfqpoint{1.152820in}{1.475052in}}%
\pgfpathlineto{\pgfqpoint{1.155944in}{1.476154in}}%
\pgfpathlineto{\pgfqpoint{1.157012in}{1.476627in}}%
\pgfpathlineto{\pgfqpoint{1.161263in}{1.477729in}}%
\pgfpathlineto{\pgfqpoint{1.162144in}{1.478254in}}%
\pgfpathlineto{\pgfqpoint{1.166430in}{1.479357in}}%
\pgfpathlineto{\pgfqpoint{1.167439in}{1.479934in}}%
\pgfpathlineto{\pgfqpoint{1.170845in}{1.481037in}}%
\pgfpathlineto{\pgfqpoint{1.171725in}{1.481352in}}%
\pgfpathlineto{\pgfqpoint{1.176340in}{1.482454in}}%
\pgfpathlineto{\pgfqpoint{1.177432in}{1.482769in}}%
\pgfpathlineto{\pgfqpoint{1.182305in}{1.483872in}}%
\pgfpathlineto{\pgfqpoint{1.183386in}{1.484397in}}%
\pgfpathlineto{\pgfqpoint{1.188047in}{1.485499in}}%
\pgfpathlineto{\pgfqpoint{1.189034in}{1.486077in}}%
\pgfpathlineto{\pgfqpoint{1.194095in}{1.487179in}}%
\pgfpathlineto{\pgfqpoint{1.194964in}{1.487494in}}%
\pgfpathlineto{\pgfqpoint{1.199484in}{1.488597in}}%
\pgfpathlineto{\pgfqpoint{1.200518in}{1.489227in}}%
\pgfpathlineto{\pgfqpoint{1.203418in}{1.490329in}}%
\pgfpathlineto{\pgfqpoint{1.204252in}{1.490697in}}%
\pgfpathlineto{\pgfqpoint{1.207211in}{1.491799in}}%
\pgfpathlineto{\pgfqpoint{1.207211in}{1.491852in}}%
\pgfpathlineto{\pgfqpoint{1.211791in}{1.492954in}}%
\pgfpathlineto{\pgfqpoint{1.212789in}{1.493322in}}%
\pgfpathlineto{\pgfqpoint{1.217192in}{1.494424in}}%
\pgfpathlineto{\pgfqpoint{1.217897in}{1.494634in}}%
\pgfpathlineto{\pgfqpoint{1.223005in}{1.495737in}}%
\pgfpathlineto{\pgfqpoint{1.223897in}{1.495842in}}%
\pgfpathlineto{\pgfqpoint{1.226621in}{1.496944in}}%
\pgfpathlineto{\pgfqpoint{1.227643in}{1.497469in}}%
\pgfpathlineto{\pgfqpoint{1.235322in}{1.498572in}}%
\pgfpathlineto{\pgfqpoint{1.236367in}{1.498729in}}%
\pgfpathlineto{\pgfqpoint{1.241757in}{1.499832in}}%
\pgfpathlineto{\pgfqpoint{1.242133in}{1.499989in}}%
\pgfpathlineto{\pgfqpoint{1.248063in}{1.501092in}}%
\pgfpathlineto{\pgfqpoint{1.248991in}{1.501354in}}%
\pgfpathlineto{\pgfqpoint{1.254028in}{1.502457in}}%
\pgfpathlineto{\pgfqpoint{1.255108in}{1.502614in}}%
\pgfpathlineto{\pgfqpoint{1.259946in}{1.503717in}}%
\pgfpathlineto{\pgfqpoint{1.260933in}{1.503927in}}%
\pgfpathlineto{\pgfqpoint{1.267696in}{1.505029in}}%
\pgfpathlineto{\pgfqpoint{1.268765in}{1.505344in}}%
\pgfpathlineto{\pgfqpoint{1.274248in}{1.506447in}}%
\pgfpathlineto{\pgfqpoint{1.274918in}{1.506552in}}%
\pgfpathlineto{\pgfqpoint{1.282128in}{1.507654in}}%
\pgfpathlineto{\pgfqpoint{1.283008in}{1.508127in}}%
\pgfpathlineto{\pgfqpoint{1.289420in}{1.509229in}}%
\pgfpathlineto{\pgfqpoint{1.290465in}{1.509649in}}%
\pgfpathlineto{\pgfqpoint{1.296723in}{1.510752in}}%
\pgfpathlineto{\pgfqpoint{1.297722in}{1.511014in}}%
\pgfpathlineto{\pgfqpoint{1.303393in}{1.512117in}}%
\pgfpathlineto{\pgfqpoint{1.304485in}{1.512379in}}%
\pgfpathlineto{\pgfqpoint{1.311084in}{1.513429in}}%
\pgfpathlineto{\pgfqpoint{1.311460in}{1.513587in}}%
\pgfpathlineto{\pgfqpoint{1.319445in}{1.514689in}}%
\pgfpathlineto{\pgfqpoint{1.320220in}{1.515109in}}%
\pgfpathlineto{\pgfqpoint{1.326338in}{1.516212in}}%
\pgfpathlineto{\pgfqpoint{1.327301in}{1.516474in}}%
\pgfpathlineto{\pgfqpoint{1.334041in}{1.517577in}}%
\pgfpathlineto{\pgfqpoint{1.334886in}{1.517839in}}%
\pgfpathlineto{\pgfqpoint{1.340664in}{1.518942in}}%
\pgfpathlineto{\pgfqpoint{1.340664in}{1.518994in}}%
\pgfpathlineto{\pgfqpoint{1.350304in}{1.520097in}}%
\pgfpathlineto{\pgfqpoint{1.351279in}{1.520464in}}%
\pgfpathlineto{\pgfqpoint{1.358007in}{1.521567in}}%
\pgfpathlineto{\pgfqpoint{1.358888in}{1.521724in}}%
\pgfpathlineto{\pgfqpoint{1.365264in}{1.522827in}}%
\pgfpathlineto{\pgfqpoint{1.366098in}{1.523037in}}%
\pgfpathlineto{\pgfqpoint{1.372697in}{1.524139in}}%
\pgfpathlineto{\pgfqpoint{1.373413in}{1.524349in}}%
\pgfpathlineto{\pgfqpoint{1.380964in}{1.525452in}}%
\pgfpathlineto{\pgfqpoint{1.382044in}{1.525977in}}%
\pgfpathlineto{\pgfqpoint{1.387222in}{1.527079in}}%
\pgfpathlineto{\pgfqpoint{1.387974in}{1.527237in}}%
\pgfpathlineto{\pgfqpoint{1.394338in}{1.528339in}}%
\pgfpathlineto{\pgfqpoint{1.395442in}{1.528654in}}%
\pgfpathlineto{\pgfqpoint{1.400174in}{1.529757in}}%
\pgfpathlineto{\pgfqpoint{1.400527in}{1.529914in}}%
\pgfpathlineto{\pgfqpoint{1.408605in}{1.531017in}}%
\pgfpathlineto{\pgfqpoint{1.409615in}{1.531227in}}%
\pgfpathlineto{\pgfqpoint{1.417201in}{1.532329in}}%
\pgfpathlineto{\pgfqpoint{1.418258in}{1.532539in}}%
\pgfpathlineto{\pgfqpoint{1.423964in}{1.533642in}}%
\pgfpathlineto{\pgfqpoint{1.424822in}{1.533852in}}%
\pgfpathlineto{\pgfqpoint{1.432008in}{1.534954in}}%
\pgfpathlineto{\pgfqpoint{1.432266in}{1.535112in}}%
\pgfpathlineto{\pgfqpoint{1.442353in}{1.536214in}}%
\pgfpathlineto{\pgfqpoint{1.442564in}{1.536319in}}%
\pgfpathlineto{\pgfqpoint{1.449504in}{1.537422in}}%
\pgfpathlineto{\pgfqpoint{1.450267in}{1.537842in}}%
\pgfpathlineto{\pgfqpoint{1.458534in}{1.538944in}}%
\pgfpathlineto{\pgfqpoint{1.459356in}{1.539102in}}%
\pgfpathlineto{\pgfqpoint{1.473083in}{1.540204in}}%
\pgfpathlineto{\pgfqpoint{1.473306in}{1.540362in}}%
\pgfpathlineto{\pgfqpoint{1.482829in}{1.541464in}}%
\pgfpathlineto{\pgfqpoint{1.483710in}{1.541569in}}%
\pgfpathlineto{\pgfqpoint{1.495405in}{1.542672in}}%
\pgfpathlineto{\pgfqpoint{1.496157in}{1.542934in}}%
\pgfpathlineto{\pgfqpoint{1.504470in}{1.544037in}}%
\pgfpathlineto{\pgfqpoint{1.505410in}{1.544299in}}%
\pgfpathlineto{\pgfqpoint{1.515121in}{1.545402in}}%
\pgfpathlineto{\pgfqpoint{1.516213in}{1.545559in}}%
\pgfpathlineto{\pgfqpoint{1.523775in}{1.546662in}}%
\pgfpathlineto{\pgfqpoint{1.524515in}{1.546819in}}%
\pgfpathlineto{\pgfqpoint{1.532006in}{1.547922in}}%
\pgfpathlineto{\pgfqpoint{1.532958in}{1.548079in}}%
\pgfpathlineto{\pgfqpoint{1.541835in}{1.549182in}}%
\pgfpathlineto{\pgfqpoint{1.542492in}{1.549287in}}%
\pgfpathlineto{\pgfqpoint{1.548258in}{1.550389in}}%
\pgfpathlineto{\pgfqpoint{1.549021in}{1.550652in}}%
\pgfpathlineto{\pgfqpoint{1.558955in}{1.551754in}}%
\pgfpathlineto{\pgfqpoint{1.559859in}{1.552069in}}%
\pgfpathlineto{\pgfqpoint{1.571132in}{1.553172in}}%
\pgfpathlineto{\pgfqpoint{1.571895in}{1.553329in}}%
\pgfpathlineto{\pgfqpoint{1.581783in}{1.554432in}}%
\pgfpathlineto{\pgfqpoint{1.582652in}{1.554589in}}%
\pgfpathlineto{\pgfqpoint{1.594253in}{1.555692in}}%
\pgfpathlineto{\pgfqpoint{1.595322in}{1.555902in}}%
\pgfpathlineto{\pgfqpoint{1.605314in}{1.557004in}}%
\pgfpathlineto{\pgfqpoint{1.605890in}{1.557162in}}%
\pgfpathlineto{\pgfqpoint{1.619534in}{1.558264in}}%
\pgfpathlineto{\pgfqpoint{1.620333in}{1.558474in}}%
\pgfpathlineto{\pgfqpoint{1.630772in}{1.559577in}}%
\pgfpathlineto{\pgfqpoint{1.630819in}{1.559682in}}%
\pgfpathlineto{\pgfqpoint{1.642738in}{1.560784in}}%
\pgfpathlineto{\pgfqpoint{1.643677in}{1.561047in}}%
\pgfpathlineto{\pgfqpoint{1.656793in}{1.562097in}}%
\pgfpathlineto{\pgfqpoint{1.657592in}{1.562307in}}%
\pgfpathlineto{\pgfqpoint{1.669041in}{1.563409in}}%
\pgfpathlineto{\pgfqpoint{1.669698in}{1.563619in}}%
\pgfpathlineto{\pgfqpoint{1.683989in}{1.564722in}}%
\pgfpathlineto{\pgfqpoint{1.685010in}{1.564879in}}%
\pgfpathlineto{\pgfqpoint{1.700311in}{1.565982in}}%
\pgfpathlineto{\pgfqpoint{1.700311in}{1.566034in}}%
\pgfpathlineto{\pgfqpoint{1.717666in}{1.567137in}}%
\pgfpathlineto{\pgfqpoint{1.718006in}{1.567294in}}%
\pgfpathlineto{\pgfqpoint{1.730125in}{1.568397in}}%
\pgfpathlineto{\pgfqpoint{1.730747in}{1.568712in}}%
\pgfpathlineto{\pgfqpoint{1.744298in}{1.569814in}}%
\pgfpathlineto{\pgfqpoint{1.745155in}{1.569972in}}%
\pgfpathlineto{\pgfqpoint{1.760138in}{1.571074in}}%
\pgfpathlineto{\pgfqpoint{1.760913in}{1.571389in}}%
\pgfpathlineto{\pgfqpoint{1.773677in}{1.572492in}}%
\pgfpathlineto{\pgfqpoint{1.774734in}{1.572754in}}%
\pgfpathlineto{\pgfqpoint{1.792406in}{1.573857in}}%
\pgfpathlineto{\pgfqpoint{1.793240in}{1.574067in}}%
\pgfpathlineto{\pgfqpoint{1.804043in}{1.575169in}}%
\pgfpathlineto{\pgfqpoint{1.804278in}{1.575274in}}%
\pgfpathlineto{\pgfqpoint{1.819355in}{1.576377in}}%
\pgfpathlineto{\pgfqpoint{1.819813in}{1.576534in}}%
\pgfpathlineto{\pgfqpoint{1.832014in}{1.577637in}}%
\pgfpathlineto{\pgfqpoint{1.833094in}{1.577794in}}%
\pgfpathlineto{\pgfqpoint{1.850508in}{1.578897in}}%
\pgfpathlineto{\pgfqpoint{1.850966in}{1.579002in}}%
\pgfpathlineto{\pgfqpoint{1.865315in}{1.580104in}}%
\pgfpathlineto{\pgfqpoint{1.866219in}{1.580367in}}%
\pgfpathlineto{\pgfqpoint{1.878279in}{1.581469in}}%
\pgfpathlineto{\pgfqpoint{1.878948in}{1.581574in}}%
\pgfpathlineto{\pgfqpoint{1.891454in}{1.582677in}}%
\pgfpathlineto{\pgfqpoint{1.892147in}{1.582887in}}%
\pgfpathlineto{\pgfqpoint{1.905075in}{1.583989in}}%
\pgfpathlineto{\pgfqpoint{1.906155in}{1.584199in}}%
\pgfpathlineto{\pgfqpoint{1.919694in}{1.585302in}}%
\pgfpathlineto{\pgfqpoint{1.920704in}{1.585512in}}%
\pgfpathlineto{\pgfqpoint{1.935558in}{1.586614in}}%
\pgfpathlineto{\pgfqpoint{1.936392in}{1.586772in}}%
\pgfpathlineto{\pgfqpoint{1.948369in}{1.587874in}}%
\pgfpathlineto{\pgfqpoint{1.949379in}{1.588242in}}%
\pgfpathlineto{\pgfqpoint{1.964691in}{1.589344in}}%
\pgfpathlineto{\pgfqpoint{1.965008in}{1.589554in}}%
\pgfpathlineto{\pgfqpoint{1.978606in}{1.590657in}}%
\pgfpathlineto{\pgfqpoint{1.979675in}{1.590867in}}%
\pgfpathlineto{\pgfqpoint{1.990513in}{1.591969in}}%
\pgfpathlineto{\pgfqpoint{1.990513in}{1.592022in}}%
\pgfpathlineto{\pgfqpoint{2.000118in}{1.593124in}}%
\pgfpathlineto{\pgfqpoint{2.001210in}{1.593387in}}%
\pgfpathlineto{\pgfqpoint{2.010581in}{1.594489in}}%
\pgfpathlineto{\pgfqpoint{2.011320in}{1.594699in}}%
\pgfpathlineto{\pgfqpoint{2.021783in}{1.595802in}}%
\pgfpathlineto{\pgfqpoint{2.022441in}{1.596169in}}%
\pgfpathlineto{\pgfqpoint{2.027243in}{1.597272in}}%
\pgfpathlineto{\pgfqpoint{2.028053in}{1.597902in}}%
\pgfpathlineto{\pgfqpoint{2.031001in}{1.599004in}}%
\pgfpathlineto{\pgfqpoint{2.031834in}{1.599424in}}%
\pgfpathlineto{\pgfqpoint{2.033044in}{1.600527in}}%
\pgfpathlineto{\pgfqpoint{2.033126in}{1.601944in}}%
\pgfpathlineto{\pgfqpoint{2.033126in}{1.601944in}}%
\pgfusepath{stroke}%
\end{pgfscope}%
\begin{pgfscope}%
\pgfsetrectcap%
\pgfsetmiterjoin%
\pgfsetlinewidth{0.803000pt}%
\definecolor{currentstroke}{rgb}{0.000000,0.000000,0.000000}%
\pgfsetstrokecolor{currentstroke}%
\pgfsetdash{}{0pt}%
\pgfpathmoveto{\pgfqpoint{0.553581in}{0.499444in}}%
\pgfpathlineto{\pgfqpoint{0.553581in}{1.654444in}}%
\pgfusepath{stroke}%
\end{pgfscope}%
\begin{pgfscope}%
\pgfsetrectcap%
\pgfsetmiterjoin%
\pgfsetlinewidth{0.803000pt}%
\definecolor{currentstroke}{rgb}{0.000000,0.000000,0.000000}%
\pgfsetstrokecolor{currentstroke}%
\pgfsetdash{}{0pt}%
\pgfpathmoveto{\pgfqpoint{2.103581in}{0.499444in}}%
\pgfpathlineto{\pgfqpoint{2.103581in}{1.654444in}}%
\pgfusepath{stroke}%
\end{pgfscope}%
\begin{pgfscope}%
\pgfsetrectcap%
\pgfsetmiterjoin%
\pgfsetlinewidth{0.803000pt}%
\definecolor{currentstroke}{rgb}{0.000000,0.000000,0.000000}%
\pgfsetstrokecolor{currentstroke}%
\pgfsetdash{}{0pt}%
\pgfpathmoveto{\pgfqpoint{0.553581in}{0.499444in}}%
\pgfpathlineto{\pgfqpoint{2.103581in}{0.499444in}}%
\pgfusepath{stroke}%
\end{pgfscope}%
\begin{pgfscope}%
\pgfsetrectcap%
\pgfsetmiterjoin%
\pgfsetlinewidth{0.803000pt}%
\definecolor{currentstroke}{rgb}{0.000000,0.000000,0.000000}%
\pgfsetstrokecolor{currentstroke}%
\pgfsetdash{}{0pt}%
\pgfpathmoveto{\pgfqpoint{0.553581in}{1.654444in}}%
\pgfpathlineto{\pgfqpoint{2.103581in}{1.654444in}}%
\pgfusepath{stroke}%
\end{pgfscope}%
\begin{pgfscope}%
\definecolor{textcolor}{rgb}{0.000000,0.000000,0.000000}%
\pgfsetstrokecolor{textcolor}%
\pgfsetfillcolor{textcolor}%
\pgftext[x=1.328581in,y=1.737778in,,base]{\color{textcolor}\rmfamily\fontsize{12.000000}{14.400000}\selectfont ROC Curve}%
\end{pgfscope}%
\begin{pgfscope}%
\pgfsetbuttcap%
\pgfsetmiterjoin%
\definecolor{currentfill}{rgb}{1.000000,1.000000,1.000000}%
\pgfsetfillcolor{currentfill}%
\pgfsetfillopacity{0.800000}%
\pgfsetlinewidth{1.003750pt}%
\definecolor{currentstroke}{rgb}{0.800000,0.800000,0.800000}%
\pgfsetstrokecolor{currentstroke}%
\pgfsetstrokeopacity{0.800000}%
\pgfsetdash{}{0pt}%
\pgfpathmoveto{\pgfqpoint{0.840525in}{0.568889in}}%
\pgfpathlineto{\pgfqpoint{2.006358in}{0.568889in}}%
\pgfpathquadraticcurveto{\pgfqpoint{2.034136in}{0.568889in}}{\pgfqpoint{2.034136in}{0.596666in}}%
\pgfpathlineto{\pgfqpoint{2.034136in}{0.791111in}}%
\pgfpathquadraticcurveto{\pgfqpoint{2.034136in}{0.818888in}}{\pgfqpoint{2.006358in}{0.818888in}}%
\pgfpathlineto{\pgfqpoint{0.840525in}{0.818888in}}%
\pgfpathquadraticcurveto{\pgfqpoint{0.812747in}{0.818888in}}{\pgfqpoint{0.812747in}{0.791111in}}%
\pgfpathlineto{\pgfqpoint{0.812747in}{0.596666in}}%
\pgfpathquadraticcurveto{\pgfqpoint{0.812747in}{0.568889in}}{\pgfqpoint{0.840525in}{0.568889in}}%
\pgfpathlineto{\pgfqpoint{0.840525in}{0.568889in}}%
\pgfpathclose%
\pgfusepath{stroke,fill}%
\end{pgfscope}%
\begin{pgfscope}%
\pgfsetrectcap%
\pgfsetroundjoin%
\pgfsetlinewidth{1.505625pt}%
\definecolor{currentstroke}{rgb}{0.000000,0.000000,0.000000}%
\pgfsetstrokecolor{currentstroke}%
\pgfsetdash{}{0pt}%
\pgfpathmoveto{\pgfqpoint{0.868303in}{0.707777in}}%
\pgfpathlineto{\pgfqpoint{1.007192in}{0.707777in}}%
\pgfpathlineto{\pgfqpoint{1.146081in}{0.707777in}}%
\pgfusepath{stroke}%
\end{pgfscope}%
\begin{pgfscope}%
\definecolor{textcolor}{rgb}{0.000000,0.000000,0.000000}%
\pgfsetstrokecolor{textcolor}%
\pgfsetfillcolor{textcolor}%
\pgftext[x=1.257192in,y=0.659166in,left,base]{\color{textcolor}\rmfamily\fontsize{10.000000}{12.000000}\selectfont AUC 0.838)}%
\end{pgfscope}%
\end{pgfpicture}%
\makeatother%
\endgroup%

\end{tabular}
\end{center}

\begin{center}
\begin{tabular}{cc}
\begin{tabular}{cc|c|c|}
	&\multicolumn{1}{c}{}& \multicolumn{2}{c}{Prediction} \cr
	&\multicolumn{1}{c}{} & \multicolumn{1}{c}{N} & \multicolumn{1}{c}{P} \cr\cline{3-4}
	\multirow{2}{*}{Actual}&N & 67.2\% & 18.5\% \vrule width 0pt height 10pt depth 2pt \cr\cline{3-4}
	&P & 3.06\% & 11.22\% \vrule width 0pt height 10pt depth 2pt \cr\cline{3-4}
\end{tabular}
&
\begin{tabular}{ll}
0.784 & Accuracy \cr 
0.785 & Balanced Accuracy \cr 
0.377 & Precision \cr 
0.784 & Balanced Precision \cr 
0.786 & Recall \cr 
0.510 & F1 \cr 
0.785 & Balanced F1 \cr 
0.543 & Gmean \cr 	\end{tabular}
\end{tabular}
\end{center}

Another way is to linearly transform the probabilities.   Whether the distribution was clustered to the left or right, or clustered at the center, is not necessarily relevant, so we want to see it spread out.  We have arbitrarily chosen a transformation to put next the original models in our results to see if it will make a better model; tuning the transformation is an avenue for future work.  We have chosen to take the 0.05 quantile of the negative class and map it to $p=0.05$, and the 0.95 quantile of the positive class and map it to $p=0.95$.  This linear transformation gives the same metrics as the shift, and the ROC curve is the same except for the two labeled medians, now at 0.305 and 0.688.

\begin{center}
\begin{tabular}{p{0.5\textwidth} p{0.5\textwidth}}
  \vspace{0pt} %% Creator: Matplotlib, PGF backend
%%
%% To include the figure in your LaTeX document, write
%%   \input{<filename>.pgf}
%%
%% Make sure the required packages are loaded in your preamble
%%   \usepackage{pgf}
%%
%% Also ensure that all the required font packages are loaded; for instance,
%% the lmodern package is sometimes necessary when using math font.
%%   \usepackage{lmodern}
%%
%% Figures using additional raster images can only be included by \input if
%% they are in the same directory as the main LaTeX file. For loading figures
%% from other directories you can use the `import` package
%%   \usepackage{import}
%%
%% and then include the figures with
%%   \import{<path to file>}{<filename>.pgf}
%%
%% Matplotlib used the following preamble
%%   
%%   \usepackage{fontspec}
%%   \makeatletter\@ifpackageloaded{underscore}{}{\usepackage[strings]{underscore}}\makeatother
%%
\begingroup%
\makeatletter%
\begin{pgfpicture}%
\pgfpathrectangle{\pgfpointorigin}{\pgfqpoint{2.563750in}{2.187638in}}%
\pgfusepath{use as bounding box, clip}%
\begin{pgfscope}%
\pgfsetbuttcap%
\pgfsetmiterjoin%
\definecolor{currentfill}{rgb}{1.000000,1.000000,1.000000}%
\pgfsetfillcolor{currentfill}%
\pgfsetlinewidth{0.000000pt}%
\definecolor{currentstroke}{rgb}{1.000000,1.000000,1.000000}%
\pgfsetstrokecolor{currentstroke}%
\pgfsetdash{}{0pt}%
\pgfpathmoveto{\pgfqpoint{-0.000000in}{0.000000in}}%
\pgfpathlineto{\pgfqpoint{2.563750in}{0.000000in}}%
\pgfpathlineto{\pgfqpoint{2.563750in}{2.187638in}}%
\pgfpathlineto{\pgfqpoint{-0.000000in}{2.187638in}}%
\pgfpathlineto{\pgfqpoint{-0.000000in}{0.000000in}}%
\pgfpathclose%
\pgfusepath{fill}%
\end{pgfscope}%
\begin{pgfscope}%
\pgfsetbuttcap%
\pgfsetmiterjoin%
\definecolor{currentfill}{rgb}{1.000000,1.000000,1.000000}%
\pgfsetfillcolor{currentfill}%
\pgfsetlinewidth{0.000000pt}%
\definecolor{currentstroke}{rgb}{0.000000,0.000000,0.000000}%
\pgfsetstrokecolor{currentstroke}%
\pgfsetstrokeopacity{0.000000}%
\pgfsetdash{}{0pt}%
\pgfpathmoveto{\pgfqpoint{0.515000in}{0.502638in}}%
\pgfpathlineto{\pgfqpoint{2.375000in}{0.502638in}}%
\pgfpathlineto{\pgfqpoint{2.375000in}{1.888638in}}%
\pgfpathlineto{\pgfqpoint{0.515000in}{1.888638in}}%
\pgfpathlineto{\pgfqpoint{0.515000in}{0.502638in}}%
\pgfpathclose%
\pgfusepath{fill}%
\end{pgfscope}%
\begin{pgfscope}%
\pgfpathrectangle{\pgfqpoint{0.515000in}{0.502638in}}{\pgfqpoint{1.860000in}{1.386000in}}%
\pgfusepath{clip}%
\pgfsetbuttcap%
\pgfsetmiterjoin%
\pgfsetlinewidth{1.003750pt}%
\definecolor{currentstroke}{rgb}{0.000000,0.000000,0.000000}%
\pgfsetstrokecolor{currentstroke}%
\pgfsetdash{}{0pt}%
\pgfpathmoveto{\pgfqpoint{0.505000in}{0.502638in}}%
\pgfpathlineto{\pgfqpoint{0.560366in}{0.502638in}}%
\pgfpathlineto{\pgfqpoint{0.560366in}{1.175784in}}%
\pgfpathlineto{\pgfqpoint{0.505000in}{1.175784in}}%
\pgfusepath{stroke}%
\end{pgfscope}%
\begin{pgfscope}%
\pgfpathrectangle{\pgfqpoint{0.515000in}{0.502638in}}{\pgfqpoint{1.860000in}{1.386000in}}%
\pgfusepath{clip}%
\pgfsetbuttcap%
\pgfsetmiterjoin%
\pgfsetlinewidth{1.003750pt}%
\definecolor{currentstroke}{rgb}{0.000000,0.000000,0.000000}%
\pgfsetstrokecolor{currentstroke}%
\pgfsetdash{}{0pt}%
\pgfpathmoveto{\pgfqpoint{0.669244in}{0.502638in}}%
\pgfpathlineto{\pgfqpoint{0.741830in}{0.502638in}}%
\pgfpathlineto{\pgfqpoint{0.741830in}{1.702791in}}%
\pgfpathlineto{\pgfqpoint{0.669244in}{1.702791in}}%
\pgfpathlineto{\pgfqpoint{0.669244in}{0.502638in}}%
\pgfpathclose%
\pgfusepath{stroke}%
\end{pgfscope}%
\begin{pgfscope}%
\pgfpathrectangle{\pgfqpoint{0.515000in}{0.502638in}}{\pgfqpoint{1.860000in}{1.386000in}}%
\pgfusepath{clip}%
\pgfsetbuttcap%
\pgfsetmiterjoin%
\pgfsetlinewidth{1.003750pt}%
\definecolor{currentstroke}{rgb}{0.000000,0.000000,0.000000}%
\pgfsetstrokecolor{currentstroke}%
\pgfsetdash{}{0pt}%
\pgfpathmoveto{\pgfqpoint{0.850708in}{0.502638in}}%
\pgfpathlineto{\pgfqpoint{0.923293in}{0.502638in}}%
\pgfpathlineto{\pgfqpoint{0.923293in}{1.822638in}}%
\pgfpathlineto{\pgfqpoint{0.850708in}{1.822638in}}%
\pgfpathlineto{\pgfqpoint{0.850708in}{0.502638in}}%
\pgfpathclose%
\pgfusepath{stroke}%
\end{pgfscope}%
\begin{pgfscope}%
\pgfpathrectangle{\pgfqpoint{0.515000in}{0.502638in}}{\pgfqpoint{1.860000in}{1.386000in}}%
\pgfusepath{clip}%
\pgfsetbuttcap%
\pgfsetmiterjoin%
\pgfsetlinewidth{1.003750pt}%
\definecolor{currentstroke}{rgb}{0.000000,0.000000,0.000000}%
\pgfsetstrokecolor{currentstroke}%
\pgfsetdash{}{0pt}%
\pgfpathmoveto{\pgfqpoint{1.032171in}{0.502638in}}%
\pgfpathlineto{\pgfqpoint{1.104756in}{0.502638in}}%
\pgfpathlineto{\pgfqpoint{1.104756in}{1.642783in}}%
\pgfpathlineto{\pgfqpoint{1.032171in}{1.642783in}}%
\pgfpathlineto{\pgfqpoint{1.032171in}{0.502638in}}%
\pgfpathclose%
\pgfusepath{stroke}%
\end{pgfscope}%
\begin{pgfscope}%
\pgfpathrectangle{\pgfqpoint{0.515000in}{0.502638in}}{\pgfqpoint{1.860000in}{1.386000in}}%
\pgfusepath{clip}%
\pgfsetbuttcap%
\pgfsetmiterjoin%
\pgfsetlinewidth{1.003750pt}%
\definecolor{currentstroke}{rgb}{0.000000,0.000000,0.000000}%
\pgfsetstrokecolor{currentstroke}%
\pgfsetdash{}{0pt}%
\pgfpathmoveto{\pgfqpoint{1.213634in}{0.502638in}}%
\pgfpathlineto{\pgfqpoint{1.286220in}{0.502638in}}%
\pgfpathlineto{\pgfqpoint{1.286220in}{1.353016in}}%
\pgfpathlineto{\pgfqpoint{1.213634in}{1.353016in}}%
\pgfpathlineto{\pgfqpoint{1.213634in}{0.502638in}}%
\pgfpathclose%
\pgfusepath{stroke}%
\end{pgfscope}%
\begin{pgfscope}%
\pgfpathrectangle{\pgfqpoint{0.515000in}{0.502638in}}{\pgfqpoint{1.860000in}{1.386000in}}%
\pgfusepath{clip}%
\pgfsetbuttcap%
\pgfsetmiterjoin%
\pgfsetlinewidth{1.003750pt}%
\definecolor{currentstroke}{rgb}{0.000000,0.000000,0.000000}%
\pgfsetstrokecolor{currentstroke}%
\pgfsetdash{}{0pt}%
\pgfpathmoveto{\pgfqpoint{1.395098in}{0.502638in}}%
\pgfpathlineto{\pgfqpoint{1.467683in}{0.502638in}}%
\pgfpathlineto{\pgfqpoint{1.467683in}{1.065538in}}%
\pgfpathlineto{\pgfqpoint{1.395098in}{1.065538in}}%
\pgfpathlineto{\pgfqpoint{1.395098in}{0.502638in}}%
\pgfpathclose%
\pgfusepath{stroke}%
\end{pgfscope}%
\begin{pgfscope}%
\pgfpathrectangle{\pgfqpoint{0.515000in}{0.502638in}}{\pgfqpoint{1.860000in}{1.386000in}}%
\pgfusepath{clip}%
\pgfsetbuttcap%
\pgfsetmiterjoin%
\pgfsetlinewidth{1.003750pt}%
\definecolor{currentstroke}{rgb}{0.000000,0.000000,0.000000}%
\pgfsetstrokecolor{currentstroke}%
\pgfsetdash{}{0pt}%
\pgfpathmoveto{\pgfqpoint{1.576561in}{0.502638in}}%
\pgfpathlineto{\pgfqpoint{1.649147in}{0.502638in}}%
\pgfpathlineto{\pgfqpoint{1.649147in}{0.918952in}}%
\pgfpathlineto{\pgfqpoint{1.576561in}{0.918952in}}%
\pgfpathlineto{\pgfqpoint{1.576561in}{0.502638in}}%
\pgfpathclose%
\pgfusepath{stroke}%
\end{pgfscope}%
\begin{pgfscope}%
\pgfpathrectangle{\pgfqpoint{0.515000in}{0.502638in}}{\pgfqpoint{1.860000in}{1.386000in}}%
\pgfusepath{clip}%
\pgfsetbuttcap%
\pgfsetmiterjoin%
\pgfsetlinewidth{1.003750pt}%
\definecolor{currentstroke}{rgb}{0.000000,0.000000,0.000000}%
\pgfsetstrokecolor{currentstroke}%
\pgfsetdash{}{0pt}%
\pgfpathmoveto{\pgfqpoint{1.758025in}{0.502638in}}%
\pgfpathlineto{\pgfqpoint{1.830610in}{0.502638in}}%
\pgfpathlineto{\pgfqpoint{1.830610in}{0.717103in}}%
\pgfpathlineto{\pgfqpoint{1.758025in}{0.717103in}}%
\pgfpathlineto{\pgfqpoint{1.758025in}{0.502638in}}%
\pgfpathclose%
\pgfusepath{stroke}%
\end{pgfscope}%
\begin{pgfscope}%
\pgfpathrectangle{\pgfqpoint{0.515000in}{0.502638in}}{\pgfqpoint{1.860000in}{1.386000in}}%
\pgfusepath{clip}%
\pgfsetbuttcap%
\pgfsetmiterjoin%
\pgfsetlinewidth{1.003750pt}%
\definecolor{currentstroke}{rgb}{0.000000,0.000000,0.000000}%
\pgfsetstrokecolor{currentstroke}%
\pgfsetdash{}{0pt}%
\pgfpathmoveto{\pgfqpoint{1.939488in}{0.502638in}}%
\pgfpathlineto{\pgfqpoint{2.012073in}{0.502638in}}%
\pgfpathlineto{\pgfqpoint{2.012073in}{0.627119in}}%
\pgfpathlineto{\pgfqpoint{1.939488in}{0.627119in}}%
\pgfpathlineto{\pgfqpoint{1.939488in}{0.502638in}}%
\pgfpathclose%
\pgfusepath{stroke}%
\end{pgfscope}%
\begin{pgfscope}%
\pgfpathrectangle{\pgfqpoint{0.515000in}{0.502638in}}{\pgfqpoint{1.860000in}{1.386000in}}%
\pgfusepath{clip}%
\pgfsetbuttcap%
\pgfsetmiterjoin%
\pgfsetlinewidth{1.003750pt}%
\definecolor{currentstroke}{rgb}{0.000000,0.000000,0.000000}%
\pgfsetstrokecolor{currentstroke}%
\pgfsetdash{}{0pt}%
\pgfpathmoveto{\pgfqpoint{2.120951in}{0.502638in}}%
\pgfpathlineto{\pgfqpoint{2.193537in}{0.502638in}}%
\pgfpathlineto{\pgfqpoint{2.193537in}{0.608084in}}%
\pgfpathlineto{\pgfqpoint{2.120951in}{0.608084in}}%
\pgfpathlineto{\pgfqpoint{2.120951in}{0.502638in}}%
\pgfpathclose%
\pgfusepath{stroke}%
\end{pgfscope}%
\begin{pgfscope}%
\pgfpathrectangle{\pgfqpoint{0.515000in}{0.502638in}}{\pgfqpoint{1.860000in}{1.386000in}}%
\pgfusepath{clip}%
\pgfsetbuttcap%
\pgfsetmiterjoin%
\definecolor{currentfill}{rgb}{0.000000,0.000000,0.000000}%
\pgfsetfillcolor{currentfill}%
\pgfsetlinewidth{0.000000pt}%
\definecolor{currentstroke}{rgb}{0.000000,0.000000,0.000000}%
\pgfsetstrokecolor{currentstroke}%
\pgfsetstrokeopacity{0.000000}%
\pgfsetdash{}{0pt}%
\pgfpathmoveto{\pgfqpoint{0.560366in}{0.502638in}}%
\pgfpathlineto{\pgfqpoint{0.632951in}{0.502638in}}%
\pgfpathlineto{\pgfqpoint{0.632951in}{0.515310in}}%
\pgfpathlineto{\pgfqpoint{0.560366in}{0.515310in}}%
\pgfpathlineto{\pgfqpoint{0.560366in}{0.502638in}}%
\pgfpathclose%
\pgfusepath{fill}%
\end{pgfscope}%
\begin{pgfscope}%
\pgfpathrectangle{\pgfqpoint{0.515000in}{0.502638in}}{\pgfqpoint{1.860000in}{1.386000in}}%
\pgfusepath{clip}%
\pgfsetbuttcap%
\pgfsetmiterjoin%
\definecolor{currentfill}{rgb}{0.000000,0.000000,0.000000}%
\pgfsetfillcolor{currentfill}%
\pgfsetlinewidth{0.000000pt}%
\definecolor{currentstroke}{rgb}{0.000000,0.000000,0.000000}%
\pgfsetstrokecolor{currentstroke}%
\pgfsetstrokeopacity{0.000000}%
\pgfsetdash{}{0pt}%
\pgfpathmoveto{\pgfqpoint{0.741830in}{0.502638in}}%
\pgfpathlineto{\pgfqpoint{0.814415in}{0.502638in}}%
\pgfpathlineto{\pgfqpoint{0.814415in}{0.525079in}}%
\pgfpathlineto{\pgfqpoint{0.741830in}{0.525079in}}%
\pgfpathlineto{\pgfqpoint{0.741830in}{0.502638in}}%
\pgfpathclose%
\pgfusepath{fill}%
\end{pgfscope}%
\begin{pgfscope}%
\pgfpathrectangle{\pgfqpoint{0.515000in}{0.502638in}}{\pgfqpoint{1.860000in}{1.386000in}}%
\pgfusepath{clip}%
\pgfsetbuttcap%
\pgfsetmiterjoin%
\definecolor{currentfill}{rgb}{0.000000,0.000000,0.000000}%
\pgfsetfillcolor{currentfill}%
\pgfsetlinewidth{0.000000pt}%
\definecolor{currentstroke}{rgb}{0.000000,0.000000,0.000000}%
\pgfsetstrokecolor{currentstroke}%
\pgfsetstrokeopacity{0.000000}%
\pgfsetdash{}{0pt}%
\pgfpathmoveto{\pgfqpoint{0.923293in}{0.502638in}}%
\pgfpathlineto{\pgfqpoint{0.995878in}{0.502638in}}%
\pgfpathlineto{\pgfqpoint{0.995878in}{0.541546in}}%
\pgfpathlineto{\pgfqpoint{0.923293in}{0.541546in}}%
\pgfpathlineto{\pgfqpoint{0.923293in}{0.502638in}}%
\pgfpathclose%
\pgfusepath{fill}%
\end{pgfscope}%
\begin{pgfscope}%
\pgfpathrectangle{\pgfqpoint{0.515000in}{0.502638in}}{\pgfqpoint{1.860000in}{1.386000in}}%
\pgfusepath{clip}%
\pgfsetbuttcap%
\pgfsetmiterjoin%
\definecolor{currentfill}{rgb}{0.000000,0.000000,0.000000}%
\pgfsetfillcolor{currentfill}%
\pgfsetlinewidth{0.000000pt}%
\definecolor{currentstroke}{rgb}{0.000000,0.000000,0.000000}%
\pgfsetstrokecolor{currentstroke}%
\pgfsetstrokeopacity{0.000000}%
\pgfsetdash{}{0pt}%
\pgfpathmoveto{\pgfqpoint{1.104756in}{0.502638in}}%
\pgfpathlineto{\pgfqpoint{1.177342in}{0.502638in}}%
\pgfpathlineto{\pgfqpoint{1.177342in}{0.564767in}}%
\pgfpathlineto{\pgfqpoint{1.104756in}{0.564767in}}%
\pgfpathlineto{\pgfqpoint{1.104756in}{0.502638in}}%
\pgfpathclose%
\pgfusepath{fill}%
\end{pgfscope}%
\begin{pgfscope}%
\pgfpathrectangle{\pgfqpoint{0.515000in}{0.502638in}}{\pgfqpoint{1.860000in}{1.386000in}}%
\pgfusepath{clip}%
\pgfsetbuttcap%
\pgfsetmiterjoin%
\definecolor{currentfill}{rgb}{0.000000,0.000000,0.000000}%
\pgfsetfillcolor{currentfill}%
\pgfsetlinewidth{0.000000pt}%
\definecolor{currentstroke}{rgb}{0.000000,0.000000,0.000000}%
\pgfsetstrokecolor{currentstroke}%
\pgfsetstrokeopacity{0.000000}%
\pgfsetdash{}{0pt}%
\pgfpathmoveto{\pgfqpoint{1.286220in}{0.502638in}}%
\pgfpathlineto{\pgfqpoint{1.358805in}{0.502638in}}%
\pgfpathlineto{\pgfqpoint{1.358805in}{0.602837in}}%
\pgfpathlineto{\pgfqpoint{1.286220in}{0.602837in}}%
\pgfpathlineto{\pgfqpoint{1.286220in}{0.502638in}}%
\pgfpathclose%
\pgfusepath{fill}%
\end{pgfscope}%
\begin{pgfscope}%
\pgfpathrectangle{\pgfqpoint{0.515000in}{0.502638in}}{\pgfqpoint{1.860000in}{1.386000in}}%
\pgfusepath{clip}%
\pgfsetbuttcap%
\pgfsetmiterjoin%
\definecolor{currentfill}{rgb}{0.000000,0.000000,0.000000}%
\pgfsetfillcolor{currentfill}%
\pgfsetlinewidth{0.000000pt}%
\definecolor{currentstroke}{rgb}{0.000000,0.000000,0.000000}%
\pgfsetstrokecolor{currentstroke}%
\pgfsetstrokeopacity{0.000000}%
\pgfsetdash{}{0pt}%
\pgfpathmoveto{\pgfqpoint{1.467683in}{0.502638in}}%
\pgfpathlineto{\pgfqpoint{1.540269in}{0.502638in}}%
\pgfpathlineto{\pgfqpoint{1.540269in}{0.649559in}}%
\pgfpathlineto{\pgfqpoint{1.467683in}{0.649559in}}%
\pgfpathlineto{\pgfqpoint{1.467683in}{0.502638in}}%
\pgfpathclose%
\pgfusepath{fill}%
\end{pgfscope}%
\begin{pgfscope}%
\pgfpathrectangle{\pgfqpoint{0.515000in}{0.502638in}}{\pgfqpoint{1.860000in}{1.386000in}}%
\pgfusepath{clip}%
\pgfsetbuttcap%
\pgfsetmiterjoin%
\definecolor{currentfill}{rgb}{0.000000,0.000000,0.000000}%
\pgfsetfillcolor{currentfill}%
\pgfsetlinewidth{0.000000pt}%
\definecolor{currentstroke}{rgb}{0.000000,0.000000,0.000000}%
\pgfsetstrokecolor{currentstroke}%
\pgfsetstrokeopacity{0.000000}%
\pgfsetdash{}{0pt}%
\pgfpathmoveto{\pgfqpoint{1.649147in}{0.502638in}}%
\pgfpathlineto{\pgfqpoint{1.721732in}{0.502638in}}%
\pgfpathlineto{\pgfqpoint{1.721732in}{0.695165in}}%
\pgfpathlineto{\pgfqpoint{1.649147in}{0.695165in}}%
\pgfpathlineto{\pgfqpoint{1.649147in}{0.502638in}}%
\pgfpathclose%
\pgfusepath{fill}%
\end{pgfscope}%
\begin{pgfscope}%
\pgfpathrectangle{\pgfqpoint{0.515000in}{0.502638in}}{\pgfqpoint{1.860000in}{1.386000in}}%
\pgfusepath{clip}%
\pgfsetbuttcap%
\pgfsetmiterjoin%
\definecolor{currentfill}{rgb}{0.000000,0.000000,0.000000}%
\pgfsetfillcolor{currentfill}%
\pgfsetlinewidth{0.000000pt}%
\definecolor{currentstroke}{rgb}{0.000000,0.000000,0.000000}%
\pgfsetstrokecolor{currentstroke}%
\pgfsetstrokeopacity{0.000000}%
\pgfsetdash{}{0pt}%
\pgfpathmoveto{\pgfqpoint{1.830610in}{0.502638in}}%
\pgfpathlineto{\pgfqpoint{1.903195in}{0.502638in}}%
\pgfpathlineto{\pgfqpoint{1.903195in}{0.719392in}}%
\pgfpathlineto{\pgfqpoint{1.830610in}{0.719392in}}%
\pgfpathlineto{\pgfqpoint{1.830610in}{0.502638in}}%
\pgfpathclose%
\pgfusepath{fill}%
\end{pgfscope}%
\begin{pgfscope}%
\pgfpathrectangle{\pgfqpoint{0.515000in}{0.502638in}}{\pgfqpoint{1.860000in}{1.386000in}}%
\pgfusepath{clip}%
\pgfsetbuttcap%
\pgfsetmiterjoin%
\definecolor{currentfill}{rgb}{0.000000,0.000000,0.000000}%
\pgfsetfillcolor{currentfill}%
\pgfsetlinewidth{0.000000pt}%
\definecolor{currentstroke}{rgb}{0.000000,0.000000,0.000000}%
\pgfsetstrokecolor{currentstroke}%
\pgfsetstrokeopacity{0.000000}%
\pgfsetdash{}{0pt}%
\pgfpathmoveto{\pgfqpoint{2.012073in}{0.502638in}}%
\pgfpathlineto{\pgfqpoint{2.084659in}{0.502638in}}%
\pgfpathlineto{\pgfqpoint{2.084659in}{0.700301in}}%
\pgfpathlineto{\pgfqpoint{2.012073in}{0.700301in}}%
\pgfpathlineto{\pgfqpoint{2.012073in}{0.502638in}}%
\pgfpathclose%
\pgfusepath{fill}%
\end{pgfscope}%
\begin{pgfscope}%
\pgfpathrectangle{\pgfqpoint{0.515000in}{0.502638in}}{\pgfqpoint{1.860000in}{1.386000in}}%
\pgfusepath{clip}%
\pgfsetbuttcap%
\pgfsetmiterjoin%
\definecolor{currentfill}{rgb}{0.000000,0.000000,0.000000}%
\pgfsetfillcolor{currentfill}%
\pgfsetlinewidth{0.000000pt}%
\definecolor{currentstroke}{rgb}{0.000000,0.000000,0.000000}%
\pgfsetstrokecolor{currentstroke}%
\pgfsetstrokeopacity{0.000000}%
\pgfsetdash{}{0pt}%
\pgfpathmoveto{\pgfqpoint{2.193537in}{0.502638in}}%
\pgfpathlineto{\pgfqpoint{2.266122in}{0.502638in}}%
\pgfpathlineto{\pgfqpoint{2.266122in}{0.622040in}}%
\pgfpathlineto{\pgfqpoint{2.193537in}{0.622040in}}%
\pgfpathlineto{\pgfqpoint{2.193537in}{0.502638in}}%
\pgfpathclose%
\pgfusepath{fill}%
\end{pgfscope}%
\begin{pgfscope}%
\pgfsetbuttcap%
\pgfsetroundjoin%
\definecolor{currentfill}{rgb}{0.000000,0.000000,0.000000}%
\pgfsetfillcolor{currentfill}%
\pgfsetlinewidth{0.803000pt}%
\definecolor{currentstroke}{rgb}{0.000000,0.000000,0.000000}%
\pgfsetstrokecolor{currentstroke}%
\pgfsetdash{}{0pt}%
\pgfsys@defobject{currentmarker}{\pgfqpoint{0.000000in}{-0.048611in}}{\pgfqpoint{0.000000in}{0.000000in}}{%
\pgfpathmoveto{\pgfqpoint{0.000000in}{0.000000in}}%
\pgfpathlineto{\pgfqpoint{0.000000in}{-0.048611in}}%
\pgfusepath{stroke,fill}%
}%
\begin{pgfscope}%
\pgfsys@transformshift{0.560366in}{0.502638in}%
\pgfsys@useobject{currentmarker}{}%
\end{pgfscope}%
\end{pgfscope}%
\begin{pgfscope}%
\definecolor{textcolor}{rgb}{0.000000,0.000000,0.000000}%
\pgfsetstrokecolor{textcolor}%
\pgfsetfillcolor{textcolor}%
\pgftext[x=0.560366in,y=0.405416in,,top]{\color{textcolor}\rmfamily\fontsize{10.000000}{12.000000}\selectfont 0.0}%
\end{pgfscope}%
\begin{pgfscope}%
\pgfsetbuttcap%
\pgfsetroundjoin%
\definecolor{currentfill}{rgb}{0.000000,0.000000,0.000000}%
\pgfsetfillcolor{currentfill}%
\pgfsetlinewidth{0.803000pt}%
\definecolor{currentstroke}{rgb}{0.000000,0.000000,0.000000}%
\pgfsetstrokecolor{currentstroke}%
\pgfsetdash{}{0pt}%
\pgfsys@defobject{currentmarker}{\pgfqpoint{0.000000in}{-0.048611in}}{\pgfqpoint{0.000000in}{0.000000in}}{%
\pgfpathmoveto{\pgfqpoint{0.000000in}{0.000000in}}%
\pgfpathlineto{\pgfqpoint{0.000000in}{-0.048611in}}%
\pgfusepath{stroke,fill}%
}%
\begin{pgfscope}%
\pgfsys@transformshift{1.014025in}{0.502638in}%
\pgfsys@useobject{currentmarker}{}%
\end{pgfscope}%
\end{pgfscope}%
\begin{pgfscope}%
\definecolor{textcolor}{rgb}{0.000000,0.000000,0.000000}%
\pgfsetstrokecolor{textcolor}%
\pgfsetfillcolor{textcolor}%
\pgftext[x=1.014025in,y=0.405416in,,top]{\color{textcolor}\rmfamily\fontsize{10.000000}{12.000000}\selectfont 0.25}%
\end{pgfscope}%
\begin{pgfscope}%
\pgfsetbuttcap%
\pgfsetroundjoin%
\definecolor{currentfill}{rgb}{0.000000,0.000000,0.000000}%
\pgfsetfillcolor{currentfill}%
\pgfsetlinewidth{0.803000pt}%
\definecolor{currentstroke}{rgb}{0.000000,0.000000,0.000000}%
\pgfsetstrokecolor{currentstroke}%
\pgfsetdash{}{0pt}%
\pgfsys@defobject{currentmarker}{\pgfqpoint{0.000000in}{-0.048611in}}{\pgfqpoint{0.000000in}{0.000000in}}{%
\pgfpathmoveto{\pgfqpoint{0.000000in}{0.000000in}}%
\pgfpathlineto{\pgfqpoint{0.000000in}{-0.048611in}}%
\pgfusepath{stroke,fill}%
}%
\begin{pgfscope}%
\pgfsys@transformshift{1.467683in}{0.502638in}%
\pgfsys@useobject{currentmarker}{}%
\end{pgfscope}%
\end{pgfscope}%
\begin{pgfscope}%
\definecolor{textcolor}{rgb}{0.000000,0.000000,0.000000}%
\pgfsetstrokecolor{textcolor}%
\pgfsetfillcolor{textcolor}%
\pgftext[x=1.467683in,y=0.405416in,,top]{\color{textcolor}\rmfamily\fontsize{10.000000}{12.000000}\selectfont 0.5}%
\end{pgfscope}%
\begin{pgfscope}%
\pgfsetbuttcap%
\pgfsetroundjoin%
\definecolor{currentfill}{rgb}{0.000000,0.000000,0.000000}%
\pgfsetfillcolor{currentfill}%
\pgfsetlinewidth{0.803000pt}%
\definecolor{currentstroke}{rgb}{0.000000,0.000000,0.000000}%
\pgfsetstrokecolor{currentstroke}%
\pgfsetdash{}{0pt}%
\pgfsys@defobject{currentmarker}{\pgfqpoint{0.000000in}{-0.048611in}}{\pgfqpoint{0.000000in}{0.000000in}}{%
\pgfpathmoveto{\pgfqpoint{0.000000in}{0.000000in}}%
\pgfpathlineto{\pgfqpoint{0.000000in}{-0.048611in}}%
\pgfusepath{stroke,fill}%
}%
\begin{pgfscope}%
\pgfsys@transformshift{1.921342in}{0.502638in}%
\pgfsys@useobject{currentmarker}{}%
\end{pgfscope}%
\end{pgfscope}%
\begin{pgfscope}%
\definecolor{textcolor}{rgb}{0.000000,0.000000,0.000000}%
\pgfsetstrokecolor{textcolor}%
\pgfsetfillcolor{textcolor}%
\pgftext[x=1.921342in,y=0.405416in,,top]{\color{textcolor}\rmfamily\fontsize{10.000000}{12.000000}\selectfont 0.75}%
\end{pgfscope}%
\begin{pgfscope}%
\pgfsetbuttcap%
\pgfsetroundjoin%
\definecolor{currentfill}{rgb}{0.000000,0.000000,0.000000}%
\pgfsetfillcolor{currentfill}%
\pgfsetlinewidth{0.803000pt}%
\definecolor{currentstroke}{rgb}{0.000000,0.000000,0.000000}%
\pgfsetstrokecolor{currentstroke}%
\pgfsetdash{}{0pt}%
\pgfsys@defobject{currentmarker}{\pgfqpoint{0.000000in}{-0.048611in}}{\pgfqpoint{0.000000in}{0.000000in}}{%
\pgfpathmoveto{\pgfqpoint{0.000000in}{0.000000in}}%
\pgfpathlineto{\pgfqpoint{0.000000in}{-0.048611in}}%
\pgfusepath{stroke,fill}%
}%
\begin{pgfscope}%
\pgfsys@transformshift{2.375000in}{0.502638in}%
\pgfsys@useobject{currentmarker}{}%
\end{pgfscope}%
\end{pgfscope}%
\begin{pgfscope}%
\definecolor{textcolor}{rgb}{0.000000,0.000000,0.000000}%
\pgfsetstrokecolor{textcolor}%
\pgfsetfillcolor{textcolor}%
\pgftext[x=2.375000in,y=0.405416in,,top]{\color{textcolor}\rmfamily\fontsize{10.000000}{12.000000}\selectfont 1.0}%
\end{pgfscope}%
\begin{pgfscope}%
\definecolor{textcolor}{rgb}{0.000000,0.000000,0.000000}%
\pgfsetstrokecolor{textcolor}%
\pgfsetfillcolor{textcolor}%
\pgftext[x=1.445000in,y=0.226528in,,top]{\color{textcolor}\rmfamily\fontsize{10.000000}{12.000000}\selectfont Range of Prediction}%
\end{pgfscope}%
\begin{pgfscope}%
\pgfsetbuttcap%
\pgfsetroundjoin%
\definecolor{currentfill}{rgb}{0.000000,0.000000,0.000000}%
\pgfsetfillcolor{currentfill}%
\pgfsetlinewidth{0.803000pt}%
\definecolor{currentstroke}{rgb}{0.000000,0.000000,0.000000}%
\pgfsetstrokecolor{currentstroke}%
\pgfsetdash{}{0pt}%
\pgfsys@defobject{currentmarker}{\pgfqpoint{-0.048611in}{0.000000in}}{\pgfqpoint{-0.000000in}{0.000000in}}{%
\pgfpathmoveto{\pgfqpoint{-0.000000in}{0.000000in}}%
\pgfpathlineto{\pgfqpoint{-0.048611in}{0.000000in}}%
\pgfusepath{stroke,fill}%
}%
\begin{pgfscope}%
\pgfsys@transformshift{0.515000in}{0.502638in}%
\pgfsys@useobject{currentmarker}{}%
\end{pgfscope}%
\end{pgfscope}%
\begin{pgfscope}%
\definecolor{textcolor}{rgb}{0.000000,0.000000,0.000000}%
\pgfsetstrokecolor{textcolor}%
\pgfsetfillcolor{textcolor}%
\pgftext[x=0.348333in, y=0.454444in, left, base]{\color{textcolor}\rmfamily\fontsize{10.000000}{12.000000}\selectfont \(\displaystyle {0}\)}%
\end{pgfscope}%
\begin{pgfscope}%
\pgfsetbuttcap%
\pgfsetroundjoin%
\definecolor{currentfill}{rgb}{0.000000,0.000000,0.000000}%
\pgfsetfillcolor{currentfill}%
\pgfsetlinewidth{0.803000pt}%
\definecolor{currentstroke}{rgb}{0.000000,0.000000,0.000000}%
\pgfsetstrokecolor{currentstroke}%
\pgfsetdash{}{0pt}%
\pgfsys@defobject{currentmarker}{\pgfqpoint{-0.048611in}{0.000000in}}{\pgfqpoint{-0.000000in}{0.000000in}}{%
\pgfpathmoveto{\pgfqpoint{-0.000000in}{0.000000in}}%
\pgfpathlineto{\pgfqpoint{-0.048611in}{0.000000in}}%
\pgfusepath{stroke,fill}%
}%
\begin{pgfscope}%
\pgfsys@transformshift{0.515000in}{0.893386in}%
\pgfsys@useobject{currentmarker}{}%
\end{pgfscope}%
\end{pgfscope}%
\begin{pgfscope}%
\definecolor{textcolor}{rgb}{0.000000,0.000000,0.000000}%
\pgfsetstrokecolor{textcolor}%
\pgfsetfillcolor{textcolor}%
\pgftext[x=0.348333in, y=0.845191in, left, base]{\color{textcolor}\rmfamily\fontsize{10.000000}{12.000000}\selectfont \(\displaystyle {5}\)}%
\end{pgfscope}%
\begin{pgfscope}%
\pgfsetbuttcap%
\pgfsetroundjoin%
\definecolor{currentfill}{rgb}{0.000000,0.000000,0.000000}%
\pgfsetfillcolor{currentfill}%
\pgfsetlinewidth{0.803000pt}%
\definecolor{currentstroke}{rgb}{0.000000,0.000000,0.000000}%
\pgfsetstrokecolor{currentstroke}%
\pgfsetdash{}{0pt}%
\pgfsys@defobject{currentmarker}{\pgfqpoint{-0.048611in}{0.000000in}}{\pgfqpoint{-0.000000in}{0.000000in}}{%
\pgfpathmoveto{\pgfqpoint{-0.000000in}{0.000000in}}%
\pgfpathlineto{\pgfqpoint{-0.048611in}{0.000000in}}%
\pgfusepath{stroke,fill}%
}%
\begin{pgfscope}%
\pgfsys@transformshift{0.515000in}{1.284133in}%
\pgfsys@useobject{currentmarker}{}%
\end{pgfscope}%
\end{pgfscope}%
\begin{pgfscope}%
\definecolor{textcolor}{rgb}{0.000000,0.000000,0.000000}%
\pgfsetstrokecolor{textcolor}%
\pgfsetfillcolor{textcolor}%
\pgftext[x=0.278889in, y=1.235939in, left, base]{\color{textcolor}\rmfamily\fontsize{10.000000}{12.000000}\selectfont \(\displaystyle {10}\)}%
\end{pgfscope}%
\begin{pgfscope}%
\pgfsetbuttcap%
\pgfsetroundjoin%
\definecolor{currentfill}{rgb}{0.000000,0.000000,0.000000}%
\pgfsetfillcolor{currentfill}%
\pgfsetlinewidth{0.803000pt}%
\definecolor{currentstroke}{rgb}{0.000000,0.000000,0.000000}%
\pgfsetstrokecolor{currentstroke}%
\pgfsetdash{}{0pt}%
\pgfsys@defobject{currentmarker}{\pgfqpoint{-0.048611in}{0.000000in}}{\pgfqpoint{-0.000000in}{0.000000in}}{%
\pgfpathmoveto{\pgfqpoint{-0.000000in}{0.000000in}}%
\pgfpathlineto{\pgfqpoint{-0.048611in}{0.000000in}}%
\pgfusepath{stroke,fill}%
}%
\begin{pgfscope}%
\pgfsys@transformshift{0.515000in}{1.674880in}%
\pgfsys@useobject{currentmarker}{}%
\end{pgfscope}%
\end{pgfscope}%
\begin{pgfscope}%
\definecolor{textcolor}{rgb}{0.000000,0.000000,0.000000}%
\pgfsetstrokecolor{textcolor}%
\pgfsetfillcolor{textcolor}%
\pgftext[x=0.278889in, y=1.626686in, left, base]{\color{textcolor}\rmfamily\fontsize{10.000000}{12.000000}\selectfont \(\displaystyle {15}\)}%
\end{pgfscope}%
\begin{pgfscope}%
\definecolor{textcolor}{rgb}{0.000000,0.000000,0.000000}%
\pgfsetstrokecolor{textcolor}%
\pgfsetfillcolor{textcolor}%
\pgftext[x=0.223333in,y=1.195638in,,bottom,rotate=90.000000]{\color{textcolor}\rmfamily\fontsize{10.000000}{12.000000}\selectfont Percent of Data Set}%
\end{pgfscope}%
\begin{pgfscope}%
\pgfsetrectcap%
\pgfsetmiterjoin%
\pgfsetlinewidth{0.803000pt}%
\definecolor{currentstroke}{rgb}{0.000000,0.000000,0.000000}%
\pgfsetstrokecolor{currentstroke}%
\pgfsetdash{}{0pt}%
\pgfpathmoveto{\pgfqpoint{0.515000in}{0.502638in}}%
\pgfpathlineto{\pgfqpoint{0.515000in}{1.888638in}}%
\pgfusepath{stroke}%
\end{pgfscope}%
\begin{pgfscope}%
\pgfsetrectcap%
\pgfsetmiterjoin%
\pgfsetlinewidth{0.803000pt}%
\definecolor{currentstroke}{rgb}{0.000000,0.000000,0.000000}%
\pgfsetstrokecolor{currentstroke}%
\pgfsetdash{}{0pt}%
\pgfpathmoveto{\pgfqpoint{2.375000in}{0.502638in}}%
\pgfpathlineto{\pgfqpoint{2.375000in}{1.888638in}}%
\pgfusepath{stroke}%
\end{pgfscope}%
\begin{pgfscope}%
\pgfsetrectcap%
\pgfsetmiterjoin%
\pgfsetlinewidth{0.803000pt}%
\definecolor{currentstroke}{rgb}{0.000000,0.000000,0.000000}%
\pgfsetstrokecolor{currentstroke}%
\pgfsetdash{}{0pt}%
\pgfpathmoveto{\pgfqpoint{0.515000in}{0.502638in}}%
\pgfpathlineto{\pgfqpoint{2.375000in}{0.502638in}}%
\pgfusepath{stroke}%
\end{pgfscope}%
\begin{pgfscope}%
\pgfsetrectcap%
\pgfsetmiterjoin%
\pgfsetlinewidth{0.803000pt}%
\definecolor{currentstroke}{rgb}{0.000000,0.000000,0.000000}%
\pgfsetstrokecolor{currentstroke}%
\pgfsetdash{}{0pt}%
\pgfpathmoveto{\pgfqpoint{0.515000in}{1.888638in}}%
\pgfpathlineto{\pgfqpoint{2.375000in}{1.888638in}}%
\pgfusepath{stroke}%
\end{pgfscope}%
\begin{pgfscope}%
\definecolor{textcolor}{rgb}{0.000000,0.000000,0.000000}%
\pgfsetstrokecolor{textcolor}%
\pgfsetfillcolor{textcolor}%
\pgftext[x=1.445000in,y=1.971972in,,base]{\color{textcolor}\rmfamily\fontsize{12.000000}{14.400000}\selectfont Probability Distribution}%
\end{pgfscope}%
\begin{pgfscope}%
\pgfsetbuttcap%
\pgfsetmiterjoin%
\definecolor{currentfill}{rgb}{1.000000,1.000000,1.000000}%
\pgfsetfillcolor{currentfill}%
\pgfsetfillopacity{0.800000}%
\pgfsetlinewidth{1.003750pt}%
\definecolor{currentstroke}{rgb}{0.800000,0.800000,0.800000}%
\pgfsetstrokecolor{currentstroke}%
\pgfsetstrokeopacity{0.800000}%
\pgfsetdash{}{0pt}%
\pgfpathmoveto{\pgfqpoint{1.598056in}{1.388639in}}%
\pgfpathlineto{\pgfqpoint{2.277778in}{1.388639in}}%
\pgfpathquadraticcurveto{\pgfqpoint{2.305556in}{1.388639in}}{\pgfqpoint{2.305556in}{1.416417in}}%
\pgfpathlineto{\pgfqpoint{2.305556in}{1.791416in}}%
\pgfpathquadraticcurveto{\pgfqpoint{2.305556in}{1.819194in}}{\pgfqpoint{2.277778in}{1.819194in}}%
\pgfpathlineto{\pgfqpoint{1.598056in}{1.819194in}}%
\pgfpathquadraticcurveto{\pgfqpoint{1.570278in}{1.819194in}}{\pgfqpoint{1.570278in}{1.791416in}}%
\pgfpathlineto{\pgfqpoint{1.570278in}{1.416417in}}%
\pgfpathquadraticcurveto{\pgfqpoint{1.570278in}{1.388639in}}{\pgfqpoint{1.598056in}{1.388639in}}%
\pgfpathlineto{\pgfqpoint{1.598056in}{1.388639in}}%
\pgfpathclose%
\pgfusepath{stroke,fill}%
\end{pgfscope}%
\begin{pgfscope}%
\pgfsetbuttcap%
\pgfsetmiterjoin%
\pgfsetlinewidth{1.003750pt}%
\definecolor{currentstroke}{rgb}{0.000000,0.000000,0.000000}%
\pgfsetstrokecolor{currentstroke}%
\pgfsetdash{}{0pt}%
\pgfpathmoveto{\pgfqpoint{1.625834in}{1.666416in}}%
\pgfpathlineto{\pgfqpoint{1.903611in}{1.666416in}}%
\pgfpathlineto{\pgfqpoint{1.903611in}{1.763638in}}%
\pgfpathlineto{\pgfqpoint{1.625834in}{1.763638in}}%
\pgfpathlineto{\pgfqpoint{1.625834in}{1.666416in}}%
\pgfpathclose%
\pgfusepath{stroke}%
\end{pgfscope}%
\begin{pgfscope}%
\definecolor{textcolor}{rgb}{0.000000,0.000000,0.000000}%
\pgfsetstrokecolor{textcolor}%
\pgfsetfillcolor{textcolor}%
\pgftext[x=2.014722in,y=1.666416in,left,base]{\color{textcolor}\rmfamily\fontsize{10.000000}{12.000000}\selectfont Neg}%
\end{pgfscope}%
\begin{pgfscope}%
\pgfsetbuttcap%
\pgfsetmiterjoin%
\definecolor{currentfill}{rgb}{0.000000,0.000000,0.000000}%
\pgfsetfillcolor{currentfill}%
\pgfsetlinewidth{0.000000pt}%
\definecolor{currentstroke}{rgb}{0.000000,0.000000,0.000000}%
\pgfsetstrokecolor{currentstroke}%
\pgfsetstrokeopacity{0.000000}%
\pgfsetdash{}{0pt}%
\pgfpathmoveto{\pgfqpoint{1.625834in}{1.471139in}}%
\pgfpathlineto{\pgfqpoint{1.903611in}{1.471139in}}%
\pgfpathlineto{\pgfqpoint{1.903611in}{1.568361in}}%
\pgfpathlineto{\pgfqpoint{1.625834in}{1.568361in}}%
\pgfpathlineto{\pgfqpoint{1.625834in}{1.471139in}}%
\pgfpathclose%
\pgfusepath{fill}%
\end{pgfscope}%
\begin{pgfscope}%
\definecolor{textcolor}{rgb}{0.000000,0.000000,0.000000}%
\pgfsetstrokecolor{textcolor}%
\pgfsetfillcolor{textcolor}%
\pgftext[x=2.014722in,y=1.471139in,left,base]{\color{textcolor}\rmfamily\fontsize{10.000000}{12.000000}\selectfont Pos}%
\end{pgfscope}%
\end{pgfpicture}%
\makeatother%
\endgroup%

  &
  \vspace{0pt} %% Creator: Matplotlib, PGF backend
%%
%% To include the figure in your LaTeX document, write
%%   \input{<filename>.pgf}
%%
%% Make sure the required packages are loaded in your preamble
%%   \usepackage{pgf}
%%
%% Also ensure that all the required font packages are loaded; for instance,
%% the lmodern package is sometimes necessary when using math font.
%%   \usepackage{lmodern}
%%
%% Figures using additional raster images can only be included by \input if
%% they are in the same directory as the main LaTeX file. For loading figures
%% from other directories you can use the `import` package
%%   \usepackage{import}
%%
%% and then include the figures with
%%   \import{<path to file>}{<filename>.pgf}
%%
%% Matplotlib used the following preamble
%%   
%%   \usepackage{fontspec}
%%   \makeatletter\@ifpackageloaded{underscore}{}{\usepackage[strings]{underscore}}\makeatother
%%
\begingroup%
\makeatletter%
\begin{pgfpicture}%
\pgfpathrectangle{\pgfpointorigin}{\pgfqpoint{2.221861in}{1.953444in}}%
\pgfusepath{use as bounding box, clip}%
\begin{pgfscope}%
\pgfsetbuttcap%
\pgfsetmiterjoin%
\definecolor{currentfill}{rgb}{1.000000,1.000000,1.000000}%
\pgfsetfillcolor{currentfill}%
\pgfsetlinewidth{0.000000pt}%
\definecolor{currentstroke}{rgb}{1.000000,1.000000,1.000000}%
\pgfsetstrokecolor{currentstroke}%
\pgfsetdash{}{0pt}%
\pgfpathmoveto{\pgfqpoint{0.000000in}{0.000000in}}%
\pgfpathlineto{\pgfqpoint{2.221861in}{0.000000in}}%
\pgfpathlineto{\pgfqpoint{2.221861in}{1.953444in}}%
\pgfpathlineto{\pgfqpoint{0.000000in}{1.953444in}}%
\pgfpathlineto{\pgfqpoint{0.000000in}{0.000000in}}%
\pgfpathclose%
\pgfusepath{fill}%
\end{pgfscope}%
\begin{pgfscope}%
\pgfsetbuttcap%
\pgfsetmiterjoin%
\definecolor{currentfill}{rgb}{1.000000,1.000000,1.000000}%
\pgfsetfillcolor{currentfill}%
\pgfsetlinewidth{0.000000pt}%
\definecolor{currentstroke}{rgb}{0.000000,0.000000,0.000000}%
\pgfsetstrokecolor{currentstroke}%
\pgfsetstrokeopacity{0.000000}%
\pgfsetdash{}{0pt}%
\pgfpathmoveto{\pgfqpoint{0.553581in}{0.499444in}}%
\pgfpathlineto{\pgfqpoint{2.103581in}{0.499444in}}%
\pgfpathlineto{\pgfqpoint{2.103581in}{1.654444in}}%
\pgfpathlineto{\pgfqpoint{0.553581in}{1.654444in}}%
\pgfpathlineto{\pgfqpoint{0.553581in}{0.499444in}}%
\pgfpathclose%
\pgfusepath{fill}%
\end{pgfscope}%
\begin{pgfscope}%
\pgfsetbuttcap%
\pgfsetroundjoin%
\definecolor{currentfill}{rgb}{0.000000,0.000000,0.000000}%
\pgfsetfillcolor{currentfill}%
\pgfsetlinewidth{0.803000pt}%
\definecolor{currentstroke}{rgb}{0.000000,0.000000,0.000000}%
\pgfsetstrokecolor{currentstroke}%
\pgfsetdash{}{0pt}%
\pgfsys@defobject{currentmarker}{\pgfqpoint{0.000000in}{-0.048611in}}{\pgfqpoint{0.000000in}{0.000000in}}{%
\pgfpathmoveto{\pgfqpoint{0.000000in}{0.000000in}}%
\pgfpathlineto{\pgfqpoint{0.000000in}{-0.048611in}}%
\pgfusepath{stroke,fill}%
}%
\begin{pgfscope}%
\pgfsys@transformshift{0.624035in}{0.499444in}%
\pgfsys@useobject{currentmarker}{}%
\end{pgfscope}%
\end{pgfscope}%
\begin{pgfscope}%
\definecolor{textcolor}{rgb}{0.000000,0.000000,0.000000}%
\pgfsetstrokecolor{textcolor}%
\pgfsetfillcolor{textcolor}%
\pgftext[x=0.624035in,y=0.402222in,,top]{\color{textcolor}\rmfamily\fontsize{10.000000}{12.000000}\selectfont \(\displaystyle {0.0}\)}%
\end{pgfscope}%
\begin{pgfscope}%
\pgfsetbuttcap%
\pgfsetroundjoin%
\definecolor{currentfill}{rgb}{0.000000,0.000000,0.000000}%
\pgfsetfillcolor{currentfill}%
\pgfsetlinewidth{0.803000pt}%
\definecolor{currentstroke}{rgb}{0.000000,0.000000,0.000000}%
\pgfsetstrokecolor{currentstroke}%
\pgfsetdash{}{0pt}%
\pgfsys@defobject{currentmarker}{\pgfqpoint{0.000000in}{-0.048611in}}{\pgfqpoint{0.000000in}{0.000000in}}{%
\pgfpathmoveto{\pgfqpoint{0.000000in}{0.000000in}}%
\pgfpathlineto{\pgfqpoint{0.000000in}{-0.048611in}}%
\pgfusepath{stroke,fill}%
}%
\begin{pgfscope}%
\pgfsys@transformshift{1.328581in}{0.499444in}%
\pgfsys@useobject{currentmarker}{}%
\end{pgfscope}%
\end{pgfscope}%
\begin{pgfscope}%
\definecolor{textcolor}{rgb}{0.000000,0.000000,0.000000}%
\pgfsetstrokecolor{textcolor}%
\pgfsetfillcolor{textcolor}%
\pgftext[x=1.328581in,y=0.402222in,,top]{\color{textcolor}\rmfamily\fontsize{10.000000}{12.000000}\selectfont \(\displaystyle {0.5}\)}%
\end{pgfscope}%
\begin{pgfscope}%
\pgfsetbuttcap%
\pgfsetroundjoin%
\definecolor{currentfill}{rgb}{0.000000,0.000000,0.000000}%
\pgfsetfillcolor{currentfill}%
\pgfsetlinewidth{0.803000pt}%
\definecolor{currentstroke}{rgb}{0.000000,0.000000,0.000000}%
\pgfsetstrokecolor{currentstroke}%
\pgfsetdash{}{0pt}%
\pgfsys@defobject{currentmarker}{\pgfqpoint{0.000000in}{-0.048611in}}{\pgfqpoint{0.000000in}{0.000000in}}{%
\pgfpathmoveto{\pgfqpoint{0.000000in}{0.000000in}}%
\pgfpathlineto{\pgfqpoint{0.000000in}{-0.048611in}}%
\pgfusepath{stroke,fill}%
}%
\begin{pgfscope}%
\pgfsys@transformshift{2.033126in}{0.499444in}%
\pgfsys@useobject{currentmarker}{}%
\end{pgfscope}%
\end{pgfscope}%
\begin{pgfscope}%
\definecolor{textcolor}{rgb}{0.000000,0.000000,0.000000}%
\pgfsetstrokecolor{textcolor}%
\pgfsetfillcolor{textcolor}%
\pgftext[x=2.033126in,y=0.402222in,,top]{\color{textcolor}\rmfamily\fontsize{10.000000}{12.000000}\selectfont \(\displaystyle {1.0}\)}%
\end{pgfscope}%
\begin{pgfscope}%
\definecolor{textcolor}{rgb}{0.000000,0.000000,0.000000}%
\pgfsetstrokecolor{textcolor}%
\pgfsetfillcolor{textcolor}%
\pgftext[x=1.328581in,y=0.223333in,,top]{\color{textcolor}\rmfamily\fontsize{10.000000}{12.000000}\selectfont False positive rate}%
\end{pgfscope}%
\begin{pgfscope}%
\pgfsetbuttcap%
\pgfsetroundjoin%
\definecolor{currentfill}{rgb}{0.000000,0.000000,0.000000}%
\pgfsetfillcolor{currentfill}%
\pgfsetlinewidth{0.803000pt}%
\definecolor{currentstroke}{rgb}{0.000000,0.000000,0.000000}%
\pgfsetstrokecolor{currentstroke}%
\pgfsetdash{}{0pt}%
\pgfsys@defobject{currentmarker}{\pgfqpoint{-0.048611in}{0.000000in}}{\pgfqpoint{-0.000000in}{0.000000in}}{%
\pgfpathmoveto{\pgfqpoint{-0.000000in}{0.000000in}}%
\pgfpathlineto{\pgfqpoint{-0.048611in}{0.000000in}}%
\pgfusepath{stroke,fill}%
}%
\begin{pgfscope}%
\pgfsys@transformshift{0.553581in}{0.551944in}%
\pgfsys@useobject{currentmarker}{}%
\end{pgfscope}%
\end{pgfscope}%
\begin{pgfscope}%
\definecolor{textcolor}{rgb}{0.000000,0.000000,0.000000}%
\pgfsetstrokecolor{textcolor}%
\pgfsetfillcolor{textcolor}%
\pgftext[x=0.278889in, y=0.503750in, left, base]{\color{textcolor}\rmfamily\fontsize{10.000000}{12.000000}\selectfont \(\displaystyle {0.0}\)}%
\end{pgfscope}%
\begin{pgfscope}%
\pgfsetbuttcap%
\pgfsetroundjoin%
\definecolor{currentfill}{rgb}{0.000000,0.000000,0.000000}%
\pgfsetfillcolor{currentfill}%
\pgfsetlinewidth{0.803000pt}%
\definecolor{currentstroke}{rgb}{0.000000,0.000000,0.000000}%
\pgfsetstrokecolor{currentstroke}%
\pgfsetdash{}{0pt}%
\pgfsys@defobject{currentmarker}{\pgfqpoint{-0.048611in}{0.000000in}}{\pgfqpoint{-0.000000in}{0.000000in}}{%
\pgfpathmoveto{\pgfqpoint{-0.000000in}{0.000000in}}%
\pgfpathlineto{\pgfqpoint{-0.048611in}{0.000000in}}%
\pgfusepath{stroke,fill}%
}%
\begin{pgfscope}%
\pgfsys@transformshift{0.553581in}{1.076944in}%
\pgfsys@useobject{currentmarker}{}%
\end{pgfscope}%
\end{pgfscope}%
\begin{pgfscope}%
\definecolor{textcolor}{rgb}{0.000000,0.000000,0.000000}%
\pgfsetstrokecolor{textcolor}%
\pgfsetfillcolor{textcolor}%
\pgftext[x=0.278889in, y=1.028750in, left, base]{\color{textcolor}\rmfamily\fontsize{10.000000}{12.000000}\selectfont \(\displaystyle {0.5}\)}%
\end{pgfscope}%
\begin{pgfscope}%
\pgfsetbuttcap%
\pgfsetroundjoin%
\definecolor{currentfill}{rgb}{0.000000,0.000000,0.000000}%
\pgfsetfillcolor{currentfill}%
\pgfsetlinewidth{0.803000pt}%
\definecolor{currentstroke}{rgb}{0.000000,0.000000,0.000000}%
\pgfsetstrokecolor{currentstroke}%
\pgfsetdash{}{0pt}%
\pgfsys@defobject{currentmarker}{\pgfqpoint{-0.048611in}{0.000000in}}{\pgfqpoint{-0.000000in}{0.000000in}}{%
\pgfpathmoveto{\pgfqpoint{-0.000000in}{0.000000in}}%
\pgfpathlineto{\pgfqpoint{-0.048611in}{0.000000in}}%
\pgfusepath{stroke,fill}%
}%
\begin{pgfscope}%
\pgfsys@transformshift{0.553581in}{1.601944in}%
\pgfsys@useobject{currentmarker}{}%
\end{pgfscope}%
\end{pgfscope}%
\begin{pgfscope}%
\definecolor{textcolor}{rgb}{0.000000,0.000000,0.000000}%
\pgfsetstrokecolor{textcolor}%
\pgfsetfillcolor{textcolor}%
\pgftext[x=0.278889in, y=1.553750in, left, base]{\color{textcolor}\rmfamily\fontsize{10.000000}{12.000000}\selectfont \(\displaystyle {1.0}\)}%
\end{pgfscope}%
\begin{pgfscope}%
\definecolor{textcolor}{rgb}{0.000000,0.000000,0.000000}%
\pgfsetstrokecolor{textcolor}%
\pgfsetfillcolor{textcolor}%
\pgftext[x=0.223333in,y=1.076944in,,bottom,rotate=90.000000]{\color{textcolor}\rmfamily\fontsize{10.000000}{12.000000}\selectfont True positive rate}%
\end{pgfscope}%
\begin{pgfscope}%
\pgfpathrectangle{\pgfqpoint{0.553581in}{0.499444in}}{\pgfqpoint{1.550000in}{1.155000in}}%
\pgfusepath{clip}%
\pgfsetbuttcap%
\pgfsetroundjoin%
\pgfsetlinewidth{1.505625pt}%
\definecolor{currentstroke}{rgb}{0.000000,0.000000,0.000000}%
\pgfsetstrokecolor{currentstroke}%
\pgfsetdash{{5.550000pt}{2.400000pt}}{0.000000pt}%
\pgfpathmoveto{\pgfqpoint{0.624035in}{0.551944in}}%
\pgfpathlineto{\pgfqpoint{2.033126in}{1.601944in}}%
\pgfusepath{stroke}%
\end{pgfscope}%
\begin{pgfscope}%
\pgfpathrectangle{\pgfqpoint{0.553581in}{0.499444in}}{\pgfqpoint{1.550000in}{1.155000in}}%
\pgfusepath{clip}%
\pgfsetrectcap%
\pgfsetroundjoin%
\pgfsetlinewidth{1.505625pt}%
\definecolor{currentstroke}{rgb}{0.000000,0.000000,0.000000}%
\pgfsetstrokecolor{currentstroke}%
\pgfsetdash{}{0pt}%
\pgfpathmoveto{\pgfqpoint{0.624035in}{0.551944in}}%
\pgfpathlineto{\pgfqpoint{0.634474in}{0.583182in}}%
\pgfpathlineto{\pgfqpoint{0.635555in}{0.589009in}}%
\pgfpathlineto{\pgfqpoint{0.636001in}{0.589954in}}%
\pgfpathlineto{\pgfqpoint{0.637069in}{0.596937in}}%
\pgfpathlineto{\pgfqpoint{0.637386in}{0.597934in}}%
\pgfpathlineto{\pgfqpoint{0.638490in}{0.605179in}}%
\pgfpathlineto{\pgfqpoint{0.638702in}{0.606124in}}%
\pgfpathlineto{\pgfqpoint{0.639805in}{0.612372in}}%
\pgfpathlineto{\pgfqpoint{0.639958in}{0.613474in}}%
\pgfpathlineto{\pgfqpoint{0.641050in}{0.619302in}}%
\pgfpathlineto{\pgfqpoint{0.641308in}{0.620404in}}%
\pgfpathlineto{\pgfqpoint{0.642400in}{0.627702in}}%
\pgfpathlineto{\pgfqpoint{0.642506in}{0.628804in}}%
\pgfpathlineto{\pgfqpoint{0.643598in}{0.636942in}}%
\pgfpathlineto{\pgfqpoint{0.643798in}{0.637992in}}%
\pgfpathlineto{\pgfqpoint{0.644902in}{0.647284in}}%
\pgfpathlineto{\pgfqpoint{0.645207in}{0.648334in}}%
\pgfpathlineto{\pgfqpoint{0.646311in}{0.656052in}}%
\pgfpathlineto{\pgfqpoint{0.646452in}{0.657049in}}%
\pgfpathlineto{\pgfqpoint{0.647520in}{0.663927in}}%
\pgfpathlineto{\pgfqpoint{0.647802in}{0.665029in}}%
\pgfpathlineto{\pgfqpoint{0.648894in}{0.671854in}}%
\pgfpathlineto{\pgfqpoint{0.648953in}{0.672747in}}%
\pgfpathlineto{\pgfqpoint{0.650056in}{0.680359in}}%
\pgfpathlineto{\pgfqpoint{0.650221in}{0.681252in}}%
\pgfpathlineto{\pgfqpoint{0.651325in}{0.689967in}}%
\pgfpathlineto{\pgfqpoint{0.651466in}{0.691017in}}%
\pgfpathlineto{\pgfqpoint{0.652569in}{0.698577in}}%
\pgfpathlineto{\pgfqpoint{0.652687in}{0.699679in}}%
\pgfpathlineto{\pgfqpoint{0.653767in}{0.707449in}}%
\pgfpathlineto{\pgfqpoint{0.654002in}{0.708552in}}%
\pgfpathlineto{\pgfqpoint{0.655094in}{0.717424in}}%
\pgfpathlineto{\pgfqpoint{0.655305in}{0.718317in}}%
\pgfpathlineto{\pgfqpoint{0.656409in}{0.726664in}}%
\pgfpathlineto{\pgfqpoint{0.656620in}{0.727767in}}%
\pgfpathlineto{\pgfqpoint{0.657724in}{0.734592in}}%
\pgfpathlineto{\pgfqpoint{0.657853in}{0.735379in}}%
\pgfpathlineto{\pgfqpoint{0.658957in}{0.745144in}}%
\pgfpathlineto{\pgfqpoint{0.659122in}{0.746247in}}%
\pgfpathlineto{\pgfqpoint{0.660214in}{0.754594in}}%
\pgfpathlineto{\pgfqpoint{0.660355in}{0.755539in}}%
\pgfpathlineto{\pgfqpoint{0.661447in}{0.763467in}}%
\pgfpathlineto{\pgfqpoint{0.661681in}{0.764569in}}%
\pgfpathlineto{\pgfqpoint{0.662785in}{0.771867in}}%
\pgfpathlineto{\pgfqpoint{0.662997in}{0.772917in}}%
\pgfpathlineto{\pgfqpoint{0.664100in}{0.780162in}}%
\pgfpathlineto{\pgfqpoint{0.664230in}{0.781264in}}%
\pgfpathlineto{\pgfqpoint{0.665333in}{0.787354in}}%
\pgfpathlineto{\pgfqpoint{0.665451in}{0.788457in}}%
\pgfpathlineto{\pgfqpoint{0.666555in}{0.795597in}}%
\pgfpathlineto{\pgfqpoint{0.666695in}{0.796542in}}%
\pgfpathlineto{\pgfqpoint{0.667799in}{0.802737in}}%
\pgfpathlineto{\pgfqpoint{0.667987in}{0.803734in}}%
\pgfpathlineto{\pgfqpoint{0.669067in}{0.810507in}}%
\pgfpathlineto{\pgfqpoint{0.669279in}{0.811609in}}%
\pgfpathlineto{\pgfqpoint{0.670383in}{0.818277in}}%
\pgfpathlineto{\pgfqpoint{0.670535in}{0.819222in}}%
\pgfpathlineto{\pgfqpoint{0.671639in}{0.826572in}}%
\pgfpathlineto{\pgfqpoint{0.671815in}{0.827674in}}%
\pgfpathlineto{\pgfqpoint{0.672907in}{0.835602in}}%
\pgfpathlineto{\pgfqpoint{0.673072in}{0.836704in}}%
\pgfpathlineto{\pgfqpoint{0.674175in}{0.842584in}}%
\pgfpathlineto{\pgfqpoint{0.674457in}{0.843634in}}%
\pgfpathlineto{\pgfqpoint{0.675549in}{0.848569in}}%
\pgfpathlineto{\pgfqpoint{0.675819in}{0.849567in}}%
\pgfpathlineto{\pgfqpoint{0.676923in}{0.856339in}}%
\pgfpathlineto{\pgfqpoint{0.677088in}{0.857284in}}%
\pgfpathlineto{\pgfqpoint{0.678180in}{0.863479in}}%
\pgfpathlineto{\pgfqpoint{0.678391in}{0.864529in}}%
\pgfpathlineto{\pgfqpoint{0.679471in}{0.871144in}}%
\pgfpathlineto{\pgfqpoint{0.679777in}{0.872194in}}%
\pgfpathlineto{\pgfqpoint{0.680869in}{0.878704in}}%
\pgfpathlineto{\pgfqpoint{0.681150in}{0.879544in}}%
\pgfpathlineto{\pgfqpoint{0.682219in}{0.885319in}}%
\pgfpathlineto{\pgfqpoint{0.682360in}{0.886107in}}%
\pgfpathlineto{\pgfqpoint{0.683452in}{0.891357in}}%
\pgfpathlineto{\pgfqpoint{0.683722in}{0.892459in}}%
\pgfpathlineto{\pgfqpoint{0.684814in}{0.898287in}}%
\pgfpathlineto{\pgfqpoint{0.685061in}{0.899284in}}%
\pgfpathlineto{\pgfqpoint{0.686153in}{0.904534in}}%
\pgfpathlineto{\pgfqpoint{0.686388in}{0.905479in}}%
\pgfpathlineto{\pgfqpoint{0.687480in}{0.910624in}}%
\pgfpathlineto{\pgfqpoint{0.687785in}{0.911727in}}%
\pgfpathlineto{\pgfqpoint{0.688889in}{0.917134in}}%
\pgfpathlineto{\pgfqpoint{0.689311in}{0.918132in}}%
\pgfpathlineto{\pgfqpoint{0.690403in}{0.925377in}}%
\pgfpathlineto{\pgfqpoint{0.690544in}{0.926322in}}%
\pgfpathlineto{\pgfqpoint{0.691648in}{0.932044in}}%
\pgfpathlineto{\pgfqpoint{0.692000in}{0.933094in}}%
\pgfpathlineto{\pgfqpoint{0.693104in}{0.938712in}}%
\pgfpathlineto{\pgfqpoint{0.693456in}{0.939814in}}%
\pgfpathlineto{\pgfqpoint{0.694548in}{0.944382in}}%
\pgfpathlineto{\pgfqpoint{0.694877in}{0.945484in}}%
\pgfpathlineto{\pgfqpoint{0.695981in}{0.950892in}}%
\pgfpathlineto{\pgfqpoint{0.696204in}{0.951994in}}%
\pgfpathlineto{\pgfqpoint{0.697308in}{0.956457in}}%
\pgfpathlineto{\pgfqpoint{0.697602in}{0.957559in}}%
\pgfpathlineto{\pgfqpoint{0.698705in}{0.962179in}}%
\pgfpathlineto{\pgfqpoint{0.698905in}{0.963282in}}%
\pgfpathlineto{\pgfqpoint{0.700009in}{0.967272in}}%
\pgfpathlineto{\pgfqpoint{0.700150in}{0.968269in}}%
\pgfpathlineto{\pgfqpoint{0.701242in}{0.973152in}}%
\pgfpathlineto{\pgfqpoint{0.701477in}{0.974097in}}%
\pgfpathlineto{\pgfqpoint{0.702580in}{0.978244in}}%
\pgfpathlineto{\pgfqpoint{0.702933in}{0.979294in}}%
\pgfpathlineto{\pgfqpoint{0.703989in}{0.983652in}}%
\pgfpathlineto{\pgfqpoint{0.704342in}{0.984754in}}%
\pgfpathlineto{\pgfqpoint{0.705434in}{0.989322in}}%
\pgfpathlineto{\pgfqpoint{0.705680in}{0.990424in}}%
\pgfpathlineto{\pgfqpoint{0.706772in}{0.995674in}}%
\pgfpathlineto{\pgfqpoint{0.707101in}{0.996724in}}%
\pgfpathlineto{\pgfqpoint{0.708205in}{1.001502in}}%
\pgfpathlineto{\pgfqpoint{0.708545in}{1.002604in}}%
\pgfpathlineto{\pgfqpoint{0.709649in}{1.006594in}}%
\pgfpathlineto{\pgfqpoint{0.709978in}{1.007592in}}%
\pgfpathlineto{\pgfqpoint{0.711023in}{1.010637in}}%
\pgfpathlineto{\pgfqpoint{0.711434in}{1.011739in}}%
\pgfpathlineto{\pgfqpoint{0.712503in}{1.016149in}}%
\pgfpathlineto{\pgfqpoint{0.712867in}{1.017252in}}%
\pgfpathlineto{\pgfqpoint{0.713970in}{1.020822in}}%
\pgfpathlineto{\pgfqpoint{0.714323in}{1.021819in}}%
\pgfpathlineto{\pgfqpoint{0.715427in}{1.025127in}}%
\pgfpathlineto{\pgfqpoint{0.715884in}{1.026124in}}%
\pgfpathlineto{\pgfqpoint{0.716977in}{1.029957in}}%
\pgfpathlineto{\pgfqpoint{0.717247in}{1.030954in}}%
\pgfpathlineto{\pgfqpoint{0.718339in}{1.036624in}}%
\pgfpathlineto{\pgfqpoint{0.718573in}{1.037674in}}%
\pgfpathlineto{\pgfqpoint{0.719677in}{1.041664in}}%
\pgfpathlineto{\pgfqpoint{0.720147in}{1.042767in}}%
\pgfpathlineto{\pgfqpoint{0.721227in}{1.045707in}}%
\pgfpathlineto{\pgfqpoint{0.721580in}{1.046757in}}%
\pgfpathlineto{\pgfqpoint{0.722660in}{1.050222in}}%
\pgfpathlineto{\pgfqpoint{0.723047in}{1.051324in}}%
\pgfpathlineto{\pgfqpoint{0.724139in}{1.054789in}}%
\pgfpathlineto{\pgfqpoint{0.724644in}{1.055892in}}%
\pgfpathlineto{\pgfqpoint{0.725736in}{1.059777in}}%
\pgfpathlineto{\pgfqpoint{0.726159in}{1.060827in}}%
\pgfpathlineto{\pgfqpoint{0.727263in}{1.063924in}}%
\pgfpathlineto{\pgfqpoint{0.727603in}{1.064974in}}%
\pgfpathlineto{\pgfqpoint{0.728672in}{1.068124in}}%
\pgfpathlineto{\pgfqpoint{0.729200in}{1.069227in}}%
\pgfpathlineto{\pgfqpoint{0.730304in}{1.072167in}}%
\pgfpathlineto{\pgfqpoint{0.730551in}{1.073269in}}%
\pgfpathlineto{\pgfqpoint{0.731631in}{1.076157in}}%
\pgfpathlineto{\pgfqpoint{0.732054in}{1.077259in}}%
\pgfpathlineto{\pgfqpoint{0.733122in}{1.080567in}}%
\pgfpathlineto{\pgfqpoint{0.733756in}{1.081669in}}%
\pgfpathlineto{\pgfqpoint{0.734848in}{1.084662in}}%
\pgfpathlineto{\pgfqpoint{0.735166in}{1.085764in}}%
\pgfpathlineto{\pgfqpoint{0.736258in}{1.089229in}}%
\pgfpathlineto{\pgfqpoint{0.736739in}{1.090332in}}%
\pgfpathlineto{\pgfqpoint{0.737808in}{1.093272in}}%
\pgfpathlineto{\pgfqpoint{0.738442in}{1.094374in}}%
\pgfpathlineto{\pgfqpoint{0.739522in}{1.098049in}}%
\pgfpathlineto{\pgfqpoint{0.739921in}{1.099152in}}%
\pgfpathlineto{\pgfqpoint{0.741025in}{1.101987in}}%
\pgfpathlineto{\pgfqpoint{0.741636in}{1.103089in}}%
\pgfpathlineto{\pgfqpoint{0.742739in}{1.105872in}}%
\pgfpathlineto{\pgfqpoint{0.743268in}{1.106974in}}%
\pgfpathlineto{\pgfqpoint{0.744360in}{1.109704in}}%
\pgfpathlineto{\pgfqpoint{0.744712in}{1.110754in}}%
\pgfpathlineto{\pgfqpoint{0.744747in}{1.110912in}}%
\pgfpathlineto{\pgfqpoint{0.744759in}{1.110912in}}%
\pgfpathlineto{\pgfqpoint{0.755363in}{1.112014in}}%
\pgfpathlineto{\pgfqpoint{0.756466in}{1.114377in}}%
\pgfpathlineto{\pgfqpoint{0.756772in}{1.115479in}}%
\pgfpathlineto{\pgfqpoint{0.757875in}{1.119627in}}%
\pgfpathlineto{\pgfqpoint{0.758592in}{1.120729in}}%
\pgfpathlineto{\pgfqpoint{0.759672in}{1.122882in}}%
\pgfpathlineto{\pgfqpoint{0.760153in}{1.123984in}}%
\pgfpathlineto{\pgfqpoint{0.761234in}{1.126819in}}%
\pgfpathlineto{\pgfqpoint{0.761727in}{1.127922in}}%
\pgfpathlineto{\pgfqpoint{0.762831in}{1.131387in}}%
\pgfpathlineto{\pgfqpoint{0.763312in}{1.132437in}}%
\pgfpathlineto{\pgfqpoint{0.764416in}{1.135219in}}%
\pgfpathlineto{\pgfqpoint{0.764897in}{1.136217in}}%
\pgfpathlineto{\pgfqpoint{0.765989in}{1.139104in}}%
\pgfpathlineto{\pgfqpoint{0.766389in}{1.140154in}}%
\pgfpathlineto{\pgfqpoint{0.767492in}{1.143199in}}%
\pgfpathlineto{\pgfqpoint{0.767997in}{1.144302in}}%
\pgfpathlineto{\pgfqpoint{0.769019in}{1.146349in}}%
\pgfpathlineto{\pgfqpoint{0.769547in}{1.147452in}}%
\pgfpathlineto{\pgfqpoint{0.770639in}{1.150182in}}%
\pgfpathlineto{\pgfqpoint{0.771273in}{1.151284in}}%
\pgfpathlineto{\pgfqpoint{0.772330in}{1.154014in}}%
\pgfpathlineto{\pgfqpoint{0.773129in}{1.155117in}}%
\pgfpathlineto{\pgfqpoint{0.774174in}{1.158004in}}%
\pgfpathlineto{\pgfqpoint{0.774855in}{1.159054in}}%
\pgfpathlineto{\pgfqpoint{0.775959in}{1.161364in}}%
\pgfpathlineto{\pgfqpoint{0.776170in}{1.162467in}}%
\pgfpathlineto{\pgfqpoint{0.777274in}{1.165197in}}%
\pgfpathlineto{\pgfqpoint{0.777779in}{1.166299in}}%
\pgfpathlineto{\pgfqpoint{0.778883in}{1.168819in}}%
\pgfpathlineto{\pgfqpoint{0.779481in}{1.169922in}}%
\pgfpathlineto{\pgfqpoint{0.780585in}{1.171339in}}%
\pgfpathlineto{\pgfqpoint{0.781008in}{1.172389in}}%
\pgfpathlineto{\pgfqpoint{0.782065in}{1.175119in}}%
\pgfpathlineto{\pgfqpoint{0.782581in}{1.176169in}}%
\pgfpathlineto{\pgfqpoint{0.783673in}{1.177797in}}%
\pgfpathlineto{\pgfqpoint{0.784261in}{1.178899in}}%
\pgfpathlineto{\pgfqpoint{0.785247in}{1.180684in}}%
\pgfpathlineto{\pgfqpoint{0.785822in}{1.181787in}}%
\pgfpathlineto{\pgfqpoint{0.786903in}{1.183887in}}%
\pgfpathlineto{\pgfqpoint{0.787830in}{1.184937in}}%
\pgfpathlineto{\pgfqpoint{0.788817in}{1.186879in}}%
\pgfpathlineto{\pgfqpoint{0.789556in}{1.187982in}}%
\pgfpathlineto{\pgfqpoint{0.790519in}{1.190029in}}%
\pgfpathlineto{\pgfqpoint{0.791353in}{1.191132in}}%
\pgfpathlineto{\pgfqpoint{0.792445in}{1.193442in}}%
\pgfpathlineto{\pgfqpoint{0.792997in}{1.194544in}}%
\pgfpathlineto{\pgfqpoint{0.794054in}{1.196067in}}%
\pgfpathlineto{\pgfqpoint{0.794876in}{1.197169in}}%
\pgfpathlineto{\pgfqpoint{0.795886in}{1.198482in}}%
\pgfpathlineto{\pgfqpoint{0.797201in}{1.199584in}}%
\pgfpathlineto{\pgfqpoint{0.798258in}{1.201579in}}%
\pgfpathlineto{\pgfqpoint{0.798739in}{1.202682in}}%
\pgfpathlineto{\pgfqpoint{0.799843in}{1.204572in}}%
\pgfpathlineto{\pgfqpoint{0.800195in}{1.205674in}}%
\pgfpathlineto{\pgfqpoint{0.801287in}{1.207564in}}%
\pgfpathlineto{\pgfqpoint{0.801874in}{1.208667in}}%
\pgfpathlineto{\pgfqpoint{0.802966in}{1.211239in}}%
\pgfpathlineto{\pgfqpoint{0.803882in}{1.212342in}}%
\pgfpathlineto{\pgfqpoint{0.804986in}{1.214337in}}%
\pgfpathlineto{\pgfqpoint{0.805714in}{1.215439in}}%
\pgfpathlineto{\pgfqpoint{0.806759in}{1.217277in}}%
\pgfpathlineto{\pgfqpoint{0.807111in}{1.218274in}}%
\pgfpathlineto{\pgfqpoint{0.808180in}{1.221109in}}%
\pgfpathlineto{\pgfqpoint{0.809084in}{1.222212in}}%
\pgfpathlineto{\pgfqpoint{0.810176in}{1.224049in}}%
\pgfpathlineto{\pgfqpoint{0.810892in}{1.225152in}}%
\pgfpathlineto{\pgfqpoint{0.811938in}{1.226517in}}%
\pgfpathlineto{\pgfqpoint{0.812478in}{1.227567in}}%
\pgfpathlineto{\pgfqpoint{0.813570in}{1.229614in}}%
\pgfpathlineto{\pgfqpoint{0.814122in}{1.230717in}}%
\pgfpathlineto{\pgfqpoint{0.815225in}{1.233027in}}%
\pgfpathlineto{\pgfqpoint{0.815942in}{1.234129in}}%
\pgfpathlineto{\pgfqpoint{0.816940in}{1.235809in}}%
\pgfpathlineto{\pgfqpoint{0.817820in}{1.236859in}}%
\pgfpathlineto{\pgfqpoint{0.818889in}{1.238697in}}%
\pgfpathlineto{\pgfqpoint{0.819429in}{1.239799in}}%
\pgfpathlineto{\pgfqpoint{0.820486in}{1.241689in}}%
\pgfpathlineto{\pgfqpoint{0.820897in}{1.242792in}}%
\pgfpathlineto{\pgfqpoint{0.821977in}{1.244367in}}%
\pgfpathlineto{\pgfqpoint{0.822881in}{1.245469in}}%
\pgfpathlineto{\pgfqpoint{0.823985in}{1.247464in}}%
\pgfpathlineto{\pgfqpoint{0.824655in}{1.248567in}}%
\pgfpathlineto{\pgfqpoint{0.825711in}{1.249827in}}%
\pgfpathlineto{\pgfqpoint{0.826369in}{1.250929in}}%
\pgfpathlineto{\pgfqpoint{0.827438in}{1.252767in}}%
\pgfpathlineto{\pgfqpoint{0.828001in}{1.253869in}}%
\pgfpathlineto{\pgfqpoint{0.829105in}{1.255812in}}%
\pgfpathlineto{\pgfqpoint{0.829845in}{1.256914in}}%
\pgfpathlineto{\pgfqpoint{0.830948in}{1.258804in}}%
\pgfpathlineto{\pgfqpoint{0.831712in}{1.259907in}}%
\pgfpathlineto{\pgfqpoint{0.832816in}{1.261797in}}%
\pgfpathlineto{\pgfqpoint{0.833861in}{1.262899in}}%
\pgfpathlineto{\pgfqpoint{0.834953in}{1.264422in}}%
\pgfpathlineto{\pgfqpoint{0.835739in}{1.265472in}}%
\pgfpathlineto{\pgfqpoint{0.836784in}{1.267257in}}%
\pgfpathlineto{\pgfqpoint{0.837477in}{1.268359in}}%
\pgfpathlineto{\pgfqpoint{0.838558in}{1.269724in}}%
\pgfpathlineto{\pgfqpoint{0.839920in}{1.270774in}}%
\pgfpathlineto{\pgfqpoint{0.841023in}{1.272664in}}%
\pgfpathlineto{\pgfqpoint{0.841975in}{1.273767in}}%
\pgfpathlineto{\pgfqpoint{0.843055in}{1.274869in}}%
\pgfpathlineto{\pgfqpoint{0.844217in}{1.275972in}}%
\pgfpathlineto{\pgfqpoint{0.845298in}{1.277809in}}%
\pgfpathlineto{\pgfqpoint{0.846002in}{1.278912in}}%
\pgfpathlineto{\pgfqpoint{0.847094in}{1.280172in}}%
\pgfpathlineto{\pgfqpoint{0.847963in}{1.281274in}}%
\pgfpathlineto{\pgfqpoint{0.849067in}{1.282902in}}%
\pgfpathlineto{\pgfqpoint{0.850124in}{1.284004in}}%
\pgfpathlineto{\pgfqpoint{0.851228in}{1.285474in}}%
\pgfpathlineto{\pgfqpoint{0.852472in}{1.286577in}}%
\pgfpathlineto{\pgfqpoint{0.853576in}{1.288624in}}%
\pgfpathlineto{\pgfqpoint{0.854645in}{1.289727in}}%
\pgfpathlineto{\pgfqpoint{0.855690in}{1.291197in}}%
\pgfpathlineto{\pgfqpoint{0.856934in}{1.292299in}}%
\pgfpathlineto{\pgfqpoint{0.857874in}{1.293297in}}%
\pgfpathlineto{\pgfqpoint{0.858766in}{1.294399in}}%
\pgfpathlineto{\pgfqpoint{0.859858in}{1.295817in}}%
\pgfpathlineto{\pgfqpoint{0.860868in}{1.296919in}}%
\pgfpathlineto{\pgfqpoint{0.861960in}{1.298179in}}%
\pgfpathlineto{\pgfqpoint{0.863205in}{1.299282in}}%
\pgfpathlineto{\pgfqpoint{0.864285in}{1.300962in}}%
\pgfpathlineto{\pgfqpoint{0.864837in}{1.302064in}}%
\pgfpathlineto{\pgfqpoint{0.865894in}{1.303482in}}%
\pgfpathlineto{\pgfqpoint{0.867009in}{1.304584in}}%
\pgfpathlineto{\pgfqpoint{0.868113in}{1.305582in}}%
\pgfpathlineto{\pgfqpoint{0.869205in}{1.306684in}}%
\pgfpathlineto{\pgfqpoint{0.870286in}{1.307682in}}%
\pgfpathlineto{\pgfqpoint{0.871131in}{1.308784in}}%
\pgfpathlineto{\pgfqpoint{0.872211in}{1.310464in}}%
\pgfpathlineto{\pgfqpoint{0.873092in}{1.311567in}}%
\pgfpathlineto{\pgfqpoint{0.874196in}{1.312932in}}%
\pgfpathlineto{\pgfqpoint{0.875523in}{1.314034in}}%
\pgfpathlineto{\pgfqpoint{0.876568in}{1.315189in}}%
\pgfpathlineto{\pgfqpoint{0.878141in}{1.316292in}}%
\pgfpathlineto{\pgfqpoint{0.879245in}{1.317919in}}%
\pgfpathlineto{\pgfqpoint{0.880513in}{1.319022in}}%
\pgfpathlineto{\pgfqpoint{0.881570in}{1.319967in}}%
\pgfpathlineto{\pgfqpoint{0.882521in}{1.321069in}}%
\pgfpathlineto{\pgfqpoint{0.883566in}{1.321962in}}%
\pgfpathlineto{\pgfqpoint{0.884564in}{1.323064in}}%
\pgfpathlineto{\pgfqpoint{0.885656in}{1.324639in}}%
\pgfpathlineto{\pgfqpoint{0.886455in}{1.325742in}}%
\pgfpathlineto{\pgfqpoint{0.887559in}{1.327054in}}%
\pgfpathlineto{\pgfqpoint{0.889250in}{1.328157in}}%
\pgfpathlineto{\pgfqpoint{0.890342in}{1.328944in}}%
\pgfpathlineto{\pgfqpoint{0.891316in}{1.330047in}}%
\pgfpathlineto{\pgfqpoint{0.892408in}{1.331307in}}%
\pgfpathlineto{\pgfqpoint{0.893864in}{1.332409in}}%
\pgfpathlineto{\pgfqpoint{0.894956in}{1.333774in}}%
\pgfpathlineto{\pgfqpoint{0.896166in}{1.334877in}}%
\pgfpathlineto{\pgfqpoint{0.897246in}{1.336032in}}%
\pgfpathlineto{\pgfqpoint{0.898667in}{1.337134in}}%
\pgfpathlineto{\pgfqpoint{0.899712in}{1.338237in}}%
\pgfpathlineto{\pgfqpoint{0.901520in}{1.339339in}}%
\pgfpathlineto{\pgfqpoint{0.902577in}{1.340284in}}%
\pgfpathlineto{\pgfqpoint{0.903446in}{1.341334in}}%
\pgfpathlineto{\pgfqpoint{0.904538in}{1.342594in}}%
\pgfpathlineto{\pgfqpoint{0.906065in}{1.343697in}}%
\pgfpathlineto{\pgfqpoint{0.907133in}{1.344484in}}%
\pgfpathlineto{\pgfqpoint{0.908625in}{1.345534in}}%
\pgfpathlineto{\pgfqpoint{0.909576in}{1.346322in}}%
\pgfpathlineto{\pgfqpoint{0.911314in}{1.347372in}}%
\pgfpathlineto{\pgfqpoint{0.912406in}{1.348894in}}%
\pgfpathlineto{\pgfqpoint{0.913674in}{1.349997in}}%
\pgfpathlineto{\pgfqpoint{0.914742in}{1.351047in}}%
\pgfpathlineto{\pgfqpoint{0.916281in}{1.352149in}}%
\pgfpathlineto{\pgfqpoint{0.917338in}{1.353042in}}%
\pgfpathlineto{\pgfqpoint{0.918993in}{1.354144in}}%
\pgfpathlineto{\pgfqpoint{0.920073in}{1.355037in}}%
\pgfpathlineto{\pgfqpoint{0.921330in}{1.356139in}}%
\pgfpathlineto{\pgfqpoint{0.922434in}{1.357662in}}%
\pgfpathlineto{\pgfqpoint{0.924042in}{1.358764in}}%
\pgfpathlineto{\pgfqpoint{0.925123in}{1.359814in}}%
\pgfpathlineto{\pgfqpoint{0.927354in}{1.360917in}}%
\pgfpathlineto{\pgfqpoint{0.928422in}{1.361967in}}%
\pgfpathlineto{\pgfqpoint{0.930242in}{1.363069in}}%
\pgfpathlineto{\pgfqpoint{0.931334in}{1.364277in}}%
\pgfpathlineto{\pgfqpoint{0.932673in}{1.365327in}}%
\pgfpathlineto{\pgfqpoint{0.933718in}{1.366534in}}%
\pgfpathlineto{\pgfqpoint{0.935397in}{1.367637in}}%
\pgfpathlineto{\pgfqpoint{0.936454in}{1.368477in}}%
\pgfpathlineto{\pgfqpoint{0.937898in}{1.369579in}}%
\pgfpathlineto{\pgfqpoint{0.938908in}{1.370262in}}%
\pgfpathlineto{\pgfqpoint{0.940822in}{1.371364in}}%
\pgfpathlineto{\pgfqpoint{0.941691in}{1.371994in}}%
\pgfpathlineto{\pgfqpoint{0.943417in}{1.373097in}}%
\pgfpathlineto{\pgfqpoint{0.944427in}{1.373937in}}%
\pgfpathlineto{\pgfqpoint{0.945754in}{1.375039in}}%
\pgfpathlineto{\pgfqpoint{0.946799in}{1.375617in}}%
\pgfpathlineto{\pgfqpoint{0.948408in}{1.376719in}}%
\pgfpathlineto{\pgfqpoint{0.949465in}{1.377717in}}%
\pgfpathlineto{\pgfqpoint{0.951203in}{1.378819in}}%
\pgfpathlineto{\pgfqpoint{0.952271in}{1.379554in}}%
\pgfpathlineto{\pgfqpoint{0.953504in}{1.380604in}}%
\pgfpathlineto{\pgfqpoint{0.954573in}{1.381182in}}%
\pgfpathlineto{\pgfqpoint{0.955970in}{1.382179in}}%
\pgfpathlineto{\pgfqpoint{0.957003in}{1.383124in}}%
\pgfpathlineto{\pgfqpoint{0.959105in}{1.384227in}}%
\pgfpathlineto{\pgfqpoint{0.960162in}{1.384909in}}%
\pgfpathlineto{\pgfqpoint{0.962217in}{1.386012in}}%
\pgfpathlineto{\pgfqpoint{0.963297in}{1.386694in}}%
\pgfpathlineto{\pgfqpoint{0.965575in}{1.387797in}}%
\pgfpathlineto{\pgfqpoint{0.966609in}{1.388374in}}%
\pgfpathlineto{\pgfqpoint{0.968030in}{1.389372in}}%
\pgfpathlineto{\pgfqpoint{0.969122in}{1.390422in}}%
\pgfpathlineto{\pgfqpoint{0.971423in}{1.391524in}}%
\pgfpathlineto{\pgfqpoint{0.972527in}{1.392522in}}%
\pgfpathlineto{\pgfqpoint{0.974652in}{1.393624in}}%
\pgfpathlineto{\pgfqpoint{0.975697in}{1.394307in}}%
\pgfpathlineto{\pgfqpoint{0.978069in}{1.395409in}}%
\pgfpathlineto{\pgfqpoint{0.979150in}{1.396092in}}%
\pgfpathlineto{\pgfqpoint{0.980782in}{1.397194in}}%
\pgfpathlineto{\pgfqpoint{0.981862in}{1.397824in}}%
\pgfpathlineto{\pgfqpoint{0.984093in}{1.398822in}}%
\pgfpathlineto{\pgfqpoint{0.985173in}{1.399767in}}%
\pgfpathlineto{\pgfqpoint{0.987581in}{1.400869in}}%
\pgfpathlineto{\pgfqpoint{0.988649in}{1.401342in}}%
\pgfpathlineto{\pgfqpoint{0.991139in}{1.402444in}}%
\pgfpathlineto{\pgfqpoint{0.992242in}{1.402864in}}%
\pgfpathlineto{\pgfqpoint{0.993816in}{1.403914in}}%
\pgfpathlineto{\pgfqpoint{0.994697in}{1.404807in}}%
\pgfpathlineto{\pgfqpoint{0.996951in}{1.405909in}}%
\pgfpathlineto{\pgfqpoint{0.998055in}{1.406749in}}%
\pgfpathlineto{\pgfqpoint{1.000216in}{1.407852in}}%
\pgfpathlineto{\pgfqpoint{1.001284in}{1.408377in}}%
\pgfpathlineto{\pgfqpoint{1.003069in}{1.409427in}}%
\pgfpathlineto{\pgfqpoint{1.004044in}{1.410214in}}%
\pgfpathlineto{\pgfqpoint{1.005746in}{1.411264in}}%
\pgfpathlineto{\pgfqpoint{1.006838in}{1.412262in}}%
\pgfpathlineto{\pgfqpoint{1.008858in}{1.413364in}}%
\pgfpathlineto{\pgfqpoint{1.009962in}{1.413837in}}%
\pgfpathlineto{\pgfqpoint{1.012909in}{1.414939in}}%
\pgfpathlineto{\pgfqpoint{1.013790in}{1.415307in}}%
\pgfpathlineto{\pgfqpoint{1.015786in}{1.416357in}}%
\pgfpathlineto{\pgfqpoint{1.016855in}{1.417092in}}%
\pgfpathlineto{\pgfqpoint{1.018604in}{1.418194in}}%
\pgfpathlineto{\pgfqpoint{1.019708in}{1.419087in}}%
\pgfpathlineto{\pgfqpoint{1.021857in}{1.420137in}}%
\pgfpathlineto{\pgfqpoint{1.022937in}{1.420767in}}%
\pgfpathlineto{\pgfqpoint{1.025673in}{1.421869in}}%
\pgfpathlineto{\pgfqpoint{1.026683in}{1.422657in}}%
\pgfpathlineto{\pgfqpoint{1.029219in}{1.423759in}}%
\pgfpathlineto{\pgfqpoint{1.030123in}{1.424599in}}%
\pgfpathlineto{\pgfqpoint{1.033071in}{1.425702in}}%
\pgfpathlineto{\pgfqpoint{1.034092in}{1.426647in}}%
\pgfpathlineto{\pgfqpoint{1.037333in}{1.427749in}}%
\pgfpathlineto{\pgfqpoint{1.038414in}{1.428484in}}%
\pgfpathlineto{\pgfqpoint{1.039917in}{1.429587in}}%
\pgfpathlineto{\pgfqpoint{1.040962in}{1.430217in}}%
\pgfpathlineto{\pgfqpoint{1.043486in}{1.431319in}}%
\pgfpathlineto{\pgfqpoint{1.044461in}{1.431897in}}%
\pgfpathlineto{\pgfqpoint{1.046622in}{1.432999in}}%
\pgfpathlineto{\pgfqpoint{1.047479in}{1.433629in}}%
\pgfpathlineto{\pgfqpoint{1.049498in}{1.434732in}}%
\pgfpathlineto{\pgfqpoint{1.050532in}{1.435204in}}%
\pgfpathlineto{\pgfqpoint{1.052035in}{1.436307in}}%
\pgfpathlineto{\pgfqpoint{1.052786in}{1.436989in}}%
\pgfpathlineto{\pgfqpoint{1.055922in}{1.438092in}}%
\pgfpathlineto{\pgfqpoint{1.056802in}{1.438354in}}%
\pgfpathlineto{\pgfqpoint{1.059573in}{1.439457in}}%
\pgfpathlineto{\pgfqpoint{1.060548in}{1.440139in}}%
\pgfpathlineto{\pgfqpoint{1.063178in}{1.441242in}}%
\pgfpathlineto{\pgfqpoint{1.064118in}{1.441872in}}%
\pgfpathlineto{\pgfqpoint{1.067370in}{1.442974in}}%
\pgfpathlineto{\pgfqpoint{1.068416in}{1.443499in}}%
\pgfpathlineto{\pgfqpoint{1.071116in}{1.444602in}}%
\pgfpathlineto{\pgfqpoint{1.072197in}{1.445284in}}%
\pgfpathlineto{\pgfqpoint{1.074498in}{1.446387in}}%
\pgfpathlineto{\pgfqpoint{1.075555in}{1.447122in}}%
\pgfpathlineto{\pgfqpoint{1.078573in}{1.448224in}}%
\pgfpathlineto{\pgfqpoint{1.079677in}{1.449064in}}%
\pgfpathlineto{\pgfqpoint{1.083575in}{1.450167in}}%
\pgfpathlineto{\pgfqpoint{1.084362in}{1.450482in}}%
\pgfpathlineto{\pgfqpoint{1.087485in}{1.451584in}}%
\pgfpathlineto{\pgfqpoint{1.088589in}{1.452267in}}%
\pgfpathlineto{\pgfqpoint{1.091442in}{1.453369in}}%
\pgfpathlineto{\pgfqpoint{1.092452in}{1.453842in}}%
\pgfpathlineto{\pgfqpoint{1.096339in}{1.454944in}}%
\pgfpathlineto{\pgfqpoint{1.097196in}{1.455312in}}%
\pgfpathlineto{\pgfqpoint{1.099721in}{1.456414in}}%
\pgfpathlineto{\pgfqpoint{1.100766in}{1.456939in}}%
\pgfpathlineto{\pgfqpoint{1.103936in}{1.458042in}}%
\pgfpathlineto{\pgfqpoint{1.104735in}{1.458252in}}%
\pgfpathlineto{\pgfqpoint{1.108716in}{1.459354in}}%
\pgfpathlineto{\pgfqpoint{1.109702in}{1.459774in}}%
\pgfpathlineto{\pgfqpoint{1.113553in}{1.460877in}}%
\pgfpathlineto{\pgfqpoint{1.114434in}{1.461297in}}%
\pgfpathlineto{\pgfqpoint{1.117346in}{1.462399in}}%
\pgfpathlineto{\pgfqpoint{1.118156in}{1.462819in}}%
\pgfpathlineto{\pgfqpoint{1.120834in}{1.463922in}}%
\pgfpathlineto{\pgfqpoint{1.121914in}{1.464499in}}%
\pgfpathlineto{\pgfqpoint{1.125437in}{1.465602in}}%
\pgfpathlineto{\pgfqpoint{1.126400in}{1.465917in}}%
\pgfpathlineto{\pgfqpoint{1.130545in}{1.467019in}}%
\pgfpathlineto{\pgfqpoint{1.131425in}{1.467334in}}%
\pgfpathlineto{\pgfqpoint{1.134044in}{1.468384in}}%
\pgfpathlineto{\pgfqpoint{1.134925in}{1.468804in}}%
\pgfpathlineto{\pgfqpoint{1.137438in}{1.469907in}}%
\pgfpathlineto{\pgfqpoint{1.138541in}{1.470642in}}%
\pgfpathlineto{\pgfqpoint{1.142557in}{1.471744in}}%
\pgfpathlineto{\pgfqpoint{1.143156in}{1.472007in}}%
\pgfpathlineto{\pgfqpoint{1.147923in}{1.473109in}}%
\pgfpathlineto{\pgfqpoint{1.148745in}{1.473529in}}%
\pgfpathlineto{\pgfqpoint{1.151740in}{1.474632in}}%
\pgfpathlineto{\pgfqpoint{1.152820in}{1.475052in}}%
\pgfpathlineto{\pgfqpoint{1.155944in}{1.476154in}}%
\pgfpathlineto{\pgfqpoint{1.157012in}{1.476627in}}%
\pgfpathlineto{\pgfqpoint{1.161263in}{1.477729in}}%
\pgfpathlineto{\pgfqpoint{1.162144in}{1.478254in}}%
\pgfpathlineto{\pgfqpoint{1.166430in}{1.479357in}}%
\pgfpathlineto{\pgfqpoint{1.167439in}{1.479934in}}%
\pgfpathlineto{\pgfqpoint{1.170845in}{1.481037in}}%
\pgfpathlineto{\pgfqpoint{1.171725in}{1.481352in}}%
\pgfpathlineto{\pgfqpoint{1.176340in}{1.482454in}}%
\pgfpathlineto{\pgfqpoint{1.177432in}{1.482769in}}%
\pgfpathlineto{\pgfqpoint{1.182305in}{1.483872in}}%
\pgfpathlineto{\pgfqpoint{1.183386in}{1.484397in}}%
\pgfpathlineto{\pgfqpoint{1.188047in}{1.485499in}}%
\pgfpathlineto{\pgfqpoint{1.189034in}{1.486077in}}%
\pgfpathlineto{\pgfqpoint{1.194095in}{1.487179in}}%
\pgfpathlineto{\pgfqpoint{1.194964in}{1.487494in}}%
\pgfpathlineto{\pgfqpoint{1.199484in}{1.488597in}}%
\pgfpathlineto{\pgfqpoint{1.200518in}{1.489227in}}%
\pgfpathlineto{\pgfqpoint{1.203418in}{1.490329in}}%
\pgfpathlineto{\pgfqpoint{1.204252in}{1.490697in}}%
\pgfpathlineto{\pgfqpoint{1.207211in}{1.491799in}}%
\pgfpathlineto{\pgfqpoint{1.207211in}{1.491852in}}%
\pgfpathlineto{\pgfqpoint{1.211791in}{1.492954in}}%
\pgfpathlineto{\pgfqpoint{1.212789in}{1.493322in}}%
\pgfpathlineto{\pgfqpoint{1.217192in}{1.494424in}}%
\pgfpathlineto{\pgfqpoint{1.217897in}{1.494634in}}%
\pgfpathlineto{\pgfqpoint{1.223005in}{1.495737in}}%
\pgfpathlineto{\pgfqpoint{1.223897in}{1.495842in}}%
\pgfpathlineto{\pgfqpoint{1.226621in}{1.496944in}}%
\pgfpathlineto{\pgfqpoint{1.227643in}{1.497469in}}%
\pgfpathlineto{\pgfqpoint{1.235322in}{1.498572in}}%
\pgfpathlineto{\pgfqpoint{1.236367in}{1.498729in}}%
\pgfpathlineto{\pgfqpoint{1.241757in}{1.499832in}}%
\pgfpathlineto{\pgfqpoint{1.242133in}{1.499989in}}%
\pgfpathlineto{\pgfqpoint{1.248063in}{1.501092in}}%
\pgfpathlineto{\pgfqpoint{1.248991in}{1.501354in}}%
\pgfpathlineto{\pgfqpoint{1.254028in}{1.502457in}}%
\pgfpathlineto{\pgfqpoint{1.255108in}{1.502614in}}%
\pgfpathlineto{\pgfqpoint{1.259946in}{1.503717in}}%
\pgfpathlineto{\pgfqpoint{1.260933in}{1.503927in}}%
\pgfpathlineto{\pgfqpoint{1.267696in}{1.505029in}}%
\pgfpathlineto{\pgfqpoint{1.268765in}{1.505344in}}%
\pgfpathlineto{\pgfqpoint{1.274248in}{1.506447in}}%
\pgfpathlineto{\pgfqpoint{1.274918in}{1.506552in}}%
\pgfpathlineto{\pgfqpoint{1.282128in}{1.507654in}}%
\pgfpathlineto{\pgfqpoint{1.283008in}{1.508127in}}%
\pgfpathlineto{\pgfqpoint{1.289420in}{1.509229in}}%
\pgfpathlineto{\pgfqpoint{1.290465in}{1.509649in}}%
\pgfpathlineto{\pgfqpoint{1.296723in}{1.510752in}}%
\pgfpathlineto{\pgfqpoint{1.297722in}{1.511014in}}%
\pgfpathlineto{\pgfqpoint{1.303393in}{1.512117in}}%
\pgfpathlineto{\pgfqpoint{1.304485in}{1.512379in}}%
\pgfpathlineto{\pgfqpoint{1.311084in}{1.513429in}}%
\pgfpathlineto{\pgfqpoint{1.311460in}{1.513587in}}%
\pgfpathlineto{\pgfqpoint{1.319445in}{1.514689in}}%
\pgfpathlineto{\pgfqpoint{1.320220in}{1.515109in}}%
\pgfpathlineto{\pgfqpoint{1.326338in}{1.516212in}}%
\pgfpathlineto{\pgfqpoint{1.327301in}{1.516474in}}%
\pgfpathlineto{\pgfqpoint{1.334041in}{1.517577in}}%
\pgfpathlineto{\pgfqpoint{1.334886in}{1.517839in}}%
\pgfpathlineto{\pgfqpoint{1.340664in}{1.518942in}}%
\pgfpathlineto{\pgfqpoint{1.340664in}{1.518994in}}%
\pgfpathlineto{\pgfqpoint{1.350304in}{1.520097in}}%
\pgfpathlineto{\pgfqpoint{1.351279in}{1.520464in}}%
\pgfpathlineto{\pgfqpoint{1.358007in}{1.521567in}}%
\pgfpathlineto{\pgfqpoint{1.358888in}{1.521724in}}%
\pgfpathlineto{\pgfqpoint{1.365264in}{1.522827in}}%
\pgfpathlineto{\pgfqpoint{1.366098in}{1.523037in}}%
\pgfpathlineto{\pgfqpoint{1.372697in}{1.524139in}}%
\pgfpathlineto{\pgfqpoint{1.373413in}{1.524349in}}%
\pgfpathlineto{\pgfqpoint{1.380964in}{1.525452in}}%
\pgfpathlineto{\pgfqpoint{1.382044in}{1.525977in}}%
\pgfpathlineto{\pgfqpoint{1.387222in}{1.527079in}}%
\pgfpathlineto{\pgfqpoint{1.387974in}{1.527237in}}%
\pgfpathlineto{\pgfqpoint{1.394338in}{1.528339in}}%
\pgfpathlineto{\pgfqpoint{1.395442in}{1.528654in}}%
\pgfpathlineto{\pgfqpoint{1.400174in}{1.529757in}}%
\pgfpathlineto{\pgfqpoint{1.400527in}{1.529914in}}%
\pgfpathlineto{\pgfqpoint{1.408605in}{1.531017in}}%
\pgfpathlineto{\pgfqpoint{1.409615in}{1.531227in}}%
\pgfpathlineto{\pgfqpoint{1.417201in}{1.532329in}}%
\pgfpathlineto{\pgfqpoint{1.418258in}{1.532539in}}%
\pgfpathlineto{\pgfqpoint{1.423964in}{1.533642in}}%
\pgfpathlineto{\pgfqpoint{1.424822in}{1.533852in}}%
\pgfpathlineto{\pgfqpoint{1.432008in}{1.534954in}}%
\pgfpathlineto{\pgfqpoint{1.432266in}{1.535112in}}%
\pgfpathlineto{\pgfqpoint{1.442353in}{1.536214in}}%
\pgfpathlineto{\pgfqpoint{1.442564in}{1.536319in}}%
\pgfpathlineto{\pgfqpoint{1.449504in}{1.537422in}}%
\pgfpathlineto{\pgfqpoint{1.450267in}{1.537842in}}%
\pgfpathlineto{\pgfqpoint{1.458534in}{1.538944in}}%
\pgfpathlineto{\pgfqpoint{1.459356in}{1.539102in}}%
\pgfpathlineto{\pgfqpoint{1.473083in}{1.540204in}}%
\pgfpathlineto{\pgfqpoint{1.473306in}{1.540362in}}%
\pgfpathlineto{\pgfqpoint{1.482829in}{1.541464in}}%
\pgfpathlineto{\pgfqpoint{1.483710in}{1.541569in}}%
\pgfpathlineto{\pgfqpoint{1.495405in}{1.542672in}}%
\pgfpathlineto{\pgfqpoint{1.496157in}{1.542934in}}%
\pgfpathlineto{\pgfqpoint{1.504470in}{1.544037in}}%
\pgfpathlineto{\pgfqpoint{1.505410in}{1.544299in}}%
\pgfpathlineto{\pgfqpoint{1.515121in}{1.545402in}}%
\pgfpathlineto{\pgfqpoint{1.516213in}{1.545559in}}%
\pgfpathlineto{\pgfqpoint{1.523775in}{1.546662in}}%
\pgfpathlineto{\pgfqpoint{1.524515in}{1.546819in}}%
\pgfpathlineto{\pgfqpoint{1.532006in}{1.547922in}}%
\pgfpathlineto{\pgfqpoint{1.532958in}{1.548079in}}%
\pgfpathlineto{\pgfqpoint{1.541835in}{1.549182in}}%
\pgfpathlineto{\pgfqpoint{1.542492in}{1.549287in}}%
\pgfpathlineto{\pgfqpoint{1.548258in}{1.550389in}}%
\pgfpathlineto{\pgfqpoint{1.549021in}{1.550652in}}%
\pgfpathlineto{\pgfqpoint{1.558955in}{1.551754in}}%
\pgfpathlineto{\pgfqpoint{1.559859in}{1.552069in}}%
\pgfpathlineto{\pgfqpoint{1.571132in}{1.553172in}}%
\pgfpathlineto{\pgfqpoint{1.571895in}{1.553329in}}%
\pgfpathlineto{\pgfqpoint{1.581783in}{1.554432in}}%
\pgfpathlineto{\pgfqpoint{1.582652in}{1.554589in}}%
\pgfpathlineto{\pgfqpoint{1.594253in}{1.555692in}}%
\pgfpathlineto{\pgfqpoint{1.595322in}{1.555902in}}%
\pgfpathlineto{\pgfqpoint{1.605314in}{1.557004in}}%
\pgfpathlineto{\pgfqpoint{1.605890in}{1.557162in}}%
\pgfpathlineto{\pgfqpoint{1.619534in}{1.558264in}}%
\pgfpathlineto{\pgfqpoint{1.620333in}{1.558474in}}%
\pgfpathlineto{\pgfqpoint{1.630772in}{1.559577in}}%
\pgfpathlineto{\pgfqpoint{1.630819in}{1.559682in}}%
\pgfpathlineto{\pgfqpoint{1.642738in}{1.560784in}}%
\pgfpathlineto{\pgfqpoint{1.643677in}{1.561047in}}%
\pgfpathlineto{\pgfqpoint{1.656793in}{1.562097in}}%
\pgfpathlineto{\pgfqpoint{1.657592in}{1.562307in}}%
\pgfpathlineto{\pgfqpoint{1.669041in}{1.563409in}}%
\pgfpathlineto{\pgfqpoint{1.669698in}{1.563619in}}%
\pgfpathlineto{\pgfqpoint{1.683989in}{1.564722in}}%
\pgfpathlineto{\pgfqpoint{1.685010in}{1.564879in}}%
\pgfpathlineto{\pgfqpoint{1.700311in}{1.565982in}}%
\pgfpathlineto{\pgfqpoint{1.700311in}{1.566034in}}%
\pgfpathlineto{\pgfqpoint{1.717666in}{1.567137in}}%
\pgfpathlineto{\pgfqpoint{1.718006in}{1.567294in}}%
\pgfpathlineto{\pgfqpoint{1.730125in}{1.568397in}}%
\pgfpathlineto{\pgfqpoint{1.730747in}{1.568712in}}%
\pgfpathlineto{\pgfqpoint{1.744298in}{1.569814in}}%
\pgfpathlineto{\pgfqpoint{1.745155in}{1.569972in}}%
\pgfpathlineto{\pgfqpoint{1.760138in}{1.571074in}}%
\pgfpathlineto{\pgfqpoint{1.760913in}{1.571389in}}%
\pgfpathlineto{\pgfqpoint{1.773677in}{1.572492in}}%
\pgfpathlineto{\pgfqpoint{1.774734in}{1.572754in}}%
\pgfpathlineto{\pgfqpoint{1.792406in}{1.573857in}}%
\pgfpathlineto{\pgfqpoint{1.793240in}{1.574067in}}%
\pgfpathlineto{\pgfqpoint{1.804043in}{1.575169in}}%
\pgfpathlineto{\pgfqpoint{1.804278in}{1.575274in}}%
\pgfpathlineto{\pgfqpoint{1.819355in}{1.576377in}}%
\pgfpathlineto{\pgfqpoint{1.819813in}{1.576534in}}%
\pgfpathlineto{\pgfqpoint{1.832014in}{1.577637in}}%
\pgfpathlineto{\pgfqpoint{1.833094in}{1.577794in}}%
\pgfpathlineto{\pgfqpoint{1.850508in}{1.578897in}}%
\pgfpathlineto{\pgfqpoint{1.850966in}{1.579002in}}%
\pgfpathlineto{\pgfqpoint{1.865315in}{1.580104in}}%
\pgfpathlineto{\pgfqpoint{1.866219in}{1.580367in}}%
\pgfpathlineto{\pgfqpoint{1.878279in}{1.581469in}}%
\pgfpathlineto{\pgfqpoint{1.878948in}{1.581574in}}%
\pgfpathlineto{\pgfqpoint{1.891454in}{1.582677in}}%
\pgfpathlineto{\pgfqpoint{1.892147in}{1.582887in}}%
\pgfpathlineto{\pgfqpoint{1.905075in}{1.583989in}}%
\pgfpathlineto{\pgfqpoint{1.906155in}{1.584199in}}%
\pgfpathlineto{\pgfqpoint{1.919694in}{1.585302in}}%
\pgfpathlineto{\pgfqpoint{1.920704in}{1.585512in}}%
\pgfpathlineto{\pgfqpoint{1.935558in}{1.586614in}}%
\pgfpathlineto{\pgfqpoint{1.936392in}{1.586772in}}%
\pgfpathlineto{\pgfqpoint{1.948369in}{1.587874in}}%
\pgfpathlineto{\pgfqpoint{1.949379in}{1.588242in}}%
\pgfpathlineto{\pgfqpoint{1.964691in}{1.589344in}}%
\pgfpathlineto{\pgfqpoint{1.965008in}{1.589554in}}%
\pgfpathlineto{\pgfqpoint{1.978606in}{1.590657in}}%
\pgfpathlineto{\pgfqpoint{1.979675in}{1.590867in}}%
\pgfpathlineto{\pgfqpoint{1.990513in}{1.591969in}}%
\pgfpathlineto{\pgfqpoint{1.990513in}{1.592022in}}%
\pgfpathlineto{\pgfqpoint{2.000118in}{1.593124in}}%
\pgfpathlineto{\pgfqpoint{2.001210in}{1.593387in}}%
\pgfpathlineto{\pgfqpoint{2.010581in}{1.594489in}}%
\pgfpathlineto{\pgfqpoint{2.011320in}{1.594699in}}%
\pgfpathlineto{\pgfqpoint{2.013739in}{1.594804in}}%
\pgfpathlineto{\pgfqpoint{2.033126in}{1.601944in}}%
\pgfpathlineto{\pgfqpoint{2.033126in}{1.601944in}}%
\pgfusepath{stroke}%
\end{pgfscope}%
\begin{pgfscope}%
\pgfsetrectcap%
\pgfsetmiterjoin%
\pgfsetlinewidth{0.803000pt}%
\definecolor{currentstroke}{rgb}{0.000000,0.000000,0.000000}%
\pgfsetstrokecolor{currentstroke}%
\pgfsetdash{}{0pt}%
\pgfpathmoveto{\pgfqpoint{0.553581in}{0.499444in}}%
\pgfpathlineto{\pgfqpoint{0.553581in}{1.654444in}}%
\pgfusepath{stroke}%
\end{pgfscope}%
\begin{pgfscope}%
\pgfsetrectcap%
\pgfsetmiterjoin%
\pgfsetlinewidth{0.803000pt}%
\definecolor{currentstroke}{rgb}{0.000000,0.000000,0.000000}%
\pgfsetstrokecolor{currentstroke}%
\pgfsetdash{}{0pt}%
\pgfpathmoveto{\pgfqpoint{2.103581in}{0.499444in}}%
\pgfpathlineto{\pgfqpoint{2.103581in}{1.654444in}}%
\pgfusepath{stroke}%
\end{pgfscope}%
\begin{pgfscope}%
\pgfsetrectcap%
\pgfsetmiterjoin%
\pgfsetlinewidth{0.803000pt}%
\definecolor{currentstroke}{rgb}{0.000000,0.000000,0.000000}%
\pgfsetstrokecolor{currentstroke}%
\pgfsetdash{}{0pt}%
\pgfpathmoveto{\pgfqpoint{0.553581in}{0.499444in}}%
\pgfpathlineto{\pgfqpoint{2.103581in}{0.499444in}}%
\pgfusepath{stroke}%
\end{pgfscope}%
\begin{pgfscope}%
\pgfsetrectcap%
\pgfsetmiterjoin%
\pgfsetlinewidth{0.803000pt}%
\definecolor{currentstroke}{rgb}{0.000000,0.000000,0.000000}%
\pgfsetstrokecolor{currentstroke}%
\pgfsetdash{}{0pt}%
\pgfpathmoveto{\pgfqpoint{0.553581in}{1.654444in}}%
\pgfpathlineto{\pgfqpoint{2.103581in}{1.654444in}}%
\pgfusepath{stroke}%
\end{pgfscope}%
\begin{pgfscope}%
\definecolor{textcolor}{rgb}{0.000000,0.000000,0.000000}%
\pgfsetstrokecolor{textcolor}%
\pgfsetfillcolor{textcolor}%
\pgftext[x=1.328581in,y=1.737778in,,base]{\color{textcolor}\rmfamily\fontsize{12.000000}{14.400000}\selectfont ROC Curve}%
\end{pgfscope}%
\begin{pgfscope}%
\pgfsetbuttcap%
\pgfsetmiterjoin%
\definecolor{currentfill}{rgb}{1.000000,1.000000,1.000000}%
\pgfsetfillcolor{currentfill}%
\pgfsetfillopacity{0.800000}%
\pgfsetlinewidth{1.003750pt}%
\definecolor{currentstroke}{rgb}{0.800000,0.800000,0.800000}%
\pgfsetstrokecolor{currentstroke}%
\pgfsetstrokeopacity{0.800000}%
\pgfsetdash{}{0pt}%
\pgfpathmoveto{\pgfqpoint{0.840525in}{0.568889in}}%
\pgfpathlineto{\pgfqpoint{2.006358in}{0.568889in}}%
\pgfpathquadraticcurveto{\pgfqpoint{2.034136in}{0.568889in}}{\pgfqpoint{2.034136in}{0.596666in}}%
\pgfpathlineto{\pgfqpoint{2.034136in}{0.791111in}}%
\pgfpathquadraticcurveto{\pgfqpoint{2.034136in}{0.818888in}}{\pgfqpoint{2.006358in}{0.818888in}}%
\pgfpathlineto{\pgfqpoint{0.840525in}{0.818888in}}%
\pgfpathquadraticcurveto{\pgfqpoint{0.812747in}{0.818888in}}{\pgfqpoint{0.812747in}{0.791111in}}%
\pgfpathlineto{\pgfqpoint{0.812747in}{0.596666in}}%
\pgfpathquadraticcurveto{\pgfqpoint{0.812747in}{0.568889in}}{\pgfqpoint{0.840525in}{0.568889in}}%
\pgfpathlineto{\pgfqpoint{0.840525in}{0.568889in}}%
\pgfpathclose%
\pgfusepath{stroke,fill}%
\end{pgfscope}%
\begin{pgfscope}%
\pgfsetrectcap%
\pgfsetroundjoin%
\pgfsetlinewidth{1.505625pt}%
\definecolor{currentstroke}{rgb}{0.000000,0.000000,0.000000}%
\pgfsetstrokecolor{currentstroke}%
\pgfsetdash{}{0pt}%
\pgfpathmoveto{\pgfqpoint{0.868303in}{0.707777in}}%
\pgfpathlineto{\pgfqpoint{1.007192in}{0.707777in}}%
\pgfpathlineto{\pgfqpoint{1.146081in}{0.707777in}}%
\pgfusepath{stroke}%
\end{pgfscope}%
\begin{pgfscope}%
\definecolor{textcolor}{rgb}{0.000000,0.000000,0.000000}%
\pgfsetstrokecolor{textcolor}%
\pgfsetfillcolor{textcolor}%
\pgftext[x=1.257192in,y=0.659166in,left,base]{\color{textcolor}\rmfamily\fontsize{10.000000}{12.000000}\selectfont AUC 0.838)}%
\end{pgfscope}%
\end{pgfpicture}%
\makeatother%
\endgroup%

\end{tabular}
\end{center}

\begin{center}
\begin{tabular}{cc}
\begin{tabular}{cc|c|c|}
	&\multicolumn{1}{c}{}& \multicolumn{2}{c}{Prediction} \cr
	&\multicolumn{1}{c}{} & \multicolumn{1}{c}{N} & \multicolumn{1}{c}{P} \cr\cline{3-4}
	\multirow{2}{*}{Actual}&N & 67.5\% & 18.2\% \vrule width 0pt height 10pt depth 2pt \cr\cline{3-4}
	&P & 3.11\% & 11.2\% \vrule width 0pt height 10pt depth 2pt \cr\cline{3-4}
\end{tabular}
&
\begin{tabular}{ll}
0.787 & Accuracy \cr 
0.785 & Balanced Accuracy \cr 
0.380 & Precision \cr 
0.786 & Balanced Precision \cr 
0.782 & Recall \cr 
0.512 & F1 \cr 
0.784 & Balanced F1 \cr 
0.547 & Gmean \cr 
	\end{tabular}
\end{tabular}
\end{center}




%%%%%
In the ideal results above, the algorithm learned a useful model from the patterns in the data.  The results below illustrate the worst case scenario, where the algorithm does not learn a good model, usually because the data does not have a pattern that predicts the target variable.  In the ROC curve, the median values of the probabilities for the two classes are so close that the labels are on top of each other.  

\begin{center}
\begin{tabular}{p{0.5\textwidth} p{0.5\textwidth}}
  \vspace{0pt} %% Creator: Matplotlib, PGF backend
%%
%% To include the figure in your LaTeX document, write
%%   \input{<filename>.pgf}
%%
%% Make sure the required packages are loaded in your preamble
%%   \usepackage{pgf}
%%
%% Also ensure that all the required font packages are loaded; for instance,
%% the lmodern package is sometimes necessary when using math font.
%%   \usepackage{lmodern}
%%
%% Figures using additional raster images can only be included by \input if
%% they are in the same directory as the main LaTeX file. For loading figures
%% from other directories you can use the `import` package
%%   \usepackage{import}
%%
%% and then include the figures with
%%   \import{<path to file>}{<filename>.pgf}
%%
%% Matplotlib used the following preamble
%%   
%%   \usepackage{fontspec}
%%   \makeatletter\@ifpackageloaded{underscore}{}{\usepackage[strings]{underscore}}\makeatother
%%
\begingroup%
\makeatletter%
\begin{pgfpicture}%
\pgfpathrectangle{\pgfpointorigin}{\pgfqpoint{2.253750in}{1.754444in}}%
\pgfusepath{use as bounding box, clip}%
\begin{pgfscope}%
\pgfsetbuttcap%
\pgfsetmiterjoin%
\definecolor{currentfill}{rgb}{1.000000,1.000000,1.000000}%
\pgfsetfillcolor{currentfill}%
\pgfsetlinewidth{0.000000pt}%
\definecolor{currentstroke}{rgb}{1.000000,1.000000,1.000000}%
\pgfsetstrokecolor{currentstroke}%
\pgfsetdash{}{0pt}%
\pgfpathmoveto{\pgfqpoint{0.000000in}{0.000000in}}%
\pgfpathlineto{\pgfqpoint{2.253750in}{0.000000in}}%
\pgfpathlineto{\pgfqpoint{2.253750in}{1.754444in}}%
\pgfpathlineto{\pgfqpoint{0.000000in}{1.754444in}}%
\pgfpathlineto{\pgfqpoint{0.000000in}{0.000000in}}%
\pgfpathclose%
\pgfusepath{fill}%
\end{pgfscope}%
\begin{pgfscope}%
\pgfsetbuttcap%
\pgfsetmiterjoin%
\definecolor{currentfill}{rgb}{1.000000,1.000000,1.000000}%
\pgfsetfillcolor{currentfill}%
\pgfsetlinewidth{0.000000pt}%
\definecolor{currentstroke}{rgb}{0.000000,0.000000,0.000000}%
\pgfsetstrokecolor{currentstroke}%
\pgfsetstrokeopacity{0.000000}%
\pgfsetdash{}{0pt}%
\pgfpathmoveto{\pgfqpoint{0.515000in}{0.499444in}}%
\pgfpathlineto{\pgfqpoint{2.065000in}{0.499444in}}%
\pgfpathlineto{\pgfqpoint{2.065000in}{1.654444in}}%
\pgfpathlineto{\pgfqpoint{0.515000in}{1.654444in}}%
\pgfpathlineto{\pgfqpoint{0.515000in}{0.499444in}}%
\pgfpathclose%
\pgfusepath{fill}%
\end{pgfscope}%
\begin{pgfscope}%
\pgfpathrectangle{\pgfqpoint{0.515000in}{0.499444in}}{\pgfqpoint{1.550000in}{1.155000in}}%
\pgfusepath{clip}%
\pgfsetbuttcap%
\pgfsetmiterjoin%
\pgfsetlinewidth{1.003750pt}%
\definecolor{currentstroke}{rgb}{0.000000,0.000000,0.000000}%
\pgfsetstrokecolor{currentstroke}%
\pgfsetdash{}{0pt}%
\pgfpathmoveto{\pgfqpoint{0.505000in}{0.499444in}}%
\pgfpathlineto{\pgfqpoint{0.552805in}{0.499444in}}%
\pgfpathlineto{\pgfqpoint{0.552805in}{1.599444in}}%
\pgfpathlineto{\pgfqpoint{0.505000in}{1.599444in}}%
\pgfusepath{stroke}%
\end{pgfscope}%
\begin{pgfscope}%
\pgfpathrectangle{\pgfqpoint{0.515000in}{0.499444in}}{\pgfqpoint{1.550000in}{1.155000in}}%
\pgfusepath{clip}%
\pgfsetbuttcap%
\pgfsetmiterjoin%
\pgfsetlinewidth{1.003750pt}%
\definecolor{currentstroke}{rgb}{0.000000,0.000000,0.000000}%
\pgfsetstrokecolor{currentstroke}%
\pgfsetdash{}{0pt}%
\pgfpathmoveto{\pgfqpoint{0.643537in}{0.499444in}}%
\pgfpathlineto{\pgfqpoint{0.704025in}{0.499444in}}%
\pgfpathlineto{\pgfqpoint{0.704025in}{1.461944in}}%
\pgfpathlineto{\pgfqpoint{0.643537in}{1.461944in}}%
\pgfpathlineto{\pgfqpoint{0.643537in}{0.499444in}}%
\pgfpathclose%
\pgfusepath{stroke}%
\end{pgfscope}%
\begin{pgfscope}%
\pgfpathrectangle{\pgfqpoint{0.515000in}{0.499444in}}{\pgfqpoint{1.550000in}{1.155000in}}%
\pgfusepath{clip}%
\pgfsetbuttcap%
\pgfsetmiterjoin%
\pgfsetlinewidth{1.003750pt}%
\definecolor{currentstroke}{rgb}{0.000000,0.000000,0.000000}%
\pgfsetstrokecolor{currentstroke}%
\pgfsetdash{}{0pt}%
\pgfpathmoveto{\pgfqpoint{0.794756in}{0.499444in}}%
\pgfpathlineto{\pgfqpoint{0.855244in}{0.499444in}}%
\pgfpathlineto{\pgfqpoint{0.855244in}{1.202222in}}%
\pgfpathlineto{\pgfqpoint{0.794756in}{1.202222in}}%
\pgfpathlineto{\pgfqpoint{0.794756in}{0.499444in}}%
\pgfpathclose%
\pgfusepath{stroke}%
\end{pgfscope}%
\begin{pgfscope}%
\pgfpathrectangle{\pgfqpoint{0.515000in}{0.499444in}}{\pgfqpoint{1.550000in}{1.155000in}}%
\pgfusepath{clip}%
\pgfsetbuttcap%
\pgfsetmiterjoin%
\pgfsetlinewidth{1.003750pt}%
\definecolor{currentstroke}{rgb}{0.000000,0.000000,0.000000}%
\pgfsetstrokecolor{currentstroke}%
\pgfsetdash{}{0pt}%
\pgfpathmoveto{\pgfqpoint{0.945976in}{0.499444in}}%
\pgfpathlineto{\pgfqpoint{1.006464in}{0.499444in}}%
\pgfpathlineto{\pgfqpoint{1.006464in}{1.461944in}}%
\pgfpathlineto{\pgfqpoint{0.945976in}{1.461944in}}%
\pgfpathlineto{\pgfqpoint{0.945976in}{0.499444in}}%
\pgfpathclose%
\pgfusepath{stroke}%
\end{pgfscope}%
\begin{pgfscope}%
\pgfpathrectangle{\pgfqpoint{0.515000in}{0.499444in}}{\pgfqpoint{1.550000in}{1.155000in}}%
\pgfusepath{clip}%
\pgfsetbuttcap%
\pgfsetmiterjoin%
\pgfsetlinewidth{1.003750pt}%
\definecolor{currentstroke}{rgb}{0.000000,0.000000,0.000000}%
\pgfsetstrokecolor{currentstroke}%
\pgfsetdash{}{0pt}%
\pgfpathmoveto{\pgfqpoint{1.097195in}{0.499444in}}%
\pgfpathlineto{\pgfqpoint{1.157683in}{0.499444in}}%
\pgfpathlineto{\pgfqpoint{1.157683in}{1.171666in}}%
\pgfpathlineto{\pgfqpoint{1.097195in}{1.171666in}}%
\pgfpathlineto{\pgfqpoint{1.097195in}{0.499444in}}%
\pgfpathclose%
\pgfusepath{stroke}%
\end{pgfscope}%
\begin{pgfscope}%
\pgfpathrectangle{\pgfqpoint{0.515000in}{0.499444in}}{\pgfqpoint{1.550000in}{1.155000in}}%
\pgfusepath{clip}%
\pgfsetbuttcap%
\pgfsetmiterjoin%
\pgfsetlinewidth{1.003750pt}%
\definecolor{currentstroke}{rgb}{0.000000,0.000000,0.000000}%
\pgfsetstrokecolor{currentstroke}%
\pgfsetdash{}{0pt}%
\pgfpathmoveto{\pgfqpoint{1.248415in}{0.499444in}}%
\pgfpathlineto{\pgfqpoint{1.308903in}{0.499444in}}%
\pgfpathlineto{\pgfqpoint{1.308903in}{1.431389in}}%
\pgfpathlineto{\pgfqpoint{1.248415in}{1.431389in}}%
\pgfpathlineto{\pgfqpoint{1.248415in}{0.499444in}}%
\pgfpathclose%
\pgfusepath{stroke}%
\end{pgfscope}%
\begin{pgfscope}%
\pgfpathrectangle{\pgfqpoint{0.515000in}{0.499444in}}{\pgfqpoint{1.550000in}{1.155000in}}%
\pgfusepath{clip}%
\pgfsetbuttcap%
\pgfsetmiterjoin%
\pgfsetlinewidth{1.003750pt}%
\definecolor{currentstroke}{rgb}{0.000000,0.000000,0.000000}%
\pgfsetstrokecolor{currentstroke}%
\pgfsetdash{}{0pt}%
\pgfpathmoveto{\pgfqpoint{1.399634in}{0.499444in}}%
\pgfpathlineto{\pgfqpoint{1.460122in}{0.499444in}}%
\pgfpathlineto{\pgfqpoint{1.460122in}{1.492500in}}%
\pgfpathlineto{\pgfqpoint{1.399634in}{1.492500in}}%
\pgfpathlineto{\pgfqpoint{1.399634in}{0.499444in}}%
\pgfpathclose%
\pgfusepath{stroke}%
\end{pgfscope}%
\begin{pgfscope}%
\pgfpathrectangle{\pgfqpoint{0.515000in}{0.499444in}}{\pgfqpoint{1.550000in}{1.155000in}}%
\pgfusepath{clip}%
\pgfsetbuttcap%
\pgfsetmiterjoin%
\pgfsetlinewidth{1.003750pt}%
\definecolor{currentstroke}{rgb}{0.000000,0.000000,0.000000}%
\pgfsetstrokecolor{currentstroke}%
\pgfsetdash{}{0pt}%
\pgfpathmoveto{\pgfqpoint{1.550854in}{0.499444in}}%
\pgfpathlineto{\pgfqpoint{1.611342in}{0.499444in}}%
\pgfpathlineto{\pgfqpoint{1.611342in}{1.324444in}}%
\pgfpathlineto{\pgfqpoint{1.550854in}{1.324444in}}%
\pgfpathlineto{\pgfqpoint{1.550854in}{0.499444in}}%
\pgfpathclose%
\pgfusepath{stroke}%
\end{pgfscope}%
\begin{pgfscope}%
\pgfpathrectangle{\pgfqpoint{0.515000in}{0.499444in}}{\pgfqpoint{1.550000in}{1.155000in}}%
\pgfusepath{clip}%
\pgfsetbuttcap%
\pgfsetmiterjoin%
\pgfsetlinewidth{1.003750pt}%
\definecolor{currentstroke}{rgb}{0.000000,0.000000,0.000000}%
\pgfsetstrokecolor{currentstroke}%
\pgfsetdash{}{0pt}%
\pgfpathmoveto{\pgfqpoint{1.702073in}{0.499444in}}%
\pgfpathlineto{\pgfqpoint{1.762561in}{0.499444in}}%
\pgfpathlineto{\pgfqpoint{1.762561in}{1.477222in}}%
\pgfpathlineto{\pgfqpoint{1.702073in}{1.477222in}}%
\pgfpathlineto{\pgfqpoint{1.702073in}{0.499444in}}%
\pgfpathclose%
\pgfusepath{stroke}%
\end{pgfscope}%
\begin{pgfscope}%
\pgfpathrectangle{\pgfqpoint{0.515000in}{0.499444in}}{\pgfqpoint{1.550000in}{1.155000in}}%
\pgfusepath{clip}%
\pgfsetbuttcap%
\pgfsetmiterjoin%
\pgfsetlinewidth{1.003750pt}%
\definecolor{currentstroke}{rgb}{0.000000,0.000000,0.000000}%
\pgfsetstrokecolor{currentstroke}%
\pgfsetdash{}{0pt}%
\pgfpathmoveto{\pgfqpoint{1.853293in}{0.499444in}}%
\pgfpathlineto{\pgfqpoint{1.913781in}{0.499444in}}%
\pgfpathlineto{\pgfqpoint{1.913781in}{1.538333in}}%
\pgfpathlineto{\pgfqpoint{1.853293in}{1.538333in}}%
\pgfpathlineto{\pgfqpoint{1.853293in}{0.499444in}}%
\pgfpathclose%
\pgfusepath{stroke}%
\end{pgfscope}%
\begin{pgfscope}%
\pgfpathrectangle{\pgfqpoint{0.515000in}{0.499444in}}{\pgfqpoint{1.550000in}{1.155000in}}%
\pgfusepath{clip}%
\pgfsetbuttcap%
\pgfsetmiterjoin%
\definecolor{currentfill}{rgb}{0.000000,0.000000,0.000000}%
\pgfsetfillcolor{currentfill}%
\pgfsetlinewidth{0.000000pt}%
\definecolor{currentstroke}{rgb}{0.000000,0.000000,0.000000}%
\pgfsetstrokecolor{currentstroke}%
\pgfsetstrokeopacity{0.000000}%
\pgfsetdash{}{0pt}%
\pgfpathmoveto{\pgfqpoint{0.552805in}{0.499444in}}%
\pgfpathlineto{\pgfqpoint{0.613293in}{0.499444in}}%
\pgfpathlineto{\pgfqpoint{0.613293in}{0.575833in}}%
\pgfpathlineto{\pgfqpoint{0.552805in}{0.575833in}}%
\pgfpathlineto{\pgfqpoint{0.552805in}{0.499444in}}%
\pgfpathclose%
\pgfusepath{fill}%
\end{pgfscope}%
\begin{pgfscope}%
\pgfpathrectangle{\pgfqpoint{0.515000in}{0.499444in}}{\pgfqpoint{1.550000in}{1.155000in}}%
\pgfusepath{clip}%
\pgfsetbuttcap%
\pgfsetmiterjoin%
\definecolor{currentfill}{rgb}{0.000000,0.000000,0.000000}%
\pgfsetfillcolor{currentfill}%
\pgfsetlinewidth{0.000000pt}%
\definecolor{currentstroke}{rgb}{0.000000,0.000000,0.000000}%
\pgfsetstrokecolor{currentstroke}%
\pgfsetstrokeopacity{0.000000}%
\pgfsetdash{}{0pt}%
\pgfpathmoveto{\pgfqpoint{0.704025in}{0.499444in}}%
\pgfpathlineto{\pgfqpoint{0.764512in}{0.499444in}}%
\pgfpathlineto{\pgfqpoint{0.764512in}{0.606389in}}%
\pgfpathlineto{\pgfqpoint{0.704025in}{0.606389in}}%
\pgfpathlineto{\pgfqpoint{0.704025in}{0.499444in}}%
\pgfpathclose%
\pgfusepath{fill}%
\end{pgfscope}%
\begin{pgfscope}%
\pgfpathrectangle{\pgfqpoint{0.515000in}{0.499444in}}{\pgfqpoint{1.550000in}{1.155000in}}%
\pgfusepath{clip}%
\pgfsetbuttcap%
\pgfsetmiterjoin%
\definecolor{currentfill}{rgb}{0.000000,0.000000,0.000000}%
\pgfsetfillcolor{currentfill}%
\pgfsetlinewidth{0.000000pt}%
\definecolor{currentstroke}{rgb}{0.000000,0.000000,0.000000}%
\pgfsetstrokecolor{currentstroke}%
\pgfsetstrokeopacity{0.000000}%
\pgfsetdash{}{0pt}%
\pgfpathmoveto{\pgfqpoint{0.855244in}{0.499444in}}%
\pgfpathlineto{\pgfqpoint{0.915732in}{0.499444in}}%
\pgfpathlineto{\pgfqpoint{0.915732in}{0.606389in}}%
\pgfpathlineto{\pgfqpoint{0.855244in}{0.606389in}}%
\pgfpathlineto{\pgfqpoint{0.855244in}{0.499444in}}%
\pgfpathclose%
\pgfusepath{fill}%
\end{pgfscope}%
\begin{pgfscope}%
\pgfpathrectangle{\pgfqpoint{0.515000in}{0.499444in}}{\pgfqpoint{1.550000in}{1.155000in}}%
\pgfusepath{clip}%
\pgfsetbuttcap%
\pgfsetmiterjoin%
\definecolor{currentfill}{rgb}{0.000000,0.000000,0.000000}%
\pgfsetfillcolor{currentfill}%
\pgfsetlinewidth{0.000000pt}%
\definecolor{currentstroke}{rgb}{0.000000,0.000000,0.000000}%
\pgfsetstrokecolor{currentstroke}%
\pgfsetstrokeopacity{0.000000}%
\pgfsetdash{}{0pt}%
\pgfpathmoveto{\pgfqpoint{1.006464in}{0.499444in}}%
\pgfpathlineto{\pgfqpoint{1.066951in}{0.499444in}}%
\pgfpathlineto{\pgfqpoint{1.066951in}{0.621666in}}%
\pgfpathlineto{\pgfqpoint{1.006464in}{0.621666in}}%
\pgfpathlineto{\pgfqpoint{1.006464in}{0.499444in}}%
\pgfpathclose%
\pgfusepath{fill}%
\end{pgfscope}%
\begin{pgfscope}%
\pgfpathrectangle{\pgfqpoint{0.515000in}{0.499444in}}{\pgfqpoint{1.550000in}{1.155000in}}%
\pgfusepath{clip}%
\pgfsetbuttcap%
\pgfsetmiterjoin%
\definecolor{currentfill}{rgb}{0.000000,0.000000,0.000000}%
\pgfsetfillcolor{currentfill}%
\pgfsetlinewidth{0.000000pt}%
\definecolor{currentstroke}{rgb}{0.000000,0.000000,0.000000}%
\pgfsetstrokecolor{currentstroke}%
\pgfsetstrokeopacity{0.000000}%
\pgfsetdash{}{0pt}%
\pgfpathmoveto{\pgfqpoint{1.157683in}{0.499444in}}%
\pgfpathlineto{\pgfqpoint{1.218171in}{0.499444in}}%
\pgfpathlineto{\pgfqpoint{1.218171in}{0.698055in}}%
\pgfpathlineto{\pgfqpoint{1.157683in}{0.698055in}}%
\pgfpathlineto{\pgfqpoint{1.157683in}{0.499444in}}%
\pgfpathclose%
\pgfusepath{fill}%
\end{pgfscope}%
\begin{pgfscope}%
\pgfpathrectangle{\pgfqpoint{0.515000in}{0.499444in}}{\pgfqpoint{1.550000in}{1.155000in}}%
\pgfusepath{clip}%
\pgfsetbuttcap%
\pgfsetmiterjoin%
\definecolor{currentfill}{rgb}{0.000000,0.000000,0.000000}%
\pgfsetfillcolor{currentfill}%
\pgfsetlinewidth{0.000000pt}%
\definecolor{currentstroke}{rgb}{0.000000,0.000000,0.000000}%
\pgfsetstrokecolor{currentstroke}%
\pgfsetstrokeopacity{0.000000}%
\pgfsetdash{}{0pt}%
\pgfpathmoveto{\pgfqpoint{1.308903in}{0.499444in}}%
\pgfpathlineto{\pgfqpoint{1.369391in}{0.499444in}}%
\pgfpathlineto{\pgfqpoint{1.369391in}{0.636944in}}%
\pgfpathlineto{\pgfqpoint{1.308903in}{0.636944in}}%
\pgfpathlineto{\pgfqpoint{1.308903in}{0.499444in}}%
\pgfpathclose%
\pgfusepath{fill}%
\end{pgfscope}%
\begin{pgfscope}%
\pgfpathrectangle{\pgfqpoint{0.515000in}{0.499444in}}{\pgfqpoint{1.550000in}{1.155000in}}%
\pgfusepath{clip}%
\pgfsetbuttcap%
\pgfsetmiterjoin%
\definecolor{currentfill}{rgb}{0.000000,0.000000,0.000000}%
\pgfsetfillcolor{currentfill}%
\pgfsetlinewidth{0.000000pt}%
\definecolor{currentstroke}{rgb}{0.000000,0.000000,0.000000}%
\pgfsetstrokecolor{currentstroke}%
\pgfsetstrokeopacity{0.000000}%
\pgfsetdash{}{0pt}%
\pgfpathmoveto{\pgfqpoint{1.460122in}{0.499444in}}%
\pgfpathlineto{\pgfqpoint{1.520610in}{0.499444in}}%
\pgfpathlineto{\pgfqpoint{1.520610in}{0.652222in}}%
\pgfpathlineto{\pgfqpoint{1.460122in}{0.652222in}}%
\pgfpathlineto{\pgfqpoint{1.460122in}{0.499444in}}%
\pgfpathclose%
\pgfusepath{fill}%
\end{pgfscope}%
\begin{pgfscope}%
\pgfpathrectangle{\pgfqpoint{0.515000in}{0.499444in}}{\pgfqpoint{1.550000in}{1.155000in}}%
\pgfusepath{clip}%
\pgfsetbuttcap%
\pgfsetmiterjoin%
\definecolor{currentfill}{rgb}{0.000000,0.000000,0.000000}%
\pgfsetfillcolor{currentfill}%
\pgfsetlinewidth{0.000000pt}%
\definecolor{currentstroke}{rgb}{0.000000,0.000000,0.000000}%
\pgfsetstrokecolor{currentstroke}%
\pgfsetstrokeopacity{0.000000}%
\pgfsetdash{}{0pt}%
\pgfpathmoveto{\pgfqpoint{1.611342in}{0.499444in}}%
\pgfpathlineto{\pgfqpoint{1.671830in}{0.499444in}}%
\pgfpathlineto{\pgfqpoint{1.671830in}{0.743889in}}%
\pgfpathlineto{\pgfqpoint{1.611342in}{0.743889in}}%
\pgfpathlineto{\pgfqpoint{1.611342in}{0.499444in}}%
\pgfpathclose%
\pgfusepath{fill}%
\end{pgfscope}%
\begin{pgfscope}%
\pgfpathrectangle{\pgfqpoint{0.515000in}{0.499444in}}{\pgfqpoint{1.550000in}{1.155000in}}%
\pgfusepath{clip}%
\pgfsetbuttcap%
\pgfsetmiterjoin%
\definecolor{currentfill}{rgb}{0.000000,0.000000,0.000000}%
\pgfsetfillcolor{currentfill}%
\pgfsetlinewidth{0.000000pt}%
\definecolor{currentstroke}{rgb}{0.000000,0.000000,0.000000}%
\pgfsetstrokecolor{currentstroke}%
\pgfsetstrokeopacity{0.000000}%
\pgfsetdash{}{0pt}%
\pgfpathmoveto{\pgfqpoint{1.762561in}{0.499444in}}%
\pgfpathlineto{\pgfqpoint{1.823049in}{0.499444in}}%
\pgfpathlineto{\pgfqpoint{1.823049in}{0.759166in}}%
\pgfpathlineto{\pgfqpoint{1.762561in}{0.759166in}}%
\pgfpathlineto{\pgfqpoint{1.762561in}{0.499444in}}%
\pgfpathclose%
\pgfusepath{fill}%
\end{pgfscope}%
\begin{pgfscope}%
\pgfpathrectangle{\pgfqpoint{0.515000in}{0.499444in}}{\pgfqpoint{1.550000in}{1.155000in}}%
\pgfusepath{clip}%
\pgfsetbuttcap%
\pgfsetmiterjoin%
\definecolor{currentfill}{rgb}{0.000000,0.000000,0.000000}%
\pgfsetfillcolor{currentfill}%
\pgfsetlinewidth{0.000000pt}%
\definecolor{currentstroke}{rgb}{0.000000,0.000000,0.000000}%
\pgfsetstrokecolor{currentstroke}%
\pgfsetstrokeopacity{0.000000}%
\pgfsetdash{}{0pt}%
\pgfpathmoveto{\pgfqpoint{1.913781in}{0.499444in}}%
\pgfpathlineto{\pgfqpoint{1.974269in}{0.499444in}}%
\pgfpathlineto{\pgfqpoint{1.974269in}{0.621666in}}%
\pgfpathlineto{\pgfqpoint{1.913781in}{0.621666in}}%
\pgfpathlineto{\pgfqpoint{1.913781in}{0.499444in}}%
\pgfpathclose%
\pgfusepath{fill}%
\end{pgfscope}%
\begin{pgfscope}%
\pgfsetbuttcap%
\pgfsetroundjoin%
\definecolor{currentfill}{rgb}{0.000000,0.000000,0.000000}%
\pgfsetfillcolor{currentfill}%
\pgfsetlinewidth{0.803000pt}%
\definecolor{currentstroke}{rgb}{0.000000,0.000000,0.000000}%
\pgfsetstrokecolor{currentstroke}%
\pgfsetdash{}{0pt}%
\pgfsys@defobject{currentmarker}{\pgfqpoint{0.000000in}{-0.048611in}}{\pgfqpoint{0.000000in}{0.000000in}}{%
\pgfpathmoveto{\pgfqpoint{0.000000in}{0.000000in}}%
\pgfpathlineto{\pgfqpoint{0.000000in}{-0.048611in}}%
\pgfusepath{stroke,fill}%
}%
\begin{pgfscope}%
\pgfsys@transformshift{0.552805in}{0.499444in}%
\pgfsys@useobject{currentmarker}{}%
\end{pgfscope}%
\end{pgfscope}%
\begin{pgfscope}%
\definecolor{textcolor}{rgb}{0.000000,0.000000,0.000000}%
\pgfsetstrokecolor{textcolor}%
\pgfsetfillcolor{textcolor}%
\pgftext[x=0.552805in,y=0.402222in,,top]{\color{textcolor}\rmfamily\fontsize{10.000000}{12.000000}\selectfont 0.0}%
\end{pgfscope}%
\begin{pgfscope}%
\pgfsetbuttcap%
\pgfsetroundjoin%
\definecolor{currentfill}{rgb}{0.000000,0.000000,0.000000}%
\pgfsetfillcolor{currentfill}%
\pgfsetlinewidth{0.803000pt}%
\definecolor{currentstroke}{rgb}{0.000000,0.000000,0.000000}%
\pgfsetstrokecolor{currentstroke}%
\pgfsetdash{}{0pt}%
\pgfsys@defobject{currentmarker}{\pgfqpoint{0.000000in}{-0.048611in}}{\pgfqpoint{0.000000in}{0.000000in}}{%
\pgfpathmoveto{\pgfqpoint{0.000000in}{0.000000in}}%
\pgfpathlineto{\pgfqpoint{0.000000in}{-0.048611in}}%
\pgfusepath{stroke,fill}%
}%
\begin{pgfscope}%
\pgfsys@transformshift{0.930854in}{0.499444in}%
\pgfsys@useobject{currentmarker}{}%
\end{pgfscope}%
\end{pgfscope}%
\begin{pgfscope}%
\definecolor{textcolor}{rgb}{0.000000,0.000000,0.000000}%
\pgfsetstrokecolor{textcolor}%
\pgfsetfillcolor{textcolor}%
\pgftext[x=0.930854in,y=0.402222in,,top]{\color{textcolor}\rmfamily\fontsize{10.000000}{12.000000}\selectfont 0.25}%
\end{pgfscope}%
\begin{pgfscope}%
\pgfsetbuttcap%
\pgfsetroundjoin%
\definecolor{currentfill}{rgb}{0.000000,0.000000,0.000000}%
\pgfsetfillcolor{currentfill}%
\pgfsetlinewidth{0.803000pt}%
\definecolor{currentstroke}{rgb}{0.000000,0.000000,0.000000}%
\pgfsetstrokecolor{currentstroke}%
\pgfsetdash{}{0pt}%
\pgfsys@defobject{currentmarker}{\pgfqpoint{0.000000in}{-0.048611in}}{\pgfqpoint{0.000000in}{0.000000in}}{%
\pgfpathmoveto{\pgfqpoint{0.000000in}{0.000000in}}%
\pgfpathlineto{\pgfqpoint{0.000000in}{-0.048611in}}%
\pgfusepath{stroke,fill}%
}%
\begin{pgfscope}%
\pgfsys@transformshift{1.308903in}{0.499444in}%
\pgfsys@useobject{currentmarker}{}%
\end{pgfscope}%
\end{pgfscope}%
\begin{pgfscope}%
\definecolor{textcolor}{rgb}{0.000000,0.000000,0.000000}%
\pgfsetstrokecolor{textcolor}%
\pgfsetfillcolor{textcolor}%
\pgftext[x=1.308903in,y=0.402222in,,top]{\color{textcolor}\rmfamily\fontsize{10.000000}{12.000000}\selectfont 0.5}%
\end{pgfscope}%
\begin{pgfscope}%
\pgfsetbuttcap%
\pgfsetroundjoin%
\definecolor{currentfill}{rgb}{0.000000,0.000000,0.000000}%
\pgfsetfillcolor{currentfill}%
\pgfsetlinewidth{0.803000pt}%
\definecolor{currentstroke}{rgb}{0.000000,0.000000,0.000000}%
\pgfsetstrokecolor{currentstroke}%
\pgfsetdash{}{0pt}%
\pgfsys@defobject{currentmarker}{\pgfqpoint{0.000000in}{-0.048611in}}{\pgfqpoint{0.000000in}{0.000000in}}{%
\pgfpathmoveto{\pgfqpoint{0.000000in}{0.000000in}}%
\pgfpathlineto{\pgfqpoint{0.000000in}{-0.048611in}}%
\pgfusepath{stroke,fill}%
}%
\begin{pgfscope}%
\pgfsys@transformshift{1.686951in}{0.499444in}%
\pgfsys@useobject{currentmarker}{}%
\end{pgfscope}%
\end{pgfscope}%
\begin{pgfscope}%
\definecolor{textcolor}{rgb}{0.000000,0.000000,0.000000}%
\pgfsetstrokecolor{textcolor}%
\pgfsetfillcolor{textcolor}%
\pgftext[x=1.686951in,y=0.402222in,,top]{\color{textcolor}\rmfamily\fontsize{10.000000}{12.000000}\selectfont 0.75}%
\end{pgfscope}%
\begin{pgfscope}%
\pgfsetbuttcap%
\pgfsetroundjoin%
\definecolor{currentfill}{rgb}{0.000000,0.000000,0.000000}%
\pgfsetfillcolor{currentfill}%
\pgfsetlinewidth{0.803000pt}%
\definecolor{currentstroke}{rgb}{0.000000,0.000000,0.000000}%
\pgfsetstrokecolor{currentstroke}%
\pgfsetdash{}{0pt}%
\pgfsys@defobject{currentmarker}{\pgfqpoint{0.000000in}{-0.048611in}}{\pgfqpoint{0.000000in}{0.000000in}}{%
\pgfpathmoveto{\pgfqpoint{0.000000in}{0.000000in}}%
\pgfpathlineto{\pgfqpoint{0.000000in}{-0.048611in}}%
\pgfusepath{stroke,fill}%
}%
\begin{pgfscope}%
\pgfsys@transformshift{2.065000in}{0.499444in}%
\pgfsys@useobject{currentmarker}{}%
\end{pgfscope}%
\end{pgfscope}%
\begin{pgfscope}%
\definecolor{textcolor}{rgb}{0.000000,0.000000,0.000000}%
\pgfsetstrokecolor{textcolor}%
\pgfsetfillcolor{textcolor}%
\pgftext[x=2.065000in,y=0.402222in,,top]{\color{textcolor}\rmfamily\fontsize{10.000000}{12.000000}\selectfont 1.0}%
\end{pgfscope}%
\begin{pgfscope}%
\definecolor{textcolor}{rgb}{0.000000,0.000000,0.000000}%
\pgfsetstrokecolor{textcolor}%
\pgfsetfillcolor{textcolor}%
\pgftext[x=1.290000in,y=0.223333in,,top]{\color{textcolor}\rmfamily\fontsize{10.000000}{12.000000}\selectfont \(\displaystyle p\)}%
\end{pgfscope}%
\begin{pgfscope}%
\pgfsetbuttcap%
\pgfsetroundjoin%
\definecolor{currentfill}{rgb}{0.000000,0.000000,0.000000}%
\pgfsetfillcolor{currentfill}%
\pgfsetlinewidth{0.803000pt}%
\definecolor{currentstroke}{rgb}{0.000000,0.000000,0.000000}%
\pgfsetstrokecolor{currentstroke}%
\pgfsetdash{}{0pt}%
\pgfsys@defobject{currentmarker}{\pgfqpoint{-0.048611in}{0.000000in}}{\pgfqpoint{-0.000000in}{0.000000in}}{%
\pgfpathmoveto{\pgfqpoint{-0.000000in}{0.000000in}}%
\pgfpathlineto{\pgfqpoint{-0.048611in}{0.000000in}}%
\pgfusepath{stroke,fill}%
}%
\begin{pgfscope}%
\pgfsys@transformshift{0.515000in}{0.499444in}%
\pgfsys@useobject{currentmarker}{}%
\end{pgfscope}%
\end{pgfscope}%
\begin{pgfscope}%
\definecolor{textcolor}{rgb}{0.000000,0.000000,0.000000}%
\pgfsetstrokecolor{textcolor}%
\pgfsetfillcolor{textcolor}%
\pgftext[x=0.348333in, y=0.451250in, left, base]{\color{textcolor}\rmfamily\fontsize{10.000000}{12.000000}\selectfont \(\displaystyle {0}\)}%
\end{pgfscope}%
\begin{pgfscope}%
\pgfsetbuttcap%
\pgfsetroundjoin%
\definecolor{currentfill}{rgb}{0.000000,0.000000,0.000000}%
\pgfsetfillcolor{currentfill}%
\pgfsetlinewidth{0.803000pt}%
\definecolor{currentstroke}{rgb}{0.000000,0.000000,0.000000}%
\pgfsetstrokecolor{currentstroke}%
\pgfsetdash{}{0pt}%
\pgfsys@defobject{currentmarker}{\pgfqpoint{-0.048611in}{0.000000in}}{\pgfqpoint{-0.000000in}{0.000000in}}{%
\pgfpathmoveto{\pgfqpoint{-0.000000in}{0.000000in}}%
\pgfpathlineto{\pgfqpoint{-0.048611in}{0.000000in}}%
\pgfusepath{stroke,fill}%
}%
\begin{pgfscope}%
\pgfsys@transformshift{0.515000in}{1.034166in}%
\pgfsys@useobject{currentmarker}{}%
\end{pgfscope}%
\end{pgfscope}%
\begin{pgfscope}%
\definecolor{textcolor}{rgb}{0.000000,0.000000,0.000000}%
\pgfsetstrokecolor{textcolor}%
\pgfsetfillcolor{textcolor}%
\pgftext[x=0.348333in, y=0.985972in, left, base]{\color{textcolor}\rmfamily\fontsize{10.000000}{12.000000}\selectfont \(\displaystyle {5}\)}%
\end{pgfscope}%
\begin{pgfscope}%
\pgfsetbuttcap%
\pgfsetroundjoin%
\definecolor{currentfill}{rgb}{0.000000,0.000000,0.000000}%
\pgfsetfillcolor{currentfill}%
\pgfsetlinewidth{0.803000pt}%
\definecolor{currentstroke}{rgb}{0.000000,0.000000,0.000000}%
\pgfsetstrokecolor{currentstroke}%
\pgfsetdash{}{0pt}%
\pgfsys@defobject{currentmarker}{\pgfqpoint{-0.048611in}{0.000000in}}{\pgfqpoint{-0.000000in}{0.000000in}}{%
\pgfpathmoveto{\pgfqpoint{-0.000000in}{0.000000in}}%
\pgfpathlineto{\pgfqpoint{-0.048611in}{0.000000in}}%
\pgfusepath{stroke,fill}%
}%
\begin{pgfscope}%
\pgfsys@transformshift{0.515000in}{1.568889in}%
\pgfsys@useobject{currentmarker}{}%
\end{pgfscope}%
\end{pgfscope}%
\begin{pgfscope}%
\definecolor{textcolor}{rgb}{0.000000,0.000000,0.000000}%
\pgfsetstrokecolor{textcolor}%
\pgfsetfillcolor{textcolor}%
\pgftext[x=0.278889in, y=1.520694in, left, base]{\color{textcolor}\rmfamily\fontsize{10.000000}{12.000000}\selectfont \(\displaystyle {10}\)}%
\end{pgfscope}%
\begin{pgfscope}%
\definecolor{textcolor}{rgb}{0.000000,0.000000,0.000000}%
\pgfsetstrokecolor{textcolor}%
\pgfsetfillcolor{textcolor}%
\pgftext[x=0.223333in,y=1.076944in,,bottom,rotate=90.000000]{\color{textcolor}\rmfamily\fontsize{10.000000}{12.000000}\selectfont Percent of Data Set}%
\end{pgfscope}%
\begin{pgfscope}%
\pgfsetrectcap%
\pgfsetmiterjoin%
\pgfsetlinewidth{0.803000pt}%
\definecolor{currentstroke}{rgb}{0.000000,0.000000,0.000000}%
\pgfsetstrokecolor{currentstroke}%
\pgfsetdash{}{0pt}%
\pgfpathmoveto{\pgfqpoint{0.515000in}{0.499444in}}%
\pgfpathlineto{\pgfqpoint{0.515000in}{1.654444in}}%
\pgfusepath{stroke}%
\end{pgfscope}%
\begin{pgfscope}%
\pgfsetrectcap%
\pgfsetmiterjoin%
\pgfsetlinewidth{0.803000pt}%
\definecolor{currentstroke}{rgb}{0.000000,0.000000,0.000000}%
\pgfsetstrokecolor{currentstroke}%
\pgfsetdash{}{0pt}%
\pgfpathmoveto{\pgfqpoint{2.065000in}{0.499444in}}%
\pgfpathlineto{\pgfqpoint{2.065000in}{1.654444in}}%
\pgfusepath{stroke}%
\end{pgfscope}%
\begin{pgfscope}%
\pgfsetrectcap%
\pgfsetmiterjoin%
\pgfsetlinewidth{0.803000pt}%
\definecolor{currentstroke}{rgb}{0.000000,0.000000,0.000000}%
\pgfsetstrokecolor{currentstroke}%
\pgfsetdash{}{0pt}%
\pgfpathmoveto{\pgfqpoint{0.515000in}{0.499444in}}%
\pgfpathlineto{\pgfqpoint{2.065000in}{0.499444in}}%
\pgfusepath{stroke}%
\end{pgfscope}%
\begin{pgfscope}%
\pgfsetrectcap%
\pgfsetmiterjoin%
\pgfsetlinewidth{0.803000pt}%
\definecolor{currentstroke}{rgb}{0.000000,0.000000,0.000000}%
\pgfsetstrokecolor{currentstroke}%
\pgfsetdash{}{0pt}%
\pgfpathmoveto{\pgfqpoint{0.515000in}{1.654444in}}%
\pgfpathlineto{\pgfqpoint{2.065000in}{1.654444in}}%
\pgfusepath{stroke}%
\end{pgfscope}%
\begin{pgfscope}%
\pgfsetbuttcap%
\pgfsetmiterjoin%
\definecolor{currentfill}{rgb}{1.000000,1.000000,1.000000}%
\pgfsetfillcolor{currentfill}%
\pgfsetfillopacity{0.800000}%
\pgfsetlinewidth{1.003750pt}%
\definecolor{currentstroke}{rgb}{0.800000,0.800000,0.800000}%
\pgfsetstrokecolor{currentstroke}%
\pgfsetstrokeopacity{0.800000}%
\pgfsetdash{}{0pt}%
\pgfpathmoveto{\pgfqpoint{1.288056in}{1.154445in}}%
\pgfpathlineto{\pgfqpoint{1.967778in}{1.154445in}}%
\pgfpathquadraticcurveto{\pgfqpoint{1.995556in}{1.154445in}}{\pgfqpoint{1.995556in}{1.182222in}}%
\pgfpathlineto{\pgfqpoint{1.995556in}{1.557222in}}%
\pgfpathquadraticcurveto{\pgfqpoint{1.995556in}{1.585000in}}{\pgfqpoint{1.967778in}{1.585000in}}%
\pgfpathlineto{\pgfqpoint{1.288056in}{1.585000in}}%
\pgfpathquadraticcurveto{\pgfqpoint{1.260278in}{1.585000in}}{\pgfqpoint{1.260278in}{1.557222in}}%
\pgfpathlineto{\pgfqpoint{1.260278in}{1.182222in}}%
\pgfpathquadraticcurveto{\pgfqpoint{1.260278in}{1.154445in}}{\pgfqpoint{1.288056in}{1.154445in}}%
\pgfpathlineto{\pgfqpoint{1.288056in}{1.154445in}}%
\pgfpathclose%
\pgfusepath{stroke,fill}%
\end{pgfscope}%
\begin{pgfscope}%
\pgfsetbuttcap%
\pgfsetmiterjoin%
\pgfsetlinewidth{1.003750pt}%
\definecolor{currentstroke}{rgb}{0.000000,0.000000,0.000000}%
\pgfsetstrokecolor{currentstroke}%
\pgfsetdash{}{0pt}%
\pgfpathmoveto{\pgfqpoint{1.315834in}{1.432222in}}%
\pgfpathlineto{\pgfqpoint{1.593611in}{1.432222in}}%
\pgfpathlineto{\pgfqpoint{1.593611in}{1.529444in}}%
\pgfpathlineto{\pgfqpoint{1.315834in}{1.529444in}}%
\pgfpathlineto{\pgfqpoint{1.315834in}{1.432222in}}%
\pgfpathclose%
\pgfusepath{stroke}%
\end{pgfscope}%
\begin{pgfscope}%
\definecolor{textcolor}{rgb}{0.000000,0.000000,0.000000}%
\pgfsetstrokecolor{textcolor}%
\pgfsetfillcolor{textcolor}%
\pgftext[x=1.704722in,y=1.432222in,left,base]{\color{textcolor}\rmfamily\fontsize{10.000000}{12.000000}\selectfont Neg}%
\end{pgfscope}%
\begin{pgfscope}%
\pgfsetbuttcap%
\pgfsetmiterjoin%
\definecolor{currentfill}{rgb}{0.000000,0.000000,0.000000}%
\pgfsetfillcolor{currentfill}%
\pgfsetlinewidth{0.000000pt}%
\definecolor{currentstroke}{rgb}{0.000000,0.000000,0.000000}%
\pgfsetstrokecolor{currentstroke}%
\pgfsetstrokeopacity{0.000000}%
\pgfsetdash{}{0pt}%
\pgfpathmoveto{\pgfqpoint{1.315834in}{1.236944in}}%
\pgfpathlineto{\pgfqpoint{1.593611in}{1.236944in}}%
\pgfpathlineto{\pgfqpoint{1.593611in}{1.334167in}}%
\pgfpathlineto{\pgfqpoint{1.315834in}{1.334167in}}%
\pgfpathlineto{\pgfqpoint{1.315834in}{1.236944in}}%
\pgfpathclose%
\pgfusepath{fill}%
\end{pgfscope}%
\begin{pgfscope}%
\definecolor{textcolor}{rgb}{0.000000,0.000000,0.000000}%
\pgfsetstrokecolor{textcolor}%
\pgfsetfillcolor{textcolor}%
\pgftext[x=1.704722in,y=1.236944in,left,base]{\color{textcolor}\rmfamily\fontsize{10.000000}{12.000000}\selectfont Pos}%
\end{pgfscope}%
\end{pgfpicture}%
\makeatother%
\endgroup%

  &
  \vspace{0pt} %% Creator: Matplotlib, PGF backend
%%
%% To include the figure in your LaTeX document, write
%%   \input{<filename>.pgf}
%%
%% Make sure the required packages are loaded in your preamble
%%   \usepackage{pgf}
%%
%% Also ensure that all the required font packages are loaded; for instance,
%% the lmodern package is sometimes necessary when using math font.
%%   \usepackage{lmodern}
%%
%% Figures using additional raster images can only be included by \input if
%% they are in the same directory as the main LaTeX file. For loading figures
%% from other directories you can use the `import` package
%%   \usepackage{import}
%%
%% and then include the figures with
%%   \import{<path to file>}{<filename>.pgf}
%%
%% Matplotlib used the following preamble
%%   
%%   \usepackage{fontspec}
%%   \makeatletter\@ifpackageloaded{underscore}{}{\usepackage[strings]{underscore}}\makeatother
%%
\begingroup%
\makeatletter%
\begin{pgfpicture}%
\pgfpathrectangle{\pgfpointorigin}{\pgfqpoint{3.144311in}{2.646444in}}%
\pgfusepath{use as bounding box, clip}%
\begin{pgfscope}%
\pgfsetbuttcap%
\pgfsetmiterjoin%
\definecolor{currentfill}{rgb}{1.000000,1.000000,1.000000}%
\pgfsetfillcolor{currentfill}%
\pgfsetlinewidth{0.000000pt}%
\definecolor{currentstroke}{rgb}{1.000000,1.000000,1.000000}%
\pgfsetstrokecolor{currentstroke}%
\pgfsetdash{}{0pt}%
\pgfpathmoveto{\pgfqpoint{0.000000in}{0.000000in}}%
\pgfpathlineto{\pgfqpoint{3.144311in}{0.000000in}}%
\pgfpathlineto{\pgfqpoint{3.144311in}{2.646444in}}%
\pgfpathlineto{\pgfqpoint{0.000000in}{2.646444in}}%
\pgfpathlineto{\pgfqpoint{0.000000in}{0.000000in}}%
\pgfpathclose%
\pgfusepath{fill}%
\end{pgfscope}%
\begin{pgfscope}%
\pgfsetbuttcap%
\pgfsetmiterjoin%
\definecolor{currentfill}{rgb}{1.000000,1.000000,1.000000}%
\pgfsetfillcolor{currentfill}%
\pgfsetlinewidth{0.000000pt}%
\definecolor{currentstroke}{rgb}{0.000000,0.000000,0.000000}%
\pgfsetstrokecolor{currentstroke}%
\pgfsetstrokeopacity{0.000000}%
\pgfsetdash{}{0pt}%
\pgfpathmoveto{\pgfqpoint{0.553581in}{0.499444in}}%
\pgfpathlineto{\pgfqpoint{3.033581in}{0.499444in}}%
\pgfpathlineto{\pgfqpoint{3.033581in}{2.347444in}}%
\pgfpathlineto{\pgfqpoint{0.553581in}{2.347444in}}%
\pgfpathlineto{\pgfqpoint{0.553581in}{0.499444in}}%
\pgfpathclose%
\pgfusepath{fill}%
\end{pgfscope}%
\begin{pgfscope}%
\pgfsetbuttcap%
\pgfsetroundjoin%
\definecolor{currentfill}{rgb}{0.000000,0.000000,0.000000}%
\pgfsetfillcolor{currentfill}%
\pgfsetlinewidth{0.803000pt}%
\definecolor{currentstroke}{rgb}{0.000000,0.000000,0.000000}%
\pgfsetstrokecolor{currentstroke}%
\pgfsetdash{}{0pt}%
\pgfsys@defobject{currentmarker}{\pgfqpoint{0.000000in}{-0.048611in}}{\pgfqpoint{0.000000in}{0.000000in}}{%
\pgfpathmoveto{\pgfqpoint{0.000000in}{0.000000in}}%
\pgfpathlineto{\pgfqpoint{0.000000in}{-0.048611in}}%
\pgfusepath{stroke,fill}%
}%
\begin{pgfscope}%
\pgfsys@transformshift{0.666308in}{0.499444in}%
\pgfsys@useobject{currentmarker}{}%
\end{pgfscope}%
\end{pgfscope}%
\begin{pgfscope}%
\definecolor{textcolor}{rgb}{0.000000,0.000000,0.000000}%
\pgfsetstrokecolor{textcolor}%
\pgfsetfillcolor{textcolor}%
\pgftext[x=0.666308in,y=0.402222in,,top]{\color{textcolor}\rmfamily\fontsize{10.000000}{12.000000}\selectfont \(\displaystyle {0.00}\)}%
\end{pgfscope}%
\begin{pgfscope}%
\pgfsetbuttcap%
\pgfsetroundjoin%
\definecolor{currentfill}{rgb}{0.000000,0.000000,0.000000}%
\pgfsetfillcolor{currentfill}%
\pgfsetlinewidth{0.803000pt}%
\definecolor{currentstroke}{rgb}{0.000000,0.000000,0.000000}%
\pgfsetstrokecolor{currentstroke}%
\pgfsetdash{}{0pt}%
\pgfsys@defobject{currentmarker}{\pgfqpoint{0.000000in}{-0.048611in}}{\pgfqpoint{0.000000in}{0.000000in}}{%
\pgfpathmoveto{\pgfqpoint{0.000000in}{0.000000in}}%
\pgfpathlineto{\pgfqpoint{0.000000in}{-0.048611in}}%
\pgfusepath{stroke,fill}%
}%
\begin{pgfscope}%
\pgfsys@transformshift{1.229944in}{0.499444in}%
\pgfsys@useobject{currentmarker}{}%
\end{pgfscope}%
\end{pgfscope}%
\begin{pgfscope}%
\definecolor{textcolor}{rgb}{0.000000,0.000000,0.000000}%
\pgfsetstrokecolor{textcolor}%
\pgfsetfillcolor{textcolor}%
\pgftext[x=1.229944in,y=0.402222in,,top]{\color{textcolor}\rmfamily\fontsize{10.000000}{12.000000}\selectfont \(\displaystyle {0.25}\)}%
\end{pgfscope}%
\begin{pgfscope}%
\pgfsetbuttcap%
\pgfsetroundjoin%
\definecolor{currentfill}{rgb}{0.000000,0.000000,0.000000}%
\pgfsetfillcolor{currentfill}%
\pgfsetlinewidth{0.803000pt}%
\definecolor{currentstroke}{rgb}{0.000000,0.000000,0.000000}%
\pgfsetstrokecolor{currentstroke}%
\pgfsetdash{}{0pt}%
\pgfsys@defobject{currentmarker}{\pgfqpoint{0.000000in}{-0.048611in}}{\pgfqpoint{0.000000in}{0.000000in}}{%
\pgfpathmoveto{\pgfqpoint{0.000000in}{0.000000in}}%
\pgfpathlineto{\pgfqpoint{0.000000in}{-0.048611in}}%
\pgfusepath{stroke,fill}%
}%
\begin{pgfscope}%
\pgfsys@transformshift{1.793581in}{0.499444in}%
\pgfsys@useobject{currentmarker}{}%
\end{pgfscope}%
\end{pgfscope}%
\begin{pgfscope}%
\definecolor{textcolor}{rgb}{0.000000,0.000000,0.000000}%
\pgfsetstrokecolor{textcolor}%
\pgfsetfillcolor{textcolor}%
\pgftext[x=1.793581in,y=0.402222in,,top]{\color{textcolor}\rmfamily\fontsize{10.000000}{12.000000}\selectfont \(\displaystyle {0.50}\)}%
\end{pgfscope}%
\begin{pgfscope}%
\pgfsetbuttcap%
\pgfsetroundjoin%
\definecolor{currentfill}{rgb}{0.000000,0.000000,0.000000}%
\pgfsetfillcolor{currentfill}%
\pgfsetlinewidth{0.803000pt}%
\definecolor{currentstroke}{rgb}{0.000000,0.000000,0.000000}%
\pgfsetstrokecolor{currentstroke}%
\pgfsetdash{}{0pt}%
\pgfsys@defobject{currentmarker}{\pgfqpoint{0.000000in}{-0.048611in}}{\pgfqpoint{0.000000in}{0.000000in}}{%
\pgfpathmoveto{\pgfqpoint{0.000000in}{0.000000in}}%
\pgfpathlineto{\pgfqpoint{0.000000in}{-0.048611in}}%
\pgfusepath{stroke,fill}%
}%
\begin{pgfscope}%
\pgfsys@transformshift{2.357217in}{0.499444in}%
\pgfsys@useobject{currentmarker}{}%
\end{pgfscope}%
\end{pgfscope}%
\begin{pgfscope}%
\definecolor{textcolor}{rgb}{0.000000,0.000000,0.000000}%
\pgfsetstrokecolor{textcolor}%
\pgfsetfillcolor{textcolor}%
\pgftext[x=2.357217in,y=0.402222in,,top]{\color{textcolor}\rmfamily\fontsize{10.000000}{12.000000}\selectfont \(\displaystyle {0.75}\)}%
\end{pgfscope}%
\begin{pgfscope}%
\pgfsetbuttcap%
\pgfsetroundjoin%
\definecolor{currentfill}{rgb}{0.000000,0.000000,0.000000}%
\pgfsetfillcolor{currentfill}%
\pgfsetlinewidth{0.803000pt}%
\definecolor{currentstroke}{rgb}{0.000000,0.000000,0.000000}%
\pgfsetstrokecolor{currentstroke}%
\pgfsetdash{}{0pt}%
\pgfsys@defobject{currentmarker}{\pgfqpoint{0.000000in}{-0.048611in}}{\pgfqpoint{0.000000in}{0.000000in}}{%
\pgfpathmoveto{\pgfqpoint{0.000000in}{0.000000in}}%
\pgfpathlineto{\pgfqpoint{0.000000in}{-0.048611in}}%
\pgfusepath{stroke,fill}%
}%
\begin{pgfscope}%
\pgfsys@transformshift{2.920853in}{0.499444in}%
\pgfsys@useobject{currentmarker}{}%
\end{pgfscope}%
\end{pgfscope}%
\begin{pgfscope}%
\definecolor{textcolor}{rgb}{0.000000,0.000000,0.000000}%
\pgfsetstrokecolor{textcolor}%
\pgfsetfillcolor{textcolor}%
\pgftext[x=2.920853in,y=0.402222in,,top]{\color{textcolor}\rmfamily\fontsize{10.000000}{12.000000}\selectfont \(\displaystyle {1.00}\)}%
\end{pgfscope}%
\begin{pgfscope}%
\definecolor{textcolor}{rgb}{0.000000,0.000000,0.000000}%
\pgfsetstrokecolor{textcolor}%
\pgfsetfillcolor{textcolor}%
\pgftext[x=1.793581in,y=0.223333in,,top]{\color{textcolor}\rmfamily\fontsize{10.000000}{12.000000}\selectfont False positive rate}%
\end{pgfscope}%
\begin{pgfscope}%
\pgfsetbuttcap%
\pgfsetroundjoin%
\definecolor{currentfill}{rgb}{0.000000,0.000000,0.000000}%
\pgfsetfillcolor{currentfill}%
\pgfsetlinewidth{0.803000pt}%
\definecolor{currentstroke}{rgb}{0.000000,0.000000,0.000000}%
\pgfsetstrokecolor{currentstroke}%
\pgfsetdash{}{0pt}%
\pgfsys@defobject{currentmarker}{\pgfqpoint{-0.048611in}{0.000000in}}{\pgfqpoint{-0.000000in}{0.000000in}}{%
\pgfpathmoveto{\pgfqpoint{-0.000000in}{0.000000in}}%
\pgfpathlineto{\pgfqpoint{-0.048611in}{0.000000in}}%
\pgfusepath{stroke,fill}%
}%
\begin{pgfscope}%
\pgfsys@transformshift{0.553581in}{0.583444in}%
\pgfsys@useobject{currentmarker}{}%
\end{pgfscope}%
\end{pgfscope}%
\begin{pgfscope}%
\definecolor{textcolor}{rgb}{0.000000,0.000000,0.000000}%
\pgfsetstrokecolor{textcolor}%
\pgfsetfillcolor{textcolor}%
\pgftext[x=0.278889in, y=0.535250in, left, base]{\color{textcolor}\rmfamily\fontsize{10.000000}{12.000000}\selectfont \(\displaystyle {0.0}\)}%
\end{pgfscope}%
\begin{pgfscope}%
\pgfsetbuttcap%
\pgfsetroundjoin%
\definecolor{currentfill}{rgb}{0.000000,0.000000,0.000000}%
\pgfsetfillcolor{currentfill}%
\pgfsetlinewidth{0.803000pt}%
\definecolor{currentstroke}{rgb}{0.000000,0.000000,0.000000}%
\pgfsetstrokecolor{currentstroke}%
\pgfsetdash{}{0pt}%
\pgfsys@defobject{currentmarker}{\pgfqpoint{-0.048611in}{0.000000in}}{\pgfqpoint{-0.000000in}{0.000000in}}{%
\pgfpathmoveto{\pgfqpoint{-0.000000in}{0.000000in}}%
\pgfpathlineto{\pgfqpoint{-0.048611in}{0.000000in}}%
\pgfusepath{stroke,fill}%
}%
\begin{pgfscope}%
\pgfsys@transformshift{0.553581in}{0.919444in}%
\pgfsys@useobject{currentmarker}{}%
\end{pgfscope}%
\end{pgfscope}%
\begin{pgfscope}%
\definecolor{textcolor}{rgb}{0.000000,0.000000,0.000000}%
\pgfsetstrokecolor{textcolor}%
\pgfsetfillcolor{textcolor}%
\pgftext[x=0.278889in, y=0.871250in, left, base]{\color{textcolor}\rmfamily\fontsize{10.000000}{12.000000}\selectfont \(\displaystyle {0.2}\)}%
\end{pgfscope}%
\begin{pgfscope}%
\pgfsetbuttcap%
\pgfsetroundjoin%
\definecolor{currentfill}{rgb}{0.000000,0.000000,0.000000}%
\pgfsetfillcolor{currentfill}%
\pgfsetlinewidth{0.803000pt}%
\definecolor{currentstroke}{rgb}{0.000000,0.000000,0.000000}%
\pgfsetstrokecolor{currentstroke}%
\pgfsetdash{}{0pt}%
\pgfsys@defobject{currentmarker}{\pgfqpoint{-0.048611in}{0.000000in}}{\pgfqpoint{-0.000000in}{0.000000in}}{%
\pgfpathmoveto{\pgfqpoint{-0.000000in}{0.000000in}}%
\pgfpathlineto{\pgfqpoint{-0.048611in}{0.000000in}}%
\pgfusepath{stroke,fill}%
}%
\begin{pgfscope}%
\pgfsys@transformshift{0.553581in}{1.255444in}%
\pgfsys@useobject{currentmarker}{}%
\end{pgfscope}%
\end{pgfscope}%
\begin{pgfscope}%
\definecolor{textcolor}{rgb}{0.000000,0.000000,0.000000}%
\pgfsetstrokecolor{textcolor}%
\pgfsetfillcolor{textcolor}%
\pgftext[x=0.278889in, y=1.207250in, left, base]{\color{textcolor}\rmfamily\fontsize{10.000000}{12.000000}\selectfont \(\displaystyle {0.4}\)}%
\end{pgfscope}%
\begin{pgfscope}%
\pgfsetbuttcap%
\pgfsetroundjoin%
\definecolor{currentfill}{rgb}{0.000000,0.000000,0.000000}%
\pgfsetfillcolor{currentfill}%
\pgfsetlinewidth{0.803000pt}%
\definecolor{currentstroke}{rgb}{0.000000,0.000000,0.000000}%
\pgfsetstrokecolor{currentstroke}%
\pgfsetdash{}{0pt}%
\pgfsys@defobject{currentmarker}{\pgfqpoint{-0.048611in}{0.000000in}}{\pgfqpoint{-0.000000in}{0.000000in}}{%
\pgfpathmoveto{\pgfqpoint{-0.000000in}{0.000000in}}%
\pgfpathlineto{\pgfqpoint{-0.048611in}{0.000000in}}%
\pgfusepath{stroke,fill}%
}%
\begin{pgfscope}%
\pgfsys@transformshift{0.553581in}{1.591444in}%
\pgfsys@useobject{currentmarker}{}%
\end{pgfscope}%
\end{pgfscope}%
\begin{pgfscope}%
\definecolor{textcolor}{rgb}{0.000000,0.000000,0.000000}%
\pgfsetstrokecolor{textcolor}%
\pgfsetfillcolor{textcolor}%
\pgftext[x=0.278889in, y=1.543250in, left, base]{\color{textcolor}\rmfamily\fontsize{10.000000}{12.000000}\selectfont \(\displaystyle {0.6}\)}%
\end{pgfscope}%
\begin{pgfscope}%
\pgfsetbuttcap%
\pgfsetroundjoin%
\definecolor{currentfill}{rgb}{0.000000,0.000000,0.000000}%
\pgfsetfillcolor{currentfill}%
\pgfsetlinewidth{0.803000pt}%
\definecolor{currentstroke}{rgb}{0.000000,0.000000,0.000000}%
\pgfsetstrokecolor{currentstroke}%
\pgfsetdash{}{0pt}%
\pgfsys@defobject{currentmarker}{\pgfqpoint{-0.048611in}{0.000000in}}{\pgfqpoint{-0.000000in}{0.000000in}}{%
\pgfpathmoveto{\pgfqpoint{-0.000000in}{0.000000in}}%
\pgfpathlineto{\pgfqpoint{-0.048611in}{0.000000in}}%
\pgfusepath{stroke,fill}%
}%
\begin{pgfscope}%
\pgfsys@transformshift{0.553581in}{1.927444in}%
\pgfsys@useobject{currentmarker}{}%
\end{pgfscope}%
\end{pgfscope}%
\begin{pgfscope}%
\definecolor{textcolor}{rgb}{0.000000,0.000000,0.000000}%
\pgfsetstrokecolor{textcolor}%
\pgfsetfillcolor{textcolor}%
\pgftext[x=0.278889in, y=1.879250in, left, base]{\color{textcolor}\rmfamily\fontsize{10.000000}{12.000000}\selectfont \(\displaystyle {0.8}\)}%
\end{pgfscope}%
\begin{pgfscope}%
\pgfsetbuttcap%
\pgfsetroundjoin%
\definecolor{currentfill}{rgb}{0.000000,0.000000,0.000000}%
\pgfsetfillcolor{currentfill}%
\pgfsetlinewidth{0.803000pt}%
\definecolor{currentstroke}{rgb}{0.000000,0.000000,0.000000}%
\pgfsetstrokecolor{currentstroke}%
\pgfsetdash{}{0pt}%
\pgfsys@defobject{currentmarker}{\pgfqpoint{-0.048611in}{0.000000in}}{\pgfqpoint{-0.000000in}{0.000000in}}{%
\pgfpathmoveto{\pgfqpoint{-0.000000in}{0.000000in}}%
\pgfpathlineto{\pgfqpoint{-0.048611in}{0.000000in}}%
\pgfusepath{stroke,fill}%
}%
\begin{pgfscope}%
\pgfsys@transformshift{0.553581in}{2.263444in}%
\pgfsys@useobject{currentmarker}{}%
\end{pgfscope}%
\end{pgfscope}%
\begin{pgfscope}%
\definecolor{textcolor}{rgb}{0.000000,0.000000,0.000000}%
\pgfsetstrokecolor{textcolor}%
\pgfsetfillcolor{textcolor}%
\pgftext[x=0.278889in, y=2.215250in, left, base]{\color{textcolor}\rmfamily\fontsize{10.000000}{12.000000}\selectfont \(\displaystyle {1.0}\)}%
\end{pgfscope}%
\begin{pgfscope}%
\definecolor{textcolor}{rgb}{0.000000,0.000000,0.000000}%
\pgfsetstrokecolor{textcolor}%
\pgfsetfillcolor{textcolor}%
\pgftext[x=0.223333in,y=1.423444in,,bottom,rotate=90.000000]{\color{textcolor}\rmfamily\fontsize{10.000000}{12.000000}\selectfont True positive rate}%
\end{pgfscope}%
\begin{pgfscope}%
\pgfpathrectangle{\pgfqpoint{0.553581in}{0.499444in}}{\pgfqpoint{2.480000in}{1.848000in}}%
\pgfusepath{clip}%
\pgfsetbuttcap%
\pgfsetroundjoin%
\pgfsetlinewidth{1.505625pt}%
\definecolor{currentstroke}{rgb}{0.000000,0.000000,0.000000}%
\pgfsetstrokecolor{currentstroke}%
\pgfsetdash{{5.550000pt}{2.400000pt}}{0.000000pt}%
\pgfpathmoveto{\pgfqpoint{0.666308in}{0.583444in}}%
\pgfpathlineto{\pgfqpoint{2.920853in}{2.263444in}}%
\pgfusepath{stroke}%
\end{pgfscope}%
\begin{pgfscope}%
\pgfpathrectangle{\pgfqpoint{0.553581in}{0.499444in}}{\pgfqpoint{2.480000in}{1.848000in}}%
\pgfusepath{clip}%
\pgfsetrectcap%
\pgfsetroundjoin%
\pgfsetlinewidth{1.505625pt}%
\definecolor{currentstroke}{rgb}{0.121569,0.466667,0.705882}%
\pgfsetstrokecolor{currentstroke}%
\pgfsetdash{}{0pt}%
\pgfpathmoveto{\pgfqpoint{0.666308in}{0.583444in}}%
\pgfpathlineto{\pgfqpoint{0.711399in}{0.583444in}}%
\pgfpathlineto{\pgfqpoint{0.711399in}{0.600244in}}%
\pgfpathlineto{\pgfqpoint{0.756490in}{0.600244in}}%
\pgfpathlineto{\pgfqpoint{0.756490in}{0.617044in}}%
\pgfpathlineto{\pgfqpoint{0.775278in}{0.617044in}}%
\pgfpathlineto{\pgfqpoint{0.775278in}{0.633844in}}%
\pgfpathlineto{\pgfqpoint{0.839156in}{0.633844in}}%
\pgfpathlineto{\pgfqpoint{0.839156in}{0.650644in}}%
\pgfpathlineto{\pgfqpoint{0.872975in}{0.650644in}}%
\pgfpathlineto{\pgfqpoint{0.872975in}{0.684244in}}%
\pgfpathlineto{\pgfqpoint{0.880490in}{0.684244in}}%
\pgfpathlineto{\pgfqpoint{0.880490in}{0.717844in}}%
\pgfpathlineto{\pgfqpoint{0.925581in}{0.717844in}}%
\pgfpathlineto{\pgfqpoint{0.925581in}{0.734644in}}%
\pgfpathlineto{\pgfqpoint{0.948126in}{0.734644in}}%
\pgfpathlineto{\pgfqpoint{0.948126in}{0.751444in}}%
\pgfpathlineto{\pgfqpoint{0.966914in}{0.751444in}}%
\pgfpathlineto{\pgfqpoint{0.966914in}{0.768244in}}%
\pgfpathlineto{\pgfqpoint{0.978187in}{0.768244in}}%
\pgfpathlineto{\pgfqpoint{0.978187in}{0.818644in}}%
\pgfpathlineto{\pgfqpoint{0.993217in}{0.818644in}}%
\pgfpathlineto{\pgfqpoint{0.993217in}{0.835444in}}%
\pgfpathlineto{\pgfqpoint{1.057096in}{0.835444in}}%
\pgfpathlineto{\pgfqpoint{1.057096in}{0.869044in}}%
\pgfpathlineto{\pgfqpoint{1.064611in}{0.869044in}}%
\pgfpathlineto{\pgfqpoint{1.064611in}{0.885844in}}%
\pgfpathlineto{\pgfqpoint{1.079641in}{0.885844in}}%
\pgfpathlineto{\pgfqpoint{1.079641in}{0.919444in}}%
\pgfpathlineto{\pgfqpoint{1.083399in}{0.919444in}}%
\pgfpathlineto{\pgfqpoint{1.083399in}{0.936244in}}%
\pgfpathlineto{\pgfqpoint{1.087156in}{0.936244in}}%
\pgfpathlineto{\pgfqpoint{1.087156in}{0.953044in}}%
\pgfpathlineto{\pgfqpoint{1.105944in}{0.953044in}}%
\pgfpathlineto{\pgfqpoint{1.105944in}{0.969844in}}%
\pgfpathlineto{\pgfqpoint{1.117217in}{0.969844in}}%
\pgfpathlineto{\pgfqpoint{1.117217in}{1.003444in}}%
\pgfpathlineto{\pgfqpoint{1.177338in}{1.003444in}}%
\pgfpathlineto{\pgfqpoint{1.177338in}{1.020244in}}%
\pgfpathlineto{\pgfqpoint{1.181096in}{1.020244in}}%
\pgfpathlineto{\pgfqpoint{1.181096in}{1.037044in}}%
\pgfpathlineto{\pgfqpoint{1.184853in}{1.037044in}}%
\pgfpathlineto{\pgfqpoint{1.184853in}{1.053844in}}%
\pgfpathlineto{\pgfqpoint{1.229944in}{1.053844in}}%
\pgfpathlineto{\pgfqpoint{1.229944in}{1.070644in}}%
\pgfpathlineto{\pgfqpoint{1.244975in}{1.070644in}}%
\pgfpathlineto{\pgfqpoint{1.244975in}{1.104244in}}%
\pgfpathlineto{\pgfqpoint{1.256247in}{1.104244in}}%
\pgfpathlineto{\pgfqpoint{1.256247in}{1.121044in}}%
\pgfpathlineto{\pgfqpoint{1.271278in}{1.121044in}}%
\pgfpathlineto{\pgfqpoint{1.271278in}{1.137844in}}%
\pgfpathlineto{\pgfqpoint{1.278793in}{1.137844in}}%
\pgfpathlineto{\pgfqpoint{1.278793in}{1.154644in}}%
\pgfpathlineto{\pgfqpoint{1.293823in}{1.154644in}}%
\pgfpathlineto{\pgfqpoint{1.293823in}{1.171444in}}%
\pgfpathlineto{\pgfqpoint{1.316369in}{1.171444in}}%
\pgfpathlineto{\pgfqpoint{1.316369in}{1.188244in}}%
\pgfpathlineto{\pgfqpoint{1.320126in}{1.188244in}}%
\pgfpathlineto{\pgfqpoint{1.320126in}{1.255444in}}%
\pgfpathlineto{\pgfqpoint{1.331399in}{1.255444in}}%
\pgfpathlineto{\pgfqpoint{1.331399in}{1.272244in}}%
\pgfpathlineto{\pgfqpoint{1.406550in}{1.272244in}}%
\pgfpathlineto{\pgfqpoint{1.406550in}{1.289044in}}%
\pgfpathlineto{\pgfqpoint{1.410308in}{1.289044in}}%
\pgfpathlineto{\pgfqpoint{1.410308in}{1.305844in}}%
\pgfpathlineto{\pgfqpoint{1.414066in}{1.305844in}}%
\pgfpathlineto{\pgfqpoint{1.414066in}{1.322644in}}%
\pgfpathlineto{\pgfqpoint{1.474187in}{1.322644in}}%
\pgfpathlineto{\pgfqpoint{1.474187in}{1.339444in}}%
\pgfpathlineto{\pgfqpoint{1.496732in}{1.339444in}}%
\pgfpathlineto{\pgfqpoint{1.496732in}{1.356244in}}%
\pgfpathlineto{\pgfqpoint{1.500490in}{1.356244in}}%
\pgfpathlineto{\pgfqpoint{1.500490in}{1.373044in}}%
\pgfpathlineto{\pgfqpoint{1.530550in}{1.373044in}}%
\pgfpathlineto{\pgfqpoint{1.530550in}{1.406644in}}%
\pgfpathlineto{\pgfqpoint{1.553096in}{1.406644in}}%
\pgfpathlineto{\pgfqpoint{1.553096in}{1.423444in}}%
\pgfpathlineto{\pgfqpoint{1.605702in}{1.423444in}}%
\pgfpathlineto{\pgfqpoint{1.605702in}{1.440244in}}%
\pgfpathlineto{\pgfqpoint{1.635763in}{1.440244in}}%
\pgfpathlineto{\pgfqpoint{1.635763in}{1.457044in}}%
\pgfpathlineto{\pgfqpoint{1.662066in}{1.457044in}}%
\pgfpathlineto{\pgfqpoint{1.662066in}{1.473844in}}%
\pgfpathlineto{\pgfqpoint{1.669581in}{1.473844in}}%
\pgfpathlineto{\pgfqpoint{1.669581in}{1.490644in}}%
\pgfpathlineto{\pgfqpoint{1.703399in}{1.490644in}}%
\pgfpathlineto{\pgfqpoint{1.703399in}{1.507444in}}%
\pgfpathlineto{\pgfqpoint{1.740975in}{1.507444in}}%
\pgfpathlineto{\pgfqpoint{1.740975in}{1.541044in}}%
\pgfpathlineto{\pgfqpoint{1.786066in}{1.541044in}}%
\pgfpathlineto{\pgfqpoint{1.786066in}{1.557844in}}%
\pgfpathlineto{\pgfqpoint{1.797338in}{1.557844in}}%
\pgfpathlineto{\pgfqpoint{1.797338in}{1.574644in}}%
\pgfpathlineto{\pgfqpoint{1.834914in}{1.574644in}}%
\pgfpathlineto{\pgfqpoint{1.834914in}{1.591444in}}%
\pgfpathlineto{\pgfqpoint{1.842429in}{1.591444in}}%
\pgfpathlineto{\pgfqpoint{1.842429in}{1.608244in}}%
\pgfpathlineto{\pgfqpoint{1.864975in}{1.608244in}}%
\pgfpathlineto{\pgfqpoint{1.864975in}{1.625044in}}%
\pgfpathlineto{\pgfqpoint{1.876247in}{1.625044in}}%
\pgfpathlineto{\pgfqpoint{1.876247in}{1.658644in}}%
\pgfpathlineto{\pgfqpoint{1.898793in}{1.658644in}}%
\pgfpathlineto{\pgfqpoint{1.898793in}{1.692244in}}%
\pgfpathlineto{\pgfqpoint{1.906308in}{1.692244in}}%
\pgfpathlineto{\pgfqpoint{1.906308in}{1.709044in}}%
\pgfpathlineto{\pgfqpoint{1.932611in}{1.709044in}}%
\pgfpathlineto{\pgfqpoint{1.932611in}{1.725844in}}%
\pgfpathlineto{\pgfqpoint{1.955156in}{1.725844in}}%
\pgfpathlineto{\pgfqpoint{1.955156in}{1.742644in}}%
\pgfpathlineto{\pgfqpoint{1.958914in}{1.742644in}}%
\pgfpathlineto{\pgfqpoint{1.958914in}{1.759444in}}%
\pgfpathlineto{\pgfqpoint{1.977702in}{1.759444in}}%
\pgfpathlineto{\pgfqpoint{1.977702in}{1.776244in}}%
\pgfpathlineto{\pgfqpoint{1.988975in}{1.776244in}}%
\pgfpathlineto{\pgfqpoint{1.988975in}{1.793044in}}%
\pgfpathlineto{\pgfqpoint{2.000247in}{1.793044in}}%
\pgfpathlineto{\pgfqpoint{2.000247in}{1.809844in}}%
\pgfpathlineto{\pgfqpoint{2.011520in}{1.809844in}}%
\pgfpathlineto{\pgfqpoint{2.011520in}{1.826644in}}%
\pgfpathlineto{\pgfqpoint{2.019035in}{1.826644in}}%
\pgfpathlineto{\pgfqpoint{2.019035in}{1.843444in}}%
\pgfpathlineto{\pgfqpoint{2.041581in}{1.843444in}}%
\pgfpathlineto{\pgfqpoint{2.041581in}{1.860244in}}%
\pgfpathlineto{\pgfqpoint{2.071641in}{1.860244in}}%
\pgfpathlineto{\pgfqpoint{2.071641in}{1.877044in}}%
\pgfpathlineto{\pgfqpoint{2.120490in}{1.877044in}}%
\pgfpathlineto{\pgfqpoint{2.120490in}{1.893844in}}%
\pgfpathlineto{\pgfqpoint{2.161823in}{1.893844in}}%
\pgfpathlineto{\pgfqpoint{2.161823in}{1.910644in}}%
\pgfpathlineto{\pgfqpoint{2.191884in}{1.910644in}}%
\pgfpathlineto{\pgfqpoint{2.191884in}{1.927444in}}%
\pgfpathlineto{\pgfqpoint{2.225702in}{1.927444in}}%
\pgfpathlineto{\pgfqpoint{2.225702in}{1.944244in}}%
\pgfpathlineto{\pgfqpoint{2.263278in}{1.944244in}}%
\pgfpathlineto{\pgfqpoint{2.263278in}{1.961044in}}%
\pgfpathlineto{\pgfqpoint{2.330914in}{1.961044in}}%
\pgfpathlineto{\pgfqpoint{2.330914in}{1.994644in}}%
\pgfpathlineto{\pgfqpoint{2.334672in}{1.994644in}}%
\pgfpathlineto{\pgfqpoint{2.334672in}{2.011444in}}%
\pgfpathlineto{\pgfqpoint{2.376005in}{2.011444in}}%
\pgfpathlineto{\pgfqpoint{2.376005in}{2.028244in}}%
\pgfpathlineto{\pgfqpoint{2.413581in}{2.028244in}}%
\pgfpathlineto{\pgfqpoint{2.413581in}{2.061844in}}%
\pgfpathlineto{\pgfqpoint{2.432369in}{2.061844in}}%
\pgfpathlineto{\pgfqpoint{2.432369in}{2.078644in}}%
\pgfpathlineto{\pgfqpoint{2.454914in}{2.078644in}}%
\pgfpathlineto{\pgfqpoint{2.454914in}{2.095444in}}%
\pgfpathlineto{\pgfqpoint{2.563884in}{2.095444in}}%
\pgfpathlineto{\pgfqpoint{2.563884in}{2.112244in}}%
\pgfpathlineto{\pgfqpoint{2.582672in}{2.112244in}}%
\pgfpathlineto{\pgfqpoint{2.582672in}{2.129044in}}%
\pgfpathlineto{\pgfqpoint{2.631520in}{2.129044in}}%
\pgfpathlineto{\pgfqpoint{2.631520in}{2.145844in}}%
\pgfpathlineto{\pgfqpoint{2.635278in}{2.145844in}}%
\pgfpathlineto{\pgfqpoint{2.635278in}{2.179444in}}%
\pgfpathlineto{\pgfqpoint{2.714187in}{2.179444in}}%
\pgfpathlineto{\pgfqpoint{2.714187in}{2.196244in}}%
\pgfpathlineto{\pgfqpoint{2.748005in}{2.196244in}}%
\pgfpathlineto{\pgfqpoint{2.748005in}{2.213044in}}%
\pgfpathlineto{\pgfqpoint{2.841944in}{2.213044in}}%
\pgfpathlineto{\pgfqpoint{2.841944in}{2.229844in}}%
\pgfpathlineto{\pgfqpoint{2.879520in}{2.229844in}}%
\pgfpathlineto{\pgfqpoint{2.879520in}{2.246644in}}%
\pgfpathlineto{\pgfqpoint{2.894550in}{2.246644in}}%
\pgfpathlineto{\pgfqpoint{2.894550in}{2.263444in}}%
\pgfpathlineto{\pgfqpoint{2.920853in}{2.263444in}}%
\pgfpathlineto{\pgfqpoint{2.920853in}{2.263444in}}%
\pgfusepath{stroke}%
\end{pgfscope}%
\begin{pgfscope}%
\pgfsetrectcap%
\pgfsetmiterjoin%
\pgfsetlinewidth{0.803000pt}%
\definecolor{currentstroke}{rgb}{0.000000,0.000000,0.000000}%
\pgfsetstrokecolor{currentstroke}%
\pgfsetdash{}{0pt}%
\pgfpathmoveto{\pgfqpoint{0.553581in}{0.499444in}}%
\pgfpathlineto{\pgfqpoint{0.553581in}{2.347444in}}%
\pgfusepath{stroke}%
\end{pgfscope}%
\begin{pgfscope}%
\pgfsetrectcap%
\pgfsetmiterjoin%
\pgfsetlinewidth{0.803000pt}%
\definecolor{currentstroke}{rgb}{0.000000,0.000000,0.000000}%
\pgfsetstrokecolor{currentstroke}%
\pgfsetdash{}{0pt}%
\pgfpathmoveto{\pgfqpoint{3.033581in}{0.499444in}}%
\pgfpathlineto{\pgfqpoint{3.033581in}{2.347444in}}%
\pgfusepath{stroke}%
\end{pgfscope}%
\begin{pgfscope}%
\pgfsetrectcap%
\pgfsetmiterjoin%
\pgfsetlinewidth{0.803000pt}%
\definecolor{currentstroke}{rgb}{0.000000,0.000000,0.000000}%
\pgfsetstrokecolor{currentstroke}%
\pgfsetdash{}{0pt}%
\pgfpathmoveto{\pgfqpoint{0.553581in}{0.499444in}}%
\pgfpathlineto{\pgfqpoint{3.033581in}{0.499444in}}%
\pgfusepath{stroke}%
\end{pgfscope}%
\begin{pgfscope}%
\pgfsetrectcap%
\pgfsetmiterjoin%
\pgfsetlinewidth{0.803000pt}%
\definecolor{currentstroke}{rgb}{0.000000,0.000000,0.000000}%
\pgfsetstrokecolor{currentstroke}%
\pgfsetdash{}{0pt}%
\pgfpathmoveto{\pgfqpoint{0.553581in}{2.347444in}}%
\pgfpathlineto{\pgfqpoint{3.033581in}{2.347444in}}%
\pgfusepath{stroke}%
\end{pgfscope}%
\begin{pgfscope}%
\pgfsetbuttcap%
\pgfsetmiterjoin%
\definecolor{currentfill}{rgb}{1.000000,1.000000,1.000000}%
\pgfsetfillcolor{currentfill}%
\pgfsetlinewidth{1.003750pt}%
\definecolor{currentstroke}{rgb}{1.000000,1.000000,1.000000}%
\pgfsetstrokecolor{currentstroke}%
\pgfsetdash{}{0pt}%
\pgfpathmoveto{\pgfqpoint{1.741783in}{1.475344in}}%
\pgfpathlineto{\pgfqpoint{2.169283in}{1.475344in}}%
\pgfpathlineto{\pgfqpoint{2.169283in}{1.709789in}}%
\pgfpathlineto{\pgfqpoint{1.741783in}{1.709789in}}%
\pgfpathlineto{\pgfqpoint{1.741783in}{1.475344in}}%
\pgfpathclose%
\pgfusepath{stroke,fill}%
\end{pgfscope}%
\begin{pgfscope}%
\definecolor{textcolor}{rgb}{0.000000,0.000000,0.000000}%
\pgfsetstrokecolor{textcolor}%
\pgfsetfillcolor{textcolor}%
\pgftext[x=1.797338in,y=1.557844in,left,base]{\color{textcolor}\rmfamily\fontsize{10.000000}{12.000000}\selectfont 0.517}%
\end{pgfscope}%
\begin{pgfscope}%
\pgfsetbuttcap%
\pgfsetmiterjoin%
\definecolor{currentfill}{rgb}{1.000000,1.000000,1.000000}%
\pgfsetfillcolor{currentfill}%
\pgfsetlinewidth{1.003750pt}%
\definecolor{currentstroke}{rgb}{1.000000,1.000000,1.000000}%
\pgfsetstrokecolor{currentstroke}%
\pgfsetdash{}{0pt}%
\pgfpathmoveto{\pgfqpoint{1.550146in}{1.340944in}}%
\pgfpathlineto{\pgfqpoint{1.977646in}{1.340944in}}%
\pgfpathlineto{\pgfqpoint{1.977646in}{1.575389in}}%
\pgfpathlineto{\pgfqpoint{1.550146in}{1.575389in}}%
\pgfpathlineto{\pgfqpoint{1.550146in}{1.340944in}}%
\pgfpathclose%
\pgfusepath{stroke,fill}%
\end{pgfscope}%
\begin{pgfscope}%
\definecolor{textcolor}{rgb}{0.000000,0.000000,0.000000}%
\pgfsetstrokecolor{textcolor}%
\pgfsetfillcolor{textcolor}%
\pgftext[x=1.605702in,y=1.423444in,left,base]{\color{textcolor}\rmfamily\fontsize{10.000000}{12.000000}\selectfont 0.612}%
\end{pgfscope}%
\begin{pgfscope}%
\definecolor{textcolor}{rgb}{0.000000,0.000000,0.000000}%
\pgfsetstrokecolor{textcolor}%
\pgfsetfillcolor{textcolor}%
\pgftext[x=1.793581in,y=2.430778in,,base]{\color{textcolor}\rmfamily\fontsize{12.000000}{14.400000}\selectfont ROC Curve}%
\end{pgfscope}%
\begin{pgfscope}%
\pgfsetbuttcap%
\pgfsetmiterjoin%
\definecolor{currentfill}{rgb}{1.000000,1.000000,1.000000}%
\pgfsetfillcolor{currentfill}%
\pgfsetfillopacity{0.800000}%
\pgfsetlinewidth{1.003750pt}%
\definecolor{currentstroke}{rgb}{0.800000,0.800000,0.800000}%
\pgfsetstrokecolor{currentstroke}%
\pgfsetstrokeopacity{0.800000}%
\pgfsetdash{}{0pt}%
\pgfpathmoveto{\pgfqpoint{0.800942in}{0.568889in}}%
\pgfpathlineto{\pgfqpoint{2.936358in}{0.568889in}}%
\pgfpathquadraticcurveto{\pgfqpoint{2.964136in}{0.568889in}}{\pgfqpoint{2.964136in}{0.596666in}}%
\pgfpathlineto{\pgfqpoint{2.964136in}{0.791111in}}%
\pgfpathquadraticcurveto{\pgfqpoint{2.964136in}{0.818888in}}{\pgfqpoint{2.936358in}{0.818888in}}%
\pgfpathlineto{\pgfqpoint{0.800942in}{0.818888in}}%
\pgfpathquadraticcurveto{\pgfqpoint{0.773164in}{0.818888in}}{\pgfqpoint{0.773164in}{0.791111in}}%
\pgfpathlineto{\pgfqpoint{0.773164in}{0.596666in}}%
\pgfpathquadraticcurveto{\pgfqpoint{0.773164in}{0.568889in}}{\pgfqpoint{0.800942in}{0.568889in}}%
\pgfpathlineto{\pgfqpoint{0.800942in}{0.568889in}}%
\pgfpathclose%
\pgfusepath{stroke,fill}%
\end{pgfscope}%
\begin{pgfscope}%
\pgfsetrectcap%
\pgfsetroundjoin%
\pgfsetlinewidth{1.505625pt}%
\definecolor{currentstroke}{rgb}{0.121569,0.466667,0.705882}%
\pgfsetstrokecolor{currentstroke}%
\pgfsetdash{}{0pt}%
\pgfpathmoveto{\pgfqpoint{0.828720in}{0.707777in}}%
\pgfpathlineto{\pgfqpoint{0.967608in}{0.707777in}}%
\pgfpathlineto{\pgfqpoint{1.106497in}{0.707777in}}%
\pgfusepath{stroke}%
\end{pgfscope}%
\begin{pgfscope}%
\definecolor{textcolor}{rgb}{0.000000,0.000000,0.000000}%
\pgfsetstrokecolor{textcolor}%
\pgfsetfillcolor{textcolor}%
\pgftext[x=1.217608in,y=0.659166in,left,base]{\color{textcolor}\rmfamily\fontsize{10.000000}{12.000000}\selectfont Area Under Curve = 0.562)}%
\end{pgfscope}%
\end{pgfpicture}%
\makeatother%
\endgroup%

\end{tabular}
\end{center}

\begin{center}
\begin{tabular}{cc}
\begin{tabular}{cc|c|c|}
	&\multicolumn{1}{c}{}& \multicolumn{2}{c}{Prediction} \cr
	&\multicolumn{1}{c}{} & \multicolumn{1}{c}{N} & \multicolumn{1}{c}{P} \cr\cline{3-4}
	\multirow{2}{*}{Actual}&N & 41.1\% & 44.6\% \vrule width 0pt height 10pt depth 2pt \cr\cline{3-4}
	&P & 5.71\% & 8.57\% \vrule width 0pt height 10pt depth 2pt \cr\cline{3-4}
\end{tabular}
&
\begin{tabular}{ll}
0.497 & Accuracy \cr 
0.540 & Balanced Accuracy \cr 
0.161 & Precision \cr 
0.536 & Balanced Precision \cr 
0.600 & Recall \cr 
0.254 & F1 \cr 
0.566 & Balanced F1 \cr 
0.278 & Gmean \cr 	\end{tabular}
\end{tabular}
\end{center}






