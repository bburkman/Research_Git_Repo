%%%
\subsection{Model Building}

To build our model we primarily used
 scikit-learn \citep{scikit-learn} and imbalanced-learn \citep{Imblearn}, and also
Keras/Tensorflow \citep{chollet2015keras} for the Focal Loss model.  

We will explain our methods with examples.  

The histogram below illustrates the kind of results we can realistically hope for in a good model.  The white boxes represent the 150,771 negative samples in the test set, and the black boxes the 26,621 positive samples.  The model evaluates each sample and gives it a probability $p \in (0,1)$ that the sample is in the positive class.  Ideally we want the model to  give most of the negative samples probabilities close to zero, and the positive samples probabilities close to one, but some will be hard to classify correctly.  The ROC curve below shows the median value of $p$ for the negative (0.338) and positive (0.658) classes.  If we choose $p=0.5$ as our threshold, we get the metrics shown.  

\noindent\begin{tabular}{@{}p{0.3\textwidth}@{\hspace{24pt}} p{0.3\textwidth} @{\hspace{24pt}} p{0.3\textwidth}}
  \vspace{0pt} %% Creator: Matplotlib, PGF backend
%%
%% To include the figure in your LaTeX document, write
%%   \input{<filename>.pgf}
%%
%% Make sure the required packages are loaded in your preamble
%%   \usepackage{pgf}
%%
%% Also ensure that all the required font packages are loaded; for instance,
%% the lmodern package is sometimes necessary when using math font.
%%   \usepackage{lmodern}
%%
%% Figures using additional raster images can only be included by \input if
%% they are in the same directory as the main LaTeX file. For loading figures
%% from other directories you can use the `import` package
%%   \usepackage{import}
%%
%% and then include the figures with
%%   \import{<path to file>}{<filename>.pgf}
%%
%% Matplotlib used the following preamble
%%   
%%   \usepackage{fontspec}
%%   \makeatletter\@ifpackageloaded{underscore}{}{\usepackage[strings]{underscore}}\makeatother
%%
\begingroup%
\makeatletter%
\begin{pgfpicture}%
\pgfpathrectangle{\pgfpointorigin}{\pgfqpoint{3.095000in}{3.044944in}}%
\pgfusepath{use as bounding box, clip}%
\begin{pgfscope}%
\pgfsetbuttcap%
\pgfsetmiterjoin%
\definecolor{currentfill}{rgb}{1.000000,1.000000,1.000000}%
\pgfsetfillcolor{currentfill}%
\pgfsetlinewidth{0.000000pt}%
\definecolor{currentstroke}{rgb}{1.000000,1.000000,1.000000}%
\pgfsetstrokecolor{currentstroke}%
\pgfsetdash{}{0pt}%
\pgfpathmoveto{\pgfqpoint{0.000000in}{0.000000in}}%
\pgfpathlineto{\pgfqpoint{3.095000in}{0.000000in}}%
\pgfpathlineto{\pgfqpoint{3.095000in}{3.044944in}}%
\pgfpathlineto{\pgfqpoint{0.000000in}{3.044944in}}%
\pgfpathlineto{\pgfqpoint{0.000000in}{0.000000in}}%
\pgfpathclose%
\pgfusepath{fill}%
\end{pgfscope}%
\begin{pgfscope}%
\pgfsetbuttcap%
\pgfsetmiterjoin%
\definecolor{currentfill}{rgb}{1.000000,1.000000,1.000000}%
\pgfsetfillcolor{currentfill}%
\pgfsetlinewidth{0.000000pt}%
\definecolor{currentstroke}{rgb}{0.000000,0.000000,0.000000}%
\pgfsetstrokecolor{currentstroke}%
\pgfsetstrokeopacity{0.000000}%
\pgfsetdash{}{0pt}%
\pgfpathmoveto{\pgfqpoint{0.515000in}{1.096944in}}%
\pgfpathlineto{\pgfqpoint{2.995000in}{1.096944in}}%
\pgfpathlineto{\pgfqpoint{2.995000in}{2.944944in}}%
\pgfpathlineto{\pgfqpoint{0.515000in}{2.944944in}}%
\pgfpathlineto{\pgfqpoint{0.515000in}{1.096944in}}%
\pgfpathclose%
\pgfusepath{fill}%
\end{pgfscope}%
\begin{pgfscope}%
\pgfpathrectangle{\pgfqpoint{0.515000in}{1.096944in}}{\pgfqpoint{2.480000in}{1.848000in}}%
\pgfusepath{clip}%
\pgfsetbuttcap%
\pgfsetmiterjoin%
\pgfsetlinewidth{1.003750pt}%
\definecolor{currentstroke}{rgb}{0.000000,0.000000,0.000000}%
\pgfsetstrokecolor{currentstroke}%
\pgfsetdash{}{0pt}%
\pgfpathmoveto{\pgfqpoint{0.505000in}{1.096944in}}%
\pgfpathlineto{\pgfqpoint{0.577627in}{1.096944in}}%
\pgfpathlineto{\pgfqpoint{0.577627in}{1.643930in}}%
\pgfpathlineto{\pgfqpoint{0.505000in}{1.643930in}}%
\pgfusepath{stroke}%
\end{pgfscope}%
\begin{pgfscope}%
\pgfpathrectangle{\pgfqpoint{0.515000in}{1.096944in}}{\pgfqpoint{2.480000in}{1.848000in}}%
\pgfusepath{clip}%
\pgfsetbuttcap%
\pgfsetmiterjoin%
\pgfsetlinewidth{1.003750pt}%
\definecolor{currentstroke}{rgb}{0.000000,0.000000,0.000000}%
\pgfsetstrokecolor{currentstroke}%
\pgfsetdash{}{0pt}%
\pgfpathmoveto{\pgfqpoint{0.727930in}{1.096944in}}%
\pgfpathlineto{\pgfqpoint{0.828132in}{1.096944in}}%
\pgfpathlineto{\pgfqpoint{0.828132in}{2.793426in}}%
\pgfpathlineto{\pgfqpoint{0.727930in}{2.793426in}}%
\pgfpathlineto{\pgfqpoint{0.727930in}{1.096944in}}%
\pgfpathclose%
\pgfusepath{stroke}%
\end{pgfscope}%
\begin{pgfscope}%
\pgfpathrectangle{\pgfqpoint{0.515000in}{1.096944in}}{\pgfqpoint{2.480000in}{1.848000in}}%
\pgfusepath{clip}%
\pgfsetbuttcap%
\pgfsetmiterjoin%
\pgfsetlinewidth{1.003750pt}%
\definecolor{currentstroke}{rgb}{0.000000,0.000000,0.000000}%
\pgfsetstrokecolor{currentstroke}%
\pgfsetdash{}{0pt}%
\pgfpathmoveto{\pgfqpoint{0.978435in}{1.096944in}}%
\pgfpathlineto{\pgfqpoint{1.078637in}{1.096944in}}%
\pgfpathlineto{\pgfqpoint{1.078637in}{2.856944in}}%
\pgfpathlineto{\pgfqpoint{0.978435in}{2.856944in}}%
\pgfpathlineto{\pgfqpoint{0.978435in}{1.096944in}}%
\pgfpathclose%
\pgfusepath{stroke}%
\end{pgfscope}%
\begin{pgfscope}%
\pgfpathrectangle{\pgfqpoint{0.515000in}{1.096944in}}{\pgfqpoint{2.480000in}{1.848000in}}%
\pgfusepath{clip}%
\pgfsetbuttcap%
\pgfsetmiterjoin%
\pgfsetlinewidth{1.003750pt}%
\definecolor{currentstroke}{rgb}{0.000000,0.000000,0.000000}%
\pgfsetstrokecolor{currentstroke}%
\pgfsetdash{}{0pt}%
\pgfpathmoveto{\pgfqpoint{1.228940in}{1.096944in}}%
\pgfpathlineto{\pgfqpoint{1.329142in}{1.096944in}}%
\pgfpathlineto{\pgfqpoint{1.329142in}{2.393983in}}%
\pgfpathlineto{\pgfqpoint{1.228940in}{2.393983in}}%
\pgfpathlineto{\pgfqpoint{1.228940in}{1.096944in}}%
\pgfpathclose%
\pgfusepath{stroke}%
\end{pgfscope}%
\begin{pgfscope}%
\pgfpathrectangle{\pgfqpoint{0.515000in}{1.096944in}}{\pgfqpoint{2.480000in}{1.848000in}}%
\pgfusepath{clip}%
\pgfsetbuttcap%
\pgfsetmiterjoin%
\pgfsetlinewidth{1.003750pt}%
\definecolor{currentstroke}{rgb}{0.000000,0.000000,0.000000}%
\pgfsetstrokecolor{currentstroke}%
\pgfsetdash{}{0pt}%
\pgfpathmoveto{\pgfqpoint{1.479445in}{1.096944in}}%
\pgfpathlineto{\pgfqpoint{1.579647in}{1.096944in}}%
\pgfpathlineto{\pgfqpoint{1.579647in}{1.874976in}}%
\pgfpathlineto{\pgfqpoint{1.479445in}{1.874976in}}%
\pgfpathlineto{\pgfqpoint{1.479445in}{1.096944in}}%
\pgfpathclose%
\pgfusepath{stroke}%
\end{pgfscope}%
\begin{pgfscope}%
\pgfpathrectangle{\pgfqpoint{0.515000in}{1.096944in}}{\pgfqpoint{2.480000in}{1.848000in}}%
\pgfusepath{clip}%
\pgfsetbuttcap%
\pgfsetmiterjoin%
\pgfsetlinewidth{1.003750pt}%
\definecolor{currentstroke}{rgb}{0.000000,0.000000,0.000000}%
\pgfsetstrokecolor{currentstroke}%
\pgfsetdash{}{0pt}%
\pgfpathmoveto{\pgfqpoint{1.729950in}{1.096944in}}%
\pgfpathlineto{\pgfqpoint{1.830152in}{1.096944in}}%
\pgfpathlineto{\pgfqpoint{1.830152in}{1.536097in}}%
\pgfpathlineto{\pgfqpoint{1.729950in}{1.536097in}}%
\pgfpathlineto{\pgfqpoint{1.729950in}{1.096944in}}%
\pgfpathclose%
\pgfusepath{stroke}%
\end{pgfscope}%
\begin{pgfscope}%
\pgfpathrectangle{\pgfqpoint{0.515000in}{1.096944in}}{\pgfqpoint{2.480000in}{1.848000in}}%
\pgfusepath{clip}%
\pgfsetbuttcap%
\pgfsetmiterjoin%
\pgfsetlinewidth{1.003750pt}%
\definecolor{currentstroke}{rgb}{0.000000,0.000000,0.000000}%
\pgfsetstrokecolor{currentstroke}%
\pgfsetdash{}{0pt}%
\pgfpathmoveto{\pgfqpoint{1.980455in}{1.096944in}}%
\pgfpathlineto{\pgfqpoint{2.080657in}{1.096944in}}%
\pgfpathlineto{\pgfqpoint{2.080657in}{1.343370in}}%
\pgfpathlineto{\pgfqpoint{1.980455in}{1.343370in}}%
\pgfpathlineto{\pgfqpoint{1.980455in}{1.096944in}}%
\pgfpathclose%
\pgfusepath{stroke}%
\end{pgfscope}%
\begin{pgfscope}%
\pgfpathrectangle{\pgfqpoint{0.515000in}{1.096944in}}{\pgfqpoint{2.480000in}{1.848000in}}%
\pgfusepath{clip}%
\pgfsetbuttcap%
\pgfsetmiterjoin%
\pgfsetlinewidth{1.003750pt}%
\definecolor{currentstroke}{rgb}{0.000000,0.000000,0.000000}%
\pgfsetstrokecolor{currentstroke}%
\pgfsetdash{}{0pt}%
\pgfpathmoveto{\pgfqpoint{2.230960in}{1.096944in}}%
\pgfpathlineto{\pgfqpoint{2.331162in}{1.096944in}}%
\pgfpathlineto{\pgfqpoint{2.331162in}{1.205472in}}%
\pgfpathlineto{\pgfqpoint{2.230960in}{1.205472in}}%
\pgfpathlineto{\pgfqpoint{2.230960in}{1.096944in}}%
\pgfpathclose%
\pgfusepath{stroke}%
\end{pgfscope}%
\begin{pgfscope}%
\pgfpathrectangle{\pgfqpoint{0.515000in}{1.096944in}}{\pgfqpoint{2.480000in}{1.848000in}}%
\pgfusepath{clip}%
\pgfsetbuttcap%
\pgfsetmiterjoin%
\pgfsetlinewidth{1.003750pt}%
\definecolor{currentstroke}{rgb}{0.000000,0.000000,0.000000}%
\pgfsetstrokecolor{currentstroke}%
\pgfsetdash{}{0pt}%
\pgfpathmoveto{\pgfqpoint{2.481465in}{1.096944in}}%
\pgfpathlineto{\pgfqpoint{2.581667in}{1.096944in}}%
\pgfpathlineto{\pgfqpoint{2.581667in}{1.149427in}}%
\pgfpathlineto{\pgfqpoint{2.481465in}{1.149427in}}%
\pgfpathlineto{\pgfqpoint{2.481465in}{1.096944in}}%
\pgfpathclose%
\pgfusepath{stroke}%
\end{pgfscope}%
\begin{pgfscope}%
\pgfpathrectangle{\pgfqpoint{0.515000in}{1.096944in}}{\pgfqpoint{2.480000in}{1.848000in}}%
\pgfusepath{clip}%
\pgfsetbuttcap%
\pgfsetmiterjoin%
\pgfsetlinewidth{1.003750pt}%
\definecolor{currentstroke}{rgb}{0.000000,0.000000,0.000000}%
\pgfsetstrokecolor{currentstroke}%
\pgfsetdash{}{0pt}%
\pgfpathmoveto{\pgfqpoint{2.731970in}{1.096944in}}%
\pgfpathlineto{\pgfqpoint{2.832172in}{1.096944in}}%
\pgfpathlineto{\pgfqpoint{2.832172in}{1.123186in}}%
\pgfpathlineto{\pgfqpoint{2.731970in}{1.123186in}}%
\pgfpathlineto{\pgfqpoint{2.731970in}{1.096944in}}%
\pgfpathclose%
\pgfusepath{stroke}%
\end{pgfscope}%
\begin{pgfscope}%
\pgfpathrectangle{\pgfqpoint{0.515000in}{1.096944in}}{\pgfqpoint{2.480000in}{1.848000in}}%
\pgfusepath{clip}%
\pgfsetbuttcap%
\pgfsetmiterjoin%
\definecolor{currentfill}{rgb}{0.000000,0.000000,0.000000}%
\pgfsetfillcolor{currentfill}%
\pgfsetlinewidth{0.000000pt}%
\definecolor{currentstroke}{rgb}{0.000000,0.000000,0.000000}%
\pgfsetstrokecolor{currentstroke}%
\pgfsetstrokeopacity{0.000000}%
\pgfsetdash{}{0pt}%
\pgfpathmoveto{\pgfqpoint{0.577627in}{1.096944in}}%
\pgfpathlineto{\pgfqpoint{0.677829in}{1.096944in}}%
\pgfpathlineto{\pgfqpoint{0.677829in}{1.103114in}}%
\pgfpathlineto{\pgfqpoint{0.577627in}{1.103114in}}%
\pgfpathlineto{\pgfqpoint{0.577627in}{1.096944in}}%
\pgfpathclose%
\pgfusepath{fill}%
\end{pgfscope}%
\begin{pgfscope}%
\pgfpathrectangle{\pgfqpoint{0.515000in}{1.096944in}}{\pgfqpoint{2.480000in}{1.848000in}}%
\pgfusepath{clip}%
\pgfsetbuttcap%
\pgfsetmiterjoin%
\definecolor{currentfill}{rgb}{0.000000,0.000000,0.000000}%
\pgfsetfillcolor{currentfill}%
\pgfsetlinewidth{0.000000pt}%
\definecolor{currentstroke}{rgb}{0.000000,0.000000,0.000000}%
\pgfsetstrokecolor{currentstroke}%
\pgfsetstrokeopacity{0.000000}%
\pgfsetdash{}{0pt}%
\pgfpathmoveto{\pgfqpoint{0.828132in}{1.096944in}}%
\pgfpathlineto{\pgfqpoint{0.928334in}{1.096944in}}%
\pgfpathlineto{\pgfqpoint{0.928334in}{1.108761in}}%
\pgfpathlineto{\pgfqpoint{0.828132in}{1.108761in}}%
\pgfpathlineto{\pgfqpoint{0.828132in}{1.096944in}}%
\pgfpathclose%
\pgfusepath{fill}%
\end{pgfscope}%
\begin{pgfscope}%
\pgfpathrectangle{\pgfqpoint{0.515000in}{1.096944in}}{\pgfqpoint{2.480000in}{1.848000in}}%
\pgfusepath{clip}%
\pgfsetbuttcap%
\pgfsetmiterjoin%
\definecolor{currentfill}{rgb}{0.000000,0.000000,0.000000}%
\pgfsetfillcolor{currentfill}%
\pgfsetlinewidth{0.000000pt}%
\definecolor{currentstroke}{rgb}{0.000000,0.000000,0.000000}%
\pgfsetstrokecolor{currentstroke}%
\pgfsetstrokeopacity{0.000000}%
\pgfsetdash{}{0pt}%
\pgfpathmoveto{\pgfqpoint{1.078637in}{1.096944in}}%
\pgfpathlineto{\pgfqpoint{1.178839in}{1.096944in}}%
\pgfpathlineto{\pgfqpoint{1.178839in}{1.126835in}}%
\pgfpathlineto{\pgfqpoint{1.078637in}{1.126835in}}%
\pgfpathlineto{\pgfqpoint{1.078637in}{1.096944in}}%
\pgfpathclose%
\pgfusepath{fill}%
\end{pgfscope}%
\begin{pgfscope}%
\pgfpathrectangle{\pgfqpoint{0.515000in}{1.096944in}}{\pgfqpoint{2.480000in}{1.848000in}}%
\pgfusepath{clip}%
\pgfsetbuttcap%
\pgfsetmiterjoin%
\definecolor{currentfill}{rgb}{0.000000,0.000000,0.000000}%
\pgfsetfillcolor{currentfill}%
\pgfsetlinewidth{0.000000pt}%
\definecolor{currentstroke}{rgb}{0.000000,0.000000,0.000000}%
\pgfsetstrokecolor{currentstroke}%
\pgfsetstrokeopacity{0.000000}%
\pgfsetdash{}{0pt}%
\pgfpathmoveto{\pgfqpoint{1.329142in}{1.096944in}}%
\pgfpathlineto{\pgfqpoint{1.429344in}{1.096944in}}%
\pgfpathlineto{\pgfqpoint{1.429344in}{1.150557in}}%
\pgfpathlineto{\pgfqpoint{1.329142in}{1.150557in}}%
\pgfpathlineto{\pgfqpoint{1.329142in}{1.096944in}}%
\pgfpathclose%
\pgfusepath{fill}%
\end{pgfscope}%
\begin{pgfscope}%
\pgfpathrectangle{\pgfqpoint{0.515000in}{1.096944in}}{\pgfqpoint{2.480000in}{1.848000in}}%
\pgfusepath{clip}%
\pgfsetbuttcap%
\pgfsetmiterjoin%
\definecolor{currentfill}{rgb}{0.000000,0.000000,0.000000}%
\pgfsetfillcolor{currentfill}%
\pgfsetlinewidth{0.000000pt}%
\definecolor{currentstroke}{rgb}{0.000000,0.000000,0.000000}%
\pgfsetstrokecolor{currentstroke}%
\pgfsetstrokeopacity{0.000000}%
\pgfsetdash{}{0pt}%
\pgfpathmoveto{\pgfqpoint{1.579647in}{1.096944in}}%
\pgfpathlineto{\pgfqpoint{1.679849in}{1.096944in}}%
\pgfpathlineto{\pgfqpoint{1.679849in}{1.207645in}}%
\pgfpathlineto{\pgfqpoint{1.579647in}{1.207645in}}%
\pgfpathlineto{\pgfqpoint{1.579647in}{1.096944in}}%
\pgfpathclose%
\pgfusepath{fill}%
\end{pgfscope}%
\begin{pgfscope}%
\pgfpathrectangle{\pgfqpoint{0.515000in}{1.096944in}}{\pgfqpoint{2.480000in}{1.848000in}}%
\pgfusepath{clip}%
\pgfsetbuttcap%
\pgfsetmiterjoin%
\definecolor{currentfill}{rgb}{0.000000,0.000000,0.000000}%
\pgfsetfillcolor{currentfill}%
\pgfsetlinewidth{0.000000pt}%
\definecolor{currentstroke}{rgb}{0.000000,0.000000,0.000000}%
\pgfsetstrokecolor{currentstroke}%
\pgfsetstrokeopacity{0.000000}%
\pgfsetdash{}{0pt}%
\pgfpathmoveto{\pgfqpoint{1.830152in}{1.096944in}}%
\pgfpathlineto{\pgfqpoint{1.930354in}{1.096944in}}%
\pgfpathlineto{\pgfqpoint{1.930354in}{1.295493in}}%
\pgfpathlineto{\pgfqpoint{1.830152in}{1.295493in}}%
\pgfpathlineto{\pgfqpoint{1.830152in}{1.096944in}}%
\pgfpathclose%
\pgfusepath{fill}%
\end{pgfscope}%
\begin{pgfscope}%
\pgfpathrectangle{\pgfqpoint{0.515000in}{1.096944in}}{\pgfqpoint{2.480000in}{1.848000in}}%
\pgfusepath{clip}%
\pgfsetbuttcap%
\pgfsetmiterjoin%
\definecolor{currentfill}{rgb}{0.000000,0.000000,0.000000}%
\pgfsetfillcolor{currentfill}%
\pgfsetlinewidth{0.000000pt}%
\definecolor{currentstroke}{rgb}{0.000000,0.000000,0.000000}%
\pgfsetstrokecolor{currentstroke}%
\pgfsetstrokeopacity{0.000000}%
\pgfsetdash{}{0pt}%
\pgfpathmoveto{\pgfqpoint{2.080657in}{1.096944in}}%
\pgfpathlineto{\pgfqpoint{2.180859in}{1.096944in}}%
\pgfpathlineto{\pgfqpoint{2.180859in}{1.417750in}}%
\pgfpathlineto{\pgfqpoint{2.080657in}{1.417750in}}%
\pgfpathlineto{\pgfqpoint{2.080657in}{1.096944in}}%
\pgfpathclose%
\pgfusepath{fill}%
\end{pgfscope}%
\begin{pgfscope}%
\pgfpathrectangle{\pgfqpoint{0.515000in}{1.096944in}}{\pgfqpoint{2.480000in}{1.848000in}}%
\pgfusepath{clip}%
\pgfsetbuttcap%
\pgfsetmiterjoin%
\definecolor{currentfill}{rgb}{0.000000,0.000000,0.000000}%
\pgfsetfillcolor{currentfill}%
\pgfsetlinewidth{0.000000pt}%
\definecolor{currentstroke}{rgb}{0.000000,0.000000,0.000000}%
\pgfsetstrokecolor{currentstroke}%
\pgfsetstrokeopacity{0.000000}%
\pgfsetdash{}{0pt}%
\pgfpathmoveto{\pgfqpoint{2.331162in}{1.096944in}}%
\pgfpathlineto{\pgfqpoint{2.431364in}{1.096944in}}%
\pgfpathlineto{\pgfqpoint{2.431364in}{1.542440in}}%
\pgfpathlineto{\pgfqpoint{2.331162in}{1.542440in}}%
\pgfpathlineto{\pgfqpoint{2.331162in}{1.096944in}}%
\pgfpathclose%
\pgfusepath{fill}%
\end{pgfscope}%
\begin{pgfscope}%
\pgfpathrectangle{\pgfqpoint{0.515000in}{1.096944in}}{\pgfqpoint{2.480000in}{1.848000in}}%
\pgfusepath{clip}%
\pgfsetbuttcap%
\pgfsetmiterjoin%
\definecolor{currentfill}{rgb}{0.000000,0.000000,0.000000}%
\pgfsetfillcolor{currentfill}%
\pgfsetlinewidth{0.000000pt}%
\definecolor{currentstroke}{rgb}{0.000000,0.000000,0.000000}%
\pgfsetstrokecolor{currentstroke}%
\pgfsetstrokeopacity{0.000000}%
\pgfsetdash{}{0pt}%
\pgfpathmoveto{\pgfqpoint{2.581667in}{1.096944in}}%
\pgfpathlineto{\pgfqpoint{2.681869in}{1.096944in}}%
\pgfpathlineto{\pgfqpoint{2.681869in}{1.515504in}}%
\pgfpathlineto{\pgfqpoint{2.581667in}{1.515504in}}%
\pgfpathlineto{\pgfqpoint{2.581667in}{1.096944in}}%
\pgfpathclose%
\pgfusepath{fill}%
\end{pgfscope}%
\begin{pgfscope}%
\pgfpathrectangle{\pgfqpoint{0.515000in}{1.096944in}}{\pgfqpoint{2.480000in}{1.848000in}}%
\pgfusepath{clip}%
\pgfsetbuttcap%
\pgfsetmiterjoin%
\definecolor{currentfill}{rgb}{0.000000,0.000000,0.000000}%
\pgfsetfillcolor{currentfill}%
\pgfsetlinewidth{0.000000pt}%
\definecolor{currentstroke}{rgb}{0.000000,0.000000,0.000000}%
\pgfsetstrokecolor{currentstroke}%
\pgfsetstrokeopacity{0.000000}%
\pgfsetdash{}{0pt}%
\pgfpathmoveto{\pgfqpoint{2.832172in}{1.096944in}}%
\pgfpathlineto{\pgfqpoint{2.932374in}{1.096944in}}%
\pgfpathlineto{\pgfqpoint{2.932374in}{1.239187in}}%
\pgfpathlineto{\pgfqpoint{2.832172in}{1.239187in}}%
\pgfpathlineto{\pgfqpoint{2.832172in}{1.096944in}}%
\pgfpathclose%
\pgfusepath{fill}%
\end{pgfscope}%
\begin{pgfscope}%
\pgfsetbuttcap%
\pgfsetroundjoin%
\definecolor{currentfill}{rgb}{0.000000,0.000000,0.000000}%
\pgfsetfillcolor{currentfill}%
\pgfsetlinewidth{0.803000pt}%
\definecolor{currentstroke}{rgb}{0.000000,0.000000,0.000000}%
\pgfsetstrokecolor{currentstroke}%
\pgfsetdash{}{0pt}%
\pgfsys@defobject{currentmarker}{\pgfqpoint{0.000000in}{-0.048611in}}{\pgfqpoint{0.000000in}{0.000000in}}{%
\pgfpathmoveto{\pgfqpoint{0.000000in}{0.000000in}}%
\pgfpathlineto{\pgfqpoint{0.000000in}{-0.048611in}}%
\pgfusepath{stroke,fill}%
}%
\begin{pgfscope}%
\pgfsys@transformshift{0.577627in}{1.096944in}%
\pgfsys@useobject{currentmarker}{}%
\end{pgfscope}%
\end{pgfscope}%
\begin{pgfscope}%
\definecolor{textcolor}{rgb}{0.000000,0.000000,0.000000}%
\pgfsetstrokecolor{textcolor}%
\pgfsetfillcolor{textcolor}%
\pgftext[x=0.612349in, y=0.282083in, left, base,rotate=90.000000]{\color{textcolor}\rmfamily\fontsize{10.000000}{12.000000}\selectfont (-0.001, 0.1]}%
\end{pgfscope}%
\begin{pgfscope}%
\pgfsetbuttcap%
\pgfsetroundjoin%
\definecolor{currentfill}{rgb}{0.000000,0.000000,0.000000}%
\pgfsetfillcolor{currentfill}%
\pgfsetlinewidth{0.803000pt}%
\definecolor{currentstroke}{rgb}{0.000000,0.000000,0.000000}%
\pgfsetstrokecolor{currentstroke}%
\pgfsetdash{}{0pt}%
\pgfsys@defobject{currentmarker}{\pgfqpoint{0.000000in}{-0.048611in}}{\pgfqpoint{0.000000in}{0.000000in}}{%
\pgfpathmoveto{\pgfqpoint{0.000000in}{0.000000in}}%
\pgfpathlineto{\pgfqpoint{0.000000in}{-0.048611in}}%
\pgfusepath{stroke,fill}%
}%
\begin{pgfscope}%
\pgfsys@transformshift{0.828132in}{1.096944in}%
\pgfsys@useobject{currentmarker}{}%
\end{pgfscope}%
\end{pgfscope}%
\begin{pgfscope}%
\definecolor{textcolor}{rgb}{0.000000,0.000000,0.000000}%
\pgfsetstrokecolor{textcolor}%
\pgfsetfillcolor{textcolor}%
\pgftext[x=0.862854in, y=0.467222in, left, base,rotate=90.000000]{\color{textcolor}\rmfamily\fontsize{10.000000}{12.000000}\selectfont (0.1, 0.2]}%
\end{pgfscope}%
\begin{pgfscope}%
\pgfsetbuttcap%
\pgfsetroundjoin%
\definecolor{currentfill}{rgb}{0.000000,0.000000,0.000000}%
\pgfsetfillcolor{currentfill}%
\pgfsetlinewidth{0.803000pt}%
\definecolor{currentstroke}{rgb}{0.000000,0.000000,0.000000}%
\pgfsetstrokecolor{currentstroke}%
\pgfsetdash{}{0pt}%
\pgfsys@defobject{currentmarker}{\pgfqpoint{0.000000in}{-0.048611in}}{\pgfqpoint{0.000000in}{0.000000in}}{%
\pgfpathmoveto{\pgfqpoint{0.000000in}{0.000000in}}%
\pgfpathlineto{\pgfqpoint{0.000000in}{-0.048611in}}%
\pgfusepath{stroke,fill}%
}%
\begin{pgfscope}%
\pgfsys@transformshift{1.078637in}{1.096944in}%
\pgfsys@useobject{currentmarker}{}%
\end{pgfscope}%
\end{pgfscope}%
\begin{pgfscope}%
\definecolor{textcolor}{rgb}{0.000000,0.000000,0.000000}%
\pgfsetstrokecolor{textcolor}%
\pgfsetfillcolor{textcolor}%
\pgftext[x=1.113359in, y=0.467222in, left, base,rotate=90.000000]{\color{textcolor}\rmfamily\fontsize{10.000000}{12.000000}\selectfont (0.2, 0.3]}%
\end{pgfscope}%
\begin{pgfscope}%
\pgfsetbuttcap%
\pgfsetroundjoin%
\definecolor{currentfill}{rgb}{0.000000,0.000000,0.000000}%
\pgfsetfillcolor{currentfill}%
\pgfsetlinewidth{0.803000pt}%
\definecolor{currentstroke}{rgb}{0.000000,0.000000,0.000000}%
\pgfsetstrokecolor{currentstroke}%
\pgfsetdash{}{0pt}%
\pgfsys@defobject{currentmarker}{\pgfqpoint{0.000000in}{-0.048611in}}{\pgfqpoint{0.000000in}{0.000000in}}{%
\pgfpathmoveto{\pgfqpoint{0.000000in}{0.000000in}}%
\pgfpathlineto{\pgfqpoint{0.000000in}{-0.048611in}}%
\pgfusepath{stroke,fill}%
}%
\begin{pgfscope}%
\pgfsys@transformshift{1.329142in}{1.096944in}%
\pgfsys@useobject{currentmarker}{}%
\end{pgfscope}%
\end{pgfscope}%
\begin{pgfscope}%
\definecolor{textcolor}{rgb}{0.000000,0.000000,0.000000}%
\pgfsetstrokecolor{textcolor}%
\pgfsetfillcolor{textcolor}%
\pgftext[x=1.363864in, y=0.467222in, left, base,rotate=90.000000]{\color{textcolor}\rmfamily\fontsize{10.000000}{12.000000}\selectfont (0.3, 0.4]}%
\end{pgfscope}%
\begin{pgfscope}%
\pgfsetbuttcap%
\pgfsetroundjoin%
\definecolor{currentfill}{rgb}{0.000000,0.000000,0.000000}%
\pgfsetfillcolor{currentfill}%
\pgfsetlinewidth{0.803000pt}%
\definecolor{currentstroke}{rgb}{0.000000,0.000000,0.000000}%
\pgfsetstrokecolor{currentstroke}%
\pgfsetdash{}{0pt}%
\pgfsys@defobject{currentmarker}{\pgfqpoint{0.000000in}{-0.048611in}}{\pgfqpoint{0.000000in}{0.000000in}}{%
\pgfpathmoveto{\pgfqpoint{0.000000in}{0.000000in}}%
\pgfpathlineto{\pgfqpoint{0.000000in}{-0.048611in}}%
\pgfusepath{stroke,fill}%
}%
\begin{pgfscope}%
\pgfsys@transformshift{1.579647in}{1.096944in}%
\pgfsys@useobject{currentmarker}{}%
\end{pgfscope}%
\end{pgfscope}%
\begin{pgfscope}%
\definecolor{textcolor}{rgb}{0.000000,0.000000,0.000000}%
\pgfsetstrokecolor{textcolor}%
\pgfsetfillcolor{textcolor}%
\pgftext[x=1.614369in, y=0.467222in, left, base,rotate=90.000000]{\color{textcolor}\rmfamily\fontsize{10.000000}{12.000000}\selectfont (0.4, 0.5]}%
\end{pgfscope}%
\begin{pgfscope}%
\pgfsetbuttcap%
\pgfsetroundjoin%
\definecolor{currentfill}{rgb}{0.000000,0.000000,0.000000}%
\pgfsetfillcolor{currentfill}%
\pgfsetlinewidth{0.803000pt}%
\definecolor{currentstroke}{rgb}{0.000000,0.000000,0.000000}%
\pgfsetstrokecolor{currentstroke}%
\pgfsetdash{}{0pt}%
\pgfsys@defobject{currentmarker}{\pgfqpoint{0.000000in}{-0.048611in}}{\pgfqpoint{0.000000in}{0.000000in}}{%
\pgfpathmoveto{\pgfqpoint{0.000000in}{0.000000in}}%
\pgfpathlineto{\pgfqpoint{0.000000in}{-0.048611in}}%
\pgfusepath{stroke,fill}%
}%
\begin{pgfscope}%
\pgfsys@transformshift{1.830152in}{1.096944in}%
\pgfsys@useobject{currentmarker}{}%
\end{pgfscope}%
\end{pgfscope}%
\begin{pgfscope}%
\definecolor{textcolor}{rgb}{0.000000,0.000000,0.000000}%
\pgfsetstrokecolor{textcolor}%
\pgfsetfillcolor{textcolor}%
\pgftext[x=1.864874in, y=0.467222in, left, base,rotate=90.000000]{\color{textcolor}\rmfamily\fontsize{10.000000}{12.000000}\selectfont (0.5, 0.6]}%
\end{pgfscope}%
\begin{pgfscope}%
\pgfsetbuttcap%
\pgfsetroundjoin%
\definecolor{currentfill}{rgb}{0.000000,0.000000,0.000000}%
\pgfsetfillcolor{currentfill}%
\pgfsetlinewidth{0.803000pt}%
\definecolor{currentstroke}{rgb}{0.000000,0.000000,0.000000}%
\pgfsetstrokecolor{currentstroke}%
\pgfsetdash{}{0pt}%
\pgfsys@defobject{currentmarker}{\pgfqpoint{0.000000in}{-0.048611in}}{\pgfqpoint{0.000000in}{0.000000in}}{%
\pgfpathmoveto{\pgfqpoint{0.000000in}{0.000000in}}%
\pgfpathlineto{\pgfqpoint{0.000000in}{-0.048611in}}%
\pgfusepath{stroke,fill}%
}%
\begin{pgfscope}%
\pgfsys@transformshift{2.080657in}{1.096944in}%
\pgfsys@useobject{currentmarker}{}%
\end{pgfscope}%
\end{pgfscope}%
\begin{pgfscope}%
\definecolor{textcolor}{rgb}{0.000000,0.000000,0.000000}%
\pgfsetstrokecolor{textcolor}%
\pgfsetfillcolor{textcolor}%
\pgftext[x=2.115379in, y=0.467222in, left, base,rotate=90.000000]{\color{textcolor}\rmfamily\fontsize{10.000000}{12.000000}\selectfont (0.6, 0.7]}%
\end{pgfscope}%
\begin{pgfscope}%
\pgfsetbuttcap%
\pgfsetroundjoin%
\definecolor{currentfill}{rgb}{0.000000,0.000000,0.000000}%
\pgfsetfillcolor{currentfill}%
\pgfsetlinewidth{0.803000pt}%
\definecolor{currentstroke}{rgb}{0.000000,0.000000,0.000000}%
\pgfsetstrokecolor{currentstroke}%
\pgfsetdash{}{0pt}%
\pgfsys@defobject{currentmarker}{\pgfqpoint{0.000000in}{-0.048611in}}{\pgfqpoint{0.000000in}{0.000000in}}{%
\pgfpathmoveto{\pgfqpoint{0.000000in}{0.000000in}}%
\pgfpathlineto{\pgfqpoint{0.000000in}{-0.048611in}}%
\pgfusepath{stroke,fill}%
}%
\begin{pgfscope}%
\pgfsys@transformshift{2.331162in}{1.096944in}%
\pgfsys@useobject{currentmarker}{}%
\end{pgfscope}%
\end{pgfscope}%
\begin{pgfscope}%
\definecolor{textcolor}{rgb}{0.000000,0.000000,0.000000}%
\pgfsetstrokecolor{textcolor}%
\pgfsetfillcolor{textcolor}%
\pgftext[x=2.365884in, y=0.467222in, left, base,rotate=90.000000]{\color{textcolor}\rmfamily\fontsize{10.000000}{12.000000}\selectfont (0.7, 0.8]}%
\end{pgfscope}%
\begin{pgfscope}%
\pgfsetbuttcap%
\pgfsetroundjoin%
\definecolor{currentfill}{rgb}{0.000000,0.000000,0.000000}%
\pgfsetfillcolor{currentfill}%
\pgfsetlinewidth{0.803000pt}%
\definecolor{currentstroke}{rgb}{0.000000,0.000000,0.000000}%
\pgfsetstrokecolor{currentstroke}%
\pgfsetdash{}{0pt}%
\pgfsys@defobject{currentmarker}{\pgfqpoint{0.000000in}{-0.048611in}}{\pgfqpoint{0.000000in}{0.000000in}}{%
\pgfpathmoveto{\pgfqpoint{0.000000in}{0.000000in}}%
\pgfpathlineto{\pgfqpoint{0.000000in}{-0.048611in}}%
\pgfusepath{stroke,fill}%
}%
\begin{pgfscope}%
\pgfsys@transformshift{2.581667in}{1.096944in}%
\pgfsys@useobject{currentmarker}{}%
\end{pgfscope}%
\end{pgfscope}%
\begin{pgfscope}%
\definecolor{textcolor}{rgb}{0.000000,0.000000,0.000000}%
\pgfsetstrokecolor{textcolor}%
\pgfsetfillcolor{textcolor}%
\pgftext[x=2.616389in, y=0.467222in, left, base,rotate=90.000000]{\color{textcolor}\rmfamily\fontsize{10.000000}{12.000000}\selectfont (0.8, 0.9]}%
\end{pgfscope}%
\begin{pgfscope}%
\pgfsetbuttcap%
\pgfsetroundjoin%
\definecolor{currentfill}{rgb}{0.000000,0.000000,0.000000}%
\pgfsetfillcolor{currentfill}%
\pgfsetlinewidth{0.803000pt}%
\definecolor{currentstroke}{rgb}{0.000000,0.000000,0.000000}%
\pgfsetstrokecolor{currentstroke}%
\pgfsetdash{}{0pt}%
\pgfsys@defobject{currentmarker}{\pgfqpoint{0.000000in}{-0.048611in}}{\pgfqpoint{0.000000in}{0.000000in}}{%
\pgfpathmoveto{\pgfqpoint{0.000000in}{0.000000in}}%
\pgfpathlineto{\pgfqpoint{0.000000in}{-0.048611in}}%
\pgfusepath{stroke,fill}%
}%
\begin{pgfscope}%
\pgfsys@transformshift{2.832172in}{1.096944in}%
\pgfsys@useobject{currentmarker}{}%
\end{pgfscope}%
\end{pgfscope}%
\begin{pgfscope}%
\definecolor{textcolor}{rgb}{0.000000,0.000000,0.000000}%
\pgfsetstrokecolor{textcolor}%
\pgfsetfillcolor{textcolor}%
\pgftext[x=2.866894in, y=0.467222in, left, base,rotate=90.000000]{\color{textcolor}\rmfamily\fontsize{10.000000}{12.000000}\selectfont (0.9, 1.0]}%
\end{pgfscope}%
\begin{pgfscope}%
\definecolor{textcolor}{rgb}{0.000000,0.000000,0.000000}%
\pgfsetstrokecolor{textcolor}%
\pgfsetfillcolor{textcolor}%
\pgftext[x=1.755000in,y=0.226527in,,top]{\color{textcolor}\rmfamily\fontsize{10.000000}{12.000000}\selectfont Range of Prediction}%
\end{pgfscope}%
\begin{pgfscope}%
\pgfsetbuttcap%
\pgfsetroundjoin%
\definecolor{currentfill}{rgb}{0.000000,0.000000,0.000000}%
\pgfsetfillcolor{currentfill}%
\pgfsetlinewidth{0.803000pt}%
\definecolor{currentstroke}{rgb}{0.000000,0.000000,0.000000}%
\pgfsetstrokecolor{currentstroke}%
\pgfsetdash{}{0pt}%
\pgfsys@defobject{currentmarker}{\pgfqpoint{-0.048611in}{0.000000in}}{\pgfqpoint{-0.000000in}{0.000000in}}{%
\pgfpathmoveto{\pgfqpoint{-0.000000in}{0.000000in}}%
\pgfpathlineto{\pgfqpoint{-0.048611in}{0.000000in}}%
\pgfusepath{stroke,fill}%
}%
\begin{pgfscope}%
\pgfsys@transformshift{0.515000in}{1.096944in}%
\pgfsys@useobject{currentmarker}{}%
\end{pgfscope}%
\end{pgfscope}%
\begin{pgfscope}%
\definecolor{textcolor}{rgb}{0.000000,0.000000,0.000000}%
\pgfsetstrokecolor{textcolor}%
\pgfsetfillcolor{textcolor}%
\pgftext[x=0.348333in, y=1.048750in, left, base]{\color{textcolor}\rmfamily\fontsize{10.000000}{12.000000}\selectfont \(\displaystyle {0}\)}%
\end{pgfscope}%
\begin{pgfscope}%
\pgfsetbuttcap%
\pgfsetroundjoin%
\definecolor{currentfill}{rgb}{0.000000,0.000000,0.000000}%
\pgfsetfillcolor{currentfill}%
\pgfsetlinewidth{0.803000pt}%
\definecolor{currentstroke}{rgb}{0.000000,0.000000,0.000000}%
\pgfsetstrokecolor{currentstroke}%
\pgfsetdash{}{0pt}%
\pgfsys@defobject{currentmarker}{\pgfqpoint{-0.048611in}{0.000000in}}{\pgfqpoint{-0.000000in}{0.000000in}}{%
\pgfpathmoveto{\pgfqpoint{-0.000000in}{0.000000in}}%
\pgfpathlineto{\pgfqpoint{-0.048611in}{0.000000in}}%
\pgfusepath{stroke,fill}%
}%
\begin{pgfscope}%
\pgfsys@transformshift{0.515000in}{1.531405in}%
\pgfsys@useobject{currentmarker}{}%
\end{pgfscope}%
\end{pgfscope}%
\begin{pgfscope}%
\definecolor{textcolor}{rgb}{0.000000,0.000000,0.000000}%
\pgfsetstrokecolor{textcolor}%
\pgfsetfillcolor{textcolor}%
\pgftext[x=0.348333in, y=1.483210in, left, base]{\color{textcolor}\rmfamily\fontsize{10.000000}{12.000000}\selectfont \(\displaystyle {5}\)}%
\end{pgfscope}%
\begin{pgfscope}%
\pgfsetbuttcap%
\pgfsetroundjoin%
\definecolor{currentfill}{rgb}{0.000000,0.000000,0.000000}%
\pgfsetfillcolor{currentfill}%
\pgfsetlinewidth{0.803000pt}%
\definecolor{currentstroke}{rgb}{0.000000,0.000000,0.000000}%
\pgfsetstrokecolor{currentstroke}%
\pgfsetdash{}{0pt}%
\pgfsys@defobject{currentmarker}{\pgfqpoint{-0.048611in}{0.000000in}}{\pgfqpoint{-0.000000in}{0.000000in}}{%
\pgfpathmoveto{\pgfqpoint{-0.000000in}{0.000000in}}%
\pgfpathlineto{\pgfqpoint{-0.048611in}{0.000000in}}%
\pgfusepath{stroke,fill}%
}%
\begin{pgfscope}%
\pgfsys@transformshift{0.515000in}{1.965865in}%
\pgfsys@useobject{currentmarker}{}%
\end{pgfscope}%
\end{pgfscope}%
\begin{pgfscope}%
\definecolor{textcolor}{rgb}{0.000000,0.000000,0.000000}%
\pgfsetstrokecolor{textcolor}%
\pgfsetfillcolor{textcolor}%
\pgftext[x=0.278889in, y=1.917671in, left, base]{\color{textcolor}\rmfamily\fontsize{10.000000}{12.000000}\selectfont \(\displaystyle {10}\)}%
\end{pgfscope}%
\begin{pgfscope}%
\pgfsetbuttcap%
\pgfsetroundjoin%
\definecolor{currentfill}{rgb}{0.000000,0.000000,0.000000}%
\pgfsetfillcolor{currentfill}%
\pgfsetlinewidth{0.803000pt}%
\definecolor{currentstroke}{rgb}{0.000000,0.000000,0.000000}%
\pgfsetstrokecolor{currentstroke}%
\pgfsetdash{}{0pt}%
\pgfsys@defobject{currentmarker}{\pgfqpoint{-0.048611in}{0.000000in}}{\pgfqpoint{-0.000000in}{0.000000in}}{%
\pgfpathmoveto{\pgfqpoint{-0.000000in}{0.000000in}}%
\pgfpathlineto{\pgfqpoint{-0.048611in}{0.000000in}}%
\pgfusepath{stroke,fill}%
}%
\begin{pgfscope}%
\pgfsys@transformshift{0.515000in}{2.400326in}%
\pgfsys@useobject{currentmarker}{}%
\end{pgfscope}%
\end{pgfscope}%
\begin{pgfscope}%
\definecolor{textcolor}{rgb}{0.000000,0.000000,0.000000}%
\pgfsetstrokecolor{textcolor}%
\pgfsetfillcolor{textcolor}%
\pgftext[x=0.278889in, y=2.352132in, left, base]{\color{textcolor}\rmfamily\fontsize{10.000000}{12.000000}\selectfont \(\displaystyle {15}\)}%
\end{pgfscope}%
\begin{pgfscope}%
\pgfsetbuttcap%
\pgfsetroundjoin%
\definecolor{currentfill}{rgb}{0.000000,0.000000,0.000000}%
\pgfsetfillcolor{currentfill}%
\pgfsetlinewidth{0.803000pt}%
\definecolor{currentstroke}{rgb}{0.000000,0.000000,0.000000}%
\pgfsetstrokecolor{currentstroke}%
\pgfsetdash{}{0pt}%
\pgfsys@defobject{currentmarker}{\pgfqpoint{-0.048611in}{0.000000in}}{\pgfqpoint{-0.000000in}{0.000000in}}{%
\pgfpathmoveto{\pgfqpoint{-0.000000in}{0.000000in}}%
\pgfpathlineto{\pgfqpoint{-0.048611in}{0.000000in}}%
\pgfusepath{stroke,fill}%
}%
\begin{pgfscope}%
\pgfsys@transformshift{0.515000in}{2.834787in}%
\pgfsys@useobject{currentmarker}{}%
\end{pgfscope}%
\end{pgfscope}%
\begin{pgfscope}%
\definecolor{textcolor}{rgb}{0.000000,0.000000,0.000000}%
\pgfsetstrokecolor{textcolor}%
\pgfsetfillcolor{textcolor}%
\pgftext[x=0.278889in, y=2.786592in, left, base]{\color{textcolor}\rmfamily\fontsize{10.000000}{12.000000}\selectfont \(\displaystyle {20}\)}%
\end{pgfscope}%
\begin{pgfscope}%
\definecolor{textcolor}{rgb}{0.000000,0.000000,0.000000}%
\pgfsetstrokecolor{textcolor}%
\pgfsetfillcolor{textcolor}%
\pgftext[x=0.223333in,y=2.020944in,,bottom,rotate=90.000000]{\color{textcolor}\rmfamily\fontsize{10.000000}{12.000000}\selectfont Percent of Data Set}%
\end{pgfscope}%
\begin{pgfscope}%
\pgfsetrectcap%
\pgfsetmiterjoin%
\pgfsetlinewidth{0.803000pt}%
\definecolor{currentstroke}{rgb}{0.000000,0.000000,0.000000}%
\pgfsetstrokecolor{currentstroke}%
\pgfsetdash{}{0pt}%
\pgfpathmoveto{\pgfqpoint{0.515000in}{1.096944in}}%
\pgfpathlineto{\pgfqpoint{0.515000in}{2.944944in}}%
\pgfusepath{stroke}%
\end{pgfscope}%
\begin{pgfscope}%
\pgfsetrectcap%
\pgfsetmiterjoin%
\pgfsetlinewidth{0.803000pt}%
\definecolor{currentstroke}{rgb}{0.000000,0.000000,0.000000}%
\pgfsetstrokecolor{currentstroke}%
\pgfsetdash{}{0pt}%
\pgfpathmoveto{\pgfqpoint{2.995000in}{1.096944in}}%
\pgfpathlineto{\pgfqpoint{2.995000in}{2.944944in}}%
\pgfusepath{stroke}%
\end{pgfscope}%
\begin{pgfscope}%
\pgfsetrectcap%
\pgfsetmiterjoin%
\pgfsetlinewidth{0.803000pt}%
\definecolor{currentstroke}{rgb}{0.000000,0.000000,0.000000}%
\pgfsetstrokecolor{currentstroke}%
\pgfsetdash{}{0pt}%
\pgfpathmoveto{\pgfqpoint{0.515000in}{1.096944in}}%
\pgfpathlineto{\pgfqpoint{2.995000in}{1.096944in}}%
\pgfusepath{stroke}%
\end{pgfscope}%
\begin{pgfscope}%
\pgfsetrectcap%
\pgfsetmiterjoin%
\pgfsetlinewidth{0.803000pt}%
\definecolor{currentstroke}{rgb}{0.000000,0.000000,0.000000}%
\pgfsetstrokecolor{currentstroke}%
\pgfsetdash{}{0pt}%
\pgfpathmoveto{\pgfqpoint{0.515000in}{2.944944in}}%
\pgfpathlineto{\pgfqpoint{2.995000in}{2.944944in}}%
\pgfusepath{stroke}%
\end{pgfscope}%
\begin{pgfscope}%
\pgfsetbuttcap%
\pgfsetmiterjoin%
\definecolor{currentfill}{rgb}{1.000000,1.000000,1.000000}%
\pgfsetfillcolor{currentfill}%
\pgfsetfillopacity{0.800000}%
\pgfsetlinewidth{1.003750pt}%
\definecolor{currentstroke}{rgb}{0.800000,0.800000,0.800000}%
\pgfsetstrokecolor{currentstroke}%
\pgfsetstrokeopacity{0.800000}%
\pgfsetdash{}{0pt}%
\pgfpathmoveto{\pgfqpoint{1.560833in}{2.444250in}}%
\pgfpathlineto{\pgfqpoint{2.897778in}{2.444250in}}%
\pgfpathquadraticcurveto{\pgfqpoint{2.925556in}{2.444250in}}{\pgfqpoint{2.925556in}{2.472028in}}%
\pgfpathlineto{\pgfqpoint{2.925556in}{2.847722in}}%
\pgfpathquadraticcurveto{\pgfqpoint{2.925556in}{2.875500in}}{\pgfqpoint{2.897778in}{2.875500in}}%
\pgfpathlineto{\pgfqpoint{1.560833in}{2.875500in}}%
\pgfpathquadraticcurveto{\pgfqpoint{1.533056in}{2.875500in}}{\pgfqpoint{1.533056in}{2.847722in}}%
\pgfpathlineto{\pgfqpoint{1.533056in}{2.472028in}}%
\pgfpathquadraticcurveto{\pgfqpoint{1.533056in}{2.444250in}}{\pgfqpoint{1.560833in}{2.444250in}}%
\pgfpathlineto{\pgfqpoint{1.560833in}{2.444250in}}%
\pgfpathclose%
\pgfusepath{stroke,fill}%
\end{pgfscope}%
\begin{pgfscope}%
\pgfsetbuttcap%
\pgfsetmiterjoin%
\pgfsetlinewidth{1.003750pt}%
\definecolor{currentstroke}{rgb}{0.000000,0.000000,0.000000}%
\pgfsetstrokecolor{currentstroke}%
\pgfsetdash{}{0pt}%
\pgfpathmoveto{\pgfqpoint{1.588611in}{2.722027in}}%
\pgfpathlineto{\pgfqpoint{1.866389in}{2.722027in}}%
\pgfpathlineto{\pgfqpoint{1.866389in}{2.819250in}}%
\pgfpathlineto{\pgfqpoint{1.588611in}{2.819250in}}%
\pgfpathlineto{\pgfqpoint{1.588611in}{2.722027in}}%
\pgfpathclose%
\pgfusepath{stroke}%
\end{pgfscope}%
\begin{pgfscope}%
\definecolor{textcolor}{rgb}{0.000000,0.000000,0.000000}%
\pgfsetstrokecolor{textcolor}%
\pgfsetfillcolor{textcolor}%
\pgftext[x=1.977500in,y=2.722027in,left,base]{\color{textcolor}\rmfamily\fontsize{10.000000}{12.000000}\selectfont Negative Class}%
\end{pgfscope}%
\begin{pgfscope}%
\pgfsetbuttcap%
\pgfsetmiterjoin%
\definecolor{currentfill}{rgb}{0.000000,0.000000,0.000000}%
\pgfsetfillcolor{currentfill}%
\pgfsetlinewidth{0.000000pt}%
\definecolor{currentstroke}{rgb}{0.000000,0.000000,0.000000}%
\pgfsetstrokecolor{currentstroke}%
\pgfsetstrokeopacity{0.000000}%
\pgfsetdash{}{0pt}%
\pgfpathmoveto{\pgfqpoint{1.588611in}{2.526750in}}%
\pgfpathlineto{\pgfqpoint{1.866389in}{2.526750in}}%
\pgfpathlineto{\pgfqpoint{1.866389in}{2.623972in}}%
\pgfpathlineto{\pgfqpoint{1.588611in}{2.623972in}}%
\pgfpathlineto{\pgfqpoint{1.588611in}{2.526750in}}%
\pgfpathclose%
\pgfusepath{fill}%
\end{pgfscope}%
\begin{pgfscope}%
\definecolor{textcolor}{rgb}{0.000000,0.000000,0.000000}%
\pgfsetstrokecolor{textcolor}%
\pgfsetfillcolor{textcolor}%
\pgftext[x=1.977500in,y=2.526750in,left,base]{\color{textcolor}\rmfamily\fontsize{10.000000}{12.000000}\selectfont Positive Class}%
\end{pgfscope}%
\end{pgfpicture}%
\makeatother%
\endgroup%

  &
  \vspace{0pt} %% Creator: Matplotlib, PGF backend
%%
%% To include the figure in your LaTeX document, write
%%   \input{<filename>.pgf}
%%
%% Make sure the required packages are loaded in your preamble
%%   \usepackage{pgf}
%%
%% Also ensure that all the required font packages are loaded; for instance,
%% the lmodern package is sometimes necessary when using math font.
%%   \usepackage{lmodern}
%%
%% Figures using additional raster images can only be included by \input if
%% they are in the same directory as the main LaTeX file. For loading figures
%% from other directories you can use the `import` package
%%   \usepackage{import}
%%
%% and then include the figures with
%%   \import{<path to file>}{<filename>.pgf}
%%
%% Matplotlib used the following preamble
%%   
%%   \usepackage{fontspec}
%%   \makeatletter\@ifpackageloaded{underscore}{}{\usepackage[strings]{underscore}}\makeatother
%%
\begingroup%
\makeatletter%
\begin{pgfpicture}%
\pgfpathrectangle{\pgfpointorigin}{\pgfqpoint{2.221861in}{1.754444in}}%
\pgfusepath{use as bounding box, clip}%
\begin{pgfscope}%
\pgfsetbuttcap%
\pgfsetmiterjoin%
\definecolor{currentfill}{rgb}{1.000000,1.000000,1.000000}%
\pgfsetfillcolor{currentfill}%
\pgfsetlinewidth{0.000000pt}%
\definecolor{currentstroke}{rgb}{1.000000,1.000000,1.000000}%
\pgfsetstrokecolor{currentstroke}%
\pgfsetdash{}{0pt}%
\pgfpathmoveto{\pgfqpoint{0.000000in}{0.000000in}}%
\pgfpathlineto{\pgfqpoint{2.221861in}{0.000000in}}%
\pgfpathlineto{\pgfqpoint{2.221861in}{1.754444in}}%
\pgfpathlineto{\pgfqpoint{0.000000in}{1.754444in}}%
\pgfpathlineto{\pgfqpoint{0.000000in}{0.000000in}}%
\pgfpathclose%
\pgfusepath{fill}%
\end{pgfscope}%
\begin{pgfscope}%
\pgfsetbuttcap%
\pgfsetmiterjoin%
\definecolor{currentfill}{rgb}{1.000000,1.000000,1.000000}%
\pgfsetfillcolor{currentfill}%
\pgfsetlinewidth{0.000000pt}%
\definecolor{currentstroke}{rgb}{0.000000,0.000000,0.000000}%
\pgfsetstrokecolor{currentstroke}%
\pgfsetstrokeopacity{0.000000}%
\pgfsetdash{}{0pt}%
\pgfpathmoveto{\pgfqpoint{0.553581in}{0.499444in}}%
\pgfpathlineto{\pgfqpoint{2.103581in}{0.499444in}}%
\pgfpathlineto{\pgfqpoint{2.103581in}{1.654444in}}%
\pgfpathlineto{\pgfqpoint{0.553581in}{1.654444in}}%
\pgfpathlineto{\pgfqpoint{0.553581in}{0.499444in}}%
\pgfpathclose%
\pgfusepath{fill}%
\end{pgfscope}%
\begin{pgfscope}%
\pgfsetbuttcap%
\pgfsetroundjoin%
\definecolor{currentfill}{rgb}{0.000000,0.000000,0.000000}%
\pgfsetfillcolor{currentfill}%
\pgfsetlinewidth{0.803000pt}%
\definecolor{currentstroke}{rgb}{0.000000,0.000000,0.000000}%
\pgfsetstrokecolor{currentstroke}%
\pgfsetdash{}{0pt}%
\pgfsys@defobject{currentmarker}{\pgfqpoint{0.000000in}{-0.048611in}}{\pgfqpoint{0.000000in}{0.000000in}}{%
\pgfpathmoveto{\pgfqpoint{0.000000in}{0.000000in}}%
\pgfpathlineto{\pgfqpoint{0.000000in}{-0.048611in}}%
\pgfusepath{stroke,fill}%
}%
\begin{pgfscope}%
\pgfsys@transformshift{0.624035in}{0.499444in}%
\pgfsys@useobject{currentmarker}{}%
\end{pgfscope}%
\end{pgfscope}%
\begin{pgfscope}%
\definecolor{textcolor}{rgb}{0.000000,0.000000,0.000000}%
\pgfsetstrokecolor{textcolor}%
\pgfsetfillcolor{textcolor}%
\pgftext[x=0.624035in,y=0.402222in,,top]{\color{textcolor}\rmfamily\fontsize{10.000000}{12.000000}\selectfont \(\displaystyle {0.0}\)}%
\end{pgfscope}%
\begin{pgfscope}%
\pgfsetbuttcap%
\pgfsetroundjoin%
\definecolor{currentfill}{rgb}{0.000000,0.000000,0.000000}%
\pgfsetfillcolor{currentfill}%
\pgfsetlinewidth{0.803000pt}%
\definecolor{currentstroke}{rgb}{0.000000,0.000000,0.000000}%
\pgfsetstrokecolor{currentstroke}%
\pgfsetdash{}{0pt}%
\pgfsys@defobject{currentmarker}{\pgfqpoint{0.000000in}{-0.048611in}}{\pgfqpoint{0.000000in}{0.000000in}}{%
\pgfpathmoveto{\pgfqpoint{0.000000in}{0.000000in}}%
\pgfpathlineto{\pgfqpoint{0.000000in}{-0.048611in}}%
\pgfusepath{stroke,fill}%
}%
\begin{pgfscope}%
\pgfsys@transformshift{1.328581in}{0.499444in}%
\pgfsys@useobject{currentmarker}{}%
\end{pgfscope}%
\end{pgfscope}%
\begin{pgfscope}%
\definecolor{textcolor}{rgb}{0.000000,0.000000,0.000000}%
\pgfsetstrokecolor{textcolor}%
\pgfsetfillcolor{textcolor}%
\pgftext[x=1.328581in,y=0.402222in,,top]{\color{textcolor}\rmfamily\fontsize{10.000000}{12.000000}\selectfont \(\displaystyle {0.5}\)}%
\end{pgfscope}%
\begin{pgfscope}%
\pgfsetbuttcap%
\pgfsetroundjoin%
\definecolor{currentfill}{rgb}{0.000000,0.000000,0.000000}%
\pgfsetfillcolor{currentfill}%
\pgfsetlinewidth{0.803000pt}%
\definecolor{currentstroke}{rgb}{0.000000,0.000000,0.000000}%
\pgfsetstrokecolor{currentstroke}%
\pgfsetdash{}{0pt}%
\pgfsys@defobject{currentmarker}{\pgfqpoint{0.000000in}{-0.048611in}}{\pgfqpoint{0.000000in}{0.000000in}}{%
\pgfpathmoveto{\pgfqpoint{0.000000in}{0.000000in}}%
\pgfpathlineto{\pgfqpoint{0.000000in}{-0.048611in}}%
\pgfusepath{stroke,fill}%
}%
\begin{pgfscope}%
\pgfsys@transformshift{2.033126in}{0.499444in}%
\pgfsys@useobject{currentmarker}{}%
\end{pgfscope}%
\end{pgfscope}%
\begin{pgfscope}%
\definecolor{textcolor}{rgb}{0.000000,0.000000,0.000000}%
\pgfsetstrokecolor{textcolor}%
\pgfsetfillcolor{textcolor}%
\pgftext[x=2.033126in,y=0.402222in,,top]{\color{textcolor}\rmfamily\fontsize{10.000000}{12.000000}\selectfont \(\displaystyle {1.0}\)}%
\end{pgfscope}%
\begin{pgfscope}%
\definecolor{textcolor}{rgb}{0.000000,0.000000,0.000000}%
\pgfsetstrokecolor{textcolor}%
\pgfsetfillcolor{textcolor}%
\pgftext[x=1.328581in,y=0.223333in,,top]{\color{textcolor}\rmfamily\fontsize{10.000000}{12.000000}\selectfont False positive rate}%
\end{pgfscope}%
\begin{pgfscope}%
\pgfsetbuttcap%
\pgfsetroundjoin%
\definecolor{currentfill}{rgb}{0.000000,0.000000,0.000000}%
\pgfsetfillcolor{currentfill}%
\pgfsetlinewidth{0.803000pt}%
\definecolor{currentstroke}{rgb}{0.000000,0.000000,0.000000}%
\pgfsetstrokecolor{currentstroke}%
\pgfsetdash{}{0pt}%
\pgfsys@defobject{currentmarker}{\pgfqpoint{-0.048611in}{0.000000in}}{\pgfqpoint{-0.000000in}{0.000000in}}{%
\pgfpathmoveto{\pgfqpoint{-0.000000in}{0.000000in}}%
\pgfpathlineto{\pgfqpoint{-0.048611in}{0.000000in}}%
\pgfusepath{stroke,fill}%
}%
\begin{pgfscope}%
\pgfsys@transformshift{0.553581in}{0.551944in}%
\pgfsys@useobject{currentmarker}{}%
\end{pgfscope}%
\end{pgfscope}%
\begin{pgfscope}%
\definecolor{textcolor}{rgb}{0.000000,0.000000,0.000000}%
\pgfsetstrokecolor{textcolor}%
\pgfsetfillcolor{textcolor}%
\pgftext[x=0.278889in, y=0.503750in, left, base]{\color{textcolor}\rmfamily\fontsize{10.000000}{12.000000}\selectfont \(\displaystyle {0.0}\)}%
\end{pgfscope}%
\begin{pgfscope}%
\pgfsetbuttcap%
\pgfsetroundjoin%
\definecolor{currentfill}{rgb}{0.000000,0.000000,0.000000}%
\pgfsetfillcolor{currentfill}%
\pgfsetlinewidth{0.803000pt}%
\definecolor{currentstroke}{rgb}{0.000000,0.000000,0.000000}%
\pgfsetstrokecolor{currentstroke}%
\pgfsetdash{}{0pt}%
\pgfsys@defobject{currentmarker}{\pgfqpoint{-0.048611in}{0.000000in}}{\pgfqpoint{-0.000000in}{0.000000in}}{%
\pgfpathmoveto{\pgfqpoint{-0.000000in}{0.000000in}}%
\pgfpathlineto{\pgfqpoint{-0.048611in}{0.000000in}}%
\pgfusepath{stroke,fill}%
}%
\begin{pgfscope}%
\pgfsys@transformshift{0.553581in}{1.076944in}%
\pgfsys@useobject{currentmarker}{}%
\end{pgfscope}%
\end{pgfscope}%
\begin{pgfscope}%
\definecolor{textcolor}{rgb}{0.000000,0.000000,0.000000}%
\pgfsetstrokecolor{textcolor}%
\pgfsetfillcolor{textcolor}%
\pgftext[x=0.278889in, y=1.028750in, left, base]{\color{textcolor}\rmfamily\fontsize{10.000000}{12.000000}\selectfont \(\displaystyle {0.5}\)}%
\end{pgfscope}%
\begin{pgfscope}%
\pgfsetbuttcap%
\pgfsetroundjoin%
\definecolor{currentfill}{rgb}{0.000000,0.000000,0.000000}%
\pgfsetfillcolor{currentfill}%
\pgfsetlinewidth{0.803000pt}%
\definecolor{currentstroke}{rgb}{0.000000,0.000000,0.000000}%
\pgfsetstrokecolor{currentstroke}%
\pgfsetdash{}{0pt}%
\pgfsys@defobject{currentmarker}{\pgfqpoint{-0.048611in}{0.000000in}}{\pgfqpoint{-0.000000in}{0.000000in}}{%
\pgfpathmoveto{\pgfqpoint{-0.000000in}{0.000000in}}%
\pgfpathlineto{\pgfqpoint{-0.048611in}{0.000000in}}%
\pgfusepath{stroke,fill}%
}%
\begin{pgfscope}%
\pgfsys@transformshift{0.553581in}{1.601944in}%
\pgfsys@useobject{currentmarker}{}%
\end{pgfscope}%
\end{pgfscope}%
\begin{pgfscope}%
\definecolor{textcolor}{rgb}{0.000000,0.000000,0.000000}%
\pgfsetstrokecolor{textcolor}%
\pgfsetfillcolor{textcolor}%
\pgftext[x=0.278889in, y=1.553750in, left, base]{\color{textcolor}\rmfamily\fontsize{10.000000}{12.000000}\selectfont \(\displaystyle {1.0}\)}%
\end{pgfscope}%
\begin{pgfscope}%
\definecolor{textcolor}{rgb}{0.000000,0.000000,0.000000}%
\pgfsetstrokecolor{textcolor}%
\pgfsetfillcolor{textcolor}%
\pgftext[x=0.223333in,y=1.076944in,,bottom,rotate=90.000000]{\color{textcolor}\rmfamily\fontsize{10.000000}{12.000000}\selectfont True positive rate}%
\end{pgfscope}%
\begin{pgfscope}%
\pgfpathrectangle{\pgfqpoint{0.553581in}{0.499444in}}{\pgfqpoint{1.550000in}{1.155000in}}%
\pgfusepath{clip}%
\pgfsetbuttcap%
\pgfsetroundjoin%
\pgfsetlinewidth{1.505625pt}%
\definecolor{currentstroke}{rgb}{0.000000,0.000000,0.000000}%
\pgfsetstrokecolor{currentstroke}%
\pgfsetdash{{5.550000pt}{2.400000pt}}{0.000000pt}%
\pgfpathmoveto{\pgfqpoint{0.624035in}{0.551944in}}%
\pgfpathlineto{\pgfqpoint{2.033126in}{1.601944in}}%
\pgfusepath{stroke}%
\end{pgfscope}%
\begin{pgfscope}%
\pgfpathrectangle{\pgfqpoint{0.553581in}{0.499444in}}{\pgfqpoint{1.550000in}{1.155000in}}%
\pgfusepath{clip}%
\pgfsetrectcap%
\pgfsetroundjoin%
\pgfsetlinewidth{1.505625pt}%
\definecolor{currentstroke}{rgb}{0.000000,0.000000,0.000000}%
\pgfsetstrokecolor{currentstroke}%
\pgfsetdash{}{0pt}%
\pgfpathmoveto{\pgfqpoint{0.624035in}{0.551944in}}%
\pgfpathlineto{\pgfqpoint{0.626119in}{0.552260in}}%
\pgfpathlineto{\pgfqpoint{0.627222in}{0.560898in}}%
\pgfpathlineto{\pgfqpoint{0.627858in}{0.562002in}}%
\pgfpathlineto{\pgfqpoint{0.628904in}{0.564132in}}%
\pgfpathlineto{\pgfqpoint{0.629465in}{0.565236in}}%
\pgfpathlineto{\pgfqpoint{0.630521in}{0.567918in}}%
\pgfpathlineto{\pgfqpoint{0.631138in}{0.569023in}}%
\pgfpathlineto{\pgfqpoint{0.632241in}{0.573085in}}%
\pgfpathlineto{\pgfqpoint{0.632652in}{0.574190in}}%
\pgfpathlineto{\pgfqpoint{0.633755in}{0.580264in}}%
\pgfpathlineto{\pgfqpoint{0.634166in}{0.581171in}}%
\pgfpathlineto{\pgfqpoint{0.635269in}{0.586378in}}%
\pgfpathlineto{\pgfqpoint{0.635447in}{0.587482in}}%
\pgfpathlineto{\pgfqpoint{0.636531in}{0.593359in}}%
\pgfpathlineto{\pgfqpoint{0.636662in}{0.594384in}}%
\pgfpathlineto{\pgfqpoint{0.637764in}{0.600222in}}%
\pgfpathlineto{\pgfqpoint{0.637979in}{0.601287in}}%
\pgfpathlineto{\pgfqpoint{0.639082in}{0.607992in}}%
\pgfpathlineto{\pgfqpoint{0.639278in}{0.608978in}}%
\pgfpathlineto{\pgfqpoint{0.640372in}{0.614027in}}%
\pgfpathlineto{\pgfqpoint{0.640587in}{0.615131in}}%
\pgfpathlineto{\pgfqpoint{0.641690in}{0.622783in}}%
\pgfpathlineto{\pgfqpoint{0.641867in}{0.623887in}}%
\pgfpathlineto{\pgfqpoint{0.642970in}{0.631381in}}%
\pgfpathlineto{\pgfqpoint{0.643092in}{0.632446in}}%
\pgfpathlineto{\pgfqpoint{0.644194in}{0.641794in}}%
\pgfpathlineto{\pgfqpoint{0.644316in}{0.642741in}}%
\pgfpathlineto{\pgfqpoint{0.645419in}{0.649841in}}%
\pgfpathlineto{\pgfqpoint{0.645559in}{0.650866in}}%
\pgfpathlineto{\pgfqpoint{0.646662in}{0.658557in}}%
\pgfpathlineto{\pgfqpoint{0.646886in}{0.659662in}}%
\pgfpathlineto{\pgfqpoint{0.647979in}{0.665736in}}%
\pgfpathlineto{\pgfqpoint{0.648232in}{0.666840in}}%
\pgfpathlineto{\pgfqpoint{0.649335in}{0.674216in}}%
\pgfpathlineto{\pgfqpoint{0.649465in}{0.675281in}}%
\pgfpathlineto{\pgfqpoint{0.650568in}{0.683091in}}%
\pgfpathlineto{\pgfqpoint{0.650736in}{0.684116in}}%
\pgfpathlineto{\pgfqpoint{0.651830in}{0.692754in}}%
\pgfpathlineto{\pgfqpoint{0.652008in}{0.693819in}}%
\pgfpathlineto{\pgfqpoint{0.653073in}{0.701471in}}%
\pgfpathlineto{\pgfqpoint{0.653260in}{0.702496in}}%
\pgfpathlineto{\pgfqpoint{0.654353in}{0.710267in}}%
\pgfpathlineto{\pgfqpoint{0.654578in}{0.711331in}}%
\pgfpathlineto{\pgfqpoint{0.655671in}{0.720877in}}%
\pgfpathlineto{\pgfqpoint{0.655793in}{0.721863in}}%
\pgfpathlineto{\pgfqpoint{0.656895in}{0.730422in}}%
\pgfpathlineto{\pgfqpoint{0.657166in}{0.731526in}}%
\pgfpathlineto{\pgfqpoint{0.658269in}{0.740677in}}%
\pgfpathlineto{\pgfqpoint{0.658372in}{0.741663in}}%
\pgfpathlineto{\pgfqpoint{0.659475in}{0.749472in}}%
\pgfpathlineto{\pgfqpoint{0.659615in}{0.750577in}}%
\pgfpathlineto{\pgfqpoint{0.660671in}{0.759136in}}%
\pgfpathlineto{\pgfqpoint{0.660914in}{0.760082in}}%
\pgfpathlineto{\pgfqpoint{0.662017in}{0.767419in}}%
\pgfpathlineto{\pgfqpoint{0.662185in}{0.768365in}}%
\pgfpathlineto{\pgfqpoint{0.663279in}{0.776806in}}%
\pgfpathlineto{\pgfqpoint{0.663466in}{0.777832in}}%
\pgfpathlineto{\pgfqpoint{0.664559in}{0.785365in}}%
\pgfpathlineto{\pgfqpoint{0.664886in}{0.786430in}}%
\pgfpathlineto{\pgfqpoint{0.665989in}{0.793530in}}%
\pgfpathlineto{\pgfqpoint{0.666204in}{0.794634in}}%
\pgfpathlineto{\pgfqpoint{0.667307in}{0.802720in}}%
\pgfpathlineto{\pgfqpoint{0.667522in}{0.803785in}}%
\pgfpathlineto{\pgfqpoint{0.668625in}{0.809110in}}%
\pgfpathlineto{\pgfqpoint{0.668802in}{0.810175in}}%
\pgfpathlineto{\pgfqpoint{0.669905in}{0.815618in}}%
\pgfpathlineto{\pgfqpoint{0.670120in}{0.816683in}}%
\pgfpathlineto{\pgfqpoint{0.671223in}{0.824847in}}%
\pgfpathlineto{\pgfqpoint{0.671475in}{0.825873in}}%
\pgfpathlineto{\pgfqpoint{0.672559in}{0.834274in}}%
\pgfpathlineto{\pgfqpoint{0.672727in}{0.835378in}}%
\pgfpathlineto{\pgfqpoint{0.673830in}{0.842439in}}%
\pgfpathlineto{\pgfqpoint{0.674036in}{0.843425in}}%
\pgfpathlineto{\pgfqpoint{0.675129in}{0.849735in}}%
\pgfpathlineto{\pgfqpoint{0.675419in}{0.850840in}}%
\pgfpathlineto{\pgfqpoint{0.676503in}{0.856520in}}%
\pgfpathlineto{\pgfqpoint{0.676681in}{0.857584in}}%
\pgfpathlineto{\pgfqpoint{0.677784in}{0.863225in}}%
\pgfpathlineto{\pgfqpoint{0.678017in}{0.864329in}}%
\pgfpathlineto{\pgfqpoint{0.679120in}{0.871508in}}%
\pgfpathlineto{\pgfqpoint{0.679372in}{0.872612in}}%
\pgfpathlineto{\pgfqpoint{0.680475in}{0.878450in}}%
\pgfpathlineto{\pgfqpoint{0.680681in}{0.879554in}}%
\pgfpathlineto{\pgfqpoint{0.681784in}{0.884839in}}%
\pgfpathlineto{\pgfqpoint{0.681914in}{0.885865in}}%
\pgfpathlineto{\pgfqpoint{0.683017in}{0.892531in}}%
\pgfpathlineto{\pgfqpoint{0.683270in}{0.893595in}}%
\pgfpathlineto{\pgfqpoint{0.684344in}{0.898171in}}%
\pgfpathlineto{\pgfqpoint{0.684615in}{0.899157in}}%
\pgfpathlineto{\pgfqpoint{0.685718in}{0.905113in}}%
\pgfpathlineto{\pgfqpoint{0.686036in}{0.906217in}}%
\pgfpathlineto{\pgfqpoint{0.687129in}{0.911660in}}%
\pgfpathlineto{\pgfqpoint{0.687428in}{0.912765in}}%
\pgfpathlineto{\pgfqpoint{0.688531in}{0.917419in}}%
\pgfpathlineto{\pgfqpoint{0.688718in}{0.918523in}}%
\pgfpathlineto{\pgfqpoint{0.689821in}{0.923572in}}%
\pgfpathlineto{\pgfqpoint{0.690017in}{0.924676in}}%
\pgfpathlineto{\pgfqpoint{0.691120in}{0.930632in}}%
\pgfpathlineto{\pgfqpoint{0.691391in}{0.931736in}}%
\pgfpathlineto{\pgfqpoint{0.692494in}{0.937692in}}%
\pgfpathlineto{\pgfqpoint{0.692709in}{0.938797in}}%
\pgfpathlineto{\pgfqpoint{0.693812in}{0.943924in}}%
\pgfpathlineto{\pgfqpoint{0.694045in}{0.944910in}}%
\pgfpathlineto{\pgfqpoint{0.695148in}{0.949288in}}%
\pgfpathlineto{\pgfqpoint{0.695401in}{0.950393in}}%
\pgfpathlineto{\pgfqpoint{0.696494in}{0.955441in}}%
\pgfpathlineto{\pgfqpoint{0.696849in}{0.956467in}}%
\pgfpathlineto{\pgfqpoint{0.697924in}{0.961516in}}%
\pgfpathlineto{\pgfqpoint{0.698260in}{0.962581in}}%
\pgfpathlineto{\pgfqpoint{0.699345in}{0.967314in}}%
\pgfpathlineto{\pgfqpoint{0.699653in}{0.968379in}}%
\pgfpathlineto{\pgfqpoint{0.700746in}{0.972204in}}%
\pgfpathlineto{\pgfqpoint{0.701008in}{0.973112in}}%
\pgfpathlineto{\pgfqpoint{0.702111in}{0.978791in}}%
\pgfpathlineto{\pgfqpoint{0.702410in}{0.979896in}}%
\pgfpathlineto{\pgfqpoint{0.703503in}{0.984944in}}%
\pgfpathlineto{\pgfqpoint{0.703831in}{0.986049in}}%
\pgfpathlineto{\pgfqpoint{0.704924in}{0.990664in}}%
\pgfpathlineto{\pgfqpoint{0.705223in}{0.991610in}}%
\pgfpathlineto{\pgfqpoint{0.706326in}{0.996541in}}%
\pgfpathlineto{\pgfqpoint{0.706690in}{0.997527in}}%
\pgfpathlineto{\pgfqpoint{0.707765in}{1.002970in}}%
\pgfpathlineto{\pgfqpoint{0.708036in}{1.004074in}}%
\pgfpathlineto{\pgfqpoint{0.709130in}{1.008373in}}%
\pgfpathlineto{\pgfqpoint{0.709391in}{1.009478in}}%
\pgfpathlineto{\pgfqpoint{0.710485in}{1.012988in}}%
\pgfpathlineto{\pgfqpoint{0.710868in}{1.014092in}}%
\pgfpathlineto{\pgfqpoint{0.711961in}{1.018747in}}%
\pgfpathlineto{\pgfqpoint{0.712242in}{1.019851in}}%
\pgfpathlineto{\pgfqpoint{0.713289in}{1.023677in}}%
\pgfpathlineto{\pgfqpoint{0.713793in}{1.024742in}}%
\pgfpathlineto{\pgfqpoint{0.714877in}{1.028568in}}%
\pgfpathlineto{\pgfqpoint{0.715186in}{1.029593in}}%
\pgfpathlineto{\pgfqpoint{0.716251in}{1.033577in}}%
\pgfpathlineto{\pgfqpoint{0.716522in}{1.034681in}}%
\pgfpathlineto{\pgfqpoint{0.717606in}{1.039493in}}%
\pgfpathlineto{\pgfqpoint{0.717924in}{1.040480in}}%
\pgfpathlineto{\pgfqpoint{0.719018in}{1.043990in}}%
\pgfpathlineto{\pgfqpoint{0.719373in}{1.045055in}}%
\pgfpathlineto{\pgfqpoint{0.720457in}{1.048881in}}%
\pgfpathlineto{\pgfqpoint{0.720803in}{1.049946in}}%
\pgfpathlineto{\pgfqpoint{0.721887in}{1.053141in}}%
\pgfpathlineto{\pgfqpoint{0.722139in}{1.054245in}}%
\pgfpathlineto{\pgfqpoint{0.723233in}{1.058268in}}%
\pgfpathlineto{\pgfqpoint{0.723635in}{1.059373in}}%
\pgfpathlineto{\pgfqpoint{0.724709in}{1.063001in}}%
\pgfpathlineto{\pgfqpoint{0.725177in}{1.064106in}}%
\pgfpathlineto{\pgfqpoint{0.726186in}{1.067892in}}%
\pgfpathlineto{\pgfqpoint{0.726541in}{1.068996in}}%
\pgfpathlineto{\pgfqpoint{0.727644in}{1.072231in}}%
\pgfpathlineto{\pgfqpoint{0.728036in}{1.073335in}}%
\pgfpathlineto{\pgfqpoint{0.729139in}{1.075899in}}%
\pgfpathlineto{\pgfqpoint{0.729532in}{1.076964in}}%
\pgfpathlineto{\pgfqpoint{0.730607in}{1.080632in}}%
\pgfpathlineto{\pgfqpoint{0.731027in}{1.081618in}}%
\pgfpathlineto{\pgfqpoint{0.732111in}{1.084655in}}%
\pgfpathlineto{\pgfqpoint{0.732364in}{1.085681in}}%
\pgfpathlineto{\pgfqpoint{0.733429in}{1.089270in}}%
\pgfpathlineto{\pgfqpoint{0.733850in}{1.090374in}}%
\pgfpathlineto{\pgfqpoint{0.734924in}{1.093372in}}%
\pgfpathlineto{\pgfqpoint{0.735326in}{1.094476in}}%
\pgfpathlineto{\pgfqpoint{0.736429in}{1.097435in}}%
\pgfpathlineto{\pgfqpoint{0.736934in}{1.098500in}}%
\pgfpathlineto{\pgfqpoint{0.738009in}{1.101615in}}%
\pgfpathlineto{\pgfqpoint{0.738513in}{1.102720in}}%
\pgfpathlineto{\pgfqpoint{0.739569in}{1.105836in}}%
\pgfpathlineto{\pgfqpoint{0.739990in}{1.106901in}}%
\pgfpathlineto{\pgfqpoint{0.741093in}{1.109938in}}%
\pgfpathlineto{\pgfqpoint{0.741766in}{1.111003in}}%
\pgfpathlineto{\pgfqpoint{0.742831in}{1.114158in}}%
\pgfpathlineto{\pgfqpoint{0.743336in}{1.115263in}}%
\pgfpathlineto{\pgfqpoint{0.744382in}{1.118181in}}%
\pgfpathlineto{\pgfqpoint{0.755205in}{1.119286in}}%
\pgfpathlineto{\pgfqpoint{0.756298in}{1.121771in}}%
\pgfpathlineto{\pgfqpoint{0.756775in}{1.122875in}}%
\pgfpathlineto{\pgfqpoint{0.757878in}{1.125912in}}%
\pgfpathlineto{\pgfqpoint{0.758373in}{1.127016in}}%
\pgfpathlineto{\pgfqpoint{0.759457in}{1.129699in}}%
\pgfpathlineto{\pgfqpoint{0.759766in}{1.130803in}}%
\pgfpathlineto{\pgfqpoint{0.760841in}{1.134116in}}%
\pgfpathlineto{\pgfqpoint{0.761289in}{1.135063in}}%
\pgfpathlineto{\pgfqpoint{0.762392in}{1.137508in}}%
\pgfpathlineto{\pgfqpoint{0.762756in}{1.138613in}}%
\pgfpathlineto{\pgfqpoint{0.763859in}{1.141926in}}%
\pgfpathlineto{\pgfqpoint{0.764299in}{1.143030in}}%
\pgfpathlineto{\pgfqpoint{0.765401in}{1.145160in}}%
\pgfpathlineto{\pgfqpoint{0.765990in}{1.146264in}}%
\pgfpathlineto{\pgfqpoint{0.767046in}{1.149144in}}%
\pgfpathlineto{\pgfqpoint{0.767514in}{1.150209in}}%
\pgfpathlineto{\pgfqpoint{0.768616in}{1.152694in}}%
\pgfpathlineto{\pgfqpoint{0.769168in}{1.153798in}}%
\pgfpathlineto{\pgfqpoint{0.770243in}{1.156914in}}%
\pgfpathlineto{\pgfqpoint{0.770887in}{1.158018in}}%
\pgfpathlineto{\pgfqpoint{0.771972in}{1.160622in}}%
\pgfpathlineto{\pgfqpoint{0.772785in}{1.161686in}}%
\pgfpathlineto{\pgfqpoint{0.773869in}{1.163777in}}%
\pgfpathlineto{\pgfqpoint{0.774327in}{1.164881in}}%
\pgfpathlineto{\pgfqpoint{0.775401in}{1.167918in}}%
\pgfpathlineto{\pgfqpoint{0.775869in}{1.169023in}}%
\pgfpathlineto{\pgfqpoint{0.776906in}{1.171823in}}%
\pgfpathlineto{\pgfqpoint{0.777504in}{1.172928in}}%
\pgfpathlineto{\pgfqpoint{0.778579in}{1.175767in}}%
\pgfpathlineto{\pgfqpoint{0.779037in}{1.176872in}}%
\pgfpathlineto{\pgfqpoint{0.780130in}{1.178489in}}%
\pgfpathlineto{\pgfqpoint{0.780663in}{1.179515in}}%
\pgfpathlineto{\pgfqpoint{0.781766in}{1.182275in}}%
\pgfpathlineto{\pgfqpoint{0.782243in}{1.183380in}}%
\pgfpathlineto{\pgfqpoint{0.783317in}{1.185707in}}%
\pgfpathlineto{\pgfqpoint{0.783747in}{1.186732in}}%
\pgfpathlineto{\pgfqpoint{0.784831in}{1.188902in}}%
\pgfpathlineto{\pgfqpoint{0.785626in}{1.190006in}}%
\pgfpathlineto{\pgfqpoint{0.786691in}{1.192097in}}%
\pgfpathlineto{\pgfqpoint{0.787289in}{1.193201in}}%
\pgfpathlineto{\pgfqpoint{0.788364in}{1.194897in}}%
\pgfpathlineto{\pgfqpoint{0.789196in}{1.196001in}}%
\pgfpathlineto{\pgfqpoint{0.790299in}{1.198053in}}%
\pgfpathlineto{\pgfqpoint{0.791046in}{1.199117in}}%
\pgfpathlineto{\pgfqpoint{0.792131in}{1.201326in}}%
\pgfpathlineto{\pgfqpoint{0.792645in}{1.202391in}}%
\pgfpathlineto{\pgfqpoint{0.793738in}{1.204048in}}%
\pgfpathlineto{\pgfqpoint{0.794561in}{1.205152in}}%
\pgfpathlineto{\pgfqpoint{0.795654in}{1.206967in}}%
\pgfpathlineto{\pgfqpoint{0.796476in}{1.208071in}}%
\pgfpathlineto{\pgfqpoint{0.797579in}{1.210201in}}%
\pgfpathlineto{\pgfqpoint{0.798065in}{1.211305in}}%
\pgfpathlineto{\pgfqpoint{0.799131in}{1.213080in}}%
\pgfpathlineto{\pgfqpoint{0.800000in}{1.214106in}}%
\pgfpathlineto{\pgfqpoint{0.801084in}{1.216275in}}%
\pgfpathlineto{\pgfqpoint{0.801570in}{1.217340in}}%
\pgfpathlineto{\pgfqpoint{0.802645in}{1.219115in}}%
\pgfpathlineto{\pgfqpoint{0.803346in}{1.220219in}}%
\pgfpathlineto{\pgfqpoint{0.804383in}{1.222112in}}%
\pgfpathlineto{\pgfqpoint{0.804953in}{1.223099in}}%
\pgfpathlineto{\pgfqpoint{0.806047in}{1.225978in}}%
\pgfpathlineto{\pgfqpoint{0.806551in}{1.227082in}}%
\pgfpathlineto{\pgfqpoint{0.807589in}{1.229094in}}%
\pgfpathlineto{\pgfqpoint{0.808327in}{1.230198in}}%
\pgfpathlineto{\pgfqpoint{0.809430in}{1.232486in}}%
\pgfpathlineto{\pgfqpoint{0.810318in}{1.233590in}}%
\pgfpathlineto{\pgfqpoint{0.811411in}{1.235286in}}%
\pgfpathlineto{\pgfqpoint{0.812103in}{1.236391in}}%
\pgfpathlineto{\pgfqpoint{0.813084in}{1.237850in}}%
\pgfpathlineto{\pgfqpoint{0.813935in}{1.238954in}}%
\pgfpathlineto{\pgfqpoint{0.815009in}{1.240808in}}%
\pgfpathlineto{\pgfqpoint{0.815888in}{1.241873in}}%
\pgfpathlineto{\pgfqpoint{0.816991in}{1.243766in}}%
\pgfpathlineto{\pgfqpoint{0.817542in}{1.244871in}}%
\pgfpathlineto{\pgfqpoint{0.818645in}{1.246646in}}%
\pgfpathlineto{\pgfqpoint{0.819383in}{1.247750in}}%
\pgfpathlineto{\pgfqpoint{0.820477in}{1.249288in}}%
\pgfpathlineto{\pgfqpoint{0.821355in}{1.250393in}}%
\pgfpathlineto{\pgfqpoint{0.822458in}{1.252957in}}%
\pgfpathlineto{\pgfqpoint{0.823252in}{1.254061in}}%
\pgfpathlineto{\pgfqpoint{0.824337in}{1.255599in}}%
\pgfpathlineto{\pgfqpoint{0.825168in}{1.256704in}}%
\pgfpathlineto{\pgfqpoint{0.826271in}{1.258123in}}%
\pgfpathlineto{\pgfqpoint{0.826897in}{1.259188in}}%
\pgfpathlineto{\pgfqpoint{0.827982in}{1.260845in}}%
\pgfpathlineto{\pgfqpoint{0.829150in}{1.261949in}}%
\pgfpathlineto{\pgfqpoint{0.830196in}{1.262935in}}%
\pgfpathlineto{\pgfqpoint{0.831028in}{1.263922in}}%
\pgfpathlineto{\pgfqpoint{0.832131in}{1.265854in}}%
\pgfpathlineto{\pgfqpoint{0.832776in}{1.266959in}}%
\pgfpathlineto{\pgfqpoint{0.833860in}{1.268418in}}%
\pgfpathlineto{\pgfqpoint{0.834776in}{1.269522in}}%
\pgfpathlineto{\pgfqpoint{0.835804in}{1.270903in}}%
\pgfpathlineto{\pgfqpoint{0.836748in}{1.271928in}}%
\pgfpathlineto{\pgfqpoint{0.837804in}{1.273269in}}%
\pgfpathlineto{\pgfqpoint{0.838617in}{1.274374in}}%
\pgfpathlineto{\pgfqpoint{0.839692in}{1.276030in}}%
\pgfpathlineto{\pgfqpoint{0.840692in}{1.277135in}}%
\pgfpathlineto{\pgfqpoint{0.841655in}{1.278239in}}%
\pgfpathlineto{\pgfqpoint{0.842570in}{1.279344in}}%
\pgfpathlineto{\pgfqpoint{0.843673in}{1.280566in}}%
\pgfpathlineto{\pgfqpoint{0.844589in}{1.281671in}}%
\pgfpathlineto{\pgfqpoint{0.845692in}{1.283288in}}%
\pgfpathlineto{\pgfqpoint{0.846785in}{1.284392in}}%
\pgfpathlineto{\pgfqpoint{0.847851in}{1.285378in}}%
\pgfpathlineto{\pgfqpoint{0.848795in}{1.286483in}}%
\pgfpathlineto{\pgfqpoint{0.849888in}{1.287390in}}%
\pgfpathlineto{\pgfqpoint{0.850757in}{1.288494in}}%
\pgfpathlineto{\pgfqpoint{0.851814in}{1.290111in}}%
\pgfpathlineto{\pgfqpoint{0.852860in}{1.291176in}}%
\pgfpathlineto{\pgfqpoint{0.853832in}{1.292754in}}%
\pgfpathlineto{\pgfqpoint{0.854458in}{1.293858in}}%
\pgfpathlineto{\pgfqpoint{0.855524in}{1.295239in}}%
\pgfpathlineto{\pgfqpoint{0.856739in}{1.296343in}}%
\pgfpathlineto{\pgfqpoint{0.857804in}{1.297763in}}%
\pgfpathlineto{\pgfqpoint{0.858730in}{1.298789in}}%
\pgfpathlineto{\pgfqpoint{0.859823in}{1.299972in}}%
\pgfpathlineto{\pgfqpoint{0.860402in}{1.301076in}}%
\pgfpathlineto{\pgfqpoint{0.861430in}{1.302417in}}%
\pgfpathlineto{\pgfqpoint{0.862169in}{1.303482in}}%
\pgfpathlineto{\pgfqpoint{0.863272in}{1.304468in}}%
\pgfpathlineto{\pgfqpoint{0.864169in}{1.305573in}}%
\pgfpathlineto{\pgfqpoint{0.865272in}{1.306993in}}%
\pgfpathlineto{\pgfqpoint{0.866337in}{1.308058in}}%
\pgfpathlineto{\pgfqpoint{0.867431in}{1.309833in}}%
\pgfpathlineto{\pgfqpoint{0.868459in}{1.310898in}}%
\pgfpathlineto{\pgfqpoint{0.869561in}{1.312436in}}%
\pgfpathlineto{\pgfqpoint{0.870487in}{1.313540in}}%
\pgfpathlineto{\pgfqpoint{0.871589in}{1.314763in}}%
\pgfpathlineto{\pgfqpoint{0.872571in}{1.315867in}}%
\pgfpathlineto{\pgfqpoint{0.873636in}{1.316696in}}%
\pgfpathlineto{\pgfqpoint{0.874328in}{1.317761in}}%
\pgfpathlineto{\pgfqpoint{0.875347in}{1.318983in}}%
\pgfpathlineto{\pgfqpoint{0.876814in}{1.320088in}}%
\pgfpathlineto{\pgfqpoint{0.877907in}{1.321113in}}%
\pgfpathlineto{\pgfqpoint{0.879104in}{1.322218in}}%
\pgfpathlineto{\pgfqpoint{0.880206in}{1.323361in}}%
\pgfpathlineto{\pgfqpoint{0.881281in}{1.324387in}}%
\pgfpathlineto{\pgfqpoint{0.882347in}{1.325570in}}%
\pgfpathlineto{\pgfqpoint{0.883664in}{1.326675in}}%
\pgfpathlineto{\pgfqpoint{0.884758in}{1.328252in}}%
\pgfpathlineto{\pgfqpoint{0.886440in}{1.329357in}}%
\pgfpathlineto{\pgfqpoint{0.887496in}{1.330382in}}%
\pgfpathlineto{\pgfqpoint{0.888805in}{1.331487in}}%
\pgfpathlineto{\pgfqpoint{0.889879in}{1.332788in}}%
\pgfpathlineto{\pgfqpoint{0.890898in}{1.333893in}}%
\pgfpathlineto{\pgfqpoint{0.891954in}{1.334681in}}%
\pgfpathlineto{\pgfqpoint{0.893337in}{1.335786in}}%
\pgfpathlineto{\pgfqpoint{0.894384in}{1.337009in}}%
\pgfpathlineto{\pgfqpoint{0.895515in}{1.338074in}}%
\pgfpathlineto{\pgfqpoint{0.896562in}{1.339178in}}%
\pgfpathlineto{\pgfqpoint{0.897562in}{1.340282in}}%
\pgfpathlineto{\pgfqpoint{0.898665in}{1.341189in}}%
\pgfpathlineto{\pgfqpoint{0.899945in}{1.342294in}}%
\pgfpathlineto{\pgfqpoint{0.900992in}{1.343162in}}%
\pgfpathlineto{\pgfqpoint{0.902225in}{1.344266in}}%
\pgfpathlineto{\pgfqpoint{0.903328in}{1.345568in}}%
\pgfpathlineto{\pgfqpoint{0.904319in}{1.346672in}}%
\pgfpathlineto{\pgfqpoint{0.905272in}{1.347776in}}%
\pgfpathlineto{\pgfqpoint{0.906375in}{1.348881in}}%
\pgfpathlineto{\pgfqpoint{0.907403in}{1.349946in}}%
\pgfpathlineto{\pgfqpoint{0.909300in}{1.351050in}}%
\pgfpathlineto{\pgfqpoint{0.910347in}{1.351839in}}%
\pgfpathlineto{\pgfqpoint{0.911637in}{1.352943in}}%
\pgfpathlineto{\pgfqpoint{0.912674in}{1.353969in}}%
\pgfpathlineto{\pgfqpoint{0.914160in}{1.355073in}}%
\pgfpathlineto{\pgfqpoint{0.915235in}{1.355862in}}%
\pgfpathlineto{\pgfqpoint{0.916730in}{1.356967in}}%
\pgfpathlineto{\pgfqpoint{0.917646in}{1.357558in}}%
\pgfpathlineto{\pgfqpoint{0.919235in}{1.358623in}}%
\pgfpathlineto{\pgfqpoint{0.920216in}{1.359885in}}%
\pgfpathlineto{\pgfqpoint{0.921226in}{1.360950in}}%
\pgfpathlineto{\pgfqpoint{0.922244in}{1.361976in}}%
\pgfpathlineto{\pgfqpoint{0.924048in}{1.363080in}}%
\pgfpathlineto{\pgfqpoint{0.925085in}{1.363751in}}%
\pgfpathlineto{\pgfqpoint{0.926366in}{1.364855in}}%
\pgfpathlineto{\pgfqpoint{0.927431in}{1.365407in}}%
\pgfpathlineto{\pgfqpoint{0.929095in}{1.366512in}}%
\pgfpathlineto{\pgfqpoint{0.930114in}{1.367300in}}%
\pgfpathlineto{\pgfqpoint{0.931095in}{1.368405in}}%
\pgfpathlineto{\pgfqpoint{0.932179in}{1.369351in}}%
\pgfpathlineto{\pgfqpoint{0.933151in}{1.370456in}}%
\pgfpathlineto{\pgfqpoint{0.934160in}{1.371205in}}%
\pgfpathlineto{\pgfqpoint{0.936646in}{1.372310in}}%
\pgfpathlineto{\pgfqpoint{0.937749in}{1.373020in}}%
\pgfpathlineto{\pgfqpoint{0.939497in}{1.374124in}}%
\pgfpathlineto{\pgfqpoint{0.940431in}{1.374992in}}%
\pgfpathlineto{\pgfqpoint{0.942104in}{1.376096in}}%
\pgfpathlineto{\pgfqpoint{0.943114in}{1.376924in}}%
\pgfpathlineto{\pgfqpoint{0.945030in}{1.377989in}}%
\pgfpathlineto{\pgfqpoint{0.946123in}{1.378936in}}%
\pgfpathlineto{\pgfqpoint{0.947880in}{1.380040in}}%
\pgfpathlineto{\pgfqpoint{0.948964in}{1.381145in}}%
\pgfpathlineto{\pgfqpoint{0.950693in}{1.382249in}}%
\pgfpathlineto{\pgfqpoint{0.951740in}{1.383275in}}%
\pgfpathlineto{\pgfqpoint{0.953469in}{1.384379in}}%
\pgfpathlineto{\pgfqpoint{0.954562in}{1.385207in}}%
\pgfpathlineto{\pgfqpoint{0.956347in}{1.386312in}}%
\pgfpathlineto{\pgfqpoint{0.957404in}{1.387022in}}%
\pgfpathlineto{\pgfqpoint{0.958712in}{1.388126in}}%
\pgfpathlineto{\pgfqpoint{0.959805in}{1.388678in}}%
\pgfpathlineto{\pgfqpoint{0.961890in}{1.389783in}}%
\pgfpathlineto{\pgfqpoint{0.962964in}{1.390453in}}%
\pgfpathlineto{\pgfqpoint{0.964450in}{1.391518in}}%
\pgfpathlineto{\pgfqpoint{0.965357in}{1.391952in}}%
\pgfpathlineto{\pgfqpoint{0.967450in}{1.393056in}}%
\pgfpathlineto{\pgfqpoint{0.968535in}{1.393806in}}%
\pgfpathlineto{\pgfqpoint{0.970264in}{1.394910in}}%
\pgfpathlineto{\pgfqpoint{0.971348in}{1.395699in}}%
\pgfpathlineto{\pgfqpoint{0.972964in}{1.396764in}}%
\pgfpathlineto{\pgfqpoint{0.974011in}{1.397711in}}%
\pgfpathlineto{\pgfqpoint{0.976413in}{1.398815in}}%
\pgfpathlineto{\pgfqpoint{0.977441in}{1.399564in}}%
\pgfpathlineto{\pgfqpoint{0.979544in}{1.400669in}}%
\pgfpathlineto{\pgfqpoint{0.980628in}{1.401339in}}%
\pgfpathlineto{\pgfqpoint{0.982264in}{1.402444in}}%
\pgfpathlineto{\pgfqpoint{0.983320in}{1.403075in}}%
\pgfpathlineto{\pgfqpoint{0.985254in}{1.404179in}}%
\pgfpathlineto{\pgfqpoint{0.986357in}{1.404731in}}%
\pgfpathlineto{\pgfqpoint{0.988282in}{1.405836in}}%
\pgfpathlineto{\pgfqpoint{0.989385in}{1.406546in}}%
\pgfpathlineto{\pgfqpoint{0.992666in}{1.407650in}}%
\pgfpathlineto{\pgfqpoint{0.993759in}{1.408676in}}%
\pgfpathlineto{\pgfqpoint{0.996703in}{1.409741in}}%
\pgfpathlineto{\pgfqpoint{0.997647in}{1.410096in}}%
\pgfpathlineto{\pgfqpoint{1.000152in}{1.411161in}}%
\pgfpathlineto{\pgfqpoint{1.001255in}{1.411397in}}%
\pgfpathlineto{\pgfqpoint{1.003451in}{1.412502in}}%
\pgfpathlineto{\pgfqpoint{1.004554in}{1.413251in}}%
\pgfpathlineto{\pgfqpoint{1.007245in}{1.414355in}}%
\pgfpathlineto{\pgfqpoint{1.008283in}{1.415065in}}%
\pgfpathlineto{\pgfqpoint{1.010227in}{1.416170in}}%
\pgfpathlineto{\pgfqpoint{1.011217in}{1.416919in}}%
\pgfpathlineto{\pgfqpoint{1.013600in}{1.418024in}}%
\pgfpathlineto{\pgfqpoint{1.014638in}{1.418694in}}%
\pgfpathlineto{\pgfqpoint{1.017703in}{1.419798in}}%
\pgfpathlineto{\pgfqpoint{1.018713in}{1.420430in}}%
\pgfpathlineto{\pgfqpoint{1.021395in}{1.421495in}}%
\pgfpathlineto{\pgfqpoint{1.022367in}{1.422559in}}%
\pgfpathlineto{\pgfqpoint{1.024255in}{1.423664in}}%
\pgfpathlineto{\pgfqpoint{1.025180in}{1.423979in}}%
\pgfpathlineto{\pgfqpoint{1.027479in}{1.425084in}}%
\pgfpathlineto{\pgfqpoint{1.028554in}{1.425557in}}%
\pgfpathlineto{\pgfqpoint{1.030292in}{1.426661in}}%
\pgfpathlineto{\pgfqpoint{1.031274in}{1.427174in}}%
\pgfpathlineto{\pgfqpoint{1.033489in}{1.428279in}}%
\pgfpathlineto{\pgfqpoint{1.034573in}{1.428713in}}%
\pgfpathlineto{\pgfqpoint{1.036750in}{1.429817in}}%
\pgfpathlineto{\pgfqpoint{1.037704in}{1.430330in}}%
\pgfpathlineto{\pgfqpoint{1.040133in}{1.431434in}}%
\pgfpathlineto{\pgfqpoint{1.041218in}{1.431947in}}%
\pgfpathlineto{\pgfqpoint{1.043274in}{1.433051in}}%
\pgfpathlineto{\pgfqpoint{1.044246in}{1.433801in}}%
\pgfpathlineto{\pgfqpoint{1.047274in}{1.434905in}}%
\pgfpathlineto{\pgfqpoint{1.048040in}{1.435260in}}%
\pgfpathlineto{\pgfqpoint{1.051040in}{1.436364in}}%
\pgfpathlineto{\pgfqpoint{1.052115in}{1.436838in}}%
\pgfpathlineto{\pgfqpoint{1.054582in}{1.437942in}}%
\pgfpathlineto{\pgfqpoint{1.055648in}{1.438613in}}%
\pgfpathlineto{\pgfqpoint{1.058386in}{1.439717in}}%
\pgfpathlineto{\pgfqpoint{1.059302in}{1.440072in}}%
\pgfpathlineto{\pgfqpoint{1.061891in}{1.441176in}}%
\pgfpathlineto{\pgfqpoint{1.062975in}{1.441452in}}%
\pgfpathlineto{\pgfqpoint{1.065180in}{1.442557in}}%
\pgfpathlineto{\pgfqpoint{1.065825in}{1.442754in}}%
\pgfpathlineto{\pgfqpoint{1.069302in}{1.443858in}}%
\pgfpathlineto{\pgfqpoint{1.070358in}{1.444371in}}%
\pgfpathlineto{\pgfqpoint{1.073723in}{1.445436in}}%
\pgfpathlineto{\pgfqpoint{1.074760in}{1.445949in}}%
\pgfpathlineto{\pgfqpoint{1.078068in}{1.447053in}}%
\pgfpathlineto{\pgfqpoint{1.079087in}{1.447408in}}%
\pgfpathlineto{\pgfqpoint{1.082293in}{1.448513in}}%
\pgfpathlineto{\pgfqpoint{1.083377in}{1.448947in}}%
\pgfpathlineto{\pgfqpoint{1.085779in}{1.450051in}}%
\pgfpathlineto{\pgfqpoint{1.086835in}{1.450248in}}%
\pgfpathlineto{\pgfqpoint{1.089928in}{1.451313in}}%
\pgfpathlineto{\pgfqpoint{1.090798in}{1.451865in}}%
\pgfpathlineto{\pgfqpoint{1.093826in}{1.452970in}}%
\pgfpathlineto{\pgfqpoint{1.094872in}{1.453246in}}%
\pgfpathlineto{\pgfqpoint{1.097900in}{1.454350in}}%
\pgfpathlineto{\pgfqpoint{1.098938in}{1.454784in}}%
\pgfpathlineto{\pgfqpoint{1.101564in}{1.455888in}}%
\pgfpathlineto{\pgfqpoint{1.102630in}{1.456165in}}%
\pgfpathlineto{\pgfqpoint{1.105592in}{1.457269in}}%
\pgfpathlineto{\pgfqpoint{1.106686in}{1.457900in}}%
\pgfpathlineto{\pgfqpoint{1.109041in}{1.459004in}}%
\pgfpathlineto{\pgfqpoint{1.110144in}{1.459754in}}%
\pgfpathlineto{\pgfqpoint{1.112620in}{1.460858in}}%
\pgfpathlineto{\pgfqpoint{1.113714in}{1.461292in}}%
\pgfpathlineto{\pgfqpoint{1.116714in}{1.462396in}}%
\pgfpathlineto{\pgfqpoint{1.117536in}{1.462673in}}%
\pgfpathlineto{\pgfqpoint{1.122190in}{1.463777in}}%
\pgfpathlineto{\pgfqpoint{1.123153in}{1.464290in}}%
\pgfpathlineto{\pgfqpoint{1.127443in}{1.465394in}}%
\pgfpathlineto{\pgfqpoint{1.128508in}{1.465946in}}%
\pgfpathlineto{\pgfqpoint{1.132331in}{1.467051in}}%
\pgfpathlineto{\pgfqpoint{1.133303in}{1.467603in}}%
\pgfpathlineto{\pgfqpoint{1.135864in}{1.468707in}}%
\pgfpathlineto{\pgfqpoint{1.136807in}{1.468944in}}%
\pgfpathlineto{\pgfqpoint{1.139452in}{1.470048in}}%
\pgfpathlineto{\pgfqpoint{1.140443in}{1.470443in}}%
\pgfpathlineto{\pgfqpoint{1.143368in}{1.471547in}}%
\pgfpathlineto{\pgfqpoint{1.144116in}{1.471981in}}%
\pgfpathlineto{\pgfqpoint{1.148462in}{1.473085in}}%
\pgfpathlineto{\pgfqpoint{1.148957in}{1.473283in}}%
\pgfpathlineto{\pgfqpoint{1.153602in}{1.474387in}}%
\pgfpathlineto{\pgfqpoint{1.154462in}{1.474781in}}%
\pgfpathlineto{\pgfqpoint{1.158312in}{1.475886in}}%
\pgfpathlineto{\pgfqpoint{1.159406in}{1.476044in}}%
\pgfpathlineto{\pgfqpoint{1.163322in}{1.477109in}}%
\pgfpathlineto{\pgfqpoint{1.164322in}{1.477542in}}%
\pgfpathlineto{\pgfqpoint{1.167387in}{1.478647in}}%
\pgfpathlineto{\pgfqpoint{1.168490in}{1.479041in}}%
\pgfpathlineto{\pgfqpoint{1.171584in}{1.480146in}}%
\pgfpathlineto{\pgfqpoint{1.172640in}{1.480619in}}%
\pgfpathlineto{\pgfqpoint{1.176098in}{1.481723in}}%
\pgfpathlineto{\pgfqpoint{1.177116in}{1.482157in}}%
\pgfpathlineto{\pgfqpoint{1.180855in}{1.483262in}}%
\pgfpathlineto{\pgfqpoint{1.181864in}{1.483893in}}%
\pgfpathlineto{\pgfqpoint{1.185406in}{1.484997in}}%
\pgfpathlineto{\pgfqpoint{1.186462in}{1.485313in}}%
\pgfpathlineto{\pgfqpoint{1.191331in}{1.486417in}}%
\pgfpathlineto{\pgfqpoint{1.192201in}{1.486614in}}%
\pgfpathlineto{\pgfqpoint{1.195995in}{1.487719in}}%
\pgfpathlineto{\pgfqpoint{1.196911in}{1.488034in}}%
\pgfpathlineto{\pgfqpoint{1.200958in}{1.489138in}}%
\pgfpathlineto{\pgfqpoint{1.201986in}{1.489454in}}%
\pgfpathlineto{\pgfqpoint{1.205696in}{1.490558in}}%
\pgfpathlineto{\pgfqpoint{1.206734in}{1.490992in}}%
\pgfpathlineto{\pgfqpoint{1.212603in}{1.492097in}}%
\pgfpathlineto{\pgfqpoint{1.213706in}{1.492333in}}%
\pgfpathlineto{\pgfqpoint{1.218182in}{1.493438in}}%
\pgfpathlineto{\pgfqpoint{1.219051in}{1.493872in}}%
\pgfpathlineto{\pgfqpoint{1.223724in}{1.494976in}}%
\pgfpathlineto{\pgfqpoint{1.224771in}{1.495252in}}%
\pgfpathlineto{\pgfqpoint{1.230238in}{1.496356in}}%
\pgfpathlineto{\pgfqpoint{1.231341in}{1.496711in}}%
\pgfpathlineto{\pgfqpoint{1.236248in}{1.497816in}}%
\pgfpathlineto{\pgfqpoint{1.237257in}{1.498131in}}%
\pgfpathlineto{\pgfqpoint{1.243388in}{1.499236in}}%
\pgfpathlineto{\pgfqpoint{1.243930in}{1.499433in}}%
\pgfpathlineto{\pgfqpoint{1.249575in}{1.500537in}}%
\pgfpathlineto{\pgfqpoint{1.250304in}{1.500774in}}%
\pgfpathlineto{\pgfqpoint{1.255613in}{1.501878in}}%
\pgfpathlineto{\pgfqpoint{1.256491in}{1.502312in}}%
\pgfpathlineto{\pgfqpoint{1.261108in}{1.503417in}}%
\pgfpathlineto{\pgfqpoint{1.262080in}{1.503614in}}%
\pgfpathlineto{\pgfqpoint{1.267089in}{1.504718in}}%
\pgfpathlineto{\pgfqpoint{1.268052in}{1.504876in}}%
\pgfpathlineto{\pgfqpoint{1.274145in}{1.505980in}}%
\pgfpathlineto{\pgfqpoint{1.274912in}{1.506178in}}%
\pgfpathlineto{\pgfqpoint{1.279286in}{1.507282in}}%
\pgfpathlineto{\pgfqpoint{1.280267in}{1.507913in}}%
\pgfpathlineto{\pgfqpoint{1.285155in}{1.509018in}}%
\pgfpathlineto{\pgfqpoint{1.285856in}{1.509333in}}%
\pgfpathlineto{\pgfqpoint{1.293463in}{1.510437in}}%
\pgfpathlineto{\pgfqpoint{1.294351in}{1.510674in}}%
\pgfpathlineto{\pgfqpoint{1.298950in}{1.511779in}}%
\pgfpathlineto{\pgfqpoint{1.299744in}{1.512212in}}%
\pgfpathlineto{\pgfqpoint{1.307221in}{1.513317in}}%
\pgfpathlineto{\pgfqpoint{1.308108in}{1.513632in}}%
\pgfpathlineto{\pgfqpoint{1.314015in}{1.514737in}}%
\pgfpathlineto{\pgfqpoint{1.315109in}{1.515092in}}%
\pgfpathlineto{\pgfqpoint{1.321744in}{1.516196in}}%
\pgfpathlineto{\pgfqpoint{1.322763in}{1.516472in}}%
\pgfpathlineto{\pgfqpoint{1.328632in}{1.517577in}}%
\pgfpathlineto{\pgfqpoint{1.329669in}{1.518010in}}%
\pgfpathlineto{\pgfqpoint{1.336539in}{1.519115in}}%
\pgfpathlineto{\pgfqpoint{1.337623in}{1.519312in}}%
\pgfpathlineto{\pgfqpoint{1.343193in}{1.520416in}}%
\pgfpathlineto{\pgfqpoint{1.343885in}{1.520574in}}%
\pgfpathlineto{\pgfqpoint{1.349866in}{1.521679in}}%
\pgfpathlineto{\pgfqpoint{1.350913in}{1.521876in}}%
\pgfpathlineto{\pgfqpoint{1.358118in}{1.522980in}}%
\pgfpathlineto{\pgfqpoint{1.359203in}{1.523335in}}%
\pgfpathlineto{\pgfqpoint{1.364698in}{1.524440in}}%
\pgfpathlineto{\pgfqpoint{1.365782in}{1.524755in}}%
\pgfpathlineto{\pgfqpoint{1.370651in}{1.525859in}}%
\pgfpathlineto{\pgfqpoint{1.371530in}{1.526017in}}%
\pgfpathlineto{\pgfqpoint{1.378352in}{1.527122in}}%
\pgfpathlineto{\pgfqpoint{1.378969in}{1.527240in}}%
\pgfpathlineto{\pgfqpoint{1.385390in}{1.528344in}}%
\pgfpathlineto{\pgfqpoint{1.386212in}{1.528423in}}%
\pgfpathlineto{\pgfqpoint{1.392324in}{1.529488in}}%
\pgfpathlineto{\pgfqpoint{1.392558in}{1.529646in}}%
\pgfpathlineto{\pgfqpoint{1.400007in}{1.530750in}}%
\pgfpathlineto{\pgfqpoint{1.401044in}{1.530948in}}%
\pgfpathlineto{\pgfqpoint{1.409652in}{1.532052in}}%
\pgfpathlineto{\pgfqpoint{1.410362in}{1.532249in}}%
\pgfpathlineto{\pgfqpoint{1.417558in}{1.533354in}}%
\pgfpathlineto{\pgfqpoint{1.418353in}{1.533432in}}%
\pgfpathlineto{\pgfqpoint{1.424914in}{1.534537in}}%
\pgfpathlineto{\pgfqpoint{1.425970in}{1.534734in}}%
\pgfpathlineto{\pgfqpoint{1.434400in}{1.535838in}}%
\pgfpathlineto{\pgfqpoint{1.434400in}{1.535878in}}%
\pgfpathlineto{\pgfqpoint{1.441512in}{1.536982in}}%
\pgfpathlineto{\pgfqpoint{1.442596in}{1.537219in}}%
\pgfpathlineto{\pgfqpoint{1.451652in}{1.538323in}}%
\pgfpathlineto{\pgfqpoint{1.452549in}{1.538521in}}%
\pgfpathlineto{\pgfqpoint{1.462886in}{1.539586in}}%
\pgfpathlineto{\pgfqpoint{1.463793in}{1.539940in}}%
\pgfpathlineto{\pgfqpoint{1.472381in}{1.541045in}}%
\pgfpathlineto{\pgfqpoint{1.473381in}{1.541242in}}%
\pgfpathlineto{\pgfqpoint{1.481082in}{1.542346in}}%
\pgfpathlineto{\pgfqpoint{1.482110in}{1.542544in}}%
\pgfpathlineto{\pgfqpoint{1.492662in}{1.543648in}}%
\pgfpathlineto{\pgfqpoint{1.493709in}{1.543924in}}%
\pgfpathlineto{\pgfqpoint{1.502485in}{1.545029in}}%
\pgfpathlineto{\pgfqpoint{1.503354in}{1.545384in}}%
\pgfpathlineto{\pgfqpoint{1.511728in}{1.546488in}}%
\pgfpathlineto{\pgfqpoint{1.512587in}{1.546882in}}%
\pgfpathlineto{\pgfqpoint{1.522560in}{1.547987in}}%
\pgfpathlineto{\pgfqpoint{1.523410in}{1.548145in}}%
\pgfpathlineto{\pgfqpoint{1.534401in}{1.549209in}}%
\pgfpathlineto{\pgfqpoint{1.535391in}{1.549446in}}%
\pgfpathlineto{\pgfqpoint{1.543242in}{1.550551in}}%
\pgfpathlineto{\pgfqpoint{1.544046in}{1.550629in}}%
\pgfpathlineto{\pgfqpoint{1.553990in}{1.551734in}}%
\pgfpathlineto{\pgfqpoint{1.553990in}{1.551773in}}%
\pgfpathlineto{\pgfqpoint{1.564840in}{1.552878in}}%
\pgfpathlineto{\pgfqpoint{1.565784in}{1.553035in}}%
\pgfpathlineto{\pgfqpoint{1.578261in}{1.554140in}}%
\pgfpathlineto{\pgfqpoint{1.579037in}{1.554298in}}%
\pgfpathlineto{\pgfqpoint{1.593691in}{1.555402in}}%
\pgfpathlineto{\pgfqpoint{1.594373in}{1.555520in}}%
\pgfpathlineto{\pgfqpoint{1.608056in}{1.556625in}}%
\pgfpathlineto{\pgfqpoint{1.609028in}{1.556861in}}%
\pgfpathlineto{\pgfqpoint{1.617019in}{1.557966in}}%
\pgfpathlineto{\pgfqpoint{1.617776in}{1.558123in}}%
\pgfpathlineto{\pgfqpoint{1.628374in}{1.559228in}}%
\pgfpathlineto{\pgfqpoint{1.629421in}{1.559425in}}%
\pgfpathlineto{\pgfqpoint{1.638552in}{1.560529in}}%
\pgfpathlineto{\pgfqpoint{1.639252in}{1.560687in}}%
\pgfpathlineto{\pgfqpoint{1.649982in}{1.561792in}}%
\pgfpathlineto{\pgfqpoint{1.650626in}{1.561949in}}%
\pgfpathlineto{\pgfqpoint{1.663028in}{1.563054in}}%
\pgfpathlineto{\pgfqpoint{1.664103in}{1.563290in}}%
\pgfpathlineto{\pgfqpoint{1.675029in}{1.564395in}}%
\pgfpathlineto{\pgfqpoint{1.675029in}{1.564434in}}%
\pgfpathlineto{\pgfqpoint{1.689001in}{1.565539in}}%
\pgfpathlineto{\pgfqpoint{1.690038in}{1.565736in}}%
\pgfpathlineto{\pgfqpoint{1.704721in}{1.566840in}}%
\pgfpathlineto{\pgfqpoint{1.704721in}{1.566919in}}%
\pgfpathlineto{\pgfqpoint{1.717954in}{1.568024in}}%
\pgfpathlineto{\pgfqpoint{1.718235in}{1.568142in}}%
\pgfpathlineto{\pgfqpoint{1.732338in}{1.569246in}}%
\pgfpathlineto{\pgfqpoint{1.733095in}{1.569365in}}%
\pgfpathlineto{\pgfqpoint{1.744366in}{1.570469in}}%
\pgfpathlineto{\pgfqpoint{1.745114in}{1.570548in}}%
\pgfpathlineto{\pgfqpoint{1.758712in}{1.571652in}}%
\pgfpathlineto{\pgfqpoint{1.759618in}{1.571810in}}%
\pgfpathlineto{\pgfqpoint{1.773217in}{1.572914in}}%
\pgfpathlineto{\pgfqpoint{1.774002in}{1.572993in}}%
\pgfpathlineto{\pgfqpoint{1.787946in}{1.574098in}}%
\pgfpathlineto{\pgfqpoint{1.788974in}{1.574216in}}%
\pgfpathlineto{\pgfqpoint{1.799525in}{1.575320in}}%
\pgfpathlineto{\pgfqpoint{1.800226in}{1.575478in}}%
\pgfpathlineto{\pgfqpoint{1.813862in}{1.576583in}}%
\pgfpathlineto{\pgfqpoint{1.813862in}{1.576622in}}%
\pgfpathlineto{\pgfqpoint{1.827591in}{1.577726in}}%
\pgfpathlineto{\pgfqpoint{1.828619in}{1.577963in}}%
\pgfpathlineto{\pgfqpoint{1.841788in}{1.579067in}}%
\pgfpathlineto{\pgfqpoint{1.842890in}{1.579186in}}%
\pgfpathlineto{\pgfqpoint{1.855199in}{1.580290in}}%
\pgfpathlineto{\pgfqpoint{1.855741in}{1.580369in}}%
\pgfpathlineto{\pgfqpoint{1.869517in}{1.581473in}}%
\pgfpathlineto{\pgfqpoint{1.870246in}{1.581552in}}%
\pgfpathlineto{\pgfqpoint{1.888003in}{1.582657in}}%
\pgfpathlineto{\pgfqpoint{1.888769in}{1.582815in}}%
\pgfpathlineto{\pgfqpoint{1.904377in}{1.583919in}}%
\pgfpathlineto{\pgfqpoint{1.905433in}{1.584116in}}%
\pgfpathlineto{\pgfqpoint{1.920760in}{1.585221in}}%
\pgfpathlineto{\pgfqpoint{1.921714in}{1.585418in}}%
\pgfpathlineto{\pgfqpoint{1.932667in}{1.586522in}}%
\pgfpathlineto{\pgfqpoint{1.933321in}{1.586601in}}%
\pgfpathlineto{\pgfqpoint{1.946106in}{1.587705in}}%
\pgfpathlineto{\pgfqpoint{1.947116in}{1.587784in}}%
\pgfpathlineto{\pgfqpoint{1.961172in}{1.588889in}}%
\pgfpathlineto{\pgfqpoint{1.962266in}{1.589046in}}%
\pgfpathlineto{\pgfqpoint{1.974518in}{1.590151in}}%
\pgfpathlineto{\pgfqpoint{1.975499in}{1.590388in}}%
\pgfpathlineto{\pgfqpoint{1.986079in}{1.591492in}}%
\pgfpathlineto{\pgfqpoint{1.986845in}{1.591650in}}%
\pgfpathlineto{\pgfqpoint{1.997425in}{1.592754in}}%
\pgfpathlineto{\pgfqpoint{1.998528in}{1.592991in}}%
\pgfpathlineto{\pgfqpoint{2.006621in}{1.594095in}}%
\pgfpathlineto{\pgfqpoint{2.007294in}{1.594292in}}%
\pgfpathlineto{\pgfqpoint{2.018219in}{1.595397in}}%
\pgfpathlineto{\pgfqpoint{2.018948in}{1.595633in}}%
\pgfpathlineto{\pgfqpoint{2.025556in}{1.596738in}}%
\pgfpathlineto{\pgfqpoint{2.025948in}{1.596974in}}%
\pgfpathlineto{\pgfqpoint{2.029715in}{1.598079in}}%
\pgfpathlineto{\pgfqpoint{2.030771in}{1.598592in}}%
\pgfpathlineto{\pgfqpoint{2.032687in}{1.599696in}}%
\pgfpathlineto{\pgfqpoint{2.033126in}{1.601944in}}%
\pgfpathlineto{\pgfqpoint{2.033126in}{1.601944in}}%
\pgfusepath{stroke}%
\end{pgfscope}%
\begin{pgfscope}%
\pgfsetrectcap%
\pgfsetmiterjoin%
\pgfsetlinewidth{0.803000pt}%
\definecolor{currentstroke}{rgb}{0.000000,0.000000,0.000000}%
\pgfsetstrokecolor{currentstroke}%
\pgfsetdash{}{0pt}%
\pgfpathmoveto{\pgfqpoint{0.553581in}{0.499444in}}%
\pgfpathlineto{\pgfqpoint{0.553581in}{1.654444in}}%
\pgfusepath{stroke}%
\end{pgfscope}%
\begin{pgfscope}%
\pgfsetrectcap%
\pgfsetmiterjoin%
\pgfsetlinewidth{0.803000pt}%
\definecolor{currentstroke}{rgb}{0.000000,0.000000,0.000000}%
\pgfsetstrokecolor{currentstroke}%
\pgfsetdash{}{0pt}%
\pgfpathmoveto{\pgfqpoint{2.103581in}{0.499444in}}%
\pgfpathlineto{\pgfqpoint{2.103581in}{1.654444in}}%
\pgfusepath{stroke}%
\end{pgfscope}%
\begin{pgfscope}%
\pgfsetrectcap%
\pgfsetmiterjoin%
\pgfsetlinewidth{0.803000pt}%
\definecolor{currentstroke}{rgb}{0.000000,0.000000,0.000000}%
\pgfsetstrokecolor{currentstroke}%
\pgfsetdash{}{0pt}%
\pgfpathmoveto{\pgfqpoint{0.553581in}{0.499444in}}%
\pgfpathlineto{\pgfqpoint{2.103581in}{0.499444in}}%
\pgfusepath{stroke}%
\end{pgfscope}%
\begin{pgfscope}%
\pgfsetrectcap%
\pgfsetmiterjoin%
\pgfsetlinewidth{0.803000pt}%
\definecolor{currentstroke}{rgb}{0.000000,0.000000,0.000000}%
\pgfsetstrokecolor{currentstroke}%
\pgfsetdash{}{0pt}%
\pgfpathmoveto{\pgfqpoint{0.553581in}{1.654444in}}%
\pgfpathlineto{\pgfqpoint{2.103581in}{1.654444in}}%
\pgfusepath{stroke}%
\end{pgfscope}%
\begin{pgfscope}%
\pgfsetbuttcap%
\pgfsetmiterjoin%
\definecolor{currentfill}{rgb}{1.000000,1.000000,1.000000}%
\pgfsetfillcolor{currentfill}%
\pgfsetlinewidth{0.000000pt}%
\definecolor{currentstroke}{rgb}{0.000000,0.000000,0.000000}%
\pgfsetstrokecolor{currentstroke}%
\pgfsetstrokeopacity{0.000000}%
\pgfsetdash{}{0pt}%
\pgfpathmoveto{\pgfqpoint{1.286928in}{1.448926in}}%
\pgfpathlineto{\pgfqpoint{1.686650in}{1.448926in}}%
\pgfpathlineto{\pgfqpoint{1.686650in}{1.655593in}}%
\pgfpathlineto{\pgfqpoint{1.286928in}{1.655593in}}%
\pgfpathlineto{\pgfqpoint{1.286928in}{1.448926in}}%
\pgfpathclose%
\pgfusepath{fill}%
\end{pgfscope}%
\begin{pgfscope}%
\definecolor{textcolor}{rgb}{0.000000,0.000000,0.000000}%
\pgfsetstrokecolor{textcolor}%
\pgfsetfillcolor{textcolor}%
\pgftext[x=1.328595in,y=1.517537in,left,base]{\color{textcolor}\rmfamily\fontsize{10.000000}{12.000000}\selectfont 0.338}%
\end{pgfscope}%
\begin{pgfscope}%
\pgfsetbuttcap%
\pgfsetmiterjoin%
\definecolor{currentfill}{rgb}{1.000000,1.000000,1.000000}%
\pgfsetfillcolor{currentfill}%
\pgfsetlinewidth{0.000000pt}%
\definecolor{currentstroke}{rgb}{0.000000,0.000000,0.000000}%
\pgfsetstrokecolor{currentstroke}%
\pgfsetstrokeopacity{0.000000}%
\pgfsetdash{}{0pt}%
\pgfpathmoveto{\pgfqpoint{0.687846in}{1.008353in}}%
\pgfpathlineto{\pgfqpoint{1.087569in}{1.008353in}}%
\pgfpathlineto{\pgfqpoint{1.087569in}{1.215019in}}%
\pgfpathlineto{\pgfqpoint{0.687846in}{1.215019in}}%
\pgfpathlineto{\pgfqpoint{0.687846in}{1.008353in}}%
\pgfpathclose%
\pgfusepath{fill}%
\end{pgfscope}%
\begin{pgfscope}%
\definecolor{textcolor}{rgb}{0.000000,0.000000,0.000000}%
\pgfsetstrokecolor{textcolor}%
\pgfsetfillcolor{textcolor}%
\pgftext[x=0.729513in,y=1.076964in,left,base]{\color{textcolor}\rmfamily\fontsize{10.000000}{12.000000}\selectfont 0.658}%
\end{pgfscope}%
\begin{pgfscope}%
\pgfsetbuttcap%
\pgfsetmiterjoin%
\definecolor{currentfill}{rgb}{1.000000,1.000000,1.000000}%
\pgfsetfillcolor{currentfill}%
\pgfsetfillopacity{0.800000}%
\pgfsetlinewidth{1.003750pt}%
\definecolor{currentstroke}{rgb}{0.800000,0.800000,0.800000}%
\pgfsetstrokecolor{currentstroke}%
\pgfsetstrokeopacity{0.800000}%
\pgfsetdash{}{0pt}%
\pgfpathmoveto{\pgfqpoint{0.840525in}{0.568889in}}%
\pgfpathlineto{\pgfqpoint{2.006358in}{0.568889in}}%
\pgfpathquadraticcurveto{\pgfqpoint{2.034136in}{0.568889in}}{\pgfqpoint{2.034136in}{0.596666in}}%
\pgfpathlineto{\pgfqpoint{2.034136in}{0.791111in}}%
\pgfpathquadraticcurveto{\pgfqpoint{2.034136in}{0.818888in}}{\pgfqpoint{2.006358in}{0.818888in}}%
\pgfpathlineto{\pgfqpoint{0.840525in}{0.818888in}}%
\pgfpathquadraticcurveto{\pgfqpoint{0.812747in}{0.818888in}}{\pgfqpoint{0.812747in}{0.791111in}}%
\pgfpathlineto{\pgfqpoint{0.812747in}{0.596666in}}%
\pgfpathquadraticcurveto{\pgfqpoint{0.812747in}{0.568889in}}{\pgfqpoint{0.840525in}{0.568889in}}%
\pgfpathlineto{\pgfqpoint{0.840525in}{0.568889in}}%
\pgfpathclose%
\pgfusepath{stroke,fill}%
\end{pgfscope}%
\begin{pgfscope}%
\pgfsetrectcap%
\pgfsetroundjoin%
\pgfsetlinewidth{1.505625pt}%
\definecolor{currentstroke}{rgb}{0.000000,0.000000,0.000000}%
\pgfsetstrokecolor{currentstroke}%
\pgfsetdash{}{0pt}%
\pgfpathmoveto{\pgfqpoint{0.868303in}{0.707777in}}%
\pgfpathlineto{\pgfqpoint{1.007192in}{0.707777in}}%
\pgfpathlineto{\pgfqpoint{1.146081in}{0.707777in}}%
\pgfusepath{stroke}%
\end{pgfscope}%
\begin{pgfscope}%
\definecolor{textcolor}{rgb}{0.000000,0.000000,0.000000}%
\pgfsetstrokecolor{textcolor}%
\pgfsetfillcolor{textcolor}%
\pgftext[x=1.257192in,y=0.659166in,left,base]{\color{textcolor}\rmfamily\fontsize{10.000000}{12.000000}\selectfont AUC 0.840)}%
\end{pgfscope}%
\end{pgfpicture}%
\makeatother%
\endgroup%

  &
\vspace{0pt} 
  
\begin{tabular}{cc|c|c|}
	&\multicolumn{1}{c}{}& \multicolumn{2}{c}{Prediction} \cr
	&\multicolumn{1}{c}{} & \multicolumn{1}{c}{N} & \multicolumn{1}{c}{P} \cr\cline{3-4}
	\multirow{2}{*}{\rotatebox[origin=c]{90}{Actual}}&N & 117,929 & 32,842 \vrule width 0pt height 10pt depth 2pt \cr\cline{3-4}
	&P & 5,928 & 20,693 \vrule width 0pt height 10pt depth 2pt \cr\cline{3-4}
\end{tabular}

\begin{center}
\begin{tabular}{ll}
0.387 & Precision \cr 
0.777 & Recall \cr 
0.516 & F1 \cr 
\end{tabular}
\end{center}
  
\end{tabular}

Note that with threshold $p=0.5$ the false positives ($FP=32,842$) are less than twice as many as the true positives ($TP=20,693$), but that does not satisfy our requirement that $\Delta FP/\Delta TP < 2.0$.  The plot below shows the rate of change as a function of $p$, and that $\Delta FP/\Delta TP = 2.0$ when $p=0.635$.  

%% Creator: Matplotlib, PGF backend
%%
%% To include the figure in your LaTeX document, write
%%   \input{<filename>.pgf}
%%
%% Make sure the required packages are loaded in your preamble
%%   \usepackage{pgf}
%%
%% Also ensure that all the required font packages are loaded; for instance,
%% the lmodern package is sometimes necessary when using math font.
%%   \usepackage{lmodern}
%%
%% Figures using additional raster images can only be included by \input if
%% they are in the same directory as the main LaTeX file. For loading figures
%% from other directories you can use the `import` package
%%   \usepackage{import}
%%
%% and then include the figures with
%%   \import{<path to file>}{<filename>.pgf}
%%
%% Matplotlib used the following preamble
%%   
%%   \usepackage{fontspec}
%%   \makeatletter\@ifpackageloaded{underscore}{}{\usepackage[strings]{underscore}}\makeatother
%%
\begingroup%
\makeatletter%
\begin{pgfpicture}%
\pgfpathrectangle{\pgfpointorigin}{\pgfqpoint{2.247807in}{1.754444in}}%
\pgfusepath{use as bounding box, clip}%
\begin{pgfscope}%
\pgfsetbuttcap%
\pgfsetmiterjoin%
\definecolor{currentfill}{rgb}{1.000000,1.000000,1.000000}%
\pgfsetfillcolor{currentfill}%
\pgfsetlinewidth{0.000000pt}%
\definecolor{currentstroke}{rgb}{1.000000,1.000000,1.000000}%
\pgfsetstrokecolor{currentstroke}%
\pgfsetdash{}{0pt}%
\pgfpathmoveto{\pgfqpoint{0.000000in}{0.000000in}}%
\pgfpathlineto{\pgfqpoint{2.247807in}{0.000000in}}%
\pgfpathlineto{\pgfqpoint{2.247807in}{1.754444in}}%
\pgfpathlineto{\pgfqpoint{0.000000in}{1.754444in}}%
\pgfpathlineto{\pgfqpoint{0.000000in}{0.000000in}}%
\pgfpathclose%
\pgfusepath{fill}%
\end{pgfscope}%
\begin{pgfscope}%
\pgfsetbuttcap%
\pgfsetmiterjoin%
\definecolor{currentfill}{rgb}{1.000000,1.000000,1.000000}%
\pgfsetfillcolor{currentfill}%
\pgfsetlinewidth{0.000000pt}%
\definecolor{currentstroke}{rgb}{0.000000,0.000000,0.000000}%
\pgfsetstrokecolor{currentstroke}%
\pgfsetstrokeopacity{0.000000}%
\pgfsetdash{}{0pt}%
\pgfpathmoveto{\pgfqpoint{0.530556in}{0.499444in}}%
\pgfpathlineto{\pgfqpoint{2.080556in}{0.499444in}}%
\pgfpathlineto{\pgfqpoint{2.080556in}{1.654444in}}%
\pgfpathlineto{\pgfqpoint{0.530556in}{1.654444in}}%
\pgfpathlineto{\pgfqpoint{0.530556in}{0.499444in}}%
\pgfpathclose%
\pgfusepath{fill}%
\end{pgfscope}%
\begin{pgfscope}%
\pgfsetbuttcap%
\pgfsetroundjoin%
\definecolor{currentfill}{rgb}{0.000000,0.000000,0.000000}%
\pgfsetfillcolor{currentfill}%
\pgfsetlinewidth{0.803000pt}%
\definecolor{currentstroke}{rgb}{0.000000,0.000000,0.000000}%
\pgfsetstrokecolor{currentstroke}%
\pgfsetdash{}{0pt}%
\pgfsys@defobject{currentmarker}{\pgfqpoint{0.000000in}{-0.048611in}}{\pgfqpoint{0.000000in}{0.000000in}}{%
\pgfpathmoveto{\pgfqpoint{0.000000in}{0.000000in}}%
\pgfpathlineto{\pgfqpoint{0.000000in}{-0.048611in}}%
\pgfusepath{stroke,fill}%
}%
\begin{pgfscope}%
\pgfsys@transformshift{0.601010in}{0.499444in}%
\pgfsys@useobject{currentmarker}{}%
\end{pgfscope}%
\end{pgfscope}%
\begin{pgfscope}%
\definecolor{textcolor}{rgb}{0.000000,0.000000,0.000000}%
\pgfsetstrokecolor{textcolor}%
\pgfsetfillcolor{textcolor}%
\pgftext[x=0.601010in,y=0.402222in,,top]{\color{textcolor}\rmfamily\fontsize{10.000000}{12.000000}\selectfont 0.057}%
\end{pgfscope}%
\begin{pgfscope}%
\pgfsetbuttcap%
\pgfsetroundjoin%
\definecolor{currentfill}{rgb}{0.000000,0.000000,0.000000}%
\pgfsetfillcolor{currentfill}%
\pgfsetlinewidth{0.803000pt}%
\definecolor{currentstroke}{rgb}{0.000000,0.000000,0.000000}%
\pgfsetstrokecolor{currentstroke}%
\pgfsetdash{}{0pt}%
\pgfsys@defobject{currentmarker}{\pgfqpoint{0.000000in}{-0.048611in}}{\pgfqpoint{0.000000in}{0.000000in}}{%
\pgfpathmoveto{\pgfqpoint{0.000000in}{0.000000in}}%
\pgfpathlineto{\pgfqpoint{0.000000in}{-0.048611in}}%
\pgfusepath{stroke,fill}%
}%
\begin{pgfscope}%
\pgfsys@transformshift{2.024334in}{0.499444in}%
\pgfsys@useobject{currentmarker}{}%
\end{pgfscope}%
\end{pgfscope}%
\begin{pgfscope}%
\definecolor{textcolor}{rgb}{0.000000,0.000000,0.000000}%
\pgfsetstrokecolor{textcolor}%
\pgfsetfillcolor{textcolor}%
\pgftext[x=2.024334in,y=0.402222in,,top]{\color{textcolor}\rmfamily\fontsize{10.000000}{12.000000}\selectfont 0.94}%
\end{pgfscope}%
\begin{pgfscope}%
\definecolor{textcolor}{rgb}{0.000000,0.000000,0.000000}%
\pgfsetstrokecolor{textcolor}%
\pgfsetfillcolor{textcolor}%
\pgftext[x=1.305556in,y=0.223333in,,top]{\color{textcolor}\rmfamily\fontsize{10.000000}{12.000000}\selectfont \(\displaystyle p\)}%
\end{pgfscope}%
\begin{pgfscope}%
\pgfsetbuttcap%
\pgfsetroundjoin%
\definecolor{currentfill}{rgb}{0.000000,0.000000,0.000000}%
\pgfsetfillcolor{currentfill}%
\pgfsetlinewidth{0.803000pt}%
\definecolor{currentstroke}{rgb}{0.000000,0.000000,0.000000}%
\pgfsetstrokecolor{currentstroke}%
\pgfsetdash{}{0pt}%
\pgfsys@defobject{currentmarker}{\pgfqpoint{-0.048611in}{0.000000in}}{\pgfqpoint{-0.000000in}{0.000000in}}{%
\pgfpathmoveto{\pgfqpoint{-0.000000in}{0.000000in}}%
\pgfpathlineto{\pgfqpoint{-0.048611in}{0.000000in}}%
\pgfusepath{stroke,fill}%
}%
\begin{pgfscope}%
\pgfsys@transformshift{0.530556in}{0.535050in}%
\pgfsys@useobject{currentmarker}{}%
\end{pgfscope}%
\end{pgfscope}%
\begin{pgfscope}%
\definecolor{textcolor}{rgb}{0.000000,0.000000,0.000000}%
\pgfsetstrokecolor{textcolor}%
\pgfsetfillcolor{textcolor}%
\pgftext[x=0.363889in, y=0.486855in, left, base]{\color{textcolor}\rmfamily\fontsize{10.000000}{12.000000}\selectfont \(\displaystyle {0}\)}%
\end{pgfscope}%
\begin{pgfscope}%
\pgfsetbuttcap%
\pgfsetroundjoin%
\definecolor{currentfill}{rgb}{0.000000,0.000000,0.000000}%
\pgfsetfillcolor{currentfill}%
\pgfsetlinewidth{0.803000pt}%
\definecolor{currentstroke}{rgb}{0.000000,0.000000,0.000000}%
\pgfsetstrokecolor{currentstroke}%
\pgfsetdash{}{0pt}%
\pgfsys@defobject{currentmarker}{\pgfqpoint{-0.048611in}{0.000000in}}{\pgfqpoint{-0.000000in}{0.000000in}}{%
\pgfpathmoveto{\pgfqpoint{-0.000000in}{0.000000in}}%
\pgfpathlineto{\pgfqpoint{-0.048611in}{0.000000in}}%
\pgfusepath{stroke,fill}%
}%
\begin{pgfscope}%
\pgfsys@transformshift{0.530556in}{0.988667in}%
\pgfsys@useobject{currentmarker}{}%
\end{pgfscope}%
\end{pgfscope}%
\begin{pgfscope}%
\definecolor{textcolor}{rgb}{0.000000,0.000000,0.000000}%
\pgfsetstrokecolor{textcolor}%
\pgfsetfillcolor{textcolor}%
\pgftext[x=0.294444in, y=0.940472in, left, base]{\color{textcolor}\rmfamily\fontsize{10.000000}{12.000000}\selectfont \(\displaystyle {20}\)}%
\end{pgfscope}%
\begin{pgfscope}%
\pgfsetbuttcap%
\pgfsetroundjoin%
\definecolor{currentfill}{rgb}{0.000000,0.000000,0.000000}%
\pgfsetfillcolor{currentfill}%
\pgfsetlinewidth{0.803000pt}%
\definecolor{currentstroke}{rgb}{0.000000,0.000000,0.000000}%
\pgfsetstrokecolor{currentstroke}%
\pgfsetdash{}{0pt}%
\pgfsys@defobject{currentmarker}{\pgfqpoint{-0.048611in}{0.000000in}}{\pgfqpoint{-0.000000in}{0.000000in}}{%
\pgfpathmoveto{\pgfqpoint{-0.000000in}{0.000000in}}%
\pgfpathlineto{\pgfqpoint{-0.048611in}{0.000000in}}%
\pgfusepath{stroke,fill}%
}%
\begin{pgfscope}%
\pgfsys@transformshift{0.530556in}{1.442284in}%
\pgfsys@useobject{currentmarker}{}%
\end{pgfscope}%
\end{pgfscope}%
\begin{pgfscope}%
\definecolor{textcolor}{rgb}{0.000000,0.000000,0.000000}%
\pgfsetstrokecolor{textcolor}%
\pgfsetfillcolor{textcolor}%
\pgftext[x=0.294444in, y=1.394090in, left, base]{\color{textcolor}\rmfamily\fontsize{10.000000}{12.000000}\selectfont \(\displaystyle {40}\)}%
\end{pgfscope}%
\begin{pgfscope}%
\definecolor{textcolor}{rgb}{0.000000,0.000000,0.000000}%
\pgfsetstrokecolor{textcolor}%
\pgfsetfillcolor{textcolor}%
\pgftext[x=0.238889in,y=1.076944in,,bottom,rotate=90.000000]{\color{textcolor}\rmfamily\fontsize{10.000000}{12.000000}\selectfont \(\displaystyle \Delta\)FP/\(\displaystyle \Delta\)TP}%
\end{pgfscope}%
\begin{pgfscope}%
\pgfpathrectangle{\pgfqpoint{0.530556in}{0.499444in}}{\pgfqpoint{1.550000in}{1.155000in}}%
\pgfusepath{clip}%
\pgfsetrectcap%
\pgfsetroundjoin%
\pgfsetlinewidth{1.505625pt}%
\definecolor{currentstroke}{rgb}{0.000000,0.000000,0.000000}%
\pgfsetstrokecolor{currentstroke}%
\pgfsetdash{}{0pt}%
\pgfpathmoveto{\pgfqpoint{0.601010in}{1.272877in}}%
\pgfpathlineto{\pgfqpoint{0.615243in}{1.295676in}}%
\pgfpathlineto{\pgfqpoint{0.629477in}{1.322318in}}%
\pgfpathlineto{\pgfqpoint{0.643710in}{1.349201in}}%
\pgfpathlineto{\pgfqpoint{0.657943in}{1.373907in}}%
\pgfpathlineto{\pgfqpoint{0.672176in}{1.401826in}}%
\pgfpathlineto{\pgfqpoint{0.686410in}{1.446886in}}%
\pgfpathlineto{\pgfqpoint{0.700643in}{1.485008in}}%
\pgfpathlineto{\pgfqpoint{0.714876in}{1.524036in}}%
\pgfpathlineto{\pgfqpoint{0.729109in}{1.557080in}}%
\pgfpathlineto{\pgfqpoint{0.743343in}{1.582906in}}%
\pgfpathlineto{\pgfqpoint{0.757576in}{1.601944in}}%
\pgfpathlineto{\pgfqpoint{0.771809in}{1.582511in}}%
\pgfpathlineto{\pgfqpoint{0.786042in}{1.558964in}}%
\pgfpathlineto{\pgfqpoint{0.800276in}{1.530713in}}%
\pgfpathlineto{\pgfqpoint{0.814509in}{1.491860in}}%
\pgfpathlineto{\pgfqpoint{0.828742in}{1.452953in}}%
\pgfpathlineto{\pgfqpoint{0.842975in}{1.415092in}}%
\pgfpathlineto{\pgfqpoint{0.857209in}{1.373402in}}%
\pgfpathlineto{\pgfqpoint{0.871442in}{1.330396in}}%
\pgfpathlineto{\pgfqpoint{0.885675in}{1.288644in}}%
\pgfpathlineto{\pgfqpoint{0.899908in}{1.245897in}}%
\pgfpathlineto{\pgfqpoint{0.914142in}{1.247274in}}%
\pgfpathlineto{\pgfqpoint{0.928375in}{1.240200in}}%
\pgfpathlineto{\pgfqpoint{0.942608in}{1.231740in}}%
\pgfpathlineto{\pgfqpoint{0.956841in}{1.220402in}}%
\pgfpathlineto{\pgfqpoint{0.971075in}{1.207322in}}%
\pgfpathlineto{\pgfqpoint{0.985308in}{1.189906in}}%
\pgfpathlineto{\pgfqpoint{0.999541in}{1.170488in}}%
\pgfpathlineto{\pgfqpoint{1.013774in}{1.147638in}}%
\pgfpathlineto{\pgfqpoint{1.028007in}{1.122992in}}%
\pgfpathlineto{\pgfqpoint{1.042241in}{1.098326in}}%
\pgfpathlineto{\pgfqpoint{1.056474in}{1.057517in}}%
\pgfpathlineto{\pgfqpoint{1.070707in}{1.021370in}}%
\pgfpathlineto{\pgfqpoint{1.084940in}{0.985477in}}%
\pgfpathlineto{\pgfqpoint{1.099174in}{0.953009in}}%
\pgfpathlineto{\pgfqpoint{1.113407in}{0.922104in}}%
\pgfpathlineto{\pgfqpoint{1.127640in}{0.893767in}}%
\pgfpathlineto{\pgfqpoint{1.141873in}{0.867298in}}%
\pgfpathlineto{\pgfqpoint{1.156107in}{0.843749in}}%
\pgfpathlineto{\pgfqpoint{1.170340in}{0.821888in}}%
\pgfpathlineto{\pgfqpoint{1.184573in}{0.801693in}}%
\pgfpathlineto{\pgfqpoint{1.198806in}{0.783256in}}%
\pgfpathlineto{\pgfqpoint{1.213040in}{0.765847in}}%
\pgfpathlineto{\pgfqpoint{1.227273in}{0.749655in}}%
\pgfpathlineto{\pgfqpoint{1.241506in}{0.734242in}}%
\pgfpathlineto{\pgfqpoint{1.255739in}{0.719675in}}%
\pgfpathlineto{\pgfqpoint{1.269973in}{0.706062in}}%
\pgfpathlineto{\pgfqpoint{1.284206in}{0.693403in}}%
\pgfpathlineto{\pgfqpoint{1.298439in}{0.681468in}}%
\pgfpathlineto{\pgfqpoint{1.312672in}{0.670361in}}%
\pgfpathlineto{\pgfqpoint{1.326906in}{0.659938in}}%
\pgfpathlineto{\pgfqpoint{1.341139in}{0.650394in}}%
\pgfpathlineto{\pgfqpoint{1.355372in}{0.641676in}}%
\pgfpathlineto{\pgfqpoint{1.369605in}{0.633677in}}%
\pgfpathlineto{\pgfqpoint{1.383839in}{0.626375in}}%
\pgfpathlineto{\pgfqpoint{1.398072in}{0.619651in}}%
\pgfpathlineto{\pgfqpoint{1.412305in}{0.613565in}}%
\pgfpathlineto{\pgfqpoint{1.426538in}{0.607894in}}%
\pgfpathlineto{\pgfqpoint{1.440771in}{0.602635in}}%
\pgfpathlineto{\pgfqpoint{1.455005in}{0.597863in}}%
\pgfpathlineto{\pgfqpoint{1.469238in}{0.593506in}}%
\pgfpathlineto{\pgfqpoint{1.483471in}{0.589519in}}%
\pgfpathlineto{\pgfqpoint{1.497704in}{0.585757in}}%
\pgfpathlineto{\pgfqpoint{1.511938in}{0.582159in}}%
\pgfpathlineto{\pgfqpoint{1.526171in}{0.578762in}}%
\pgfpathlineto{\pgfqpoint{1.540404in}{0.575562in}}%
\pgfpathlineto{\pgfqpoint{1.554637in}{0.572618in}}%
\pgfpathlineto{\pgfqpoint{1.568871in}{0.569964in}}%
\pgfpathlineto{\pgfqpoint{1.583104in}{0.567510in}}%
\pgfpathlineto{\pgfqpoint{1.597337in}{0.565244in}}%
\pgfpathlineto{\pgfqpoint{1.611570in}{0.563171in}}%
\pgfpathlineto{\pgfqpoint{1.625804in}{0.561210in}}%
\pgfpathlineto{\pgfqpoint{1.640037in}{0.559476in}}%
\pgfpathlineto{\pgfqpoint{1.654270in}{0.558003in}}%
\pgfpathlineto{\pgfqpoint{1.668503in}{0.556655in}}%
\pgfpathlineto{\pgfqpoint{1.682737in}{0.555499in}}%
\pgfpathlineto{\pgfqpoint{1.696970in}{0.554469in}}%
\pgfpathlineto{\pgfqpoint{1.711203in}{0.553542in}}%
\pgfpathlineto{\pgfqpoint{1.725436in}{0.552731in}}%
\pgfpathlineto{\pgfqpoint{1.739670in}{0.552030in}}%
\pgfpathlineto{\pgfqpoint{1.753903in}{0.551944in}}%
\pgfpathlineto{\pgfqpoint{1.768136in}{0.551963in}}%
\pgfpathlineto{\pgfqpoint{1.782369in}{0.552054in}}%
\pgfpathlineto{\pgfqpoint{1.796603in}{0.552228in}}%
\pgfpathlineto{\pgfqpoint{1.810836in}{0.552561in}}%
\pgfpathlineto{\pgfqpoint{1.825069in}{0.553006in}}%
\pgfpathlineto{\pgfqpoint{1.839302in}{0.553578in}}%
\pgfpathlineto{\pgfqpoint{1.853535in}{0.554303in}}%
\pgfpathlineto{\pgfqpoint{1.867769in}{0.555301in}}%
\pgfpathlineto{\pgfqpoint{1.882002in}{0.556527in}}%
\pgfpathlineto{\pgfqpoint{1.896235in}{0.556349in}}%
\pgfpathlineto{\pgfqpoint{1.910468in}{0.556293in}}%
\pgfpathlineto{\pgfqpoint{1.924702in}{0.556340in}}%
\pgfpathlineto{\pgfqpoint{1.938935in}{0.556440in}}%
\pgfpathlineto{\pgfqpoint{1.953168in}{0.556605in}}%
\pgfpathlineto{\pgfqpoint{1.967401in}{0.556690in}}%
\pgfpathlineto{\pgfqpoint{1.981635in}{0.556693in}}%
\pgfpathlineto{\pgfqpoint{1.995868in}{0.556540in}}%
\pgfpathlineto{\pgfqpoint{2.010101in}{0.555985in}}%
\pgfusepath{stroke}%
\end{pgfscope}%
\begin{pgfscope}%
\pgfpathrectangle{\pgfqpoint{0.530556in}{0.499444in}}{\pgfqpoint{1.550000in}{1.155000in}}%
\pgfusepath{clip}%
\pgfsetbuttcap%
\pgfsetroundjoin%
\pgfsetlinewidth{1.505625pt}%
\definecolor{currentstroke}{rgb}{0.000000,0.000000,0.000000}%
\pgfsetstrokecolor{currentstroke}%
\pgfsetdash{{5.550000pt}{2.400000pt}}{0.000000pt}%
\pgfpathmoveto{\pgfqpoint{0.530556in}{0.580412in}}%
\pgfpathlineto{\pgfqpoint{2.080556in}{0.580412in}}%
\pgfusepath{stroke}%
\end{pgfscope}%
\begin{pgfscope}%
\pgfpathrectangle{\pgfqpoint{0.530556in}{0.499444in}}{\pgfqpoint{1.550000in}{1.155000in}}%
\pgfusepath{clip}%
\pgfsetrectcap%
\pgfsetroundjoin%
\pgfsetlinewidth{1.505625pt}%
\definecolor{currentstroke}{rgb}{0.121569,0.466667,0.705882}%
\pgfsetstrokecolor{currentstroke}%
\pgfsetdash{}{0pt}%
\pgfpathmoveto{\pgfqpoint{1.526171in}{0.580412in}}%
\pgfusepath{stroke}%
\end{pgfscope}%
\begin{pgfscope}%
\pgfpathrectangle{\pgfqpoint{0.530556in}{0.499444in}}{\pgfqpoint{1.550000in}{1.155000in}}%
\pgfusepath{clip}%
\pgfsetbuttcap%
\pgfsetroundjoin%
\definecolor{currentfill}{rgb}{0.000000,0.000000,0.000000}%
\pgfsetfillcolor{currentfill}%
\pgfsetlinewidth{1.003750pt}%
\definecolor{currentstroke}{rgb}{0.000000,0.000000,0.000000}%
\pgfsetstrokecolor{currentstroke}%
\pgfsetdash{}{0pt}%
\pgfsys@defobject{currentmarker}{\pgfqpoint{-0.041667in}{-0.041667in}}{\pgfqpoint{0.041667in}{0.041667in}}{%
\pgfpathmoveto{\pgfqpoint{0.000000in}{-0.041667in}}%
\pgfpathcurveto{\pgfqpoint{0.011050in}{-0.041667in}}{\pgfqpoint{0.021649in}{-0.037276in}}{\pgfqpoint{0.029463in}{-0.029463in}}%
\pgfpathcurveto{\pgfqpoint{0.037276in}{-0.021649in}}{\pgfqpoint{0.041667in}{-0.011050in}}{\pgfqpoint{0.041667in}{0.000000in}}%
\pgfpathcurveto{\pgfqpoint{0.041667in}{0.011050in}}{\pgfqpoint{0.037276in}{0.021649in}}{\pgfqpoint{0.029463in}{0.029463in}}%
\pgfpathcurveto{\pgfqpoint{0.021649in}{0.037276in}}{\pgfqpoint{0.011050in}{0.041667in}}{\pgfqpoint{0.000000in}{0.041667in}}%
\pgfpathcurveto{\pgfqpoint{-0.011050in}{0.041667in}}{\pgfqpoint{-0.021649in}{0.037276in}}{\pgfqpoint{-0.029463in}{0.029463in}}%
\pgfpathcurveto{\pgfqpoint{-0.037276in}{0.021649in}}{\pgfqpoint{-0.041667in}{0.011050in}}{\pgfqpoint{-0.041667in}{0.000000in}}%
\pgfpathcurveto{\pgfqpoint{-0.041667in}{-0.011050in}}{\pgfqpoint{-0.037276in}{-0.021649in}}{\pgfqpoint{-0.029463in}{-0.029463in}}%
\pgfpathcurveto{\pgfqpoint{-0.021649in}{-0.037276in}}{\pgfqpoint{-0.011050in}{-0.041667in}}{\pgfqpoint{0.000000in}{-0.041667in}}%
\pgfpathlineto{\pgfqpoint{0.000000in}{-0.041667in}}%
\pgfpathclose%
\pgfusepath{stroke,fill}%
}%
\begin{pgfscope}%
\pgfsys@transformshift{1.526171in}{0.580412in}%
\pgfsys@useobject{currentmarker}{}%
\end{pgfscope}%
\end{pgfscope}%
\begin{pgfscope}%
\pgfsetrectcap%
\pgfsetmiterjoin%
\pgfsetlinewidth{0.803000pt}%
\definecolor{currentstroke}{rgb}{0.000000,0.000000,0.000000}%
\pgfsetstrokecolor{currentstroke}%
\pgfsetdash{}{0pt}%
\pgfpathmoveto{\pgfqpoint{0.530556in}{0.499444in}}%
\pgfpathlineto{\pgfqpoint{0.530556in}{1.654444in}}%
\pgfusepath{stroke}%
\end{pgfscope}%
\begin{pgfscope}%
\pgfsetrectcap%
\pgfsetmiterjoin%
\pgfsetlinewidth{0.803000pt}%
\definecolor{currentstroke}{rgb}{0.000000,0.000000,0.000000}%
\pgfsetstrokecolor{currentstroke}%
\pgfsetdash{}{0pt}%
\pgfpathmoveto{\pgfqpoint{2.080556in}{0.499444in}}%
\pgfpathlineto{\pgfqpoint{2.080556in}{1.654444in}}%
\pgfusepath{stroke}%
\end{pgfscope}%
\begin{pgfscope}%
\pgfsetrectcap%
\pgfsetmiterjoin%
\pgfsetlinewidth{0.803000pt}%
\definecolor{currentstroke}{rgb}{0.000000,0.000000,0.000000}%
\pgfsetstrokecolor{currentstroke}%
\pgfsetdash{}{0pt}%
\pgfpathmoveto{\pgfqpoint{0.530556in}{0.499444in}}%
\pgfpathlineto{\pgfqpoint{2.080556in}{0.499444in}}%
\pgfusepath{stroke}%
\end{pgfscope}%
\begin{pgfscope}%
\pgfsetrectcap%
\pgfsetmiterjoin%
\pgfsetlinewidth{0.803000pt}%
\definecolor{currentstroke}{rgb}{0.000000,0.000000,0.000000}%
\pgfsetstrokecolor{currentstroke}%
\pgfsetdash{}{0pt}%
\pgfpathmoveto{\pgfqpoint{0.530556in}{1.654444in}}%
\pgfpathlineto{\pgfqpoint{2.080556in}{1.654444in}}%
\pgfusepath{stroke}%
\end{pgfscope}%
\begin{pgfscope}%
\pgfsetbuttcap%
\pgfsetmiterjoin%
\definecolor{currentfill}{rgb}{1.000000,1.000000,1.000000}%
\pgfsetfillcolor{currentfill}%
\pgfsetfillopacity{0.800000}%
\pgfsetlinewidth{1.003750pt}%
\definecolor{currentstroke}{rgb}{0.800000,0.800000,0.800000}%
\pgfsetstrokecolor{currentstroke}%
\pgfsetstrokeopacity{0.800000}%
\pgfsetdash{}{0pt}%
\pgfpathmoveto{\pgfqpoint{0.811987in}{1.126667in}}%
\pgfpathlineto{\pgfqpoint{1.983333in}{1.126667in}}%
\pgfpathquadraticcurveto{\pgfqpoint{2.011111in}{1.126667in}}{\pgfqpoint{2.011111in}{1.154444in}}%
\pgfpathlineto{\pgfqpoint{2.011111in}{1.557222in}}%
\pgfpathquadraticcurveto{\pgfqpoint{2.011111in}{1.585000in}}{\pgfqpoint{1.983333in}{1.585000in}}%
\pgfpathlineto{\pgfqpoint{0.811987in}{1.585000in}}%
\pgfpathquadraticcurveto{\pgfqpoint{0.784210in}{1.585000in}}{\pgfqpoint{0.784210in}{1.557222in}}%
\pgfpathlineto{\pgfqpoint{0.784210in}{1.154444in}}%
\pgfpathquadraticcurveto{\pgfqpoint{0.784210in}{1.126667in}}{\pgfqpoint{0.811987in}{1.126667in}}%
\pgfpathlineto{\pgfqpoint{0.811987in}{1.126667in}}%
\pgfpathclose%
\pgfusepath{stroke,fill}%
\end{pgfscope}%
\begin{pgfscope}%
\pgfsetrectcap%
\pgfsetroundjoin%
\pgfsetlinewidth{1.505625pt}%
\definecolor{currentstroke}{rgb}{0.000000,0.000000,0.000000}%
\pgfsetstrokecolor{currentstroke}%
\pgfsetdash{}{0pt}%
\pgfpathmoveto{\pgfqpoint{0.839765in}{1.473889in}}%
\pgfpathlineto{\pgfqpoint{0.978654in}{1.473889in}}%
\pgfpathlineto{\pgfqpoint{1.117543in}{1.473889in}}%
\pgfusepath{stroke}%
\end{pgfscope}%
\begin{pgfscope}%
\definecolor{textcolor}{rgb}{0.000000,0.000000,0.000000}%
\pgfsetstrokecolor{textcolor}%
\pgfsetfillcolor{textcolor}%
\pgftext[x=1.228654in,y=1.425277in,left,base]{\color{textcolor}\rmfamily\fontsize{10.000000}{12.000000}\selectfont \(\displaystyle \Delta FP/\Delta TP\)}%
\end{pgfscope}%
\begin{pgfscope}%
\pgfsetrectcap%
\pgfsetroundjoin%
\pgfsetlinewidth{1.505625pt}%
\definecolor{currentstroke}{rgb}{0.121569,0.466667,0.705882}%
\pgfsetstrokecolor{currentstroke}%
\pgfsetdash{}{0pt}%
\pgfpathmoveto{\pgfqpoint{0.839765in}{1.265555in}}%
\pgfpathlineto{\pgfqpoint{0.978654in}{1.265555in}}%
\pgfpathlineto{\pgfqpoint{1.117543in}{1.265555in}}%
\pgfusepath{stroke}%
\end{pgfscope}%
\begin{pgfscope}%
\pgfsetbuttcap%
\pgfsetroundjoin%
\definecolor{currentfill}{rgb}{0.000000,0.000000,0.000000}%
\pgfsetfillcolor{currentfill}%
\pgfsetlinewidth{1.003750pt}%
\definecolor{currentstroke}{rgb}{0.000000,0.000000,0.000000}%
\pgfsetstrokecolor{currentstroke}%
\pgfsetdash{}{0pt}%
\pgfsys@defobject{currentmarker}{\pgfqpoint{-0.041667in}{-0.041667in}}{\pgfqpoint{0.041667in}{0.041667in}}{%
\pgfpathmoveto{\pgfqpoint{0.000000in}{-0.041667in}}%
\pgfpathcurveto{\pgfqpoint{0.011050in}{-0.041667in}}{\pgfqpoint{0.021649in}{-0.037276in}}{\pgfqpoint{0.029463in}{-0.029463in}}%
\pgfpathcurveto{\pgfqpoint{0.037276in}{-0.021649in}}{\pgfqpoint{0.041667in}{-0.011050in}}{\pgfqpoint{0.041667in}{0.000000in}}%
\pgfpathcurveto{\pgfqpoint{0.041667in}{0.011050in}}{\pgfqpoint{0.037276in}{0.021649in}}{\pgfqpoint{0.029463in}{0.029463in}}%
\pgfpathcurveto{\pgfqpoint{0.021649in}{0.037276in}}{\pgfqpoint{0.011050in}{0.041667in}}{\pgfqpoint{0.000000in}{0.041667in}}%
\pgfpathcurveto{\pgfqpoint{-0.011050in}{0.041667in}}{\pgfqpoint{-0.021649in}{0.037276in}}{\pgfqpoint{-0.029463in}{0.029463in}}%
\pgfpathcurveto{\pgfqpoint{-0.037276in}{0.021649in}}{\pgfqpoint{-0.041667in}{0.011050in}}{\pgfqpoint{-0.041667in}{0.000000in}}%
\pgfpathcurveto{\pgfqpoint{-0.041667in}{-0.011050in}}{\pgfqpoint{-0.037276in}{-0.021649in}}{\pgfqpoint{-0.029463in}{-0.029463in}}%
\pgfpathcurveto{\pgfqpoint{-0.021649in}{-0.037276in}}{\pgfqpoint{-0.011050in}{-0.041667in}}{\pgfqpoint{0.000000in}{-0.041667in}}%
\pgfpathlineto{\pgfqpoint{0.000000in}{-0.041667in}}%
\pgfpathclose%
\pgfusepath{stroke,fill}%
}%
\begin{pgfscope}%
\pgfsys@transformshift{0.978654in}{1.265555in}%
\pgfsys@useobject{currentmarker}{}%
\end{pgfscope}%
\end{pgfscope}%
\begin{pgfscope}%
\definecolor{textcolor}{rgb}{0.000000,0.000000,0.000000}%
\pgfsetstrokecolor{textcolor}%
\pgfsetfillcolor{textcolor}%
\pgftext[x=1.228654in,y=1.216944in,left,base]{\color{textcolor}\rmfamily\fontsize{10.000000}{12.000000}\selectfont (0.631,2)}%
\end{pgfscope}%
\end{pgfpicture}%
\makeatother%
\endgroup%


We want $p=0.635$ to be our threshold while still using tools that assume $p=0.5$ is the threshold, so we will linearly transform the probabilities, mapping $0.635 \to 0.5$ while keeping 0.0 at 0.0, shown below in Example 2. Note that the linear transformation of the probabilities has no affect on the shape of the ROC curve nor the area under it.  The 0.243 and 0.52 are the median transformed probabilities for the negative and positive classes, respectively.  

\noindent\begin{tabular}{@{}p{0.3\textwidth}@{\hspace{24pt}} p{0.3\textwidth} @{\hspace{24pt}} p{0.3\textwidth}}
  \vspace{0pt} %% Creator: Matplotlib, PGF backend
%%
%% To include the figure in your LaTeX document, write
%%   \input{<filename>.pgf}
%%
%% Make sure the required packages are loaded in your preamble
%%   \usepackage{pgf}
%%
%% Also ensure that all the required font packages are loaded; for instance,
%% the lmodern package is sometimes necessary when using math font.
%%   \usepackage{lmodern}
%%
%% Figures using additional raster images can only be included by \input if
%% they are in the same directory as the main LaTeX file. For loading figures
%% from other directories you can use the `import` package
%%   \usepackage{import}
%%
%% and then include the figures with
%%   \import{<path to file>}{<filename>.pgf}
%%
%% Matplotlib used the following preamble
%%   
%%   \usepackage{fontspec}
%%   \makeatletter\@ifpackageloaded{underscore}{}{\usepackage[strings]{underscore}}\makeatother
%%
\begingroup%
\makeatletter%
\begin{pgfpicture}%
\pgfpathrectangle{\pgfpointorigin}{\pgfqpoint{2.253750in}{1.953444in}}%
\pgfusepath{use as bounding box, clip}%
\begin{pgfscope}%
\pgfsetbuttcap%
\pgfsetmiterjoin%
\definecolor{currentfill}{rgb}{1.000000,1.000000,1.000000}%
\pgfsetfillcolor{currentfill}%
\pgfsetlinewidth{0.000000pt}%
\definecolor{currentstroke}{rgb}{1.000000,1.000000,1.000000}%
\pgfsetstrokecolor{currentstroke}%
\pgfsetdash{}{0pt}%
\pgfpathmoveto{\pgfqpoint{0.000000in}{0.000000in}}%
\pgfpathlineto{\pgfqpoint{2.253750in}{0.000000in}}%
\pgfpathlineto{\pgfqpoint{2.253750in}{1.953444in}}%
\pgfpathlineto{\pgfqpoint{0.000000in}{1.953444in}}%
\pgfpathlineto{\pgfqpoint{0.000000in}{0.000000in}}%
\pgfpathclose%
\pgfusepath{fill}%
\end{pgfscope}%
\begin{pgfscope}%
\pgfsetbuttcap%
\pgfsetmiterjoin%
\definecolor{currentfill}{rgb}{1.000000,1.000000,1.000000}%
\pgfsetfillcolor{currentfill}%
\pgfsetlinewidth{0.000000pt}%
\definecolor{currentstroke}{rgb}{0.000000,0.000000,0.000000}%
\pgfsetstrokecolor{currentstroke}%
\pgfsetstrokeopacity{0.000000}%
\pgfsetdash{}{0pt}%
\pgfpathmoveto{\pgfqpoint{0.515000in}{0.499444in}}%
\pgfpathlineto{\pgfqpoint{2.065000in}{0.499444in}}%
\pgfpathlineto{\pgfqpoint{2.065000in}{1.654444in}}%
\pgfpathlineto{\pgfqpoint{0.515000in}{1.654444in}}%
\pgfpathlineto{\pgfqpoint{0.515000in}{0.499444in}}%
\pgfpathclose%
\pgfusepath{fill}%
\end{pgfscope}%
\begin{pgfscope}%
\pgfpathrectangle{\pgfqpoint{0.515000in}{0.499444in}}{\pgfqpoint{1.550000in}{1.155000in}}%
\pgfusepath{clip}%
\pgfsetbuttcap%
\pgfsetmiterjoin%
\pgfsetlinewidth{1.003750pt}%
\definecolor{currentstroke}{rgb}{0.000000,0.000000,0.000000}%
\pgfsetstrokecolor{currentstroke}%
\pgfsetdash{}{0pt}%
\pgfpathmoveto{\pgfqpoint{0.505000in}{0.499444in}}%
\pgfpathlineto{\pgfqpoint{0.552805in}{0.499444in}}%
\pgfpathlineto{\pgfqpoint{0.552805in}{1.044377in}}%
\pgfpathlineto{\pgfqpoint{0.505000in}{1.044377in}}%
\pgfusepath{stroke}%
\end{pgfscope}%
\begin{pgfscope}%
\pgfpathrectangle{\pgfqpoint{0.515000in}{0.499444in}}{\pgfqpoint{1.550000in}{1.155000in}}%
\pgfusepath{clip}%
\pgfsetbuttcap%
\pgfsetmiterjoin%
\pgfsetlinewidth{1.003750pt}%
\definecolor{currentstroke}{rgb}{0.000000,0.000000,0.000000}%
\pgfsetstrokecolor{currentstroke}%
\pgfsetdash{}{0pt}%
\pgfpathmoveto{\pgfqpoint{0.643537in}{0.499444in}}%
\pgfpathlineto{\pgfqpoint{0.704025in}{0.499444in}}%
\pgfpathlineto{\pgfqpoint{0.704025in}{1.599444in}}%
\pgfpathlineto{\pgfqpoint{0.643537in}{1.599444in}}%
\pgfpathlineto{\pgfqpoint{0.643537in}{0.499444in}}%
\pgfpathclose%
\pgfusepath{stroke}%
\end{pgfscope}%
\begin{pgfscope}%
\pgfpathrectangle{\pgfqpoint{0.515000in}{0.499444in}}{\pgfqpoint{1.550000in}{1.155000in}}%
\pgfusepath{clip}%
\pgfsetbuttcap%
\pgfsetmiterjoin%
\pgfsetlinewidth{1.003750pt}%
\definecolor{currentstroke}{rgb}{0.000000,0.000000,0.000000}%
\pgfsetstrokecolor{currentstroke}%
\pgfsetdash{}{0pt}%
\pgfpathmoveto{\pgfqpoint{0.794756in}{0.499444in}}%
\pgfpathlineto{\pgfqpoint{0.855244in}{0.499444in}}%
\pgfpathlineto{\pgfqpoint{0.855244in}{1.551408in}}%
\pgfpathlineto{\pgfqpoint{0.794756in}{1.551408in}}%
\pgfpathlineto{\pgfqpoint{0.794756in}{0.499444in}}%
\pgfpathclose%
\pgfusepath{stroke}%
\end{pgfscope}%
\begin{pgfscope}%
\pgfpathrectangle{\pgfqpoint{0.515000in}{0.499444in}}{\pgfqpoint{1.550000in}{1.155000in}}%
\pgfusepath{clip}%
\pgfsetbuttcap%
\pgfsetmiterjoin%
\pgfsetlinewidth{1.003750pt}%
\definecolor{currentstroke}{rgb}{0.000000,0.000000,0.000000}%
\pgfsetstrokecolor{currentstroke}%
\pgfsetdash{}{0pt}%
\pgfpathmoveto{\pgfqpoint{0.945976in}{0.499444in}}%
\pgfpathlineto{\pgfqpoint{1.006464in}{0.499444in}}%
\pgfpathlineto{\pgfqpoint{1.006464in}{1.215792in}}%
\pgfpathlineto{\pgfqpoint{0.945976in}{1.215792in}}%
\pgfpathlineto{\pgfqpoint{0.945976in}{0.499444in}}%
\pgfpathclose%
\pgfusepath{stroke}%
\end{pgfscope}%
\begin{pgfscope}%
\pgfpathrectangle{\pgfqpoint{0.515000in}{0.499444in}}{\pgfqpoint{1.550000in}{1.155000in}}%
\pgfusepath{clip}%
\pgfsetbuttcap%
\pgfsetmiterjoin%
\pgfsetlinewidth{1.003750pt}%
\definecolor{currentstroke}{rgb}{0.000000,0.000000,0.000000}%
\pgfsetstrokecolor{currentstroke}%
\pgfsetdash{}{0pt}%
\pgfpathmoveto{\pgfqpoint{1.097195in}{0.499444in}}%
\pgfpathlineto{\pgfqpoint{1.157683in}{0.499444in}}%
\pgfpathlineto{\pgfqpoint{1.157683in}{0.939437in}}%
\pgfpathlineto{\pgfqpoint{1.097195in}{0.939437in}}%
\pgfpathlineto{\pgfqpoint{1.097195in}{0.499444in}}%
\pgfpathclose%
\pgfusepath{stroke}%
\end{pgfscope}%
\begin{pgfscope}%
\pgfpathrectangle{\pgfqpoint{0.515000in}{0.499444in}}{\pgfqpoint{1.550000in}{1.155000in}}%
\pgfusepath{clip}%
\pgfsetbuttcap%
\pgfsetmiterjoin%
\pgfsetlinewidth{1.003750pt}%
\definecolor{currentstroke}{rgb}{0.000000,0.000000,0.000000}%
\pgfsetstrokecolor{currentstroke}%
\pgfsetdash{}{0pt}%
\pgfpathmoveto{\pgfqpoint{1.248415in}{0.499444in}}%
\pgfpathlineto{\pgfqpoint{1.308903in}{0.499444in}}%
\pgfpathlineto{\pgfqpoint{1.308903in}{0.704537in}}%
\pgfpathlineto{\pgfqpoint{1.248415in}{0.704537in}}%
\pgfpathlineto{\pgfqpoint{1.248415in}{0.499444in}}%
\pgfpathclose%
\pgfusepath{stroke}%
\end{pgfscope}%
\begin{pgfscope}%
\pgfpathrectangle{\pgfqpoint{0.515000in}{0.499444in}}{\pgfqpoint{1.550000in}{1.155000in}}%
\pgfusepath{clip}%
\pgfsetbuttcap%
\pgfsetmiterjoin%
\pgfsetlinewidth{1.003750pt}%
\definecolor{currentstroke}{rgb}{0.000000,0.000000,0.000000}%
\pgfsetstrokecolor{currentstroke}%
\pgfsetdash{}{0pt}%
\pgfpathmoveto{\pgfqpoint{1.399634in}{0.499444in}}%
\pgfpathlineto{\pgfqpoint{1.460122in}{0.499444in}}%
\pgfpathlineto{\pgfqpoint{1.460122in}{0.597240in}}%
\pgfpathlineto{\pgfqpoint{1.399634in}{0.597240in}}%
\pgfpathlineto{\pgfqpoint{1.399634in}{0.499444in}}%
\pgfpathclose%
\pgfusepath{stroke}%
\end{pgfscope}%
\begin{pgfscope}%
\pgfpathrectangle{\pgfqpoint{0.515000in}{0.499444in}}{\pgfqpoint{1.550000in}{1.155000in}}%
\pgfusepath{clip}%
\pgfsetbuttcap%
\pgfsetmiterjoin%
\pgfsetlinewidth{1.003750pt}%
\definecolor{currentstroke}{rgb}{0.000000,0.000000,0.000000}%
\pgfsetstrokecolor{currentstroke}%
\pgfsetdash{}{0pt}%
\pgfpathmoveto{\pgfqpoint{1.550854in}{0.499444in}}%
\pgfpathlineto{\pgfqpoint{1.611342in}{0.499444in}}%
\pgfpathlineto{\pgfqpoint{1.611342in}{0.543890in}}%
\pgfpathlineto{\pgfqpoint{1.550854in}{0.543890in}}%
\pgfpathlineto{\pgfqpoint{1.550854in}{0.499444in}}%
\pgfpathclose%
\pgfusepath{stroke}%
\end{pgfscope}%
\begin{pgfscope}%
\pgfpathrectangle{\pgfqpoint{0.515000in}{0.499444in}}{\pgfqpoint{1.550000in}{1.155000in}}%
\pgfusepath{clip}%
\pgfsetbuttcap%
\pgfsetmiterjoin%
\pgfsetlinewidth{1.003750pt}%
\definecolor{currentstroke}{rgb}{0.000000,0.000000,0.000000}%
\pgfsetstrokecolor{currentstroke}%
\pgfsetdash{}{0pt}%
\pgfpathmoveto{\pgfqpoint{1.702073in}{0.499444in}}%
\pgfpathlineto{\pgfqpoint{1.762561in}{0.499444in}}%
\pgfpathlineto{\pgfqpoint{1.762561in}{0.500535in}}%
\pgfpathlineto{\pgfqpoint{1.702073in}{0.500535in}}%
\pgfpathlineto{\pgfqpoint{1.702073in}{0.499444in}}%
\pgfpathclose%
\pgfusepath{stroke}%
\end{pgfscope}%
\begin{pgfscope}%
\pgfpathrectangle{\pgfqpoint{0.515000in}{0.499444in}}{\pgfqpoint{1.550000in}{1.155000in}}%
\pgfusepath{clip}%
\pgfsetbuttcap%
\pgfsetmiterjoin%
\pgfsetlinewidth{1.003750pt}%
\definecolor{currentstroke}{rgb}{0.000000,0.000000,0.000000}%
\pgfsetstrokecolor{currentstroke}%
\pgfsetdash{}{0pt}%
\pgfpathmoveto{\pgfqpoint{1.853293in}{0.499444in}}%
\pgfpathlineto{\pgfqpoint{1.913781in}{0.499444in}}%
\pgfpathlineto{\pgfqpoint{1.913781in}{0.499444in}}%
\pgfpathlineto{\pgfqpoint{1.853293in}{0.499444in}}%
\pgfpathlineto{\pgfqpoint{1.853293in}{0.499444in}}%
\pgfpathclose%
\pgfusepath{stroke}%
\end{pgfscope}%
\begin{pgfscope}%
\pgfpathrectangle{\pgfqpoint{0.515000in}{0.499444in}}{\pgfqpoint{1.550000in}{1.155000in}}%
\pgfusepath{clip}%
\pgfsetbuttcap%
\pgfsetmiterjoin%
\definecolor{currentfill}{rgb}{0.000000,0.000000,0.000000}%
\pgfsetfillcolor{currentfill}%
\pgfsetlinewidth{0.000000pt}%
\definecolor{currentstroke}{rgb}{0.000000,0.000000,0.000000}%
\pgfsetstrokecolor{currentstroke}%
\pgfsetstrokeopacity{0.000000}%
\pgfsetdash{}{0pt}%
\pgfpathmoveto{\pgfqpoint{0.552805in}{0.499444in}}%
\pgfpathlineto{\pgfqpoint{0.613293in}{0.499444in}}%
\pgfpathlineto{\pgfqpoint{0.613293in}{0.511726in}}%
\pgfpathlineto{\pgfqpoint{0.552805in}{0.511726in}}%
\pgfpathlineto{\pgfqpoint{0.552805in}{0.499444in}}%
\pgfpathclose%
\pgfusepath{fill}%
\end{pgfscope}%
\begin{pgfscope}%
\pgfpathrectangle{\pgfqpoint{0.515000in}{0.499444in}}{\pgfqpoint{1.550000in}{1.155000in}}%
\pgfusepath{clip}%
\pgfsetbuttcap%
\pgfsetmiterjoin%
\definecolor{currentfill}{rgb}{0.000000,0.000000,0.000000}%
\pgfsetfillcolor{currentfill}%
\pgfsetlinewidth{0.000000pt}%
\definecolor{currentstroke}{rgb}{0.000000,0.000000,0.000000}%
\pgfsetstrokecolor{currentstroke}%
\pgfsetstrokeopacity{0.000000}%
\pgfsetdash{}{0pt}%
\pgfpathmoveto{\pgfqpoint{0.704025in}{0.499444in}}%
\pgfpathlineto{\pgfqpoint{0.764512in}{0.499444in}}%
\pgfpathlineto{\pgfqpoint{0.764512in}{0.524852in}}%
\pgfpathlineto{\pgfqpoint{0.704025in}{0.524852in}}%
\pgfpathlineto{\pgfqpoint{0.704025in}{0.499444in}}%
\pgfpathclose%
\pgfusepath{fill}%
\end{pgfscope}%
\begin{pgfscope}%
\pgfpathrectangle{\pgfqpoint{0.515000in}{0.499444in}}{\pgfqpoint{1.550000in}{1.155000in}}%
\pgfusepath{clip}%
\pgfsetbuttcap%
\pgfsetmiterjoin%
\definecolor{currentfill}{rgb}{0.000000,0.000000,0.000000}%
\pgfsetfillcolor{currentfill}%
\pgfsetlinewidth{0.000000pt}%
\definecolor{currentstroke}{rgb}{0.000000,0.000000,0.000000}%
\pgfsetstrokecolor{currentstroke}%
\pgfsetstrokeopacity{0.000000}%
\pgfsetdash{}{0pt}%
\pgfpathmoveto{\pgfqpoint{0.855244in}{0.499444in}}%
\pgfpathlineto{\pgfqpoint{0.915732in}{0.499444in}}%
\pgfpathlineto{\pgfqpoint{0.915732in}{0.549627in}}%
\pgfpathlineto{\pgfqpoint{0.855244in}{0.549627in}}%
\pgfpathlineto{\pgfqpoint{0.855244in}{0.499444in}}%
\pgfpathclose%
\pgfusepath{fill}%
\end{pgfscope}%
\begin{pgfscope}%
\pgfpathrectangle{\pgfqpoint{0.515000in}{0.499444in}}{\pgfqpoint{1.550000in}{1.155000in}}%
\pgfusepath{clip}%
\pgfsetbuttcap%
\pgfsetmiterjoin%
\definecolor{currentfill}{rgb}{0.000000,0.000000,0.000000}%
\pgfsetfillcolor{currentfill}%
\pgfsetlinewidth{0.000000pt}%
\definecolor{currentstroke}{rgb}{0.000000,0.000000,0.000000}%
\pgfsetstrokecolor{currentstroke}%
\pgfsetstrokeopacity{0.000000}%
\pgfsetdash{}{0pt}%
\pgfpathmoveto{\pgfqpoint{1.006464in}{0.499444in}}%
\pgfpathlineto{\pgfqpoint{1.066951in}{0.499444in}}%
\pgfpathlineto{\pgfqpoint{1.066951in}{0.592806in}}%
\pgfpathlineto{\pgfqpoint{1.006464in}{0.592806in}}%
\pgfpathlineto{\pgfqpoint{1.006464in}{0.499444in}}%
\pgfpathclose%
\pgfusepath{fill}%
\end{pgfscope}%
\begin{pgfscope}%
\pgfpathrectangle{\pgfqpoint{0.515000in}{0.499444in}}{\pgfqpoint{1.550000in}{1.155000in}}%
\pgfusepath{clip}%
\pgfsetbuttcap%
\pgfsetmiterjoin%
\definecolor{currentfill}{rgb}{0.000000,0.000000,0.000000}%
\pgfsetfillcolor{currentfill}%
\pgfsetlinewidth{0.000000pt}%
\definecolor{currentstroke}{rgb}{0.000000,0.000000,0.000000}%
\pgfsetstrokecolor{currentstroke}%
\pgfsetstrokeopacity{0.000000}%
\pgfsetdash{}{0pt}%
\pgfpathmoveto{\pgfqpoint{1.157683in}{0.499444in}}%
\pgfpathlineto{\pgfqpoint{1.218171in}{0.499444in}}%
\pgfpathlineto{\pgfqpoint{1.218171in}{0.651153in}}%
\pgfpathlineto{\pgfqpoint{1.157683in}{0.651153in}}%
\pgfpathlineto{\pgfqpoint{1.157683in}{0.499444in}}%
\pgfpathclose%
\pgfusepath{fill}%
\end{pgfscope}%
\begin{pgfscope}%
\pgfpathrectangle{\pgfqpoint{0.515000in}{0.499444in}}{\pgfqpoint{1.550000in}{1.155000in}}%
\pgfusepath{clip}%
\pgfsetbuttcap%
\pgfsetmiterjoin%
\definecolor{currentfill}{rgb}{0.000000,0.000000,0.000000}%
\pgfsetfillcolor{currentfill}%
\pgfsetlinewidth{0.000000pt}%
\definecolor{currentstroke}{rgb}{0.000000,0.000000,0.000000}%
\pgfsetstrokecolor{currentstroke}%
\pgfsetstrokeopacity{0.000000}%
\pgfsetdash{}{0pt}%
\pgfpathmoveto{\pgfqpoint{1.308903in}{0.499444in}}%
\pgfpathlineto{\pgfqpoint{1.369391in}{0.499444in}}%
\pgfpathlineto{\pgfqpoint{1.369391in}{0.681909in}}%
\pgfpathlineto{\pgfqpoint{1.308903in}{0.681909in}}%
\pgfpathlineto{\pgfqpoint{1.308903in}{0.499444in}}%
\pgfpathclose%
\pgfusepath{fill}%
\end{pgfscope}%
\begin{pgfscope}%
\pgfpathrectangle{\pgfqpoint{0.515000in}{0.499444in}}{\pgfqpoint{1.550000in}{1.155000in}}%
\pgfusepath{clip}%
\pgfsetbuttcap%
\pgfsetmiterjoin%
\definecolor{currentfill}{rgb}{0.000000,0.000000,0.000000}%
\pgfsetfillcolor{currentfill}%
\pgfsetlinewidth{0.000000pt}%
\definecolor{currentstroke}{rgb}{0.000000,0.000000,0.000000}%
\pgfsetstrokecolor{currentstroke}%
\pgfsetstrokeopacity{0.000000}%
\pgfsetdash{}{0pt}%
\pgfpathmoveto{\pgfqpoint{1.460122in}{0.499444in}}%
\pgfpathlineto{\pgfqpoint{1.520610in}{0.499444in}}%
\pgfpathlineto{\pgfqpoint{1.520610in}{0.646085in}}%
\pgfpathlineto{\pgfqpoint{1.460122in}{0.646085in}}%
\pgfpathlineto{\pgfqpoint{1.460122in}{0.499444in}}%
\pgfpathclose%
\pgfusepath{fill}%
\end{pgfscope}%
\begin{pgfscope}%
\pgfpathrectangle{\pgfqpoint{0.515000in}{0.499444in}}{\pgfqpoint{1.550000in}{1.155000in}}%
\pgfusepath{clip}%
\pgfsetbuttcap%
\pgfsetmiterjoin%
\definecolor{currentfill}{rgb}{0.000000,0.000000,0.000000}%
\pgfsetfillcolor{currentfill}%
\pgfsetlinewidth{0.000000pt}%
\definecolor{currentstroke}{rgb}{0.000000,0.000000,0.000000}%
\pgfsetstrokecolor{currentstroke}%
\pgfsetstrokeopacity{0.000000}%
\pgfsetdash{}{0pt}%
\pgfpathmoveto{\pgfqpoint{1.611342in}{0.499444in}}%
\pgfpathlineto{\pgfqpoint{1.671830in}{0.499444in}}%
\pgfpathlineto{\pgfqpoint{1.671830in}{0.537838in}}%
\pgfpathlineto{\pgfqpoint{1.611342in}{0.537838in}}%
\pgfpathlineto{\pgfqpoint{1.611342in}{0.499444in}}%
\pgfpathclose%
\pgfusepath{fill}%
\end{pgfscope}%
\begin{pgfscope}%
\pgfpathrectangle{\pgfqpoint{0.515000in}{0.499444in}}{\pgfqpoint{1.550000in}{1.155000in}}%
\pgfusepath{clip}%
\pgfsetbuttcap%
\pgfsetmiterjoin%
\definecolor{currentfill}{rgb}{0.000000,0.000000,0.000000}%
\pgfsetfillcolor{currentfill}%
\pgfsetlinewidth{0.000000pt}%
\definecolor{currentstroke}{rgb}{0.000000,0.000000,0.000000}%
\pgfsetstrokecolor{currentstroke}%
\pgfsetstrokeopacity{0.000000}%
\pgfsetdash{}{0pt}%
\pgfpathmoveto{\pgfqpoint{1.762561in}{0.499444in}}%
\pgfpathlineto{\pgfqpoint{1.823049in}{0.499444in}}%
\pgfpathlineto{\pgfqpoint{1.823049in}{0.499444in}}%
\pgfpathlineto{\pgfqpoint{1.762561in}{0.499444in}}%
\pgfpathlineto{\pgfqpoint{1.762561in}{0.499444in}}%
\pgfpathclose%
\pgfusepath{fill}%
\end{pgfscope}%
\begin{pgfscope}%
\pgfpathrectangle{\pgfqpoint{0.515000in}{0.499444in}}{\pgfqpoint{1.550000in}{1.155000in}}%
\pgfusepath{clip}%
\pgfsetbuttcap%
\pgfsetmiterjoin%
\definecolor{currentfill}{rgb}{0.000000,0.000000,0.000000}%
\pgfsetfillcolor{currentfill}%
\pgfsetlinewidth{0.000000pt}%
\definecolor{currentstroke}{rgb}{0.000000,0.000000,0.000000}%
\pgfsetstrokecolor{currentstroke}%
\pgfsetstrokeopacity{0.000000}%
\pgfsetdash{}{0pt}%
\pgfpathmoveto{\pgfqpoint{1.913781in}{0.499444in}}%
\pgfpathlineto{\pgfqpoint{1.974269in}{0.499444in}}%
\pgfpathlineto{\pgfqpoint{1.974269in}{0.499444in}}%
\pgfpathlineto{\pgfqpoint{1.913781in}{0.499444in}}%
\pgfpathlineto{\pgfqpoint{1.913781in}{0.499444in}}%
\pgfpathclose%
\pgfusepath{fill}%
\end{pgfscope}%
\begin{pgfscope}%
\pgfsetbuttcap%
\pgfsetroundjoin%
\definecolor{currentfill}{rgb}{0.000000,0.000000,0.000000}%
\pgfsetfillcolor{currentfill}%
\pgfsetlinewidth{0.803000pt}%
\definecolor{currentstroke}{rgb}{0.000000,0.000000,0.000000}%
\pgfsetstrokecolor{currentstroke}%
\pgfsetdash{}{0pt}%
\pgfsys@defobject{currentmarker}{\pgfqpoint{0.000000in}{-0.048611in}}{\pgfqpoint{0.000000in}{0.000000in}}{%
\pgfpathmoveto{\pgfqpoint{0.000000in}{0.000000in}}%
\pgfpathlineto{\pgfqpoint{0.000000in}{-0.048611in}}%
\pgfusepath{stroke,fill}%
}%
\begin{pgfscope}%
\pgfsys@transformshift{0.552805in}{0.499444in}%
\pgfsys@useobject{currentmarker}{}%
\end{pgfscope}%
\end{pgfscope}%
\begin{pgfscope}%
\definecolor{textcolor}{rgb}{0.000000,0.000000,0.000000}%
\pgfsetstrokecolor{textcolor}%
\pgfsetfillcolor{textcolor}%
\pgftext[x=0.552805in,y=0.402222in,,top]{\color{textcolor}\rmfamily\fontsize{10.000000}{12.000000}\selectfont 0.0}%
\end{pgfscope}%
\begin{pgfscope}%
\pgfsetbuttcap%
\pgfsetroundjoin%
\definecolor{currentfill}{rgb}{0.000000,0.000000,0.000000}%
\pgfsetfillcolor{currentfill}%
\pgfsetlinewidth{0.803000pt}%
\definecolor{currentstroke}{rgb}{0.000000,0.000000,0.000000}%
\pgfsetstrokecolor{currentstroke}%
\pgfsetdash{}{0pt}%
\pgfsys@defobject{currentmarker}{\pgfqpoint{0.000000in}{-0.048611in}}{\pgfqpoint{0.000000in}{0.000000in}}{%
\pgfpathmoveto{\pgfqpoint{0.000000in}{0.000000in}}%
\pgfpathlineto{\pgfqpoint{0.000000in}{-0.048611in}}%
\pgfusepath{stroke,fill}%
}%
\begin{pgfscope}%
\pgfsys@transformshift{0.930854in}{0.499444in}%
\pgfsys@useobject{currentmarker}{}%
\end{pgfscope}%
\end{pgfscope}%
\begin{pgfscope}%
\definecolor{textcolor}{rgb}{0.000000,0.000000,0.000000}%
\pgfsetstrokecolor{textcolor}%
\pgfsetfillcolor{textcolor}%
\pgftext[x=0.930854in,y=0.402222in,,top]{\color{textcolor}\rmfamily\fontsize{10.000000}{12.000000}\selectfont 0.25}%
\end{pgfscope}%
\begin{pgfscope}%
\pgfsetbuttcap%
\pgfsetroundjoin%
\definecolor{currentfill}{rgb}{0.000000,0.000000,0.000000}%
\pgfsetfillcolor{currentfill}%
\pgfsetlinewidth{0.803000pt}%
\definecolor{currentstroke}{rgb}{0.000000,0.000000,0.000000}%
\pgfsetstrokecolor{currentstroke}%
\pgfsetdash{}{0pt}%
\pgfsys@defobject{currentmarker}{\pgfqpoint{0.000000in}{-0.048611in}}{\pgfqpoint{0.000000in}{0.000000in}}{%
\pgfpathmoveto{\pgfqpoint{0.000000in}{0.000000in}}%
\pgfpathlineto{\pgfqpoint{0.000000in}{-0.048611in}}%
\pgfusepath{stroke,fill}%
}%
\begin{pgfscope}%
\pgfsys@transformshift{1.308903in}{0.499444in}%
\pgfsys@useobject{currentmarker}{}%
\end{pgfscope}%
\end{pgfscope}%
\begin{pgfscope}%
\definecolor{textcolor}{rgb}{0.000000,0.000000,0.000000}%
\pgfsetstrokecolor{textcolor}%
\pgfsetfillcolor{textcolor}%
\pgftext[x=1.308903in,y=0.402222in,,top]{\color{textcolor}\rmfamily\fontsize{10.000000}{12.000000}\selectfont 0.5}%
\end{pgfscope}%
\begin{pgfscope}%
\pgfsetbuttcap%
\pgfsetroundjoin%
\definecolor{currentfill}{rgb}{0.000000,0.000000,0.000000}%
\pgfsetfillcolor{currentfill}%
\pgfsetlinewidth{0.803000pt}%
\definecolor{currentstroke}{rgb}{0.000000,0.000000,0.000000}%
\pgfsetstrokecolor{currentstroke}%
\pgfsetdash{}{0pt}%
\pgfsys@defobject{currentmarker}{\pgfqpoint{0.000000in}{-0.048611in}}{\pgfqpoint{0.000000in}{0.000000in}}{%
\pgfpathmoveto{\pgfqpoint{0.000000in}{0.000000in}}%
\pgfpathlineto{\pgfqpoint{0.000000in}{-0.048611in}}%
\pgfusepath{stroke,fill}%
}%
\begin{pgfscope}%
\pgfsys@transformshift{1.686951in}{0.499444in}%
\pgfsys@useobject{currentmarker}{}%
\end{pgfscope}%
\end{pgfscope}%
\begin{pgfscope}%
\definecolor{textcolor}{rgb}{0.000000,0.000000,0.000000}%
\pgfsetstrokecolor{textcolor}%
\pgfsetfillcolor{textcolor}%
\pgftext[x=1.686951in,y=0.402222in,,top]{\color{textcolor}\rmfamily\fontsize{10.000000}{12.000000}\selectfont 0.75}%
\end{pgfscope}%
\begin{pgfscope}%
\pgfsetbuttcap%
\pgfsetroundjoin%
\definecolor{currentfill}{rgb}{0.000000,0.000000,0.000000}%
\pgfsetfillcolor{currentfill}%
\pgfsetlinewidth{0.803000pt}%
\definecolor{currentstroke}{rgb}{0.000000,0.000000,0.000000}%
\pgfsetstrokecolor{currentstroke}%
\pgfsetdash{}{0pt}%
\pgfsys@defobject{currentmarker}{\pgfqpoint{0.000000in}{-0.048611in}}{\pgfqpoint{0.000000in}{0.000000in}}{%
\pgfpathmoveto{\pgfqpoint{0.000000in}{0.000000in}}%
\pgfpathlineto{\pgfqpoint{0.000000in}{-0.048611in}}%
\pgfusepath{stroke,fill}%
}%
\begin{pgfscope}%
\pgfsys@transformshift{2.065000in}{0.499444in}%
\pgfsys@useobject{currentmarker}{}%
\end{pgfscope}%
\end{pgfscope}%
\begin{pgfscope}%
\definecolor{textcolor}{rgb}{0.000000,0.000000,0.000000}%
\pgfsetstrokecolor{textcolor}%
\pgfsetfillcolor{textcolor}%
\pgftext[x=2.065000in,y=0.402222in,,top]{\color{textcolor}\rmfamily\fontsize{10.000000}{12.000000}\selectfont 1.0}%
\end{pgfscope}%
\begin{pgfscope}%
\definecolor{textcolor}{rgb}{0.000000,0.000000,0.000000}%
\pgfsetstrokecolor{textcolor}%
\pgfsetfillcolor{textcolor}%
\pgftext[x=1.290000in,y=0.223333in,,top]{\color{textcolor}\rmfamily\fontsize{10.000000}{12.000000}\selectfont \(\displaystyle p\)}%
\end{pgfscope}%
\begin{pgfscope}%
\pgfsetbuttcap%
\pgfsetroundjoin%
\definecolor{currentfill}{rgb}{0.000000,0.000000,0.000000}%
\pgfsetfillcolor{currentfill}%
\pgfsetlinewidth{0.803000pt}%
\definecolor{currentstroke}{rgb}{0.000000,0.000000,0.000000}%
\pgfsetstrokecolor{currentstroke}%
\pgfsetdash{}{0pt}%
\pgfsys@defobject{currentmarker}{\pgfqpoint{-0.048611in}{0.000000in}}{\pgfqpoint{-0.000000in}{0.000000in}}{%
\pgfpathmoveto{\pgfqpoint{-0.000000in}{0.000000in}}%
\pgfpathlineto{\pgfqpoint{-0.048611in}{0.000000in}}%
\pgfusepath{stroke,fill}%
}%
\begin{pgfscope}%
\pgfsys@transformshift{0.515000in}{0.499444in}%
\pgfsys@useobject{currentmarker}{}%
\end{pgfscope}%
\end{pgfscope}%
\begin{pgfscope}%
\definecolor{textcolor}{rgb}{0.000000,0.000000,0.000000}%
\pgfsetstrokecolor{textcolor}%
\pgfsetfillcolor{textcolor}%
\pgftext[x=0.348333in, y=0.451250in, left, base]{\color{textcolor}\rmfamily\fontsize{10.000000}{12.000000}\selectfont \(\displaystyle {0}\)}%
\end{pgfscope}%
\begin{pgfscope}%
\pgfsetbuttcap%
\pgfsetroundjoin%
\definecolor{currentfill}{rgb}{0.000000,0.000000,0.000000}%
\pgfsetfillcolor{currentfill}%
\pgfsetlinewidth{0.803000pt}%
\definecolor{currentstroke}{rgb}{0.000000,0.000000,0.000000}%
\pgfsetstrokecolor{currentstroke}%
\pgfsetdash{}{0pt}%
\pgfsys@defobject{currentmarker}{\pgfqpoint{-0.048611in}{0.000000in}}{\pgfqpoint{-0.000000in}{0.000000in}}{%
\pgfpathmoveto{\pgfqpoint{-0.000000in}{0.000000in}}%
\pgfpathlineto{\pgfqpoint{-0.048611in}{0.000000in}}%
\pgfusepath{stroke,fill}%
}%
\begin{pgfscope}%
\pgfsys@transformshift{0.515000in}{0.992118in}%
\pgfsys@useobject{currentmarker}{}%
\end{pgfscope}%
\end{pgfscope}%
\begin{pgfscope}%
\definecolor{textcolor}{rgb}{0.000000,0.000000,0.000000}%
\pgfsetstrokecolor{textcolor}%
\pgfsetfillcolor{textcolor}%
\pgftext[x=0.278889in, y=0.943924in, left, base]{\color{textcolor}\rmfamily\fontsize{10.000000}{12.000000}\selectfont \(\displaystyle {10}\)}%
\end{pgfscope}%
\begin{pgfscope}%
\pgfsetbuttcap%
\pgfsetroundjoin%
\definecolor{currentfill}{rgb}{0.000000,0.000000,0.000000}%
\pgfsetfillcolor{currentfill}%
\pgfsetlinewidth{0.803000pt}%
\definecolor{currentstroke}{rgb}{0.000000,0.000000,0.000000}%
\pgfsetstrokecolor{currentstroke}%
\pgfsetdash{}{0pt}%
\pgfsys@defobject{currentmarker}{\pgfqpoint{-0.048611in}{0.000000in}}{\pgfqpoint{-0.000000in}{0.000000in}}{%
\pgfpathmoveto{\pgfqpoint{-0.000000in}{0.000000in}}%
\pgfpathlineto{\pgfqpoint{-0.048611in}{0.000000in}}%
\pgfusepath{stroke,fill}%
}%
\begin{pgfscope}%
\pgfsys@transformshift{0.515000in}{1.484792in}%
\pgfsys@useobject{currentmarker}{}%
\end{pgfscope}%
\end{pgfscope}%
\begin{pgfscope}%
\definecolor{textcolor}{rgb}{0.000000,0.000000,0.000000}%
\pgfsetstrokecolor{textcolor}%
\pgfsetfillcolor{textcolor}%
\pgftext[x=0.278889in, y=1.436597in, left, base]{\color{textcolor}\rmfamily\fontsize{10.000000}{12.000000}\selectfont \(\displaystyle {20}\)}%
\end{pgfscope}%
\begin{pgfscope}%
\definecolor{textcolor}{rgb}{0.000000,0.000000,0.000000}%
\pgfsetstrokecolor{textcolor}%
\pgfsetfillcolor{textcolor}%
\pgftext[x=0.223333in,y=1.076944in,,bottom,rotate=90.000000]{\color{textcolor}\rmfamily\fontsize{10.000000}{12.000000}\selectfont Percent of Data Set}%
\end{pgfscope}%
\begin{pgfscope}%
\pgfsetrectcap%
\pgfsetmiterjoin%
\pgfsetlinewidth{0.803000pt}%
\definecolor{currentstroke}{rgb}{0.000000,0.000000,0.000000}%
\pgfsetstrokecolor{currentstroke}%
\pgfsetdash{}{0pt}%
\pgfpathmoveto{\pgfqpoint{0.515000in}{0.499444in}}%
\pgfpathlineto{\pgfqpoint{0.515000in}{1.654444in}}%
\pgfusepath{stroke}%
\end{pgfscope}%
\begin{pgfscope}%
\pgfsetrectcap%
\pgfsetmiterjoin%
\pgfsetlinewidth{0.803000pt}%
\definecolor{currentstroke}{rgb}{0.000000,0.000000,0.000000}%
\pgfsetstrokecolor{currentstroke}%
\pgfsetdash{}{0pt}%
\pgfpathmoveto{\pgfqpoint{2.065000in}{0.499444in}}%
\pgfpathlineto{\pgfqpoint{2.065000in}{1.654444in}}%
\pgfusepath{stroke}%
\end{pgfscope}%
\begin{pgfscope}%
\pgfsetrectcap%
\pgfsetmiterjoin%
\pgfsetlinewidth{0.803000pt}%
\definecolor{currentstroke}{rgb}{0.000000,0.000000,0.000000}%
\pgfsetstrokecolor{currentstroke}%
\pgfsetdash{}{0pt}%
\pgfpathmoveto{\pgfqpoint{0.515000in}{0.499444in}}%
\pgfpathlineto{\pgfqpoint{2.065000in}{0.499444in}}%
\pgfusepath{stroke}%
\end{pgfscope}%
\begin{pgfscope}%
\pgfsetrectcap%
\pgfsetmiterjoin%
\pgfsetlinewidth{0.803000pt}%
\definecolor{currentstroke}{rgb}{0.000000,0.000000,0.000000}%
\pgfsetstrokecolor{currentstroke}%
\pgfsetdash{}{0pt}%
\pgfpathmoveto{\pgfqpoint{0.515000in}{1.654444in}}%
\pgfpathlineto{\pgfqpoint{2.065000in}{1.654444in}}%
\pgfusepath{stroke}%
\end{pgfscope}%
\begin{pgfscope}%
\definecolor{textcolor}{rgb}{0.000000,0.000000,0.000000}%
\pgfsetstrokecolor{textcolor}%
\pgfsetfillcolor{textcolor}%
\pgftext[x=1.290000in,y=1.737778in,,base]{\color{textcolor}\rmfamily\fontsize{12.000000}{14.400000}\selectfont Probability Distribution}%
\end{pgfscope}%
\begin{pgfscope}%
\pgfsetbuttcap%
\pgfsetmiterjoin%
\definecolor{currentfill}{rgb}{1.000000,1.000000,1.000000}%
\pgfsetfillcolor{currentfill}%
\pgfsetfillopacity{0.800000}%
\pgfsetlinewidth{1.003750pt}%
\definecolor{currentstroke}{rgb}{0.800000,0.800000,0.800000}%
\pgfsetstrokecolor{currentstroke}%
\pgfsetstrokeopacity{0.800000}%
\pgfsetdash{}{0pt}%
\pgfpathmoveto{\pgfqpoint{1.288056in}{1.154445in}}%
\pgfpathlineto{\pgfqpoint{1.967778in}{1.154445in}}%
\pgfpathquadraticcurveto{\pgfqpoint{1.995556in}{1.154445in}}{\pgfqpoint{1.995556in}{1.182222in}}%
\pgfpathlineto{\pgfqpoint{1.995556in}{1.557222in}}%
\pgfpathquadraticcurveto{\pgfqpoint{1.995556in}{1.585000in}}{\pgfqpoint{1.967778in}{1.585000in}}%
\pgfpathlineto{\pgfqpoint{1.288056in}{1.585000in}}%
\pgfpathquadraticcurveto{\pgfqpoint{1.260278in}{1.585000in}}{\pgfqpoint{1.260278in}{1.557222in}}%
\pgfpathlineto{\pgfqpoint{1.260278in}{1.182222in}}%
\pgfpathquadraticcurveto{\pgfqpoint{1.260278in}{1.154445in}}{\pgfqpoint{1.288056in}{1.154445in}}%
\pgfpathlineto{\pgfqpoint{1.288056in}{1.154445in}}%
\pgfpathclose%
\pgfusepath{stroke,fill}%
\end{pgfscope}%
\begin{pgfscope}%
\pgfsetbuttcap%
\pgfsetmiterjoin%
\pgfsetlinewidth{1.003750pt}%
\definecolor{currentstroke}{rgb}{0.000000,0.000000,0.000000}%
\pgfsetstrokecolor{currentstroke}%
\pgfsetdash{}{0pt}%
\pgfpathmoveto{\pgfqpoint{1.315834in}{1.432222in}}%
\pgfpathlineto{\pgfqpoint{1.593611in}{1.432222in}}%
\pgfpathlineto{\pgfqpoint{1.593611in}{1.529444in}}%
\pgfpathlineto{\pgfqpoint{1.315834in}{1.529444in}}%
\pgfpathlineto{\pgfqpoint{1.315834in}{1.432222in}}%
\pgfpathclose%
\pgfusepath{stroke}%
\end{pgfscope}%
\begin{pgfscope}%
\definecolor{textcolor}{rgb}{0.000000,0.000000,0.000000}%
\pgfsetstrokecolor{textcolor}%
\pgfsetfillcolor{textcolor}%
\pgftext[x=1.704722in,y=1.432222in,left,base]{\color{textcolor}\rmfamily\fontsize{10.000000}{12.000000}\selectfont Neg}%
\end{pgfscope}%
\begin{pgfscope}%
\pgfsetbuttcap%
\pgfsetmiterjoin%
\definecolor{currentfill}{rgb}{0.000000,0.000000,0.000000}%
\pgfsetfillcolor{currentfill}%
\pgfsetlinewidth{0.000000pt}%
\definecolor{currentstroke}{rgb}{0.000000,0.000000,0.000000}%
\pgfsetstrokecolor{currentstroke}%
\pgfsetstrokeopacity{0.000000}%
\pgfsetdash{}{0pt}%
\pgfpathmoveto{\pgfqpoint{1.315834in}{1.236944in}}%
\pgfpathlineto{\pgfqpoint{1.593611in}{1.236944in}}%
\pgfpathlineto{\pgfqpoint{1.593611in}{1.334167in}}%
\pgfpathlineto{\pgfqpoint{1.315834in}{1.334167in}}%
\pgfpathlineto{\pgfqpoint{1.315834in}{1.236944in}}%
\pgfpathclose%
\pgfusepath{fill}%
\end{pgfscope}%
\begin{pgfscope}%
\definecolor{textcolor}{rgb}{0.000000,0.000000,0.000000}%
\pgfsetstrokecolor{textcolor}%
\pgfsetfillcolor{textcolor}%
\pgftext[x=1.704722in,y=1.236944in,left,base]{\color{textcolor}\rmfamily\fontsize{10.000000}{12.000000}\selectfont Pos}%
\end{pgfscope}%
\end{pgfpicture}%
\makeatother%
\endgroup%

  &
  \vspace{0pt} %% Creator: Matplotlib, PGF backend
%%
%% To include the figure in your LaTeX document, write
%%   \input{<filename>.pgf}
%%
%% Make sure the required packages are loaded in your preamble
%%   \usepackage{pgf}
%%
%% Also ensure that all the required font packages are loaded; for instance,
%% the lmodern package is sometimes necessary when using math font.
%%   \usepackage{lmodern}
%%
%% Figures using additional raster images can only be included by \input if
%% they are in the same directory as the main LaTeX file. For loading figures
%% from other directories you can use the `import` package
%%   \usepackage{import}
%%
%% and then include the figures with
%%   \import{<path to file>}{<filename>.pgf}
%%
%% Matplotlib used the following preamble
%%   
%%   \usepackage{fontspec}
%%   \makeatletter\@ifpackageloaded{underscore}{}{\usepackage[strings]{underscore}}\makeatother
%%
\begingroup%
\makeatletter%
\begin{pgfpicture}%
\pgfpathrectangle{\pgfpointorigin}{\pgfqpoint{2.221861in}{1.754444in}}%
\pgfusepath{use as bounding box, clip}%
\begin{pgfscope}%
\pgfsetbuttcap%
\pgfsetmiterjoin%
\definecolor{currentfill}{rgb}{1.000000,1.000000,1.000000}%
\pgfsetfillcolor{currentfill}%
\pgfsetlinewidth{0.000000pt}%
\definecolor{currentstroke}{rgb}{1.000000,1.000000,1.000000}%
\pgfsetstrokecolor{currentstroke}%
\pgfsetdash{}{0pt}%
\pgfpathmoveto{\pgfqpoint{0.000000in}{0.000000in}}%
\pgfpathlineto{\pgfqpoint{2.221861in}{0.000000in}}%
\pgfpathlineto{\pgfqpoint{2.221861in}{1.754444in}}%
\pgfpathlineto{\pgfqpoint{0.000000in}{1.754444in}}%
\pgfpathlineto{\pgfqpoint{0.000000in}{0.000000in}}%
\pgfpathclose%
\pgfusepath{fill}%
\end{pgfscope}%
\begin{pgfscope}%
\pgfsetbuttcap%
\pgfsetmiterjoin%
\definecolor{currentfill}{rgb}{1.000000,1.000000,1.000000}%
\pgfsetfillcolor{currentfill}%
\pgfsetlinewidth{0.000000pt}%
\definecolor{currentstroke}{rgb}{0.000000,0.000000,0.000000}%
\pgfsetstrokecolor{currentstroke}%
\pgfsetstrokeopacity{0.000000}%
\pgfsetdash{}{0pt}%
\pgfpathmoveto{\pgfqpoint{0.553581in}{0.499444in}}%
\pgfpathlineto{\pgfqpoint{2.103581in}{0.499444in}}%
\pgfpathlineto{\pgfqpoint{2.103581in}{1.654444in}}%
\pgfpathlineto{\pgfqpoint{0.553581in}{1.654444in}}%
\pgfpathlineto{\pgfqpoint{0.553581in}{0.499444in}}%
\pgfpathclose%
\pgfusepath{fill}%
\end{pgfscope}%
\begin{pgfscope}%
\pgfsetbuttcap%
\pgfsetroundjoin%
\definecolor{currentfill}{rgb}{0.000000,0.000000,0.000000}%
\pgfsetfillcolor{currentfill}%
\pgfsetlinewidth{0.803000pt}%
\definecolor{currentstroke}{rgb}{0.000000,0.000000,0.000000}%
\pgfsetstrokecolor{currentstroke}%
\pgfsetdash{}{0pt}%
\pgfsys@defobject{currentmarker}{\pgfqpoint{0.000000in}{-0.048611in}}{\pgfqpoint{0.000000in}{0.000000in}}{%
\pgfpathmoveto{\pgfqpoint{0.000000in}{0.000000in}}%
\pgfpathlineto{\pgfqpoint{0.000000in}{-0.048611in}}%
\pgfusepath{stroke,fill}%
}%
\begin{pgfscope}%
\pgfsys@transformshift{0.624035in}{0.499444in}%
\pgfsys@useobject{currentmarker}{}%
\end{pgfscope}%
\end{pgfscope}%
\begin{pgfscope}%
\definecolor{textcolor}{rgb}{0.000000,0.000000,0.000000}%
\pgfsetstrokecolor{textcolor}%
\pgfsetfillcolor{textcolor}%
\pgftext[x=0.624035in,y=0.402222in,,top]{\color{textcolor}\rmfamily\fontsize{10.000000}{12.000000}\selectfont \(\displaystyle {0.0}\)}%
\end{pgfscope}%
\begin{pgfscope}%
\pgfsetbuttcap%
\pgfsetroundjoin%
\definecolor{currentfill}{rgb}{0.000000,0.000000,0.000000}%
\pgfsetfillcolor{currentfill}%
\pgfsetlinewidth{0.803000pt}%
\definecolor{currentstroke}{rgb}{0.000000,0.000000,0.000000}%
\pgfsetstrokecolor{currentstroke}%
\pgfsetdash{}{0pt}%
\pgfsys@defobject{currentmarker}{\pgfqpoint{0.000000in}{-0.048611in}}{\pgfqpoint{0.000000in}{0.000000in}}{%
\pgfpathmoveto{\pgfqpoint{0.000000in}{0.000000in}}%
\pgfpathlineto{\pgfqpoint{0.000000in}{-0.048611in}}%
\pgfusepath{stroke,fill}%
}%
\begin{pgfscope}%
\pgfsys@transformshift{1.328581in}{0.499444in}%
\pgfsys@useobject{currentmarker}{}%
\end{pgfscope}%
\end{pgfscope}%
\begin{pgfscope}%
\definecolor{textcolor}{rgb}{0.000000,0.000000,0.000000}%
\pgfsetstrokecolor{textcolor}%
\pgfsetfillcolor{textcolor}%
\pgftext[x=1.328581in,y=0.402222in,,top]{\color{textcolor}\rmfamily\fontsize{10.000000}{12.000000}\selectfont \(\displaystyle {0.5}\)}%
\end{pgfscope}%
\begin{pgfscope}%
\pgfsetbuttcap%
\pgfsetroundjoin%
\definecolor{currentfill}{rgb}{0.000000,0.000000,0.000000}%
\pgfsetfillcolor{currentfill}%
\pgfsetlinewidth{0.803000pt}%
\definecolor{currentstroke}{rgb}{0.000000,0.000000,0.000000}%
\pgfsetstrokecolor{currentstroke}%
\pgfsetdash{}{0pt}%
\pgfsys@defobject{currentmarker}{\pgfqpoint{0.000000in}{-0.048611in}}{\pgfqpoint{0.000000in}{0.000000in}}{%
\pgfpathmoveto{\pgfqpoint{0.000000in}{0.000000in}}%
\pgfpathlineto{\pgfqpoint{0.000000in}{-0.048611in}}%
\pgfusepath{stroke,fill}%
}%
\begin{pgfscope}%
\pgfsys@transformshift{2.033126in}{0.499444in}%
\pgfsys@useobject{currentmarker}{}%
\end{pgfscope}%
\end{pgfscope}%
\begin{pgfscope}%
\definecolor{textcolor}{rgb}{0.000000,0.000000,0.000000}%
\pgfsetstrokecolor{textcolor}%
\pgfsetfillcolor{textcolor}%
\pgftext[x=2.033126in,y=0.402222in,,top]{\color{textcolor}\rmfamily\fontsize{10.000000}{12.000000}\selectfont \(\displaystyle {1.0}\)}%
\end{pgfscope}%
\begin{pgfscope}%
\definecolor{textcolor}{rgb}{0.000000,0.000000,0.000000}%
\pgfsetstrokecolor{textcolor}%
\pgfsetfillcolor{textcolor}%
\pgftext[x=1.328581in,y=0.223333in,,top]{\color{textcolor}\rmfamily\fontsize{10.000000}{12.000000}\selectfont False positive rate}%
\end{pgfscope}%
\begin{pgfscope}%
\pgfsetbuttcap%
\pgfsetroundjoin%
\definecolor{currentfill}{rgb}{0.000000,0.000000,0.000000}%
\pgfsetfillcolor{currentfill}%
\pgfsetlinewidth{0.803000pt}%
\definecolor{currentstroke}{rgb}{0.000000,0.000000,0.000000}%
\pgfsetstrokecolor{currentstroke}%
\pgfsetdash{}{0pt}%
\pgfsys@defobject{currentmarker}{\pgfqpoint{-0.048611in}{0.000000in}}{\pgfqpoint{-0.000000in}{0.000000in}}{%
\pgfpathmoveto{\pgfqpoint{-0.000000in}{0.000000in}}%
\pgfpathlineto{\pgfqpoint{-0.048611in}{0.000000in}}%
\pgfusepath{stroke,fill}%
}%
\begin{pgfscope}%
\pgfsys@transformshift{0.553581in}{0.551944in}%
\pgfsys@useobject{currentmarker}{}%
\end{pgfscope}%
\end{pgfscope}%
\begin{pgfscope}%
\definecolor{textcolor}{rgb}{0.000000,0.000000,0.000000}%
\pgfsetstrokecolor{textcolor}%
\pgfsetfillcolor{textcolor}%
\pgftext[x=0.278889in, y=0.503750in, left, base]{\color{textcolor}\rmfamily\fontsize{10.000000}{12.000000}\selectfont \(\displaystyle {0.0}\)}%
\end{pgfscope}%
\begin{pgfscope}%
\pgfsetbuttcap%
\pgfsetroundjoin%
\definecolor{currentfill}{rgb}{0.000000,0.000000,0.000000}%
\pgfsetfillcolor{currentfill}%
\pgfsetlinewidth{0.803000pt}%
\definecolor{currentstroke}{rgb}{0.000000,0.000000,0.000000}%
\pgfsetstrokecolor{currentstroke}%
\pgfsetdash{}{0pt}%
\pgfsys@defobject{currentmarker}{\pgfqpoint{-0.048611in}{0.000000in}}{\pgfqpoint{-0.000000in}{0.000000in}}{%
\pgfpathmoveto{\pgfqpoint{-0.000000in}{0.000000in}}%
\pgfpathlineto{\pgfqpoint{-0.048611in}{0.000000in}}%
\pgfusepath{stroke,fill}%
}%
\begin{pgfscope}%
\pgfsys@transformshift{0.553581in}{1.076944in}%
\pgfsys@useobject{currentmarker}{}%
\end{pgfscope}%
\end{pgfscope}%
\begin{pgfscope}%
\definecolor{textcolor}{rgb}{0.000000,0.000000,0.000000}%
\pgfsetstrokecolor{textcolor}%
\pgfsetfillcolor{textcolor}%
\pgftext[x=0.278889in, y=1.028750in, left, base]{\color{textcolor}\rmfamily\fontsize{10.000000}{12.000000}\selectfont \(\displaystyle {0.5}\)}%
\end{pgfscope}%
\begin{pgfscope}%
\pgfsetbuttcap%
\pgfsetroundjoin%
\definecolor{currentfill}{rgb}{0.000000,0.000000,0.000000}%
\pgfsetfillcolor{currentfill}%
\pgfsetlinewidth{0.803000pt}%
\definecolor{currentstroke}{rgb}{0.000000,0.000000,0.000000}%
\pgfsetstrokecolor{currentstroke}%
\pgfsetdash{}{0pt}%
\pgfsys@defobject{currentmarker}{\pgfqpoint{-0.048611in}{0.000000in}}{\pgfqpoint{-0.000000in}{0.000000in}}{%
\pgfpathmoveto{\pgfqpoint{-0.000000in}{0.000000in}}%
\pgfpathlineto{\pgfqpoint{-0.048611in}{0.000000in}}%
\pgfusepath{stroke,fill}%
}%
\begin{pgfscope}%
\pgfsys@transformshift{0.553581in}{1.601944in}%
\pgfsys@useobject{currentmarker}{}%
\end{pgfscope}%
\end{pgfscope}%
\begin{pgfscope}%
\definecolor{textcolor}{rgb}{0.000000,0.000000,0.000000}%
\pgfsetstrokecolor{textcolor}%
\pgfsetfillcolor{textcolor}%
\pgftext[x=0.278889in, y=1.553750in, left, base]{\color{textcolor}\rmfamily\fontsize{10.000000}{12.000000}\selectfont \(\displaystyle {1.0}\)}%
\end{pgfscope}%
\begin{pgfscope}%
\definecolor{textcolor}{rgb}{0.000000,0.000000,0.000000}%
\pgfsetstrokecolor{textcolor}%
\pgfsetfillcolor{textcolor}%
\pgftext[x=0.223333in,y=1.076944in,,bottom,rotate=90.000000]{\color{textcolor}\rmfamily\fontsize{10.000000}{12.000000}\selectfont True positive rate}%
\end{pgfscope}%
\begin{pgfscope}%
\pgfpathrectangle{\pgfqpoint{0.553581in}{0.499444in}}{\pgfqpoint{1.550000in}{1.155000in}}%
\pgfusepath{clip}%
\pgfsetbuttcap%
\pgfsetroundjoin%
\pgfsetlinewidth{1.505625pt}%
\definecolor{currentstroke}{rgb}{0.000000,0.000000,0.000000}%
\pgfsetstrokecolor{currentstroke}%
\pgfsetdash{{5.550000pt}{2.400000pt}}{0.000000pt}%
\pgfpathmoveto{\pgfqpoint{0.624035in}{0.551944in}}%
\pgfpathlineto{\pgfqpoint{2.033126in}{1.601944in}}%
\pgfusepath{stroke}%
\end{pgfscope}%
\begin{pgfscope}%
\pgfpathrectangle{\pgfqpoint{0.553581in}{0.499444in}}{\pgfqpoint{1.550000in}{1.155000in}}%
\pgfusepath{clip}%
\pgfsetrectcap%
\pgfsetroundjoin%
\pgfsetlinewidth{1.505625pt}%
\definecolor{currentstroke}{rgb}{0.000000,0.000000,0.000000}%
\pgfsetstrokecolor{currentstroke}%
\pgfsetdash{}{0pt}%
\pgfpathmoveto{\pgfqpoint{0.624035in}{0.551944in}}%
\pgfpathlineto{\pgfqpoint{0.627493in}{0.553009in}}%
\pgfpathlineto{\pgfqpoint{0.628559in}{0.554390in}}%
\pgfpathlineto{\pgfqpoint{0.629063in}{0.555494in}}%
\pgfpathlineto{\pgfqpoint{0.630166in}{0.558452in}}%
\pgfpathlineto{\pgfqpoint{0.630531in}{0.559557in}}%
\pgfpathlineto{\pgfqpoint{0.631624in}{0.564132in}}%
\pgfpathlineto{\pgfqpoint{0.631970in}{0.565197in}}%
\pgfpathlineto{\pgfqpoint{0.633073in}{0.569338in}}%
\pgfpathlineto{\pgfqpoint{0.633306in}{0.570443in}}%
\pgfpathlineto{\pgfqpoint{0.634409in}{0.576044in}}%
\pgfpathlineto{\pgfqpoint{0.634755in}{0.577148in}}%
\pgfpathlineto{\pgfqpoint{0.635858in}{0.582236in}}%
\pgfpathlineto{\pgfqpoint{0.636035in}{0.583104in}}%
\pgfpathlineto{\pgfqpoint{0.637138in}{0.589967in}}%
\pgfpathlineto{\pgfqpoint{0.637344in}{0.591032in}}%
\pgfpathlineto{\pgfqpoint{0.638447in}{0.596790in}}%
\pgfpathlineto{\pgfqpoint{0.638559in}{0.597895in}}%
\pgfpathlineto{\pgfqpoint{0.639652in}{0.604758in}}%
\pgfpathlineto{\pgfqpoint{0.639877in}{0.605823in}}%
\pgfpathlineto{\pgfqpoint{0.640979in}{0.612449in}}%
\pgfpathlineto{\pgfqpoint{0.641138in}{0.613553in}}%
\pgfpathlineto{\pgfqpoint{0.642241in}{0.621363in}}%
\pgfpathlineto{\pgfqpoint{0.642456in}{0.622467in}}%
\pgfpathlineto{\pgfqpoint{0.643549in}{0.631026in}}%
\pgfpathlineto{\pgfqpoint{0.643802in}{0.631973in}}%
\pgfpathlineto{\pgfqpoint{0.644895in}{0.639428in}}%
\pgfpathlineto{\pgfqpoint{0.645073in}{0.640532in}}%
\pgfpathlineto{\pgfqpoint{0.646138in}{0.649131in}}%
\pgfpathlineto{\pgfqpoint{0.656213in}{0.650235in}}%
\pgfpathlineto{\pgfqpoint{0.657316in}{0.658636in}}%
\pgfpathlineto{\pgfqpoint{0.657475in}{0.659662in}}%
\pgfpathlineto{\pgfqpoint{0.658559in}{0.668812in}}%
\pgfpathlineto{\pgfqpoint{0.658746in}{0.669877in}}%
\pgfpathlineto{\pgfqpoint{0.659849in}{0.678989in}}%
\pgfpathlineto{\pgfqpoint{0.659942in}{0.680014in}}%
\pgfpathlineto{\pgfqpoint{0.661045in}{0.688415in}}%
\pgfpathlineto{\pgfqpoint{0.661176in}{0.689520in}}%
\pgfpathlineto{\pgfqpoint{0.662279in}{0.697527in}}%
\pgfpathlineto{\pgfqpoint{0.662475in}{0.698631in}}%
\pgfpathlineto{\pgfqpoint{0.663578in}{0.705691in}}%
\pgfpathlineto{\pgfqpoint{0.663821in}{0.706717in}}%
\pgfpathlineto{\pgfqpoint{0.664924in}{0.715394in}}%
\pgfpathlineto{\pgfqpoint{0.665129in}{0.716459in}}%
\pgfpathlineto{\pgfqpoint{0.666223in}{0.724269in}}%
\pgfpathlineto{\pgfqpoint{0.666372in}{0.725334in}}%
\pgfpathlineto{\pgfqpoint{0.667475in}{0.732473in}}%
\pgfpathlineto{\pgfqpoint{0.667643in}{0.733538in}}%
\pgfpathlineto{\pgfqpoint{0.668737in}{0.741189in}}%
\pgfpathlineto{\pgfqpoint{0.668942in}{0.742254in}}%
\pgfpathlineto{\pgfqpoint{0.670026in}{0.751326in}}%
\pgfpathlineto{\pgfqpoint{0.670213in}{0.752431in}}%
\pgfpathlineto{\pgfqpoint{0.671316in}{0.759964in}}%
\pgfpathlineto{\pgfqpoint{0.671606in}{0.761029in}}%
\pgfpathlineto{\pgfqpoint{0.672709in}{0.766788in}}%
\pgfpathlineto{\pgfqpoint{0.672961in}{0.767892in}}%
\pgfpathlineto{\pgfqpoint{0.674064in}{0.775978in}}%
\pgfpathlineto{\pgfqpoint{0.674241in}{0.777043in}}%
\pgfpathlineto{\pgfqpoint{0.675307in}{0.783235in}}%
\pgfpathlineto{\pgfqpoint{0.675634in}{0.784340in}}%
\pgfpathlineto{\pgfqpoint{0.676737in}{0.791124in}}%
\pgfpathlineto{\pgfqpoint{0.676877in}{0.791991in}}%
\pgfpathlineto{\pgfqpoint{0.677980in}{0.799643in}}%
\pgfpathlineto{\pgfqpoint{0.678101in}{0.800669in}}%
\pgfpathlineto{\pgfqpoint{0.679195in}{0.806861in}}%
\pgfpathlineto{\pgfqpoint{0.679466in}{0.807966in}}%
\pgfpathlineto{\pgfqpoint{0.680569in}{0.813685in}}%
\pgfpathlineto{\pgfqpoint{0.680746in}{0.814632in}}%
\pgfpathlineto{\pgfqpoint{0.681840in}{0.821652in}}%
\pgfpathlineto{\pgfqpoint{0.682017in}{0.822717in}}%
\pgfpathlineto{\pgfqpoint{0.683111in}{0.829383in}}%
\pgfpathlineto{\pgfqpoint{0.683316in}{0.830448in}}%
\pgfpathlineto{\pgfqpoint{0.684410in}{0.836522in}}%
\pgfpathlineto{\pgfqpoint{0.684578in}{0.837469in}}%
\pgfpathlineto{\pgfqpoint{0.685681in}{0.843306in}}%
\pgfpathlineto{\pgfqpoint{0.685914in}{0.844411in}}%
\pgfpathlineto{\pgfqpoint{0.686999in}{0.851116in}}%
\pgfpathlineto{\pgfqpoint{0.687195in}{0.851786in}}%
\pgfpathlineto{\pgfqpoint{0.688298in}{0.859951in}}%
\pgfpathlineto{\pgfqpoint{0.688466in}{0.860937in}}%
\pgfpathlineto{\pgfqpoint{0.689569in}{0.867603in}}%
\pgfpathlineto{\pgfqpoint{0.689765in}{0.868668in}}%
\pgfpathlineto{\pgfqpoint{0.690858in}{0.874505in}}%
\pgfpathlineto{\pgfqpoint{0.691073in}{0.875610in}}%
\pgfpathlineto{\pgfqpoint{0.692176in}{0.881684in}}%
\pgfpathlineto{\pgfqpoint{0.692410in}{0.882709in}}%
\pgfpathlineto{\pgfqpoint{0.693513in}{0.889296in}}%
\pgfpathlineto{\pgfqpoint{0.693709in}{0.890401in}}%
\pgfpathlineto{\pgfqpoint{0.694812in}{0.895804in}}%
\pgfpathlineto{\pgfqpoint{0.695176in}{0.896909in}}%
\pgfpathlineto{\pgfqpoint{0.696279in}{0.902549in}}%
\pgfpathlineto{\pgfqpoint{0.696569in}{0.903653in}}%
\pgfpathlineto{\pgfqpoint{0.697672in}{0.909846in}}%
\pgfpathlineto{\pgfqpoint{0.697915in}{0.910911in}}%
\pgfpathlineto{\pgfqpoint{0.698989in}{0.917458in}}%
\pgfpathlineto{\pgfqpoint{0.699279in}{0.918523in}}%
\pgfpathlineto{\pgfqpoint{0.700354in}{0.923454in}}%
\pgfpathlineto{\pgfqpoint{0.700569in}{0.924479in}}%
\pgfpathlineto{\pgfqpoint{0.701672in}{0.929685in}}%
\pgfpathlineto{\pgfqpoint{0.702027in}{0.930790in}}%
\pgfpathlineto{\pgfqpoint{0.703130in}{0.937180in}}%
\pgfpathlineto{\pgfqpoint{0.703335in}{0.938284in}}%
\pgfpathlineto{\pgfqpoint{0.704429in}{0.943609in}}%
\pgfpathlineto{\pgfqpoint{0.704709in}{0.944634in}}%
\pgfpathlineto{\pgfqpoint{0.705765in}{0.949446in}}%
\pgfpathlineto{\pgfqpoint{0.706214in}{0.950472in}}%
\pgfpathlineto{\pgfqpoint{0.707307in}{0.955165in}}%
\pgfpathlineto{\pgfqpoint{0.707634in}{0.956270in}}%
\pgfpathlineto{\pgfqpoint{0.708718in}{0.961200in}}%
\pgfpathlineto{\pgfqpoint{0.708989in}{0.962265in}}%
\pgfpathlineto{\pgfqpoint{0.710083in}{0.966604in}}%
\pgfpathlineto{\pgfqpoint{0.710242in}{0.967708in}}%
\pgfpathlineto{\pgfqpoint{0.711335in}{0.972165in}}%
\pgfpathlineto{\pgfqpoint{0.711532in}{0.973269in}}%
\pgfpathlineto{\pgfqpoint{0.712625in}{0.978042in}}%
\pgfpathlineto{\pgfqpoint{0.712896in}{0.979107in}}%
\pgfpathlineto{\pgfqpoint{0.713980in}{0.983801in}}%
\pgfpathlineto{\pgfqpoint{0.714251in}{0.984905in}}%
\pgfpathlineto{\pgfqpoint{0.715354in}{0.990348in}}%
\pgfpathlineto{\pgfqpoint{0.715597in}{0.991452in}}%
\pgfpathlineto{\pgfqpoint{0.716681in}{0.996067in}}%
\pgfpathlineto{\pgfqpoint{0.717074in}{0.997172in}}%
\pgfpathlineto{\pgfqpoint{0.718177in}{1.001786in}}%
\pgfpathlineto{\pgfqpoint{0.718476in}{1.002851in}}%
\pgfpathlineto{\pgfqpoint{0.719578in}{1.007506in}}%
\pgfpathlineto{\pgfqpoint{0.719905in}{1.008531in}}%
\pgfpathlineto{\pgfqpoint{0.720999in}{1.012988in}}%
\pgfpathlineto{\pgfqpoint{0.721401in}{1.014092in}}%
\pgfpathlineto{\pgfqpoint{0.722504in}{1.017840in}}%
\pgfpathlineto{\pgfqpoint{0.722765in}{1.018904in}}%
\pgfpathlineto{\pgfqpoint{0.723859in}{1.022691in}}%
\pgfpathlineto{\pgfqpoint{0.724111in}{1.023756in}}%
\pgfpathlineto{\pgfqpoint{0.725205in}{1.026990in}}%
\pgfpathlineto{\pgfqpoint{0.725606in}{1.028055in}}%
\pgfpathlineto{\pgfqpoint{0.726709in}{1.031526in}}%
\pgfpathlineto{\pgfqpoint{0.727074in}{1.032630in}}%
\pgfpathlineto{\pgfqpoint{0.728177in}{1.036023in}}%
\pgfpathlineto{\pgfqpoint{0.728392in}{1.037009in}}%
\pgfpathlineto{\pgfqpoint{0.729485in}{1.041702in}}%
\pgfpathlineto{\pgfqpoint{0.729775in}{1.042767in}}%
\pgfpathlineto{\pgfqpoint{0.730878in}{1.046435in}}%
\pgfpathlineto{\pgfqpoint{0.731214in}{1.047540in}}%
\pgfpathlineto{\pgfqpoint{0.732308in}{1.050853in}}%
\pgfpathlineto{\pgfqpoint{0.732775in}{1.051957in}}%
\pgfpathlineto{\pgfqpoint{0.733878in}{1.055152in}}%
\pgfpathlineto{\pgfqpoint{0.734093in}{1.056178in}}%
\pgfpathlineto{\pgfqpoint{0.735177in}{1.060043in}}%
\pgfpathlineto{\pgfqpoint{0.735551in}{1.061108in}}%
\pgfpathlineto{\pgfqpoint{0.736653in}{1.064618in}}%
\pgfpathlineto{\pgfqpoint{0.737046in}{1.065683in}}%
\pgfpathlineto{\pgfqpoint{0.738130in}{1.068681in}}%
\pgfpathlineto{\pgfqpoint{0.738551in}{1.069785in}}%
\pgfpathlineto{\pgfqpoint{0.739644in}{1.073887in}}%
\pgfpathlineto{\pgfqpoint{0.740027in}{1.074992in}}%
\pgfpathlineto{\pgfqpoint{0.741130in}{1.077871in}}%
\pgfpathlineto{\pgfqpoint{0.741551in}{1.078975in}}%
\pgfpathlineto{\pgfqpoint{0.742644in}{1.081973in}}%
\pgfpathlineto{\pgfqpoint{0.743233in}{1.083077in}}%
\pgfpathlineto{\pgfqpoint{0.744336in}{1.086036in}}%
\pgfpathlineto{\pgfqpoint{0.744934in}{1.087061in}}%
\pgfpathlineto{\pgfqpoint{0.746037in}{1.090769in}}%
\pgfpathlineto{\pgfqpoint{0.746448in}{1.091873in}}%
\pgfpathlineto{\pgfqpoint{0.747523in}{1.095778in}}%
\pgfpathlineto{\pgfqpoint{0.747887in}{1.096843in}}%
\pgfpathlineto{\pgfqpoint{0.748934in}{1.100551in}}%
\pgfpathlineto{\pgfqpoint{0.749280in}{1.101655in}}%
\pgfpathlineto{\pgfqpoint{0.750364in}{1.104574in}}%
\pgfpathlineto{\pgfqpoint{0.750756in}{1.105678in}}%
\pgfpathlineto{\pgfqpoint{0.751840in}{1.108360in}}%
\pgfpathlineto{\pgfqpoint{0.752298in}{1.109465in}}%
\pgfpathlineto{\pgfqpoint{0.753373in}{1.112107in}}%
\pgfpathlineto{\pgfqpoint{0.753999in}{1.113172in}}%
\pgfpathlineto{\pgfqpoint{0.755083in}{1.115696in}}%
\pgfpathlineto{\pgfqpoint{0.755569in}{1.116801in}}%
\pgfpathlineto{\pgfqpoint{0.756672in}{1.119838in}}%
\pgfpathlineto{\pgfqpoint{0.756981in}{1.120706in}}%
\pgfpathlineto{\pgfqpoint{0.758065in}{1.124532in}}%
\pgfpathlineto{\pgfqpoint{0.758551in}{1.125636in}}%
\pgfpathlineto{\pgfqpoint{0.759607in}{1.128949in}}%
\pgfpathlineto{\pgfqpoint{0.760168in}{1.130054in}}%
\pgfpathlineto{\pgfqpoint{0.761270in}{1.132854in}}%
\pgfpathlineto{\pgfqpoint{0.761999in}{1.133958in}}%
\pgfpathlineto{\pgfqpoint{0.763046in}{1.136680in}}%
\pgfpathlineto{\pgfqpoint{0.763822in}{1.137784in}}%
\pgfpathlineto{\pgfqpoint{0.764925in}{1.139875in}}%
\pgfpathlineto{\pgfqpoint{0.765355in}{1.140979in}}%
\pgfpathlineto{\pgfqpoint{0.766448in}{1.143622in}}%
\pgfpathlineto{\pgfqpoint{0.767065in}{1.144726in}}%
\pgfpathlineto{\pgfqpoint{0.768121in}{1.146896in}}%
\pgfpathlineto{\pgfqpoint{0.768672in}{1.148000in}}%
\pgfpathlineto{\pgfqpoint{0.769710in}{1.150485in}}%
\pgfpathlineto{\pgfqpoint{0.770327in}{1.151589in}}%
\pgfpathlineto{\pgfqpoint{0.771429in}{1.154587in}}%
\pgfpathlineto{\pgfqpoint{0.772000in}{1.155691in}}%
\pgfpathlineto{\pgfqpoint{0.773084in}{1.157506in}}%
\pgfpathlineto{\pgfqpoint{0.773813in}{1.158571in}}%
\pgfpathlineto{\pgfqpoint{0.774906in}{1.161016in}}%
\pgfpathlineto{\pgfqpoint{0.775476in}{1.162120in}}%
\pgfpathlineto{\pgfqpoint{0.776570in}{1.164250in}}%
\pgfpathlineto{\pgfqpoint{0.777140in}{1.165355in}}%
\pgfpathlineto{\pgfqpoint{0.778177in}{1.167642in}}%
\pgfpathlineto{\pgfqpoint{0.778850in}{1.168747in}}%
\pgfpathlineto{\pgfqpoint{0.779953in}{1.170600in}}%
\pgfpathlineto{\pgfqpoint{0.780392in}{1.171508in}}%
\pgfpathlineto{\pgfqpoint{0.781476in}{1.173874in}}%
\pgfpathlineto{\pgfqpoint{0.781953in}{1.174979in}}%
\pgfpathlineto{\pgfqpoint{0.783056in}{1.176793in}}%
\pgfpathlineto{\pgfqpoint{0.783635in}{1.177897in}}%
\pgfpathlineto{\pgfqpoint{0.784729in}{1.179396in}}%
\pgfpathlineto{\pgfqpoint{0.785374in}{1.180501in}}%
\pgfpathlineto{\pgfqpoint{0.786439in}{1.182512in}}%
\pgfpathlineto{\pgfqpoint{0.787121in}{1.183617in}}%
\pgfpathlineto{\pgfqpoint{0.788224in}{1.185589in}}%
\pgfpathlineto{\pgfqpoint{0.788934in}{1.186693in}}%
\pgfpathlineto{\pgfqpoint{0.789878in}{1.187916in}}%
\pgfpathlineto{\pgfqpoint{0.790607in}{1.188981in}}%
\pgfpathlineto{\pgfqpoint{0.791617in}{1.191189in}}%
\pgfpathlineto{\pgfqpoint{0.792271in}{1.192294in}}%
\pgfpathlineto{\pgfqpoint{0.793364in}{1.193990in}}%
\pgfpathlineto{\pgfqpoint{0.793916in}{1.195055in}}%
\pgfpathlineto{\pgfqpoint{0.795009in}{1.196790in}}%
\pgfpathlineto{\pgfqpoint{0.795551in}{1.197855in}}%
\pgfpathlineto{\pgfqpoint{0.796626in}{1.199709in}}%
\pgfpathlineto{\pgfqpoint{0.797346in}{1.200813in}}%
\pgfpathlineto{\pgfqpoint{0.798439in}{1.203101in}}%
\pgfpathlineto{\pgfqpoint{0.799196in}{1.204206in}}%
\pgfpathlineto{\pgfqpoint{0.800299in}{1.206257in}}%
\pgfpathlineto{\pgfqpoint{0.800720in}{1.207361in}}%
\pgfpathlineto{\pgfqpoint{0.801785in}{1.209294in}}%
\pgfpathlineto{\pgfqpoint{0.802467in}{1.210398in}}%
\pgfpathlineto{\pgfqpoint{0.803561in}{1.212843in}}%
\pgfpathlineto{\pgfqpoint{0.804168in}{1.213948in}}%
\pgfpathlineto{\pgfqpoint{0.805252in}{1.215723in}}%
\pgfpathlineto{\pgfqpoint{0.806103in}{1.216827in}}%
\pgfpathlineto{\pgfqpoint{0.807206in}{1.219312in}}%
\pgfpathlineto{\pgfqpoint{0.807841in}{1.220416in}}%
\pgfpathlineto{\pgfqpoint{0.808897in}{1.222073in}}%
\pgfpathlineto{\pgfqpoint{0.809421in}{1.223177in}}%
\pgfpathlineto{\pgfqpoint{0.810514in}{1.224755in}}%
\pgfpathlineto{\pgfqpoint{0.811327in}{1.225859in}}%
\pgfpathlineto{\pgfqpoint{0.812421in}{1.228029in}}%
\pgfpathlineto{\pgfqpoint{0.812888in}{1.229133in}}%
\pgfpathlineto{\pgfqpoint{0.813907in}{1.230593in}}%
\pgfpathlineto{\pgfqpoint{0.814776in}{1.231697in}}%
\pgfpathlineto{\pgfqpoint{0.815879in}{1.233354in}}%
\pgfpathlineto{\pgfqpoint{0.816505in}{1.234340in}}%
\pgfpathlineto{\pgfqpoint{0.817598in}{1.236272in}}%
\pgfpathlineto{\pgfqpoint{0.817981in}{1.237377in}}%
\pgfpathlineto{\pgfqpoint{0.819075in}{1.239191in}}%
\pgfpathlineto{\pgfqpoint{0.819748in}{1.240295in}}%
\pgfpathlineto{\pgfqpoint{0.820832in}{1.241400in}}%
\pgfpathlineto{\pgfqpoint{0.821654in}{1.242465in}}%
\pgfpathlineto{\pgfqpoint{0.822692in}{1.243806in}}%
\pgfpathlineto{\pgfqpoint{0.823486in}{1.244831in}}%
\pgfpathlineto{\pgfqpoint{0.824552in}{1.246527in}}%
\pgfpathlineto{\pgfqpoint{0.825477in}{1.247632in}}%
\pgfpathlineto{\pgfqpoint{0.826580in}{1.249880in}}%
\pgfpathlineto{\pgfqpoint{0.827103in}{1.250984in}}%
\pgfpathlineto{\pgfqpoint{0.828196in}{1.252523in}}%
\pgfpathlineto{\pgfqpoint{0.829243in}{1.253627in}}%
\pgfpathlineto{\pgfqpoint{0.830299in}{1.255402in}}%
\pgfpathlineto{\pgfqpoint{0.831253in}{1.256467in}}%
\pgfpathlineto{\pgfqpoint{0.832346in}{1.258045in}}%
\pgfpathlineto{\pgfqpoint{0.833533in}{1.259149in}}%
\pgfpathlineto{\pgfqpoint{0.834608in}{1.260490in}}%
\pgfpathlineto{\pgfqpoint{0.835542in}{1.261594in}}%
\pgfpathlineto{\pgfqpoint{0.836598in}{1.262738in}}%
\pgfpathlineto{\pgfqpoint{0.837785in}{1.263843in}}%
\pgfpathlineto{\pgfqpoint{0.838879in}{1.265815in}}%
\pgfpathlineto{\pgfqpoint{0.839683in}{1.266919in}}%
\pgfpathlineto{\pgfqpoint{0.840739in}{1.268260in}}%
\pgfpathlineto{\pgfqpoint{0.841337in}{1.269365in}}%
\pgfpathlineto{\pgfqpoint{0.842412in}{1.270863in}}%
\pgfpathlineto{\pgfqpoint{0.843458in}{1.271968in}}%
\pgfpathlineto{\pgfqpoint{0.844542in}{1.273112in}}%
\pgfpathlineto{\pgfqpoint{0.845860in}{1.274216in}}%
\pgfpathlineto{\pgfqpoint{0.846935in}{1.275439in}}%
\pgfpathlineto{\pgfqpoint{0.847748in}{1.276504in}}%
\pgfpathlineto{\pgfqpoint{0.848842in}{1.278318in}}%
\pgfpathlineto{\pgfqpoint{0.849636in}{1.279383in}}%
\pgfpathlineto{\pgfqpoint{0.850739in}{1.280448in}}%
\pgfpathlineto{\pgfqpoint{0.851935in}{1.281513in}}%
\pgfpathlineto{\pgfqpoint{0.853038in}{1.283091in}}%
\pgfpathlineto{\pgfqpoint{0.854225in}{1.284195in}}%
\pgfpathlineto{\pgfqpoint{0.855225in}{1.285418in}}%
\pgfpathlineto{\pgfqpoint{0.856094in}{1.286522in}}%
\pgfpathlineto{\pgfqpoint{0.857038in}{1.287982in}}%
\pgfpathlineto{\pgfqpoint{0.857860in}{1.289086in}}%
\pgfpathlineto{\pgfqpoint{0.858944in}{1.290506in}}%
\pgfpathlineto{\pgfqpoint{0.859954in}{1.291610in}}%
\pgfpathlineto{\pgfqpoint{0.861057in}{1.293109in}}%
\pgfpathlineto{\pgfqpoint{0.861786in}{1.294174in}}%
\pgfpathlineto{\pgfqpoint{0.862888in}{1.295515in}}%
\pgfpathlineto{\pgfqpoint{0.863524in}{1.296619in}}%
\pgfpathlineto{\pgfqpoint{0.864608in}{1.297605in}}%
\pgfpathlineto{\pgfqpoint{0.865748in}{1.298710in}}%
\pgfpathlineto{\pgfqpoint{0.866842in}{1.300011in}}%
\pgfpathlineto{\pgfqpoint{0.868122in}{1.301116in}}%
\pgfpathlineto{\pgfqpoint{0.869206in}{1.302417in}}%
\pgfpathlineto{\pgfqpoint{0.869973in}{1.303522in}}%
\pgfpathlineto{\pgfqpoint{0.871038in}{1.304508in}}%
\pgfpathlineto{\pgfqpoint{0.872786in}{1.305573in}}%
\pgfpathlineto{\pgfqpoint{0.873823in}{1.306520in}}%
\pgfpathlineto{\pgfqpoint{0.874870in}{1.307545in}}%
\pgfpathlineto{\pgfqpoint{0.875954in}{1.309004in}}%
\pgfpathlineto{\pgfqpoint{0.876935in}{1.310109in}}%
\pgfpathlineto{\pgfqpoint{0.877973in}{1.311292in}}%
\pgfpathlineto{\pgfqpoint{0.878973in}{1.312357in}}%
\pgfpathlineto{\pgfqpoint{0.880076in}{1.313580in}}%
\pgfpathlineto{\pgfqpoint{0.881216in}{1.314684in}}%
\pgfpathlineto{\pgfqpoint{0.882291in}{1.315867in}}%
\pgfpathlineto{\pgfqpoint{0.883534in}{1.316972in}}%
\pgfpathlineto{\pgfqpoint{0.884618in}{1.318037in}}%
\pgfpathlineto{\pgfqpoint{0.885870in}{1.319141in}}%
\pgfpathlineto{\pgfqpoint{0.886973in}{1.320679in}}%
\pgfpathlineto{\pgfqpoint{0.887907in}{1.321744in}}%
\pgfpathlineto{\pgfqpoint{0.889001in}{1.323125in}}%
\pgfpathlineto{\pgfqpoint{0.890870in}{1.324229in}}%
\pgfpathlineto{\pgfqpoint{0.891926in}{1.325294in}}%
\pgfpathlineto{\pgfqpoint{0.893216in}{1.326399in}}%
\pgfpathlineto{\pgfqpoint{0.894319in}{1.327306in}}%
\pgfpathlineto{\pgfqpoint{0.895823in}{1.328410in}}%
\pgfpathlineto{\pgfqpoint{0.896908in}{1.329357in}}%
\pgfpathlineto{\pgfqpoint{0.897795in}{1.330461in}}%
\pgfpathlineto{\pgfqpoint{0.898898in}{1.331368in}}%
\pgfpathlineto{\pgfqpoint{0.900590in}{1.332473in}}%
\pgfpathlineto{\pgfqpoint{0.901683in}{1.333735in}}%
\pgfpathlineto{\pgfqpoint{0.902674in}{1.334839in}}%
\pgfpathlineto{\pgfqpoint{0.903758in}{1.336023in}}%
\pgfpathlineto{\pgfqpoint{0.905085in}{1.337127in}}%
\pgfpathlineto{\pgfqpoint{0.906141in}{1.338310in}}%
\pgfpathlineto{\pgfqpoint{0.907440in}{1.339415in}}%
\pgfpathlineto{\pgfqpoint{0.908253in}{1.340046in}}%
\pgfpathlineto{\pgfqpoint{0.910067in}{1.341150in}}%
\pgfpathlineto{\pgfqpoint{0.911132in}{1.342254in}}%
\pgfpathlineto{\pgfqpoint{0.912282in}{1.343359in}}%
\pgfpathlineto{\pgfqpoint{0.913338in}{1.344187in}}%
\pgfpathlineto{\pgfqpoint{0.914908in}{1.345292in}}%
\pgfpathlineto{\pgfqpoint{0.916001in}{1.346396in}}%
\pgfpathlineto{\pgfqpoint{0.917169in}{1.347500in}}%
\pgfpathlineto{\pgfqpoint{0.918226in}{1.348486in}}%
\pgfpathlineto{\pgfqpoint{0.919627in}{1.349591in}}%
\pgfpathlineto{\pgfqpoint{0.920721in}{1.350537in}}%
\pgfpathlineto{\pgfqpoint{0.922104in}{1.351642in}}%
\pgfpathlineto{\pgfqpoint{0.923020in}{1.352983in}}%
\pgfpathlineto{\pgfqpoint{0.924515in}{1.354087in}}%
\pgfpathlineto{\pgfqpoint{0.925618in}{1.355073in}}%
\pgfpathlineto{\pgfqpoint{0.926618in}{1.356059in}}%
\pgfpathlineto{\pgfqpoint{0.927684in}{1.357045in}}%
\pgfpathlineto{\pgfqpoint{0.929188in}{1.358150in}}%
\pgfpathlineto{\pgfqpoint{0.930160in}{1.359215in}}%
\pgfpathlineto{\pgfqpoint{0.932039in}{1.360319in}}%
\pgfpathlineto{\pgfqpoint{0.933067in}{1.361147in}}%
\pgfpathlineto{\pgfqpoint{0.934871in}{1.362252in}}%
\pgfpathlineto{\pgfqpoint{0.935964in}{1.363238in}}%
\pgfpathlineto{\pgfqpoint{0.937544in}{1.364342in}}%
\pgfpathlineto{\pgfqpoint{0.938618in}{1.365289in}}%
\pgfpathlineto{\pgfqpoint{0.939843in}{1.366393in}}%
\pgfpathlineto{\pgfqpoint{0.940871in}{1.367300in}}%
\pgfpathlineto{\pgfqpoint{0.942534in}{1.368405in}}%
\pgfpathlineto{\pgfqpoint{0.943618in}{1.369194in}}%
\pgfpathlineto{\pgfqpoint{0.945656in}{1.370298in}}%
\pgfpathlineto{\pgfqpoint{0.946702in}{1.371087in}}%
\pgfpathlineto{\pgfqpoint{0.948039in}{1.372191in}}%
\pgfpathlineto{\pgfqpoint{0.949076in}{1.372822in}}%
\pgfpathlineto{\pgfqpoint{0.950908in}{1.373927in}}%
\pgfpathlineto{\pgfqpoint{0.952011in}{1.374716in}}%
\pgfpathlineto{\pgfqpoint{0.953665in}{1.375820in}}%
\pgfpathlineto{\pgfqpoint{0.954768in}{1.376727in}}%
\pgfpathlineto{\pgfqpoint{0.956544in}{1.377753in}}%
\pgfpathlineto{\pgfqpoint{0.957600in}{1.378660in}}%
\pgfpathlineto{\pgfqpoint{0.959749in}{1.379764in}}%
\pgfpathlineto{\pgfqpoint{0.960843in}{1.381026in}}%
\pgfpathlineto{\pgfqpoint{0.962749in}{1.382131in}}%
\pgfpathlineto{\pgfqpoint{0.963805in}{1.382841in}}%
\pgfpathlineto{\pgfqpoint{0.965516in}{1.383945in}}%
\pgfpathlineto{\pgfqpoint{0.966478in}{1.384695in}}%
\pgfpathlineto{\pgfqpoint{0.968376in}{1.385799in}}%
\pgfpathlineto{\pgfqpoint{0.969432in}{1.386588in}}%
\pgfpathlineto{\pgfqpoint{0.971806in}{1.387692in}}%
\pgfpathlineto{\pgfqpoint{0.972834in}{1.388402in}}%
\pgfpathlineto{\pgfqpoint{0.974637in}{1.389507in}}%
\pgfpathlineto{\pgfqpoint{0.975656in}{1.390217in}}%
\pgfpathlineto{\pgfqpoint{0.978049in}{1.391321in}}%
\pgfpathlineto{\pgfqpoint{0.978918in}{1.391755in}}%
\pgfpathlineto{\pgfqpoint{0.981011in}{1.392859in}}%
\pgfpathlineto{\pgfqpoint{0.981955in}{1.393451in}}%
\pgfpathlineto{\pgfqpoint{0.984105in}{1.394555in}}%
\pgfpathlineto{\pgfqpoint{0.985208in}{1.395226in}}%
\pgfpathlineto{\pgfqpoint{0.987245in}{1.396330in}}%
\pgfpathlineto{\pgfqpoint{0.988348in}{1.397080in}}%
\pgfpathlineto{\pgfqpoint{0.989853in}{1.398184in}}%
\pgfpathlineto{\pgfqpoint{0.990871in}{1.398815in}}%
\pgfpathlineto{\pgfqpoint{0.993329in}{1.399919in}}%
\pgfpathlineto{\pgfqpoint{0.994404in}{1.400551in}}%
\pgfpathlineto{\pgfqpoint{0.996797in}{1.401655in}}%
\pgfpathlineto{\pgfqpoint{0.997834in}{1.402207in}}%
\pgfpathlineto{\pgfqpoint{0.999955in}{1.403312in}}%
\pgfpathlineto{\pgfqpoint{1.000890in}{1.404021in}}%
\pgfpathlineto{\pgfqpoint{1.003096in}{1.405126in}}%
\pgfpathlineto{\pgfqpoint{1.004152in}{1.405718in}}%
\pgfpathlineto{\pgfqpoint{1.006600in}{1.406822in}}%
\pgfpathlineto{\pgfqpoint{1.007554in}{1.407374in}}%
\pgfpathlineto{\pgfqpoint{1.009133in}{1.408478in}}%
\pgfpathlineto{\pgfqpoint{1.010096in}{1.409188in}}%
\pgfpathlineto{\pgfqpoint{1.012217in}{1.410293in}}%
\pgfpathlineto{\pgfqpoint{1.013311in}{1.411003in}}%
\pgfpathlineto{\pgfqpoint{1.015395in}{1.412107in}}%
\pgfpathlineto{\pgfqpoint{1.016404in}{1.412620in}}%
\pgfpathlineto{\pgfqpoint{1.018853in}{1.413724in}}%
\pgfpathlineto{\pgfqpoint{1.019956in}{1.414316in}}%
\pgfpathlineto{\pgfqpoint{1.022283in}{1.415420in}}%
\pgfpathlineto{\pgfqpoint{1.023367in}{1.416170in}}%
\pgfpathlineto{\pgfqpoint{1.025788in}{1.417274in}}%
\pgfpathlineto{\pgfqpoint{1.026573in}{1.417866in}}%
\pgfpathlineto{\pgfqpoint{1.029180in}{1.418931in}}%
\pgfpathlineto{\pgfqpoint{1.030189in}{1.419246in}}%
\pgfpathlineto{\pgfqpoint{1.032545in}{1.420351in}}%
\pgfpathlineto{\pgfqpoint{1.033647in}{1.420824in}}%
\pgfpathlineto{\pgfqpoint{1.036461in}{1.421928in}}%
\pgfpathlineto{\pgfqpoint{1.037535in}{1.422441in}}%
\pgfpathlineto{\pgfqpoint{1.039984in}{1.423546in}}%
\pgfpathlineto{\pgfqpoint{1.040975in}{1.424255in}}%
\pgfpathlineto{\pgfqpoint{1.043928in}{1.425360in}}%
\pgfpathlineto{\pgfqpoint{1.045021in}{1.425991in}}%
\pgfpathlineto{\pgfqpoint{1.046984in}{1.427095in}}%
\pgfpathlineto{\pgfqpoint{1.047928in}{1.427529in}}%
\pgfpathlineto{\pgfqpoint{1.050685in}{1.428634in}}%
\pgfpathlineto{\pgfqpoint{1.051704in}{1.429344in}}%
\pgfpathlineto{\pgfqpoint{1.054685in}{1.430448in}}%
\pgfpathlineto{\pgfqpoint{1.055722in}{1.430961in}}%
\pgfpathlineto{\pgfqpoint{1.058012in}{1.432065in}}%
\pgfpathlineto{\pgfqpoint{1.059068in}{1.432420in}}%
\pgfpathlineto{\pgfqpoint{1.061367in}{1.433524in}}%
\pgfpathlineto{\pgfqpoint{1.062321in}{1.433998in}}%
\pgfpathlineto{\pgfqpoint{1.064975in}{1.435102in}}%
\pgfpathlineto{\pgfqpoint{1.065863in}{1.435260in}}%
\pgfpathlineto{\pgfqpoint{1.069237in}{1.436364in}}%
\pgfpathlineto{\pgfqpoint{1.070237in}{1.436838in}}%
\pgfpathlineto{\pgfqpoint{1.072984in}{1.437942in}}%
\pgfpathlineto{\pgfqpoint{1.073956in}{1.438336in}}%
\pgfpathlineto{\pgfqpoint{1.076517in}{1.439441in}}%
\pgfpathlineto{\pgfqpoint{1.077442in}{1.439993in}}%
\pgfpathlineto{\pgfqpoint{1.080797in}{1.441097in}}%
\pgfpathlineto{\pgfqpoint{1.081751in}{1.441531in}}%
\pgfpathlineto{\pgfqpoint{1.084583in}{1.442636in}}%
\pgfpathlineto{\pgfqpoint{1.085611in}{1.443188in}}%
\pgfpathlineto{\pgfqpoint{1.088284in}{1.444292in}}%
\pgfpathlineto{\pgfqpoint{1.089237in}{1.444687in}}%
\pgfpathlineto{\pgfqpoint{1.092676in}{1.445752in}}%
\pgfpathlineto{\pgfqpoint{1.093685in}{1.446422in}}%
\pgfpathlineto{\pgfqpoint{1.097031in}{1.447487in}}%
\pgfpathlineto{\pgfqpoint{1.098115in}{1.448079in}}%
\pgfpathlineto{\pgfqpoint{1.101471in}{1.449183in}}%
\pgfpathlineto{\pgfqpoint{1.102433in}{1.449735in}}%
\pgfpathlineto{\pgfqpoint{1.105723in}{1.450840in}}%
\pgfpathlineto{\pgfqpoint{1.106788in}{1.451195in}}%
\pgfpathlineto{\pgfqpoint{1.110293in}{1.452299in}}%
\pgfpathlineto{\pgfqpoint{1.111377in}{1.453009in}}%
\pgfpathlineto{\pgfqpoint{1.113891in}{1.454114in}}%
\pgfpathlineto{\pgfqpoint{1.114994in}{1.454587in}}%
\pgfpathlineto{\pgfqpoint{1.118265in}{1.455691in}}%
\pgfpathlineto{\pgfqpoint{1.119134in}{1.456283in}}%
\pgfpathlineto{\pgfqpoint{1.123125in}{1.457387in}}%
\pgfpathlineto{\pgfqpoint{1.124172in}{1.457742in}}%
\pgfpathlineto{\pgfqpoint{1.127284in}{1.458847in}}%
\pgfpathlineto{\pgfqpoint{1.128228in}{1.459123in}}%
\pgfpathlineto{\pgfqpoint{1.131602in}{1.460227in}}%
\pgfpathlineto{\pgfqpoint{1.132705in}{1.460819in}}%
\pgfpathlineto{\pgfqpoint{1.136443in}{1.461923in}}%
\pgfpathlineto{\pgfqpoint{1.137527in}{1.462436in}}%
\pgfpathlineto{\pgfqpoint{1.140340in}{1.463540in}}%
\pgfpathlineto{\pgfqpoint{1.141200in}{1.463895in}}%
\pgfpathlineto{\pgfqpoint{1.144434in}{1.465000in}}%
\pgfpathlineto{\pgfqpoint{1.145527in}{1.465473in}}%
\pgfpathlineto{\pgfqpoint{1.150163in}{1.466577in}}%
\pgfpathlineto{\pgfqpoint{1.151256in}{1.466932in}}%
\pgfpathlineto{\pgfqpoint{1.156312in}{1.468037in}}%
\pgfpathlineto{\pgfqpoint{1.157387in}{1.468549in}}%
\pgfpathlineto{\pgfqpoint{1.160967in}{1.469614in}}%
\pgfpathlineto{\pgfqpoint{1.162032in}{1.470246in}}%
\pgfpathlineto{\pgfqpoint{1.165013in}{1.471350in}}%
\pgfpathlineto{\pgfqpoint{1.166098in}{1.471744in}}%
\pgfpathlineto{\pgfqpoint{1.169621in}{1.472849in}}%
\pgfpathlineto{\pgfqpoint{1.170668in}{1.473283in}}%
\pgfpathlineto{\pgfqpoint{1.174985in}{1.474387in}}%
\pgfpathlineto{\pgfqpoint{1.175957in}{1.474703in}}%
\pgfpathlineto{\pgfqpoint{1.179593in}{1.475807in}}%
\pgfpathlineto{\pgfqpoint{1.180696in}{1.476122in}}%
\pgfpathlineto{\pgfqpoint{1.184098in}{1.477227in}}%
\pgfpathlineto{\pgfqpoint{1.185116in}{1.477582in}}%
\pgfpathlineto{\pgfqpoint{1.190388in}{1.478686in}}%
\pgfpathlineto{\pgfqpoint{1.191425in}{1.479120in}}%
\pgfpathlineto{\pgfqpoint{1.195453in}{1.480224in}}%
\pgfpathlineto{\pgfqpoint{1.196528in}{1.480501in}}%
\pgfpathlineto{\pgfqpoint{1.200247in}{1.481605in}}%
\pgfpathlineto{\pgfqpoint{1.201285in}{1.482039in}}%
\pgfpathlineto{\pgfqpoint{1.205911in}{1.483104in}}%
\pgfpathlineto{\pgfqpoint{1.206855in}{1.483577in}}%
\pgfpathlineto{\pgfqpoint{1.212892in}{1.484681in}}%
\pgfpathlineto{\pgfqpoint{1.213921in}{1.484958in}}%
\pgfpathlineto{\pgfqpoint{1.217874in}{1.486062in}}%
\pgfpathlineto{\pgfqpoint{1.218967in}{1.486496in}}%
\pgfpathlineto{\pgfqpoint{1.222556in}{1.487600in}}%
\pgfpathlineto{\pgfqpoint{1.223594in}{1.487837in}}%
\pgfpathlineto{\pgfqpoint{1.227566in}{1.488941in}}%
\pgfpathlineto{\pgfqpoint{1.228631in}{1.489296in}}%
\pgfpathlineto{\pgfqpoint{1.231846in}{1.490401in}}%
\pgfpathlineto{\pgfqpoint{1.232911in}{1.490598in}}%
\pgfpathlineto{\pgfqpoint{1.240323in}{1.491702in}}%
\pgfpathlineto{\pgfqpoint{1.241407in}{1.492136in}}%
\pgfpathlineto{\pgfqpoint{1.246164in}{1.493241in}}%
\pgfpathlineto{\pgfqpoint{1.246968in}{1.493477in}}%
\pgfpathlineto{\pgfqpoint{1.251827in}{1.494582in}}%
\pgfpathlineto{\pgfqpoint{1.252827in}{1.494976in}}%
\pgfpathlineto{\pgfqpoint{1.259089in}{1.496080in}}%
\pgfpathlineto{\pgfqpoint{1.259968in}{1.496356in}}%
\pgfpathlineto{\pgfqpoint{1.265295in}{1.497461in}}%
\pgfpathlineto{\pgfqpoint{1.266267in}{1.497895in}}%
\pgfpathlineto{\pgfqpoint{1.270949in}{1.498999in}}%
\pgfpathlineto{\pgfqpoint{1.271846in}{1.499472in}}%
\pgfpathlineto{\pgfqpoint{1.277641in}{1.500577in}}%
\pgfpathlineto{\pgfqpoint{1.278669in}{1.501011in}}%
\pgfpathlineto{\pgfqpoint{1.282847in}{1.502115in}}%
\pgfpathlineto{\pgfqpoint{1.283725in}{1.502786in}}%
\pgfpathlineto{\pgfqpoint{1.289725in}{1.503890in}}%
\pgfpathlineto{\pgfqpoint{1.290781in}{1.504206in}}%
\pgfpathlineto{\pgfqpoint{1.297342in}{1.505310in}}%
\pgfpathlineto{\pgfqpoint{1.298314in}{1.505468in}}%
\pgfpathlineto{\pgfqpoint{1.303211in}{1.506572in}}%
\pgfpathlineto{\pgfqpoint{1.304211in}{1.506927in}}%
\pgfpathlineto{\pgfqpoint{1.311342in}{1.508031in}}%
\pgfpathlineto{\pgfqpoint{1.312426in}{1.508347in}}%
\pgfpathlineto{\pgfqpoint{1.319847in}{1.509451in}}%
\pgfpathlineto{\pgfqpoint{1.320931in}{1.509885in}}%
\pgfpathlineto{\pgfqpoint{1.325866in}{1.510990in}}%
\pgfpathlineto{\pgfqpoint{1.326492in}{1.511226in}}%
\pgfpathlineto{\pgfqpoint{1.333324in}{1.512331in}}%
\pgfpathlineto{\pgfqpoint{1.334370in}{1.512607in}}%
\pgfpathlineto{\pgfqpoint{1.342670in}{1.513711in}}%
\pgfpathlineto{\pgfqpoint{1.343539in}{1.513869in}}%
\pgfpathlineto{\pgfqpoint{1.348053in}{1.514973in}}%
\pgfpathlineto{\pgfqpoint{1.349081in}{1.515328in}}%
\pgfpathlineto{\pgfqpoint{1.356445in}{1.516433in}}%
\pgfpathlineto{\pgfqpoint{1.357399in}{1.516630in}}%
\pgfpathlineto{\pgfqpoint{1.366137in}{1.517734in}}%
\pgfpathlineto{\pgfqpoint{1.367090in}{1.518050in}}%
\pgfpathlineto{\pgfqpoint{1.373707in}{1.519154in}}%
\pgfpathlineto{\pgfqpoint{1.374810in}{1.519391in}}%
\pgfpathlineto{\pgfqpoint{1.381119in}{1.520495in}}%
\pgfpathlineto{\pgfqpoint{1.382072in}{1.520693in}}%
\pgfpathlineto{\pgfqpoint{1.389633in}{1.521797in}}%
\pgfpathlineto{\pgfqpoint{1.390736in}{1.521955in}}%
\pgfpathlineto{\pgfqpoint{1.397212in}{1.523059in}}%
\pgfpathlineto{\pgfqpoint{1.398138in}{1.523138in}}%
\pgfpathlineto{\pgfqpoint{1.403979in}{1.524242in}}%
\pgfpathlineto{\pgfqpoint{1.404642in}{1.524361in}}%
\pgfpathlineto{\pgfqpoint{1.413100in}{1.525465in}}%
\pgfpathlineto{\pgfqpoint{1.413988in}{1.525662in}}%
\pgfpathlineto{\pgfqpoint{1.422951in}{1.526767in}}%
\pgfpathlineto{\pgfqpoint{1.423970in}{1.526924in}}%
\pgfpathlineto{\pgfqpoint{1.431661in}{1.528029in}}%
\pgfpathlineto{\pgfqpoint{1.432026in}{1.528147in}}%
\pgfpathlineto{\pgfqpoint{1.441671in}{1.529252in}}%
\pgfpathlineto{\pgfqpoint{1.442475in}{1.529409in}}%
\pgfpathlineto{\pgfqpoint{1.455456in}{1.530514in}}%
\pgfpathlineto{\pgfqpoint{1.456475in}{1.530632in}}%
\pgfpathlineto{\pgfqpoint{1.466662in}{1.531736in}}%
\pgfpathlineto{\pgfqpoint{1.467699in}{1.532013in}}%
\pgfpathlineto{\pgfqpoint{1.477288in}{1.533117in}}%
\pgfpathlineto{\pgfqpoint{1.477297in}{1.533196in}}%
\pgfpathlineto{\pgfqpoint{1.487157in}{1.534300in}}%
\pgfpathlineto{\pgfqpoint{1.488232in}{1.534616in}}%
\pgfpathlineto{\pgfqpoint{1.496615in}{1.535720in}}%
\pgfpathlineto{\pgfqpoint{1.497139in}{1.535917in}}%
\pgfpathlineto{\pgfqpoint{1.508559in}{1.537022in}}%
\pgfpathlineto{\pgfqpoint{1.509382in}{1.537101in}}%
\pgfpathlineto{\pgfqpoint{1.517681in}{1.538205in}}%
\pgfpathlineto{\pgfqpoint{1.517737in}{1.538284in}}%
\pgfpathlineto{\pgfqpoint{1.529877in}{1.539388in}}%
\pgfpathlineto{\pgfqpoint{1.530933in}{1.539625in}}%
\pgfpathlineto{\pgfqpoint{1.538691in}{1.540729in}}%
\pgfpathlineto{\pgfqpoint{1.539597in}{1.540927in}}%
\pgfpathlineto{\pgfqpoint{1.549859in}{1.542031in}}%
\pgfpathlineto{\pgfqpoint{1.550952in}{1.542268in}}%
\pgfpathlineto{\pgfqpoint{1.559345in}{1.543372in}}%
\pgfpathlineto{\pgfqpoint{1.560336in}{1.543490in}}%
\pgfpathlineto{\pgfqpoint{1.571560in}{1.544595in}}%
\pgfpathlineto{\pgfqpoint{1.572579in}{1.544752in}}%
\pgfpathlineto{\pgfqpoint{1.580943in}{1.545857in}}%
\pgfpathlineto{\pgfqpoint{1.581224in}{1.545936in}}%
\pgfpathlineto{\pgfqpoint{1.592205in}{1.547040in}}%
\pgfpathlineto{\pgfqpoint{1.592887in}{1.547237in}}%
\pgfpathlineto{\pgfqpoint{1.603757in}{1.548342in}}%
\pgfpathlineto{\pgfqpoint{1.604056in}{1.548421in}}%
\pgfpathlineto{\pgfqpoint{1.612327in}{1.549525in}}%
\pgfpathlineto{\pgfqpoint{1.612570in}{1.549604in}}%
\pgfpathlineto{\pgfqpoint{1.629720in}{1.550708in}}%
\pgfpathlineto{\pgfqpoint{1.630514in}{1.550787in}}%
\pgfpathlineto{\pgfqpoint{1.639318in}{1.551379in}}%
\pgfpathlineto{\pgfqpoint{1.640281in}{1.560806in}}%
\pgfpathlineto{\pgfqpoint{1.653505in}{1.561910in}}%
\pgfpathlineto{\pgfqpoint{1.653589in}{1.561989in}}%
\pgfpathlineto{\pgfqpoint{1.666430in}{1.563093in}}%
\pgfpathlineto{\pgfqpoint{1.667281in}{1.563251in}}%
\pgfpathlineto{\pgfqpoint{1.680365in}{1.564355in}}%
\pgfpathlineto{\pgfqpoint{1.681374in}{1.564632in}}%
\pgfpathlineto{\pgfqpoint{1.692608in}{1.565736in}}%
\pgfpathlineto{\pgfqpoint{1.693683in}{1.565854in}}%
\pgfpathlineto{\pgfqpoint{1.704758in}{1.566959in}}%
\pgfpathlineto{\pgfqpoint{1.705655in}{1.567156in}}%
\pgfpathlineto{\pgfqpoint{1.715627in}{1.568260in}}%
\pgfpathlineto{\pgfqpoint{1.716702in}{1.568379in}}%
\pgfpathlineto{\pgfqpoint{1.729721in}{1.569483in}}%
\pgfpathlineto{\pgfqpoint{1.730796in}{1.569601in}}%
\pgfpathlineto{\pgfqpoint{1.745375in}{1.570706in}}%
\pgfpathlineto{\pgfqpoint{1.745628in}{1.570824in}}%
\pgfpathlineto{\pgfqpoint{1.762282in}{1.571928in}}%
\pgfpathlineto{\pgfqpoint{1.762488in}{1.572047in}}%
\pgfpathlineto{\pgfqpoint{1.773693in}{1.573151in}}%
\pgfpathlineto{\pgfqpoint{1.774506in}{1.573388in}}%
\pgfpathlineto{\pgfqpoint{1.786890in}{1.574492in}}%
\pgfpathlineto{\pgfqpoint{1.787815in}{1.574729in}}%
\pgfpathlineto{\pgfqpoint{1.797852in}{1.575833in}}%
\pgfpathlineto{\pgfqpoint{1.797852in}{1.575873in}}%
\pgfpathlineto{\pgfqpoint{1.810750in}{1.576977in}}%
\pgfpathlineto{\pgfqpoint{1.811479in}{1.577095in}}%
\pgfpathlineto{\pgfqpoint{1.825956in}{1.578200in}}%
\pgfpathlineto{\pgfqpoint{1.826451in}{1.578318in}}%
\pgfpathlineto{\pgfqpoint{1.839993in}{1.579422in}}%
\pgfpathlineto{\pgfqpoint{1.840844in}{1.579620in}}%
\pgfpathlineto{\pgfqpoint{1.855535in}{1.580724in}}%
\pgfpathlineto{\pgfqpoint{1.856517in}{1.580921in}}%
\pgfpathlineto{\pgfqpoint{1.870582in}{1.582026in}}%
\pgfpathlineto{\pgfqpoint{1.871582in}{1.582223in}}%
\pgfpathlineto{\pgfqpoint{1.884424in}{1.583327in}}%
\pgfpathlineto{\pgfqpoint{1.885349in}{1.583446in}}%
\pgfpathlineto{\pgfqpoint{1.903807in}{1.584550in}}%
\pgfpathlineto{\pgfqpoint{1.904611in}{1.584787in}}%
\pgfpathlineto{\pgfqpoint{1.922433in}{1.585891in}}%
\pgfpathlineto{\pgfqpoint{1.923489in}{1.585970in}}%
\pgfpathlineto{\pgfqpoint{1.934218in}{1.587074in}}%
\pgfpathlineto{\pgfqpoint{1.935312in}{1.587153in}}%
\pgfpathlineto{\pgfqpoint{1.948835in}{1.588258in}}%
\pgfpathlineto{\pgfqpoint{1.949480in}{1.588376in}}%
\pgfpathlineto{\pgfqpoint{1.964910in}{1.589480in}}%
\pgfpathlineto{\pgfqpoint{1.965752in}{1.589638in}}%
\pgfpathlineto{\pgfqpoint{1.975574in}{1.590742in}}%
\pgfpathlineto{\pgfqpoint{1.976284in}{1.590900in}}%
\pgfpathlineto{\pgfqpoint{1.987434in}{1.592005in}}%
\pgfpathlineto{\pgfqpoint{1.988490in}{1.592281in}}%
\pgfpathlineto{\pgfqpoint{2.000387in}{1.593385in}}%
\pgfpathlineto{\pgfqpoint{2.001444in}{1.593661in}}%
\pgfpathlineto{\pgfqpoint{2.008930in}{1.594766in}}%
\pgfpathlineto{\pgfqpoint{2.010032in}{1.594963in}}%
\pgfpathlineto{\pgfqpoint{2.018135in}{1.596067in}}%
\pgfpathlineto{\pgfqpoint{2.018995in}{1.596264in}}%
\pgfpathlineto{\pgfqpoint{2.024892in}{1.597369in}}%
\pgfpathlineto{\pgfqpoint{2.025677in}{1.597566in}}%
\pgfpathlineto{\pgfqpoint{2.025874in}{1.597566in}}%
\pgfpathlineto{\pgfqpoint{2.033126in}{1.601944in}}%
\pgfpathlineto{\pgfqpoint{2.033126in}{1.601944in}}%
\pgfusepath{stroke}%
\end{pgfscope}%
\begin{pgfscope}%
\pgfsetrectcap%
\pgfsetmiterjoin%
\pgfsetlinewidth{0.803000pt}%
\definecolor{currentstroke}{rgb}{0.000000,0.000000,0.000000}%
\pgfsetstrokecolor{currentstroke}%
\pgfsetdash{}{0pt}%
\pgfpathmoveto{\pgfqpoint{0.553581in}{0.499444in}}%
\pgfpathlineto{\pgfqpoint{0.553581in}{1.654444in}}%
\pgfusepath{stroke}%
\end{pgfscope}%
\begin{pgfscope}%
\pgfsetrectcap%
\pgfsetmiterjoin%
\pgfsetlinewidth{0.803000pt}%
\definecolor{currentstroke}{rgb}{0.000000,0.000000,0.000000}%
\pgfsetstrokecolor{currentstroke}%
\pgfsetdash{}{0pt}%
\pgfpathmoveto{\pgfqpoint{2.103581in}{0.499444in}}%
\pgfpathlineto{\pgfqpoint{2.103581in}{1.654444in}}%
\pgfusepath{stroke}%
\end{pgfscope}%
\begin{pgfscope}%
\pgfsetrectcap%
\pgfsetmiterjoin%
\pgfsetlinewidth{0.803000pt}%
\definecolor{currentstroke}{rgb}{0.000000,0.000000,0.000000}%
\pgfsetstrokecolor{currentstroke}%
\pgfsetdash{}{0pt}%
\pgfpathmoveto{\pgfqpoint{0.553581in}{0.499444in}}%
\pgfpathlineto{\pgfqpoint{2.103581in}{0.499444in}}%
\pgfusepath{stroke}%
\end{pgfscope}%
\begin{pgfscope}%
\pgfsetrectcap%
\pgfsetmiterjoin%
\pgfsetlinewidth{0.803000pt}%
\definecolor{currentstroke}{rgb}{0.000000,0.000000,0.000000}%
\pgfsetstrokecolor{currentstroke}%
\pgfsetdash{}{0pt}%
\pgfpathmoveto{\pgfqpoint{0.553581in}{1.654444in}}%
\pgfpathlineto{\pgfqpoint{2.103581in}{1.654444in}}%
\pgfusepath{stroke}%
\end{pgfscope}%
\begin{pgfscope}%
\pgfsetbuttcap%
\pgfsetmiterjoin%
\definecolor{currentfill}{rgb}{1.000000,1.000000,1.000000}%
\pgfsetfillcolor{currentfill}%
\pgfsetlinewidth{0.000000pt}%
\definecolor{currentstroke}{rgb}{0.000000,0.000000,0.000000}%
\pgfsetstrokecolor{currentstroke}%
\pgfsetstrokeopacity{0.000000}%
\pgfsetdash{}{0pt}%
\pgfpathmoveto{\pgfqpoint{1.287031in}{1.442891in}}%
\pgfpathlineto{\pgfqpoint{1.686753in}{1.442891in}}%
\pgfpathlineto{\pgfqpoint{1.686753in}{1.649558in}}%
\pgfpathlineto{\pgfqpoint{1.287031in}{1.649558in}}%
\pgfpathlineto{\pgfqpoint{1.287031in}{1.442891in}}%
\pgfpathclose%
\pgfusepath{fill}%
\end{pgfscope}%
\begin{pgfscope}%
\definecolor{textcolor}{rgb}{0.000000,0.000000,0.000000}%
\pgfsetstrokecolor{textcolor}%
\pgfsetfillcolor{textcolor}%
\pgftext[x=1.328698in,y=1.511502in,left,base]{\color{textcolor}\rmfamily\fontsize{10.000000}{12.000000}\selectfont 0.244}%
\end{pgfscope}%
\begin{pgfscope}%
\pgfsetbuttcap%
\pgfsetmiterjoin%
\definecolor{currentfill}{rgb}{1.000000,1.000000,1.000000}%
\pgfsetfillcolor{currentfill}%
\pgfsetlinewidth{0.000000pt}%
\definecolor{currentstroke}{rgb}{0.000000,0.000000,0.000000}%
\pgfsetstrokecolor{currentstroke}%
\pgfsetstrokeopacity{0.000000}%
\pgfsetdash{}{0pt}%
\pgfpathmoveto{\pgfqpoint{0.699043in}{1.008353in}}%
\pgfpathlineto{\pgfqpoint{1.098765in}{1.008353in}}%
\pgfpathlineto{\pgfqpoint{1.098765in}{1.215019in}}%
\pgfpathlineto{\pgfqpoint{0.699043in}{1.215019in}}%
\pgfpathlineto{\pgfqpoint{0.699043in}{1.008353in}}%
\pgfpathclose%
\pgfusepath{fill}%
\end{pgfscope}%
\begin{pgfscope}%
\definecolor{textcolor}{rgb}{0.000000,0.000000,0.000000}%
\pgfsetstrokecolor{textcolor}%
\pgfsetfillcolor{textcolor}%
\pgftext[x=0.740709in,y=1.076964in,left,base]{\color{textcolor}\rmfamily\fontsize{10.000000}{12.000000}\selectfont 0.522}%
\end{pgfscope}%
\begin{pgfscope}%
\pgfsetbuttcap%
\pgfsetmiterjoin%
\definecolor{currentfill}{rgb}{1.000000,1.000000,1.000000}%
\pgfsetfillcolor{currentfill}%
\pgfsetfillopacity{0.800000}%
\pgfsetlinewidth{1.003750pt}%
\definecolor{currentstroke}{rgb}{0.800000,0.800000,0.800000}%
\pgfsetstrokecolor{currentstroke}%
\pgfsetstrokeopacity{0.800000}%
\pgfsetdash{}{0pt}%
\pgfpathmoveto{\pgfqpoint{0.832747in}{0.568889in}}%
\pgfpathlineto{\pgfqpoint{2.006358in}{0.568889in}}%
\pgfpathquadraticcurveto{\pgfqpoint{2.034136in}{0.568889in}}{\pgfqpoint{2.034136in}{0.596666in}}%
\pgfpathlineto{\pgfqpoint{2.034136in}{0.776388in}}%
\pgfpathquadraticcurveto{\pgfqpoint{2.034136in}{0.804166in}}{\pgfqpoint{2.006358in}{0.804166in}}%
\pgfpathlineto{\pgfqpoint{0.832747in}{0.804166in}}%
\pgfpathquadraticcurveto{\pgfqpoint{0.804970in}{0.804166in}}{\pgfqpoint{0.804970in}{0.776388in}}%
\pgfpathlineto{\pgfqpoint{0.804970in}{0.596666in}}%
\pgfpathquadraticcurveto{\pgfqpoint{0.804970in}{0.568889in}}{\pgfqpoint{0.832747in}{0.568889in}}%
\pgfpathlineto{\pgfqpoint{0.832747in}{0.568889in}}%
\pgfpathclose%
\pgfusepath{stroke,fill}%
\end{pgfscope}%
\begin{pgfscope}%
\pgfsetrectcap%
\pgfsetroundjoin%
\pgfsetlinewidth{1.505625pt}%
\definecolor{currentstroke}{rgb}{0.000000,0.000000,0.000000}%
\pgfsetstrokecolor{currentstroke}%
\pgfsetdash{}{0pt}%
\pgfpathmoveto{\pgfqpoint{0.860525in}{0.700000in}}%
\pgfpathlineto{\pgfqpoint{0.999414in}{0.700000in}}%
\pgfpathlineto{\pgfqpoint{1.138303in}{0.700000in}}%
\pgfusepath{stroke}%
\end{pgfscope}%
\begin{pgfscope}%
\definecolor{textcolor}{rgb}{0.000000,0.000000,0.000000}%
\pgfsetstrokecolor{textcolor}%
\pgfsetfillcolor{textcolor}%
\pgftext[x=1.249414in,y=0.651388in,left,base]{\color{textcolor}\rmfamily\fontsize{10.000000}{12.000000}\selectfont AUC=0.832}%
\end{pgfscope}%
\end{pgfpicture}%
\makeatother%
\endgroup%

  &
\vspace{0pt} 
  
\begin{tabular}{cc|c|c|}
	&\multicolumn{1}{c}{}& \multicolumn{2}{c}{Prediction} \cr
	&\multicolumn{1}{c}{} & \multicolumn{1}{c}{N} & \multicolumn{1}{c}{P} \cr\cline{3-4}
	\multirow{2}{*}{\rotatebox[origin=c]{90}{Actual}}&N & 136,348 & 14,423 \vrule width 0pt height 10pt depth 2pt \cr\cline{3-4}
	&P & 12,017 & 14,604 \vrule width 0pt height 10pt depth 2pt \cr\cline{3-4}
\end{tabular}

\begin{center}
\begin{tabular}{ll}
0.503 & Precision \cr 
0.549 & Recall \cr 
0.525 & F1 \cr 
\end{tabular}
\end{center}
  
\end{tabular}

We have decided that we want $\Delta FP/\Delta TP < 2.0$; we had $FP/TP < 2.0$, and now have $FP/TP \approx 1$, sending fewer ambulances to crashes that need one.  Why is this better?  If we look at the change in FP and TP from changing the threshold $p = 0.5 \to 0.635$, 

$$\frac{\Delta FP}{\Delta TP} = \frac{32,842 - 14,423}{20,693-14,604} = \frac{18,419}{6,089} \approx 3$$

By changing the threshold, we have not sent some ambulances that were needed, but have not sent three times as many ambulances that were not needed.  Given our goal of $\Delta FP/\Delta TP < 2$, this tradeoff is appropriate.  

It sometimes happens that, using $p=0.5$, a model will recommend that we never send an ambulance, as in Example 3 below, which is a linear transformation of Example 1, $f(x) = 0.5x$.  (One cause of such a model is the class weight being too low.) Such a model can still be useful if it separates the negative and positive classes well, because we can move the threshold.   


\noindent\begin{tabular}{@{}p{0.3\textwidth}@{\hspace{24pt}} p{0.3\textwidth} @{\hspace{24pt}} p{0.3\textwidth}}
  \vspace{0pt} %% Creator: Matplotlib, PGF backend
%%
%% To include the figure in your LaTeX document, write
%%   \input{<filename>.pgf}
%%
%% Make sure the required packages are loaded in your preamble
%%   \usepackage{pgf}
%%
%% Also ensure that all the required font packages are loaded; for instance,
%% the lmodern package is sometimes necessary when using math font.
%%   \usepackage{lmodern}
%%
%% Figures using additional raster images can only be included by \input if
%% they are in the same directory as the main LaTeX file. For loading figures
%% from other directories you can use the `import` package
%%   \usepackage{import}
%%
%% and then include the figures with
%%   \import{<path to file>}{<filename>.pgf}
%%
%% Matplotlib used the following preamble
%%   
%%   \usepackage{fontspec}
%%   \makeatletter\@ifpackageloaded{underscore}{}{\usepackage[strings]{underscore}}\makeatother
%%
\begingroup%
\makeatletter%
\begin{pgfpicture}%
\pgfpathrectangle{\pgfpointorigin}{\pgfqpoint{2.253750in}{1.754444in}}%
\pgfusepath{use as bounding box, clip}%
\begin{pgfscope}%
\pgfsetbuttcap%
\pgfsetmiterjoin%
\definecolor{currentfill}{rgb}{1.000000,1.000000,1.000000}%
\pgfsetfillcolor{currentfill}%
\pgfsetlinewidth{0.000000pt}%
\definecolor{currentstroke}{rgb}{1.000000,1.000000,1.000000}%
\pgfsetstrokecolor{currentstroke}%
\pgfsetdash{}{0pt}%
\pgfpathmoveto{\pgfqpoint{0.000000in}{0.000000in}}%
\pgfpathlineto{\pgfqpoint{2.253750in}{0.000000in}}%
\pgfpathlineto{\pgfqpoint{2.253750in}{1.754444in}}%
\pgfpathlineto{\pgfqpoint{0.000000in}{1.754444in}}%
\pgfpathlineto{\pgfqpoint{0.000000in}{0.000000in}}%
\pgfpathclose%
\pgfusepath{fill}%
\end{pgfscope}%
\begin{pgfscope}%
\pgfsetbuttcap%
\pgfsetmiterjoin%
\definecolor{currentfill}{rgb}{1.000000,1.000000,1.000000}%
\pgfsetfillcolor{currentfill}%
\pgfsetlinewidth{0.000000pt}%
\definecolor{currentstroke}{rgb}{0.000000,0.000000,0.000000}%
\pgfsetstrokecolor{currentstroke}%
\pgfsetstrokeopacity{0.000000}%
\pgfsetdash{}{0pt}%
\pgfpathmoveto{\pgfqpoint{0.515000in}{0.499444in}}%
\pgfpathlineto{\pgfqpoint{2.065000in}{0.499444in}}%
\pgfpathlineto{\pgfqpoint{2.065000in}{1.654444in}}%
\pgfpathlineto{\pgfqpoint{0.515000in}{1.654444in}}%
\pgfpathlineto{\pgfqpoint{0.515000in}{0.499444in}}%
\pgfpathclose%
\pgfusepath{fill}%
\end{pgfscope}%
\begin{pgfscope}%
\pgfpathrectangle{\pgfqpoint{0.515000in}{0.499444in}}{\pgfqpoint{1.550000in}{1.155000in}}%
\pgfusepath{clip}%
\pgfsetbuttcap%
\pgfsetmiterjoin%
\pgfsetlinewidth{1.003750pt}%
\definecolor{currentstroke}{rgb}{0.000000,0.000000,0.000000}%
\pgfsetstrokecolor{currentstroke}%
\pgfsetdash{}{0pt}%
\pgfpathmoveto{\pgfqpoint{0.505000in}{0.499444in}}%
\pgfpathlineto{\pgfqpoint{0.552805in}{0.499444in}}%
\pgfpathlineto{\pgfqpoint{0.552805in}{0.956876in}}%
\pgfpathlineto{\pgfqpoint{0.505000in}{0.956876in}}%
\pgfusepath{stroke}%
\end{pgfscope}%
\begin{pgfscope}%
\pgfpathrectangle{\pgfqpoint{0.515000in}{0.499444in}}{\pgfqpoint{1.550000in}{1.155000in}}%
\pgfusepath{clip}%
\pgfsetbuttcap%
\pgfsetmiterjoin%
\pgfsetlinewidth{1.003750pt}%
\definecolor{currentstroke}{rgb}{0.000000,0.000000,0.000000}%
\pgfsetstrokecolor{currentstroke}%
\pgfsetdash{}{0pt}%
\pgfpathmoveto{\pgfqpoint{0.643537in}{0.499444in}}%
\pgfpathlineto{\pgfqpoint{0.704025in}{0.499444in}}%
\pgfpathlineto{\pgfqpoint{0.704025in}{1.599444in}}%
\pgfpathlineto{\pgfqpoint{0.643537in}{1.599444in}}%
\pgfpathlineto{\pgfqpoint{0.643537in}{0.499444in}}%
\pgfpathclose%
\pgfusepath{stroke}%
\end{pgfscope}%
\begin{pgfscope}%
\pgfpathrectangle{\pgfqpoint{0.515000in}{0.499444in}}{\pgfqpoint{1.550000in}{1.155000in}}%
\pgfusepath{clip}%
\pgfsetbuttcap%
\pgfsetmiterjoin%
\pgfsetlinewidth{1.003750pt}%
\definecolor{currentstroke}{rgb}{0.000000,0.000000,0.000000}%
\pgfsetstrokecolor{currentstroke}%
\pgfsetdash{}{0pt}%
\pgfpathmoveto{\pgfqpoint{0.794756in}{0.499444in}}%
\pgfpathlineto{\pgfqpoint{0.855244in}{0.499444in}}%
\pgfpathlineto{\pgfqpoint{0.855244in}{1.124995in}}%
\pgfpathlineto{\pgfqpoint{0.794756in}{1.124995in}}%
\pgfpathlineto{\pgfqpoint{0.794756in}{0.499444in}}%
\pgfpathclose%
\pgfusepath{stroke}%
\end{pgfscope}%
\begin{pgfscope}%
\pgfpathrectangle{\pgfqpoint{0.515000in}{0.499444in}}{\pgfqpoint{1.550000in}{1.155000in}}%
\pgfusepath{clip}%
\pgfsetbuttcap%
\pgfsetmiterjoin%
\pgfsetlinewidth{1.003750pt}%
\definecolor{currentstroke}{rgb}{0.000000,0.000000,0.000000}%
\pgfsetstrokecolor{currentstroke}%
\pgfsetdash{}{0pt}%
\pgfpathmoveto{\pgfqpoint{0.945976in}{0.499444in}}%
\pgfpathlineto{\pgfqpoint{1.006464in}{0.499444in}}%
\pgfpathlineto{\pgfqpoint{1.006464in}{0.718484in}}%
\pgfpathlineto{\pgfqpoint{0.945976in}{0.718484in}}%
\pgfpathlineto{\pgfqpoint{0.945976in}{0.499444in}}%
\pgfpathclose%
\pgfusepath{stroke}%
\end{pgfscope}%
\begin{pgfscope}%
\pgfpathrectangle{\pgfqpoint{0.515000in}{0.499444in}}{\pgfqpoint{1.550000in}{1.155000in}}%
\pgfusepath{clip}%
\pgfsetbuttcap%
\pgfsetmiterjoin%
\pgfsetlinewidth{1.003750pt}%
\definecolor{currentstroke}{rgb}{0.000000,0.000000,0.000000}%
\pgfsetstrokecolor{currentstroke}%
\pgfsetdash{}{0pt}%
\pgfpathmoveto{\pgfqpoint{1.097195in}{0.499444in}}%
\pgfpathlineto{\pgfqpoint{1.157683in}{0.499444in}}%
\pgfpathlineto{\pgfqpoint{1.157683in}{0.576393in}}%
\pgfpathlineto{\pgfqpoint{1.097195in}{0.576393in}}%
\pgfpathlineto{\pgfqpoint{1.097195in}{0.499444in}}%
\pgfpathclose%
\pgfusepath{stroke}%
\end{pgfscope}%
\begin{pgfscope}%
\pgfpathrectangle{\pgfqpoint{0.515000in}{0.499444in}}{\pgfqpoint{1.550000in}{1.155000in}}%
\pgfusepath{clip}%
\pgfsetbuttcap%
\pgfsetmiterjoin%
\pgfsetlinewidth{1.003750pt}%
\definecolor{currentstroke}{rgb}{0.000000,0.000000,0.000000}%
\pgfsetstrokecolor{currentstroke}%
\pgfsetdash{}{0pt}%
\pgfpathmoveto{\pgfqpoint{1.248415in}{0.499444in}}%
\pgfpathlineto{\pgfqpoint{1.308903in}{0.499444in}}%
\pgfpathlineto{\pgfqpoint{1.308903in}{0.499444in}}%
\pgfpathlineto{\pgfqpoint{1.248415in}{0.499444in}}%
\pgfpathlineto{\pgfqpoint{1.248415in}{0.499444in}}%
\pgfpathclose%
\pgfusepath{stroke}%
\end{pgfscope}%
\begin{pgfscope}%
\pgfpathrectangle{\pgfqpoint{0.515000in}{0.499444in}}{\pgfqpoint{1.550000in}{1.155000in}}%
\pgfusepath{clip}%
\pgfsetbuttcap%
\pgfsetmiterjoin%
\pgfsetlinewidth{1.003750pt}%
\definecolor{currentstroke}{rgb}{0.000000,0.000000,0.000000}%
\pgfsetstrokecolor{currentstroke}%
\pgfsetdash{}{0pt}%
\pgfpathmoveto{\pgfqpoint{1.399634in}{0.499444in}}%
\pgfpathlineto{\pgfqpoint{1.460122in}{0.499444in}}%
\pgfpathlineto{\pgfqpoint{1.460122in}{0.499444in}}%
\pgfpathlineto{\pgfqpoint{1.399634in}{0.499444in}}%
\pgfpathlineto{\pgfqpoint{1.399634in}{0.499444in}}%
\pgfpathclose%
\pgfusepath{stroke}%
\end{pgfscope}%
\begin{pgfscope}%
\pgfpathrectangle{\pgfqpoint{0.515000in}{0.499444in}}{\pgfqpoint{1.550000in}{1.155000in}}%
\pgfusepath{clip}%
\pgfsetbuttcap%
\pgfsetmiterjoin%
\pgfsetlinewidth{1.003750pt}%
\definecolor{currentstroke}{rgb}{0.000000,0.000000,0.000000}%
\pgfsetstrokecolor{currentstroke}%
\pgfsetdash{}{0pt}%
\pgfpathmoveto{\pgfqpoint{1.550854in}{0.499444in}}%
\pgfpathlineto{\pgfqpoint{1.611342in}{0.499444in}}%
\pgfpathlineto{\pgfqpoint{1.611342in}{0.499444in}}%
\pgfpathlineto{\pgfqpoint{1.550854in}{0.499444in}}%
\pgfpathlineto{\pgfqpoint{1.550854in}{0.499444in}}%
\pgfpathclose%
\pgfusepath{stroke}%
\end{pgfscope}%
\begin{pgfscope}%
\pgfpathrectangle{\pgfqpoint{0.515000in}{0.499444in}}{\pgfqpoint{1.550000in}{1.155000in}}%
\pgfusepath{clip}%
\pgfsetbuttcap%
\pgfsetmiterjoin%
\pgfsetlinewidth{1.003750pt}%
\definecolor{currentstroke}{rgb}{0.000000,0.000000,0.000000}%
\pgfsetstrokecolor{currentstroke}%
\pgfsetdash{}{0pt}%
\pgfpathmoveto{\pgfqpoint{1.702073in}{0.499444in}}%
\pgfpathlineto{\pgfqpoint{1.762561in}{0.499444in}}%
\pgfpathlineto{\pgfqpoint{1.762561in}{0.499444in}}%
\pgfpathlineto{\pgfqpoint{1.702073in}{0.499444in}}%
\pgfpathlineto{\pgfqpoint{1.702073in}{0.499444in}}%
\pgfpathclose%
\pgfusepath{stroke}%
\end{pgfscope}%
\begin{pgfscope}%
\pgfpathrectangle{\pgfqpoint{0.515000in}{0.499444in}}{\pgfqpoint{1.550000in}{1.155000in}}%
\pgfusepath{clip}%
\pgfsetbuttcap%
\pgfsetmiterjoin%
\pgfsetlinewidth{1.003750pt}%
\definecolor{currentstroke}{rgb}{0.000000,0.000000,0.000000}%
\pgfsetstrokecolor{currentstroke}%
\pgfsetdash{}{0pt}%
\pgfpathmoveto{\pgfqpoint{1.853293in}{0.499444in}}%
\pgfpathlineto{\pgfqpoint{1.913781in}{0.499444in}}%
\pgfpathlineto{\pgfqpoint{1.913781in}{0.499444in}}%
\pgfpathlineto{\pgfqpoint{1.853293in}{0.499444in}}%
\pgfpathlineto{\pgfqpoint{1.853293in}{0.499444in}}%
\pgfpathclose%
\pgfusepath{stroke}%
\end{pgfscope}%
\begin{pgfscope}%
\pgfpathrectangle{\pgfqpoint{0.515000in}{0.499444in}}{\pgfqpoint{1.550000in}{1.155000in}}%
\pgfusepath{clip}%
\pgfsetbuttcap%
\pgfsetmiterjoin%
\definecolor{currentfill}{rgb}{0.000000,0.000000,0.000000}%
\pgfsetfillcolor{currentfill}%
\pgfsetlinewidth{0.000000pt}%
\definecolor{currentstroke}{rgb}{0.000000,0.000000,0.000000}%
\pgfsetstrokecolor{currentstroke}%
\pgfsetstrokeopacity{0.000000}%
\pgfsetdash{}{0pt}%
\pgfpathmoveto{\pgfqpoint{0.552805in}{0.499444in}}%
\pgfpathlineto{\pgfqpoint{0.613293in}{0.499444in}}%
\pgfpathlineto{\pgfqpoint{0.613293in}{0.511480in}}%
\pgfpathlineto{\pgfqpoint{0.552805in}{0.511480in}}%
\pgfpathlineto{\pgfqpoint{0.552805in}{0.499444in}}%
\pgfpathclose%
\pgfusepath{fill}%
\end{pgfscope}%
\begin{pgfscope}%
\pgfpathrectangle{\pgfqpoint{0.515000in}{0.499444in}}{\pgfqpoint{1.550000in}{1.155000in}}%
\pgfusepath{clip}%
\pgfsetbuttcap%
\pgfsetmiterjoin%
\definecolor{currentfill}{rgb}{0.000000,0.000000,0.000000}%
\pgfsetfillcolor{currentfill}%
\pgfsetlinewidth{0.000000pt}%
\definecolor{currentstroke}{rgb}{0.000000,0.000000,0.000000}%
\pgfsetstrokecolor{currentstroke}%
\pgfsetstrokeopacity{0.000000}%
\pgfsetdash{}{0pt}%
\pgfpathmoveto{\pgfqpoint{0.704025in}{0.499444in}}%
\pgfpathlineto{\pgfqpoint{0.764512in}{0.499444in}}%
\pgfpathlineto{\pgfqpoint{0.764512in}{0.544068in}}%
\pgfpathlineto{\pgfqpoint{0.704025in}{0.544068in}}%
\pgfpathlineto{\pgfqpoint{0.704025in}{0.499444in}}%
\pgfpathclose%
\pgfusepath{fill}%
\end{pgfscope}%
\begin{pgfscope}%
\pgfpathrectangle{\pgfqpoint{0.515000in}{0.499444in}}{\pgfqpoint{1.550000in}{1.155000in}}%
\pgfusepath{clip}%
\pgfsetbuttcap%
\pgfsetmiterjoin%
\definecolor{currentfill}{rgb}{0.000000,0.000000,0.000000}%
\pgfsetfillcolor{currentfill}%
\pgfsetlinewidth{0.000000pt}%
\definecolor{currentstroke}{rgb}{0.000000,0.000000,0.000000}%
\pgfsetstrokecolor{currentstroke}%
\pgfsetstrokeopacity{0.000000}%
\pgfsetdash{}{0pt}%
\pgfpathmoveto{\pgfqpoint{0.855244in}{0.499444in}}%
\pgfpathlineto{\pgfqpoint{0.915732in}{0.499444in}}%
\pgfpathlineto{\pgfqpoint{0.915732in}{0.613518in}}%
\pgfpathlineto{\pgfqpoint{0.855244in}{0.613518in}}%
\pgfpathlineto{\pgfqpoint{0.855244in}{0.499444in}}%
\pgfpathclose%
\pgfusepath{fill}%
\end{pgfscope}%
\begin{pgfscope}%
\pgfpathrectangle{\pgfqpoint{0.515000in}{0.499444in}}{\pgfqpoint{1.550000in}{1.155000in}}%
\pgfusepath{clip}%
\pgfsetbuttcap%
\pgfsetmiterjoin%
\definecolor{currentfill}{rgb}{0.000000,0.000000,0.000000}%
\pgfsetfillcolor{currentfill}%
\pgfsetlinewidth{0.000000pt}%
\definecolor{currentstroke}{rgb}{0.000000,0.000000,0.000000}%
\pgfsetstrokecolor{currentstroke}%
\pgfsetstrokeopacity{0.000000}%
\pgfsetdash{}{0pt}%
\pgfpathmoveto{\pgfqpoint{1.006464in}{0.499444in}}%
\pgfpathlineto{\pgfqpoint{1.066951in}{0.499444in}}%
\pgfpathlineto{\pgfqpoint{1.066951in}{0.690516in}}%
\pgfpathlineto{\pgfqpoint{1.006464in}{0.690516in}}%
\pgfpathlineto{\pgfqpoint{1.006464in}{0.499444in}}%
\pgfpathclose%
\pgfusepath{fill}%
\end{pgfscope}%
\begin{pgfscope}%
\pgfpathrectangle{\pgfqpoint{0.515000in}{0.499444in}}{\pgfqpoint{1.550000in}{1.155000in}}%
\pgfusepath{clip}%
\pgfsetbuttcap%
\pgfsetmiterjoin%
\definecolor{currentfill}{rgb}{0.000000,0.000000,0.000000}%
\pgfsetfillcolor{currentfill}%
\pgfsetlinewidth{0.000000pt}%
\definecolor{currentstroke}{rgb}{0.000000,0.000000,0.000000}%
\pgfsetstrokecolor{currentstroke}%
\pgfsetstrokeopacity{0.000000}%
\pgfsetdash{}{0pt}%
\pgfpathmoveto{\pgfqpoint{1.157683in}{0.499444in}}%
\pgfpathlineto{\pgfqpoint{1.218171in}{0.499444in}}%
\pgfpathlineto{\pgfqpoint{1.218171in}{0.575340in}}%
\pgfpathlineto{\pgfqpoint{1.157683in}{0.575340in}}%
\pgfpathlineto{\pgfqpoint{1.157683in}{0.499444in}}%
\pgfpathclose%
\pgfusepath{fill}%
\end{pgfscope}%
\begin{pgfscope}%
\pgfpathrectangle{\pgfqpoint{0.515000in}{0.499444in}}{\pgfqpoint{1.550000in}{1.155000in}}%
\pgfusepath{clip}%
\pgfsetbuttcap%
\pgfsetmiterjoin%
\definecolor{currentfill}{rgb}{0.000000,0.000000,0.000000}%
\pgfsetfillcolor{currentfill}%
\pgfsetlinewidth{0.000000pt}%
\definecolor{currentstroke}{rgb}{0.000000,0.000000,0.000000}%
\pgfsetstrokecolor{currentstroke}%
\pgfsetstrokeopacity{0.000000}%
\pgfsetdash{}{0pt}%
\pgfpathmoveto{\pgfqpoint{1.308903in}{0.499444in}}%
\pgfpathlineto{\pgfqpoint{1.369391in}{0.499444in}}%
\pgfpathlineto{\pgfqpoint{1.369391in}{0.499444in}}%
\pgfpathlineto{\pgfqpoint{1.308903in}{0.499444in}}%
\pgfpathlineto{\pgfqpoint{1.308903in}{0.499444in}}%
\pgfpathclose%
\pgfusepath{fill}%
\end{pgfscope}%
\begin{pgfscope}%
\pgfpathrectangle{\pgfqpoint{0.515000in}{0.499444in}}{\pgfqpoint{1.550000in}{1.155000in}}%
\pgfusepath{clip}%
\pgfsetbuttcap%
\pgfsetmiterjoin%
\definecolor{currentfill}{rgb}{0.000000,0.000000,0.000000}%
\pgfsetfillcolor{currentfill}%
\pgfsetlinewidth{0.000000pt}%
\definecolor{currentstroke}{rgb}{0.000000,0.000000,0.000000}%
\pgfsetstrokecolor{currentstroke}%
\pgfsetstrokeopacity{0.000000}%
\pgfsetdash{}{0pt}%
\pgfpathmoveto{\pgfqpoint{1.460122in}{0.499444in}}%
\pgfpathlineto{\pgfqpoint{1.520610in}{0.499444in}}%
\pgfpathlineto{\pgfqpoint{1.520610in}{0.499444in}}%
\pgfpathlineto{\pgfqpoint{1.460122in}{0.499444in}}%
\pgfpathlineto{\pgfqpoint{1.460122in}{0.499444in}}%
\pgfpathclose%
\pgfusepath{fill}%
\end{pgfscope}%
\begin{pgfscope}%
\pgfpathrectangle{\pgfqpoint{0.515000in}{0.499444in}}{\pgfqpoint{1.550000in}{1.155000in}}%
\pgfusepath{clip}%
\pgfsetbuttcap%
\pgfsetmiterjoin%
\definecolor{currentfill}{rgb}{0.000000,0.000000,0.000000}%
\pgfsetfillcolor{currentfill}%
\pgfsetlinewidth{0.000000pt}%
\definecolor{currentstroke}{rgb}{0.000000,0.000000,0.000000}%
\pgfsetstrokecolor{currentstroke}%
\pgfsetstrokeopacity{0.000000}%
\pgfsetdash{}{0pt}%
\pgfpathmoveto{\pgfqpoint{1.611342in}{0.499444in}}%
\pgfpathlineto{\pgfqpoint{1.671830in}{0.499444in}}%
\pgfpathlineto{\pgfqpoint{1.671830in}{0.499444in}}%
\pgfpathlineto{\pgfqpoint{1.611342in}{0.499444in}}%
\pgfpathlineto{\pgfqpoint{1.611342in}{0.499444in}}%
\pgfpathclose%
\pgfusepath{fill}%
\end{pgfscope}%
\begin{pgfscope}%
\pgfpathrectangle{\pgfqpoint{0.515000in}{0.499444in}}{\pgfqpoint{1.550000in}{1.155000in}}%
\pgfusepath{clip}%
\pgfsetbuttcap%
\pgfsetmiterjoin%
\definecolor{currentfill}{rgb}{0.000000,0.000000,0.000000}%
\pgfsetfillcolor{currentfill}%
\pgfsetlinewidth{0.000000pt}%
\definecolor{currentstroke}{rgb}{0.000000,0.000000,0.000000}%
\pgfsetstrokecolor{currentstroke}%
\pgfsetstrokeopacity{0.000000}%
\pgfsetdash{}{0pt}%
\pgfpathmoveto{\pgfqpoint{1.762561in}{0.499444in}}%
\pgfpathlineto{\pgfqpoint{1.823049in}{0.499444in}}%
\pgfpathlineto{\pgfqpoint{1.823049in}{0.499444in}}%
\pgfpathlineto{\pgfqpoint{1.762561in}{0.499444in}}%
\pgfpathlineto{\pgfqpoint{1.762561in}{0.499444in}}%
\pgfpathclose%
\pgfusepath{fill}%
\end{pgfscope}%
\begin{pgfscope}%
\pgfpathrectangle{\pgfqpoint{0.515000in}{0.499444in}}{\pgfqpoint{1.550000in}{1.155000in}}%
\pgfusepath{clip}%
\pgfsetbuttcap%
\pgfsetmiterjoin%
\definecolor{currentfill}{rgb}{0.000000,0.000000,0.000000}%
\pgfsetfillcolor{currentfill}%
\pgfsetlinewidth{0.000000pt}%
\definecolor{currentstroke}{rgb}{0.000000,0.000000,0.000000}%
\pgfsetstrokecolor{currentstroke}%
\pgfsetstrokeopacity{0.000000}%
\pgfsetdash{}{0pt}%
\pgfpathmoveto{\pgfqpoint{1.913781in}{0.499444in}}%
\pgfpathlineto{\pgfqpoint{1.974269in}{0.499444in}}%
\pgfpathlineto{\pgfqpoint{1.974269in}{0.499444in}}%
\pgfpathlineto{\pgfqpoint{1.913781in}{0.499444in}}%
\pgfpathlineto{\pgfqpoint{1.913781in}{0.499444in}}%
\pgfpathclose%
\pgfusepath{fill}%
\end{pgfscope}%
\begin{pgfscope}%
\pgfsetbuttcap%
\pgfsetroundjoin%
\definecolor{currentfill}{rgb}{0.000000,0.000000,0.000000}%
\pgfsetfillcolor{currentfill}%
\pgfsetlinewidth{0.803000pt}%
\definecolor{currentstroke}{rgb}{0.000000,0.000000,0.000000}%
\pgfsetstrokecolor{currentstroke}%
\pgfsetdash{}{0pt}%
\pgfsys@defobject{currentmarker}{\pgfqpoint{0.000000in}{-0.048611in}}{\pgfqpoint{0.000000in}{0.000000in}}{%
\pgfpathmoveto{\pgfqpoint{0.000000in}{0.000000in}}%
\pgfpathlineto{\pgfqpoint{0.000000in}{-0.048611in}}%
\pgfusepath{stroke,fill}%
}%
\begin{pgfscope}%
\pgfsys@transformshift{0.552805in}{0.499444in}%
\pgfsys@useobject{currentmarker}{}%
\end{pgfscope}%
\end{pgfscope}%
\begin{pgfscope}%
\definecolor{textcolor}{rgb}{0.000000,0.000000,0.000000}%
\pgfsetstrokecolor{textcolor}%
\pgfsetfillcolor{textcolor}%
\pgftext[x=0.552805in,y=0.402222in,,top]{\color{textcolor}\rmfamily\fontsize{10.000000}{12.000000}\selectfont 0.0}%
\end{pgfscope}%
\begin{pgfscope}%
\pgfsetbuttcap%
\pgfsetroundjoin%
\definecolor{currentfill}{rgb}{0.000000,0.000000,0.000000}%
\pgfsetfillcolor{currentfill}%
\pgfsetlinewidth{0.803000pt}%
\definecolor{currentstroke}{rgb}{0.000000,0.000000,0.000000}%
\pgfsetstrokecolor{currentstroke}%
\pgfsetdash{}{0pt}%
\pgfsys@defobject{currentmarker}{\pgfqpoint{0.000000in}{-0.048611in}}{\pgfqpoint{0.000000in}{0.000000in}}{%
\pgfpathmoveto{\pgfqpoint{0.000000in}{0.000000in}}%
\pgfpathlineto{\pgfqpoint{0.000000in}{-0.048611in}}%
\pgfusepath{stroke,fill}%
}%
\begin{pgfscope}%
\pgfsys@transformshift{0.930854in}{0.499444in}%
\pgfsys@useobject{currentmarker}{}%
\end{pgfscope}%
\end{pgfscope}%
\begin{pgfscope}%
\definecolor{textcolor}{rgb}{0.000000,0.000000,0.000000}%
\pgfsetstrokecolor{textcolor}%
\pgfsetfillcolor{textcolor}%
\pgftext[x=0.930854in,y=0.402222in,,top]{\color{textcolor}\rmfamily\fontsize{10.000000}{12.000000}\selectfont 0.25}%
\end{pgfscope}%
\begin{pgfscope}%
\pgfsetbuttcap%
\pgfsetroundjoin%
\definecolor{currentfill}{rgb}{0.000000,0.000000,0.000000}%
\pgfsetfillcolor{currentfill}%
\pgfsetlinewidth{0.803000pt}%
\definecolor{currentstroke}{rgb}{0.000000,0.000000,0.000000}%
\pgfsetstrokecolor{currentstroke}%
\pgfsetdash{}{0pt}%
\pgfsys@defobject{currentmarker}{\pgfqpoint{0.000000in}{-0.048611in}}{\pgfqpoint{0.000000in}{0.000000in}}{%
\pgfpathmoveto{\pgfqpoint{0.000000in}{0.000000in}}%
\pgfpathlineto{\pgfqpoint{0.000000in}{-0.048611in}}%
\pgfusepath{stroke,fill}%
}%
\begin{pgfscope}%
\pgfsys@transformshift{1.308903in}{0.499444in}%
\pgfsys@useobject{currentmarker}{}%
\end{pgfscope}%
\end{pgfscope}%
\begin{pgfscope}%
\definecolor{textcolor}{rgb}{0.000000,0.000000,0.000000}%
\pgfsetstrokecolor{textcolor}%
\pgfsetfillcolor{textcolor}%
\pgftext[x=1.308903in,y=0.402222in,,top]{\color{textcolor}\rmfamily\fontsize{10.000000}{12.000000}\selectfont 0.5}%
\end{pgfscope}%
\begin{pgfscope}%
\pgfsetbuttcap%
\pgfsetroundjoin%
\definecolor{currentfill}{rgb}{0.000000,0.000000,0.000000}%
\pgfsetfillcolor{currentfill}%
\pgfsetlinewidth{0.803000pt}%
\definecolor{currentstroke}{rgb}{0.000000,0.000000,0.000000}%
\pgfsetstrokecolor{currentstroke}%
\pgfsetdash{}{0pt}%
\pgfsys@defobject{currentmarker}{\pgfqpoint{0.000000in}{-0.048611in}}{\pgfqpoint{0.000000in}{0.000000in}}{%
\pgfpathmoveto{\pgfqpoint{0.000000in}{0.000000in}}%
\pgfpathlineto{\pgfqpoint{0.000000in}{-0.048611in}}%
\pgfusepath{stroke,fill}%
}%
\begin{pgfscope}%
\pgfsys@transformshift{1.686951in}{0.499444in}%
\pgfsys@useobject{currentmarker}{}%
\end{pgfscope}%
\end{pgfscope}%
\begin{pgfscope}%
\definecolor{textcolor}{rgb}{0.000000,0.000000,0.000000}%
\pgfsetstrokecolor{textcolor}%
\pgfsetfillcolor{textcolor}%
\pgftext[x=1.686951in,y=0.402222in,,top]{\color{textcolor}\rmfamily\fontsize{10.000000}{12.000000}\selectfont 0.75}%
\end{pgfscope}%
\begin{pgfscope}%
\pgfsetbuttcap%
\pgfsetroundjoin%
\definecolor{currentfill}{rgb}{0.000000,0.000000,0.000000}%
\pgfsetfillcolor{currentfill}%
\pgfsetlinewidth{0.803000pt}%
\definecolor{currentstroke}{rgb}{0.000000,0.000000,0.000000}%
\pgfsetstrokecolor{currentstroke}%
\pgfsetdash{}{0pt}%
\pgfsys@defobject{currentmarker}{\pgfqpoint{0.000000in}{-0.048611in}}{\pgfqpoint{0.000000in}{0.000000in}}{%
\pgfpathmoveto{\pgfqpoint{0.000000in}{0.000000in}}%
\pgfpathlineto{\pgfqpoint{0.000000in}{-0.048611in}}%
\pgfusepath{stroke,fill}%
}%
\begin{pgfscope}%
\pgfsys@transformshift{2.065000in}{0.499444in}%
\pgfsys@useobject{currentmarker}{}%
\end{pgfscope}%
\end{pgfscope}%
\begin{pgfscope}%
\definecolor{textcolor}{rgb}{0.000000,0.000000,0.000000}%
\pgfsetstrokecolor{textcolor}%
\pgfsetfillcolor{textcolor}%
\pgftext[x=2.065000in,y=0.402222in,,top]{\color{textcolor}\rmfamily\fontsize{10.000000}{12.000000}\selectfont 1.0}%
\end{pgfscope}%
\begin{pgfscope}%
\definecolor{textcolor}{rgb}{0.000000,0.000000,0.000000}%
\pgfsetstrokecolor{textcolor}%
\pgfsetfillcolor{textcolor}%
\pgftext[x=1.290000in,y=0.223333in,,top]{\color{textcolor}\rmfamily\fontsize{10.000000}{12.000000}\selectfont \(\displaystyle p\)}%
\end{pgfscope}%
\begin{pgfscope}%
\pgfsetbuttcap%
\pgfsetroundjoin%
\definecolor{currentfill}{rgb}{0.000000,0.000000,0.000000}%
\pgfsetfillcolor{currentfill}%
\pgfsetlinewidth{0.803000pt}%
\definecolor{currentstroke}{rgb}{0.000000,0.000000,0.000000}%
\pgfsetstrokecolor{currentstroke}%
\pgfsetdash{}{0pt}%
\pgfsys@defobject{currentmarker}{\pgfqpoint{-0.048611in}{0.000000in}}{\pgfqpoint{-0.000000in}{0.000000in}}{%
\pgfpathmoveto{\pgfqpoint{-0.000000in}{0.000000in}}%
\pgfpathlineto{\pgfqpoint{-0.048611in}{0.000000in}}%
\pgfusepath{stroke,fill}%
}%
\begin{pgfscope}%
\pgfsys@transformshift{0.515000in}{0.499444in}%
\pgfsys@useobject{currentmarker}{}%
\end{pgfscope}%
\end{pgfscope}%
\begin{pgfscope}%
\definecolor{textcolor}{rgb}{0.000000,0.000000,0.000000}%
\pgfsetstrokecolor{textcolor}%
\pgfsetfillcolor{textcolor}%
\pgftext[x=0.348333in, y=0.451250in, left, base]{\color{textcolor}\rmfamily\fontsize{10.000000}{12.000000}\selectfont \(\displaystyle {0}\)}%
\end{pgfscope}%
\begin{pgfscope}%
\pgfsetbuttcap%
\pgfsetroundjoin%
\definecolor{currentfill}{rgb}{0.000000,0.000000,0.000000}%
\pgfsetfillcolor{currentfill}%
\pgfsetlinewidth{0.803000pt}%
\definecolor{currentstroke}{rgb}{0.000000,0.000000,0.000000}%
\pgfsetstrokecolor{currentstroke}%
\pgfsetdash{}{0pt}%
\pgfsys@defobject{currentmarker}{\pgfqpoint{-0.048611in}{0.000000in}}{\pgfqpoint{-0.000000in}{0.000000in}}{%
\pgfpathmoveto{\pgfqpoint{-0.000000in}{0.000000in}}%
\pgfpathlineto{\pgfqpoint{-0.048611in}{0.000000in}}%
\pgfusepath{stroke,fill}%
}%
\begin{pgfscope}%
\pgfsys@transformshift{0.515000in}{0.791111in}%
\pgfsys@useobject{currentmarker}{}%
\end{pgfscope}%
\end{pgfscope}%
\begin{pgfscope}%
\definecolor{textcolor}{rgb}{0.000000,0.000000,0.000000}%
\pgfsetstrokecolor{textcolor}%
\pgfsetfillcolor{textcolor}%
\pgftext[x=0.278889in, y=0.742917in, left, base]{\color{textcolor}\rmfamily\fontsize{10.000000}{12.000000}\selectfont \(\displaystyle {10}\)}%
\end{pgfscope}%
\begin{pgfscope}%
\pgfsetbuttcap%
\pgfsetroundjoin%
\definecolor{currentfill}{rgb}{0.000000,0.000000,0.000000}%
\pgfsetfillcolor{currentfill}%
\pgfsetlinewidth{0.803000pt}%
\definecolor{currentstroke}{rgb}{0.000000,0.000000,0.000000}%
\pgfsetstrokecolor{currentstroke}%
\pgfsetdash{}{0pt}%
\pgfsys@defobject{currentmarker}{\pgfqpoint{-0.048611in}{0.000000in}}{\pgfqpoint{-0.000000in}{0.000000in}}{%
\pgfpathmoveto{\pgfqpoint{-0.000000in}{0.000000in}}%
\pgfpathlineto{\pgfqpoint{-0.048611in}{0.000000in}}%
\pgfusepath{stroke,fill}%
}%
\begin{pgfscope}%
\pgfsys@transformshift{0.515000in}{1.082779in}%
\pgfsys@useobject{currentmarker}{}%
\end{pgfscope}%
\end{pgfscope}%
\begin{pgfscope}%
\definecolor{textcolor}{rgb}{0.000000,0.000000,0.000000}%
\pgfsetstrokecolor{textcolor}%
\pgfsetfillcolor{textcolor}%
\pgftext[x=0.278889in, y=1.034584in, left, base]{\color{textcolor}\rmfamily\fontsize{10.000000}{12.000000}\selectfont \(\displaystyle {20}\)}%
\end{pgfscope}%
\begin{pgfscope}%
\pgfsetbuttcap%
\pgfsetroundjoin%
\definecolor{currentfill}{rgb}{0.000000,0.000000,0.000000}%
\pgfsetfillcolor{currentfill}%
\pgfsetlinewidth{0.803000pt}%
\definecolor{currentstroke}{rgb}{0.000000,0.000000,0.000000}%
\pgfsetstrokecolor{currentstroke}%
\pgfsetdash{}{0pt}%
\pgfsys@defobject{currentmarker}{\pgfqpoint{-0.048611in}{0.000000in}}{\pgfqpoint{-0.000000in}{0.000000in}}{%
\pgfpathmoveto{\pgfqpoint{-0.000000in}{0.000000in}}%
\pgfpathlineto{\pgfqpoint{-0.048611in}{0.000000in}}%
\pgfusepath{stroke,fill}%
}%
\begin{pgfscope}%
\pgfsys@transformshift{0.515000in}{1.374446in}%
\pgfsys@useobject{currentmarker}{}%
\end{pgfscope}%
\end{pgfscope}%
\begin{pgfscope}%
\definecolor{textcolor}{rgb}{0.000000,0.000000,0.000000}%
\pgfsetstrokecolor{textcolor}%
\pgfsetfillcolor{textcolor}%
\pgftext[x=0.278889in, y=1.326251in, left, base]{\color{textcolor}\rmfamily\fontsize{10.000000}{12.000000}\selectfont \(\displaystyle {30}\)}%
\end{pgfscope}%
\begin{pgfscope}%
\definecolor{textcolor}{rgb}{0.000000,0.000000,0.000000}%
\pgfsetstrokecolor{textcolor}%
\pgfsetfillcolor{textcolor}%
\pgftext[x=0.223333in,y=1.076944in,,bottom,rotate=90.000000]{\color{textcolor}\rmfamily\fontsize{10.000000}{12.000000}\selectfont Percent of Data Set}%
\end{pgfscope}%
\begin{pgfscope}%
\pgfsetrectcap%
\pgfsetmiterjoin%
\pgfsetlinewidth{0.803000pt}%
\definecolor{currentstroke}{rgb}{0.000000,0.000000,0.000000}%
\pgfsetstrokecolor{currentstroke}%
\pgfsetdash{}{0pt}%
\pgfpathmoveto{\pgfqpoint{0.515000in}{0.499444in}}%
\pgfpathlineto{\pgfqpoint{0.515000in}{1.654444in}}%
\pgfusepath{stroke}%
\end{pgfscope}%
\begin{pgfscope}%
\pgfsetrectcap%
\pgfsetmiterjoin%
\pgfsetlinewidth{0.803000pt}%
\definecolor{currentstroke}{rgb}{0.000000,0.000000,0.000000}%
\pgfsetstrokecolor{currentstroke}%
\pgfsetdash{}{0pt}%
\pgfpathmoveto{\pgfqpoint{2.065000in}{0.499444in}}%
\pgfpathlineto{\pgfqpoint{2.065000in}{1.654444in}}%
\pgfusepath{stroke}%
\end{pgfscope}%
\begin{pgfscope}%
\pgfsetrectcap%
\pgfsetmiterjoin%
\pgfsetlinewidth{0.803000pt}%
\definecolor{currentstroke}{rgb}{0.000000,0.000000,0.000000}%
\pgfsetstrokecolor{currentstroke}%
\pgfsetdash{}{0pt}%
\pgfpathmoveto{\pgfqpoint{0.515000in}{0.499444in}}%
\pgfpathlineto{\pgfqpoint{2.065000in}{0.499444in}}%
\pgfusepath{stroke}%
\end{pgfscope}%
\begin{pgfscope}%
\pgfsetrectcap%
\pgfsetmiterjoin%
\pgfsetlinewidth{0.803000pt}%
\definecolor{currentstroke}{rgb}{0.000000,0.000000,0.000000}%
\pgfsetstrokecolor{currentstroke}%
\pgfsetdash{}{0pt}%
\pgfpathmoveto{\pgfqpoint{0.515000in}{1.654444in}}%
\pgfpathlineto{\pgfqpoint{2.065000in}{1.654444in}}%
\pgfusepath{stroke}%
\end{pgfscope}%
\begin{pgfscope}%
\pgfsetbuttcap%
\pgfsetmiterjoin%
\definecolor{currentfill}{rgb}{1.000000,1.000000,1.000000}%
\pgfsetfillcolor{currentfill}%
\pgfsetfillopacity{0.800000}%
\pgfsetlinewidth{1.003750pt}%
\definecolor{currentstroke}{rgb}{0.800000,0.800000,0.800000}%
\pgfsetstrokecolor{currentstroke}%
\pgfsetstrokeopacity{0.800000}%
\pgfsetdash{}{0pt}%
\pgfpathmoveto{\pgfqpoint{1.288056in}{1.154445in}}%
\pgfpathlineto{\pgfqpoint{1.967778in}{1.154445in}}%
\pgfpathquadraticcurveto{\pgfqpoint{1.995556in}{1.154445in}}{\pgfqpoint{1.995556in}{1.182222in}}%
\pgfpathlineto{\pgfqpoint{1.995556in}{1.557222in}}%
\pgfpathquadraticcurveto{\pgfqpoint{1.995556in}{1.585000in}}{\pgfqpoint{1.967778in}{1.585000in}}%
\pgfpathlineto{\pgfqpoint{1.288056in}{1.585000in}}%
\pgfpathquadraticcurveto{\pgfqpoint{1.260278in}{1.585000in}}{\pgfqpoint{1.260278in}{1.557222in}}%
\pgfpathlineto{\pgfqpoint{1.260278in}{1.182222in}}%
\pgfpathquadraticcurveto{\pgfqpoint{1.260278in}{1.154445in}}{\pgfqpoint{1.288056in}{1.154445in}}%
\pgfpathlineto{\pgfqpoint{1.288056in}{1.154445in}}%
\pgfpathclose%
\pgfusepath{stroke,fill}%
\end{pgfscope}%
\begin{pgfscope}%
\pgfsetbuttcap%
\pgfsetmiterjoin%
\pgfsetlinewidth{1.003750pt}%
\definecolor{currentstroke}{rgb}{0.000000,0.000000,0.000000}%
\pgfsetstrokecolor{currentstroke}%
\pgfsetdash{}{0pt}%
\pgfpathmoveto{\pgfqpoint{1.315834in}{1.432222in}}%
\pgfpathlineto{\pgfqpoint{1.593611in}{1.432222in}}%
\pgfpathlineto{\pgfqpoint{1.593611in}{1.529444in}}%
\pgfpathlineto{\pgfqpoint{1.315834in}{1.529444in}}%
\pgfpathlineto{\pgfqpoint{1.315834in}{1.432222in}}%
\pgfpathclose%
\pgfusepath{stroke}%
\end{pgfscope}%
\begin{pgfscope}%
\definecolor{textcolor}{rgb}{0.000000,0.000000,0.000000}%
\pgfsetstrokecolor{textcolor}%
\pgfsetfillcolor{textcolor}%
\pgftext[x=1.704722in,y=1.432222in,left,base]{\color{textcolor}\rmfamily\fontsize{10.000000}{12.000000}\selectfont Neg}%
\end{pgfscope}%
\begin{pgfscope}%
\pgfsetbuttcap%
\pgfsetmiterjoin%
\definecolor{currentfill}{rgb}{0.000000,0.000000,0.000000}%
\pgfsetfillcolor{currentfill}%
\pgfsetlinewidth{0.000000pt}%
\definecolor{currentstroke}{rgb}{0.000000,0.000000,0.000000}%
\pgfsetstrokecolor{currentstroke}%
\pgfsetstrokeopacity{0.000000}%
\pgfsetdash{}{0pt}%
\pgfpathmoveto{\pgfqpoint{1.315834in}{1.236944in}}%
\pgfpathlineto{\pgfqpoint{1.593611in}{1.236944in}}%
\pgfpathlineto{\pgfqpoint{1.593611in}{1.334167in}}%
\pgfpathlineto{\pgfqpoint{1.315834in}{1.334167in}}%
\pgfpathlineto{\pgfqpoint{1.315834in}{1.236944in}}%
\pgfpathclose%
\pgfusepath{fill}%
\end{pgfscope}%
\begin{pgfscope}%
\definecolor{textcolor}{rgb}{0.000000,0.000000,0.000000}%
\pgfsetstrokecolor{textcolor}%
\pgfsetfillcolor{textcolor}%
\pgftext[x=1.704722in,y=1.236944in,left,base]{\color{textcolor}\rmfamily\fontsize{10.000000}{12.000000}\selectfont Pos}%
\end{pgfscope}%
\end{pgfpicture}%
\makeatother%
\endgroup%

%  &
%  \vspace{0pt} %% Creator: Matplotlib, PGF backend
%%
%% To include the figure in your LaTeX document, write
%%   \input{<filename>.pgf}
%%
%% Make sure the required packages are loaded in your preamble
%%   \usepackage{pgf}
%%
%% Also ensure that all the required font packages are loaded; for instance,
%% the lmodern package is sometimes necessary when using math font.
%%   \usepackage{lmodern}
%%
%% Figures using additional raster images can only be included by \input if
%% they are in the same directory as the main LaTeX file. For loading figures
%% from other directories you can use the `import` package
%%   \usepackage{import}
%%
%% and then include the figures with
%%   \import{<path to file>}{<filename>.pgf}
%%
%% Matplotlib used the following preamble
%%   
%%   \usepackage{fontspec}
%%   \makeatletter\@ifpackageloaded{underscore}{}{\usepackage[strings]{underscore}}\makeatother
%%
\begingroup%
\makeatletter%
\begin{pgfpicture}%
\pgfpathrectangle{\pgfpointorigin}{\pgfqpoint{2.221861in}{1.754444in}}%
\pgfusepath{use as bounding box, clip}%
\begin{pgfscope}%
\pgfsetbuttcap%
\pgfsetmiterjoin%
\definecolor{currentfill}{rgb}{1.000000,1.000000,1.000000}%
\pgfsetfillcolor{currentfill}%
\pgfsetlinewidth{0.000000pt}%
\definecolor{currentstroke}{rgb}{1.000000,1.000000,1.000000}%
\pgfsetstrokecolor{currentstroke}%
\pgfsetdash{}{0pt}%
\pgfpathmoveto{\pgfqpoint{0.000000in}{0.000000in}}%
\pgfpathlineto{\pgfqpoint{2.221861in}{0.000000in}}%
\pgfpathlineto{\pgfqpoint{2.221861in}{1.754444in}}%
\pgfpathlineto{\pgfqpoint{0.000000in}{1.754444in}}%
\pgfpathlineto{\pgfqpoint{0.000000in}{0.000000in}}%
\pgfpathclose%
\pgfusepath{fill}%
\end{pgfscope}%
\begin{pgfscope}%
\pgfsetbuttcap%
\pgfsetmiterjoin%
\definecolor{currentfill}{rgb}{1.000000,1.000000,1.000000}%
\pgfsetfillcolor{currentfill}%
\pgfsetlinewidth{0.000000pt}%
\definecolor{currentstroke}{rgb}{0.000000,0.000000,0.000000}%
\pgfsetstrokecolor{currentstroke}%
\pgfsetstrokeopacity{0.000000}%
\pgfsetdash{}{0pt}%
\pgfpathmoveto{\pgfqpoint{0.553581in}{0.499444in}}%
\pgfpathlineto{\pgfqpoint{2.103581in}{0.499444in}}%
\pgfpathlineto{\pgfqpoint{2.103581in}{1.654444in}}%
\pgfpathlineto{\pgfqpoint{0.553581in}{1.654444in}}%
\pgfpathlineto{\pgfqpoint{0.553581in}{0.499444in}}%
\pgfpathclose%
\pgfusepath{fill}%
\end{pgfscope}%
\begin{pgfscope}%
\pgfsetbuttcap%
\pgfsetroundjoin%
\definecolor{currentfill}{rgb}{0.000000,0.000000,0.000000}%
\pgfsetfillcolor{currentfill}%
\pgfsetlinewidth{0.803000pt}%
\definecolor{currentstroke}{rgb}{0.000000,0.000000,0.000000}%
\pgfsetstrokecolor{currentstroke}%
\pgfsetdash{}{0pt}%
\pgfsys@defobject{currentmarker}{\pgfqpoint{0.000000in}{-0.048611in}}{\pgfqpoint{0.000000in}{0.000000in}}{%
\pgfpathmoveto{\pgfqpoint{0.000000in}{0.000000in}}%
\pgfpathlineto{\pgfqpoint{0.000000in}{-0.048611in}}%
\pgfusepath{stroke,fill}%
}%
\begin{pgfscope}%
\pgfsys@transformshift{0.624035in}{0.499444in}%
\pgfsys@useobject{currentmarker}{}%
\end{pgfscope}%
\end{pgfscope}%
\begin{pgfscope}%
\definecolor{textcolor}{rgb}{0.000000,0.000000,0.000000}%
\pgfsetstrokecolor{textcolor}%
\pgfsetfillcolor{textcolor}%
\pgftext[x=0.624035in,y=0.402222in,,top]{\color{textcolor}\rmfamily\fontsize{10.000000}{12.000000}\selectfont \(\displaystyle {0.0}\)}%
\end{pgfscope}%
\begin{pgfscope}%
\pgfsetbuttcap%
\pgfsetroundjoin%
\definecolor{currentfill}{rgb}{0.000000,0.000000,0.000000}%
\pgfsetfillcolor{currentfill}%
\pgfsetlinewidth{0.803000pt}%
\definecolor{currentstroke}{rgb}{0.000000,0.000000,0.000000}%
\pgfsetstrokecolor{currentstroke}%
\pgfsetdash{}{0pt}%
\pgfsys@defobject{currentmarker}{\pgfqpoint{0.000000in}{-0.048611in}}{\pgfqpoint{0.000000in}{0.000000in}}{%
\pgfpathmoveto{\pgfqpoint{0.000000in}{0.000000in}}%
\pgfpathlineto{\pgfqpoint{0.000000in}{-0.048611in}}%
\pgfusepath{stroke,fill}%
}%
\begin{pgfscope}%
\pgfsys@transformshift{1.328581in}{0.499444in}%
\pgfsys@useobject{currentmarker}{}%
\end{pgfscope}%
\end{pgfscope}%
\begin{pgfscope}%
\definecolor{textcolor}{rgb}{0.000000,0.000000,0.000000}%
\pgfsetstrokecolor{textcolor}%
\pgfsetfillcolor{textcolor}%
\pgftext[x=1.328581in,y=0.402222in,,top]{\color{textcolor}\rmfamily\fontsize{10.000000}{12.000000}\selectfont \(\displaystyle {0.5}\)}%
\end{pgfscope}%
\begin{pgfscope}%
\pgfsetbuttcap%
\pgfsetroundjoin%
\definecolor{currentfill}{rgb}{0.000000,0.000000,0.000000}%
\pgfsetfillcolor{currentfill}%
\pgfsetlinewidth{0.803000pt}%
\definecolor{currentstroke}{rgb}{0.000000,0.000000,0.000000}%
\pgfsetstrokecolor{currentstroke}%
\pgfsetdash{}{0pt}%
\pgfsys@defobject{currentmarker}{\pgfqpoint{0.000000in}{-0.048611in}}{\pgfqpoint{0.000000in}{0.000000in}}{%
\pgfpathmoveto{\pgfqpoint{0.000000in}{0.000000in}}%
\pgfpathlineto{\pgfqpoint{0.000000in}{-0.048611in}}%
\pgfusepath{stroke,fill}%
}%
\begin{pgfscope}%
\pgfsys@transformshift{2.033126in}{0.499444in}%
\pgfsys@useobject{currentmarker}{}%
\end{pgfscope}%
\end{pgfscope}%
\begin{pgfscope}%
\definecolor{textcolor}{rgb}{0.000000,0.000000,0.000000}%
\pgfsetstrokecolor{textcolor}%
\pgfsetfillcolor{textcolor}%
\pgftext[x=2.033126in,y=0.402222in,,top]{\color{textcolor}\rmfamily\fontsize{10.000000}{12.000000}\selectfont \(\displaystyle {1.0}\)}%
\end{pgfscope}%
\begin{pgfscope}%
\definecolor{textcolor}{rgb}{0.000000,0.000000,0.000000}%
\pgfsetstrokecolor{textcolor}%
\pgfsetfillcolor{textcolor}%
\pgftext[x=1.328581in,y=0.223333in,,top]{\color{textcolor}\rmfamily\fontsize{10.000000}{12.000000}\selectfont False positive rate}%
\end{pgfscope}%
\begin{pgfscope}%
\pgfsetbuttcap%
\pgfsetroundjoin%
\definecolor{currentfill}{rgb}{0.000000,0.000000,0.000000}%
\pgfsetfillcolor{currentfill}%
\pgfsetlinewidth{0.803000pt}%
\definecolor{currentstroke}{rgb}{0.000000,0.000000,0.000000}%
\pgfsetstrokecolor{currentstroke}%
\pgfsetdash{}{0pt}%
\pgfsys@defobject{currentmarker}{\pgfqpoint{-0.048611in}{0.000000in}}{\pgfqpoint{-0.000000in}{0.000000in}}{%
\pgfpathmoveto{\pgfqpoint{-0.000000in}{0.000000in}}%
\pgfpathlineto{\pgfqpoint{-0.048611in}{0.000000in}}%
\pgfusepath{stroke,fill}%
}%
\begin{pgfscope}%
\pgfsys@transformshift{0.553581in}{0.551944in}%
\pgfsys@useobject{currentmarker}{}%
\end{pgfscope}%
\end{pgfscope}%
\begin{pgfscope}%
\definecolor{textcolor}{rgb}{0.000000,0.000000,0.000000}%
\pgfsetstrokecolor{textcolor}%
\pgfsetfillcolor{textcolor}%
\pgftext[x=0.278889in, y=0.503750in, left, base]{\color{textcolor}\rmfamily\fontsize{10.000000}{12.000000}\selectfont \(\displaystyle {0.0}\)}%
\end{pgfscope}%
\begin{pgfscope}%
\pgfsetbuttcap%
\pgfsetroundjoin%
\definecolor{currentfill}{rgb}{0.000000,0.000000,0.000000}%
\pgfsetfillcolor{currentfill}%
\pgfsetlinewidth{0.803000pt}%
\definecolor{currentstroke}{rgb}{0.000000,0.000000,0.000000}%
\pgfsetstrokecolor{currentstroke}%
\pgfsetdash{}{0pt}%
\pgfsys@defobject{currentmarker}{\pgfqpoint{-0.048611in}{0.000000in}}{\pgfqpoint{-0.000000in}{0.000000in}}{%
\pgfpathmoveto{\pgfqpoint{-0.000000in}{0.000000in}}%
\pgfpathlineto{\pgfqpoint{-0.048611in}{0.000000in}}%
\pgfusepath{stroke,fill}%
}%
\begin{pgfscope}%
\pgfsys@transformshift{0.553581in}{1.076944in}%
\pgfsys@useobject{currentmarker}{}%
\end{pgfscope}%
\end{pgfscope}%
\begin{pgfscope}%
\definecolor{textcolor}{rgb}{0.000000,0.000000,0.000000}%
\pgfsetstrokecolor{textcolor}%
\pgfsetfillcolor{textcolor}%
\pgftext[x=0.278889in, y=1.028750in, left, base]{\color{textcolor}\rmfamily\fontsize{10.000000}{12.000000}\selectfont \(\displaystyle {0.5}\)}%
\end{pgfscope}%
\begin{pgfscope}%
\pgfsetbuttcap%
\pgfsetroundjoin%
\definecolor{currentfill}{rgb}{0.000000,0.000000,0.000000}%
\pgfsetfillcolor{currentfill}%
\pgfsetlinewidth{0.803000pt}%
\definecolor{currentstroke}{rgb}{0.000000,0.000000,0.000000}%
\pgfsetstrokecolor{currentstroke}%
\pgfsetdash{}{0pt}%
\pgfsys@defobject{currentmarker}{\pgfqpoint{-0.048611in}{0.000000in}}{\pgfqpoint{-0.000000in}{0.000000in}}{%
\pgfpathmoveto{\pgfqpoint{-0.000000in}{0.000000in}}%
\pgfpathlineto{\pgfqpoint{-0.048611in}{0.000000in}}%
\pgfusepath{stroke,fill}%
}%
\begin{pgfscope}%
\pgfsys@transformshift{0.553581in}{1.601944in}%
\pgfsys@useobject{currentmarker}{}%
\end{pgfscope}%
\end{pgfscope}%
\begin{pgfscope}%
\definecolor{textcolor}{rgb}{0.000000,0.000000,0.000000}%
\pgfsetstrokecolor{textcolor}%
\pgfsetfillcolor{textcolor}%
\pgftext[x=0.278889in, y=1.553750in, left, base]{\color{textcolor}\rmfamily\fontsize{10.000000}{12.000000}\selectfont \(\displaystyle {1.0}\)}%
\end{pgfscope}%
\begin{pgfscope}%
\definecolor{textcolor}{rgb}{0.000000,0.000000,0.000000}%
\pgfsetstrokecolor{textcolor}%
\pgfsetfillcolor{textcolor}%
\pgftext[x=0.223333in,y=1.076944in,,bottom,rotate=90.000000]{\color{textcolor}\rmfamily\fontsize{10.000000}{12.000000}\selectfont True positive rate}%
\end{pgfscope}%
\begin{pgfscope}%
\pgfpathrectangle{\pgfqpoint{0.553581in}{0.499444in}}{\pgfqpoint{1.550000in}{1.155000in}}%
\pgfusepath{clip}%
\pgfsetbuttcap%
\pgfsetroundjoin%
\pgfsetlinewidth{1.505625pt}%
\definecolor{currentstroke}{rgb}{0.000000,0.000000,0.000000}%
\pgfsetstrokecolor{currentstroke}%
\pgfsetdash{{5.550000pt}{2.400000pt}}{0.000000pt}%
\pgfpathmoveto{\pgfqpoint{0.624035in}{0.551944in}}%
\pgfpathlineto{\pgfqpoint{2.033126in}{1.601944in}}%
\pgfusepath{stroke}%
\end{pgfscope}%
\begin{pgfscope}%
\pgfpathrectangle{\pgfqpoint{0.553581in}{0.499444in}}{\pgfqpoint{1.550000in}{1.155000in}}%
\pgfusepath{clip}%
\pgfsetrectcap%
\pgfsetroundjoin%
\pgfsetlinewidth{1.505625pt}%
\definecolor{currentstroke}{rgb}{0.000000,0.000000,0.000000}%
\pgfsetstrokecolor{currentstroke}%
\pgfsetdash{}{0pt}%
\pgfpathmoveto{\pgfqpoint{0.624035in}{0.551944in}}%
\pgfpathlineto{\pgfqpoint{0.627493in}{0.553009in}}%
\pgfpathlineto{\pgfqpoint{0.628559in}{0.554390in}}%
\pgfpathlineto{\pgfqpoint{0.629063in}{0.555494in}}%
\pgfpathlineto{\pgfqpoint{0.630166in}{0.558452in}}%
\pgfpathlineto{\pgfqpoint{0.630531in}{0.559557in}}%
\pgfpathlineto{\pgfqpoint{0.631624in}{0.564132in}}%
\pgfpathlineto{\pgfqpoint{0.631970in}{0.565197in}}%
\pgfpathlineto{\pgfqpoint{0.633073in}{0.569338in}}%
\pgfpathlineto{\pgfqpoint{0.633306in}{0.570443in}}%
\pgfpathlineto{\pgfqpoint{0.634409in}{0.576044in}}%
\pgfpathlineto{\pgfqpoint{0.634755in}{0.577148in}}%
\pgfpathlineto{\pgfqpoint{0.635858in}{0.582236in}}%
\pgfpathlineto{\pgfqpoint{0.636035in}{0.583104in}}%
\pgfpathlineto{\pgfqpoint{0.637138in}{0.589967in}}%
\pgfpathlineto{\pgfqpoint{0.637344in}{0.591032in}}%
\pgfpathlineto{\pgfqpoint{0.638447in}{0.596790in}}%
\pgfpathlineto{\pgfqpoint{0.638559in}{0.597895in}}%
\pgfpathlineto{\pgfqpoint{0.639652in}{0.604758in}}%
\pgfpathlineto{\pgfqpoint{0.639877in}{0.605823in}}%
\pgfpathlineto{\pgfqpoint{0.640979in}{0.612449in}}%
\pgfpathlineto{\pgfqpoint{0.641138in}{0.613553in}}%
\pgfpathlineto{\pgfqpoint{0.642241in}{0.621363in}}%
\pgfpathlineto{\pgfqpoint{0.642456in}{0.622467in}}%
\pgfpathlineto{\pgfqpoint{0.643549in}{0.631026in}}%
\pgfpathlineto{\pgfqpoint{0.643802in}{0.631973in}}%
\pgfpathlineto{\pgfqpoint{0.644895in}{0.639428in}}%
\pgfpathlineto{\pgfqpoint{0.645073in}{0.640532in}}%
\pgfpathlineto{\pgfqpoint{0.646138in}{0.649131in}}%
\pgfpathlineto{\pgfqpoint{0.656213in}{0.650235in}}%
\pgfpathlineto{\pgfqpoint{0.657316in}{0.658636in}}%
\pgfpathlineto{\pgfqpoint{0.657475in}{0.659662in}}%
\pgfpathlineto{\pgfqpoint{0.658559in}{0.668812in}}%
\pgfpathlineto{\pgfqpoint{0.658746in}{0.669877in}}%
\pgfpathlineto{\pgfqpoint{0.659849in}{0.678989in}}%
\pgfpathlineto{\pgfqpoint{0.659942in}{0.680014in}}%
\pgfpathlineto{\pgfqpoint{0.661045in}{0.688415in}}%
\pgfpathlineto{\pgfqpoint{0.661176in}{0.689520in}}%
\pgfpathlineto{\pgfqpoint{0.662279in}{0.697527in}}%
\pgfpathlineto{\pgfqpoint{0.662475in}{0.698631in}}%
\pgfpathlineto{\pgfqpoint{0.663578in}{0.705691in}}%
\pgfpathlineto{\pgfqpoint{0.663821in}{0.706717in}}%
\pgfpathlineto{\pgfqpoint{0.664924in}{0.715394in}}%
\pgfpathlineto{\pgfqpoint{0.665129in}{0.716459in}}%
\pgfpathlineto{\pgfqpoint{0.666223in}{0.724269in}}%
\pgfpathlineto{\pgfqpoint{0.666372in}{0.725334in}}%
\pgfpathlineto{\pgfqpoint{0.667475in}{0.732473in}}%
\pgfpathlineto{\pgfqpoint{0.667643in}{0.733538in}}%
\pgfpathlineto{\pgfqpoint{0.668737in}{0.741189in}}%
\pgfpathlineto{\pgfqpoint{0.668942in}{0.742254in}}%
\pgfpathlineto{\pgfqpoint{0.670026in}{0.751326in}}%
\pgfpathlineto{\pgfqpoint{0.670213in}{0.752431in}}%
\pgfpathlineto{\pgfqpoint{0.671316in}{0.759964in}}%
\pgfpathlineto{\pgfqpoint{0.671606in}{0.761029in}}%
\pgfpathlineto{\pgfqpoint{0.672709in}{0.766788in}}%
\pgfpathlineto{\pgfqpoint{0.672961in}{0.767892in}}%
\pgfpathlineto{\pgfqpoint{0.674064in}{0.775978in}}%
\pgfpathlineto{\pgfqpoint{0.674241in}{0.777043in}}%
\pgfpathlineto{\pgfqpoint{0.675307in}{0.783235in}}%
\pgfpathlineto{\pgfqpoint{0.675634in}{0.784340in}}%
\pgfpathlineto{\pgfqpoint{0.676737in}{0.791124in}}%
\pgfpathlineto{\pgfqpoint{0.676877in}{0.791991in}}%
\pgfpathlineto{\pgfqpoint{0.677980in}{0.799643in}}%
\pgfpathlineto{\pgfqpoint{0.678101in}{0.800669in}}%
\pgfpathlineto{\pgfqpoint{0.679195in}{0.806861in}}%
\pgfpathlineto{\pgfqpoint{0.679466in}{0.807966in}}%
\pgfpathlineto{\pgfqpoint{0.680569in}{0.813685in}}%
\pgfpathlineto{\pgfqpoint{0.680746in}{0.814632in}}%
\pgfpathlineto{\pgfqpoint{0.681840in}{0.821652in}}%
\pgfpathlineto{\pgfqpoint{0.682017in}{0.822717in}}%
\pgfpathlineto{\pgfqpoint{0.683111in}{0.829383in}}%
\pgfpathlineto{\pgfqpoint{0.683316in}{0.830448in}}%
\pgfpathlineto{\pgfqpoint{0.684410in}{0.836522in}}%
\pgfpathlineto{\pgfqpoint{0.684578in}{0.837469in}}%
\pgfpathlineto{\pgfqpoint{0.685681in}{0.843306in}}%
\pgfpathlineto{\pgfqpoint{0.685914in}{0.844411in}}%
\pgfpathlineto{\pgfqpoint{0.686999in}{0.851116in}}%
\pgfpathlineto{\pgfqpoint{0.687195in}{0.851786in}}%
\pgfpathlineto{\pgfqpoint{0.688298in}{0.859951in}}%
\pgfpathlineto{\pgfqpoint{0.688466in}{0.860937in}}%
\pgfpathlineto{\pgfqpoint{0.689569in}{0.867603in}}%
\pgfpathlineto{\pgfqpoint{0.689765in}{0.868668in}}%
\pgfpathlineto{\pgfqpoint{0.690858in}{0.874505in}}%
\pgfpathlineto{\pgfqpoint{0.691073in}{0.875610in}}%
\pgfpathlineto{\pgfqpoint{0.692176in}{0.881684in}}%
\pgfpathlineto{\pgfqpoint{0.692410in}{0.882709in}}%
\pgfpathlineto{\pgfqpoint{0.693513in}{0.889296in}}%
\pgfpathlineto{\pgfqpoint{0.693709in}{0.890401in}}%
\pgfpathlineto{\pgfqpoint{0.694812in}{0.895804in}}%
\pgfpathlineto{\pgfqpoint{0.695176in}{0.896909in}}%
\pgfpathlineto{\pgfqpoint{0.696279in}{0.902549in}}%
\pgfpathlineto{\pgfqpoint{0.696569in}{0.903653in}}%
\pgfpathlineto{\pgfqpoint{0.697672in}{0.909846in}}%
\pgfpathlineto{\pgfqpoint{0.697915in}{0.910911in}}%
\pgfpathlineto{\pgfqpoint{0.698989in}{0.917458in}}%
\pgfpathlineto{\pgfqpoint{0.699279in}{0.918523in}}%
\pgfpathlineto{\pgfqpoint{0.700354in}{0.923454in}}%
\pgfpathlineto{\pgfqpoint{0.700569in}{0.924479in}}%
\pgfpathlineto{\pgfqpoint{0.701672in}{0.929685in}}%
\pgfpathlineto{\pgfqpoint{0.702027in}{0.930790in}}%
\pgfpathlineto{\pgfqpoint{0.703130in}{0.937180in}}%
\pgfpathlineto{\pgfqpoint{0.703335in}{0.938284in}}%
\pgfpathlineto{\pgfqpoint{0.704429in}{0.943609in}}%
\pgfpathlineto{\pgfqpoint{0.704709in}{0.944634in}}%
\pgfpathlineto{\pgfqpoint{0.705765in}{0.949446in}}%
\pgfpathlineto{\pgfqpoint{0.706214in}{0.950472in}}%
\pgfpathlineto{\pgfqpoint{0.707307in}{0.955165in}}%
\pgfpathlineto{\pgfqpoint{0.707634in}{0.956270in}}%
\pgfpathlineto{\pgfqpoint{0.708718in}{0.961200in}}%
\pgfpathlineto{\pgfqpoint{0.708989in}{0.962265in}}%
\pgfpathlineto{\pgfqpoint{0.710083in}{0.966604in}}%
\pgfpathlineto{\pgfqpoint{0.710242in}{0.967708in}}%
\pgfpathlineto{\pgfqpoint{0.711335in}{0.972165in}}%
\pgfpathlineto{\pgfqpoint{0.711532in}{0.973269in}}%
\pgfpathlineto{\pgfqpoint{0.712625in}{0.978042in}}%
\pgfpathlineto{\pgfqpoint{0.712896in}{0.979107in}}%
\pgfpathlineto{\pgfqpoint{0.713980in}{0.983801in}}%
\pgfpathlineto{\pgfqpoint{0.714251in}{0.984905in}}%
\pgfpathlineto{\pgfqpoint{0.715354in}{0.990348in}}%
\pgfpathlineto{\pgfqpoint{0.715597in}{0.991452in}}%
\pgfpathlineto{\pgfqpoint{0.716681in}{0.996067in}}%
\pgfpathlineto{\pgfqpoint{0.717074in}{0.997172in}}%
\pgfpathlineto{\pgfqpoint{0.718177in}{1.001786in}}%
\pgfpathlineto{\pgfqpoint{0.718476in}{1.002851in}}%
\pgfpathlineto{\pgfqpoint{0.719578in}{1.007506in}}%
\pgfpathlineto{\pgfqpoint{0.719905in}{1.008531in}}%
\pgfpathlineto{\pgfqpoint{0.720999in}{1.012988in}}%
\pgfpathlineto{\pgfqpoint{0.721401in}{1.014092in}}%
\pgfpathlineto{\pgfqpoint{0.722504in}{1.017840in}}%
\pgfpathlineto{\pgfqpoint{0.722765in}{1.018904in}}%
\pgfpathlineto{\pgfqpoint{0.723859in}{1.022691in}}%
\pgfpathlineto{\pgfqpoint{0.724111in}{1.023756in}}%
\pgfpathlineto{\pgfqpoint{0.725205in}{1.026990in}}%
\pgfpathlineto{\pgfqpoint{0.725606in}{1.028055in}}%
\pgfpathlineto{\pgfqpoint{0.726709in}{1.031526in}}%
\pgfpathlineto{\pgfqpoint{0.727074in}{1.032630in}}%
\pgfpathlineto{\pgfqpoint{0.728177in}{1.036023in}}%
\pgfpathlineto{\pgfqpoint{0.728392in}{1.037009in}}%
\pgfpathlineto{\pgfqpoint{0.729485in}{1.041702in}}%
\pgfpathlineto{\pgfqpoint{0.729775in}{1.042767in}}%
\pgfpathlineto{\pgfqpoint{0.730878in}{1.046435in}}%
\pgfpathlineto{\pgfqpoint{0.731214in}{1.047540in}}%
\pgfpathlineto{\pgfqpoint{0.732308in}{1.050853in}}%
\pgfpathlineto{\pgfqpoint{0.732775in}{1.051957in}}%
\pgfpathlineto{\pgfqpoint{0.733878in}{1.055152in}}%
\pgfpathlineto{\pgfqpoint{0.734093in}{1.056178in}}%
\pgfpathlineto{\pgfqpoint{0.735177in}{1.060043in}}%
\pgfpathlineto{\pgfqpoint{0.735551in}{1.061108in}}%
\pgfpathlineto{\pgfqpoint{0.736653in}{1.064618in}}%
\pgfpathlineto{\pgfqpoint{0.737046in}{1.065683in}}%
\pgfpathlineto{\pgfqpoint{0.738130in}{1.068681in}}%
\pgfpathlineto{\pgfqpoint{0.738551in}{1.069785in}}%
\pgfpathlineto{\pgfqpoint{0.739644in}{1.073887in}}%
\pgfpathlineto{\pgfqpoint{0.740027in}{1.074992in}}%
\pgfpathlineto{\pgfqpoint{0.741130in}{1.077871in}}%
\pgfpathlineto{\pgfqpoint{0.741551in}{1.078975in}}%
\pgfpathlineto{\pgfqpoint{0.742644in}{1.081973in}}%
\pgfpathlineto{\pgfqpoint{0.743233in}{1.083077in}}%
\pgfpathlineto{\pgfqpoint{0.744336in}{1.086036in}}%
\pgfpathlineto{\pgfqpoint{0.744934in}{1.087061in}}%
\pgfpathlineto{\pgfqpoint{0.746037in}{1.090769in}}%
\pgfpathlineto{\pgfqpoint{0.746448in}{1.091873in}}%
\pgfpathlineto{\pgfqpoint{0.747523in}{1.095778in}}%
\pgfpathlineto{\pgfqpoint{0.747887in}{1.096843in}}%
\pgfpathlineto{\pgfqpoint{0.748934in}{1.100551in}}%
\pgfpathlineto{\pgfqpoint{0.749280in}{1.101655in}}%
\pgfpathlineto{\pgfqpoint{0.750364in}{1.104574in}}%
\pgfpathlineto{\pgfqpoint{0.750756in}{1.105678in}}%
\pgfpathlineto{\pgfqpoint{0.751840in}{1.108360in}}%
\pgfpathlineto{\pgfqpoint{0.752298in}{1.109465in}}%
\pgfpathlineto{\pgfqpoint{0.753373in}{1.112107in}}%
\pgfpathlineto{\pgfqpoint{0.753999in}{1.113172in}}%
\pgfpathlineto{\pgfqpoint{0.755083in}{1.115696in}}%
\pgfpathlineto{\pgfqpoint{0.755569in}{1.116801in}}%
\pgfpathlineto{\pgfqpoint{0.756672in}{1.119838in}}%
\pgfpathlineto{\pgfqpoint{0.756981in}{1.120706in}}%
\pgfpathlineto{\pgfqpoint{0.758065in}{1.124532in}}%
\pgfpathlineto{\pgfqpoint{0.758551in}{1.125636in}}%
\pgfpathlineto{\pgfqpoint{0.759607in}{1.128949in}}%
\pgfpathlineto{\pgfqpoint{0.760168in}{1.130054in}}%
\pgfpathlineto{\pgfqpoint{0.761270in}{1.132854in}}%
\pgfpathlineto{\pgfqpoint{0.761999in}{1.133958in}}%
\pgfpathlineto{\pgfqpoint{0.763046in}{1.136680in}}%
\pgfpathlineto{\pgfqpoint{0.763822in}{1.137784in}}%
\pgfpathlineto{\pgfqpoint{0.764925in}{1.139875in}}%
\pgfpathlineto{\pgfqpoint{0.765355in}{1.140979in}}%
\pgfpathlineto{\pgfqpoint{0.766448in}{1.143622in}}%
\pgfpathlineto{\pgfqpoint{0.767065in}{1.144726in}}%
\pgfpathlineto{\pgfqpoint{0.768121in}{1.146896in}}%
\pgfpathlineto{\pgfqpoint{0.768672in}{1.148000in}}%
\pgfpathlineto{\pgfqpoint{0.769710in}{1.150485in}}%
\pgfpathlineto{\pgfqpoint{0.770327in}{1.151589in}}%
\pgfpathlineto{\pgfqpoint{0.771429in}{1.154587in}}%
\pgfpathlineto{\pgfqpoint{0.772000in}{1.155691in}}%
\pgfpathlineto{\pgfqpoint{0.773084in}{1.157506in}}%
\pgfpathlineto{\pgfqpoint{0.773813in}{1.158571in}}%
\pgfpathlineto{\pgfqpoint{0.774906in}{1.161016in}}%
\pgfpathlineto{\pgfqpoint{0.775476in}{1.162120in}}%
\pgfpathlineto{\pgfqpoint{0.776570in}{1.164250in}}%
\pgfpathlineto{\pgfqpoint{0.777140in}{1.165355in}}%
\pgfpathlineto{\pgfqpoint{0.778177in}{1.167642in}}%
\pgfpathlineto{\pgfqpoint{0.778850in}{1.168747in}}%
\pgfpathlineto{\pgfqpoint{0.779953in}{1.170600in}}%
\pgfpathlineto{\pgfqpoint{0.780392in}{1.171508in}}%
\pgfpathlineto{\pgfqpoint{0.781476in}{1.173874in}}%
\pgfpathlineto{\pgfqpoint{0.781953in}{1.174979in}}%
\pgfpathlineto{\pgfqpoint{0.783056in}{1.176793in}}%
\pgfpathlineto{\pgfqpoint{0.783635in}{1.177897in}}%
\pgfpathlineto{\pgfqpoint{0.784729in}{1.179396in}}%
\pgfpathlineto{\pgfqpoint{0.785374in}{1.180501in}}%
\pgfpathlineto{\pgfqpoint{0.786439in}{1.182512in}}%
\pgfpathlineto{\pgfqpoint{0.787121in}{1.183617in}}%
\pgfpathlineto{\pgfqpoint{0.788224in}{1.185589in}}%
\pgfpathlineto{\pgfqpoint{0.788934in}{1.186693in}}%
\pgfpathlineto{\pgfqpoint{0.789878in}{1.187916in}}%
\pgfpathlineto{\pgfqpoint{0.790607in}{1.188981in}}%
\pgfpathlineto{\pgfqpoint{0.791617in}{1.191189in}}%
\pgfpathlineto{\pgfqpoint{0.792271in}{1.192294in}}%
\pgfpathlineto{\pgfqpoint{0.793364in}{1.193990in}}%
\pgfpathlineto{\pgfqpoint{0.793916in}{1.195055in}}%
\pgfpathlineto{\pgfqpoint{0.795009in}{1.196790in}}%
\pgfpathlineto{\pgfqpoint{0.795551in}{1.197855in}}%
\pgfpathlineto{\pgfqpoint{0.796626in}{1.199709in}}%
\pgfpathlineto{\pgfqpoint{0.797346in}{1.200813in}}%
\pgfpathlineto{\pgfqpoint{0.798439in}{1.203101in}}%
\pgfpathlineto{\pgfqpoint{0.799196in}{1.204206in}}%
\pgfpathlineto{\pgfqpoint{0.800299in}{1.206257in}}%
\pgfpathlineto{\pgfqpoint{0.800720in}{1.207361in}}%
\pgfpathlineto{\pgfqpoint{0.801785in}{1.209294in}}%
\pgfpathlineto{\pgfqpoint{0.802467in}{1.210398in}}%
\pgfpathlineto{\pgfqpoint{0.803561in}{1.212843in}}%
\pgfpathlineto{\pgfqpoint{0.804168in}{1.213948in}}%
\pgfpathlineto{\pgfqpoint{0.805252in}{1.215723in}}%
\pgfpathlineto{\pgfqpoint{0.806103in}{1.216827in}}%
\pgfpathlineto{\pgfqpoint{0.807206in}{1.219312in}}%
\pgfpathlineto{\pgfqpoint{0.807841in}{1.220416in}}%
\pgfpathlineto{\pgfqpoint{0.808897in}{1.222073in}}%
\pgfpathlineto{\pgfqpoint{0.809421in}{1.223177in}}%
\pgfpathlineto{\pgfqpoint{0.810514in}{1.224755in}}%
\pgfpathlineto{\pgfqpoint{0.811327in}{1.225859in}}%
\pgfpathlineto{\pgfqpoint{0.812421in}{1.228029in}}%
\pgfpathlineto{\pgfqpoint{0.812888in}{1.229133in}}%
\pgfpathlineto{\pgfqpoint{0.813907in}{1.230593in}}%
\pgfpathlineto{\pgfqpoint{0.814776in}{1.231697in}}%
\pgfpathlineto{\pgfqpoint{0.815879in}{1.233354in}}%
\pgfpathlineto{\pgfqpoint{0.816505in}{1.234340in}}%
\pgfpathlineto{\pgfqpoint{0.817598in}{1.236272in}}%
\pgfpathlineto{\pgfqpoint{0.817981in}{1.237377in}}%
\pgfpathlineto{\pgfqpoint{0.819075in}{1.239191in}}%
\pgfpathlineto{\pgfqpoint{0.819748in}{1.240295in}}%
\pgfpathlineto{\pgfqpoint{0.820832in}{1.241400in}}%
\pgfpathlineto{\pgfqpoint{0.821654in}{1.242465in}}%
\pgfpathlineto{\pgfqpoint{0.822692in}{1.243806in}}%
\pgfpathlineto{\pgfqpoint{0.823486in}{1.244831in}}%
\pgfpathlineto{\pgfqpoint{0.824552in}{1.246527in}}%
\pgfpathlineto{\pgfqpoint{0.825477in}{1.247632in}}%
\pgfpathlineto{\pgfqpoint{0.826580in}{1.249880in}}%
\pgfpathlineto{\pgfqpoint{0.827103in}{1.250984in}}%
\pgfpathlineto{\pgfqpoint{0.828196in}{1.252523in}}%
\pgfpathlineto{\pgfqpoint{0.829243in}{1.253627in}}%
\pgfpathlineto{\pgfqpoint{0.830299in}{1.255402in}}%
\pgfpathlineto{\pgfqpoint{0.831253in}{1.256467in}}%
\pgfpathlineto{\pgfqpoint{0.832346in}{1.258045in}}%
\pgfpathlineto{\pgfqpoint{0.833533in}{1.259149in}}%
\pgfpathlineto{\pgfqpoint{0.834608in}{1.260490in}}%
\pgfpathlineto{\pgfqpoint{0.835542in}{1.261594in}}%
\pgfpathlineto{\pgfqpoint{0.836598in}{1.262738in}}%
\pgfpathlineto{\pgfqpoint{0.837785in}{1.263843in}}%
\pgfpathlineto{\pgfqpoint{0.838879in}{1.265815in}}%
\pgfpathlineto{\pgfqpoint{0.839683in}{1.266919in}}%
\pgfpathlineto{\pgfqpoint{0.840739in}{1.268260in}}%
\pgfpathlineto{\pgfqpoint{0.841337in}{1.269365in}}%
\pgfpathlineto{\pgfqpoint{0.842412in}{1.270863in}}%
\pgfpathlineto{\pgfqpoint{0.843458in}{1.271968in}}%
\pgfpathlineto{\pgfqpoint{0.844542in}{1.273112in}}%
\pgfpathlineto{\pgfqpoint{0.845860in}{1.274216in}}%
\pgfpathlineto{\pgfqpoint{0.846935in}{1.275439in}}%
\pgfpathlineto{\pgfqpoint{0.847748in}{1.276504in}}%
\pgfpathlineto{\pgfqpoint{0.848842in}{1.278318in}}%
\pgfpathlineto{\pgfqpoint{0.849636in}{1.279383in}}%
\pgfpathlineto{\pgfqpoint{0.850739in}{1.280448in}}%
\pgfpathlineto{\pgfqpoint{0.851935in}{1.281513in}}%
\pgfpathlineto{\pgfqpoint{0.853038in}{1.283091in}}%
\pgfpathlineto{\pgfqpoint{0.854225in}{1.284195in}}%
\pgfpathlineto{\pgfqpoint{0.855225in}{1.285418in}}%
\pgfpathlineto{\pgfqpoint{0.856094in}{1.286522in}}%
\pgfpathlineto{\pgfqpoint{0.857038in}{1.287982in}}%
\pgfpathlineto{\pgfqpoint{0.857860in}{1.289086in}}%
\pgfpathlineto{\pgfqpoint{0.858944in}{1.290506in}}%
\pgfpathlineto{\pgfqpoint{0.859954in}{1.291610in}}%
\pgfpathlineto{\pgfqpoint{0.861057in}{1.293109in}}%
\pgfpathlineto{\pgfqpoint{0.861786in}{1.294174in}}%
\pgfpathlineto{\pgfqpoint{0.862888in}{1.295515in}}%
\pgfpathlineto{\pgfqpoint{0.863524in}{1.296619in}}%
\pgfpathlineto{\pgfqpoint{0.864608in}{1.297605in}}%
\pgfpathlineto{\pgfqpoint{0.865748in}{1.298710in}}%
\pgfpathlineto{\pgfqpoint{0.866842in}{1.300011in}}%
\pgfpathlineto{\pgfqpoint{0.868122in}{1.301116in}}%
\pgfpathlineto{\pgfqpoint{0.869206in}{1.302417in}}%
\pgfpathlineto{\pgfqpoint{0.869973in}{1.303522in}}%
\pgfpathlineto{\pgfqpoint{0.871038in}{1.304508in}}%
\pgfpathlineto{\pgfqpoint{0.872786in}{1.305573in}}%
\pgfpathlineto{\pgfqpoint{0.873823in}{1.306520in}}%
\pgfpathlineto{\pgfqpoint{0.874870in}{1.307545in}}%
\pgfpathlineto{\pgfqpoint{0.875954in}{1.309004in}}%
\pgfpathlineto{\pgfqpoint{0.876935in}{1.310109in}}%
\pgfpathlineto{\pgfqpoint{0.877973in}{1.311292in}}%
\pgfpathlineto{\pgfqpoint{0.878973in}{1.312357in}}%
\pgfpathlineto{\pgfqpoint{0.880076in}{1.313580in}}%
\pgfpathlineto{\pgfqpoint{0.881216in}{1.314684in}}%
\pgfpathlineto{\pgfqpoint{0.882291in}{1.315867in}}%
\pgfpathlineto{\pgfqpoint{0.883534in}{1.316972in}}%
\pgfpathlineto{\pgfqpoint{0.884618in}{1.318037in}}%
\pgfpathlineto{\pgfqpoint{0.885870in}{1.319141in}}%
\pgfpathlineto{\pgfqpoint{0.886973in}{1.320679in}}%
\pgfpathlineto{\pgfqpoint{0.887907in}{1.321744in}}%
\pgfpathlineto{\pgfqpoint{0.889001in}{1.323125in}}%
\pgfpathlineto{\pgfqpoint{0.890870in}{1.324229in}}%
\pgfpathlineto{\pgfqpoint{0.891926in}{1.325294in}}%
\pgfpathlineto{\pgfqpoint{0.893216in}{1.326399in}}%
\pgfpathlineto{\pgfqpoint{0.894319in}{1.327306in}}%
\pgfpathlineto{\pgfqpoint{0.895823in}{1.328410in}}%
\pgfpathlineto{\pgfqpoint{0.896908in}{1.329357in}}%
\pgfpathlineto{\pgfqpoint{0.897795in}{1.330461in}}%
\pgfpathlineto{\pgfqpoint{0.898898in}{1.331368in}}%
\pgfpathlineto{\pgfqpoint{0.900590in}{1.332473in}}%
\pgfpathlineto{\pgfqpoint{0.901683in}{1.333735in}}%
\pgfpathlineto{\pgfqpoint{0.902674in}{1.334839in}}%
\pgfpathlineto{\pgfqpoint{0.903758in}{1.336023in}}%
\pgfpathlineto{\pgfqpoint{0.905085in}{1.337127in}}%
\pgfpathlineto{\pgfqpoint{0.906141in}{1.338310in}}%
\pgfpathlineto{\pgfqpoint{0.907440in}{1.339415in}}%
\pgfpathlineto{\pgfqpoint{0.908253in}{1.340046in}}%
\pgfpathlineto{\pgfqpoint{0.910067in}{1.341150in}}%
\pgfpathlineto{\pgfqpoint{0.911132in}{1.342254in}}%
\pgfpathlineto{\pgfqpoint{0.912282in}{1.343359in}}%
\pgfpathlineto{\pgfqpoint{0.913338in}{1.344187in}}%
\pgfpathlineto{\pgfqpoint{0.914908in}{1.345292in}}%
\pgfpathlineto{\pgfqpoint{0.916001in}{1.346396in}}%
\pgfpathlineto{\pgfqpoint{0.917169in}{1.347500in}}%
\pgfpathlineto{\pgfqpoint{0.918226in}{1.348486in}}%
\pgfpathlineto{\pgfqpoint{0.919627in}{1.349591in}}%
\pgfpathlineto{\pgfqpoint{0.920721in}{1.350537in}}%
\pgfpathlineto{\pgfqpoint{0.922104in}{1.351642in}}%
\pgfpathlineto{\pgfqpoint{0.923020in}{1.352983in}}%
\pgfpathlineto{\pgfqpoint{0.924515in}{1.354087in}}%
\pgfpathlineto{\pgfqpoint{0.925618in}{1.355073in}}%
\pgfpathlineto{\pgfqpoint{0.926618in}{1.356059in}}%
\pgfpathlineto{\pgfqpoint{0.927684in}{1.357045in}}%
\pgfpathlineto{\pgfqpoint{0.929188in}{1.358150in}}%
\pgfpathlineto{\pgfqpoint{0.930160in}{1.359215in}}%
\pgfpathlineto{\pgfqpoint{0.932039in}{1.360319in}}%
\pgfpathlineto{\pgfqpoint{0.933067in}{1.361147in}}%
\pgfpathlineto{\pgfqpoint{0.934871in}{1.362252in}}%
\pgfpathlineto{\pgfqpoint{0.935964in}{1.363238in}}%
\pgfpathlineto{\pgfqpoint{0.937544in}{1.364342in}}%
\pgfpathlineto{\pgfqpoint{0.938618in}{1.365289in}}%
\pgfpathlineto{\pgfqpoint{0.939843in}{1.366393in}}%
\pgfpathlineto{\pgfqpoint{0.940871in}{1.367300in}}%
\pgfpathlineto{\pgfqpoint{0.942534in}{1.368405in}}%
\pgfpathlineto{\pgfqpoint{0.943618in}{1.369194in}}%
\pgfpathlineto{\pgfqpoint{0.945656in}{1.370298in}}%
\pgfpathlineto{\pgfqpoint{0.946702in}{1.371087in}}%
\pgfpathlineto{\pgfqpoint{0.948039in}{1.372191in}}%
\pgfpathlineto{\pgfqpoint{0.949076in}{1.372822in}}%
\pgfpathlineto{\pgfqpoint{0.950908in}{1.373927in}}%
\pgfpathlineto{\pgfqpoint{0.952011in}{1.374716in}}%
\pgfpathlineto{\pgfqpoint{0.953665in}{1.375820in}}%
\pgfpathlineto{\pgfqpoint{0.954768in}{1.376727in}}%
\pgfpathlineto{\pgfqpoint{0.956544in}{1.377753in}}%
\pgfpathlineto{\pgfqpoint{0.957600in}{1.378660in}}%
\pgfpathlineto{\pgfqpoint{0.959749in}{1.379764in}}%
\pgfpathlineto{\pgfqpoint{0.960843in}{1.381026in}}%
\pgfpathlineto{\pgfqpoint{0.962749in}{1.382131in}}%
\pgfpathlineto{\pgfqpoint{0.963805in}{1.382841in}}%
\pgfpathlineto{\pgfqpoint{0.965516in}{1.383945in}}%
\pgfpathlineto{\pgfqpoint{0.966478in}{1.384695in}}%
\pgfpathlineto{\pgfqpoint{0.968376in}{1.385799in}}%
\pgfpathlineto{\pgfqpoint{0.969432in}{1.386588in}}%
\pgfpathlineto{\pgfqpoint{0.971806in}{1.387692in}}%
\pgfpathlineto{\pgfqpoint{0.972834in}{1.388402in}}%
\pgfpathlineto{\pgfqpoint{0.974637in}{1.389507in}}%
\pgfpathlineto{\pgfqpoint{0.975656in}{1.390217in}}%
\pgfpathlineto{\pgfqpoint{0.978049in}{1.391321in}}%
\pgfpathlineto{\pgfqpoint{0.978918in}{1.391755in}}%
\pgfpathlineto{\pgfqpoint{0.981011in}{1.392859in}}%
\pgfpathlineto{\pgfqpoint{0.981955in}{1.393451in}}%
\pgfpathlineto{\pgfqpoint{0.984105in}{1.394555in}}%
\pgfpathlineto{\pgfqpoint{0.985208in}{1.395226in}}%
\pgfpathlineto{\pgfqpoint{0.987245in}{1.396330in}}%
\pgfpathlineto{\pgfqpoint{0.988348in}{1.397080in}}%
\pgfpathlineto{\pgfqpoint{0.989853in}{1.398184in}}%
\pgfpathlineto{\pgfqpoint{0.990871in}{1.398815in}}%
\pgfpathlineto{\pgfqpoint{0.993329in}{1.399919in}}%
\pgfpathlineto{\pgfqpoint{0.994404in}{1.400551in}}%
\pgfpathlineto{\pgfqpoint{0.996797in}{1.401655in}}%
\pgfpathlineto{\pgfqpoint{0.997834in}{1.402207in}}%
\pgfpathlineto{\pgfqpoint{0.999955in}{1.403312in}}%
\pgfpathlineto{\pgfqpoint{1.000890in}{1.404021in}}%
\pgfpathlineto{\pgfqpoint{1.003096in}{1.405126in}}%
\pgfpathlineto{\pgfqpoint{1.004152in}{1.405718in}}%
\pgfpathlineto{\pgfqpoint{1.006600in}{1.406822in}}%
\pgfpathlineto{\pgfqpoint{1.007554in}{1.407374in}}%
\pgfpathlineto{\pgfqpoint{1.009133in}{1.408478in}}%
\pgfpathlineto{\pgfqpoint{1.010096in}{1.409188in}}%
\pgfpathlineto{\pgfqpoint{1.012217in}{1.410293in}}%
\pgfpathlineto{\pgfqpoint{1.013311in}{1.411003in}}%
\pgfpathlineto{\pgfqpoint{1.015395in}{1.412107in}}%
\pgfpathlineto{\pgfqpoint{1.016404in}{1.412620in}}%
\pgfpathlineto{\pgfqpoint{1.018853in}{1.413724in}}%
\pgfpathlineto{\pgfqpoint{1.019956in}{1.414316in}}%
\pgfpathlineto{\pgfqpoint{1.022283in}{1.415420in}}%
\pgfpathlineto{\pgfqpoint{1.023367in}{1.416170in}}%
\pgfpathlineto{\pgfqpoint{1.025788in}{1.417274in}}%
\pgfpathlineto{\pgfqpoint{1.026573in}{1.417866in}}%
\pgfpathlineto{\pgfqpoint{1.029180in}{1.418931in}}%
\pgfpathlineto{\pgfqpoint{1.030189in}{1.419246in}}%
\pgfpathlineto{\pgfqpoint{1.032545in}{1.420351in}}%
\pgfpathlineto{\pgfqpoint{1.033647in}{1.420824in}}%
\pgfpathlineto{\pgfqpoint{1.036461in}{1.421928in}}%
\pgfpathlineto{\pgfqpoint{1.037535in}{1.422441in}}%
\pgfpathlineto{\pgfqpoint{1.039984in}{1.423546in}}%
\pgfpathlineto{\pgfqpoint{1.040975in}{1.424255in}}%
\pgfpathlineto{\pgfqpoint{1.043928in}{1.425360in}}%
\pgfpathlineto{\pgfqpoint{1.045021in}{1.425991in}}%
\pgfpathlineto{\pgfqpoint{1.046984in}{1.427095in}}%
\pgfpathlineto{\pgfqpoint{1.047928in}{1.427529in}}%
\pgfpathlineto{\pgfqpoint{1.050685in}{1.428634in}}%
\pgfpathlineto{\pgfqpoint{1.051704in}{1.429344in}}%
\pgfpathlineto{\pgfqpoint{1.054685in}{1.430448in}}%
\pgfpathlineto{\pgfqpoint{1.055722in}{1.430961in}}%
\pgfpathlineto{\pgfqpoint{1.058012in}{1.432065in}}%
\pgfpathlineto{\pgfqpoint{1.059068in}{1.432420in}}%
\pgfpathlineto{\pgfqpoint{1.061367in}{1.433524in}}%
\pgfpathlineto{\pgfqpoint{1.062321in}{1.433998in}}%
\pgfpathlineto{\pgfqpoint{1.064975in}{1.435102in}}%
\pgfpathlineto{\pgfqpoint{1.065863in}{1.435260in}}%
\pgfpathlineto{\pgfqpoint{1.069237in}{1.436364in}}%
\pgfpathlineto{\pgfqpoint{1.070237in}{1.436838in}}%
\pgfpathlineto{\pgfqpoint{1.072984in}{1.437942in}}%
\pgfpathlineto{\pgfqpoint{1.073956in}{1.438336in}}%
\pgfpathlineto{\pgfqpoint{1.076517in}{1.439441in}}%
\pgfpathlineto{\pgfqpoint{1.077442in}{1.439993in}}%
\pgfpathlineto{\pgfqpoint{1.080797in}{1.441097in}}%
\pgfpathlineto{\pgfqpoint{1.081751in}{1.441531in}}%
\pgfpathlineto{\pgfqpoint{1.084583in}{1.442636in}}%
\pgfpathlineto{\pgfqpoint{1.085611in}{1.443188in}}%
\pgfpathlineto{\pgfqpoint{1.088284in}{1.444292in}}%
\pgfpathlineto{\pgfqpoint{1.089237in}{1.444687in}}%
\pgfpathlineto{\pgfqpoint{1.092676in}{1.445752in}}%
\pgfpathlineto{\pgfqpoint{1.093685in}{1.446422in}}%
\pgfpathlineto{\pgfqpoint{1.097031in}{1.447487in}}%
\pgfpathlineto{\pgfqpoint{1.098115in}{1.448079in}}%
\pgfpathlineto{\pgfqpoint{1.101471in}{1.449183in}}%
\pgfpathlineto{\pgfqpoint{1.102433in}{1.449735in}}%
\pgfpathlineto{\pgfqpoint{1.105723in}{1.450840in}}%
\pgfpathlineto{\pgfqpoint{1.106788in}{1.451195in}}%
\pgfpathlineto{\pgfqpoint{1.110293in}{1.452299in}}%
\pgfpathlineto{\pgfqpoint{1.111377in}{1.453009in}}%
\pgfpathlineto{\pgfqpoint{1.113891in}{1.454114in}}%
\pgfpathlineto{\pgfqpoint{1.114994in}{1.454587in}}%
\pgfpathlineto{\pgfqpoint{1.118265in}{1.455691in}}%
\pgfpathlineto{\pgfqpoint{1.119134in}{1.456283in}}%
\pgfpathlineto{\pgfqpoint{1.123125in}{1.457387in}}%
\pgfpathlineto{\pgfqpoint{1.124172in}{1.457742in}}%
\pgfpathlineto{\pgfqpoint{1.127284in}{1.458847in}}%
\pgfpathlineto{\pgfqpoint{1.128228in}{1.459123in}}%
\pgfpathlineto{\pgfqpoint{1.131602in}{1.460227in}}%
\pgfpathlineto{\pgfqpoint{1.132705in}{1.460819in}}%
\pgfpathlineto{\pgfqpoint{1.136443in}{1.461923in}}%
\pgfpathlineto{\pgfqpoint{1.137527in}{1.462436in}}%
\pgfpathlineto{\pgfqpoint{1.140340in}{1.463540in}}%
\pgfpathlineto{\pgfqpoint{1.141200in}{1.463895in}}%
\pgfpathlineto{\pgfqpoint{1.144434in}{1.465000in}}%
\pgfpathlineto{\pgfqpoint{1.145527in}{1.465473in}}%
\pgfpathlineto{\pgfqpoint{1.150163in}{1.466577in}}%
\pgfpathlineto{\pgfqpoint{1.151256in}{1.466932in}}%
\pgfpathlineto{\pgfqpoint{1.156312in}{1.468037in}}%
\pgfpathlineto{\pgfqpoint{1.157387in}{1.468549in}}%
\pgfpathlineto{\pgfqpoint{1.160967in}{1.469614in}}%
\pgfpathlineto{\pgfqpoint{1.162032in}{1.470246in}}%
\pgfpathlineto{\pgfqpoint{1.165013in}{1.471350in}}%
\pgfpathlineto{\pgfqpoint{1.166098in}{1.471744in}}%
\pgfpathlineto{\pgfqpoint{1.169621in}{1.472849in}}%
\pgfpathlineto{\pgfqpoint{1.170668in}{1.473283in}}%
\pgfpathlineto{\pgfqpoint{1.174985in}{1.474387in}}%
\pgfpathlineto{\pgfqpoint{1.175957in}{1.474703in}}%
\pgfpathlineto{\pgfqpoint{1.179593in}{1.475807in}}%
\pgfpathlineto{\pgfqpoint{1.180696in}{1.476122in}}%
\pgfpathlineto{\pgfqpoint{1.184098in}{1.477227in}}%
\pgfpathlineto{\pgfqpoint{1.185116in}{1.477582in}}%
\pgfpathlineto{\pgfqpoint{1.190388in}{1.478686in}}%
\pgfpathlineto{\pgfqpoint{1.191425in}{1.479120in}}%
\pgfpathlineto{\pgfqpoint{1.195453in}{1.480224in}}%
\pgfpathlineto{\pgfqpoint{1.196528in}{1.480501in}}%
\pgfpathlineto{\pgfqpoint{1.200247in}{1.481605in}}%
\pgfpathlineto{\pgfqpoint{1.201285in}{1.482039in}}%
\pgfpathlineto{\pgfqpoint{1.205911in}{1.483104in}}%
\pgfpathlineto{\pgfqpoint{1.206855in}{1.483577in}}%
\pgfpathlineto{\pgfqpoint{1.212892in}{1.484681in}}%
\pgfpathlineto{\pgfqpoint{1.213921in}{1.484958in}}%
\pgfpathlineto{\pgfqpoint{1.217874in}{1.486062in}}%
\pgfpathlineto{\pgfqpoint{1.218967in}{1.486496in}}%
\pgfpathlineto{\pgfqpoint{1.222556in}{1.487600in}}%
\pgfpathlineto{\pgfqpoint{1.223594in}{1.487837in}}%
\pgfpathlineto{\pgfqpoint{1.227566in}{1.488941in}}%
\pgfpathlineto{\pgfqpoint{1.228631in}{1.489296in}}%
\pgfpathlineto{\pgfqpoint{1.231846in}{1.490401in}}%
\pgfpathlineto{\pgfqpoint{1.232911in}{1.490598in}}%
\pgfpathlineto{\pgfqpoint{1.240323in}{1.491702in}}%
\pgfpathlineto{\pgfqpoint{1.241407in}{1.492136in}}%
\pgfpathlineto{\pgfqpoint{1.246164in}{1.493241in}}%
\pgfpathlineto{\pgfqpoint{1.246968in}{1.493477in}}%
\pgfpathlineto{\pgfqpoint{1.251827in}{1.494582in}}%
\pgfpathlineto{\pgfqpoint{1.252827in}{1.494976in}}%
\pgfpathlineto{\pgfqpoint{1.259089in}{1.496080in}}%
\pgfpathlineto{\pgfqpoint{1.259968in}{1.496356in}}%
\pgfpathlineto{\pgfqpoint{1.265295in}{1.497461in}}%
\pgfpathlineto{\pgfqpoint{1.266267in}{1.497895in}}%
\pgfpathlineto{\pgfqpoint{1.270949in}{1.498999in}}%
\pgfpathlineto{\pgfqpoint{1.271846in}{1.499472in}}%
\pgfpathlineto{\pgfqpoint{1.277641in}{1.500577in}}%
\pgfpathlineto{\pgfqpoint{1.278669in}{1.501011in}}%
\pgfpathlineto{\pgfqpoint{1.282847in}{1.502115in}}%
\pgfpathlineto{\pgfqpoint{1.283725in}{1.502786in}}%
\pgfpathlineto{\pgfqpoint{1.289725in}{1.503890in}}%
\pgfpathlineto{\pgfqpoint{1.290781in}{1.504206in}}%
\pgfpathlineto{\pgfqpoint{1.297342in}{1.505310in}}%
\pgfpathlineto{\pgfqpoint{1.298314in}{1.505468in}}%
\pgfpathlineto{\pgfqpoint{1.303211in}{1.506572in}}%
\pgfpathlineto{\pgfqpoint{1.304211in}{1.506927in}}%
\pgfpathlineto{\pgfqpoint{1.311342in}{1.508031in}}%
\pgfpathlineto{\pgfqpoint{1.312426in}{1.508347in}}%
\pgfpathlineto{\pgfqpoint{1.319847in}{1.509451in}}%
\pgfpathlineto{\pgfqpoint{1.320931in}{1.509885in}}%
\pgfpathlineto{\pgfqpoint{1.325866in}{1.510990in}}%
\pgfpathlineto{\pgfqpoint{1.326492in}{1.511226in}}%
\pgfpathlineto{\pgfqpoint{1.333324in}{1.512331in}}%
\pgfpathlineto{\pgfqpoint{1.334370in}{1.512607in}}%
\pgfpathlineto{\pgfqpoint{1.342670in}{1.513711in}}%
\pgfpathlineto{\pgfqpoint{1.343539in}{1.513869in}}%
\pgfpathlineto{\pgfqpoint{1.348053in}{1.514973in}}%
\pgfpathlineto{\pgfqpoint{1.349081in}{1.515328in}}%
\pgfpathlineto{\pgfqpoint{1.356445in}{1.516433in}}%
\pgfpathlineto{\pgfqpoint{1.357399in}{1.516630in}}%
\pgfpathlineto{\pgfqpoint{1.366137in}{1.517734in}}%
\pgfpathlineto{\pgfqpoint{1.367090in}{1.518050in}}%
\pgfpathlineto{\pgfqpoint{1.373707in}{1.519154in}}%
\pgfpathlineto{\pgfqpoint{1.374810in}{1.519391in}}%
\pgfpathlineto{\pgfqpoint{1.381119in}{1.520495in}}%
\pgfpathlineto{\pgfqpoint{1.382072in}{1.520693in}}%
\pgfpathlineto{\pgfqpoint{1.389633in}{1.521797in}}%
\pgfpathlineto{\pgfqpoint{1.390736in}{1.521955in}}%
\pgfpathlineto{\pgfqpoint{1.397212in}{1.523059in}}%
\pgfpathlineto{\pgfqpoint{1.398138in}{1.523138in}}%
\pgfpathlineto{\pgfqpoint{1.403979in}{1.524242in}}%
\pgfpathlineto{\pgfqpoint{1.404642in}{1.524361in}}%
\pgfpathlineto{\pgfqpoint{1.413100in}{1.525465in}}%
\pgfpathlineto{\pgfqpoint{1.413988in}{1.525662in}}%
\pgfpathlineto{\pgfqpoint{1.422951in}{1.526767in}}%
\pgfpathlineto{\pgfqpoint{1.423970in}{1.526924in}}%
\pgfpathlineto{\pgfqpoint{1.431661in}{1.528029in}}%
\pgfpathlineto{\pgfqpoint{1.432026in}{1.528147in}}%
\pgfpathlineto{\pgfqpoint{1.441671in}{1.529252in}}%
\pgfpathlineto{\pgfqpoint{1.442475in}{1.529409in}}%
\pgfpathlineto{\pgfqpoint{1.455456in}{1.530514in}}%
\pgfpathlineto{\pgfqpoint{1.456475in}{1.530632in}}%
\pgfpathlineto{\pgfqpoint{1.466662in}{1.531736in}}%
\pgfpathlineto{\pgfqpoint{1.467699in}{1.532013in}}%
\pgfpathlineto{\pgfqpoint{1.477288in}{1.533117in}}%
\pgfpathlineto{\pgfqpoint{1.477297in}{1.533196in}}%
\pgfpathlineto{\pgfqpoint{1.487157in}{1.534300in}}%
\pgfpathlineto{\pgfqpoint{1.488232in}{1.534616in}}%
\pgfpathlineto{\pgfqpoint{1.496615in}{1.535720in}}%
\pgfpathlineto{\pgfqpoint{1.497139in}{1.535917in}}%
\pgfpathlineto{\pgfqpoint{1.508559in}{1.537022in}}%
\pgfpathlineto{\pgfqpoint{1.509382in}{1.537101in}}%
\pgfpathlineto{\pgfqpoint{1.517681in}{1.538205in}}%
\pgfpathlineto{\pgfqpoint{1.517737in}{1.538284in}}%
\pgfpathlineto{\pgfqpoint{1.529877in}{1.539388in}}%
\pgfpathlineto{\pgfqpoint{1.530933in}{1.539625in}}%
\pgfpathlineto{\pgfqpoint{1.538691in}{1.540729in}}%
\pgfpathlineto{\pgfqpoint{1.539597in}{1.540927in}}%
\pgfpathlineto{\pgfqpoint{1.549859in}{1.542031in}}%
\pgfpathlineto{\pgfqpoint{1.550952in}{1.542268in}}%
\pgfpathlineto{\pgfqpoint{1.559345in}{1.543372in}}%
\pgfpathlineto{\pgfqpoint{1.560336in}{1.543490in}}%
\pgfpathlineto{\pgfqpoint{1.571560in}{1.544595in}}%
\pgfpathlineto{\pgfqpoint{1.572579in}{1.544752in}}%
\pgfpathlineto{\pgfqpoint{1.580943in}{1.545857in}}%
\pgfpathlineto{\pgfqpoint{1.581224in}{1.545936in}}%
\pgfpathlineto{\pgfqpoint{1.592205in}{1.547040in}}%
\pgfpathlineto{\pgfqpoint{1.592887in}{1.547237in}}%
\pgfpathlineto{\pgfqpoint{1.603757in}{1.548342in}}%
\pgfpathlineto{\pgfqpoint{1.604056in}{1.548421in}}%
\pgfpathlineto{\pgfqpoint{1.612327in}{1.549525in}}%
\pgfpathlineto{\pgfqpoint{1.612570in}{1.549604in}}%
\pgfpathlineto{\pgfqpoint{1.629720in}{1.550708in}}%
\pgfpathlineto{\pgfqpoint{1.630514in}{1.550787in}}%
\pgfpathlineto{\pgfqpoint{1.639318in}{1.551379in}}%
\pgfpathlineto{\pgfqpoint{1.640281in}{1.560806in}}%
\pgfpathlineto{\pgfqpoint{1.653505in}{1.561910in}}%
\pgfpathlineto{\pgfqpoint{1.653589in}{1.561989in}}%
\pgfpathlineto{\pgfqpoint{1.666430in}{1.563093in}}%
\pgfpathlineto{\pgfqpoint{1.667281in}{1.563251in}}%
\pgfpathlineto{\pgfqpoint{1.680365in}{1.564355in}}%
\pgfpathlineto{\pgfqpoint{1.681374in}{1.564632in}}%
\pgfpathlineto{\pgfqpoint{1.692608in}{1.565736in}}%
\pgfpathlineto{\pgfqpoint{1.693683in}{1.565854in}}%
\pgfpathlineto{\pgfqpoint{1.704758in}{1.566959in}}%
\pgfpathlineto{\pgfqpoint{1.705655in}{1.567156in}}%
\pgfpathlineto{\pgfqpoint{1.715627in}{1.568260in}}%
\pgfpathlineto{\pgfqpoint{1.716702in}{1.568379in}}%
\pgfpathlineto{\pgfqpoint{1.729721in}{1.569483in}}%
\pgfpathlineto{\pgfqpoint{1.730796in}{1.569601in}}%
\pgfpathlineto{\pgfqpoint{1.745375in}{1.570706in}}%
\pgfpathlineto{\pgfqpoint{1.745628in}{1.570824in}}%
\pgfpathlineto{\pgfqpoint{1.762282in}{1.571928in}}%
\pgfpathlineto{\pgfqpoint{1.762488in}{1.572047in}}%
\pgfpathlineto{\pgfqpoint{1.773693in}{1.573151in}}%
\pgfpathlineto{\pgfqpoint{1.774506in}{1.573388in}}%
\pgfpathlineto{\pgfqpoint{1.786890in}{1.574492in}}%
\pgfpathlineto{\pgfqpoint{1.787815in}{1.574729in}}%
\pgfpathlineto{\pgfqpoint{1.797852in}{1.575833in}}%
\pgfpathlineto{\pgfqpoint{1.797852in}{1.575873in}}%
\pgfpathlineto{\pgfqpoint{1.810750in}{1.576977in}}%
\pgfpathlineto{\pgfqpoint{1.811479in}{1.577095in}}%
\pgfpathlineto{\pgfqpoint{1.825956in}{1.578200in}}%
\pgfpathlineto{\pgfqpoint{1.826451in}{1.578318in}}%
\pgfpathlineto{\pgfqpoint{1.839993in}{1.579422in}}%
\pgfpathlineto{\pgfqpoint{1.840844in}{1.579620in}}%
\pgfpathlineto{\pgfqpoint{1.855535in}{1.580724in}}%
\pgfpathlineto{\pgfqpoint{1.856517in}{1.580921in}}%
\pgfpathlineto{\pgfqpoint{1.870582in}{1.582026in}}%
\pgfpathlineto{\pgfqpoint{1.871582in}{1.582223in}}%
\pgfpathlineto{\pgfqpoint{1.884424in}{1.583327in}}%
\pgfpathlineto{\pgfqpoint{1.885349in}{1.583446in}}%
\pgfpathlineto{\pgfqpoint{1.903807in}{1.584550in}}%
\pgfpathlineto{\pgfqpoint{1.904611in}{1.584787in}}%
\pgfpathlineto{\pgfqpoint{1.922433in}{1.585891in}}%
\pgfpathlineto{\pgfqpoint{1.923489in}{1.585970in}}%
\pgfpathlineto{\pgfqpoint{1.934218in}{1.587074in}}%
\pgfpathlineto{\pgfqpoint{1.935312in}{1.587153in}}%
\pgfpathlineto{\pgfqpoint{1.948835in}{1.588258in}}%
\pgfpathlineto{\pgfqpoint{1.949480in}{1.588376in}}%
\pgfpathlineto{\pgfqpoint{1.964910in}{1.589480in}}%
\pgfpathlineto{\pgfqpoint{1.965752in}{1.589638in}}%
\pgfpathlineto{\pgfqpoint{1.975574in}{1.590742in}}%
\pgfpathlineto{\pgfqpoint{1.976284in}{1.590900in}}%
\pgfpathlineto{\pgfqpoint{1.987434in}{1.592005in}}%
\pgfpathlineto{\pgfqpoint{1.988490in}{1.592281in}}%
\pgfpathlineto{\pgfqpoint{2.000387in}{1.593385in}}%
\pgfpathlineto{\pgfqpoint{2.001444in}{1.593661in}}%
\pgfpathlineto{\pgfqpoint{2.008930in}{1.594766in}}%
\pgfpathlineto{\pgfqpoint{2.010032in}{1.594963in}}%
\pgfpathlineto{\pgfqpoint{2.018135in}{1.596067in}}%
\pgfpathlineto{\pgfqpoint{2.018995in}{1.596264in}}%
\pgfpathlineto{\pgfqpoint{2.024892in}{1.597369in}}%
\pgfpathlineto{\pgfqpoint{2.025958in}{1.597605in}}%
\pgfpathlineto{\pgfqpoint{2.029584in}{1.598710in}}%
\pgfpathlineto{\pgfqpoint{2.030481in}{1.599144in}}%
\pgfpathlineto{\pgfqpoint{2.032799in}{1.600209in}}%
\pgfpathlineto{\pgfqpoint{2.033126in}{1.601944in}}%
\pgfpathlineto{\pgfqpoint{2.033126in}{1.601944in}}%
\pgfusepath{stroke}%
\end{pgfscope}%
\begin{pgfscope}%
\pgfsetrectcap%
\pgfsetmiterjoin%
\pgfsetlinewidth{0.803000pt}%
\definecolor{currentstroke}{rgb}{0.000000,0.000000,0.000000}%
\pgfsetstrokecolor{currentstroke}%
\pgfsetdash{}{0pt}%
\pgfpathmoveto{\pgfqpoint{0.553581in}{0.499444in}}%
\pgfpathlineto{\pgfqpoint{0.553581in}{1.654444in}}%
\pgfusepath{stroke}%
\end{pgfscope}%
\begin{pgfscope}%
\pgfsetrectcap%
\pgfsetmiterjoin%
\pgfsetlinewidth{0.803000pt}%
\definecolor{currentstroke}{rgb}{0.000000,0.000000,0.000000}%
\pgfsetstrokecolor{currentstroke}%
\pgfsetdash{}{0pt}%
\pgfpathmoveto{\pgfqpoint{2.103581in}{0.499444in}}%
\pgfpathlineto{\pgfqpoint{2.103581in}{1.654444in}}%
\pgfusepath{stroke}%
\end{pgfscope}%
\begin{pgfscope}%
\pgfsetrectcap%
\pgfsetmiterjoin%
\pgfsetlinewidth{0.803000pt}%
\definecolor{currentstroke}{rgb}{0.000000,0.000000,0.000000}%
\pgfsetstrokecolor{currentstroke}%
\pgfsetdash{}{0pt}%
\pgfpathmoveto{\pgfqpoint{0.553581in}{0.499444in}}%
\pgfpathlineto{\pgfqpoint{2.103581in}{0.499444in}}%
\pgfusepath{stroke}%
\end{pgfscope}%
\begin{pgfscope}%
\pgfsetrectcap%
\pgfsetmiterjoin%
\pgfsetlinewidth{0.803000pt}%
\definecolor{currentstroke}{rgb}{0.000000,0.000000,0.000000}%
\pgfsetstrokecolor{currentstroke}%
\pgfsetdash{}{0pt}%
\pgfpathmoveto{\pgfqpoint{0.553581in}{1.654444in}}%
\pgfpathlineto{\pgfqpoint{2.103581in}{1.654444in}}%
\pgfusepath{stroke}%
\end{pgfscope}%
\begin{pgfscope}%
\pgfsetbuttcap%
\pgfsetmiterjoin%
\definecolor{currentfill}{rgb}{1.000000,1.000000,1.000000}%
\pgfsetfillcolor{currentfill}%
\pgfsetlinewidth{0.000000pt}%
\definecolor{currentstroke}{rgb}{0.000000,0.000000,0.000000}%
\pgfsetstrokecolor{currentstroke}%
\pgfsetstrokeopacity{0.000000}%
\pgfsetdash{}{0pt}%
\pgfpathmoveto{\pgfqpoint{0.686183in}{0.966899in}}%
\pgfpathlineto{\pgfqpoint{1.085905in}{0.966899in}}%
\pgfpathlineto{\pgfqpoint{1.085905in}{1.173565in}}%
\pgfpathlineto{\pgfqpoint{0.686183in}{1.173565in}}%
\pgfpathlineto{\pgfqpoint{0.686183in}{0.966899in}}%
\pgfpathclose%
\pgfusepath{fill}%
\end{pgfscope}%
\begin{pgfscope}%
\definecolor{textcolor}{rgb}{0.000000,0.000000,0.000000}%
\pgfsetstrokecolor{textcolor}%
\pgfsetfillcolor{textcolor}%
\pgftext[x=0.727850in,y=1.035510in,left,base]{\color{textcolor}\rmfamily\fontsize{10.000000}{12.000000}\selectfont 0.337}%
\end{pgfscope}%
\begin{pgfscope}%
\pgfsetbuttcap%
\pgfsetmiterjoin%
\definecolor{currentfill}{rgb}{1.000000,1.000000,1.000000}%
\pgfsetfillcolor{currentfill}%
\pgfsetlinewidth{0.000000pt}%
\definecolor{currentstroke}{rgb}{0.000000,0.000000,0.000000}%
\pgfsetstrokecolor{currentstroke}%
\pgfsetstrokeopacity{0.000000}%
\pgfsetdash{}{0pt}%
\pgfpathmoveto{\pgfqpoint{0.582378in}{0.483333in}}%
\pgfpathlineto{\pgfqpoint{0.843211in}{0.483333in}}%
\pgfpathlineto{\pgfqpoint{0.843211in}{0.690000in}}%
\pgfpathlineto{\pgfqpoint{0.582378in}{0.690000in}}%
\pgfpathlineto{\pgfqpoint{0.582378in}{0.483333in}}%
\pgfpathclose%
\pgfusepath{fill}%
\end{pgfscope}%
\begin{pgfscope}%
\definecolor{textcolor}{rgb}{0.000000,0.000000,0.000000}%
\pgfsetstrokecolor{textcolor}%
\pgfsetfillcolor{textcolor}%
\pgftext[x=0.624045in,y=0.551944in,left,base]{\color{textcolor}\rmfamily\fontsize{10.000000}{12.000000}\selectfont 0.5}%
\end{pgfscope}%
\begin{pgfscope}%
\pgfsetbuttcap%
\pgfsetmiterjoin%
\definecolor{currentfill}{rgb}{1.000000,1.000000,1.000000}%
\pgfsetfillcolor{currentfill}%
\pgfsetlinewidth{0.000000pt}%
\definecolor{currentstroke}{rgb}{0.000000,0.000000,0.000000}%
\pgfsetstrokecolor{currentstroke}%
\pgfsetstrokeopacity{0.000000}%
\pgfsetdash{}{0pt}%
\pgfpathmoveto{\pgfqpoint{0.582378in}{0.483333in}}%
\pgfpathlineto{\pgfqpoint{0.982100in}{0.483333in}}%
\pgfpathlineto{\pgfqpoint{0.982100in}{0.690000in}}%
\pgfpathlineto{\pgfqpoint{0.582378in}{0.690000in}}%
\pgfpathlineto{\pgfqpoint{0.582378in}{0.483333in}}%
\pgfpathclose%
\pgfusepath{fill}%
\end{pgfscope}%
\begin{pgfscope}%
\definecolor{textcolor}{rgb}{0.000000,0.000000,0.000000}%
\pgfsetstrokecolor{textcolor}%
\pgfsetfillcolor{textcolor}%
\pgftext[x=0.624045in,y=0.551944in,left,base]{\color{textcolor}\rmfamily\fontsize{10.000000}{12.000000}\selectfont 0.656}%
\end{pgfscope}%
\begin{pgfscope}%
\pgfsetbuttcap%
\pgfsetmiterjoin%
\definecolor{currentfill}{rgb}{1.000000,1.000000,1.000000}%
\pgfsetfillcolor{currentfill}%
\pgfsetfillopacity{0.800000}%
\pgfsetlinewidth{1.003750pt}%
\definecolor{currentstroke}{rgb}{0.800000,0.800000,0.800000}%
\pgfsetstrokecolor{currentstroke}%
\pgfsetstrokeopacity{0.800000}%
\pgfsetdash{}{0pt}%
\pgfpathmoveto{\pgfqpoint{0.832747in}{0.568889in}}%
\pgfpathlineto{\pgfqpoint{2.006358in}{0.568889in}}%
\pgfpathquadraticcurveto{\pgfqpoint{2.034136in}{0.568889in}}{\pgfqpoint{2.034136in}{0.596666in}}%
\pgfpathlineto{\pgfqpoint{2.034136in}{0.776388in}}%
\pgfpathquadraticcurveto{\pgfqpoint{2.034136in}{0.804166in}}{\pgfqpoint{2.006358in}{0.804166in}}%
\pgfpathlineto{\pgfqpoint{0.832747in}{0.804166in}}%
\pgfpathquadraticcurveto{\pgfqpoint{0.804970in}{0.804166in}}{\pgfqpoint{0.804970in}{0.776388in}}%
\pgfpathlineto{\pgfqpoint{0.804970in}{0.596666in}}%
\pgfpathquadraticcurveto{\pgfqpoint{0.804970in}{0.568889in}}{\pgfqpoint{0.832747in}{0.568889in}}%
\pgfpathlineto{\pgfqpoint{0.832747in}{0.568889in}}%
\pgfpathclose%
\pgfusepath{stroke,fill}%
\end{pgfscope}%
\begin{pgfscope}%
\pgfsetrectcap%
\pgfsetroundjoin%
\pgfsetlinewidth{1.505625pt}%
\definecolor{currentstroke}{rgb}{0.000000,0.000000,0.000000}%
\pgfsetstrokecolor{currentstroke}%
\pgfsetdash{}{0pt}%
\pgfpathmoveto{\pgfqpoint{0.860525in}{0.700000in}}%
\pgfpathlineto{\pgfqpoint{0.999414in}{0.700000in}}%
\pgfpathlineto{\pgfqpoint{1.138303in}{0.700000in}}%
\pgfusepath{stroke}%
\end{pgfscope}%
\begin{pgfscope}%
\definecolor{textcolor}{rgb}{0.000000,0.000000,0.000000}%
\pgfsetstrokecolor{textcolor}%
\pgfsetfillcolor{textcolor}%
\pgftext[x=1.249414in,y=0.651388in,left,base]{\color{textcolor}\rmfamily\fontsize{10.000000}{12.000000}\selectfont AUC=0.832}%
\end{pgfscope}%
\end{pgfpicture}%
\makeatother%
\endgroup%

  &
\vspace{0pt} 
  
\begin{tabular}{cc|c|c|}
	&\multicolumn{1}{c}{}& \multicolumn{2}{c}{Prediction} \cr
	&\multicolumn{1}{c}{} & \multicolumn{1}{c}{N} & \multicolumn{1}{c}{P} \cr\cline{3-4}
	\multirow{2}{*}{\rotatebox[origin=c]{90}{Actual}}&N & 150,771    &  0 \vrule width 0pt height 10pt depth 2pt \cr\cline{3-4}
	&P & 22,621 & 0 \vrule width 0pt height 10pt depth 2pt \cr\cline{3-4}
\end{tabular}

\begin{center}
\begin{tabular}{ll}
und & Precision \cr 
0.000 & Recall \cr 
und & F1 \cr 
\end{tabular}
\end{center}
  
\end{tabular}

Similarly, if the class weight is too high, we may get a model like Example 4 that sends an ambulance to every reported crash.  Example 4 is Example 1 transformed with $f(x) = 0.5x + 0.5$

\noindent\begin{tabular}{@{}p{0.3\textwidth}@{\hspace{24pt}} p{0.3\textwidth} @{\hspace{24pt}} p{0.3\textwidth}}
  \vspace{0pt} %% Creator: Matplotlib, PGF backend
%%
%% To include the figure in your LaTeX document, write
%%   \input{<filename>.pgf}
%%
%% Make sure the required packages are loaded in your preamble
%%   \usepackage{pgf}
%%
%% Also ensure that all the required font packages are loaded; for instance,
%% the lmodern package is sometimes necessary when using math font.
%%   \usepackage{lmodern}
%%
%% Figures using additional raster images can only be included by \input if
%% they are in the same directory as the main LaTeX file. For loading figures
%% from other directories you can use the `import` package
%%   \usepackage{import}
%%
%% and then include the figures with
%%   \import{<path to file>}{<filename>.pgf}
%%
%% Matplotlib used the following preamble
%%   
%%   \usepackage{fontspec}
%%   \makeatletter\@ifpackageloaded{underscore}{}{\usepackage[strings]{underscore}}\makeatother
%%
\begingroup%
\makeatletter%
\begin{pgfpicture}%
\pgfpathrectangle{\pgfpointorigin}{\pgfqpoint{2.253750in}{1.953444in}}%
\pgfusepath{use as bounding box, clip}%
\begin{pgfscope}%
\pgfsetbuttcap%
\pgfsetmiterjoin%
\definecolor{currentfill}{rgb}{1.000000,1.000000,1.000000}%
\pgfsetfillcolor{currentfill}%
\pgfsetlinewidth{0.000000pt}%
\definecolor{currentstroke}{rgb}{1.000000,1.000000,1.000000}%
\pgfsetstrokecolor{currentstroke}%
\pgfsetdash{}{0pt}%
\pgfpathmoveto{\pgfqpoint{0.000000in}{0.000000in}}%
\pgfpathlineto{\pgfqpoint{2.253750in}{0.000000in}}%
\pgfpathlineto{\pgfqpoint{2.253750in}{1.953444in}}%
\pgfpathlineto{\pgfqpoint{0.000000in}{1.953444in}}%
\pgfpathlineto{\pgfqpoint{0.000000in}{0.000000in}}%
\pgfpathclose%
\pgfusepath{fill}%
\end{pgfscope}%
\begin{pgfscope}%
\pgfsetbuttcap%
\pgfsetmiterjoin%
\definecolor{currentfill}{rgb}{1.000000,1.000000,1.000000}%
\pgfsetfillcolor{currentfill}%
\pgfsetlinewidth{0.000000pt}%
\definecolor{currentstroke}{rgb}{0.000000,0.000000,0.000000}%
\pgfsetstrokecolor{currentstroke}%
\pgfsetstrokeopacity{0.000000}%
\pgfsetdash{}{0pt}%
\pgfpathmoveto{\pgfqpoint{0.515000in}{0.499444in}}%
\pgfpathlineto{\pgfqpoint{2.065000in}{0.499444in}}%
\pgfpathlineto{\pgfqpoint{2.065000in}{1.654444in}}%
\pgfpathlineto{\pgfqpoint{0.515000in}{1.654444in}}%
\pgfpathlineto{\pgfqpoint{0.515000in}{0.499444in}}%
\pgfpathclose%
\pgfusepath{fill}%
\end{pgfscope}%
\begin{pgfscope}%
\pgfpathrectangle{\pgfqpoint{0.515000in}{0.499444in}}{\pgfqpoint{1.550000in}{1.155000in}}%
\pgfusepath{clip}%
\pgfsetbuttcap%
\pgfsetmiterjoin%
\pgfsetlinewidth{1.003750pt}%
\definecolor{currentstroke}{rgb}{0.000000,0.000000,0.000000}%
\pgfsetstrokecolor{currentstroke}%
\pgfsetdash{}{0pt}%
\pgfpathmoveto{\pgfqpoint{0.505000in}{0.499444in}}%
\pgfpathlineto{\pgfqpoint{0.552805in}{0.499444in}}%
\pgfpathlineto{\pgfqpoint{0.552805in}{0.499444in}}%
\pgfpathlineto{\pgfqpoint{0.505000in}{0.499444in}}%
\pgfusepath{stroke}%
\end{pgfscope}%
\begin{pgfscope}%
\pgfpathrectangle{\pgfqpoint{0.515000in}{0.499444in}}{\pgfqpoint{1.550000in}{1.155000in}}%
\pgfusepath{clip}%
\pgfsetbuttcap%
\pgfsetmiterjoin%
\pgfsetlinewidth{1.003750pt}%
\definecolor{currentstroke}{rgb}{0.000000,0.000000,0.000000}%
\pgfsetstrokecolor{currentstroke}%
\pgfsetdash{}{0pt}%
\pgfpathmoveto{\pgfqpoint{0.643537in}{0.499444in}}%
\pgfpathlineto{\pgfqpoint{0.704025in}{0.499444in}}%
\pgfpathlineto{\pgfqpoint{0.704025in}{0.499444in}}%
\pgfpathlineto{\pgfqpoint{0.643537in}{0.499444in}}%
\pgfpathlineto{\pgfqpoint{0.643537in}{0.499444in}}%
\pgfpathclose%
\pgfusepath{stroke}%
\end{pgfscope}%
\begin{pgfscope}%
\pgfpathrectangle{\pgfqpoint{0.515000in}{0.499444in}}{\pgfqpoint{1.550000in}{1.155000in}}%
\pgfusepath{clip}%
\pgfsetbuttcap%
\pgfsetmiterjoin%
\pgfsetlinewidth{1.003750pt}%
\definecolor{currentstroke}{rgb}{0.000000,0.000000,0.000000}%
\pgfsetstrokecolor{currentstroke}%
\pgfsetdash{}{0pt}%
\pgfpathmoveto{\pgfqpoint{0.794756in}{0.499444in}}%
\pgfpathlineto{\pgfqpoint{0.855244in}{0.499444in}}%
\pgfpathlineto{\pgfqpoint{0.855244in}{0.499444in}}%
\pgfpathlineto{\pgfqpoint{0.794756in}{0.499444in}}%
\pgfpathlineto{\pgfqpoint{0.794756in}{0.499444in}}%
\pgfpathclose%
\pgfusepath{stroke}%
\end{pgfscope}%
\begin{pgfscope}%
\pgfpathrectangle{\pgfqpoint{0.515000in}{0.499444in}}{\pgfqpoint{1.550000in}{1.155000in}}%
\pgfusepath{clip}%
\pgfsetbuttcap%
\pgfsetmiterjoin%
\pgfsetlinewidth{1.003750pt}%
\definecolor{currentstroke}{rgb}{0.000000,0.000000,0.000000}%
\pgfsetstrokecolor{currentstroke}%
\pgfsetdash{}{0pt}%
\pgfpathmoveto{\pgfqpoint{0.945976in}{0.499444in}}%
\pgfpathlineto{\pgfqpoint{1.006464in}{0.499444in}}%
\pgfpathlineto{\pgfqpoint{1.006464in}{0.499444in}}%
\pgfpathlineto{\pgfqpoint{0.945976in}{0.499444in}}%
\pgfpathlineto{\pgfqpoint{0.945976in}{0.499444in}}%
\pgfpathclose%
\pgfusepath{stroke}%
\end{pgfscope}%
\begin{pgfscope}%
\pgfpathrectangle{\pgfqpoint{0.515000in}{0.499444in}}{\pgfqpoint{1.550000in}{1.155000in}}%
\pgfusepath{clip}%
\pgfsetbuttcap%
\pgfsetmiterjoin%
\pgfsetlinewidth{1.003750pt}%
\definecolor{currentstroke}{rgb}{0.000000,0.000000,0.000000}%
\pgfsetstrokecolor{currentstroke}%
\pgfsetdash{}{0pt}%
\pgfpathmoveto{\pgfqpoint{1.097195in}{0.499444in}}%
\pgfpathlineto{\pgfqpoint{1.157683in}{0.499444in}}%
\pgfpathlineto{\pgfqpoint{1.157683in}{0.499444in}}%
\pgfpathlineto{\pgfqpoint{1.097195in}{0.499444in}}%
\pgfpathlineto{\pgfqpoint{1.097195in}{0.499444in}}%
\pgfpathclose%
\pgfusepath{stroke}%
\end{pgfscope}%
\begin{pgfscope}%
\pgfpathrectangle{\pgfqpoint{0.515000in}{0.499444in}}{\pgfqpoint{1.550000in}{1.155000in}}%
\pgfusepath{clip}%
\pgfsetbuttcap%
\pgfsetmiterjoin%
\pgfsetlinewidth{1.003750pt}%
\definecolor{currentstroke}{rgb}{0.000000,0.000000,0.000000}%
\pgfsetstrokecolor{currentstroke}%
\pgfsetdash{}{0pt}%
\pgfpathmoveto{\pgfqpoint{1.248415in}{0.499444in}}%
\pgfpathlineto{\pgfqpoint{1.308903in}{0.499444in}}%
\pgfpathlineto{\pgfqpoint{1.308903in}{0.958323in}}%
\pgfpathlineto{\pgfqpoint{1.248415in}{0.958323in}}%
\pgfpathlineto{\pgfqpoint{1.248415in}{0.499444in}}%
\pgfpathclose%
\pgfusepath{stroke}%
\end{pgfscope}%
\begin{pgfscope}%
\pgfpathrectangle{\pgfqpoint{0.515000in}{0.499444in}}{\pgfqpoint{1.550000in}{1.155000in}}%
\pgfusepath{clip}%
\pgfsetbuttcap%
\pgfsetmiterjoin%
\pgfsetlinewidth{1.003750pt}%
\definecolor{currentstroke}{rgb}{0.000000,0.000000,0.000000}%
\pgfsetstrokecolor{currentstroke}%
\pgfsetdash{}{0pt}%
\pgfpathmoveto{\pgfqpoint{1.399634in}{0.499444in}}%
\pgfpathlineto{\pgfqpoint{1.460122in}{0.499444in}}%
\pgfpathlineto{\pgfqpoint{1.460122in}{1.599444in}}%
\pgfpathlineto{\pgfqpoint{1.399634in}{1.599444in}}%
\pgfpathlineto{\pgfqpoint{1.399634in}{0.499444in}}%
\pgfpathclose%
\pgfusepath{stroke}%
\end{pgfscope}%
\begin{pgfscope}%
\pgfpathrectangle{\pgfqpoint{0.515000in}{0.499444in}}{\pgfqpoint{1.550000in}{1.155000in}}%
\pgfusepath{clip}%
\pgfsetbuttcap%
\pgfsetmiterjoin%
\pgfsetlinewidth{1.003750pt}%
\definecolor{currentstroke}{rgb}{0.000000,0.000000,0.000000}%
\pgfsetstrokecolor{currentstroke}%
\pgfsetdash{}{0pt}%
\pgfpathmoveto{\pgfqpoint{1.550854in}{0.499444in}}%
\pgfpathlineto{\pgfqpoint{1.611342in}{0.499444in}}%
\pgfpathlineto{\pgfqpoint{1.611342in}{1.129057in}}%
\pgfpathlineto{\pgfqpoint{1.550854in}{1.129057in}}%
\pgfpathlineto{\pgfqpoint{1.550854in}{0.499444in}}%
\pgfpathclose%
\pgfusepath{stroke}%
\end{pgfscope}%
\begin{pgfscope}%
\pgfpathrectangle{\pgfqpoint{0.515000in}{0.499444in}}{\pgfqpoint{1.550000in}{1.155000in}}%
\pgfusepath{clip}%
\pgfsetbuttcap%
\pgfsetmiterjoin%
\pgfsetlinewidth{1.003750pt}%
\definecolor{currentstroke}{rgb}{0.000000,0.000000,0.000000}%
\pgfsetstrokecolor{currentstroke}%
\pgfsetdash{}{0pt}%
\pgfpathmoveto{\pgfqpoint{1.702073in}{0.499444in}}%
\pgfpathlineto{\pgfqpoint{1.762561in}{0.499444in}}%
\pgfpathlineto{\pgfqpoint{1.762561in}{0.740633in}}%
\pgfpathlineto{\pgfqpoint{1.702073in}{0.740633in}}%
\pgfpathlineto{\pgfqpoint{1.702073in}{0.499444in}}%
\pgfpathclose%
\pgfusepath{stroke}%
\end{pgfscope}%
\begin{pgfscope}%
\pgfpathrectangle{\pgfqpoint{0.515000in}{0.499444in}}{\pgfqpoint{1.550000in}{1.155000in}}%
\pgfusepath{clip}%
\pgfsetbuttcap%
\pgfsetmiterjoin%
\pgfsetlinewidth{1.003750pt}%
\definecolor{currentstroke}{rgb}{0.000000,0.000000,0.000000}%
\pgfsetstrokecolor{currentstroke}%
\pgfsetdash{}{0pt}%
\pgfpathmoveto{\pgfqpoint{1.853293in}{0.499444in}}%
\pgfpathlineto{\pgfqpoint{1.913781in}{0.499444in}}%
\pgfpathlineto{\pgfqpoint{1.913781in}{0.557512in}}%
\pgfpathlineto{\pgfqpoint{1.853293in}{0.557512in}}%
\pgfpathlineto{\pgfqpoint{1.853293in}{0.499444in}}%
\pgfpathclose%
\pgfusepath{stroke}%
\end{pgfscope}%
\begin{pgfscope}%
\pgfpathrectangle{\pgfqpoint{0.515000in}{0.499444in}}{\pgfqpoint{1.550000in}{1.155000in}}%
\pgfusepath{clip}%
\pgfsetbuttcap%
\pgfsetmiterjoin%
\definecolor{currentfill}{rgb}{0.000000,0.000000,0.000000}%
\pgfsetfillcolor{currentfill}%
\pgfsetlinewidth{0.000000pt}%
\definecolor{currentstroke}{rgb}{0.000000,0.000000,0.000000}%
\pgfsetstrokecolor{currentstroke}%
\pgfsetstrokeopacity{0.000000}%
\pgfsetdash{}{0pt}%
\pgfpathmoveto{\pgfqpoint{0.552805in}{0.499444in}}%
\pgfpathlineto{\pgfqpoint{0.613293in}{0.499444in}}%
\pgfpathlineto{\pgfqpoint{0.613293in}{0.499444in}}%
\pgfpathlineto{\pgfqpoint{0.552805in}{0.499444in}}%
\pgfpathlineto{\pgfqpoint{0.552805in}{0.499444in}}%
\pgfpathclose%
\pgfusepath{fill}%
\end{pgfscope}%
\begin{pgfscope}%
\pgfpathrectangle{\pgfqpoint{0.515000in}{0.499444in}}{\pgfqpoint{1.550000in}{1.155000in}}%
\pgfusepath{clip}%
\pgfsetbuttcap%
\pgfsetmiterjoin%
\definecolor{currentfill}{rgb}{0.000000,0.000000,0.000000}%
\pgfsetfillcolor{currentfill}%
\pgfsetlinewidth{0.000000pt}%
\definecolor{currentstroke}{rgb}{0.000000,0.000000,0.000000}%
\pgfsetstrokecolor{currentstroke}%
\pgfsetstrokeopacity{0.000000}%
\pgfsetdash{}{0pt}%
\pgfpathmoveto{\pgfqpoint{0.704025in}{0.499444in}}%
\pgfpathlineto{\pgfqpoint{0.764512in}{0.499444in}}%
\pgfpathlineto{\pgfqpoint{0.764512in}{0.499444in}}%
\pgfpathlineto{\pgfqpoint{0.704025in}{0.499444in}}%
\pgfpathlineto{\pgfqpoint{0.704025in}{0.499444in}}%
\pgfpathclose%
\pgfusepath{fill}%
\end{pgfscope}%
\begin{pgfscope}%
\pgfpathrectangle{\pgfqpoint{0.515000in}{0.499444in}}{\pgfqpoint{1.550000in}{1.155000in}}%
\pgfusepath{clip}%
\pgfsetbuttcap%
\pgfsetmiterjoin%
\definecolor{currentfill}{rgb}{0.000000,0.000000,0.000000}%
\pgfsetfillcolor{currentfill}%
\pgfsetlinewidth{0.000000pt}%
\definecolor{currentstroke}{rgb}{0.000000,0.000000,0.000000}%
\pgfsetstrokecolor{currentstroke}%
\pgfsetstrokeopacity{0.000000}%
\pgfsetdash{}{0pt}%
\pgfpathmoveto{\pgfqpoint{0.855244in}{0.499444in}}%
\pgfpathlineto{\pgfqpoint{0.915732in}{0.499444in}}%
\pgfpathlineto{\pgfqpoint{0.915732in}{0.499444in}}%
\pgfpathlineto{\pgfqpoint{0.855244in}{0.499444in}}%
\pgfpathlineto{\pgfqpoint{0.855244in}{0.499444in}}%
\pgfpathclose%
\pgfusepath{fill}%
\end{pgfscope}%
\begin{pgfscope}%
\pgfpathrectangle{\pgfqpoint{0.515000in}{0.499444in}}{\pgfqpoint{1.550000in}{1.155000in}}%
\pgfusepath{clip}%
\pgfsetbuttcap%
\pgfsetmiterjoin%
\definecolor{currentfill}{rgb}{0.000000,0.000000,0.000000}%
\pgfsetfillcolor{currentfill}%
\pgfsetlinewidth{0.000000pt}%
\definecolor{currentstroke}{rgb}{0.000000,0.000000,0.000000}%
\pgfsetstrokecolor{currentstroke}%
\pgfsetstrokeopacity{0.000000}%
\pgfsetdash{}{0pt}%
\pgfpathmoveto{\pgfqpoint{1.006464in}{0.499444in}}%
\pgfpathlineto{\pgfqpoint{1.066951in}{0.499444in}}%
\pgfpathlineto{\pgfqpoint{1.066951in}{0.499444in}}%
\pgfpathlineto{\pgfqpoint{1.006464in}{0.499444in}}%
\pgfpathlineto{\pgfqpoint{1.006464in}{0.499444in}}%
\pgfpathclose%
\pgfusepath{fill}%
\end{pgfscope}%
\begin{pgfscope}%
\pgfpathrectangle{\pgfqpoint{0.515000in}{0.499444in}}{\pgfqpoint{1.550000in}{1.155000in}}%
\pgfusepath{clip}%
\pgfsetbuttcap%
\pgfsetmiterjoin%
\definecolor{currentfill}{rgb}{0.000000,0.000000,0.000000}%
\pgfsetfillcolor{currentfill}%
\pgfsetlinewidth{0.000000pt}%
\definecolor{currentstroke}{rgb}{0.000000,0.000000,0.000000}%
\pgfsetstrokecolor{currentstroke}%
\pgfsetstrokeopacity{0.000000}%
\pgfsetdash{}{0pt}%
\pgfpathmoveto{\pgfqpoint{1.157683in}{0.499444in}}%
\pgfpathlineto{\pgfqpoint{1.218171in}{0.499444in}}%
\pgfpathlineto{\pgfqpoint{1.218171in}{0.499444in}}%
\pgfpathlineto{\pgfqpoint{1.157683in}{0.499444in}}%
\pgfpathlineto{\pgfqpoint{1.157683in}{0.499444in}}%
\pgfpathclose%
\pgfusepath{fill}%
\end{pgfscope}%
\begin{pgfscope}%
\pgfpathrectangle{\pgfqpoint{0.515000in}{0.499444in}}{\pgfqpoint{1.550000in}{1.155000in}}%
\pgfusepath{clip}%
\pgfsetbuttcap%
\pgfsetmiterjoin%
\definecolor{currentfill}{rgb}{0.000000,0.000000,0.000000}%
\pgfsetfillcolor{currentfill}%
\pgfsetlinewidth{0.000000pt}%
\definecolor{currentstroke}{rgb}{0.000000,0.000000,0.000000}%
\pgfsetstrokecolor{currentstroke}%
\pgfsetstrokeopacity{0.000000}%
\pgfsetdash{}{0pt}%
\pgfpathmoveto{\pgfqpoint{1.308903in}{0.499444in}}%
\pgfpathlineto{\pgfqpoint{1.369391in}{0.499444in}}%
\pgfpathlineto{\pgfqpoint{1.369391in}{0.512226in}}%
\pgfpathlineto{\pgfqpoint{1.308903in}{0.512226in}}%
\pgfpathlineto{\pgfqpoint{1.308903in}{0.499444in}}%
\pgfpathclose%
\pgfusepath{fill}%
\end{pgfscope}%
\begin{pgfscope}%
\pgfpathrectangle{\pgfqpoint{0.515000in}{0.499444in}}{\pgfqpoint{1.550000in}{1.155000in}}%
\pgfusepath{clip}%
\pgfsetbuttcap%
\pgfsetmiterjoin%
\definecolor{currentfill}{rgb}{0.000000,0.000000,0.000000}%
\pgfsetfillcolor{currentfill}%
\pgfsetlinewidth{0.000000pt}%
\definecolor{currentstroke}{rgb}{0.000000,0.000000,0.000000}%
\pgfsetstrokecolor{currentstroke}%
\pgfsetstrokeopacity{0.000000}%
\pgfsetdash{}{0pt}%
\pgfpathmoveto{\pgfqpoint{1.460122in}{0.499444in}}%
\pgfpathlineto{\pgfqpoint{1.520610in}{0.499444in}}%
\pgfpathlineto{\pgfqpoint{1.520610in}{0.539811in}}%
\pgfpathlineto{\pgfqpoint{1.460122in}{0.539811in}}%
\pgfpathlineto{\pgfqpoint{1.460122in}{0.499444in}}%
\pgfpathclose%
\pgfusepath{fill}%
\end{pgfscope}%
\begin{pgfscope}%
\pgfpathrectangle{\pgfqpoint{0.515000in}{0.499444in}}{\pgfqpoint{1.550000in}{1.155000in}}%
\pgfusepath{clip}%
\pgfsetbuttcap%
\pgfsetmiterjoin%
\definecolor{currentfill}{rgb}{0.000000,0.000000,0.000000}%
\pgfsetfillcolor{currentfill}%
\pgfsetlinewidth{0.000000pt}%
\definecolor{currentstroke}{rgb}{0.000000,0.000000,0.000000}%
\pgfsetstrokecolor{currentstroke}%
\pgfsetstrokeopacity{0.000000}%
\pgfsetdash{}{0pt}%
\pgfpathmoveto{\pgfqpoint{1.611342in}{0.499444in}}%
\pgfpathlineto{\pgfqpoint{1.671830in}{0.499444in}}%
\pgfpathlineto{\pgfqpoint{1.671830in}{0.620149in}}%
\pgfpathlineto{\pgfqpoint{1.611342in}{0.620149in}}%
\pgfpathlineto{\pgfqpoint{1.611342in}{0.499444in}}%
\pgfpathclose%
\pgfusepath{fill}%
\end{pgfscope}%
\begin{pgfscope}%
\pgfpathrectangle{\pgfqpoint{0.515000in}{0.499444in}}{\pgfqpoint{1.550000in}{1.155000in}}%
\pgfusepath{clip}%
\pgfsetbuttcap%
\pgfsetmiterjoin%
\definecolor{currentfill}{rgb}{0.000000,0.000000,0.000000}%
\pgfsetfillcolor{currentfill}%
\pgfsetlinewidth{0.000000pt}%
\definecolor{currentstroke}{rgb}{0.000000,0.000000,0.000000}%
\pgfsetstrokecolor{currentstroke}%
\pgfsetstrokeopacity{0.000000}%
\pgfsetdash{}{0pt}%
\pgfpathmoveto{\pgfqpoint{1.762561in}{0.499444in}}%
\pgfpathlineto{\pgfqpoint{1.823049in}{0.499444in}}%
\pgfpathlineto{\pgfqpoint{1.823049in}{0.686562in}}%
\pgfpathlineto{\pgfqpoint{1.762561in}{0.686562in}}%
\pgfpathlineto{\pgfqpoint{1.762561in}{0.499444in}}%
\pgfpathclose%
\pgfusepath{fill}%
\end{pgfscope}%
\begin{pgfscope}%
\pgfpathrectangle{\pgfqpoint{0.515000in}{0.499444in}}{\pgfqpoint{1.550000in}{1.155000in}}%
\pgfusepath{clip}%
\pgfsetbuttcap%
\pgfsetmiterjoin%
\definecolor{currentfill}{rgb}{0.000000,0.000000,0.000000}%
\pgfsetfillcolor{currentfill}%
\pgfsetlinewidth{0.000000pt}%
\definecolor{currentstroke}{rgb}{0.000000,0.000000,0.000000}%
\pgfsetstrokecolor{currentstroke}%
\pgfsetstrokeopacity{0.000000}%
\pgfsetdash{}{0pt}%
\pgfpathmoveto{\pgfqpoint{1.913781in}{0.499444in}}%
\pgfpathlineto{\pgfqpoint{1.974269in}{0.499444in}}%
\pgfpathlineto{\pgfqpoint{1.974269in}{0.577718in}}%
\pgfpathlineto{\pgfqpoint{1.913781in}{0.577718in}}%
\pgfpathlineto{\pgfqpoint{1.913781in}{0.499444in}}%
\pgfpathclose%
\pgfusepath{fill}%
\end{pgfscope}%
\begin{pgfscope}%
\pgfsetbuttcap%
\pgfsetroundjoin%
\definecolor{currentfill}{rgb}{0.000000,0.000000,0.000000}%
\pgfsetfillcolor{currentfill}%
\pgfsetlinewidth{0.803000pt}%
\definecolor{currentstroke}{rgb}{0.000000,0.000000,0.000000}%
\pgfsetstrokecolor{currentstroke}%
\pgfsetdash{}{0pt}%
\pgfsys@defobject{currentmarker}{\pgfqpoint{0.000000in}{-0.048611in}}{\pgfqpoint{0.000000in}{0.000000in}}{%
\pgfpathmoveto{\pgfqpoint{0.000000in}{0.000000in}}%
\pgfpathlineto{\pgfqpoint{0.000000in}{-0.048611in}}%
\pgfusepath{stroke,fill}%
}%
\begin{pgfscope}%
\pgfsys@transformshift{0.552805in}{0.499444in}%
\pgfsys@useobject{currentmarker}{}%
\end{pgfscope}%
\end{pgfscope}%
\begin{pgfscope}%
\definecolor{textcolor}{rgb}{0.000000,0.000000,0.000000}%
\pgfsetstrokecolor{textcolor}%
\pgfsetfillcolor{textcolor}%
\pgftext[x=0.552805in,y=0.402222in,,top]{\color{textcolor}\rmfamily\fontsize{10.000000}{12.000000}\selectfont 0.0}%
\end{pgfscope}%
\begin{pgfscope}%
\pgfsetbuttcap%
\pgfsetroundjoin%
\definecolor{currentfill}{rgb}{0.000000,0.000000,0.000000}%
\pgfsetfillcolor{currentfill}%
\pgfsetlinewidth{0.803000pt}%
\definecolor{currentstroke}{rgb}{0.000000,0.000000,0.000000}%
\pgfsetstrokecolor{currentstroke}%
\pgfsetdash{}{0pt}%
\pgfsys@defobject{currentmarker}{\pgfqpoint{0.000000in}{-0.048611in}}{\pgfqpoint{0.000000in}{0.000000in}}{%
\pgfpathmoveto{\pgfqpoint{0.000000in}{0.000000in}}%
\pgfpathlineto{\pgfqpoint{0.000000in}{-0.048611in}}%
\pgfusepath{stroke,fill}%
}%
\begin{pgfscope}%
\pgfsys@transformshift{0.930854in}{0.499444in}%
\pgfsys@useobject{currentmarker}{}%
\end{pgfscope}%
\end{pgfscope}%
\begin{pgfscope}%
\definecolor{textcolor}{rgb}{0.000000,0.000000,0.000000}%
\pgfsetstrokecolor{textcolor}%
\pgfsetfillcolor{textcolor}%
\pgftext[x=0.930854in,y=0.402222in,,top]{\color{textcolor}\rmfamily\fontsize{10.000000}{12.000000}\selectfont 0.25}%
\end{pgfscope}%
\begin{pgfscope}%
\pgfsetbuttcap%
\pgfsetroundjoin%
\definecolor{currentfill}{rgb}{0.000000,0.000000,0.000000}%
\pgfsetfillcolor{currentfill}%
\pgfsetlinewidth{0.803000pt}%
\definecolor{currentstroke}{rgb}{0.000000,0.000000,0.000000}%
\pgfsetstrokecolor{currentstroke}%
\pgfsetdash{}{0pt}%
\pgfsys@defobject{currentmarker}{\pgfqpoint{0.000000in}{-0.048611in}}{\pgfqpoint{0.000000in}{0.000000in}}{%
\pgfpathmoveto{\pgfqpoint{0.000000in}{0.000000in}}%
\pgfpathlineto{\pgfqpoint{0.000000in}{-0.048611in}}%
\pgfusepath{stroke,fill}%
}%
\begin{pgfscope}%
\pgfsys@transformshift{1.308903in}{0.499444in}%
\pgfsys@useobject{currentmarker}{}%
\end{pgfscope}%
\end{pgfscope}%
\begin{pgfscope}%
\definecolor{textcolor}{rgb}{0.000000,0.000000,0.000000}%
\pgfsetstrokecolor{textcolor}%
\pgfsetfillcolor{textcolor}%
\pgftext[x=1.308903in,y=0.402222in,,top]{\color{textcolor}\rmfamily\fontsize{10.000000}{12.000000}\selectfont 0.5}%
\end{pgfscope}%
\begin{pgfscope}%
\pgfsetbuttcap%
\pgfsetroundjoin%
\definecolor{currentfill}{rgb}{0.000000,0.000000,0.000000}%
\pgfsetfillcolor{currentfill}%
\pgfsetlinewidth{0.803000pt}%
\definecolor{currentstroke}{rgb}{0.000000,0.000000,0.000000}%
\pgfsetstrokecolor{currentstroke}%
\pgfsetdash{}{0pt}%
\pgfsys@defobject{currentmarker}{\pgfqpoint{0.000000in}{-0.048611in}}{\pgfqpoint{0.000000in}{0.000000in}}{%
\pgfpathmoveto{\pgfqpoint{0.000000in}{0.000000in}}%
\pgfpathlineto{\pgfqpoint{0.000000in}{-0.048611in}}%
\pgfusepath{stroke,fill}%
}%
\begin{pgfscope}%
\pgfsys@transformshift{1.686951in}{0.499444in}%
\pgfsys@useobject{currentmarker}{}%
\end{pgfscope}%
\end{pgfscope}%
\begin{pgfscope}%
\definecolor{textcolor}{rgb}{0.000000,0.000000,0.000000}%
\pgfsetstrokecolor{textcolor}%
\pgfsetfillcolor{textcolor}%
\pgftext[x=1.686951in,y=0.402222in,,top]{\color{textcolor}\rmfamily\fontsize{10.000000}{12.000000}\selectfont 0.75}%
\end{pgfscope}%
\begin{pgfscope}%
\pgfsetbuttcap%
\pgfsetroundjoin%
\definecolor{currentfill}{rgb}{0.000000,0.000000,0.000000}%
\pgfsetfillcolor{currentfill}%
\pgfsetlinewidth{0.803000pt}%
\definecolor{currentstroke}{rgb}{0.000000,0.000000,0.000000}%
\pgfsetstrokecolor{currentstroke}%
\pgfsetdash{}{0pt}%
\pgfsys@defobject{currentmarker}{\pgfqpoint{0.000000in}{-0.048611in}}{\pgfqpoint{0.000000in}{0.000000in}}{%
\pgfpathmoveto{\pgfqpoint{0.000000in}{0.000000in}}%
\pgfpathlineto{\pgfqpoint{0.000000in}{-0.048611in}}%
\pgfusepath{stroke,fill}%
}%
\begin{pgfscope}%
\pgfsys@transformshift{2.065000in}{0.499444in}%
\pgfsys@useobject{currentmarker}{}%
\end{pgfscope}%
\end{pgfscope}%
\begin{pgfscope}%
\definecolor{textcolor}{rgb}{0.000000,0.000000,0.000000}%
\pgfsetstrokecolor{textcolor}%
\pgfsetfillcolor{textcolor}%
\pgftext[x=2.065000in,y=0.402222in,,top]{\color{textcolor}\rmfamily\fontsize{10.000000}{12.000000}\selectfont 1.0}%
\end{pgfscope}%
\begin{pgfscope}%
\definecolor{textcolor}{rgb}{0.000000,0.000000,0.000000}%
\pgfsetstrokecolor{textcolor}%
\pgfsetfillcolor{textcolor}%
\pgftext[x=1.290000in,y=0.223333in,,top]{\color{textcolor}\rmfamily\fontsize{10.000000}{12.000000}\selectfont \(\displaystyle p\)}%
\end{pgfscope}%
\begin{pgfscope}%
\pgfsetbuttcap%
\pgfsetroundjoin%
\definecolor{currentfill}{rgb}{0.000000,0.000000,0.000000}%
\pgfsetfillcolor{currentfill}%
\pgfsetlinewidth{0.803000pt}%
\definecolor{currentstroke}{rgb}{0.000000,0.000000,0.000000}%
\pgfsetstrokecolor{currentstroke}%
\pgfsetdash{}{0pt}%
\pgfsys@defobject{currentmarker}{\pgfqpoint{-0.048611in}{0.000000in}}{\pgfqpoint{-0.000000in}{0.000000in}}{%
\pgfpathmoveto{\pgfqpoint{-0.000000in}{0.000000in}}%
\pgfpathlineto{\pgfqpoint{-0.048611in}{0.000000in}}%
\pgfusepath{stroke,fill}%
}%
\begin{pgfscope}%
\pgfsys@transformshift{0.515000in}{0.499444in}%
\pgfsys@useobject{currentmarker}{}%
\end{pgfscope}%
\end{pgfscope}%
\begin{pgfscope}%
\definecolor{textcolor}{rgb}{0.000000,0.000000,0.000000}%
\pgfsetstrokecolor{textcolor}%
\pgfsetfillcolor{textcolor}%
\pgftext[x=0.348333in, y=0.451250in, left, base]{\color{textcolor}\rmfamily\fontsize{10.000000}{12.000000}\selectfont \(\displaystyle {0}\)}%
\end{pgfscope}%
\begin{pgfscope}%
\pgfsetbuttcap%
\pgfsetroundjoin%
\definecolor{currentfill}{rgb}{0.000000,0.000000,0.000000}%
\pgfsetfillcolor{currentfill}%
\pgfsetlinewidth{0.803000pt}%
\definecolor{currentstroke}{rgb}{0.000000,0.000000,0.000000}%
\pgfsetstrokecolor{currentstroke}%
\pgfsetdash{}{0pt}%
\pgfsys@defobject{currentmarker}{\pgfqpoint{-0.048611in}{0.000000in}}{\pgfqpoint{-0.000000in}{0.000000in}}{%
\pgfpathmoveto{\pgfqpoint{-0.000000in}{0.000000in}}%
\pgfpathlineto{\pgfqpoint{-0.048611in}{0.000000in}}%
\pgfusepath{stroke,fill}%
}%
\begin{pgfscope}%
\pgfsys@transformshift{0.515000in}{0.792144in}%
\pgfsys@useobject{currentmarker}{}%
\end{pgfscope}%
\end{pgfscope}%
\begin{pgfscope}%
\definecolor{textcolor}{rgb}{0.000000,0.000000,0.000000}%
\pgfsetstrokecolor{textcolor}%
\pgfsetfillcolor{textcolor}%
\pgftext[x=0.278889in, y=0.743949in, left, base]{\color{textcolor}\rmfamily\fontsize{10.000000}{12.000000}\selectfont \(\displaystyle {10}\)}%
\end{pgfscope}%
\begin{pgfscope}%
\pgfsetbuttcap%
\pgfsetroundjoin%
\definecolor{currentfill}{rgb}{0.000000,0.000000,0.000000}%
\pgfsetfillcolor{currentfill}%
\pgfsetlinewidth{0.803000pt}%
\definecolor{currentstroke}{rgb}{0.000000,0.000000,0.000000}%
\pgfsetstrokecolor{currentstroke}%
\pgfsetdash{}{0pt}%
\pgfsys@defobject{currentmarker}{\pgfqpoint{-0.048611in}{0.000000in}}{\pgfqpoint{-0.000000in}{0.000000in}}{%
\pgfpathmoveto{\pgfqpoint{-0.000000in}{0.000000in}}%
\pgfpathlineto{\pgfqpoint{-0.048611in}{0.000000in}}%
\pgfusepath{stroke,fill}%
}%
\begin{pgfscope}%
\pgfsys@transformshift{0.515000in}{1.084843in}%
\pgfsys@useobject{currentmarker}{}%
\end{pgfscope}%
\end{pgfscope}%
\begin{pgfscope}%
\definecolor{textcolor}{rgb}{0.000000,0.000000,0.000000}%
\pgfsetstrokecolor{textcolor}%
\pgfsetfillcolor{textcolor}%
\pgftext[x=0.278889in, y=1.036648in, left, base]{\color{textcolor}\rmfamily\fontsize{10.000000}{12.000000}\selectfont \(\displaystyle {20}\)}%
\end{pgfscope}%
\begin{pgfscope}%
\pgfsetbuttcap%
\pgfsetroundjoin%
\definecolor{currentfill}{rgb}{0.000000,0.000000,0.000000}%
\pgfsetfillcolor{currentfill}%
\pgfsetlinewidth{0.803000pt}%
\definecolor{currentstroke}{rgb}{0.000000,0.000000,0.000000}%
\pgfsetstrokecolor{currentstroke}%
\pgfsetdash{}{0pt}%
\pgfsys@defobject{currentmarker}{\pgfqpoint{-0.048611in}{0.000000in}}{\pgfqpoint{-0.000000in}{0.000000in}}{%
\pgfpathmoveto{\pgfqpoint{-0.000000in}{0.000000in}}%
\pgfpathlineto{\pgfqpoint{-0.048611in}{0.000000in}}%
\pgfusepath{stroke,fill}%
}%
\begin{pgfscope}%
\pgfsys@transformshift{0.515000in}{1.377542in}%
\pgfsys@useobject{currentmarker}{}%
\end{pgfscope}%
\end{pgfscope}%
\begin{pgfscope}%
\definecolor{textcolor}{rgb}{0.000000,0.000000,0.000000}%
\pgfsetstrokecolor{textcolor}%
\pgfsetfillcolor{textcolor}%
\pgftext[x=0.278889in, y=1.329348in, left, base]{\color{textcolor}\rmfamily\fontsize{10.000000}{12.000000}\selectfont \(\displaystyle {30}\)}%
\end{pgfscope}%
\begin{pgfscope}%
\definecolor{textcolor}{rgb}{0.000000,0.000000,0.000000}%
\pgfsetstrokecolor{textcolor}%
\pgfsetfillcolor{textcolor}%
\pgftext[x=0.223333in,y=1.076944in,,bottom,rotate=90.000000]{\color{textcolor}\rmfamily\fontsize{10.000000}{12.000000}\selectfont Percent of Data Set}%
\end{pgfscope}%
\begin{pgfscope}%
\pgfsetrectcap%
\pgfsetmiterjoin%
\pgfsetlinewidth{0.803000pt}%
\definecolor{currentstroke}{rgb}{0.000000,0.000000,0.000000}%
\pgfsetstrokecolor{currentstroke}%
\pgfsetdash{}{0pt}%
\pgfpathmoveto{\pgfqpoint{0.515000in}{0.499444in}}%
\pgfpathlineto{\pgfqpoint{0.515000in}{1.654444in}}%
\pgfusepath{stroke}%
\end{pgfscope}%
\begin{pgfscope}%
\pgfsetrectcap%
\pgfsetmiterjoin%
\pgfsetlinewidth{0.803000pt}%
\definecolor{currentstroke}{rgb}{0.000000,0.000000,0.000000}%
\pgfsetstrokecolor{currentstroke}%
\pgfsetdash{}{0pt}%
\pgfpathmoveto{\pgfqpoint{2.065000in}{0.499444in}}%
\pgfpathlineto{\pgfqpoint{2.065000in}{1.654444in}}%
\pgfusepath{stroke}%
\end{pgfscope}%
\begin{pgfscope}%
\pgfsetrectcap%
\pgfsetmiterjoin%
\pgfsetlinewidth{0.803000pt}%
\definecolor{currentstroke}{rgb}{0.000000,0.000000,0.000000}%
\pgfsetstrokecolor{currentstroke}%
\pgfsetdash{}{0pt}%
\pgfpathmoveto{\pgfqpoint{0.515000in}{0.499444in}}%
\pgfpathlineto{\pgfqpoint{2.065000in}{0.499444in}}%
\pgfusepath{stroke}%
\end{pgfscope}%
\begin{pgfscope}%
\pgfsetrectcap%
\pgfsetmiterjoin%
\pgfsetlinewidth{0.803000pt}%
\definecolor{currentstroke}{rgb}{0.000000,0.000000,0.000000}%
\pgfsetstrokecolor{currentstroke}%
\pgfsetdash{}{0pt}%
\pgfpathmoveto{\pgfqpoint{0.515000in}{1.654444in}}%
\pgfpathlineto{\pgfqpoint{2.065000in}{1.654444in}}%
\pgfusepath{stroke}%
\end{pgfscope}%
\begin{pgfscope}%
\definecolor{textcolor}{rgb}{0.000000,0.000000,0.000000}%
\pgfsetstrokecolor{textcolor}%
\pgfsetfillcolor{textcolor}%
\pgftext[x=1.290000in,y=1.737778in,,base]{\color{textcolor}\rmfamily\fontsize{12.000000}{14.400000}\selectfont Probability Distribution}%
\end{pgfscope}%
\begin{pgfscope}%
\pgfsetbuttcap%
\pgfsetmiterjoin%
\definecolor{currentfill}{rgb}{1.000000,1.000000,1.000000}%
\pgfsetfillcolor{currentfill}%
\pgfsetfillopacity{0.800000}%
\pgfsetlinewidth{1.003750pt}%
\definecolor{currentstroke}{rgb}{0.800000,0.800000,0.800000}%
\pgfsetstrokecolor{currentstroke}%
\pgfsetstrokeopacity{0.800000}%
\pgfsetdash{}{0pt}%
\pgfpathmoveto{\pgfqpoint{0.612223in}{1.154445in}}%
\pgfpathlineto{\pgfqpoint{1.291945in}{1.154445in}}%
\pgfpathquadraticcurveto{\pgfqpoint{1.319722in}{1.154445in}}{\pgfqpoint{1.319722in}{1.182222in}}%
\pgfpathlineto{\pgfqpoint{1.319722in}{1.557222in}}%
\pgfpathquadraticcurveto{\pgfqpoint{1.319722in}{1.585000in}}{\pgfqpoint{1.291945in}{1.585000in}}%
\pgfpathlineto{\pgfqpoint{0.612223in}{1.585000in}}%
\pgfpathquadraticcurveto{\pgfqpoint{0.584445in}{1.585000in}}{\pgfqpoint{0.584445in}{1.557222in}}%
\pgfpathlineto{\pgfqpoint{0.584445in}{1.182222in}}%
\pgfpathquadraticcurveto{\pgfqpoint{0.584445in}{1.154445in}}{\pgfqpoint{0.612223in}{1.154445in}}%
\pgfpathlineto{\pgfqpoint{0.612223in}{1.154445in}}%
\pgfpathclose%
\pgfusepath{stroke,fill}%
\end{pgfscope}%
\begin{pgfscope}%
\pgfsetbuttcap%
\pgfsetmiterjoin%
\pgfsetlinewidth{1.003750pt}%
\definecolor{currentstroke}{rgb}{0.000000,0.000000,0.000000}%
\pgfsetstrokecolor{currentstroke}%
\pgfsetdash{}{0pt}%
\pgfpathmoveto{\pgfqpoint{0.640000in}{1.432222in}}%
\pgfpathlineto{\pgfqpoint{0.917778in}{1.432222in}}%
\pgfpathlineto{\pgfqpoint{0.917778in}{1.529444in}}%
\pgfpathlineto{\pgfqpoint{0.640000in}{1.529444in}}%
\pgfpathlineto{\pgfqpoint{0.640000in}{1.432222in}}%
\pgfpathclose%
\pgfusepath{stroke}%
\end{pgfscope}%
\begin{pgfscope}%
\definecolor{textcolor}{rgb}{0.000000,0.000000,0.000000}%
\pgfsetstrokecolor{textcolor}%
\pgfsetfillcolor{textcolor}%
\pgftext[x=1.028889in,y=1.432222in,left,base]{\color{textcolor}\rmfamily\fontsize{10.000000}{12.000000}\selectfont Neg}%
\end{pgfscope}%
\begin{pgfscope}%
\pgfsetbuttcap%
\pgfsetmiterjoin%
\definecolor{currentfill}{rgb}{0.000000,0.000000,0.000000}%
\pgfsetfillcolor{currentfill}%
\pgfsetlinewidth{0.000000pt}%
\definecolor{currentstroke}{rgb}{0.000000,0.000000,0.000000}%
\pgfsetstrokecolor{currentstroke}%
\pgfsetstrokeopacity{0.000000}%
\pgfsetdash{}{0pt}%
\pgfpathmoveto{\pgfqpoint{0.640000in}{1.236944in}}%
\pgfpathlineto{\pgfqpoint{0.917778in}{1.236944in}}%
\pgfpathlineto{\pgfqpoint{0.917778in}{1.334167in}}%
\pgfpathlineto{\pgfqpoint{0.640000in}{1.334167in}}%
\pgfpathlineto{\pgfqpoint{0.640000in}{1.236944in}}%
\pgfpathclose%
\pgfusepath{fill}%
\end{pgfscope}%
\begin{pgfscope}%
\definecolor{textcolor}{rgb}{0.000000,0.000000,0.000000}%
\pgfsetstrokecolor{textcolor}%
\pgfsetfillcolor{textcolor}%
\pgftext[x=1.028889in,y=1.236944in,left,base]{\color{textcolor}\rmfamily\fontsize{10.000000}{12.000000}\selectfont Pos}%
\end{pgfscope}%
\end{pgfpicture}%
\makeatother%
\endgroup%

%  &
%  \vspace{0pt} %% Creator: Matplotlib, PGF backend
%%
%% To include the figure in your LaTeX document, write
%%   \input{<filename>.pgf}
%%
%% Make sure the required packages are loaded in your preamble
%%   \usepackage{pgf}
%%
%% Also ensure that all the required font packages are loaded; for instance,
%% the lmodern package is sometimes necessary when using math font.
%%   \usepackage{lmodern}
%%
%% Figures using additional raster images can only be included by \input if
%% they are in the same directory as the main LaTeX file. For loading figures
%% from other directories you can use the `import` package
%%   \usepackage{import}
%%
%% and then include the figures with
%%   \import{<path to file>}{<filename>.pgf}
%%
%% Matplotlib used the following preamble
%%   
%%   \usepackage{fontspec}
%%   \makeatletter\@ifpackageloaded{underscore}{}{\usepackage[strings]{underscore}}\makeatother
%%
\begingroup%
\makeatletter%
\begin{pgfpicture}%
\pgfpathrectangle{\pgfpointorigin}{\pgfqpoint{2.221861in}{1.754444in}}%
\pgfusepath{use as bounding box, clip}%
\begin{pgfscope}%
\pgfsetbuttcap%
\pgfsetmiterjoin%
\definecolor{currentfill}{rgb}{1.000000,1.000000,1.000000}%
\pgfsetfillcolor{currentfill}%
\pgfsetlinewidth{0.000000pt}%
\definecolor{currentstroke}{rgb}{1.000000,1.000000,1.000000}%
\pgfsetstrokecolor{currentstroke}%
\pgfsetdash{}{0pt}%
\pgfpathmoveto{\pgfqpoint{0.000000in}{0.000000in}}%
\pgfpathlineto{\pgfqpoint{2.221861in}{0.000000in}}%
\pgfpathlineto{\pgfqpoint{2.221861in}{1.754444in}}%
\pgfpathlineto{\pgfqpoint{0.000000in}{1.754444in}}%
\pgfpathlineto{\pgfqpoint{0.000000in}{0.000000in}}%
\pgfpathclose%
\pgfusepath{fill}%
\end{pgfscope}%
\begin{pgfscope}%
\pgfsetbuttcap%
\pgfsetmiterjoin%
\definecolor{currentfill}{rgb}{1.000000,1.000000,1.000000}%
\pgfsetfillcolor{currentfill}%
\pgfsetlinewidth{0.000000pt}%
\definecolor{currentstroke}{rgb}{0.000000,0.000000,0.000000}%
\pgfsetstrokecolor{currentstroke}%
\pgfsetstrokeopacity{0.000000}%
\pgfsetdash{}{0pt}%
\pgfpathmoveto{\pgfqpoint{0.553581in}{0.499444in}}%
\pgfpathlineto{\pgfqpoint{2.103581in}{0.499444in}}%
\pgfpathlineto{\pgfqpoint{2.103581in}{1.654444in}}%
\pgfpathlineto{\pgfqpoint{0.553581in}{1.654444in}}%
\pgfpathlineto{\pgfqpoint{0.553581in}{0.499444in}}%
\pgfpathclose%
\pgfusepath{fill}%
\end{pgfscope}%
\begin{pgfscope}%
\pgfsetbuttcap%
\pgfsetroundjoin%
\definecolor{currentfill}{rgb}{0.000000,0.000000,0.000000}%
\pgfsetfillcolor{currentfill}%
\pgfsetlinewidth{0.803000pt}%
\definecolor{currentstroke}{rgb}{0.000000,0.000000,0.000000}%
\pgfsetstrokecolor{currentstroke}%
\pgfsetdash{}{0pt}%
\pgfsys@defobject{currentmarker}{\pgfqpoint{0.000000in}{-0.048611in}}{\pgfqpoint{0.000000in}{0.000000in}}{%
\pgfpathmoveto{\pgfqpoint{0.000000in}{0.000000in}}%
\pgfpathlineto{\pgfqpoint{0.000000in}{-0.048611in}}%
\pgfusepath{stroke,fill}%
}%
\begin{pgfscope}%
\pgfsys@transformshift{0.624035in}{0.499444in}%
\pgfsys@useobject{currentmarker}{}%
\end{pgfscope}%
\end{pgfscope}%
\begin{pgfscope}%
\definecolor{textcolor}{rgb}{0.000000,0.000000,0.000000}%
\pgfsetstrokecolor{textcolor}%
\pgfsetfillcolor{textcolor}%
\pgftext[x=0.624035in,y=0.402222in,,top]{\color{textcolor}\rmfamily\fontsize{10.000000}{12.000000}\selectfont \(\displaystyle {0.0}\)}%
\end{pgfscope}%
\begin{pgfscope}%
\pgfsetbuttcap%
\pgfsetroundjoin%
\definecolor{currentfill}{rgb}{0.000000,0.000000,0.000000}%
\pgfsetfillcolor{currentfill}%
\pgfsetlinewidth{0.803000pt}%
\definecolor{currentstroke}{rgb}{0.000000,0.000000,0.000000}%
\pgfsetstrokecolor{currentstroke}%
\pgfsetdash{}{0pt}%
\pgfsys@defobject{currentmarker}{\pgfqpoint{0.000000in}{-0.048611in}}{\pgfqpoint{0.000000in}{0.000000in}}{%
\pgfpathmoveto{\pgfqpoint{0.000000in}{0.000000in}}%
\pgfpathlineto{\pgfqpoint{0.000000in}{-0.048611in}}%
\pgfusepath{stroke,fill}%
}%
\begin{pgfscope}%
\pgfsys@transformshift{1.328581in}{0.499444in}%
\pgfsys@useobject{currentmarker}{}%
\end{pgfscope}%
\end{pgfscope}%
\begin{pgfscope}%
\definecolor{textcolor}{rgb}{0.000000,0.000000,0.000000}%
\pgfsetstrokecolor{textcolor}%
\pgfsetfillcolor{textcolor}%
\pgftext[x=1.328581in,y=0.402222in,,top]{\color{textcolor}\rmfamily\fontsize{10.000000}{12.000000}\selectfont \(\displaystyle {0.5}\)}%
\end{pgfscope}%
\begin{pgfscope}%
\pgfsetbuttcap%
\pgfsetroundjoin%
\definecolor{currentfill}{rgb}{0.000000,0.000000,0.000000}%
\pgfsetfillcolor{currentfill}%
\pgfsetlinewidth{0.803000pt}%
\definecolor{currentstroke}{rgb}{0.000000,0.000000,0.000000}%
\pgfsetstrokecolor{currentstroke}%
\pgfsetdash{}{0pt}%
\pgfsys@defobject{currentmarker}{\pgfqpoint{0.000000in}{-0.048611in}}{\pgfqpoint{0.000000in}{0.000000in}}{%
\pgfpathmoveto{\pgfqpoint{0.000000in}{0.000000in}}%
\pgfpathlineto{\pgfqpoint{0.000000in}{-0.048611in}}%
\pgfusepath{stroke,fill}%
}%
\begin{pgfscope}%
\pgfsys@transformshift{2.033126in}{0.499444in}%
\pgfsys@useobject{currentmarker}{}%
\end{pgfscope}%
\end{pgfscope}%
\begin{pgfscope}%
\definecolor{textcolor}{rgb}{0.000000,0.000000,0.000000}%
\pgfsetstrokecolor{textcolor}%
\pgfsetfillcolor{textcolor}%
\pgftext[x=2.033126in,y=0.402222in,,top]{\color{textcolor}\rmfamily\fontsize{10.000000}{12.000000}\selectfont \(\displaystyle {1.0}\)}%
\end{pgfscope}%
\begin{pgfscope}%
\definecolor{textcolor}{rgb}{0.000000,0.000000,0.000000}%
\pgfsetstrokecolor{textcolor}%
\pgfsetfillcolor{textcolor}%
\pgftext[x=1.328581in,y=0.223333in,,top]{\color{textcolor}\rmfamily\fontsize{10.000000}{12.000000}\selectfont False positive rate}%
\end{pgfscope}%
\begin{pgfscope}%
\pgfsetbuttcap%
\pgfsetroundjoin%
\definecolor{currentfill}{rgb}{0.000000,0.000000,0.000000}%
\pgfsetfillcolor{currentfill}%
\pgfsetlinewidth{0.803000pt}%
\definecolor{currentstroke}{rgb}{0.000000,0.000000,0.000000}%
\pgfsetstrokecolor{currentstroke}%
\pgfsetdash{}{0pt}%
\pgfsys@defobject{currentmarker}{\pgfqpoint{-0.048611in}{0.000000in}}{\pgfqpoint{-0.000000in}{0.000000in}}{%
\pgfpathmoveto{\pgfqpoint{-0.000000in}{0.000000in}}%
\pgfpathlineto{\pgfqpoint{-0.048611in}{0.000000in}}%
\pgfusepath{stroke,fill}%
}%
\begin{pgfscope}%
\pgfsys@transformshift{0.553581in}{0.551944in}%
\pgfsys@useobject{currentmarker}{}%
\end{pgfscope}%
\end{pgfscope}%
\begin{pgfscope}%
\definecolor{textcolor}{rgb}{0.000000,0.000000,0.000000}%
\pgfsetstrokecolor{textcolor}%
\pgfsetfillcolor{textcolor}%
\pgftext[x=0.278889in, y=0.503750in, left, base]{\color{textcolor}\rmfamily\fontsize{10.000000}{12.000000}\selectfont \(\displaystyle {0.0}\)}%
\end{pgfscope}%
\begin{pgfscope}%
\pgfsetbuttcap%
\pgfsetroundjoin%
\definecolor{currentfill}{rgb}{0.000000,0.000000,0.000000}%
\pgfsetfillcolor{currentfill}%
\pgfsetlinewidth{0.803000pt}%
\definecolor{currentstroke}{rgb}{0.000000,0.000000,0.000000}%
\pgfsetstrokecolor{currentstroke}%
\pgfsetdash{}{0pt}%
\pgfsys@defobject{currentmarker}{\pgfqpoint{-0.048611in}{0.000000in}}{\pgfqpoint{-0.000000in}{0.000000in}}{%
\pgfpathmoveto{\pgfqpoint{-0.000000in}{0.000000in}}%
\pgfpathlineto{\pgfqpoint{-0.048611in}{0.000000in}}%
\pgfusepath{stroke,fill}%
}%
\begin{pgfscope}%
\pgfsys@transformshift{0.553581in}{1.076944in}%
\pgfsys@useobject{currentmarker}{}%
\end{pgfscope}%
\end{pgfscope}%
\begin{pgfscope}%
\definecolor{textcolor}{rgb}{0.000000,0.000000,0.000000}%
\pgfsetstrokecolor{textcolor}%
\pgfsetfillcolor{textcolor}%
\pgftext[x=0.278889in, y=1.028750in, left, base]{\color{textcolor}\rmfamily\fontsize{10.000000}{12.000000}\selectfont \(\displaystyle {0.5}\)}%
\end{pgfscope}%
\begin{pgfscope}%
\pgfsetbuttcap%
\pgfsetroundjoin%
\definecolor{currentfill}{rgb}{0.000000,0.000000,0.000000}%
\pgfsetfillcolor{currentfill}%
\pgfsetlinewidth{0.803000pt}%
\definecolor{currentstroke}{rgb}{0.000000,0.000000,0.000000}%
\pgfsetstrokecolor{currentstroke}%
\pgfsetdash{}{0pt}%
\pgfsys@defobject{currentmarker}{\pgfqpoint{-0.048611in}{0.000000in}}{\pgfqpoint{-0.000000in}{0.000000in}}{%
\pgfpathmoveto{\pgfqpoint{-0.000000in}{0.000000in}}%
\pgfpathlineto{\pgfqpoint{-0.048611in}{0.000000in}}%
\pgfusepath{stroke,fill}%
}%
\begin{pgfscope}%
\pgfsys@transformshift{0.553581in}{1.601944in}%
\pgfsys@useobject{currentmarker}{}%
\end{pgfscope}%
\end{pgfscope}%
\begin{pgfscope}%
\definecolor{textcolor}{rgb}{0.000000,0.000000,0.000000}%
\pgfsetstrokecolor{textcolor}%
\pgfsetfillcolor{textcolor}%
\pgftext[x=0.278889in, y=1.553750in, left, base]{\color{textcolor}\rmfamily\fontsize{10.000000}{12.000000}\selectfont \(\displaystyle {1.0}\)}%
\end{pgfscope}%
\begin{pgfscope}%
\definecolor{textcolor}{rgb}{0.000000,0.000000,0.000000}%
\pgfsetstrokecolor{textcolor}%
\pgfsetfillcolor{textcolor}%
\pgftext[x=0.223333in,y=1.076944in,,bottom,rotate=90.000000]{\color{textcolor}\rmfamily\fontsize{10.000000}{12.000000}\selectfont True positive rate}%
\end{pgfscope}%
\begin{pgfscope}%
\pgfpathrectangle{\pgfqpoint{0.553581in}{0.499444in}}{\pgfqpoint{1.550000in}{1.155000in}}%
\pgfusepath{clip}%
\pgfsetbuttcap%
\pgfsetroundjoin%
\pgfsetlinewidth{1.505625pt}%
\definecolor{currentstroke}{rgb}{0.000000,0.000000,0.000000}%
\pgfsetstrokecolor{currentstroke}%
\pgfsetdash{{5.550000pt}{2.400000pt}}{0.000000pt}%
\pgfpathmoveto{\pgfqpoint{0.624035in}{0.551944in}}%
\pgfpathlineto{\pgfqpoint{2.033126in}{1.601944in}}%
\pgfusepath{stroke}%
\end{pgfscope}%
\begin{pgfscope}%
\pgfpathrectangle{\pgfqpoint{0.553581in}{0.499444in}}{\pgfqpoint{1.550000in}{1.155000in}}%
\pgfusepath{clip}%
\pgfsetrectcap%
\pgfsetroundjoin%
\pgfsetlinewidth{1.505625pt}%
\definecolor{currentstroke}{rgb}{0.000000,0.000000,0.000000}%
\pgfsetstrokecolor{currentstroke}%
\pgfsetdash{}{0pt}%
\pgfpathmoveto{\pgfqpoint{0.624035in}{0.551944in}}%
\pgfpathlineto{\pgfqpoint{0.627493in}{0.553009in}}%
\pgfpathlineto{\pgfqpoint{0.628559in}{0.554390in}}%
\pgfpathlineto{\pgfqpoint{0.629063in}{0.555494in}}%
\pgfpathlineto{\pgfqpoint{0.630166in}{0.558452in}}%
\pgfpathlineto{\pgfqpoint{0.630531in}{0.559557in}}%
\pgfpathlineto{\pgfqpoint{0.631624in}{0.564132in}}%
\pgfpathlineto{\pgfqpoint{0.631970in}{0.565197in}}%
\pgfpathlineto{\pgfqpoint{0.633073in}{0.569338in}}%
\pgfpathlineto{\pgfqpoint{0.633306in}{0.570443in}}%
\pgfpathlineto{\pgfqpoint{0.634409in}{0.576044in}}%
\pgfpathlineto{\pgfqpoint{0.634755in}{0.577148in}}%
\pgfpathlineto{\pgfqpoint{0.635858in}{0.582236in}}%
\pgfpathlineto{\pgfqpoint{0.636035in}{0.583104in}}%
\pgfpathlineto{\pgfqpoint{0.637138in}{0.589967in}}%
\pgfpathlineto{\pgfqpoint{0.637344in}{0.591032in}}%
\pgfpathlineto{\pgfqpoint{0.638447in}{0.596790in}}%
\pgfpathlineto{\pgfqpoint{0.638559in}{0.597895in}}%
\pgfpathlineto{\pgfqpoint{0.639652in}{0.604758in}}%
\pgfpathlineto{\pgfqpoint{0.639877in}{0.605823in}}%
\pgfpathlineto{\pgfqpoint{0.640979in}{0.612449in}}%
\pgfpathlineto{\pgfqpoint{0.641138in}{0.613553in}}%
\pgfpathlineto{\pgfqpoint{0.642241in}{0.621363in}}%
\pgfpathlineto{\pgfqpoint{0.642456in}{0.622467in}}%
\pgfpathlineto{\pgfqpoint{0.643549in}{0.631026in}}%
\pgfpathlineto{\pgfqpoint{0.643802in}{0.631973in}}%
\pgfpathlineto{\pgfqpoint{0.644895in}{0.639428in}}%
\pgfpathlineto{\pgfqpoint{0.645073in}{0.640532in}}%
\pgfpathlineto{\pgfqpoint{0.646138in}{0.649131in}}%
\pgfpathlineto{\pgfqpoint{0.656213in}{0.650235in}}%
\pgfpathlineto{\pgfqpoint{0.657316in}{0.658636in}}%
\pgfpathlineto{\pgfqpoint{0.657475in}{0.659662in}}%
\pgfpathlineto{\pgfqpoint{0.658559in}{0.668812in}}%
\pgfpathlineto{\pgfqpoint{0.658746in}{0.669877in}}%
\pgfpathlineto{\pgfqpoint{0.659849in}{0.678989in}}%
\pgfpathlineto{\pgfqpoint{0.659942in}{0.680014in}}%
\pgfpathlineto{\pgfqpoint{0.661045in}{0.688415in}}%
\pgfpathlineto{\pgfqpoint{0.661176in}{0.689520in}}%
\pgfpathlineto{\pgfqpoint{0.662279in}{0.697527in}}%
\pgfpathlineto{\pgfqpoint{0.662475in}{0.698631in}}%
\pgfpathlineto{\pgfqpoint{0.663578in}{0.705691in}}%
\pgfpathlineto{\pgfqpoint{0.663821in}{0.706717in}}%
\pgfpathlineto{\pgfqpoint{0.664924in}{0.715394in}}%
\pgfpathlineto{\pgfqpoint{0.665129in}{0.716459in}}%
\pgfpathlineto{\pgfqpoint{0.666223in}{0.724269in}}%
\pgfpathlineto{\pgfqpoint{0.666372in}{0.725334in}}%
\pgfpathlineto{\pgfqpoint{0.667475in}{0.732473in}}%
\pgfpathlineto{\pgfqpoint{0.667643in}{0.733538in}}%
\pgfpathlineto{\pgfqpoint{0.668737in}{0.741189in}}%
\pgfpathlineto{\pgfqpoint{0.668942in}{0.742254in}}%
\pgfpathlineto{\pgfqpoint{0.670026in}{0.751326in}}%
\pgfpathlineto{\pgfqpoint{0.670213in}{0.752431in}}%
\pgfpathlineto{\pgfqpoint{0.671316in}{0.759964in}}%
\pgfpathlineto{\pgfqpoint{0.671606in}{0.761029in}}%
\pgfpathlineto{\pgfqpoint{0.672709in}{0.766788in}}%
\pgfpathlineto{\pgfqpoint{0.672961in}{0.767892in}}%
\pgfpathlineto{\pgfqpoint{0.674064in}{0.775978in}}%
\pgfpathlineto{\pgfqpoint{0.674241in}{0.777043in}}%
\pgfpathlineto{\pgfqpoint{0.675307in}{0.783235in}}%
\pgfpathlineto{\pgfqpoint{0.675634in}{0.784340in}}%
\pgfpathlineto{\pgfqpoint{0.676737in}{0.791124in}}%
\pgfpathlineto{\pgfqpoint{0.676877in}{0.791991in}}%
\pgfpathlineto{\pgfqpoint{0.677980in}{0.799643in}}%
\pgfpathlineto{\pgfqpoint{0.678101in}{0.800669in}}%
\pgfpathlineto{\pgfqpoint{0.679195in}{0.806861in}}%
\pgfpathlineto{\pgfqpoint{0.679466in}{0.807966in}}%
\pgfpathlineto{\pgfqpoint{0.680569in}{0.813685in}}%
\pgfpathlineto{\pgfqpoint{0.680746in}{0.814632in}}%
\pgfpathlineto{\pgfqpoint{0.681840in}{0.821652in}}%
\pgfpathlineto{\pgfqpoint{0.682017in}{0.822717in}}%
\pgfpathlineto{\pgfqpoint{0.683111in}{0.829383in}}%
\pgfpathlineto{\pgfqpoint{0.683316in}{0.830448in}}%
\pgfpathlineto{\pgfqpoint{0.684410in}{0.836522in}}%
\pgfpathlineto{\pgfqpoint{0.684578in}{0.837469in}}%
\pgfpathlineto{\pgfqpoint{0.685681in}{0.843306in}}%
\pgfpathlineto{\pgfqpoint{0.685914in}{0.844411in}}%
\pgfpathlineto{\pgfqpoint{0.686999in}{0.851116in}}%
\pgfpathlineto{\pgfqpoint{0.687195in}{0.851786in}}%
\pgfpathlineto{\pgfqpoint{0.688298in}{0.859951in}}%
\pgfpathlineto{\pgfqpoint{0.688466in}{0.860937in}}%
\pgfpathlineto{\pgfqpoint{0.689569in}{0.867603in}}%
\pgfpathlineto{\pgfqpoint{0.689765in}{0.868668in}}%
\pgfpathlineto{\pgfqpoint{0.690858in}{0.874505in}}%
\pgfpathlineto{\pgfqpoint{0.691073in}{0.875610in}}%
\pgfpathlineto{\pgfqpoint{0.692176in}{0.881684in}}%
\pgfpathlineto{\pgfqpoint{0.692410in}{0.882709in}}%
\pgfpathlineto{\pgfqpoint{0.693513in}{0.889296in}}%
\pgfpathlineto{\pgfqpoint{0.693709in}{0.890401in}}%
\pgfpathlineto{\pgfqpoint{0.694812in}{0.895804in}}%
\pgfpathlineto{\pgfqpoint{0.695176in}{0.896909in}}%
\pgfpathlineto{\pgfqpoint{0.696279in}{0.902549in}}%
\pgfpathlineto{\pgfqpoint{0.696569in}{0.903653in}}%
\pgfpathlineto{\pgfqpoint{0.697672in}{0.909846in}}%
\pgfpathlineto{\pgfqpoint{0.697915in}{0.910911in}}%
\pgfpathlineto{\pgfqpoint{0.698989in}{0.917458in}}%
\pgfpathlineto{\pgfqpoint{0.699279in}{0.918523in}}%
\pgfpathlineto{\pgfqpoint{0.700354in}{0.923454in}}%
\pgfpathlineto{\pgfqpoint{0.700569in}{0.924479in}}%
\pgfpathlineto{\pgfqpoint{0.701672in}{0.929685in}}%
\pgfpathlineto{\pgfqpoint{0.702027in}{0.930790in}}%
\pgfpathlineto{\pgfqpoint{0.703130in}{0.937180in}}%
\pgfpathlineto{\pgfqpoint{0.703335in}{0.938284in}}%
\pgfpathlineto{\pgfqpoint{0.704429in}{0.943609in}}%
\pgfpathlineto{\pgfqpoint{0.704709in}{0.944634in}}%
\pgfpathlineto{\pgfqpoint{0.705765in}{0.949446in}}%
\pgfpathlineto{\pgfqpoint{0.706214in}{0.950472in}}%
\pgfpathlineto{\pgfqpoint{0.707307in}{0.955165in}}%
\pgfpathlineto{\pgfqpoint{0.707634in}{0.956270in}}%
\pgfpathlineto{\pgfqpoint{0.708718in}{0.961200in}}%
\pgfpathlineto{\pgfqpoint{0.708989in}{0.962265in}}%
\pgfpathlineto{\pgfqpoint{0.710083in}{0.966604in}}%
\pgfpathlineto{\pgfqpoint{0.710242in}{0.967708in}}%
\pgfpathlineto{\pgfqpoint{0.711335in}{0.972165in}}%
\pgfpathlineto{\pgfqpoint{0.711532in}{0.973269in}}%
\pgfpathlineto{\pgfqpoint{0.712625in}{0.978042in}}%
\pgfpathlineto{\pgfqpoint{0.712896in}{0.979107in}}%
\pgfpathlineto{\pgfqpoint{0.713980in}{0.983801in}}%
\pgfpathlineto{\pgfqpoint{0.714251in}{0.984905in}}%
\pgfpathlineto{\pgfqpoint{0.715354in}{0.990348in}}%
\pgfpathlineto{\pgfqpoint{0.715597in}{0.991452in}}%
\pgfpathlineto{\pgfqpoint{0.716681in}{0.996067in}}%
\pgfpathlineto{\pgfqpoint{0.717074in}{0.997172in}}%
\pgfpathlineto{\pgfqpoint{0.718177in}{1.001786in}}%
\pgfpathlineto{\pgfqpoint{0.718476in}{1.002851in}}%
\pgfpathlineto{\pgfqpoint{0.719578in}{1.007506in}}%
\pgfpathlineto{\pgfqpoint{0.719905in}{1.008531in}}%
\pgfpathlineto{\pgfqpoint{0.720999in}{1.012988in}}%
\pgfpathlineto{\pgfqpoint{0.721401in}{1.014092in}}%
\pgfpathlineto{\pgfqpoint{0.722504in}{1.017840in}}%
\pgfpathlineto{\pgfqpoint{0.722765in}{1.018904in}}%
\pgfpathlineto{\pgfqpoint{0.723859in}{1.022691in}}%
\pgfpathlineto{\pgfqpoint{0.724111in}{1.023756in}}%
\pgfpathlineto{\pgfqpoint{0.725205in}{1.026990in}}%
\pgfpathlineto{\pgfqpoint{0.725606in}{1.028055in}}%
\pgfpathlineto{\pgfqpoint{0.726709in}{1.031526in}}%
\pgfpathlineto{\pgfqpoint{0.727074in}{1.032630in}}%
\pgfpathlineto{\pgfqpoint{0.728177in}{1.036023in}}%
\pgfpathlineto{\pgfqpoint{0.728392in}{1.037009in}}%
\pgfpathlineto{\pgfqpoint{0.729485in}{1.041702in}}%
\pgfpathlineto{\pgfqpoint{0.729775in}{1.042767in}}%
\pgfpathlineto{\pgfqpoint{0.730878in}{1.046435in}}%
\pgfpathlineto{\pgfqpoint{0.731214in}{1.047540in}}%
\pgfpathlineto{\pgfqpoint{0.732308in}{1.050853in}}%
\pgfpathlineto{\pgfqpoint{0.732775in}{1.051957in}}%
\pgfpathlineto{\pgfqpoint{0.733878in}{1.055152in}}%
\pgfpathlineto{\pgfqpoint{0.734093in}{1.056178in}}%
\pgfpathlineto{\pgfqpoint{0.735177in}{1.060043in}}%
\pgfpathlineto{\pgfqpoint{0.735551in}{1.061108in}}%
\pgfpathlineto{\pgfqpoint{0.736653in}{1.064618in}}%
\pgfpathlineto{\pgfqpoint{0.737046in}{1.065683in}}%
\pgfpathlineto{\pgfqpoint{0.738130in}{1.068681in}}%
\pgfpathlineto{\pgfqpoint{0.738551in}{1.069785in}}%
\pgfpathlineto{\pgfqpoint{0.739644in}{1.073887in}}%
\pgfpathlineto{\pgfqpoint{0.740027in}{1.074992in}}%
\pgfpathlineto{\pgfqpoint{0.741130in}{1.077871in}}%
\pgfpathlineto{\pgfqpoint{0.741551in}{1.078975in}}%
\pgfpathlineto{\pgfqpoint{0.742644in}{1.081973in}}%
\pgfpathlineto{\pgfqpoint{0.743233in}{1.083077in}}%
\pgfpathlineto{\pgfqpoint{0.744336in}{1.086036in}}%
\pgfpathlineto{\pgfqpoint{0.744934in}{1.087061in}}%
\pgfpathlineto{\pgfqpoint{0.746037in}{1.090769in}}%
\pgfpathlineto{\pgfqpoint{0.746448in}{1.091873in}}%
\pgfpathlineto{\pgfqpoint{0.747523in}{1.095778in}}%
\pgfpathlineto{\pgfqpoint{0.747887in}{1.096843in}}%
\pgfpathlineto{\pgfqpoint{0.748934in}{1.100551in}}%
\pgfpathlineto{\pgfqpoint{0.749280in}{1.101655in}}%
\pgfpathlineto{\pgfqpoint{0.750364in}{1.104574in}}%
\pgfpathlineto{\pgfqpoint{0.750756in}{1.105678in}}%
\pgfpathlineto{\pgfqpoint{0.751840in}{1.108360in}}%
\pgfpathlineto{\pgfqpoint{0.752298in}{1.109465in}}%
\pgfpathlineto{\pgfqpoint{0.753373in}{1.112107in}}%
\pgfpathlineto{\pgfqpoint{0.753999in}{1.113172in}}%
\pgfpathlineto{\pgfqpoint{0.755083in}{1.115696in}}%
\pgfpathlineto{\pgfqpoint{0.755569in}{1.116801in}}%
\pgfpathlineto{\pgfqpoint{0.756672in}{1.119838in}}%
\pgfpathlineto{\pgfqpoint{0.756981in}{1.120706in}}%
\pgfpathlineto{\pgfqpoint{0.758065in}{1.124532in}}%
\pgfpathlineto{\pgfqpoint{0.758551in}{1.125636in}}%
\pgfpathlineto{\pgfqpoint{0.759607in}{1.128949in}}%
\pgfpathlineto{\pgfqpoint{0.760168in}{1.130054in}}%
\pgfpathlineto{\pgfqpoint{0.761270in}{1.132854in}}%
\pgfpathlineto{\pgfqpoint{0.761999in}{1.133958in}}%
\pgfpathlineto{\pgfqpoint{0.763046in}{1.136680in}}%
\pgfpathlineto{\pgfqpoint{0.763822in}{1.137784in}}%
\pgfpathlineto{\pgfqpoint{0.764925in}{1.139875in}}%
\pgfpathlineto{\pgfqpoint{0.765355in}{1.140979in}}%
\pgfpathlineto{\pgfqpoint{0.766448in}{1.143622in}}%
\pgfpathlineto{\pgfqpoint{0.767065in}{1.144726in}}%
\pgfpathlineto{\pgfqpoint{0.768121in}{1.146896in}}%
\pgfpathlineto{\pgfqpoint{0.768672in}{1.148000in}}%
\pgfpathlineto{\pgfqpoint{0.769710in}{1.150485in}}%
\pgfpathlineto{\pgfqpoint{0.770327in}{1.151589in}}%
\pgfpathlineto{\pgfqpoint{0.771429in}{1.154587in}}%
\pgfpathlineto{\pgfqpoint{0.772000in}{1.155691in}}%
\pgfpathlineto{\pgfqpoint{0.773084in}{1.157506in}}%
\pgfpathlineto{\pgfqpoint{0.773813in}{1.158571in}}%
\pgfpathlineto{\pgfqpoint{0.774906in}{1.161016in}}%
\pgfpathlineto{\pgfqpoint{0.775476in}{1.162120in}}%
\pgfpathlineto{\pgfqpoint{0.776570in}{1.164250in}}%
\pgfpathlineto{\pgfqpoint{0.777140in}{1.165355in}}%
\pgfpathlineto{\pgfqpoint{0.778177in}{1.167642in}}%
\pgfpathlineto{\pgfqpoint{0.778850in}{1.168747in}}%
\pgfpathlineto{\pgfqpoint{0.779953in}{1.170600in}}%
\pgfpathlineto{\pgfqpoint{0.780392in}{1.171508in}}%
\pgfpathlineto{\pgfqpoint{0.781476in}{1.173874in}}%
\pgfpathlineto{\pgfqpoint{0.781953in}{1.174979in}}%
\pgfpathlineto{\pgfqpoint{0.783056in}{1.176793in}}%
\pgfpathlineto{\pgfqpoint{0.783635in}{1.177897in}}%
\pgfpathlineto{\pgfqpoint{0.784729in}{1.179396in}}%
\pgfpathlineto{\pgfqpoint{0.785374in}{1.180501in}}%
\pgfpathlineto{\pgfqpoint{0.786439in}{1.182512in}}%
\pgfpathlineto{\pgfqpoint{0.787121in}{1.183617in}}%
\pgfpathlineto{\pgfqpoint{0.788224in}{1.185589in}}%
\pgfpathlineto{\pgfqpoint{0.788934in}{1.186693in}}%
\pgfpathlineto{\pgfqpoint{0.789878in}{1.187916in}}%
\pgfpathlineto{\pgfqpoint{0.790607in}{1.188981in}}%
\pgfpathlineto{\pgfqpoint{0.791617in}{1.191189in}}%
\pgfpathlineto{\pgfqpoint{0.792271in}{1.192294in}}%
\pgfpathlineto{\pgfqpoint{0.793364in}{1.193990in}}%
\pgfpathlineto{\pgfqpoint{0.793916in}{1.195055in}}%
\pgfpathlineto{\pgfqpoint{0.795009in}{1.196790in}}%
\pgfpathlineto{\pgfqpoint{0.795551in}{1.197855in}}%
\pgfpathlineto{\pgfqpoint{0.796626in}{1.199709in}}%
\pgfpathlineto{\pgfqpoint{0.797346in}{1.200813in}}%
\pgfpathlineto{\pgfqpoint{0.798439in}{1.203101in}}%
\pgfpathlineto{\pgfqpoint{0.799196in}{1.204206in}}%
\pgfpathlineto{\pgfqpoint{0.800299in}{1.206257in}}%
\pgfpathlineto{\pgfqpoint{0.800720in}{1.207361in}}%
\pgfpathlineto{\pgfqpoint{0.801785in}{1.209294in}}%
\pgfpathlineto{\pgfqpoint{0.802467in}{1.210398in}}%
\pgfpathlineto{\pgfqpoint{0.803561in}{1.212843in}}%
\pgfpathlineto{\pgfqpoint{0.804168in}{1.213948in}}%
\pgfpathlineto{\pgfqpoint{0.805252in}{1.215723in}}%
\pgfpathlineto{\pgfqpoint{0.806103in}{1.216827in}}%
\pgfpathlineto{\pgfqpoint{0.807206in}{1.219312in}}%
\pgfpathlineto{\pgfqpoint{0.807841in}{1.220416in}}%
\pgfpathlineto{\pgfqpoint{0.808897in}{1.222073in}}%
\pgfpathlineto{\pgfqpoint{0.809421in}{1.223177in}}%
\pgfpathlineto{\pgfqpoint{0.810514in}{1.224755in}}%
\pgfpathlineto{\pgfqpoint{0.811327in}{1.225859in}}%
\pgfpathlineto{\pgfqpoint{0.812421in}{1.228029in}}%
\pgfpathlineto{\pgfqpoint{0.812888in}{1.229133in}}%
\pgfpathlineto{\pgfqpoint{0.813907in}{1.230593in}}%
\pgfpathlineto{\pgfqpoint{0.814776in}{1.231697in}}%
\pgfpathlineto{\pgfqpoint{0.815879in}{1.233354in}}%
\pgfpathlineto{\pgfqpoint{0.816505in}{1.234340in}}%
\pgfpathlineto{\pgfqpoint{0.817598in}{1.236272in}}%
\pgfpathlineto{\pgfqpoint{0.817981in}{1.237377in}}%
\pgfpathlineto{\pgfqpoint{0.819075in}{1.239191in}}%
\pgfpathlineto{\pgfqpoint{0.819748in}{1.240295in}}%
\pgfpathlineto{\pgfqpoint{0.820832in}{1.241400in}}%
\pgfpathlineto{\pgfqpoint{0.821654in}{1.242465in}}%
\pgfpathlineto{\pgfqpoint{0.822692in}{1.243806in}}%
\pgfpathlineto{\pgfqpoint{0.823486in}{1.244831in}}%
\pgfpathlineto{\pgfqpoint{0.824552in}{1.246527in}}%
\pgfpathlineto{\pgfqpoint{0.825477in}{1.247632in}}%
\pgfpathlineto{\pgfqpoint{0.826580in}{1.249880in}}%
\pgfpathlineto{\pgfqpoint{0.827103in}{1.250984in}}%
\pgfpathlineto{\pgfqpoint{0.828196in}{1.252523in}}%
\pgfpathlineto{\pgfqpoint{0.829243in}{1.253627in}}%
\pgfpathlineto{\pgfqpoint{0.830299in}{1.255402in}}%
\pgfpathlineto{\pgfqpoint{0.831253in}{1.256467in}}%
\pgfpathlineto{\pgfqpoint{0.832346in}{1.258045in}}%
\pgfpathlineto{\pgfqpoint{0.833533in}{1.259149in}}%
\pgfpathlineto{\pgfqpoint{0.834608in}{1.260490in}}%
\pgfpathlineto{\pgfqpoint{0.835542in}{1.261594in}}%
\pgfpathlineto{\pgfqpoint{0.836598in}{1.262738in}}%
\pgfpathlineto{\pgfqpoint{0.837785in}{1.263843in}}%
\pgfpathlineto{\pgfqpoint{0.838879in}{1.265815in}}%
\pgfpathlineto{\pgfqpoint{0.839683in}{1.266919in}}%
\pgfpathlineto{\pgfqpoint{0.840739in}{1.268260in}}%
\pgfpathlineto{\pgfqpoint{0.841337in}{1.269365in}}%
\pgfpathlineto{\pgfqpoint{0.842412in}{1.270863in}}%
\pgfpathlineto{\pgfqpoint{0.843458in}{1.271968in}}%
\pgfpathlineto{\pgfqpoint{0.844542in}{1.273112in}}%
\pgfpathlineto{\pgfqpoint{0.845860in}{1.274216in}}%
\pgfpathlineto{\pgfqpoint{0.846935in}{1.275439in}}%
\pgfpathlineto{\pgfqpoint{0.847748in}{1.276504in}}%
\pgfpathlineto{\pgfqpoint{0.848842in}{1.278318in}}%
\pgfpathlineto{\pgfqpoint{0.849636in}{1.279383in}}%
\pgfpathlineto{\pgfqpoint{0.850739in}{1.280448in}}%
\pgfpathlineto{\pgfqpoint{0.851935in}{1.281513in}}%
\pgfpathlineto{\pgfqpoint{0.853038in}{1.283091in}}%
\pgfpathlineto{\pgfqpoint{0.854225in}{1.284195in}}%
\pgfpathlineto{\pgfqpoint{0.855225in}{1.285418in}}%
\pgfpathlineto{\pgfqpoint{0.856094in}{1.286522in}}%
\pgfpathlineto{\pgfqpoint{0.857038in}{1.287982in}}%
\pgfpathlineto{\pgfqpoint{0.857860in}{1.289086in}}%
\pgfpathlineto{\pgfqpoint{0.858944in}{1.290506in}}%
\pgfpathlineto{\pgfqpoint{0.859954in}{1.291610in}}%
\pgfpathlineto{\pgfqpoint{0.861057in}{1.293109in}}%
\pgfpathlineto{\pgfqpoint{0.861786in}{1.294174in}}%
\pgfpathlineto{\pgfqpoint{0.862888in}{1.295515in}}%
\pgfpathlineto{\pgfqpoint{0.863524in}{1.296619in}}%
\pgfpathlineto{\pgfqpoint{0.864608in}{1.297605in}}%
\pgfpathlineto{\pgfqpoint{0.865748in}{1.298710in}}%
\pgfpathlineto{\pgfqpoint{0.866842in}{1.300011in}}%
\pgfpathlineto{\pgfqpoint{0.868122in}{1.301116in}}%
\pgfpathlineto{\pgfqpoint{0.869206in}{1.302417in}}%
\pgfpathlineto{\pgfqpoint{0.869973in}{1.303522in}}%
\pgfpathlineto{\pgfqpoint{0.871038in}{1.304508in}}%
\pgfpathlineto{\pgfqpoint{0.872786in}{1.305573in}}%
\pgfpathlineto{\pgfqpoint{0.873823in}{1.306520in}}%
\pgfpathlineto{\pgfqpoint{0.874870in}{1.307545in}}%
\pgfpathlineto{\pgfqpoint{0.875954in}{1.309004in}}%
\pgfpathlineto{\pgfqpoint{0.876935in}{1.310109in}}%
\pgfpathlineto{\pgfqpoint{0.877973in}{1.311292in}}%
\pgfpathlineto{\pgfqpoint{0.878973in}{1.312357in}}%
\pgfpathlineto{\pgfqpoint{0.880076in}{1.313580in}}%
\pgfpathlineto{\pgfqpoint{0.881216in}{1.314684in}}%
\pgfpathlineto{\pgfqpoint{0.882291in}{1.315867in}}%
\pgfpathlineto{\pgfqpoint{0.883534in}{1.316972in}}%
\pgfpathlineto{\pgfqpoint{0.884618in}{1.318037in}}%
\pgfpathlineto{\pgfqpoint{0.885870in}{1.319141in}}%
\pgfpathlineto{\pgfqpoint{0.886973in}{1.320679in}}%
\pgfpathlineto{\pgfqpoint{0.887907in}{1.321744in}}%
\pgfpathlineto{\pgfqpoint{0.889001in}{1.323125in}}%
\pgfpathlineto{\pgfqpoint{0.890870in}{1.324229in}}%
\pgfpathlineto{\pgfqpoint{0.891926in}{1.325294in}}%
\pgfpathlineto{\pgfqpoint{0.893216in}{1.326399in}}%
\pgfpathlineto{\pgfqpoint{0.894319in}{1.327306in}}%
\pgfpathlineto{\pgfqpoint{0.895823in}{1.328410in}}%
\pgfpathlineto{\pgfqpoint{0.896908in}{1.329357in}}%
\pgfpathlineto{\pgfqpoint{0.897795in}{1.330461in}}%
\pgfpathlineto{\pgfqpoint{0.898898in}{1.331368in}}%
\pgfpathlineto{\pgfqpoint{0.900590in}{1.332473in}}%
\pgfpathlineto{\pgfqpoint{0.901683in}{1.333735in}}%
\pgfpathlineto{\pgfqpoint{0.902674in}{1.334839in}}%
\pgfpathlineto{\pgfqpoint{0.903758in}{1.336023in}}%
\pgfpathlineto{\pgfqpoint{0.905085in}{1.337127in}}%
\pgfpathlineto{\pgfqpoint{0.906141in}{1.338310in}}%
\pgfpathlineto{\pgfqpoint{0.907440in}{1.339415in}}%
\pgfpathlineto{\pgfqpoint{0.908253in}{1.340046in}}%
\pgfpathlineto{\pgfqpoint{0.910067in}{1.341150in}}%
\pgfpathlineto{\pgfqpoint{0.911132in}{1.342254in}}%
\pgfpathlineto{\pgfqpoint{0.912282in}{1.343359in}}%
\pgfpathlineto{\pgfqpoint{0.913338in}{1.344187in}}%
\pgfpathlineto{\pgfqpoint{0.914908in}{1.345292in}}%
\pgfpathlineto{\pgfqpoint{0.916001in}{1.346396in}}%
\pgfpathlineto{\pgfqpoint{0.917169in}{1.347500in}}%
\pgfpathlineto{\pgfqpoint{0.918226in}{1.348486in}}%
\pgfpathlineto{\pgfqpoint{0.919627in}{1.349591in}}%
\pgfpathlineto{\pgfqpoint{0.920721in}{1.350537in}}%
\pgfpathlineto{\pgfqpoint{0.922104in}{1.351642in}}%
\pgfpathlineto{\pgfqpoint{0.923020in}{1.352983in}}%
\pgfpathlineto{\pgfqpoint{0.924515in}{1.354087in}}%
\pgfpathlineto{\pgfqpoint{0.925618in}{1.355073in}}%
\pgfpathlineto{\pgfqpoint{0.926618in}{1.356059in}}%
\pgfpathlineto{\pgfqpoint{0.927684in}{1.357045in}}%
\pgfpathlineto{\pgfqpoint{0.929188in}{1.358150in}}%
\pgfpathlineto{\pgfqpoint{0.930160in}{1.359215in}}%
\pgfpathlineto{\pgfqpoint{0.932039in}{1.360319in}}%
\pgfpathlineto{\pgfqpoint{0.933067in}{1.361147in}}%
\pgfpathlineto{\pgfqpoint{0.934871in}{1.362252in}}%
\pgfpathlineto{\pgfqpoint{0.935964in}{1.363238in}}%
\pgfpathlineto{\pgfqpoint{0.937544in}{1.364342in}}%
\pgfpathlineto{\pgfqpoint{0.938618in}{1.365289in}}%
\pgfpathlineto{\pgfqpoint{0.939843in}{1.366393in}}%
\pgfpathlineto{\pgfqpoint{0.940871in}{1.367300in}}%
\pgfpathlineto{\pgfqpoint{0.942534in}{1.368405in}}%
\pgfpathlineto{\pgfqpoint{0.943618in}{1.369194in}}%
\pgfpathlineto{\pgfqpoint{0.945656in}{1.370298in}}%
\pgfpathlineto{\pgfqpoint{0.946702in}{1.371087in}}%
\pgfpathlineto{\pgfqpoint{0.948039in}{1.372191in}}%
\pgfpathlineto{\pgfqpoint{0.949076in}{1.372822in}}%
\pgfpathlineto{\pgfqpoint{0.950908in}{1.373927in}}%
\pgfpathlineto{\pgfqpoint{0.952011in}{1.374716in}}%
\pgfpathlineto{\pgfqpoint{0.953665in}{1.375820in}}%
\pgfpathlineto{\pgfqpoint{0.954768in}{1.376727in}}%
\pgfpathlineto{\pgfqpoint{0.956544in}{1.377753in}}%
\pgfpathlineto{\pgfqpoint{0.957600in}{1.378660in}}%
\pgfpathlineto{\pgfqpoint{0.959749in}{1.379764in}}%
\pgfpathlineto{\pgfqpoint{0.960843in}{1.381026in}}%
\pgfpathlineto{\pgfqpoint{0.962749in}{1.382131in}}%
\pgfpathlineto{\pgfqpoint{0.963805in}{1.382841in}}%
\pgfpathlineto{\pgfqpoint{0.965516in}{1.383945in}}%
\pgfpathlineto{\pgfqpoint{0.966478in}{1.384695in}}%
\pgfpathlineto{\pgfqpoint{0.968376in}{1.385799in}}%
\pgfpathlineto{\pgfqpoint{0.969432in}{1.386588in}}%
\pgfpathlineto{\pgfqpoint{0.971806in}{1.387692in}}%
\pgfpathlineto{\pgfqpoint{0.972834in}{1.388402in}}%
\pgfpathlineto{\pgfqpoint{0.974637in}{1.389507in}}%
\pgfpathlineto{\pgfqpoint{0.975656in}{1.390217in}}%
\pgfpathlineto{\pgfqpoint{0.978049in}{1.391321in}}%
\pgfpathlineto{\pgfqpoint{0.978918in}{1.391755in}}%
\pgfpathlineto{\pgfqpoint{0.981011in}{1.392859in}}%
\pgfpathlineto{\pgfqpoint{0.981955in}{1.393451in}}%
\pgfpathlineto{\pgfqpoint{0.984105in}{1.394555in}}%
\pgfpathlineto{\pgfqpoint{0.985208in}{1.395226in}}%
\pgfpathlineto{\pgfqpoint{0.987245in}{1.396330in}}%
\pgfpathlineto{\pgfqpoint{0.988348in}{1.397080in}}%
\pgfpathlineto{\pgfqpoint{0.989853in}{1.398184in}}%
\pgfpathlineto{\pgfqpoint{0.990871in}{1.398815in}}%
\pgfpathlineto{\pgfqpoint{0.993329in}{1.399919in}}%
\pgfpathlineto{\pgfqpoint{0.994404in}{1.400551in}}%
\pgfpathlineto{\pgfqpoint{0.996797in}{1.401655in}}%
\pgfpathlineto{\pgfqpoint{0.997834in}{1.402207in}}%
\pgfpathlineto{\pgfqpoint{0.999955in}{1.403312in}}%
\pgfpathlineto{\pgfqpoint{1.000890in}{1.404021in}}%
\pgfpathlineto{\pgfqpoint{1.003096in}{1.405126in}}%
\pgfpathlineto{\pgfqpoint{1.004152in}{1.405718in}}%
\pgfpathlineto{\pgfqpoint{1.006600in}{1.406822in}}%
\pgfpathlineto{\pgfqpoint{1.007554in}{1.407374in}}%
\pgfpathlineto{\pgfqpoint{1.009133in}{1.408478in}}%
\pgfpathlineto{\pgfqpoint{1.010096in}{1.409188in}}%
\pgfpathlineto{\pgfqpoint{1.012217in}{1.410293in}}%
\pgfpathlineto{\pgfqpoint{1.013311in}{1.411003in}}%
\pgfpathlineto{\pgfqpoint{1.015395in}{1.412107in}}%
\pgfpathlineto{\pgfqpoint{1.016404in}{1.412620in}}%
\pgfpathlineto{\pgfqpoint{1.018853in}{1.413724in}}%
\pgfpathlineto{\pgfqpoint{1.019956in}{1.414316in}}%
\pgfpathlineto{\pgfqpoint{1.022283in}{1.415420in}}%
\pgfpathlineto{\pgfqpoint{1.023367in}{1.416170in}}%
\pgfpathlineto{\pgfqpoint{1.025788in}{1.417274in}}%
\pgfpathlineto{\pgfqpoint{1.026573in}{1.417866in}}%
\pgfpathlineto{\pgfqpoint{1.029180in}{1.418931in}}%
\pgfpathlineto{\pgfqpoint{1.030189in}{1.419246in}}%
\pgfpathlineto{\pgfqpoint{1.032545in}{1.420351in}}%
\pgfpathlineto{\pgfqpoint{1.033647in}{1.420824in}}%
\pgfpathlineto{\pgfqpoint{1.036461in}{1.421928in}}%
\pgfpathlineto{\pgfqpoint{1.037535in}{1.422441in}}%
\pgfpathlineto{\pgfqpoint{1.039984in}{1.423546in}}%
\pgfpathlineto{\pgfqpoint{1.040975in}{1.424255in}}%
\pgfpathlineto{\pgfqpoint{1.043928in}{1.425360in}}%
\pgfpathlineto{\pgfqpoint{1.045021in}{1.425991in}}%
\pgfpathlineto{\pgfqpoint{1.046984in}{1.427095in}}%
\pgfpathlineto{\pgfqpoint{1.047928in}{1.427529in}}%
\pgfpathlineto{\pgfqpoint{1.050685in}{1.428634in}}%
\pgfpathlineto{\pgfqpoint{1.051704in}{1.429344in}}%
\pgfpathlineto{\pgfqpoint{1.054685in}{1.430448in}}%
\pgfpathlineto{\pgfqpoint{1.055722in}{1.430961in}}%
\pgfpathlineto{\pgfqpoint{1.058012in}{1.432065in}}%
\pgfpathlineto{\pgfqpoint{1.059068in}{1.432420in}}%
\pgfpathlineto{\pgfqpoint{1.061367in}{1.433524in}}%
\pgfpathlineto{\pgfqpoint{1.062321in}{1.433998in}}%
\pgfpathlineto{\pgfqpoint{1.064975in}{1.435102in}}%
\pgfpathlineto{\pgfqpoint{1.065863in}{1.435260in}}%
\pgfpathlineto{\pgfqpoint{1.069237in}{1.436364in}}%
\pgfpathlineto{\pgfqpoint{1.070237in}{1.436838in}}%
\pgfpathlineto{\pgfqpoint{1.072984in}{1.437942in}}%
\pgfpathlineto{\pgfqpoint{1.073956in}{1.438336in}}%
\pgfpathlineto{\pgfqpoint{1.076517in}{1.439441in}}%
\pgfpathlineto{\pgfqpoint{1.077442in}{1.439993in}}%
\pgfpathlineto{\pgfqpoint{1.080797in}{1.441097in}}%
\pgfpathlineto{\pgfqpoint{1.081751in}{1.441531in}}%
\pgfpathlineto{\pgfqpoint{1.084583in}{1.442636in}}%
\pgfpathlineto{\pgfqpoint{1.085611in}{1.443188in}}%
\pgfpathlineto{\pgfqpoint{1.088284in}{1.444292in}}%
\pgfpathlineto{\pgfqpoint{1.089237in}{1.444687in}}%
\pgfpathlineto{\pgfqpoint{1.092676in}{1.445752in}}%
\pgfpathlineto{\pgfqpoint{1.093685in}{1.446422in}}%
\pgfpathlineto{\pgfqpoint{1.097031in}{1.447487in}}%
\pgfpathlineto{\pgfqpoint{1.098115in}{1.448079in}}%
\pgfpathlineto{\pgfqpoint{1.101471in}{1.449183in}}%
\pgfpathlineto{\pgfqpoint{1.102433in}{1.449735in}}%
\pgfpathlineto{\pgfqpoint{1.105723in}{1.450840in}}%
\pgfpathlineto{\pgfqpoint{1.106788in}{1.451195in}}%
\pgfpathlineto{\pgfqpoint{1.110293in}{1.452299in}}%
\pgfpathlineto{\pgfqpoint{1.111377in}{1.453009in}}%
\pgfpathlineto{\pgfqpoint{1.113891in}{1.454114in}}%
\pgfpathlineto{\pgfqpoint{1.114994in}{1.454587in}}%
\pgfpathlineto{\pgfqpoint{1.118265in}{1.455691in}}%
\pgfpathlineto{\pgfqpoint{1.119134in}{1.456283in}}%
\pgfpathlineto{\pgfqpoint{1.123125in}{1.457387in}}%
\pgfpathlineto{\pgfqpoint{1.124172in}{1.457742in}}%
\pgfpathlineto{\pgfqpoint{1.127284in}{1.458847in}}%
\pgfpathlineto{\pgfqpoint{1.128228in}{1.459123in}}%
\pgfpathlineto{\pgfqpoint{1.131602in}{1.460227in}}%
\pgfpathlineto{\pgfqpoint{1.132705in}{1.460819in}}%
\pgfpathlineto{\pgfqpoint{1.136443in}{1.461923in}}%
\pgfpathlineto{\pgfqpoint{1.137527in}{1.462436in}}%
\pgfpathlineto{\pgfqpoint{1.140340in}{1.463540in}}%
\pgfpathlineto{\pgfqpoint{1.141200in}{1.463895in}}%
\pgfpathlineto{\pgfqpoint{1.144434in}{1.465000in}}%
\pgfpathlineto{\pgfqpoint{1.145527in}{1.465473in}}%
\pgfpathlineto{\pgfqpoint{1.150163in}{1.466577in}}%
\pgfpathlineto{\pgfqpoint{1.151256in}{1.466932in}}%
\pgfpathlineto{\pgfqpoint{1.156312in}{1.468037in}}%
\pgfpathlineto{\pgfqpoint{1.157387in}{1.468549in}}%
\pgfpathlineto{\pgfqpoint{1.160967in}{1.469614in}}%
\pgfpathlineto{\pgfqpoint{1.162032in}{1.470246in}}%
\pgfpathlineto{\pgfqpoint{1.165013in}{1.471350in}}%
\pgfpathlineto{\pgfqpoint{1.166098in}{1.471744in}}%
\pgfpathlineto{\pgfqpoint{1.169621in}{1.472849in}}%
\pgfpathlineto{\pgfqpoint{1.170668in}{1.473283in}}%
\pgfpathlineto{\pgfqpoint{1.174985in}{1.474387in}}%
\pgfpathlineto{\pgfqpoint{1.175957in}{1.474703in}}%
\pgfpathlineto{\pgfqpoint{1.179593in}{1.475807in}}%
\pgfpathlineto{\pgfqpoint{1.180696in}{1.476122in}}%
\pgfpathlineto{\pgfqpoint{1.184098in}{1.477227in}}%
\pgfpathlineto{\pgfqpoint{1.185116in}{1.477582in}}%
\pgfpathlineto{\pgfqpoint{1.190388in}{1.478686in}}%
\pgfpathlineto{\pgfqpoint{1.191425in}{1.479120in}}%
\pgfpathlineto{\pgfqpoint{1.195453in}{1.480224in}}%
\pgfpathlineto{\pgfqpoint{1.196528in}{1.480501in}}%
\pgfpathlineto{\pgfqpoint{1.200247in}{1.481605in}}%
\pgfpathlineto{\pgfqpoint{1.201285in}{1.482039in}}%
\pgfpathlineto{\pgfqpoint{1.205911in}{1.483104in}}%
\pgfpathlineto{\pgfqpoint{1.206855in}{1.483577in}}%
\pgfpathlineto{\pgfqpoint{1.212892in}{1.484681in}}%
\pgfpathlineto{\pgfqpoint{1.213921in}{1.484958in}}%
\pgfpathlineto{\pgfqpoint{1.217874in}{1.486062in}}%
\pgfpathlineto{\pgfqpoint{1.218967in}{1.486496in}}%
\pgfpathlineto{\pgfqpoint{1.222556in}{1.487600in}}%
\pgfpathlineto{\pgfqpoint{1.223594in}{1.487837in}}%
\pgfpathlineto{\pgfqpoint{1.227566in}{1.488941in}}%
\pgfpathlineto{\pgfqpoint{1.228631in}{1.489296in}}%
\pgfpathlineto{\pgfqpoint{1.231846in}{1.490401in}}%
\pgfpathlineto{\pgfqpoint{1.232911in}{1.490598in}}%
\pgfpathlineto{\pgfqpoint{1.240323in}{1.491702in}}%
\pgfpathlineto{\pgfqpoint{1.241407in}{1.492136in}}%
\pgfpathlineto{\pgfqpoint{1.246164in}{1.493241in}}%
\pgfpathlineto{\pgfqpoint{1.246968in}{1.493477in}}%
\pgfpathlineto{\pgfqpoint{1.251827in}{1.494582in}}%
\pgfpathlineto{\pgfqpoint{1.252827in}{1.494976in}}%
\pgfpathlineto{\pgfqpoint{1.259089in}{1.496080in}}%
\pgfpathlineto{\pgfqpoint{1.259968in}{1.496356in}}%
\pgfpathlineto{\pgfqpoint{1.265295in}{1.497461in}}%
\pgfpathlineto{\pgfqpoint{1.266267in}{1.497895in}}%
\pgfpathlineto{\pgfqpoint{1.270949in}{1.498999in}}%
\pgfpathlineto{\pgfqpoint{1.271846in}{1.499472in}}%
\pgfpathlineto{\pgfqpoint{1.277641in}{1.500577in}}%
\pgfpathlineto{\pgfqpoint{1.278669in}{1.501011in}}%
\pgfpathlineto{\pgfqpoint{1.282847in}{1.502115in}}%
\pgfpathlineto{\pgfqpoint{1.283725in}{1.502786in}}%
\pgfpathlineto{\pgfqpoint{1.289725in}{1.503890in}}%
\pgfpathlineto{\pgfqpoint{1.290781in}{1.504206in}}%
\pgfpathlineto{\pgfqpoint{1.297342in}{1.505310in}}%
\pgfpathlineto{\pgfqpoint{1.298314in}{1.505468in}}%
\pgfpathlineto{\pgfqpoint{1.303211in}{1.506572in}}%
\pgfpathlineto{\pgfqpoint{1.304211in}{1.506927in}}%
\pgfpathlineto{\pgfqpoint{1.311342in}{1.508031in}}%
\pgfpathlineto{\pgfqpoint{1.312426in}{1.508347in}}%
\pgfpathlineto{\pgfqpoint{1.319847in}{1.509451in}}%
\pgfpathlineto{\pgfqpoint{1.320931in}{1.509885in}}%
\pgfpathlineto{\pgfqpoint{1.325866in}{1.510990in}}%
\pgfpathlineto{\pgfqpoint{1.326492in}{1.511226in}}%
\pgfpathlineto{\pgfqpoint{1.333324in}{1.512331in}}%
\pgfpathlineto{\pgfqpoint{1.334370in}{1.512607in}}%
\pgfpathlineto{\pgfqpoint{1.342670in}{1.513711in}}%
\pgfpathlineto{\pgfqpoint{1.343539in}{1.513869in}}%
\pgfpathlineto{\pgfqpoint{1.348053in}{1.514973in}}%
\pgfpathlineto{\pgfqpoint{1.349081in}{1.515328in}}%
\pgfpathlineto{\pgfqpoint{1.356445in}{1.516433in}}%
\pgfpathlineto{\pgfqpoint{1.357399in}{1.516630in}}%
\pgfpathlineto{\pgfqpoint{1.366137in}{1.517734in}}%
\pgfpathlineto{\pgfqpoint{1.367090in}{1.518050in}}%
\pgfpathlineto{\pgfqpoint{1.373707in}{1.519154in}}%
\pgfpathlineto{\pgfqpoint{1.374810in}{1.519391in}}%
\pgfpathlineto{\pgfqpoint{1.381119in}{1.520495in}}%
\pgfpathlineto{\pgfqpoint{1.382072in}{1.520693in}}%
\pgfpathlineto{\pgfqpoint{1.389633in}{1.521797in}}%
\pgfpathlineto{\pgfqpoint{1.390736in}{1.521955in}}%
\pgfpathlineto{\pgfqpoint{1.397212in}{1.523059in}}%
\pgfpathlineto{\pgfqpoint{1.398138in}{1.523138in}}%
\pgfpathlineto{\pgfqpoint{1.403979in}{1.524242in}}%
\pgfpathlineto{\pgfqpoint{1.404642in}{1.524361in}}%
\pgfpathlineto{\pgfqpoint{1.413100in}{1.525465in}}%
\pgfpathlineto{\pgfqpoint{1.413988in}{1.525662in}}%
\pgfpathlineto{\pgfqpoint{1.422951in}{1.526767in}}%
\pgfpathlineto{\pgfqpoint{1.423970in}{1.526924in}}%
\pgfpathlineto{\pgfqpoint{1.431661in}{1.528029in}}%
\pgfpathlineto{\pgfqpoint{1.432026in}{1.528147in}}%
\pgfpathlineto{\pgfqpoint{1.441671in}{1.529252in}}%
\pgfpathlineto{\pgfqpoint{1.442475in}{1.529409in}}%
\pgfpathlineto{\pgfqpoint{1.455456in}{1.530514in}}%
\pgfpathlineto{\pgfqpoint{1.456475in}{1.530632in}}%
\pgfpathlineto{\pgfqpoint{1.466662in}{1.531736in}}%
\pgfpathlineto{\pgfqpoint{1.467699in}{1.532013in}}%
\pgfpathlineto{\pgfqpoint{1.477288in}{1.533117in}}%
\pgfpathlineto{\pgfqpoint{1.477297in}{1.533196in}}%
\pgfpathlineto{\pgfqpoint{1.487157in}{1.534300in}}%
\pgfpathlineto{\pgfqpoint{1.488232in}{1.534616in}}%
\pgfpathlineto{\pgfqpoint{1.496615in}{1.535720in}}%
\pgfpathlineto{\pgfqpoint{1.497139in}{1.535917in}}%
\pgfpathlineto{\pgfqpoint{1.508559in}{1.537022in}}%
\pgfpathlineto{\pgfqpoint{1.509382in}{1.537101in}}%
\pgfpathlineto{\pgfqpoint{1.517681in}{1.538205in}}%
\pgfpathlineto{\pgfqpoint{1.517737in}{1.538284in}}%
\pgfpathlineto{\pgfqpoint{1.529877in}{1.539388in}}%
\pgfpathlineto{\pgfqpoint{1.530933in}{1.539625in}}%
\pgfpathlineto{\pgfqpoint{1.538691in}{1.540729in}}%
\pgfpathlineto{\pgfqpoint{1.539597in}{1.540927in}}%
\pgfpathlineto{\pgfqpoint{1.549859in}{1.542031in}}%
\pgfpathlineto{\pgfqpoint{1.550952in}{1.542268in}}%
\pgfpathlineto{\pgfqpoint{1.559345in}{1.543372in}}%
\pgfpathlineto{\pgfqpoint{1.560336in}{1.543490in}}%
\pgfpathlineto{\pgfqpoint{1.571560in}{1.544595in}}%
\pgfpathlineto{\pgfqpoint{1.572579in}{1.544752in}}%
\pgfpathlineto{\pgfqpoint{1.580943in}{1.545857in}}%
\pgfpathlineto{\pgfqpoint{1.581224in}{1.545936in}}%
\pgfpathlineto{\pgfqpoint{1.592205in}{1.547040in}}%
\pgfpathlineto{\pgfqpoint{1.592887in}{1.547237in}}%
\pgfpathlineto{\pgfqpoint{1.603757in}{1.548342in}}%
\pgfpathlineto{\pgfqpoint{1.604056in}{1.548421in}}%
\pgfpathlineto{\pgfqpoint{1.612327in}{1.549525in}}%
\pgfpathlineto{\pgfqpoint{1.612570in}{1.549604in}}%
\pgfpathlineto{\pgfqpoint{1.629720in}{1.550708in}}%
\pgfpathlineto{\pgfqpoint{1.630514in}{1.550787in}}%
\pgfpathlineto{\pgfqpoint{1.639318in}{1.551379in}}%
\pgfpathlineto{\pgfqpoint{1.640281in}{1.560806in}}%
\pgfpathlineto{\pgfqpoint{1.653505in}{1.561910in}}%
\pgfpathlineto{\pgfqpoint{1.653589in}{1.561989in}}%
\pgfpathlineto{\pgfqpoint{1.666430in}{1.563093in}}%
\pgfpathlineto{\pgfqpoint{1.667281in}{1.563251in}}%
\pgfpathlineto{\pgfqpoint{1.680365in}{1.564355in}}%
\pgfpathlineto{\pgfqpoint{1.681374in}{1.564632in}}%
\pgfpathlineto{\pgfqpoint{1.692608in}{1.565736in}}%
\pgfpathlineto{\pgfqpoint{1.693683in}{1.565854in}}%
\pgfpathlineto{\pgfqpoint{1.704758in}{1.566959in}}%
\pgfpathlineto{\pgfqpoint{1.705655in}{1.567156in}}%
\pgfpathlineto{\pgfqpoint{1.715627in}{1.568260in}}%
\pgfpathlineto{\pgfqpoint{1.716702in}{1.568379in}}%
\pgfpathlineto{\pgfqpoint{1.729721in}{1.569483in}}%
\pgfpathlineto{\pgfqpoint{1.730796in}{1.569601in}}%
\pgfpathlineto{\pgfqpoint{1.745375in}{1.570706in}}%
\pgfpathlineto{\pgfqpoint{1.745628in}{1.570824in}}%
\pgfpathlineto{\pgfqpoint{1.762282in}{1.571928in}}%
\pgfpathlineto{\pgfqpoint{1.762488in}{1.572047in}}%
\pgfpathlineto{\pgfqpoint{1.773693in}{1.573151in}}%
\pgfpathlineto{\pgfqpoint{1.774506in}{1.573388in}}%
\pgfpathlineto{\pgfqpoint{1.786890in}{1.574492in}}%
\pgfpathlineto{\pgfqpoint{1.787815in}{1.574729in}}%
\pgfpathlineto{\pgfqpoint{1.797852in}{1.575833in}}%
\pgfpathlineto{\pgfqpoint{1.797852in}{1.575873in}}%
\pgfpathlineto{\pgfqpoint{1.810750in}{1.576977in}}%
\pgfpathlineto{\pgfqpoint{1.811479in}{1.577095in}}%
\pgfpathlineto{\pgfqpoint{1.825956in}{1.578200in}}%
\pgfpathlineto{\pgfqpoint{1.826451in}{1.578318in}}%
\pgfpathlineto{\pgfqpoint{1.839993in}{1.579422in}}%
\pgfpathlineto{\pgfqpoint{1.840844in}{1.579620in}}%
\pgfpathlineto{\pgfqpoint{1.855535in}{1.580724in}}%
\pgfpathlineto{\pgfqpoint{1.856517in}{1.580921in}}%
\pgfpathlineto{\pgfqpoint{1.870582in}{1.582026in}}%
\pgfpathlineto{\pgfqpoint{1.871582in}{1.582223in}}%
\pgfpathlineto{\pgfqpoint{1.884424in}{1.583327in}}%
\pgfpathlineto{\pgfqpoint{1.885349in}{1.583446in}}%
\pgfpathlineto{\pgfqpoint{1.903807in}{1.584550in}}%
\pgfpathlineto{\pgfqpoint{1.904611in}{1.584787in}}%
\pgfpathlineto{\pgfqpoint{1.922433in}{1.585891in}}%
\pgfpathlineto{\pgfqpoint{1.923489in}{1.585970in}}%
\pgfpathlineto{\pgfqpoint{1.934218in}{1.587074in}}%
\pgfpathlineto{\pgfqpoint{1.935312in}{1.587153in}}%
\pgfpathlineto{\pgfqpoint{1.948835in}{1.588258in}}%
\pgfpathlineto{\pgfqpoint{1.949480in}{1.588376in}}%
\pgfpathlineto{\pgfqpoint{1.964910in}{1.589480in}}%
\pgfpathlineto{\pgfqpoint{1.965752in}{1.589638in}}%
\pgfpathlineto{\pgfqpoint{1.975574in}{1.590742in}}%
\pgfpathlineto{\pgfqpoint{1.976284in}{1.590900in}}%
\pgfpathlineto{\pgfqpoint{1.987434in}{1.592005in}}%
\pgfpathlineto{\pgfqpoint{1.988490in}{1.592281in}}%
\pgfpathlineto{\pgfqpoint{2.000387in}{1.593385in}}%
\pgfpathlineto{\pgfqpoint{2.001444in}{1.593661in}}%
\pgfpathlineto{\pgfqpoint{2.008930in}{1.594766in}}%
\pgfpathlineto{\pgfqpoint{2.010032in}{1.594963in}}%
\pgfpathlineto{\pgfqpoint{2.018135in}{1.596067in}}%
\pgfpathlineto{\pgfqpoint{2.018995in}{1.596264in}}%
\pgfpathlineto{\pgfqpoint{2.024892in}{1.597369in}}%
\pgfpathlineto{\pgfqpoint{2.025958in}{1.597605in}}%
\pgfpathlineto{\pgfqpoint{2.029584in}{1.598710in}}%
\pgfpathlineto{\pgfqpoint{2.030481in}{1.599144in}}%
\pgfpathlineto{\pgfqpoint{2.032799in}{1.600209in}}%
\pgfpathlineto{\pgfqpoint{2.033126in}{1.601944in}}%
\pgfpathlineto{\pgfqpoint{2.033126in}{1.601944in}}%
\pgfusepath{stroke}%
\end{pgfscope}%
\begin{pgfscope}%
\pgfsetrectcap%
\pgfsetmiterjoin%
\pgfsetlinewidth{0.803000pt}%
\definecolor{currentstroke}{rgb}{0.000000,0.000000,0.000000}%
\pgfsetstrokecolor{currentstroke}%
\pgfsetdash{}{0pt}%
\pgfpathmoveto{\pgfqpoint{0.553581in}{0.499444in}}%
\pgfpathlineto{\pgfqpoint{0.553581in}{1.654444in}}%
\pgfusepath{stroke}%
\end{pgfscope}%
\begin{pgfscope}%
\pgfsetrectcap%
\pgfsetmiterjoin%
\pgfsetlinewidth{0.803000pt}%
\definecolor{currentstroke}{rgb}{0.000000,0.000000,0.000000}%
\pgfsetstrokecolor{currentstroke}%
\pgfsetdash{}{0pt}%
\pgfpathmoveto{\pgfqpoint{2.103581in}{0.499444in}}%
\pgfpathlineto{\pgfqpoint{2.103581in}{1.654444in}}%
\pgfusepath{stroke}%
\end{pgfscope}%
\begin{pgfscope}%
\pgfsetrectcap%
\pgfsetmiterjoin%
\pgfsetlinewidth{0.803000pt}%
\definecolor{currentstroke}{rgb}{0.000000,0.000000,0.000000}%
\pgfsetstrokecolor{currentstroke}%
\pgfsetdash{}{0pt}%
\pgfpathmoveto{\pgfqpoint{0.553581in}{0.499444in}}%
\pgfpathlineto{\pgfqpoint{2.103581in}{0.499444in}}%
\pgfusepath{stroke}%
\end{pgfscope}%
\begin{pgfscope}%
\pgfsetrectcap%
\pgfsetmiterjoin%
\pgfsetlinewidth{0.803000pt}%
\definecolor{currentstroke}{rgb}{0.000000,0.000000,0.000000}%
\pgfsetstrokecolor{currentstroke}%
\pgfsetdash{}{0pt}%
\pgfpathmoveto{\pgfqpoint{0.553581in}{1.654444in}}%
\pgfpathlineto{\pgfqpoint{2.103581in}{1.654444in}}%
\pgfusepath{stroke}%
\end{pgfscope}%
\begin{pgfscope}%
\pgfsetbuttcap%
\pgfsetmiterjoin%
\definecolor{currentfill}{rgb}{1.000000,1.000000,1.000000}%
\pgfsetfillcolor{currentfill}%
\pgfsetlinewidth{0.000000pt}%
\definecolor{currentstroke}{rgb}{0.000000,0.000000,0.000000}%
\pgfsetstrokecolor{currentstroke}%
\pgfsetstrokeopacity{0.000000}%
\pgfsetdash{}{0pt}%
\pgfpathmoveto{\pgfqpoint{0.686183in}{0.966899in}}%
\pgfpathlineto{\pgfqpoint{1.085905in}{0.966899in}}%
\pgfpathlineto{\pgfqpoint{1.085905in}{1.173565in}}%
\pgfpathlineto{\pgfqpoint{0.686183in}{1.173565in}}%
\pgfpathlineto{\pgfqpoint{0.686183in}{0.966899in}}%
\pgfpathclose%
\pgfusepath{fill}%
\end{pgfscope}%
\begin{pgfscope}%
\definecolor{textcolor}{rgb}{0.000000,0.000000,0.000000}%
\pgfsetstrokecolor{textcolor}%
\pgfsetfillcolor{textcolor}%
\pgftext[x=0.727850in,y=1.035510in,left,base]{\color{textcolor}\rmfamily\fontsize{10.000000}{12.000000}\selectfont 0.337}%
\end{pgfscope}%
\begin{pgfscope}%
\pgfsetbuttcap%
\pgfsetmiterjoin%
\definecolor{currentfill}{rgb}{1.000000,1.000000,1.000000}%
\pgfsetfillcolor{currentfill}%
\pgfsetlinewidth{0.000000pt}%
\definecolor{currentstroke}{rgb}{0.000000,0.000000,0.000000}%
\pgfsetstrokecolor{currentstroke}%
\pgfsetstrokeopacity{0.000000}%
\pgfsetdash{}{0pt}%
\pgfpathmoveto{\pgfqpoint{0.582378in}{0.483333in}}%
\pgfpathlineto{\pgfqpoint{0.843211in}{0.483333in}}%
\pgfpathlineto{\pgfqpoint{0.843211in}{0.690000in}}%
\pgfpathlineto{\pgfqpoint{0.582378in}{0.690000in}}%
\pgfpathlineto{\pgfqpoint{0.582378in}{0.483333in}}%
\pgfpathclose%
\pgfusepath{fill}%
\end{pgfscope}%
\begin{pgfscope}%
\definecolor{textcolor}{rgb}{0.000000,0.000000,0.000000}%
\pgfsetstrokecolor{textcolor}%
\pgfsetfillcolor{textcolor}%
\pgftext[x=0.624045in,y=0.551944in,left,base]{\color{textcolor}\rmfamily\fontsize{10.000000}{12.000000}\selectfont 0.5}%
\end{pgfscope}%
\begin{pgfscope}%
\pgfsetbuttcap%
\pgfsetmiterjoin%
\definecolor{currentfill}{rgb}{1.000000,1.000000,1.000000}%
\pgfsetfillcolor{currentfill}%
\pgfsetlinewidth{0.000000pt}%
\definecolor{currentstroke}{rgb}{0.000000,0.000000,0.000000}%
\pgfsetstrokecolor{currentstroke}%
\pgfsetstrokeopacity{0.000000}%
\pgfsetdash{}{0pt}%
\pgfpathmoveto{\pgfqpoint{0.582378in}{0.483333in}}%
\pgfpathlineto{\pgfqpoint{0.982100in}{0.483333in}}%
\pgfpathlineto{\pgfqpoint{0.982100in}{0.690000in}}%
\pgfpathlineto{\pgfqpoint{0.582378in}{0.690000in}}%
\pgfpathlineto{\pgfqpoint{0.582378in}{0.483333in}}%
\pgfpathclose%
\pgfusepath{fill}%
\end{pgfscope}%
\begin{pgfscope}%
\definecolor{textcolor}{rgb}{0.000000,0.000000,0.000000}%
\pgfsetstrokecolor{textcolor}%
\pgfsetfillcolor{textcolor}%
\pgftext[x=0.624045in,y=0.551944in,left,base]{\color{textcolor}\rmfamily\fontsize{10.000000}{12.000000}\selectfont 0.656}%
\end{pgfscope}%
\begin{pgfscope}%
\pgfsetbuttcap%
\pgfsetmiterjoin%
\definecolor{currentfill}{rgb}{1.000000,1.000000,1.000000}%
\pgfsetfillcolor{currentfill}%
\pgfsetfillopacity{0.800000}%
\pgfsetlinewidth{1.003750pt}%
\definecolor{currentstroke}{rgb}{0.800000,0.800000,0.800000}%
\pgfsetstrokecolor{currentstroke}%
\pgfsetstrokeopacity{0.800000}%
\pgfsetdash{}{0pt}%
\pgfpathmoveto{\pgfqpoint{0.832747in}{0.568889in}}%
\pgfpathlineto{\pgfqpoint{2.006358in}{0.568889in}}%
\pgfpathquadraticcurveto{\pgfqpoint{2.034136in}{0.568889in}}{\pgfqpoint{2.034136in}{0.596666in}}%
\pgfpathlineto{\pgfqpoint{2.034136in}{0.776388in}}%
\pgfpathquadraticcurveto{\pgfqpoint{2.034136in}{0.804166in}}{\pgfqpoint{2.006358in}{0.804166in}}%
\pgfpathlineto{\pgfqpoint{0.832747in}{0.804166in}}%
\pgfpathquadraticcurveto{\pgfqpoint{0.804970in}{0.804166in}}{\pgfqpoint{0.804970in}{0.776388in}}%
\pgfpathlineto{\pgfqpoint{0.804970in}{0.596666in}}%
\pgfpathquadraticcurveto{\pgfqpoint{0.804970in}{0.568889in}}{\pgfqpoint{0.832747in}{0.568889in}}%
\pgfpathlineto{\pgfqpoint{0.832747in}{0.568889in}}%
\pgfpathclose%
\pgfusepath{stroke,fill}%
\end{pgfscope}%
\begin{pgfscope}%
\pgfsetrectcap%
\pgfsetroundjoin%
\pgfsetlinewidth{1.505625pt}%
\definecolor{currentstroke}{rgb}{0.000000,0.000000,0.000000}%
\pgfsetstrokecolor{currentstroke}%
\pgfsetdash{}{0pt}%
\pgfpathmoveto{\pgfqpoint{0.860525in}{0.700000in}}%
\pgfpathlineto{\pgfqpoint{0.999414in}{0.700000in}}%
\pgfpathlineto{\pgfqpoint{1.138303in}{0.700000in}}%
\pgfusepath{stroke}%
\end{pgfscope}%
\begin{pgfscope}%
\definecolor{textcolor}{rgb}{0.000000,0.000000,0.000000}%
\pgfsetstrokecolor{textcolor}%
\pgfsetfillcolor{textcolor}%
\pgftext[x=1.249414in,y=0.651388in,left,base]{\color{textcolor}\rmfamily\fontsize{10.000000}{12.000000}\selectfont AUC=0.832}%
\end{pgfscope}%
\end{pgfpicture}%
\makeatother%
\endgroup%

  &
\vspace{0pt} 
  
\begin{tabular}{cc|c|c|}
	&\multicolumn{1}{c}{}& \multicolumn{2}{c}{Prediction} \cr
	&\multicolumn{1}{c}{} & \multicolumn{1}{c}{N} & \multicolumn{1}{c}{P} \cr\cline{3-4}
	\multirow{2}{*}{\rotatebox[origin=c]{90}{Actual}}&N & 0 & 150,771   \vrule width 0pt height 10pt depth 2pt \cr\cline{3-4}
	&P & 0 & 22,621 \vrule width 0pt height 10pt depth 2pt \cr\cline{3-4}
\end{tabular}

\begin{center}
\begin{tabular}{ll}
0.150 & Precision \cr 
1.000 & Recall \cr 
0.261 & F1 \cr 
\end{tabular}
\end{center}
  
\end{tabular}

Some models, like the Easy Ensemble Classifier and RUSBoost, give a tight range of probabilities.  Example 5 below is the probabilities from Example 1 linearly transformed with $f(x) = 0.2x + 0.4$ to have range $p \in [0,4,0.6]$.  As a model it gives the same decisions and metrics as Example 1, but is not as useful as a data visualization for comparing models, which is why we dilate the probabilities to go to $p=0.0$.

\noindent\begin{tabular}{@{}p{0.3\textwidth}@{\hspace{24pt}} p{0.3\textwidth} @{\hspace{24pt}} p{0.3\textwidth}}
  \vspace{0pt} %% Creator: Matplotlib, PGF backend
%%
%% To include the figure in your LaTeX document, write
%%   \input{<filename>.pgf}
%%
%% Make sure the required packages are loaded in your preamble
%%   \usepackage{pgf}
%%
%% Also ensure that all the required font packages are loaded; for instance,
%% the lmodern package is sometimes necessary when using math font.
%%   \usepackage{lmodern}
%%
%% Figures using additional raster images can only be included by \input if
%% they are in the same directory as the main LaTeX file. For loading figures
%% from other directories you can use the `import` package
%%   \usepackage{import}
%%
%% and then include the figures with
%%   \import{<path to file>}{<filename>.pgf}
%%
%% Matplotlib used the following preamble
%%   
%%   \usepackage{fontspec}
%%   \makeatletter\@ifpackageloaded{underscore}{}{\usepackage[strings]{underscore}}\makeatother
%%
\begingroup%
\makeatletter%
\begin{pgfpicture}%
\pgfpathrectangle{\pgfpointorigin}{\pgfqpoint{2.253750in}{1.754444in}}%
\pgfusepath{use as bounding box, clip}%
\begin{pgfscope}%
\pgfsetbuttcap%
\pgfsetmiterjoin%
\definecolor{currentfill}{rgb}{1.000000,1.000000,1.000000}%
\pgfsetfillcolor{currentfill}%
\pgfsetlinewidth{0.000000pt}%
\definecolor{currentstroke}{rgb}{1.000000,1.000000,1.000000}%
\pgfsetstrokecolor{currentstroke}%
\pgfsetdash{}{0pt}%
\pgfpathmoveto{\pgfqpoint{0.000000in}{0.000000in}}%
\pgfpathlineto{\pgfqpoint{2.253750in}{0.000000in}}%
\pgfpathlineto{\pgfqpoint{2.253750in}{1.754444in}}%
\pgfpathlineto{\pgfqpoint{0.000000in}{1.754444in}}%
\pgfpathlineto{\pgfqpoint{0.000000in}{0.000000in}}%
\pgfpathclose%
\pgfusepath{fill}%
\end{pgfscope}%
\begin{pgfscope}%
\pgfsetbuttcap%
\pgfsetmiterjoin%
\definecolor{currentfill}{rgb}{1.000000,1.000000,1.000000}%
\pgfsetfillcolor{currentfill}%
\pgfsetlinewidth{0.000000pt}%
\definecolor{currentstroke}{rgb}{0.000000,0.000000,0.000000}%
\pgfsetstrokecolor{currentstroke}%
\pgfsetstrokeopacity{0.000000}%
\pgfsetdash{}{0pt}%
\pgfpathmoveto{\pgfqpoint{0.515000in}{0.499444in}}%
\pgfpathlineto{\pgfqpoint{2.065000in}{0.499444in}}%
\pgfpathlineto{\pgfqpoint{2.065000in}{1.654444in}}%
\pgfpathlineto{\pgfqpoint{0.515000in}{1.654444in}}%
\pgfpathlineto{\pgfqpoint{0.515000in}{0.499444in}}%
\pgfpathclose%
\pgfusepath{fill}%
\end{pgfscope}%
\begin{pgfscope}%
\pgfpathrectangle{\pgfqpoint{0.515000in}{0.499444in}}{\pgfqpoint{1.550000in}{1.155000in}}%
\pgfusepath{clip}%
\pgfsetbuttcap%
\pgfsetmiterjoin%
\pgfsetlinewidth{1.003750pt}%
\definecolor{currentstroke}{rgb}{0.000000,0.000000,0.000000}%
\pgfsetstrokecolor{currentstroke}%
\pgfsetdash{}{0pt}%
\pgfpathmoveto{\pgfqpoint{0.505000in}{0.499444in}}%
\pgfpathlineto{\pgfqpoint{0.552805in}{0.499444in}}%
\pgfpathlineto{\pgfqpoint{0.552805in}{0.499444in}}%
\pgfpathlineto{\pgfqpoint{0.505000in}{0.499444in}}%
\pgfusepath{stroke}%
\end{pgfscope}%
\begin{pgfscope}%
\pgfpathrectangle{\pgfqpoint{0.515000in}{0.499444in}}{\pgfqpoint{1.550000in}{1.155000in}}%
\pgfusepath{clip}%
\pgfsetbuttcap%
\pgfsetmiterjoin%
\pgfsetlinewidth{1.003750pt}%
\definecolor{currentstroke}{rgb}{0.000000,0.000000,0.000000}%
\pgfsetstrokecolor{currentstroke}%
\pgfsetdash{}{0pt}%
\pgfpathmoveto{\pgfqpoint{0.643537in}{0.499444in}}%
\pgfpathlineto{\pgfqpoint{0.704025in}{0.499444in}}%
\pgfpathlineto{\pgfqpoint{0.704025in}{0.499444in}}%
\pgfpathlineto{\pgfqpoint{0.643537in}{0.499444in}}%
\pgfpathlineto{\pgfqpoint{0.643537in}{0.499444in}}%
\pgfpathclose%
\pgfusepath{stroke}%
\end{pgfscope}%
\begin{pgfscope}%
\pgfpathrectangle{\pgfqpoint{0.515000in}{0.499444in}}{\pgfqpoint{1.550000in}{1.155000in}}%
\pgfusepath{clip}%
\pgfsetbuttcap%
\pgfsetmiterjoin%
\pgfsetlinewidth{1.003750pt}%
\definecolor{currentstroke}{rgb}{0.000000,0.000000,0.000000}%
\pgfsetstrokecolor{currentstroke}%
\pgfsetdash{}{0pt}%
\pgfpathmoveto{\pgfqpoint{0.794756in}{0.499444in}}%
\pgfpathlineto{\pgfqpoint{0.855244in}{0.499444in}}%
\pgfpathlineto{\pgfqpoint{0.855244in}{0.499444in}}%
\pgfpathlineto{\pgfqpoint{0.794756in}{0.499444in}}%
\pgfpathlineto{\pgfqpoint{0.794756in}{0.499444in}}%
\pgfpathclose%
\pgfusepath{stroke}%
\end{pgfscope}%
\begin{pgfscope}%
\pgfpathrectangle{\pgfqpoint{0.515000in}{0.499444in}}{\pgfqpoint{1.550000in}{1.155000in}}%
\pgfusepath{clip}%
\pgfsetbuttcap%
\pgfsetmiterjoin%
\pgfsetlinewidth{1.003750pt}%
\definecolor{currentstroke}{rgb}{0.000000,0.000000,0.000000}%
\pgfsetstrokecolor{currentstroke}%
\pgfsetdash{}{0pt}%
\pgfpathmoveto{\pgfqpoint{0.945976in}{0.499444in}}%
\pgfpathlineto{\pgfqpoint{1.006464in}{0.499444in}}%
\pgfpathlineto{\pgfqpoint{1.006464in}{0.499444in}}%
\pgfpathlineto{\pgfqpoint{0.945976in}{0.499444in}}%
\pgfpathlineto{\pgfqpoint{0.945976in}{0.499444in}}%
\pgfpathclose%
\pgfusepath{stroke}%
\end{pgfscope}%
\begin{pgfscope}%
\pgfpathrectangle{\pgfqpoint{0.515000in}{0.499444in}}{\pgfqpoint{1.550000in}{1.155000in}}%
\pgfusepath{clip}%
\pgfsetbuttcap%
\pgfsetmiterjoin%
\pgfsetlinewidth{1.003750pt}%
\definecolor{currentstroke}{rgb}{0.000000,0.000000,0.000000}%
\pgfsetstrokecolor{currentstroke}%
\pgfsetdash{}{0pt}%
\pgfpathmoveto{\pgfqpoint{1.097195in}{0.499444in}}%
\pgfpathlineto{\pgfqpoint{1.157683in}{0.499444in}}%
\pgfpathlineto{\pgfqpoint{1.157683in}{1.599444in}}%
\pgfpathlineto{\pgfqpoint{1.097195in}{1.599444in}}%
\pgfpathlineto{\pgfqpoint{1.097195in}{0.499444in}}%
\pgfpathclose%
\pgfusepath{stroke}%
\end{pgfscope}%
\begin{pgfscope}%
\pgfpathrectangle{\pgfqpoint{0.515000in}{0.499444in}}{\pgfqpoint{1.550000in}{1.155000in}}%
\pgfusepath{clip}%
\pgfsetbuttcap%
\pgfsetmiterjoin%
\pgfsetlinewidth{1.003750pt}%
\definecolor{currentstroke}{rgb}{0.000000,0.000000,0.000000}%
\pgfsetstrokecolor{currentstroke}%
\pgfsetdash{}{0pt}%
\pgfpathmoveto{\pgfqpoint{1.248415in}{0.499444in}}%
\pgfpathlineto{\pgfqpoint{1.308903in}{0.499444in}}%
\pgfpathlineto{\pgfqpoint{1.308903in}{0.805783in}}%
\pgfpathlineto{\pgfqpoint{1.248415in}{0.805783in}}%
\pgfpathlineto{\pgfqpoint{1.248415in}{0.499444in}}%
\pgfpathclose%
\pgfusepath{stroke}%
\end{pgfscope}%
\begin{pgfscope}%
\pgfpathrectangle{\pgfqpoint{0.515000in}{0.499444in}}{\pgfqpoint{1.550000in}{1.155000in}}%
\pgfusepath{clip}%
\pgfsetbuttcap%
\pgfsetmiterjoin%
\pgfsetlinewidth{1.003750pt}%
\definecolor{currentstroke}{rgb}{0.000000,0.000000,0.000000}%
\pgfsetstrokecolor{currentstroke}%
\pgfsetdash{}{0pt}%
\pgfpathmoveto{\pgfqpoint{1.399634in}{0.499444in}}%
\pgfpathlineto{\pgfqpoint{1.460122in}{0.499444in}}%
\pgfpathlineto{\pgfqpoint{1.460122in}{0.499444in}}%
\pgfpathlineto{\pgfqpoint{1.399634in}{0.499444in}}%
\pgfpathlineto{\pgfqpoint{1.399634in}{0.499444in}}%
\pgfpathclose%
\pgfusepath{stroke}%
\end{pgfscope}%
\begin{pgfscope}%
\pgfpathrectangle{\pgfqpoint{0.515000in}{0.499444in}}{\pgfqpoint{1.550000in}{1.155000in}}%
\pgfusepath{clip}%
\pgfsetbuttcap%
\pgfsetmiterjoin%
\pgfsetlinewidth{1.003750pt}%
\definecolor{currentstroke}{rgb}{0.000000,0.000000,0.000000}%
\pgfsetstrokecolor{currentstroke}%
\pgfsetdash{}{0pt}%
\pgfpathmoveto{\pgfqpoint{1.550854in}{0.499444in}}%
\pgfpathlineto{\pgfqpoint{1.611342in}{0.499444in}}%
\pgfpathlineto{\pgfqpoint{1.611342in}{0.499444in}}%
\pgfpathlineto{\pgfqpoint{1.550854in}{0.499444in}}%
\pgfpathlineto{\pgfqpoint{1.550854in}{0.499444in}}%
\pgfpathclose%
\pgfusepath{stroke}%
\end{pgfscope}%
\begin{pgfscope}%
\pgfpathrectangle{\pgfqpoint{0.515000in}{0.499444in}}{\pgfqpoint{1.550000in}{1.155000in}}%
\pgfusepath{clip}%
\pgfsetbuttcap%
\pgfsetmiterjoin%
\pgfsetlinewidth{1.003750pt}%
\definecolor{currentstroke}{rgb}{0.000000,0.000000,0.000000}%
\pgfsetstrokecolor{currentstroke}%
\pgfsetdash{}{0pt}%
\pgfpathmoveto{\pgfqpoint{1.702073in}{0.499444in}}%
\pgfpathlineto{\pgfqpoint{1.762561in}{0.499444in}}%
\pgfpathlineto{\pgfqpoint{1.762561in}{0.499444in}}%
\pgfpathlineto{\pgfqpoint{1.702073in}{0.499444in}}%
\pgfpathlineto{\pgfqpoint{1.702073in}{0.499444in}}%
\pgfpathclose%
\pgfusepath{stroke}%
\end{pgfscope}%
\begin{pgfscope}%
\pgfpathrectangle{\pgfqpoint{0.515000in}{0.499444in}}{\pgfqpoint{1.550000in}{1.155000in}}%
\pgfusepath{clip}%
\pgfsetbuttcap%
\pgfsetmiterjoin%
\pgfsetlinewidth{1.003750pt}%
\definecolor{currentstroke}{rgb}{0.000000,0.000000,0.000000}%
\pgfsetstrokecolor{currentstroke}%
\pgfsetdash{}{0pt}%
\pgfpathmoveto{\pgfqpoint{1.853293in}{0.499444in}}%
\pgfpathlineto{\pgfqpoint{1.913781in}{0.499444in}}%
\pgfpathlineto{\pgfqpoint{1.913781in}{0.499444in}}%
\pgfpathlineto{\pgfqpoint{1.853293in}{0.499444in}}%
\pgfpathlineto{\pgfqpoint{1.853293in}{0.499444in}}%
\pgfpathclose%
\pgfusepath{stroke}%
\end{pgfscope}%
\begin{pgfscope}%
\pgfpathrectangle{\pgfqpoint{0.515000in}{0.499444in}}{\pgfqpoint{1.550000in}{1.155000in}}%
\pgfusepath{clip}%
\pgfsetbuttcap%
\pgfsetmiterjoin%
\definecolor{currentfill}{rgb}{0.000000,0.000000,0.000000}%
\pgfsetfillcolor{currentfill}%
\pgfsetlinewidth{0.000000pt}%
\definecolor{currentstroke}{rgb}{0.000000,0.000000,0.000000}%
\pgfsetstrokecolor{currentstroke}%
\pgfsetstrokeopacity{0.000000}%
\pgfsetdash{}{0pt}%
\pgfpathmoveto{\pgfqpoint{0.552805in}{0.499444in}}%
\pgfpathlineto{\pgfqpoint{0.613293in}{0.499444in}}%
\pgfpathlineto{\pgfqpoint{0.613293in}{0.499444in}}%
\pgfpathlineto{\pgfqpoint{0.552805in}{0.499444in}}%
\pgfpathlineto{\pgfqpoint{0.552805in}{0.499444in}}%
\pgfpathclose%
\pgfusepath{fill}%
\end{pgfscope}%
\begin{pgfscope}%
\pgfpathrectangle{\pgfqpoint{0.515000in}{0.499444in}}{\pgfqpoint{1.550000in}{1.155000in}}%
\pgfusepath{clip}%
\pgfsetbuttcap%
\pgfsetmiterjoin%
\definecolor{currentfill}{rgb}{0.000000,0.000000,0.000000}%
\pgfsetfillcolor{currentfill}%
\pgfsetlinewidth{0.000000pt}%
\definecolor{currentstroke}{rgb}{0.000000,0.000000,0.000000}%
\pgfsetstrokecolor{currentstroke}%
\pgfsetstrokeopacity{0.000000}%
\pgfsetdash{}{0pt}%
\pgfpathmoveto{\pgfqpoint{0.704025in}{0.499444in}}%
\pgfpathlineto{\pgfqpoint{0.764512in}{0.499444in}}%
\pgfpathlineto{\pgfqpoint{0.764512in}{0.499444in}}%
\pgfpathlineto{\pgfqpoint{0.704025in}{0.499444in}}%
\pgfpathlineto{\pgfqpoint{0.704025in}{0.499444in}}%
\pgfpathclose%
\pgfusepath{fill}%
\end{pgfscope}%
\begin{pgfscope}%
\pgfpathrectangle{\pgfqpoint{0.515000in}{0.499444in}}{\pgfqpoint{1.550000in}{1.155000in}}%
\pgfusepath{clip}%
\pgfsetbuttcap%
\pgfsetmiterjoin%
\definecolor{currentfill}{rgb}{0.000000,0.000000,0.000000}%
\pgfsetfillcolor{currentfill}%
\pgfsetlinewidth{0.000000pt}%
\definecolor{currentstroke}{rgb}{0.000000,0.000000,0.000000}%
\pgfsetstrokecolor{currentstroke}%
\pgfsetstrokeopacity{0.000000}%
\pgfsetdash{}{0pt}%
\pgfpathmoveto{\pgfqpoint{0.855244in}{0.499444in}}%
\pgfpathlineto{\pgfqpoint{0.915732in}{0.499444in}}%
\pgfpathlineto{\pgfqpoint{0.915732in}{0.499444in}}%
\pgfpathlineto{\pgfqpoint{0.855244in}{0.499444in}}%
\pgfpathlineto{\pgfqpoint{0.855244in}{0.499444in}}%
\pgfpathclose%
\pgfusepath{fill}%
\end{pgfscope}%
\begin{pgfscope}%
\pgfpathrectangle{\pgfqpoint{0.515000in}{0.499444in}}{\pgfqpoint{1.550000in}{1.155000in}}%
\pgfusepath{clip}%
\pgfsetbuttcap%
\pgfsetmiterjoin%
\definecolor{currentfill}{rgb}{0.000000,0.000000,0.000000}%
\pgfsetfillcolor{currentfill}%
\pgfsetlinewidth{0.000000pt}%
\definecolor{currentstroke}{rgb}{0.000000,0.000000,0.000000}%
\pgfsetstrokecolor{currentstroke}%
\pgfsetstrokeopacity{0.000000}%
\pgfsetdash{}{0pt}%
\pgfpathmoveto{\pgfqpoint{1.006464in}{0.499444in}}%
\pgfpathlineto{\pgfqpoint{1.066951in}{0.499444in}}%
\pgfpathlineto{\pgfqpoint{1.066951in}{0.499444in}}%
\pgfpathlineto{\pgfqpoint{1.006464in}{0.499444in}}%
\pgfpathlineto{\pgfqpoint{1.006464in}{0.499444in}}%
\pgfpathclose%
\pgfusepath{fill}%
\end{pgfscope}%
\begin{pgfscope}%
\pgfpathrectangle{\pgfqpoint{0.515000in}{0.499444in}}{\pgfqpoint{1.550000in}{1.155000in}}%
\pgfusepath{clip}%
\pgfsetbuttcap%
\pgfsetmiterjoin%
\definecolor{currentfill}{rgb}{0.000000,0.000000,0.000000}%
\pgfsetfillcolor{currentfill}%
\pgfsetlinewidth{0.000000pt}%
\definecolor{currentstroke}{rgb}{0.000000,0.000000,0.000000}%
\pgfsetstrokecolor{currentstroke}%
\pgfsetstrokeopacity{0.000000}%
\pgfsetdash{}{0pt}%
\pgfpathmoveto{\pgfqpoint{1.157683in}{0.499444in}}%
\pgfpathlineto{\pgfqpoint{1.218171in}{0.499444in}}%
\pgfpathlineto{\pgfqpoint{1.218171in}{0.554738in}}%
\pgfpathlineto{\pgfqpoint{1.157683in}{0.554738in}}%
\pgfpathlineto{\pgfqpoint{1.157683in}{0.499444in}}%
\pgfpathclose%
\pgfusepath{fill}%
\end{pgfscope}%
\begin{pgfscope}%
\pgfpathrectangle{\pgfqpoint{0.515000in}{0.499444in}}{\pgfqpoint{1.550000in}{1.155000in}}%
\pgfusepath{clip}%
\pgfsetbuttcap%
\pgfsetmiterjoin%
\definecolor{currentfill}{rgb}{0.000000,0.000000,0.000000}%
\pgfsetfillcolor{currentfill}%
\pgfsetlinewidth{0.000000pt}%
\definecolor{currentstroke}{rgb}{0.000000,0.000000,0.000000}%
\pgfsetstrokecolor{currentstroke}%
\pgfsetstrokeopacity{0.000000}%
\pgfsetdash{}{0pt}%
\pgfpathmoveto{\pgfqpoint{1.308903in}{0.499444in}}%
\pgfpathlineto{\pgfqpoint{1.369391in}{0.499444in}}%
\pgfpathlineto{\pgfqpoint{1.369391in}{0.692461in}}%
\pgfpathlineto{\pgfqpoint{1.308903in}{0.692461in}}%
\pgfpathlineto{\pgfqpoint{1.308903in}{0.499444in}}%
\pgfpathclose%
\pgfusepath{fill}%
\end{pgfscope}%
\begin{pgfscope}%
\pgfpathrectangle{\pgfqpoint{0.515000in}{0.499444in}}{\pgfqpoint{1.550000in}{1.155000in}}%
\pgfusepath{clip}%
\pgfsetbuttcap%
\pgfsetmiterjoin%
\definecolor{currentfill}{rgb}{0.000000,0.000000,0.000000}%
\pgfsetfillcolor{currentfill}%
\pgfsetlinewidth{0.000000pt}%
\definecolor{currentstroke}{rgb}{0.000000,0.000000,0.000000}%
\pgfsetstrokecolor{currentstroke}%
\pgfsetstrokeopacity{0.000000}%
\pgfsetdash{}{0pt}%
\pgfpathmoveto{\pgfqpoint{1.460122in}{0.499444in}}%
\pgfpathlineto{\pgfqpoint{1.520610in}{0.499444in}}%
\pgfpathlineto{\pgfqpoint{1.520610in}{0.499444in}}%
\pgfpathlineto{\pgfqpoint{1.460122in}{0.499444in}}%
\pgfpathlineto{\pgfqpoint{1.460122in}{0.499444in}}%
\pgfpathclose%
\pgfusepath{fill}%
\end{pgfscope}%
\begin{pgfscope}%
\pgfpathrectangle{\pgfqpoint{0.515000in}{0.499444in}}{\pgfqpoint{1.550000in}{1.155000in}}%
\pgfusepath{clip}%
\pgfsetbuttcap%
\pgfsetmiterjoin%
\definecolor{currentfill}{rgb}{0.000000,0.000000,0.000000}%
\pgfsetfillcolor{currentfill}%
\pgfsetlinewidth{0.000000pt}%
\definecolor{currentstroke}{rgb}{0.000000,0.000000,0.000000}%
\pgfsetstrokecolor{currentstroke}%
\pgfsetstrokeopacity{0.000000}%
\pgfsetdash{}{0pt}%
\pgfpathmoveto{\pgfqpoint{1.611342in}{0.499444in}}%
\pgfpathlineto{\pgfqpoint{1.671830in}{0.499444in}}%
\pgfpathlineto{\pgfqpoint{1.671830in}{0.499444in}}%
\pgfpathlineto{\pgfqpoint{1.611342in}{0.499444in}}%
\pgfpathlineto{\pgfqpoint{1.611342in}{0.499444in}}%
\pgfpathclose%
\pgfusepath{fill}%
\end{pgfscope}%
\begin{pgfscope}%
\pgfpathrectangle{\pgfqpoint{0.515000in}{0.499444in}}{\pgfqpoint{1.550000in}{1.155000in}}%
\pgfusepath{clip}%
\pgfsetbuttcap%
\pgfsetmiterjoin%
\definecolor{currentfill}{rgb}{0.000000,0.000000,0.000000}%
\pgfsetfillcolor{currentfill}%
\pgfsetlinewidth{0.000000pt}%
\definecolor{currentstroke}{rgb}{0.000000,0.000000,0.000000}%
\pgfsetstrokecolor{currentstroke}%
\pgfsetstrokeopacity{0.000000}%
\pgfsetdash{}{0pt}%
\pgfpathmoveto{\pgfqpoint{1.762561in}{0.499444in}}%
\pgfpathlineto{\pgfqpoint{1.823049in}{0.499444in}}%
\pgfpathlineto{\pgfqpoint{1.823049in}{0.499444in}}%
\pgfpathlineto{\pgfqpoint{1.762561in}{0.499444in}}%
\pgfpathlineto{\pgfqpoint{1.762561in}{0.499444in}}%
\pgfpathclose%
\pgfusepath{fill}%
\end{pgfscope}%
\begin{pgfscope}%
\pgfpathrectangle{\pgfqpoint{0.515000in}{0.499444in}}{\pgfqpoint{1.550000in}{1.155000in}}%
\pgfusepath{clip}%
\pgfsetbuttcap%
\pgfsetmiterjoin%
\definecolor{currentfill}{rgb}{0.000000,0.000000,0.000000}%
\pgfsetfillcolor{currentfill}%
\pgfsetlinewidth{0.000000pt}%
\definecolor{currentstroke}{rgb}{0.000000,0.000000,0.000000}%
\pgfsetstrokecolor{currentstroke}%
\pgfsetstrokeopacity{0.000000}%
\pgfsetdash{}{0pt}%
\pgfpathmoveto{\pgfqpoint{1.913781in}{0.499444in}}%
\pgfpathlineto{\pgfqpoint{1.974269in}{0.499444in}}%
\pgfpathlineto{\pgfqpoint{1.974269in}{0.499444in}}%
\pgfpathlineto{\pgfqpoint{1.913781in}{0.499444in}}%
\pgfpathlineto{\pgfqpoint{1.913781in}{0.499444in}}%
\pgfpathclose%
\pgfusepath{fill}%
\end{pgfscope}%
\begin{pgfscope}%
\pgfsetbuttcap%
\pgfsetroundjoin%
\definecolor{currentfill}{rgb}{0.000000,0.000000,0.000000}%
\pgfsetfillcolor{currentfill}%
\pgfsetlinewidth{0.803000pt}%
\definecolor{currentstroke}{rgb}{0.000000,0.000000,0.000000}%
\pgfsetstrokecolor{currentstroke}%
\pgfsetdash{}{0pt}%
\pgfsys@defobject{currentmarker}{\pgfqpoint{0.000000in}{-0.048611in}}{\pgfqpoint{0.000000in}{0.000000in}}{%
\pgfpathmoveto{\pgfqpoint{0.000000in}{0.000000in}}%
\pgfpathlineto{\pgfqpoint{0.000000in}{-0.048611in}}%
\pgfusepath{stroke,fill}%
}%
\begin{pgfscope}%
\pgfsys@transformshift{0.552805in}{0.499444in}%
\pgfsys@useobject{currentmarker}{}%
\end{pgfscope}%
\end{pgfscope}%
\begin{pgfscope}%
\definecolor{textcolor}{rgb}{0.000000,0.000000,0.000000}%
\pgfsetstrokecolor{textcolor}%
\pgfsetfillcolor{textcolor}%
\pgftext[x=0.552805in,y=0.402222in,,top]{\color{textcolor}\rmfamily\fontsize{10.000000}{12.000000}\selectfont 0.0}%
\end{pgfscope}%
\begin{pgfscope}%
\pgfsetbuttcap%
\pgfsetroundjoin%
\definecolor{currentfill}{rgb}{0.000000,0.000000,0.000000}%
\pgfsetfillcolor{currentfill}%
\pgfsetlinewidth{0.803000pt}%
\definecolor{currentstroke}{rgb}{0.000000,0.000000,0.000000}%
\pgfsetstrokecolor{currentstroke}%
\pgfsetdash{}{0pt}%
\pgfsys@defobject{currentmarker}{\pgfqpoint{0.000000in}{-0.048611in}}{\pgfqpoint{0.000000in}{0.000000in}}{%
\pgfpathmoveto{\pgfqpoint{0.000000in}{0.000000in}}%
\pgfpathlineto{\pgfqpoint{0.000000in}{-0.048611in}}%
\pgfusepath{stroke,fill}%
}%
\begin{pgfscope}%
\pgfsys@transformshift{0.930854in}{0.499444in}%
\pgfsys@useobject{currentmarker}{}%
\end{pgfscope}%
\end{pgfscope}%
\begin{pgfscope}%
\definecolor{textcolor}{rgb}{0.000000,0.000000,0.000000}%
\pgfsetstrokecolor{textcolor}%
\pgfsetfillcolor{textcolor}%
\pgftext[x=0.930854in,y=0.402222in,,top]{\color{textcolor}\rmfamily\fontsize{10.000000}{12.000000}\selectfont 0.25}%
\end{pgfscope}%
\begin{pgfscope}%
\pgfsetbuttcap%
\pgfsetroundjoin%
\definecolor{currentfill}{rgb}{0.000000,0.000000,0.000000}%
\pgfsetfillcolor{currentfill}%
\pgfsetlinewidth{0.803000pt}%
\definecolor{currentstroke}{rgb}{0.000000,0.000000,0.000000}%
\pgfsetstrokecolor{currentstroke}%
\pgfsetdash{}{0pt}%
\pgfsys@defobject{currentmarker}{\pgfqpoint{0.000000in}{-0.048611in}}{\pgfqpoint{0.000000in}{0.000000in}}{%
\pgfpathmoveto{\pgfqpoint{0.000000in}{0.000000in}}%
\pgfpathlineto{\pgfqpoint{0.000000in}{-0.048611in}}%
\pgfusepath{stroke,fill}%
}%
\begin{pgfscope}%
\pgfsys@transformshift{1.308903in}{0.499444in}%
\pgfsys@useobject{currentmarker}{}%
\end{pgfscope}%
\end{pgfscope}%
\begin{pgfscope}%
\definecolor{textcolor}{rgb}{0.000000,0.000000,0.000000}%
\pgfsetstrokecolor{textcolor}%
\pgfsetfillcolor{textcolor}%
\pgftext[x=1.308903in,y=0.402222in,,top]{\color{textcolor}\rmfamily\fontsize{10.000000}{12.000000}\selectfont 0.5}%
\end{pgfscope}%
\begin{pgfscope}%
\pgfsetbuttcap%
\pgfsetroundjoin%
\definecolor{currentfill}{rgb}{0.000000,0.000000,0.000000}%
\pgfsetfillcolor{currentfill}%
\pgfsetlinewidth{0.803000pt}%
\definecolor{currentstroke}{rgb}{0.000000,0.000000,0.000000}%
\pgfsetstrokecolor{currentstroke}%
\pgfsetdash{}{0pt}%
\pgfsys@defobject{currentmarker}{\pgfqpoint{0.000000in}{-0.048611in}}{\pgfqpoint{0.000000in}{0.000000in}}{%
\pgfpathmoveto{\pgfqpoint{0.000000in}{0.000000in}}%
\pgfpathlineto{\pgfqpoint{0.000000in}{-0.048611in}}%
\pgfusepath{stroke,fill}%
}%
\begin{pgfscope}%
\pgfsys@transformshift{1.686951in}{0.499444in}%
\pgfsys@useobject{currentmarker}{}%
\end{pgfscope}%
\end{pgfscope}%
\begin{pgfscope}%
\definecolor{textcolor}{rgb}{0.000000,0.000000,0.000000}%
\pgfsetstrokecolor{textcolor}%
\pgfsetfillcolor{textcolor}%
\pgftext[x=1.686951in,y=0.402222in,,top]{\color{textcolor}\rmfamily\fontsize{10.000000}{12.000000}\selectfont 0.75}%
\end{pgfscope}%
\begin{pgfscope}%
\pgfsetbuttcap%
\pgfsetroundjoin%
\definecolor{currentfill}{rgb}{0.000000,0.000000,0.000000}%
\pgfsetfillcolor{currentfill}%
\pgfsetlinewidth{0.803000pt}%
\definecolor{currentstroke}{rgb}{0.000000,0.000000,0.000000}%
\pgfsetstrokecolor{currentstroke}%
\pgfsetdash{}{0pt}%
\pgfsys@defobject{currentmarker}{\pgfqpoint{0.000000in}{-0.048611in}}{\pgfqpoint{0.000000in}{0.000000in}}{%
\pgfpathmoveto{\pgfqpoint{0.000000in}{0.000000in}}%
\pgfpathlineto{\pgfqpoint{0.000000in}{-0.048611in}}%
\pgfusepath{stroke,fill}%
}%
\begin{pgfscope}%
\pgfsys@transformshift{2.065000in}{0.499444in}%
\pgfsys@useobject{currentmarker}{}%
\end{pgfscope}%
\end{pgfscope}%
\begin{pgfscope}%
\definecolor{textcolor}{rgb}{0.000000,0.000000,0.000000}%
\pgfsetstrokecolor{textcolor}%
\pgfsetfillcolor{textcolor}%
\pgftext[x=2.065000in,y=0.402222in,,top]{\color{textcolor}\rmfamily\fontsize{10.000000}{12.000000}\selectfont 1.0}%
\end{pgfscope}%
\begin{pgfscope}%
\definecolor{textcolor}{rgb}{0.000000,0.000000,0.000000}%
\pgfsetstrokecolor{textcolor}%
\pgfsetfillcolor{textcolor}%
\pgftext[x=1.290000in,y=0.223333in,,top]{\color{textcolor}\rmfamily\fontsize{10.000000}{12.000000}\selectfont \(\displaystyle p\)}%
\end{pgfscope}%
\begin{pgfscope}%
\pgfsetbuttcap%
\pgfsetroundjoin%
\definecolor{currentfill}{rgb}{0.000000,0.000000,0.000000}%
\pgfsetfillcolor{currentfill}%
\pgfsetlinewidth{0.803000pt}%
\definecolor{currentstroke}{rgb}{0.000000,0.000000,0.000000}%
\pgfsetstrokecolor{currentstroke}%
\pgfsetdash{}{0pt}%
\pgfsys@defobject{currentmarker}{\pgfqpoint{-0.048611in}{0.000000in}}{\pgfqpoint{-0.000000in}{0.000000in}}{%
\pgfpathmoveto{\pgfqpoint{-0.000000in}{0.000000in}}%
\pgfpathlineto{\pgfqpoint{-0.048611in}{0.000000in}}%
\pgfusepath{stroke,fill}%
}%
\begin{pgfscope}%
\pgfsys@transformshift{0.515000in}{0.499444in}%
\pgfsys@useobject{currentmarker}{}%
\end{pgfscope}%
\end{pgfscope}%
\begin{pgfscope}%
\definecolor{textcolor}{rgb}{0.000000,0.000000,0.000000}%
\pgfsetstrokecolor{textcolor}%
\pgfsetfillcolor{textcolor}%
\pgftext[x=0.348333in, y=0.451250in, left, base]{\color{textcolor}\rmfamily\fontsize{10.000000}{12.000000}\selectfont \(\displaystyle {0}\)}%
\end{pgfscope}%
\begin{pgfscope}%
\pgfsetbuttcap%
\pgfsetroundjoin%
\definecolor{currentfill}{rgb}{0.000000,0.000000,0.000000}%
\pgfsetfillcolor{currentfill}%
\pgfsetlinewidth{0.803000pt}%
\definecolor{currentstroke}{rgb}{0.000000,0.000000,0.000000}%
\pgfsetstrokecolor{currentstroke}%
\pgfsetdash{}{0pt}%
\pgfsys@defobject{currentmarker}{\pgfqpoint{-0.048611in}{0.000000in}}{\pgfqpoint{-0.000000in}{0.000000in}}{%
\pgfpathmoveto{\pgfqpoint{-0.000000in}{0.000000in}}%
\pgfpathlineto{\pgfqpoint{-0.048611in}{0.000000in}}%
\pgfusepath{stroke,fill}%
}%
\begin{pgfscope}%
\pgfsys@transformshift{0.515000in}{0.830374in}%
\pgfsys@useobject{currentmarker}{}%
\end{pgfscope}%
\end{pgfscope}%
\begin{pgfscope}%
\definecolor{textcolor}{rgb}{0.000000,0.000000,0.000000}%
\pgfsetstrokecolor{textcolor}%
\pgfsetfillcolor{textcolor}%
\pgftext[x=0.278889in, y=0.782180in, left, base]{\color{textcolor}\rmfamily\fontsize{10.000000}{12.000000}\selectfont \(\displaystyle {20}\)}%
\end{pgfscope}%
\begin{pgfscope}%
\pgfsetbuttcap%
\pgfsetroundjoin%
\definecolor{currentfill}{rgb}{0.000000,0.000000,0.000000}%
\pgfsetfillcolor{currentfill}%
\pgfsetlinewidth{0.803000pt}%
\definecolor{currentstroke}{rgb}{0.000000,0.000000,0.000000}%
\pgfsetstrokecolor{currentstroke}%
\pgfsetdash{}{0pt}%
\pgfsys@defobject{currentmarker}{\pgfqpoint{-0.048611in}{0.000000in}}{\pgfqpoint{-0.000000in}{0.000000in}}{%
\pgfpathmoveto{\pgfqpoint{-0.000000in}{0.000000in}}%
\pgfpathlineto{\pgfqpoint{-0.048611in}{0.000000in}}%
\pgfusepath{stroke,fill}%
}%
\begin{pgfscope}%
\pgfsys@transformshift{0.515000in}{1.161304in}%
\pgfsys@useobject{currentmarker}{}%
\end{pgfscope}%
\end{pgfscope}%
\begin{pgfscope}%
\definecolor{textcolor}{rgb}{0.000000,0.000000,0.000000}%
\pgfsetstrokecolor{textcolor}%
\pgfsetfillcolor{textcolor}%
\pgftext[x=0.278889in, y=1.113110in, left, base]{\color{textcolor}\rmfamily\fontsize{10.000000}{12.000000}\selectfont \(\displaystyle {40}\)}%
\end{pgfscope}%
\begin{pgfscope}%
\pgfsetbuttcap%
\pgfsetroundjoin%
\definecolor{currentfill}{rgb}{0.000000,0.000000,0.000000}%
\pgfsetfillcolor{currentfill}%
\pgfsetlinewidth{0.803000pt}%
\definecolor{currentstroke}{rgb}{0.000000,0.000000,0.000000}%
\pgfsetstrokecolor{currentstroke}%
\pgfsetdash{}{0pt}%
\pgfsys@defobject{currentmarker}{\pgfqpoint{-0.048611in}{0.000000in}}{\pgfqpoint{-0.000000in}{0.000000in}}{%
\pgfpathmoveto{\pgfqpoint{-0.000000in}{0.000000in}}%
\pgfpathlineto{\pgfqpoint{-0.048611in}{0.000000in}}%
\pgfusepath{stroke,fill}%
}%
\begin{pgfscope}%
\pgfsys@transformshift{0.515000in}{1.492234in}%
\pgfsys@useobject{currentmarker}{}%
\end{pgfscope}%
\end{pgfscope}%
\begin{pgfscope}%
\definecolor{textcolor}{rgb}{0.000000,0.000000,0.000000}%
\pgfsetstrokecolor{textcolor}%
\pgfsetfillcolor{textcolor}%
\pgftext[x=0.278889in, y=1.444040in, left, base]{\color{textcolor}\rmfamily\fontsize{10.000000}{12.000000}\selectfont \(\displaystyle {60}\)}%
\end{pgfscope}%
\begin{pgfscope}%
\definecolor{textcolor}{rgb}{0.000000,0.000000,0.000000}%
\pgfsetstrokecolor{textcolor}%
\pgfsetfillcolor{textcolor}%
\pgftext[x=0.223333in,y=1.076944in,,bottom,rotate=90.000000]{\color{textcolor}\rmfamily\fontsize{10.000000}{12.000000}\selectfont Percent of Data Set}%
\end{pgfscope}%
\begin{pgfscope}%
\pgfsetrectcap%
\pgfsetmiterjoin%
\pgfsetlinewidth{0.803000pt}%
\definecolor{currentstroke}{rgb}{0.000000,0.000000,0.000000}%
\pgfsetstrokecolor{currentstroke}%
\pgfsetdash{}{0pt}%
\pgfpathmoveto{\pgfqpoint{0.515000in}{0.499444in}}%
\pgfpathlineto{\pgfqpoint{0.515000in}{1.654444in}}%
\pgfusepath{stroke}%
\end{pgfscope}%
\begin{pgfscope}%
\pgfsetrectcap%
\pgfsetmiterjoin%
\pgfsetlinewidth{0.803000pt}%
\definecolor{currentstroke}{rgb}{0.000000,0.000000,0.000000}%
\pgfsetstrokecolor{currentstroke}%
\pgfsetdash{}{0pt}%
\pgfpathmoveto{\pgfqpoint{2.065000in}{0.499444in}}%
\pgfpathlineto{\pgfqpoint{2.065000in}{1.654444in}}%
\pgfusepath{stroke}%
\end{pgfscope}%
\begin{pgfscope}%
\pgfsetrectcap%
\pgfsetmiterjoin%
\pgfsetlinewidth{0.803000pt}%
\definecolor{currentstroke}{rgb}{0.000000,0.000000,0.000000}%
\pgfsetstrokecolor{currentstroke}%
\pgfsetdash{}{0pt}%
\pgfpathmoveto{\pgfqpoint{0.515000in}{0.499444in}}%
\pgfpathlineto{\pgfqpoint{2.065000in}{0.499444in}}%
\pgfusepath{stroke}%
\end{pgfscope}%
\begin{pgfscope}%
\pgfsetrectcap%
\pgfsetmiterjoin%
\pgfsetlinewidth{0.803000pt}%
\definecolor{currentstroke}{rgb}{0.000000,0.000000,0.000000}%
\pgfsetstrokecolor{currentstroke}%
\pgfsetdash{}{0pt}%
\pgfpathmoveto{\pgfqpoint{0.515000in}{1.654444in}}%
\pgfpathlineto{\pgfqpoint{2.065000in}{1.654444in}}%
\pgfusepath{stroke}%
\end{pgfscope}%
\begin{pgfscope}%
\pgfsetbuttcap%
\pgfsetmiterjoin%
\definecolor{currentfill}{rgb}{1.000000,1.000000,1.000000}%
\pgfsetfillcolor{currentfill}%
\pgfsetfillopacity{0.800000}%
\pgfsetlinewidth{1.003750pt}%
\definecolor{currentstroke}{rgb}{0.800000,0.800000,0.800000}%
\pgfsetstrokecolor{currentstroke}%
\pgfsetstrokeopacity{0.800000}%
\pgfsetdash{}{0pt}%
\pgfpathmoveto{\pgfqpoint{1.288056in}{1.154445in}}%
\pgfpathlineto{\pgfqpoint{1.967778in}{1.154445in}}%
\pgfpathquadraticcurveto{\pgfqpoint{1.995556in}{1.154445in}}{\pgfqpoint{1.995556in}{1.182222in}}%
\pgfpathlineto{\pgfqpoint{1.995556in}{1.557222in}}%
\pgfpathquadraticcurveto{\pgfqpoint{1.995556in}{1.585000in}}{\pgfqpoint{1.967778in}{1.585000in}}%
\pgfpathlineto{\pgfqpoint{1.288056in}{1.585000in}}%
\pgfpathquadraticcurveto{\pgfqpoint{1.260278in}{1.585000in}}{\pgfqpoint{1.260278in}{1.557222in}}%
\pgfpathlineto{\pgfqpoint{1.260278in}{1.182222in}}%
\pgfpathquadraticcurveto{\pgfqpoint{1.260278in}{1.154445in}}{\pgfqpoint{1.288056in}{1.154445in}}%
\pgfpathlineto{\pgfqpoint{1.288056in}{1.154445in}}%
\pgfpathclose%
\pgfusepath{stroke,fill}%
\end{pgfscope}%
\begin{pgfscope}%
\pgfsetbuttcap%
\pgfsetmiterjoin%
\pgfsetlinewidth{1.003750pt}%
\definecolor{currentstroke}{rgb}{0.000000,0.000000,0.000000}%
\pgfsetstrokecolor{currentstroke}%
\pgfsetdash{}{0pt}%
\pgfpathmoveto{\pgfqpoint{1.315834in}{1.432222in}}%
\pgfpathlineto{\pgfqpoint{1.593611in}{1.432222in}}%
\pgfpathlineto{\pgfqpoint{1.593611in}{1.529444in}}%
\pgfpathlineto{\pgfqpoint{1.315834in}{1.529444in}}%
\pgfpathlineto{\pgfqpoint{1.315834in}{1.432222in}}%
\pgfpathclose%
\pgfusepath{stroke}%
\end{pgfscope}%
\begin{pgfscope}%
\definecolor{textcolor}{rgb}{0.000000,0.000000,0.000000}%
\pgfsetstrokecolor{textcolor}%
\pgfsetfillcolor{textcolor}%
\pgftext[x=1.704722in,y=1.432222in,left,base]{\color{textcolor}\rmfamily\fontsize{10.000000}{12.000000}\selectfont Neg}%
\end{pgfscope}%
\begin{pgfscope}%
\pgfsetbuttcap%
\pgfsetmiterjoin%
\definecolor{currentfill}{rgb}{0.000000,0.000000,0.000000}%
\pgfsetfillcolor{currentfill}%
\pgfsetlinewidth{0.000000pt}%
\definecolor{currentstroke}{rgb}{0.000000,0.000000,0.000000}%
\pgfsetstrokecolor{currentstroke}%
\pgfsetstrokeopacity{0.000000}%
\pgfsetdash{}{0pt}%
\pgfpathmoveto{\pgfqpoint{1.315834in}{1.236944in}}%
\pgfpathlineto{\pgfqpoint{1.593611in}{1.236944in}}%
\pgfpathlineto{\pgfqpoint{1.593611in}{1.334167in}}%
\pgfpathlineto{\pgfqpoint{1.315834in}{1.334167in}}%
\pgfpathlineto{\pgfqpoint{1.315834in}{1.236944in}}%
\pgfpathclose%
\pgfusepath{fill}%
\end{pgfscope}%
\begin{pgfscope}%
\definecolor{textcolor}{rgb}{0.000000,0.000000,0.000000}%
\pgfsetstrokecolor{textcolor}%
\pgfsetfillcolor{textcolor}%
\pgftext[x=1.704722in,y=1.236944in,left,base]{\color{textcolor}\rmfamily\fontsize{10.000000}{12.000000}\selectfont Pos}%
\end{pgfscope}%
\end{pgfpicture}%
\makeatother%
\endgroup%

%  &
%  \vspace{0pt} %% Creator: Matplotlib, PGF backend
%%
%% To include the figure in your LaTeX document, write
%%   \input{<filename>.pgf}
%%
%% Make sure the required packages are loaded in your preamble
%%   \usepackage{pgf}
%%
%% Also ensure that all the required font packages are loaded; for instance,
%% the lmodern package is sometimes necessary when using math font.
%%   \usepackage{lmodern}
%%
%% Figures using additional raster images can only be included by \input if
%% they are in the same directory as the main LaTeX file. For loading figures
%% from other directories you can use the `import` package
%%   \usepackage{import}
%%
%% and then include the figures with
%%   \import{<path to file>}{<filename>.pgf}
%%
%% Matplotlib used the following preamble
%%   
%%   \usepackage{fontspec}
%%   \makeatletter\@ifpackageloaded{underscore}{}{\usepackage[strings]{underscore}}\makeatother
%%
\begingroup%
\makeatletter%
\begin{pgfpicture}%
\pgfpathrectangle{\pgfpointorigin}{\pgfqpoint{2.221861in}{1.754444in}}%
\pgfusepath{use as bounding box, clip}%
\begin{pgfscope}%
\pgfsetbuttcap%
\pgfsetmiterjoin%
\definecolor{currentfill}{rgb}{1.000000,1.000000,1.000000}%
\pgfsetfillcolor{currentfill}%
\pgfsetlinewidth{0.000000pt}%
\definecolor{currentstroke}{rgb}{1.000000,1.000000,1.000000}%
\pgfsetstrokecolor{currentstroke}%
\pgfsetdash{}{0pt}%
\pgfpathmoveto{\pgfqpoint{0.000000in}{0.000000in}}%
\pgfpathlineto{\pgfqpoint{2.221861in}{0.000000in}}%
\pgfpathlineto{\pgfqpoint{2.221861in}{1.754444in}}%
\pgfpathlineto{\pgfqpoint{0.000000in}{1.754444in}}%
\pgfpathlineto{\pgfqpoint{0.000000in}{0.000000in}}%
\pgfpathclose%
\pgfusepath{fill}%
\end{pgfscope}%
\begin{pgfscope}%
\pgfsetbuttcap%
\pgfsetmiterjoin%
\definecolor{currentfill}{rgb}{1.000000,1.000000,1.000000}%
\pgfsetfillcolor{currentfill}%
\pgfsetlinewidth{0.000000pt}%
\definecolor{currentstroke}{rgb}{0.000000,0.000000,0.000000}%
\pgfsetstrokecolor{currentstroke}%
\pgfsetstrokeopacity{0.000000}%
\pgfsetdash{}{0pt}%
\pgfpathmoveto{\pgfqpoint{0.553581in}{0.499444in}}%
\pgfpathlineto{\pgfqpoint{2.103581in}{0.499444in}}%
\pgfpathlineto{\pgfqpoint{2.103581in}{1.654444in}}%
\pgfpathlineto{\pgfqpoint{0.553581in}{1.654444in}}%
\pgfpathlineto{\pgfqpoint{0.553581in}{0.499444in}}%
\pgfpathclose%
\pgfusepath{fill}%
\end{pgfscope}%
\begin{pgfscope}%
\pgfsetbuttcap%
\pgfsetroundjoin%
\definecolor{currentfill}{rgb}{0.000000,0.000000,0.000000}%
\pgfsetfillcolor{currentfill}%
\pgfsetlinewidth{0.803000pt}%
\definecolor{currentstroke}{rgb}{0.000000,0.000000,0.000000}%
\pgfsetstrokecolor{currentstroke}%
\pgfsetdash{}{0pt}%
\pgfsys@defobject{currentmarker}{\pgfqpoint{0.000000in}{-0.048611in}}{\pgfqpoint{0.000000in}{0.000000in}}{%
\pgfpathmoveto{\pgfqpoint{0.000000in}{0.000000in}}%
\pgfpathlineto{\pgfqpoint{0.000000in}{-0.048611in}}%
\pgfusepath{stroke,fill}%
}%
\begin{pgfscope}%
\pgfsys@transformshift{0.624035in}{0.499444in}%
\pgfsys@useobject{currentmarker}{}%
\end{pgfscope}%
\end{pgfscope}%
\begin{pgfscope}%
\definecolor{textcolor}{rgb}{0.000000,0.000000,0.000000}%
\pgfsetstrokecolor{textcolor}%
\pgfsetfillcolor{textcolor}%
\pgftext[x=0.624035in,y=0.402222in,,top]{\color{textcolor}\rmfamily\fontsize{10.000000}{12.000000}\selectfont \(\displaystyle {0.0}\)}%
\end{pgfscope}%
\begin{pgfscope}%
\pgfsetbuttcap%
\pgfsetroundjoin%
\definecolor{currentfill}{rgb}{0.000000,0.000000,0.000000}%
\pgfsetfillcolor{currentfill}%
\pgfsetlinewidth{0.803000pt}%
\definecolor{currentstroke}{rgb}{0.000000,0.000000,0.000000}%
\pgfsetstrokecolor{currentstroke}%
\pgfsetdash{}{0pt}%
\pgfsys@defobject{currentmarker}{\pgfqpoint{0.000000in}{-0.048611in}}{\pgfqpoint{0.000000in}{0.000000in}}{%
\pgfpathmoveto{\pgfqpoint{0.000000in}{0.000000in}}%
\pgfpathlineto{\pgfqpoint{0.000000in}{-0.048611in}}%
\pgfusepath{stroke,fill}%
}%
\begin{pgfscope}%
\pgfsys@transformshift{1.328581in}{0.499444in}%
\pgfsys@useobject{currentmarker}{}%
\end{pgfscope}%
\end{pgfscope}%
\begin{pgfscope}%
\definecolor{textcolor}{rgb}{0.000000,0.000000,0.000000}%
\pgfsetstrokecolor{textcolor}%
\pgfsetfillcolor{textcolor}%
\pgftext[x=1.328581in,y=0.402222in,,top]{\color{textcolor}\rmfamily\fontsize{10.000000}{12.000000}\selectfont \(\displaystyle {0.5}\)}%
\end{pgfscope}%
\begin{pgfscope}%
\pgfsetbuttcap%
\pgfsetroundjoin%
\definecolor{currentfill}{rgb}{0.000000,0.000000,0.000000}%
\pgfsetfillcolor{currentfill}%
\pgfsetlinewidth{0.803000pt}%
\definecolor{currentstroke}{rgb}{0.000000,0.000000,0.000000}%
\pgfsetstrokecolor{currentstroke}%
\pgfsetdash{}{0pt}%
\pgfsys@defobject{currentmarker}{\pgfqpoint{0.000000in}{-0.048611in}}{\pgfqpoint{0.000000in}{0.000000in}}{%
\pgfpathmoveto{\pgfqpoint{0.000000in}{0.000000in}}%
\pgfpathlineto{\pgfqpoint{0.000000in}{-0.048611in}}%
\pgfusepath{stroke,fill}%
}%
\begin{pgfscope}%
\pgfsys@transformshift{2.033126in}{0.499444in}%
\pgfsys@useobject{currentmarker}{}%
\end{pgfscope}%
\end{pgfscope}%
\begin{pgfscope}%
\definecolor{textcolor}{rgb}{0.000000,0.000000,0.000000}%
\pgfsetstrokecolor{textcolor}%
\pgfsetfillcolor{textcolor}%
\pgftext[x=2.033126in,y=0.402222in,,top]{\color{textcolor}\rmfamily\fontsize{10.000000}{12.000000}\selectfont \(\displaystyle {1.0}\)}%
\end{pgfscope}%
\begin{pgfscope}%
\definecolor{textcolor}{rgb}{0.000000,0.000000,0.000000}%
\pgfsetstrokecolor{textcolor}%
\pgfsetfillcolor{textcolor}%
\pgftext[x=1.328581in,y=0.223333in,,top]{\color{textcolor}\rmfamily\fontsize{10.000000}{12.000000}\selectfont False positive rate}%
\end{pgfscope}%
\begin{pgfscope}%
\pgfsetbuttcap%
\pgfsetroundjoin%
\definecolor{currentfill}{rgb}{0.000000,0.000000,0.000000}%
\pgfsetfillcolor{currentfill}%
\pgfsetlinewidth{0.803000pt}%
\definecolor{currentstroke}{rgb}{0.000000,0.000000,0.000000}%
\pgfsetstrokecolor{currentstroke}%
\pgfsetdash{}{0pt}%
\pgfsys@defobject{currentmarker}{\pgfqpoint{-0.048611in}{0.000000in}}{\pgfqpoint{-0.000000in}{0.000000in}}{%
\pgfpathmoveto{\pgfqpoint{-0.000000in}{0.000000in}}%
\pgfpathlineto{\pgfqpoint{-0.048611in}{0.000000in}}%
\pgfusepath{stroke,fill}%
}%
\begin{pgfscope}%
\pgfsys@transformshift{0.553581in}{0.551944in}%
\pgfsys@useobject{currentmarker}{}%
\end{pgfscope}%
\end{pgfscope}%
\begin{pgfscope}%
\definecolor{textcolor}{rgb}{0.000000,0.000000,0.000000}%
\pgfsetstrokecolor{textcolor}%
\pgfsetfillcolor{textcolor}%
\pgftext[x=0.278889in, y=0.503750in, left, base]{\color{textcolor}\rmfamily\fontsize{10.000000}{12.000000}\selectfont \(\displaystyle {0.0}\)}%
\end{pgfscope}%
\begin{pgfscope}%
\pgfsetbuttcap%
\pgfsetroundjoin%
\definecolor{currentfill}{rgb}{0.000000,0.000000,0.000000}%
\pgfsetfillcolor{currentfill}%
\pgfsetlinewidth{0.803000pt}%
\definecolor{currentstroke}{rgb}{0.000000,0.000000,0.000000}%
\pgfsetstrokecolor{currentstroke}%
\pgfsetdash{}{0pt}%
\pgfsys@defobject{currentmarker}{\pgfqpoint{-0.048611in}{0.000000in}}{\pgfqpoint{-0.000000in}{0.000000in}}{%
\pgfpathmoveto{\pgfqpoint{-0.000000in}{0.000000in}}%
\pgfpathlineto{\pgfqpoint{-0.048611in}{0.000000in}}%
\pgfusepath{stroke,fill}%
}%
\begin{pgfscope}%
\pgfsys@transformshift{0.553581in}{1.076944in}%
\pgfsys@useobject{currentmarker}{}%
\end{pgfscope}%
\end{pgfscope}%
\begin{pgfscope}%
\definecolor{textcolor}{rgb}{0.000000,0.000000,0.000000}%
\pgfsetstrokecolor{textcolor}%
\pgfsetfillcolor{textcolor}%
\pgftext[x=0.278889in, y=1.028750in, left, base]{\color{textcolor}\rmfamily\fontsize{10.000000}{12.000000}\selectfont \(\displaystyle {0.5}\)}%
\end{pgfscope}%
\begin{pgfscope}%
\pgfsetbuttcap%
\pgfsetroundjoin%
\definecolor{currentfill}{rgb}{0.000000,0.000000,0.000000}%
\pgfsetfillcolor{currentfill}%
\pgfsetlinewidth{0.803000pt}%
\definecolor{currentstroke}{rgb}{0.000000,0.000000,0.000000}%
\pgfsetstrokecolor{currentstroke}%
\pgfsetdash{}{0pt}%
\pgfsys@defobject{currentmarker}{\pgfqpoint{-0.048611in}{0.000000in}}{\pgfqpoint{-0.000000in}{0.000000in}}{%
\pgfpathmoveto{\pgfqpoint{-0.000000in}{0.000000in}}%
\pgfpathlineto{\pgfqpoint{-0.048611in}{0.000000in}}%
\pgfusepath{stroke,fill}%
}%
\begin{pgfscope}%
\pgfsys@transformshift{0.553581in}{1.601944in}%
\pgfsys@useobject{currentmarker}{}%
\end{pgfscope}%
\end{pgfscope}%
\begin{pgfscope}%
\definecolor{textcolor}{rgb}{0.000000,0.000000,0.000000}%
\pgfsetstrokecolor{textcolor}%
\pgfsetfillcolor{textcolor}%
\pgftext[x=0.278889in, y=1.553750in, left, base]{\color{textcolor}\rmfamily\fontsize{10.000000}{12.000000}\selectfont \(\displaystyle {1.0}\)}%
\end{pgfscope}%
\begin{pgfscope}%
\definecolor{textcolor}{rgb}{0.000000,0.000000,0.000000}%
\pgfsetstrokecolor{textcolor}%
\pgfsetfillcolor{textcolor}%
\pgftext[x=0.223333in,y=1.076944in,,bottom,rotate=90.000000]{\color{textcolor}\rmfamily\fontsize{10.000000}{12.000000}\selectfont True positive rate}%
\end{pgfscope}%
\begin{pgfscope}%
\pgfpathrectangle{\pgfqpoint{0.553581in}{0.499444in}}{\pgfqpoint{1.550000in}{1.155000in}}%
\pgfusepath{clip}%
\pgfsetbuttcap%
\pgfsetroundjoin%
\pgfsetlinewidth{1.505625pt}%
\definecolor{currentstroke}{rgb}{0.000000,0.000000,0.000000}%
\pgfsetstrokecolor{currentstroke}%
\pgfsetdash{{5.550000pt}{2.400000pt}}{0.000000pt}%
\pgfpathmoveto{\pgfqpoint{0.624035in}{0.551944in}}%
\pgfpathlineto{\pgfqpoint{2.033126in}{1.601944in}}%
\pgfusepath{stroke}%
\end{pgfscope}%
\begin{pgfscope}%
\pgfpathrectangle{\pgfqpoint{0.553581in}{0.499444in}}{\pgfqpoint{1.550000in}{1.155000in}}%
\pgfusepath{clip}%
\pgfsetrectcap%
\pgfsetroundjoin%
\pgfsetlinewidth{1.505625pt}%
\definecolor{currentstroke}{rgb}{0.000000,0.000000,0.000000}%
\pgfsetstrokecolor{currentstroke}%
\pgfsetdash{}{0pt}%
\pgfpathmoveto{\pgfqpoint{0.624035in}{0.551944in}}%
\pgfpathlineto{\pgfqpoint{0.627493in}{0.553009in}}%
\pgfpathlineto{\pgfqpoint{0.628559in}{0.554390in}}%
\pgfpathlineto{\pgfqpoint{0.629063in}{0.555494in}}%
\pgfpathlineto{\pgfqpoint{0.630166in}{0.558452in}}%
\pgfpathlineto{\pgfqpoint{0.630531in}{0.559557in}}%
\pgfpathlineto{\pgfqpoint{0.631624in}{0.564132in}}%
\pgfpathlineto{\pgfqpoint{0.631970in}{0.565197in}}%
\pgfpathlineto{\pgfqpoint{0.633073in}{0.569338in}}%
\pgfpathlineto{\pgfqpoint{0.633306in}{0.570443in}}%
\pgfpathlineto{\pgfqpoint{0.634409in}{0.576044in}}%
\pgfpathlineto{\pgfqpoint{0.634755in}{0.577148in}}%
\pgfpathlineto{\pgfqpoint{0.635858in}{0.582236in}}%
\pgfpathlineto{\pgfqpoint{0.636035in}{0.583104in}}%
\pgfpathlineto{\pgfqpoint{0.637138in}{0.589967in}}%
\pgfpathlineto{\pgfqpoint{0.637344in}{0.591032in}}%
\pgfpathlineto{\pgfqpoint{0.638447in}{0.596790in}}%
\pgfpathlineto{\pgfqpoint{0.638559in}{0.597895in}}%
\pgfpathlineto{\pgfqpoint{0.639652in}{0.604758in}}%
\pgfpathlineto{\pgfqpoint{0.639877in}{0.605823in}}%
\pgfpathlineto{\pgfqpoint{0.640979in}{0.612449in}}%
\pgfpathlineto{\pgfqpoint{0.641138in}{0.613553in}}%
\pgfpathlineto{\pgfqpoint{0.642241in}{0.621363in}}%
\pgfpathlineto{\pgfqpoint{0.642456in}{0.622467in}}%
\pgfpathlineto{\pgfqpoint{0.643549in}{0.631026in}}%
\pgfpathlineto{\pgfqpoint{0.643802in}{0.631973in}}%
\pgfpathlineto{\pgfqpoint{0.644895in}{0.639428in}}%
\pgfpathlineto{\pgfqpoint{0.645073in}{0.640532in}}%
\pgfpathlineto{\pgfqpoint{0.646138in}{0.649131in}}%
\pgfpathlineto{\pgfqpoint{0.656213in}{0.650235in}}%
\pgfpathlineto{\pgfqpoint{0.657316in}{0.658636in}}%
\pgfpathlineto{\pgfqpoint{0.657475in}{0.659662in}}%
\pgfpathlineto{\pgfqpoint{0.658559in}{0.668812in}}%
\pgfpathlineto{\pgfqpoint{0.658746in}{0.669877in}}%
\pgfpathlineto{\pgfqpoint{0.659849in}{0.678989in}}%
\pgfpathlineto{\pgfqpoint{0.659942in}{0.680014in}}%
\pgfpathlineto{\pgfqpoint{0.661045in}{0.688415in}}%
\pgfpathlineto{\pgfqpoint{0.661176in}{0.689520in}}%
\pgfpathlineto{\pgfqpoint{0.662279in}{0.697527in}}%
\pgfpathlineto{\pgfqpoint{0.662475in}{0.698631in}}%
\pgfpathlineto{\pgfqpoint{0.663578in}{0.705691in}}%
\pgfpathlineto{\pgfqpoint{0.663821in}{0.706717in}}%
\pgfpathlineto{\pgfqpoint{0.664924in}{0.715394in}}%
\pgfpathlineto{\pgfqpoint{0.665129in}{0.716459in}}%
\pgfpathlineto{\pgfqpoint{0.666223in}{0.724269in}}%
\pgfpathlineto{\pgfqpoint{0.666372in}{0.725334in}}%
\pgfpathlineto{\pgfqpoint{0.667475in}{0.732473in}}%
\pgfpathlineto{\pgfqpoint{0.667643in}{0.733538in}}%
\pgfpathlineto{\pgfqpoint{0.668737in}{0.741189in}}%
\pgfpathlineto{\pgfqpoint{0.668942in}{0.742254in}}%
\pgfpathlineto{\pgfqpoint{0.670026in}{0.751326in}}%
\pgfpathlineto{\pgfqpoint{0.670213in}{0.752431in}}%
\pgfpathlineto{\pgfqpoint{0.671316in}{0.759964in}}%
\pgfpathlineto{\pgfqpoint{0.671606in}{0.761029in}}%
\pgfpathlineto{\pgfqpoint{0.672709in}{0.766788in}}%
\pgfpathlineto{\pgfqpoint{0.672961in}{0.767892in}}%
\pgfpathlineto{\pgfqpoint{0.674064in}{0.775978in}}%
\pgfpathlineto{\pgfqpoint{0.674241in}{0.777043in}}%
\pgfpathlineto{\pgfqpoint{0.675307in}{0.783235in}}%
\pgfpathlineto{\pgfqpoint{0.675634in}{0.784340in}}%
\pgfpathlineto{\pgfqpoint{0.676737in}{0.791124in}}%
\pgfpathlineto{\pgfqpoint{0.676877in}{0.791991in}}%
\pgfpathlineto{\pgfqpoint{0.677980in}{0.799643in}}%
\pgfpathlineto{\pgfqpoint{0.678101in}{0.800669in}}%
\pgfpathlineto{\pgfqpoint{0.679195in}{0.806861in}}%
\pgfpathlineto{\pgfqpoint{0.679466in}{0.807966in}}%
\pgfpathlineto{\pgfqpoint{0.680569in}{0.813685in}}%
\pgfpathlineto{\pgfqpoint{0.680746in}{0.814632in}}%
\pgfpathlineto{\pgfqpoint{0.681840in}{0.821652in}}%
\pgfpathlineto{\pgfqpoint{0.682017in}{0.822717in}}%
\pgfpathlineto{\pgfqpoint{0.683111in}{0.829383in}}%
\pgfpathlineto{\pgfqpoint{0.683316in}{0.830448in}}%
\pgfpathlineto{\pgfqpoint{0.684410in}{0.836522in}}%
\pgfpathlineto{\pgfqpoint{0.684578in}{0.837469in}}%
\pgfpathlineto{\pgfqpoint{0.685681in}{0.843306in}}%
\pgfpathlineto{\pgfqpoint{0.685914in}{0.844411in}}%
\pgfpathlineto{\pgfqpoint{0.686999in}{0.851116in}}%
\pgfpathlineto{\pgfqpoint{0.687195in}{0.851786in}}%
\pgfpathlineto{\pgfqpoint{0.688298in}{0.859951in}}%
\pgfpathlineto{\pgfqpoint{0.688466in}{0.860937in}}%
\pgfpathlineto{\pgfqpoint{0.689569in}{0.867603in}}%
\pgfpathlineto{\pgfqpoint{0.689765in}{0.868668in}}%
\pgfpathlineto{\pgfqpoint{0.690858in}{0.874505in}}%
\pgfpathlineto{\pgfqpoint{0.691073in}{0.875610in}}%
\pgfpathlineto{\pgfqpoint{0.692176in}{0.881684in}}%
\pgfpathlineto{\pgfqpoint{0.692410in}{0.882709in}}%
\pgfpathlineto{\pgfqpoint{0.693513in}{0.889296in}}%
\pgfpathlineto{\pgfqpoint{0.693709in}{0.890401in}}%
\pgfpathlineto{\pgfqpoint{0.694812in}{0.895804in}}%
\pgfpathlineto{\pgfqpoint{0.695176in}{0.896909in}}%
\pgfpathlineto{\pgfqpoint{0.696279in}{0.902549in}}%
\pgfpathlineto{\pgfqpoint{0.696569in}{0.903653in}}%
\pgfpathlineto{\pgfqpoint{0.697672in}{0.909846in}}%
\pgfpathlineto{\pgfqpoint{0.697915in}{0.910911in}}%
\pgfpathlineto{\pgfqpoint{0.698989in}{0.917458in}}%
\pgfpathlineto{\pgfqpoint{0.699279in}{0.918523in}}%
\pgfpathlineto{\pgfqpoint{0.700354in}{0.923454in}}%
\pgfpathlineto{\pgfqpoint{0.700569in}{0.924479in}}%
\pgfpathlineto{\pgfqpoint{0.701672in}{0.929685in}}%
\pgfpathlineto{\pgfqpoint{0.702027in}{0.930790in}}%
\pgfpathlineto{\pgfqpoint{0.703130in}{0.937180in}}%
\pgfpathlineto{\pgfqpoint{0.703335in}{0.938284in}}%
\pgfpathlineto{\pgfqpoint{0.704429in}{0.943609in}}%
\pgfpathlineto{\pgfqpoint{0.704709in}{0.944634in}}%
\pgfpathlineto{\pgfqpoint{0.705765in}{0.949446in}}%
\pgfpathlineto{\pgfqpoint{0.706214in}{0.950472in}}%
\pgfpathlineto{\pgfqpoint{0.707307in}{0.955165in}}%
\pgfpathlineto{\pgfqpoint{0.707634in}{0.956270in}}%
\pgfpathlineto{\pgfqpoint{0.708718in}{0.961200in}}%
\pgfpathlineto{\pgfqpoint{0.708989in}{0.962265in}}%
\pgfpathlineto{\pgfqpoint{0.710083in}{0.966604in}}%
\pgfpathlineto{\pgfqpoint{0.710242in}{0.967708in}}%
\pgfpathlineto{\pgfqpoint{0.711335in}{0.972165in}}%
\pgfpathlineto{\pgfqpoint{0.711532in}{0.973269in}}%
\pgfpathlineto{\pgfqpoint{0.712625in}{0.978042in}}%
\pgfpathlineto{\pgfqpoint{0.712896in}{0.979107in}}%
\pgfpathlineto{\pgfqpoint{0.713980in}{0.983801in}}%
\pgfpathlineto{\pgfqpoint{0.714251in}{0.984905in}}%
\pgfpathlineto{\pgfqpoint{0.715354in}{0.990348in}}%
\pgfpathlineto{\pgfqpoint{0.715597in}{0.991452in}}%
\pgfpathlineto{\pgfqpoint{0.716681in}{0.996067in}}%
\pgfpathlineto{\pgfqpoint{0.717074in}{0.997172in}}%
\pgfpathlineto{\pgfqpoint{0.718177in}{1.001786in}}%
\pgfpathlineto{\pgfqpoint{0.718476in}{1.002851in}}%
\pgfpathlineto{\pgfqpoint{0.719578in}{1.007506in}}%
\pgfpathlineto{\pgfqpoint{0.719905in}{1.008531in}}%
\pgfpathlineto{\pgfqpoint{0.720999in}{1.012988in}}%
\pgfpathlineto{\pgfqpoint{0.721401in}{1.014092in}}%
\pgfpathlineto{\pgfqpoint{0.722504in}{1.017840in}}%
\pgfpathlineto{\pgfqpoint{0.722765in}{1.018904in}}%
\pgfpathlineto{\pgfqpoint{0.723859in}{1.022691in}}%
\pgfpathlineto{\pgfqpoint{0.724111in}{1.023756in}}%
\pgfpathlineto{\pgfqpoint{0.725205in}{1.026990in}}%
\pgfpathlineto{\pgfqpoint{0.725606in}{1.028055in}}%
\pgfpathlineto{\pgfqpoint{0.726709in}{1.031526in}}%
\pgfpathlineto{\pgfqpoint{0.727074in}{1.032630in}}%
\pgfpathlineto{\pgfqpoint{0.728177in}{1.036023in}}%
\pgfpathlineto{\pgfqpoint{0.728392in}{1.037009in}}%
\pgfpathlineto{\pgfqpoint{0.729485in}{1.041702in}}%
\pgfpathlineto{\pgfqpoint{0.729775in}{1.042767in}}%
\pgfpathlineto{\pgfqpoint{0.730878in}{1.046435in}}%
\pgfpathlineto{\pgfqpoint{0.731214in}{1.047540in}}%
\pgfpathlineto{\pgfqpoint{0.732308in}{1.050853in}}%
\pgfpathlineto{\pgfqpoint{0.732775in}{1.051957in}}%
\pgfpathlineto{\pgfqpoint{0.733878in}{1.055152in}}%
\pgfpathlineto{\pgfqpoint{0.734093in}{1.056178in}}%
\pgfpathlineto{\pgfqpoint{0.735177in}{1.060043in}}%
\pgfpathlineto{\pgfqpoint{0.735551in}{1.061108in}}%
\pgfpathlineto{\pgfqpoint{0.736653in}{1.064618in}}%
\pgfpathlineto{\pgfqpoint{0.737046in}{1.065683in}}%
\pgfpathlineto{\pgfqpoint{0.738130in}{1.068681in}}%
\pgfpathlineto{\pgfqpoint{0.738551in}{1.069785in}}%
\pgfpathlineto{\pgfqpoint{0.739644in}{1.073887in}}%
\pgfpathlineto{\pgfqpoint{0.740027in}{1.074992in}}%
\pgfpathlineto{\pgfqpoint{0.741130in}{1.077871in}}%
\pgfpathlineto{\pgfqpoint{0.741551in}{1.078975in}}%
\pgfpathlineto{\pgfqpoint{0.742644in}{1.081973in}}%
\pgfpathlineto{\pgfqpoint{0.743233in}{1.083077in}}%
\pgfpathlineto{\pgfqpoint{0.744336in}{1.086036in}}%
\pgfpathlineto{\pgfqpoint{0.744934in}{1.087061in}}%
\pgfpathlineto{\pgfqpoint{0.746037in}{1.090769in}}%
\pgfpathlineto{\pgfqpoint{0.746448in}{1.091873in}}%
\pgfpathlineto{\pgfqpoint{0.747523in}{1.095778in}}%
\pgfpathlineto{\pgfqpoint{0.747887in}{1.096843in}}%
\pgfpathlineto{\pgfqpoint{0.748934in}{1.100551in}}%
\pgfpathlineto{\pgfqpoint{0.749280in}{1.101655in}}%
\pgfpathlineto{\pgfqpoint{0.750364in}{1.104574in}}%
\pgfpathlineto{\pgfqpoint{0.750756in}{1.105678in}}%
\pgfpathlineto{\pgfqpoint{0.751840in}{1.108360in}}%
\pgfpathlineto{\pgfqpoint{0.752298in}{1.109465in}}%
\pgfpathlineto{\pgfqpoint{0.753373in}{1.112107in}}%
\pgfpathlineto{\pgfqpoint{0.753999in}{1.113172in}}%
\pgfpathlineto{\pgfqpoint{0.755083in}{1.115696in}}%
\pgfpathlineto{\pgfqpoint{0.755569in}{1.116801in}}%
\pgfpathlineto{\pgfqpoint{0.756672in}{1.119838in}}%
\pgfpathlineto{\pgfqpoint{0.756981in}{1.120706in}}%
\pgfpathlineto{\pgfqpoint{0.758065in}{1.124532in}}%
\pgfpathlineto{\pgfqpoint{0.758551in}{1.125636in}}%
\pgfpathlineto{\pgfqpoint{0.759607in}{1.128949in}}%
\pgfpathlineto{\pgfqpoint{0.760168in}{1.130054in}}%
\pgfpathlineto{\pgfqpoint{0.761270in}{1.132854in}}%
\pgfpathlineto{\pgfqpoint{0.761999in}{1.133958in}}%
\pgfpathlineto{\pgfqpoint{0.763046in}{1.136680in}}%
\pgfpathlineto{\pgfqpoint{0.763822in}{1.137784in}}%
\pgfpathlineto{\pgfqpoint{0.764925in}{1.139875in}}%
\pgfpathlineto{\pgfqpoint{0.765355in}{1.140979in}}%
\pgfpathlineto{\pgfqpoint{0.766448in}{1.143622in}}%
\pgfpathlineto{\pgfqpoint{0.767065in}{1.144726in}}%
\pgfpathlineto{\pgfqpoint{0.768121in}{1.146896in}}%
\pgfpathlineto{\pgfqpoint{0.768672in}{1.148000in}}%
\pgfpathlineto{\pgfqpoint{0.769710in}{1.150485in}}%
\pgfpathlineto{\pgfqpoint{0.770327in}{1.151589in}}%
\pgfpathlineto{\pgfqpoint{0.771429in}{1.154587in}}%
\pgfpathlineto{\pgfqpoint{0.772000in}{1.155691in}}%
\pgfpathlineto{\pgfqpoint{0.773084in}{1.157506in}}%
\pgfpathlineto{\pgfqpoint{0.773813in}{1.158571in}}%
\pgfpathlineto{\pgfqpoint{0.774906in}{1.161016in}}%
\pgfpathlineto{\pgfqpoint{0.775476in}{1.162120in}}%
\pgfpathlineto{\pgfqpoint{0.776570in}{1.164250in}}%
\pgfpathlineto{\pgfqpoint{0.777140in}{1.165355in}}%
\pgfpathlineto{\pgfqpoint{0.778177in}{1.167642in}}%
\pgfpathlineto{\pgfqpoint{0.778850in}{1.168747in}}%
\pgfpathlineto{\pgfqpoint{0.779953in}{1.170600in}}%
\pgfpathlineto{\pgfqpoint{0.780392in}{1.171508in}}%
\pgfpathlineto{\pgfqpoint{0.781476in}{1.173874in}}%
\pgfpathlineto{\pgfqpoint{0.781953in}{1.174979in}}%
\pgfpathlineto{\pgfqpoint{0.783056in}{1.176793in}}%
\pgfpathlineto{\pgfqpoint{0.783635in}{1.177897in}}%
\pgfpathlineto{\pgfqpoint{0.784729in}{1.179396in}}%
\pgfpathlineto{\pgfqpoint{0.785374in}{1.180501in}}%
\pgfpathlineto{\pgfqpoint{0.786439in}{1.182512in}}%
\pgfpathlineto{\pgfqpoint{0.787121in}{1.183617in}}%
\pgfpathlineto{\pgfqpoint{0.788224in}{1.185589in}}%
\pgfpathlineto{\pgfqpoint{0.788934in}{1.186693in}}%
\pgfpathlineto{\pgfqpoint{0.789878in}{1.187916in}}%
\pgfpathlineto{\pgfqpoint{0.790607in}{1.188981in}}%
\pgfpathlineto{\pgfqpoint{0.791617in}{1.191189in}}%
\pgfpathlineto{\pgfqpoint{0.792271in}{1.192294in}}%
\pgfpathlineto{\pgfqpoint{0.793364in}{1.193990in}}%
\pgfpathlineto{\pgfqpoint{0.793916in}{1.195055in}}%
\pgfpathlineto{\pgfqpoint{0.795009in}{1.196790in}}%
\pgfpathlineto{\pgfqpoint{0.795551in}{1.197855in}}%
\pgfpathlineto{\pgfqpoint{0.796626in}{1.199709in}}%
\pgfpathlineto{\pgfqpoint{0.797346in}{1.200813in}}%
\pgfpathlineto{\pgfqpoint{0.798439in}{1.203101in}}%
\pgfpathlineto{\pgfqpoint{0.799196in}{1.204206in}}%
\pgfpathlineto{\pgfqpoint{0.800299in}{1.206257in}}%
\pgfpathlineto{\pgfqpoint{0.800720in}{1.207361in}}%
\pgfpathlineto{\pgfqpoint{0.801785in}{1.209294in}}%
\pgfpathlineto{\pgfqpoint{0.802467in}{1.210398in}}%
\pgfpathlineto{\pgfqpoint{0.803561in}{1.212843in}}%
\pgfpathlineto{\pgfqpoint{0.804168in}{1.213948in}}%
\pgfpathlineto{\pgfqpoint{0.805252in}{1.215723in}}%
\pgfpathlineto{\pgfqpoint{0.806103in}{1.216827in}}%
\pgfpathlineto{\pgfqpoint{0.807206in}{1.219312in}}%
\pgfpathlineto{\pgfqpoint{0.807841in}{1.220416in}}%
\pgfpathlineto{\pgfqpoint{0.808897in}{1.222073in}}%
\pgfpathlineto{\pgfqpoint{0.809421in}{1.223177in}}%
\pgfpathlineto{\pgfqpoint{0.810514in}{1.224755in}}%
\pgfpathlineto{\pgfqpoint{0.811327in}{1.225859in}}%
\pgfpathlineto{\pgfqpoint{0.812421in}{1.228029in}}%
\pgfpathlineto{\pgfqpoint{0.812888in}{1.229133in}}%
\pgfpathlineto{\pgfqpoint{0.813907in}{1.230593in}}%
\pgfpathlineto{\pgfqpoint{0.814776in}{1.231697in}}%
\pgfpathlineto{\pgfqpoint{0.815879in}{1.233354in}}%
\pgfpathlineto{\pgfqpoint{0.816505in}{1.234340in}}%
\pgfpathlineto{\pgfqpoint{0.817598in}{1.236272in}}%
\pgfpathlineto{\pgfqpoint{0.817981in}{1.237377in}}%
\pgfpathlineto{\pgfqpoint{0.819075in}{1.239191in}}%
\pgfpathlineto{\pgfqpoint{0.819748in}{1.240295in}}%
\pgfpathlineto{\pgfqpoint{0.820832in}{1.241400in}}%
\pgfpathlineto{\pgfqpoint{0.821654in}{1.242465in}}%
\pgfpathlineto{\pgfqpoint{0.822692in}{1.243806in}}%
\pgfpathlineto{\pgfqpoint{0.823486in}{1.244831in}}%
\pgfpathlineto{\pgfqpoint{0.824552in}{1.246527in}}%
\pgfpathlineto{\pgfqpoint{0.825477in}{1.247632in}}%
\pgfpathlineto{\pgfqpoint{0.826580in}{1.249880in}}%
\pgfpathlineto{\pgfqpoint{0.827103in}{1.250984in}}%
\pgfpathlineto{\pgfqpoint{0.828196in}{1.252523in}}%
\pgfpathlineto{\pgfqpoint{0.829243in}{1.253627in}}%
\pgfpathlineto{\pgfqpoint{0.830299in}{1.255402in}}%
\pgfpathlineto{\pgfqpoint{0.831253in}{1.256467in}}%
\pgfpathlineto{\pgfqpoint{0.832346in}{1.258045in}}%
\pgfpathlineto{\pgfqpoint{0.833533in}{1.259149in}}%
\pgfpathlineto{\pgfqpoint{0.834608in}{1.260490in}}%
\pgfpathlineto{\pgfqpoint{0.835542in}{1.261594in}}%
\pgfpathlineto{\pgfqpoint{0.836598in}{1.262738in}}%
\pgfpathlineto{\pgfqpoint{0.837785in}{1.263843in}}%
\pgfpathlineto{\pgfqpoint{0.838879in}{1.265815in}}%
\pgfpathlineto{\pgfqpoint{0.839683in}{1.266919in}}%
\pgfpathlineto{\pgfqpoint{0.840739in}{1.268260in}}%
\pgfpathlineto{\pgfqpoint{0.841337in}{1.269365in}}%
\pgfpathlineto{\pgfqpoint{0.842412in}{1.270863in}}%
\pgfpathlineto{\pgfqpoint{0.843458in}{1.271968in}}%
\pgfpathlineto{\pgfqpoint{0.844542in}{1.273112in}}%
\pgfpathlineto{\pgfqpoint{0.845860in}{1.274216in}}%
\pgfpathlineto{\pgfqpoint{0.846935in}{1.275439in}}%
\pgfpathlineto{\pgfqpoint{0.847748in}{1.276504in}}%
\pgfpathlineto{\pgfqpoint{0.848842in}{1.278318in}}%
\pgfpathlineto{\pgfqpoint{0.849636in}{1.279383in}}%
\pgfpathlineto{\pgfqpoint{0.850739in}{1.280448in}}%
\pgfpathlineto{\pgfqpoint{0.851935in}{1.281513in}}%
\pgfpathlineto{\pgfqpoint{0.853038in}{1.283091in}}%
\pgfpathlineto{\pgfqpoint{0.854225in}{1.284195in}}%
\pgfpathlineto{\pgfqpoint{0.855225in}{1.285418in}}%
\pgfpathlineto{\pgfqpoint{0.856094in}{1.286522in}}%
\pgfpathlineto{\pgfqpoint{0.857038in}{1.287982in}}%
\pgfpathlineto{\pgfqpoint{0.857860in}{1.289086in}}%
\pgfpathlineto{\pgfqpoint{0.858944in}{1.290506in}}%
\pgfpathlineto{\pgfqpoint{0.859954in}{1.291610in}}%
\pgfpathlineto{\pgfqpoint{0.861057in}{1.293109in}}%
\pgfpathlineto{\pgfqpoint{0.861786in}{1.294174in}}%
\pgfpathlineto{\pgfqpoint{0.862888in}{1.295515in}}%
\pgfpathlineto{\pgfqpoint{0.863524in}{1.296619in}}%
\pgfpathlineto{\pgfqpoint{0.864608in}{1.297605in}}%
\pgfpathlineto{\pgfqpoint{0.865748in}{1.298710in}}%
\pgfpathlineto{\pgfqpoint{0.866842in}{1.300011in}}%
\pgfpathlineto{\pgfqpoint{0.868122in}{1.301116in}}%
\pgfpathlineto{\pgfqpoint{0.869206in}{1.302417in}}%
\pgfpathlineto{\pgfqpoint{0.869973in}{1.303522in}}%
\pgfpathlineto{\pgfqpoint{0.871038in}{1.304508in}}%
\pgfpathlineto{\pgfqpoint{0.872786in}{1.305573in}}%
\pgfpathlineto{\pgfqpoint{0.873823in}{1.306520in}}%
\pgfpathlineto{\pgfqpoint{0.874870in}{1.307545in}}%
\pgfpathlineto{\pgfqpoint{0.875954in}{1.309004in}}%
\pgfpathlineto{\pgfqpoint{0.876935in}{1.310109in}}%
\pgfpathlineto{\pgfqpoint{0.877973in}{1.311292in}}%
\pgfpathlineto{\pgfqpoint{0.878973in}{1.312357in}}%
\pgfpathlineto{\pgfqpoint{0.880076in}{1.313580in}}%
\pgfpathlineto{\pgfqpoint{0.881216in}{1.314684in}}%
\pgfpathlineto{\pgfqpoint{0.882291in}{1.315867in}}%
\pgfpathlineto{\pgfqpoint{0.883534in}{1.316972in}}%
\pgfpathlineto{\pgfqpoint{0.884618in}{1.318037in}}%
\pgfpathlineto{\pgfqpoint{0.885870in}{1.319141in}}%
\pgfpathlineto{\pgfqpoint{0.886973in}{1.320679in}}%
\pgfpathlineto{\pgfqpoint{0.887907in}{1.321744in}}%
\pgfpathlineto{\pgfqpoint{0.889001in}{1.323125in}}%
\pgfpathlineto{\pgfqpoint{0.890870in}{1.324229in}}%
\pgfpathlineto{\pgfqpoint{0.891926in}{1.325294in}}%
\pgfpathlineto{\pgfqpoint{0.893216in}{1.326399in}}%
\pgfpathlineto{\pgfqpoint{0.894319in}{1.327306in}}%
\pgfpathlineto{\pgfqpoint{0.895823in}{1.328410in}}%
\pgfpathlineto{\pgfqpoint{0.896908in}{1.329357in}}%
\pgfpathlineto{\pgfqpoint{0.897795in}{1.330461in}}%
\pgfpathlineto{\pgfqpoint{0.898898in}{1.331368in}}%
\pgfpathlineto{\pgfqpoint{0.900590in}{1.332473in}}%
\pgfpathlineto{\pgfqpoint{0.901683in}{1.333735in}}%
\pgfpathlineto{\pgfqpoint{0.902674in}{1.334839in}}%
\pgfpathlineto{\pgfqpoint{0.903758in}{1.336023in}}%
\pgfpathlineto{\pgfqpoint{0.905085in}{1.337127in}}%
\pgfpathlineto{\pgfqpoint{0.906141in}{1.338310in}}%
\pgfpathlineto{\pgfqpoint{0.907440in}{1.339415in}}%
\pgfpathlineto{\pgfqpoint{0.908253in}{1.340046in}}%
\pgfpathlineto{\pgfqpoint{0.910067in}{1.341150in}}%
\pgfpathlineto{\pgfqpoint{0.911132in}{1.342254in}}%
\pgfpathlineto{\pgfqpoint{0.912282in}{1.343359in}}%
\pgfpathlineto{\pgfqpoint{0.913338in}{1.344187in}}%
\pgfpathlineto{\pgfqpoint{0.914908in}{1.345292in}}%
\pgfpathlineto{\pgfqpoint{0.916001in}{1.346396in}}%
\pgfpathlineto{\pgfqpoint{0.917169in}{1.347500in}}%
\pgfpathlineto{\pgfqpoint{0.918226in}{1.348486in}}%
\pgfpathlineto{\pgfqpoint{0.919627in}{1.349591in}}%
\pgfpathlineto{\pgfqpoint{0.920721in}{1.350537in}}%
\pgfpathlineto{\pgfqpoint{0.922104in}{1.351642in}}%
\pgfpathlineto{\pgfqpoint{0.923020in}{1.352983in}}%
\pgfpathlineto{\pgfqpoint{0.924515in}{1.354087in}}%
\pgfpathlineto{\pgfqpoint{0.925618in}{1.355073in}}%
\pgfpathlineto{\pgfqpoint{0.926618in}{1.356059in}}%
\pgfpathlineto{\pgfqpoint{0.927684in}{1.357045in}}%
\pgfpathlineto{\pgfqpoint{0.929188in}{1.358150in}}%
\pgfpathlineto{\pgfqpoint{0.930160in}{1.359215in}}%
\pgfpathlineto{\pgfqpoint{0.932039in}{1.360319in}}%
\pgfpathlineto{\pgfqpoint{0.933067in}{1.361147in}}%
\pgfpathlineto{\pgfqpoint{0.934871in}{1.362252in}}%
\pgfpathlineto{\pgfqpoint{0.935964in}{1.363238in}}%
\pgfpathlineto{\pgfqpoint{0.937544in}{1.364342in}}%
\pgfpathlineto{\pgfqpoint{0.938618in}{1.365289in}}%
\pgfpathlineto{\pgfqpoint{0.939843in}{1.366393in}}%
\pgfpathlineto{\pgfqpoint{0.940871in}{1.367300in}}%
\pgfpathlineto{\pgfqpoint{0.942534in}{1.368405in}}%
\pgfpathlineto{\pgfqpoint{0.943618in}{1.369194in}}%
\pgfpathlineto{\pgfqpoint{0.945656in}{1.370298in}}%
\pgfpathlineto{\pgfqpoint{0.946702in}{1.371087in}}%
\pgfpathlineto{\pgfqpoint{0.948039in}{1.372191in}}%
\pgfpathlineto{\pgfqpoint{0.949076in}{1.372822in}}%
\pgfpathlineto{\pgfqpoint{0.950908in}{1.373927in}}%
\pgfpathlineto{\pgfqpoint{0.952011in}{1.374716in}}%
\pgfpathlineto{\pgfqpoint{0.953665in}{1.375820in}}%
\pgfpathlineto{\pgfqpoint{0.954768in}{1.376727in}}%
\pgfpathlineto{\pgfqpoint{0.956544in}{1.377753in}}%
\pgfpathlineto{\pgfqpoint{0.957600in}{1.378660in}}%
\pgfpathlineto{\pgfqpoint{0.959749in}{1.379764in}}%
\pgfpathlineto{\pgfqpoint{0.960843in}{1.381026in}}%
\pgfpathlineto{\pgfqpoint{0.962749in}{1.382131in}}%
\pgfpathlineto{\pgfqpoint{0.963805in}{1.382841in}}%
\pgfpathlineto{\pgfqpoint{0.965516in}{1.383945in}}%
\pgfpathlineto{\pgfqpoint{0.966478in}{1.384695in}}%
\pgfpathlineto{\pgfqpoint{0.968376in}{1.385799in}}%
\pgfpathlineto{\pgfqpoint{0.969432in}{1.386588in}}%
\pgfpathlineto{\pgfqpoint{0.971806in}{1.387692in}}%
\pgfpathlineto{\pgfqpoint{0.972834in}{1.388402in}}%
\pgfpathlineto{\pgfqpoint{0.974637in}{1.389507in}}%
\pgfpathlineto{\pgfqpoint{0.975656in}{1.390217in}}%
\pgfpathlineto{\pgfqpoint{0.978049in}{1.391321in}}%
\pgfpathlineto{\pgfqpoint{0.978918in}{1.391755in}}%
\pgfpathlineto{\pgfqpoint{0.981011in}{1.392859in}}%
\pgfpathlineto{\pgfqpoint{0.981955in}{1.393451in}}%
\pgfpathlineto{\pgfqpoint{0.984105in}{1.394555in}}%
\pgfpathlineto{\pgfqpoint{0.985208in}{1.395226in}}%
\pgfpathlineto{\pgfqpoint{0.987245in}{1.396330in}}%
\pgfpathlineto{\pgfqpoint{0.988348in}{1.397080in}}%
\pgfpathlineto{\pgfqpoint{0.989853in}{1.398184in}}%
\pgfpathlineto{\pgfqpoint{0.990871in}{1.398815in}}%
\pgfpathlineto{\pgfqpoint{0.993329in}{1.399919in}}%
\pgfpathlineto{\pgfqpoint{0.994404in}{1.400551in}}%
\pgfpathlineto{\pgfqpoint{0.996797in}{1.401655in}}%
\pgfpathlineto{\pgfqpoint{0.997834in}{1.402207in}}%
\pgfpathlineto{\pgfqpoint{0.999955in}{1.403312in}}%
\pgfpathlineto{\pgfqpoint{1.000890in}{1.404021in}}%
\pgfpathlineto{\pgfqpoint{1.003096in}{1.405126in}}%
\pgfpathlineto{\pgfqpoint{1.004152in}{1.405718in}}%
\pgfpathlineto{\pgfqpoint{1.006600in}{1.406822in}}%
\pgfpathlineto{\pgfqpoint{1.007554in}{1.407374in}}%
\pgfpathlineto{\pgfqpoint{1.009133in}{1.408478in}}%
\pgfpathlineto{\pgfqpoint{1.010096in}{1.409188in}}%
\pgfpathlineto{\pgfqpoint{1.012217in}{1.410293in}}%
\pgfpathlineto{\pgfqpoint{1.013311in}{1.411003in}}%
\pgfpathlineto{\pgfqpoint{1.015395in}{1.412107in}}%
\pgfpathlineto{\pgfqpoint{1.016404in}{1.412620in}}%
\pgfpathlineto{\pgfqpoint{1.018853in}{1.413724in}}%
\pgfpathlineto{\pgfqpoint{1.019956in}{1.414316in}}%
\pgfpathlineto{\pgfqpoint{1.022283in}{1.415420in}}%
\pgfpathlineto{\pgfqpoint{1.023367in}{1.416170in}}%
\pgfpathlineto{\pgfqpoint{1.025788in}{1.417274in}}%
\pgfpathlineto{\pgfqpoint{1.026573in}{1.417866in}}%
\pgfpathlineto{\pgfqpoint{1.029180in}{1.418931in}}%
\pgfpathlineto{\pgfqpoint{1.030189in}{1.419246in}}%
\pgfpathlineto{\pgfqpoint{1.032545in}{1.420351in}}%
\pgfpathlineto{\pgfqpoint{1.033647in}{1.420824in}}%
\pgfpathlineto{\pgfqpoint{1.036461in}{1.421928in}}%
\pgfpathlineto{\pgfqpoint{1.037535in}{1.422441in}}%
\pgfpathlineto{\pgfqpoint{1.039984in}{1.423546in}}%
\pgfpathlineto{\pgfqpoint{1.040975in}{1.424255in}}%
\pgfpathlineto{\pgfqpoint{1.043928in}{1.425360in}}%
\pgfpathlineto{\pgfqpoint{1.045021in}{1.425991in}}%
\pgfpathlineto{\pgfqpoint{1.046984in}{1.427095in}}%
\pgfpathlineto{\pgfqpoint{1.047928in}{1.427529in}}%
\pgfpathlineto{\pgfqpoint{1.050685in}{1.428634in}}%
\pgfpathlineto{\pgfqpoint{1.051704in}{1.429344in}}%
\pgfpathlineto{\pgfqpoint{1.054685in}{1.430448in}}%
\pgfpathlineto{\pgfqpoint{1.055722in}{1.430961in}}%
\pgfpathlineto{\pgfqpoint{1.058012in}{1.432065in}}%
\pgfpathlineto{\pgfqpoint{1.059068in}{1.432420in}}%
\pgfpathlineto{\pgfqpoint{1.061367in}{1.433524in}}%
\pgfpathlineto{\pgfqpoint{1.062321in}{1.433998in}}%
\pgfpathlineto{\pgfqpoint{1.064975in}{1.435102in}}%
\pgfpathlineto{\pgfqpoint{1.065863in}{1.435260in}}%
\pgfpathlineto{\pgfqpoint{1.069237in}{1.436364in}}%
\pgfpathlineto{\pgfqpoint{1.070237in}{1.436838in}}%
\pgfpathlineto{\pgfqpoint{1.072984in}{1.437942in}}%
\pgfpathlineto{\pgfqpoint{1.073956in}{1.438336in}}%
\pgfpathlineto{\pgfqpoint{1.076517in}{1.439441in}}%
\pgfpathlineto{\pgfqpoint{1.077442in}{1.439993in}}%
\pgfpathlineto{\pgfqpoint{1.080797in}{1.441097in}}%
\pgfpathlineto{\pgfqpoint{1.081751in}{1.441531in}}%
\pgfpathlineto{\pgfqpoint{1.084583in}{1.442636in}}%
\pgfpathlineto{\pgfqpoint{1.085611in}{1.443188in}}%
\pgfpathlineto{\pgfqpoint{1.088284in}{1.444292in}}%
\pgfpathlineto{\pgfqpoint{1.089237in}{1.444687in}}%
\pgfpathlineto{\pgfqpoint{1.092676in}{1.445752in}}%
\pgfpathlineto{\pgfqpoint{1.093685in}{1.446422in}}%
\pgfpathlineto{\pgfqpoint{1.097031in}{1.447487in}}%
\pgfpathlineto{\pgfqpoint{1.098115in}{1.448079in}}%
\pgfpathlineto{\pgfqpoint{1.101471in}{1.449183in}}%
\pgfpathlineto{\pgfqpoint{1.102433in}{1.449735in}}%
\pgfpathlineto{\pgfqpoint{1.105723in}{1.450840in}}%
\pgfpathlineto{\pgfqpoint{1.106788in}{1.451195in}}%
\pgfpathlineto{\pgfqpoint{1.110293in}{1.452299in}}%
\pgfpathlineto{\pgfqpoint{1.111377in}{1.453009in}}%
\pgfpathlineto{\pgfqpoint{1.113891in}{1.454114in}}%
\pgfpathlineto{\pgfqpoint{1.114994in}{1.454587in}}%
\pgfpathlineto{\pgfqpoint{1.118265in}{1.455691in}}%
\pgfpathlineto{\pgfqpoint{1.119134in}{1.456283in}}%
\pgfpathlineto{\pgfqpoint{1.123125in}{1.457387in}}%
\pgfpathlineto{\pgfqpoint{1.124172in}{1.457742in}}%
\pgfpathlineto{\pgfqpoint{1.127284in}{1.458847in}}%
\pgfpathlineto{\pgfqpoint{1.128228in}{1.459123in}}%
\pgfpathlineto{\pgfqpoint{1.131602in}{1.460227in}}%
\pgfpathlineto{\pgfqpoint{1.132705in}{1.460819in}}%
\pgfpathlineto{\pgfqpoint{1.136443in}{1.461923in}}%
\pgfpathlineto{\pgfqpoint{1.137527in}{1.462436in}}%
\pgfpathlineto{\pgfqpoint{1.140340in}{1.463540in}}%
\pgfpathlineto{\pgfqpoint{1.141200in}{1.463895in}}%
\pgfpathlineto{\pgfqpoint{1.144434in}{1.465000in}}%
\pgfpathlineto{\pgfqpoint{1.145527in}{1.465473in}}%
\pgfpathlineto{\pgfqpoint{1.150163in}{1.466577in}}%
\pgfpathlineto{\pgfqpoint{1.151256in}{1.466932in}}%
\pgfpathlineto{\pgfqpoint{1.156312in}{1.468037in}}%
\pgfpathlineto{\pgfqpoint{1.157387in}{1.468549in}}%
\pgfpathlineto{\pgfqpoint{1.160967in}{1.469614in}}%
\pgfpathlineto{\pgfqpoint{1.162032in}{1.470246in}}%
\pgfpathlineto{\pgfqpoint{1.165013in}{1.471350in}}%
\pgfpathlineto{\pgfqpoint{1.166098in}{1.471744in}}%
\pgfpathlineto{\pgfqpoint{1.169621in}{1.472849in}}%
\pgfpathlineto{\pgfqpoint{1.170668in}{1.473283in}}%
\pgfpathlineto{\pgfqpoint{1.174985in}{1.474387in}}%
\pgfpathlineto{\pgfqpoint{1.175957in}{1.474703in}}%
\pgfpathlineto{\pgfqpoint{1.179593in}{1.475807in}}%
\pgfpathlineto{\pgfqpoint{1.180696in}{1.476122in}}%
\pgfpathlineto{\pgfqpoint{1.184098in}{1.477227in}}%
\pgfpathlineto{\pgfqpoint{1.185116in}{1.477582in}}%
\pgfpathlineto{\pgfqpoint{1.190388in}{1.478686in}}%
\pgfpathlineto{\pgfqpoint{1.191425in}{1.479120in}}%
\pgfpathlineto{\pgfqpoint{1.195453in}{1.480224in}}%
\pgfpathlineto{\pgfqpoint{1.196528in}{1.480501in}}%
\pgfpathlineto{\pgfqpoint{1.200247in}{1.481605in}}%
\pgfpathlineto{\pgfqpoint{1.201285in}{1.482039in}}%
\pgfpathlineto{\pgfqpoint{1.205911in}{1.483104in}}%
\pgfpathlineto{\pgfqpoint{1.206855in}{1.483577in}}%
\pgfpathlineto{\pgfqpoint{1.212892in}{1.484681in}}%
\pgfpathlineto{\pgfqpoint{1.213921in}{1.484958in}}%
\pgfpathlineto{\pgfqpoint{1.217874in}{1.486062in}}%
\pgfpathlineto{\pgfqpoint{1.218967in}{1.486496in}}%
\pgfpathlineto{\pgfqpoint{1.222556in}{1.487600in}}%
\pgfpathlineto{\pgfqpoint{1.223594in}{1.487837in}}%
\pgfpathlineto{\pgfqpoint{1.227566in}{1.488941in}}%
\pgfpathlineto{\pgfqpoint{1.228631in}{1.489296in}}%
\pgfpathlineto{\pgfqpoint{1.231846in}{1.490401in}}%
\pgfpathlineto{\pgfqpoint{1.232911in}{1.490598in}}%
\pgfpathlineto{\pgfqpoint{1.240323in}{1.491702in}}%
\pgfpathlineto{\pgfqpoint{1.241407in}{1.492136in}}%
\pgfpathlineto{\pgfqpoint{1.246164in}{1.493241in}}%
\pgfpathlineto{\pgfqpoint{1.246968in}{1.493477in}}%
\pgfpathlineto{\pgfqpoint{1.251827in}{1.494582in}}%
\pgfpathlineto{\pgfqpoint{1.252827in}{1.494976in}}%
\pgfpathlineto{\pgfqpoint{1.259089in}{1.496080in}}%
\pgfpathlineto{\pgfqpoint{1.259968in}{1.496356in}}%
\pgfpathlineto{\pgfqpoint{1.265295in}{1.497461in}}%
\pgfpathlineto{\pgfqpoint{1.266267in}{1.497895in}}%
\pgfpathlineto{\pgfqpoint{1.270949in}{1.498999in}}%
\pgfpathlineto{\pgfqpoint{1.271846in}{1.499472in}}%
\pgfpathlineto{\pgfqpoint{1.277641in}{1.500577in}}%
\pgfpathlineto{\pgfqpoint{1.278669in}{1.501011in}}%
\pgfpathlineto{\pgfqpoint{1.282847in}{1.502115in}}%
\pgfpathlineto{\pgfqpoint{1.283725in}{1.502786in}}%
\pgfpathlineto{\pgfqpoint{1.289725in}{1.503890in}}%
\pgfpathlineto{\pgfqpoint{1.290781in}{1.504206in}}%
\pgfpathlineto{\pgfqpoint{1.297342in}{1.505310in}}%
\pgfpathlineto{\pgfqpoint{1.298314in}{1.505468in}}%
\pgfpathlineto{\pgfqpoint{1.303211in}{1.506572in}}%
\pgfpathlineto{\pgfqpoint{1.304211in}{1.506927in}}%
\pgfpathlineto{\pgfqpoint{1.311342in}{1.508031in}}%
\pgfpathlineto{\pgfqpoint{1.312426in}{1.508347in}}%
\pgfpathlineto{\pgfqpoint{1.319847in}{1.509451in}}%
\pgfpathlineto{\pgfqpoint{1.320931in}{1.509885in}}%
\pgfpathlineto{\pgfqpoint{1.325866in}{1.510990in}}%
\pgfpathlineto{\pgfqpoint{1.326492in}{1.511226in}}%
\pgfpathlineto{\pgfqpoint{1.333324in}{1.512331in}}%
\pgfpathlineto{\pgfqpoint{1.334370in}{1.512607in}}%
\pgfpathlineto{\pgfqpoint{1.342670in}{1.513711in}}%
\pgfpathlineto{\pgfqpoint{1.343539in}{1.513869in}}%
\pgfpathlineto{\pgfqpoint{1.348053in}{1.514973in}}%
\pgfpathlineto{\pgfqpoint{1.349081in}{1.515328in}}%
\pgfpathlineto{\pgfqpoint{1.356445in}{1.516433in}}%
\pgfpathlineto{\pgfqpoint{1.357399in}{1.516630in}}%
\pgfpathlineto{\pgfqpoint{1.366137in}{1.517734in}}%
\pgfpathlineto{\pgfqpoint{1.367090in}{1.518050in}}%
\pgfpathlineto{\pgfqpoint{1.373707in}{1.519154in}}%
\pgfpathlineto{\pgfqpoint{1.374810in}{1.519391in}}%
\pgfpathlineto{\pgfqpoint{1.381119in}{1.520495in}}%
\pgfpathlineto{\pgfqpoint{1.382072in}{1.520693in}}%
\pgfpathlineto{\pgfqpoint{1.389633in}{1.521797in}}%
\pgfpathlineto{\pgfqpoint{1.390736in}{1.521955in}}%
\pgfpathlineto{\pgfqpoint{1.397212in}{1.523059in}}%
\pgfpathlineto{\pgfqpoint{1.398138in}{1.523138in}}%
\pgfpathlineto{\pgfqpoint{1.403979in}{1.524242in}}%
\pgfpathlineto{\pgfqpoint{1.404642in}{1.524361in}}%
\pgfpathlineto{\pgfqpoint{1.413100in}{1.525465in}}%
\pgfpathlineto{\pgfqpoint{1.413988in}{1.525662in}}%
\pgfpathlineto{\pgfqpoint{1.422951in}{1.526767in}}%
\pgfpathlineto{\pgfqpoint{1.423970in}{1.526924in}}%
\pgfpathlineto{\pgfqpoint{1.431661in}{1.528029in}}%
\pgfpathlineto{\pgfqpoint{1.432026in}{1.528147in}}%
\pgfpathlineto{\pgfqpoint{1.441671in}{1.529252in}}%
\pgfpathlineto{\pgfqpoint{1.442475in}{1.529409in}}%
\pgfpathlineto{\pgfqpoint{1.455456in}{1.530514in}}%
\pgfpathlineto{\pgfqpoint{1.456475in}{1.530632in}}%
\pgfpathlineto{\pgfqpoint{1.466662in}{1.531736in}}%
\pgfpathlineto{\pgfqpoint{1.467699in}{1.532013in}}%
\pgfpathlineto{\pgfqpoint{1.477288in}{1.533117in}}%
\pgfpathlineto{\pgfqpoint{1.477297in}{1.533196in}}%
\pgfpathlineto{\pgfqpoint{1.487157in}{1.534300in}}%
\pgfpathlineto{\pgfqpoint{1.488232in}{1.534616in}}%
\pgfpathlineto{\pgfqpoint{1.496615in}{1.535720in}}%
\pgfpathlineto{\pgfqpoint{1.497139in}{1.535917in}}%
\pgfpathlineto{\pgfqpoint{1.508559in}{1.537022in}}%
\pgfpathlineto{\pgfqpoint{1.509382in}{1.537101in}}%
\pgfpathlineto{\pgfqpoint{1.517681in}{1.538205in}}%
\pgfpathlineto{\pgfqpoint{1.517737in}{1.538284in}}%
\pgfpathlineto{\pgfqpoint{1.529877in}{1.539388in}}%
\pgfpathlineto{\pgfqpoint{1.530933in}{1.539625in}}%
\pgfpathlineto{\pgfqpoint{1.538691in}{1.540729in}}%
\pgfpathlineto{\pgfqpoint{1.539597in}{1.540927in}}%
\pgfpathlineto{\pgfqpoint{1.549859in}{1.542031in}}%
\pgfpathlineto{\pgfqpoint{1.550952in}{1.542268in}}%
\pgfpathlineto{\pgfqpoint{1.559345in}{1.543372in}}%
\pgfpathlineto{\pgfqpoint{1.560336in}{1.543490in}}%
\pgfpathlineto{\pgfqpoint{1.571560in}{1.544595in}}%
\pgfpathlineto{\pgfqpoint{1.572579in}{1.544752in}}%
\pgfpathlineto{\pgfqpoint{1.580943in}{1.545857in}}%
\pgfpathlineto{\pgfqpoint{1.581224in}{1.545936in}}%
\pgfpathlineto{\pgfqpoint{1.592205in}{1.547040in}}%
\pgfpathlineto{\pgfqpoint{1.592887in}{1.547237in}}%
\pgfpathlineto{\pgfqpoint{1.603757in}{1.548342in}}%
\pgfpathlineto{\pgfqpoint{1.604056in}{1.548421in}}%
\pgfpathlineto{\pgfqpoint{1.612327in}{1.549525in}}%
\pgfpathlineto{\pgfqpoint{1.612570in}{1.549604in}}%
\pgfpathlineto{\pgfqpoint{1.629720in}{1.550708in}}%
\pgfpathlineto{\pgfqpoint{1.630514in}{1.550787in}}%
\pgfpathlineto{\pgfqpoint{1.639318in}{1.551379in}}%
\pgfpathlineto{\pgfqpoint{1.640281in}{1.560806in}}%
\pgfpathlineto{\pgfqpoint{1.653505in}{1.561910in}}%
\pgfpathlineto{\pgfqpoint{1.653589in}{1.561989in}}%
\pgfpathlineto{\pgfqpoint{1.666430in}{1.563093in}}%
\pgfpathlineto{\pgfqpoint{1.667281in}{1.563251in}}%
\pgfpathlineto{\pgfqpoint{1.680365in}{1.564355in}}%
\pgfpathlineto{\pgfqpoint{1.681374in}{1.564632in}}%
\pgfpathlineto{\pgfqpoint{1.692608in}{1.565736in}}%
\pgfpathlineto{\pgfqpoint{1.693683in}{1.565854in}}%
\pgfpathlineto{\pgfqpoint{1.704758in}{1.566959in}}%
\pgfpathlineto{\pgfqpoint{1.705655in}{1.567156in}}%
\pgfpathlineto{\pgfqpoint{1.715627in}{1.568260in}}%
\pgfpathlineto{\pgfqpoint{1.716702in}{1.568379in}}%
\pgfpathlineto{\pgfqpoint{1.729721in}{1.569483in}}%
\pgfpathlineto{\pgfqpoint{1.730796in}{1.569601in}}%
\pgfpathlineto{\pgfqpoint{1.745375in}{1.570706in}}%
\pgfpathlineto{\pgfqpoint{1.745628in}{1.570824in}}%
\pgfpathlineto{\pgfqpoint{1.762282in}{1.571928in}}%
\pgfpathlineto{\pgfqpoint{1.762488in}{1.572047in}}%
\pgfpathlineto{\pgfqpoint{1.773693in}{1.573151in}}%
\pgfpathlineto{\pgfqpoint{1.774506in}{1.573388in}}%
\pgfpathlineto{\pgfqpoint{1.786890in}{1.574492in}}%
\pgfpathlineto{\pgfqpoint{1.787815in}{1.574729in}}%
\pgfpathlineto{\pgfqpoint{1.797852in}{1.575833in}}%
\pgfpathlineto{\pgfqpoint{1.797852in}{1.575873in}}%
\pgfpathlineto{\pgfqpoint{1.810750in}{1.576977in}}%
\pgfpathlineto{\pgfqpoint{1.811479in}{1.577095in}}%
\pgfpathlineto{\pgfqpoint{1.825956in}{1.578200in}}%
\pgfpathlineto{\pgfqpoint{1.826451in}{1.578318in}}%
\pgfpathlineto{\pgfqpoint{1.839993in}{1.579422in}}%
\pgfpathlineto{\pgfqpoint{1.840844in}{1.579620in}}%
\pgfpathlineto{\pgfqpoint{1.855535in}{1.580724in}}%
\pgfpathlineto{\pgfqpoint{1.856517in}{1.580921in}}%
\pgfpathlineto{\pgfqpoint{1.870582in}{1.582026in}}%
\pgfpathlineto{\pgfqpoint{1.871582in}{1.582223in}}%
\pgfpathlineto{\pgfqpoint{1.884424in}{1.583327in}}%
\pgfpathlineto{\pgfqpoint{1.885349in}{1.583446in}}%
\pgfpathlineto{\pgfqpoint{1.903807in}{1.584550in}}%
\pgfpathlineto{\pgfqpoint{1.904611in}{1.584787in}}%
\pgfpathlineto{\pgfqpoint{1.922433in}{1.585891in}}%
\pgfpathlineto{\pgfqpoint{1.923489in}{1.585970in}}%
\pgfpathlineto{\pgfqpoint{1.934218in}{1.587074in}}%
\pgfpathlineto{\pgfqpoint{1.935312in}{1.587153in}}%
\pgfpathlineto{\pgfqpoint{1.948835in}{1.588258in}}%
\pgfpathlineto{\pgfqpoint{1.949480in}{1.588376in}}%
\pgfpathlineto{\pgfqpoint{1.964910in}{1.589480in}}%
\pgfpathlineto{\pgfqpoint{1.965752in}{1.589638in}}%
\pgfpathlineto{\pgfqpoint{1.975574in}{1.590742in}}%
\pgfpathlineto{\pgfqpoint{1.976284in}{1.590900in}}%
\pgfpathlineto{\pgfqpoint{1.987434in}{1.592005in}}%
\pgfpathlineto{\pgfqpoint{1.988490in}{1.592281in}}%
\pgfpathlineto{\pgfqpoint{2.000387in}{1.593385in}}%
\pgfpathlineto{\pgfqpoint{2.001444in}{1.593661in}}%
\pgfpathlineto{\pgfqpoint{2.008930in}{1.594766in}}%
\pgfpathlineto{\pgfqpoint{2.010032in}{1.594963in}}%
\pgfpathlineto{\pgfqpoint{2.018135in}{1.596067in}}%
\pgfpathlineto{\pgfqpoint{2.018995in}{1.596264in}}%
\pgfpathlineto{\pgfqpoint{2.024892in}{1.597369in}}%
\pgfpathlineto{\pgfqpoint{2.025958in}{1.597605in}}%
\pgfpathlineto{\pgfqpoint{2.029584in}{1.598710in}}%
\pgfpathlineto{\pgfqpoint{2.030481in}{1.599144in}}%
\pgfpathlineto{\pgfqpoint{2.032799in}{1.600209in}}%
\pgfpathlineto{\pgfqpoint{2.033126in}{1.601944in}}%
\pgfpathlineto{\pgfqpoint{2.033126in}{1.601944in}}%
\pgfusepath{stroke}%
\end{pgfscope}%
\begin{pgfscope}%
\pgfsetrectcap%
\pgfsetmiterjoin%
\pgfsetlinewidth{0.803000pt}%
\definecolor{currentstroke}{rgb}{0.000000,0.000000,0.000000}%
\pgfsetstrokecolor{currentstroke}%
\pgfsetdash{}{0pt}%
\pgfpathmoveto{\pgfqpoint{0.553581in}{0.499444in}}%
\pgfpathlineto{\pgfqpoint{0.553581in}{1.654444in}}%
\pgfusepath{stroke}%
\end{pgfscope}%
\begin{pgfscope}%
\pgfsetrectcap%
\pgfsetmiterjoin%
\pgfsetlinewidth{0.803000pt}%
\definecolor{currentstroke}{rgb}{0.000000,0.000000,0.000000}%
\pgfsetstrokecolor{currentstroke}%
\pgfsetdash{}{0pt}%
\pgfpathmoveto{\pgfqpoint{2.103581in}{0.499444in}}%
\pgfpathlineto{\pgfqpoint{2.103581in}{1.654444in}}%
\pgfusepath{stroke}%
\end{pgfscope}%
\begin{pgfscope}%
\pgfsetrectcap%
\pgfsetmiterjoin%
\pgfsetlinewidth{0.803000pt}%
\definecolor{currentstroke}{rgb}{0.000000,0.000000,0.000000}%
\pgfsetstrokecolor{currentstroke}%
\pgfsetdash{}{0pt}%
\pgfpathmoveto{\pgfqpoint{0.553581in}{0.499444in}}%
\pgfpathlineto{\pgfqpoint{2.103581in}{0.499444in}}%
\pgfusepath{stroke}%
\end{pgfscope}%
\begin{pgfscope}%
\pgfsetrectcap%
\pgfsetmiterjoin%
\pgfsetlinewidth{0.803000pt}%
\definecolor{currentstroke}{rgb}{0.000000,0.000000,0.000000}%
\pgfsetstrokecolor{currentstroke}%
\pgfsetdash{}{0pt}%
\pgfpathmoveto{\pgfqpoint{0.553581in}{1.654444in}}%
\pgfpathlineto{\pgfqpoint{2.103581in}{1.654444in}}%
\pgfusepath{stroke}%
\end{pgfscope}%
\begin{pgfscope}%
\pgfsetbuttcap%
\pgfsetmiterjoin%
\definecolor{currentfill}{rgb}{1.000000,1.000000,1.000000}%
\pgfsetfillcolor{currentfill}%
\pgfsetlinewidth{0.000000pt}%
\definecolor{currentstroke}{rgb}{0.000000,0.000000,0.000000}%
\pgfsetstrokecolor{currentstroke}%
\pgfsetstrokeopacity{0.000000}%
\pgfsetdash{}{0pt}%
\pgfpathmoveto{\pgfqpoint{0.686183in}{0.966899in}}%
\pgfpathlineto{\pgfqpoint{1.085905in}{0.966899in}}%
\pgfpathlineto{\pgfqpoint{1.085905in}{1.173565in}}%
\pgfpathlineto{\pgfqpoint{0.686183in}{1.173565in}}%
\pgfpathlineto{\pgfqpoint{0.686183in}{0.966899in}}%
\pgfpathclose%
\pgfusepath{fill}%
\end{pgfscope}%
\begin{pgfscope}%
\definecolor{textcolor}{rgb}{0.000000,0.000000,0.000000}%
\pgfsetstrokecolor{textcolor}%
\pgfsetfillcolor{textcolor}%
\pgftext[x=0.727850in,y=1.035510in,left,base]{\color{textcolor}\rmfamily\fontsize{10.000000}{12.000000}\selectfont 0.337}%
\end{pgfscope}%
\begin{pgfscope}%
\pgfsetbuttcap%
\pgfsetmiterjoin%
\definecolor{currentfill}{rgb}{1.000000,1.000000,1.000000}%
\pgfsetfillcolor{currentfill}%
\pgfsetlinewidth{0.000000pt}%
\definecolor{currentstroke}{rgb}{0.000000,0.000000,0.000000}%
\pgfsetstrokecolor{currentstroke}%
\pgfsetstrokeopacity{0.000000}%
\pgfsetdash{}{0pt}%
\pgfpathmoveto{\pgfqpoint{0.582378in}{0.483333in}}%
\pgfpathlineto{\pgfqpoint{0.843211in}{0.483333in}}%
\pgfpathlineto{\pgfqpoint{0.843211in}{0.690000in}}%
\pgfpathlineto{\pgfqpoint{0.582378in}{0.690000in}}%
\pgfpathlineto{\pgfqpoint{0.582378in}{0.483333in}}%
\pgfpathclose%
\pgfusepath{fill}%
\end{pgfscope}%
\begin{pgfscope}%
\definecolor{textcolor}{rgb}{0.000000,0.000000,0.000000}%
\pgfsetstrokecolor{textcolor}%
\pgfsetfillcolor{textcolor}%
\pgftext[x=0.624045in,y=0.551944in,left,base]{\color{textcolor}\rmfamily\fontsize{10.000000}{12.000000}\selectfont 0.5}%
\end{pgfscope}%
\begin{pgfscope}%
\pgfsetbuttcap%
\pgfsetmiterjoin%
\definecolor{currentfill}{rgb}{1.000000,1.000000,1.000000}%
\pgfsetfillcolor{currentfill}%
\pgfsetlinewidth{0.000000pt}%
\definecolor{currentstroke}{rgb}{0.000000,0.000000,0.000000}%
\pgfsetstrokecolor{currentstroke}%
\pgfsetstrokeopacity{0.000000}%
\pgfsetdash{}{0pt}%
\pgfpathmoveto{\pgfqpoint{0.582378in}{0.483333in}}%
\pgfpathlineto{\pgfqpoint{0.982100in}{0.483333in}}%
\pgfpathlineto{\pgfqpoint{0.982100in}{0.690000in}}%
\pgfpathlineto{\pgfqpoint{0.582378in}{0.690000in}}%
\pgfpathlineto{\pgfqpoint{0.582378in}{0.483333in}}%
\pgfpathclose%
\pgfusepath{fill}%
\end{pgfscope}%
\begin{pgfscope}%
\definecolor{textcolor}{rgb}{0.000000,0.000000,0.000000}%
\pgfsetstrokecolor{textcolor}%
\pgfsetfillcolor{textcolor}%
\pgftext[x=0.624045in,y=0.551944in,left,base]{\color{textcolor}\rmfamily\fontsize{10.000000}{12.000000}\selectfont 0.656}%
\end{pgfscope}%
\begin{pgfscope}%
\pgfsetbuttcap%
\pgfsetmiterjoin%
\definecolor{currentfill}{rgb}{1.000000,1.000000,1.000000}%
\pgfsetfillcolor{currentfill}%
\pgfsetfillopacity{0.800000}%
\pgfsetlinewidth{1.003750pt}%
\definecolor{currentstroke}{rgb}{0.800000,0.800000,0.800000}%
\pgfsetstrokecolor{currentstroke}%
\pgfsetstrokeopacity{0.800000}%
\pgfsetdash{}{0pt}%
\pgfpathmoveto{\pgfqpoint{0.832747in}{0.568889in}}%
\pgfpathlineto{\pgfqpoint{2.006358in}{0.568889in}}%
\pgfpathquadraticcurveto{\pgfqpoint{2.034136in}{0.568889in}}{\pgfqpoint{2.034136in}{0.596666in}}%
\pgfpathlineto{\pgfqpoint{2.034136in}{0.776388in}}%
\pgfpathquadraticcurveto{\pgfqpoint{2.034136in}{0.804166in}}{\pgfqpoint{2.006358in}{0.804166in}}%
\pgfpathlineto{\pgfqpoint{0.832747in}{0.804166in}}%
\pgfpathquadraticcurveto{\pgfqpoint{0.804970in}{0.804166in}}{\pgfqpoint{0.804970in}{0.776388in}}%
\pgfpathlineto{\pgfqpoint{0.804970in}{0.596666in}}%
\pgfpathquadraticcurveto{\pgfqpoint{0.804970in}{0.568889in}}{\pgfqpoint{0.832747in}{0.568889in}}%
\pgfpathlineto{\pgfqpoint{0.832747in}{0.568889in}}%
\pgfpathclose%
\pgfusepath{stroke,fill}%
\end{pgfscope}%
\begin{pgfscope}%
\pgfsetrectcap%
\pgfsetroundjoin%
\pgfsetlinewidth{1.505625pt}%
\definecolor{currentstroke}{rgb}{0.000000,0.000000,0.000000}%
\pgfsetstrokecolor{currentstroke}%
\pgfsetdash{}{0pt}%
\pgfpathmoveto{\pgfqpoint{0.860525in}{0.700000in}}%
\pgfpathlineto{\pgfqpoint{0.999414in}{0.700000in}}%
\pgfpathlineto{\pgfqpoint{1.138303in}{0.700000in}}%
\pgfusepath{stroke}%
\end{pgfscope}%
\begin{pgfscope}%
\definecolor{textcolor}{rgb}{0.000000,0.000000,0.000000}%
\pgfsetstrokecolor{textcolor}%
\pgfsetfillcolor{textcolor}%
\pgftext[x=1.249414in,y=0.651388in,left,base]{\color{textcolor}\rmfamily\fontsize{10.000000}{12.000000}\selectfont AUC=0.832}%
\end{pgfscope}%
\end{pgfpicture}%
\makeatother%
\endgroup%

  &
\vspace{0pt} 
  
\begin{tabular}{cc|c|c|}
	&\multicolumn{1}{c}{}& \multicolumn{2}{c}{Prediction} \cr
	&\multicolumn{1}{c}{} & \multicolumn{1}{c}{N} & \multicolumn{1}{c}{P} \cr\cline{3-4}
	\multirow{2}{*}{\rotatebox[origin=c]{90}{Actual}}&N & 117,929 & 32,842 \vrule width 0pt height 10pt depth 2pt \cr\cline{3-4}
	&P & 5,928 & 20,693 \vrule width 0pt height 10pt depth 2pt \cr\cline{3-4}
\end{tabular}

\begin{center}
\begin{tabular}{ll}
0.387 & Precision \cr 
0.777 & Recall \cr 
0.516 & F1 \cr 
\end{tabular}
\end{center}
  
\end{tabular}

Because Examples 3, 4, and 5 are linear transformations of Example 1, when we find the value of decision threshold $p$ that makes $\Delta FP/\Delta TP = 2.0$ and linearly transform the probabilities, each of them becomes the distribution in Example 2.  


