%%%
\subsection{Model Building}

To build our model we primarily used
 scikit-learn \citep{scikit-learn} and imbalanced-learn \citep{Imblearn}, and also
Keras/Tensorflow \citep{chollet2015keras} for the Focal Loss model.  

We will explain our methods with examples.  

The histogram below illustrates the kind of results we can realistically hope for in a good model.  The white boxes represent the 150,771 negative samples in the test set, and the black boxes the 26,621 positive samples.  The model evaluates each sample and gives it a probability $p \in (0,1)$ that the sample is in the positive class.  Ideally we want the model to  give most of the negative samples probabilities close to zero, and the positive samples probabilities close to one, but some will be hard to classify correctly.  The ROC curve below shows the median value of $p$ for the negative (0.338) and positive (0.658) classes.  If we choose $p=0.5$ as our threshold, we get the metrics shown.  

\noindent\begin{tabular}{@{}p{0.3\textwidth}@{\hspace{24pt}} p{0.3\textwidth} @{\hspace{24pt}} p{0.3\textwidth}}
  \vspace{0pt} \input{../Keras/Images/Ideal_Pred.pgf}
  &
  \vspace{0pt} %% Creator: Matplotlib, PGF backend
%%
%% To include the figure in your LaTeX document, write
%%   \input{<filename>.pgf}
%%
%% Make sure the required packages are loaded in your preamble
%%   \usepackage{pgf}
%%
%% Also ensure that all the required font packages are loaded; for instance,
%% the lmodern package is sometimes necessary when using math font.
%%   \usepackage{lmodern}
%%
%% Figures using additional raster images can only be included by \input if
%% they are in the same directory as the main LaTeX file. For loading figures
%% from other directories you can use the `import` package
%%   \usepackage{import}
%%
%% and then include the figures with
%%   \import{<path to file>}{<filename>.pgf}
%%
%% Matplotlib used the following preamble
%%   
%%   \usepackage{fontspec}
%%   \makeatletter\@ifpackageloaded{underscore}{}{\usepackage[strings]{underscore}}\makeatother
%%
\begingroup%
\makeatletter%
\begin{pgfpicture}%
\pgfpathrectangle{\pgfpointorigin}{\pgfqpoint{3.144311in}{2.646444in}}%
\pgfusepath{use as bounding box, clip}%
\begin{pgfscope}%
\pgfsetbuttcap%
\pgfsetmiterjoin%
\definecolor{currentfill}{rgb}{1.000000,1.000000,1.000000}%
\pgfsetfillcolor{currentfill}%
\pgfsetlinewidth{0.000000pt}%
\definecolor{currentstroke}{rgb}{1.000000,1.000000,1.000000}%
\pgfsetstrokecolor{currentstroke}%
\pgfsetdash{}{0pt}%
\pgfpathmoveto{\pgfqpoint{0.000000in}{0.000000in}}%
\pgfpathlineto{\pgfqpoint{3.144311in}{0.000000in}}%
\pgfpathlineto{\pgfqpoint{3.144311in}{2.646444in}}%
\pgfpathlineto{\pgfqpoint{0.000000in}{2.646444in}}%
\pgfpathlineto{\pgfqpoint{0.000000in}{0.000000in}}%
\pgfpathclose%
\pgfusepath{fill}%
\end{pgfscope}%
\begin{pgfscope}%
\pgfsetbuttcap%
\pgfsetmiterjoin%
\definecolor{currentfill}{rgb}{1.000000,1.000000,1.000000}%
\pgfsetfillcolor{currentfill}%
\pgfsetlinewidth{0.000000pt}%
\definecolor{currentstroke}{rgb}{0.000000,0.000000,0.000000}%
\pgfsetstrokecolor{currentstroke}%
\pgfsetstrokeopacity{0.000000}%
\pgfsetdash{}{0pt}%
\pgfpathmoveto{\pgfqpoint{0.553581in}{0.499444in}}%
\pgfpathlineto{\pgfqpoint{3.033581in}{0.499444in}}%
\pgfpathlineto{\pgfqpoint{3.033581in}{2.347444in}}%
\pgfpathlineto{\pgfqpoint{0.553581in}{2.347444in}}%
\pgfpathlineto{\pgfqpoint{0.553581in}{0.499444in}}%
\pgfpathclose%
\pgfusepath{fill}%
\end{pgfscope}%
\begin{pgfscope}%
\pgfsetbuttcap%
\pgfsetroundjoin%
\definecolor{currentfill}{rgb}{0.000000,0.000000,0.000000}%
\pgfsetfillcolor{currentfill}%
\pgfsetlinewidth{0.803000pt}%
\definecolor{currentstroke}{rgb}{0.000000,0.000000,0.000000}%
\pgfsetstrokecolor{currentstroke}%
\pgfsetdash{}{0pt}%
\pgfsys@defobject{currentmarker}{\pgfqpoint{0.000000in}{-0.048611in}}{\pgfqpoint{0.000000in}{0.000000in}}{%
\pgfpathmoveto{\pgfqpoint{0.000000in}{0.000000in}}%
\pgfpathlineto{\pgfqpoint{0.000000in}{-0.048611in}}%
\pgfusepath{stroke,fill}%
}%
\begin{pgfscope}%
\pgfsys@transformshift{0.666308in}{0.499444in}%
\pgfsys@useobject{currentmarker}{}%
\end{pgfscope}%
\end{pgfscope}%
\begin{pgfscope}%
\definecolor{textcolor}{rgb}{0.000000,0.000000,0.000000}%
\pgfsetstrokecolor{textcolor}%
\pgfsetfillcolor{textcolor}%
\pgftext[x=0.666308in,y=0.402222in,,top]{\color{textcolor}\rmfamily\fontsize{10.000000}{12.000000}\selectfont \(\displaystyle {0.00}\)}%
\end{pgfscope}%
\begin{pgfscope}%
\pgfsetbuttcap%
\pgfsetroundjoin%
\definecolor{currentfill}{rgb}{0.000000,0.000000,0.000000}%
\pgfsetfillcolor{currentfill}%
\pgfsetlinewidth{0.803000pt}%
\definecolor{currentstroke}{rgb}{0.000000,0.000000,0.000000}%
\pgfsetstrokecolor{currentstroke}%
\pgfsetdash{}{0pt}%
\pgfsys@defobject{currentmarker}{\pgfqpoint{0.000000in}{-0.048611in}}{\pgfqpoint{0.000000in}{0.000000in}}{%
\pgfpathmoveto{\pgfqpoint{0.000000in}{0.000000in}}%
\pgfpathlineto{\pgfqpoint{0.000000in}{-0.048611in}}%
\pgfusepath{stroke,fill}%
}%
\begin{pgfscope}%
\pgfsys@transformshift{1.229944in}{0.499444in}%
\pgfsys@useobject{currentmarker}{}%
\end{pgfscope}%
\end{pgfscope}%
\begin{pgfscope}%
\definecolor{textcolor}{rgb}{0.000000,0.000000,0.000000}%
\pgfsetstrokecolor{textcolor}%
\pgfsetfillcolor{textcolor}%
\pgftext[x=1.229944in,y=0.402222in,,top]{\color{textcolor}\rmfamily\fontsize{10.000000}{12.000000}\selectfont \(\displaystyle {0.25}\)}%
\end{pgfscope}%
\begin{pgfscope}%
\pgfsetbuttcap%
\pgfsetroundjoin%
\definecolor{currentfill}{rgb}{0.000000,0.000000,0.000000}%
\pgfsetfillcolor{currentfill}%
\pgfsetlinewidth{0.803000pt}%
\definecolor{currentstroke}{rgb}{0.000000,0.000000,0.000000}%
\pgfsetstrokecolor{currentstroke}%
\pgfsetdash{}{0pt}%
\pgfsys@defobject{currentmarker}{\pgfqpoint{0.000000in}{-0.048611in}}{\pgfqpoint{0.000000in}{0.000000in}}{%
\pgfpathmoveto{\pgfqpoint{0.000000in}{0.000000in}}%
\pgfpathlineto{\pgfqpoint{0.000000in}{-0.048611in}}%
\pgfusepath{stroke,fill}%
}%
\begin{pgfscope}%
\pgfsys@transformshift{1.793581in}{0.499444in}%
\pgfsys@useobject{currentmarker}{}%
\end{pgfscope}%
\end{pgfscope}%
\begin{pgfscope}%
\definecolor{textcolor}{rgb}{0.000000,0.000000,0.000000}%
\pgfsetstrokecolor{textcolor}%
\pgfsetfillcolor{textcolor}%
\pgftext[x=1.793581in,y=0.402222in,,top]{\color{textcolor}\rmfamily\fontsize{10.000000}{12.000000}\selectfont \(\displaystyle {0.50}\)}%
\end{pgfscope}%
\begin{pgfscope}%
\pgfsetbuttcap%
\pgfsetroundjoin%
\definecolor{currentfill}{rgb}{0.000000,0.000000,0.000000}%
\pgfsetfillcolor{currentfill}%
\pgfsetlinewidth{0.803000pt}%
\definecolor{currentstroke}{rgb}{0.000000,0.000000,0.000000}%
\pgfsetstrokecolor{currentstroke}%
\pgfsetdash{}{0pt}%
\pgfsys@defobject{currentmarker}{\pgfqpoint{0.000000in}{-0.048611in}}{\pgfqpoint{0.000000in}{0.000000in}}{%
\pgfpathmoveto{\pgfqpoint{0.000000in}{0.000000in}}%
\pgfpathlineto{\pgfqpoint{0.000000in}{-0.048611in}}%
\pgfusepath{stroke,fill}%
}%
\begin{pgfscope}%
\pgfsys@transformshift{2.357217in}{0.499444in}%
\pgfsys@useobject{currentmarker}{}%
\end{pgfscope}%
\end{pgfscope}%
\begin{pgfscope}%
\definecolor{textcolor}{rgb}{0.000000,0.000000,0.000000}%
\pgfsetstrokecolor{textcolor}%
\pgfsetfillcolor{textcolor}%
\pgftext[x=2.357217in,y=0.402222in,,top]{\color{textcolor}\rmfamily\fontsize{10.000000}{12.000000}\selectfont \(\displaystyle {0.75}\)}%
\end{pgfscope}%
\begin{pgfscope}%
\pgfsetbuttcap%
\pgfsetroundjoin%
\definecolor{currentfill}{rgb}{0.000000,0.000000,0.000000}%
\pgfsetfillcolor{currentfill}%
\pgfsetlinewidth{0.803000pt}%
\definecolor{currentstroke}{rgb}{0.000000,0.000000,0.000000}%
\pgfsetstrokecolor{currentstroke}%
\pgfsetdash{}{0pt}%
\pgfsys@defobject{currentmarker}{\pgfqpoint{0.000000in}{-0.048611in}}{\pgfqpoint{0.000000in}{0.000000in}}{%
\pgfpathmoveto{\pgfqpoint{0.000000in}{0.000000in}}%
\pgfpathlineto{\pgfqpoint{0.000000in}{-0.048611in}}%
\pgfusepath{stroke,fill}%
}%
\begin{pgfscope}%
\pgfsys@transformshift{2.920853in}{0.499444in}%
\pgfsys@useobject{currentmarker}{}%
\end{pgfscope}%
\end{pgfscope}%
\begin{pgfscope}%
\definecolor{textcolor}{rgb}{0.000000,0.000000,0.000000}%
\pgfsetstrokecolor{textcolor}%
\pgfsetfillcolor{textcolor}%
\pgftext[x=2.920853in,y=0.402222in,,top]{\color{textcolor}\rmfamily\fontsize{10.000000}{12.000000}\selectfont \(\displaystyle {1.00}\)}%
\end{pgfscope}%
\begin{pgfscope}%
\definecolor{textcolor}{rgb}{0.000000,0.000000,0.000000}%
\pgfsetstrokecolor{textcolor}%
\pgfsetfillcolor{textcolor}%
\pgftext[x=1.793581in,y=0.223333in,,top]{\color{textcolor}\rmfamily\fontsize{10.000000}{12.000000}\selectfont False positive rate}%
\end{pgfscope}%
\begin{pgfscope}%
\pgfsetbuttcap%
\pgfsetroundjoin%
\definecolor{currentfill}{rgb}{0.000000,0.000000,0.000000}%
\pgfsetfillcolor{currentfill}%
\pgfsetlinewidth{0.803000pt}%
\definecolor{currentstroke}{rgb}{0.000000,0.000000,0.000000}%
\pgfsetstrokecolor{currentstroke}%
\pgfsetdash{}{0pt}%
\pgfsys@defobject{currentmarker}{\pgfqpoint{-0.048611in}{0.000000in}}{\pgfqpoint{-0.000000in}{0.000000in}}{%
\pgfpathmoveto{\pgfqpoint{-0.000000in}{0.000000in}}%
\pgfpathlineto{\pgfqpoint{-0.048611in}{0.000000in}}%
\pgfusepath{stroke,fill}%
}%
\begin{pgfscope}%
\pgfsys@transformshift{0.553581in}{0.583444in}%
\pgfsys@useobject{currentmarker}{}%
\end{pgfscope}%
\end{pgfscope}%
\begin{pgfscope}%
\definecolor{textcolor}{rgb}{0.000000,0.000000,0.000000}%
\pgfsetstrokecolor{textcolor}%
\pgfsetfillcolor{textcolor}%
\pgftext[x=0.278889in, y=0.535250in, left, base]{\color{textcolor}\rmfamily\fontsize{10.000000}{12.000000}\selectfont \(\displaystyle {0.0}\)}%
\end{pgfscope}%
\begin{pgfscope}%
\pgfsetbuttcap%
\pgfsetroundjoin%
\definecolor{currentfill}{rgb}{0.000000,0.000000,0.000000}%
\pgfsetfillcolor{currentfill}%
\pgfsetlinewidth{0.803000pt}%
\definecolor{currentstroke}{rgb}{0.000000,0.000000,0.000000}%
\pgfsetstrokecolor{currentstroke}%
\pgfsetdash{}{0pt}%
\pgfsys@defobject{currentmarker}{\pgfqpoint{-0.048611in}{0.000000in}}{\pgfqpoint{-0.000000in}{0.000000in}}{%
\pgfpathmoveto{\pgfqpoint{-0.000000in}{0.000000in}}%
\pgfpathlineto{\pgfqpoint{-0.048611in}{0.000000in}}%
\pgfusepath{stroke,fill}%
}%
\begin{pgfscope}%
\pgfsys@transformshift{0.553581in}{0.919444in}%
\pgfsys@useobject{currentmarker}{}%
\end{pgfscope}%
\end{pgfscope}%
\begin{pgfscope}%
\definecolor{textcolor}{rgb}{0.000000,0.000000,0.000000}%
\pgfsetstrokecolor{textcolor}%
\pgfsetfillcolor{textcolor}%
\pgftext[x=0.278889in, y=0.871250in, left, base]{\color{textcolor}\rmfamily\fontsize{10.000000}{12.000000}\selectfont \(\displaystyle {0.2}\)}%
\end{pgfscope}%
\begin{pgfscope}%
\pgfsetbuttcap%
\pgfsetroundjoin%
\definecolor{currentfill}{rgb}{0.000000,0.000000,0.000000}%
\pgfsetfillcolor{currentfill}%
\pgfsetlinewidth{0.803000pt}%
\definecolor{currentstroke}{rgb}{0.000000,0.000000,0.000000}%
\pgfsetstrokecolor{currentstroke}%
\pgfsetdash{}{0pt}%
\pgfsys@defobject{currentmarker}{\pgfqpoint{-0.048611in}{0.000000in}}{\pgfqpoint{-0.000000in}{0.000000in}}{%
\pgfpathmoveto{\pgfqpoint{-0.000000in}{0.000000in}}%
\pgfpathlineto{\pgfqpoint{-0.048611in}{0.000000in}}%
\pgfusepath{stroke,fill}%
}%
\begin{pgfscope}%
\pgfsys@transformshift{0.553581in}{1.255444in}%
\pgfsys@useobject{currentmarker}{}%
\end{pgfscope}%
\end{pgfscope}%
\begin{pgfscope}%
\definecolor{textcolor}{rgb}{0.000000,0.000000,0.000000}%
\pgfsetstrokecolor{textcolor}%
\pgfsetfillcolor{textcolor}%
\pgftext[x=0.278889in, y=1.207250in, left, base]{\color{textcolor}\rmfamily\fontsize{10.000000}{12.000000}\selectfont \(\displaystyle {0.4}\)}%
\end{pgfscope}%
\begin{pgfscope}%
\pgfsetbuttcap%
\pgfsetroundjoin%
\definecolor{currentfill}{rgb}{0.000000,0.000000,0.000000}%
\pgfsetfillcolor{currentfill}%
\pgfsetlinewidth{0.803000pt}%
\definecolor{currentstroke}{rgb}{0.000000,0.000000,0.000000}%
\pgfsetstrokecolor{currentstroke}%
\pgfsetdash{}{0pt}%
\pgfsys@defobject{currentmarker}{\pgfqpoint{-0.048611in}{0.000000in}}{\pgfqpoint{-0.000000in}{0.000000in}}{%
\pgfpathmoveto{\pgfqpoint{-0.000000in}{0.000000in}}%
\pgfpathlineto{\pgfqpoint{-0.048611in}{0.000000in}}%
\pgfusepath{stroke,fill}%
}%
\begin{pgfscope}%
\pgfsys@transformshift{0.553581in}{1.591444in}%
\pgfsys@useobject{currentmarker}{}%
\end{pgfscope}%
\end{pgfscope}%
\begin{pgfscope}%
\definecolor{textcolor}{rgb}{0.000000,0.000000,0.000000}%
\pgfsetstrokecolor{textcolor}%
\pgfsetfillcolor{textcolor}%
\pgftext[x=0.278889in, y=1.543250in, left, base]{\color{textcolor}\rmfamily\fontsize{10.000000}{12.000000}\selectfont \(\displaystyle {0.6}\)}%
\end{pgfscope}%
\begin{pgfscope}%
\pgfsetbuttcap%
\pgfsetroundjoin%
\definecolor{currentfill}{rgb}{0.000000,0.000000,0.000000}%
\pgfsetfillcolor{currentfill}%
\pgfsetlinewidth{0.803000pt}%
\definecolor{currentstroke}{rgb}{0.000000,0.000000,0.000000}%
\pgfsetstrokecolor{currentstroke}%
\pgfsetdash{}{0pt}%
\pgfsys@defobject{currentmarker}{\pgfqpoint{-0.048611in}{0.000000in}}{\pgfqpoint{-0.000000in}{0.000000in}}{%
\pgfpathmoveto{\pgfqpoint{-0.000000in}{0.000000in}}%
\pgfpathlineto{\pgfqpoint{-0.048611in}{0.000000in}}%
\pgfusepath{stroke,fill}%
}%
\begin{pgfscope}%
\pgfsys@transformshift{0.553581in}{1.927444in}%
\pgfsys@useobject{currentmarker}{}%
\end{pgfscope}%
\end{pgfscope}%
\begin{pgfscope}%
\definecolor{textcolor}{rgb}{0.000000,0.000000,0.000000}%
\pgfsetstrokecolor{textcolor}%
\pgfsetfillcolor{textcolor}%
\pgftext[x=0.278889in, y=1.879250in, left, base]{\color{textcolor}\rmfamily\fontsize{10.000000}{12.000000}\selectfont \(\displaystyle {0.8}\)}%
\end{pgfscope}%
\begin{pgfscope}%
\pgfsetbuttcap%
\pgfsetroundjoin%
\definecolor{currentfill}{rgb}{0.000000,0.000000,0.000000}%
\pgfsetfillcolor{currentfill}%
\pgfsetlinewidth{0.803000pt}%
\definecolor{currentstroke}{rgb}{0.000000,0.000000,0.000000}%
\pgfsetstrokecolor{currentstroke}%
\pgfsetdash{}{0pt}%
\pgfsys@defobject{currentmarker}{\pgfqpoint{-0.048611in}{0.000000in}}{\pgfqpoint{-0.000000in}{0.000000in}}{%
\pgfpathmoveto{\pgfqpoint{-0.000000in}{0.000000in}}%
\pgfpathlineto{\pgfqpoint{-0.048611in}{0.000000in}}%
\pgfusepath{stroke,fill}%
}%
\begin{pgfscope}%
\pgfsys@transformshift{0.553581in}{2.263444in}%
\pgfsys@useobject{currentmarker}{}%
\end{pgfscope}%
\end{pgfscope}%
\begin{pgfscope}%
\definecolor{textcolor}{rgb}{0.000000,0.000000,0.000000}%
\pgfsetstrokecolor{textcolor}%
\pgfsetfillcolor{textcolor}%
\pgftext[x=0.278889in, y=2.215250in, left, base]{\color{textcolor}\rmfamily\fontsize{10.000000}{12.000000}\selectfont \(\displaystyle {1.0}\)}%
\end{pgfscope}%
\begin{pgfscope}%
\definecolor{textcolor}{rgb}{0.000000,0.000000,0.000000}%
\pgfsetstrokecolor{textcolor}%
\pgfsetfillcolor{textcolor}%
\pgftext[x=0.223333in,y=1.423444in,,bottom,rotate=90.000000]{\color{textcolor}\rmfamily\fontsize{10.000000}{12.000000}\selectfont True positive rate}%
\end{pgfscope}%
\begin{pgfscope}%
\pgfpathrectangle{\pgfqpoint{0.553581in}{0.499444in}}{\pgfqpoint{2.480000in}{1.848000in}}%
\pgfusepath{clip}%
\pgfsetbuttcap%
\pgfsetroundjoin%
\pgfsetlinewidth{1.505625pt}%
\definecolor{currentstroke}{rgb}{0.000000,0.000000,0.000000}%
\pgfsetstrokecolor{currentstroke}%
\pgfsetdash{{5.550000pt}{2.400000pt}}{0.000000pt}%
\pgfpathmoveto{\pgfqpoint{0.666308in}{0.583444in}}%
\pgfpathlineto{\pgfqpoint{2.920853in}{2.263444in}}%
\pgfusepath{stroke}%
\end{pgfscope}%
\begin{pgfscope}%
\pgfpathrectangle{\pgfqpoint{0.553581in}{0.499444in}}{\pgfqpoint{2.480000in}{1.848000in}}%
\pgfusepath{clip}%
\pgfsetrectcap%
\pgfsetroundjoin%
\pgfsetlinewidth{1.505625pt}%
\definecolor{currentstroke}{rgb}{0.121569,0.466667,0.705882}%
\pgfsetstrokecolor{currentstroke}%
\pgfsetdash{}{0pt}%
\pgfpathmoveto{\pgfqpoint{0.666308in}{0.583444in}}%
\pgfpathlineto{\pgfqpoint{0.667266in}{0.584536in}}%
\pgfpathlineto{\pgfqpoint{0.668365in}{0.593356in}}%
\pgfpathlineto{\pgfqpoint{0.668563in}{0.594448in}}%
\pgfpathlineto{\pgfqpoint{0.669662in}{0.606040in}}%
\pgfpathlineto{\pgfqpoint{0.669774in}{0.606796in}}%
\pgfpathlineto{\pgfqpoint{0.670873in}{0.633676in}}%
\pgfpathlineto{\pgfqpoint{0.670986in}{0.634600in}}%
\pgfpathlineto{\pgfqpoint{0.672085in}{0.655432in}}%
\pgfpathlineto{\pgfqpoint{0.672142in}{0.655936in}}%
\pgfpathlineto{\pgfqpoint{0.673241in}{0.682144in}}%
\pgfpathlineto{\pgfqpoint{0.673297in}{0.682816in}}%
\pgfpathlineto{\pgfqpoint{0.674396in}{0.710200in}}%
\pgfpathlineto{\pgfqpoint{0.674509in}{0.711040in}}%
\pgfpathlineto{\pgfqpoint{0.675580in}{0.745732in}}%
\pgfpathlineto{\pgfqpoint{0.675636in}{0.745732in}}%
\pgfpathlineto{\pgfqpoint{0.676735in}{0.778072in}}%
\pgfpathlineto{\pgfqpoint{0.676792in}{0.778828in}}%
\pgfpathlineto{\pgfqpoint{0.677891in}{0.807724in}}%
\pgfpathlineto{\pgfqpoint{0.677919in}{0.807724in}}%
\pgfpathlineto{\pgfqpoint{0.679018in}{0.844180in}}%
\pgfpathlineto{\pgfqpoint{0.679103in}{0.844768in}}%
\pgfpathlineto{\pgfqpoint{0.680202in}{0.875848in}}%
\pgfpathlineto{\pgfqpoint{0.680343in}{0.876688in}}%
\pgfpathlineto{\pgfqpoint{0.681442in}{0.910204in}}%
\pgfpathlineto{\pgfqpoint{0.681526in}{0.911296in}}%
\pgfpathlineto{\pgfqpoint{0.682625in}{0.947920in}}%
\pgfpathlineto{\pgfqpoint{0.682710in}{0.949012in}}%
\pgfpathlineto{\pgfqpoint{0.683809in}{0.974884in}}%
\pgfpathlineto{\pgfqpoint{0.683950in}{0.975556in}}%
\pgfpathlineto{\pgfqpoint{0.685049in}{0.998488in}}%
\pgfpathlineto{\pgfqpoint{0.685133in}{0.999244in}}%
\pgfpathlineto{\pgfqpoint{0.686233in}{1.017052in}}%
\pgfpathlineto{\pgfqpoint{0.686317in}{1.017976in}}%
\pgfpathlineto{\pgfqpoint{0.687360in}{1.042840in}}%
\pgfpathlineto{\pgfqpoint{0.687501in}{1.043848in}}%
\pgfpathlineto{\pgfqpoint{0.688600in}{1.064260in}}%
\pgfpathlineto{\pgfqpoint{0.688656in}{1.065184in}}%
\pgfpathlineto{\pgfqpoint{0.689755in}{1.085176in}}%
\pgfpathlineto{\pgfqpoint{0.689868in}{1.086268in}}%
\pgfpathlineto{\pgfqpoint{0.690967in}{1.109536in}}%
\pgfpathlineto{\pgfqpoint{0.691080in}{1.110544in}}%
\pgfpathlineto{\pgfqpoint{0.692179in}{1.129780in}}%
\pgfpathlineto{\pgfqpoint{0.692263in}{1.130368in}}%
\pgfpathlineto{\pgfqpoint{0.693306in}{1.149268in}}%
\pgfpathlineto{\pgfqpoint{0.693475in}{1.149688in}}%
\pgfpathlineto{\pgfqpoint{0.694574in}{1.170856in}}%
\pgfpathlineto{\pgfqpoint{0.694603in}{1.170856in}}%
\pgfpathlineto{\pgfqpoint{0.695702in}{1.190596in}}%
\pgfpathlineto{\pgfqpoint{0.695758in}{1.190596in}}%
\pgfpathlineto{\pgfqpoint{0.696857in}{1.210756in}}%
\pgfpathlineto{\pgfqpoint{0.696913in}{1.211596in}}%
\pgfpathlineto{\pgfqpoint{0.698013in}{1.235200in}}%
\pgfpathlineto{\pgfqpoint{0.698069in}{1.235956in}}%
\pgfpathlineto{\pgfqpoint{0.699168in}{1.252504in}}%
\pgfpathlineto{\pgfqpoint{0.699196in}{1.252504in}}%
\pgfpathlineto{\pgfqpoint{0.700295in}{1.271740in}}%
\pgfpathlineto{\pgfqpoint{0.700380in}{1.272748in}}%
\pgfpathlineto{\pgfqpoint{0.701451in}{1.288288in}}%
\pgfpathlineto{\pgfqpoint{0.701563in}{1.288960in}}%
\pgfpathlineto{\pgfqpoint{0.702634in}{1.304920in}}%
\pgfpathlineto{\pgfqpoint{0.702832in}{1.306012in}}%
\pgfpathlineto{\pgfqpoint{0.703931in}{1.321132in}}%
\pgfpathlineto{\pgfqpoint{0.704043in}{1.322140in}}%
\pgfpathlineto{\pgfqpoint{0.705143in}{1.337932in}}%
\pgfpathlineto{\pgfqpoint{0.705283in}{1.338856in}}%
\pgfpathlineto{\pgfqpoint{0.706383in}{1.354900in}}%
\pgfpathlineto{\pgfqpoint{0.706523in}{1.355992in}}%
\pgfpathlineto{\pgfqpoint{0.707594in}{1.371784in}}%
\pgfpathlineto{\pgfqpoint{0.707707in}{1.372372in}}%
\pgfpathlineto{\pgfqpoint{0.708806in}{1.385980in}}%
\pgfpathlineto{\pgfqpoint{0.708975in}{1.387072in}}%
\pgfpathlineto{\pgfqpoint{0.710046in}{1.397320in}}%
\pgfpathlineto{\pgfqpoint{0.710131in}{1.397572in}}%
\pgfpathlineto{\pgfqpoint{0.711230in}{1.411012in}}%
\pgfpathlineto{\pgfqpoint{0.711343in}{1.411432in}}%
\pgfpathlineto{\pgfqpoint{0.712385in}{1.423948in}}%
\pgfpathlineto{\pgfqpoint{0.712667in}{1.424956in}}%
\pgfpathlineto{\pgfqpoint{0.713738in}{1.434616in}}%
\pgfpathlineto{\pgfqpoint{0.713935in}{1.435456in}}%
\pgfpathlineto{\pgfqpoint{0.715034in}{1.447384in}}%
\pgfpathlineto{\pgfqpoint{0.715175in}{1.447552in}}%
\pgfpathlineto{\pgfqpoint{0.716274in}{1.459648in}}%
\pgfpathlineto{\pgfqpoint{0.716415in}{1.460740in}}%
\pgfpathlineto{\pgfqpoint{0.717514in}{1.472164in}}%
\pgfpathlineto{\pgfqpoint{0.717627in}{1.472584in}}%
\pgfpathlineto{\pgfqpoint{0.718726in}{1.483252in}}%
\pgfpathlineto{\pgfqpoint{0.718867in}{1.484008in}}%
\pgfpathlineto{\pgfqpoint{0.719966in}{1.494256in}}%
\pgfpathlineto{\pgfqpoint{0.720107in}{1.495348in}}%
\pgfpathlineto{\pgfqpoint{0.721206in}{1.505092in}}%
\pgfpathlineto{\pgfqpoint{0.721432in}{1.506184in}}%
\pgfpathlineto{\pgfqpoint{0.722531in}{1.518028in}}%
\pgfpathlineto{\pgfqpoint{0.722672in}{1.519120in}}%
\pgfpathlineto{\pgfqpoint{0.723771in}{1.527520in}}%
\pgfpathlineto{\pgfqpoint{0.723883in}{1.527940in}}%
\pgfpathlineto{\pgfqpoint{0.724983in}{1.539700in}}%
\pgfpathlineto{\pgfqpoint{0.725152in}{1.540540in}}%
\pgfpathlineto{\pgfqpoint{0.726251in}{1.549024in}}%
\pgfpathlineto{\pgfqpoint{0.726476in}{1.549864in}}%
\pgfpathlineto{\pgfqpoint{0.727519in}{1.560364in}}%
\pgfpathlineto{\pgfqpoint{0.727801in}{1.561204in}}%
\pgfpathlineto{\pgfqpoint{0.728815in}{1.568176in}}%
\pgfpathlineto{\pgfqpoint{0.729238in}{1.569268in}}%
\pgfpathlineto{\pgfqpoint{0.730337in}{1.580020in}}%
\pgfpathlineto{\pgfqpoint{0.730563in}{1.580692in}}%
\pgfpathlineto{\pgfqpoint{0.731633in}{1.588420in}}%
\pgfpathlineto{\pgfqpoint{0.731887in}{1.589512in}}%
\pgfpathlineto{\pgfqpoint{0.732986in}{1.597492in}}%
\pgfpathlineto{\pgfqpoint{0.733240in}{1.598584in}}%
\pgfpathlineto{\pgfqpoint{0.734339in}{1.606900in}}%
\pgfpathlineto{\pgfqpoint{0.734621in}{1.607992in}}%
\pgfpathlineto{\pgfqpoint{0.735692in}{1.614796in}}%
\pgfpathlineto{\pgfqpoint{0.736086in}{1.615888in}}%
\pgfpathlineto{\pgfqpoint{0.737185in}{1.624624in}}%
\pgfpathlineto{\pgfqpoint{0.737326in}{1.625464in}}%
\pgfpathlineto{\pgfqpoint{0.738425in}{1.631932in}}%
\pgfpathlineto{\pgfqpoint{0.738623in}{1.632688in}}%
\pgfpathlineto{\pgfqpoint{0.739665in}{1.638988in}}%
\pgfpathlineto{\pgfqpoint{0.740003in}{1.639828in}}%
\pgfpathlineto{\pgfqpoint{0.741103in}{1.646128in}}%
\pgfpathlineto{\pgfqpoint{0.741356in}{1.647220in}}%
\pgfpathlineto{\pgfqpoint{0.742455in}{1.654612in}}%
\pgfpathlineto{\pgfqpoint{0.742596in}{1.655452in}}%
\pgfpathlineto{\pgfqpoint{0.743667in}{1.661416in}}%
\pgfpathlineto{\pgfqpoint{0.743893in}{1.662508in}}%
\pgfpathlineto{\pgfqpoint{0.744992in}{1.667128in}}%
\pgfpathlineto{\pgfqpoint{0.745273in}{1.668220in}}%
\pgfpathlineto{\pgfqpoint{0.746373in}{1.675444in}}%
\pgfpathlineto{\pgfqpoint{0.746683in}{1.676116in}}%
\pgfpathlineto{\pgfqpoint{0.747782in}{1.680652in}}%
\pgfpathlineto{\pgfqpoint{0.748120in}{1.681744in}}%
\pgfpathlineto{\pgfqpoint{0.749191in}{1.689136in}}%
\pgfpathlineto{\pgfqpoint{0.749529in}{1.690228in}}%
\pgfpathlineto{\pgfqpoint{0.750515in}{1.695016in}}%
\pgfpathlineto{\pgfqpoint{0.750938in}{1.696108in}}%
\pgfpathlineto{\pgfqpoint{0.752037in}{1.702072in}}%
\pgfpathlineto{\pgfqpoint{0.752263in}{1.702996in}}%
\pgfpathlineto{\pgfqpoint{0.753305in}{1.707784in}}%
\pgfpathlineto{\pgfqpoint{0.753531in}{1.708792in}}%
\pgfpathlineto{\pgfqpoint{0.754630in}{1.713664in}}%
\pgfpathlineto{\pgfqpoint{0.755137in}{1.714672in}}%
\pgfpathlineto{\pgfqpoint{0.756236in}{1.718284in}}%
\pgfpathlineto{\pgfqpoint{0.756433in}{1.719292in}}%
\pgfpathlineto{\pgfqpoint{0.757533in}{1.723576in}}%
\pgfpathlineto{\pgfqpoint{0.757786in}{1.724500in}}%
\pgfpathlineto{\pgfqpoint{0.758857in}{1.729876in}}%
\pgfpathlineto{\pgfqpoint{0.759111in}{1.730968in}}%
\pgfpathlineto{\pgfqpoint{0.760210in}{1.737016in}}%
\pgfpathlineto{\pgfqpoint{0.760463in}{1.737940in}}%
\pgfpathlineto{\pgfqpoint{0.761534in}{1.741720in}}%
\pgfpathlineto{\pgfqpoint{0.762126in}{1.742728in}}%
\pgfpathlineto{\pgfqpoint{0.763225in}{1.747096in}}%
\pgfpathlineto{\pgfqpoint{0.763676in}{1.748188in}}%
\pgfpathlineto{\pgfqpoint{0.764775in}{1.752472in}}%
\pgfpathlineto{\pgfqpoint{0.765113in}{1.753564in}}%
\pgfpathlineto{\pgfqpoint{0.765536in}{1.756252in}}%
\pgfpathlineto{\pgfqpoint{0.771877in}{1.757260in}}%
\pgfpathlineto{\pgfqpoint{0.772920in}{1.760200in}}%
\pgfpathlineto{\pgfqpoint{0.773371in}{1.761124in}}%
\pgfpathlineto{\pgfqpoint{0.774470in}{1.765240in}}%
\pgfpathlineto{\pgfqpoint{0.774780in}{1.765996in}}%
\pgfpathlineto{\pgfqpoint{0.775879in}{1.770196in}}%
\pgfpathlineto{\pgfqpoint{0.776076in}{1.771204in}}%
\pgfpathlineto{\pgfqpoint{0.777175in}{1.776664in}}%
\pgfpathlineto{\pgfqpoint{0.777542in}{1.777672in}}%
\pgfpathlineto{\pgfqpoint{0.778641in}{1.781200in}}%
\pgfpathlineto{\pgfqpoint{0.778951in}{1.782208in}}%
\pgfpathlineto{\pgfqpoint{0.780050in}{1.787752in}}%
\pgfpathlineto{\pgfqpoint{0.780388in}{1.788676in}}%
\pgfpathlineto{\pgfqpoint{0.781459in}{1.792456in}}%
\pgfpathlineto{\pgfqpoint{0.781882in}{1.793548in}}%
\pgfpathlineto{\pgfqpoint{0.782896in}{1.797664in}}%
\pgfpathlineto{\pgfqpoint{0.783291in}{1.798756in}}%
\pgfpathlineto{\pgfqpoint{0.784390in}{1.802872in}}%
\pgfpathlineto{\pgfqpoint{0.784813in}{1.803880in}}%
\pgfpathlineto{\pgfqpoint{0.785855in}{1.806568in}}%
\pgfpathlineto{\pgfqpoint{0.786363in}{1.807576in}}%
\pgfpathlineto{\pgfqpoint{0.787462in}{1.811440in}}%
\pgfpathlineto{\pgfqpoint{0.787715in}{1.812532in}}%
\pgfpathlineto{\pgfqpoint{0.788814in}{1.816480in}}%
\pgfpathlineto{\pgfqpoint{0.789209in}{1.817572in}}%
\pgfpathlineto{\pgfqpoint{0.790308in}{1.820932in}}%
\pgfpathlineto{\pgfqpoint{0.790900in}{1.822024in}}%
\pgfpathlineto{\pgfqpoint{0.791971in}{1.826140in}}%
\pgfpathlineto{\pgfqpoint{0.792534in}{1.827232in}}%
\pgfpathlineto{\pgfqpoint{0.793605in}{1.832104in}}%
\pgfpathlineto{\pgfqpoint{0.794028in}{1.833196in}}%
\pgfpathlineto{\pgfqpoint{0.795099in}{1.835128in}}%
\pgfpathlineto{\pgfqpoint{0.795916in}{1.836220in}}%
\pgfpathlineto{\pgfqpoint{0.796987in}{1.840168in}}%
\pgfpathlineto{\pgfqpoint{0.797494in}{1.841260in}}%
\pgfpathlineto{\pgfqpoint{0.798593in}{1.843276in}}%
\pgfpathlineto{\pgfqpoint{0.799016in}{1.844200in}}%
\pgfpathlineto{\pgfqpoint{0.800031in}{1.847980in}}%
\pgfpathlineto{\pgfqpoint{0.800397in}{1.848988in}}%
\pgfpathlineto{\pgfqpoint{0.801496in}{1.853356in}}%
\pgfpathlineto{\pgfqpoint{0.802285in}{1.854280in}}%
\pgfpathlineto{\pgfqpoint{0.803356in}{1.857052in}}%
\pgfpathlineto{\pgfqpoint{0.803948in}{1.858144in}}%
\pgfpathlineto{\pgfqpoint{0.805047in}{1.861756in}}%
\pgfpathlineto{\pgfqpoint{0.805498in}{1.862764in}}%
\pgfpathlineto{\pgfqpoint{0.806569in}{1.866124in}}%
\pgfpathlineto{\pgfqpoint{0.807583in}{1.867132in}}%
\pgfpathlineto{\pgfqpoint{0.808683in}{1.870576in}}%
\pgfpathlineto{\pgfqpoint{0.809472in}{1.871668in}}%
\pgfpathlineto{\pgfqpoint{0.810571in}{1.873432in}}%
\pgfpathlineto{\pgfqpoint{0.811050in}{1.874440in}}%
\pgfpathlineto{\pgfqpoint{0.812121in}{1.878052in}}%
\pgfpathlineto{\pgfqpoint{0.812600in}{1.879144in}}%
\pgfpathlineto{\pgfqpoint{0.813699in}{1.883344in}}%
\pgfpathlineto{\pgfqpoint{0.814403in}{1.884436in}}%
\pgfpathlineto{\pgfqpoint{0.815418in}{1.887292in}}%
\pgfpathlineto{\pgfqpoint{0.815813in}{1.888132in}}%
\pgfpathlineto{\pgfqpoint{0.816771in}{1.890064in}}%
\pgfpathlineto{\pgfqpoint{0.817447in}{1.891072in}}%
\pgfpathlineto{\pgfqpoint{0.818490in}{1.893844in}}%
\pgfpathlineto{\pgfqpoint{0.818884in}{1.894768in}}%
\pgfpathlineto{\pgfqpoint{0.819983in}{1.896616in}}%
\pgfpathlineto{\pgfqpoint{0.821026in}{1.897708in}}%
\pgfpathlineto{\pgfqpoint{0.822125in}{1.900480in}}%
\pgfpathlineto{\pgfqpoint{0.822717in}{1.901572in}}%
\pgfpathlineto{\pgfqpoint{0.823816in}{1.904596in}}%
\pgfpathlineto{\pgfqpoint{0.824013in}{1.905520in}}%
\pgfpathlineto{\pgfqpoint{0.825056in}{1.908628in}}%
\pgfpathlineto{\pgfqpoint{0.826268in}{1.909636in}}%
\pgfpathlineto{\pgfqpoint{0.827311in}{1.911232in}}%
\pgfpathlineto{\pgfqpoint{0.828438in}{1.912324in}}%
\pgfpathlineto{\pgfqpoint{0.829537in}{1.914424in}}%
\pgfpathlineto{\pgfqpoint{0.830016in}{1.915516in}}%
\pgfpathlineto{\pgfqpoint{0.831087in}{1.917784in}}%
\pgfpathlineto{\pgfqpoint{0.831538in}{1.918708in}}%
\pgfpathlineto{\pgfqpoint{0.832524in}{1.920976in}}%
\pgfpathlineto{\pgfqpoint{0.833003in}{1.921984in}}%
\pgfpathlineto{\pgfqpoint{0.834046in}{1.924168in}}%
\pgfpathlineto{\pgfqpoint{0.834779in}{1.925176in}}%
\pgfpathlineto{\pgfqpoint{0.835878in}{1.928284in}}%
\pgfpathlineto{\pgfqpoint{0.836442in}{1.929376in}}%
\pgfpathlineto{\pgfqpoint{0.837541in}{1.931308in}}%
\pgfpathlineto{\pgfqpoint{0.838386in}{1.932400in}}%
\pgfpathlineto{\pgfqpoint{0.839485in}{1.934668in}}%
\pgfpathlineto{\pgfqpoint{0.840133in}{1.935676in}}%
\pgfpathlineto{\pgfqpoint{0.841120in}{1.938532in}}%
\pgfpathlineto{\pgfqpoint{0.842163in}{1.939624in}}%
\pgfpathlineto{\pgfqpoint{0.843262in}{1.941304in}}%
\pgfpathlineto{\pgfqpoint{0.844361in}{1.942396in}}%
\pgfpathlineto{\pgfqpoint{0.845460in}{1.945084in}}%
\pgfpathlineto{\pgfqpoint{0.846333in}{1.946176in}}%
\pgfpathlineto{\pgfqpoint{0.847433in}{1.949032in}}%
\pgfpathlineto{\pgfqpoint{0.848165in}{1.950040in}}%
\pgfpathlineto{\pgfqpoint{0.849236in}{1.951720in}}%
\pgfpathlineto{\pgfqpoint{0.849772in}{1.952728in}}%
\pgfpathlineto{\pgfqpoint{0.850843in}{1.954996in}}%
\pgfpathlineto{\pgfqpoint{0.851942in}{1.956088in}}%
\pgfpathlineto{\pgfqpoint{0.853013in}{1.958104in}}%
\pgfpathlineto{\pgfqpoint{0.853745in}{1.959196in}}%
\pgfpathlineto{\pgfqpoint{0.854816in}{1.960540in}}%
\pgfpathlineto{\pgfqpoint{0.855183in}{1.961380in}}%
\pgfpathlineto{\pgfqpoint{0.856282in}{1.963732in}}%
\pgfpathlineto{\pgfqpoint{0.857071in}{1.964824in}}%
\pgfpathlineto{\pgfqpoint{0.858113in}{1.966756in}}%
\pgfpathlineto{\pgfqpoint{0.858621in}{1.967848in}}%
\pgfpathlineto{\pgfqpoint{0.859663in}{1.969444in}}%
\pgfpathlineto{\pgfqpoint{0.860255in}{1.970536in}}%
\pgfpathlineto{\pgfqpoint{0.861298in}{1.972132in}}%
\pgfpathlineto{\pgfqpoint{0.862538in}{1.973224in}}%
\pgfpathlineto{\pgfqpoint{0.863609in}{1.974988in}}%
\pgfpathlineto{\pgfqpoint{0.864680in}{1.975912in}}%
\pgfpathlineto{\pgfqpoint{0.865751in}{1.978600in}}%
\pgfpathlineto{\pgfqpoint{0.866371in}{1.979692in}}%
\pgfpathlineto{\pgfqpoint{0.867470in}{1.981456in}}%
\pgfpathlineto{\pgfqpoint{0.868090in}{1.982464in}}%
\pgfpathlineto{\pgfqpoint{0.869189in}{1.983892in}}%
\pgfpathlineto{\pgfqpoint{0.870795in}{1.984900in}}%
\pgfpathlineto{\pgfqpoint{0.871782in}{1.985992in}}%
\pgfpathlineto{\pgfqpoint{0.872712in}{1.987084in}}%
\pgfpathlineto{\pgfqpoint{0.873416in}{1.987672in}}%
\pgfpathlineto{\pgfqpoint{0.875248in}{1.988764in}}%
\pgfpathlineto{\pgfqpoint{0.876347in}{1.990528in}}%
\pgfpathlineto{\pgfqpoint{0.877023in}{1.991620in}}%
\pgfpathlineto{\pgfqpoint{0.878094in}{1.993132in}}%
\pgfpathlineto{\pgfqpoint{0.878883in}{1.994224in}}%
\pgfpathlineto{\pgfqpoint{0.879983in}{1.995820in}}%
\pgfpathlineto{\pgfqpoint{0.880743in}{1.996912in}}%
\pgfpathlineto{\pgfqpoint{0.881730in}{1.997584in}}%
\pgfpathlineto{\pgfqpoint{0.882885in}{1.998592in}}%
\pgfpathlineto{\pgfqpoint{0.883984in}{1.999936in}}%
\pgfpathlineto{\pgfqpoint{0.885224in}{2.001028in}}%
\pgfpathlineto{\pgfqpoint{0.886098in}{2.001952in}}%
\pgfpathlineto{\pgfqpoint{0.887169in}{2.003044in}}%
\pgfpathlineto{\pgfqpoint{0.888268in}{2.005564in}}%
\pgfpathlineto{\pgfqpoint{0.889480in}{2.006656in}}%
\pgfpathlineto{\pgfqpoint{0.890579in}{2.008084in}}%
\pgfpathlineto{\pgfqpoint{0.891593in}{2.009176in}}%
\pgfpathlineto{\pgfqpoint{0.892664in}{2.010100in}}%
\pgfpathlineto{\pgfqpoint{0.893735in}{2.011192in}}%
\pgfpathlineto{\pgfqpoint{0.894834in}{2.012620in}}%
\pgfpathlineto{\pgfqpoint{0.895990in}{2.013712in}}%
\pgfpathlineto{\pgfqpoint{0.896892in}{2.015476in}}%
\pgfpathlineto{\pgfqpoint{0.898526in}{2.016568in}}%
\pgfpathlineto{\pgfqpoint{0.899597in}{2.017828in}}%
\pgfpathlineto{\pgfqpoint{0.900893in}{2.018920in}}%
\pgfpathlineto{\pgfqpoint{0.901908in}{2.020432in}}%
\pgfpathlineto{\pgfqpoint{0.902894in}{2.021524in}}%
\pgfpathlineto{\pgfqpoint{0.903909in}{2.022532in}}%
\pgfpathlineto{\pgfqpoint{0.904754in}{2.023624in}}%
\pgfpathlineto{\pgfqpoint{0.905515in}{2.024884in}}%
\pgfpathlineto{\pgfqpoint{0.907093in}{2.025976in}}%
\pgfpathlineto{\pgfqpoint{0.908136in}{2.026648in}}%
\pgfpathlineto{\pgfqpoint{0.909348in}{2.027740in}}%
\pgfpathlineto{\pgfqpoint{0.910419in}{2.028244in}}%
\pgfpathlineto{\pgfqpoint{0.911800in}{2.029336in}}%
\pgfpathlineto{\pgfqpoint{0.912730in}{2.031100in}}%
\pgfpathlineto{\pgfqpoint{0.914759in}{2.032192in}}%
\pgfpathlineto{\pgfqpoint{0.915858in}{2.032948in}}%
\pgfpathlineto{\pgfqpoint{0.917211in}{2.034040in}}%
\pgfpathlineto{\pgfqpoint{0.918141in}{2.034964in}}%
\pgfpathlineto{\pgfqpoint{0.919719in}{2.036056in}}%
\pgfpathlineto{\pgfqpoint{0.920536in}{2.037064in}}%
\pgfpathlineto{\pgfqpoint{0.922453in}{2.038072in}}%
\pgfpathlineto{\pgfqpoint{0.923326in}{2.038996in}}%
\pgfpathlineto{\pgfqpoint{0.925017in}{2.040088in}}%
\pgfpathlineto{\pgfqpoint{0.926003in}{2.041180in}}%
\pgfpathlineto{\pgfqpoint{0.926990in}{2.042272in}}%
\pgfpathlineto{\pgfqpoint{0.927610in}{2.043028in}}%
\pgfpathlineto{\pgfqpoint{0.929244in}{2.044120in}}%
\pgfpathlineto{\pgfqpoint{0.930174in}{2.044540in}}%
\pgfpathlineto{\pgfqpoint{0.932316in}{2.045632in}}%
\pgfpathlineto{\pgfqpoint{0.933387in}{2.046556in}}%
\pgfpathlineto{\pgfqpoint{0.934796in}{2.047648in}}%
\pgfpathlineto{\pgfqpoint{0.935839in}{2.048992in}}%
\pgfpathlineto{\pgfqpoint{0.938009in}{2.050084in}}%
\pgfpathlineto{\pgfqpoint{0.938770in}{2.050672in}}%
\pgfpathlineto{\pgfqpoint{0.940855in}{2.051764in}}%
\pgfpathlineto{\pgfqpoint{0.941870in}{2.052688in}}%
\pgfpathlineto{\pgfqpoint{0.943786in}{2.053780in}}%
\pgfpathlineto{\pgfqpoint{0.944829in}{2.054872in}}%
\pgfpathlineto{\pgfqpoint{0.946830in}{2.055964in}}%
\pgfpathlineto{\pgfqpoint{0.947901in}{2.057476in}}%
\pgfpathlineto{\pgfqpoint{0.949845in}{2.058568in}}%
\pgfpathlineto{\pgfqpoint{0.950888in}{2.059408in}}%
\pgfpathlineto{\pgfqpoint{0.952241in}{2.060500in}}%
\pgfpathlineto{\pgfqpoint{0.953312in}{2.061508in}}%
\pgfpathlineto{\pgfqpoint{0.955284in}{2.062600in}}%
\pgfpathlineto{\pgfqpoint{0.956327in}{2.063356in}}%
\pgfpathlineto{\pgfqpoint{0.957990in}{2.064364in}}%
\pgfpathlineto{\pgfqpoint{0.958948in}{2.066044in}}%
\pgfpathlineto{\pgfqpoint{0.962048in}{2.067136in}}%
\pgfpathlineto{\pgfqpoint{0.962893in}{2.068144in}}%
\pgfpathlineto{\pgfqpoint{0.965683in}{2.069236in}}%
\pgfpathlineto{\pgfqpoint{0.966783in}{2.070412in}}%
\pgfpathlineto{\pgfqpoint{0.968135in}{2.071504in}}%
\pgfpathlineto{\pgfqpoint{0.969178in}{2.072428in}}%
\pgfpathlineto{\pgfqpoint{0.970700in}{2.073520in}}%
\pgfpathlineto{\pgfqpoint{0.971771in}{2.075032in}}%
\pgfpathlineto{\pgfqpoint{0.973377in}{2.076040in}}%
\pgfpathlineto{\pgfqpoint{0.974476in}{2.077468in}}%
\pgfpathlineto{\pgfqpoint{0.976364in}{2.078560in}}%
\pgfpathlineto{\pgfqpoint{0.977097in}{2.079736in}}%
\pgfpathlineto{\pgfqpoint{0.979436in}{2.080828in}}%
\pgfpathlineto{\pgfqpoint{0.980479in}{2.082004in}}%
\pgfpathlineto{\pgfqpoint{0.982733in}{2.083096in}}%
\pgfpathlineto{\pgfqpoint{0.982818in}{2.083264in}}%
\pgfpathlineto{\pgfqpoint{0.986228in}{2.084356in}}%
\pgfpathlineto{\pgfqpoint{0.987130in}{2.085196in}}%
\pgfpathlineto{\pgfqpoint{0.988821in}{2.086288in}}%
\pgfpathlineto{\pgfqpoint{0.989863in}{2.086876in}}%
\pgfpathlineto{\pgfqpoint{0.992287in}{2.087968in}}%
\pgfpathlineto{\pgfqpoint{0.993273in}{2.088892in}}%
\pgfpathlineto{\pgfqpoint{0.994570in}{2.089984in}}%
\pgfpathlineto{\pgfqpoint{0.995669in}{2.090320in}}%
\pgfpathlineto{\pgfqpoint{0.997444in}{2.091328in}}%
\pgfpathlineto{\pgfqpoint{0.997923in}{2.091916in}}%
\pgfpathlineto{\pgfqpoint{0.999783in}{2.092924in}}%
\pgfpathlineto{\pgfqpoint{1.000883in}{2.093680in}}%
\pgfpathlineto{\pgfqpoint{1.002461in}{2.094688in}}%
\pgfpathlineto{\pgfqpoint{1.003475in}{2.095360in}}%
\pgfpathlineto{\pgfqpoint{1.004462in}{2.096452in}}%
\pgfpathlineto{\pgfqpoint{1.005251in}{2.096956in}}%
\pgfpathlineto{\pgfqpoint{1.007421in}{2.098048in}}%
\pgfpathlineto{\pgfqpoint{1.008435in}{2.098888in}}%
\pgfpathlineto{\pgfqpoint{1.010887in}{2.099980in}}%
\pgfpathlineto{\pgfqpoint{1.011958in}{2.100568in}}%
\pgfpathlineto{\pgfqpoint{1.014973in}{2.101660in}}%
\pgfpathlineto{\pgfqpoint{1.015791in}{2.102164in}}%
\pgfpathlineto{\pgfqpoint{1.018693in}{2.103256in}}%
\pgfpathlineto{\pgfqpoint{1.019623in}{2.104348in}}%
\pgfpathlineto{\pgfqpoint{1.023400in}{2.105440in}}%
\pgfpathlineto{\pgfqpoint{1.024358in}{2.105860in}}%
\pgfpathlineto{\pgfqpoint{1.027373in}{2.106952in}}%
\pgfpathlineto{\pgfqpoint{1.028219in}{2.107372in}}%
\pgfpathlineto{\pgfqpoint{1.031037in}{2.108464in}}%
\pgfpathlineto{\pgfqpoint{1.031939in}{2.109136in}}%
\pgfpathlineto{\pgfqpoint{1.034222in}{2.110228in}}%
\pgfpathlineto{\pgfqpoint{1.035236in}{2.110984in}}%
\pgfpathlineto{\pgfqpoint{1.038139in}{2.112076in}}%
\pgfpathlineto{\pgfqpoint{1.039238in}{2.113000in}}%
\pgfpathlineto{\pgfqpoint{1.041887in}{2.114008in}}%
\pgfpathlineto{\pgfqpoint{1.042845in}{2.115100in}}%
\pgfpathlineto{\pgfqpoint{1.045297in}{2.116192in}}%
\pgfpathlineto{\pgfqpoint{1.046002in}{2.116696in}}%
\pgfpathlineto{\pgfqpoint{1.050060in}{2.117788in}}%
\pgfpathlineto{\pgfqpoint{1.050990in}{2.118208in}}%
\pgfpathlineto{\pgfqpoint{1.052793in}{2.119300in}}%
\pgfpathlineto{\pgfqpoint{1.053554in}{2.119888in}}%
\pgfpathlineto{\pgfqpoint{1.056232in}{2.120980in}}%
\pgfpathlineto{\pgfqpoint{1.057077in}{2.121400in}}%
\pgfpathlineto{\pgfqpoint{1.059726in}{2.122492in}}%
\pgfpathlineto{\pgfqpoint{1.060628in}{2.123080in}}%
\pgfpathlineto{\pgfqpoint{1.062995in}{2.124172in}}%
\pgfpathlineto{\pgfqpoint{1.063953in}{2.125012in}}%
\pgfpathlineto{\pgfqpoint{1.067673in}{2.126104in}}%
\pgfpathlineto{\pgfqpoint{1.068575in}{2.126440in}}%
\pgfpathlineto{\pgfqpoint{1.070914in}{2.127448in}}%
\pgfpathlineto{\pgfqpoint{1.071478in}{2.127952in}}%
\pgfpathlineto{\pgfqpoint{1.074663in}{2.129044in}}%
\pgfpathlineto{\pgfqpoint{1.075311in}{2.129548in}}%
\pgfpathlineto{\pgfqpoint{1.079679in}{2.130556in}}%
\pgfpathlineto{\pgfqpoint{1.080778in}{2.131144in}}%
\pgfpathlineto{\pgfqpoint{1.082694in}{2.132236in}}%
\pgfpathlineto{\pgfqpoint{1.083793in}{2.132572in}}%
\pgfpathlineto{\pgfqpoint{1.086893in}{2.133664in}}%
\pgfpathlineto{\pgfqpoint{1.087823in}{2.134084in}}%
\pgfpathlineto{\pgfqpoint{1.091938in}{2.135092in}}%
\pgfpathlineto{\pgfqpoint{1.092981in}{2.135680in}}%
\pgfpathlineto{\pgfqpoint{1.097293in}{2.136772in}}%
\pgfpathlineto{\pgfqpoint{1.098335in}{2.137108in}}%
\pgfpathlineto{\pgfqpoint{1.102901in}{2.138200in}}%
\pgfpathlineto{\pgfqpoint{1.103972in}{2.139208in}}%
\pgfpathlineto{\pgfqpoint{1.108283in}{2.140300in}}%
\pgfpathlineto{\pgfqpoint{1.109298in}{2.140720in}}%
\pgfpathlineto{\pgfqpoint{1.112821in}{2.141812in}}%
\pgfpathlineto{\pgfqpoint{1.113723in}{2.142232in}}%
\pgfpathlineto{\pgfqpoint{1.116710in}{2.143324in}}%
\pgfpathlineto{\pgfqpoint{1.116879in}{2.143492in}}%
\pgfpathlineto{\pgfqpoint{1.122797in}{2.144584in}}%
\pgfpathlineto{\pgfqpoint{1.123586in}{2.144836in}}%
\pgfpathlineto{\pgfqpoint{1.129786in}{2.145928in}}%
\pgfpathlineto{\pgfqpoint{1.130519in}{2.146684in}}%
\pgfpathlineto{\pgfqpoint{1.138043in}{2.147776in}}%
\pgfpathlineto{\pgfqpoint{1.139030in}{2.148196in}}%
\pgfpathlineto{\pgfqpoint{1.141425in}{2.149288in}}%
\pgfpathlineto{\pgfqpoint{1.142496in}{2.149960in}}%
\pgfpathlineto{\pgfqpoint{1.146639in}{2.151052in}}%
\pgfpathlineto{\pgfqpoint{1.147428in}{2.151472in}}%
\pgfpathlineto{\pgfqpoint{1.152022in}{2.152564in}}%
\pgfpathlineto{\pgfqpoint{1.152529in}{2.152816in}}%
\pgfpathlineto{\pgfqpoint{1.156277in}{2.153824in}}%
\pgfpathlineto{\pgfqpoint{1.157263in}{2.154244in}}%
\pgfpathlineto{\pgfqpoint{1.159631in}{2.155336in}}%
\pgfpathlineto{\pgfqpoint{1.160702in}{2.155672in}}%
\pgfpathlineto{\pgfqpoint{1.166310in}{2.156764in}}%
\pgfpathlineto{\pgfqpoint{1.167353in}{2.157268in}}%
\pgfpathlineto{\pgfqpoint{1.172876in}{2.158360in}}%
\pgfpathlineto{\pgfqpoint{1.173383in}{2.158612in}}%
\pgfpathlineto{\pgfqpoint{1.178738in}{2.159704in}}%
\pgfpathlineto{\pgfqpoint{1.179837in}{2.160208in}}%
\pgfpathlineto{\pgfqpoint{1.183980in}{2.161300in}}%
\pgfpathlineto{\pgfqpoint{1.185023in}{2.161636in}}%
\pgfpathlineto{\pgfqpoint{1.190433in}{2.162728in}}%
\pgfpathlineto{\pgfqpoint{1.190772in}{2.162980in}}%
\pgfpathlineto{\pgfqpoint{1.196803in}{2.164072in}}%
\pgfpathlineto{\pgfqpoint{1.197873in}{2.164492in}}%
\pgfpathlineto{\pgfqpoint{1.203200in}{2.165584in}}%
\pgfpathlineto{\pgfqpoint{1.203707in}{2.165752in}}%
\pgfpathlineto{\pgfqpoint{1.208639in}{2.166844in}}%
\pgfpathlineto{\pgfqpoint{1.209682in}{2.167180in}}%
\pgfpathlineto{\pgfqpoint{1.215121in}{2.168272in}}%
\pgfpathlineto{\pgfqpoint{1.216023in}{2.168608in}}%
\pgfpathlineto{\pgfqpoint{1.221913in}{2.169700in}}%
\pgfpathlineto{\pgfqpoint{1.222673in}{2.170036in}}%
\pgfpathlineto{\pgfqpoint{1.228423in}{2.171128in}}%
\pgfpathlineto{\pgfqpoint{1.228423in}{2.171212in}}%
\pgfpathlineto{\pgfqpoint{1.234313in}{2.172304in}}%
\pgfpathlineto{\pgfqpoint{1.234820in}{2.172640in}}%
\pgfpathlineto{\pgfqpoint{1.242880in}{2.173732in}}%
\pgfpathlineto{\pgfqpoint{1.243162in}{2.173984in}}%
\pgfpathlineto{\pgfqpoint{1.251870in}{2.175076in}}%
\pgfpathlineto{\pgfqpoint{1.252743in}{2.175496in}}%
\pgfpathlineto{\pgfqpoint{1.257196in}{2.176588in}}%
\pgfpathlineto{\pgfqpoint{1.257957in}{2.176924in}}%
\pgfpathlineto{\pgfqpoint{1.271090in}{2.178016in}}%
\pgfpathlineto{\pgfqpoint{1.271851in}{2.178436in}}%
\pgfpathlineto{\pgfqpoint{1.280982in}{2.179528in}}%
\pgfpathlineto{\pgfqpoint{1.281320in}{2.179948in}}%
\pgfpathlineto{\pgfqpoint{1.288619in}{2.181040in}}%
\pgfpathlineto{\pgfqpoint{1.289464in}{2.181208in}}%
\pgfpathlineto{\pgfqpoint{1.294593in}{2.182300in}}%
\pgfpathlineto{\pgfqpoint{1.295552in}{2.182720in}}%
\pgfpathlineto{\pgfqpoint{1.304654in}{2.183812in}}%
\pgfpathlineto{\pgfqpoint{1.304654in}{2.183896in}}%
\pgfpathlineto{\pgfqpoint{1.311756in}{2.184988in}}%
\pgfpathlineto{\pgfqpoint{1.312545in}{2.185156in}}%
\pgfpathlineto{\pgfqpoint{1.320380in}{2.186248in}}%
\pgfpathlineto{\pgfqpoint{1.321394in}{2.186584in}}%
\pgfpathlineto{\pgfqpoint{1.328581in}{2.187676in}}%
\pgfpathlineto{\pgfqpoint{1.329116in}{2.187928in}}%
\pgfpathlineto{\pgfqpoint{1.337993in}{2.189020in}}%
\pgfpathlineto{\pgfqpoint{1.337993in}{2.189104in}}%
\pgfpathlineto{\pgfqpoint{1.354367in}{2.190196in}}%
\pgfpathlineto{\pgfqpoint{1.354367in}{2.190280in}}%
\pgfpathlineto{\pgfqpoint{1.370600in}{2.191372in}}%
\pgfpathlineto{\pgfqpoint{1.371614in}{2.191792in}}%
\pgfpathlineto{\pgfqpoint{1.379984in}{2.192884in}}%
\pgfpathlineto{\pgfqpoint{1.380633in}{2.193052in}}%
\pgfpathlineto{\pgfqpoint{1.388383in}{2.194144in}}%
\pgfpathlineto{\pgfqpoint{1.389031in}{2.194480in}}%
\pgfpathlineto{\pgfqpoint{1.396696in}{2.195572in}}%
\pgfpathlineto{\pgfqpoint{1.397232in}{2.195740in}}%
\pgfpathlineto{\pgfqpoint{1.407208in}{2.196832in}}%
\pgfpathlineto{\pgfqpoint{1.407377in}{2.197000in}}%
\pgfpathlineto{\pgfqpoint{1.415014in}{2.198092in}}%
\pgfpathlineto{\pgfqpoint{1.415353in}{2.198260in}}%
\pgfpathlineto{\pgfqpoint{1.427330in}{2.199352in}}%
\pgfpathlineto{\pgfqpoint{1.428288in}{2.199688in}}%
\pgfpathlineto{\pgfqpoint{1.441731in}{2.200780in}}%
\pgfpathlineto{\pgfqpoint{1.442238in}{2.201032in}}%
\pgfpathlineto{\pgfqpoint{1.456075in}{2.202124in}}%
\pgfpathlineto{\pgfqpoint{1.456075in}{2.202208in}}%
\pgfpathlineto{\pgfqpoint{1.468109in}{2.203300in}}%
\pgfpathlineto{\pgfqpoint{1.468109in}{2.203384in}}%
\pgfpathlineto{\pgfqpoint{1.481101in}{2.204476in}}%
\pgfpathlineto{\pgfqpoint{1.481101in}{2.204560in}}%
\pgfpathlineto{\pgfqpoint{1.491753in}{2.205652in}}%
\pgfpathlineto{\pgfqpoint{1.492261in}{2.205988in}}%
\pgfpathlineto{\pgfqpoint{1.509846in}{2.207080in}}%
\pgfpathlineto{\pgfqpoint{1.510184in}{2.207248in}}%
\pgfpathlineto{\pgfqpoint{1.524050in}{2.208340in}}%
\pgfpathlineto{\pgfqpoint{1.524670in}{2.208676in}}%
\pgfpathlineto{\pgfqpoint{1.537464in}{2.209768in}}%
\pgfpathlineto{\pgfqpoint{1.537464in}{2.209852in}}%
\pgfpathlineto{\pgfqpoint{1.549836in}{2.210944in}}%
\pgfpathlineto{\pgfqpoint{1.549836in}{2.211028in}}%
\pgfpathlineto{\pgfqpoint{1.563504in}{2.212120in}}%
\pgfpathlineto{\pgfqpoint{1.564463in}{2.212456in}}%
\pgfpathlineto{\pgfqpoint{1.578328in}{2.213548in}}%
\pgfpathlineto{\pgfqpoint{1.579343in}{2.213716in}}%
\pgfpathlineto{\pgfqpoint{1.588868in}{2.214808in}}%
\pgfpathlineto{\pgfqpoint{1.589347in}{2.214976in}}%
\pgfpathlineto{\pgfqpoint{1.610061in}{2.216068in}}%
\pgfpathlineto{\pgfqpoint{1.610765in}{2.216236in}}%
\pgfpathlineto{\pgfqpoint{1.624349in}{2.217328in}}%
\pgfpathlineto{\pgfqpoint{1.624349in}{2.217412in}}%
\pgfpathlineto{\pgfqpoint{1.633142in}{2.218504in}}%
\pgfpathlineto{\pgfqpoint{1.633142in}{2.218588in}}%
\pgfpathlineto{\pgfqpoint{1.656053in}{2.219680in}}%
\pgfpathlineto{\pgfqpoint{1.656730in}{2.219848in}}%
\pgfpathlineto{\pgfqpoint{1.668904in}{2.220940in}}%
\pgfpathlineto{\pgfqpoint{1.669806in}{2.221192in}}%
\pgfpathlineto{\pgfqpoint{1.685306in}{2.222284in}}%
\pgfpathlineto{\pgfqpoint{1.686152in}{2.222536in}}%
\pgfpathlineto{\pgfqpoint{1.710642in}{2.223628in}}%
\pgfpathlineto{\pgfqpoint{1.710867in}{2.223796in}}%
\pgfpathlineto{\pgfqpoint{1.736823in}{2.224888in}}%
\pgfpathlineto{\pgfqpoint{1.736823in}{2.224972in}}%
\pgfpathlineto{\pgfqpoint{1.769175in}{2.226064in}}%
\pgfpathlineto{\pgfqpoint{1.769175in}{2.226148in}}%
\pgfpathlineto{\pgfqpoint{1.795131in}{2.227240in}}%
\pgfpathlineto{\pgfqpoint{1.795131in}{2.227324in}}%
\pgfpathlineto{\pgfqpoint{1.815788in}{2.228416in}}%
\pgfpathlineto{\pgfqpoint{1.816295in}{2.228584in}}%
\pgfpathlineto{\pgfqpoint{1.833430in}{2.229676in}}%
\pgfpathlineto{\pgfqpoint{1.833430in}{2.229760in}}%
\pgfpathlineto{\pgfqpoint{1.855299in}{2.230852in}}%
\pgfpathlineto{\pgfqpoint{1.855299in}{2.230936in}}%
\pgfpathlineto{\pgfqpoint{1.884242in}{2.232028in}}%
\pgfpathlineto{\pgfqpoint{1.884242in}{2.232112in}}%
\pgfpathlineto{\pgfqpoint{1.914171in}{2.233204in}}%
\pgfpathlineto{\pgfqpoint{1.914171in}{2.233288in}}%
\pgfpathlineto{\pgfqpoint{1.943733in}{2.234380in}}%
\pgfpathlineto{\pgfqpoint{1.944635in}{2.234548in}}%
\pgfpathlineto{\pgfqpoint{1.970027in}{2.235640in}}%
\pgfpathlineto{\pgfqpoint{1.970027in}{2.235724in}}%
\pgfpathlineto{\pgfqpoint{1.990769in}{2.236816in}}%
\pgfpathlineto{\pgfqpoint{1.991333in}{2.236984in}}%
\pgfpathlineto{\pgfqpoint{2.016583in}{2.238076in}}%
\pgfpathlineto{\pgfqpoint{2.016668in}{2.238244in}}%
\pgfpathlineto{\pgfqpoint{2.033013in}{2.239336in}}%
\pgfpathlineto{\pgfqpoint{2.033943in}{2.239504in}}%
\pgfpathlineto{\pgfqpoint{2.060547in}{2.240596in}}%
\pgfpathlineto{\pgfqpoint{2.060547in}{2.240680in}}%
\pgfpathlineto{\pgfqpoint{2.084051in}{2.241772in}}%
\pgfpathlineto{\pgfqpoint{2.084502in}{2.241940in}}%
\pgfpathlineto{\pgfqpoint{2.112853in}{2.243032in}}%
\pgfpathlineto{\pgfqpoint{2.113191in}{2.243200in}}%
\pgfpathlineto{\pgfqpoint{2.139400in}{2.244292in}}%
\pgfpathlineto{\pgfqpoint{2.139851in}{2.244460in}}%
\pgfpathlineto{\pgfqpoint{2.179503in}{2.245552in}}%
\pgfpathlineto{\pgfqpoint{2.179503in}{2.245636in}}%
\pgfpathlineto{\pgfqpoint{2.216393in}{2.246728in}}%
\pgfpathlineto{\pgfqpoint{2.216393in}{2.246812in}}%
\pgfpathlineto{\pgfqpoint{2.273517in}{2.247904in}}%
\pgfpathlineto{\pgfqpoint{2.273517in}{2.247988in}}%
\pgfpathlineto{\pgfqpoint{2.336588in}{2.249080in}}%
\pgfpathlineto{\pgfqpoint{2.336588in}{2.249164in}}%
\pgfpathlineto{\pgfqpoint{2.380101in}{2.250256in}}%
\pgfpathlineto{\pgfqpoint{2.381059in}{2.250424in}}%
\pgfpathlineto{\pgfqpoint{2.450978in}{2.251516in}}%
\pgfpathlineto{\pgfqpoint{2.450978in}{2.251600in}}%
\pgfpathlineto{\pgfqpoint{2.504157in}{2.252692in}}%
\pgfpathlineto{\pgfqpoint{2.504157in}{2.252776in}}%
\pgfpathlineto{\pgfqpoint{2.564635in}{2.253868in}}%
\pgfpathlineto{\pgfqpoint{2.564635in}{2.253952in}}%
\pgfpathlineto{\pgfqpoint{2.638021in}{2.255044in}}%
\pgfpathlineto{\pgfqpoint{2.638021in}{2.255128in}}%
\pgfpathlineto{\pgfqpoint{2.675023in}{2.256220in}}%
\pgfpathlineto{\pgfqpoint{2.675023in}{2.256304in}}%
\pgfpathlineto{\pgfqpoint{2.741476in}{2.257396in}}%
\pgfpathlineto{\pgfqpoint{2.741476in}{2.257480in}}%
\pgfpathlineto{\pgfqpoint{2.791922in}{2.258572in}}%
\pgfpathlineto{\pgfqpoint{2.791922in}{2.258656in}}%
\pgfpathlineto{\pgfqpoint{2.842452in}{2.259748in}}%
\pgfpathlineto{\pgfqpoint{2.842452in}{2.259832in}}%
\pgfpathlineto{\pgfqpoint{2.899435in}{2.260924in}}%
\pgfpathlineto{\pgfqpoint{2.899435in}{2.261008in}}%
\pgfpathlineto{\pgfqpoint{2.917331in}{2.262100in}}%
\pgfpathlineto{\pgfqpoint{2.918035in}{2.262436in}}%
\pgfpathlineto{\pgfqpoint{2.920853in}{2.263192in}}%
\pgfpathlineto{\pgfqpoint{2.920853in}{2.263444in}}%
\pgfusepath{stroke}%
\end{pgfscope}%
\begin{pgfscope}%
\pgfsetrectcap%
\pgfsetmiterjoin%
\pgfsetlinewidth{0.803000pt}%
\definecolor{currentstroke}{rgb}{0.000000,0.000000,0.000000}%
\pgfsetstrokecolor{currentstroke}%
\pgfsetdash{}{0pt}%
\pgfpathmoveto{\pgfqpoint{0.553581in}{0.499444in}}%
\pgfpathlineto{\pgfqpoint{0.553581in}{2.347444in}}%
\pgfusepath{stroke}%
\end{pgfscope}%
\begin{pgfscope}%
\pgfsetrectcap%
\pgfsetmiterjoin%
\pgfsetlinewidth{0.803000pt}%
\definecolor{currentstroke}{rgb}{0.000000,0.000000,0.000000}%
\pgfsetstrokecolor{currentstroke}%
\pgfsetdash{}{0pt}%
\pgfpathmoveto{\pgfqpoint{3.033581in}{0.499444in}}%
\pgfpathlineto{\pgfqpoint{3.033581in}{2.347444in}}%
\pgfusepath{stroke}%
\end{pgfscope}%
\begin{pgfscope}%
\pgfsetrectcap%
\pgfsetmiterjoin%
\pgfsetlinewidth{0.803000pt}%
\definecolor{currentstroke}{rgb}{0.000000,0.000000,0.000000}%
\pgfsetstrokecolor{currentstroke}%
\pgfsetdash{}{0pt}%
\pgfpathmoveto{\pgfqpoint{0.553581in}{0.499444in}}%
\pgfpathlineto{\pgfqpoint{3.033581in}{0.499444in}}%
\pgfusepath{stroke}%
\end{pgfscope}%
\begin{pgfscope}%
\pgfsetrectcap%
\pgfsetmiterjoin%
\pgfsetlinewidth{0.803000pt}%
\definecolor{currentstroke}{rgb}{0.000000,0.000000,0.000000}%
\pgfsetstrokecolor{currentstroke}%
\pgfsetdash{}{0pt}%
\pgfpathmoveto{\pgfqpoint{0.553581in}{2.347444in}}%
\pgfpathlineto{\pgfqpoint{3.033581in}{2.347444in}}%
\pgfusepath{stroke}%
\end{pgfscope}%
\begin{pgfscope}%
\pgfsetbuttcap%
\pgfsetmiterjoin%
\definecolor{currentfill}{rgb}{1.000000,1.000000,1.000000}%
\pgfsetfillcolor{currentfill}%
\pgfsetlinewidth{1.003750pt}%
\definecolor{currentstroke}{rgb}{1.000000,1.000000,1.000000}%
\pgfsetstrokecolor{currentstroke}%
\pgfsetdash{}{0pt}%
\pgfpathmoveto{\pgfqpoint{1.738673in}{2.144572in}}%
\pgfpathlineto{\pgfqpoint{2.166173in}{2.144572in}}%
\pgfpathlineto{\pgfqpoint{2.166173in}{2.379017in}}%
\pgfpathlineto{\pgfqpoint{1.738673in}{2.379017in}}%
\pgfpathlineto{\pgfqpoint{1.738673in}{2.144572in}}%
\pgfpathclose%
\pgfusepath{stroke,fill}%
\end{pgfscope}%
\begin{pgfscope}%
\definecolor{textcolor}{rgb}{0.000000,0.000000,0.000000}%
\pgfsetstrokecolor{textcolor}%
\pgfsetfillcolor{textcolor}%
\pgftext[x=1.794229in,y=2.227072in,left,base]{\color{textcolor}\rmfamily\fontsize{10.000000}{12.000000}\selectfont 0.268}%
\end{pgfscope}%
\begin{pgfscope}%
\pgfsetbuttcap%
\pgfsetmiterjoin%
\definecolor{currentfill}{rgb}{1.000000,1.000000,1.000000}%
\pgfsetfillcolor{currentfill}%
\pgfsetlinewidth{1.003750pt}%
\definecolor{currentstroke}{rgb}{1.000000,1.000000,1.000000}%
\pgfsetstrokecolor{currentstroke}%
\pgfsetdash{}{0pt}%
\pgfpathmoveto{\pgfqpoint{0.656773in}{1.340860in}}%
\pgfpathlineto{\pgfqpoint{1.084273in}{1.340860in}}%
\pgfpathlineto{\pgfqpoint{1.084273in}{1.575305in}}%
\pgfpathlineto{\pgfqpoint{0.656773in}{1.575305in}}%
\pgfpathlineto{\pgfqpoint{0.656773in}{1.340860in}}%
\pgfpathclose%
\pgfusepath{stroke,fill}%
\end{pgfscope}%
\begin{pgfscope}%
\definecolor{textcolor}{rgb}{0.000000,0.000000,0.000000}%
\pgfsetstrokecolor{textcolor}%
\pgfsetfillcolor{textcolor}%
\pgftext[x=0.712329in,y=1.423360in,left,base]{\color{textcolor}\rmfamily\fontsize{10.000000}{12.000000}\selectfont 0.733}%
\end{pgfscope}%
\begin{pgfscope}%
\definecolor{textcolor}{rgb}{0.000000,0.000000,0.000000}%
\pgfsetstrokecolor{textcolor}%
\pgfsetfillcolor{textcolor}%
\pgftext[x=1.793581in,y=2.430778in,,base]{\color{textcolor}\rmfamily\fontsize{12.000000}{14.400000}\selectfont ROC Curve}%
\end{pgfscope}%
\begin{pgfscope}%
\pgfsetbuttcap%
\pgfsetmiterjoin%
\definecolor{currentfill}{rgb}{1.000000,1.000000,1.000000}%
\pgfsetfillcolor{currentfill}%
\pgfsetfillopacity{0.800000}%
\pgfsetlinewidth{1.003750pt}%
\definecolor{currentstroke}{rgb}{0.800000,0.800000,0.800000}%
\pgfsetstrokecolor{currentstroke}%
\pgfsetstrokeopacity{0.800000}%
\pgfsetdash{}{0pt}%
\pgfpathmoveto{\pgfqpoint{0.800942in}{0.568889in}}%
\pgfpathlineto{\pgfqpoint{2.936358in}{0.568889in}}%
\pgfpathquadraticcurveto{\pgfqpoint{2.964136in}{0.568889in}}{\pgfqpoint{2.964136in}{0.596666in}}%
\pgfpathlineto{\pgfqpoint{2.964136in}{0.791111in}}%
\pgfpathquadraticcurveto{\pgfqpoint{2.964136in}{0.818888in}}{\pgfqpoint{2.936358in}{0.818888in}}%
\pgfpathlineto{\pgfqpoint{0.800942in}{0.818888in}}%
\pgfpathquadraticcurveto{\pgfqpoint{0.773164in}{0.818888in}}{\pgfqpoint{0.773164in}{0.791111in}}%
\pgfpathlineto{\pgfqpoint{0.773164in}{0.596666in}}%
\pgfpathquadraticcurveto{\pgfqpoint{0.773164in}{0.568889in}}{\pgfqpoint{0.800942in}{0.568889in}}%
\pgfpathlineto{\pgfqpoint{0.800942in}{0.568889in}}%
\pgfpathclose%
\pgfusepath{stroke,fill}%
\end{pgfscope}%
\begin{pgfscope}%
\pgfsetrectcap%
\pgfsetroundjoin%
\pgfsetlinewidth{1.505625pt}%
\definecolor{currentstroke}{rgb}{0.121569,0.466667,0.705882}%
\pgfsetstrokecolor{currentstroke}%
\pgfsetdash{}{0pt}%
\pgfpathmoveto{\pgfqpoint{0.828720in}{0.707777in}}%
\pgfpathlineto{\pgfqpoint{0.967608in}{0.707777in}}%
\pgfpathlineto{\pgfqpoint{1.106497in}{0.707777in}}%
\pgfusepath{stroke}%
\end{pgfscope}%
\begin{pgfscope}%
\definecolor{textcolor}{rgb}{0.000000,0.000000,0.000000}%
\pgfsetstrokecolor{textcolor}%
\pgfsetfillcolor{textcolor}%
\pgftext[x=1.217608in,y=0.659166in,left,base]{\color{textcolor}\rmfamily\fontsize{10.000000}{12.000000}\selectfont Area Under Curve = 0.937)}%
\end{pgfscope}%
\end{pgfpicture}%
\makeatother%
\endgroup%

  &
\vspace{0pt} 
  
\begin{tabular}{cc|c|c|}
	&\multicolumn{1}{c}{}& \multicolumn{2}{c}{Prediction} \cr
	&\multicolumn{1}{c}{} & \multicolumn{1}{c}{N} & \multicolumn{1}{c}{P} \cr\cline{3-4}
	\multirow{2}{*}{\rotatebox[origin=c]{90}{Actual}}&N & 117,929 & 32,842 \vrule width 0pt height 10pt depth 2pt \cr\cline{3-4}
	&P & 5,928 & 20,693 \vrule width 0pt height 10pt depth 2pt \cr\cline{3-4}
\end{tabular}

\begin{center}
\begin{tabular}{ll}
0.387 & Precision \cr 
0.777 & Recall \cr 
0.516 & F1 \cr 
\end{tabular}
\end{center}
  
\end{tabular}

Note that with threshold $p=0.5$ the false positives ($FP=32,842$) are less than twice as many as the true positives ($TP=20,693$), but that does not satisfy our requirement that $\Delta FP/\Delta TP < 2.0$.  The plot below shows the rate of change as a function of $p$, and that $\Delta FP/\Delta TP = 2.0$ when $p=0.635$.  

\input{../Keras/Images/Ideal_Shift_to_FP_equals_r_TP_FP_TP.pgf}

We want $p=0.635$ to be our threshold while still using tools that assume $p=0.5$ is the threshold, so we will linearly transform the probabilities, mapping $0.635 \to 0.5$ while keeping 0.0 at 0.0, shown below in Example 2. Note that the linear transformation of the probabilities has no affect on the shape of the ROC curve nor the area under it.  The 0.243 and 0.52 are the median transformed probabilities for the negative and positive classes, respectively.  

\noindent\begin{tabular}{@{}p{0.3\textwidth}@{\hspace{24pt}} p{0.3\textwidth} @{\hspace{24pt}} p{0.3\textwidth}}
  \vspace{0pt} \input{../Keras/Images/Ideal_Shift_to_FP_equals_r_TP_Pred.pgf}
  &
  \vspace{0pt} \input{../Keras/Images/Ideal_Shift_to_FP_equals_r_TP_ROC.pgf}
  &
\vspace{0pt} 
  
\begin{tabular}{cc|c|c|}
	&\multicolumn{1}{c}{}& \multicolumn{2}{c}{Prediction} \cr
	&\multicolumn{1}{c}{} & \multicolumn{1}{c}{N} & \multicolumn{1}{c}{P} \cr\cline{3-4}
	\multirow{2}{*}{\rotatebox[origin=c]{90}{Actual}}&N & 136,348 & 14,423 \vrule width 0pt height 10pt depth 2pt \cr\cline{3-4}
	&P & 12,017 & 14,604 \vrule width 0pt height 10pt depth 2pt \cr\cline{3-4}
\end{tabular}

\begin{center}
\begin{tabular}{ll}
0.503 & Precision \cr 
0.549 & Recall \cr 
0.525 & F1 \cr 
\end{tabular}
\end{center}
  
\end{tabular}

We have decided that we want $\Delta FP/\Delta TP < 2.0$; we had $FP/TP < 2.0$, and now have $FP/TP \approx 1$, sending fewer ambulances to crashes that need one.  Why is this better?  If we look at the change in FP and TP from changing the threshold $p = 0.5 \to 0.635$, 

$$\frac{\Delta FP}{\Delta TP} = \frac{32,842 - 14,423}{20,693-14,604} = \frac{18,419}{6,089} \approx 3$$

By changing the threshold, we have not sent some ambulances that were needed, but have not sent three times as many ambulances that were not needed.  Given our goal of $\Delta FP/\Delta TP < 2$, this tradeoff is appropriate.  

It sometimes happens that, using $p=0.5$, a model will recommend that we never send an ambulance, as in Example 3 below, which is a linear transformation of Example 1, $f(x) = 0.5x$.  (One cause of such a model is the class weight being too low.) Such a model can still be useful if it separates the negative and positive classes well, because we can move the threshold.   


\noindent\begin{tabular}{@{}p{0.3\textwidth}@{\hspace{24pt}} p{0.3\textwidth} @{\hspace{24pt}} p{0.3\textwidth}}
  \vspace{0pt} \input{../Keras/Images/Ideal_Left_Pred.pgf}
%  &
%  \vspace{0pt} \input{../Keras/Images/Ideal_Left_ROC.pgf}
  &
\vspace{0pt} 
  
\begin{tabular}{cc|c|c|}
	&\multicolumn{1}{c}{}& \multicolumn{2}{c}{Prediction} \cr
	&\multicolumn{1}{c}{} & \multicolumn{1}{c}{N} & \multicolumn{1}{c}{P} \cr\cline{3-4}
	\multirow{2}{*}{\rotatebox[origin=c]{90}{Actual}}&N & 150,771    &  0 \vrule width 0pt height 10pt depth 2pt \cr\cline{3-4}
	&P & 22,621 & 0 \vrule width 0pt height 10pt depth 2pt \cr\cline{3-4}
\end{tabular}

\begin{center}
\begin{tabular}{ll}
und & Precision \cr 
0.000 & Recall \cr 
und & F1 \cr 
\end{tabular}
\end{center}
  
\end{tabular}

Similarly, if the class weight is too high, we may get a model like Example 4 that sends an ambulance to every reported crash.  Example 4 is Example 1 transformed with $f(x) = 0.5x + 0.5$

\noindent\begin{tabular}{@{}p{0.3\textwidth}@{\hspace{24pt}} p{0.3\textwidth} @{\hspace{24pt}} p{0.3\textwidth}}
  \vspace{0pt} \input{../Keras/Images/Ideal_Right_Pred.pgf}
%  &
%  \vspace{0pt} \input{../Keras/Images/Ideal_Left_ROC.pgf}
  &
\vspace{0pt} 
  
\begin{tabular}{cc|c|c|}
	&\multicolumn{1}{c}{}& \multicolumn{2}{c}{Prediction} \cr
	&\multicolumn{1}{c}{} & \multicolumn{1}{c}{N} & \multicolumn{1}{c}{P} \cr\cline{3-4}
	\multirow{2}{*}{\rotatebox[origin=c]{90}{Actual}}&N & 0 & 150,771   \vrule width 0pt height 10pt depth 2pt \cr\cline{3-4}
	&P & 0 & 22,621 \vrule width 0pt height 10pt depth 2pt \cr\cline{3-4}
\end{tabular}

\begin{center}
\begin{tabular}{ll}
0.150 & Precision \cr 
1.000 & Recall \cr 
0.261 & F1 \cr 
\end{tabular}
\end{center}
  
\end{tabular}

Some models, like the Easy Ensemble Classifier and RUSBoost, give a tight range of probabilities.  Example 5 below is the probabilities from Example 1 linearly transformed with $f(x) = 0.2x + 0.4$ to have range $p \in [0,4,0.6]$.  As a model it gives the same decisions and metrics as Example 1, but is not as useful as a data visualization for comparing models, which is why we dilate the probabilities to go to $p=0.0$.

\noindent\begin{tabular}{@{}p{0.3\textwidth}@{\hspace{24pt}} p{0.3\textwidth} @{\hspace{24pt}} p{0.3\textwidth}}
  \vspace{0pt} \input{../Keras/Images/Ideal_Tight_Pred.pgf}
%  &
%  \vspace{0pt} \input{../Keras/Images/Ideal_Left_ROC.pgf}
  &
\vspace{0pt} 
  
\begin{tabular}{cc|c|c|}
	&\multicolumn{1}{c}{}& \multicolumn{2}{c}{Prediction} \cr
	&\multicolumn{1}{c}{} & \multicolumn{1}{c}{N} & \multicolumn{1}{c}{P} \cr\cline{3-4}
	\multirow{2}{*}{\rotatebox[origin=c]{90}{Actual}}&N & 117,929 & 32,842 \vrule width 0pt height 10pt depth 2pt \cr\cline{3-4}
	&P & 5,928 & 20,693 \vrule width 0pt height 10pt depth 2pt \cr\cline{3-4}
\end{tabular}

\begin{center}
\begin{tabular}{ll}
0.387 & Precision \cr 
0.777 & Recall \cr 
0.516 & F1 \cr 
\end{tabular}
\end{center}
  
\end{tabular}

Because Examples 3, 4, and 5 are linear transformations of Example 1, when we find the value of decision threshold $p$ that makes $\Delta FP/\Delta TP = 2.0$ and linearly transform the probabilities, each of them becomes the distribution in Example 2.  


