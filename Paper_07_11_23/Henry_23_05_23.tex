\subsection{Abstract}

I made most of the changes you suggested for the abstract.  (Above)

I changed ``police'' to ``emergency dispatcher'' rather than ``911 dispatcher'' so as to not annoy the people (like the editor) in countries that use 119.  

I think the abstract is now too long, but that's a problem for later.  

\subsection{Threshold, Prior and Posterior Probabilities}

Thank you for your summary of the decision problem.  I will adapt the language here. I will try to answer your questions through an example.   Please let me know if I don't understand it well.  

\

I'm trying to understand the $\hat{p}$ you used.  I understand $p$ as the probability that a given sample is in Class 1.  Because you related $\hat{p}$ to $\pi_1$, I suspect what $\hat{p}$ means is the percentage of samples the model (with our choice of decision threshold) classifies as being in Class 1.  Is that correct?  In the big chart below, 

$$\hat{p} = \frac{FP + TP}{TN + FP + FN + TP}$$

\

Using the CRSS dataset, our problem has two outcomes.  

\begin{center}
\begin{tabular}{llll}
	Class 0 & ``No ambulance needed'' & $\pi_0 \approx 85\%$ \cr
	Class 1 & ``Ambulance needed'' & $\pi_1 \approx 15\%$ \cr
\end{tabular}
\end{center}

We will run several models, but a particular model gives, for each input vector (sample), a probability $p$ that the sample is in Class 1.  We want to pick a discrimination threshold (decision threshold) $\theta$ such that if for a particular crash notification $p > \theta$, then our recommendation system recommends that the emergency dispatcher send an ambulance.  

Below is the test results of one of our best models, the Balanced Random Forest Classifier with no class weights ($\alpha = 0.5$).  The histogram gives, for each range of $p$, the percent of the total dataset with Class 0 (Neg) and Class 1 in that range of $p$.  

Until we choose the discrimination threshold $\theta$, we can't have a confusion matrix.  Once we choose $\theta$, then all of the negative samples with $p< \theta$ are true negatives, and all of the negative samples with $p > \theta$ are false positives; conversely for the positive samples.  

\


\noindent\begin{tabular}{@{\hspace{-6pt}}p{4.5in} @{\hspace{-6pt}}p{2.0in}}
	\vskip 0pt
	\qquad \qquad Raw Model Output
	
	%% Creator: Matplotlib, PGF backend
%%
%% To include the figure in your LaTeX document, write
%%   \input{<filename>.pgf}
%%
%% Make sure the required packages are loaded in your preamble
%%   \usepackage{pgf}
%%
%% Also ensure that all the required font packages are loaded; for instance,
%% the lmodern package is sometimes necessary when using math font.
%%   \usepackage{lmodern}
%%
%% Figures using additional raster images can only be included by \input if
%% they are in the same directory as the main LaTeX file. For loading figures
%% from other directories you can use the `import` package
%%   \usepackage{import}
%%
%% and then include the figures with
%%   \import{<path to file>}{<filename>.pgf}
%%
%% Matplotlib used the following preamble
%%   
%%   \usepackage{fontspec}
%%   \makeatletter\@ifpackageloaded{underscore}{}{\usepackage[strings]{underscore}}\makeatother
%%
\begingroup%
\makeatletter%
\begin{pgfpicture}%
\pgfpathrectangle{\pgfpointorigin}{\pgfqpoint{4.509306in}{1.754444in}}%
\pgfusepath{use as bounding box, clip}%
\begin{pgfscope}%
\pgfsetbuttcap%
\pgfsetmiterjoin%
\definecolor{currentfill}{rgb}{1.000000,1.000000,1.000000}%
\pgfsetfillcolor{currentfill}%
\pgfsetlinewidth{0.000000pt}%
\definecolor{currentstroke}{rgb}{1.000000,1.000000,1.000000}%
\pgfsetstrokecolor{currentstroke}%
\pgfsetdash{}{0pt}%
\pgfpathmoveto{\pgfqpoint{0.000000in}{0.000000in}}%
\pgfpathlineto{\pgfqpoint{4.509306in}{0.000000in}}%
\pgfpathlineto{\pgfqpoint{4.509306in}{1.754444in}}%
\pgfpathlineto{\pgfqpoint{0.000000in}{1.754444in}}%
\pgfpathlineto{\pgfqpoint{0.000000in}{0.000000in}}%
\pgfpathclose%
\pgfusepath{fill}%
\end{pgfscope}%
\begin{pgfscope}%
\pgfsetbuttcap%
\pgfsetmiterjoin%
\definecolor{currentfill}{rgb}{1.000000,1.000000,1.000000}%
\pgfsetfillcolor{currentfill}%
\pgfsetlinewidth{0.000000pt}%
\definecolor{currentstroke}{rgb}{0.000000,0.000000,0.000000}%
\pgfsetstrokecolor{currentstroke}%
\pgfsetstrokeopacity{0.000000}%
\pgfsetdash{}{0pt}%
\pgfpathmoveto{\pgfqpoint{0.445556in}{0.499444in}}%
\pgfpathlineto{\pgfqpoint{4.320556in}{0.499444in}}%
\pgfpathlineto{\pgfqpoint{4.320556in}{1.654444in}}%
\pgfpathlineto{\pgfqpoint{0.445556in}{1.654444in}}%
\pgfpathlineto{\pgfqpoint{0.445556in}{0.499444in}}%
\pgfpathclose%
\pgfusepath{fill}%
\end{pgfscope}%
\begin{pgfscope}%
\pgfpathrectangle{\pgfqpoint{0.445556in}{0.499444in}}{\pgfqpoint{3.875000in}{1.155000in}}%
\pgfusepath{clip}%
\pgfsetbuttcap%
\pgfsetmiterjoin%
\pgfsetlinewidth{1.003750pt}%
\definecolor{currentstroke}{rgb}{0.000000,0.000000,0.000000}%
\pgfsetstrokecolor{currentstroke}%
\pgfsetdash{}{0pt}%
\pgfpathmoveto{\pgfqpoint{0.435556in}{0.499444in}}%
\pgfpathlineto{\pgfqpoint{0.483922in}{0.499444in}}%
\pgfpathlineto{\pgfqpoint{0.483922in}{0.618288in}}%
\pgfpathlineto{\pgfqpoint{0.435556in}{0.618288in}}%
\pgfusepath{stroke}%
\end{pgfscope}%
\begin{pgfscope}%
\pgfpathrectangle{\pgfqpoint{0.445556in}{0.499444in}}{\pgfqpoint{3.875000in}{1.155000in}}%
\pgfusepath{clip}%
\pgfsetbuttcap%
\pgfsetmiterjoin%
\pgfsetlinewidth{1.003750pt}%
\definecolor{currentstroke}{rgb}{0.000000,0.000000,0.000000}%
\pgfsetstrokecolor{currentstroke}%
\pgfsetdash{}{0pt}%
\pgfpathmoveto{\pgfqpoint{0.576001in}{0.499444in}}%
\pgfpathlineto{\pgfqpoint{0.637387in}{0.499444in}}%
\pgfpathlineto{\pgfqpoint{0.637387in}{0.801196in}}%
\pgfpathlineto{\pgfqpoint{0.576001in}{0.801196in}}%
\pgfpathlineto{\pgfqpoint{0.576001in}{0.499444in}}%
\pgfpathclose%
\pgfusepath{stroke}%
\end{pgfscope}%
\begin{pgfscope}%
\pgfpathrectangle{\pgfqpoint{0.445556in}{0.499444in}}{\pgfqpoint{3.875000in}{1.155000in}}%
\pgfusepath{clip}%
\pgfsetbuttcap%
\pgfsetmiterjoin%
\pgfsetlinewidth{1.003750pt}%
\definecolor{currentstroke}{rgb}{0.000000,0.000000,0.000000}%
\pgfsetstrokecolor{currentstroke}%
\pgfsetdash{}{0pt}%
\pgfpathmoveto{\pgfqpoint{0.729467in}{0.499444in}}%
\pgfpathlineto{\pgfqpoint{0.790853in}{0.499444in}}%
\pgfpathlineto{\pgfqpoint{0.790853in}{1.026194in}}%
\pgfpathlineto{\pgfqpoint{0.729467in}{1.026194in}}%
\pgfpathlineto{\pgfqpoint{0.729467in}{0.499444in}}%
\pgfpathclose%
\pgfusepath{stroke}%
\end{pgfscope}%
\begin{pgfscope}%
\pgfpathrectangle{\pgfqpoint{0.445556in}{0.499444in}}{\pgfqpoint{3.875000in}{1.155000in}}%
\pgfusepath{clip}%
\pgfsetbuttcap%
\pgfsetmiterjoin%
\pgfsetlinewidth{1.003750pt}%
\definecolor{currentstroke}{rgb}{0.000000,0.000000,0.000000}%
\pgfsetstrokecolor{currentstroke}%
\pgfsetdash{}{0pt}%
\pgfpathmoveto{\pgfqpoint{0.882932in}{0.499444in}}%
\pgfpathlineto{\pgfqpoint{0.944318in}{0.499444in}}%
\pgfpathlineto{\pgfqpoint{0.944318in}{1.203686in}}%
\pgfpathlineto{\pgfqpoint{0.882932in}{1.203686in}}%
\pgfpathlineto{\pgfqpoint{0.882932in}{0.499444in}}%
\pgfpathclose%
\pgfusepath{stroke}%
\end{pgfscope}%
\begin{pgfscope}%
\pgfpathrectangle{\pgfqpoint{0.445556in}{0.499444in}}{\pgfqpoint{3.875000in}{1.155000in}}%
\pgfusepath{clip}%
\pgfsetbuttcap%
\pgfsetmiterjoin%
\pgfsetlinewidth{1.003750pt}%
\definecolor{currentstroke}{rgb}{0.000000,0.000000,0.000000}%
\pgfsetstrokecolor{currentstroke}%
\pgfsetdash{}{0pt}%
\pgfpathmoveto{\pgfqpoint{1.036397in}{0.499444in}}%
\pgfpathlineto{\pgfqpoint{1.097783in}{0.499444in}}%
\pgfpathlineto{\pgfqpoint{1.097783in}{1.348758in}}%
\pgfpathlineto{\pgfqpoint{1.036397in}{1.348758in}}%
\pgfpathlineto{\pgfqpoint{1.036397in}{0.499444in}}%
\pgfpathclose%
\pgfusepath{stroke}%
\end{pgfscope}%
\begin{pgfscope}%
\pgfpathrectangle{\pgfqpoint{0.445556in}{0.499444in}}{\pgfqpoint{3.875000in}{1.155000in}}%
\pgfusepath{clip}%
\pgfsetbuttcap%
\pgfsetmiterjoin%
\pgfsetlinewidth{1.003750pt}%
\definecolor{currentstroke}{rgb}{0.000000,0.000000,0.000000}%
\pgfsetstrokecolor{currentstroke}%
\pgfsetdash{}{0pt}%
\pgfpathmoveto{\pgfqpoint{1.189863in}{0.499444in}}%
\pgfpathlineto{\pgfqpoint{1.251249in}{0.499444in}}%
\pgfpathlineto{\pgfqpoint{1.251249in}{1.455609in}}%
\pgfpathlineto{\pgfqpoint{1.189863in}{1.455609in}}%
\pgfpathlineto{\pgfqpoint{1.189863in}{0.499444in}}%
\pgfpathclose%
\pgfusepath{stroke}%
\end{pgfscope}%
\begin{pgfscope}%
\pgfpathrectangle{\pgfqpoint{0.445556in}{0.499444in}}{\pgfqpoint{3.875000in}{1.155000in}}%
\pgfusepath{clip}%
\pgfsetbuttcap%
\pgfsetmiterjoin%
\pgfsetlinewidth{1.003750pt}%
\definecolor{currentstroke}{rgb}{0.000000,0.000000,0.000000}%
\pgfsetstrokecolor{currentstroke}%
\pgfsetdash{}{0pt}%
\pgfpathmoveto{\pgfqpoint{1.343328in}{0.499444in}}%
\pgfpathlineto{\pgfqpoint{1.404714in}{0.499444in}}%
\pgfpathlineto{\pgfqpoint{1.404714in}{1.549075in}}%
\pgfpathlineto{\pgfqpoint{1.343328in}{1.549075in}}%
\pgfpathlineto{\pgfqpoint{1.343328in}{0.499444in}}%
\pgfpathclose%
\pgfusepath{stroke}%
\end{pgfscope}%
\begin{pgfscope}%
\pgfpathrectangle{\pgfqpoint{0.445556in}{0.499444in}}{\pgfqpoint{3.875000in}{1.155000in}}%
\pgfusepath{clip}%
\pgfsetbuttcap%
\pgfsetmiterjoin%
\pgfsetlinewidth{1.003750pt}%
\definecolor{currentstroke}{rgb}{0.000000,0.000000,0.000000}%
\pgfsetstrokecolor{currentstroke}%
\pgfsetdash{}{0pt}%
\pgfpathmoveto{\pgfqpoint{1.496793in}{0.499444in}}%
\pgfpathlineto{\pgfqpoint{1.558179in}{0.499444in}}%
\pgfpathlineto{\pgfqpoint{1.558179in}{1.599444in}}%
\pgfpathlineto{\pgfqpoint{1.496793in}{1.599444in}}%
\pgfpathlineto{\pgfqpoint{1.496793in}{0.499444in}}%
\pgfpathclose%
\pgfusepath{stroke}%
\end{pgfscope}%
\begin{pgfscope}%
\pgfpathrectangle{\pgfqpoint{0.445556in}{0.499444in}}{\pgfqpoint{3.875000in}{1.155000in}}%
\pgfusepath{clip}%
\pgfsetbuttcap%
\pgfsetmiterjoin%
\pgfsetlinewidth{1.003750pt}%
\definecolor{currentstroke}{rgb}{0.000000,0.000000,0.000000}%
\pgfsetstrokecolor{currentstroke}%
\pgfsetdash{}{0pt}%
\pgfpathmoveto{\pgfqpoint{1.650259in}{0.499444in}}%
\pgfpathlineto{\pgfqpoint{1.711645in}{0.499444in}}%
\pgfpathlineto{\pgfqpoint{1.711645in}{1.597742in}}%
\pgfpathlineto{\pgfqpoint{1.650259in}{1.597742in}}%
\pgfpathlineto{\pgfqpoint{1.650259in}{0.499444in}}%
\pgfpathclose%
\pgfusepath{stroke}%
\end{pgfscope}%
\begin{pgfscope}%
\pgfpathrectangle{\pgfqpoint{0.445556in}{0.499444in}}{\pgfqpoint{3.875000in}{1.155000in}}%
\pgfusepath{clip}%
\pgfsetbuttcap%
\pgfsetmiterjoin%
\pgfsetlinewidth{1.003750pt}%
\definecolor{currentstroke}{rgb}{0.000000,0.000000,0.000000}%
\pgfsetstrokecolor{currentstroke}%
\pgfsetdash{}{0pt}%
\pgfpathmoveto{\pgfqpoint{1.803724in}{0.499444in}}%
\pgfpathlineto{\pgfqpoint{1.865110in}{0.499444in}}%
\pgfpathlineto{\pgfqpoint{1.865110in}{1.571977in}}%
\pgfpathlineto{\pgfqpoint{1.803724in}{1.571977in}}%
\pgfpathlineto{\pgfqpoint{1.803724in}{0.499444in}}%
\pgfpathclose%
\pgfusepath{stroke}%
\end{pgfscope}%
\begin{pgfscope}%
\pgfpathrectangle{\pgfqpoint{0.445556in}{0.499444in}}{\pgfqpoint{3.875000in}{1.155000in}}%
\pgfusepath{clip}%
\pgfsetbuttcap%
\pgfsetmiterjoin%
\pgfsetlinewidth{1.003750pt}%
\definecolor{currentstroke}{rgb}{0.000000,0.000000,0.000000}%
\pgfsetstrokecolor{currentstroke}%
\pgfsetdash{}{0pt}%
\pgfpathmoveto{\pgfqpoint{1.957189in}{0.499444in}}%
\pgfpathlineto{\pgfqpoint{2.018575in}{0.499444in}}%
\pgfpathlineto{\pgfqpoint{2.018575in}{1.535535in}}%
\pgfpathlineto{\pgfqpoint{1.957189in}{1.535535in}}%
\pgfpathlineto{\pgfqpoint{1.957189in}{0.499444in}}%
\pgfpathclose%
\pgfusepath{stroke}%
\end{pgfscope}%
\begin{pgfscope}%
\pgfpathrectangle{\pgfqpoint{0.445556in}{0.499444in}}{\pgfqpoint{3.875000in}{1.155000in}}%
\pgfusepath{clip}%
\pgfsetbuttcap%
\pgfsetmiterjoin%
\pgfsetlinewidth{1.003750pt}%
\definecolor{currentstroke}{rgb}{0.000000,0.000000,0.000000}%
\pgfsetstrokecolor{currentstroke}%
\pgfsetdash{}{0pt}%
\pgfpathmoveto{\pgfqpoint{2.110655in}{0.499444in}}%
\pgfpathlineto{\pgfqpoint{2.172041in}{0.499444in}}%
\pgfpathlineto{\pgfqpoint{2.172041in}{1.450812in}}%
\pgfpathlineto{\pgfqpoint{2.110655in}{1.450812in}}%
\pgfpathlineto{\pgfqpoint{2.110655in}{0.499444in}}%
\pgfpathclose%
\pgfusepath{stroke}%
\end{pgfscope}%
\begin{pgfscope}%
\pgfpathrectangle{\pgfqpoint{0.445556in}{0.499444in}}{\pgfqpoint{3.875000in}{1.155000in}}%
\pgfusepath{clip}%
\pgfsetbuttcap%
\pgfsetmiterjoin%
\pgfsetlinewidth{1.003750pt}%
\definecolor{currentstroke}{rgb}{0.000000,0.000000,0.000000}%
\pgfsetstrokecolor{currentstroke}%
\pgfsetdash{}{0pt}%
\pgfpathmoveto{\pgfqpoint{2.264120in}{0.499444in}}%
\pgfpathlineto{\pgfqpoint{2.325506in}{0.499444in}}%
\pgfpathlineto{\pgfqpoint{2.325506in}{1.379398in}}%
\pgfpathlineto{\pgfqpoint{2.264120in}{1.379398in}}%
\pgfpathlineto{\pgfqpoint{2.264120in}{0.499444in}}%
\pgfpathclose%
\pgfusepath{stroke}%
\end{pgfscope}%
\begin{pgfscope}%
\pgfpathrectangle{\pgfqpoint{0.445556in}{0.499444in}}{\pgfqpoint{3.875000in}{1.155000in}}%
\pgfusepath{clip}%
\pgfsetbuttcap%
\pgfsetmiterjoin%
\pgfsetlinewidth{1.003750pt}%
\definecolor{currentstroke}{rgb}{0.000000,0.000000,0.000000}%
\pgfsetstrokecolor{currentstroke}%
\pgfsetdash{}{0pt}%
\pgfpathmoveto{\pgfqpoint{2.417585in}{0.499444in}}%
\pgfpathlineto{\pgfqpoint{2.478972in}{0.499444in}}%
\pgfpathlineto{\pgfqpoint{2.478972in}{1.248329in}}%
\pgfpathlineto{\pgfqpoint{2.417585in}{1.248329in}}%
\pgfpathlineto{\pgfqpoint{2.417585in}{0.499444in}}%
\pgfpathclose%
\pgfusepath{stroke}%
\end{pgfscope}%
\begin{pgfscope}%
\pgfpathrectangle{\pgfqpoint{0.445556in}{0.499444in}}{\pgfqpoint{3.875000in}{1.155000in}}%
\pgfusepath{clip}%
\pgfsetbuttcap%
\pgfsetmiterjoin%
\pgfsetlinewidth{1.003750pt}%
\definecolor{currentstroke}{rgb}{0.000000,0.000000,0.000000}%
\pgfsetstrokecolor{currentstroke}%
\pgfsetdash{}{0pt}%
\pgfpathmoveto{\pgfqpoint{2.571051in}{0.499444in}}%
\pgfpathlineto{\pgfqpoint{2.632437in}{0.499444in}}%
\pgfpathlineto{\pgfqpoint{2.632437in}{1.129640in}}%
\pgfpathlineto{\pgfqpoint{2.571051in}{1.129640in}}%
\pgfpathlineto{\pgfqpoint{2.571051in}{0.499444in}}%
\pgfpathclose%
\pgfusepath{stroke}%
\end{pgfscope}%
\begin{pgfscope}%
\pgfpathrectangle{\pgfqpoint{0.445556in}{0.499444in}}{\pgfqpoint{3.875000in}{1.155000in}}%
\pgfusepath{clip}%
\pgfsetbuttcap%
\pgfsetmiterjoin%
\pgfsetlinewidth{1.003750pt}%
\definecolor{currentstroke}{rgb}{0.000000,0.000000,0.000000}%
\pgfsetstrokecolor{currentstroke}%
\pgfsetdash{}{0pt}%
\pgfpathmoveto{\pgfqpoint{2.724516in}{0.499444in}}%
\pgfpathlineto{\pgfqpoint{2.785902in}{0.499444in}}%
\pgfpathlineto{\pgfqpoint{2.785902in}{1.011880in}}%
\pgfpathlineto{\pgfqpoint{2.724516in}{1.011880in}}%
\pgfpathlineto{\pgfqpoint{2.724516in}{0.499444in}}%
\pgfpathclose%
\pgfusepath{stroke}%
\end{pgfscope}%
\begin{pgfscope}%
\pgfpathrectangle{\pgfqpoint{0.445556in}{0.499444in}}{\pgfqpoint{3.875000in}{1.155000in}}%
\pgfusepath{clip}%
\pgfsetbuttcap%
\pgfsetmiterjoin%
\pgfsetlinewidth{1.003750pt}%
\definecolor{currentstroke}{rgb}{0.000000,0.000000,0.000000}%
\pgfsetstrokecolor{currentstroke}%
\pgfsetdash{}{0pt}%
\pgfpathmoveto{\pgfqpoint{2.877981in}{0.499444in}}%
\pgfpathlineto{\pgfqpoint{2.939368in}{0.499444in}}%
\pgfpathlineto{\pgfqpoint{2.939368in}{0.912921in}}%
\pgfpathlineto{\pgfqpoint{2.877981in}{0.912921in}}%
\pgfpathlineto{\pgfqpoint{2.877981in}{0.499444in}}%
\pgfpathclose%
\pgfusepath{stroke}%
\end{pgfscope}%
\begin{pgfscope}%
\pgfpathrectangle{\pgfqpoint{0.445556in}{0.499444in}}{\pgfqpoint{3.875000in}{1.155000in}}%
\pgfusepath{clip}%
\pgfsetbuttcap%
\pgfsetmiterjoin%
\pgfsetlinewidth{1.003750pt}%
\definecolor{currentstroke}{rgb}{0.000000,0.000000,0.000000}%
\pgfsetstrokecolor{currentstroke}%
\pgfsetdash{}{0pt}%
\pgfpathmoveto{\pgfqpoint{3.031447in}{0.499444in}}%
\pgfpathlineto{\pgfqpoint{3.092833in}{0.499444in}}%
\pgfpathlineto{\pgfqpoint{3.092833in}{0.805374in}}%
\pgfpathlineto{\pgfqpoint{3.031447in}{0.805374in}}%
\pgfpathlineto{\pgfqpoint{3.031447in}{0.499444in}}%
\pgfpathclose%
\pgfusepath{stroke}%
\end{pgfscope}%
\begin{pgfscope}%
\pgfpathrectangle{\pgfqpoint{0.445556in}{0.499444in}}{\pgfqpoint{3.875000in}{1.155000in}}%
\pgfusepath{clip}%
\pgfsetbuttcap%
\pgfsetmiterjoin%
\pgfsetlinewidth{1.003750pt}%
\definecolor{currentstroke}{rgb}{0.000000,0.000000,0.000000}%
\pgfsetstrokecolor{currentstroke}%
\pgfsetdash{}{0pt}%
\pgfpathmoveto{\pgfqpoint{3.184912in}{0.499444in}}%
\pgfpathlineto{\pgfqpoint{3.246298in}{0.499444in}}%
\pgfpathlineto{\pgfqpoint{3.246298in}{0.730710in}}%
\pgfpathlineto{\pgfqpoint{3.184912in}{0.730710in}}%
\pgfpathlineto{\pgfqpoint{3.184912in}{0.499444in}}%
\pgfpathclose%
\pgfusepath{stroke}%
\end{pgfscope}%
\begin{pgfscope}%
\pgfpathrectangle{\pgfqpoint{0.445556in}{0.499444in}}{\pgfqpoint{3.875000in}{1.155000in}}%
\pgfusepath{clip}%
\pgfsetbuttcap%
\pgfsetmiterjoin%
\pgfsetlinewidth{1.003750pt}%
\definecolor{currentstroke}{rgb}{0.000000,0.000000,0.000000}%
\pgfsetstrokecolor{currentstroke}%
\pgfsetdash{}{0pt}%
\pgfpathmoveto{\pgfqpoint{3.338377in}{0.499444in}}%
\pgfpathlineto{\pgfqpoint{3.399764in}{0.499444in}}%
\pgfpathlineto{\pgfqpoint{3.399764in}{0.672603in}}%
\pgfpathlineto{\pgfqpoint{3.338377in}{0.672603in}}%
\pgfpathlineto{\pgfqpoint{3.338377in}{0.499444in}}%
\pgfpathclose%
\pgfusepath{stroke}%
\end{pgfscope}%
\begin{pgfscope}%
\pgfpathrectangle{\pgfqpoint{0.445556in}{0.499444in}}{\pgfqpoint{3.875000in}{1.155000in}}%
\pgfusepath{clip}%
\pgfsetbuttcap%
\pgfsetmiterjoin%
\pgfsetlinewidth{1.003750pt}%
\definecolor{currentstroke}{rgb}{0.000000,0.000000,0.000000}%
\pgfsetstrokecolor{currentstroke}%
\pgfsetdash{}{0pt}%
\pgfpathmoveto{\pgfqpoint{3.491843in}{0.499444in}}%
\pgfpathlineto{\pgfqpoint{3.553229in}{0.499444in}}%
\pgfpathlineto{\pgfqpoint{3.553229in}{0.616663in}}%
\pgfpathlineto{\pgfqpoint{3.491843in}{0.616663in}}%
\pgfpathlineto{\pgfqpoint{3.491843in}{0.499444in}}%
\pgfpathclose%
\pgfusepath{stroke}%
\end{pgfscope}%
\begin{pgfscope}%
\pgfpathrectangle{\pgfqpoint{0.445556in}{0.499444in}}{\pgfqpoint{3.875000in}{1.155000in}}%
\pgfusepath{clip}%
\pgfsetbuttcap%
\pgfsetmiterjoin%
\pgfsetlinewidth{1.003750pt}%
\definecolor{currentstroke}{rgb}{0.000000,0.000000,0.000000}%
\pgfsetstrokecolor{currentstroke}%
\pgfsetdash{}{0pt}%
\pgfpathmoveto{\pgfqpoint{3.645308in}{0.499444in}}%
\pgfpathlineto{\pgfqpoint{3.706694in}{0.499444in}}%
\pgfpathlineto{\pgfqpoint{3.706694in}{0.574031in}}%
\pgfpathlineto{\pgfqpoint{3.645308in}{0.574031in}}%
\pgfpathlineto{\pgfqpoint{3.645308in}{0.499444in}}%
\pgfpathclose%
\pgfusepath{stroke}%
\end{pgfscope}%
\begin{pgfscope}%
\pgfpathrectangle{\pgfqpoint{0.445556in}{0.499444in}}{\pgfqpoint{3.875000in}{1.155000in}}%
\pgfusepath{clip}%
\pgfsetbuttcap%
\pgfsetmiterjoin%
\pgfsetlinewidth{1.003750pt}%
\definecolor{currentstroke}{rgb}{0.000000,0.000000,0.000000}%
\pgfsetstrokecolor{currentstroke}%
\pgfsetdash{}{0pt}%
\pgfpathmoveto{\pgfqpoint{3.798774in}{0.499444in}}%
\pgfpathlineto{\pgfqpoint{3.860160in}{0.499444in}}%
\pgfpathlineto{\pgfqpoint{3.860160in}{0.550587in}}%
\pgfpathlineto{\pgfqpoint{3.798774in}{0.550587in}}%
\pgfpathlineto{\pgfqpoint{3.798774in}{0.499444in}}%
\pgfpathclose%
\pgfusepath{stroke}%
\end{pgfscope}%
\begin{pgfscope}%
\pgfpathrectangle{\pgfqpoint{0.445556in}{0.499444in}}{\pgfqpoint{3.875000in}{1.155000in}}%
\pgfusepath{clip}%
\pgfsetbuttcap%
\pgfsetmiterjoin%
\pgfsetlinewidth{1.003750pt}%
\definecolor{currentstroke}{rgb}{0.000000,0.000000,0.000000}%
\pgfsetstrokecolor{currentstroke}%
\pgfsetdash{}{0pt}%
\pgfpathmoveto{\pgfqpoint{3.952239in}{0.499444in}}%
\pgfpathlineto{\pgfqpoint{4.013625in}{0.499444in}}%
\pgfpathlineto{\pgfqpoint{4.013625in}{0.526679in}}%
\pgfpathlineto{\pgfqpoint{3.952239in}{0.526679in}}%
\pgfpathlineto{\pgfqpoint{3.952239in}{0.499444in}}%
\pgfpathclose%
\pgfusepath{stroke}%
\end{pgfscope}%
\begin{pgfscope}%
\pgfpathrectangle{\pgfqpoint{0.445556in}{0.499444in}}{\pgfqpoint{3.875000in}{1.155000in}}%
\pgfusepath{clip}%
\pgfsetbuttcap%
\pgfsetmiterjoin%
\pgfsetlinewidth{1.003750pt}%
\definecolor{currentstroke}{rgb}{0.000000,0.000000,0.000000}%
\pgfsetstrokecolor{currentstroke}%
\pgfsetdash{}{0pt}%
\pgfpathmoveto{\pgfqpoint{4.105704in}{0.499444in}}%
\pgfpathlineto{\pgfqpoint{4.167090in}{0.499444in}}%
\pgfpathlineto{\pgfqpoint{4.167090in}{0.506640in}}%
\pgfpathlineto{\pgfqpoint{4.105704in}{0.506640in}}%
\pgfpathlineto{\pgfqpoint{4.105704in}{0.499444in}}%
\pgfpathclose%
\pgfusepath{stroke}%
\end{pgfscope}%
\begin{pgfscope}%
\pgfpathrectangle{\pgfqpoint{0.445556in}{0.499444in}}{\pgfqpoint{3.875000in}{1.155000in}}%
\pgfusepath{clip}%
\pgfsetbuttcap%
\pgfsetmiterjoin%
\definecolor{currentfill}{rgb}{0.000000,0.000000,0.000000}%
\pgfsetfillcolor{currentfill}%
\pgfsetlinewidth{0.000000pt}%
\definecolor{currentstroke}{rgb}{0.000000,0.000000,0.000000}%
\pgfsetstrokecolor{currentstroke}%
\pgfsetstrokeopacity{0.000000}%
\pgfsetdash{}{0pt}%
\pgfpathmoveto{\pgfqpoint{0.483922in}{0.499444in}}%
\pgfpathlineto{\pgfqpoint{0.545308in}{0.499444in}}%
\pgfpathlineto{\pgfqpoint{0.545308in}{0.500527in}}%
\pgfpathlineto{\pgfqpoint{0.483922in}{0.500527in}}%
\pgfpathlineto{\pgfqpoint{0.483922in}{0.499444in}}%
\pgfpathclose%
\pgfusepath{fill}%
\end{pgfscope}%
\begin{pgfscope}%
\pgfpathrectangle{\pgfqpoint{0.445556in}{0.499444in}}{\pgfqpoint{3.875000in}{1.155000in}}%
\pgfusepath{clip}%
\pgfsetbuttcap%
\pgfsetmiterjoin%
\definecolor{currentfill}{rgb}{0.000000,0.000000,0.000000}%
\pgfsetfillcolor{currentfill}%
\pgfsetlinewidth{0.000000pt}%
\definecolor{currentstroke}{rgb}{0.000000,0.000000,0.000000}%
\pgfsetstrokecolor{currentstroke}%
\pgfsetstrokeopacity{0.000000}%
\pgfsetdash{}{0pt}%
\pgfpathmoveto{\pgfqpoint{0.637387in}{0.499444in}}%
\pgfpathlineto{\pgfqpoint{0.698774in}{0.499444in}}%
\pgfpathlineto{\pgfqpoint{0.698774in}{0.502230in}}%
\pgfpathlineto{\pgfqpoint{0.637387in}{0.502230in}}%
\pgfpathlineto{\pgfqpoint{0.637387in}{0.499444in}}%
\pgfpathclose%
\pgfusepath{fill}%
\end{pgfscope}%
\begin{pgfscope}%
\pgfpathrectangle{\pgfqpoint{0.445556in}{0.499444in}}{\pgfqpoint{3.875000in}{1.155000in}}%
\pgfusepath{clip}%
\pgfsetbuttcap%
\pgfsetmiterjoin%
\definecolor{currentfill}{rgb}{0.000000,0.000000,0.000000}%
\pgfsetfillcolor{currentfill}%
\pgfsetlinewidth{0.000000pt}%
\definecolor{currentstroke}{rgb}{0.000000,0.000000,0.000000}%
\pgfsetstrokecolor{currentstroke}%
\pgfsetstrokeopacity{0.000000}%
\pgfsetdash{}{0pt}%
\pgfpathmoveto{\pgfqpoint{0.790853in}{0.499444in}}%
\pgfpathlineto{\pgfqpoint{0.852239in}{0.499444in}}%
\pgfpathlineto{\pgfqpoint{0.852239in}{0.506330in}}%
\pgfpathlineto{\pgfqpoint{0.790853in}{0.506330in}}%
\pgfpathlineto{\pgfqpoint{0.790853in}{0.499444in}}%
\pgfpathclose%
\pgfusepath{fill}%
\end{pgfscope}%
\begin{pgfscope}%
\pgfpathrectangle{\pgfqpoint{0.445556in}{0.499444in}}{\pgfqpoint{3.875000in}{1.155000in}}%
\pgfusepath{clip}%
\pgfsetbuttcap%
\pgfsetmiterjoin%
\definecolor{currentfill}{rgb}{0.000000,0.000000,0.000000}%
\pgfsetfillcolor{currentfill}%
\pgfsetlinewidth{0.000000pt}%
\definecolor{currentstroke}{rgb}{0.000000,0.000000,0.000000}%
\pgfsetstrokecolor{currentstroke}%
\pgfsetstrokeopacity{0.000000}%
\pgfsetdash{}{0pt}%
\pgfpathmoveto{\pgfqpoint{0.944318in}{0.499444in}}%
\pgfpathlineto{\pgfqpoint{1.005704in}{0.499444in}}%
\pgfpathlineto{\pgfqpoint{1.005704in}{0.514068in}}%
\pgfpathlineto{\pgfqpoint{0.944318in}{0.514068in}}%
\pgfpathlineto{\pgfqpoint{0.944318in}{0.499444in}}%
\pgfpathclose%
\pgfusepath{fill}%
\end{pgfscope}%
\begin{pgfscope}%
\pgfpathrectangle{\pgfqpoint{0.445556in}{0.499444in}}{\pgfqpoint{3.875000in}{1.155000in}}%
\pgfusepath{clip}%
\pgfsetbuttcap%
\pgfsetmiterjoin%
\definecolor{currentfill}{rgb}{0.000000,0.000000,0.000000}%
\pgfsetfillcolor{currentfill}%
\pgfsetlinewidth{0.000000pt}%
\definecolor{currentstroke}{rgb}{0.000000,0.000000,0.000000}%
\pgfsetstrokecolor{currentstroke}%
\pgfsetstrokeopacity{0.000000}%
\pgfsetdash{}{0pt}%
\pgfpathmoveto{\pgfqpoint{1.097783in}{0.499444in}}%
\pgfpathlineto{\pgfqpoint{1.159170in}{0.499444in}}%
\pgfpathlineto{\pgfqpoint{1.159170in}{0.524977in}}%
\pgfpathlineto{\pgfqpoint{1.097783in}{0.524977in}}%
\pgfpathlineto{\pgfqpoint{1.097783in}{0.499444in}}%
\pgfpathclose%
\pgfusepath{fill}%
\end{pgfscope}%
\begin{pgfscope}%
\pgfpathrectangle{\pgfqpoint{0.445556in}{0.499444in}}{\pgfqpoint{3.875000in}{1.155000in}}%
\pgfusepath{clip}%
\pgfsetbuttcap%
\pgfsetmiterjoin%
\definecolor{currentfill}{rgb}{0.000000,0.000000,0.000000}%
\pgfsetfillcolor{currentfill}%
\pgfsetlinewidth{0.000000pt}%
\definecolor{currentstroke}{rgb}{0.000000,0.000000,0.000000}%
\pgfsetstrokecolor{currentstroke}%
\pgfsetstrokeopacity{0.000000}%
\pgfsetdash{}{0pt}%
\pgfpathmoveto{\pgfqpoint{1.251249in}{0.499444in}}%
\pgfpathlineto{\pgfqpoint{1.312635in}{0.499444in}}%
\pgfpathlineto{\pgfqpoint{1.312635in}{0.535267in}}%
\pgfpathlineto{\pgfqpoint{1.251249in}{0.535267in}}%
\pgfpathlineto{\pgfqpoint{1.251249in}{0.499444in}}%
\pgfpathclose%
\pgfusepath{fill}%
\end{pgfscope}%
\begin{pgfscope}%
\pgfpathrectangle{\pgfqpoint{0.445556in}{0.499444in}}{\pgfqpoint{3.875000in}{1.155000in}}%
\pgfusepath{clip}%
\pgfsetbuttcap%
\pgfsetmiterjoin%
\definecolor{currentfill}{rgb}{0.000000,0.000000,0.000000}%
\pgfsetfillcolor{currentfill}%
\pgfsetlinewidth{0.000000pt}%
\definecolor{currentstroke}{rgb}{0.000000,0.000000,0.000000}%
\pgfsetstrokecolor{currentstroke}%
\pgfsetstrokeopacity{0.000000}%
\pgfsetdash{}{0pt}%
\pgfpathmoveto{\pgfqpoint{1.404714in}{0.499444in}}%
\pgfpathlineto{\pgfqpoint{1.466100in}{0.499444in}}%
\pgfpathlineto{\pgfqpoint{1.466100in}{0.549659in}}%
\pgfpathlineto{\pgfqpoint{1.404714in}{0.549659in}}%
\pgfpathlineto{\pgfqpoint{1.404714in}{0.499444in}}%
\pgfpathclose%
\pgfusepath{fill}%
\end{pgfscope}%
\begin{pgfscope}%
\pgfpathrectangle{\pgfqpoint{0.445556in}{0.499444in}}{\pgfqpoint{3.875000in}{1.155000in}}%
\pgfusepath{clip}%
\pgfsetbuttcap%
\pgfsetmiterjoin%
\definecolor{currentfill}{rgb}{0.000000,0.000000,0.000000}%
\pgfsetfillcolor{currentfill}%
\pgfsetlinewidth{0.000000pt}%
\definecolor{currentstroke}{rgb}{0.000000,0.000000,0.000000}%
\pgfsetstrokecolor{currentstroke}%
\pgfsetstrokeopacity{0.000000}%
\pgfsetdash{}{0pt}%
\pgfpathmoveto{\pgfqpoint{1.558179in}{0.499444in}}%
\pgfpathlineto{\pgfqpoint{1.619566in}{0.499444in}}%
\pgfpathlineto{\pgfqpoint{1.619566in}{0.568228in}}%
\pgfpathlineto{\pgfqpoint{1.558179in}{0.568228in}}%
\pgfpathlineto{\pgfqpoint{1.558179in}{0.499444in}}%
\pgfpathclose%
\pgfusepath{fill}%
\end{pgfscope}%
\begin{pgfscope}%
\pgfpathrectangle{\pgfqpoint{0.445556in}{0.499444in}}{\pgfqpoint{3.875000in}{1.155000in}}%
\pgfusepath{clip}%
\pgfsetbuttcap%
\pgfsetmiterjoin%
\definecolor{currentfill}{rgb}{0.000000,0.000000,0.000000}%
\pgfsetfillcolor{currentfill}%
\pgfsetlinewidth{0.000000pt}%
\definecolor{currentstroke}{rgb}{0.000000,0.000000,0.000000}%
\pgfsetstrokecolor{currentstroke}%
\pgfsetstrokeopacity{0.000000}%
\pgfsetdash{}{0pt}%
\pgfpathmoveto{\pgfqpoint{1.711645in}{0.499444in}}%
\pgfpathlineto{\pgfqpoint{1.773031in}{0.499444in}}%
\pgfpathlineto{\pgfqpoint{1.773031in}{0.590743in}}%
\pgfpathlineto{\pgfqpoint{1.711645in}{0.590743in}}%
\pgfpathlineto{\pgfqpoint{1.711645in}{0.499444in}}%
\pgfpathclose%
\pgfusepath{fill}%
\end{pgfscope}%
\begin{pgfscope}%
\pgfpathrectangle{\pgfqpoint{0.445556in}{0.499444in}}{\pgfqpoint{3.875000in}{1.155000in}}%
\pgfusepath{clip}%
\pgfsetbuttcap%
\pgfsetmiterjoin%
\definecolor{currentfill}{rgb}{0.000000,0.000000,0.000000}%
\pgfsetfillcolor{currentfill}%
\pgfsetlinewidth{0.000000pt}%
\definecolor{currentstroke}{rgb}{0.000000,0.000000,0.000000}%
\pgfsetstrokecolor{currentstroke}%
\pgfsetstrokeopacity{0.000000}%
\pgfsetdash{}{0pt}%
\pgfpathmoveto{\pgfqpoint{1.865110in}{0.499444in}}%
\pgfpathlineto{\pgfqpoint{1.926496in}{0.499444in}}%
\pgfpathlineto{\pgfqpoint{1.926496in}{0.605521in}}%
\pgfpathlineto{\pgfqpoint{1.865110in}{0.605521in}}%
\pgfpathlineto{\pgfqpoint{1.865110in}{0.499444in}}%
\pgfpathclose%
\pgfusepath{fill}%
\end{pgfscope}%
\begin{pgfscope}%
\pgfpathrectangle{\pgfqpoint{0.445556in}{0.499444in}}{\pgfqpoint{3.875000in}{1.155000in}}%
\pgfusepath{clip}%
\pgfsetbuttcap%
\pgfsetmiterjoin%
\definecolor{currentfill}{rgb}{0.000000,0.000000,0.000000}%
\pgfsetfillcolor{currentfill}%
\pgfsetlinewidth{0.000000pt}%
\definecolor{currentstroke}{rgb}{0.000000,0.000000,0.000000}%
\pgfsetstrokecolor{currentstroke}%
\pgfsetstrokeopacity{0.000000}%
\pgfsetdash{}{0pt}%
\pgfpathmoveto{\pgfqpoint{2.018575in}{0.499444in}}%
\pgfpathlineto{\pgfqpoint{2.079962in}{0.499444in}}%
\pgfpathlineto{\pgfqpoint{2.079962in}{0.624632in}}%
\pgfpathlineto{\pgfqpoint{2.018575in}{0.624632in}}%
\pgfpathlineto{\pgfqpoint{2.018575in}{0.499444in}}%
\pgfpathclose%
\pgfusepath{fill}%
\end{pgfscope}%
\begin{pgfscope}%
\pgfpathrectangle{\pgfqpoint{0.445556in}{0.499444in}}{\pgfqpoint{3.875000in}{1.155000in}}%
\pgfusepath{clip}%
\pgfsetbuttcap%
\pgfsetmiterjoin%
\definecolor{currentfill}{rgb}{0.000000,0.000000,0.000000}%
\pgfsetfillcolor{currentfill}%
\pgfsetlinewidth{0.000000pt}%
\definecolor{currentstroke}{rgb}{0.000000,0.000000,0.000000}%
\pgfsetstrokecolor{currentstroke}%
\pgfsetstrokeopacity{0.000000}%
\pgfsetdash{}{0pt}%
\pgfpathmoveto{\pgfqpoint{2.172041in}{0.499444in}}%
\pgfpathlineto{\pgfqpoint{2.233427in}{0.499444in}}%
\pgfpathlineto{\pgfqpoint{2.233427in}{0.643356in}}%
\pgfpathlineto{\pgfqpoint{2.172041in}{0.643356in}}%
\pgfpathlineto{\pgfqpoint{2.172041in}{0.499444in}}%
\pgfpathclose%
\pgfusepath{fill}%
\end{pgfscope}%
\begin{pgfscope}%
\pgfpathrectangle{\pgfqpoint{0.445556in}{0.499444in}}{\pgfqpoint{3.875000in}{1.155000in}}%
\pgfusepath{clip}%
\pgfsetbuttcap%
\pgfsetmiterjoin%
\definecolor{currentfill}{rgb}{0.000000,0.000000,0.000000}%
\pgfsetfillcolor{currentfill}%
\pgfsetlinewidth{0.000000pt}%
\definecolor{currentstroke}{rgb}{0.000000,0.000000,0.000000}%
\pgfsetstrokecolor{currentstroke}%
\pgfsetstrokeopacity{0.000000}%
\pgfsetdash{}{0pt}%
\pgfpathmoveto{\pgfqpoint{2.325506in}{0.499444in}}%
\pgfpathlineto{\pgfqpoint{2.386892in}{0.499444in}}%
\pgfpathlineto{\pgfqpoint{2.386892in}{0.662622in}}%
\pgfpathlineto{\pgfqpoint{2.325506in}{0.662622in}}%
\pgfpathlineto{\pgfqpoint{2.325506in}{0.499444in}}%
\pgfpathclose%
\pgfusepath{fill}%
\end{pgfscope}%
\begin{pgfscope}%
\pgfpathrectangle{\pgfqpoint{0.445556in}{0.499444in}}{\pgfqpoint{3.875000in}{1.155000in}}%
\pgfusepath{clip}%
\pgfsetbuttcap%
\pgfsetmiterjoin%
\definecolor{currentfill}{rgb}{0.000000,0.000000,0.000000}%
\pgfsetfillcolor{currentfill}%
\pgfsetlinewidth{0.000000pt}%
\definecolor{currentstroke}{rgb}{0.000000,0.000000,0.000000}%
\pgfsetstrokecolor{currentstroke}%
\pgfsetstrokeopacity{0.000000}%
\pgfsetdash{}{0pt}%
\pgfpathmoveto{\pgfqpoint{2.478972in}{0.499444in}}%
\pgfpathlineto{\pgfqpoint{2.540358in}{0.499444in}}%
\pgfpathlineto{\pgfqpoint{2.540358in}{0.672216in}}%
\pgfpathlineto{\pgfqpoint{2.478972in}{0.672216in}}%
\pgfpathlineto{\pgfqpoint{2.478972in}{0.499444in}}%
\pgfpathclose%
\pgfusepath{fill}%
\end{pgfscope}%
\begin{pgfscope}%
\pgfpathrectangle{\pgfqpoint{0.445556in}{0.499444in}}{\pgfqpoint{3.875000in}{1.155000in}}%
\pgfusepath{clip}%
\pgfsetbuttcap%
\pgfsetmiterjoin%
\definecolor{currentfill}{rgb}{0.000000,0.000000,0.000000}%
\pgfsetfillcolor{currentfill}%
\pgfsetlinewidth{0.000000pt}%
\definecolor{currentstroke}{rgb}{0.000000,0.000000,0.000000}%
\pgfsetstrokecolor{currentstroke}%
\pgfsetstrokeopacity{0.000000}%
\pgfsetdash{}{0pt}%
\pgfpathmoveto{\pgfqpoint{2.632437in}{0.499444in}}%
\pgfpathlineto{\pgfqpoint{2.693823in}{0.499444in}}%
\pgfpathlineto{\pgfqpoint{2.693823in}{0.684286in}}%
\pgfpathlineto{\pgfqpoint{2.632437in}{0.684286in}}%
\pgfpathlineto{\pgfqpoint{2.632437in}{0.499444in}}%
\pgfpathclose%
\pgfusepath{fill}%
\end{pgfscope}%
\begin{pgfscope}%
\pgfpathrectangle{\pgfqpoint{0.445556in}{0.499444in}}{\pgfqpoint{3.875000in}{1.155000in}}%
\pgfusepath{clip}%
\pgfsetbuttcap%
\pgfsetmiterjoin%
\definecolor{currentfill}{rgb}{0.000000,0.000000,0.000000}%
\pgfsetfillcolor{currentfill}%
\pgfsetlinewidth{0.000000pt}%
\definecolor{currentstroke}{rgb}{0.000000,0.000000,0.000000}%
\pgfsetstrokecolor{currentstroke}%
\pgfsetstrokeopacity{0.000000}%
\pgfsetdash{}{0pt}%
\pgfpathmoveto{\pgfqpoint{2.785902in}{0.499444in}}%
\pgfpathlineto{\pgfqpoint{2.847288in}{0.499444in}}%
\pgfpathlineto{\pgfqpoint{2.847288in}{0.686607in}}%
\pgfpathlineto{\pgfqpoint{2.785902in}{0.686607in}}%
\pgfpathlineto{\pgfqpoint{2.785902in}{0.499444in}}%
\pgfpathclose%
\pgfusepath{fill}%
\end{pgfscope}%
\begin{pgfscope}%
\pgfpathrectangle{\pgfqpoint{0.445556in}{0.499444in}}{\pgfqpoint{3.875000in}{1.155000in}}%
\pgfusepath{clip}%
\pgfsetbuttcap%
\pgfsetmiterjoin%
\definecolor{currentfill}{rgb}{0.000000,0.000000,0.000000}%
\pgfsetfillcolor{currentfill}%
\pgfsetlinewidth{0.000000pt}%
\definecolor{currentstroke}{rgb}{0.000000,0.000000,0.000000}%
\pgfsetstrokecolor{currentstroke}%
\pgfsetstrokeopacity{0.000000}%
\pgfsetdash{}{0pt}%
\pgfpathmoveto{\pgfqpoint{2.939368in}{0.499444in}}%
\pgfpathlineto{\pgfqpoint{3.000754in}{0.499444in}}%
\pgfpathlineto{\pgfqpoint{3.000754in}{0.682816in}}%
\pgfpathlineto{\pgfqpoint{2.939368in}{0.682816in}}%
\pgfpathlineto{\pgfqpoint{2.939368in}{0.499444in}}%
\pgfpathclose%
\pgfusepath{fill}%
\end{pgfscope}%
\begin{pgfscope}%
\pgfpathrectangle{\pgfqpoint{0.445556in}{0.499444in}}{\pgfqpoint{3.875000in}{1.155000in}}%
\pgfusepath{clip}%
\pgfsetbuttcap%
\pgfsetmiterjoin%
\definecolor{currentfill}{rgb}{0.000000,0.000000,0.000000}%
\pgfsetfillcolor{currentfill}%
\pgfsetlinewidth{0.000000pt}%
\definecolor{currentstroke}{rgb}{0.000000,0.000000,0.000000}%
\pgfsetstrokecolor{currentstroke}%
\pgfsetstrokeopacity{0.000000}%
\pgfsetdash{}{0pt}%
\pgfpathmoveto{\pgfqpoint{3.092833in}{0.499444in}}%
\pgfpathlineto{\pgfqpoint{3.154219in}{0.499444in}}%
\pgfpathlineto{\pgfqpoint{3.154219in}{0.681037in}}%
\pgfpathlineto{\pgfqpoint{3.092833in}{0.681037in}}%
\pgfpathlineto{\pgfqpoint{3.092833in}{0.499444in}}%
\pgfpathclose%
\pgfusepath{fill}%
\end{pgfscope}%
\begin{pgfscope}%
\pgfpathrectangle{\pgfqpoint{0.445556in}{0.499444in}}{\pgfqpoint{3.875000in}{1.155000in}}%
\pgfusepath{clip}%
\pgfsetbuttcap%
\pgfsetmiterjoin%
\definecolor{currentfill}{rgb}{0.000000,0.000000,0.000000}%
\pgfsetfillcolor{currentfill}%
\pgfsetlinewidth{0.000000pt}%
\definecolor{currentstroke}{rgb}{0.000000,0.000000,0.000000}%
\pgfsetstrokecolor{currentstroke}%
\pgfsetstrokeopacity{0.000000}%
\pgfsetdash{}{0pt}%
\pgfpathmoveto{\pgfqpoint{3.246298in}{0.499444in}}%
\pgfpathlineto{\pgfqpoint{3.307684in}{0.499444in}}%
\pgfpathlineto{\pgfqpoint{3.307684in}{0.677787in}}%
\pgfpathlineto{\pgfqpoint{3.246298in}{0.677787in}}%
\pgfpathlineto{\pgfqpoint{3.246298in}{0.499444in}}%
\pgfpathclose%
\pgfusepath{fill}%
\end{pgfscope}%
\begin{pgfscope}%
\pgfpathrectangle{\pgfqpoint{0.445556in}{0.499444in}}{\pgfqpoint{3.875000in}{1.155000in}}%
\pgfusepath{clip}%
\pgfsetbuttcap%
\pgfsetmiterjoin%
\definecolor{currentfill}{rgb}{0.000000,0.000000,0.000000}%
\pgfsetfillcolor{currentfill}%
\pgfsetlinewidth{0.000000pt}%
\definecolor{currentstroke}{rgb}{0.000000,0.000000,0.000000}%
\pgfsetstrokecolor{currentstroke}%
\pgfsetstrokeopacity{0.000000}%
\pgfsetdash{}{0pt}%
\pgfpathmoveto{\pgfqpoint{3.399764in}{0.499444in}}%
\pgfpathlineto{\pgfqpoint{3.461150in}{0.499444in}}%
\pgfpathlineto{\pgfqpoint{3.461150in}{0.659527in}}%
\pgfpathlineto{\pgfqpoint{3.399764in}{0.659527in}}%
\pgfpathlineto{\pgfqpoint{3.399764in}{0.499444in}}%
\pgfpathclose%
\pgfusepath{fill}%
\end{pgfscope}%
\begin{pgfscope}%
\pgfpathrectangle{\pgfqpoint{0.445556in}{0.499444in}}{\pgfqpoint{3.875000in}{1.155000in}}%
\pgfusepath{clip}%
\pgfsetbuttcap%
\pgfsetmiterjoin%
\definecolor{currentfill}{rgb}{0.000000,0.000000,0.000000}%
\pgfsetfillcolor{currentfill}%
\pgfsetlinewidth{0.000000pt}%
\definecolor{currentstroke}{rgb}{0.000000,0.000000,0.000000}%
\pgfsetstrokecolor{currentstroke}%
\pgfsetstrokeopacity{0.000000}%
\pgfsetdash{}{0pt}%
\pgfpathmoveto{\pgfqpoint{3.553229in}{0.499444in}}%
\pgfpathlineto{\pgfqpoint{3.614615in}{0.499444in}}%
\pgfpathlineto{\pgfqpoint{3.614615in}{0.657206in}}%
\pgfpathlineto{\pgfqpoint{3.553229in}{0.657206in}}%
\pgfpathlineto{\pgfqpoint{3.553229in}{0.499444in}}%
\pgfpathclose%
\pgfusepath{fill}%
\end{pgfscope}%
\begin{pgfscope}%
\pgfpathrectangle{\pgfqpoint{0.445556in}{0.499444in}}{\pgfqpoint{3.875000in}{1.155000in}}%
\pgfusepath{clip}%
\pgfsetbuttcap%
\pgfsetmiterjoin%
\definecolor{currentfill}{rgb}{0.000000,0.000000,0.000000}%
\pgfsetfillcolor{currentfill}%
\pgfsetlinewidth{0.000000pt}%
\definecolor{currentstroke}{rgb}{0.000000,0.000000,0.000000}%
\pgfsetstrokecolor{currentstroke}%
\pgfsetstrokeopacity{0.000000}%
\pgfsetdash{}{0pt}%
\pgfpathmoveto{\pgfqpoint{3.706694in}{0.499444in}}%
\pgfpathlineto{\pgfqpoint{3.768080in}{0.499444in}}%
\pgfpathlineto{\pgfqpoint{3.768080in}{0.639333in}}%
\pgfpathlineto{\pgfqpoint{3.706694in}{0.639333in}}%
\pgfpathlineto{\pgfqpoint{3.706694in}{0.499444in}}%
\pgfpathclose%
\pgfusepath{fill}%
\end{pgfscope}%
\begin{pgfscope}%
\pgfpathrectangle{\pgfqpoint{0.445556in}{0.499444in}}{\pgfqpoint{3.875000in}{1.155000in}}%
\pgfusepath{clip}%
\pgfsetbuttcap%
\pgfsetmiterjoin%
\definecolor{currentfill}{rgb}{0.000000,0.000000,0.000000}%
\pgfsetfillcolor{currentfill}%
\pgfsetlinewidth{0.000000pt}%
\definecolor{currentstroke}{rgb}{0.000000,0.000000,0.000000}%
\pgfsetstrokecolor{currentstroke}%
\pgfsetstrokeopacity{0.000000}%
\pgfsetdash{}{0pt}%
\pgfpathmoveto{\pgfqpoint{3.860160in}{0.499444in}}%
\pgfpathlineto{\pgfqpoint{3.921546in}{0.499444in}}%
\pgfpathlineto{\pgfqpoint{3.921546in}{0.623626in}}%
\pgfpathlineto{\pgfqpoint{3.860160in}{0.623626in}}%
\pgfpathlineto{\pgfqpoint{3.860160in}{0.499444in}}%
\pgfpathclose%
\pgfusepath{fill}%
\end{pgfscope}%
\begin{pgfscope}%
\pgfpathrectangle{\pgfqpoint{0.445556in}{0.499444in}}{\pgfqpoint{3.875000in}{1.155000in}}%
\pgfusepath{clip}%
\pgfsetbuttcap%
\pgfsetmiterjoin%
\definecolor{currentfill}{rgb}{0.000000,0.000000,0.000000}%
\pgfsetfillcolor{currentfill}%
\pgfsetlinewidth{0.000000pt}%
\definecolor{currentstroke}{rgb}{0.000000,0.000000,0.000000}%
\pgfsetstrokecolor{currentstroke}%
\pgfsetstrokeopacity{0.000000}%
\pgfsetdash{}{0pt}%
\pgfpathmoveto{\pgfqpoint{4.013625in}{0.499444in}}%
\pgfpathlineto{\pgfqpoint{4.075011in}{0.499444in}}%
\pgfpathlineto{\pgfqpoint{4.075011in}{0.580840in}}%
\pgfpathlineto{\pgfqpoint{4.013625in}{0.580840in}}%
\pgfpathlineto{\pgfqpoint{4.013625in}{0.499444in}}%
\pgfpathclose%
\pgfusepath{fill}%
\end{pgfscope}%
\begin{pgfscope}%
\pgfpathrectangle{\pgfqpoint{0.445556in}{0.499444in}}{\pgfqpoint{3.875000in}{1.155000in}}%
\pgfusepath{clip}%
\pgfsetbuttcap%
\pgfsetmiterjoin%
\definecolor{currentfill}{rgb}{0.000000,0.000000,0.000000}%
\pgfsetfillcolor{currentfill}%
\pgfsetlinewidth{0.000000pt}%
\definecolor{currentstroke}{rgb}{0.000000,0.000000,0.000000}%
\pgfsetstrokecolor{currentstroke}%
\pgfsetstrokeopacity{0.000000}%
\pgfsetdash{}{0pt}%
\pgfpathmoveto{\pgfqpoint{4.167090in}{0.499444in}}%
\pgfpathlineto{\pgfqpoint{4.228476in}{0.499444in}}%
\pgfpathlineto{\pgfqpoint{4.228476in}{0.529774in}}%
\pgfpathlineto{\pgfqpoint{4.167090in}{0.529774in}}%
\pgfpathlineto{\pgfqpoint{4.167090in}{0.499444in}}%
\pgfpathclose%
\pgfusepath{fill}%
\end{pgfscope}%
\begin{pgfscope}%
\pgfsetbuttcap%
\pgfsetroundjoin%
\definecolor{currentfill}{rgb}{0.000000,0.000000,0.000000}%
\pgfsetfillcolor{currentfill}%
\pgfsetlinewidth{0.803000pt}%
\definecolor{currentstroke}{rgb}{0.000000,0.000000,0.000000}%
\pgfsetstrokecolor{currentstroke}%
\pgfsetdash{}{0pt}%
\pgfsys@defobject{currentmarker}{\pgfqpoint{0.000000in}{-0.048611in}}{\pgfqpoint{0.000000in}{0.000000in}}{%
\pgfpathmoveto{\pgfqpoint{0.000000in}{0.000000in}}%
\pgfpathlineto{\pgfqpoint{0.000000in}{-0.048611in}}%
\pgfusepath{stroke,fill}%
}%
\begin{pgfscope}%
\pgfsys@transformshift{0.483922in}{0.499444in}%
\pgfsys@useobject{currentmarker}{}%
\end{pgfscope}%
\end{pgfscope}%
\begin{pgfscope}%
\definecolor{textcolor}{rgb}{0.000000,0.000000,0.000000}%
\pgfsetstrokecolor{textcolor}%
\pgfsetfillcolor{textcolor}%
\pgftext[x=0.483922in,y=0.402222in,,top]{\color{textcolor}\rmfamily\fontsize{10.000000}{12.000000}\selectfont 0.0}%
\end{pgfscope}%
\begin{pgfscope}%
\pgfsetbuttcap%
\pgfsetroundjoin%
\definecolor{currentfill}{rgb}{0.000000,0.000000,0.000000}%
\pgfsetfillcolor{currentfill}%
\pgfsetlinewidth{0.803000pt}%
\definecolor{currentstroke}{rgb}{0.000000,0.000000,0.000000}%
\pgfsetstrokecolor{currentstroke}%
\pgfsetdash{}{0pt}%
\pgfsys@defobject{currentmarker}{\pgfqpoint{0.000000in}{-0.048611in}}{\pgfqpoint{0.000000in}{0.000000in}}{%
\pgfpathmoveto{\pgfqpoint{0.000000in}{0.000000in}}%
\pgfpathlineto{\pgfqpoint{0.000000in}{-0.048611in}}%
\pgfusepath{stroke,fill}%
}%
\begin{pgfscope}%
\pgfsys@transformshift{0.867585in}{0.499444in}%
\pgfsys@useobject{currentmarker}{}%
\end{pgfscope}%
\end{pgfscope}%
\begin{pgfscope}%
\definecolor{textcolor}{rgb}{0.000000,0.000000,0.000000}%
\pgfsetstrokecolor{textcolor}%
\pgfsetfillcolor{textcolor}%
\pgftext[x=0.867585in,y=0.402222in,,top]{\color{textcolor}\rmfamily\fontsize{10.000000}{12.000000}\selectfont 0.1}%
\end{pgfscope}%
\begin{pgfscope}%
\pgfsetbuttcap%
\pgfsetroundjoin%
\definecolor{currentfill}{rgb}{0.000000,0.000000,0.000000}%
\pgfsetfillcolor{currentfill}%
\pgfsetlinewidth{0.803000pt}%
\definecolor{currentstroke}{rgb}{0.000000,0.000000,0.000000}%
\pgfsetstrokecolor{currentstroke}%
\pgfsetdash{}{0pt}%
\pgfsys@defobject{currentmarker}{\pgfqpoint{0.000000in}{-0.048611in}}{\pgfqpoint{0.000000in}{0.000000in}}{%
\pgfpathmoveto{\pgfqpoint{0.000000in}{0.000000in}}%
\pgfpathlineto{\pgfqpoint{0.000000in}{-0.048611in}}%
\pgfusepath{stroke,fill}%
}%
\begin{pgfscope}%
\pgfsys@transformshift{1.251249in}{0.499444in}%
\pgfsys@useobject{currentmarker}{}%
\end{pgfscope}%
\end{pgfscope}%
\begin{pgfscope}%
\definecolor{textcolor}{rgb}{0.000000,0.000000,0.000000}%
\pgfsetstrokecolor{textcolor}%
\pgfsetfillcolor{textcolor}%
\pgftext[x=1.251249in,y=0.402222in,,top]{\color{textcolor}\rmfamily\fontsize{10.000000}{12.000000}\selectfont 0.2}%
\end{pgfscope}%
\begin{pgfscope}%
\pgfsetbuttcap%
\pgfsetroundjoin%
\definecolor{currentfill}{rgb}{0.000000,0.000000,0.000000}%
\pgfsetfillcolor{currentfill}%
\pgfsetlinewidth{0.803000pt}%
\definecolor{currentstroke}{rgb}{0.000000,0.000000,0.000000}%
\pgfsetstrokecolor{currentstroke}%
\pgfsetdash{}{0pt}%
\pgfsys@defobject{currentmarker}{\pgfqpoint{0.000000in}{-0.048611in}}{\pgfqpoint{0.000000in}{0.000000in}}{%
\pgfpathmoveto{\pgfqpoint{0.000000in}{0.000000in}}%
\pgfpathlineto{\pgfqpoint{0.000000in}{-0.048611in}}%
\pgfusepath{stroke,fill}%
}%
\begin{pgfscope}%
\pgfsys@transformshift{1.634912in}{0.499444in}%
\pgfsys@useobject{currentmarker}{}%
\end{pgfscope}%
\end{pgfscope}%
\begin{pgfscope}%
\definecolor{textcolor}{rgb}{0.000000,0.000000,0.000000}%
\pgfsetstrokecolor{textcolor}%
\pgfsetfillcolor{textcolor}%
\pgftext[x=1.634912in,y=0.402222in,,top]{\color{textcolor}\rmfamily\fontsize{10.000000}{12.000000}\selectfont 0.3}%
\end{pgfscope}%
\begin{pgfscope}%
\pgfsetbuttcap%
\pgfsetroundjoin%
\definecolor{currentfill}{rgb}{0.000000,0.000000,0.000000}%
\pgfsetfillcolor{currentfill}%
\pgfsetlinewidth{0.803000pt}%
\definecolor{currentstroke}{rgb}{0.000000,0.000000,0.000000}%
\pgfsetstrokecolor{currentstroke}%
\pgfsetdash{}{0pt}%
\pgfsys@defobject{currentmarker}{\pgfqpoint{0.000000in}{-0.048611in}}{\pgfqpoint{0.000000in}{0.000000in}}{%
\pgfpathmoveto{\pgfqpoint{0.000000in}{0.000000in}}%
\pgfpathlineto{\pgfqpoint{0.000000in}{-0.048611in}}%
\pgfusepath{stroke,fill}%
}%
\begin{pgfscope}%
\pgfsys@transformshift{2.018575in}{0.499444in}%
\pgfsys@useobject{currentmarker}{}%
\end{pgfscope}%
\end{pgfscope}%
\begin{pgfscope}%
\definecolor{textcolor}{rgb}{0.000000,0.000000,0.000000}%
\pgfsetstrokecolor{textcolor}%
\pgfsetfillcolor{textcolor}%
\pgftext[x=2.018575in,y=0.402222in,,top]{\color{textcolor}\rmfamily\fontsize{10.000000}{12.000000}\selectfont 0.4}%
\end{pgfscope}%
\begin{pgfscope}%
\pgfsetbuttcap%
\pgfsetroundjoin%
\definecolor{currentfill}{rgb}{0.000000,0.000000,0.000000}%
\pgfsetfillcolor{currentfill}%
\pgfsetlinewidth{0.803000pt}%
\definecolor{currentstroke}{rgb}{0.000000,0.000000,0.000000}%
\pgfsetstrokecolor{currentstroke}%
\pgfsetdash{}{0pt}%
\pgfsys@defobject{currentmarker}{\pgfqpoint{0.000000in}{-0.048611in}}{\pgfqpoint{0.000000in}{0.000000in}}{%
\pgfpathmoveto{\pgfqpoint{0.000000in}{0.000000in}}%
\pgfpathlineto{\pgfqpoint{0.000000in}{-0.048611in}}%
\pgfusepath{stroke,fill}%
}%
\begin{pgfscope}%
\pgfsys@transformshift{2.402239in}{0.499444in}%
\pgfsys@useobject{currentmarker}{}%
\end{pgfscope}%
\end{pgfscope}%
\begin{pgfscope}%
\definecolor{textcolor}{rgb}{0.000000,0.000000,0.000000}%
\pgfsetstrokecolor{textcolor}%
\pgfsetfillcolor{textcolor}%
\pgftext[x=2.402239in,y=0.402222in,,top]{\color{textcolor}\rmfamily\fontsize{10.000000}{12.000000}\selectfont 0.5}%
\end{pgfscope}%
\begin{pgfscope}%
\pgfsetbuttcap%
\pgfsetroundjoin%
\definecolor{currentfill}{rgb}{0.000000,0.000000,0.000000}%
\pgfsetfillcolor{currentfill}%
\pgfsetlinewidth{0.803000pt}%
\definecolor{currentstroke}{rgb}{0.000000,0.000000,0.000000}%
\pgfsetstrokecolor{currentstroke}%
\pgfsetdash{}{0pt}%
\pgfsys@defobject{currentmarker}{\pgfqpoint{0.000000in}{-0.048611in}}{\pgfqpoint{0.000000in}{0.000000in}}{%
\pgfpathmoveto{\pgfqpoint{0.000000in}{0.000000in}}%
\pgfpathlineto{\pgfqpoint{0.000000in}{-0.048611in}}%
\pgfusepath{stroke,fill}%
}%
\begin{pgfscope}%
\pgfsys@transformshift{2.785902in}{0.499444in}%
\pgfsys@useobject{currentmarker}{}%
\end{pgfscope}%
\end{pgfscope}%
\begin{pgfscope}%
\definecolor{textcolor}{rgb}{0.000000,0.000000,0.000000}%
\pgfsetstrokecolor{textcolor}%
\pgfsetfillcolor{textcolor}%
\pgftext[x=2.785902in,y=0.402222in,,top]{\color{textcolor}\rmfamily\fontsize{10.000000}{12.000000}\selectfont 0.6}%
\end{pgfscope}%
\begin{pgfscope}%
\pgfsetbuttcap%
\pgfsetroundjoin%
\definecolor{currentfill}{rgb}{0.000000,0.000000,0.000000}%
\pgfsetfillcolor{currentfill}%
\pgfsetlinewidth{0.803000pt}%
\definecolor{currentstroke}{rgb}{0.000000,0.000000,0.000000}%
\pgfsetstrokecolor{currentstroke}%
\pgfsetdash{}{0pt}%
\pgfsys@defobject{currentmarker}{\pgfqpoint{0.000000in}{-0.048611in}}{\pgfqpoint{0.000000in}{0.000000in}}{%
\pgfpathmoveto{\pgfqpoint{0.000000in}{0.000000in}}%
\pgfpathlineto{\pgfqpoint{0.000000in}{-0.048611in}}%
\pgfusepath{stroke,fill}%
}%
\begin{pgfscope}%
\pgfsys@transformshift{3.169566in}{0.499444in}%
\pgfsys@useobject{currentmarker}{}%
\end{pgfscope}%
\end{pgfscope}%
\begin{pgfscope}%
\definecolor{textcolor}{rgb}{0.000000,0.000000,0.000000}%
\pgfsetstrokecolor{textcolor}%
\pgfsetfillcolor{textcolor}%
\pgftext[x=3.169566in,y=0.402222in,,top]{\color{textcolor}\rmfamily\fontsize{10.000000}{12.000000}\selectfont 0.7}%
\end{pgfscope}%
\begin{pgfscope}%
\pgfsetbuttcap%
\pgfsetroundjoin%
\definecolor{currentfill}{rgb}{0.000000,0.000000,0.000000}%
\pgfsetfillcolor{currentfill}%
\pgfsetlinewidth{0.803000pt}%
\definecolor{currentstroke}{rgb}{0.000000,0.000000,0.000000}%
\pgfsetstrokecolor{currentstroke}%
\pgfsetdash{}{0pt}%
\pgfsys@defobject{currentmarker}{\pgfqpoint{0.000000in}{-0.048611in}}{\pgfqpoint{0.000000in}{0.000000in}}{%
\pgfpathmoveto{\pgfqpoint{0.000000in}{0.000000in}}%
\pgfpathlineto{\pgfqpoint{0.000000in}{-0.048611in}}%
\pgfusepath{stroke,fill}%
}%
\begin{pgfscope}%
\pgfsys@transformshift{3.553229in}{0.499444in}%
\pgfsys@useobject{currentmarker}{}%
\end{pgfscope}%
\end{pgfscope}%
\begin{pgfscope}%
\definecolor{textcolor}{rgb}{0.000000,0.000000,0.000000}%
\pgfsetstrokecolor{textcolor}%
\pgfsetfillcolor{textcolor}%
\pgftext[x=3.553229in,y=0.402222in,,top]{\color{textcolor}\rmfamily\fontsize{10.000000}{12.000000}\selectfont 0.8}%
\end{pgfscope}%
\begin{pgfscope}%
\pgfsetbuttcap%
\pgfsetroundjoin%
\definecolor{currentfill}{rgb}{0.000000,0.000000,0.000000}%
\pgfsetfillcolor{currentfill}%
\pgfsetlinewidth{0.803000pt}%
\definecolor{currentstroke}{rgb}{0.000000,0.000000,0.000000}%
\pgfsetstrokecolor{currentstroke}%
\pgfsetdash{}{0pt}%
\pgfsys@defobject{currentmarker}{\pgfqpoint{0.000000in}{-0.048611in}}{\pgfqpoint{0.000000in}{0.000000in}}{%
\pgfpathmoveto{\pgfqpoint{0.000000in}{0.000000in}}%
\pgfpathlineto{\pgfqpoint{0.000000in}{-0.048611in}}%
\pgfusepath{stroke,fill}%
}%
\begin{pgfscope}%
\pgfsys@transformshift{3.936892in}{0.499444in}%
\pgfsys@useobject{currentmarker}{}%
\end{pgfscope}%
\end{pgfscope}%
\begin{pgfscope}%
\definecolor{textcolor}{rgb}{0.000000,0.000000,0.000000}%
\pgfsetstrokecolor{textcolor}%
\pgfsetfillcolor{textcolor}%
\pgftext[x=3.936892in,y=0.402222in,,top]{\color{textcolor}\rmfamily\fontsize{10.000000}{12.000000}\selectfont 0.9}%
\end{pgfscope}%
\begin{pgfscope}%
\pgfsetbuttcap%
\pgfsetroundjoin%
\definecolor{currentfill}{rgb}{0.000000,0.000000,0.000000}%
\pgfsetfillcolor{currentfill}%
\pgfsetlinewidth{0.803000pt}%
\definecolor{currentstroke}{rgb}{0.000000,0.000000,0.000000}%
\pgfsetstrokecolor{currentstroke}%
\pgfsetdash{}{0pt}%
\pgfsys@defobject{currentmarker}{\pgfqpoint{0.000000in}{-0.048611in}}{\pgfqpoint{0.000000in}{0.000000in}}{%
\pgfpathmoveto{\pgfqpoint{0.000000in}{0.000000in}}%
\pgfpathlineto{\pgfqpoint{0.000000in}{-0.048611in}}%
\pgfusepath{stroke,fill}%
}%
\begin{pgfscope}%
\pgfsys@transformshift{4.320556in}{0.499444in}%
\pgfsys@useobject{currentmarker}{}%
\end{pgfscope}%
\end{pgfscope}%
\begin{pgfscope}%
\definecolor{textcolor}{rgb}{0.000000,0.000000,0.000000}%
\pgfsetstrokecolor{textcolor}%
\pgfsetfillcolor{textcolor}%
\pgftext[x=4.320556in,y=0.402222in,,top]{\color{textcolor}\rmfamily\fontsize{10.000000}{12.000000}\selectfont 1.0}%
\end{pgfscope}%
\begin{pgfscope}%
\definecolor{textcolor}{rgb}{0.000000,0.000000,0.000000}%
\pgfsetstrokecolor{textcolor}%
\pgfsetfillcolor{textcolor}%
\pgftext[x=2.383056in,y=0.223333in,,top]{\color{textcolor}\rmfamily\fontsize{10.000000}{12.000000}\selectfont \(\displaystyle p\)}%
\end{pgfscope}%
\begin{pgfscope}%
\pgfsetbuttcap%
\pgfsetroundjoin%
\definecolor{currentfill}{rgb}{0.000000,0.000000,0.000000}%
\pgfsetfillcolor{currentfill}%
\pgfsetlinewidth{0.803000pt}%
\definecolor{currentstroke}{rgb}{0.000000,0.000000,0.000000}%
\pgfsetstrokecolor{currentstroke}%
\pgfsetdash{}{0pt}%
\pgfsys@defobject{currentmarker}{\pgfqpoint{-0.048611in}{0.000000in}}{\pgfqpoint{-0.000000in}{0.000000in}}{%
\pgfpathmoveto{\pgfqpoint{-0.000000in}{0.000000in}}%
\pgfpathlineto{\pgfqpoint{-0.048611in}{0.000000in}}%
\pgfusepath{stroke,fill}%
}%
\begin{pgfscope}%
\pgfsys@transformshift{0.445556in}{0.499444in}%
\pgfsys@useobject{currentmarker}{}%
\end{pgfscope}%
\end{pgfscope}%
\begin{pgfscope}%
\definecolor{textcolor}{rgb}{0.000000,0.000000,0.000000}%
\pgfsetstrokecolor{textcolor}%
\pgfsetfillcolor{textcolor}%
\pgftext[x=0.278889in, y=0.451250in, left, base]{\color{textcolor}\rmfamily\fontsize{10.000000}{12.000000}\selectfont \(\displaystyle {0}\)}%
\end{pgfscope}%
\begin{pgfscope}%
\pgfsetbuttcap%
\pgfsetroundjoin%
\definecolor{currentfill}{rgb}{0.000000,0.000000,0.000000}%
\pgfsetfillcolor{currentfill}%
\pgfsetlinewidth{0.803000pt}%
\definecolor{currentstroke}{rgb}{0.000000,0.000000,0.000000}%
\pgfsetstrokecolor{currentstroke}%
\pgfsetdash{}{0pt}%
\pgfsys@defobject{currentmarker}{\pgfqpoint{-0.048611in}{0.000000in}}{\pgfqpoint{-0.000000in}{0.000000in}}{%
\pgfpathmoveto{\pgfqpoint{-0.000000in}{0.000000in}}%
\pgfpathlineto{\pgfqpoint{-0.048611in}{0.000000in}}%
\pgfusepath{stroke,fill}%
}%
\begin{pgfscope}%
\pgfsys@transformshift{0.445556in}{0.830705in}%
\pgfsys@useobject{currentmarker}{}%
\end{pgfscope}%
\end{pgfscope}%
\begin{pgfscope}%
\definecolor{textcolor}{rgb}{0.000000,0.000000,0.000000}%
\pgfsetstrokecolor{textcolor}%
\pgfsetfillcolor{textcolor}%
\pgftext[x=0.278889in, y=0.782511in, left, base]{\color{textcolor}\rmfamily\fontsize{10.000000}{12.000000}\selectfont \(\displaystyle {2}\)}%
\end{pgfscope}%
\begin{pgfscope}%
\pgfsetbuttcap%
\pgfsetroundjoin%
\definecolor{currentfill}{rgb}{0.000000,0.000000,0.000000}%
\pgfsetfillcolor{currentfill}%
\pgfsetlinewidth{0.803000pt}%
\definecolor{currentstroke}{rgb}{0.000000,0.000000,0.000000}%
\pgfsetstrokecolor{currentstroke}%
\pgfsetdash{}{0pt}%
\pgfsys@defobject{currentmarker}{\pgfqpoint{-0.048611in}{0.000000in}}{\pgfqpoint{-0.000000in}{0.000000in}}{%
\pgfpathmoveto{\pgfqpoint{-0.000000in}{0.000000in}}%
\pgfpathlineto{\pgfqpoint{-0.048611in}{0.000000in}}%
\pgfusepath{stroke,fill}%
}%
\begin{pgfscope}%
\pgfsys@transformshift{0.445556in}{1.161966in}%
\pgfsys@useobject{currentmarker}{}%
\end{pgfscope}%
\end{pgfscope}%
\begin{pgfscope}%
\definecolor{textcolor}{rgb}{0.000000,0.000000,0.000000}%
\pgfsetstrokecolor{textcolor}%
\pgfsetfillcolor{textcolor}%
\pgftext[x=0.278889in, y=1.113772in, left, base]{\color{textcolor}\rmfamily\fontsize{10.000000}{12.000000}\selectfont \(\displaystyle {4}\)}%
\end{pgfscope}%
\begin{pgfscope}%
\pgfsetbuttcap%
\pgfsetroundjoin%
\definecolor{currentfill}{rgb}{0.000000,0.000000,0.000000}%
\pgfsetfillcolor{currentfill}%
\pgfsetlinewidth{0.803000pt}%
\definecolor{currentstroke}{rgb}{0.000000,0.000000,0.000000}%
\pgfsetstrokecolor{currentstroke}%
\pgfsetdash{}{0pt}%
\pgfsys@defobject{currentmarker}{\pgfqpoint{-0.048611in}{0.000000in}}{\pgfqpoint{-0.000000in}{0.000000in}}{%
\pgfpathmoveto{\pgfqpoint{-0.000000in}{0.000000in}}%
\pgfpathlineto{\pgfqpoint{-0.048611in}{0.000000in}}%
\pgfusepath{stroke,fill}%
}%
\begin{pgfscope}%
\pgfsys@transformshift{0.445556in}{1.493228in}%
\pgfsys@useobject{currentmarker}{}%
\end{pgfscope}%
\end{pgfscope}%
\begin{pgfscope}%
\definecolor{textcolor}{rgb}{0.000000,0.000000,0.000000}%
\pgfsetstrokecolor{textcolor}%
\pgfsetfillcolor{textcolor}%
\pgftext[x=0.278889in, y=1.445033in, left, base]{\color{textcolor}\rmfamily\fontsize{10.000000}{12.000000}\selectfont \(\displaystyle {6}\)}%
\end{pgfscope}%
\begin{pgfscope}%
\definecolor{textcolor}{rgb}{0.000000,0.000000,0.000000}%
\pgfsetstrokecolor{textcolor}%
\pgfsetfillcolor{textcolor}%
\pgftext[x=0.223333in,y=1.076944in,,bottom,rotate=90.000000]{\color{textcolor}\rmfamily\fontsize{10.000000}{12.000000}\selectfont Percent of Data Set}%
\end{pgfscope}%
\begin{pgfscope}%
\pgfsetrectcap%
\pgfsetmiterjoin%
\pgfsetlinewidth{0.803000pt}%
\definecolor{currentstroke}{rgb}{0.000000,0.000000,0.000000}%
\pgfsetstrokecolor{currentstroke}%
\pgfsetdash{}{0pt}%
\pgfpathmoveto{\pgfqpoint{0.445556in}{0.499444in}}%
\pgfpathlineto{\pgfqpoint{0.445556in}{1.654444in}}%
\pgfusepath{stroke}%
\end{pgfscope}%
\begin{pgfscope}%
\pgfsetrectcap%
\pgfsetmiterjoin%
\pgfsetlinewidth{0.803000pt}%
\definecolor{currentstroke}{rgb}{0.000000,0.000000,0.000000}%
\pgfsetstrokecolor{currentstroke}%
\pgfsetdash{}{0pt}%
\pgfpathmoveto{\pgfqpoint{4.320556in}{0.499444in}}%
\pgfpathlineto{\pgfqpoint{4.320556in}{1.654444in}}%
\pgfusepath{stroke}%
\end{pgfscope}%
\begin{pgfscope}%
\pgfsetrectcap%
\pgfsetmiterjoin%
\pgfsetlinewidth{0.803000pt}%
\definecolor{currentstroke}{rgb}{0.000000,0.000000,0.000000}%
\pgfsetstrokecolor{currentstroke}%
\pgfsetdash{}{0pt}%
\pgfpathmoveto{\pgfqpoint{0.445556in}{0.499444in}}%
\pgfpathlineto{\pgfqpoint{4.320556in}{0.499444in}}%
\pgfusepath{stroke}%
\end{pgfscope}%
\begin{pgfscope}%
\pgfsetrectcap%
\pgfsetmiterjoin%
\pgfsetlinewidth{0.803000pt}%
\definecolor{currentstroke}{rgb}{0.000000,0.000000,0.000000}%
\pgfsetstrokecolor{currentstroke}%
\pgfsetdash{}{0pt}%
\pgfpathmoveto{\pgfqpoint{0.445556in}{1.654444in}}%
\pgfpathlineto{\pgfqpoint{4.320556in}{1.654444in}}%
\pgfusepath{stroke}%
\end{pgfscope}%
\begin{pgfscope}%
\pgfsetbuttcap%
\pgfsetmiterjoin%
\definecolor{currentfill}{rgb}{1.000000,1.000000,1.000000}%
\pgfsetfillcolor{currentfill}%
\pgfsetfillopacity{0.800000}%
\pgfsetlinewidth{1.003750pt}%
\definecolor{currentstroke}{rgb}{0.800000,0.800000,0.800000}%
\pgfsetstrokecolor{currentstroke}%
\pgfsetstrokeopacity{0.800000}%
\pgfsetdash{}{0pt}%
\pgfpathmoveto{\pgfqpoint{3.543611in}{1.154445in}}%
\pgfpathlineto{\pgfqpoint{4.223333in}{1.154445in}}%
\pgfpathquadraticcurveto{\pgfqpoint{4.251111in}{1.154445in}}{\pgfqpoint{4.251111in}{1.182222in}}%
\pgfpathlineto{\pgfqpoint{4.251111in}{1.557222in}}%
\pgfpathquadraticcurveto{\pgfqpoint{4.251111in}{1.585000in}}{\pgfqpoint{4.223333in}{1.585000in}}%
\pgfpathlineto{\pgfqpoint{3.543611in}{1.585000in}}%
\pgfpathquadraticcurveto{\pgfqpoint{3.515833in}{1.585000in}}{\pgfqpoint{3.515833in}{1.557222in}}%
\pgfpathlineto{\pgfqpoint{3.515833in}{1.182222in}}%
\pgfpathquadraticcurveto{\pgfqpoint{3.515833in}{1.154445in}}{\pgfqpoint{3.543611in}{1.154445in}}%
\pgfpathlineto{\pgfqpoint{3.543611in}{1.154445in}}%
\pgfpathclose%
\pgfusepath{stroke,fill}%
\end{pgfscope}%
\begin{pgfscope}%
\pgfsetbuttcap%
\pgfsetmiterjoin%
\pgfsetlinewidth{1.003750pt}%
\definecolor{currentstroke}{rgb}{0.000000,0.000000,0.000000}%
\pgfsetstrokecolor{currentstroke}%
\pgfsetdash{}{0pt}%
\pgfpathmoveto{\pgfqpoint{3.571389in}{1.432222in}}%
\pgfpathlineto{\pgfqpoint{3.849167in}{1.432222in}}%
\pgfpathlineto{\pgfqpoint{3.849167in}{1.529444in}}%
\pgfpathlineto{\pgfqpoint{3.571389in}{1.529444in}}%
\pgfpathlineto{\pgfqpoint{3.571389in}{1.432222in}}%
\pgfpathclose%
\pgfusepath{stroke}%
\end{pgfscope}%
\begin{pgfscope}%
\definecolor{textcolor}{rgb}{0.000000,0.000000,0.000000}%
\pgfsetstrokecolor{textcolor}%
\pgfsetfillcolor{textcolor}%
\pgftext[x=3.960278in,y=1.432222in,left,base]{\color{textcolor}\rmfamily\fontsize{10.000000}{12.000000}\selectfont Neg}%
\end{pgfscope}%
\begin{pgfscope}%
\pgfsetbuttcap%
\pgfsetmiterjoin%
\definecolor{currentfill}{rgb}{0.000000,0.000000,0.000000}%
\pgfsetfillcolor{currentfill}%
\pgfsetlinewidth{0.000000pt}%
\definecolor{currentstroke}{rgb}{0.000000,0.000000,0.000000}%
\pgfsetstrokecolor{currentstroke}%
\pgfsetstrokeopacity{0.000000}%
\pgfsetdash{}{0pt}%
\pgfpathmoveto{\pgfqpoint{3.571389in}{1.236944in}}%
\pgfpathlineto{\pgfqpoint{3.849167in}{1.236944in}}%
\pgfpathlineto{\pgfqpoint{3.849167in}{1.334167in}}%
\pgfpathlineto{\pgfqpoint{3.571389in}{1.334167in}}%
\pgfpathlineto{\pgfqpoint{3.571389in}{1.236944in}}%
\pgfpathclose%
\pgfusepath{fill}%
\end{pgfscope}%
\begin{pgfscope}%
\definecolor{textcolor}{rgb}{0.000000,0.000000,0.000000}%
\pgfsetstrokecolor{textcolor}%
\pgfsetfillcolor{textcolor}%
\pgftext[x=3.960278in,y=1.236944in,left,base]{\color{textcolor}\rmfamily\fontsize{10.000000}{12.000000}\selectfont Pos}%
\end{pgfscope}%
\end{pgfpicture}%
\makeatother%
\endgroup%

&
	\vskip 0pt
	\qquad \qquad ROC Curve
	
	%% Creator: Matplotlib, PGF backend
%%
%% To include the figure in your LaTeX document, write
%%   \input{<filename>.pgf}
%%
%% Make sure the required packages are loaded in your preamble
%%   \usepackage{pgf}
%%
%% Also ensure that all the required font packages are loaded; for instance,
%% the lmodern package is sometimes necessary when using math font.
%%   \usepackage{lmodern}
%%
%% Figures using additional raster images can only be included by \input if
%% they are in the same directory as the main LaTeX file. For loading figures
%% from other directories you can use the `import` package
%%   \usepackage{import}
%%
%% and then include the figures with
%%   \import{<path to file>}{<filename>.pgf}
%%
%% Matplotlib used the following preamble
%%   
%%   \usepackage{fontspec}
%%   \makeatletter\@ifpackageloaded{underscore}{}{\usepackage[strings]{underscore}}\makeatother
%%
\begingroup%
\makeatletter%
\begin{pgfpicture}%
\pgfpathrectangle{\pgfpointorigin}{\pgfqpoint{2.221861in}{1.754444in}}%
\pgfusepath{use as bounding box, clip}%
\begin{pgfscope}%
\pgfsetbuttcap%
\pgfsetmiterjoin%
\definecolor{currentfill}{rgb}{1.000000,1.000000,1.000000}%
\pgfsetfillcolor{currentfill}%
\pgfsetlinewidth{0.000000pt}%
\definecolor{currentstroke}{rgb}{1.000000,1.000000,1.000000}%
\pgfsetstrokecolor{currentstroke}%
\pgfsetdash{}{0pt}%
\pgfpathmoveto{\pgfqpoint{0.000000in}{0.000000in}}%
\pgfpathlineto{\pgfqpoint{2.221861in}{0.000000in}}%
\pgfpathlineto{\pgfqpoint{2.221861in}{1.754444in}}%
\pgfpathlineto{\pgfqpoint{0.000000in}{1.754444in}}%
\pgfpathlineto{\pgfqpoint{0.000000in}{0.000000in}}%
\pgfpathclose%
\pgfusepath{fill}%
\end{pgfscope}%
\begin{pgfscope}%
\pgfsetbuttcap%
\pgfsetmiterjoin%
\definecolor{currentfill}{rgb}{1.000000,1.000000,1.000000}%
\pgfsetfillcolor{currentfill}%
\pgfsetlinewidth{0.000000pt}%
\definecolor{currentstroke}{rgb}{0.000000,0.000000,0.000000}%
\pgfsetstrokecolor{currentstroke}%
\pgfsetstrokeopacity{0.000000}%
\pgfsetdash{}{0pt}%
\pgfpathmoveto{\pgfqpoint{0.553581in}{0.499444in}}%
\pgfpathlineto{\pgfqpoint{2.103581in}{0.499444in}}%
\pgfpathlineto{\pgfqpoint{2.103581in}{1.654444in}}%
\pgfpathlineto{\pgfqpoint{0.553581in}{1.654444in}}%
\pgfpathlineto{\pgfqpoint{0.553581in}{0.499444in}}%
\pgfpathclose%
\pgfusepath{fill}%
\end{pgfscope}%
\begin{pgfscope}%
\pgfsetbuttcap%
\pgfsetroundjoin%
\definecolor{currentfill}{rgb}{0.000000,0.000000,0.000000}%
\pgfsetfillcolor{currentfill}%
\pgfsetlinewidth{0.803000pt}%
\definecolor{currentstroke}{rgb}{0.000000,0.000000,0.000000}%
\pgfsetstrokecolor{currentstroke}%
\pgfsetdash{}{0pt}%
\pgfsys@defobject{currentmarker}{\pgfqpoint{0.000000in}{-0.048611in}}{\pgfqpoint{0.000000in}{0.000000in}}{%
\pgfpathmoveto{\pgfqpoint{0.000000in}{0.000000in}}%
\pgfpathlineto{\pgfqpoint{0.000000in}{-0.048611in}}%
\pgfusepath{stroke,fill}%
}%
\begin{pgfscope}%
\pgfsys@transformshift{0.624035in}{0.499444in}%
\pgfsys@useobject{currentmarker}{}%
\end{pgfscope}%
\end{pgfscope}%
\begin{pgfscope}%
\definecolor{textcolor}{rgb}{0.000000,0.000000,0.000000}%
\pgfsetstrokecolor{textcolor}%
\pgfsetfillcolor{textcolor}%
\pgftext[x=0.624035in,y=0.402222in,,top]{\color{textcolor}\rmfamily\fontsize{10.000000}{12.000000}\selectfont \(\displaystyle {0.0}\)}%
\end{pgfscope}%
\begin{pgfscope}%
\pgfsetbuttcap%
\pgfsetroundjoin%
\definecolor{currentfill}{rgb}{0.000000,0.000000,0.000000}%
\pgfsetfillcolor{currentfill}%
\pgfsetlinewidth{0.803000pt}%
\definecolor{currentstroke}{rgb}{0.000000,0.000000,0.000000}%
\pgfsetstrokecolor{currentstroke}%
\pgfsetdash{}{0pt}%
\pgfsys@defobject{currentmarker}{\pgfqpoint{0.000000in}{-0.048611in}}{\pgfqpoint{0.000000in}{0.000000in}}{%
\pgfpathmoveto{\pgfqpoint{0.000000in}{0.000000in}}%
\pgfpathlineto{\pgfqpoint{0.000000in}{-0.048611in}}%
\pgfusepath{stroke,fill}%
}%
\begin{pgfscope}%
\pgfsys@transformshift{1.328581in}{0.499444in}%
\pgfsys@useobject{currentmarker}{}%
\end{pgfscope}%
\end{pgfscope}%
\begin{pgfscope}%
\definecolor{textcolor}{rgb}{0.000000,0.000000,0.000000}%
\pgfsetstrokecolor{textcolor}%
\pgfsetfillcolor{textcolor}%
\pgftext[x=1.328581in,y=0.402222in,,top]{\color{textcolor}\rmfamily\fontsize{10.000000}{12.000000}\selectfont \(\displaystyle {0.5}\)}%
\end{pgfscope}%
\begin{pgfscope}%
\pgfsetbuttcap%
\pgfsetroundjoin%
\definecolor{currentfill}{rgb}{0.000000,0.000000,0.000000}%
\pgfsetfillcolor{currentfill}%
\pgfsetlinewidth{0.803000pt}%
\definecolor{currentstroke}{rgb}{0.000000,0.000000,0.000000}%
\pgfsetstrokecolor{currentstroke}%
\pgfsetdash{}{0pt}%
\pgfsys@defobject{currentmarker}{\pgfqpoint{0.000000in}{-0.048611in}}{\pgfqpoint{0.000000in}{0.000000in}}{%
\pgfpathmoveto{\pgfqpoint{0.000000in}{0.000000in}}%
\pgfpathlineto{\pgfqpoint{0.000000in}{-0.048611in}}%
\pgfusepath{stroke,fill}%
}%
\begin{pgfscope}%
\pgfsys@transformshift{2.033126in}{0.499444in}%
\pgfsys@useobject{currentmarker}{}%
\end{pgfscope}%
\end{pgfscope}%
\begin{pgfscope}%
\definecolor{textcolor}{rgb}{0.000000,0.000000,0.000000}%
\pgfsetstrokecolor{textcolor}%
\pgfsetfillcolor{textcolor}%
\pgftext[x=2.033126in,y=0.402222in,,top]{\color{textcolor}\rmfamily\fontsize{10.000000}{12.000000}\selectfont \(\displaystyle {1.0}\)}%
\end{pgfscope}%
\begin{pgfscope}%
\definecolor{textcolor}{rgb}{0.000000,0.000000,0.000000}%
\pgfsetstrokecolor{textcolor}%
\pgfsetfillcolor{textcolor}%
\pgftext[x=1.328581in,y=0.223333in,,top]{\color{textcolor}\rmfamily\fontsize{10.000000}{12.000000}\selectfont False positive rate}%
\end{pgfscope}%
\begin{pgfscope}%
\pgfsetbuttcap%
\pgfsetroundjoin%
\definecolor{currentfill}{rgb}{0.000000,0.000000,0.000000}%
\pgfsetfillcolor{currentfill}%
\pgfsetlinewidth{0.803000pt}%
\definecolor{currentstroke}{rgb}{0.000000,0.000000,0.000000}%
\pgfsetstrokecolor{currentstroke}%
\pgfsetdash{}{0pt}%
\pgfsys@defobject{currentmarker}{\pgfqpoint{-0.048611in}{0.000000in}}{\pgfqpoint{-0.000000in}{0.000000in}}{%
\pgfpathmoveto{\pgfqpoint{-0.000000in}{0.000000in}}%
\pgfpathlineto{\pgfqpoint{-0.048611in}{0.000000in}}%
\pgfusepath{stroke,fill}%
}%
\begin{pgfscope}%
\pgfsys@transformshift{0.553581in}{0.551944in}%
\pgfsys@useobject{currentmarker}{}%
\end{pgfscope}%
\end{pgfscope}%
\begin{pgfscope}%
\definecolor{textcolor}{rgb}{0.000000,0.000000,0.000000}%
\pgfsetstrokecolor{textcolor}%
\pgfsetfillcolor{textcolor}%
\pgftext[x=0.278889in, y=0.503750in, left, base]{\color{textcolor}\rmfamily\fontsize{10.000000}{12.000000}\selectfont \(\displaystyle {0.0}\)}%
\end{pgfscope}%
\begin{pgfscope}%
\pgfsetbuttcap%
\pgfsetroundjoin%
\definecolor{currentfill}{rgb}{0.000000,0.000000,0.000000}%
\pgfsetfillcolor{currentfill}%
\pgfsetlinewidth{0.803000pt}%
\definecolor{currentstroke}{rgb}{0.000000,0.000000,0.000000}%
\pgfsetstrokecolor{currentstroke}%
\pgfsetdash{}{0pt}%
\pgfsys@defobject{currentmarker}{\pgfqpoint{-0.048611in}{0.000000in}}{\pgfqpoint{-0.000000in}{0.000000in}}{%
\pgfpathmoveto{\pgfqpoint{-0.000000in}{0.000000in}}%
\pgfpathlineto{\pgfqpoint{-0.048611in}{0.000000in}}%
\pgfusepath{stroke,fill}%
}%
\begin{pgfscope}%
\pgfsys@transformshift{0.553581in}{1.076944in}%
\pgfsys@useobject{currentmarker}{}%
\end{pgfscope}%
\end{pgfscope}%
\begin{pgfscope}%
\definecolor{textcolor}{rgb}{0.000000,0.000000,0.000000}%
\pgfsetstrokecolor{textcolor}%
\pgfsetfillcolor{textcolor}%
\pgftext[x=0.278889in, y=1.028750in, left, base]{\color{textcolor}\rmfamily\fontsize{10.000000}{12.000000}\selectfont \(\displaystyle {0.5}\)}%
\end{pgfscope}%
\begin{pgfscope}%
\pgfsetbuttcap%
\pgfsetroundjoin%
\definecolor{currentfill}{rgb}{0.000000,0.000000,0.000000}%
\pgfsetfillcolor{currentfill}%
\pgfsetlinewidth{0.803000pt}%
\definecolor{currentstroke}{rgb}{0.000000,0.000000,0.000000}%
\pgfsetstrokecolor{currentstroke}%
\pgfsetdash{}{0pt}%
\pgfsys@defobject{currentmarker}{\pgfqpoint{-0.048611in}{0.000000in}}{\pgfqpoint{-0.000000in}{0.000000in}}{%
\pgfpathmoveto{\pgfqpoint{-0.000000in}{0.000000in}}%
\pgfpathlineto{\pgfqpoint{-0.048611in}{0.000000in}}%
\pgfusepath{stroke,fill}%
}%
\begin{pgfscope}%
\pgfsys@transformshift{0.553581in}{1.601944in}%
\pgfsys@useobject{currentmarker}{}%
\end{pgfscope}%
\end{pgfscope}%
\begin{pgfscope}%
\definecolor{textcolor}{rgb}{0.000000,0.000000,0.000000}%
\pgfsetstrokecolor{textcolor}%
\pgfsetfillcolor{textcolor}%
\pgftext[x=0.278889in, y=1.553750in, left, base]{\color{textcolor}\rmfamily\fontsize{10.000000}{12.000000}\selectfont \(\displaystyle {1.0}\)}%
\end{pgfscope}%
\begin{pgfscope}%
\definecolor{textcolor}{rgb}{0.000000,0.000000,0.000000}%
\pgfsetstrokecolor{textcolor}%
\pgfsetfillcolor{textcolor}%
\pgftext[x=0.223333in,y=1.076944in,,bottom,rotate=90.000000]{\color{textcolor}\rmfamily\fontsize{10.000000}{12.000000}\selectfont True positive rate}%
\end{pgfscope}%
\begin{pgfscope}%
\pgfpathrectangle{\pgfqpoint{0.553581in}{0.499444in}}{\pgfqpoint{1.550000in}{1.155000in}}%
\pgfusepath{clip}%
\pgfsetbuttcap%
\pgfsetroundjoin%
\pgfsetlinewidth{1.505625pt}%
\definecolor{currentstroke}{rgb}{0.000000,0.000000,0.000000}%
\pgfsetstrokecolor{currentstroke}%
\pgfsetdash{{5.550000pt}{2.400000pt}}{0.000000pt}%
\pgfpathmoveto{\pgfqpoint{0.624035in}{0.551944in}}%
\pgfpathlineto{\pgfqpoint{2.033126in}{1.601944in}}%
\pgfusepath{stroke}%
\end{pgfscope}%
\begin{pgfscope}%
\pgfpathrectangle{\pgfqpoint{0.553581in}{0.499444in}}{\pgfqpoint{1.550000in}{1.155000in}}%
\pgfusepath{clip}%
\pgfsetrectcap%
\pgfsetroundjoin%
\pgfsetlinewidth{1.505625pt}%
\definecolor{currentstroke}{rgb}{0.000000,0.000000,0.000000}%
\pgfsetstrokecolor{currentstroke}%
\pgfsetdash{}{0pt}%
\pgfpathmoveto{\pgfqpoint{0.624035in}{0.551944in}}%
\pgfpathlineto{\pgfqpoint{0.625818in}{0.578454in}}%
\pgfpathlineto{\pgfqpoint{0.635863in}{0.672977in}}%
\pgfpathlineto{\pgfqpoint{0.642954in}{0.717523in}}%
\pgfpathlineto{\pgfqpoint{0.652061in}{0.766011in}}%
\pgfpathlineto{\pgfqpoint{0.660082in}{0.797456in}}%
\pgfpathlineto{\pgfqpoint{0.680307in}{0.866246in}}%
\pgfpathlineto{\pgfqpoint{0.692854in}{0.901510in}}%
\pgfpathlineto{\pgfqpoint{0.692924in}{0.901789in}}%
\pgfpathlineto{\pgfqpoint{0.715377in}{0.955057in}}%
\pgfpathlineto{\pgfqpoint{0.723835in}{0.974645in}}%
\pgfpathlineto{\pgfqpoint{0.765613in}{1.048215in}}%
\pgfpathlineto{\pgfqpoint{0.817389in}{1.123306in}}%
\pgfpathlineto{\pgfqpoint{0.847119in}{1.160680in}}%
\pgfpathlineto{\pgfqpoint{0.879485in}{1.196068in}}%
\pgfpathlineto{\pgfqpoint{0.880837in}{1.197248in}}%
\pgfpathlineto{\pgfqpoint{0.881064in}{1.197465in}}%
\pgfpathlineto{\pgfqpoint{0.936850in}{1.250454in}}%
\pgfpathlineto{\pgfqpoint{0.977643in}{1.283607in}}%
\pgfpathlineto{\pgfqpoint{0.999563in}{1.300153in}}%
\pgfpathlineto{\pgfqpoint{1.092633in}{1.363292in}}%
\pgfpathlineto{\pgfqpoint{1.116703in}{1.376827in}}%
\pgfpathlineto{\pgfqpoint{1.141766in}{1.389989in}}%
\pgfpathlineto{\pgfqpoint{1.167416in}{1.404609in}}%
\pgfpathlineto{\pgfqpoint{1.246452in}{1.440215in}}%
\pgfpathlineto{\pgfqpoint{1.273478in}{1.452507in}}%
\pgfpathlineto{\pgfqpoint{1.299753in}{1.462689in}}%
\pgfpathlineto{\pgfqpoint{1.438524in}{1.510618in}}%
\pgfpathlineto{\pgfqpoint{1.465792in}{1.519403in}}%
\pgfpathlineto{\pgfqpoint{1.521602in}{1.533651in}}%
\pgfpathlineto{\pgfqpoint{1.576935in}{1.547000in}}%
\pgfpathlineto{\pgfqpoint{1.707928in}{1.571678in}}%
\pgfpathlineto{\pgfqpoint{1.802459in}{1.584933in}}%
\pgfpathlineto{\pgfqpoint{1.865414in}{1.591762in}}%
\pgfpathlineto{\pgfqpoint{1.936570in}{1.597629in}}%
\pgfpathlineto{\pgfqpoint{1.989793in}{1.600392in}}%
\pgfpathlineto{\pgfqpoint{2.033126in}{1.601944in}}%
\pgfpathlineto{\pgfqpoint{2.033126in}{1.601944in}}%
\pgfusepath{stroke}%
\end{pgfscope}%
\begin{pgfscope}%
\pgfsetrectcap%
\pgfsetmiterjoin%
\pgfsetlinewidth{0.803000pt}%
\definecolor{currentstroke}{rgb}{0.000000,0.000000,0.000000}%
\pgfsetstrokecolor{currentstroke}%
\pgfsetdash{}{0pt}%
\pgfpathmoveto{\pgfqpoint{0.553581in}{0.499444in}}%
\pgfpathlineto{\pgfqpoint{0.553581in}{1.654444in}}%
\pgfusepath{stroke}%
\end{pgfscope}%
\begin{pgfscope}%
\pgfsetrectcap%
\pgfsetmiterjoin%
\pgfsetlinewidth{0.803000pt}%
\definecolor{currentstroke}{rgb}{0.000000,0.000000,0.000000}%
\pgfsetstrokecolor{currentstroke}%
\pgfsetdash{}{0pt}%
\pgfpathmoveto{\pgfqpoint{2.103581in}{0.499444in}}%
\pgfpathlineto{\pgfqpoint{2.103581in}{1.654444in}}%
\pgfusepath{stroke}%
\end{pgfscope}%
\begin{pgfscope}%
\pgfsetrectcap%
\pgfsetmiterjoin%
\pgfsetlinewidth{0.803000pt}%
\definecolor{currentstroke}{rgb}{0.000000,0.000000,0.000000}%
\pgfsetstrokecolor{currentstroke}%
\pgfsetdash{}{0pt}%
\pgfpathmoveto{\pgfqpoint{0.553581in}{0.499444in}}%
\pgfpathlineto{\pgfqpoint{2.103581in}{0.499444in}}%
\pgfusepath{stroke}%
\end{pgfscope}%
\begin{pgfscope}%
\pgfsetrectcap%
\pgfsetmiterjoin%
\pgfsetlinewidth{0.803000pt}%
\definecolor{currentstroke}{rgb}{0.000000,0.000000,0.000000}%
\pgfsetstrokecolor{currentstroke}%
\pgfsetdash{}{0pt}%
\pgfpathmoveto{\pgfqpoint{0.553581in}{1.654444in}}%
\pgfpathlineto{\pgfqpoint{2.103581in}{1.654444in}}%
\pgfusepath{stroke}%
\end{pgfscope}%
\begin{pgfscope}%
\pgfsetbuttcap%
\pgfsetmiterjoin%
\definecolor{currentfill}{rgb}{1.000000,1.000000,1.000000}%
\pgfsetfillcolor{currentfill}%
\pgfsetfillopacity{0.800000}%
\pgfsetlinewidth{1.003750pt}%
\definecolor{currentstroke}{rgb}{0.800000,0.800000,0.800000}%
\pgfsetstrokecolor{currentstroke}%
\pgfsetstrokeopacity{0.800000}%
\pgfsetdash{}{0pt}%
\pgfpathmoveto{\pgfqpoint{0.832747in}{0.568889in}}%
\pgfpathlineto{\pgfqpoint{2.006358in}{0.568889in}}%
\pgfpathquadraticcurveto{\pgfqpoint{2.034136in}{0.568889in}}{\pgfqpoint{2.034136in}{0.596666in}}%
\pgfpathlineto{\pgfqpoint{2.034136in}{0.776388in}}%
\pgfpathquadraticcurveto{\pgfqpoint{2.034136in}{0.804166in}}{\pgfqpoint{2.006358in}{0.804166in}}%
\pgfpathlineto{\pgfqpoint{0.832747in}{0.804166in}}%
\pgfpathquadraticcurveto{\pgfqpoint{0.804970in}{0.804166in}}{\pgfqpoint{0.804970in}{0.776388in}}%
\pgfpathlineto{\pgfqpoint{0.804970in}{0.596666in}}%
\pgfpathquadraticcurveto{\pgfqpoint{0.804970in}{0.568889in}}{\pgfqpoint{0.832747in}{0.568889in}}%
\pgfpathlineto{\pgfqpoint{0.832747in}{0.568889in}}%
\pgfpathclose%
\pgfusepath{stroke,fill}%
\end{pgfscope}%
\begin{pgfscope}%
\pgfsetrectcap%
\pgfsetroundjoin%
\pgfsetlinewidth{1.505625pt}%
\definecolor{currentstroke}{rgb}{0.000000,0.000000,0.000000}%
\pgfsetstrokecolor{currentstroke}%
\pgfsetdash{}{0pt}%
\pgfpathmoveto{\pgfqpoint{0.860525in}{0.700000in}}%
\pgfpathlineto{\pgfqpoint{0.999414in}{0.700000in}}%
\pgfpathlineto{\pgfqpoint{1.138303in}{0.700000in}}%
\pgfusepath{stroke}%
\end{pgfscope}%
\begin{pgfscope}%
\definecolor{textcolor}{rgb}{0.000000,0.000000,0.000000}%
\pgfsetstrokecolor{textcolor}%
\pgfsetfillcolor{textcolor}%
\pgftext[x=1.249414in,y=0.651388in,left,base]{\color{textcolor}\rmfamily\fontsize{10.000000}{12.000000}\selectfont AUC=0.799}%
\end{pgfscope}%
\end{pgfpicture}%
\makeatother%
\endgroup%

\end{tabular}

\
	
In the table below,

\begin{itemize}
	\item The first three columns are the same information as in the histogram.  ``Neg'' (``Pos'') is the number of negative (positive) samples whose probability of being in Class 1 is in that range of $p$.  
	\item The fourth column, ``Neg/Pos,'' is the number of unneeded ambulances sent for each needed ambulance sent, for samples with $p$ in that range.    This is the marginal cost of having $\theta$ below this value of $p$.  These values are given in this plot, emphasizing where Neg/Pos = $\Delta FP/\Delta TP = 2$. 
	
%% Creator: Matplotlib, PGF backend
%%
%% To include the figure in your LaTeX document, write
%%   \input{<filename>.pgf}
%%
%% Make sure the required packages are loaded in your preamble
%%   \usepackage{pgf}
%%
%% Also ensure that all the required font packages are loaded; for instance,
%% the lmodern package is sometimes necessary when using math font.
%%   \usepackage{lmodern}
%%
%% Figures using additional raster images can only be included by \input if
%% they are in the same directory as the main LaTeX file. For loading figures
%% from other directories you can use the `import` package
%%   \usepackage{import}
%%
%% and then include the figures with
%%   \import{<path to file>}{<filename>.pgf}
%%
%% Matplotlib used the following preamble
%%   
%%   \usepackage{fontspec}
%%   \makeatletter\@ifpackageloaded{underscore}{}{\usepackage[strings]{underscore}}\makeatother
%%
\begingroup%
\makeatletter%
\begin{pgfpicture}%
\pgfpathrectangle{\pgfpointorigin}{\pgfqpoint{2.317251in}{1.754444in}}%
\pgfusepath{use as bounding box, clip}%
\begin{pgfscope}%
\pgfsetbuttcap%
\pgfsetmiterjoin%
\definecolor{currentfill}{rgb}{1.000000,1.000000,1.000000}%
\pgfsetfillcolor{currentfill}%
\pgfsetlinewidth{0.000000pt}%
\definecolor{currentstroke}{rgb}{1.000000,1.000000,1.000000}%
\pgfsetstrokecolor{currentstroke}%
\pgfsetdash{}{0pt}%
\pgfpathmoveto{\pgfqpoint{0.000000in}{0.000000in}}%
\pgfpathlineto{\pgfqpoint{2.317251in}{0.000000in}}%
\pgfpathlineto{\pgfqpoint{2.317251in}{1.754444in}}%
\pgfpathlineto{\pgfqpoint{0.000000in}{1.754444in}}%
\pgfpathlineto{\pgfqpoint{0.000000in}{0.000000in}}%
\pgfpathclose%
\pgfusepath{fill}%
\end{pgfscope}%
\begin{pgfscope}%
\pgfsetbuttcap%
\pgfsetmiterjoin%
\definecolor{currentfill}{rgb}{1.000000,1.000000,1.000000}%
\pgfsetfillcolor{currentfill}%
\pgfsetlinewidth{0.000000pt}%
\definecolor{currentstroke}{rgb}{0.000000,0.000000,0.000000}%
\pgfsetstrokecolor{currentstroke}%
\pgfsetstrokeopacity{0.000000}%
\pgfsetdash{}{0pt}%
\pgfpathmoveto{\pgfqpoint{0.600000in}{0.499444in}}%
\pgfpathlineto{\pgfqpoint{2.150000in}{0.499444in}}%
\pgfpathlineto{\pgfqpoint{2.150000in}{1.654444in}}%
\pgfpathlineto{\pgfqpoint{0.600000in}{1.654444in}}%
\pgfpathlineto{\pgfqpoint{0.600000in}{0.499444in}}%
\pgfpathclose%
\pgfusepath{fill}%
\end{pgfscope}%
\begin{pgfscope}%
\pgfsetbuttcap%
\pgfsetroundjoin%
\definecolor{currentfill}{rgb}{0.000000,0.000000,0.000000}%
\pgfsetfillcolor{currentfill}%
\pgfsetlinewidth{0.803000pt}%
\definecolor{currentstroke}{rgb}{0.000000,0.000000,0.000000}%
\pgfsetstrokecolor{currentstroke}%
\pgfsetdash{}{0pt}%
\pgfsys@defobject{currentmarker}{\pgfqpoint{0.000000in}{-0.048611in}}{\pgfqpoint{0.000000in}{0.000000in}}{%
\pgfpathmoveto{\pgfqpoint{0.000000in}{0.000000in}}%
\pgfpathlineto{\pgfqpoint{0.000000in}{-0.048611in}}%
\pgfusepath{stroke,fill}%
}%
\begin{pgfscope}%
\pgfsys@transformshift{0.670455in}{0.499444in}%
\pgfsys@useobject{currentmarker}{}%
\end{pgfscope}%
\end{pgfscope}%
\begin{pgfscope}%
\definecolor{textcolor}{rgb}{0.000000,0.000000,0.000000}%
\pgfsetstrokecolor{textcolor}%
\pgfsetfillcolor{textcolor}%
\pgftext[x=0.670455in,y=0.402222in,,top]{\color{textcolor}\rmfamily\fontsize{10.000000}{12.000000}\selectfont 0.03}%
\end{pgfscope}%
\begin{pgfscope}%
\pgfsetbuttcap%
\pgfsetroundjoin%
\definecolor{currentfill}{rgb}{0.000000,0.000000,0.000000}%
\pgfsetfillcolor{currentfill}%
\pgfsetlinewidth{0.803000pt}%
\definecolor{currentstroke}{rgb}{0.000000,0.000000,0.000000}%
\pgfsetstrokecolor{currentstroke}%
\pgfsetdash{}{0pt}%
\pgfsys@defobject{currentmarker}{\pgfqpoint{0.000000in}{-0.048611in}}{\pgfqpoint{0.000000in}{0.000000in}}{%
\pgfpathmoveto{\pgfqpoint{0.000000in}{0.000000in}}%
\pgfpathlineto{\pgfqpoint{0.000000in}{-0.048611in}}%
\pgfusepath{stroke,fill}%
}%
\begin{pgfscope}%
\pgfsys@transformshift{2.093779in}{0.499444in}%
\pgfsys@useobject{currentmarker}{}%
\end{pgfscope}%
\end{pgfscope}%
\begin{pgfscope}%
\definecolor{textcolor}{rgb}{0.000000,0.000000,0.000000}%
\pgfsetstrokecolor{textcolor}%
\pgfsetfillcolor{textcolor}%
\pgftext[x=2.093779in,y=0.402222in,,top]{\color{textcolor}\rmfamily\fontsize{10.000000}{12.000000}\selectfont 0.98}%
\end{pgfscope}%
\begin{pgfscope}%
\definecolor{textcolor}{rgb}{0.000000,0.000000,0.000000}%
\pgfsetstrokecolor{textcolor}%
\pgfsetfillcolor{textcolor}%
\pgftext[x=1.375000in,y=0.223333in,,top]{\color{textcolor}\rmfamily\fontsize{10.000000}{12.000000}\selectfont \(\displaystyle p\)}%
\end{pgfscope}%
\begin{pgfscope}%
\pgfsetbuttcap%
\pgfsetroundjoin%
\definecolor{currentfill}{rgb}{0.000000,0.000000,0.000000}%
\pgfsetfillcolor{currentfill}%
\pgfsetlinewidth{0.803000pt}%
\definecolor{currentstroke}{rgb}{0.000000,0.000000,0.000000}%
\pgfsetstrokecolor{currentstroke}%
\pgfsetdash{}{0pt}%
\pgfsys@defobject{currentmarker}{\pgfqpoint{-0.048611in}{0.000000in}}{\pgfqpoint{-0.000000in}{0.000000in}}{%
\pgfpathmoveto{\pgfqpoint{-0.000000in}{0.000000in}}%
\pgfpathlineto{\pgfqpoint{-0.048611in}{0.000000in}}%
\pgfusepath{stroke,fill}%
}%
\begin{pgfscope}%
\pgfsys@transformshift{0.600000in}{0.548323in}%
\pgfsys@useobject{currentmarker}{}%
\end{pgfscope}%
\end{pgfscope}%
\begin{pgfscope}%
\definecolor{textcolor}{rgb}{0.000000,0.000000,0.000000}%
\pgfsetstrokecolor{textcolor}%
\pgfsetfillcolor{textcolor}%
\pgftext[x=0.433333in, y=0.500129in, left, base]{\color{textcolor}\rmfamily\fontsize{10.000000}{12.000000}\selectfont \(\displaystyle {0}\)}%
\end{pgfscope}%
\begin{pgfscope}%
\pgfsetbuttcap%
\pgfsetroundjoin%
\definecolor{currentfill}{rgb}{0.000000,0.000000,0.000000}%
\pgfsetfillcolor{currentfill}%
\pgfsetlinewidth{0.803000pt}%
\definecolor{currentstroke}{rgb}{0.000000,0.000000,0.000000}%
\pgfsetstrokecolor{currentstroke}%
\pgfsetdash{}{0pt}%
\pgfsys@defobject{currentmarker}{\pgfqpoint{-0.048611in}{0.000000in}}{\pgfqpoint{-0.000000in}{0.000000in}}{%
\pgfpathmoveto{\pgfqpoint{-0.000000in}{0.000000in}}%
\pgfpathlineto{\pgfqpoint{-0.048611in}{0.000000in}}%
\pgfusepath{stroke,fill}%
}%
\begin{pgfscope}%
\pgfsys@transformshift{0.600000in}{1.062723in}%
\pgfsys@useobject{currentmarker}{}%
\end{pgfscope}%
\end{pgfscope}%
\begin{pgfscope}%
\definecolor{textcolor}{rgb}{0.000000,0.000000,0.000000}%
\pgfsetstrokecolor{textcolor}%
\pgfsetfillcolor{textcolor}%
\pgftext[x=0.363889in, y=1.014528in, left, base]{\color{textcolor}\rmfamily\fontsize{10.000000}{12.000000}\selectfont \(\displaystyle {50}\)}%
\end{pgfscope}%
\begin{pgfscope}%
\pgfsetbuttcap%
\pgfsetroundjoin%
\definecolor{currentfill}{rgb}{0.000000,0.000000,0.000000}%
\pgfsetfillcolor{currentfill}%
\pgfsetlinewidth{0.803000pt}%
\definecolor{currentstroke}{rgb}{0.000000,0.000000,0.000000}%
\pgfsetstrokecolor{currentstroke}%
\pgfsetdash{}{0pt}%
\pgfsys@defobject{currentmarker}{\pgfqpoint{-0.048611in}{0.000000in}}{\pgfqpoint{-0.000000in}{0.000000in}}{%
\pgfpathmoveto{\pgfqpoint{-0.000000in}{0.000000in}}%
\pgfpathlineto{\pgfqpoint{-0.048611in}{0.000000in}}%
\pgfusepath{stroke,fill}%
}%
\begin{pgfscope}%
\pgfsys@transformshift{0.600000in}{1.577123in}%
\pgfsys@useobject{currentmarker}{}%
\end{pgfscope}%
\end{pgfscope}%
\begin{pgfscope}%
\definecolor{textcolor}{rgb}{0.000000,0.000000,0.000000}%
\pgfsetstrokecolor{textcolor}%
\pgfsetfillcolor{textcolor}%
\pgftext[x=0.294444in, y=1.528928in, left, base]{\color{textcolor}\rmfamily\fontsize{10.000000}{12.000000}\selectfont \(\displaystyle {100}\)}%
\end{pgfscope}%
\begin{pgfscope}%
\definecolor{textcolor}{rgb}{0.000000,0.000000,0.000000}%
\pgfsetstrokecolor{textcolor}%
\pgfsetfillcolor{textcolor}%
\pgftext[x=0.238889in,y=1.076944in,,bottom,rotate=90.000000]{\color{textcolor}\rmfamily\fontsize{10.000000}{12.000000}\selectfont \(\displaystyle \Delta\)FP/\(\displaystyle \Delta\)TP}%
\end{pgfscope}%
\begin{pgfscope}%
\pgfpathrectangle{\pgfqpoint{0.600000in}{0.499444in}}{\pgfqpoint{1.550000in}{1.155000in}}%
\pgfusepath{clip}%
\pgfsetrectcap%
\pgfsetroundjoin%
\pgfsetlinewidth{1.505625pt}%
\definecolor{currentstroke}{rgb}{0.000000,0.000000,0.000000}%
\pgfsetstrokecolor{currentstroke}%
\pgfsetdash{}{0pt}%
\pgfpathmoveto{\pgfqpoint{0.670455in}{1.601944in}}%
\pgfpathlineto{\pgfqpoint{0.684688in}{1.573841in}}%
\pgfpathlineto{\pgfqpoint{0.698921in}{1.537066in}}%
\pgfpathlineto{\pgfqpoint{0.713155in}{1.503260in}}%
\pgfpathlineto{\pgfqpoint{0.727388in}{1.472551in}}%
\pgfpathlineto{\pgfqpoint{0.741621in}{1.442546in}}%
\pgfpathlineto{\pgfqpoint{0.755854in}{1.385587in}}%
\pgfpathlineto{\pgfqpoint{0.770088in}{1.329631in}}%
\pgfpathlineto{\pgfqpoint{0.784321in}{1.276183in}}%
\pgfpathlineto{\pgfqpoint{0.798554in}{1.214555in}}%
\pgfpathlineto{\pgfqpoint{0.812787in}{1.156260in}}%
\pgfpathlineto{\pgfqpoint{0.827021in}{1.105196in}}%
\pgfpathlineto{\pgfqpoint{0.841254in}{1.065195in}}%
\pgfpathlineto{\pgfqpoint{0.855487in}{1.027947in}}%
\pgfpathlineto{\pgfqpoint{0.869720in}{0.991293in}}%
\pgfpathlineto{\pgfqpoint{0.883954in}{0.957959in}}%
\pgfpathlineto{\pgfqpoint{0.898187in}{0.928478in}}%
\pgfpathlineto{\pgfqpoint{0.912420in}{0.900936in}}%
\pgfpathlineto{\pgfqpoint{0.926653in}{0.877049in}}%
\pgfpathlineto{\pgfqpoint{0.940887in}{0.857451in}}%
\pgfpathlineto{\pgfqpoint{0.955120in}{0.838696in}}%
\pgfpathlineto{\pgfqpoint{0.969353in}{0.821290in}}%
\pgfpathlineto{\pgfqpoint{0.983586in}{0.803916in}}%
\pgfpathlineto{\pgfqpoint{0.997820in}{0.787307in}}%
\pgfpathlineto{\pgfqpoint{1.012053in}{0.773108in}}%
\pgfpathlineto{\pgfqpoint{1.026286in}{0.760037in}}%
\pgfpathlineto{\pgfqpoint{1.040519in}{0.747762in}}%
\pgfpathlineto{\pgfqpoint{1.054752in}{0.736533in}}%
\pgfpathlineto{\pgfqpoint{1.068986in}{0.725560in}}%
\pgfpathlineto{\pgfqpoint{1.083219in}{0.715394in}}%
\pgfpathlineto{\pgfqpoint{1.097452in}{0.706082in}}%
\pgfpathlineto{\pgfqpoint{1.111685in}{0.697385in}}%
\pgfpathlineto{\pgfqpoint{1.125919in}{0.688967in}}%
\pgfpathlineto{\pgfqpoint{1.140152in}{0.681563in}}%
\pgfpathlineto{\pgfqpoint{1.154385in}{0.674752in}}%
\pgfpathlineto{\pgfqpoint{1.168618in}{0.668287in}}%
\pgfpathlineto{\pgfqpoint{1.182852in}{0.662451in}}%
\pgfpathlineto{\pgfqpoint{1.197085in}{0.657085in}}%
\pgfpathlineto{\pgfqpoint{1.211318in}{0.651784in}}%
\pgfpathlineto{\pgfqpoint{1.225551in}{0.646742in}}%
\pgfpathlineto{\pgfqpoint{1.239785in}{0.642096in}}%
\pgfpathlineto{\pgfqpoint{1.254018in}{0.637795in}}%
\pgfpathlineto{\pgfqpoint{1.268251in}{0.633598in}}%
\pgfpathlineto{\pgfqpoint{1.282484in}{0.629491in}}%
\pgfpathlineto{\pgfqpoint{1.296718in}{0.625387in}}%
\pgfpathlineto{\pgfqpoint{1.310951in}{0.621674in}}%
\pgfpathlineto{\pgfqpoint{1.325184in}{0.617966in}}%
\pgfpathlineto{\pgfqpoint{1.339417in}{0.614407in}}%
\pgfpathlineto{\pgfqpoint{1.353651in}{0.611304in}}%
\pgfpathlineto{\pgfqpoint{1.367884in}{0.608276in}}%
\pgfpathlineto{\pgfqpoint{1.382117in}{0.605388in}}%
\pgfpathlineto{\pgfqpoint{1.396350in}{0.602558in}}%
\pgfpathlineto{\pgfqpoint{1.410584in}{0.599809in}}%
\pgfpathlineto{\pgfqpoint{1.424817in}{0.597192in}}%
\pgfpathlineto{\pgfqpoint{1.439050in}{0.594711in}}%
\pgfpathlineto{\pgfqpoint{1.453283in}{0.592287in}}%
\pgfpathlineto{\pgfqpoint{1.467516in}{0.590063in}}%
\pgfpathlineto{\pgfqpoint{1.481750in}{0.587929in}}%
\pgfpathlineto{\pgfqpoint{1.495983in}{0.585888in}}%
\pgfpathlineto{\pgfqpoint{1.510216in}{0.583922in}}%
\pgfpathlineto{\pgfqpoint{1.524449in}{0.582110in}}%
\pgfpathlineto{\pgfqpoint{1.538683in}{0.580377in}}%
\pgfpathlineto{\pgfqpoint{1.552916in}{0.578720in}}%
\pgfpathlineto{\pgfqpoint{1.567149in}{0.577155in}}%
\pgfpathlineto{\pgfqpoint{1.581382in}{0.575712in}}%
\pgfpathlineto{\pgfqpoint{1.595616in}{0.574331in}}%
\pgfpathlineto{\pgfqpoint{1.609849in}{0.572991in}}%
\pgfpathlineto{\pgfqpoint{1.624082in}{0.571708in}}%
\pgfpathlineto{\pgfqpoint{1.638315in}{0.570431in}}%
\pgfpathlineto{\pgfqpoint{1.652549in}{0.569263in}}%
\pgfpathlineto{\pgfqpoint{1.666782in}{0.568089in}}%
\pgfpathlineto{\pgfqpoint{1.681015in}{0.567022in}}%
\pgfpathlineto{\pgfqpoint{1.695248in}{0.565952in}}%
\pgfpathlineto{\pgfqpoint{1.709482in}{0.564947in}}%
\pgfpathlineto{\pgfqpoint{1.723715in}{0.563971in}}%
\pgfpathlineto{\pgfqpoint{1.737948in}{0.563060in}}%
\pgfpathlineto{\pgfqpoint{1.752181in}{0.562168in}}%
\pgfpathlineto{\pgfqpoint{1.766415in}{0.561351in}}%
\pgfpathlineto{\pgfqpoint{1.780648in}{0.560588in}}%
\pgfpathlineto{\pgfqpoint{1.794881in}{0.559844in}}%
\pgfpathlineto{\pgfqpoint{1.809114in}{0.559139in}}%
\pgfpathlineto{\pgfqpoint{1.823348in}{0.558432in}}%
\pgfpathlineto{\pgfqpoint{1.837581in}{0.557757in}}%
\pgfpathlineto{\pgfqpoint{1.851814in}{0.557106in}}%
\pgfpathlineto{\pgfqpoint{1.866047in}{0.556494in}}%
\pgfpathlineto{\pgfqpoint{1.880280in}{0.555914in}}%
\pgfpathlineto{\pgfqpoint{1.894514in}{0.555401in}}%
\pgfpathlineto{\pgfqpoint{1.908747in}{0.554906in}}%
\pgfpathlineto{\pgfqpoint{1.922980in}{0.554447in}}%
\pgfpathlineto{\pgfqpoint{1.937213in}{0.554030in}}%
\pgfpathlineto{\pgfqpoint{1.951447in}{0.553667in}}%
\pgfpathlineto{\pgfqpoint{1.965680in}{0.553351in}}%
\pgfpathlineto{\pgfqpoint{1.979913in}{0.553052in}}%
\pgfpathlineto{\pgfqpoint{1.994146in}{0.552784in}}%
\pgfpathlineto{\pgfqpoint{2.008380in}{0.552548in}}%
\pgfpathlineto{\pgfqpoint{2.022613in}{0.552337in}}%
\pgfpathlineto{\pgfqpoint{2.036846in}{0.552218in}}%
\pgfpathlineto{\pgfqpoint{2.051079in}{0.552112in}}%
\pgfpathlineto{\pgfqpoint{2.065313in}{0.552020in}}%
\pgfpathlineto{\pgfqpoint{2.079546in}{0.551944in}}%
\pgfusepath{stroke}%
\end{pgfscope}%
\begin{pgfscope}%
\pgfpathrectangle{\pgfqpoint{0.600000in}{0.499444in}}{\pgfqpoint{1.550000in}{1.155000in}}%
\pgfusepath{clip}%
\pgfsetbuttcap%
\pgfsetroundjoin%
\pgfsetlinewidth{1.505625pt}%
\definecolor{currentstroke}{rgb}{0.000000,0.000000,0.000000}%
\pgfsetstrokecolor{currentstroke}%
\pgfsetdash{{5.550000pt}{2.400000pt}}{0.000000pt}%
\pgfpathmoveto{\pgfqpoint{0.600000in}{0.568899in}}%
\pgfpathlineto{\pgfqpoint{2.150000in}{0.568899in}}%
\pgfusepath{stroke}%
\end{pgfscope}%
\begin{pgfscope}%
\pgfpathrectangle{\pgfqpoint{0.600000in}{0.499444in}}{\pgfqpoint{1.550000in}{1.155000in}}%
\pgfusepath{clip}%
\pgfsetrectcap%
\pgfsetroundjoin%
\pgfsetlinewidth{1.505625pt}%
\definecolor{currentstroke}{rgb}{0.121569,0.466667,0.705882}%
\pgfsetstrokecolor{currentstroke}%
\pgfsetdash{}{0pt}%
\pgfpathmoveto{\pgfqpoint{1.652549in}{0.568899in}}%
\pgfusepath{stroke}%
\end{pgfscope}%
\begin{pgfscope}%
\pgfpathrectangle{\pgfqpoint{0.600000in}{0.499444in}}{\pgfqpoint{1.550000in}{1.155000in}}%
\pgfusepath{clip}%
\pgfsetbuttcap%
\pgfsetroundjoin%
\definecolor{currentfill}{rgb}{0.000000,0.000000,0.000000}%
\pgfsetfillcolor{currentfill}%
\pgfsetlinewidth{1.003750pt}%
\definecolor{currentstroke}{rgb}{0.000000,0.000000,0.000000}%
\pgfsetstrokecolor{currentstroke}%
\pgfsetdash{}{0pt}%
\pgfsys@defobject{currentmarker}{\pgfqpoint{-0.041667in}{-0.041667in}}{\pgfqpoint{0.041667in}{0.041667in}}{%
\pgfpathmoveto{\pgfqpoint{0.000000in}{-0.041667in}}%
\pgfpathcurveto{\pgfqpoint{0.011050in}{-0.041667in}}{\pgfqpoint{0.021649in}{-0.037276in}}{\pgfqpoint{0.029463in}{-0.029463in}}%
\pgfpathcurveto{\pgfqpoint{0.037276in}{-0.021649in}}{\pgfqpoint{0.041667in}{-0.011050in}}{\pgfqpoint{0.041667in}{0.000000in}}%
\pgfpathcurveto{\pgfqpoint{0.041667in}{0.011050in}}{\pgfqpoint{0.037276in}{0.021649in}}{\pgfqpoint{0.029463in}{0.029463in}}%
\pgfpathcurveto{\pgfqpoint{0.021649in}{0.037276in}}{\pgfqpoint{0.011050in}{0.041667in}}{\pgfqpoint{0.000000in}{0.041667in}}%
\pgfpathcurveto{\pgfqpoint{-0.011050in}{0.041667in}}{\pgfqpoint{-0.021649in}{0.037276in}}{\pgfqpoint{-0.029463in}{0.029463in}}%
\pgfpathcurveto{\pgfqpoint{-0.037276in}{0.021649in}}{\pgfqpoint{-0.041667in}{0.011050in}}{\pgfqpoint{-0.041667in}{0.000000in}}%
\pgfpathcurveto{\pgfqpoint{-0.041667in}{-0.011050in}}{\pgfqpoint{-0.037276in}{-0.021649in}}{\pgfqpoint{-0.029463in}{-0.029463in}}%
\pgfpathcurveto{\pgfqpoint{-0.021649in}{-0.037276in}}{\pgfqpoint{-0.011050in}{-0.041667in}}{\pgfqpoint{0.000000in}{-0.041667in}}%
\pgfpathlineto{\pgfqpoint{0.000000in}{-0.041667in}}%
\pgfpathclose%
\pgfusepath{stroke,fill}%
}%
\begin{pgfscope}%
\pgfsys@transformshift{1.652549in}{0.568899in}%
\pgfsys@useobject{currentmarker}{}%
\end{pgfscope}%
\end{pgfscope}%
\begin{pgfscope}%
\pgfsetrectcap%
\pgfsetmiterjoin%
\pgfsetlinewidth{0.803000pt}%
\definecolor{currentstroke}{rgb}{0.000000,0.000000,0.000000}%
\pgfsetstrokecolor{currentstroke}%
\pgfsetdash{}{0pt}%
\pgfpathmoveto{\pgfqpoint{0.600000in}{0.499444in}}%
\pgfpathlineto{\pgfqpoint{0.600000in}{1.654444in}}%
\pgfusepath{stroke}%
\end{pgfscope}%
\begin{pgfscope}%
\pgfsetrectcap%
\pgfsetmiterjoin%
\pgfsetlinewidth{0.803000pt}%
\definecolor{currentstroke}{rgb}{0.000000,0.000000,0.000000}%
\pgfsetstrokecolor{currentstroke}%
\pgfsetdash{}{0pt}%
\pgfpathmoveto{\pgfqpoint{2.150000in}{0.499444in}}%
\pgfpathlineto{\pgfqpoint{2.150000in}{1.654444in}}%
\pgfusepath{stroke}%
\end{pgfscope}%
\begin{pgfscope}%
\pgfsetrectcap%
\pgfsetmiterjoin%
\pgfsetlinewidth{0.803000pt}%
\definecolor{currentstroke}{rgb}{0.000000,0.000000,0.000000}%
\pgfsetstrokecolor{currentstroke}%
\pgfsetdash{}{0pt}%
\pgfpathmoveto{\pgfqpoint{0.600000in}{0.499444in}}%
\pgfpathlineto{\pgfqpoint{2.150000in}{0.499444in}}%
\pgfusepath{stroke}%
\end{pgfscope}%
\begin{pgfscope}%
\pgfsetrectcap%
\pgfsetmiterjoin%
\pgfsetlinewidth{0.803000pt}%
\definecolor{currentstroke}{rgb}{0.000000,0.000000,0.000000}%
\pgfsetstrokecolor{currentstroke}%
\pgfsetdash{}{0pt}%
\pgfpathmoveto{\pgfqpoint{0.600000in}{1.654444in}}%
\pgfpathlineto{\pgfqpoint{2.150000in}{1.654444in}}%
\pgfusepath{stroke}%
\end{pgfscope}%
\begin{pgfscope}%
\pgfsetbuttcap%
\pgfsetmiterjoin%
\definecolor{currentfill}{rgb}{1.000000,1.000000,1.000000}%
\pgfsetfillcolor{currentfill}%
\pgfsetfillopacity{0.800000}%
\pgfsetlinewidth{1.003750pt}%
\definecolor{currentstroke}{rgb}{0.800000,0.800000,0.800000}%
\pgfsetstrokecolor{currentstroke}%
\pgfsetstrokeopacity{0.800000}%
\pgfsetdash{}{0pt}%
\pgfpathmoveto{\pgfqpoint{0.881432in}{1.126667in}}%
\pgfpathlineto{\pgfqpoint{2.052778in}{1.126667in}}%
\pgfpathquadraticcurveto{\pgfqpoint{2.080556in}{1.126667in}}{\pgfqpoint{2.080556in}{1.154444in}}%
\pgfpathlineto{\pgfqpoint{2.080556in}{1.557222in}}%
\pgfpathquadraticcurveto{\pgfqpoint{2.080556in}{1.585000in}}{\pgfqpoint{2.052778in}{1.585000in}}%
\pgfpathlineto{\pgfqpoint{0.881432in}{1.585000in}}%
\pgfpathquadraticcurveto{\pgfqpoint{0.853654in}{1.585000in}}{\pgfqpoint{0.853654in}{1.557222in}}%
\pgfpathlineto{\pgfqpoint{0.853654in}{1.154444in}}%
\pgfpathquadraticcurveto{\pgfqpoint{0.853654in}{1.126667in}}{\pgfqpoint{0.881432in}{1.126667in}}%
\pgfpathlineto{\pgfqpoint{0.881432in}{1.126667in}}%
\pgfpathclose%
\pgfusepath{stroke,fill}%
\end{pgfscope}%
\begin{pgfscope}%
\pgfsetrectcap%
\pgfsetroundjoin%
\pgfsetlinewidth{1.505625pt}%
\definecolor{currentstroke}{rgb}{0.000000,0.000000,0.000000}%
\pgfsetstrokecolor{currentstroke}%
\pgfsetdash{}{0pt}%
\pgfpathmoveto{\pgfqpoint{0.909210in}{1.473889in}}%
\pgfpathlineto{\pgfqpoint{1.048099in}{1.473889in}}%
\pgfpathlineto{\pgfqpoint{1.186988in}{1.473889in}}%
\pgfusepath{stroke}%
\end{pgfscope}%
\begin{pgfscope}%
\definecolor{textcolor}{rgb}{0.000000,0.000000,0.000000}%
\pgfsetstrokecolor{textcolor}%
\pgfsetfillcolor{textcolor}%
\pgftext[x=1.298099in,y=1.425277in,left,base]{\color{textcolor}\rmfamily\fontsize{10.000000}{12.000000}\selectfont \(\displaystyle \Delta FP/\Delta TP\)}%
\end{pgfscope}%
\begin{pgfscope}%
\pgfsetrectcap%
\pgfsetroundjoin%
\pgfsetlinewidth{1.505625pt}%
\definecolor{currentstroke}{rgb}{0.121569,0.466667,0.705882}%
\pgfsetstrokecolor{currentstroke}%
\pgfsetdash{}{0pt}%
\pgfpathmoveto{\pgfqpoint{0.909210in}{1.265555in}}%
\pgfpathlineto{\pgfqpoint{1.048099in}{1.265555in}}%
\pgfpathlineto{\pgfqpoint{1.186988in}{1.265555in}}%
\pgfusepath{stroke}%
\end{pgfscope}%
\begin{pgfscope}%
\pgfsetbuttcap%
\pgfsetroundjoin%
\definecolor{currentfill}{rgb}{0.000000,0.000000,0.000000}%
\pgfsetfillcolor{currentfill}%
\pgfsetlinewidth{1.003750pt}%
\definecolor{currentstroke}{rgb}{0.000000,0.000000,0.000000}%
\pgfsetstrokecolor{currentstroke}%
\pgfsetdash{}{0pt}%
\pgfsys@defobject{currentmarker}{\pgfqpoint{-0.041667in}{-0.041667in}}{\pgfqpoint{0.041667in}{0.041667in}}{%
\pgfpathmoveto{\pgfqpoint{0.000000in}{-0.041667in}}%
\pgfpathcurveto{\pgfqpoint{0.011050in}{-0.041667in}}{\pgfqpoint{0.021649in}{-0.037276in}}{\pgfqpoint{0.029463in}{-0.029463in}}%
\pgfpathcurveto{\pgfqpoint{0.037276in}{-0.021649in}}{\pgfqpoint{0.041667in}{-0.011050in}}{\pgfqpoint{0.041667in}{0.000000in}}%
\pgfpathcurveto{\pgfqpoint{0.041667in}{0.011050in}}{\pgfqpoint{0.037276in}{0.021649in}}{\pgfqpoint{0.029463in}{0.029463in}}%
\pgfpathcurveto{\pgfqpoint{0.021649in}{0.037276in}}{\pgfqpoint{0.011050in}{0.041667in}}{\pgfqpoint{0.000000in}{0.041667in}}%
\pgfpathcurveto{\pgfqpoint{-0.011050in}{0.041667in}}{\pgfqpoint{-0.021649in}{0.037276in}}{\pgfqpoint{-0.029463in}{0.029463in}}%
\pgfpathcurveto{\pgfqpoint{-0.037276in}{0.021649in}}{\pgfqpoint{-0.041667in}{0.011050in}}{\pgfqpoint{-0.041667in}{0.000000in}}%
\pgfpathcurveto{\pgfqpoint{-0.041667in}{-0.011050in}}{\pgfqpoint{-0.037276in}{-0.021649in}}{\pgfqpoint{-0.029463in}{-0.029463in}}%
\pgfpathcurveto{\pgfqpoint{-0.021649in}{-0.037276in}}{\pgfqpoint{-0.011050in}{-0.041667in}}{\pgfqpoint{0.000000in}{-0.041667in}}%
\pgfpathlineto{\pgfqpoint{0.000000in}{-0.041667in}}%
\pgfpathclose%
\pgfusepath{stroke,fill}%
}%
\begin{pgfscope}%
\pgfsys@transformshift{1.048099in}{1.265555in}%
\pgfsys@useobject{currentmarker}{}%
\end{pgfscope}%
\end{pgfscope}%
\begin{pgfscope}%
\definecolor{textcolor}{rgb}{0.000000,0.000000,0.000000}%
\pgfsetstrokecolor{textcolor}%
\pgfsetfillcolor{textcolor}%
\pgftext[x=1.298099in,y=1.216944in,left,base]{\color{textcolor}\rmfamily\fontsize{10.000000}{12.000000}\selectfont (0.685,2)}%
\end{pgfscope}%
\end{pgfpicture}%
\makeatother%
\endgroup%
	

The ``TN,'' ``FP,'' ``FN,'' and ``TP'' columns are the confusion matrix if we choose $\theta$ in that range of $p$.  

	\item The seventh column, ``FP/TP'' is related, but it's the total cost in terms of how many unneeded ambulances we'd send for each one we needed if $\theta$ were that value of $p$.  
	
	\item The last three columns are Precision, Recall, and $\hat{p}$ (If I understand $\hat{p}$ correctly).
\end{itemize}



\begin{center}
\begin{tabular}{rrrrrrrrrrrrrr}
\toprule
{} &     Neg &    Pos & mPrec &       TN &       FP &      FN &      TP &  Prec &   Rec & $\hat{p}$ \\
p    &         &        &       &          &          &         &         &       &       &           \\
\midrule
0.00 &     107 &      0 &  0.00 &      107 &  180,138 &       0 &  33,825 &  0.16 &  1.00 &      1.00 \\
0.05 &   2,287 &     20 &  0.01 &    2,394 &  177,851 &      20 &  33,805 &  0.16 &  1.00 &      0.99 \\
0.10 &   6,177 &     62 &  0.01 &    8,571 &  171,674 &      82 &  33,743 &  0.16 &  1.00 &      0.96 \\
0.15 &  10,435 &    188 &  0.02 &   19,006 &  161,239 &     270 &  33,555 &  0.17 &  0.99 &      0.91 \\
0.20 &  13,424 &    388 &  0.03 &   32,430 &  147,815 &     658 &  33,167 &  0.18 &  0.98 &      0.85 \\
0.25 &  15,746 &    619 &  0.04 &   48,176 &  132,069 &   1,277 &  32,548 &  0.20 &  0.96 &      0.77 \\
0.30 &  17,256 &    923 &  0.05 &   65,432 &  114,813 &   2,200 &  31,625 &  0.22 &  0.93 &      0.68 \\
0.35 &  17,841 &  1,320 &  0.07 &   83,273 &   96,972 &   3,520 &  30,305 &  0.24 &  0.90 &      0.59 \\
0.40 &  17,355 &  1,690 &  0.09 &  100,628 &   79,617 &   5,210 &  28,615 &  0.26 &  0.85 &      0.51 \\
0.45 &  16,597 &  2,042 &  0.11 &  117,225 &   63,020 &   7,252 &  26,573 &  0.30 &  0.79 &      0.42 \\
0.50 &  14,984 &  2,470 &  0.14 &  132,209 &   48,036 &   9,722 &  24,103 &  0.33 &  0.71 &      0.34 \\
0.55 &  12,810 &  2,725 &  0.18 &  145,019 &   35,226 &  12,447 &  21,378 &  0.38 &  0.63 &      0.26 \\
0.60 &  10,493 &  2,972 &  0.22 &  155,512 &   24,733 &  15,419 &  18,406 &  0.43 &  0.54 &      0.20 \\
0.65 &   8,062 &  3,037 &  0.27 &  163,574 &   16,671 &  18,456 &  15,369 &  0.48 &  0.45 &      0.15 \\
0.70 &   6,040 &  2,953 &  0.33 &  169,614 &   10,631 &  21,409 &  12,416 &  0.54 &  0.37 &      0.11 \\
0.75 &   4,144 &  2,893 &  0.41 &  173,758 &    6,487 &  24,302 &   9,523 &  0.59 &  0.28 &      0.07 \\
0.80 &   2,902 &  2,627 &  0.48 &  176,660 &    3,585 &  26,929 &   6,896 &  0.66 &  0.20 &      0.05 \\
0.85 &   1,801 &  2,530 &  0.58 &  178,461 &    1,784 &  29,459 &   4,366 &  0.71 &  0.13 &      0.03 \\
0.90 &   1,043 &  2,166 &  0.67 &  179,504 &      741 &  31,625 &   2,200 &  0.75 &  0.07 &      0.01 \\
0.95 &     587 &  1,597 &  0.73 &  180,091 &      154 &  33,222 &     603 &  0.80 &  0.02 &      0.00 \\
1.00 &     154 &    603 &  0.80 &  180,245 &        0 &  33,825 &       0 &   nan &  0.00 &      0.00 \\
\bottomrule
\end{tabular}
}
\end{center}

Here I've zoomed in on three smaller ranges.  Interestingly, it's not useful to zoom in further, because most values of $p$ given by this implementation of the classifier are only given to two digits.  

\begin{center}
\begin{tabular}{rrrrrrrrrrrrrr}
\toprule
{} &    Neg &  Pos & Neg/Pos &       TN &       FP &      FN &      TP & FP/TP & Prec. &  Rec. & $\hat{p}$ \\
p    &        &      &         &          &          &         &         &       &       &       &           \\
\midrule
0.48 &  2,990 &  453 &    6.60 &  126,315 &   53,930 &   8,688 &  25,137 &  2.15 &  0.32 &  0.74 &      0.37 \\
0.49 &  3,020 &  533 &    5.67 &  129,335 &   50,910 &   9,221 &  24,604 &  2.07 &  0.33 &  0.73 &      0.35 \\
0.50 &  2,874 &  501 &    5.74 &  132,209 &   48,036 &   9,722 &  24,103 &  1.99 &  0.33 &  0.71 &      0.34 \\
0.51 &  2,804 &  533 &    5.26 &  135,013 &   45,232 &  10,255 &  23,570 &  1.92 &  0.34 &  0.70 &      0.32 \\
\hline
0.64 &  1,582 &  586 &    2.70 &  162,135 &   18,110 &  17,838 &  15,987 &  1.13 &  0.47 &  0.47 &      0.16 \\
0.65 &  1,439 &  618 &    2.33 &  163,574 &   16,671 &  18,456 &  15,369 &  1.08 &  0.48 &  0.45 &      0.15 \\
0.66 &  1,376 &  561 &    2.45 &  164,950 &   15,295 &  19,017 &  14,808 &  1.03 &  0.49 &  0.44 &      0.14 \\
0.67 &  1,288 &  637 &    2.02 &  166,238 &   14,007 &  19,654 &  14,171 &  0.99 &  0.50 &  0.42 &      0.13 \\
0.68 &  1,241 &  554 &    2.24 &  167,479 &   12,766 &  20,208 &  13,617 &  0.94 &  0.52 &  0.40 &      0.12 \\
0.69 &  1,082 &  631 &    1.71 &  168,561 &   11,684 &  20,839 &  12,986 &  0.90 &  0.53 &  0.38 &      0.12 \\
0.70 &  1,053 &  570 &    1.85 &  169,614 &   10,631 &  21,409 &  12,416 &  0.86 &  0.54 &  0.37 &      0.11 \\
0.71 &    922 &  587 &    1.57 &  170,536 &    9,709 &  21,996 &  11,829 &  0.82 &  0.55 &  0.35 &      0.10 \\
\hline
0.76 &    664 &  558 &    1.19 &  174,422 &    5,823 &  24,860 &   8,965 &  0.65 &  0.61 &  0.27 &      0.07 \\
0.77 &    627 &  524 &    1.20 &  175,049 &    5,196 &  25,384 &   8,441 &  0.62 &  0.62 &  0.25 &      0.06 \\
0.78 &    585 &  532 &    1.10 &  175,634 &    4,611 &  25,916 &   7,909 &  0.58 &  0.63 &  0.23 &      0.06 \\
0.79 &    568 &  529 &    1.07 &  176,202 &    4,043 &  26,445 &   7,380 &  0.55 &  0.65 &  0.22 &      0.05 \\
0.80 &    458 &  484 &    0.95 &  176,660 &    3,585 &  26,929 &   6,896 &  0.52 &  0.66 &  0.20 &      0.05 \\
0.81 &    429 &  514 &    0.83 &  177,089 &    3,156 &  27,443 &   6,382 &  0.49 &  0.67 &  0.19 &      0.04 \\
0.82 &    399 &  535 &    0.75 &  177,488 &    2,757 &  27,978 &   5,847 &  0.47 &  0.68 &  0.17 &      0.04 \\
\bottomrule
\end{tabular}
\end{center}

How shall we choose $\theta$, the discrimination threshold?  

For example, if we choose the default $\theta = 0.50$, we would send ambulances to 34\% of the automatically reported crashes.  The total cost is 1.99 unneeded ambulances per needed ambulance.  The marginal cost, the difference between making $\theta = 0.50$ and making $\theta = 0.51$, is over 5 (5.74) unneeded ambulances per needed ambulance.  We would be sending an ambulance to each crash with at least a $1/(5.74+1) \approx 15\%$ chance of needing an ambulance.  

One goal of my analysis is figuring out how to choose $\theta$ given some marginal probability that an ambulance is needed, given explicitly or implicitly by the people funding the emergency services.  I think this is actually how the decision is likely to be made by the politicians:  We're willing to send an ambulance early (before an eyewitness report) if there is some probability that it's needed.  

The option I'm exploring is sending an ambulance when there's at least a 33\% chance it will be needed, which happens when Neg/Pos = 2, at about $\theta = 0.68$.  The total cost would be 0.99 unneeded ambulance, and we would be sending an ambulance to 13.16\% of the crashes.  

If we wanted there to be at least a 50\% chance that an ambulance is needed, then we would choose $\theta = 0.80$, where Neg/Pos = 1.  

If we wanted $\hat{p} \approx \pi_1$, then we would choose $\theta = 0.65$.  

\

Does that decision-making method make sense?

\subsection{Other Questions:  Dataset}

The CRSS dataset intentionally over represents more serious crashes; in all police-reported crashes Class 1 is much smaller (2-3\%), but it is also true that very minor crashes (parking lot fender benders) have a similar deceleration profile to hard braking, so minor crashes are less likely to be detected by the phone.   Also, the automated report only goes to the emergency dispatcher if the phone's owner does not respond promptly to the phone.  So we are going to wave our hands and say that the CRSS data is the best approximation we have to the set of crashes reported by automated cell phone reports.  Does that approach seem reasonable?

%%%%%
\subsection{Other Questions:  Validation Set}

I've done my work so far with the data split 70/30 into training and test, and then I did that twice, splitting with a different random seed, to compare results and see whether the differences were within randomness.  Should I do a 60/20/20 training/validation/test set, or use 5-fold cross-validation on the training set?  Or does it matter?  

\newpage
\subsection{Other Questions:  Class Weights}

Below are the raw model outputs \verb|y_proba| for the (neural network) Keras Binary Crossentropy Classifier with three different class weights $\alpha$.  

I had thought the point of class weights was that class weights would put more weight on the misclassified elements of the positive class and make the algorithm would do a better job of separating the two classes.  Using the area under the ROC curve (AUC) as a measure of how well the model separates the classes, the difference between these three models with different class weights is within randomness.  It seems that raising the class weight just pushes both classes together to the right with no useful effect.  Do class weights basically have the same effect as shifting the decision threshold $\theta$?

On the next page I show that if you linearly transform the $p$ values so that $p=0.5$ where $\Delta FP/\Delta TP = 2.0$, then you get nearly the same confusion matrix.  The AUC is invariant under the transformation.  

\
	
Model 1:  $\alpha = 0.5$ for no class weights

\noindent\begin{tabular}{@{\hspace{-6pt}}p{4.5in} @{\hspace{-6pt}}p{2.0in}}
	\vskip 0pt
	\qquad \qquad Raw Model Output
	
	%% Creator: Matplotlib, PGF backend
%%
%% To include the figure in your LaTeX document, write
%%   \input{<filename>.pgf}
%%
%% Make sure the required packages are loaded in your preamble
%%   \usepackage{pgf}
%%
%% Also ensure that all the required font packages are loaded; for instance,
%% the lmodern package is sometimes necessary when using math font.
%%   \usepackage{lmodern}
%%
%% Figures using additional raster images can only be included by \input if
%% they are in the same directory as the main LaTeX file. For loading figures
%% from other directories you can use the `import` package
%%   \usepackage{import}
%%
%% and then include the figures with
%%   \import{<path to file>}{<filename>.pgf}
%%
%% Matplotlib used the following preamble
%%   
%%   \usepackage{fontspec}
%%   \makeatletter\@ifpackageloaded{underscore}{}{\usepackage[strings]{underscore}}\makeatother
%%
\begingroup%
\makeatletter%
\begin{pgfpicture}%
\pgfpathrectangle{\pgfpointorigin}{\pgfqpoint{4.102500in}{1.754444in}}%
\pgfusepath{use as bounding box, clip}%
\begin{pgfscope}%
\pgfsetbuttcap%
\pgfsetmiterjoin%
\definecolor{currentfill}{rgb}{1.000000,1.000000,1.000000}%
\pgfsetfillcolor{currentfill}%
\pgfsetlinewidth{0.000000pt}%
\definecolor{currentstroke}{rgb}{1.000000,1.000000,1.000000}%
\pgfsetstrokecolor{currentstroke}%
\pgfsetdash{}{0pt}%
\pgfpathmoveto{\pgfqpoint{0.000000in}{0.000000in}}%
\pgfpathlineto{\pgfqpoint{4.102500in}{0.000000in}}%
\pgfpathlineto{\pgfqpoint{4.102500in}{1.754444in}}%
\pgfpathlineto{\pgfqpoint{0.000000in}{1.754444in}}%
\pgfpathlineto{\pgfqpoint{0.000000in}{0.000000in}}%
\pgfpathclose%
\pgfusepath{fill}%
\end{pgfscope}%
\begin{pgfscope}%
\pgfsetbuttcap%
\pgfsetmiterjoin%
\definecolor{currentfill}{rgb}{1.000000,1.000000,1.000000}%
\pgfsetfillcolor{currentfill}%
\pgfsetlinewidth{0.000000pt}%
\definecolor{currentstroke}{rgb}{0.000000,0.000000,0.000000}%
\pgfsetstrokecolor{currentstroke}%
\pgfsetstrokeopacity{0.000000}%
\pgfsetdash{}{0pt}%
\pgfpathmoveto{\pgfqpoint{0.515000in}{0.499444in}}%
\pgfpathlineto{\pgfqpoint{4.002500in}{0.499444in}}%
\pgfpathlineto{\pgfqpoint{4.002500in}{1.654444in}}%
\pgfpathlineto{\pgfqpoint{0.515000in}{1.654444in}}%
\pgfpathlineto{\pgfqpoint{0.515000in}{0.499444in}}%
\pgfpathclose%
\pgfusepath{fill}%
\end{pgfscope}%
\begin{pgfscope}%
\pgfpathrectangle{\pgfqpoint{0.515000in}{0.499444in}}{\pgfqpoint{3.487500in}{1.155000in}}%
\pgfusepath{clip}%
\pgfsetbuttcap%
\pgfsetmiterjoin%
\pgfsetlinewidth{1.003750pt}%
\definecolor{currentstroke}{rgb}{0.000000,0.000000,0.000000}%
\pgfsetstrokecolor{currentstroke}%
\pgfsetdash{}{0pt}%
\pgfpathmoveto{\pgfqpoint{0.610114in}{0.499444in}}%
\pgfpathlineto{\pgfqpoint{0.673523in}{0.499444in}}%
\pgfpathlineto{\pgfqpoint{0.673523in}{0.499444in}}%
\pgfpathlineto{\pgfqpoint{0.610114in}{0.499444in}}%
\pgfpathlineto{\pgfqpoint{0.610114in}{0.499444in}}%
\pgfpathclose%
\pgfusepath{stroke}%
\end{pgfscope}%
\begin{pgfscope}%
\pgfpathrectangle{\pgfqpoint{0.515000in}{0.499444in}}{\pgfqpoint{3.487500in}{1.155000in}}%
\pgfusepath{clip}%
\pgfsetbuttcap%
\pgfsetmiterjoin%
\pgfsetlinewidth{1.003750pt}%
\definecolor{currentstroke}{rgb}{0.000000,0.000000,0.000000}%
\pgfsetstrokecolor{currentstroke}%
\pgfsetdash{}{0pt}%
\pgfpathmoveto{\pgfqpoint{0.768637in}{0.499444in}}%
\pgfpathlineto{\pgfqpoint{0.832046in}{0.499444in}}%
\pgfpathlineto{\pgfqpoint{0.832046in}{1.599444in}}%
\pgfpathlineto{\pgfqpoint{0.768637in}{1.599444in}}%
\pgfpathlineto{\pgfqpoint{0.768637in}{0.499444in}}%
\pgfpathclose%
\pgfusepath{stroke}%
\end{pgfscope}%
\begin{pgfscope}%
\pgfpathrectangle{\pgfqpoint{0.515000in}{0.499444in}}{\pgfqpoint{3.487500in}{1.155000in}}%
\pgfusepath{clip}%
\pgfsetbuttcap%
\pgfsetmiterjoin%
\pgfsetlinewidth{1.003750pt}%
\definecolor{currentstroke}{rgb}{0.000000,0.000000,0.000000}%
\pgfsetstrokecolor{currentstroke}%
\pgfsetdash{}{0pt}%
\pgfpathmoveto{\pgfqpoint{0.927159in}{0.499444in}}%
\pgfpathlineto{\pgfqpoint{0.990568in}{0.499444in}}%
\pgfpathlineto{\pgfqpoint{0.990568in}{1.346951in}}%
\pgfpathlineto{\pgfqpoint{0.927159in}{1.346951in}}%
\pgfpathlineto{\pgfqpoint{0.927159in}{0.499444in}}%
\pgfpathclose%
\pgfusepath{stroke}%
\end{pgfscope}%
\begin{pgfscope}%
\pgfpathrectangle{\pgfqpoint{0.515000in}{0.499444in}}{\pgfqpoint{3.487500in}{1.155000in}}%
\pgfusepath{clip}%
\pgfsetbuttcap%
\pgfsetmiterjoin%
\pgfsetlinewidth{1.003750pt}%
\definecolor{currentstroke}{rgb}{0.000000,0.000000,0.000000}%
\pgfsetstrokecolor{currentstroke}%
\pgfsetdash{}{0pt}%
\pgfpathmoveto{\pgfqpoint{1.085682in}{0.499444in}}%
\pgfpathlineto{\pgfqpoint{1.149091in}{0.499444in}}%
\pgfpathlineto{\pgfqpoint{1.149091in}{1.032251in}}%
\pgfpathlineto{\pgfqpoint{1.085682in}{1.032251in}}%
\pgfpathlineto{\pgfqpoint{1.085682in}{0.499444in}}%
\pgfpathclose%
\pgfusepath{stroke}%
\end{pgfscope}%
\begin{pgfscope}%
\pgfpathrectangle{\pgfqpoint{0.515000in}{0.499444in}}{\pgfqpoint{3.487500in}{1.155000in}}%
\pgfusepath{clip}%
\pgfsetbuttcap%
\pgfsetmiterjoin%
\pgfsetlinewidth{1.003750pt}%
\definecolor{currentstroke}{rgb}{0.000000,0.000000,0.000000}%
\pgfsetstrokecolor{currentstroke}%
\pgfsetdash{}{0pt}%
\pgfpathmoveto{\pgfqpoint{1.244205in}{0.499444in}}%
\pgfpathlineto{\pgfqpoint{1.307614in}{0.499444in}}%
\pgfpathlineto{\pgfqpoint{1.307614in}{0.848431in}}%
\pgfpathlineto{\pgfqpoint{1.244205in}{0.848431in}}%
\pgfpathlineto{\pgfqpoint{1.244205in}{0.499444in}}%
\pgfpathclose%
\pgfusepath{stroke}%
\end{pgfscope}%
\begin{pgfscope}%
\pgfpathrectangle{\pgfqpoint{0.515000in}{0.499444in}}{\pgfqpoint{3.487500in}{1.155000in}}%
\pgfusepath{clip}%
\pgfsetbuttcap%
\pgfsetmiterjoin%
\pgfsetlinewidth{1.003750pt}%
\definecolor{currentstroke}{rgb}{0.000000,0.000000,0.000000}%
\pgfsetstrokecolor{currentstroke}%
\pgfsetdash{}{0pt}%
\pgfpathmoveto{\pgfqpoint{1.402728in}{0.499444in}}%
\pgfpathlineto{\pgfqpoint{1.466137in}{0.499444in}}%
\pgfpathlineto{\pgfqpoint{1.466137in}{0.740608in}}%
\pgfpathlineto{\pgfqpoint{1.402728in}{0.740608in}}%
\pgfpathlineto{\pgfqpoint{1.402728in}{0.499444in}}%
\pgfpathclose%
\pgfusepath{stroke}%
\end{pgfscope}%
\begin{pgfscope}%
\pgfpathrectangle{\pgfqpoint{0.515000in}{0.499444in}}{\pgfqpoint{3.487500in}{1.155000in}}%
\pgfusepath{clip}%
\pgfsetbuttcap%
\pgfsetmiterjoin%
\pgfsetlinewidth{1.003750pt}%
\definecolor{currentstroke}{rgb}{0.000000,0.000000,0.000000}%
\pgfsetstrokecolor{currentstroke}%
\pgfsetdash{}{0pt}%
\pgfpathmoveto{\pgfqpoint{1.561250in}{0.499444in}}%
\pgfpathlineto{\pgfqpoint{1.624659in}{0.499444in}}%
\pgfpathlineto{\pgfqpoint{1.624659in}{0.660907in}}%
\pgfpathlineto{\pgfqpoint{1.561250in}{0.660907in}}%
\pgfpathlineto{\pgfqpoint{1.561250in}{0.499444in}}%
\pgfpathclose%
\pgfusepath{stroke}%
\end{pgfscope}%
\begin{pgfscope}%
\pgfpathrectangle{\pgfqpoint{0.515000in}{0.499444in}}{\pgfqpoint{3.487500in}{1.155000in}}%
\pgfusepath{clip}%
\pgfsetbuttcap%
\pgfsetmiterjoin%
\pgfsetlinewidth{1.003750pt}%
\definecolor{currentstroke}{rgb}{0.000000,0.000000,0.000000}%
\pgfsetstrokecolor{currentstroke}%
\pgfsetdash{}{0pt}%
\pgfpathmoveto{\pgfqpoint{1.719773in}{0.499444in}}%
\pgfpathlineto{\pgfqpoint{1.783182in}{0.499444in}}%
\pgfpathlineto{\pgfqpoint{1.783182in}{0.611530in}}%
\pgfpathlineto{\pgfqpoint{1.719773in}{0.611530in}}%
\pgfpathlineto{\pgfqpoint{1.719773in}{0.499444in}}%
\pgfpathclose%
\pgfusepath{stroke}%
\end{pgfscope}%
\begin{pgfscope}%
\pgfpathrectangle{\pgfqpoint{0.515000in}{0.499444in}}{\pgfqpoint{3.487500in}{1.155000in}}%
\pgfusepath{clip}%
\pgfsetbuttcap%
\pgfsetmiterjoin%
\pgfsetlinewidth{1.003750pt}%
\definecolor{currentstroke}{rgb}{0.000000,0.000000,0.000000}%
\pgfsetstrokecolor{currentstroke}%
\pgfsetdash{}{0pt}%
\pgfpathmoveto{\pgfqpoint{1.878296in}{0.499444in}}%
\pgfpathlineto{\pgfqpoint{1.941705in}{0.499444in}}%
\pgfpathlineto{\pgfqpoint{1.941705in}{0.577063in}}%
\pgfpathlineto{\pgfqpoint{1.878296in}{0.577063in}}%
\pgfpathlineto{\pgfqpoint{1.878296in}{0.499444in}}%
\pgfpathclose%
\pgfusepath{stroke}%
\end{pgfscope}%
\begin{pgfscope}%
\pgfpathrectangle{\pgfqpoint{0.515000in}{0.499444in}}{\pgfqpoint{3.487500in}{1.155000in}}%
\pgfusepath{clip}%
\pgfsetbuttcap%
\pgfsetmiterjoin%
\pgfsetlinewidth{1.003750pt}%
\definecolor{currentstroke}{rgb}{0.000000,0.000000,0.000000}%
\pgfsetstrokecolor{currentstroke}%
\pgfsetdash{}{0pt}%
\pgfpathmoveto{\pgfqpoint{2.036818in}{0.499444in}}%
\pgfpathlineto{\pgfqpoint{2.100228in}{0.499444in}}%
\pgfpathlineto{\pgfqpoint{2.100228in}{0.554326in}}%
\pgfpathlineto{\pgfqpoint{2.036818in}{0.554326in}}%
\pgfpathlineto{\pgfqpoint{2.036818in}{0.499444in}}%
\pgfpathclose%
\pgfusepath{stroke}%
\end{pgfscope}%
\begin{pgfscope}%
\pgfpathrectangle{\pgfqpoint{0.515000in}{0.499444in}}{\pgfqpoint{3.487500in}{1.155000in}}%
\pgfusepath{clip}%
\pgfsetbuttcap%
\pgfsetmiterjoin%
\pgfsetlinewidth{1.003750pt}%
\definecolor{currentstroke}{rgb}{0.000000,0.000000,0.000000}%
\pgfsetstrokecolor{currentstroke}%
\pgfsetdash{}{0pt}%
\pgfpathmoveto{\pgfqpoint{2.195341in}{0.499444in}}%
\pgfpathlineto{\pgfqpoint{2.258750in}{0.499444in}}%
\pgfpathlineto{\pgfqpoint{2.258750in}{0.536953in}}%
\pgfpathlineto{\pgfqpoint{2.195341in}{0.536953in}}%
\pgfpathlineto{\pgfqpoint{2.195341in}{0.499444in}}%
\pgfpathclose%
\pgfusepath{stroke}%
\end{pgfscope}%
\begin{pgfscope}%
\pgfpathrectangle{\pgfqpoint{0.515000in}{0.499444in}}{\pgfqpoint{3.487500in}{1.155000in}}%
\pgfusepath{clip}%
\pgfsetbuttcap%
\pgfsetmiterjoin%
\pgfsetlinewidth{1.003750pt}%
\definecolor{currentstroke}{rgb}{0.000000,0.000000,0.000000}%
\pgfsetstrokecolor{currentstroke}%
\pgfsetdash{}{0pt}%
\pgfpathmoveto{\pgfqpoint{2.353864in}{0.499444in}}%
\pgfpathlineto{\pgfqpoint{2.417273in}{0.499444in}}%
\pgfpathlineto{\pgfqpoint{2.417273in}{0.528186in}}%
\pgfpathlineto{\pgfqpoint{2.353864in}{0.528186in}}%
\pgfpathlineto{\pgfqpoint{2.353864in}{0.499444in}}%
\pgfpathclose%
\pgfusepath{stroke}%
\end{pgfscope}%
\begin{pgfscope}%
\pgfpathrectangle{\pgfqpoint{0.515000in}{0.499444in}}{\pgfqpoint{3.487500in}{1.155000in}}%
\pgfusepath{clip}%
\pgfsetbuttcap%
\pgfsetmiterjoin%
\pgfsetlinewidth{1.003750pt}%
\definecolor{currentstroke}{rgb}{0.000000,0.000000,0.000000}%
\pgfsetstrokecolor{currentstroke}%
\pgfsetdash{}{0pt}%
\pgfpathmoveto{\pgfqpoint{2.512387in}{0.499444in}}%
\pgfpathlineto{\pgfqpoint{2.575796in}{0.499444in}}%
\pgfpathlineto{\pgfqpoint{2.575796in}{0.519540in}}%
\pgfpathlineto{\pgfqpoint{2.512387in}{0.519540in}}%
\pgfpathlineto{\pgfqpoint{2.512387in}{0.499444in}}%
\pgfpathclose%
\pgfusepath{stroke}%
\end{pgfscope}%
\begin{pgfscope}%
\pgfpathrectangle{\pgfqpoint{0.515000in}{0.499444in}}{\pgfqpoint{3.487500in}{1.155000in}}%
\pgfusepath{clip}%
\pgfsetbuttcap%
\pgfsetmiterjoin%
\pgfsetlinewidth{1.003750pt}%
\definecolor{currentstroke}{rgb}{0.000000,0.000000,0.000000}%
\pgfsetstrokecolor{currentstroke}%
\pgfsetdash{}{0pt}%
\pgfpathmoveto{\pgfqpoint{2.670909in}{0.499444in}}%
\pgfpathlineto{\pgfqpoint{2.734318in}{0.499444in}}%
\pgfpathlineto{\pgfqpoint{2.734318in}{0.513475in}}%
\pgfpathlineto{\pgfqpoint{2.670909in}{0.513475in}}%
\pgfpathlineto{\pgfqpoint{2.670909in}{0.499444in}}%
\pgfpathclose%
\pgfusepath{stroke}%
\end{pgfscope}%
\begin{pgfscope}%
\pgfpathrectangle{\pgfqpoint{0.515000in}{0.499444in}}{\pgfqpoint{3.487500in}{1.155000in}}%
\pgfusepath{clip}%
\pgfsetbuttcap%
\pgfsetmiterjoin%
\pgfsetlinewidth{1.003750pt}%
\definecolor{currentstroke}{rgb}{0.000000,0.000000,0.000000}%
\pgfsetstrokecolor{currentstroke}%
\pgfsetdash{}{0pt}%
\pgfpathmoveto{\pgfqpoint{2.829432in}{0.499444in}}%
\pgfpathlineto{\pgfqpoint{2.892841in}{0.499444in}}%
\pgfpathlineto{\pgfqpoint{2.892841in}{0.509932in}}%
\pgfpathlineto{\pgfqpoint{2.829432in}{0.509932in}}%
\pgfpathlineto{\pgfqpoint{2.829432in}{0.499444in}}%
\pgfpathclose%
\pgfusepath{stroke}%
\end{pgfscope}%
\begin{pgfscope}%
\pgfpathrectangle{\pgfqpoint{0.515000in}{0.499444in}}{\pgfqpoint{3.487500in}{1.155000in}}%
\pgfusepath{clip}%
\pgfsetbuttcap%
\pgfsetmiterjoin%
\pgfsetlinewidth{1.003750pt}%
\definecolor{currentstroke}{rgb}{0.000000,0.000000,0.000000}%
\pgfsetstrokecolor{currentstroke}%
\pgfsetdash{}{0pt}%
\pgfpathmoveto{\pgfqpoint{2.987955in}{0.499444in}}%
\pgfpathlineto{\pgfqpoint{3.051364in}{0.499444in}}%
\pgfpathlineto{\pgfqpoint{3.051364in}{0.507630in}}%
\pgfpathlineto{\pgfqpoint{2.987955in}{0.507630in}}%
\pgfpathlineto{\pgfqpoint{2.987955in}{0.499444in}}%
\pgfpathclose%
\pgfusepath{stroke}%
\end{pgfscope}%
\begin{pgfscope}%
\pgfpathrectangle{\pgfqpoint{0.515000in}{0.499444in}}{\pgfqpoint{3.487500in}{1.155000in}}%
\pgfusepath{clip}%
\pgfsetbuttcap%
\pgfsetmiterjoin%
\pgfsetlinewidth{1.003750pt}%
\definecolor{currentstroke}{rgb}{0.000000,0.000000,0.000000}%
\pgfsetstrokecolor{currentstroke}%
\pgfsetdash{}{0pt}%
\pgfpathmoveto{\pgfqpoint{3.146478in}{0.499444in}}%
\pgfpathlineto{\pgfqpoint{3.209887in}{0.499444in}}%
\pgfpathlineto{\pgfqpoint{3.209887in}{0.505709in}}%
\pgfpathlineto{\pgfqpoint{3.146478in}{0.505709in}}%
\pgfpathlineto{\pgfqpoint{3.146478in}{0.499444in}}%
\pgfpathclose%
\pgfusepath{stroke}%
\end{pgfscope}%
\begin{pgfscope}%
\pgfpathrectangle{\pgfqpoint{0.515000in}{0.499444in}}{\pgfqpoint{3.487500in}{1.155000in}}%
\pgfusepath{clip}%
\pgfsetbuttcap%
\pgfsetmiterjoin%
\pgfsetlinewidth{1.003750pt}%
\definecolor{currentstroke}{rgb}{0.000000,0.000000,0.000000}%
\pgfsetstrokecolor{currentstroke}%
\pgfsetdash{}{0pt}%
\pgfpathmoveto{\pgfqpoint{3.305000in}{0.499444in}}%
\pgfpathlineto{\pgfqpoint{3.368409in}{0.499444in}}%
\pgfpathlineto{\pgfqpoint{3.368409in}{0.503147in}}%
\pgfpathlineto{\pgfqpoint{3.305000in}{0.503147in}}%
\pgfpathlineto{\pgfqpoint{3.305000in}{0.499444in}}%
\pgfpathclose%
\pgfusepath{stroke}%
\end{pgfscope}%
\begin{pgfscope}%
\pgfpathrectangle{\pgfqpoint{0.515000in}{0.499444in}}{\pgfqpoint{3.487500in}{1.155000in}}%
\pgfusepath{clip}%
\pgfsetbuttcap%
\pgfsetmiterjoin%
\pgfsetlinewidth{1.003750pt}%
\definecolor{currentstroke}{rgb}{0.000000,0.000000,0.000000}%
\pgfsetstrokecolor{currentstroke}%
\pgfsetdash{}{0pt}%
\pgfpathmoveto{\pgfqpoint{3.463523in}{0.499444in}}%
\pgfpathlineto{\pgfqpoint{3.526932in}{0.499444in}}%
\pgfpathlineto{\pgfqpoint{3.526932in}{0.501266in}}%
\pgfpathlineto{\pgfqpoint{3.463523in}{0.501266in}}%
\pgfpathlineto{\pgfqpoint{3.463523in}{0.499444in}}%
\pgfpathclose%
\pgfusepath{stroke}%
\end{pgfscope}%
\begin{pgfscope}%
\pgfpathrectangle{\pgfqpoint{0.515000in}{0.499444in}}{\pgfqpoint{3.487500in}{1.155000in}}%
\pgfusepath{clip}%
\pgfsetbuttcap%
\pgfsetmiterjoin%
\pgfsetlinewidth{1.003750pt}%
\definecolor{currentstroke}{rgb}{0.000000,0.000000,0.000000}%
\pgfsetstrokecolor{currentstroke}%
\pgfsetdash{}{0pt}%
\pgfpathmoveto{\pgfqpoint{3.622046in}{0.499444in}}%
\pgfpathlineto{\pgfqpoint{3.685455in}{0.499444in}}%
\pgfpathlineto{\pgfqpoint{3.685455in}{0.499744in}}%
\pgfpathlineto{\pgfqpoint{3.622046in}{0.499744in}}%
\pgfpathlineto{\pgfqpoint{3.622046in}{0.499444in}}%
\pgfpathclose%
\pgfusepath{stroke}%
\end{pgfscope}%
\begin{pgfscope}%
\pgfpathrectangle{\pgfqpoint{0.515000in}{0.499444in}}{\pgfqpoint{3.487500in}{1.155000in}}%
\pgfusepath{clip}%
\pgfsetbuttcap%
\pgfsetmiterjoin%
\pgfsetlinewidth{1.003750pt}%
\definecolor{currentstroke}{rgb}{0.000000,0.000000,0.000000}%
\pgfsetstrokecolor{currentstroke}%
\pgfsetdash{}{0pt}%
\pgfpathmoveto{\pgfqpoint{3.780568in}{0.499444in}}%
\pgfpathlineto{\pgfqpoint{3.843978in}{0.499444in}}%
\pgfpathlineto{\pgfqpoint{3.843978in}{0.499444in}}%
\pgfpathlineto{\pgfqpoint{3.780568in}{0.499444in}}%
\pgfpathlineto{\pgfqpoint{3.780568in}{0.499444in}}%
\pgfpathclose%
\pgfusepath{stroke}%
\end{pgfscope}%
\begin{pgfscope}%
\pgfpathrectangle{\pgfqpoint{0.515000in}{0.499444in}}{\pgfqpoint{3.487500in}{1.155000in}}%
\pgfusepath{clip}%
\pgfsetbuttcap%
\pgfsetmiterjoin%
\definecolor{currentfill}{rgb}{0.000000,0.000000,0.000000}%
\pgfsetfillcolor{currentfill}%
\pgfsetlinewidth{0.000000pt}%
\definecolor{currentstroke}{rgb}{0.000000,0.000000,0.000000}%
\pgfsetstrokecolor{currentstroke}%
\pgfsetstrokeopacity{0.000000}%
\pgfsetdash{}{0pt}%
\pgfpathmoveto{\pgfqpoint{0.673523in}{0.499444in}}%
\pgfpathlineto{\pgfqpoint{0.736932in}{0.499444in}}%
\pgfpathlineto{\pgfqpoint{0.736932in}{0.499444in}}%
\pgfpathlineto{\pgfqpoint{0.673523in}{0.499444in}}%
\pgfpathlineto{\pgfqpoint{0.673523in}{0.499444in}}%
\pgfpathclose%
\pgfusepath{fill}%
\end{pgfscope}%
\begin{pgfscope}%
\pgfpathrectangle{\pgfqpoint{0.515000in}{0.499444in}}{\pgfqpoint{3.487500in}{1.155000in}}%
\pgfusepath{clip}%
\pgfsetbuttcap%
\pgfsetmiterjoin%
\definecolor{currentfill}{rgb}{0.000000,0.000000,0.000000}%
\pgfsetfillcolor{currentfill}%
\pgfsetlinewidth{0.000000pt}%
\definecolor{currentstroke}{rgb}{0.000000,0.000000,0.000000}%
\pgfsetstrokecolor{currentstroke}%
\pgfsetstrokeopacity{0.000000}%
\pgfsetdash{}{0pt}%
\pgfpathmoveto{\pgfqpoint{0.832046in}{0.499444in}}%
\pgfpathlineto{\pgfqpoint{0.895455in}{0.499444in}}%
\pgfpathlineto{\pgfqpoint{0.895455in}{0.535532in}}%
\pgfpathlineto{\pgfqpoint{0.832046in}{0.535532in}}%
\pgfpathlineto{\pgfqpoint{0.832046in}{0.499444in}}%
\pgfpathclose%
\pgfusepath{fill}%
\end{pgfscope}%
\begin{pgfscope}%
\pgfpathrectangle{\pgfqpoint{0.515000in}{0.499444in}}{\pgfqpoint{3.487500in}{1.155000in}}%
\pgfusepath{clip}%
\pgfsetbuttcap%
\pgfsetmiterjoin%
\definecolor{currentfill}{rgb}{0.000000,0.000000,0.000000}%
\pgfsetfillcolor{currentfill}%
\pgfsetlinewidth{0.000000pt}%
\definecolor{currentstroke}{rgb}{0.000000,0.000000,0.000000}%
\pgfsetstrokecolor{currentstroke}%
\pgfsetstrokeopacity{0.000000}%
\pgfsetdash{}{0pt}%
\pgfpathmoveto{\pgfqpoint{0.990568in}{0.499444in}}%
\pgfpathlineto{\pgfqpoint{1.053978in}{0.499444in}}%
\pgfpathlineto{\pgfqpoint{1.053978in}{0.575602in}}%
\pgfpathlineto{\pgfqpoint{0.990568in}{0.575602in}}%
\pgfpathlineto{\pgfqpoint{0.990568in}{0.499444in}}%
\pgfpathclose%
\pgfusepath{fill}%
\end{pgfscope}%
\begin{pgfscope}%
\pgfpathrectangle{\pgfqpoint{0.515000in}{0.499444in}}{\pgfqpoint{3.487500in}{1.155000in}}%
\pgfusepath{clip}%
\pgfsetbuttcap%
\pgfsetmiterjoin%
\definecolor{currentfill}{rgb}{0.000000,0.000000,0.000000}%
\pgfsetfillcolor{currentfill}%
\pgfsetlinewidth{0.000000pt}%
\definecolor{currentstroke}{rgb}{0.000000,0.000000,0.000000}%
\pgfsetstrokecolor{currentstroke}%
\pgfsetstrokeopacity{0.000000}%
\pgfsetdash{}{0pt}%
\pgfpathmoveto{\pgfqpoint{1.149091in}{0.499444in}}%
\pgfpathlineto{\pgfqpoint{1.212500in}{0.499444in}}%
\pgfpathlineto{\pgfqpoint{1.212500in}{0.576243in}}%
\pgfpathlineto{\pgfqpoint{1.149091in}{0.576243in}}%
\pgfpathlineto{\pgfqpoint{1.149091in}{0.499444in}}%
\pgfpathclose%
\pgfusepath{fill}%
\end{pgfscope}%
\begin{pgfscope}%
\pgfpathrectangle{\pgfqpoint{0.515000in}{0.499444in}}{\pgfqpoint{3.487500in}{1.155000in}}%
\pgfusepath{clip}%
\pgfsetbuttcap%
\pgfsetmiterjoin%
\definecolor{currentfill}{rgb}{0.000000,0.000000,0.000000}%
\pgfsetfillcolor{currentfill}%
\pgfsetlinewidth{0.000000pt}%
\definecolor{currentstroke}{rgb}{0.000000,0.000000,0.000000}%
\pgfsetstrokecolor{currentstroke}%
\pgfsetstrokeopacity{0.000000}%
\pgfsetdash{}{0pt}%
\pgfpathmoveto{\pgfqpoint{1.307614in}{0.499444in}}%
\pgfpathlineto{\pgfqpoint{1.371023in}{0.499444in}}%
\pgfpathlineto{\pgfqpoint{1.371023in}{0.571940in}}%
\pgfpathlineto{\pgfqpoint{1.307614in}{0.571940in}}%
\pgfpathlineto{\pgfqpoint{1.307614in}{0.499444in}}%
\pgfpathclose%
\pgfusepath{fill}%
\end{pgfscope}%
\begin{pgfscope}%
\pgfpathrectangle{\pgfqpoint{0.515000in}{0.499444in}}{\pgfqpoint{3.487500in}{1.155000in}}%
\pgfusepath{clip}%
\pgfsetbuttcap%
\pgfsetmiterjoin%
\definecolor{currentfill}{rgb}{0.000000,0.000000,0.000000}%
\pgfsetfillcolor{currentfill}%
\pgfsetlinewidth{0.000000pt}%
\definecolor{currentstroke}{rgb}{0.000000,0.000000,0.000000}%
\pgfsetstrokecolor{currentstroke}%
\pgfsetstrokeopacity{0.000000}%
\pgfsetdash{}{0pt}%
\pgfpathmoveto{\pgfqpoint{1.466137in}{0.499444in}}%
\pgfpathlineto{\pgfqpoint{1.529546in}{0.499444in}}%
\pgfpathlineto{\pgfqpoint{1.529546in}{0.564794in}}%
\pgfpathlineto{\pgfqpoint{1.466137in}{0.564794in}}%
\pgfpathlineto{\pgfqpoint{1.466137in}{0.499444in}}%
\pgfpathclose%
\pgfusepath{fill}%
\end{pgfscope}%
\begin{pgfscope}%
\pgfpathrectangle{\pgfqpoint{0.515000in}{0.499444in}}{\pgfqpoint{3.487500in}{1.155000in}}%
\pgfusepath{clip}%
\pgfsetbuttcap%
\pgfsetmiterjoin%
\definecolor{currentfill}{rgb}{0.000000,0.000000,0.000000}%
\pgfsetfillcolor{currentfill}%
\pgfsetlinewidth{0.000000pt}%
\definecolor{currentstroke}{rgb}{0.000000,0.000000,0.000000}%
\pgfsetstrokecolor{currentstroke}%
\pgfsetstrokeopacity{0.000000}%
\pgfsetdash{}{0pt}%
\pgfpathmoveto{\pgfqpoint{1.624659in}{0.499444in}}%
\pgfpathlineto{\pgfqpoint{1.688068in}{0.499444in}}%
\pgfpathlineto{\pgfqpoint{1.688068in}{0.557128in}}%
\pgfpathlineto{\pgfqpoint{1.624659in}{0.557128in}}%
\pgfpathlineto{\pgfqpoint{1.624659in}{0.499444in}}%
\pgfpathclose%
\pgfusepath{fill}%
\end{pgfscope}%
\begin{pgfscope}%
\pgfpathrectangle{\pgfqpoint{0.515000in}{0.499444in}}{\pgfqpoint{3.487500in}{1.155000in}}%
\pgfusepath{clip}%
\pgfsetbuttcap%
\pgfsetmiterjoin%
\definecolor{currentfill}{rgb}{0.000000,0.000000,0.000000}%
\pgfsetfillcolor{currentfill}%
\pgfsetlinewidth{0.000000pt}%
\definecolor{currentstroke}{rgb}{0.000000,0.000000,0.000000}%
\pgfsetstrokecolor{currentstroke}%
\pgfsetstrokeopacity{0.000000}%
\pgfsetdash{}{0pt}%
\pgfpathmoveto{\pgfqpoint{1.783182in}{0.499444in}}%
\pgfpathlineto{\pgfqpoint{1.846591in}{0.499444in}}%
\pgfpathlineto{\pgfqpoint{1.846591in}{0.548922in}}%
\pgfpathlineto{\pgfqpoint{1.783182in}{0.548922in}}%
\pgfpathlineto{\pgfqpoint{1.783182in}{0.499444in}}%
\pgfpathclose%
\pgfusepath{fill}%
\end{pgfscope}%
\begin{pgfscope}%
\pgfpathrectangle{\pgfqpoint{0.515000in}{0.499444in}}{\pgfqpoint{3.487500in}{1.155000in}}%
\pgfusepath{clip}%
\pgfsetbuttcap%
\pgfsetmiterjoin%
\definecolor{currentfill}{rgb}{0.000000,0.000000,0.000000}%
\pgfsetfillcolor{currentfill}%
\pgfsetlinewidth{0.000000pt}%
\definecolor{currentstroke}{rgb}{0.000000,0.000000,0.000000}%
\pgfsetstrokecolor{currentstroke}%
\pgfsetstrokeopacity{0.000000}%
\pgfsetdash{}{0pt}%
\pgfpathmoveto{\pgfqpoint{1.941705in}{0.499444in}}%
\pgfpathlineto{\pgfqpoint{2.005114in}{0.499444in}}%
\pgfpathlineto{\pgfqpoint{2.005114in}{0.542597in}}%
\pgfpathlineto{\pgfqpoint{1.941705in}{0.542597in}}%
\pgfpathlineto{\pgfqpoint{1.941705in}{0.499444in}}%
\pgfpathclose%
\pgfusepath{fill}%
\end{pgfscope}%
\begin{pgfscope}%
\pgfpathrectangle{\pgfqpoint{0.515000in}{0.499444in}}{\pgfqpoint{3.487500in}{1.155000in}}%
\pgfusepath{clip}%
\pgfsetbuttcap%
\pgfsetmiterjoin%
\definecolor{currentfill}{rgb}{0.000000,0.000000,0.000000}%
\pgfsetfillcolor{currentfill}%
\pgfsetlinewidth{0.000000pt}%
\definecolor{currentstroke}{rgb}{0.000000,0.000000,0.000000}%
\pgfsetstrokecolor{currentstroke}%
\pgfsetstrokeopacity{0.000000}%
\pgfsetdash{}{0pt}%
\pgfpathmoveto{\pgfqpoint{2.100228in}{0.499444in}}%
\pgfpathlineto{\pgfqpoint{2.163637in}{0.499444in}}%
\pgfpathlineto{\pgfqpoint{2.163637in}{0.534451in}}%
\pgfpathlineto{\pgfqpoint{2.100228in}{0.534451in}}%
\pgfpathlineto{\pgfqpoint{2.100228in}{0.499444in}}%
\pgfpathclose%
\pgfusepath{fill}%
\end{pgfscope}%
\begin{pgfscope}%
\pgfpathrectangle{\pgfqpoint{0.515000in}{0.499444in}}{\pgfqpoint{3.487500in}{1.155000in}}%
\pgfusepath{clip}%
\pgfsetbuttcap%
\pgfsetmiterjoin%
\definecolor{currentfill}{rgb}{0.000000,0.000000,0.000000}%
\pgfsetfillcolor{currentfill}%
\pgfsetlinewidth{0.000000pt}%
\definecolor{currentstroke}{rgb}{0.000000,0.000000,0.000000}%
\pgfsetstrokecolor{currentstroke}%
\pgfsetstrokeopacity{0.000000}%
\pgfsetdash{}{0pt}%
\pgfpathmoveto{\pgfqpoint{2.258750in}{0.499444in}}%
\pgfpathlineto{\pgfqpoint{2.322159in}{0.499444in}}%
\pgfpathlineto{\pgfqpoint{2.322159in}{0.528907in}}%
\pgfpathlineto{\pgfqpoint{2.258750in}{0.528907in}}%
\pgfpathlineto{\pgfqpoint{2.258750in}{0.499444in}}%
\pgfpathclose%
\pgfusepath{fill}%
\end{pgfscope}%
\begin{pgfscope}%
\pgfpathrectangle{\pgfqpoint{0.515000in}{0.499444in}}{\pgfqpoint{3.487500in}{1.155000in}}%
\pgfusepath{clip}%
\pgfsetbuttcap%
\pgfsetmiterjoin%
\definecolor{currentfill}{rgb}{0.000000,0.000000,0.000000}%
\pgfsetfillcolor{currentfill}%
\pgfsetlinewidth{0.000000pt}%
\definecolor{currentstroke}{rgb}{0.000000,0.000000,0.000000}%
\pgfsetstrokecolor{currentstroke}%
\pgfsetstrokeopacity{0.000000}%
\pgfsetdash{}{0pt}%
\pgfpathmoveto{\pgfqpoint{2.417273in}{0.499444in}}%
\pgfpathlineto{\pgfqpoint{2.480682in}{0.499444in}}%
\pgfpathlineto{\pgfqpoint{2.480682in}{0.525304in}}%
\pgfpathlineto{\pgfqpoint{2.417273in}{0.525304in}}%
\pgfpathlineto{\pgfqpoint{2.417273in}{0.499444in}}%
\pgfpathclose%
\pgfusepath{fill}%
\end{pgfscope}%
\begin{pgfscope}%
\pgfpathrectangle{\pgfqpoint{0.515000in}{0.499444in}}{\pgfqpoint{3.487500in}{1.155000in}}%
\pgfusepath{clip}%
\pgfsetbuttcap%
\pgfsetmiterjoin%
\definecolor{currentfill}{rgb}{0.000000,0.000000,0.000000}%
\pgfsetfillcolor{currentfill}%
\pgfsetlinewidth{0.000000pt}%
\definecolor{currentstroke}{rgb}{0.000000,0.000000,0.000000}%
\pgfsetstrokecolor{currentstroke}%
\pgfsetstrokeopacity{0.000000}%
\pgfsetdash{}{0pt}%
\pgfpathmoveto{\pgfqpoint{2.575796in}{0.499444in}}%
\pgfpathlineto{\pgfqpoint{2.639205in}{0.499444in}}%
\pgfpathlineto{\pgfqpoint{2.639205in}{0.521501in}}%
\pgfpathlineto{\pgfqpoint{2.575796in}{0.521501in}}%
\pgfpathlineto{\pgfqpoint{2.575796in}{0.499444in}}%
\pgfpathclose%
\pgfusepath{fill}%
\end{pgfscope}%
\begin{pgfscope}%
\pgfpathrectangle{\pgfqpoint{0.515000in}{0.499444in}}{\pgfqpoint{3.487500in}{1.155000in}}%
\pgfusepath{clip}%
\pgfsetbuttcap%
\pgfsetmiterjoin%
\definecolor{currentfill}{rgb}{0.000000,0.000000,0.000000}%
\pgfsetfillcolor{currentfill}%
\pgfsetlinewidth{0.000000pt}%
\definecolor{currentstroke}{rgb}{0.000000,0.000000,0.000000}%
\pgfsetstrokecolor{currentstroke}%
\pgfsetstrokeopacity{0.000000}%
\pgfsetdash{}{0pt}%
\pgfpathmoveto{\pgfqpoint{2.734318in}{0.499444in}}%
\pgfpathlineto{\pgfqpoint{2.797728in}{0.499444in}}%
\pgfpathlineto{\pgfqpoint{2.797728in}{0.518299in}}%
\pgfpathlineto{\pgfqpoint{2.734318in}{0.518299in}}%
\pgfpathlineto{\pgfqpoint{2.734318in}{0.499444in}}%
\pgfpathclose%
\pgfusepath{fill}%
\end{pgfscope}%
\begin{pgfscope}%
\pgfpathrectangle{\pgfqpoint{0.515000in}{0.499444in}}{\pgfqpoint{3.487500in}{1.155000in}}%
\pgfusepath{clip}%
\pgfsetbuttcap%
\pgfsetmiterjoin%
\definecolor{currentfill}{rgb}{0.000000,0.000000,0.000000}%
\pgfsetfillcolor{currentfill}%
\pgfsetlinewidth{0.000000pt}%
\definecolor{currentstroke}{rgb}{0.000000,0.000000,0.000000}%
\pgfsetstrokecolor{currentstroke}%
\pgfsetstrokeopacity{0.000000}%
\pgfsetdash{}{0pt}%
\pgfpathmoveto{\pgfqpoint{2.892841in}{0.499444in}}%
\pgfpathlineto{\pgfqpoint{2.956250in}{0.499444in}}%
\pgfpathlineto{\pgfqpoint{2.956250in}{0.517418in}}%
\pgfpathlineto{\pgfqpoint{2.892841in}{0.517418in}}%
\pgfpathlineto{\pgfqpoint{2.892841in}{0.499444in}}%
\pgfpathclose%
\pgfusepath{fill}%
\end{pgfscope}%
\begin{pgfscope}%
\pgfpathrectangle{\pgfqpoint{0.515000in}{0.499444in}}{\pgfqpoint{3.487500in}{1.155000in}}%
\pgfusepath{clip}%
\pgfsetbuttcap%
\pgfsetmiterjoin%
\definecolor{currentfill}{rgb}{0.000000,0.000000,0.000000}%
\pgfsetfillcolor{currentfill}%
\pgfsetlinewidth{0.000000pt}%
\definecolor{currentstroke}{rgb}{0.000000,0.000000,0.000000}%
\pgfsetstrokecolor{currentstroke}%
\pgfsetstrokeopacity{0.000000}%
\pgfsetdash{}{0pt}%
\pgfpathmoveto{\pgfqpoint{3.051364in}{0.499444in}}%
\pgfpathlineto{\pgfqpoint{3.114773in}{0.499444in}}%
\pgfpathlineto{\pgfqpoint{3.114773in}{0.515116in}}%
\pgfpathlineto{\pgfqpoint{3.051364in}{0.515116in}}%
\pgfpathlineto{\pgfqpoint{3.051364in}{0.499444in}}%
\pgfpathclose%
\pgfusepath{fill}%
\end{pgfscope}%
\begin{pgfscope}%
\pgfpathrectangle{\pgfqpoint{0.515000in}{0.499444in}}{\pgfqpoint{3.487500in}{1.155000in}}%
\pgfusepath{clip}%
\pgfsetbuttcap%
\pgfsetmiterjoin%
\definecolor{currentfill}{rgb}{0.000000,0.000000,0.000000}%
\pgfsetfillcolor{currentfill}%
\pgfsetlinewidth{0.000000pt}%
\definecolor{currentstroke}{rgb}{0.000000,0.000000,0.000000}%
\pgfsetstrokecolor{currentstroke}%
\pgfsetstrokeopacity{0.000000}%
\pgfsetdash{}{0pt}%
\pgfpathmoveto{\pgfqpoint{3.209887in}{0.499444in}}%
\pgfpathlineto{\pgfqpoint{3.273296in}{0.499444in}}%
\pgfpathlineto{\pgfqpoint{3.273296in}{0.513695in}}%
\pgfpathlineto{\pgfqpoint{3.209887in}{0.513695in}}%
\pgfpathlineto{\pgfqpoint{3.209887in}{0.499444in}}%
\pgfpathclose%
\pgfusepath{fill}%
\end{pgfscope}%
\begin{pgfscope}%
\pgfpathrectangle{\pgfqpoint{0.515000in}{0.499444in}}{\pgfqpoint{3.487500in}{1.155000in}}%
\pgfusepath{clip}%
\pgfsetbuttcap%
\pgfsetmiterjoin%
\definecolor{currentfill}{rgb}{0.000000,0.000000,0.000000}%
\pgfsetfillcolor{currentfill}%
\pgfsetlinewidth{0.000000pt}%
\definecolor{currentstroke}{rgb}{0.000000,0.000000,0.000000}%
\pgfsetstrokecolor{currentstroke}%
\pgfsetstrokeopacity{0.000000}%
\pgfsetdash{}{0pt}%
\pgfpathmoveto{\pgfqpoint{3.368409in}{0.499444in}}%
\pgfpathlineto{\pgfqpoint{3.431818in}{0.499444in}}%
\pgfpathlineto{\pgfqpoint{3.431818in}{0.511653in}}%
\pgfpathlineto{\pgfqpoint{3.368409in}{0.511653in}}%
\pgfpathlineto{\pgfqpoint{3.368409in}{0.499444in}}%
\pgfpathclose%
\pgfusepath{fill}%
\end{pgfscope}%
\begin{pgfscope}%
\pgfpathrectangle{\pgfqpoint{0.515000in}{0.499444in}}{\pgfqpoint{3.487500in}{1.155000in}}%
\pgfusepath{clip}%
\pgfsetbuttcap%
\pgfsetmiterjoin%
\definecolor{currentfill}{rgb}{0.000000,0.000000,0.000000}%
\pgfsetfillcolor{currentfill}%
\pgfsetlinewidth{0.000000pt}%
\definecolor{currentstroke}{rgb}{0.000000,0.000000,0.000000}%
\pgfsetstrokecolor{currentstroke}%
\pgfsetstrokeopacity{0.000000}%
\pgfsetdash{}{0pt}%
\pgfpathmoveto{\pgfqpoint{3.526932in}{0.499444in}}%
\pgfpathlineto{\pgfqpoint{3.590341in}{0.499444in}}%
\pgfpathlineto{\pgfqpoint{3.590341in}{0.505949in}}%
\pgfpathlineto{\pgfqpoint{3.526932in}{0.505949in}}%
\pgfpathlineto{\pgfqpoint{3.526932in}{0.499444in}}%
\pgfpathclose%
\pgfusepath{fill}%
\end{pgfscope}%
\begin{pgfscope}%
\pgfpathrectangle{\pgfqpoint{0.515000in}{0.499444in}}{\pgfqpoint{3.487500in}{1.155000in}}%
\pgfusepath{clip}%
\pgfsetbuttcap%
\pgfsetmiterjoin%
\definecolor{currentfill}{rgb}{0.000000,0.000000,0.000000}%
\pgfsetfillcolor{currentfill}%
\pgfsetlinewidth{0.000000pt}%
\definecolor{currentstroke}{rgb}{0.000000,0.000000,0.000000}%
\pgfsetstrokecolor{currentstroke}%
\pgfsetstrokeopacity{0.000000}%
\pgfsetdash{}{0pt}%
\pgfpathmoveto{\pgfqpoint{3.685455in}{0.499444in}}%
\pgfpathlineto{\pgfqpoint{3.748864in}{0.499444in}}%
\pgfpathlineto{\pgfqpoint{3.748864in}{0.501386in}}%
\pgfpathlineto{\pgfqpoint{3.685455in}{0.501386in}}%
\pgfpathlineto{\pgfqpoint{3.685455in}{0.499444in}}%
\pgfpathclose%
\pgfusepath{fill}%
\end{pgfscope}%
\begin{pgfscope}%
\pgfpathrectangle{\pgfqpoint{0.515000in}{0.499444in}}{\pgfqpoint{3.487500in}{1.155000in}}%
\pgfusepath{clip}%
\pgfsetbuttcap%
\pgfsetmiterjoin%
\definecolor{currentfill}{rgb}{0.000000,0.000000,0.000000}%
\pgfsetfillcolor{currentfill}%
\pgfsetlinewidth{0.000000pt}%
\definecolor{currentstroke}{rgb}{0.000000,0.000000,0.000000}%
\pgfsetstrokecolor{currentstroke}%
\pgfsetstrokeopacity{0.000000}%
\pgfsetdash{}{0pt}%
\pgfpathmoveto{\pgfqpoint{3.843978in}{0.499444in}}%
\pgfpathlineto{\pgfqpoint{3.907387in}{0.499444in}}%
\pgfpathlineto{\pgfqpoint{3.907387in}{0.499464in}}%
\pgfpathlineto{\pgfqpoint{3.843978in}{0.499464in}}%
\pgfpathlineto{\pgfqpoint{3.843978in}{0.499444in}}%
\pgfpathclose%
\pgfusepath{fill}%
\end{pgfscope}%
\begin{pgfscope}%
\pgfsetbuttcap%
\pgfsetroundjoin%
\definecolor{currentfill}{rgb}{0.000000,0.000000,0.000000}%
\pgfsetfillcolor{currentfill}%
\pgfsetlinewidth{0.803000pt}%
\definecolor{currentstroke}{rgb}{0.000000,0.000000,0.000000}%
\pgfsetstrokecolor{currentstroke}%
\pgfsetdash{}{0pt}%
\pgfsys@defobject{currentmarker}{\pgfqpoint{0.000000in}{-0.048611in}}{\pgfqpoint{0.000000in}{0.000000in}}{%
\pgfpathmoveto{\pgfqpoint{0.000000in}{0.000000in}}%
\pgfpathlineto{\pgfqpoint{0.000000in}{-0.048611in}}%
\pgfusepath{stroke,fill}%
}%
\begin{pgfscope}%
\pgfsys@transformshift{0.515000in}{0.499444in}%
\pgfsys@useobject{currentmarker}{}%
\end{pgfscope}%
\end{pgfscope}%
\begin{pgfscope}%
\pgfsetbuttcap%
\pgfsetroundjoin%
\definecolor{currentfill}{rgb}{0.000000,0.000000,0.000000}%
\pgfsetfillcolor{currentfill}%
\pgfsetlinewidth{0.803000pt}%
\definecolor{currentstroke}{rgb}{0.000000,0.000000,0.000000}%
\pgfsetstrokecolor{currentstroke}%
\pgfsetdash{}{0pt}%
\pgfsys@defobject{currentmarker}{\pgfqpoint{0.000000in}{-0.048611in}}{\pgfqpoint{0.000000in}{0.000000in}}{%
\pgfpathmoveto{\pgfqpoint{0.000000in}{0.000000in}}%
\pgfpathlineto{\pgfqpoint{0.000000in}{-0.048611in}}%
\pgfusepath{stroke,fill}%
}%
\begin{pgfscope}%
\pgfsys@transformshift{0.673523in}{0.499444in}%
\pgfsys@useobject{currentmarker}{}%
\end{pgfscope}%
\end{pgfscope}%
\begin{pgfscope}%
\definecolor{textcolor}{rgb}{0.000000,0.000000,0.000000}%
\pgfsetstrokecolor{textcolor}%
\pgfsetfillcolor{textcolor}%
\pgftext[x=0.673523in,y=0.402222in,,top]{\color{textcolor}\rmfamily\fontsize{10.000000}{12.000000}\selectfont 0.0}%
\end{pgfscope}%
\begin{pgfscope}%
\pgfsetbuttcap%
\pgfsetroundjoin%
\definecolor{currentfill}{rgb}{0.000000,0.000000,0.000000}%
\pgfsetfillcolor{currentfill}%
\pgfsetlinewidth{0.803000pt}%
\definecolor{currentstroke}{rgb}{0.000000,0.000000,0.000000}%
\pgfsetstrokecolor{currentstroke}%
\pgfsetdash{}{0pt}%
\pgfsys@defobject{currentmarker}{\pgfqpoint{0.000000in}{-0.048611in}}{\pgfqpoint{0.000000in}{0.000000in}}{%
\pgfpathmoveto{\pgfqpoint{0.000000in}{0.000000in}}%
\pgfpathlineto{\pgfqpoint{0.000000in}{-0.048611in}}%
\pgfusepath{stroke,fill}%
}%
\begin{pgfscope}%
\pgfsys@transformshift{0.832046in}{0.499444in}%
\pgfsys@useobject{currentmarker}{}%
\end{pgfscope}%
\end{pgfscope}%
\begin{pgfscope}%
\pgfsetbuttcap%
\pgfsetroundjoin%
\definecolor{currentfill}{rgb}{0.000000,0.000000,0.000000}%
\pgfsetfillcolor{currentfill}%
\pgfsetlinewidth{0.803000pt}%
\definecolor{currentstroke}{rgb}{0.000000,0.000000,0.000000}%
\pgfsetstrokecolor{currentstroke}%
\pgfsetdash{}{0pt}%
\pgfsys@defobject{currentmarker}{\pgfqpoint{0.000000in}{-0.048611in}}{\pgfqpoint{0.000000in}{0.000000in}}{%
\pgfpathmoveto{\pgfqpoint{0.000000in}{0.000000in}}%
\pgfpathlineto{\pgfqpoint{0.000000in}{-0.048611in}}%
\pgfusepath{stroke,fill}%
}%
\begin{pgfscope}%
\pgfsys@transformshift{0.990568in}{0.499444in}%
\pgfsys@useobject{currentmarker}{}%
\end{pgfscope}%
\end{pgfscope}%
\begin{pgfscope}%
\definecolor{textcolor}{rgb}{0.000000,0.000000,0.000000}%
\pgfsetstrokecolor{textcolor}%
\pgfsetfillcolor{textcolor}%
\pgftext[x=0.990568in,y=0.402222in,,top]{\color{textcolor}\rmfamily\fontsize{10.000000}{12.000000}\selectfont 0.1}%
\end{pgfscope}%
\begin{pgfscope}%
\pgfsetbuttcap%
\pgfsetroundjoin%
\definecolor{currentfill}{rgb}{0.000000,0.000000,0.000000}%
\pgfsetfillcolor{currentfill}%
\pgfsetlinewidth{0.803000pt}%
\definecolor{currentstroke}{rgb}{0.000000,0.000000,0.000000}%
\pgfsetstrokecolor{currentstroke}%
\pgfsetdash{}{0pt}%
\pgfsys@defobject{currentmarker}{\pgfqpoint{0.000000in}{-0.048611in}}{\pgfqpoint{0.000000in}{0.000000in}}{%
\pgfpathmoveto{\pgfqpoint{0.000000in}{0.000000in}}%
\pgfpathlineto{\pgfqpoint{0.000000in}{-0.048611in}}%
\pgfusepath{stroke,fill}%
}%
\begin{pgfscope}%
\pgfsys@transformshift{1.149091in}{0.499444in}%
\pgfsys@useobject{currentmarker}{}%
\end{pgfscope}%
\end{pgfscope}%
\begin{pgfscope}%
\pgfsetbuttcap%
\pgfsetroundjoin%
\definecolor{currentfill}{rgb}{0.000000,0.000000,0.000000}%
\pgfsetfillcolor{currentfill}%
\pgfsetlinewidth{0.803000pt}%
\definecolor{currentstroke}{rgb}{0.000000,0.000000,0.000000}%
\pgfsetstrokecolor{currentstroke}%
\pgfsetdash{}{0pt}%
\pgfsys@defobject{currentmarker}{\pgfqpoint{0.000000in}{-0.048611in}}{\pgfqpoint{0.000000in}{0.000000in}}{%
\pgfpathmoveto{\pgfqpoint{0.000000in}{0.000000in}}%
\pgfpathlineto{\pgfqpoint{0.000000in}{-0.048611in}}%
\pgfusepath{stroke,fill}%
}%
\begin{pgfscope}%
\pgfsys@transformshift{1.307614in}{0.499444in}%
\pgfsys@useobject{currentmarker}{}%
\end{pgfscope}%
\end{pgfscope}%
\begin{pgfscope}%
\definecolor{textcolor}{rgb}{0.000000,0.000000,0.000000}%
\pgfsetstrokecolor{textcolor}%
\pgfsetfillcolor{textcolor}%
\pgftext[x=1.307614in,y=0.402222in,,top]{\color{textcolor}\rmfamily\fontsize{10.000000}{12.000000}\selectfont 0.2}%
\end{pgfscope}%
\begin{pgfscope}%
\pgfsetbuttcap%
\pgfsetroundjoin%
\definecolor{currentfill}{rgb}{0.000000,0.000000,0.000000}%
\pgfsetfillcolor{currentfill}%
\pgfsetlinewidth{0.803000pt}%
\definecolor{currentstroke}{rgb}{0.000000,0.000000,0.000000}%
\pgfsetstrokecolor{currentstroke}%
\pgfsetdash{}{0pt}%
\pgfsys@defobject{currentmarker}{\pgfqpoint{0.000000in}{-0.048611in}}{\pgfqpoint{0.000000in}{0.000000in}}{%
\pgfpathmoveto{\pgfqpoint{0.000000in}{0.000000in}}%
\pgfpathlineto{\pgfqpoint{0.000000in}{-0.048611in}}%
\pgfusepath{stroke,fill}%
}%
\begin{pgfscope}%
\pgfsys@transformshift{1.466137in}{0.499444in}%
\pgfsys@useobject{currentmarker}{}%
\end{pgfscope}%
\end{pgfscope}%
\begin{pgfscope}%
\pgfsetbuttcap%
\pgfsetroundjoin%
\definecolor{currentfill}{rgb}{0.000000,0.000000,0.000000}%
\pgfsetfillcolor{currentfill}%
\pgfsetlinewidth{0.803000pt}%
\definecolor{currentstroke}{rgb}{0.000000,0.000000,0.000000}%
\pgfsetstrokecolor{currentstroke}%
\pgfsetdash{}{0pt}%
\pgfsys@defobject{currentmarker}{\pgfqpoint{0.000000in}{-0.048611in}}{\pgfqpoint{0.000000in}{0.000000in}}{%
\pgfpathmoveto{\pgfqpoint{0.000000in}{0.000000in}}%
\pgfpathlineto{\pgfqpoint{0.000000in}{-0.048611in}}%
\pgfusepath{stroke,fill}%
}%
\begin{pgfscope}%
\pgfsys@transformshift{1.624659in}{0.499444in}%
\pgfsys@useobject{currentmarker}{}%
\end{pgfscope}%
\end{pgfscope}%
\begin{pgfscope}%
\definecolor{textcolor}{rgb}{0.000000,0.000000,0.000000}%
\pgfsetstrokecolor{textcolor}%
\pgfsetfillcolor{textcolor}%
\pgftext[x=1.624659in,y=0.402222in,,top]{\color{textcolor}\rmfamily\fontsize{10.000000}{12.000000}\selectfont 0.3}%
\end{pgfscope}%
\begin{pgfscope}%
\pgfsetbuttcap%
\pgfsetroundjoin%
\definecolor{currentfill}{rgb}{0.000000,0.000000,0.000000}%
\pgfsetfillcolor{currentfill}%
\pgfsetlinewidth{0.803000pt}%
\definecolor{currentstroke}{rgb}{0.000000,0.000000,0.000000}%
\pgfsetstrokecolor{currentstroke}%
\pgfsetdash{}{0pt}%
\pgfsys@defobject{currentmarker}{\pgfqpoint{0.000000in}{-0.048611in}}{\pgfqpoint{0.000000in}{0.000000in}}{%
\pgfpathmoveto{\pgfqpoint{0.000000in}{0.000000in}}%
\pgfpathlineto{\pgfqpoint{0.000000in}{-0.048611in}}%
\pgfusepath{stroke,fill}%
}%
\begin{pgfscope}%
\pgfsys@transformshift{1.783182in}{0.499444in}%
\pgfsys@useobject{currentmarker}{}%
\end{pgfscope}%
\end{pgfscope}%
\begin{pgfscope}%
\pgfsetbuttcap%
\pgfsetroundjoin%
\definecolor{currentfill}{rgb}{0.000000,0.000000,0.000000}%
\pgfsetfillcolor{currentfill}%
\pgfsetlinewidth{0.803000pt}%
\definecolor{currentstroke}{rgb}{0.000000,0.000000,0.000000}%
\pgfsetstrokecolor{currentstroke}%
\pgfsetdash{}{0pt}%
\pgfsys@defobject{currentmarker}{\pgfqpoint{0.000000in}{-0.048611in}}{\pgfqpoint{0.000000in}{0.000000in}}{%
\pgfpathmoveto{\pgfqpoint{0.000000in}{0.000000in}}%
\pgfpathlineto{\pgfqpoint{0.000000in}{-0.048611in}}%
\pgfusepath{stroke,fill}%
}%
\begin{pgfscope}%
\pgfsys@transformshift{1.941705in}{0.499444in}%
\pgfsys@useobject{currentmarker}{}%
\end{pgfscope}%
\end{pgfscope}%
\begin{pgfscope}%
\definecolor{textcolor}{rgb}{0.000000,0.000000,0.000000}%
\pgfsetstrokecolor{textcolor}%
\pgfsetfillcolor{textcolor}%
\pgftext[x=1.941705in,y=0.402222in,,top]{\color{textcolor}\rmfamily\fontsize{10.000000}{12.000000}\selectfont 0.4}%
\end{pgfscope}%
\begin{pgfscope}%
\pgfsetbuttcap%
\pgfsetroundjoin%
\definecolor{currentfill}{rgb}{0.000000,0.000000,0.000000}%
\pgfsetfillcolor{currentfill}%
\pgfsetlinewidth{0.803000pt}%
\definecolor{currentstroke}{rgb}{0.000000,0.000000,0.000000}%
\pgfsetstrokecolor{currentstroke}%
\pgfsetdash{}{0pt}%
\pgfsys@defobject{currentmarker}{\pgfqpoint{0.000000in}{-0.048611in}}{\pgfqpoint{0.000000in}{0.000000in}}{%
\pgfpathmoveto{\pgfqpoint{0.000000in}{0.000000in}}%
\pgfpathlineto{\pgfqpoint{0.000000in}{-0.048611in}}%
\pgfusepath{stroke,fill}%
}%
\begin{pgfscope}%
\pgfsys@transformshift{2.100228in}{0.499444in}%
\pgfsys@useobject{currentmarker}{}%
\end{pgfscope}%
\end{pgfscope}%
\begin{pgfscope}%
\pgfsetbuttcap%
\pgfsetroundjoin%
\definecolor{currentfill}{rgb}{0.000000,0.000000,0.000000}%
\pgfsetfillcolor{currentfill}%
\pgfsetlinewidth{0.803000pt}%
\definecolor{currentstroke}{rgb}{0.000000,0.000000,0.000000}%
\pgfsetstrokecolor{currentstroke}%
\pgfsetdash{}{0pt}%
\pgfsys@defobject{currentmarker}{\pgfqpoint{0.000000in}{-0.048611in}}{\pgfqpoint{0.000000in}{0.000000in}}{%
\pgfpathmoveto{\pgfqpoint{0.000000in}{0.000000in}}%
\pgfpathlineto{\pgfqpoint{0.000000in}{-0.048611in}}%
\pgfusepath{stroke,fill}%
}%
\begin{pgfscope}%
\pgfsys@transformshift{2.258750in}{0.499444in}%
\pgfsys@useobject{currentmarker}{}%
\end{pgfscope}%
\end{pgfscope}%
\begin{pgfscope}%
\definecolor{textcolor}{rgb}{0.000000,0.000000,0.000000}%
\pgfsetstrokecolor{textcolor}%
\pgfsetfillcolor{textcolor}%
\pgftext[x=2.258750in,y=0.402222in,,top]{\color{textcolor}\rmfamily\fontsize{10.000000}{12.000000}\selectfont 0.5}%
\end{pgfscope}%
\begin{pgfscope}%
\pgfsetbuttcap%
\pgfsetroundjoin%
\definecolor{currentfill}{rgb}{0.000000,0.000000,0.000000}%
\pgfsetfillcolor{currentfill}%
\pgfsetlinewidth{0.803000pt}%
\definecolor{currentstroke}{rgb}{0.000000,0.000000,0.000000}%
\pgfsetstrokecolor{currentstroke}%
\pgfsetdash{}{0pt}%
\pgfsys@defobject{currentmarker}{\pgfqpoint{0.000000in}{-0.048611in}}{\pgfqpoint{0.000000in}{0.000000in}}{%
\pgfpathmoveto{\pgfqpoint{0.000000in}{0.000000in}}%
\pgfpathlineto{\pgfqpoint{0.000000in}{-0.048611in}}%
\pgfusepath{stroke,fill}%
}%
\begin{pgfscope}%
\pgfsys@transformshift{2.417273in}{0.499444in}%
\pgfsys@useobject{currentmarker}{}%
\end{pgfscope}%
\end{pgfscope}%
\begin{pgfscope}%
\pgfsetbuttcap%
\pgfsetroundjoin%
\definecolor{currentfill}{rgb}{0.000000,0.000000,0.000000}%
\pgfsetfillcolor{currentfill}%
\pgfsetlinewidth{0.803000pt}%
\definecolor{currentstroke}{rgb}{0.000000,0.000000,0.000000}%
\pgfsetstrokecolor{currentstroke}%
\pgfsetdash{}{0pt}%
\pgfsys@defobject{currentmarker}{\pgfqpoint{0.000000in}{-0.048611in}}{\pgfqpoint{0.000000in}{0.000000in}}{%
\pgfpathmoveto{\pgfqpoint{0.000000in}{0.000000in}}%
\pgfpathlineto{\pgfqpoint{0.000000in}{-0.048611in}}%
\pgfusepath{stroke,fill}%
}%
\begin{pgfscope}%
\pgfsys@transformshift{2.575796in}{0.499444in}%
\pgfsys@useobject{currentmarker}{}%
\end{pgfscope}%
\end{pgfscope}%
\begin{pgfscope}%
\definecolor{textcolor}{rgb}{0.000000,0.000000,0.000000}%
\pgfsetstrokecolor{textcolor}%
\pgfsetfillcolor{textcolor}%
\pgftext[x=2.575796in,y=0.402222in,,top]{\color{textcolor}\rmfamily\fontsize{10.000000}{12.000000}\selectfont 0.6}%
\end{pgfscope}%
\begin{pgfscope}%
\pgfsetbuttcap%
\pgfsetroundjoin%
\definecolor{currentfill}{rgb}{0.000000,0.000000,0.000000}%
\pgfsetfillcolor{currentfill}%
\pgfsetlinewidth{0.803000pt}%
\definecolor{currentstroke}{rgb}{0.000000,0.000000,0.000000}%
\pgfsetstrokecolor{currentstroke}%
\pgfsetdash{}{0pt}%
\pgfsys@defobject{currentmarker}{\pgfqpoint{0.000000in}{-0.048611in}}{\pgfqpoint{0.000000in}{0.000000in}}{%
\pgfpathmoveto{\pgfqpoint{0.000000in}{0.000000in}}%
\pgfpathlineto{\pgfqpoint{0.000000in}{-0.048611in}}%
\pgfusepath{stroke,fill}%
}%
\begin{pgfscope}%
\pgfsys@transformshift{2.734318in}{0.499444in}%
\pgfsys@useobject{currentmarker}{}%
\end{pgfscope}%
\end{pgfscope}%
\begin{pgfscope}%
\pgfsetbuttcap%
\pgfsetroundjoin%
\definecolor{currentfill}{rgb}{0.000000,0.000000,0.000000}%
\pgfsetfillcolor{currentfill}%
\pgfsetlinewidth{0.803000pt}%
\definecolor{currentstroke}{rgb}{0.000000,0.000000,0.000000}%
\pgfsetstrokecolor{currentstroke}%
\pgfsetdash{}{0pt}%
\pgfsys@defobject{currentmarker}{\pgfqpoint{0.000000in}{-0.048611in}}{\pgfqpoint{0.000000in}{0.000000in}}{%
\pgfpathmoveto{\pgfqpoint{0.000000in}{0.000000in}}%
\pgfpathlineto{\pgfqpoint{0.000000in}{-0.048611in}}%
\pgfusepath{stroke,fill}%
}%
\begin{pgfscope}%
\pgfsys@transformshift{2.892841in}{0.499444in}%
\pgfsys@useobject{currentmarker}{}%
\end{pgfscope}%
\end{pgfscope}%
\begin{pgfscope}%
\definecolor{textcolor}{rgb}{0.000000,0.000000,0.000000}%
\pgfsetstrokecolor{textcolor}%
\pgfsetfillcolor{textcolor}%
\pgftext[x=2.892841in,y=0.402222in,,top]{\color{textcolor}\rmfamily\fontsize{10.000000}{12.000000}\selectfont 0.7}%
\end{pgfscope}%
\begin{pgfscope}%
\pgfsetbuttcap%
\pgfsetroundjoin%
\definecolor{currentfill}{rgb}{0.000000,0.000000,0.000000}%
\pgfsetfillcolor{currentfill}%
\pgfsetlinewidth{0.803000pt}%
\definecolor{currentstroke}{rgb}{0.000000,0.000000,0.000000}%
\pgfsetstrokecolor{currentstroke}%
\pgfsetdash{}{0pt}%
\pgfsys@defobject{currentmarker}{\pgfqpoint{0.000000in}{-0.048611in}}{\pgfqpoint{0.000000in}{0.000000in}}{%
\pgfpathmoveto{\pgfqpoint{0.000000in}{0.000000in}}%
\pgfpathlineto{\pgfqpoint{0.000000in}{-0.048611in}}%
\pgfusepath{stroke,fill}%
}%
\begin{pgfscope}%
\pgfsys@transformshift{3.051364in}{0.499444in}%
\pgfsys@useobject{currentmarker}{}%
\end{pgfscope}%
\end{pgfscope}%
\begin{pgfscope}%
\pgfsetbuttcap%
\pgfsetroundjoin%
\definecolor{currentfill}{rgb}{0.000000,0.000000,0.000000}%
\pgfsetfillcolor{currentfill}%
\pgfsetlinewidth{0.803000pt}%
\definecolor{currentstroke}{rgb}{0.000000,0.000000,0.000000}%
\pgfsetstrokecolor{currentstroke}%
\pgfsetdash{}{0pt}%
\pgfsys@defobject{currentmarker}{\pgfqpoint{0.000000in}{-0.048611in}}{\pgfqpoint{0.000000in}{0.000000in}}{%
\pgfpathmoveto{\pgfqpoint{0.000000in}{0.000000in}}%
\pgfpathlineto{\pgfqpoint{0.000000in}{-0.048611in}}%
\pgfusepath{stroke,fill}%
}%
\begin{pgfscope}%
\pgfsys@transformshift{3.209887in}{0.499444in}%
\pgfsys@useobject{currentmarker}{}%
\end{pgfscope}%
\end{pgfscope}%
\begin{pgfscope}%
\definecolor{textcolor}{rgb}{0.000000,0.000000,0.000000}%
\pgfsetstrokecolor{textcolor}%
\pgfsetfillcolor{textcolor}%
\pgftext[x=3.209887in,y=0.402222in,,top]{\color{textcolor}\rmfamily\fontsize{10.000000}{12.000000}\selectfont 0.8}%
\end{pgfscope}%
\begin{pgfscope}%
\pgfsetbuttcap%
\pgfsetroundjoin%
\definecolor{currentfill}{rgb}{0.000000,0.000000,0.000000}%
\pgfsetfillcolor{currentfill}%
\pgfsetlinewidth{0.803000pt}%
\definecolor{currentstroke}{rgb}{0.000000,0.000000,0.000000}%
\pgfsetstrokecolor{currentstroke}%
\pgfsetdash{}{0pt}%
\pgfsys@defobject{currentmarker}{\pgfqpoint{0.000000in}{-0.048611in}}{\pgfqpoint{0.000000in}{0.000000in}}{%
\pgfpathmoveto{\pgfqpoint{0.000000in}{0.000000in}}%
\pgfpathlineto{\pgfqpoint{0.000000in}{-0.048611in}}%
\pgfusepath{stroke,fill}%
}%
\begin{pgfscope}%
\pgfsys@transformshift{3.368409in}{0.499444in}%
\pgfsys@useobject{currentmarker}{}%
\end{pgfscope}%
\end{pgfscope}%
\begin{pgfscope}%
\pgfsetbuttcap%
\pgfsetroundjoin%
\definecolor{currentfill}{rgb}{0.000000,0.000000,0.000000}%
\pgfsetfillcolor{currentfill}%
\pgfsetlinewidth{0.803000pt}%
\definecolor{currentstroke}{rgb}{0.000000,0.000000,0.000000}%
\pgfsetstrokecolor{currentstroke}%
\pgfsetdash{}{0pt}%
\pgfsys@defobject{currentmarker}{\pgfqpoint{0.000000in}{-0.048611in}}{\pgfqpoint{0.000000in}{0.000000in}}{%
\pgfpathmoveto{\pgfqpoint{0.000000in}{0.000000in}}%
\pgfpathlineto{\pgfqpoint{0.000000in}{-0.048611in}}%
\pgfusepath{stroke,fill}%
}%
\begin{pgfscope}%
\pgfsys@transformshift{3.526932in}{0.499444in}%
\pgfsys@useobject{currentmarker}{}%
\end{pgfscope}%
\end{pgfscope}%
\begin{pgfscope}%
\definecolor{textcolor}{rgb}{0.000000,0.000000,0.000000}%
\pgfsetstrokecolor{textcolor}%
\pgfsetfillcolor{textcolor}%
\pgftext[x=3.526932in,y=0.402222in,,top]{\color{textcolor}\rmfamily\fontsize{10.000000}{12.000000}\selectfont 0.9}%
\end{pgfscope}%
\begin{pgfscope}%
\pgfsetbuttcap%
\pgfsetroundjoin%
\definecolor{currentfill}{rgb}{0.000000,0.000000,0.000000}%
\pgfsetfillcolor{currentfill}%
\pgfsetlinewidth{0.803000pt}%
\definecolor{currentstroke}{rgb}{0.000000,0.000000,0.000000}%
\pgfsetstrokecolor{currentstroke}%
\pgfsetdash{}{0pt}%
\pgfsys@defobject{currentmarker}{\pgfqpoint{0.000000in}{-0.048611in}}{\pgfqpoint{0.000000in}{0.000000in}}{%
\pgfpathmoveto{\pgfqpoint{0.000000in}{0.000000in}}%
\pgfpathlineto{\pgfqpoint{0.000000in}{-0.048611in}}%
\pgfusepath{stroke,fill}%
}%
\begin{pgfscope}%
\pgfsys@transformshift{3.685455in}{0.499444in}%
\pgfsys@useobject{currentmarker}{}%
\end{pgfscope}%
\end{pgfscope}%
\begin{pgfscope}%
\pgfsetbuttcap%
\pgfsetroundjoin%
\definecolor{currentfill}{rgb}{0.000000,0.000000,0.000000}%
\pgfsetfillcolor{currentfill}%
\pgfsetlinewidth{0.803000pt}%
\definecolor{currentstroke}{rgb}{0.000000,0.000000,0.000000}%
\pgfsetstrokecolor{currentstroke}%
\pgfsetdash{}{0pt}%
\pgfsys@defobject{currentmarker}{\pgfqpoint{0.000000in}{-0.048611in}}{\pgfqpoint{0.000000in}{0.000000in}}{%
\pgfpathmoveto{\pgfqpoint{0.000000in}{0.000000in}}%
\pgfpathlineto{\pgfqpoint{0.000000in}{-0.048611in}}%
\pgfusepath{stroke,fill}%
}%
\begin{pgfscope}%
\pgfsys@transformshift{3.843978in}{0.499444in}%
\pgfsys@useobject{currentmarker}{}%
\end{pgfscope}%
\end{pgfscope}%
\begin{pgfscope}%
\definecolor{textcolor}{rgb}{0.000000,0.000000,0.000000}%
\pgfsetstrokecolor{textcolor}%
\pgfsetfillcolor{textcolor}%
\pgftext[x=3.843978in,y=0.402222in,,top]{\color{textcolor}\rmfamily\fontsize{10.000000}{12.000000}\selectfont 1.0}%
\end{pgfscope}%
\begin{pgfscope}%
\pgfsetbuttcap%
\pgfsetroundjoin%
\definecolor{currentfill}{rgb}{0.000000,0.000000,0.000000}%
\pgfsetfillcolor{currentfill}%
\pgfsetlinewidth{0.803000pt}%
\definecolor{currentstroke}{rgb}{0.000000,0.000000,0.000000}%
\pgfsetstrokecolor{currentstroke}%
\pgfsetdash{}{0pt}%
\pgfsys@defobject{currentmarker}{\pgfqpoint{0.000000in}{-0.048611in}}{\pgfqpoint{0.000000in}{0.000000in}}{%
\pgfpathmoveto{\pgfqpoint{0.000000in}{0.000000in}}%
\pgfpathlineto{\pgfqpoint{0.000000in}{-0.048611in}}%
\pgfusepath{stroke,fill}%
}%
\begin{pgfscope}%
\pgfsys@transformshift{4.002500in}{0.499444in}%
\pgfsys@useobject{currentmarker}{}%
\end{pgfscope}%
\end{pgfscope}%
\begin{pgfscope}%
\definecolor{textcolor}{rgb}{0.000000,0.000000,0.000000}%
\pgfsetstrokecolor{textcolor}%
\pgfsetfillcolor{textcolor}%
\pgftext[x=2.258750in,y=0.223333in,,top]{\color{textcolor}\rmfamily\fontsize{10.000000}{12.000000}\selectfont \(\displaystyle p\)}%
\end{pgfscope}%
\begin{pgfscope}%
\pgfsetbuttcap%
\pgfsetroundjoin%
\definecolor{currentfill}{rgb}{0.000000,0.000000,0.000000}%
\pgfsetfillcolor{currentfill}%
\pgfsetlinewidth{0.803000pt}%
\definecolor{currentstroke}{rgb}{0.000000,0.000000,0.000000}%
\pgfsetstrokecolor{currentstroke}%
\pgfsetdash{}{0pt}%
\pgfsys@defobject{currentmarker}{\pgfqpoint{-0.048611in}{0.000000in}}{\pgfqpoint{-0.000000in}{0.000000in}}{%
\pgfpathmoveto{\pgfqpoint{-0.000000in}{0.000000in}}%
\pgfpathlineto{\pgfqpoint{-0.048611in}{0.000000in}}%
\pgfusepath{stroke,fill}%
}%
\begin{pgfscope}%
\pgfsys@transformshift{0.515000in}{0.499444in}%
\pgfsys@useobject{currentmarker}{}%
\end{pgfscope}%
\end{pgfscope}%
\begin{pgfscope}%
\definecolor{textcolor}{rgb}{0.000000,0.000000,0.000000}%
\pgfsetstrokecolor{textcolor}%
\pgfsetfillcolor{textcolor}%
\pgftext[x=0.348333in, y=0.451250in, left, base]{\color{textcolor}\rmfamily\fontsize{10.000000}{12.000000}\selectfont \(\displaystyle {0}\)}%
\end{pgfscope}%
\begin{pgfscope}%
\pgfsetbuttcap%
\pgfsetroundjoin%
\definecolor{currentfill}{rgb}{0.000000,0.000000,0.000000}%
\pgfsetfillcolor{currentfill}%
\pgfsetlinewidth{0.803000pt}%
\definecolor{currentstroke}{rgb}{0.000000,0.000000,0.000000}%
\pgfsetstrokecolor{currentstroke}%
\pgfsetdash{}{0pt}%
\pgfsys@defobject{currentmarker}{\pgfqpoint{-0.048611in}{0.000000in}}{\pgfqpoint{-0.000000in}{0.000000in}}{%
\pgfpathmoveto{\pgfqpoint{-0.000000in}{0.000000in}}%
\pgfpathlineto{\pgfqpoint{-0.048611in}{0.000000in}}%
\pgfusepath{stroke,fill}%
}%
\begin{pgfscope}%
\pgfsys@transformshift{0.515000in}{0.927911in}%
\pgfsys@useobject{currentmarker}{}%
\end{pgfscope}%
\end{pgfscope}%
\begin{pgfscope}%
\definecolor{textcolor}{rgb}{0.000000,0.000000,0.000000}%
\pgfsetstrokecolor{textcolor}%
\pgfsetfillcolor{textcolor}%
\pgftext[x=0.278889in, y=0.879717in, left, base]{\color{textcolor}\rmfamily\fontsize{10.000000}{12.000000}\selectfont \(\displaystyle {10}\)}%
\end{pgfscope}%
\begin{pgfscope}%
\pgfsetbuttcap%
\pgfsetroundjoin%
\definecolor{currentfill}{rgb}{0.000000,0.000000,0.000000}%
\pgfsetfillcolor{currentfill}%
\pgfsetlinewidth{0.803000pt}%
\definecolor{currentstroke}{rgb}{0.000000,0.000000,0.000000}%
\pgfsetstrokecolor{currentstroke}%
\pgfsetdash{}{0pt}%
\pgfsys@defobject{currentmarker}{\pgfqpoint{-0.048611in}{0.000000in}}{\pgfqpoint{-0.000000in}{0.000000in}}{%
\pgfpathmoveto{\pgfqpoint{-0.000000in}{0.000000in}}%
\pgfpathlineto{\pgfqpoint{-0.048611in}{0.000000in}}%
\pgfusepath{stroke,fill}%
}%
\begin{pgfscope}%
\pgfsys@transformshift{0.515000in}{1.356379in}%
\pgfsys@useobject{currentmarker}{}%
\end{pgfscope}%
\end{pgfscope}%
\begin{pgfscope}%
\definecolor{textcolor}{rgb}{0.000000,0.000000,0.000000}%
\pgfsetstrokecolor{textcolor}%
\pgfsetfillcolor{textcolor}%
\pgftext[x=0.278889in, y=1.308184in, left, base]{\color{textcolor}\rmfamily\fontsize{10.000000}{12.000000}\selectfont \(\displaystyle {20}\)}%
\end{pgfscope}%
\begin{pgfscope}%
\definecolor{textcolor}{rgb}{0.000000,0.000000,0.000000}%
\pgfsetstrokecolor{textcolor}%
\pgfsetfillcolor{textcolor}%
\pgftext[x=0.223333in,y=1.076944in,,bottom,rotate=90.000000]{\color{textcolor}\rmfamily\fontsize{10.000000}{12.000000}\selectfont Percent of Data Set}%
\end{pgfscope}%
\begin{pgfscope}%
\pgfsetrectcap%
\pgfsetmiterjoin%
\pgfsetlinewidth{0.803000pt}%
\definecolor{currentstroke}{rgb}{0.000000,0.000000,0.000000}%
\pgfsetstrokecolor{currentstroke}%
\pgfsetdash{}{0pt}%
\pgfpathmoveto{\pgfqpoint{0.515000in}{0.499444in}}%
\pgfpathlineto{\pgfqpoint{0.515000in}{1.654444in}}%
\pgfusepath{stroke}%
\end{pgfscope}%
\begin{pgfscope}%
\pgfsetrectcap%
\pgfsetmiterjoin%
\pgfsetlinewidth{0.803000pt}%
\definecolor{currentstroke}{rgb}{0.000000,0.000000,0.000000}%
\pgfsetstrokecolor{currentstroke}%
\pgfsetdash{}{0pt}%
\pgfpathmoveto{\pgfqpoint{4.002500in}{0.499444in}}%
\pgfpathlineto{\pgfqpoint{4.002500in}{1.654444in}}%
\pgfusepath{stroke}%
\end{pgfscope}%
\begin{pgfscope}%
\pgfsetrectcap%
\pgfsetmiterjoin%
\pgfsetlinewidth{0.803000pt}%
\definecolor{currentstroke}{rgb}{0.000000,0.000000,0.000000}%
\pgfsetstrokecolor{currentstroke}%
\pgfsetdash{}{0pt}%
\pgfpathmoveto{\pgfqpoint{0.515000in}{0.499444in}}%
\pgfpathlineto{\pgfqpoint{4.002500in}{0.499444in}}%
\pgfusepath{stroke}%
\end{pgfscope}%
\begin{pgfscope}%
\pgfsetrectcap%
\pgfsetmiterjoin%
\pgfsetlinewidth{0.803000pt}%
\definecolor{currentstroke}{rgb}{0.000000,0.000000,0.000000}%
\pgfsetstrokecolor{currentstroke}%
\pgfsetdash{}{0pt}%
\pgfpathmoveto{\pgfqpoint{0.515000in}{1.654444in}}%
\pgfpathlineto{\pgfqpoint{4.002500in}{1.654444in}}%
\pgfusepath{stroke}%
\end{pgfscope}%
\begin{pgfscope}%
\pgfsetbuttcap%
\pgfsetmiterjoin%
\definecolor{currentfill}{rgb}{1.000000,1.000000,1.000000}%
\pgfsetfillcolor{currentfill}%
\pgfsetfillopacity{0.800000}%
\pgfsetlinewidth{1.003750pt}%
\definecolor{currentstroke}{rgb}{0.800000,0.800000,0.800000}%
\pgfsetstrokecolor{currentstroke}%
\pgfsetstrokeopacity{0.800000}%
\pgfsetdash{}{0pt}%
\pgfpathmoveto{\pgfqpoint{3.225556in}{1.154445in}}%
\pgfpathlineto{\pgfqpoint{3.905278in}{1.154445in}}%
\pgfpathquadraticcurveto{\pgfqpoint{3.933056in}{1.154445in}}{\pgfqpoint{3.933056in}{1.182222in}}%
\pgfpathlineto{\pgfqpoint{3.933056in}{1.557222in}}%
\pgfpathquadraticcurveto{\pgfqpoint{3.933056in}{1.585000in}}{\pgfqpoint{3.905278in}{1.585000in}}%
\pgfpathlineto{\pgfqpoint{3.225556in}{1.585000in}}%
\pgfpathquadraticcurveto{\pgfqpoint{3.197778in}{1.585000in}}{\pgfqpoint{3.197778in}{1.557222in}}%
\pgfpathlineto{\pgfqpoint{3.197778in}{1.182222in}}%
\pgfpathquadraticcurveto{\pgfqpoint{3.197778in}{1.154445in}}{\pgfqpoint{3.225556in}{1.154445in}}%
\pgfpathlineto{\pgfqpoint{3.225556in}{1.154445in}}%
\pgfpathclose%
\pgfusepath{stroke,fill}%
\end{pgfscope}%
\begin{pgfscope}%
\pgfsetbuttcap%
\pgfsetmiterjoin%
\pgfsetlinewidth{1.003750pt}%
\definecolor{currentstroke}{rgb}{0.000000,0.000000,0.000000}%
\pgfsetstrokecolor{currentstroke}%
\pgfsetdash{}{0pt}%
\pgfpathmoveto{\pgfqpoint{3.253334in}{1.432222in}}%
\pgfpathlineto{\pgfqpoint{3.531111in}{1.432222in}}%
\pgfpathlineto{\pgfqpoint{3.531111in}{1.529444in}}%
\pgfpathlineto{\pgfqpoint{3.253334in}{1.529444in}}%
\pgfpathlineto{\pgfqpoint{3.253334in}{1.432222in}}%
\pgfpathclose%
\pgfusepath{stroke}%
\end{pgfscope}%
\begin{pgfscope}%
\definecolor{textcolor}{rgb}{0.000000,0.000000,0.000000}%
\pgfsetstrokecolor{textcolor}%
\pgfsetfillcolor{textcolor}%
\pgftext[x=3.642223in,y=1.432222in,left,base]{\color{textcolor}\rmfamily\fontsize{10.000000}{12.000000}\selectfont Neg}%
\end{pgfscope}%
\begin{pgfscope}%
\pgfsetbuttcap%
\pgfsetmiterjoin%
\definecolor{currentfill}{rgb}{0.000000,0.000000,0.000000}%
\pgfsetfillcolor{currentfill}%
\pgfsetlinewidth{0.000000pt}%
\definecolor{currentstroke}{rgb}{0.000000,0.000000,0.000000}%
\pgfsetstrokecolor{currentstroke}%
\pgfsetstrokeopacity{0.000000}%
\pgfsetdash{}{0pt}%
\pgfpathmoveto{\pgfqpoint{3.253334in}{1.236944in}}%
\pgfpathlineto{\pgfqpoint{3.531111in}{1.236944in}}%
\pgfpathlineto{\pgfqpoint{3.531111in}{1.334167in}}%
\pgfpathlineto{\pgfqpoint{3.253334in}{1.334167in}}%
\pgfpathlineto{\pgfqpoint{3.253334in}{1.236944in}}%
\pgfpathclose%
\pgfusepath{fill}%
\end{pgfscope}%
\begin{pgfscope}%
\definecolor{textcolor}{rgb}{0.000000,0.000000,0.000000}%
\pgfsetstrokecolor{textcolor}%
\pgfsetfillcolor{textcolor}%
\pgftext[x=3.642223in,y=1.236944in,left,base]{\color{textcolor}\rmfamily\fontsize{10.000000}{12.000000}\selectfont Pos}%
\end{pgfscope}%
\end{pgfpicture}%
\makeatother%
\endgroup%

&
	\vskip 0pt
	\qquad \qquad ROC Curve
	
	%% Creator: Matplotlib, PGF backend
%%
%% To include the figure in your LaTeX document, write
%%   \input{<filename>.pgf}
%%
%% Make sure the required packages are loaded in your preamble
%%   \usepackage{pgf}
%%
%% Also ensure that all the required font packages are loaded; for instance,
%% the lmodern package is sometimes necessary when using math font.
%%   \usepackage{lmodern}
%%
%% Figures using additional raster images can only be included by \input if
%% they are in the same directory as the main LaTeX file. For loading figures
%% from other directories you can use the `import` package
%%   \usepackage{import}
%%
%% and then include the figures with
%%   \import{<path to file>}{<filename>.pgf}
%%
%% Matplotlib used the following preamble
%%   
%%   \usepackage{fontspec}
%%   \makeatletter\@ifpackageloaded{underscore}{}{\usepackage[strings]{underscore}}\makeatother
%%
\begingroup%
\makeatletter%
\begin{pgfpicture}%
\pgfpathrectangle{\pgfpointorigin}{\pgfqpoint{2.221861in}{1.754444in}}%
\pgfusepath{use as bounding box, clip}%
\begin{pgfscope}%
\pgfsetbuttcap%
\pgfsetmiterjoin%
\definecolor{currentfill}{rgb}{1.000000,1.000000,1.000000}%
\pgfsetfillcolor{currentfill}%
\pgfsetlinewidth{0.000000pt}%
\definecolor{currentstroke}{rgb}{1.000000,1.000000,1.000000}%
\pgfsetstrokecolor{currentstroke}%
\pgfsetdash{}{0pt}%
\pgfpathmoveto{\pgfqpoint{0.000000in}{0.000000in}}%
\pgfpathlineto{\pgfqpoint{2.221861in}{0.000000in}}%
\pgfpathlineto{\pgfqpoint{2.221861in}{1.754444in}}%
\pgfpathlineto{\pgfqpoint{0.000000in}{1.754444in}}%
\pgfpathlineto{\pgfqpoint{0.000000in}{0.000000in}}%
\pgfpathclose%
\pgfusepath{fill}%
\end{pgfscope}%
\begin{pgfscope}%
\pgfsetbuttcap%
\pgfsetmiterjoin%
\definecolor{currentfill}{rgb}{1.000000,1.000000,1.000000}%
\pgfsetfillcolor{currentfill}%
\pgfsetlinewidth{0.000000pt}%
\definecolor{currentstroke}{rgb}{0.000000,0.000000,0.000000}%
\pgfsetstrokecolor{currentstroke}%
\pgfsetstrokeopacity{0.000000}%
\pgfsetdash{}{0pt}%
\pgfpathmoveto{\pgfqpoint{0.553581in}{0.499444in}}%
\pgfpathlineto{\pgfqpoint{2.103581in}{0.499444in}}%
\pgfpathlineto{\pgfqpoint{2.103581in}{1.654444in}}%
\pgfpathlineto{\pgfqpoint{0.553581in}{1.654444in}}%
\pgfpathlineto{\pgfqpoint{0.553581in}{0.499444in}}%
\pgfpathclose%
\pgfusepath{fill}%
\end{pgfscope}%
\begin{pgfscope}%
\pgfsetbuttcap%
\pgfsetroundjoin%
\definecolor{currentfill}{rgb}{0.000000,0.000000,0.000000}%
\pgfsetfillcolor{currentfill}%
\pgfsetlinewidth{0.803000pt}%
\definecolor{currentstroke}{rgb}{0.000000,0.000000,0.000000}%
\pgfsetstrokecolor{currentstroke}%
\pgfsetdash{}{0pt}%
\pgfsys@defobject{currentmarker}{\pgfqpoint{0.000000in}{-0.048611in}}{\pgfqpoint{0.000000in}{0.000000in}}{%
\pgfpathmoveto{\pgfqpoint{0.000000in}{0.000000in}}%
\pgfpathlineto{\pgfqpoint{0.000000in}{-0.048611in}}%
\pgfusepath{stroke,fill}%
}%
\begin{pgfscope}%
\pgfsys@transformshift{0.624035in}{0.499444in}%
\pgfsys@useobject{currentmarker}{}%
\end{pgfscope}%
\end{pgfscope}%
\begin{pgfscope}%
\definecolor{textcolor}{rgb}{0.000000,0.000000,0.000000}%
\pgfsetstrokecolor{textcolor}%
\pgfsetfillcolor{textcolor}%
\pgftext[x=0.624035in,y=0.402222in,,top]{\color{textcolor}\rmfamily\fontsize{10.000000}{12.000000}\selectfont \(\displaystyle {0.0}\)}%
\end{pgfscope}%
\begin{pgfscope}%
\pgfsetbuttcap%
\pgfsetroundjoin%
\definecolor{currentfill}{rgb}{0.000000,0.000000,0.000000}%
\pgfsetfillcolor{currentfill}%
\pgfsetlinewidth{0.803000pt}%
\definecolor{currentstroke}{rgb}{0.000000,0.000000,0.000000}%
\pgfsetstrokecolor{currentstroke}%
\pgfsetdash{}{0pt}%
\pgfsys@defobject{currentmarker}{\pgfqpoint{0.000000in}{-0.048611in}}{\pgfqpoint{0.000000in}{0.000000in}}{%
\pgfpathmoveto{\pgfqpoint{0.000000in}{0.000000in}}%
\pgfpathlineto{\pgfqpoint{0.000000in}{-0.048611in}}%
\pgfusepath{stroke,fill}%
}%
\begin{pgfscope}%
\pgfsys@transformshift{1.328581in}{0.499444in}%
\pgfsys@useobject{currentmarker}{}%
\end{pgfscope}%
\end{pgfscope}%
\begin{pgfscope}%
\definecolor{textcolor}{rgb}{0.000000,0.000000,0.000000}%
\pgfsetstrokecolor{textcolor}%
\pgfsetfillcolor{textcolor}%
\pgftext[x=1.328581in,y=0.402222in,,top]{\color{textcolor}\rmfamily\fontsize{10.000000}{12.000000}\selectfont \(\displaystyle {0.5}\)}%
\end{pgfscope}%
\begin{pgfscope}%
\pgfsetbuttcap%
\pgfsetroundjoin%
\definecolor{currentfill}{rgb}{0.000000,0.000000,0.000000}%
\pgfsetfillcolor{currentfill}%
\pgfsetlinewidth{0.803000pt}%
\definecolor{currentstroke}{rgb}{0.000000,0.000000,0.000000}%
\pgfsetstrokecolor{currentstroke}%
\pgfsetdash{}{0pt}%
\pgfsys@defobject{currentmarker}{\pgfqpoint{0.000000in}{-0.048611in}}{\pgfqpoint{0.000000in}{0.000000in}}{%
\pgfpathmoveto{\pgfqpoint{0.000000in}{0.000000in}}%
\pgfpathlineto{\pgfqpoint{0.000000in}{-0.048611in}}%
\pgfusepath{stroke,fill}%
}%
\begin{pgfscope}%
\pgfsys@transformshift{2.033126in}{0.499444in}%
\pgfsys@useobject{currentmarker}{}%
\end{pgfscope}%
\end{pgfscope}%
\begin{pgfscope}%
\definecolor{textcolor}{rgb}{0.000000,0.000000,0.000000}%
\pgfsetstrokecolor{textcolor}%
\pgfsetfillcolor{textcolor}%
\pgftext[x=2.033126in,y=0.402222in,,top]{\color{textcolor}\rmfamily\fontsize{10.000000}{12.000000}\selectfont \(\displaystyle {1.0}\)}%
\end{pgfscope}%
\begin{pgfscope}%
\definecolor{textcolor}{rgb}{0.000000,0.000000,0.000000}%
\pgfsetstrokecolor{textcolor}%
\pgfsetfillcolor{textcolor}%
\pgftext[x=1.328581in,y=0.223333in,,top]{\color{textcolor}\rmfamily\fontsize{10.000000}{12.000000}\selectfont False positive rate}%
\end{pgfscope}%
\begin{pgfscope}%
\pgfsetbuttcap%
\pgfsetroundjoin%
\definecolor{currentfill}{rgb}{0.000000,0.000000,0.000000}%
\pgfsetfillcolor{currentfill}%
\pgfsetlinewidth{0.803000pt}%
\definecolor{currentstroke}{rgb}{0.000000,0.000000,0.000000}%
\pgfsetstrokecolor{currentstroke}%
\pgfsetdash{}{0pt}%
\pgfsys@defobject{currentmarker}{\pgfqpoint{-0.048611in}{0.000000in}}{\pgfqpoint{-0.000000in}{0.000000in}}{%
\pgfpathmoveto{\pgfqpoint{-0.000000in}{0.000000in}}%
\pgfpathlineto{\pgfqpoint{-0.048611in}{0.000000in}}%
\pgfusepath{stroke,fill}%
}%
\begin{pgfscope}%
\pgfsys@transformshift{0.553581in}{0.551944in}%
\pgfsys@useobject{currentmarker}{}%
\end{pgfscope}%
\end{pgfscope}%
\begin{pgfscope}%
\definecolor{textcolor}{rgb}{0.000000,0.000000,0.000000}%
\pgfsetstrokecolor{textcolor}%
\pgfsetfillcolor{textcolor}%
\pgftext[x=0.278889in, y=0.503750in, left, base]{\color{textcolor}\rmfamily\fontsize{10.000000}{12.000000}\selectfont \(\displaystyle {0.0}\)}%
\end{pgfscope}%
\begin{pgfscope}%
\pgfsetbuttcap%
\pgfsetroundjoin%
\definecolor{currentfill}{rgb}{0.000000,0.000000,0.000000}%
\pgfsetfillcolor{currentfill}%
\pgfsetlinewidth{0.803000pt}%
\definecolor{currentstroke}{rgb}{0.000000,0.000000,0.000000}%
\pgfsetstrokecolor{currentstroke}%
\pgfsetdash{}{0pt}%
\pgfsys@defobject{currentmarker}{\pgfqpoint{-0.048611in}{0.000000in}}{\pgfqpoint{-0.000000in}{0.000000in}}{%
\pgfpathmoveto{\pgfqpoint{-0.000000in}{0.000000in}}%
\pgfpathlineto{\pgfqpoint{-0.048611in}{0.000000in}}%
\pgfusepath{stroke,fill}%
}%
\begin{pgfscope}%
\pgfsys@transformshift{0.553581in}{1.076944in}%
\pgfsys@useobject{currentmarker}{}%
\end{pgfscope}%
\end{pgfscope}%
\begin{pgfscope}%
\definecolor{textcolor}{rgb}{0.000000,0.000000,0.000000}%
\pgfsetstrokecolor{textcolor}%
\pgfsetfillcolor{textcolor}%
\pgftext[x=0.278889in, y=1.028750in, left, base]{\color{textcolor}\rmfamily\fontsize{10.000000}{12.000000}\selectfont \(\displaystyle {0.5}\)}%
\end{pgfscope}%
\begin{pgfscope}%
\pgfsetbuttcap%
\pgfsetroundjoin%
\definecolor{currentfill}{rgb}{0.000000,0.000000,0.000000}%
\pgfsetfillcolor{currentfill}%
\pgfsetlinewidth{0.803000pt}%
\definecolor{currentstroke}{rgb}{0.000000,0.000000,0.000000}%
\pgfsetstrokecolor{currentstroke}%
\pgfsetdash{}{0pt}%
\pgfsys@defobject{currentmarker}{\pgfqpoint{-0.048611in}{0.000000in}}{\pgfqpoint{-0.000000in}{0.000000in}}{%
\pgfpathmoveto{\pgfqpoint{-0.000000in}{0.000000in}}%
\pgfpathlineto{\pgfqpoint{-0.048611in}{0.000000in}}%
\pgfusepath{stroke,fill}%
}%
\begin{pgfscope}%
\pgfsys@transformshift{0.553581in}{1.601944in}%
\pgfsys@useobject{currentmarker}{}%
\end{pgfscope}%
\end{pgfscope}%
\begin{pgfscope}%
\definecolor{textcolor}{rgb}{0.000000,0.000000,0.000000}%
\pgfsetstrokecolor{textcolor}%
\pgfsetfillcolor{textcolor}%
\pgftext[x=0.278889in, y=1.553750in, left, base]{\color{textcolor}\rmfamily\fontsize{10.000000}{12.000000}\selectfont \(\displaystyle {1.0}\)}%
\end{pgfscope}%
\begin{pgfscope}%
\definecolor{textcolor}{rgb}{0.000000,0.000000,0.000000}%
\pgfsetstrokecolor{textcolor}%
\pgfsetfillcolor{textcolor}%
\pgftext[x=0.223333in,y=1.076944in,,bottom,rotate=90.000000]{\color{textcolor}\rmfamily\fontsize{10.000000}{12.000000}\selectfont True positive rate}%
\end{pgfscope}%
\begin{pgfscope}%
\pgfpathrectangle{\pgfqpoint{0.553581in}{0.499444in}}{\pgfqpoint{1.550000in}{1.155000in}}%
\pgfusepath{clip}%
\pgfsetbuttcap%
\pgfsetroundjoin%
\pgfsetlinewidth{1.505625pt}%
\definecolor{currentstroke}{rgb}{0.000000,0.000000,0.000000}%
\pgfsetstrokecolor{currentstroke}%
\pgfsetdash{{5.550000pt}{2.400000pt}}{0.000000pt}%
\pgfpathmoveto{\pgfqpoint{0.624035in}{0.551944in}}%
\pgfpathlineto{\pgfqpoint{2.033126in}{1.601944in}}%
\pgfusepath{stroke}%
\end{pgfscope}%
\begin{pgfscope}%
\pgfpathrectangle{\pgfqpoint{0.553581in}{0.499444in}}{\pgfqpoint{1.550000in}{1.155000in}}%
\pgfusepath{clip}%
\pgfsetrectcap%
\pgfsetroundjoin%
\pgfsetlinewidth{1.505625pt}%
\definecolor{currentstroke}{rgb}{0.000000,0.000000,0.000000}%
\pgfsetstrokecolor{currentstroke}%
\pgfsetdash{}{0pt}%
\pgfpathmoveto{\pgfqpoint{0.624035in}{0.551944in}}%
\pgfpathlineto{\pgfqpoint{0.625145in}{0.568893in}}%
\pgfpathlineto{\pgfqpoint{0.625239in}{0.569949in}}%
\pgfpathlineto{\pgfqpoint{0.626341in}{0.584414in}}%
\pgfpathlineto{\pgfqpoint{0.626388in}{0.585345in}}%
\pgfpathlineto{\pgfqpoint{0.627498in}{0.595310in}}%
\pgfpathlineto{\pgfqpoint{0.627616in}{0.596396in}}%
\pgfpathlineto{\pgfqpoint{0.628726in}{0.605306in}}%
\pgfpathlineto{\pgfqpoint{0.628781in}{0.606237in}}%
\pgfpathlineto{\pgfqpoint{0.629891in}{0.616698in}}%
\pgfpathlineto{\pgfqpoint{0.629992in}{0.617785in}}%
\pgfpathlineto{\pgfqpoint{0.631102in}{0.624986in}}%
\pgfpathlineto{\pgfqpoint{0.631235in}{0.625887in}}%
\pgfpathlineto{\pgfqpoint{0.632330in}{0.633150in}}%
\pgfpathlineto{\pgfqpoint{0.632533in}{0.634175in}}%
\pgfpathlineto{\pgfqpoint{0.633643in}{0.643208in}}%
\pgfpathlineto{\pgfqpoint{0.633885in}{0.644263in}}%
\pgfpathlineto{\pgfqpoint{0.634996in}{0.651869in}}%
\pgfpathlineto{\pgfqpoint{0.635128in}{0.652924in}}%
\pgfpathlineto{\pgfqpoint{0.636239in}{0.659785in}}%
\pgfpathlineto{\pgfqpoint{0.636442in}{0.660840in}}%
\pgfpathlineto{\pgfqpoint{0.637552in}{0.666862in}}%
\pgfpathlineto{\pgfqpoint{0.637833in}{0.667949in}}%
\pgfpathlineto{\pgfqpoint{0.638943in}{0.674219in}}%
\pgfpathlineto{\pgfqpoint{0.639084in}{0.675306in}}%
\pgfpathlineto{\pgfqpoint{0.640194in}{0.680955in}}%
\pgfpathlineto{\pgfqpoint{0.640444in}{0.682042in}}%
\pgfpathlineto{\pgfqpoint{0.641547in}{0.687567in}}%
\pgfpathlineto{\pgfqpoint{0.641766in}{0.688654in}}%
\pgfpathlineto{\pgfqpoint{0.642868in}{0.693124in}}%
\pgfpathlineto{\pgfqpoint{0.643040in}{0.694210in}}%
\pgfpathlineto{\pgfqpoint{0.644111in}{0.699549in}}%
\pgfpathlineto{\pgfqpoint{0.644431in}{0.700574in}}%
\pgfpathlineto{\pgfqpoint{0.645542in}{0.705696in}}%
\pgfpathlineto{\pgfqpoint{0.645901in}{0.706782in}}%
\pgfpathlineto{\pgfqpoint{0.647003in}{0.711687in}}%
\pgfpathlineto{\pgfqpoint{0.647308in}{0.712773in}}%
\pgfpathlineto{\pgfqpoint{0.648418in}{0.717585in}}%
\pgfpathlineto{\pgfqpoint{0.648661in}{0.718609in}}%
\pgfpathlineto{\pgfqpoint{0.649771in}{0.723390in}}%
\pgfpathlineto{\pgfqpoint{0.650076in}{0.724476in}}%
\pgfpathlineto{\pgfqpoint{0.651186in}{0.727953in}}%
\pgfpathlineto{\pgfqpoint{0.651483in}{0.729040in}}%
\pgfpathlineto{\pgfqpoint{0.652593in}{0.733820in}}%
\pgfpathlineto{\pgfqpoint{0.652921in}{0.734813in}}%
\pgfpathlineto{\pgfqpoint{0.654016in}{0.740059in}}%
\pgfpathlineto{\pgfqpoint{0.654462in}{0.741115in}}%
\pgfpathlineto{\pgfqpoint{0.655556in}{0.745740in}}%
\pgfpathlineto{\pgfqpoint{0.655861in}{0.746827in}}%
\pgfpathlineto{\pgfqpoint{0.656971in}{0.750334in}}%
\pgfpathlineto{\pgfqpoint{0.657323in}{0.751390in}}%
\pgfpathlineto{\pgfqpoint{0.658425in}{0.754711in}}%
\pgfpathlineto{\pgfqpoint{0.658800in}{0.755736in}}%
\pgfpathlineto{\pgfqpoint{0.659895in}{0.759492in}}%
\pgfpathlineto{\pgfqpoint{0.660161in}{0.760516in}}%
\pgfpathlineto{\pgfqpoint{0.661271in}{0.765173in}}%
\pgfpathlineto{\pgfqpoint{0.661544in}{0.766228in}}%
\pgfpathlineto{\pgfqpoint{0.662647in}{0.769922in}}%
\pgfpathlineto{\pgfqpoint{0.663037in}{0.770853in}}%
\pgfpathlineto{\pgfqpoint{0.664148in}{0.774144in}}%
\pgfpathlineto{\pgfqpoint{0.664531in}{0.775199in}}%
\pgfpathlineto{\pgfqpoint{0.665641in}{0.778769in}}%
\pgfpathlineto{\pgfqpoint{0.666094in}{0.779855in}}%
\pgfpathlineto{\pgfqpoint{0.667204in}{0.783239in}}%
\pgfpathlineto{\pgfqpoint{0.667462in}{0.784294in}}%
\pgfpathlineto{\pgfqpoint{0.668565in}{0.787802in}}%
\pgfpathlineto{\pgfqpoint{0.669010in}{0.788889in}}%
\pgfpathlineto{\pgfqpoint{0.670120in}{0.792521in}}%
\pgfpathlineto{\pgfqpoint{0.670511in}{0.793607in}}%
\pgfpathlineto{\pgfqpoint{0.671582in}{0.797860in}}%
\pgfpathlineto{\pgfqpoint{0.671981in}{0.798915in}}%
\pgfpathlineto{\pgfqpoint{0.671981in}{0.798946in}}%
\pgfpathlineto{\pgfqpoint{0.673091in}{0.802113in}}%
\pgfpathlineto{\pgfqpoint{0.673568in}{0.803199in}}%
\pgfpathlineto{\pgfqpoint{0.674670in}{0.806241in}}%
\pgfpathlineto{\pgfqpoint{0.675163in}{0.807328in}}%
\pgfpathlineto{\pgfqpoint{0.676265in}{0.810370in}}%
\pgfpathlineto{\pgfqpoint{0.676562in}{0.811425in}}%
\pgfpathlineto{\pgfqpoint{0.677672in}{0.814467in}}%
\pgfpathlineto{\pgfqpoint{0.678079in}{0.815461in}}%
\pgfpathlineto{\pgfqpoint{0.679181in}{0.819093in}}%
\pgfpathlineto{\pgfqpoint{0.679689in}{0.820148in}}%
\pgfpathlineto{\pgfqpoint{0.680776in}{0.823066in}}%
\pgfpathlineto{\pgfqpoint{0.681104in}{0.824091in}}%
\pgfpathlineto{\pgfqpoint{0.682214in}{0.826760in}}%
\pgfpathlineto{\pgfqpoint{0.682668in}{0.827847in}}%
\pgfpathlineto{\pgfqpoint{0.683778in}{0.831013in}}%
\pgfpathlineto{\pgfqpoint{0.684239in}{0.832099in}}%
\pgfpathlineto{\pgfqpoint{0.685302in}{0.834831in}}%
\pgfpathlineto{\pgfqpoint{0.685888in}{0.835918in}}%
\pgfpathlineto{\pgfqpoint{0.686991in}{0.839332in}}%
\pgfpathlineto{\pgfqpoint{0.687475in}{0.840419in}}%
\pgfpathlineto{\pgfqpoint{0.688578in}{0.843523in}}%
\pgfpathlineto{\pgfqpoint{0.689039in}{0.844578in}}%
\pgfpathlineto{\pgfqpoint{0.690141in}{0.846844in}}%
\pgfpathlineto{\pgfqpoint{0.690790in}{0.847931in}}%
\pgfpathlineto{\pgfqpoint{0.691900in}{0.850414in}}%
\pgfpathlineto{\pgfqpoint{0.692393in}{0.851439in}}%
\pgfpathlineto{\pgfqpoint{0.693495in}{0.853922in}}%
\pgfpathlineto{\pgfqpoint{0.693792in}{0.854915in}}%
\pgfpathlineto{\pgfqpoint{0.694871in}{0.857926in}}%
\pgfpathlineto{\pgfqpoint{0.695301in}{0.859013in}}%
\pgfpathlineto{\pgfqpoint{0.696395in}{0.861372in}}%
\pgfpathlineto{\pgfqpoint{0.696880in}{0.862459in}}%
\pgfpathlineto{\pgfqpoint{0.697990in}{0.865345in}}%
\pgfpathlineto{\pgfqpoint{0.698475in}{0.866370in}}%
\pgfpathlineto{\pgfqpoint{0.699538in}{0.869381in}}%
\pgfpathlineto{\pgfqpoint{0.700023in}{0.870405in}}%
\pgfpathlineto{\pgfqpoint{0.701133in}{0.873013in}}%
\pgfpathlineto{\pgfqpoint{0.701672in}{0.874037in}}%
\pgfpathlineto{\pgfqpoint{0.702751in}{0.876521in}}%
\pgfpathlineto{\pgfqpoint{0.703181in}{0.877607in}}%
\pgfpathlineto{\pgfqpoint{0.704291in}{0.880494in}}%
\pgfpathlineto{\pgfqpoint{0.704838in}{0.881581in}}%
\pgfpathlineto{\pgfqpoint{0.705941in}{0.884561in}}%
\pgfpathlineto{\pgfqpoint{0.706543in}{0.885585in}}%
\pgfpathlineto{\pgfqpoint{0.707653in}{0.887975in}}%
\pgfpathlineto{\pgfqpoint{0.708075in}{0.889062in}}%
\pgfpathlineto{\pgfqpoint{0.709185in}{0.891762in}}%
\pgfpathlineto{\pgfqpoint{0.709842in}{0.892849in}}%
\pgfpathlineto{\pgfqpoint{0.710928in}{0.895860in}}%
\pgfpathlineto{\pgfqpoint{0.711476in}{0.896946in}}%
\pgfpathlineto{\pgfqpoint{0.712547in}{0.899554in}}%
\pgfpathlineto{\pgfqpoint{0.713297in}{0.900640in}}%
\pgfpathlineto{\pgfqpoint{0.714392in}{0.902844in}}%
\pgfpathlineto{\pgfqpoint{0.714853in}{0.903745in}}%
\pgfpathlineto{\pgfqpoint{0.714853in}{0.903900in}}%
\pgfpathlineto{\pgfqpoint{0.715955in}{0.906290in}}%
\pgfpathlineto{\pgfqpoint{0.716643in}{0.907377in}}%
\pgfpathlineto{\pgfqpoint{0.717753in}{0.909363in}}%
\pgfpathlineto{\pgfqpoint{0.718128in}{0.910450in}}%
\pgfpathlineto{\pgfqpoint{0.719238in}{0.912964in}}%
\pgfpathlineto{\pgfqpoint{0.719723in}{0.914051in}}%
\pgfpathlineto{\pgfqpoint{0.720825in}{0.916161in}}%
\pgfpathlineto{\pgfqpoint{0.721631in}{0.917248in}}%
\pgfpathlineto{\pgfqpoint{0.722725in}{0.919142in}}%
\pgfpathlineto{\pgfqpoint{0.723225in}{0.920228in}}%
\pgfpathlineto{\pgfqpoint{0.724320in}{0.923115in}}%
\pgfpathlineto{\pgfqpoint{0.724875in}{0.924201in}}%
\pgfpathlineto{\pgfqpoint{0.725954in}{0.926312in}}%
\pgfpathlineto{\pgfqpoint{0.726525in}{0.927399in}}%
\pgfpathlineto{\pgfqpoint{0.727627in}{0.929789in}}%
\pgfpathlineto{\pgfqpoint{0.728221in}{0.930875in}}%
\pgfpathlineto{\pgfqpoint{0.729323in}{0.933079in}}%
\pgfpathlineto{\pgfqpoint{0.729824in}{0.934166in}}%
\pgfpathlineto{\pgfqpoint{0.730887in}{0.936184in}}%
\pgfpathlineto{\pgfqpoint{0.731637in}{0.937270in}}%
\pgfpathlineto{\pgfqpoint{0.732732in}{0.940033in}}%
\pgfpathlineto{\pgfqpoint{0.733240in}{0.941119in}}%
\pgfpathlineto{\pgfqpoint{0.734334in}{0.943447in}}%
\pgfpathlineto{\pgfqpoint{0.734975in}{0.944534in}}%
\pgfpathlineto{\pgfqpoint{0.736078in}{0.947017in}}%
\pgfpathlineto{\pgfqpoint{0.736953in}{0.948104in}}%
\pgfpathlineto{\pgfqpoint{0.738040in}{0.949997in}}%
\pgfpathlineto{\pgfqpoint{0.738470in}{0.951084in}}%
\pgfpathlineto{\pgfqpoint{0.739549in}{0.953381in}}%
\pgfpathlineto{\pgfqpoint{0.739572in}{0.953381in}}%
\pgfpathlineto{\pgfqpoint{0.740049in}{0.954467in}}%
\pgfpathlineto{\pgfqpoint{0.741159in}{0.955957in}}%
\pgfpathlineto{\pgfqpoint{0.741706in}{0.957044in}}%
\pgfpathlineto{\pgfqpoint{0.742770in}{0.959031in}}%
\pgfpathlineto{\pgfqpoint{0.742801in}{0.959031in}}%
\pgfpathlineto{\pgfqpoint{0.743340in}{0.960117in}}%
\pgfpathlineto{\pgfqpoint{0.744450in}{0.961980in}}%
\pgfpathlineto{\pgfqpoint{0.745334in}{0.963066in}}%
\pgfpathlineto{\pgfqpoint{0.746413in}{0.965953in}}%
\pgfpathlineto{\pgfqpoint{0.746944in}{0.967040in}}%
\pgfpathlineto{\pgfqpoint{0.748046in}{0.969181in}}%
\pgfpathlineto{\pgfqpoint{0.748688in}{0.970268in}}%
\pgfpathlineto{\pgfqpoint{0.749790in}{0.971882in}}%
\pgfpathlineto{\pgfqpoint{0.750533in}{0.972969in}}%
\pgfpathlineto{\pgfqpoint{0.751627in}{0.974924in}}%
\pgfpathlineto{\pgfqpoint{0.752714in}{0.976011in}}%
\pgfpathlineto{\pgfqpoint{0.753808in}{0.977501in}}%
\pgfpathlineto{\pgfqpoint{0.754535in}{0.978587in}}%
\pgfpathlineto{\pgfqpoint{0.755630in}{0.980481in}}%
\pgfpathlineto{\pgfqpoint{0.756318in}{0.981567in}}%
\pgfpathlineto{\pgfqpoint{0.757412in}{0.983275in}}%
\pgfpathlineto{\pgfqpoint{0.757428in}{0.983275in}}%
\pgfpathlineto{\pgfqpoint{0.758092in}{0.984330in}}%
\pgfpathlineto{\pgfqpoint{0.759179in}{0.985975in}}%
\pgfpathlineto{\pgfqpoint{0.759945in}{0.987062in}}%
\pgfpathlineto{\pgfqpoint{0.761024in}{0.988738in}}%
\pgfpathlineto{\pgfqpoint{0.761884in}{0.989824in}}%
\pgfpathlineto{\pgfqpoint{0.762963in}{0.991749in}}%
\pgfpathlineto{\pgfqpoint{0.763705in}{0.992836in}}%
\pgfpathlineto{\pgfqpoint{0.764815in}{0.994543in}}%
\pgfpathlineto{\pgfqpoint{0.765433in}{0.995629in}}%
\pgfpathlineto{\pgfqpoint{0.766512in}{0.997647in}}%
\pgfpathlineto{\pgfqpoint{0.767270in}{0.998702in}}%
\pgfpathlineto{\pgfqpoint{0.768380in}{1.001062in}}%
\pgfpathlineto{\pgfqpoint{0.768998in}{1.002117in}}%
\pgfpathlineto{\pgfqpoint{0.770069in}{1.004197in}}%
\pgfpathlineto{\pgfqpoint{0.770600in}{1.005221in}}%
\pgfpathlineto{\pgfqpoint{0.771679in}{1.006804in}}%
\pgfpathlineto{\pgfqpoint{0.771695in}{1.006804in}}%
\pgfpathlineto{\pgfqpoint{0.772289in}{1.007891in}}%
\pgfpathlineto{\pgfqpoint{0.773383in}{1.009412in}}%
\pgfpathlineto{\pgfqpoint{0.774032in}{1.010498in}}%
\pgfpathlineto{\pgfqpoint{0.775142in}{1.011895in}}%
\pgfpathlineto{\pgfqpoint{0.775987in}{1.012951in}}%
\pgfpathlineto{\pgfqpoint{0.777097in}{1.014844in}}%
\pgfpathlineto{\pgfqpoint{0.777754in}{1.015931in}}%
\pgfpathlineto{\pgfqpoint{0.778840in}{1.017762in}}%
\pgfpathlineto{\pgfqpoint{0.779599in}{1.018849in}}%
\pgfpathlineto{\pgfqpoint{0.780709in}{1.021053in}}%
\pgfpathlineto{\pgfqpoint{0.781608in}{1.022139in}}%
\pgfpathlineto{\pgfqpoint{0.782640in}{1.023691in}}%
\pgfpathlineto{\pgfqpoint{0.783437in}{1.024778in}}%
\pgfpathlineto{\pgfqpoint{0.784508in}{1.026485in}}%
\pgfpathlineto{\pgfqpoint{0.785274in}{1.027572in}}%
\pgfpathlineto{\pgfqpoint{0.786376in}{1.029434in}}%
\pgfpathlineto{\pgfqpoint{0.786853in}{1.030490in}}%
\pgfpathlineto{\pgfqpoint{0.787948in}{1.032166in}}%
\pgfpathlineto{\pgfqpoint{0.788659in}{1.033252in}}%
\pgfpathlineto{\pgfqpoint{0.789769in}{1.035208in}}%
\pgfpathlineto{\pgfqpoint{0.790434in}{1.036294in}}%
\pgfpathlineto{\pgfqpoint{0.791520in}{1.038343in}}%
\pgfpathlineto{\pgfqpoint{0.791544in}{1.038343in}}%
\pgfpathlineto{\pgfqpoint{0.792068in}{1.039430in}}%
\pgfpathlineto{\pgfqpoint{0.793131in}{1.040951in}}%
\pgfpathlineto{\pgfqpoint{0.793960in}{1.042037in}}%
\pgfpathlineto{\pgfqpoint{0.795070in}{1.043527in}}%
\pgfpathlineto{\pgfqpoint{0.795914in}{1.044583in}}%
\pgfpathlineto{\pgfqpoint{0.797016in}{1.046166in}}%
\pgfpathlineto{\pgfqpoint{0.798236in}{1.047252in}}%
\pgfpathlineto{\pgfqpoint{0.799276in}{1.048711in}}%
\pgfpathlineto{\pgfqpoint{0.799322in}{1.048711in}}%
\pgfpathlineto{\pgfqpoint{0.800120in}{1.049798in}}%
\pgfpathlineto{\pgfqpoint{0.801167in}{1.051567in}}%
\pgfpathlineto{\pgfqpoint{0.801918in}{1.052654in}}%
\pgfpathlineto{\pgfqpoint{0.802997in}{1.054237in}}%
\pgfpathlineto{\pgfqpoint{0.803896in}{1.055323in}}%
\pgfpathlineto{\pgfqpoint{0.804935in}{1.056906in}}%
\pgfpathlineto{\pgfqpoint{0.805670in}{1.057962in}}%
\pgfpathlineto{\pgfqpoint{0.806749in}{1.059390in}}%
\pgfpathlineto{\pgfqpoint{0.808055in}{1.060445in}}%
\pgfpathlineto{\pgfqpoint{0.809165in}{1.061997in}}%
\pgfpathlineto{\pgfqpoint{0.810126in}{1.063084in}}%
\pgfpathlineto{\pgfqpoint{0.811236in}{1.065040in}}%
\pgfpathlineto{\pgfqpoint{0.812057in}{1.066064in}}%
\pgfpathlineto{\pgfqpoint{0.813152in}{1.067368in}}%
\pgfpathlineto{\pgfqpoint{0.814184in}{1.068454in}}%
\pgfpathlineto{\pgfqpoint{0.815270in}{1.070099in}}%
\pgfpathlineto{\pgfqpoint{0.816044in}{1.071155in}}%
\pgfpathlineto{\pgfqpoint{0.817154in}{1.072210in}}%
\pgfpathlineto{\pgfqpoint{0.817999in}{1.073266in}}%
\pgfpathlineto{\pgfqpoint{0.819101in}{1.074973in}}%
\pgfpathlineto{\pgfqpoint{0.819883in}{1.075997in}}%
\pgfpathlineto{\pgfqpoint{0.820962in}{1.078046in}}%
\pgfpathlineto{\pgfqpoint{0.821814in}{1.079102in}}%
\pgfpathlineto{\pgfqpoint{0.822924in}{1.080902in}}%
\pgfpathlineto{\pgfqpoint{0.823909in}{1.081989in}}%
\pgfpathlineto{\pgfqpoint{0.825011in}{1.083323in}}%
\pgfpathlineto{\pgfqpoint{0.825613in}{1.084410in}}%
\pgfpathlineto{\pgfqpoint{0.826692in}{1.086210in}}%
\pgfpathlineto{\pgfqpoint{0.827833in}{1.087297in}}%
\pgfpathlineto{\pgfqpoint{0.828936in}{1.088569in}}%
\pgfpathlineto{\pgfqpoint{0.830030in}{1.089656in}}%
\pgfpathlineto{\pgfqpoint{0.831132in}{1.091332in}}%
\pgfpathlineto{\pgfqpoint{0.832188in}{1.092419in}}%
\pgfpathlineto{\pgfqpoint{0.833235in}{1.094064in}}%
\pgfpathlineto{\pgfqpoint{0.833962in}{1.095119in}}%
\pgfpathlineto{\pgfqpoint{0.835065in}{1.096423in}}%
\pgfpathlineto{\pgfqpoint{0.836034in}{1.097479in}}%
\pgfpathlineto{\pgfqpoint{0.837136in}{1.099341in}}%
\pgfpathlineto{\pgfqpoint{0.838082in}{1.100428in}}%
\pgfpathlineto{\pgfqpoint{0.839192in}{1.102135in}}%
\pgfpathlineto{\pgfqpoint{0.839701in}{1.103221in}}%
\pgfpathlineto{\pgfqpoint{0.840748in}{1.104246in}}%
\pgfpathlineto{\pgfqpoint{0.841639in}{1.105332in}}%
\pgfpathlineto{\pgfqpoint{0.842726in}{1.106729in}}%
\pgfpathlineto{\pgfqpoint{0.843688in}{1.107753in}}%
\pgfpathlineto{\pgfqpoint{0.844782in}{1.108809in}}%
\pgfpathlineto{\pgfqpoint{0.845658in}{1.109895in}}%
\pgfpathlineto{\pgfqpoint{0.846768in}{1.111541in}}%
\pgfpathlineto{\pgfqpoint{0.847534in}{1.112627in}}%
\pgfpathlineto{\pgfqpoint{0.848636in}{1.113900in}}%
\pgfpathlineto{\pgfqpoint{0.849519in}{1.114986in}}%
\pgfpathlineto{\pgfqpoint{0.850630in}{1.116042in}}%
\pgfpathlineto{\pgfqpoint{0.851615in}{1.117128in}}%
\pgfpathlineto{\pgfqpoint{0.852748in}{1.118649in}}%
\pgfpathlineto{\pgfqpoint{0.853796in}{1.119705in}}%
\pgfpathlineto{\pgfqpoint{0.854875in}{1.120822in}}%
\pgfpathlineto{\pgfqpoint{0.855867in}{1.121909in}}%
\pgfpathlineto{\pgfqpoint{0.856962in}{1.123306in}}%
\pgfpathlineto{\pgfqpoint{0.857900in}{1.124392in}}%
\pgfpathlineto{\pgfqpoint{0.859002in}{1.126317in}}%
\pgfpathlineto{\pgfqpoint{0.860339in}{1.127372in}}%
\pgfpathlineto{\pgfqpoint{0.861441in}{1.128583in}}%
\pgfpathlineto{\pgfqpoint{0.862684in}{1.129669in}}%
\pgfpathlineto{\pgfqpoint{0.863771in}{1.130911in}}%
\pgfpathlineto{\pgfqpoint{0.864803in}{1.131966in}}%
\pgfpathlineto{\pgfqpoint{0.865874in}{1.133239in}}%
\pgfpathlineto{\pgfqpoint{0.867187in}{1.134326in}}%
\pgfpathlineto{\pgfqpoint{0.868282in}{1.135785in}}%
\pgfpathlineto{\pgfqpoint{0.869759in}{1.136871in}}%
\pgfpathlineto{\pgfqpoint{0.870869in}{1.138082in}}%
\pgfpathlineto{\pgfqpoint{0.871987in}{1.139137in}}%
\pgfpathlineto{\pgfqpoint{0.873098in}{1.140348in}}%
\pgfpathlineto{\pgfqpoint{0.874200in}{1.141434in}}%
\pgfpathlineto{\pgfqpoint{0.875310in}{1.142521in}}%
\pgfpathlineto{\pgfqpoint{0.876506in}{1.143607in}}%
\pgfpathlineto{\pgfqpoint{0.877616in}{1.144787in}}%
\pgfpathlineto{\pgfqpoint{0.878523in}{1.145842in}}%
\pgfpathlineto{\pgfqpoint{0.879633in}{1.147332in}}%
\pgfpathlineto{\pgfqpoint{0.880696in}{1.148419in}}%
\pgfpathlineto{\pgfqpoint{0.881720in}{1.149226in}}%
\pgfpathlineto{\pgfqpoint{0.883143in}{1.150312in}}%
\pgfpathlineto{\pgfqpoint{0.884253in}{1.151616in}}%
\pgfpathlineto{\pgfqpoint{0.885418in}{1.152702in}}%
\pgfpathlineto{\pgfqpoint{0.886481in}{1.153944in}}%
\pgfpathlineto{\pgfqpoint{0.886520in}{1.153944in}}%
\pgfpathlineto{\pgfqpoint{0.887818in}{1.155031in}}%
\pgfpathlineto{\pgfqpoint{0.888928in}{1.156521in}}%
\pgfpathlineto{\pgfqpoint{0.889929in}{1.157607in}}%
\pgfpathlineto{\pgfqpoint{0.891039in}{1.158600in}}%
\pgfpathlineto{\pgfqpoint{0.891891in}{1.159687in}}%
\pgfpathlineto{\pgfqpoint{0.892962in}{1.160680in}}%
\pgfpathlineto{\pgfqpoint{0.894002in}{1.161767in}}%
\pgfpathlineto{\pgfqpoint{0.895089in}{1.162698in}}%
\pgfpathlineto{\pgfqpoint{0.896034in}{1.163785in}}%
\pgfpathlineto{\pgfqpoint{0.897113in}{1.165119in}}%
\pgfpathlineto{\pgfqpoint{0.898067in}{1.166206in}}%
\pgfpathlineto{\pgfqpoint{0.899130in}{1.167572in}}%
\pgfpathlineto{\pgfqpoint{0.900279in}{1.168658in}}%
\pgfpathlineto{\pgfqpoint{0.901390in}{1.169838in}}%
\pgfpathlineto{\pgfqpoint{0.902672in}{1.170924in}}%
\pgfpathlineto{\pgfqpoint{0.903774in}{1.172042in}}%
\pgfpathlineto{\pgfqpoint{0.904853in}{1.173128in}}%
\pgfpathlineto{\pgfqpoint{0.905900in}{1.174122in}}%
\pgfpathlineto{\pgfqpoint{0.906721in}{1.175208in}}%
\pgfpathlineto{\pgfqpoint{0.907800in}{1.176388in}}%
\pgfpathlineto{\pgfqpoint{0.909090in}{1.177474in}}%
\pgfpathlineto{\pgfqpoint{0.910169in}{1.178064in}}%
\pgfpathlineto{\pgfqpoint{0.911310in}{1.179119in}}%
\pgfpathlineto{\pgfqpoint{0.912412in}{1.180082in}}%
\pgfpathlineto{\pgfqpoint{0.913499in}{1.181106in}}%
\pgfpathlineto{\pgfqpoint{0.914609in}{1.182503in}}%
\pgfpathlineto{\pgfqpoint{0.916079in}{1.183558in}}%
\pgfpathlineto{\pgfqpoint{0.917189in}{1.185017in}}%
\pgfpathlineto{\pgfqpoint{0.918080in}{1.186104in}}%
\pgfpathlineto{\pgfqpoint{0.919175in}{1.187625in}}%
\pgfpathlineto{\pgfqpoint{0.920183in}{1.188711in}}%
\pgfpathlineto{\pgfqpoint{0.921207in}{1.189643in}}%
\pgfpathlineto{\pgfqpoint{0.922450in}{1.190698in}}%
\pgfpathlineto{\pgfqpoint{0.923545in}{1.191722in}}%
\pgfpathlineto{\pgfqpoint{0.924889in}{1.192809in}}%
\pgfpathlineto{\pgfqpoint{0.925992in}{1.194299in}}%
\pgfpathlineto{\pgfqpoint{0.927149in}{1.195354in}}%
\pgfpathlineto{\pgfqpoint{0.928259in}{1.196534in}}%
\pgfpathlineto{\pgfqpoint{0.929486in}{1.197620in}}%
\pgfpathlineto{\pgfqpoint{0.930463in}{1.198428in}}%
\pgfpathlineto{\pgfqpoint{0.931878in}{1.199514in}}%
\pgfpathlineto{\pgfqpoint{0.932942in}{1.200414in}}%
\pgfpathlineto{\pgfqpoint{0.932981in}{1.200414in}}%
\pgfpathlineto{\pgfqpoint{0.934646in}{1.201439in}}%
\pgfpathlineto{\pgfqpoint{0.935646in}{1.202370in}}%
\pgfpathlineto{\pgfqpoint{0.937194in}{1.203456in}}%
\pgfpathlineto{\pgfqpoint{0.938289in}{1.204605in}}%
\pgfpathlineto{\pgfqpoint{0.939375in}{1.205691in}}%
\pgfpathlineto{\pgfqpoint{0.940478in}{1.206654in}}%
\pgfpathlineto{\pgfqpoint{0.941588in}{1.207709in}}%
\pgfpathlineto{\pgfqpoint{0.942682in}{1.208858in}}%
\pgfpathlineto{\pgfqpoint{0.944347in}{1.209944in}}%
\pgfpathlineto{\pgfqpoint{0.945426in}{1.210875in}}%
\pgfpathlineto{\pgfqpoint{0.946372in}{1.211962in}}%
\pgfpathlineto{\pgfqpoint{0.947467in}{1.213048in}}%
\pgfpathlineto{\pgfqpoint{0.948600in}{1.214135in}}%
\pgfpathlineto{\pgfqpoint{0.949695in}{1.215190in}}%
\pgfpathlineto{\pgfqpoint{0.950891in}{1.216277in}}%
\pgfpathlineto{\pgfqpoint{0.952001in}{1.217487in}}%
\pgfpathlineto{\pgfqpoint{0.953588in}{1.218574in}}%
\pgfpathlineto{\pgfqpoint{0.954659in}{1.219505in}}%
\pgfpathlineto{\pgfqpoint{0.954690in}{1.219505in}}%
\pgfpathlineto{\pgfqpoint{0.955417in}{1.220561in}}%
\pgfpathlineto{\pgfqpoint{0.956520in}{1.222020in}}%
\pgfpathlineto{\pgfqpoint{0.957778in}{1.223106in}}%
\pgfpathlineto{\pgfqpoint{0.958865in}{1.224037in}}%
\pgfpathlineto{\pgfqpoint{0.960014in}{1.225093in}}%
\pgfpathlineto{\pgfqpoint{0.961069in}{1.226086in}}%
\pgfpathlineto{\pgfqpoint{0.962461in}{1.227173in}}%
\pgfpathlineto{\pgfqpoint{0.963548in}{1.228073in}}%
\pgfpathlineto{\pgfqpoint{0.965111in}{1.229128in}}%
\pgfpathlineto{\pgfqpoint{0.966198in}{1.230153in}}%
\pgfpathlineto{\pgfqpoint{0.967347in}{1.231239in}}%
\pgfpathlineto{\pgfqpoint{0.968387in}{1.232294in}}%
\pgfpathlineto{\pgfqpoint{0.969880in}{1.233381in}}%
\pgfpathlineto{\pgfqpoint{0.970935in}{1.234467in}}%
\pgfpathlineto{\pgfqpoint{0.971952in}{1.235554in}}%
\pgfpathlineto{\pgfqpoint{0.973062in}{1.236392in}}%
\pgfpathlineto{\pgfqpoint{0.974438in}{1.237479in}}%
\pgfpathlineto{\pgfqpoint{0.975430in}{1.238255in}}%
\pgfpathlineto{\pgfqpoint{0.976908in}{1.239341in}}%
\pgfpathlineto{\pgfqpoint{0.978018in}{1.240396in}}%
\pgfpathlineto{\pgfqpoint{0.979644in}{1.241483in}}%
\pgfpathlineto{\pgfqpoint{0.980746in}{1.242694in}}%
\pgfpathlineto{\pgfqpoint{0.982193in}{1.243780in}}%
\pgfpathlineto{\pgfqpoint{0.983295in}{1.244960in}}%
\pgfpathlineto{\pgfqpoint{0.984444in}{1.246046in}}%
\pgfpathlineto{\pgfqpoint{0.985546in}{1.246946in}}%
\pgfpathlineto{\pgfqpoint{0.987352in}{1.248002in}}%
\pgfpathlineto{\pgfqpoint{0.988392in}{1.249026in}}%
\pgfpathlineto{\pgfqpoint{0.989971in}{1.250113in}}%
\pgfpathlineto{\pgfqpoint{0.991066in}{1.250982in}}%
\pgfpathlineto{\pgfqpoint{0.992184in}{1.252068in}}%
\pgfpathlineto{\pgfqpoint{0.993098in}{1.252751in}}%
\pgfpathlineto{\pgfqpoint{0.993176in}{1.252751in}}%
\pgfpathlineto{\pgfqpoint{0.994889in}{1.253838in}}%
\pgfpathlineto{\pgfqpoint{0.995967in}{1.254645in}}%
\pgfpathlineto{\pgfqpoint{0.997203in}{1.255731in}}%
\pgfpathlineto{\pgfqpoint{0.998258in}{1.256414in}}%
\pgfpathlineto{\pgfqpoint{0.999407in}{1.257501in}}%
\pgfpathlineto{\pgfqpoint{1.000455in}{1.258401in}}%
\pgfpathlineto{\pgfqpoint{1.000509in}{1.258401in}}%
\pgfpathlineto{\pgfqpoint{1.002143in}{1.259456in}}%
\pgfpathlineto{\pgfqpoint{1.003214in}{1.260481in}}%
\pgfpathlineto{\pgfqpoint{1.005020in}{1.261536in}}%
\pgfpathlineto{\pgfqpoint{1.006130in}{1.262498in}}%
\pgfpathlineto{\pgfqpoint{1.007451in}{1.263585in}}%
\pgfpathlineto{\pgfqpoint{1.008554in}{1.264392in}}%
\pgfpathlineto{\pgfqpoint{1.010188in}{1.265447in}}%
\pgfpathlineto{\pgfqpoint{1.011274in}{1.266317in}}%
\pgfpathlineto{\pgfqpoint{1.012861in}{1.267403in}}%
\pgfpathlineto{\pgfqpoint{1.013948in}{1.268490in}}%
\pgfpathlineto{\pgfqpoint{1.015379in}{1.269576in}}%
\pgfpathlineto{\pgfqpoint{1.016489in}{1.271035in}}%
\pgfpathlineto{\pgfqpoint{1.017505in}{1.272122in}}%
\pgfpathlineto{\pgfqpoint{1.018615in}{1.273208in}}%
\pgfpathlineto{\pgfqpoint{1.020116in}{1.274263in}}%
\pgfpathlineto{\pgfqpoint{1.021171in}{1.274915in}}%
\pgfpathlineto{\pgfqpoint{1.022719in}{1.276002in}}%
\pgfpathlineto{\pgfqpoint{1.023822in}{1.277306in}}%
\pgfpathlineto{\pgfqpoint{1.025096in}{1.278392in}}%
\pgfpathlineto{\pgfqpoint{1.026206in}{1.279572in}}%
\pgfpathlineto{\pgfqpoint{1.027472in}{1.280658in}}%
\pgfpathlineto{\pgfqpoint{1.028567in}{1.281745in}}%
\pgfpathlineto{\pgfqpoint{1.029974in}{1.282831in}}%
\pgfpathlineto{\pgfqpoint{1.031084in}{1.283980in}}%
\pgfpathlineto{\pgfqpoint{1.032812in}{1.285066in}}%
\pgfpathlineto{\pgfqpoint{1.033867in}{1.286401in}}%
\pgfpathlineto{\pgfqpoint{1.035384in}{1.287487in}}%
\pgfpathlineto{\pgfqpoint{1.036478in}{1.288388in}}%
\pgfpathlineto{\pgfqpoint{1.037901in}{1.289474in}}%
\pgfpathlineto{\pgfqpoint{1.038925in}{1.290095in}}%
\pgfpathlineto{\pgfqpoint{1.040137in}{1.291181in}}%
\pgfpathlineto{\pgfqpoint{1.041153in}{1.291833in}}%
\pgfpathlineto{\pgfqpoint{1.042686in}{1.292920in}}%
\pgfpathlineto{\pgfqpoint{1.043780in}{1.294161in}}%
\pgfpathlineto{\pgfqpoint{1.044953in}{1.295217in}}%
\pgfpathlineto{\pgfqpoint{1.045985in}{1.296086in}}%
\pgfpathlineto{\pgfqpoint{1.046008in}{1.296086in}}%
\pgfpathlineto{\pgfqpoint{1.047462in}{1.297142in}}%
\pgfpathlineto{\pgfqpoint{1.048400in}{1.297762in}}%
\pgfpathlineto{\pgfqpoint{1.050613in}{1.298849in}}%
\pgfpathlineto{\pgfqpoint{1.051684in}{1.299842in}}%
\pgfpathlineto{\pgfqpoint{1.053380in}{1.300929in}}%
\pgfpathlineto{\pgfqpoint{1.054412in}{1.301674in}}%
\pgfpathlineto{\pgfqpoint{1.055796in}{1.302760in}}%
\pgfpathlineto{\pgfqpoint{1.056804in}{1.303474in}}%
\pgfpathlineto{\pgfqpoint{1.058282in}{1.304561in}}%
\pgfpathlineto{\pgfqpoint{1.059314in}{1.305554in}}%
\pgfpathlineto{\pgfqpoint{1.061143in}{1.306640in}}%
\pgfpathlineto{\pgfqpoint{1.062245in}{1.307354in}}%
\pgfpathlineto{\pgfqpoint{1.064012in}{1.308441in}}%
\pgfpathlineto{\pgfqpoint{1.065107in}{1.309062in}}%
\pgfpathlineto{\pgfqpoint{1.066350in}{1.310117in}}%
\pgfpathlineto{\pgfqpoint{1.067421in}{1.311079in}}%
\pgfpathlineto{\pgfqpoint{1.068718in}{1.312166in}}%
\pgfpathlineto{\pgfqpoint{1.069797in}{1.313035in}}%
\pgfpathlineto{\pgfqpoint{1.071165in}{1.314122in}}%
\pgfpathlineto{\pgfqpoint{1.072197in}{1.314804in}}%
\pgfpathlineto{\pgfqpoint{1.073823in}{1.315891in}}%
\pgfpathlineto{\pgfqpoint{1.074910in}{1.316760in}}%
\pgfpathlineto{\pgfqpoint{1.076872in}{1.317816in}}%
\pgfpathlineto{\pgfqpoint{1.077982in}{1.318871in}}%
\pgfpathlineto{\pgfqpoint{1.079585in}{1.319957in}}%
\pgfpathlineto{\pgfqpoint{1.080625in}{1.320671in}}%
\pgfpathlineto{\pgfqpoint{1.080679in}{1.320671in}}%
\pgfpathlineto{\pgfqpoint{1.082133in}{1.321758in}}%
\pgfpathlineto{\pgfqpoint{1.083204in}{1.322627in}}%
\pgfpathlineto{\pgfqpoint{1.084823in}{1.323714in}}%
\pgfpathlineto{\pgfqpoint{1.085909in}{1.324490in}}%
\pgfpathlineto{\pgfqpoint{1.088427in}{1.325576in}}%
\pgfpathlineto{\pgfqpoint{1.089380in}{1.326197in}}%
\pgfpathlineto{\pgfqpoint{1.091468in}{1.327283in}}%
\pgfpathlineto{\pgfqpoint{1.092570in}{1.328277in}}%
\pgfpathlineto{\pgfqpoint{1.094454in}{1.329363in}}%
\pgfpathlineto{\pgfqpoint{1.095564in}{1.330015in}}%
\pgfpathlineto{\pgfqpoint{1.097135in}{1.331102in}}%
\pgfpathlineto{\pgfqpoint{1.098222in}{1.332033in}}%
\pgfpathlineto{\pgfqpoint{1.099762in}{1.333119in}}%
\pgfpathlineto{\pgfqpoint{1.100810in}{1.333926in}}%
\pgfpathlineto{\pgfqpoint{1.102663in}{1.334982in}}%
\pgfpathlineto{\pgfqpoint{1.103757in}{1.335851in}}%
\pgfpathlineto{\pgfqpoint{1.105352in}{1.336938in}}%
\pgfpathlineto{\pgfqpoint{1.106462in}{1.337496in}}%
\pgfpathlineto{\pgfqpoint{1.107642in}{1.338583in}}%
\pgfpathlineto{\pgfqpoint{1.108667in}{1.339576in}}%
\pgfpathlineto{\pgfqpoint{1.108690in}{1.339576in}}%
\pgfpathlineto{\pgfqpoint{1.110644in}{1.340663in}}%
\pgfpathlineto{\pgfqpoint{1.111747in}{1.341221in}}%
\pgfpathlineto{\pgfqpoint{1.113138in}{1.342308in}}%
\pgfpathlineto{\pgfqpoint{1.114233in}{1.342867in}}%
\pgfpathlineto{\pgfqpoint{1.115765in}{1.343953in}}%
\pgfpathlineto{\pgfqpoint{1.116805in}{1.344698in}}%
\pgfpathlineto{\pgfqpoint{1.118321in}{1.345753in}}%
\pgfpathlineto{\pgfqpoint{1.119338in}{1.346685in}}%
\pgfpathlineto{\pgfqpoint{1.122535in}{1.347771in}}%
\pgfpathlineto{\pgfqpoint{1.123622in}{1.348765in}}%
\pgfpathlineto{\pgfqpoint{1.125529in}{1.349820in}}%
\pgfpathlineto{\pgfqpoint{1.126553in}{1.350503in}}%
\pgfpathlineto{\pgfqpoint{1.128484in}{1.351589in}}%
\pgfpathlineto{\pgfqpoint{1.129469in}{1.352148in}}%
\pgfpathlineto{\pgfqpoint{1.132377in}{1.353235in}}%
\pgfpathlineto{\pgfqpoint{1.133441in}{1.354073in}}%
\pgfpathlineto{\pgfqpoint{1.135418in}{1.355159in}}%
\pgfpathlineto{\pgfqpoint{1.136466in}{1.355594in}}%
\pgfpathlineto{\pgfqpoint{1.138413in}{1.356680in}}%
\pgfpathlineto{\pgfqpoint{1.139515in}{1.357456in}}%
\pgfpathlineto{\pgfqpoint{1.141790in}{1.358543in}}%
\pgfpathlineto{\pgfqpoint{1.142869in}{1.359288in}}%
\pgfpathlineto{\pgfqpoint{1.144581in}{1.360374in}}%
\pgfpathlineto{\pgfqpoint{1.145660in}{1.361088in}}%
\pgfpathlineto{\pgfqpoint{1.147403in}{1.362144in}}%
\pgfpathlineto{\pgfqpoint{1.148458in}{1.363075in}}%
\pgfpathlineto{\pgfqpoint{1.148474in}{1.363075in}}%
\pgfpathlineto{\pgfqpoint{1.149647in}{1.364161in}}%
\pgfpathlineto{\pgfqpoint{1.150655in}{1.364813in}}%
\pgfpathlineto{\pgfqpoint{1.152508in}{1.365900in}}%
\pgfpathlineto{\pgfqpoint{1.153579in}{1.366614in}}%
\pgfpathlineto{\pgfqpoint{1.155103in}{1.367669in}}%
\pgfpathlineto{\pgfqpoint{1.156057in}{1.368259in}}%
\pgfpathlineto{\pgfqpoint{1.156143in}{1.368259in}}%
\pgfpathlineto{\pgfqpoint{1.157363in}{1.369345in}}%
\pgfpathlineto{\pgfqpoint{1.158426in}{1.369966in}}%
\pgfpathlineto{\pgfqpoint{1.158449in}{1.369966in}}%
\pgfpathlineto{\pgfqpoint{1.160239in}{1.371053in}}%
\pgfpathlineto{\pgfqpoint{1.161326in}{1.371798in}}%
\pgfpathlineto{\pgfqpoint{1.163023in}{1.372853in}}%
\pgfpathlineto{\pgfqpoint{1.164047in}{1.373722in}}%
\pgfpathlineto{\pgfqpoint{1.166048in}{1.374809in}}%
\pgfpathlineto{\pgfqpoint{1.167135in}{1.375461in}}%
\pgfpathlineto{\pgfqpoint{1.169480in}{1.376547in}}%
\pgfpathlineto{\pgfqpoint{1.170528in}{1.377230in}}%
\pgfpathlineto{\pgfqpoint{1.173272in}{1.378317in}}%
\pgfpathlineto{\pgfqpoint{1.174335in}{1.378969in}}%
\pgfpathlineto{\pgfqpoint{1.176375in}{1.380055in}}%
\pgfpathlineto{\pgfqpoint{1.177462in}{1.380893in}}%
\pgfpathlineto{\pgfqpoint{1.180221in}{1.381980in}}%
\pgfpathlineto{\pgfqpoint{1.181316in}{1.382476in}}%
\pgfpathlineto{\pgfqpoint{1.183497in}{1.383563in}}%
\pgfpathlineto{\pgfqpoint{1.184591in}{1.384091in}}%
\pgfpathlineto{\pgfqpoint{1.184599in}{1.384091in}}%
\pgfpathlineto{\pgfqpoint{1.186929in}{1.385177in}}%
\pgfpathlineto{\pgfqpoint{1.188039in}{1.385984in}}%
\pgfpathlineto{\pgfqpoint{1.190408in}{1.387071in}}%
\pgfpathlineto{\pgfqpoint{1.191494in}{1.387722in}}%
\pgfpathlineto{\pgfqpoint{1.194106in}{1.388809in}}%
\pgfpathlineto{\pgfqpoint{1.195177in}{1.389523in}}%
\pgfpathlineto{\pgfqpoint{1.197420in}{1.390609in}}%
\pgfpathlineto{\pgfqpoint{1.198436in}{1.391416in}}%
\pgfpathlineto{\pgfqpoint{1.200485in}{1.392472in}}%
\pgfpathlineto{\pgfqpoint{1.201587in}{1.393341in}}%
\pgfpathlineto{\pgfqpoint{1.203752in}{1.394428in}}%
\pgfpathlineto{\pgfqpoint{1.204777in}{1.394893in}}%
\pgfpathlineto{\pgfqpoint{1.206942in}{1.395980in}}%
\pgfpathlineto{\pgfqpoint{1.208044in}{1.396818in}}%
\pgfpathlineto{\pgfqpoint{1.210405in}{1.397904in}}%
\pgfpathlineto{\pgfqpoint{1.211422in}{1.398339in}}%
\pgfpathlineto{\pgfqpoint{1.213869in}{1.399394in}}%
\pgfpathlineto{\pgfqpoint{1.214971in}{1.400077in}}%
\pgfpathlineto{\pgfqpoint{1.217183in}{1.401164in}}%
\pgfpathlineto{\pgfqpoint{1.218285in}{1.401722in}}%
\pgfpathlineto{\pgfqpoint{1.220482in}{1.402809in}}%
\pgfpathlineto{\pgfqpoint{1.221444in}{1.403244in}}%
\pgfpathlineto{\pgfqpoint{1.223703in}{1.404330in}}%
\pgfpathlineto{\pgfqpoint{1.224329in}{1.404796in}}%
\pgfpathlineto{\pgfqpoint{1.224563in}{1.404796in}}%
\pgfpathlineto{\pgfqpoint{1.227800in}{1.405882in}}%
\pgfpathlineto{\pgfqpoint{1.228886in}{1.406534in}}%
\pgfpathlineto{\pgfqpoint{1.231357in}{1.407620in}}%
\pgfpathlineto{\pgfqpoint{1.232459in}{1.408241in}}%
\pgfpathlineto{\pgfqpoint{1.235117in}{1.409328in}}%
\pgfpathlineto{\pgfqpoint{1.236102in}{1.410011in}}%
\pgfpathlineto{\pgfqpoint{1.238385in}{1.411097in}}%
\pgfpathlineto{\pgfqpoint{1.239487in}{1.411532in}}%
\pgfpathlineto{\pgfqpoint{1.242028in}{1.412618in}}%
\pgfpathlineto{\pgfqpoint{1.243091in}{1.413301in}}%
\pgfpathlineto{\pgfqpoint{1.244826in}{1.414388in}}%
\pgfpathlineto{\pgfqpoint{1.245929in}{1.415102in}}%
\pgfpathlineto{\pgfqpoint{1.248032in}{1.416188in}}%
\pgfpathlineto{\pgfqpoint{1.249142in}{1.416871in}}%
\pgfpathlineto{\pgfqpoint{1.251088in}{1.417957in}}%
\pgfpathlineto{\pgfqpoint{1.252183in}{1.418454in}}%
\pgfpathlineto{\pgfqpoint{1.254810in}{1.419541in}}%
\pgfpathlineto{\pgfqpoint{1.255912in}{1.420068in}}%
\pgfpathlineto{\pgfqpoint{1.257593in}{1.421155in}}%
\pgfpathlineto{\pgfqpoint{1.258648in}{1.421714in}}%
\pgfpathlineto{\pgfqpoint{1.261064in}{1.422800in}}%
\pgfpathlineto{\pgfqpoint{1.262119in}{1.423204in}}%
\pgfpathlineto{\pgfqpoint{1.264988in}{1.424290in}}%
\pgfpathlineto{\pgfqpoint{1.266059in}{1.424756in}}%
\pgfpathlineto{\pgfqpoint{1.268529in}{1.425842in}}%
\pgfpathlineto{\pgfqpoint{1.269624in}{1.426401in}}%
\pgfpathlineto{\pgfqpoint{1.271649in}{1.427487in}}%
\pgfpathlineto{\pgfqpoint{1.272509in}{1.427891in}}%
\pgfpathlineto{\pgfqpoint{1.275112in}{1.428977in}}%
\pgfpathlineto{\pgfqpoint{1.276183in}{1.429722in}}%
\pgfpathlineto{\pgfqpoint{1.278497in}{1.430809in}}%
\pgfpathlineto{\pgfqpoint{1.279584in}{1.431368in}}%
\pgfpathlineto{\pgfqpoint{1.282070in}{1.432454in}}%
\pgfpathlineto{\pgfqpoint{1.283141in}{1.432951in}}%
\pgfpathlineto{\pgfqpoint{1.284603in}{1.434037in}}%
\pgfpathlineto{\pgfqpoint{1.285681in}{1.434813in}}%
\pgfpathlineto{\pgfqpoint{1.288550in}{1.435900in}}%
\pgfpathlineto{\pgfqpoint{1.289410in}{1.436272in}}%
\pgfpathlineto{\pgfqpoint{1.291466in}{1.437359in}}%
\pgfpathlineto{\pgfqpoint{1.292514in}{1.437855in}}%
\pgfpathlineto{\pgfqpoint{1.295203in}{1.438942in}}%
\pgfpathlineto{\pgfqpoint{1.296251in}{1.439314in}}%
\pgfpathlineto{\pgfqpoint{1.298987in}{1.440401in}}%
\pgfpathlineto{\pgfqpoint{1.299988in}{1.440960in}}%
\pgfpathlineto{\pgfqpoint{1.302216in}{1.442046in}}%
\pgfpathlineto{\pgfqpoint{1.303326in}{1.442543in}}%
\pgfpathlineto{\pgfqpoint{1.306164in}{1.443629in}}%
\pgfpathlineto{\pgfqpoint{1.307227in}{1.444064in}}%
\pgfpathlineto{\pgfqpoint{1.310878in}{1.445150in}}%
\pgfpathlineto{\pgfqpoint{1.311784in}{1.445492in}}%
\pgfpathlineto{\pgfqpoint{1.315084in}{1.446578in}}%
\pgfpathlineto{\pgfqpoint{1.316061in}{1.447137in}}%
\pgfpathlineto{\pgfqpoint{1.317820in}{1.448224in}}%
\pgfpathlineto{\pgfqpoint{1.318883in}{1.448938in}}%
\pgfpathlineto{\pgfqpoint{1.320751in}{1.450024in}}%
\pgfpathlineto{\pgfqpoint{1.321854in}{1.450334in}}%
\pgfpathlineto{\pgfqpoint{1.324332in}{1.451421in}}%
\pgfpathlineto{\pgfqpoint{1.325411in}{1.452011in}}%
\pgfpathlineto{\pgfqpoint{1.327850in}{1.453097in}}%
\pgfpathlineto{\pgfqpoint{1.328788in}{1.453563in}}%
\pgfpathlineto{\pgfqpoint{1.331813in}{1.454649in}}%
\pgfpathlineto{\pgfqpoint{1.332837in}{1.455053in}}%
\pgfpathlineto{\pgfqpoint{1.336043in}{1.456139in}}%
\pgfpathlineto{\pgfqpoint{1.337106in}{1.457040in}}%
\pgfpathlineto{\pgfqpoint{1.339240in}{1.458126in}}%
\pgfpathlineto{\pgfqpoint{1.340295in}{1.458716in}}%
\pgfpathlineto{\pgfqpoint{1.340311in}{1.458716in}}%
\pgfpathlineto{\pgfqpoint{1.342086in}{1.459802in}}%
\pgfpathlineto{\pgfqpoint{1.343188in}{1.460299in}}%
\pgfpathlineto{\pgfqpoint{1.345776in}{1.461385in}}%
\pgfpathlineto{\pgfqpoint{1.346800in}{1.461944in}}%
\pgfpathlineto{\pgfqpoint{1.350513in}{1.463031in}}%
\pgfpathlineto{\pgfqpoint{1.351576in}{1.463496in}}%
\pgfpathlineto{\pgfqpoint{1.354054in}{1.464552in}}%
\pgfpathlineto{\pgfqpoint{1.355094in}{1.465017in}}%
\pgfpathlineto{\pgfqpoint{1.358401in}{1.466104in}}%
\pgfpathlineto{\pgfqpoint{1.359308in}{1.466663in}}%
\pgfpathlineto{\pgfqpoint{1.359355in}{1.466663in}}%
\pgfpathlineto{\pgfqpoint{1.363068in}{1.467749in}}%
\pgfpathlineto{\pgfqpoint{1.363975in}{1.468153in}}%
\pgfpathlineto{\pgfqpoint{1.364085in}{1.468153in}}%
\pgfpathlineto{\pgfqpoint{1.366453in}{1.469239in}}%
\pgfpathlineto{\pgfqpoint{1.367438in}{1.469829in}}%
\pgfpathlineto{\pgfqpoint{1.370518in}{1.470915in}}%
\pgfpathlineto{\pgfqpoint{1.371402in}{1.471505in}}%
\pgfpathlineto{\pgfqpoint{1.371621in}{1.471505in}}%
\pgfpathlineto{\pgfqpoint{1.374107in}{1.472592in}}%
\pgfpathlineto{\pgfqpoint{1.375092in}{1.473026in}}%
\pgfpathlineto{\pgfqpoint{1.375107in}{1.473026in}}%
\pgfpathlineto{\pgfqpoint{1.377148in}{1.474113in}}%
\pgfpathlineto{\pgfqpoint{1.378172in}{1.474702in}}%
\pgfpathlineto{\pgfqpoint{1.380822in}{1.475789in}}%
\pgfpathlineto{\pgfqpoint{1.381877in}{1.476255in}}%
\pgfpathlineto{\pgfqpoint{1.384754in}{1.477341in}}%
\pgfpathlineto{\pgfqpoint{1.385841in}{1.477776in}}%
\pgfpathlineto{\pgfqpoint{1.385857in}{1.477776in}}%
\pgfpathlineto{\pgfqpoint{1.390047in}{1.478862in}}%
\pgfpathlineto{\pgfqpoint{1.391016in}{1.479142in}}%
\pgfpathlineto{\pgfqpoint{1.391141in}{1.479142in}}%
\pgfpathlineto{\pgfqpoint{1.394902in}{1.480228in}}%
\pgfpathlineto{\pgfqpoint{1.395965in}{1.480632in}}%
\pgfpathlineto{\pgfqpoint{1.399092in}{1.481718in}}%
\pgfpathlineto{\pgfqpoint{1.400179in}{1.482122in}}%
\pgfpathlineto{\pgfqpoint{1.400194in}{1.482122in}}%
\pgfpathlineto{\pgfqpoint{1.403251in}{1.483208in}}%
\pgfpathlineto{\pgfqpoint{1.404361in}{1.483767in}}%
\pgfpathlineto{\pgfqpoint{1.406988in}{1.484853in}}%
\pgfpathlineto{\pgfqpoint{1.408012in}{1.485474in}}%
\pgfpathlineto{\pgfqpoint{1.411741in}{1.486561in}}%
\pgfpathlineto{\pgfqpoint{1.412804in}{1.486933in}}%
\pgfpathlineto{\pgfqpoint{1.415579in}{1.488020in}}%
\pgfpathlineto{\pgfqpoint{1.416682in}{1.488578in}}%
\pgfpathlineto{\pgfqpoint{1.420544in}{1.489665in}}%
\pgfpathlineto{\pgfqpoint{1.421591in}{1.490130in}}%
\pgfpathlineto{\pgfqpoint{1.424468in}{1.491217in}}%
\pgfpathlineto{\pgfqpoint{1.425461in}{1.491714in}}%
\pgfpathlineto{\pgfqpoint{1.425531in}{1.491714in}}%
\pgfpathlineto{\pgfqpoint{1.428353in}{1.492800in}}%
\pgfpathlineto{\pgfqpoint{1.429370in}{1.493173in}}%
\pgfpathlineto{\pgfqpoint{1.429463in}{1.493173in}}%
\pgfpathlineto{\pgfqpoint{1.432442in}{1.494259in}}%
\pgfpathlineto{\pgfqpoint{1.433380in}{1.494600in}}%
\pgfpathlineto{\pgfqpoint{1.433466in}{1.494600in}}%
\pgfpathlineto{\pgfqpoint{1.437179in}{1.495687in}}%
\pgfpathlineto{\pgfqpoint{1.438282in}{1.496059in}}%
\pgfpathlineto{\pgfqpoint{1.441213in}{1.497146in}}%
\pgfpathlineto{\pgfqpoint{1.442300in}{1.497581in}}%
\pgfpathlineto{\pgfqpoint{1.445286in}{1.498667in}}%
\pgfpathlineto{\pgfqpoint{1.446310in}{1.499226in}}%
\pgfpathlineto{\pgfqpoint{1.449281in}{1.500312in}}%
\pgfpathlineto{\pgfqpoint{1.450368in}{1.500902in}}%
\pgfpathlineto{\pgfqpoint{1.453354in}{1.501989in}}%
\pgfpathlineto{\pgfqpoint{1.454433in}{1.502330in}}%
\pgfpathlineto{\pgfqpoint{1.458561in}{1.503385in}}%
\pgfpathlineto{\pgfqpoint{1.459632in}{1.503789in}}%
\pgfpathlineto{\pgfqpoint{1.462923in}{1.504875in}}%
\pgfpathlineto{\pgfqpoint{1.463822in}{1.505124in}}%
\pgfpathlineto{\pgfqpoint{1.463900in}{1.505124in}}%
\pgfpathlineto{\pgfqpoint{1.467301in}{1.506210in}}%
\pgfpathlineto{\pgfqpoint{1.468395in}{1.506645in}}%
\pgfpathlineto{\pgfqpoint{1.470874in}{1.507731in}}%
\pgfpathlineto{\pgfqpoint{1.471819in}{1.508042in}}%
\pgfpathlineto{\pgfqpoint{1.471945in}{1.508042in}}%
\pgfpathlineto{\pgfqpoint{1.474915in}{1.509097in}}%
\pgfpathlineto{\pgfqpoint{1.476018in}{1.509408in}}%
\pgfpathlineto{\pgfqpoint{1.476025in}{1.509408in}}%
\pgfpathlineto{\pgfqpoint{1.478425in}{1.510494in}}%
\pgfpathlineto{\pgfqpoint{1.479629in}{1.510898in}}%
\pgfpathlineto{\pgfqpoint{1.484038in}{1.511984in}}%
\pgfpathlineto{\pgfqpoint{1.485070in}{1.512481in}}%
\pgfpathlineto{\pgfqpoint{1.488815in}{1.513567in}}%
\pgfpathlineto{\pgfqpoint{1.489823in}{1.514033in}}%
\pgfpathlineto{\pgfqpoint{1.492786in}{1.515119in}}%
\pgfpathlineto{\pgfqpoint{1.493826in}{1.515399in}}%
\pgfpathlineto{\pgfqpoint{1.497696in}{1.516485in}}%
\pgfpathlineto{\pgfqpoint{1.498790in}{1.516796in}}%
\pgfpathlineto{\pgfqpoint{1.501761in}{1.517820in}}%
\pgfpathlineto{\pgfqpoint{1.501761in}{1.517851in}}%
\pgfpathlineto{\pgfqpoint{1.502801in}{1.518286in}}%
\pgfpathlineto{\pgfqpoint{1.507366in}{1.519372in}}%
\pgfpathlineto{\pgfqpoint{1.508429in}{1.519807in}}%
\pgfpathlineto{\pgfqpoint{1.511729in}{1.520893in}}%
\pgfpathlineto{\pgfqpoint{1.512768in}{1.521328in}}%
\pgfpathlineto{\pgfqpoint{1.516411in}{1.522383in}}%
\pgfpathlineto{\pgfqpoint{1.517467in}{1.522694in}}%
\pgfpathlineto{\pgfqpoint{1.521196in}{1.523749in}}%
\pgfpathlineto{\pgfqpoint{1.522228in}{1.524122in}}%
\pgfpathlineto{\pgfqpoint{1.527309in}{1.525208in}}%
\pgfpathlineto{\pgfqpoint{1.528396in}{1.525612in}}%
\pgfpathlineto{\pgfqpoint{1.532265in}{1.526698in}}%
\pgfpathlineto{\pgfqpoint{1.533250in}{1.526977in}}%
\pgfpathlineto{\pgfqpoint{1.536847in}{1.528064in}}%
\pgfpathlineto{\pgfqpoint{1.537957in}{1.528281in}}%
\pgfpathlineto{\pgfqpoint{1.541350in}{1.529337in}}%
\pgfpathlineto{\pgfqpoint{1.542374in}{1.529678in}}%
\pgfpathlineto{\pgfqpoint{1.546095in}{1.530765in}}%
\pgfpathlineto{\pgfqpoint{1.546994in}{1.531137in}}%
\pgfpathlineto{\pgfqpoint{1.551747in}{1.532224in}}%
\pgfpathlineto{\pgfqpoint{1.552787in}{1.532410in}}%
\pgfpathlineto{\pgfqpoint{1.556883in}{1.533496in}}%
\pgfpathlineto{\pgfqpoint{1.557962in}{1.533931in}}%
\pgfpathlineto{\pgfqpoint{1.557978in}{1.533931in}}%
\pgfpathlineto{\pgfqpoint{1.561902in}{1.535017in}}%
\pgfpathlineto{\pgfqpoint{1.562911in}{1.535452in}}%
\pgfpathlineto{\pgfqpoint{1.567234in}{1.536538in}}%
\pgfpathlineto{\pgfqpoint{1.568297in}{1.536911in}}%
\pgfpathlineto{\pgfqpoint{1.573019in}{1.537966in}}%
\pgfpathlineto{\pgfqpoint{1.574090in}{1.538308in}}%
\pgfpathlineto{\pgfqpoint{1.578655in}{1.539394in}}%
\pgfpathlineto{\pgfqpoint{1.579601in}{1.539643in}}%
\pgfpathlineto{\pgfqpoint{1.583651in}{1.540729in}}%
\pgfpathlineto{\pgfqpoint{1.584737in}{1.541008in}}%
\pgfpathlineto{\pgfqpoint{1.588779in}{1.542095in}}%
\pgfpathlineto{\pgfqpoint{1.589874in}{1.542312in}}%
\pgfpathlineto{\pgfqpoint{1.593048in}{1.543399in}}%
\pgfpathlineto{\pgfqpoint{1.594119in}{1.543492in}}%
\pgfpathlineto{\pgfqpoint{1.594134in}{1.543492in}}%
\pgfpathlineto{\pgfqpoint{1.597996in}{1.544547in}}%
\pgfpathlineto{\pgfqpoint{1.599012in}{1.544951in}}%
\pgfpathlineto{\pgfqpoint{1.603789in}{1.546037in}}%
\pgfpathlineto{\pgfqpoint{1.604852in}{1.546317in}}%
\pgfpathlineto{\pgfqpoint{1.610841in}{1.547403in}}%
\pgfpathlineto{\pgfqpoint{1.611388in}{1.547651in}}%
\pgfpathlineto{\pgfqpoint{1.611708in}{1.547651in}}%
\pgfpathlineto{\pgfqpoint{1.617306in}{1.548738in}}%
\pgfpathlineto{\pgfqpoint{1.618361in}{1.549110in}}%
\pgfpathlineto{\pgfqpoint{1.623091in}{1.550197in}}%
\pgfpathlineto{\pgfqpoint{1.623966in}{1.550569in}}%
\pgfpathlineto{\pgfqpoint{1.624193in}{1.550569in}}%
\pgfpathlineto{\pgfqpoint{1.630181in}{1.551656in}}%
\pgfpathlineto{\pgfqpoint{1.631135in}{1.551780in}}%
\pgfpathlineto{\pgfqpoint{1.631206in}{1.551780in}}%
\pgfpathlineto{\pgfqpoint{1.636084in}{1.552836in}}%
\pgfpathlineto{\pgfqpoint{1.637053in}{1.553208in}}%
\pgfpathlineto{\pgfqpoint{1.637170in}{1.553208in}}%
\pgfpathlineto{\pgfqpoint{1.641157in}{1.554294in}}%
\pgfpathlineto{\pgfqpoint{1.642252in}{1.554481in}}%
\pgfpathlineto{\pgfqpoint{1.646919in}{1.555567in}}%
\pgfpathlineto{\pgfqpoint{1.648013in}{1.555878in}}%
\pgfpathlineto{\pgfqpoint{1.652524in}{1.556964in}}%
\pgfpathlineto{\pgfqpoint{1.653603in}{1.557150in}}%
\pgfpathlineto{\pgfqpoint{1.660655in}{1.558237in}}%
\pgfpathlineto{\pgfqpoint{1.661710in}{1.558454in}}%
\pgfpathlineto{\pgfqpoint{1.661726in}{1.558454in}}%
\pgfpathlineto{\pgfqpoint{1.667956in}{1.559541in}}%
\pgfpathlineto{\pgfqpoint{1.668980in}{1.559727in}}%
\pgfpathlineto{\pgfqpoint{1.674851in}{1.560813in}}%
\pgfpathlineto{\pgfqpoint{1.675883in}{1.560938in}}%
\pgfpathlineto{\pgfqpoint{1.675946in}{1.560938in}}%
\pgfpathlineto{\pgfqpoint{1.681020in}{1.562024in}}%
\pgfpathlineto{\pgfqpoint{1.682020in}{1.562334in}}%
\pgfpathlineto{\pgfqpoint{1.686554in}{1.563421in}}%
\pgfpathlineto{\pgfqpoint{1.687625in}{1.563669in}}%
\pgfpathlineto{\pgfqpoint{1.694622in}{1.564756in}}%
\pgfpathlineto{\pgfqpoint{1.695513in}{1.565035in}}%
\pgfpathlineto{\pgfqpoint{1.700657in}{1.566122in}}%
\pgfpathlineto{\pgfqpoint{1.701760in}{1.566277in}}%
\pgfpathlineto{\pgfqpoint{1.706583in}{1.567363in}}%
\pgfpathlineto{\pgfqpoint{1.707428in}{1.567674in}}%
\pgfpathlineto{\pgfqpoint{1.713815in}{1.568760in}}%
\pgfpathlineto{\pgfqpoint{1.714917in}{1.569040in}}%
\pgfpathlineto{\pgfqpoint{1.720194in}{1.570126in}}%
\pgfpathlineto{\pgfqpoint{1.721241in}{1.570374in}}%
\pgfpathlineto{\pgfqpoint{1.728645in}{1.571461in}}%
\pgfpathlineto{\pgfqpoint{1.729614in}{1.571616in}}%
\pgfpathlineto{\pgfqpoint{1.735477in}{1.572702in}}%
\pgfpathlineto{\pgfqpoint{1.736501in}{1.573199in}}%
\pgfpathlineto{\pgfqpoint{1.742685in}{1.574286in}}%
\pgfpathlineto{\pgfqpoint{1.743709in}{1.574565in}}%
\pgfpathlineto{\pgfqpoint{1.750706in}{1.575651in}}%
\pgfpathlineto{\pgfqpoint{1.751761in}{1.575838in}}%
\pgfpathlineto{\pgfqpoint{1.756249in}{1.576924in}}%
\pgfpathlineto{\pgfqpoint{1.756819in}{1.577079in}}%
\pgfpathlineto{\pgfqpoint{1.763832in}{1.578166in}}%
\pgfpathlineto{\pgfqpoint{1.764848in}{1.578445in}}%
\pgfpathlineto{\pgfqpoint{1.764934in}{1.578445in}}%
\pgfpathlineto{\pgfqpoint{1.772150in}{1.579532in}}%
\pgfpathlineto{\pgfqpoint{1.773197in}{1.579780in}}%
\pgfpathlineto{\pgfqpoint{1.773260in}{1.579780in}}%
\pgfpathlineto{\pgfqpoint{1.780155in}{1.580867in}}%
\pgfpathlineto{\pgfqpoint{1.781031in}{1.581146in}}%
\pgfpathlineto{\pgfqpoint{1.790834in}{1.582232in}}%
\pgfpathlineto{\pgfqpoint{1.791897in}{1.582388in}}%
\pgfpathlineto{\pgfqpoint{1.800966in}{1.583474in}}%
\pgfpathlineto{\pgfqpoint{1.802013in}{1.583660in}}%
\pgfpathlineto{\pgfqpoint{1.812215in}{1.584747in}}%
\pgfpathlineto{\pgfqpoint{1.813192in}{1.585057in}}%
\pgfpathlineto{\pgfqpoint{1.822409in}{1.586144in}}%
\pgfpathlineto{\pgfqpoint{1.822409in}{1.586175in}}%
\pgfpathlineto{\pgfqpoint{1.823113in}{1.586175in}}%
\pgfpathlineto{\pgfqpoint{1.834066in}{1.587261in}}%
\pgfpathlineto{\pgfqpoint{1.834511in}{1.587447in}}%
\pgfpathlineto{\pgfqpoint{1.834965in}{1.587447in}}%
\pgfpathlineto{\pgfqpoint{1.842524in}{1.588534in}}%
\pgfpathlineto{\pgfqpoint{1.843580in}{1.588751in}}%
\pgfpathlineto{\pgfqpoint{1.851952in}{1.589838in}}%
\pgfpathlineto{\pgfqpoint{1.852937in}{1.589993in}}%
\pgfpathlineto{\pgfqpoint{1.853055in}{1.589993in}}%
\pgfpathlineto{\pgfqpoint{1.865453in}{1.591079in}}%
\pgfpathlineto{\pgfqpoint{1.866462in}{1.591266in}}%
\pgfpathlineto{\pgfqpoint{1.876656in}{1.592352in}}%
\pgfpathlineto{\pgfqpoint{1.877680in}{1.592507in}}%
\pgfpathlineto{\pgfqpoint{1.892565in}{1.593594in}}%
\pgfpathlineto{\pgfqpoint{1.893284in}{1.593687in}}%
\pgfpathlineto{\pgfqpoint{1.893628in}{1.593687in}}%
\pgfpathlineto{\pgfqpoint{1.909091in}{1.594773in}}%
\pgfpathlineto{\pgfqpoint{1.909865in}{1.594898in}}%
\pgfpathlineto{\pgfqpoint{1.910108in}{1.594898in}}%
\pgfpathlineto{\pgfqpoint{1.920310in}{1.595984in}}%
\pgfpathlineto{\pgfqpoint{1.920427in}{1.596046in}}%
\pgfpathlineto{\pgfqpoint{1.920497in}{1.596046in}}%
\pgfpathlineto{\pgfqpoint{1.935578in}{1.597133in}}%
\pgfpathlineto{\pgfqpoint{1.936484in}{1.597350in}}%
\pgfpathlineto{\pgfqpoint{1.951127in}{1.598436in}}%
\pgfpathlineto{\pgfqpoint{1.951987in}{1.598498in}}%
\pgfpathlineto{\pgfqpoint{1.968388in}{1.599585in}}%
\pgfpathlineto{\pgfqpoint{1.969451in}{1.599709in}}%
\pgfpathlineto{\pgfqpoint{1.991599in}{1.600796in}}%
\pgfpathlineto{\pgfqpoint{1.992334in}{1.600889in}}%
\pgfpathlineto{\pgfqpoint{1.992654in}{1.600889in}}%
\pgfpathlineto{\pgfqpoint{2.033126in}{1.601944in}}%
\pgfpathlineto{\pgfqpoint{2.033126in}{1.601944in}}%
\pgfusepath{stroke}%
\end{pgfscope}%
\begin{pgfscope}%
\pgfsetrectcap%
\pgfsetmiterjoin%
\pgfsetlinewidth{0.803000pt}%
\definecolor{currentstroke}{rgb}{0.000000,0.000000,0.000000}%
\pgfsetstrokecolor{currentstroke}%
\pgfsetdash{}{0pt}%
\pgfpathmoveto{\pgfqpoint{0.553581in}{0.499444in}}%
\pgfpathlineto{\pgfqpoint{0.553581in}{1.654444in}}%
\pgfusepath{stroke}%
\end{pgfscope}%
\begin{pgfscope}%
\pgfsetrectcap%
\pgfsetmiterjoin%
\pgfsetlinewidth{0.803000pt}%
\definecolor{currentstroke}{rgb}{0.000000,0.000000,0.000000}%
\pgfsetstrokecolor{currentstroke}%
\pgfsetdash{}{0pt}%
\pgfpathmoveto{\pgfqpoint{2.103581in}{0.499444in}}%
\pgfpathlineto{\pgfqpoint{2.103581in}{1.654444in}}%
\pgfusepath{stroke}%
\end{pgfscope}%
\begin{pgfscope}%
\pgfsetrectcap%
\pgfsetmiterjoin%
\pgfsetlinewidth{0.803000pt}%
\definecolor{currentstroke}{rgb}{0.000000,0.000000,0.000000}%
\pgfsetstrokecolor{currentstroke}%
\pgfsetdash{}{0pt}%
\pgfpathmoveto{\pgfqpoint{0.553581in}{0.499444in}}%
\pgfpathlineto{\pgfqpoint{2.103581in}{0.499444in}}%
\pgfusepath{stroke}%
\end{pgfscope}%
\begin{pgfscope}%
\pgfsetrectcap%
\pgfsetmiterjoin%
\pgfsetlinewidth{0.803000pt}%
\definecolor{currentstroke}{rgb}{0.000000,0.000000,0.000000}%
\pgfsetstrokecolor{currentstroke}%
\pgfsetdash{}{0pt}%
\pgfpathmoveto{\pgfqpoint{0.553581in}{1.654444in}}%
\pgfpathlineto{\pgfqpoint{2.103581in}{1.654444in}}%
\pgfusepath{stroke}%
\end{pgfscope}%
\begin{pgfscope}%
\pgfsetbuttcap%
\pgfsetmiterjoin%
\definecolor{currentfill}{rgb}{1.000000,1.000000,1.000000}%
\pgfsetfillcolor{currentfill}%
\pgfsetfillopacity{0.800000}%
\pgfsetlinewidth{1.003750pt}%
\definecolor{currentstroke}{rgb}{0.800000,0.800000,0.800000}%
\pgfsetstrokecolor{currentstroke}%
\pgfsetstrokeopacity{0.800000}%
\pgfsetdash{}{0pt}%
\pgfpathmoveto{\pgfqpoint{0.832747in}{0.568889in}}%
\pgfpathlineto{\pgfqpoint{2.006358in}{0.568889in}}%
\pgfpathquadraticcurveto{\pgfqpoint{2.034136in}{0.568889in}}{\pgfqpoint{2.034136in}{0.596666in}}%
\pgfpathlineto{\pgfqpoint{2.034136in}{0.776388in}}%
\pgfpathquadraticcurveto{\pgfqpoint{2.034136in}{0.804166in}}{\pgfqpoint{2.006358in}{0.804166in}}%
\pgfpathlineto{\pgfqpoint{0.832747in}{0.804166in}}%
\pgfpathquadraticcurveto{\pgfqpoint{0.804970in}{0.804166in}}{\pgfqpoint{0.804970in}{0.776388in}}%
\pgfpathlineto{\pgfqpoint{0.804970in}{0.596666in}}%
\pgfpathquadraticcurveto{\pgfqpoint{0.804970in}{0.568889in}}{\pgfqpoint{0.832747in}{0.568889in}}%
\pgfpathlineto{\pgfqpoint{0.832747in}{0.568889in}}%
\pgfpathclose%
\pgfusepath{stroke,fill}%
\end{pgfscope}%
\begin{pgfscope}%
\pgfsetrectcap%
\pgfsetroundjoin%
\pgfsetlinewidth{1.505625pt}%
\definecolor{currentstroke}{rgb}{0.000000,0.000000,0.000000}%
\pgfsetstrokecolor{currentstroke}%
\pgfsetdash{}{0pt}%
\pgfpathmoveto{\pgfqpoint{0.860525in}{0.700000in}}%
\pgfpathlineto{\pgfqpoint{0.999414in}{0.700000in}}%
\pgfpathlineto{\pgfqpoint{1.138303in}{0.700000in}}%
\pgfusepath{stroke}%
\end{pgfscope}%
\begin{pgfscope}%
\definecolor{textcolor}{rgb}{0.000000,0.000000,0.000000}%
\pgfsetstrokecolor{textcolor}%
\pgfsetfillcolor{textcolor}%
\pgftext[x=1.249414in,y=0.651388in,left,base]{\color{textcolor}\rmfamily\fontsize{10.000000}{12.000000}\selectfont AUC=0.778}%
\end{pgfscope}%
\end{pgfpicture}%
\makeatother%
\endgroup%

\end{tabular}

\

Model 2:  $\alpha = 0.67$ for 33\% chance the ambulance is needed

\noindent\begin{tabular}{@{\hspace{-6pt}}p{4.5in} @{\hspace{-6pt}}p{2.0in}}
	\vskip 0pt
	\qquad \qquad Raw Model Output
	
	%% Creator: Matplotlib, PGF backend
%%
%% To include the figure in your LaTeX document, write
%%   \input{<filename>.pgf}
%%
%% Make sure the required packages are loaded in your preamble
%%   \usepackage{pgf}
%%
%% Also ensure that all the required font packages are loaded; for instance,
%% the lmodern package is sometimes necessary when using math font.
%%   \usepackage{lmodern}
%%
%% Figures using additional raster images can only be included by \input if
%% they are in the same directory as the main LaTeX file. For loading figures
%% from other directories you can use the `import` package
%%   \usepackage{import}
%%
%% and then include the figures with
%%   \import{<path to file>}{<filename>.pgf}
%%
%% Matplotlib used the following preamble
%%   
%%   \usepackage{fontspec}
%%   \makeatletter\@ifpackageloaded{underscore}{}{\usepackage[strings]{underscore}}\makeatother
%%
\begingroup%
\makeatletter%
\begin{pgfpicture}%
\pgfpathrectangle{\pgfpointorigin}{\pgfqpoint{4.509306in}{1.754444in}}%
\pgfusepath{use as bounding box, clip}%
\begin{pgfscope}%
\pgfsetbuttcap%
\pgfsetmiterjoin%
\definecolor{currentfill}{rgb}{1.000000,1.000000,1.000000}%
\pgfsetfillcolor{currentfill}%
\pgfsetlinewidth{0.000000pt}%
\definecolor{currentstroke}{rgb}{1.000000,1.000000,1.000000}%
\pgfsetstrokecolor{currentstroke}%
\pgfsetdash{}{0pt}%
\pgfpathmoveto{\pgfqpoint{0.000000in}{0.000000in}}%
\pgfpathlineto{\pgfqpoint{4.509306in}{0.000000in}}%
\pgfpathlineto{\pgfqpoint{4.509306in}{1.754444in}}%
\pgfpathlineto{\pgfqpoint{0.000000in}{1.754444in}}%
\pgfpathlineto{\pgfqpoint{0.000000in}{0.000000in}}%
\pgfpathclose%
\pgfusepath{fill}%
\end{pgfscope}%
\begin{pgfscope}%
\pgfsetbuttcap%
\pgfsetmiterjoin%
\definecolor{currentfill}{rgb}{1.000000,1.000000,1.000000}%
\pgfsetfillcolor{currentfill}%
\pgfsetlinewidth{0.000000pt}%
\definecolor{currentstroke}{rgb}{0.000000,0.000000,0.000000}%
\pgfsetstrokecolor{currentstroke}%
\pgfsetstrokeopacity{0.000000}%
\pgfsetdash{}{0pt}%
\pgfpathmoveto{\pgfqpoint{0.445556in}{0.499444in}}%
\pgfpathlineto{\pgfqpoint{4.320556in}{0.499444in}}%
\pgfpathlineto{\pgfqpoint{4.320556in}{1.654444in}}%
\pgfpathlineto{\pgfqpoint{0.445556in}{1.654444in}}%
\pgfpathlineto{\pgfqpoint{0.445556in}{0.499444in}}%
\pgfpathclose%
\pgfusepath{fill}%
\end{pgfscope}%
\begin{pgfscope}%
\pgfpathrectangle{\pgfqpoint{0.445556in}{0.499444in}}{\pgfqpoint{3.875000in}{1.155000in}}%
\pgfusepath{clip}%
\pgfsetbuttcap%
\pgfsetmiterjoin%
\pgfsetlinewidth{1.003750pt}%
\definecolor{currentstroke}{rgb}{0.000000,0.000000,0.000000}%
\pgfsetstrokecolor{currentstroke}%
\pgfsetdash{}{0pt}%
\pgfpathmoveto{\pgfqpoint{0.435556in}{0.499444in}}%
\pgfpathlineto{\pgfqpoint{0.483922in}{0.499444in}}%
\pgfpathlineto{\pgfqpoint{0.483922in}{1.196060in}}%
\pgfpathlineto{\pgfqpoint{0.435556in}{1.196060in}}%
\pgfusepath{stroke}%
\end{pgfscope}%
\begin{pgfscope}%
\pgfpathrectangle{\pgfqpoint{0.445556in}{0.499444in}}{\pgfqpoint{3.875000in}{1.155000in}}%
\pgfusepath{clip}%
\pgfsetbuttcap%
\pgfsetmiterjoin%
\pgfsetlinewidth{1.003750pt}%
\definecolor{currentstroke}{rgb}{0.000000,0.000000,0.000000}%
\pgfsetstrokecolor{currentstroke}%
\pgfsetdash{}{0pt}%
\pgfpathmoveto{\pgfqpoint{0.576001in}{0.499444in}}%
\pgfpathlineto{\pgfqpoint{0.637387in}{0.499444in}}%
\pgfpathlineto{\pgfqpoint{0.637387in}{1.565132in}}%
\pgfpathlineto{\pgfqpoint{0.576001in}{1.565132in}}%
\pgfpathlineto{\pgfqpoint{0.576001in}{0.499444in}}%
\pgfpathclose%
\pgfusepath{stroke}%
\end{pgfscope}%
\begin{pgfscope}%
\pgfpathrectangle{\pgfqpoint{0.445556in}{0.499444in}}{\pgfqpoint{3.875000in}{1.155000in}}%
\pgfusepath{clip}%
\pgfsetbuttcap%
\pgfsetmiterjoin%
\pgfsetlinewidth{1.003750pt}%
\definecolor{currentstroke}{rgb}{0.000000,0.000000,0.000000}%
\pgfsetstrokecolor{currentstroke}%
\pgfsetdash{}{0pt}%
\pgfpathmoveto{\pgfqpoint{0.729467in}{0.499444in}}%
\pgfpathlineto{\pgfqpoint{0.790853in}{0.499444in}}%
\pgfpathlineto{\pgfqpoint{0.790853in}{1.599444in}}%
\pgfpathlineto{\pgfqpoint{0.729467in}{1.599444in}}%
\pgfpathlineto{\pgfqpoint{0.729467in}{0.499444in}}%
\pgfpathclose%
\pgfusepath{stroke}%
\end{pgfscope}%
\begin{pgfscope}%
\pgfpathrectangle{\pgfqpoint{0.445556in}{0.499444in}}{\pgfqpoint{3.875000in}{1.155000in}}%
\pgfusepath{clip}%
\pgfsetbuttcap%
\pgfsetmiterjoin%
\pgfsetlinewidth{1.003750pt}%
\definecolor{currentstroke}{rgb}{0.000000,0.000000,0.000000}%
\pgfsetstrokecolor{currentstroke}%
\pgfsetdash{}{0pt}%
\pgfpathmoveto{\pgfqpoint{0.882932in}{0.499444in}}%
\pgfpathlineto{\pgfqpoint{0.944318in}{0.499444in}}%
\pgfpathlineto{\pgfqpoint{0.944318in}{1.555920in}}%
\pgfpathlineto{\pgfqpoint{0.882932in}{1.555920in}}%
\pgfpathlineto{\pgfqpoint{0.882932in}{0.499444in}}%
\pgfpathclose%
\pgfusepath{stroke}%
\end{pgfscope}%
\begin{pgfscope}%
\pgfpathrectangle{\pgfqpoint{0.445556in}{0.499444in}}{\pgfqpoint{3.875000in}{1.155000in}}%
\pgfusepath{clip}%
\pgfsetbuttcap%
\pgfsetmiterjoin%
\pgfsetlinewidth{1.003750pt}%
\definecolor{currentstroke}{rgb}{0.000000,0.000000,0.000000}%
\pgfsetstrokecolor{currentstroke}%
\pgfsetdash{}{0pt}%
\pgfpathmoveto{\pgfqpoint{1.036397in}{0.499444in}}%
\pgfpathlineto{\pgfqpoint{1.097783in}{0.499444in}}%
\pgfpathlineto{\pgfqpoint{1.097783in}{1.498579in}}%
\pgfpathlineto{\pgfqpoint{1.036397in}{1.498579in}}%
\pgfpathlineto{\pgfqpoint{1.036397in}{0.499444in}}%
\pgfpathclose%
\pgfusepath{stroke}%
\end{pgfscope}%
\begin{pgfscope}%
\pgfpathrectangle{\pgfqpoint{0.445556in}{0.499444in}}{\pgfqpoint{3.875000in}{1.155000in}}%
\pgfusepath{clip}%
\pgfsetbuttcap%
\pgfsetmiterjoin%
\pgfsetlinewidth{1.003750pt}%
\definecolor{currentstroke}{rgb}{0.000000,0.000000,0.000000}%
\pgfsetstrokecolor{currentstroke}%
\pgfsetdash{}{0pt}%
\pgfpathmoveto{\pgfqpoint{1.189863in}{0.499444in}}%
\pgfpathlineto{\pgfqpoint{1.251249in}{0.499444in}}%
\pgfpathlineto{\pgfqpoint{1.251249in}{1.466723in}}%
\pgfpathlineto{\pgfqpoint{1.189863in}{1.466723in}}%
\pgfpathlineto{\pgfqpoint{1.189863in}{0.499444in}}%
\pgfpathclose%
\pgfusepath{stroke}%
\end{pgfscope}%
\begin{pgfscope}%
\pgfpathrectangle{\pgfqpoint{0.445556in}{0.499444in}}{\pgfqpoint{3.875000in}{1.155000in}}%
\pgfusepath{clip}%
\pgfsetbuttcap%
\pgfsetmiterjoin%
\pgfsetlinewidth{1.003750pt}%
\definecolor{currentstroke}{rgb}{0.000000,0.000000,0.000000}%
\pgfsetstrokecolor{currentstroke}%
\pgfsetdash{}{0pt}%
\pgfpathmoveto{\pgfqpoint{1.343328in}{0.499444in}}%
\pgfpathlineto{\pgfqpoint{1.404714in}{0.499444in}}%
\pgfpathlineto{\pgfqpoint{1.404714in}{1.406695in}}%
\pgfpathlineto{\pgfqpoint{1.343328in}{1.406695in}}%
\pgfpathlineto{\pgfqpoint{1.343328in}{0.499444in}}%
\pgfpathclose%
\pgfusepath{stroke}%
\end{pgfscope}%
\begin{pgfscope}%
\pgfpathrectangle{\pgfqpoint{0.445556in}{0.499444in}}{\pgfqpoint{3.875000in}{1.155000in}}%
\pgfusepath{clip}%
\pgfsetbuttcap%
\pgfsetmiterjoin%
\pgfsetlinewidth{1.003750pt}%
\definecolor{currentstroke}{rgb}{0.000000,0.000000,0.000000}%
\pgfsetstrokecolor{currentstroke}%
\pgfsetdash{}{0pt}%
\pgfpathmoveto{\pgfqpoint{1.496793in}{0.499444in}}%
\pgfpathlineto{\pgfqpoint{1.558179in}{0.499444in}}%
\pgfpathlineto{\pgfqpoint{1.558179in}{1.343289in}}%
\pgfpathlineto{\pgfqpoint{1.496793in}{1.343289in}}%
\pgfpathlineto{\pgfqpoint{1.496793in}{0.499444in}}%
\pgfpathclose%
\pgfusepath{stroke}%
\end{pgfscope}%
\begin{pgfscope}%
\pgfpathrectangle{\pgfqpoint{0.445556in}{0.499444in}}{\pgfqpoint{3.875000in}{1.155000in}}%
\pgfusepath{clip}%
\pgfsetbuttcap%
\pgfsetmiterjoin%
\pgfsetlinewidth{1.003750pt}%
\definecolor{currentstroke}{rgb}{0.000000,0.000000,0.000000}%
\pgfsetstrokecolor{currentstroke}%
\pgfsetdash{}{0pt}%
\pgfpathmoveto{\pgfqpoint{1.650259in}{0.499444in}}%
\pgfpathlineto{\pgfqpoint{1.711645in}{0.499444in}}%
\pgfpathlineto{\pgfqpoint{1.711645in}{1.288097in}}%
\pgfpathlineto{\pgfqpoint{1.650259in}{1.288097in}}%
\pgfpathlineto{\pgfqpoint{1.650259in}{0.499444in}}%
\pgfpathclose%
\pgfusepath{stroke}%
\end{pgfscope}%
\begin{pgfscope}%
\pgfpathrectangle{\pgfqpoint{0.445556in}{0.499444in}}{\pgfqpoint{3.875000in}{1.155000in}}%
\pgfusepath{clip}%
\pgfsetbuttcap%
\pgfsetmiterjoin%
\pgfsetlinewidth{1.003750pt}%
\definecolor{currentstroke}{rgb}{0.000000,0.000000,0.000000}%
\pgfsetstrokecolor{currentstroke}%
\pgfsetdash{}{0pt}%
\pgfpathmoveto{\pgfqpoint{1.803724in}{0.499444in}}%
\pgfpathlineto{\pgfqpoint{1.865110in}{0.499444in}}%
\pgfpathlineto{\pgfqpoint{1.865110in}{1.219549in}}%
\pgfpathlineto{\pgfqpoint{1.803724in}{1.219549in}}%
\pgfpathlineto{\pgfqpoint{1.803724in}{0.499444in}}%
\pgfpathclose%
\pgfusepath{stroke}%
\end{pgfscope}%
\begin{pgfscope}%
\pgfpathrectangle{\pgfqpoint{0.445556in}{0.499444in}}{\pgfqpoint{3.875000in}{1.155000in}}%
\pgfusepath{clip}%
\pgfsetbuttcap%
\pgfsetmiterjoin%
\pgfsetlinewidth{1.003750pt}%
\definecolor{currentstroke}{rgb}{0.000000,0.000000,0.000000}%
\pgfsetstrokecolor{currentstroke}%
\pgfsetdash{}{0pt}%
\pgfpathmoveto{\pgfqpoint{1.957189in}{0.499444in}}%
\pgfpathlineto{\pgfqpoint{2.018575in}{0.499444in}}%
\pgfpathlineto{\pgfqpoint{2.018575in}{1.159751in}}%
\pgfpathlineto{\pgfqpoint{1.957189in}{1.159751in}}%
\pgfpathlineto{\pgfqpoint{1.957189in}{0.499444in}}%
\pgfpathclose%
\pgfusepath{stroke}%
\end{pgfscope}%
\begin{pgfscope}%
\pgfpathrectangle{\pgfqpoint{0.445556in}{0.499444in}}{\pgfqpoint{3.875000in}{1.155000in}}%
\pgfusepath{clip}%
\pgfsetbuttcap%
\pgfsetmiterjoin%
\pgfsetlinewidth{1.003750pt}%
\definecolor{currentstroke}{rgb}{0.000000,0.000000,0.000000}%
\pgfsetstrokecolor{currentstroke}%
\pgfsetdash{}{0pt}%
\pgfpathmoveto{\pgfqpoint{2.110655in}{0.499444in}}%
\pgfpathlineto{\pgfqpoint{2.172041in}{0.499444in}}%
\pgfpathlineto{\pgfqpoint{2.172041in}{1.097420in}}%
\pgfpathlineto{\pgfqpoint{2.110655in}{1.097420in}}%
\pgfpathlineto{\pgfqpoint{2.110655in}{0.499444in}}%
\pgfpathclose%
\pgfusepath{stroke}%
\end{pgfscope}%
\begin{pgfscope}%
\pgfpathrectangle{\pgfqpoint{0.445556in}{0.499444in}}{\pgfqpoint{3.875000in}{1.155000in}}%
\pgfusepath{clip}%
\pgfsetbuttcap%
\pgfsetmiterjoin%
\pgfsetlinewidth{1.003750pt}%
\definecolor{currentstroke}{rgb}{0.000000,0.000000,0.000000}%
\pgfsetstrokecolor{currentstroke}%
\pgfsetdash{}{0pt}%
\pgfpathmoveto{\pgfqpoint{2.264120in}{0.499444in}}%
\pgfpathlineto{\pgfqpoint{2.325506in}{0.499444in}}%
\pgfpathlineto{\pgfqpoint{2.325506in}{1.048523in}}%
\pgfpathlineto{\pgfqpoint{2.264120in}{1.048523in}}%
\pgfpathlineto{\pgfqpoint{2.264120in}{0.499444in}}%
\pgfpathclose%
\pgfusepath{stroke}%
\end{pgfscope}%
\begin{pgfscope}%
\pgfpathrectangle{\pgfqpoint{0.445556in}{0.499444in}}{\pgfqpoint{3.875000in}{1.155000in}}%
\pgfusepath{clip}%
\pgfsetbuttcap%
\pgfsetmiterjoin%
\pgfsetlinewidth{1.003750pt}%
\definecolor{currentstroke}{rgb}{0.000000,0.000000,0.000000}%
\pgfsetstrokecolor{currentstroke}%
\pgfsetdash{}{0pt}%
\pgfpathmoveto{\pgfqpoint{2.417585in}{0.499444in}}%
\pgfpathlineto{\pgfqpoint{2.478972in}{0.499444in}}%
\pgfpathlineto{\pgfqpoint{2.478972in}{0.993178in}}%
\pgfpathlineto{\pgfqpoint{2.417585in}{0.993178in}}%
\pgfpathlineto{\pgfqpoint{2.417585in}{0.499444in}}%
\pgfpathclose%
\pgfusepath{stroke}%
\end{pgfscope}%
\begin{pgfscope}%
\pgfpathrectangle{\pgfqpoint{0.445556in}{0.499444in}}{\pgfqpoint{3.875000in}{1.155000in}}%
\pgfusepath{clip}%
\pgfsetbuttcap%
\pgfsetmiterjoin%
\pgfsetlinewidth{1.003750pt}%
\definecolor{currentstroke}{rgb}{0.000000,0.000000,0.000000}%
\pgfsetstrokecolor{currentstroke}%
\pgfsetdash{}{0pt}%
\pgfpathmoveto{\pgfqpoint{2.571051in}{0.499444in}}%
\pgfpathlineto{\pgfqpoint{2.632437in}{0.499444in}}%
\pgfpathlineto{\pgfqpoint{2.632437in}{0.958097in}}%
\pgfpathlineto{\pgfqpoint{2.571051in}{0.958097in}}%
\pgfpathlineto{\pgfqpoint{2.571051in}{0.499444in}}%
\pgfpathclose%
\pgfusepath{stroke}%
\end{pgfscope}%
\begin{pgfscope}%
\pgfpathrectangle{\pgfqpoint{0.445556in}{0.499444in}}{\pgfqpoint{3.875000in}{1.155000in}}%
\pgfusepath{clip}%
\pgfsetbuttcap%
\pgfsetmiterjoin%
\pgfsetlinewidth{1.003750pt}%
\definecolor{currentstroke}{rgb}{0.000000,0.000000,0.000000}%
\pgfsetstrokecolor{currentstroke}%
\pgfsetdash{}{0pt}%
\pgfpathmoveto{\pgfqpoint{2.724516in}{0.499444in}}%
\pgfpathlineto{\pgfqpoint{2.785902in}{0.499444in}}%
\pgfpathlineto{\pgfqpoint{2.785902in}{0.906513in}}%
\pgfpathlineto{\pgfqpoint{2.724516in}{0.906513in}}%
\pgfpathlineto{\pgfqpoint{2.724516in}{0.499444in}}%
\pgfpathclose%
\pgfusepath{stroke}%
\end{pgfscope}%
\begin{pgfscope}%
\pgfpathrectangle{\pgfqpoint{0.445556in}{0.499444in}}{\pgfqpoint{3.875000in}{1.155000in}}%
\pgfusepath{clip}%
\pgfsetbuttcap%
\pgfsetmiterjoin%
\pgfsetlinewidth{1.003750pt}%
\definecolor{currentstroke}{rgb}{0.000000,0.000000,0.000000}%
\pgfsetstrokecolor{currentstroke}%
\pgfsetdash{}{0pt}%
\pgfpathmoveto{\pgfqpoint{2.877981in}{0.499444in}}%
\pgfpathlineto{\pgfqpoint{2.939368in}{0.499444in}}%
\pgfpathlineto{\pgfqpoint{2.939368in}{0.860379in}}%
\pgfpathlineto{\pgfqpoint{2.877981in}{0.860379in}}%
\pgfpathlineto{\pgfqpoint{2.877981in}{0.499444in}}%
\pgfpathclose%
\pgfusepath{stroke}%
\end{pgfscope}%
\begin{pgfscope}%
\pgfpathrectangle{\pgfqpoint{0.445556in}{0.499444in}}{\pgfqpoint{3.875000in}{1.155000in}}%
\pgfusepath{clip}%
\pgfsetbuttcap%
\pgfsetmiterjoin%
\pgfsetlinewidth{1.003750pt}%
\definecolor{currentstroke}{rgb}{0.000000,0.000000,0.000000}%
\pgfsetstrokecolor{currentstroke}%
\pgfsetdash{}{0pt}%
\pgfpathmoveto{\pgfqpoint{3.031447in}{0.499444in}}%
\pgfpathlineto{\pgfqpoint{3.092833in}{0.499444in}}%
\pgfpathlineto{\pgfqpoint{3.092833in}{0.810791in}}%
\pgfpathlineto{\pgfqpoint{3.031447in}{0.810791in}}%
\pgfpathlineto{\pgfqpoint{3.031447in}{0.499444in}}%
\pgfpathclose%
\pgfusepath{stroke}%
\end{pgfscope}%
\begin{pgfscope}%
\pgfpathrectangle{\pgfqpoint{0.445556in}{0.499444in}}{\pgfqpoint{3.875000in}{1.155000in}}%
\pgfusepath{clip}%
\pgfsetbuttcap%
\pgfsetmiterjoin%
\pgfsetlinewidth{1.003750pt}%
\definecolor{currentstroke}{rgb}{0.000000,0.000000,0.000000}%
\pgfsetstrokecolor{currentstroke}%
\pgfsetdash{}{0pt}%
\pgfpathmoveto{\pgfqpoint{3.184912in}{0.499444in}}%
\pgfpathlineto{\pgfqpoint{3.246298in}{0.499444in}}%
\pgfpathlineto{\pgfqpoint{3.246298in}{0.746234in}}%
\pgfpathlineto{\pgfqpoint{3.184912in}{0.746234in}}%
\pgfpathlineto{\pgfqpoint{3.184912in}{0.499444in}}%
\pgfpathclose%
\pgfusepath{stroke}%
\end{pgfscope}%
\begin{pgfscope}%
\pgfpathrectangle{\pgfqpoint{0.445556in}{0.499444in}}{\pgfqpoint{3.875000in}{1.155000in}}%
\pgfusepath{clip}%
\pgfsetbuttcap%
\pgfsetmiterjoin%
\pgfsetlinewidth{1.003750pt}%
\definecolor{currentstroke}{rgb}{0.000000,0.000000,0.000000}%
\pgfsetstrokecolor{currentstroke}%
\pgfsetdash{}{0pt}%
\pgfpathmoveto{\pgfqpoint{3.338377in}{0.499444in}}%
\pgfpathlineto{\pgfqpoint{3.399764in}{0.499444in}}%
\pgfpathlineto{\pgfqpoint{3.399764in}{0.706318in}}%
\pgfpathlineto{\pgfqpoint{3.338377in}{0.706318in}}%
\pgfpathlineto{\pgfqpoint{3.338377in}{0.499444in}}%
\pgfpathclose%
\pgfusepath{stroke}%
\end{pgfscope}%
\begin{pgfscope}%
\pgfpathrectangle{\pgfqpoint{0.445556in}{0.499444in}}{\pgfqpoint{3.875000in}{1.155000in}}%
\pgfusepath{clip}%
\pgfsetbuttcap%
\pgfsetmiterjoin%
\pgfsetlinewidth{1.003750pt}%
\definecolor{currentstroke}{rgb}{0.000000,0.000000,0.000000}%
\pgfsetstrokecolor{currentstroke}%
\pgfsetdash{}{0pt}%
\pgfpathmoveto{\pgfqpoint{3.491843in}{0.499444in}}%
\pgfpathlineto{\pgfqpoint{3.553229in}{0.499444in}}%
\pgfpathlineto{\pgfqpoint{3.553229in}{0.652661in}}%
\pgfpathlineto{\pgfqpoint{3.491843in}{0.652661in}}%
\pgfpathlineto{\pgfqpoint{3.491843in}{0.499444in}}%
\pgfpathclose%
\pgfusepath{stroke}%
\end{pgfscope}%
\begin{pgfscope}%
\pgfpathrectangle{\pgfqpoint{0.445556in}{0.499444in}}{\pgfqpoint{3.875000in}{1.155000in}}%
\pgfusepath{clip}%
\pgfsetbuttcap%
\pgfsetmiterjoin%
\pgfsetlinewidth{1.003750pt}%
\definecolor{currentstroke}{rgb}{0.000000,0.000000,0.000000}%
\pgfsetstrokecolor{currentstroke}%
\pgfsetdash{}{0pt}%
\pgfpathmoveto{\pgfqpoint{3.645308in}{0.499444in}}%
\pgfpathlineto{\pgfqpoint{3.706694in}{0.499444in}}%
\pgfpathlineto{\pgfqpoint{3.706694in}{0.605222in}}%
\pgfpathlineto{\pgfqpoint{3.645308in}{0.605222in}}%
\pgfpathlineto{\pgfqpoint{3.645308in}{0.499444in}}%
\pgfpathclose%
\pgfusepath{stroke}%
\end{pgfscope}%
\begin{pgfscope}%
\pgfpathrectangle{\pgfqpoint{0.445556in}{0.499444in}}{\pgfqpoint{3.875000in}{1.155000in}}%
\pgfusepath{clip}%
\pgfsetbuttcap%
\pgfsetmiterjoin%
\pgfsetlinewidth{1.003750pt}%
\definecolor{currentstroke}{rgb}{0.000000,0.000000,0.000000}%
\pgfsetstrokecolor{currentstroke}%
\pgfsetdash{}{0pt}%
\pgfpathmoveto{\pgfqpoint{3.798774in}{0.499444in}}%
\pgfpathlineto{\pgfqpoint{3.860160in}{0.499444in}}%
\pgfpathlineto{\pgfqpoint{3.860160in}{0.570142in}}%
\pgfpathlineto{\pgfqpoint{3.798774in}{0.570142in}}%
\pgfpathlineto{\pgfqpoint{3.798774in}{0.499444in}}%
\pgfpathclose%
\pgfusepath{stroke}%
\end{pgfscope}%
\begin{pgfscope}%
\pgfpathrectangle{\pgfqpoint{0.445556in}{0.499444in}}{\pgfqpoint{3.875000in}{1.155000in}}%
\pgfusepath{clip}%
\pgfsetbuttcap%
\pgfsetmiterjoin%
\pgfsetlinewidth{1.003750pt}%
\definecolor{currentstroke}{rgb}{0.000000,0.000000,0.000000}%
\pgfsetstrokecolor{currentstroke}%
\pgfsetdash{}{0pt}%
\pgfpathmoveto{\pgfqpoint{3.952239in}{0.499444in}}%
\pgfpathlineto{\pgfqpoint{4.013625in}{0.499444in}}%
\pgfpathlineto{\pgfqpoint{4.013625in}{0.554943in}}%
\pgfpathlineto{\pgfqpoint{3.952239in}{0.554943in}}%
\pgfpathlineto{\pgfqpoint{3.952239in}{0.499444in}}%
\pgfpathclose%
\pgfusepath{stroke}%
\end{pgfscope}%
\begin{pgfscope}%
\pgfpathrectangle{\pgfqpoint{0.445556in}{0.499444in}}{\pgfqpoint{3.875000in}{1.155000in}}%
\pgfusepath{clip}%
\pgfsetbuttcap%
\pgfsetmiterjoin%
\pgfsetlinewidth{1.003750pt}%
\definecolor{currentstroke}{rgb}{0.000000,0.000000,0.000000}%
\pgfsetstrokecolor{currentstroke}%
\pgfsetdash{}{0pt}%
\pgfpathmoveto{\pgfqpoint{4.105704in}{0.499444in}}%
\pgfpathlineto{\pgfqpoint{4.167090in}{0.499444in}}%
\pgfpathlineto{\pgfqpoint{4.167090in}{0.512417in}}%
\pgfpathlineto{\pgfqpoint{4.105704in}{0.512417in}}%
\pgfpathlineto{\pgfqpoint{4.105704in}{0.499444in}}%
\pgfpathclose%
\pgfusepath{stroke}%
\end{pgfscope}%
\begin{pgfscope}%
\pgfpathrectangle{\pgfqpoint{0.445556in}{0.499444in}}{\pgfqpoint{3.875000in}{1.155000in}}%
\pgfusepath{clip}%
\pgfsetbuttcap%
\pgfsetmiterjoin%
\definecolor{currentfill}{rgb}{0.000000,0.000000,0.000000}%
\pgfsetfillcolor{currentfill}%
\pgfsetlinewidth{0.000000pt}%
\definecolor{currentstroke}{rgb}{0.000000,0.000000,0.000000}%
\pgfsetstrokecolor{currentstroke}%
\pgfsetstrokeopacity{0.000000}%
\pgfsetdash{}{0pt}%
\pgfpathmoveto{\pgfqpoint{0.483922in}{0.499444in}}%
\pgfpathlineto{\pgfqpoint{0.545308in}{0.499444in}}%
\pgfpathlineto{\pgfqpoint{0.545308in}{0.506506in}}%
\pgfpathlineto{\pgfqpoint{0.483922in}{0.506506in}}%
\pgfpathlineto{\pgfqpoint{0.483922in}{0.499444in}}%
\pgfpathclose%
\pgfusepath{fill}%
\end{pgfscope}%
\begin{pgfscope}%
\pgfpathrectangle{\pgfqpoint{0.445556in}{0.499444in}}{\pgfqpoint{3.875000in}{1.155000in}}%
\pgfusepath{clip}%
\pgfsetbuttcap%
\pgfsetmiterjoin%
\definecolor{currentfill}{rgb}{0.000000,0.000000,0.000000}%
\pgfsetfillcolor{currentfill}%
\pgfsetlinewidth{0.000000pt}%
\definecolor{currentstroke}{rgb}{0.000000,0.000000,0.000000}%
\pgfsetstrokecolor{currentstroke}%
\pgfsetstrokeopacity{0.000000}%
\pgfsetdash{}{0pt}%
\pgfpathmoveto{\pgfqpoint{0.637387in}{0.499444in}}%
\pgfpathlineto{\pgfqpoint{0.698774in}{0.499444in}}%
\pgfpathlineto{\pgfqpoint{0.698774in}{0.518865in}}%
\pgfpathlineto{\pgfqpoint{0.637387in}{0.518865in}}%
\pgfpathlineto{\pgfqpoint{0.637387in}{0.499444in}}%
\pgfpathclose%
\pgfusepath{fill}%
\end{pgfscope}%
\begin{pgfscope}%
\pgfpathrectangle{\pgfqpoint{0.445556in}{0.499444in}}{\pgfqpoint{3.875000in}{1.155000in}}%
\pgfusepath{clip}%
\pgfsetbuttcap%
\pgfsetmiterjoin%
\definecolor{currentfill}{rgb}{0.000000,0.000000,0.000000}%
\pgfsetfillcolor{currentfill}%
\pgfsetlinewidth{0.000000pt}%
\definecolor{currentstroke}{rgb}{0.000000,0.000000,0.000000}%
\pgfsetstrokecolor{currentstroke}%
\pgfsetstrokeopacity{0.000000}%
\pgfsetdash{}{0pt}%
\pgfpathmoveto{\pgfqpoint{0.790853in}{0.499444in}}%
\pgfpathlineto{\pgfqpoint{0.852239in}{0.499444in}}%
\pgfpathlineto{\pgfqpoint{0.852239in}{0.538593in}}%
\pgfpathlineto{\pgfqpoint{0.790853in}{0.538593in}}%
\pgfpathlineto{\pgfqpoint{0.790853in}{0.499444in}}%
\pgfpathclose%
\pgfusepath{fill}%
\end{pgfscope}%
\begin{pgfscope}%
\pgfpathrectangle{\pgfqpoint{0.445556in}{0.499444in}}{\pgfqpoint{3.875000in}{1.155000in}}%
\pgfusepath{clip}%
\pgfsetbuttcap%
\pgfsetmiterjoin%
\definecolor{currentfill}{rgb}{0.000000,0.000000,0.000000}%
\pgfsetfillcolor{currentfill}%
\pgfsetlinewidth{0.000000pt}%
\definecolor{currentstroke}{rgb}{0.000000,0.000000,0.000000}%
\pgfsetstrokecolor{currentstroke}%
\pgfsetstrokeopacity{0.000000}%
\pgfsetdash{}{0pt}%
\pgfpathmoveto{\pgfqpoint{0.944318in}{0.499444in}}%
\pgfpathlineto{\pgfqpoint{1.005704in}{0.499444in}}%
\pgfpathlineto{\pgfqpoint{1.005704in}{0.551105in}}%
\pgfpathlineto{\pgfqpoint{0.944318in}{0.551105in}}%
\pgfpathlineto{\pgfqpoint{0.944318in}{0.499444in}}%
\pgfpathclose%
\pgfusepath{fill}%
\end{pgfscope}%
\begin{pgfscope}%
\pgfpathrectangle{\pgfqpoint{0.445556in}{0.499444in}}{\pgfqpoint{3.875000in}{1.155000in}}%
\pgfusepath{clip}%
\pgfsetbuttcap%
\pgfsetmiterjoin%
\definecolor{currentfill}{rgb}{0.000000,0.000000,0.000000}%
\pgfsetfillcolor{currentfill}%
\pgfsetlinewidth{0.000000pt}%
\definecolor{currentstroke}{rgb}{0.000000,0.000000,0.000000}%
\pgfsetstrokecolor{currentstroke}%
\pgfsetstrokeopacity{0.000000}%
\pgfsetdash{}{0pt}%
\pgfpathmoveto{\pgfqpoint{1.097783in}{0.499444in}}%
\pgfpathlineto{\pgfqpoint{1.159170in}{0.499444in}}%
\pgfpathlineto{\pgfqpoint{1.159170in}{0.562082in}}%
\pgfpathlineto{\pgfqpoint{1.097783in}{0.562082in}}%
\pgfpathlineto{\pgfqpoint{1.097783in}{0.499444in}}%
\pgfpathclose%
\pgfusepath{fill}%
\end{pgfscope}%
\begin{pgfscope}%
\pgfpathrectangle{\pgfqpoint{0.445556in}{0.499444in}}{\pgfqpoint{3.875000in}{1.155000in}}%
\pgfusepath{clip}%
\pgfsetbuttcap%
\pgfsetmiterjoin%
\definecolor{currentfill}{rgb}{0.000000,0.000000,0.000000}%
\pgfsetfillcolor{currentfill}%
\pgfsetlinewidth{0.000000pt}%
\definecolor{currentstroke}{rgb}{0.000000,0.000000,0.000000}%
\pgfsetstrokecolor{currentstroke}%
\pgfsetstrokeopacity{0.000000}%
\pgfsetdash{}{0pt}%
\pgfpathmoveto{\pgfqpoint{1.251249in}{0.499444in}}%
\pgfpathlineto{\pgfqpoint{1.312635in}{0.499444in}}%
\pgfpathlineto{\pgfqpoint{1.312635in}{0.576590in}}%
\pgfpathlineto{\pgfqpoint{1.251249in}{0.576590in}}%
\pgfpathlineto{\pgfqpoint{1.251249in}{0.499444in}}%
\pgfpathclose%
\pgfusepath{fill}%
\end{pgfscope}%
\begin{pgfscope}%
\pgfpathrectangle{\pgfqpoint{0.445556in}{0.499444in}}{\pgfqpoint{3.875000in}{1.155000in}}%
\pgfusepath{clip}%
\pgfsetbuttcap%
\pgfsetmiterjoin%
\definecolor{currentfill}{rgb}{0.000000,0.000000,0.000000}%
\pgfsetfillcolor{currentfill}%
\pgfsetlinewidth{0.000000pt}%
\definecolor{currentstroke}{rgb}{0.000000,0.000000,0.000000}%
\pgfsetstrokecolor{currentstroke}%
\pgfsetstrokeopacity{0.000000}%
\pgfsetdash{}{0pt}%
\pgfpathmoveto{\pgfqpoint{1.404714in}{0.499444in}}%
\pgfpathlineto{\pgfqpoint{1.466100in}{0.499444in}}%
\pgfpathlineto{\pgfqpoint{1.466100in}{0.587874in}}%
\pgfpathlineto{\pgfqpoint{1.404714in}{0.587874in}}%
\pgfpathlineto{\pgfqpoint{1.404714in}{0.499444in}}%
\pgfpathclose%
\pgfusepath{fill}%
\end{pgfscope}%
\begin{pgfscope}%
\pgfpathrectangle{\pgfqpoint{0.445556in}{0.499444in}}{\pgfqpoint{3.875000in}{1.155000in}}%
\pgfusepath{clip}%
\pgfsetbuttcap%
\pgfsetmiterjoin%
\definecolor{currentfill}{rgb}{0.000000,0.000000,0.000000}%
\pgfsetfillcolor{currentfill}%
\pgfsetlinewidth{0.000000pt}%
\definecolor{currentstroke}{rgb}{0.000000,0.000000,0.000000}%
\pgfsetstrokecolor{currentstroke}%
\pgfsetstrokeopacity{0.000000}%
\pgfsetdash{}{0pt}%
\pgfpathmoveto{\pgfqpoint{1.558179in}{0.499444in}}%
\pgfpathlineto{\pgfqpoint{1.619566in}{0.499444in}}%
\pgfpathlineto{\pgfqpoint{1.619566in}{0.598621in}}%
\pgfpathlineto{\pgfqpoint{1.558179in}{0.598621in}}%
\pgfpathlineto{\pgfqpoint{1.558179in}{0.499444in}}%
\pgfpathclose%
\pgfusepath{fill}%
\end{pgfscope}%
\begin{pgfscope}%
\pgfpathrectangle{\pgfqpoint{0.445556in}{0.499444in}}{\pgfqpoint{3.875000in}{1.155000in}}%
\pgfusepath{clip}%
\pgfsetbuttcap%
\pgfsetmiterjoin%
\definecolor{currentfill}{rgb}{0.000000,0.000000,0.000000}%
\pgfsetfillcolor{currentfill}%
\pgfsetlinewidth{0.000000pt}%
\definecolor{currentstroke}{rgb}{0.000000,0.000000,0.000000}%
\pgfsetstrokecolor{currentstroke}%
\pgfsetstrokeopacity{0.000000}%
\pgfsetdash{}{0pt}%
\pgfpathmoveto{\pgfqpoint{1.711645in}{0.499444in}}%
\pgfpathlineto{\pgfqpoint{1.773031in}{0.499444in}}%
\pgfpathlineto{\pgfqpoint{1.773031in}{0.601307in}}%
\pgfpathlineto{\pgfqpoint{1.711645in}{0.601307in}}%
\pgfpathlineto{\pgfqpoint{1.711645in}{0.499444in}}%
\pgfpathclose%
\pgfusepath{fill}%
\end{pgfscope}%
\begin{pgfscope}%
\pgfpathrectangle{\pgfqpoint{0.445556in}{0.499444in}}{\pgfqpoint{3.875000in}{1.155000in}}%
\pgfusepath{clip}%
\pgfsetbuttcap%
\pgfsetmiterjoin%
\definecolor{currentfill}{rgb}{0.000000,0.000000,0.000000}%
\pgfsetfillcolor{currentfill}%
\pgfsetlinewidth{0.000000pt}%
\definecolor{currentstroke}{rgb}{0.000000,0.000000,0.000000}%
\pgfsetstrokecolor{currentstroke}%
\pgfsetstrokeopacity{0.000000}%
\pgfsetdash{}{0pt}%
\pgfpathmoveto{\pgfqpoint{1.865110in}{0.499444in}}%
\pgfpathlineto{\pgfqpoint{1.926496in}{0.499444in}}%
\pgfpathlineto{\pgfqpoint{1.926496in}{0.612668in}}%
\pgfpathlineto{\pgfqpoint{1.865110in}{0.612668in}}%
\pgfpathlineto{\pgfqpoint{1.865110in}{0.499444in}}%
\pgfpathclose%
\pgfusepath{fill}%
\end{pgfscope}%
\begin{pgfscope}%
\pgfpathrectangle{\pgfqpoint{0.445556in}{0.499444in}}{\pgfqpoint{3.875000in}{1.155000in}}%
\pgfusepath{clip}%
\pgfsetbuttcap%
\pgfsetmiterjoin%
\definecolor{currentfill}{rgb}{0.000000,0.000000,0.000000}%
\pgfsetfillcolor{currentfill}%
\pgfsetlinewidth{0.000000pt}%
\definecolor{currentstroke}{rgb}{0.000000,0.000000,0.000000}%
\pgfsetstrokecolor{currentstroke}%
\pgfsetstrokeopacity{0.000000}%
\pgfsetdash{}{0pt}%
\pgfpathmoveto{\pgfqpoint{2.018575in}{0.499444in}}%
\pgfpathlineto{\pgfqpoint{2.079962in}{0.499444in}}%
\pgfpathlineto{\pgfqpoint{2.079962in}{0.620037in}}%
\pgfpathlineto{\pgfqpoint{2.018575in}{0.620037in}}%
\pgfpathlineto{\pgfqpoint{2.018575in}{0.499444in}}%
\pgfpathclose%
\pgfusepath{fill}%
\end{pgfscope}%
\begin{pgfscope}%
\pgfpathrectangle{\pgfqpoint{0.445556in}{0.499444in}}{\pgfqpoint{3.875000in}{1.155000in}}%
\pgfusepath{clip}%
\pgfsetbuttcap%
\pgfsetmiterjoin%
\definecolor{currentfill}{rgb}{0.000000,0.000000,0.000000}%
\pgfsetfillcolor{currentfill}%
\pgfsetlinewidth{0.000000pt}%
\definecolor{currentstroke}{rgb}{0.000000,0.000000,0.000000}%
\pgfsetstrokecolor{currentstroke}%
\pgfsetstrokeopacity{0.000000}%
\pgfsetdash{}{0pt}%
\pgfpathmoveto{\pgfqpoint{2.172041in}{0.499444in}}%
\pgfpathlineto{\pgfqpoint{2.233427in}{0.499444in}}%
\pgfpathlineto{\pgfqpoint{2.233427in}{0.625180in}}%
\pgfpathlineto{\pgfqpoint{2.172041in}{0.625180in}}%
\pgfpathlineto{\pgfqpoint{2.172041in}{0.499444in}}%
\pgfpathclose%
\pgfusepath{fill}%
\end{pgfscope}%
\begin{pgfscope}%
\pgfpathrectangle{\pgfqpoint{0.445556in}{0.499444in}}{\pgfqpoint{3.875000in}{1.155000in}}%
\pgfusepath{clip}%
\pgfsetbuttcap%
\pgfsetmiterjoin%
\definecolor{currentfill}{rgb}{0.000000,0.000000,0.000000}%
\pgfsetfillcolor{currentfill}%
\pgfsetlinewidth{0.000000pt}%
\definecolor{currentstroke}{rgb}{0.000000,0.000000,0.000000}%
\pgfsetstrokecolor{currentstroke}%
\pgfsetstrokeopacity{0.000000}%
\pgfsetdash{}{0pt}%
\pgfpathmoveto{\pgfqpoint{2.325506in}{0.499444in}}%
\pgfpathlineto{\pgfqpoint{2.386892in}{0.499444in}}%
\pgfpathlineto{\pgfqpoint{2.386892in}{0.628251in}}%
\pgfpathlineto{\pgfqpoint{2.325506in}{0.628251in}}%
\pgfpathlineto{\pgfqpoint{2.325506in}{0.499444in}}%
\pgfpathclose%
\pgfusepath{fill}%
\end{pgfscope}%
\begin{pgfscope}%
\pgfpathrectangle{\pgfqpoint{0.445556in}{0.499444in}}{\pgfqpoint{3.875000in}{1.155000in}}%
\pgfusepath{clip}%
\pgfsetbuttcap%
\pgfsetmiterjoin%
\definecolor{currentfill}{rgb}{0.000000,0.000000,0.000000}%
\pgfsetfillcolor{currentfill}%
\pgfsetlinewidth{0.000000pt}%
\definecolor{currentstroke}{rgb}{0.000000,0.000000,0.000000}%
\pgfsetstrokecolor{currentstroke}%
\pgfsetstrokeopacity{0.000000}%
\pgfsetdash{}{0pt}%
\pgfpathmoveto{\pgfqpoint{2.478972in}{0.499444in}}%
\pgfpathlineto{\pgfqpoint{2.540358in}{0.499444in}}%
\pgfpathlineto{\pgfqpoint{2.540358in}{0.637616in}}%
\pgfpathlineto{\pgfqpoint{2.478972in}{0.637616in}}%
\pgfpathlineto{\pgfqpoint{2.478972in}{0.499444in}}%
\pgfpathclose%
\pgfusepath{fill}%
\end{pgfscope}%
\begin{pgfscope}%
\pgfpathrectangle{\pgfqpoint{0.445556in}{0.499444in}}{\pgfqpoint{3.875000in}{1.155000in}}%
\pgfusepath{clip}%
\pgfsetbuttcap%
\pgfsetmiterjoin%
\definecolor{currentfill}{rgb}{0.000000,0.000000,0.000000}%
\pgfsetfillcolor{currentfill}%
\pgfsetlinewidth{0.000000pt}%
\definecolor{currentstroke}{rgb}{0.000000,0.000000,0.000000}%
\pgfsetstrokecolor{currentstroke}%
\pgfsetstrokeopacity{0.000000}%
\pgfsetdash{}{0pt}%
\pgfpathmoveto{\pgfqpoint{2.632437in}{0.499444in}}%
\pgfpathlineto{\pgfqpoint{2.693823in}{0.499444in}}%
\pgfpathlineto{\pgfqpoint{2.693823in}{0.638537in}}%
\pgfpathlineto{\pgfqpoint{2.632437in}{0.638537in}}%
\pgfpathlineto{\pgfqpoint{2.632437in}{0.499444in}}%
\pgfpathclose%
\pgfusepath{fill}%
\end{pgfscope}%
\begin{pgfscope}%
\pgfpathrectangle{\pgfqpoint{0.445556in}{0.499444in}}{\pgfqpoint{3.875000in}{1.155000in}}%
\pgfusepath{clip}%
\pgfsetbuttcap%
\pgfsetmiterjoin%
\definecolor{currentfill}{rgb}{0.000000,0.000000,0.000000}%
\pgfsetfillcolor{currentfill}%
\pgfsetlinewidth{0.000000pt}%
\definecolor{currentstroke}{rgb}{0.000000,0.000000,0.000000}%
\pgfsetstrokecolor{currentstroke}%
\pgfsetstrokeopacity{0.000000}%
\pgfsetdash{}{0pt}%
\pgfpathmoveto{\pgfqpoint{2.785902in}{0.499444in}}%
\pgfpathlineto{\pgfqpoint{2.847288in}{0.499444in}}%
\pgfpathlineto{\pgfqpoint{2.847288in}{0.640840in}}%
\pgfpathlineto{\pgfqpoint{2.785902in}{0.640840in}}%
\pgfpathlineto{\pgfqpoint{2.785902in}{0.499444in}}%
\pgfpathclose%
\pgfusepath{fill}%
\end{pgfscope}%
\begin{pgfscope}%
\pgfpathrectangle{\pgfqpoint{0.445556in}{0.499444in}}{\pgfqpoint{3.875000in}{1.155000in}}%
\pgfusepath{clip}%
\pgfsetbuttcap%
\pgfsetmiterjoin%
\definecolor{currentfill}{rgb}{0.000000,0.000000,0.000000}%
\pgfsetfillcolor{currentfill}%
\pgfsetlinewidth{0.000000pt}%
\definecolor{currentstroke}{rgb}{0.000000,0.000000,0.000000}%
\pgfsetstrokecolor{currentstroke}%
\pgfsetstrokeopacity{0.000000}%
\pgfsetdash{}{0pt}%
\pgfpathmoveto{\pgfqpoint{2.939368in}{0.499444in}}%
\pgfpathlineto{\pgfqpoint{3.000754in}{0.499444in}}%
\pgfpathlineto{\pgfqpoint{3.000754in}{0.650589in}}%
\pgfpathlineto{\pgfqpoint{2.939368in}{0.650589in}}%
\pgfpathlineto{\pgfqpoint{2.939368in}{0.499444in}}%
\pgfpathclose%
\pgfusepath{fill}%
\end{pgfscope}%
\begin{pgfscope}%
\pgfpathrectangle{\pgfqpoint{0.445556in}{0.499444in}}{\pgfqpoint{3.875000in}{1.155000in}}%
\pgfusepath{clip}%
\pgfsetbuttcap%
\pgfsetmiterjoin%
\definecolor{currentfill}{rgb}{0.000000,0.000000,0.000000}%
\pgfsetfillcolor{currentfill}%
\pgfsetlinewidth{0.000000pt}%
\definecolor{currentstroke}{rgb}{0.000000,0.000000,0.000000}%
\pgfsetstrokecolor{currentstroke}%
\pgfsetstrokeopacity{0.000000}%
\pgfsetdash{}{0pt}%
\pgfpathmoveto{\pgfqpoint{3.092833in}{0.499444in}}%
\pgfpathlineto{\pgfqpoint{3.154219in}{0.499444in}}%
\pgfpathlineto{\pgfqpoint{3.154219in}{0.646367in}}%
\pgfpathlineto{\pgfqpoint{3.092833in}{0.646367in}}%
\pgfpathlineto{\pgfqpoint{3.092833in}{0.499444in}}%
\pgfpathclose%
\pgfusepath{fill}%
\end{pgfscope}%
\begin{pgfscope}%
\pgfpathrectangle{\pgfqpoint{0.445556in}{0.499444in}}{\pgfqpoint{3.875000in}{1.155000in}}%
\pgfusepath{clip}%
\pgfsetbuttcap%
\pgfsetmiterjoin%
\definecolor{currentfill}{rgb}{0.000000,0.000000,0.000000}%
\pgfsetfillcolor{currentfill}%
\pgfsetlinewidth{0.000000pt}%
\definecolor{currentstroke}{rgb}{0.000000,0.000000,0.000000}%
\pgfsetstrokecolor{currentstroke}%
\pgfsetstrokeopacity{0.000000}%
\pgfsetdash{}{0pt}%
\pgfpathmoveto{\pgfqpoint{3.246298in}{0.499444in}}%
\pgfpathlineto{\pgfqpoint{3.307684in}{0.499444in}}%
\pgfpathlineto{\pgfqpoint{3.307684in}{0.650051in}}%
\pgfpathlineto{\pgfqpoint{3.246298in}{0.650051in}}%
\pgfpathlineto{\pgfqpoint{3.246298in}{0.499444in}}%
\pgfpathclose%
\pgfusepath{fill}%
\end{pgfscope}%
\begin{pgfscope}%
\pgfpathrectangle{\pgfqpoint{0.445556in}{0.499444in}}{\pgfqpoint{3.875000in}{1.155000in}}%
\pgfusepath{clip}%
\pgfsetbuttcap%
\pgfsetmiterjoin%
\definecolor{currentfill}{rgb}{0.000000,0.000000,0.000000}%
\pgfsetfillcolor{currentfill}%
\pgfsetlinewidth{0.000000pt}%
\definecolor{currentstroke}{rgb}{0.000000,0.000000,0.000000}%
\pgfsetstrokecolor{currentstroke}%
\pgfsetstrokeopacity{0.000000}%
\pgfsetdash{}{0pt}%
\pgfpathmoveto{\pgfqpoint{3.399764in}{0.499444in}}%
\pgfpathlineto{\pgfqpoint{3.461150in}{0.499444in}}%
\pgfpathlineto{\pgfqpoint{3.461150in}{0.649975in}}%
\pgfpathlineto{\pgfqpoint{3.399764in}{0.649975in}}%
\pgfpathlineto{\pgfqpoint{3.399764in}{0.499444in}}%
\pgfpathclose%
\pgfusepath{fill}%
\end{pgfscope}%
\begin{pgfscope}%
\pgfpathrectangle{\pgfqpoint{0.445556in}{0.499444in}}{\pgfqpoint{3.875000in}{1.155000in}}%
\pgfusepath{clip}%
\pgfsetbuttcap%
\pgfsetmiterjoin%
\definecolor{currentfill}{rgb}{0.000000,0.000000,0.000000}%
\pgfsetfillcolor{currentfill}%
\pgfsetlinewidth{0.000000pt}%
\definecolor{currentstroke}{rgb}{0.000000,0.000000,0.000000}%
\pgfsetstrokecolor{currentstroke}%
\pgfsetstrokeopacity{0.000000}%
\pgfsetdash{}{0pt}%
\pgfpathmoveto{\pgfqpoint{3.553229in}{0.499444in}}%
\pgfpathlineto{\pgfqpoint{3.614615in}{0.499444in}}%
\pgfpathlineto{\pgfqpoint{3.614615in}{0.626409in}}%
\pgfpathlineto{\pgfqpoint{3.553229in}{0.626409in}}%
\pgfpathlineto{\pgfqpoint{3.553229in}{0.499444in}}%
\pgfpathclose%
\pgfusepath{fill}%
\end{pgfscope}%
\begin{pgfscope}%
\pgfpathrectangle{\pgfqpoint{0.445556in}{0.499444in}}{\pgfqpoint{3.875000in}{1.155000in}}%
\pgfusepath{clip}%
\pgfsetbuttcap%
\pgfsetmiterjoin%
\definecolor{currentfill}{rgb}{0.000000,0.000000,0.000000}%
\pgfsetfillcolor{currentfill}%
\pgfsetlinewidth{0.000000pt}%
\definecolor{currentstroke}{rgb}{0.000000,0.000000,0.000000}%
\pgfsetstrokecolor{currentstroke}%
\pgfsetstrokeopacity{0.000000}%
\pgfsetdash{}{0pt}%
\pgfpathmoveto{\pgfqpoint{3.706694in}{0.499444in}}%
\pgfpathlineto{\pgfqpoint{3.768080in}{0.499444in}}%
\pgfpathlineto{\pgfqpoint{3.768080in}{0.625718in}}%
\pgfpathlineto{\pgfqpoint{3.706694in}{0.625718in}}%
\pgfpathlineto{\pgfqpoint{3.706694in}{0.499444in}}%
\pgfpathclose%
\pgfusepath{fill}%
\end{pgfscope}%
\begin{pgfscope}%
\pgfpathrectangle{\pgfqpoint{0.445556in}{0.499444in}}{\pgfqpoint{3.875000in}{1.155000in}}%
\pgfusepath{clip}%
\pgfsetbuttcap%
\pgfsetmiterjoin%
\definecolor{currentfill}{rgb}{0.000000,0.000000,0.000000}%
\pgfsetfillcolor{currentfill}%
\pgfsetlinewidth{0.000000pt}%
\definecolor{currentstroke}{rgb}{0.000000,0.000000,0.000000}%
\pgfsetstrokecolor{currentstroke}%
\pgfsetstrokeopacity{0.000000}%
\pgfsetdash{}{0pt}%
\pgfpathmoveto{\pgfqpoint{3.860160in}{0.499444in}}%
\pgfpathlineto{\pgfqpoint{3.921546in}{0.499444in}}%
\pgfpathlineto{\pgfqpoint{3.921546in}{0.608984in}}%
\pgfpathlineto{\pgfqpoint{3.860160in}{0.608984in}}%
\pgfpathlineto{\pgfqpoint{3.860160in}{0.499444in}}%
\pgfpathclose%
\pgfusepath{fill}%
\end{pgfscope}%
\begin{pgfscope}%
\pgfpathrectangle{\pgfqpoint{0.445556in}{0.499444in}}{\pgfqpoint{3.875000in}{1.155000in}}%
\pgfusepath{clip}%
\pgfsetbuttcap%
\pgfsetmiterjoin%
\definecolor{currentfill}{rgb}{0.000000,0.000000,0.000000}%
\pgfsetfillcolor{currentfill}%
\pgfsetlinewidth{0.000000pt}%
\definecolor{currentstroke}{rgb}{0.000000,0.000000,0.000000}%
\pgfsetstrokecolor{currentstroke}%
\pgfsetstrokeopacity{0.000000}%
\pgfsetdash{}{0pt}%
\pgfpathmoveto{\pgfqpoint{4.013625in}{0.499444in}}%
\pgfpathlineto{\pgfqpoint{4.075011in}{0.499444in}}%
\pgfpathlineto{\pgfqpoint{4.075011in}{0.629402in}}%
\pgfpathlineto{\pgfqpoint{4.013625in}{0.629402in}}%
\pgfpathlineto{\pgfqpoint{4.013625in}{0.499444in}}%
\pgfpathclose%
\pgfusepath{fill}%
\end{pgfscope}%
\begin{pgfscope}%
\pgfpathrectangle{\pgfqpoint{0.445556in}{0.499444in}}{\pgfqpoint{3.875000in}{1.155000in}}%
\pgfusepath{clip}%
\pgfsetbuttcap%
\pgfsetmiterjoin%
\definecolor{currentfill}{rgb}{0.000000,0.000000,0.000000}%
\pgfsetfillcolor{currentfill}%
\pgfsetlinewidth{0.000000pt}%
\definecolor{currentstroke}{rgb}{0.000000,0.000000,0.000000}%
\pgfsetstrokecolor{currentstroke}%
\pgfsetstrokeopacity{0.000000}%
\pgfsetdash{}{0pt}%
\pgfpathmoveto{\pgfqpoint{4.167090in}{0.499444in}}%
\pgfpathlineto{\pgfqpoint{4.228476in}{0.499444in}}%
\pgfpathlineto{\pgfqpoint{4.228476in}{0.550414in}}%
\pgfpathlineto{\pgfqpoint{4.167090in}{0.550414in}}%
\pgfpathlineto{\pgfqpoint{4.167090in}{0.499444in}}%
\pgfpathclose%
\pgfusepath{fill}%
\end{pgfscope}%
\begin{pgfscope}%
\pgfsetbuttcap%
\pgfsetroundjoin%
\definecolor{currentfill}{rgb}{0.000000,0.000000,0.000000}%
\pgfsetfillcolor{currentfill}%
\pgfsetlinewidth{0.803000pt}%
\definecolor{currentstroke}{rgb}{0.000000,0.000000,0.000000}%
\pgfsetstrokecolor{currentstroke}%
\pgfsetdash{}{0pt}%
\pgfsys@defobject{currentmarker}{\pgfqpoint{0.000000in}{-0.048611in}}{\pgfqpoint{0.000000in}{0.000000in}}{%
\pgfpathmoveto{\pgfqpoint{0.000000in}{0.000000in}}%
\pgfpathlineto{\pgfqpoint{0.000000in}{-0.048611in}}%
\pgfusepath{stroke,fill}%
}%
\begin{pgfscope}%
\pgfsys@transformshift{0.483922in}{0.499444in}%
\pgfsys@useobject{currentmarker}{}%
\end{pgfscope}%
\end{pgfscope}%
\begin{pgfscope}%
\definecolor{textcolor}{rgb}{0.000000,0.000000,0.000000}%
\pgfsetstrokecolor{textcolor}%
\pgfsetfillcolor{textcolor}%
\pgftext[x=0.483922in,y=0.402222in,,top]{\color{textcolor}\rmfamily\fontsize{10.000000}{12.000000}\selectfont 0.0}%
\end{pgfscope}%
\begin{pgfscope}%
\pgfsetbuttcap%
\pgfsetroundjoin%
\definecolor{currentfill}{rgb}{0.000000,0.000000,0.000000}%
\pgfsetfillcolor{currentfill}%
\pgfsetlinewidth{0.803000pt}%
\definecolor{currentstroke}{rgb}{0.000000,0.000000,0.000000}%
\pgfsetstrokecolor{currentstroke}%
\pgfsetdash{}{0pt}%
\pgfsys@defobject{currentmarker}{\pgfqpoint{0.000000in}{-0.048611in}}{\pgfqpoint{0.000000in}{0.000000in}}{%
\pgfpathmoveto{\pgfqpoint{0.000000in}{0.000000in}}%
\pgfpathlineto{\pgfqpoint{0.000000in}{-0.048611in}}%
\pgfusepath{stroke,fill}%
}%
\begin{pgfscope}%
\pgfsys@transformshift{0.867585in}{0.499444in}%
\pgfsys@useobject{currentmarker}{}%
\end{pgfscope}%
\end{pgfscope}%
\begin{pgfscope}%
\definecolor{textcolor}{rgb}{0.000000,0.000000,0.000000}%
\pgfsetstrokecolor{textcolor}%
\pgfsetfillcolor{textcolor}%
\pgftext[x=0.867585in,y=0.402222in,,top]{\color{textcolor}\rmfamily\fontsize{10.000000}{12.000000}\selectfont 0.1}%
\end{pgfscope}%
\begin{pgfscope}%
\pgfsetbuttcap%
\pgfsetroundjoin%
\definecolor{currentfill}{rgb}{0.000000,0.000000,0.000000}%
\pgfsetfillcolor{currentfill}%
\pgfsetlinewidth{0.803000pt}%
\definecolor{currentstroke}{rgb}{0.000000,0.000000,0.000000}%
\pgfsetstrokecolor{currentstroke}%
\pgfsetdash{}{0pt}%
\pgfsys@defobject{currentmarker}{\pgfqpoint{0.000000in}{-0.048611in}}{\pgfqpoint{0.000000in}{0.000000in}}{%
\pgfpathmoveto{\pgfqpoint{0.000000in}{0.000000in}}%
\pgfpathlineto{\pgfqpoint{0.000000in}{-0.048611in}}%
\pgfusepath{stroke,fill}%
}%
\begin{pgfscope}%
\pgfsys@transformshift{1.251249in}{0.499444in}%
\pgfsys@useobject{currentmarker}{}%
\end{pgfscope}%
\end{pgfscope}%
\begin{pgfscope}%
\definecolor{textcolor}{rgb}{0.000000,0.000000,0.000000}%
\pgfsetstrokecolor{textcolor}%
\pgfsetfillcolor{textcolor}%
\pgftext[x=1.251249in,y=0.402222in,,top]{\color{textcolor}\rmfamily\fontsize{10.000000}{12.000000}\selectfont 0.2}%
\end{pgfscope}%
\begin{pgfscope}%
\pgfsetbuttcap%
\pgfsetroundjoin%
\definecolor{currentfill}{rgb}{0.000000,0.000000,0.000000}%
\pgfsetfillcolor{currentfill}%
\pgfsetlinewidth{0.803000pt}%
\definecolor{currentstroke}{rgb}{0.000000,0.000000,0.000000}%
\pgfsetstrokecolor{currentstroke}%
\pgfsetdash{}{0pt}%
\pgfsys@defobject{currentmarker}{\pgfqpoint{0.000000in}{-0.048611in}}{\pgfqpoint{0.000000in}{0.000000in}}{%
\pgfpathmoveto{\pgfqpoint{0.000000in}{0.000000in}}%
\pgfpathlineto{\pgfqpoint{0.000000in}{-0.048611in}}%
\pgfusepath{stroke,fill}%
}%
\begin{pgfscope}%
\pgfsys@transformshift{1.634912in}{0.499444in}%
\pgfsys@useobject{currentmarker}{}%
\end{pgfscope}%
\end{pgfscope}%
\begin{pgfscope}%
\definecolor{textcolor}{rgb}{0.000000,0.000000,0.000000}%
\pgfsetstrokecolor{textcolor}%
\pgfsetfillcolor{textcolor}%
\pgftext[x=1.634912in,y=0.402222in,,top]{\color{textcolor}\rmfamily\fontsize{10.000000}{12.000000}\selectfont 0.3}%
\end{pgfscope}%
\begin{pgfscope}%
\pgfsetbuttcap%
\pgfsetroundjoin%
\definecolor{currentfill}{rgb}{0.000000,0.000000,0.000000}%
\pgfsetfillcolor{currentfill}%
\pgfsetlinewidth{0.803000pt}%
\definecolor{currentstroke}{rgb}{0.000000,0.000000,0.000000}%
\pgfsetstrokecolor{currentstroke}%
\pgfsetdash{}{0pt}%
\pgfsys@defobject{currentmarker}{\pgfqpoint{0.000000in}{-0.048611in}}{\pgfqpoint{0.000000in}{0.000000in}}{%
\pgfpathmoveto{\pgfqpoint{0.000000in}{0.000000in}}%
\pgfpathlineto{\pgfqpoint{0.000000in}{-0.048611in}}%
\pgfusepath{stroke,fill}%
}%
\begin{pgfscope}%
\pgfsys@transformshift{2.018575in}{0.499444in}%
\pgfsys@useobject{currentmarker}{}%
\end{pgfscope}%
\end{pgfscope}%
\begin{pgfscope}%
\definecolor{textcolor}{rgb}{0.000000,0.000000,0.000000}%
\pgfsetstrokecolor{textcolor}%
\pgfsetfillcolor{textcolor}%
\pgftext[x=2.018575in,y=0.402222in,,top]{\color{textcolor}\rmfamily\fontsize{10.000000}{12.000000}\selectfont 0.4}%
\end{pgfscope}%
\begin{pgfscope}%
\pgfsetbuttcap%
\pgfsetroundjoin%
\definecolor{currentfill}{rgb}{0.000000,0.000000,0.000000}%
\pgfsetfillcolor{currentfill}%
\pgfsetlinewidth{0.803000pt}%
\definecolor{currentstroke}{rgb}{0.000000,0.000000,0.000000}%
\pgfsetstrokecolor{currentstroke}%
\pgfsetdash{}{0pt}%
\pgfsys@defobject{currentmarker}{\pgfqpoint{0.000000in}{-0.048611in}}{\pgfqpoint{0.000000in}{0.000000in}}{%
\pgfpathmoveto{\pgfqpoint{0.000000in}{0.000000in}}%
\pgfpathlineto{\pgfqpoint{0.000000in}{-0.048611in}}%
\pgfusepath{stroke,fill}%
}%
\begin{pgfscope}%
\pgfsys@transformshift{2.402239in}{0.499444in}%
\pgfsys@useobject{currentmarker}{}%
\end{pgfscope}%
\end{pgfscope}%
\begin{pgfscope}%
\definecolor{textcolor}{rgb}{0.000000,0.000000,0.000000}%
\pgfsetstrokecolor{textcolor}%
\pgfsetfillcolor{textcolor}%
\pgftext[x=2.402239in,y=0.402222in,,top]{\color{textcolor}\rmfamily\fontsize{10.000000}{12.000000}\selectfont 0.5}%
\end{pgfscope}%
\begin{pgfscope}%
\pgfsetbuttcap%
\pgfsetroundjoin%
\definecolor{currentfill}{rgb}{0.000000,0.000000,0.000000}%
\pgfsetfillcolor{currentfill}%
\pgfsetlinewidth{0.803000pt}%
\definecolor{currentstroke}{rgb}{0.000000,0.000000,0.000000}%
\pgfsetstrokecolor{currentstroke}%
\pgfsetdash{}{0pt}%
\pgfsys@defobject{currentmarker}{\pgfqpoint{0.000000in}{-0.048611in}}{\pgfqpoint{0.000000in}{0.000000in}}{%
\pgfpathmoveto{\pgfqpoint{0.000000in}{0.000000in}}%
\pgfpathlineto{\pgfqpoint{0.000000in}{-0.048611in}}%
\pgfusepath{stroke,fill}%
}%
\begin{pgfscope}%
\pgfsys@transformshift{2.785902in}{0.499444in}%
\pgfsys@useobject{currentmarker}{}%
\end{pgfscope}%
\end{pgfscope}%
\begin{pgfscope}%
\definecolor{textcolor}{rgb}{0.000000,0.000000,0.000000}%
\pgfsetstrokecolor{textcolor}%
\pgfsetfillcolor{textcolor}%
\pgftext[x=2.785902in,y=0.402222in,,top]{\color{textcolor}\rmfamily\fontsize{10.000000}{12.000000}\selectfont 0.6}%
\end{pgfscope}%
\begin{pgfscope}%
\pgfsetbuttcap%
\pgfsetroundjoin%
\definecolor{currentfill}{rgb}{0.000000,0.000000,0.000000}%
\pgfsetfillcolor{currentfill}%
\pgfsetlinewidth{0.803000pt}%
\definecolor{currentstroke}{rgb}{0.000000,0.000000,0.000000}%
\pgfsetstrokecolor{currentstroke}%
\pgfsetdash{}{0pt}%
\pgfsys@defobject{currentmarker}{\pgfqpoint{0.000000in}{-0.048611in}}{\pgfqpoint{0.000000in}{0.000000in}}{%
\pgfpathmoveto{\pgfqpoint{0.000000in}{0.000000in}}%
\pgfpathlineto{\pgfqpoint{0.000000in}{-0.048611in}}%
\pgfusepath{stroke,fill}%
}%
\begin{pgfscope}%
\pgfsys@transformshift{3.169566in}{0.499444in}%
\pgfsys@useobject{currentmarker}{}%
\end{pgfscope}%
\end{pgfscope}%
\begin{pgfscope}%
\definecolor{textcolor}{rgb}{0.000000,0.000000,0.000000}%
\pgfsetstrokecolor{textcolor}%
\pgfsetfillcolor{textcolor}%
\pgftext[x=3.169566in,y=0.402222in,,top]{\color{textcolor}\rmfamily\fontsize{10.000000}{12.000000}\selectfont 0.7}%
\end{pgfscope}%
\begin{pgfscope}%
\pgfsetbuttcap%
\pgfsetroundjoin%
\definecolor{currentfill}{rgb}{0.000000,0.000000,0.000000}%
\pgfsetfillcolor{currentfill}%
\pgfsetlinewidth{0.803000pt}%
\definecolor{currentstroke}{rgb}{0.000000,0.000000,0.000000}%
\pgfsetstrokecolor{currentstroke}%
\pgfsetdash{}{0pt}%
\pgfsys@defobject{currentmarker}{\pgfqpoint{0.000000in}{-0.048611in}}{\pgfqpoint{0.000000in}{0.000000in}}{%
\pgfpathmoveto{\pgfqpoint{0.000000in}{0.000000in}}%
\pgfpathlineto{\pgfqpoint{0.000000in}{-0.048611in}}%
\pgfusepath{stroke,fill}%
}%
\begin{pgfscope}%
\pgfsys@transformshift{3.553229in}{0.499444in}%
\pgfsys@useobject{currentmarker}{}%
\end{pgfscope}%
\end{pgfscope}%
\begin{pgfscope}%
\definecolor{textcolor}{rgb}{0.000000,0.000000,0.000000}%
\pgfsetstrokecolor{textcolor}%
\pgfsetfillcolor{textcolor}%
\pgftext[x=3.553229in,y=0.402222in,,top]{\color{textcolor}\rmfamily\fontsize{10.000000}{12.000000}\selectfont 0.8}%
\end{pgfscope}%
\begin{pgfscope}%
\pgfsetbuttcap%
\pgfsetroundjoin%
\definecolor{currentfill}{rgb}{0.000000,0.000000,0.000000}%
\pgfsetfillcolor{currentfill}%
\pgfsetlinewidth{0.803000pt}%
\definecolor{currentstroke}{rgb}{0.000000,0.000000,0.000000}%
\pgfsetstrokecolor{currentstroke}%
\pgfsetdash{}{0pt}%
\pgfsys@defobject{currentmarker}{\pgfqpoint{0.000000in}{-0.048611in}}{\pgfqpoint{0.000000in}{0.000000in}}{%
\pgfpathmoveto{\pgfqpoint{0.000000in}{0.000000in}}%
\pgfpathlineto{\pgfqpoint{0.000000in}{-0.048611in}}%
\pgfusepath{stroke,fill}%
}%
\begin{pgfscope}%
\pgfsys@transformshift{3.936892in}{0.499444in}%
\pgfsys@useobject{currentmarker}{}%
\end{pgfscope}%
\end{pgfscope}%
\begin{pgfscope}%
\definecolor{textcolor}{rgb}{0.000000,0.000000,0.000000}%
\pgfsetstrokecolor{textcolor}%
\pgfsetfillcolor{textcolor}%
\pgftext[x=3.936892in,y=0.402222in,,top]{\color{textcolor}\rmfamily\fontsize{10.000000}{12.000000}\selectfont 0.9}%
\end{pgfscope}%
\begin{pgfscope}%
\pgfsetbuttcap%
\pgfsetroundjoin%
\definecolor{currentfill}{rgb}{0.000000,0.000000,0.000000}%
\pgfsetfillcolor{currentfill}%
\pgfsetlinewidth{0.803000pt}%
\definecolor{currentstroke}{rgb}{0.000000,0.000000,0.000000}%
\pgfsetstrokecolor{currentstroke}%
\pgfsetdash{}{0pt}%
\pgfsys@defobject{currentmarker}{\pgfqpoint{0.000000in}{-0.048611in}}{\pgfqpoint{0.000000in}{0.000000in}}{%
\pgfpathmoveto{\pgfqpoint{0.000000in}{0.000000in}}%
\pgfpathlineto{\pgfqpoint{0.000000in}{-0.048611in}}%
\pgfusepath{stroke,fill}%
}%
\begin{pgfscope}%
\pgfsys@transformshift{4.320556in}{0.499444in}%
\pgfsys@useobject{currentmarker}{}%
\end{pgfscope}%
\end{pgfscope}%
\begin{pgfscope}%
\definecolor{textcolor}{rgb}{0.000000,0.000000,0.000000}%
\pgfsetstrokecolor{textcolor}%
\pgfsetfillcolor{textcolor}%
\pgftext[x=4.320556in,y=0.402222in,,top]{\color{textcolor}\rmfamily\fontsize{10.000000}{12.000000}\selectfont 1.0}%
\end{pgfscope}%
\begin{pgfscope}%
\definecolor{textcolor}{rgb}{0.000000,0.000000,0.000000}%
\pgfsetstrokecolor{textcolor}%
\pgfsetfillcolor{textcolor}%
\pgftext[x=2.383056in,y=0.223333in,,top]{\color{textcolor}\rmfamily\fontsize{10.000000}{12.000000}\selectfont \(\displaystyle p\)}%
\end{pgfscope}%
\begin{pgfscope}%
\pgfsetbuttcap%
\pgfsetroundjoin%
\definecolor{currentfill}{rgb}{0.000000,0.000000,0.000000}%
\pgfsetfillcolor{currentfill}%
\pgfsetlinewidth{0.803000pt}%
\definecolor{currentstroke}{rgb}{0.000000,0.000000,0.000000}%
\pgfsetstrokecolor{currentstroke}%
\pgfsetdash{}{0pt}%
\pgfsys@defobject{currentmarker}{\pgfqpoint{-0.048611in}{0.000000in}}{\pgfqpoint{-0.000000in}{0.000000in}}{%
\pgfpathmoveto{\pgfqpoint{-0.000000in}{0.000000in}}%
\pgfpathlineto{\pgfqpoint{-0.048611in}{0.000000in}}%
\pgfusepath{stroke,fill}%
}%
\begin{pgfscope}%
\pgfsys@transformshift{0.445556in}{0.499444in}%
\pgfsys@useobject{currentmarker}{}%
\end{pgfscope}%
\end{pgfscope}%
\begin{pgfscope}%
\definecolor{textcolor}{rgb}{0.000000,0.000000,0.000000}%
\pgfsetstrokecolor{textcolor}%
\pgfsetfillcolor{textcolor}%
\pgftext[x=0.278889in, y=0.451250in, left, base]{\color{textcolor}\rmfamily\fontsize{10.000000}{12.000000}\selectfont \(\displaystyle {0}\)}%
\end{pgfscope}%
\begin{pgfscope}%
\pgfsetbuttcap%
\pgfsetroundjoin%
\definecolor{currentfill}{rgb}{0.000000,0.000000,0.000000}%
\pgfsetfillcolor{currentfill}%
\pgfsetlinewidth{0.803000pt}%
\definecolor{currentstroke}{rgb}{0.000000,0.000000,0.000000}%
\pgfsetstrokecolor{currentstroke}%
\pgfsetdash{}{0pt}%
\pgfsys@defobject{currentmarker}{\pgfqpoint{-0.048611in}{0.000000in}}{\pgfqpoint{-0.000000in}{0.000000in}}{%
\pgfpathmoveto{\pgfqpoint{-0.000000in}{0.000000in}}%
\pgfpathlineto{\pgfqpoint{-0.048611in}{0.000000in}}%
\pgfusepath{stroke,fill}%
}%
\begin{pgfscope}%
\pgfsys@transformshift{0.445556in}{0.828093in}%
\pgfsys@useobject{currentmarker}{}%
\end{pgfscope}%
\end{pgfscope}%
\begin{pgfscope}%
\definecolor{textcolor}{rgb}{0.000000,0.000000,0.000000}%
\pgfsetstrokecolor{textcolor}%
\pgfsetfillcolor{textcolor}%
\pgftext[x=0.278889in, y=0.779899in, left, base]{\color{textcolor}\rmfamily\fontsize{10.000000}{12.000000}\selectfont \(\displaystyle {2}\)}%
\end{pgfscope}%
\begin{pgfscope}%
\pgfsetbuttcap%
\pgfsetroundjoin%
\definecolor{currentfill}{rgb}{0.000000,0.000000,0.000000}%
\pgfsetfillcolor{currentfill}%
\pgfsetlinewidth{0.803000pt}%
\definecolor{currentstroke}{rgb}{0.000000,0.000000,0.000000}%
\pgfsetstrokecolor{currentstroke}%
\pgfsetdash{}{0pt}%
\pgfsys@defobject{currentmarker}{\pgfqpoint{-0.048611in}{0.000000in}}{\pgfqpoint{-0.000000in}{0.000000in}}{%
\pgfpathmoveto{\pgfqpoint{-0.000000in}{0.000000in}}%
\pgfpathlineto{\pgfqpoint{-0.048611in}{0.000000in}}%
\pgfusepath{stroke,fill}%
}%
\begin{pgfscope}%
\pgfsys@transformshift{0.445556in}{1.156742in}%
\pgfsys@useobject{currentmarker}{}%
\end{pgfscope}%
\end{pgfscope}%
\begin{pgfscope}%
\definecolor{textcolor}{rgb}{0.000000,0.000000,0.000000}%
\pgfsetstrokecolor{textcolor}%
\pgfsetfillcolor{textcolor}%
\pgftext[x=0.278889in, y=1.108548in, left, base]{\color{textcolor}\rmfamily\fontsize{10.000000}{12.000000}\selectfont \(\displaystyle {4}\)}%
\end{pgfscope}%
\begin{pgfscope}%
\pgfsetbuttcap%
\pgfsetroundjoin%
\definecolor{currentfill}{rgb}{0.000000,0.000000,0.000000}%
\pgfsetfillcolor{currentfill}%
\pgfsetlinewidth{0.803000pt}%
\definecolor{currentstroke}{rgb}{0.000000,0.000000,0.000000}%
\pgfsetstrokecolor{currentstroke}%
\pgfsetdash{}{0pt}%
\pgfsys@defobject{currentmarker}{\pgfqpoint{-0.048611in}{0.000000in}}{\pgfqpoint{-0.000000in}{0.000000in}}{%
\pgfpathmoveto{\pgfqpoint{-0.000000in}{0.000000in}}%
\pgfpathlineto{\pgfqpoint{-0.048611in}{0.000000in}}%
\pgfusepath{stroke,fill}%
}%
\begin{pgfscope}%
\pgfsys@transformshift{0.445556in}{1.485391in}%
\pgfsys@useobject{currentmarker}{}%
\end{pgfscope}%
\end{pgfscope}%
\begin{pgfscope}%
\definecolor{textcolor}{rgb}{0.000000,0.000000,0.000000}%
\pgfsetstrokecolor{textcolor}%
\pgfsetfillcolor{textcolor}%
\pgftext[x=0.278889in, y=1.437197in, left, base]{\color{textcolor}\rmfamily\fontsize{10.000000}{12.000000}\selectfont \(\displaystyle {6}\)}%
\end{pgfscope}%
\begin{pgfscope}%
\definecolor{textcolor}{rgb}{0.000000,0.000000,0.000000}%
\pgfsetstrokecolor{textcolor}%
\pgfsetfillcolor{textcolor}%
\pgftext[x=0.223333in,y=1.076944in,,bottom,rotate=90.000000]{\color{textcolor}\rmfamily\fontsize{10.000000}{12.000000}\selectfont Percent of Data Set}%
\end{pgfscope}%
\begin{pgfscope}%
\pgfsetrectcap%
\pgfsetmiterjoin%
\pgfsetlinewidth{0.803000pt}%
\definecolor{currentstroke}{rgb}{0.000000,0.000000,0.000000}%
\pgfsetstrokecolor{currentstroke}%
\pgfsetdash{}{0pt}%
\pgfpathmoveto{\pgfqpoint{0.445556in}{0.499444in}}%
\pgfpathlineto{\pgfqpoint{0.445556in}{1.654444in}}%
\pgfusepath{stroke}%
\end{pgfscope}%
\begin{pgfscope}%
\pgfsetrectcap%
\pgfsetmiterjoin%
\pgfsetlinewidth{0.803000pt}%
\definecolor{currentstroke}{rgb}{0.000000,0.000000,0.000000}%
\pgfsetstrokecolor{currentstroke}%
\pgfsetdash{}{0pt}%
\pgfpathmoveto{\pgfqpoint{4.320556in}{0.499444in}}%
\pgfpathlineto{\pgfqpoint{4.320556in}{1.654444in}}%
\pgfusepath{stroke}%
\end{pgfscope}%
\begin{pgfscope}%
\pgfsetrectcap%
\pgfsetmiterjoin%
\pgfsetlinewidth{0.803000pt}%
\definecolor{currentstroke}{rgb}{0.000000,0.000000,0.000000}%
\pgfsetstrokecolor{currentstroke}%
\pgfsetdash{}{0pt}%
\pgfpathmoveto{\pgfqpoint{0.445556in}{0.499444in}}%
\pgfpathlineto{\pgfqpoint{4.320556in}{0.499444in}}%
\pgfusepath{stroke}%
\end{pgfscope}%
\begin{pgfscope}%
\pgfsetrectcap%
\pgfsetmiterjoin%
\pgfsetlinewidth{0.803000pt}%
\definecolor{currentstroke}{rgb}{0.000000,0.000000,0.000000}%
\pgfsetstrokecolor{currentstroke}%
\pgfsetdash{}{0pt}%
\pgfpathmoveto{\pgfqpoint{0.445556in}{1.654444in}}%
\pgfpathlineto{\pgfqpoint{4.320556in}{1.654444in}}%
\pgfusepath{stroke}%
\end{pgfscope}%
\begin{pgfscope}%
\pgfsetbuttcap%
\pgfsetmiterjoin%
\definecolor{currentfill}{rgb}{1.000000,1.000000,1.000000}%
\pgfsetfillcolor{currentfill}%
\pgfsetfillopacity{0.800000}%
\pgfsetlinewidth{1.003750pt}%
\definecolor{currentstroke}{rgb}{0.800000,0.800000,0.800000}%
\pgfsetstrokecolor{currentstroke}%
\pgfsetstrokeopacity{0.800000}%
\pgfsetdash{}{0pt}%
\pgfpathmoveto{\pgfqpoint{3.543611in}{1.154445in}}%
\pgfpathlineto{\pgfqpoint{4.223333in}{1.154445in}}%
\pgfpathquadraticcurveto{\pgfqpoint{4.251111in}{1.154445in}}{\pgfqpoint{4.251111in}{1.182222in}}%
\pgfpathlineto{\pgfqpoint{4.251111in}{1.557222in}}%
\pgfpathquadraticcurveto{\pgfqpoint{4.251111in}{1.585000in}}{\pgfqpoint{4.223333in}{1.585000in}}%
\pgfpathlineto{\pgfqpoint{3.543611in}{1.585000in}}%
\pgfpathquadraticcurveto{\pgfqpoint{3.515833in}{1.585000in}}{\pgfqpoint{3.515833in}{1.557222in}}%
\pgfpathlineto{\pgfqpoint{3.515833in}{1.182222in}}%
\pgfpathquadraticcurveto{\pgfqpoint{3.515833in}{1.154445in}}{\pgfqpoint{3.543611in}{1.154445in}}%
\pgfpathlineto{\pgfqpoint{3.543611in}{1.154445in}}%
\pgfpathclose%
\pgfusepath{stroke,fill}%
\end{pgfscope}%
\begin{pgfscope}%
\pgfsetbuttcap%
\pgfsetmiterjoin%
\pgfsetlinewidth{1.003750pt}%
\definecolor{currentstroke}{rgb}{0.000000,0.000000,0.000000}%
\pgfsetstrokecolor{currentstroke}%
\pgfsetdash{}{0pt}%
\pgfpathmoveto{\pgfqpoint{3.571389in}{1.432222in}}%
\pgfpathlineto{\pgfqpoint{3.849167in}{1.432222in}}%
\pgfpathlineto{\pgfqpoint{3.849167in}{1.529444in}}%
\pgfpathlineto{\pgfqpoint{3.571389in}{1.529444in}}%
\pgfpathlineto{\pgfqpoint{3.571389in}{1.432222in}}%
\pgfpathclose%
\pgfusepath{stroke}%
\end{pgfscope}%
\begin{pgfscope}%
\definecolor{textcolor}{rgb}{0.000000,0.000000,0.000000}%
\pgfsetstrokecolor{textcolor}%
\pgfsetfillcolor{textcolor}%
\pgftext[x=3.960278in,y=1.432222in,left,base]{\color{textcolor}\rmfamily\fontsize{10.000000}{12.000000}\selectfont Neg}%
\end{pgfscope}%
\begin{pgfscope}%
\pgfsetbuttcap%
\pgfsetmiterjoin%
\definecolor{currentfill}{rgb}{0.000000,0.000000,0.000000}%
\pgfsetfillcolor{currentfill}%
\pgfsetlinewidth{0.000000pt}%
\definecolor{currentstroke}{rgb}{0.000000,0.000000,0.000000}%
\pgfsetstrokecolor{currentstroke}%
\pgfsetstrokeopacity{0.000000}%
\pgfsetdash{}{0pt}%
\pgfpathmoveto{\pgfqpoint{3.571389in}{1.236944in}}%
\pgfpathlineto{\pgfqpoint{3.849167in}{1.236944in}}%
\pgfpathlineto{\pgfqpoint{3.849167in}{1.334167in}}%
\pgfpathlineto{\pgfqpoint{3.571389in}{1.334167in}}%
\pgfpathlineto{\pgfqpoint{3.571389in}{1.236944in}}%
\pgfpathclose%
\pgfusepath{fill}%
\end{pgfscope}%
\begin{pgfscope}%
\definecolor{textcolor}{rgb}{0.000000,0.000000,0.000000}%
\pgfsetstrokecolor{textcolor}%
\pgfsetfillcolor{textcolor}%
\pgftext[x=3.960278in,y=1.236944in,left,base]{\color{textcolor}\rmfamily\fontsize{10.000000}{12.000000}\selectfont Pos}%
\end{pgfscope}%
\end{pgfpicture}%
\makeatother%
\endgroup%

&
	\vskip 0pt
	\qquad \qquad ROC Curve
	
	%% Creator: Matplotlib, PGF backend
%%
%% To include the figure in your LaTeX document, write
%%   \input{<filename>.pgf}
%%
%% Make sure the required packages are loaded in your preamble
%%   \usepackage{pgf}
%%
%% Also ensure that all the required font packages are loaded; for instance,
%% the lmodern package is sometimes necessary when using math font.
%%   \usepackage{lmodern}
%%
%% Figures using additional raster images can only be included by \input if
%% they are in the same directory as the main LaTeX file. For loading figures
%% from other directories you can use the `import` package
%%   \usepackage{import}
%%
%% and then include the figures with
%%   \import{<path to file>}{<filename>.pgf}
%%
%% Matplotlib used the following preamble
%%   
%%   \usepackage{fontspec}
%%   \makeatletter\@ifpackageloaded{underscore}{}{\usepackage[strings]{underscore}}\makeatother
%%
\begingroup%
\makeatletter%
\begin{pgfpicture}%
\pgfpathrectangle{\pgfpointorigin}{\pgfqpoint{2.221861in}{1.754444in}}%
\pgfusepath{use as bounding box, clip}%
\begin{pgfscope}%
\pgfsetbuttcap%
\pgfsetmiterjoin%
\definecolor{currentfill}{rgb}{1.000000,1.000000,1.000000}%
\pgfsetfillcolor{currentfill}%
\pgfsetlinewidth{0.000000pt}%
\definecolor{currentstroke}{rgb}{1.000000,1.000000,1.000000}%
\pgfsetstrokecolor{currentstroke}%
\pgfsetdash{}{0pt}%
\pgfpathmoveto{\pgfqpoint{0.000000in}{0.000000in}}%
\pgfpathlineto{\pgfqpoint{2.221861in}{0.000000in}}%
\pgfpathlineto{\pgfqpoint{2.221861in}{1.754444in}}%
\pgfpathlineto{\pgfqpoint{0.000000in}{1.754444in}}%
\pgfpathlineto{\pgfqpoint{0.000000in}{0.000000in}}%
\pgfpathclose%
\pgfusepath{fill}%
\end{pgfscope}%
\begin{pgfscope}%
\pgfsetbuttcap%
\pgfsetmiterjoin%
\definecolor{currentfill}{rgb}{1.000000,1.000000,1.000000}%
\pgfsetfillcolor{currentfill}%
\pgfsetlinewidth{0.000000pt}%
\definecolor{currentstroke}{rgb}{0.000000,0.000000,0.000000}%
\pgfsetstrokecolor{currentstroke}%
\pgfsetstrokeopacity{0.000000}%
\pgfsetdash{}{0pt}%
\pgfpathmoveto{\pgfqpoint{0.553581in}{0.499444in}}%
\pgfpathlineto{\pgfqpoint{2.103581in}{0.499444in}}%
\pgfpathlineto{\pgfqpoint{2.103581in}{1.654444in}}%
\pgfpathlineto{\pgfqpoint{0.553581in}{1.654444in}}%
\pgfpathlineto{\pgfqpoint{0.553581in}{0.499444in}}%
\pgfpathclose%
\pgfusepath{fill}%
\end{pgfscope}%
\begin{pgfscope}%
\pgfsetbuttcap%
\pgfsetroundjoin%
\definecolor{currentfill}{rgb}{0.000000,0.000000,0.000000}%
\pgfsetfillcolor{currentfill}%
\pgfsetlinewidth{0.803000pt}%
\definecolor{currentstroke}{rgb}{0.000000,0.000000,0.000000}%
\pgfsetstrokecolor{currentstroke}%
\pgfsetdash{}{0pt}%
\pgfsys@defobject{currentmarker}{\pgfqpoint{0.000000in}{-0.048611in}}{\pgfqpoint{0.000000in}{0.000000in}}{%
\pgfpathmoveto{\pgfqpoint{0.000000in}{0.000000in}}%
\pgfpathlineto{\pgfqpoint{0.000000in}{-0.048611in}}%
\pgfusepath{stroke,fill}%
}%
\begin{pgfscope}%
\pgfsys@transformshift{0.624035in}{0.499444in}%
\pgfsys@useobject{currentmarker}{}%
\end{pgfscope}%
\end{pgfscope}%
\begin{pgfscope}%
\definecolor{textcolor}{rgb}{0.000000,0.000000,0.000000}%
\pgfsetstrokecolor{textcolor}%
\pgfsetfillcolor{textcolor}%
\pgftext[x=0.624035in,y=0.402222in,,top]{\color{textcolor}\rmfamily\fontsize{10.000000}{12.000000}\selectfont \(\displaystyle {0.0}\)}%
\end{pgfscope}%
\begin{pgfscope}%
\pgfsetbuttcap%
\pgfsetroundjoin%
\definecolor{currentfill}{rgb}{0.000000,0.000000,0.000000}%
\pgfsetfillcolor{currentfill}%
\pgfsetlinewidth{0.803000pt}%
\definecolor{currentstroke}{rgb}{0.000000,0.000000,0.000000}%
\pgfsetstrokecolor{currentstroke}%
\pgfsetdash{}{0pt}%
\pgfsys@defobject{currentmarker}{\pgfqpoint{0.000000in}{-0.048611in}}{\pgfqpoint{0.000000in}{0.000000in}}{%
\pgfpathmoveto{\pgfqpoint{0.000000in}{0.000000in}}%
\pgfpathlineto{\pgfqpoint{0.000000in}{-0.048611in}}%
\pgfusepath{stroke,fill}%
}%
\begin{pgfscope}%
\pgfsys@transformshift{1.328581in}{0.499444in}%
\pgfsys@useobject{currentmarker}{}%
\end{pgfscope}%
\end{pgfscope}%
\begin{pgfscope}%
\definecolor{textcolor}{rgb}{0.000000,0.000000,0.000000}%
\pgfsetstrokecolor{textcolor}%
\pgfsetfillcolor{textcolor}%
\pgftext[x=1.328581in,y=0.402222in,,top]{\color{textcolor}\rmfamily\fontsize{10.000000}{12.000000}\selectfont \(\displaystyle {0.5}\)}%
\end{pgfscope}%
\begin{pgfscope}%
\pgfsetbuttcap%
\pgfsetroundjoin%
\definecolor{currentfill}{rgb}{0.000000,0.000000,0.000000}%
\pgfsetfillcolor{currentfill}%
\pgfsetlinewidth{0.803000pt}%
\definecolor{currentstroke}{rgb}{0.000000,0.000000,0.000000}%
\pgfsetstrokecolor{currentstroke}%
\pgfsetdash{}{0pt}%
\pgfsys@defobject{currentmarker}{\pgfqpoint{0.000000in}{-0.048611in}}{\pgfqpoint{0.000000in}{0.000000in}}{%
\pgfpathmoveto{\pgfqpoint{0.000000in}{0.000000in}}%
\pgfpathlineto{\pgfqpoint{0.000000in}{-0.048611in}}%
\pgfusepath{stroke,fill}%
}%
\begin{pgfscope}%
\pgfsys@transformshift{2.033126in}{0.499444in}%
\pgfsys@useobject{currentmarker}{}%
\end{pgfscope}%
\end{pgfscope}%
\begin{pgfscope}%
\definecolor{textcolor}{rgb}{0.000000,0.000000,0.000000}%
\pgfsetstrokecolor{textcolor}%
\pgfsetfillcolor{textcolor}%
\pgftext[x=2.033126in,y=0.402222in,,top]{\color{textcolor}\rmfamily\fontsize{10.000000}{12.000000}\selectfont \(\displaystyle {1.0}\)}%
\end{pgfscope}%
\begin{pgfscope}%
\definecolor{textcolor}{rgb}{0.000000,0.000000,0.000000}%
\pgfsetstrokecolor{textcolor}%
\pgfsetfillcolor{textcolor}%
\pgftext[x=1.328581in,y=0.223333in,,top]{\color{textcolor}\rmfamily\fontsize{10.000000}{12.000000}\selectfont False positive rate}%
\end{pgfscope}%
\begin{pgfscope}%
\pgfsetbuttcap%
\pgfsetroundjoin%
\definecolor{currentfill}{rgb}{0.000000,0.000000,0.000000}%
\pgfsetfillcolor{currentfill}%
\pgfsetlinewidth{0.803000pt}%
\definecolor{currentstroke}{rgb}{0.000000,0.000000,0.000000}%
\pgfsetstrokecolor{currentstroke}%
\pgfsetdash{}{0pt}%
\pgfsys@defobject{currentmarker}{\pgfqpoint{-0.048611in}{0.000000in}}{\pgfqpoint{-0.000000in}{0.000000in}}{%
\pgfpathmoveto{\pgfqpoint{-0.000000in}{0.000000in}}%
\pgfpathlineto{\pgfqpoint{-0.048611in}{0.000000in}}%
\pgfusepath{stroke,fill}%
}%
\begin{pgfscope}%
\pgfsys@transformshift{0.553581in}{0.551944in}%
\pgfsys@useobject{currentmarker}{}%
\end{pgfscope}%
\end{pgfscope}%
\begin{pgfscope}%
\definecolor{textcolor}{rgb}{0.000000,0.000000,0.000000}%
\pgfsetstrokecolor{textcolor}%
\pgfsetfillcolor{textcolor}%
\pgftext[x=0.278889in, y=0.503750in, left, base]{\color{textcolor}\rmfamily\fontsize{10.000000}{12.000000}\selectfont \(\displaystyle {0.0}\)}%
\end{pgfscope}%
\begin{pgfscope}%
\pgfsetbuttcap%
\pgfsetroundjoin%
\definecolor{currentfill}{rgb}{0.000000,0.000000,0.000000}%
\pgfsetfillcolor{currentfill}%
\pgfsetlinewidth{0.803000pt}%
\definecolor{currentstroke}{rgb}{0.000000,0.000000,0.000000}%
\pgfsetstrokecolor{currentstroke}%
\pgfsetdash{}{0pt}%
\pgfsys@defobject{currentmarker}{\pgfqpoint{-0.048611in}{0.000000in}}{\pgfqpoint{-0.000000in}{0.000000in}}{%
\pgfpathmoveto{\pgfqpoint{-0.000000in}{0.000000in}}%
\pgfpathlineto{\pgfqpoint{-0.048611in}{0.000000in}}%
\pgfusepath{stroke,fill}%
}%
\begin{pgfscope}%
\pgfsys@transformshift{0.553581in}{1.076944in}%
\pgfsys@useobject{currentmarker}{}%
\end{pgfscope}%
\end{pgfscope}%
\begin{pgfscope}%
\definecolor{textcolor}{rgb}{0.000000,0.000000,0.000000}%
\pgfsetstrokecolor{textcolor}%
\pgfsetfillcolor{textcolor}%
\pgftext[x=0.278889in, y=1.028750in, left, base]{\color{textcolor}\rmfamily\fontsize{10.000000}{12.000000}\selectfont \(\displaystyle {0.5}\)}%
\end{pgfscope}%
\begin{pgfscope}%
\pgfsetbuttcap%
\pgfsetroundjoin%
\definecolor{currentfill}{rgb}{0.000000,0.000000,0.000000}%
\pgfsetfillcolor{currentfill}%
\pgfsetlinewidth{0.803000pt}%
\definecolor{currentstroke}{rgb}{0.000000,0.000000,0.000000}%
\pgfsetstrokecolor{currentstroke}%
\pgfsetdash{}{0pt}%
\pgfsys@defobject{currentmarker}{\pgfqpoint{-0.048611in}{0.000000in}}{\pgfqpoint{-0.000000in}{0.000000in}}{%
\pgfpathmoveto{\pgfqpoint{-0.000000in}{0.000000in}}%
\pgfpathlineto{\pgfqpoint{-0.048611in}{0.000000in}}%
\pgfusepath{stroke,fill}%
}%
\begin{pgfscope}%
\pgfsys@transformshift{0.553581in}{1.601944in}%
\pgfsys@useobject{currentmarker}{}%
\end{pgfscope}%
\end{pgfscope}%
\begin{pgfscope}%
\definecolor{textcolor}{rgb}{0.000000,0.000000,0.000000}%
\pgfsetstrokecolor{textcolor}%
\pgfsetfillcolor{textcolor}%
\pgftext[x=0.278889in, y=1.553750in, left, base]{\color{textcolor}\rmfamily\fontsize{10.000000}{12.000000}\selectfont \(\displaystyle {1.0}\)}%
\end{pgfscope}%
\begin{pgfscope}%
\definecolor{textcolor}{rgb}{0.000000,0.000000,0.000000}%
\pgfsetstrokecolor{textcolor}%
\pgfsetfillcolor{textcolor}%
\pgftext[x=0.223333in,y=1.076944in,,bottom,rotate=90.000000]{\color{textcolor}\rmfamily\fontsize{10.000000}{12.000000}\selectfont True positive rate}%
\end{pgfscope}%
\begin{pgfscope}%
\pgfpathrectangle{\pgfqpoint{0.553581in}{0.499444in}}{\pgfqpoint{1.550000in}{1.155000in}}%
\pgfusepath{clip}%
\pgfsetbuttcap%
\pgfsetroundjoin%
\pgfsetlinewidth{1.505625pt}%
\definecolor{currentstroke}{rgb}{0.000000,0.000000,0.000000}%
\pgfsetstrokecolor{currentstroke}%
\pgfsetdash{{5.550000pt}{2.400000pt}}{0.000000pt}%
\pgfpathmoveto{\pgfqpoint{0.624035in}{0.551944in}}%
\pgfpathlineto{\pgfqpoint{2.033126in}{1.601944in}}%
\pgfusepath{stroke}%
\end{pgfscope}%
\begin{pgfscope}%
\pgfpathrectangle{\pgfqpoint{0.553581in}{0.499444in}}{\pgfqpoint{1.550000in}{1.155000in}}%
\pgfusepath{clip}%
\pgfsetrectcap%
\pgfsetroundjoin%
\pgfsetlinewidth{1.505625pt}%
\definecolor{currentstroke}{rgb}{0.000000,0.000000,0.000000}%
\pgfsetstrokecolor{currentstroke}%
\pgfsetdash{}{0pt}%
\pgfpathmoveto{\pgfqpoint{0.624035in}{0.551944in}}%
\pgfpathlineto{\pgfqpoint{0.625145in}{0.569669in}}%
\pgfpathlineto{\pgfqpoint{0.625239in}{0.570663in}}%
\pgfpathlineto{\pgfqpoint{0.626334in}{0.584445in}}%
\pgfpathlineto{\pgfqpoint{0.626482in}{0.585470in}}%
\pgfpathlineto{\pgfqpoint{0.627592in}{0.597887in}}%
\pgfpathlineto{\pgfqpoint{0.627733in}{0.598911in}}%
\pgfpathlineto{\pgfqpoint{0.628843in}{0.608534in}}%
\pgfpathlineto{\pgfqpoint{0.629007in}{0.609558in}}%
\pgfpathlineto{\pgfqpoint{0.630110in}{0.618716in}}%
\pgfpathlineto{\pgfqpoint{0.630321in}{0.619740in}}%
\pgfpathlineto{\pgfqpoint{0.631431in}{0.628773in}}%
\pgfpathlineto{\pgfqpoint{0.631618in}{0.629829in}}%
\pgfpathlineto{\pgfqpoint{0.632728in}{0.635944in}}%
\pgfpathlineto{\pgfqpoint{0.632908in}{0.637031in}}%
\pgfpathlineto{\pgfqpoint{0.634018in}{0.645940in}}%
\pgfpathlineto{\pgfqpoint{0.634292in}{0.646964in}}%
\pgfpathlineto{\pgfqpoint{0.635363in}{0.654135in}}%
\pgfpathlineto{\pgfqpoint{0.635590in}{0.655190in}}%
\pgfpathlineto{\pgfqpoint{0.636700in}{0.660964in}}%
\pgfpathlineto{\pgfqpoint{0.636841in}{0.661989in}}%
\pgfpathlineto{\pgfqpoint{0.637943in}{0.667793in}}%
\pgfpathlineto{\pgfqpoint{0.638123in}{0.668694in}}%
\pgfpathlineto{\pgfqpoint{0.639233in}{0.675088in}}%
\pgfpathlineto{\pgfqpoint{0.639491in}{0.675989in}}%
\pgfpathlineto{\pgfqpoint{0.640593in}{0.681142in}}%
\pgfpathlineto{\pgfqpoint{0.640867in}{0.682166in}}%
\pgfpathlineto{\pgfqpoint{0.640867in}{0.682197in}}%
\pgfpathlineto{\pgfqpoint{0.641969in}{0.688064in}}%
\pgfpathlineto{\pgfqpoint{0.642297in}{0.689088in}}%
\pgfpathlineto{\pgfqpoint{0.643407in}{0.694614in}}%
\pgfpathlineto{\pgfqpoint{0.643626in}{0.695669in}}%
\pgfpathlineto{\pgfqpoint{0.644705in}{0.701412in}}%
\pgfpathlineto{\pgfqpoint{0.644721in}{0.701412in}}%
\pgfpathlineto{\pgfqpoint{0.644893in}{0.702436in}}%
\pgfpathlineto{\pgfqpoint{0.646003in}{0.707589in}}%
\pgfpathlineto{\pgfqpoint{0.646284in}{0.708645in}}%
\pgfpathlineto{\pgfqpoint{0.647394in}{0.713829in}}%
\pgfpathlineto{\pgfqpoint{0.647668in}{0.714853in}}%
\pgfpathlineto{\pgfqpoint{0.648778in}{0.719541in}}%
\pgfpathlineto{\pgfqpoint{0.648997in}{0.720565in}}%
\pgfpathlineto{\pgfqpoint{0.650107in}{0.724725in}}%
\pgfpathlineto{\pgfqpoint{0.650623in}{0.725811in}}%
\pgfpathlineto{\pgfqpoint{0.651718in}{0.730126in}}%
\pgfpathlineto{\pgfqpoint{0.652015in}{0.731212in}}%
\pgfpathlineto{\pgfqpoint{0.653109in}{0.736459in}}%
\pgfpathlineto{\pgfqpoint{0.653531in}{0.737545in}}%
\pgfpathlineto{\pgfqpoint{0.654633in}{0.741860in}}%
\pgfpathlineto{\pgfqpoint{0.654931in}{0.742729in}}%
\pgfpathlineto{\pgfqpoint{0.656009in}{0.746671in}}%
\pgfpathlineto{\pgfqpoint{0.656564in}{0.747758in}}%
\pgfpathlineto{\pgfqpoint{0.657659in}{0.750924in}}%
\pgfpathlineto{\pgfqpoint{0.658042in}{0.751949in}}%
\pgfpathlineto{\pgfqpoint{0.659136in}{0.755332in}}%
\pgfpathlineto{\pgfqpoint{0.659488in}{0.756419in}}%
\pgfpathlineto{\pgfqpoint{0.660583in}{0.759802in}}%
\pgfpathlineto{\pgfqpoint{0.660895in}{0.760827in}}%
\pgfpathlineto{\pgfqpoint{0.661990in}{0.764459in}}%
\pgfpathlineto{\pgfqpoint{0.662318in}{0.765483in}}%
\pgfpathlineto{\pgfqpoint{0.663428in}{0.768649in}}%
\pgfpathlineto{\pgfqpoint{0.663718in}{0.769612in}}%
\pgfpathlineto{\pgfqpoint{0.664828in}{0.772933in}}%
\pgfpathlineto{\pgfqpoint{0.665156in}{0.773989in}}%
\pgfpathlineto{\pgfqpoint{0.666266in}{0.777434in}}%
\pgfpathlineto{\pgfqpoint{0.666727in}{0.778459in}}%
\pgfpathlineto{\pgfqpoint{0.667837in}{0.782494in}}%
\pgfpathlineto{\pgfqpoint{0.668095in}{0.783549in}}%
\pgfpathlineto{\pgfqpoint{0.669206in}{0.786623in}}%
\pgfpathlineto{\pgfqpoint{0.669503in}{0.787647in}}%
\pgfpathlineto{\pgfqpoint{0.670589in}{0.790813in}}%
\pgfpathlineto{\pgfqpoint{0.671160in}{0.791900in}}%
\pgfpathlineto{\pgfqpoint{0.672262in}{0.795873in}}%
\pgfpathlineto{\pgfqpoint{0.672716in}{0.796960in}}%
\pgfpathlineto{\pgfqpoint{0.673787in}{0.800498in}}%
\pgfpathlineto{\pgfqpoint{0.673818in}{0.800498in}}%
\pgfpathlineto{\pgfqpoint{0.674045in}{0.801554in}}%
\pgfpathlineto{\pgfqpoint{0.675139in}{0.804627in}}%
\pgfpathlineto{\pgfqpoint{0.675765in}{0.805714in}}%
\pgfpathlineto{\pgfqpoint{0.676851in}{0.809066in}}%
\pgfpathlineto{\pgfqpoint{0.677289in}{0.810122in}}%
\pgfpathlineto{\pgfqpoint{0.678391in}{0.813722in}}%
\pgfpathlineto{\pgfqpoint{0.678939in}{0.814809in}}%
\pgfpathlineto{\pgfqpoint{0.680049in}{0.817696in}}%
\pgfpathlineto{\pgfqpoint{0.680338in}{0.818720in}}%
\pgfpathlineto{\pgfqpoint{0.681425in}{0.821669in}}%
\pgfpathlineto{\pgfqpoint{0.681706in}{0.822725in}}%
\pgfpathlineto{\pgfqpoint{0.682816in}{0.825518in}}%
\pgfpathlineto{\pgfqpoint{0.683520in}{0.826543in}}%
\pgfpathlineto{\pgfqpoint{0.684630in}{0.829057in}}%
\pgfpathlineto{\pgfqpoint{0.688640in}{0.840667in}}%
\pgfpathlineto{\pgfqpoint{0.689180in}{0.841753in}}%
\pgfpathlineto{\pgfqpoint{0.690282in}{0.844082in}}%
\pgfpathlineto{\pgfqpoint{0.690290in}{0.844082in}}%
\pgfpathlineto{\pgfqpoint{0.690939in}{0.845168in}}%
\pgfpathlineto{\pgfqpoint{0.692049in}{0.849110in}}%
\pgfpathlineto{\pgfqpoint{0.692729in}{0.850197in}}%
\pgfpathlineto{\pgfqpoint{0.693831in}{0.853053in}}%
\pgfpathlineto{\pgfqpoint{0.694355in}{0.854139in}}%
\pgfpathlineto{\pgfqpoint{0.695449in}{0.856250in}}%
\pgfpathlineto{\pgfqpoint{0.695879in}{0.857337in}}%
\pgfpathlineto{\pgfqpoint{0.696974in}{0.860161in}}%
\pgfpathlineto{\pgfqpoint{0.697427in}{0.861217in}}%
\pgfpathlineto{\pgfqpoint{0.698530in}{0.863514in}}%
\pgfpathlineto{\pgfqpoint{0.698920in}{0.864600in}}%
\pgfpathlineto{\pgfqpoint{0.700015in}{0.867270in}}%
\pgfpathlineto{\pgfqpoint{0.700468in}{0.868326in}}%
\pgfpathlineto{\pgfqpoint{0.701578in}{0.871275in}}%
\pgfpathlineto{\pgfqpoint{0.702024in}{0.872361in}}%
\pgfpathlineto{\pgfqpoint{0.703134in}{0.875434in}}%
\pgfpathlineto{\pgfqpoint{0.703548in}{0.876490in}}%
\pgfpathlineto{\pgfqpoint{0.704651in}{0.879439in}}%
\pgfpathlineto{\pgfqpoint{0.705042in}{0.880525in}}%
\pgfpathlineto{\pgfqpoint{0.706136in}{0.883412in}}%
\pgfpathlineto{\pgfqpoint{0.706629in}{0.884498in}}%
\pgfpathlineto{\pgfqpoint{0.707739in}{0.886671in}}%
\pgfpathlineto{\pgfqpoint{0.708106in}{0.887758in}}%
\pgfpathlineto{\pgfqpoint{0.709216in}{0.890490in}}%
\pgfpathlineto{\pgfqpoint{0.709857in}{0.891576in}}%
\pgfpathlineto{\pgfqpoint{0.710936in}{0.893811in}}%
\pgfpathlineto{\pgfqpoint{0.711311in}{0.894867in}}%
\pgfpathlineto{\pgfqpoint{0.712422in}{0.898002in}}%
\pgfpathlineto{\pgfqpoint{0.713078in}{0.899088in}}%
\pgfpathlineto{\pgfqpoint{0.714188in}{0.901168in}}%
\pgfpathlineto{\pgfqpoint{0.714681in}{0.902255in}}%
\pgfpathlineto{\pgfqpoint{0.715760in}{0.904800in}}%
\pgfpathlineto{\pgfqpoint{0.716315in}{0.905793in}}%
\pgfpathlineto{\pgfqpoint{0.717394in}{0.908463in}}%
\pgfpathlineto{\pgfqpoint{0.718105in}{0.909425in}}%
\pgfpathlineto{\pgfqpoint{0.719215in}{0.911567in}}%
\pgfpathlineto{\pgfqpoint{0.719770in}{0.912654in}}%
\pgfpathlineto{\pgfqpoint{0.720865in}{0.914765in}}%
\pgfpathlineto{\pgfqpoint{0.721451in}{0.915820in}}%
\pgfpathlineto{\pgfqpoint{0.722561in}{0.918024in}}%
\pgfpathlineto{\pgfqpoint{0.723218in}{0.919048in}}%
\pgfpathlineto{\pgfqpoint{0.724320in}{0.921345in}}%
\pgfpathlineto{\pgfqpoint{0.725125in}{0.922432in}}%
\pgfpathlineto{\pgfqpoint{0.726212in}{0.924481in}}%
\pgfpathlineto{\pgfqpoint{0.726806in}{0.925567in}}%
\pgfpathlineto{\pgfqpoint{0.727908in}{0.927647in}}%
\pgfpathlineto{\pgfqpoint{0.728885in}{0.928671in}}%
\pgfpathlineto{\pgfqpoint{0.729964in}{0.931031in}}%
\pgfpathlineto{\pgfqpoint{0.729996in}{0.931031in}}%
\pgfpathlineto{\pgfqpoint{0.730848in}{0.932086in}}%
\pgfpathlineto{\pgfqpoint{0.731841in}{0.933607in}}%
\pgfpathlineto{\pgfqpoint{0.732419in}{0.934632in}}%
\pgfpathlineto{\pgfqpoint{0.732419in}{0.934694in}}%
\pgfpathlineto{\pgfqpoint{0.733467in}{0.936556in}}%
\pgfpathlineto{\pgfqpoint{0.734186in}{0.937643in}}%
\pgfpathlineto{\pgfqpoint{0.735280in}{0.939722in}}%
\pgfpathlineto{\pgfqpoint{0.736109in}{0.940809in}}%
\pgfpathlineto{\pgfqpoint{0.737196in}{0.942951in}}%
\pgfpathlineto{\pgfqpoint{0.737672in}{0.944006in}}%
\pgfpathlineto{\pgfqpoint{0.738751in}{0.945962in}}%
\pgfpathlineto{\pgfqpoint{0.739322in}{0.947017in}}%
\pgfpathlineto{\pgfqpoint{0.740424in}{0.949159in}}%
\pgfpathlineto{\pgfqpoint{0.741018in}{0.950246in}}%
\pgfpathlineto{\pgfqpoint{0.742113in}{0.952077in}}%
\pgfpathlineto{\pgfqpoint{0.742613in}{0.953164in}}%
\pgfpathlineto{\pgfqpoint{0.743661in}{0.956175in}}%
\pgfpathlineto{\pgfqpoint{0.744661in}{0.957261in}}%
\pgfpathlineto{\pgfqpoint{0.745772in}{0.959310in}}%
\pgfpathlineto{\pgfqpoint{0.746272in}{0.960365in}}%
\pgfpathlineto{\pgfqpoint{0.747335in}{0.962073in}}%
\pgfpathlineto{\pgfqpoint{0.747921in}{0.963159in}}%
\pgfpathlineto{\pgfqpoint{0.749008in}{0.965239in}}%
\pgfpathlineto{\pgfqpoint{0.749571in}{0.966263in}}%
\pgfpathlineto{\pgfqpoint{0.750681in}{0.967971in}}%
\pgfpathlineto{\pgfqpoint{0.751447in}{0.969026in}}%
\pgfpathlineto{\pgfqpoint{0.752557in}{0.971541in}}%
\pgfpathlineto{\pgfqpoint{0.753245in}{0.972627in}}%
\pgfpathlineto{\pgfqpoint{0.754355in}{0.974428in}}%
\pgfpathlineto{\pgfqpoint{0.754926in}{0.975452in}}%
\pgfpathlineto{\pgfqpoint{0.754926in}{0.975514in}}%
\pgfpathlineto{\pgfqpoint{0.756841in}{0.978867in}}%
\pgfpathlineto{\pgfqpoint{0.757420in}{0.979922in}}%
\pgfpathlineto{\pgfqpoint{0.758514in}{0.981660in}}%
\pgfpathlineto{\pgfqpoint{0.759445in}{0.982716in}}%
\pgfpathlineto{\pgfqpoint{0.760531in}{0.984796in}}%
\pgfpathlineto{\pgfqpoint{0.760555in}{0.984796in}}%
\pgfpathlineto{\pgfqpoint{0.761071in}{0.985851in}}%
\pgfpathlineto{\pgfqpoint{0.762157in}{0.987434in}}%
\pgfpathlineto{\pgfqpoint{0.762923in}{0.988490in}}%
\pgfpathlineto{\pgfqpoint{0.764026in}{0.990104in}}%
\pgfpathlineto{\pgfqpoint{0.764886in}{0.991190in}}%
\pgfpathlineto{\pgfqpoint{0.765957in}{0.993425in}}%
\pgfpathlineto{\pgfqpoint{0.766535in}{0.994512in}}%
\pgfpathlineto{\pgfqpoint{0.767630in}{0.996343in}}%
\pgfpathlineto{\pgfqpoint{0.768591in}{0.997430in}}%
\pgfpathlineto{\pgfqpoint{0.769678in}{0.999727in}}%
\pgfpathlineto{\pgfqpoint{0.770561in}{1.000813in}}%
\pgfpathlineto{\pgfqpoint{0.771671in}{1.002583in}}%
\pgfpathlineto{\pgfqpoint{0.772477in}{1.003669in}}%
\pgfpathlineto{\pgfqpoint{0.773563in}{1.005035in}}%
\pgfpathlineto{\pgfqpoint{0.774118in}{1.006122in}}%
\pgfpathlineto{\pgfqpoint{0.775221in}{1.007674in}}%
\pgfpathlineto{\pgfqpoint{0.775901in}{1.008729in}}%
\pgfpathlineto{\pgfqpoint{0.777011in}{1.010778in}}%
\pgfpathlineto{\pgfqpoint{0.777730in}{1.011802in}}%
\pgfpathlineto{\pgfqpoint{0.778840in}{1.013820in}}%
\pgfpathlineto{\pgfqpoint{0.779942in}{1.014906in}}%
\pgfpathlineto{\pgfqpoint{0.781053in}{1.016334in}}%
\pgfpathlineto{\pgfqpoint{0.781952in}{1.017421in}}%
\pgfpathlineto{\pgfqpoint{0.783038in}{1.018973in}}%
\pgfpathlineto{\pgfqpoint{0.784000in}{1.020059in}}%
\pgfpathlineto{\pgfqpoint{0.785110in}{1.022232in}}%
\pgfpathlineto{\pgfqpoint{0.786064in}{1.023319in}}%
\pgfpathlineto{\pgfqpoint{0.787158in}{1.024902in}}%
\pgfpathlineto{\pgfqpoint{0.788057in}{1.025989in}}%
\pgfpathlineto{\pgfqpoint{0.789167in}{1.027913in}}%
\pgfpathlineto{\pgfqpoint{0.789832in}{1.029000in}}%
\pgfpathlineto{\pgfqpoint{0.790934in}{1.031173in}}%
\pgfpathlineto{\pgfqpoint{0.791692in}{1.032166in}}%
\pgfpathlineto{\pgfqpoint{0.792795in}{1.033873in}}%
\pgfpathlineto{\pgfqpoint{0.793874in}{1.034960in}}%
\pgfpathlineto{\pgfqpoint{0.794984in}{1.036388in}}%
\pgfpathlineto{\pgfqpoint{0.795969in}{1.037443in}}%
\pgfpathlineto{\pgfqpoint{0.797079in}{1.039275in}}%
\pgfpathlineto{\pgfqpoint{0.797868in}{1.040361in}}%
\pgfpathlineto{\pgfqpoint{0.798853in}{1.041541in}}%
\pgfpathlineto{\pgfqpoint{0.799807in}{1.042596in}}%
\pgfpathlineto{\pgfqpoint{0.800776in}{1.044117in}}%
\pgfpathlineto{\pgfqpoint{0.801668in}{1.045204in}}%
\pgfpathlineto{\pgfqpoint{0.802747in}{1.047283in}}%
\pgfpathlineto{\pgfqpoint{0.803606in}{1.048370in}}%
\pgfpathlineto{\pgfqpoint{0.804638in}{1.050108in}}%
\pgfpathlineto{\pgfqpoint{0.804701in}{1.050108in}}%
\pgfpathlineto{\pgfqpoint{0.805475in}{1.051164in}}%
\pgfpathlineto{\pgfqpoint{0.806522in}{1.052188in}}%
\pgfpathlineto{\pgfqpoint{0.807234in}{1.053244in}}%
\pgfpathlineto{\pgfqpoint{0.808266in}{1.054547in}}%
\pgfpathlineto{\pgfqpoint{0.809196in}{1.055634in}}%
\pgfpathlineto{\pgfqpoint{0.810306in}{1.057310in}}%
\pgfpathlineto{\pgfqpoint{0.811111in}{1.058396in}}%
\pgfpathlineto{\pgfqpoint{0.812214in}{1.059887in}}%
\pgfpathlineto{\pgfqpoint{0.813300in}{1.060942in}}%
\pgfpathlineto{\pgfqpoint{0.814410in}{1.062339in}}%
\pgfpathlineto{\pgfqpoint{0.815302in}{1.063394in}}%
\pgfpathlineto{\pgfqpoint{0.816388in}{1.065226in}}%
\pgfpathlineto{\pgfqpoint{0.817147in}{1.066312in}}%
\pgfpathlineto{\pgfqpoint{0.818233in}{1.067957in}}%
\pgfpathlineto{\pgfqpoint{0.819039in}{1.069013in}}%
\pgfpathlineto{\pgfqpoint{0.820117in}{1.070441in}}%
\pgfpathlineto{\pgfqpoint{0.820923in}{1.071496in}}%
\pgfpathlineto{\pgfqpoint{0.821978in}{1.073017in}}%
\pgfpathlineto{\pgfqpoint{0.823111in}{1.074011in}}%
\pgfpathlineto{\pgfqpoint{0.824198in}{1.075345in}}%
\pgfpathlineto{\pgfqpoint{0.825160in}{1.076401in}}%
\pgfpathlineto{\pgfqpoint{0.826207in}{1.077643in}}%
\pgfpathlineto{\pgfqpoint{0.826903in}{1.078729in}}%
\pgfpathlineto{\pgfqpoint{0.827998in}{1.079909in}}%
\pgfpathlineto{\pgfqpoint{0.828717in}{1.080995in}}%
\pgfpathlineto{\pgfqpoint{0.829819in}{1.082516in}}%
\pgfpathlineto{\pgfqpoint{0.830914in}{1.083603in}}%
\pgfpathlineto{\pgfqpoint{0.832016in}{1.084938in}}%
\pgfpathlineto{\pgfqpoint{0.832743in}{1.086024in}}%
\pgfpathlineto{\pgfqpoint{0.833822in}{1.087421in}}%
\pgfpathlineto{\pgfqpoint{0.834854in}{1.088507in}}%
\pgfpathlineto{\pgfqpoint{0.835932in}{1.089997in}}%
\pgfpathlineto{\pgfqpoint{0.836831in}{1.091084in}}%
\pgfpathlineto{\pgfqpoint{0.837942in}{1.092357in}}%
\pgfpathlineto{\pgfqpoint{0.838676in}{1.093443in}}%
\pgfpathlineto{\pgfqpoint{0.839755in}{1.095430in}}%
\pgfpathlineto{\pgfqpoint{0.839779in}{1.095430in}}%
\pgfpathlineto{\pgfqpoint{0.840365in}{1.096454in}}%
\pgfpathlineto{\pgfqpoint{0.840365in}{1.096485in}}%
\pgfpathlineto{\pgfqpoint{0.841475in}{1.098286in}}%
\pgfpathlineto{\pgfqpoint{0.842319in}{1.099341in}}%
\pgfpathlineto{\pgfqpoint{0.843406in}{1.100986in}}%
\pgfpathlineto{\pgfqpoint{0.844618in}{1.102073in}}%
\pgfpathlineto{\pgfqpoint{0.845728in}{1.103501in}}%
\pgfpathlineto{\pgfqpoint{0.846486in}{1.104587in}}%
\pgfpathlineto{\pgfqpoint{0.847589in}{1.105705in}}%
\pgfpathlineto{\pgfqpoint{0.848667in}{1.106791in}}%
\pgfpathlineto{\pgfqpoint{0.849770in}{1.108064in}}%
\pgfpathlineto{\pgfqpoint{0.850770in}{1.109150in}}%
\pgfpathlineto{\pgfqpoint{0.851880in}{1.110889in}}%
\pgfpathlineto{\pgfqpoint{0.853116in}{1.111975in}}%
\pgfpathlineto{\pgfqpoint{0.854148in}{1.112906in}}%
\pgfpathlineto{\pgfqpoint{0.854171in}{1.112906in}}%
\pgfpathlineto{\pgfqpoint{0.855508in}{1.113993in}}%
\pgfpathlineto{\pgfqpoint{0.856587in}{1.115545in}}%
\pgfpathlineto{\pgfqpoint{0.857650in}{1.116632in}}%
\pgfpathlineto{\pgfqpoint{0.858729in}{1.118028in}}%
\pgfpathlineto{\pgfqpoint{0.859894in}{1.119115in}}%
\pgfpathlineto{\pgfqpoint{0.860972in}{1.120543in}}%
\pgfpathlineto{\pgfqpoint{0.862278in}{1.121629in}}%
\pgfpathlineto{\pgfqpoint{0.863341in}{1.122654in}}%
\pgfpathlineto{\pgfqpoint{0.864545in}{1.123740in}}%
\pgfpathlineto{\pgfqpoint{0.865569in}{1.125230in}}%
\pgfpathlineto{\pgfqpoint{0.866523in}{1.126286in}}%
\pgfpathlineto{\pgfqpoint{0.867625in}{1.127900in}}%
\pgfpathlineto{\pgfqpoint{0.868430in}{1.128924in}}%
\pgfpathlineto{\pgfqpoint{0.869517in}{1.130073in}}%
\pgfpathlineto{\pgfqpoint{0.870408in}{1.131159in}}%
\pgfpathlineto{\pgfqpoint{0.871503in}{1.132649in}}%
\pgfpathlineto{\pgfqpoint{0.871511in}{1.132649in}}%
\pgfpathlineto{\pgfqpoint{0.872238in}{1.133705in}}%
\pgfpathlineto{\pgfqpoint{0.873340in}{1.135102in}}%
\pgfpathlineto{\pgfqpoint{0.874145in}{1.136188in}}%
\pgfpathlineto{\pgfqpoint{0.875177in}{1.137492in}}%
\pgfpathlineto{\pgfqpoint{0.876514in}{1.138578in}}%
\pgfpathlineto{\pgfqpoint{0.877546in}{1.140161in}}%
\pgfpathlineto{\pgfqpoint{0.878492in}{1.141248in}}%
\pgfpathlineto{\pgfqpoint{0.879594in}{1.142334in}}%
\pgfpathlineto{\pgfqpoint{0.880344in}{1.143390in}}%
\pgfpathlineto{\pgfqpoint{0.881415in}{1.144569in}}%
\pgfpathlineto{\pgfqpoint{0.882354in}{1.145656in}}%
\pgfpathlineto{\pgfqpoint{0.883464in}{1.147239in}}%
\pgfpathlineto{\pgfqpoint{0.884285in}{1.148326in}}%
\pgfpathlineto{\pgfqpoint{0.885379in}{1.149412in}}%
\pgfpathlineto{\pgfqpoint{0.886372in}{1.150467in}}%
\pgfpathlineto{\pgfqpoint{0.887466in}{1.151864in}}%
\pgfpathlineto{\pgfqpoint{0.888834in}{1.152951in}}%
\pgfpathlineto{\pgfqpoint{0.889898in}{1.154130in}}%
\pgfpathlineto{\pgfqpoint{0.891430in}{1.155217in}}%
\pgfpathlineto{\pgfqpoint{0.892532in}{1.156272in}}%
\pgfpathlineto{\pgfqpoint{0.893627in}{1.157359in}}%
\pgfpathlineto{\pgfqpoint{0.894729in}{1.158663in}}%
\pgfpathlineto{\pgfqpoint{0.895565in}{1.159749in}}%
\pgfpathlineto{\pgfqpoint{0.896652in}{1.160929in}}%
\pgfpathlineto{\pgfqpoint{0.897747in}{1.161984in}}%
\pgfpathlineto{\pgfqpoint{0.898833in}{1.163412in}}%
\pgfpathlineto{\pgfqpoint{0.900162in}{1.164467in}}%
\pgfpathlineto{\pgfqpoint{0.901272in}{1.165989in}}%
\pgfpathlineto{\pgfqpoint{0.902367in}{1.167075in}}%
\pgfpathlineto{\pgfqpoint{0.903477in}{1.168596in}}%
\pgfpathlineto{\pgfqpoint{0.904720in}{1.169651in}}%
\pgfpathlineto{\pgfqpoint{0.905611in}{1.170676in}}%
\pgfpathlineto{\pgfqpoint{0.905775in}{1.170676in}}%
\pgfpathlineto{\pgfqpoint{0.906956in}{1.171762in}}%
\pgfpathlineto{\pgfqpoint{0.908027in}{1.172818in}}%
\pgfpathlineto{\pgfqpoint{0.909137in}{1.173904in}}%
\pgfpathlineto{\pgfqpoint{0.910247in}{1.175425in}}%
\pgfpathlineto{\pgfqpoint{0.911107in}{1.176512in}}%
\pgfpathlineto{\pgfqpoint{0.912201in}{1.177598in}}%
\pgfpathlineto{\pgfqpoint{0.913319in}{1.178685in}}%
\pgfpathlineto{\pgfqpoint{0.914390in}{1.180020in}}%
\pgfpathlineto{\pgfqpoint{0.915250in}{1.181044in}}%
\pgfpathlineto{\pgfqpoint{0.916306in}{1.182037in}}%
\pgfpathlineto{\pgfqpoint{0.917416in}{1.183124in}}%
\pgfpathlineto{\pgfqpoint{0.918518in}{1.184428in}}%
\pgfpathlineto{\pgfqpoint{0.919964in}{1.185514in}}%
\pgfpathlineto{\pgfqpoint{0.921059in}{1.186538in}}%
\pgfpathlineto{\pgfqpoint{0.921872in}{1.187625in}}%
\pgfpathlineto{\pgfqpoint{0.922974in}{1.188742in}}%
\pgfpathlineto{\pgfqpoint{0.923811in}{1.189829in}}%
\pgfpathlineto{\pgfqpoint{0.924858in}{1.191102in}}%
\pgfpathlineto{\pgfqpoint{0.926078in}{1.192126in}}%
\pgfpathlineto{\pgfqpoint{0.927102in}{1.193244in}}%
\pgfpathlineto{\pgfqpoint{0.928579in}{1.194330in}}%
\pgfpathlineto{\pgfqpoint{0.929666in}{1.194982in}}%
\pgfpathlineto{\pgfqpoint{0.930886in}{1.196068in}}%
\pgfpathlineto{\pgfqpoint{0.931972in}{1.197465in}}%
\pgfpathlineto{\pgfqpoint{0.933153in}{1.198552in}}%
\pgfpathlineto{\pgfqpoint{0.934231in}{1.200166in}}%
\pgfpathlineto{\pgfqpoint{0.935451in}{1.201252in}}%
\pgfpathlineto{\pgfqpoint{0.936561in}{1.201997in}}%
\pgfpathlineto{\pgfqpoint{0.937734in}{1.203053in}}%
\pgfpathlineto{\pgfqpoint{0.938844in}{1.203922in}}%
\pgfpathlineto{\pgfqpoint{0.939985in}{1.205008in}}%
\pgfpathlineto{\pgfqpoint{0.941095in}{1.206126in}}%
\pgfpathlineto{\pgfqpoint{0.942557in}{1.207212in}}%
\pgfpathlineto{\pgfqpoint{0.943613in}{1.208268in}}%
\pgfpathlineto{\pgfqpoint{0.944871in}{1.209354in}}%
\pgfpathlineto{\pgfqpoint{0.945966in}{1.210472in}}%
\pgfpathlineto{\pgfqpoint{0.947326in}{1.211558in}}%
\pgfpathlineto{\pgfqpoint{0.948428in}{1.212521in}}%
\pgfpathlineto{\pgfqpoint{0.949757in}{1.213576in}}%
\pgfpathlineto{\pgfqpoint{0.950750in}{1.214600in}}%
\pgfpathlineto{\pgfqpoint{0.952056in}{1.215687in}}%
\pgfpathlineto{\pgfqpoint{0.953158in}{1.216370in}}%
\pgfpathlineto{\pgfqpoint{0.954362in}{1.217425in}}%
\pgfpathlineto{\pgfqpoint{0.955472in}{1.218450in}}%
\pgfpathlineto{\pgfqpoint{0.956746in}{1.219505in}}%
\pgfpathlineto{\pgfqpoint{0.957849in}{1.221057in}}%
\pgfpathlineto{\pgfqpoint{0.958865in}{1.222144in}}%
\pgfpathlineto{\pgfqpoint{0.959967in}{1.223230in}}%
\pgfpathlineto{\pgfqpoint{0.961304in}{1.224317in}}%
\pgfpathlineto{\pgfqpoint{0.962344in}{1.225714in}}%
\pgfpathlineto{\pgfqpoint{0.963681in}{1.226800in}}%
\pgfpathlineto{\pgfqpoint{0.964720in}{1.227669in}}%
\pgfpathlineto{\pgfqpoint{0.965916in}{1.228756in}}%
\pgfpathlineto{\pgfqpoint{0.966956in}{1.229997in}}%
\pgfpathlineto{\pgfqpoint{0.966987in}{1.229997in}}%
\pgfpathlineto{\pgfqpoint{0.968207in}{1.231084in}}%
\pgfpathlineto{\pgfqpoint{0.969317in}{1.231891in}}%
\pgfpathlineto{\pgfqpoint{0.970419in}{1.232977in}}%
\pgfpathlineto{\pgfqpoint{0.971397in}{1.233660in}}%
\pgfpathlineto{\pgfqpoint{0.972780in}{1.234747in}}%
\pgfpathlineto{\pgfqpoint{0.973890in}{1.235957in}}%
\pgfpathlineto{\pgfqpoint{0.975055in}{1.237044in}}%
\pgfpathlineto{\pgfqpoint{0.976087in}{1.238317in}}%
\pgfpathlineto{\pgfqpoint{0.976126in}{1.238317in}}%
\pgfpathlineto{\pgfqpoint{0.977823in}{1.239403in}}%
\pgfpathlineto{\pgfqpoint{0.978925in}{1.240272in}}%
\pgfpathlineto{\pgfqpoint{0.980262in}{1.241359in}}%
\pgfpathlineto{\pgfqpoint{0.981348in}{1.242663in}}%
\pgfpathlineto{\pgfqpoint{0.982607in}{1.243749in}}%
\pgfpathlineto{\pgfqpoint{0.983709in}{1.244649in}}%
\pgfpathlineto{\pgfqpoint{0.985578in}{1.245736in}}%
\pgfpathlineto{\pgfqpoint{0.986657in}{1.246729in}}%
\pgfpathlineto{\pgfqpoint{0.987978in}{1.247816in}}%
\pgfpathlineto{\pgfqpoint{0.989017in}{1.248840in}}%
\pgfpathlineto{\pgfqpoint{0.990112in}{1.249926in}}%
\pgfpathlineto{\pgfqpoint{0.991222in}{1.250889in}}%
\pgfpathlineto{\pgfqpoint{0.992715in}{1.251975in}}%
\pgfpathlineto{\pgfqpoint{0.993786in}{1.253310in}}%
\pgfpathlineto{\pgfqpoint{0.995139in}{1.254396in}}%
\pgfpathlineto{\pgfqpoint{0.996194in}{1.255514in}}%
\pgfpathlineto{\pgfqpoint{0.997578in}{1.256600in}}%
\pgfpathlineto{\pgfqpoint{0.998649in}{1.257470in}}%
\pgfpathlineto{\pgfqpoint{1.000181in}{1.258525in}}%
\pgfpathlineto{\pgfqpoint{1.001283in}{1.259456in}}%
\pgfpathlineto{\pgfqpoint{1.002737in}{1.260543in}}%
\pgfpathlineto{\pgfqpoint{1.003793in}{1.261505in}}%
\pgfpathlineto{\pgfqpoint{1.005184in}{1.262592in}}%
\pgfpathlineto{\pgfqpoint{1.006287in}{1.263492in}}%
\pgfpathlineto{\pgfqpoint{1.007983in}{1.264578in}}%
\pgfpathlineto{\pgfqpoint{1.009093in}{1.265354in}}%
\pgfpathlineto{\pgfqpoint{1.010266in}{1.266410in}}%
\pgfpathlineto{\pgfqpoint{1.011353in}{1.267558in}}%
\pgfpathlineto{\pgfqpoint{1.012682in}{1.268645in}}%
\pgfpathlineto{\pgfqpoint{1.013792in}{1.269700in}}%
\pgfpathlineto{\pgfqpoint{1.015003in}{1.270787in}}%
\pgfpathlineto{\pgfqpoint{1.016043in}{1.271966in}}%
\pgfpathlineto{\pgfqpoint{1.017255in}{1.273053in}}%
\pgfpathlineto{\pgfqpoint{1.018326in}{1.274201in}}%
\pgfpathlineto{\pgfqpoint{1.020038in}{1.275288in}}%
\pgfpathlineto{\pgfqpoint{1.021125in}{1.276095in}}%
\pgfpathlineto{\pgfqpoint{1.022563in}{1.277181in}}%
\pgfpathlineto{\pgfqpoint{1.023665in}{1.278268in}}%
\pgfpathlineto{\pgfqpoint{1.025002in}{1.279354in}}%
\pgfpathlineto{\pgfqpoint{1.026112in}{1.280130in}}%
\pgfpathlineto{\pgfqpoint{1.027902in}{1.281217in}}%
\pgfpathlineto{\pgfqpoint{1.028981in}{1.282210in}}%
\pgfpathlineto{\pgfqpoint{1.030435in}{1.283297in}}%
\pgfpathlineto{\pgfqpoint{1.031530in}{1.284166in}}%
\pgfpathlineto{\pgfqpoint{1.033344in}{1.285252in}}%
\pgfpathlineto{\pgfqpoint{1.034438in}{1.286277in}}%
\pgfpathlineto{\pgfqpoint{1.035939in}{1.287363in}}%
\pgfpathlineto{\pgfqpoint{1.036916in}{1.288263in}}%
\pgfpathlineto{\pgfqpoint{1.038378in}{1.289350in}}%
\pgfpathlineto{\pgfqpoint{1.039480in}{1.290095in}}%
\pgfpathlineto{\pgfqpoint{1.040755in}{1.291181in}}%
\pgfpathlineto{\pgfqpoint{1.041818in}{1.292237in}}%
\pgfpathlineto{\pgfqpoint{1.043788in}{1.293323in}}%
\pgfpathlineto{\pgfqpoint{1.044843in}{1.294441in}}%
\pgfpathlineto{\pgfqpoint{1.046845in}{1.295527in}}%
\pgfpathlineto{\pgfqpoint{1.047939in}{1.296365in}}%
\pgfpathlineto{\pgfqpoint{1.049401in}{1.297452in}}%
\pgfpathlineto{\pgfqpoint{1.050409in}{1.298414in}}%
\pgfpathlineto{\pgfqpoint{1.050464in}{1.298414in}}%
\pgfpathlineto{\pgfqpoint{1.051965in}{1.299501in}}%
\pgfpathlineto{\pgfqpoint{1.053067in}{1.300308in}}%
\pgfpathlineto{\pgfqpoint{1.054561in}{1.301394in}}%
\pgfpathlineto{\pgfqpoint{1.055608in}{1.302263in}}%
\pgfpathlineto{\pgfqpoint{1.057172in}{1.303350in}}%
\pgfpathlineto{\pgfqpoint{1.058196in}{1.304188in}}%
\pgfpathlineto{\pgfqpoint{1.058266in}{1.304188in}}%
\pgfpathlineto{\pgfqpoint{1.059705in}{1.305275in}}%
\pgfpathlineto{\pgfqpoint{1.060815in}{1.306051in}}%
\pgfpathlineto{\pgfqpoint{1.062535in}{1.307137in}}%
\pgfpathlineto{\pgfqpoint{1.063567in}{1.307944in}}%
\pgfpathlineto{\pgfqpoint{1.065419in}{1.309000in}}%
\pgfpathlineto{\pgfqpoint{1.066522in}{1.309683in}}%
\pgfpathlineto{\pgfqpoint{1.067796in}{1.310738in}}%
\pgfpathlineto{\pgfqpoint{1.068875in}{1.311669in}}%
\pgfpathlineto{\pgfqpoint{1.070305in}{1.312756in}}%
\pgfpathlineto{\pgfqpoint{1.071408in}{1.313532in}}%
\pgfpathlineto{\pgfqpoint{1.072901in}{1.314618in}}%
\pgfpathlineto{\pgfqpoint{1.073956in}{1.315425in}}%
\pgfpathlineto{\pgfqpoint{1.075457in}{1.316512in}}%
\pgfpathlineto{\pgfqpoint{1.076567in}{1.317536in}}%
\pgfpathlineto{\pgfqpoint{1.077967in}{1.318623in}}%
\pgfpathlineto{\pgfqpoint{1.079069in}{1.319492in}}%
\pgfpathlineto{\pgfqpoint{1.080726in}{1.320578in}}%
\pgfpathlineto{\pgfqpoint{1.081813in}{1.321541in}}%
\pgfpathlineto{\pgfqpoint{1.083158in}{1.322627in}}%
\pgfpathlineto{\pgfqpoint{1.084174in}{1.323403in}}%
\pgfpathlineto{\pgfqpoint{1.085909in}{1.324490in}}%
\pgfpathlineto{\pgfqpoint{1.087004in}{1.325359in}}%
\pgfpathlineto{\pgfqpoint{1.088356in}{1.326445in}}%
\pgfpathlineto{\pgfqpoint{1.089466in}{1.327345in}}%
\pgfpathlineto{\pgfqpoint{1.091358in}{1.328432in}}%
\pgfpathlineto{\pgfqpoint{1.092320in}{1.329332in}}%
\pgfpathlineto{\pgfqpoint{1.093922in}{1.330357in}}%
\pgfpathlineto{\pgfqpoint{1.095009in}{1.331412in}}%
\pgfpathlineto{\pgfqpoint{1.097393in}{1.332498in}}%
\pgfpathlineto{\pgfqpoint{1.098472in}{1.333244in}}%
\pgfpathlineto{\pgfqpoint{1.100216in}{1.334330in}}%
\pgfpathlineto{\pgfqpoint{1.101279in}{1.334951in}}%
\pgfpathlineto{\pgfqpoint{1.101294in}{1.334951in}}%
\pgfpathlineto{\pgfqpoint{1.103124in}{1.336037in}}%
\pgfpathlineto{\pgfqpoint{1.104234in}{1.336782in}}%
\pgfpathlineto{\pgfqpoint{1.105962in}{1.337869in}}%
\pgfpathlineto{\pgfqpoint{1.107064in}{1.338769in}}%
\pgfpathlineto{\pgfqpoint{1.108182in}{1.339855in}}%
\pgfpathlineto{\pgfqpoint{1.109206in}{1.340663in}}%
\pgfpathlineto{\pgfqpoint{1.110918in}{1.341749in}}%
\pgfpathlineto{\pgfqpoint{1.112020in}{1.342494in}}%
\pgfpathlineto{\pgfqpoint{1.113365in}{1.343581in}}%
\pgfpathlineto{\pgfqpoint{1.114428in}{1.344388in}}%
\pgfpathlineto{\pgfqpoint{1.115757in}{1.345443in}}%
\pgfpathlineto{\pgfqpoint{1.116867in}{1.346126in}}%
\pgfpathlineto{\pgfqpoint{1.118353in}{1.347212in}}%
\pgfpathlineto{\pgfqpoint{1.119455in}{1.348206in}}%
\pgfpathlineto{\pgfqpoint{1.120854in}{1.349292in}}%
\pgfpathlineto{\pgfqpoint{1.121933in}{1.349665in}}%
\pgfpathlineto{\pgfqpoint{1.123786in}{1.350751in}}%
\pgfpathlineto{\pgfqpoint{1.124826in}{1.351279in}}%
\pgfpathlineto{\pgfqpoint{1.126749in}{1.352334in}}%
\pgfpathlineto{\pgfqpoint{1.127859in}{1.352831in}}%
\pgfpathlineto{\pgfqpoint{1.129352in}{1.353918in}}%
\pgfpathlineto{\pgfqpoint{1.130400in}{1.354849in}}%
\pgfpathlineto{\pgfqpoint{1.130446in}{1.354849in}}%
\pgfpathlineto{\pgfqpoint{1.131979in}{1.355935in}}%
\pgfpathlineto{\pgfqpoint{1.133073in}{1.356711in}}%
\pgfpathlineto{\pgfqpoint{1.135731in}{1.357798in}}%
\pgfpathlineto{\pgfqpoint{1.136826in}{1.358357in}}%
\pgfpathlineto{\pgfqpoint{1.138139in}{1.359412in}}%
\pgfpathlineto{\pgfqpoint{1.139241in}{1.360126in}}%
\pgfpathlineto{\pgfqpoint{1.140992in}{1.361212in}}%
\pgfpathlineto{\pgfqpoint{1.142103in}{1.361926in}}%
\pgfpathlineto{\pgfqpoint{1.143432in}{1.363013in}}%
\pgfpathlineto{\pgfqpoint{1.144463in}{1.363758in}}%
\pgfpathlineto{\pgfqpoint{1.146621in}{1.364813in}}%
\pgfpathlineto{\pgfqpoint{1.147731in}{1.365838in}}%
\pgfpathlineto{\pgfqpoint{1.150022in}{1.366924in}}%
\pgfpathlineto{\pgfqpoint{1.151101in}{1.367700in}}%
\pgfpathlineto{\pgfqpoint{1.153290in}{1.368787in}}%
\pgfpathlineto{\pgfqpoint{1.154400in}{1.369563in}}%
\pgfpathlineto{\pgfqpoint{1.156151in}{1.370649in}}%
\pgfpathlineto{\pgfqpoint{1.157198in}{1.371146in}}%
\pgfpathlineto{\pgfqpoint{1.159419in}{1.372232in}}%
\pgfpathlineto{\pgfqpoint{1.160333in}{1.372853in}}%
\pgfpathlineto{\pgfqpoint{1.160380in}{1.372853in}}%
\pgfpathlineto{\pgfqpoint{1.162655in}{1.373940in}}%
\pgfpathlineto{\pgfqpoint{1.163703in}{1.374561in}}%
\pgfpathlineto{\pgfqpoint{1.163734in}{1.374561in}}%
\pgfpathlineto{\pgfqpoint{1.165344in}{1.375647in}}%
\pgfpathlineto{\pgfqpoint{1.166306in}{1.376206in}}%
\pgfpathlineto{\pgfqpoint{1.167737in}{1.377292in}}%
\pgfpathlineto{\pgfqpoint{1.168847in}{1.378224in}}%
\pgfpathlineto{\pgfqpoint{1.171341in}{1.379310in}}%
\pgfpathlineto{\pgfqpoint{1.172419in}{1.379776in}}%
\pgfpathlineto{\pgfqpoint{1.174960in}{1.380862in}}%
\pgfpathlineto{\pgfqpoint{1.176070in}{1.381297in}}%
\pgfpathlineto{\pgfqpoint{1.178118in}{1.382383in}}%
\pgfpathlineto{\pgfqpoint{1.179205in}{1.382849in}}%
\pgfpathlineto{\pgfqpoint{1.181175in}{1.383935in}}%
\pgfpathlineto{\pgfqpoint{1.182145in}{1.384556in}}%
\pgfpathlineto{\pgfqpoint{1.182199in}{1.384556in}}%
\pgfpathlineto{\pgfqpoint{1.184052in}{1.385643in}}%
\pgfpathlineto{\pgfqpoint{1.185139in}{1.386357in}}%
\pgfpathlineto{\pgfqpoint{1.186835in}{1.387412in}}%
\pgfpathlineto{\pgfqpoint{1.187922in}{1.388095in}}%
\pgfpathlineto{\pgfqpoint{1.189954in}{1.389181in}}%
\pgfpathlineto{\pgfqpoint{1.191064in}{1.389926in}}%
\pgfpathlineto{\pgfqpoint{1.193425in}{1.390982in}}%
\pgfpathlineto{\pgfqpoint{1.194426in}{1.391696in}}%
\pgfpathlineto{\pgfqpoint{1.196459in}{1.392782in}}%
\pgfpathlineto{\pgfqpoint{1.197420in}{1.393341in}}%
\pgfpathlineto{\pgfqpoint{1.200188in}{1.394428in}}%
\pgfpathlineto{\pgfqpoint{1.201259in}{1.395079in}}%
\pgfpathlineto{\pgfqpoint{1.201274in}{1.395079in}}%
\pgfpathlineto{\pgfqpoint{1.203323in}{1.396166in}}%
\pgfpathlineto{\pgfqpoint{1.204261in}{1.396694in}}%
\pgfpathlineto{\pgfqpoint{1.204417in}{1.396694in}}%
\pgfpathlineto{\pgfqpoint{1.205996in}{1.397780in}}%
\pgfpathlineto{\pgfqpoint{1.207106in}{1.398339in}}%
\pgfpathlineto{\pgfqpoint{1.208795in}{1.399425in}}%
\pgfpathlineto{\pgfqpoint{1.209882in}{1.399984in}}%
\pgfpathlineto{\pgfqpoint{1.211445in}{1.401071in}}%
\pgfpathlineto{\pgfqpoint{1.212516in}{1.402033in}}%
\pgfpathlineto{\pgfqpoint{1.214869in}{1.403119in}}%
\pgfpathlineto{\pgfqpoint{1.215799in}{1.403771in}}%
\pgfpathlineto{\pgfqpoint{1.215870in}{1.403771in}}%
\pgfpathlineto{\pgfqpoint{1.217910in}{1.404858in}}%
\pgfpathlineto{\pgfqpoint{1.219013in}{1.405354in}}%
\pgfpathlineto{\pgfqpoint{1.220779in}{1.406441in}}%
\pgfpathlineto{\pgfqpoint{1.221741in}{1.406906in}}%
\pgfpathlineto{\pgfqpoint{1.224321in}{1.407993in}}%
\pgfpathlineto{\pgfqpoint{1.225290in}{1.408459in}}%
\pgfpathlineto{\pgfqpoint{1.228214in}{1.409545in}}%
\pgfpathlineto{\pgfqpoint{1.229324in}{1.410259in}}%
\pgfpathlineto{\pgfqpoint{1.231669in}{1.411345in}}%
\pgfpathlineto{\pgfqpoint{1.232561in}{1.411811in}}%
\pgfpathlineto{\pgfqpoint{1.235297in}{1.412898in}}%
\pgfpathlineto{\pgfqpoint{1.236391in}{1.413643in}}%
\pgfpathlineto{\pgfqpoint{1.238885in}{1.414729in}}%
\pgfpathlineto{\pgfqpoint{1.239917in}{1.415102in}}%
\pgfpathlineto{\pgfqpoint{1.239987in}{1.415102in}}%
\pgfpathlineto{\pgfqpoint{1.242661in}{1.416188in}}%
\pgfpathlineto{\pgfqpoint{1.243701in}{1.416592in}}%
\pgfpathlineto{\pgfqpoint{1.246265in}{1.417678in}}%
\pgfpathlineto{\pgfqpoint{1.247305in}{1.418206in}}%
\pgfpathlineto{\pgfqpoint{1.249142in}{1.419292in}}%
\pgfpathlineto{\pgfqpoint{1.250205in}{1.419665in}}%
\pgfpathlineto{\pgfqpoint{1.252284in}{1.420751in}}%
\pgfpathlineto{\pgfqpoint{1.253254in}{1.421186in}}%
\pgfpathlineto{\pgfqpoint{1.255849in}{1.422272in}}%
\pgfpathlineto{\pgfqpoint{1.256866in}{1.423048in}}%
\pgfpathlineto{\pgfqpoint{1.259242in}{1.424135in}}%
\pgfpathlineto{\pgfqpoint{1.260282in}{1.424632in}}%
\pgfpathlineto{\pgfqpoint{1.262400in}{1.425687in}}%
\pgfpathlineto{\pgfqpoint{1.263479in}{1.426153in}}%
\pgfpathlineto{\pgfqpoint{1.265316in}{1.427239in}}%
\pgfpathlineto{\pgfqpoint{1.266403in}{1.427736in}}%
\pgfpathlineto{\pgfqpoint{1.268655in}{1.428822in}}%
\pgfpathlineto{\pgfqpoint{1.269749in}{1.429474in}}%
\pgfpathlineto{\pgfqpoint{1.271602in}{1.430561in}}%
\pgfpathlineto{\pgfqpoint{1.272595in}{1.431212in}}%
\pgfpathlineto{\pgfqpoint{1.275128in}{1.432299in}}%
\pgfpathlineto{\pgfqpoint{1.276089in}{1.432920in}}%
\pgfpathlineto{\pgfqpoint{1.278090in}{1.434006in}}%
\pgfpathlineto{\pgfqpoint{1.279201in}{1.434503in}}%
\pgfpathlineto{\pgfqpoint{1.281577in}{1.435589in}}%
\pgfpathlineto{\pgfqpoint{1.282679in}{1.436148in}}%
\pgfpathlineto{\pgfqpoint{1.284649in}{1.437235in}}%
\pgfpathlineto{\pgfqpoint{1.285728in}{1.437793in}}%
\pgfpathlineto{\pgfqpoint{1.288433in}{1.438880in}}%
\pgfpathlineto{\pgfqpoint{1.289473in}{1.439314in}}%
\pgfpathlineto{\pgfqpoint{1.292663in}{1.440401in}}%
\pgfpathlineto{\pgfqpoint{1.293749in}{1.441084in}}%
\pgfpathlineto{\pgfqpoint{1.295993in}{1.442170in}}%
\pgfpathlineto{\pgfqpoint{1.296994in}{1.442760in}}%
\pgfpathlineto{\pgfqpoint{1.297025in}{1.442760in}}%
\pgfpathlineto{\pgfqpoint{1.299229in}{1.443785in}}%
\pgfpathlineto{\pgfqpoint{1.300324in}{1.444561in}}%
\pgfpathlineto{\pgfqpoint{1.302661in}{1.445647in}}%
\pgfpathlineto{\pgfqpoint{1.303678in}{1.446144in}}%
\pgfpathlineto{\pgfqpoint{1.306805in}{1.447168in}}%
\pgfpathlineto{\pgfqpoint{1.307876in}{1.447665in}}%
\pgfpathlineto{\pgfqpoint{1.310346in}{1.448720in}}%
\pgfpathlineto{\pgfqpoint{1.311253in}{1.449124in}}%
\pgfpathlineto{\pgfqpoint{1.313356in}{1.450210in}}%
\pgfpathlineto{\pgfqpoint{1.314614in}{1.450707in}}%
\pgfpathlineto{\pgfqpoint{1.316991in}{1.451793in}}%
\pgfpathlineto{\pgfqpoint{1.317976in}{1.452507in}}%
\pgfpathlineto{\pgfqpoint{1.320681in}{1.453594in}}%
\pgfpathlineto{\pgfqpoint{1.321736in}{1.454028in}}%
\pgfpathlineto{\pgfqpoint{1.321791in}{1.454028in}}%
\pgfpathlineto{\pgfqpoint{1.322964in}{1.454494in}}%
\pgfpathlineto{\pgfqpoint{1.325106in}{1.455581in}}%
\pgfpathlineto{\pgfqpoint{1.326020in}{1.456108in}}%
\pgfpathlineto{\pgfqpoint{1.328952in}{1.457195in}}%
\pgfpathlineto{\pgfqpoint{1.330000in}{1.457691in}}%
\pgfpathlineto{\pgfqpoint{1.333392in}{1.458778in}}%
\pgfpathlineto{\pgfqpoint{1.334495in}{1.459337in}}%
\pgfpathlineto{\pgfqpoint{1.337035in}{1.460423in}}%
\pgfpathlineto{\pgfqpoint{1.337817in}{1.460951in}}%
\pgfpathlineto{\pgfqpoint{1.338028in}{1.460951in}}%
\pgfpathlineto{\pgfqpoint{1.340553in}{1.462037in}}%
\pgfpathlineto{\pgfqpoint{1.341585in}{1.462379in}}%
\pgfpathlineto{\pgfqpoint{1.341632in}{1.462379in}}%
\pgfpathlineto{\pgfqpoint{1.344322in}{1.463465in}}%
\pgfpathlineto{\pgfqpoint{1.345299in}{1.463900in}}%
\pgfpathlineto{\pgfqpoint{1.345432in}{1.463900in}}%
\pgfpathlineto{\pgfqpoint{1.348105in}{1.464955in}}%
\pgfpathlineto{\pgfqpoint{1.349059in}{1.465576in}}%
\pgfpathlineto{\pgfqpoint{1.352045in}{1.466663in}}%
\pgfpathlineto{\pgfqpoint{1.353124in}{1.467252in}}%
\pgfpathlineto{\pgfqpoint{1.355649in}{1.468339in}}%
\pgfpathlineto{\pgfqpoint{1.356752in}{1.468836in}}%
\pgfpathlineto{\pgfqpoint{1.359621in}{1.469922in}}%
\pgfpathlineto{\pgfqpoint{1.360684in}{1.470698in}}%
\pgfpathlineto{\pgfqpoint{1.363858in}{1.471785in}}%
\pgfpathlineto{\pgfqpoint{1.364898in}{1.472374in}}%
\pgfpathlineto{\pgfqpoint{1.367939in}{1.473430in}}%
\pgfpathlineto{\pgfqpoint{1.367939in}{1.473461in}}%
\pgfpathlineto{\pgfqpoint{1.368877in}{1.473802in}}%
\pgfpathlineto{\pgfqpoint{1.371668in}{1.474889in}}%
\pgfpathlineto{\pgfqpoint{1.372715in}{1.475354in}}%
\pgfpathlineto{\pgfqpoint{1.372770in}{1.475354in}}%
\pgfpathlineto{\pgfqpoint{1.375694in}{1.476441in}}%
\pgfpathlineto{\pgfqpoint{1.376475in}{1.476720in}}%
\pgfpathlineto{\pgfqpoint{1.379352in}{1.477807in}}%
\pgfpathlineto{\pgfqpoint{1.380439in}{1.478179in}}%
\pgfpathlineto{\pgfqpoint{1.383590in}{1.479266in}}%
\pgfpathlineto{\pgfqpoint{1.384645in}{1.479669in}}%
\pgfpathlineto{\pgfqpoint{1.384676in}{1.479669in}}%
\pgfpathlineto{\pgfqpoint{1.388569in}{1.480756in}}%
\pgfpathlineto{\pgfqpoint{1.389586in}{1.481159in}}%
\pgfpathlineto{\pgfqpoint{1.389625in}{1.481159in}}%
\pgfpathlineto{\pgfqpoint{1.391908in}{1.482246in}}%
\pgfpathlineto{\pgfqpoint{1.392846in}{1.482711in}}%
\pgfpathlineto{\pgfqpoint{1.395809in}{1.483798in}}%
\pgfpathlineto{\pgfqpoint{1.396911in}{1.484139in}}%
\pgfpathlineto{\pgfqpoint{1.399303in}{1.485195in}}%
\pgfpathlineto{\pgfqpoint{1.400296in}{1.485505in}}%
\pgfpathlineto{\pgfqpoint{1.400405in}{1.485505in}}%
\pgfpathlineto{\pgfqpoint{1.404095in}{1.486592in}}%
\pgfpathlineto{\pgfqpoint{1.405182in}{1.487337in}}%
\pgfpathlineto{\pgfqpoint{1.408575in}{1.488423in}}%
\pgfpathlineto{\pgfqpoint{1.409552in}{1.488827in}}%
\pgfpathlineto{\pgfqpoint{1.412226in}{1.489913in}}%
\pgfpathlineto{\pgfqpoint{1.413336in}{1.490286in}}%
\pgfpathlineto{\pgfqpoint{1.416471in}{1.491372in}}%
\pgfpathlineto{\pgfqpoint{1.417370in}{1.491776in}}%
\pgfpathlineto{\pgfqpoint{1.417456in}{1.491776in}}%
\pgfpathlineto{\pgfqpoint{1.420614in}{1.492862in}}%
\pgfpathlineto{\pgfqpoint{1.421701in}{1.493297in}}%
\pgfpathlineto{\pgfqpoint{1.421716in}{1.493297in}}%
\pgfpathlineto{\pgfqpoint{1.424570in}{1.494352in}}%
\pgfpathlineto{\pgfqpoint{1.424570in}{1.494383in}}%
\pgfpathlineto{\pgfqpoint{1.425844in}{1.495097in}}%
\pgfpathlineto{\pgfqpoint{1.429112in}{1.496184in}}%
\pgfpathlineto{\pgfqpoint{1.429964in}{1.496494in}}%
\pgfpathlineto{\pgfqpoint{1.429987in}{1.496494in}}%
\pgfpathlineto{\pgfqpoint{1.432426in}{1.497581in}}%
\pgfpathlineto{\pgfqpoint{1.433443in}{1.497984in}}%
\pgfpathlineto{\pgfqpoint{1.437813in}{1.499071in}}%
\pgfpathlineto{\pgfqpoint{1.438884in}{1.499536in}}%
\pgfpathlineto{\pgfqpoint{1.442910in}{1.500623in}}%
\pgfpathlineto{\pgfqpoint{1.443957in}{1.500995in}}%
\pgfpathlineto{\pgfqpoint{1.443973in}{1.500995in}}%
\pgfpathlineto{\pgfqpoint{1.448046in}{1.502051in}}%
\pgfpathlineto{\pgfqpoint{1.448929in}{1.502237in}}%
\pgfpathlineto{\pgfqpoint{1.453448in}{1.503323in}}%
\pgfpathlineto{\pgfqpoint{1.454542in}{1.503696in}}%
\pgfpathlineto{\pgfqpoint{1.457826in}{1.504782in}}%
\pgfpathlineto{\pgfqpoint{1.458827in}{1.505062in}}%
\pgfpathlineto{\pgfqpoint{1.461703in}{1.506148in}}%
\pgfpathlineto{\pgfqpoint{1.462751in}{1.506552in}}%
\pgfpathlineto{\pgfqpoint{1.466136in}{1.507638in}}%
\pgfpathlineto{\pgfqpoint{1.467230in}{1.508042in}}%
\pgfpathlineto{\pgfqpoint{1.470905in}{1.509128in}}%
\pgfpathlineto{\pgfqpoint{1.471929in}{1.509594in}}%
\pgfpathlineto{\pgfqpoint{1.475290in}{1.510680in}}%
\pgfpathlineto{\pgfqpoint{1.476401in}{1.511363in}}%
\pgfpathlineto{\pgfqpoint{1.479950in}{1.512419in}}%
\pgfpathlineto{\pgfqpoint{1.481013in}{1.512853in}}%
\pgfpathlineto{\pgfqpoint{1.485031in}{1.513940in}}%
\pgfpathlineto{\pgfqpoint{1.486040in}{1.514343in}}%
\pgfpathlineto{\pgfqpoint{1.488971in}{1.515430in}}%
\pgfpathlineto{\pgfqpoint{1.490058in}{1.515678in}}%
\pgfpathlineto{\pgfqpoint{1.492888in}{1.516734in}}%
\pgfpathlineto{\pgfqpoint{1.493896in}{1.517168in}}%
\pgfpathlineto{\pgfqpoint{1.496554in}{1.518255in}}%
\pgfpathlineto{\pgfqpoint{1.497633in}{1.518627in}}%
\pgfpathlineto{\pgfqpoint{1.501589in}{1.519714in}}%
\pgfpathlineto{\pgfqpoint{1.502613in}{1.520086in}}%
\pgfpathlineto{\pgfqpoint{1.502629in}{1.520086in}}%
\pgfpathlineto{\pgfqpoint{1.506155in}{1.521173in}}%
\pgfpathlineto{\pgfqpoint{1.507226in}{1.521421in}}%
\pgfpathlineto{\pgfqpoint{1.511603in}{1.522507in}}%
\pgfpathlineto{\pgfqpoint{1.512690in}{1.522880in}}%
\pgfpathlineto{\pgfqpoint{1.515731in}{1.523966in}}%
\pgfpathlineto{\pgfqpoint{1.516818in}{1.524432in}}%
\pgfpathlineto{\pgfqpoint{1.521094in}{1.525518in}}%
\pgfpathlineto{\pgfqpoint{1.522071in}{1.525953in}}%
\pgfpathlineto{\pgfqpoint{1.522095in}{1.525953in}}%
\pgfpathlineto{\pgfqpoint{1.524596in}{1.527040in}}%
\pgfpathlineto{\pgfqpoint{1.525691in}{1.527381in}}%
\pgfpathlineto{\pgfqpoint{1.529772in}{1.528467in}}%
\pgfpathlineto{\pgfqpoint{1.530647in}{1.528933in}}%
\pgfpathlineto{\pgfqpoint{1.530874in}{1.528933in}}%
\pgfpathlineto{\pgfqpoint{1.535752in}{1.530020in}}%
\pgfpathlineto{\pgfqpoint{1.536706in}{1.530175in}}%
\pgfpathlineto{\pgfqpoint{1.540505in}{1.531230in}}%
\pgfpathlineto{\pgfqpoint{1.541451in}{1.531541in}}%
\pgfpathlineto{\pgfqpoint{1.545759in}{1.532627in}}%
\pgfpathlineto{\pgfqpoint{1.546728in}{1.532875in}}%
\pgfpathlineto{\pgfqpoint{1.546869in}{1.532875in}}%
\pgfpathlineto{\pgfqpoint{1.549746in}{1.533962in}}%
\pgfpathlineto{\pgfqpoint{1.550488in}{1.534272in}}%
\pgfpathlineto{\pgfqpoint{1.556039in}{1.535359in}}%
\pgfpathlineto{\pgfqpoint{1.557149in}{1.535669in}}%
\pgfpathlineto{\pgfqpoint{1.561136in}{1.536756in}}%
\pgfpathlineto{\pgfqpoint{1.562043in}{1.537283in}}%
\pgfpathlineto{\pgfqpoint{1.566116in}{1.538370in}}%
\pgfpathlineto{\pgfqpoint{1.567226in}{1.538680in}}%
\pgfpathlineto{\pgfqpoint{1.572370in}{1.539767in}}%
\pgfpathlineto{\pgfqpoint{1.573433in}{1.540046in}}%
\pgfpathlineto{\pgfqpoint{1.577365in}{1.541133in}}%
\pgfpathlineto{\pgfqpoint{1.578319in}{1.541412in}}%
\pgfpathlineto{\pgfqpoint{1.578476in}{1.541412in}}%
\pgfpathlineto{\pgfqpoint{1.582463in}{1.542498in}}%
\pgfpathlineto{\pgfqpoint{1.583440in}{1.542809in}}%
\pgfpathlineto{\pgfqpoint{1.588693in}{1.543895in}}%
\pgfpathlineto{\pgfqpoint{1.589678in}{1.544206in}}%
\pgfpathlineto{\pgfqpoint{1.594502in}{1.545292in}}%
\pgfpathlineto{\pgfqpoint{1.595604in}{1.545634in}}%
\pgfpathlineto{\pgfqpoint{1.600506in}{1.546720in}}%
\pgfpathlineto{\pgfqpoint{1.601592in}{1.547062in}}%
\pgfpathlineto{\pgfqpoint{1.605548in}{1.548148in}}%
\pgfpathlineto{\pgfqpoint{1.606048in}{1.548396in}}%
\pgfpathlineto{\pgfqpoint{1.606635in}{1.548396in}}%
\pgfpathlineto{\pgfqpoint{1.611966in}{1.549483in}}%
\pgfpathlineto{\pgfqpoint{1.612623in}{1.549855in}}%
\pgfpathlineto{\pgfqpoint{1.612967in}{1.549855in}}%
\pgfpathlineto{\pgfqpoint{1.617618in}{1.550942in}}%
\pgfpathlineto{\pgfqpoint{1.618689in}{1.551252in}}%
\pgfpathlineto{\pgfqpoint{1.623841in}{1.552339in}}%
\pgfpathlineto{\pgfqpoint{1.624983in}{1.552556in}}%
\pgfpathlineto{\pgfqpoint{1.630103in}{1.553643in}}%
\pgfpathlineto{\pgfqpoint{1.631127in}{1.553984in}}%
\pgfpathlineto{\pgfqpoint{1.636084in}{1.555040in}}%
\pgfpathlineto{\pgfqpoint{1.636959in}{1.555350in}}%
\pgfpathlineto{\pgfqpoint{1.637045in}{1.555350in}}%
\pgfpathlineto{\pgfqpoint{1.642439in}{1.556405in}}%
\pgfpathlineto{\pgfqpoint{1.643487in}{1.556778in}}%
\pgfpathlineto{\pgfqpoint{1.643542in}{1.556778in}}%
\pgfpathlineto{\pgfqpoint{1.648615in}{1.557833in}}%
\pgfpathlineto{\pgfqpoint{1.649694in}{1.558175in}}%
\pgfpathlineto{\pgfqpoint{1.649726in}{1.558175in}}%
\pgfpathlineto{\pgfqpoint{1.655526in}{1.559261in}}%
\pgfpathlineto{\pgfqpoint{1.656550in}{1.559479in}}%
\pgfpathlineto{\pgfqpoint{1.661186in}{1.560565in}}%
\pgfpathlineto{\pgfqpoint{1.662242in}{1.560906in}}%
\pgfpathlineto{\pgfqpoint{1.667370in}{1.561993in}}%
\pgfpathlineto{\pgfqpoint{1.668050in}{1.562117in}}%
\pgfpathlineto{\pgfqpoint{1.668472in}{1.562117in}}%
\pgfpathlineto{\pgfqpoint{1.674828in}{1.563204in}}%
\pgfpathlineto{\pgfqpoint{1.675664in}{1.563390in}}%
\pgfpathlineto{\pgfqpoint{1.682083in}{1.564476in}}%
\pgfpathlineto{\pgfqpoint{1.682794in}{1.564694in}}%
\pgfpathlineto{\pgfqpoint{1.683013in}{1.564694in}}%
\pgfpathlineto{\pgfqpoint{1.688751in}{1.565780in}}%
\pgfpathlineto{\pgfqpoint{1.689721in}{1.566091in}}%
\pgfpathlineto{\pgfqpoint{1.698250in}{1.567177in}}%
\pgfpathlineto{\pgfqpoint{1.699352in}{1.567394in}}%
\pgfpathlineto{\pgfqpoint{1.705004in}{1.568450in}}%
\pgfpathlineto{\pgfqpoint{1.706091in}{1.568760in}}%
\pgfpathlineto{\pgfqpoint{1.711055in}{1.569847in}}%
\pgfpathlineto{\pgfqpoint{1.712087in}{1.570033in}}%
\pgfpathlineto{\pgfqpoint{1.712118in}{1.570033in}}%
\pgfpathlineto{\pgfqpoint{1.717293in}{1.571119in}}%
\pgfpathlineto{\pgfqpoint{1.718388in}{1.571368in}}%
\pgfpathlineto{\pgfqpoint{1.726112in}{1.572454in}}%
\pgfpathlineto{\pgfqpoint{1.727136in}{1.572796in}}%
\pgfpathlineto{\pgfqpoint{1.733992in}{1.573882in}}%
\pgfpathlineto{\pgfqpoint{1.735055in}{1.574193in}}%
\pgfpathlineto{\pgfqpoint{1.741043in}{1.575279in}}%
\pgfpathlineto{\pgfqpoint{1.742138in}{1.575527in}}%
\pgfpathlineto{\pgfqpoint{1.748408in}{1.576614in}}%
\pgfpathlineto{\pgfqpoint{1.749494in}{1.576831in}}%
\pgfpathlineto{\pgfqpoint{1.749510in}{1.576831in}}%
\pgfpathlineto{\pgfqpoint{1.756116in}{1.577918in}}%
\pgfpathlineto{\pgfqpoint{1.756991in}{1.578104in}}%
\pgfpathlineto{\pgfqpoint{1.764371in}{1.579190in}}%
\pgfpathlineto{\pgfqpoint{1.765411in}{1.579501in}}%
\pgfpathlineto{\pgfqpoint{1.772846in}{1.580587in}}%
\pgfpathlineto{\pgfqpoint{1.773604in}{1.580773in}}%
\pgfpathlineto{\pgfqpoint{1.780679in}{1.581860in}}%
\pgfpathlineto{\pgfqpoint{1.781726in}{1.582232in}}%
\pgfpathlineto{\pgfqpoint{1.781758in}{1.582232in}}%
\pgfpathlineto{\pgfqpoint{1.790412in}{1.583319in}}%
\pgfpathlineto{\pgfqpoint{1.791506in}{1.583474in}}%
\pgfpathlineto{\pgfqpoint{1.798847in}{1.584561in}}%
\pgfpathlineto{\pgfqpoint{1.799957in}{1.584747in}}%
\pgfpathlineto{\pgfqpoint{1.806704in}{1.585833in}}%
\pgfpathlineto{\pgfqpoint{1.807806in}{1.585989in}}%
\pgfpathlineto{\pgfqpoint{1.817789in}{1.587075in}}%
\pgfpathlineto{\pgfqpoint{1.818774in}{1.587292in}}%
\pgfpathlineto{\pgfqpoint{1.824747in}{1.588379in}}%
\pgfpathlineto{\pgfqpoint{1.825450in}{1.588441in}}%
\pgfpathlineto{\pgfqpoint{1.836223in}{1.589527in}}%
\pgfpathlineto{\pgfqpoint{1.837161in}{1.589651in}}%
\pgfpathlineto{\pgfqpoint{1.848669in}{1.590738in}}%
\pgfpathlineto{\pgfqpoint{1.849740in}{1.590986in}}%
\pgfpathlineto{\pgfqpoint{1.862170in}{1.592073in}}%
\pgfpathlineto{\pgfqpoint{1.862811in}{1.592135in}}%
\pgfpathlineto{\pgfqpoint{1.862850in}{1.592135in}}%
\pgfpathlineto{\pgfqpoint{1.879908in}{1.593221in}}%
\pgfpathlineto{\pgfqpoint{1.880643in}{1.593314in}}%
\pgfpathlineto{\pgfqpoint{1.895332in}{1.594401in}}%
\pgfpathlineto{\pgfqpoint{1.896435in}{1.594618in}}%
\pgfpathlineto{\pgfqpoint{1.911093in}{1.595705in}}%
\pgfpathlineto{\pgfqpoint{1.911390in}{1.595767in}}%
\pgfpathlineto{\pgfqpoint{1.912164in}{1.595767in}}%
\pgfpathlineto{\pgfqpoint{1.922084in}{1.596853in}}%
\pgfpathlineto{\pgfqpoint{1.922905in}{1.596977in}}%
\pgfpathlineto{\pgfqpoint{1.944896in}{1.598064in}}%
\pgfpathlineto{\pgfqpoint{1.945999in}{1.598188in}}%
\pgfpathlineto{\pgfqpoint{1.966661in}{1.599275in}}%
\pgfpathlineto{\pgfqpoint{1.966661in}{1.599306in}}%
\pgfpathlineto{\pgfqpoint{1.967505in}{1.599306in}}%
\pgfpathlineto{\pgfqpoint{1.986728in}{1.600392in}}%
\pgfpathlineto{\pgfqpoint{1.986728in}{1.600423in}}%
\pgfpathlineto{\pgfqpoint{1.987753in}{1.600423in}}%
\pgfpathlineto{\pgfqpoint{2.017139in}{1.601510in}}%
\pgfpathlineto{\pgfqpoint{2.017413in}{1.601572in}}%
\pgfpathlineto{\pgfqpoint{2.033126in}{1.601944in}}%
\pgfpathlineto{\pgfqpoint{2.033126in}{1.601944in}}%
\pgfusepath{stroke}%
\end{pgfscope}%
\begin{pgfscope}%
\pgfsetrectcap%
\pgfsetmiterjoin%
\pgfsetlinewidth{0.803000pt}%
\definecolor{currentstroke}{rgb}{0.000000,0.000000,0.000000}%
\pgfsetstrokecolor{currentstroke}%
\pgfsetdash{}{0pt}%
\pgfpathmoveto{\pgfqpoint{0.553581in}{0.499444in}}%
\pgfpathlineto{\pgfqpoint{0.553581in}{1.654444in}}%
\pgfusepath{stroke}%
\end{pgfscope}%
\begin{pgfscope}%
\pgfsetrectcap%
\pgfsetmiterjoin%
\pgfsetlinewidth{0.803000pt}%
\definecolor{currentstroke}{rgb}{0.000000,0.000000,0.000000}%
\pgfsetstrokecolor{currentstroke}%
\pgfsetdash{}{0pt}%
\pgfpathmoveto{\pgfqpoint{2.103581in}{0.499444in}}%
\pgfpathlineto{\pgfqpoint{2.103581in}{1.654444in}}%
\pgfusepath{stroke}%
\end{pgfscope}%
\begin{pgfscope}%
\pgfsetrectcap%
\pgfsetmiterjoin%
\pgfsetlinewidth{0.803000pt}%
\definecolor{currentstroke}{rgb}{0.000000,0.000000,0.000000}%
\pgfsetstrokecolor{currentstroke}%
\pgfsetdash{}{0pt}%
\pgfpathmoveto{\pgfqpoint{0.553581in}{0.499444in}}%
\pgfpathlineto{\pgfqpoint{2.103581in}{0.499444in}}%
\pgfusepath{stroke}%
\end{pgfscope}%
\begin{pgfscope}%
\pgfsetrectcap%
\pgfsetmiterjoin%
\pgfsetlinewidth{0.803000pt}%
\definecolor{currentstroke}{rgb}{0.000000,0.000000,0.000000}%
\pgfsetstrokecolor{currentstroke}%
\pgfsetdash{}{0pt}%
\pgfpathmoveto{\pgfqpoint{0.553581in}{1.654444in}}%
\pgfpathlineto{\pgfqpoint{2.103581in}{1.654444in}}%
\pgfusepath{stroke}%
\end{pgfscope}%
\begin{pgfscope}%
\pgfsetbuttcap%
\pgfsetmiterjoin%
\definecolor{currentfill}{rgb}{1.000000,1.000000,1.000000}%
\pgfsetfillcolor{currentfill}%
\pgfsetfillopacity{0.800000}%
\pgfsetlinewidth{1.003750pt}%
\definecolor{currentstroke}{rgb}{0.800000,0.800000,0.800000}%
\pgfsetstrokecolor{currentstroke}%
\pgfsetstrokeopacity{0.800000}%
\pgfsetdash{}{0pt}%
\pgfpathmoveto{\pgfqpoint{0.832747in}{0.568889in}}%
\pgfpathlineto{\pgfqpoint{2.006358in}{0.568889in}}%
\pgfpathquadraticcurveto{\pgfqpoint{2.034136in}{0.568889in}}{\pgfqpoint{2.034136in}{0.596666in}}%
\pgfpathlineto{\pgfqpoint{2.034136in}{0.776388in}}%
\pgfpathquadraticcurveto{\pgfqpoint{2.034136in}{0.804166in}}{\pgfqpoint{2.006358in}{0.804166in}}%
\pgfpathlineto{\pgfqpoint{0.832747in}{0.804166in}}%
\pgfpathquadraticcurveto{\pgfqpoint{0.804970in}{0.804166in}}{\pgfqpoint{0.804970in}{0.776388in}}%
\pgfpathlineto{\pgfqpoint{0.804970in}{0.596666in}}%
\pgfpathquadraticcurveto{\pgfqpoint{0.804970in}{0.568889in}}{\pgfqpoint{0.832747in}{0.568889in}}%
\pgfpathlineto{\pgfqpoint{0.832747in}{0.568889in}}%
\pgfpathclose%
\pgfusepath{stroke,fill}%
\end{pgfscope}%
\begin{pgfscope}%
\pgfsetrectcap%
\pgfsetroundjoin%
\pgfsetlinewidth{1.505625pt}%
\definecolor{currentstroke}{rgb}{0.000000,0.000000,0.000000}%
\pgfsetstrokecolor{currentstroke}%
\pgfsetdash{}{0pt}%
\pgfpathmoveto{\pgfqpoint{0.860525in}{0.700000in}}%
\pgfpathlineto{\pgfqpoint{0.999414in}{0.700000in}}%
\pgfpathlineto{\pgfqpoint{1.138303in}{0.700000in}}%
\pgfusepath{stroke}%
\end{pgfscope}%
\begin{pgfscope}%
\definecolor{textcolor}{rgb}{0.000000,0.000000,0.000000}%
\pgfsetstrokecolor{textcolor}%
\pgfsetfillcolor{textcolor}%
\pgftext[x=1.249414in,y=0.651388in,left,base]{\color{textcolor}\rmfamily\fontsize{10.000000}{12.000000}\selectfont AUC=0.778}%
\end{pgfscope}%
\end{pgfpicture}%
\makeatother%
\endgroup%

\end{tabular}

	

\

Model 3:  $\alpha = \pi_0 = 0.84$ for class balance

\noindent\begin{tabular}{@{\hspace{-6pt}}p{4.5in} @{\hspace{-6pt}}p{2.0in}}
	\vskip 0pt
	\qquad \qquad Raw Model Output
	
	%% Creator: Matplotlib, PGF backend
%%
%% To include the figure in your LaTeX document, write
%%   \input{<filename>.pgf}
%%
%% Make sure the required packages are loaded in your preamble
%%   \usepackage{pgf}
%%
%% Also ensure that all the required font packages are loaded; for instance,
%% the lmodern package is sometimes necessary when using math font.
%%   \usepackage{lmodern}
%%
%% Figures using additional raster images can only be included by \input if
%% they are in the same directory as the main LaTeX file. For loading figures
%% from other directories you can use the `import` package
%%   \usepackage{import}
%%
%% and then include the figures with
%%   \import{<path to file>}{<filename>.pgf}
%%
%% Matplotlib used the following preamble
%%   
%%   \usepackage{fontspec}
%%   \makeatletter\@ifpackageloaded{underscore}{}{\usepackage[strings]{underscore}}\makeatother
%%
\begingroup%
\makeatletter%
\begin{pgfpicture}%
\pgfpathrectangle{\pgfpointorigin}{\pgfqpoint{4.509306in}{1.754444in}}%
\pgfusepath{use as bounding box, clip}%
\begin{pgfscope}%
\pgfsetbuttcap%
\pgfsetmiterjoin%
\definecolor{currentfill}{rgb}{1.000000,1.000000,1.000000}%
\pgfsetfillcolor{currentfill}%
\pgfsetlinewidth{0.000000pt}%
\definecolor{currentstroke}{rgb}{1.000000,1.000000,1.000000}%
\pgfsetstrokecolor{currentstroke}%
\pgfsetdash{}{0pt}%
\pgfpathmoveto{\pgfqpoint{0.000000in}{0.000000in}}%
\pgfpathlineto{\pgfqpoint{4.509306in}{0.000000in}}%
\pgfpathlineto{\pgfqpoint{4.509306in}{1.754444in}}%
\pgfpathlineto{\pgfqpoint{0.000000in}{1.754444in}}%
\pgfpathlineto{\pgfqpoint{0.000000in}{0.000000in}}%
\pgfpathclose%
\pgfusepath{fill}%
\end{pgfscope}%
\begin{pgfscope}%
\pgfsetbuttcap%
\pgfsetmiterjoin%
\definecolor{currentfill}{rgb}{1.000000,1.000000,1.000000}%
\pgfsetfillcolor{currentfill}%
\pgfsetlinewidth{0.000000pt}%
\definecolor{currentstroke}{rgb}{0.000000,0.000000,0.000000}%
\pgfsetstrokecolor{currentstroke}%
\pgfsetstrokeopacity{0.000000}%
\pgfsetdash{}{0pt}%
\pgfpathmoveto{\pgfqpoint{0.445556in}{0.499444in}}%
\pgfpathlineto{\pgfqpoint{4.320556in}{0.499444in}}%
\pgfpathlineto{\pgfqpoint{4.320556in}{1.654444in}}%
\pgfpathlineto{\pgfqpoint{0.445556in}{1.654444in}}%
\pgfpathlineto{\pgfqpoint{0.445556in}{0.499444in}}%
\pgfpathclose%
\pgfusepath{fill}%
\end{pgfscope}%
\begin{pgfscope}%
\pgfpathrectangle{\pgfqpoint{0.445556in}{0.499444in}}{\pgfqpoint{3.875000in}{1.155000in}}%
\pgfusepath{clip}%
\pgfsetbuttcap%
\pgfsetmiterjoin%
\pgfsetlinewidth{1.003750pt}%
\definecolor{currentstroke}{rgb}{0.000000,0.000000,0.000000}%
\pgfsetstrokecolor{currentstroke}%
\pgfsetdash{}{0pt}%
\pgfpathmoveto{\pgfqpoint{0.435556in}{0.499444in}}%
\pgfpathlineto{\pgfqpoint{0.483922in}{0.499444in}}%
\pgfpathlineto{\pgfqpoint{0.483922in}{0.905316in}}%
\pgfpathlineto{\pgfqpoint{0.435556in}{0.905316in}}%
\pgfusepath{stroke}%
\end{pgfscope}%
\begin{pgfscope}%
\pgfpathrectangle{\pgfqpoint{0.445556in}{0.499444in}}{\pgfqpoint{3.875000in}{1.155000in}}%
\pgfusepath{clip}%
\pgfsetbuttcap%
\pgfsetmiterjoin%
\pgfsetlinewidth{1.003750pt}%
\definecolor{currentstroke}{rgb}{0.000000,0.000000,0.000000}%
\pgfsetstrokecolor{currentstroke}%
\pgfsetdash{}{0pt}%
\pgfpathmoveto{\pgfqpoint{0.576001in}{0.499444in}}%
\pgfpathlineto{\pgfqpoint{0.637387in}{0.499444in}}%
\pgfpathlineto{\pgfqpoint{0.637387in}{1.095204in}}%
\pgfpathlineto{\pgfqpoint{0.576001in}{1.095204in}}%
\pgfpathlineto{\pgfqpoint{0.576001in}{0.499444in}}%
\pgfpathclose%
\pgfusepath{stroke}%
\end{pgfscope}%
\begin{pgfscope}%
\pgfpathrectangle{\pgfqpoint{0.445556in}{0.499444in}}{\pgfqpoint{3.875000in}{1.155000in}}%
\pgfusepath{clip}%
\pgfsetbuttcap%
\pgfsetmiterjoin%
\pgfsetlinewidth{1.003750pt}%
\definecolor{currentstroke}{rgb}{0.000000,0.000000,0.000000}%
\pgfsetstrokecolor{currentstroke}%
\pgfsetdash{}{0pt}%
\pgfpathmoveto{\pgfqpoint{0.729467in}{0.499444in}}%
\pgfpathlineto{\pgfqpoint{0.790853in}{0.499444in}}%
\pgfpathlineto{\pgfqpoint{0.790853in}{1.209750in}}%
\pgfpathlineto{\pgfqpoint{0.729467in}{1.209750in}}%
\pgfpathlineto{\pgfqpoint{0.729467in}{0.499444in}}%
\pgfpathclose%
\pgfusepath{stroke}%
\end{pgfscope}%
\begin{pgfscope}%
\pgfpathrectangle{\pgfqpoint{0.445556in}{0.499444in}}{\pgfqpoint{3.875000in}{1.155000in}}%
\pgfusepath{clip}%
\pgfsetbuttcap%
\pgfsetmiterjoin%
\pgfsetlinewidth{1.003750pt}%
\definecolor{currentstroke}{rgb}{0.000000,0.000000,0.000000}%
\pgfsetstrokecolor{currentstroke}%
\pgfsetdash{}{0pt}%
\pgfpathmoveto{\pgfqpoint{0.882932in}{0.499444in}}%
\pgfpathlineto{\pgfqpoint{0.944318in}{0.499444in}}%
\pgfpathlineto{\pgfqpoint{0.944318in}{1.285327in}}%
\pgfpathlineto{\pgfqpoint{0.882932in}{1.285327in}}%
\pgfpathlineto{\pgfqpoint{0.882932in}{0.499444in}}%
\pgfpathclose%
\pgfusepath{stroke}%
\end{pgfscope}%
\begin{pgfscope}%
\pgfpathrectangle{\pgfqpoint{0.445556in}{0.499444in}}{\pgfqpoint{3.875000in}{1.155000in}}%
\pgfusepath{clip}%
\pgfsetbuttcap%
\pgfsetmiterjoin%
\pgfsetlinewidth{1.003750pt}%
\definecolor{currentstroke}{rgb}{0.000000,0.000000,0.000000}%
\pgfsetstrokecolor{currentstroke}%
\pgfsetdash{}{0pt}%
\pgfpathmoveto{\pgfqpoint{1.036397in}{0.499444in}}%
\pgfpathlineto{\pgfqpoint{1.097783in}{0.499444in}}%
\pgfpathlineto{\pgfqpoint{1.097783in}{1.316621in}}%
\pgfpathlineto{\pgfqpoint{1.036397in}{1.316621in}}%
\pgfpathlineto{\pgfqpoint{1.036397in}{0.499444in}}%
\pgfpathclose%
\pgfusepath{stroke}%
\end{pgfscope}%
\begin{pgfscope}%
\pgfpathrectangle{\pgfqpoint{0.445556in}{0.499444in}}{\pgfqpoint{3.875000in}{1.155000in}}%
\pgfusepath{clip}%
\pgfsetbuttcap%
\pgfsetmiterjoin%
\pgfsetlinewidth{1.003750pt}%
\definecolor{currentstroke}{rgb}{0.000000,0.000000,0.000000}%
\pgfsetstrokecolor{currentstroke}%
\pgfsetdash{}{0pt}%
\pgfpathmoveto{\pgfqpoint{1.189863in}{0.499444in}}%
\pgfpathlineto{\pgfqpoint{1.251249in}{0.499444in}}%
\pgfpathlineto{\pgfqpoint{1.251249in}{1.329256in}}%
\pgfpathlineto{\pgfqpoint{1.189863in}{1.329256in}}%
\pgfpathlineto{\pgfqpoint{1.189863in}{0.499444in}}%
\pgfpathclose%
\pgfusepath{stroke}%
\end{pgfscope}%
\begin{pgfscope}%
\pgfpathrectangle{\pgfqpoint{0.445556in}{0.499444in}}{\pgfqpoint{3.875000in}{1.155000in}}%
\pgfusepath{clip}%
\pgfsetbuttcap%
\pgfsetmiterjoin%
\pgfsetlinewidth{1.003750pt}%
\definecolor{currentstroke}{rgb}{0.000000,0.000000,0.000000}%
\pgfsetstrokecolor{currentstroke}%
\pgfsetdash{}{0pt}%
\pgfpathmoveto{\pgfqpoint{1.343328in}{0.499444in}}%
\pgfpathlineto{\pgfqpoint{1.404714in}{0.499444in}}%
\pgfpathlineto{\pgfqpoint{1.404714in}{1.369643in}}%
\pgfpathlineto{\pgfqpoint{1.343328in}{1.369643in}}%
\pgfpathlineto{\pgfqpoint{1.343328in}{0.499444in}}%
\pgfpathclose%
\pgfusepath{stroke}%
\end{pgfscope}%
\begin{pgfscope}%
\pgfpathrectangle{\pgfqpoint{0.445556in}{0.499444in}}{\pgfqpoint{3.875000in}{1.155000in}}%
\pgfusepath{clip}%
\pgfsetbuttcap%
\pgfsetmiterjoin%
\pgfsetlinewidth{1.003750pt}%
\definecolor{currentstroke}{rgb}{0.000000,0.000000,0.000000}%
\pgfsetstrokecolor{currentstroke}%
\pgfsetdash{}{0pt}%
\pgfpathmoveto{\pgfqpoint{1.496793in}{0.499444in}}%
\pgfpathlineto{\pgfqpoint{1.558179in}{0.499444in}}%
\pgfpathlineto{\pgfqpoint{1.558179in}{1.389836in}}%
\pgfpathlineto{\pgfqpoint{1.496793in}{1.389836in}}%
\pgfpathlineto{\pgfqpoint{1.496793in}{0.499444in}}%
\pgfpathclose%
\pgfusepath{stroke}%
\end{pgfscope}%
\begin{pgfscope}%
\pgfpathrectangle{\pgfqpoint{0.445556in}{0.499444in}}{\pgfqpoint{3.875000in}{1.155000in}}%
\pgfusepath{clip}%
\pgfsetbuttcap%
\pgfsetmiterjoin%
\pgfsetlinewidth{1.003750pt}%
\definecolor{currentstroke}{rgb}{0.000000,0.000000,0.000000}%
\pgfsetstrokecolor{currentstroke}%
\pgfsetdash{}{0pt}%
\pgfpathmoveto{\pgfqpoint{1.650259in}{0.499444in}}%
\pgfpathlineto{\pgfqpoint{1.711645in}{0.499444in}}%
\pgfpathlineto{\pgfqpoint{1.711645in}{1.423728in}}%
\pgfpathlineto{\pgfqpoint{1.650259in}{1.423728in}}%
\pgfpathlineto{\pgfqpoint{1.650259in}{0.499444in}}%
\pgfpathclose%
\pgfusepath{stroke}%
\end{pgfscope}%
\begin{pgfscope}%
\pgfpathrectangle{\pgfqpoint{0.445556in}{0.499444in}}{\pgfqpoint{3.875000in}{1.155000in}}%
\pgfusepath{clip}%
\pgfsetbuttcap%
\pgfsetmiterjoin%
\pgfsetlinewidth{1.003750pt}%
\definecolor{currentstroke}{rgb}{0.000000,0.000000,0.000000}%
\pgfsetstrokecolor{currentstroke}%
\pgfsetdash{}{0pt}%
\pgfpathmoveto{\pgfqpoint{1.803724in}{0.499444in}}%
\pgfpathlineto{\pgfqpoint{1.865110in}{0.499444in}}%
\pgfpathlineto{\pgfqpoint{1.865110in}{1.450534in}}%
\pgfpathlineto{\pgfqpoint{1.803724in}{1.450534in}}%
\pgfpathlineto{\pgfqpoint{1.803724in}{0.499444in}}%
\pgfpathclose%
\pgfusepath{stroke}%
\end{pgfscope}%
\begin{pgfscope}%
\pgfpathrectangle{\pgfqpoint{0.445556in}{0.499444in}}{\pgfqpoint{3.875000in}{1.155000in}}%
\pgfusepath{clip}%
\pgfsetbuttcap%
\pgfsetmiterjoin%
\pgfsetlinewidth{1.003750pt}%
\definecolor{currentstroke}{rgb}{0.000000,0.000000,0.000000}%
\pgfsetstrokecolor{currentstroke}%
\pgfsetdash{}{0pt}%
\pgfpathmoveto{\pgfqpoint{1.957189in}{0.499444in}}%
\pgfpathlineto{\pgfqpoint{2.018575in}{0.499444in}}%
\pgfpathlineto{\pgfqpoint{2.018575in}{1.477104in}}%
\pgfpathlineto{\pgfqpoint{1.957189in}{1.477104in}}%
\pgfpathlineto{\pgfqpoint{1.957189in}{0.499444in}}%
\pgfpathclose%
\pgfusepath{stroke}%
\end{pgfscope}%
\begin{pgfscope}%
\pgfpathrectangle{\pgfqpoint{0.445556in}{0.499444in}}{\pgfqpoint{3.875000in}{1.155000in}}%
\pgfusepath{clip}%
\pgfsetbuttcap%
\pgfsetmiterjoin%
\pgfsetlinewidth{1.003750pt}%
\definecolor{currentstroke}{rgb}{0.000000,0.000000,0.000000}%
\pgfsetstrokecolor{currentstroke}%
\pgfsetdash{}{0pt}%
\pgfpathmoveto{\pgfqpoint{2.110655in}{0.499444in}}%
\pgfpathlineto{\pgfqpoint{2.172041in}{0.499444in}}%
\pgfpathlineto{\pgfqpoint{2.172041in}{1.501312in}}%
\pgfpathlineto{\pgfqpoint{2.110655in}{1.501312in}}%
\pgfpathlineto{\pgfqpoint{2.110655in}{0.499444in}}%
\pgfpathclose%
\pgfusepath{stroke}%
\end{pgfscope}%
\begin{pgfscope}%
\pgfpathrectangle{\pgfqpoint{0.445556in}{0.499444in}}{\pgfqpoint{3.875000in}{1.155000in}}%
\pgfusepath{clip}%
\pgfsetbuttcap%
\pgfsetmiterjoin%
\pgfsetlinewidth{1.003750pt}%
\definecolor{currentstroke}{rgb}{0.000000,0.000000,0.000000}%
\pgfsetstrokecolor{currentstroke}%
\pgfsetdash{}{0pt}%
\pgfpathmoveto{\pgfqpoint{2.264120in}{0.499444in}}%
\pgfpathlineto{\pgfqpoint{2.325506in}{0.499444in}}%
\pgfpathlineto{\pgfqpoint{2.325506in}{1.532960in}}%
\pgfpathlineto{\pgfqpoint{2.264120in}{1.532960in}}%
\pgfpathlineto{\pgfqpoint{2.264120in}{0.499444in}}%
\pgfpathclose%
\pgfusepath{stroke}%
\end{pgfscope}%
\begin{pgfscope}%
\pgfpathrectangle{\pgfqpoint{0.445556in}{0.499444in}}{\pgfqpoint{3.875000in}{1.155000in}}%
\pgfusepath{clip}%
\pgfsetbuttcap%
\pgfsetmiterjoin%
\pgfsetlinewidth{1.003750pt}%
\definecolor{currentstroke}{rgb}{0.000000,0.000000,0.000000}%
\pgfsetstrokecolor{currentstroke}%
\pgfsetdash{}{0pt}%
\pgfpathmoveto{\pgfqpoint{2.417585in}{0.499444in}}%
\pgfpathlineto{\pgfqpoint{2.478972in}{0.499444in}}%
\pgfpathlineto{\pgfqpoint{2.478972in}{1.543352in}}%
\pgfpathlineto{\pgfqpoint{2.417585in}{1.543352in}}%
\pgfpathlineto{\pgfqpoint{2.417585in}{0.499444in}}%
\pgfpathclose%
\pgfusepath{stroke}%
\end{pgfscope}%
\begin{pgfscope}%
\pgfpathrectangle{\pgfqpoint{0.445556in}{0.499444in}}{\pgfqpoint{3.875000in}{1.155000in}}%
\pgfusepath{clip}%
\pgfsetbuttcap%
\pgfsetmiterjoin%
\pgfsetlinewidth{1.003750pt}%
\definecolor{currentstroke}{rgb}{0.000000,0.000000,0.000000}%
\pgfsetstrokecolor{currentstroke}%
\pgfsetdash{}{0pt}%
\pgfpathmoveto{\pgfqpoint{2.571051in}{0.499444in}}%
\pgfpathlineto{\pgfqpoint{2.632437in}{0.499444in}}%
\pgfpathlineto{\pgfqpoint{2.632437in}{1.562718in}}%
\pgfpathlineto{\pgfqpoint{2.571051in}{1.562718in}}%
\pgfpathlineto{\pgfqpoint{2.571051in}{0.499444in}}%
\pgfpathclose%
\pgfusepath{stroke}%
\end{pgfscope}%
\begin{pgfscope}%
\pgfpathrectangle{\pgfqpoint{0.445556in}{0.499444in}}{\pgfqpoint{3.875000in}{1.155000in}}%
\pgfusepath{clip}%
\pgfsetbuttcap%
\pgfsetmiterjoin%
\pgfsetlinewidth{1.003750pt}%
\definecolor{currentstroke}{rgb}{0.000000,0.000000,0.000000}%
\pgfsetstrokecolor{currentstroke}%
\pgfsetdash{}{0pt}%
\pgfpathmoveto{\pgfqpoint{2.724516in}{0.499444in}}%
\pgfpathlineto{\pgfqpoint{2.785902in}{0.499444in}}%
\pgfpathlineto{\pgfqpoint{2.785902in}{1.589052in}}%
\pgfpathlineto{\pgfqpoint{2.724516in}{1.589052in}}%
\pgfpathlineto{\pgfqpoint{2.724516in}{0.499444in}}%
\pgfpathclose%
\pgfusepath{stroke}%
\end{pgfscope}%
\begin{pgfscope}%
\pgfpathrectangle{\pgfqpoint{0.445556in}{0.499444in}}{\pgfqpoint{3.875000in}{1.155000in}}%
\pgfusepath{clip}%
\pgfsetbuttcap%
\pgfsetmiterjoin%
\pgfsetlinewidth{1.003750pt}%
\definecolor{currentstroke}{rgb}{0.000000,0.000000,0.000000}%
\pgfsetstrokecolor{currentstroke}%
\pgfsetdash{}{0pt}%
\pgfpathmoveto{\pgfqpoint{2.877981in}{0.499444in}}%
\pgfpathlineto{\pgfqpoint{2.939368in}{0.499444in}}%
\pgfpathlineto{\pgfqpoint{2.939368in}{1.599444in}}%
\pgfpathlineto{\pgfqpoint{2.877981in}{1.599444in}}%
\pgfpathlineto{\pgfqpoint{2.877981in}{0.499444in}}%
\pgfpathclose%
\pgfusepath{stroke}%
\end{pgfscope}%
\begin{pgfscope}%
\pgfpathrectangle{\pgfqpoint{0.445556in}{0.499444in}}{\pgfqpoint{3.875000in}{1.155000in}}%
\pgfusepath{clip}%
\pgfsetbuttcap%
\pgfsetmiterjoin%
\pgfsetlinewidth{1.003750pt}%
\definecolor{currentstroke}{rgb}{0.000000,0.000000,0.000000}%
\pgfsetstrokecolor{currentstroke}%
\pgfsetdash{}{0pt}%
\pgfpathmoveto{\pgfqpoint{3.031447in}{0.499444in}}%
\pgfpathlineto{\pgfqpoint{3.092833in}{0.499444in}}%
\pgfpathlineto{\pgfqpoint{3.092833in}{1.557759in}}%
\pgfpathlineto{\pgfqpoint{3.031447in}{1.557759in}}%
\pgfpathlineto{\pgfqpoint{3.031447in}{0.499444in}}%
\pgfpathclose%
\pgfusepath{stroke}%
\end{pgfscope}%
\begin{pgfscope}%
\pgfpathrectangle{\pgfqpoint{0.445556in}{0.499444in}}{\pgfqpoint{3.875000in}{1.155000in}}%
\pgfusepath{clip}%
\pgfsetbuttcap%
\pgfsetmiterjoin%
\pgfsetlinewidth{1.003750pt}%
\definecolor{currentstroke}{rgb}{0.000000,0.000000,0.000000}%
\pgfsetstrokecolor{currentstroke}%
\pgfsetdash{}{0pt}%
\pgfpathmoveto{\pgfqpoint{3.184912in}{0.499444in}}%
\pgfpathlineto{\pgfqpoint{3.246298in}{0.499444in}}%
\pgfpathlineto{\pgfqpoint{3.246298in}{1.555043in}}%
\pgfpathlineto{\pgfqpoint{3.184912in}{1.555043in}}%
\pgfpathlineto{\pgfqpoint{3.184912in}{0.499444in}}%
\pgfpathclose%
\pgfusepath{stroke}%
\end{pgfscope}%
\begin{pgfscope}%
\pgfpathrectangle{\pgfqpoint{0.445556in}{0.499444in}}{\pgfqpoint{3.875000in}{1.155000in}}%
\pgfusepath{clip}%
\pgfsetbuttcap%
\pgfsetmiterjoin%
\pgfsetlinewidth{1.003750pt}%
\definecolor{currentstroke}{rgb}{0.000000,0.000000,0.000000}%
\pgfsetstrokecolor{currentstroke}%
\pgfsetdash{}{0pt}%
\pgfpathmoveto{\pgfqpoint{3.338377in}{0.499444in}}%
\pgfpathlineto{\pgfqpoint{3.399764in}{0.499444in}}%
\pgfpathlineto{\pgfqpoint{3.399764in}{1.526111in}}%
\pgfpathlineto{\pgfqpoint{3.338377in}{1.526111in}}%
\pgfpathlineto{\pgfqpoint{3.338377in}{0.499444in}}%
\pgfpathclose%
\pgfusepath{stroke}%
\end{pgfscope}%
\begin{pgfscope}%
\pgfpathrectangle{\pgfqpoint{0.445556in}{0.499444in}}{\pgfqpoint{3.875000in}{1.155000in}}%
\pgfusepath{clip}%
\pgfsetbuttcap%
\pgfsetmiterjoin%
\pgfsetlinewidth{1.003750pt}%
\definecolor{currentstroke}{rgb}{0.000000,0.000000,0.000000}%
\pgfsetstrokecolor{currentstroke}%
\pgfsetdash{}{0pt}%
\pgfpathmoveto{\pgfqpoint{3.491843in}{0.499444in}}%
\pgfpathlineto{\pgfqpoint{3.553229in}{0.499444in}}%
\pgfpathlineto{\pgfqpoint{3.553229in}{1.448999in}}%
\pgfpathlineto{\pgfqpoint{3.491843in}{1.448999in}}%
\pgfpathlineto{\pgfqpoint{3.491843in}{0.499444in}}%
\pgfpathclose%
\pgfusepath{stroke}%
\end{pgfscope}%
\begin{pgfscope}%
\pgfpathrectangle{\pgfqpoint{0.445556in}{0.499444in}}{\pgfqpoint{3.875000in}{1.155000in}}%
\pgfusepath{clip}%
\pgfsetbuttcap%
\pgfsetmiterjoin%
\pgfsetlinewidth{1.003750pt}%
\definecolor{currentstroke}{rgb}{0.000000,0.000000,0.000000}%
\pgfsetstrokecolor{currentstroke}%
\pgfsetdash{}{0pt}%
\pgfpathmoveto{\pgfqpoint{3.645308in}{0.499444in}}%
\pgfpathlineto{\pgfqpoint{3.706694in}{0.499444in}}%
\pgfpathlineto{\pgfqpoint{3.706694in}{1.348032in}}%
\pgfpathlineto{\pgfqpoint{3.645308in}{1.348032in}}%
\pgfpathlineto{\pgfqpoint{3.645308in}{0.499444in}}%
\pgfpathclose%
\pgfusepath{stroke}%
\end{pgfscope}%
\begin{pgfscope}%
\pgfpathrectangle{\pgfqpoint{0.445556in}{0.499444in}}{\pgfqpoint{3.875000in}{1.155000in}}%
\pgfusepath{clip}%
\pgfsetbuttcap%
\pgfsetmiterjoin%
\pgfsetlinewidth{1.003750pt}%
\definecolor{currentstroke}{rgb}{0.000000,0.000000,0.000000}%
\pgfsetstrokecolor{currentstroke}%
\pgfsetdash{}{0pt}%
\pgfpathmoveto{\pgfqpoint{3.798774in}{0.499444in}}%
\pgfpathlineto{\pgfqpoint{3.860160in}{0.499444in}}%
\pgfpathlineto{\pgfqpoint{3.860160in}{1.173143in}}%
\pgfpathlineto{\pgfqpoint{3.798774in}{1.173143in}}%
\pgfpathlineto{\pgfqpoint{3.798774in}{0.499444in}}%
\pgfpathclose%
\pgfusepath{stroke}%
\end{pgfscope}%
\begin{pgfscope}%
\pgfpathrectangle{\pgfqpoint{0.445556in}{0.499444in}}{\pgfqpoint{3.875000in}{1.155000in}}%
\pgfusepath{clip}%
\pgfsetbuttcap%
\pgfsetmiterjoin%
\pgfsetlinewidth{1.003750pt}%
\definecolor{currentstroke}{rgb}{0.000000,0.000000,0.000000}%
\pgfsetstrokecolor{currentstroke}%
\pgfsetdash{}{0pt}%
\pgfpathmoveto{\pgfqpoint{3.952239in}{0.499444in}}%
\pgfpathlineto{\pgfqpoint{4.013625in}{0.499444in}}%
\pgfpathlineto{\pgfqpoint{4.013625in}{0.920432in}}%
\pgfpathlineto{\pgfqpoint{3.952239in}{0.920432in}}%
\pgfpathlineto{\pgfqpoint{3.952239in}{0.499444in}}%
\pgfpathclose%
\pgfusepath{stroke}%
\end{pgfscope}%
\begin{pgfscope}%
\pgfpathrectangle{\pgfqpoint{0.445556in}{0.499444in}}{\pgfqpoint{3.875000in}{1.155000in}}%
\pgfusepath{clip}%
\pgfsetbuttcap%
\pgfsetmiterjoin%
\pgfsetlinewidth{1.003750pt}%
\definecolor{currentstroke}{rgb}{0.000000,0.000000,0.000000}%
\pgfsetstrokecolor{currentstroke}%
\pgfsetdash{}{0pt}%
\pgfpathmoveto{\pgfqpoint{4.105704in}{0.499444in}}%
\pgfpathlineto{\pgfqpoint{4.167090in}{0.499444in}}%
\pgfpathlineto{\pgfqpoint{4.167090in}{0.660400in}}%
\pgfpathlineto{\pgfqpoint{4.105704in}{0.660400in}}%
\pgfpathlineto{\pgfqpoint{4.105704in}{0.499444in}}%
\pgfpathclose%
\pgfusepath{stroke}%
\end{pgfscope}%
\begin{pgfscope}%
\pgfpathrectangle{\pgfqpoint{0.445556in}{0.499444in}}{\pgfqpoint{3.875000in}{1.155000in}}%
\pgfusepath{clip}%
\pgfsetbuttcap%
\pgfsetmiterjoin%
\definecolor{currentfill}{rgb}{0.000000,0.000000,0.000000}%
\pgfsetfillcolor{currentfill}%
\pgfsetlinewidth{0.000000pt}%
\definecolor{currentstroke}{rgb}{0.000000,0.000000,0.000000}%
\pgfsetstrokecolor{currentstroke}%
\pgfsetstrokeopacity{0.000000}%
\pgfsetdash{}{0pt}%
\pgfpathmoveto{\pgfqpoint{0.483922in}{0.499444in}}%
\pgfpathlineto{\pgfqpoint{0.545308in}{0.499444in}}%
\pgfpathlineto{\pgfqpoint{0.545308in}{0.502514in}}%
\pgfpathlineto{\pgfqpoint{0.483922in}{0.502514in}}%
\pgfpathlineto{\pgfqpoint{0.483922in}{0.499444in}}%
\pgfpathclose%
\pgfusepath{fill}%
\end{pgfscope}%
\begin{pgfscope}%
\pgfpathrectangle{\pgfqpoint{0.445556in}{0.499444in}}{\pgfqpoint{3.875000in}{1.155000in}}%
\pgfusepath{clip}%
\pgfsetbuttcap%
\pgfsetmiterjoin%
\definecolor{currentfill}{rgb}{0.000000,0.000000,0.000000}%
\pgfsetfillcolor{currentfill}%
\pgfsetlinewidth{0.000000pt}%
\definecolor{currentstroke}{rgb}{0.000000,0.000000,0.000000}%
\pgfsetstrokecolor{currentstroke}%
\pgfsetstrokeopacity{0.000000}%
\pgfsetdash{}{0pt}%
\pgfpathmoveto{\pgfqpoint{0.637387in}{0.499444in}}%
\pgfpathlineto{\pgfqpoint{0.698774in}{0.499444in}}%
\pgfpathlineto{\pgfqpoint{0.698774in}{0.506411in}}%
\pgfpathlineto{\pgfqpoint{0.637387in}{0.506411in}}%
\pgfpathlineto{\pgfqpoint{0.637387in}{0.499444in}}%
\pgfpathclose%
\pgfusepath{fill}%
\end{pgfscope}%
\begin{pgfscope}%
\pgfpathrectangle{\pgfqpoint{0.445556in}{0.499444in}}{\pgfqpoint{3.875000in}{1.155000in}}%
\pgfusepath{clip}%
\pgfsetbuttcap%
\pgfsetmiterjoin%
\definecolor{currentfill}{rgb}{0.000000,0.000000,0.000000}%
\pgfsetfillcolor{currentfill}%
\pgfsetlinewidth{0.000000pt}%
\definecolor{currentstroke}{rgb}{0.000000,0.000000,0.000000}%
\pgfsetstrokecolor{currentstroke}%
\pgfsetstrokeopacity{0.000000}%
\pgfsetdash{}{0pt}%
\pgfpathmoveto{\pgfqpoint{0.790853in}{0.499444in}}%
\pgfpathlineto{\pgfqpoint{0.852239in}{0.499444in}}%
\pgfpathlineto{\pgfqpoint{0.852239in}{0.511489in}}%
\pgfpathlineto{\pgfqpoint{0.790853in}{0.511489in}}%
\pgfpathlineto{\pgfqpoint{0.790853in}{0.499444in}}%
\pgfpathclose%
\pgfusepath{fill}%
\end{pgfscope}%
\begin{pgfscope}%
\pgfpathrectangle{\pgfqpoint{0.445556in}{0.499444in}}{\pgfqpoint{3.875000in}{1.155000in}}%
\pgfusepath{clip}%
\pgfsetbuttcap%
\pgfsetmiterjoin%
\definecolor{currentfill}{rgb}{0.000000,0.000000,0.000000}%
\pgfsetfillcolor{currentfill}%
\pgfsetlinewidth{0.000000pt}%
\definecolor{currentstroke}{rgb}{0.000000,0.000000,0.000000}%
\pgfsetstrokecolor{currentstroke}%
\pgfsetstrokeopacity{0.000000}%
\pgfsetdash{}{0pt}%
\pgfpathmoveto{\pgfqpoint{0.944318in}{0.499444in}}%
\pgfpathlineto{\pgfqpoint{1.005704in}{0.499444in}}%
\pgfpathlineto{\pgfqpoint{1.005704in}{0.515268in}}%
\pgfpathlineto{\pgfqpoint{0.944318in}{0.515268in}}%
\pgfpathlineto{\pgfqpoint{0.944318in}{0.499444in}}%
\pgfpathclose%
\pgfusepath{fill}%
\end{pgfscope}%
\begin{pgfscope}%
\pgfpathrectangle{\pgfqpoint{0.445556in}{0.499444in}}{\pgfqpoint{3.875000in}{1.155000in}}%
\pgfusepath{clip}%
\pgfsetbuttcap%
\pgfsetmiterjoin%
\definecolor{currentfill}{rgb}{0.000000,0.000000,0.000000}%
\pgfsetfillcolor{currentfill}%
\pgfsetlinewidth{0.000000pt}%
\definecolor{currentstroke}{rgb}{0.000000,0.000000,0.000000}%
\pgfsetstrokecolor{currentstroke}%
\pgfsetstrokeopacity{0.000000}%
\pgfsetdash{}{0pt}%
\pgfpathmoveto{\pgfqpoint{1.097783in}{0.499444in}}%
\pgfpathlineto{\pgfqpoint{1.159170in}{0.499444in}}%
\pgfpathlineto{\pgfqpoint{1.159170in}{0.523416in}}%
\pgfpathlineto{\pgfqpoint{1.097783in}{0.523416in}}%
\pgfpathlineto{\pgfqpoint{1.097783in}{0.499444in}}%
\pgfpathclose%
\pgfusepath{fill}%
\end{pgfscope}%
\begin{pgfscope}%
\pgfpathrectangle{\pgfqpoint{0.445556in}{0.499444in}}{\pgfqpoint{3.875000in}{1.155000in}}%
\pgfusepath{clip}%
\pgfsetbuttcap%
\pgfsetmiterjoin%
\definecolor{currentfill}{rgb}{0.000000,0.000000,0.000000}%
\pgfsetfillcolor{currentfill}%
\pgfsetlinewidth{0.000000pt}%
\definecolor{currentstroke}{rgb}{0.000000,0.000000,0.000000}%
\pgfsetstrokecolor{currentstroke}%
\pgfsetstrokeopacity{0.000000}%
\pgfsetdash{}{0pt}%
\pgfpathmoveto{\pgfqpoint{1.251249in}{0.499444in}}%
\pgfpathlineto{\pgfqpoint{1.312635in}{0.499444in}}%
\pgfpathlineto{\pgfqpoint{1.312635in}{0.530029in}}%
\pgfpathlineto{\pgfqpoint{1.251249in}{0.530029in}}%
\pgfpathlineto{\pgfqpoint{1.251249in}{0.499444in}}%
\pgfpathclose%
\pgfusepath{fill}%
\end{pgfscope}%
\begin{pgfscope}%
\pgfpathrectangle{\pgfqpoint{0.445556in}{0.499444in}}{\pgfqpoint{3.875000in}{1.155000in}}%
\pgfusepath{clip}%
\pgfsetbuttcap%
\pgfsetmiterjoin%
\definecolor{currentfill}{rgb}{0.000000,0.000000,0.000000}%
\pgfsetfillcolor{currentfill}%
\pgfsetlinewidth{0.000000pt}%
\definecolor{currentstroke}{rgb}{0.000000,0.000000,0.000000}%
\pgfsetstrokecolor{currentstroke}%
\pgfsetstrokeopacity{0.000000}%
\pgfsetdash{}{0pt}%
\pgfpathmoveto{\pgfqpoint{1.404714in}{0.499444in}}%
\pgfpathlineto{\pgfqpoint{1.466100in}{0.499444in}}%
\pgfpathlineto{\pgfqpoint{1.466100in}{0.542547in}}%
\pgfpathlineto{\pgfqpoint{1.404714in}{0.542547in}}%
\pgfpathlineto{\pgfqpoint{1.404714in}{0.499444in}}%
\pgfpathclose%
\pgfusepath{fill}%
\end{pgfscope}%
\begin{pgfscope}%
\pgfpathrectangle{\pgfqpoint{0.445556in}{0.499444in}}{\pgfqpoint{3.875000in}{1.155000in}}%
\pgfusepath{clip}%
\pgfsetbuttcap%
\pgfsetmiterjoin%
\definecolor{currentfill}{rgb}{0.000000,0.000000,0.000000}%
\pgfsetfillcolor{currentfill}%
\pgfsetlinewidth{0.000000pt}%
\definecolor{currentstroke}{rgb}{0.000000,0.000000,0.000000}%
\pgfsetstrokecolor{currentstroke}%
\pgfsetstrokeopacity{0.000000}%
\pgfsetdash{}{0pt}%
\pgfpathmoveto{\pgfqpoint{1.558179in}{0.499444in}}%
\pgfpathlineto{\pgfqpoint{1.619566in}{0.499444in}}%
\pgfpathlineto{\pgfqpoint{1.619566in}{0.543019in}}%
\pgfpathlineto{\pgfqpoint{1.558179in}{0.543019in}}%
\pgfpathlineto{\pgfqpoint{1.558179in}{0.499444in}}%
\pgfpathclose%
\pgfusepath{fill}%
\end{pgfscope}%
\begin{pgfscope}%
\pgfpathrectangle{\pgfqpoint{0.445556in}{0.499444in}}{\pgfqpoint{3.875000in}{1.155000in}}%
\pgfusepath{clip}%
\pgfsetbuttcap%
\pgfsetmiterjoin%
\definecolor{currentfill}{rgb}{0.000000,0.000000,0.000000}%
\pgfsetfillcolor{currentfill}%
\pgfsetlinewidth{0.000000pt}%
\definecolor{currentstroke}{rgb}{0.000000,0.000000,0.000000}%
\pgfsetstrokecolor{currentstroke}%
\pgfsetstrokeopacity{0.000000}%
\pgfsetdash{}{0pt}%
\pgfpathmoveto{\pgfqpoint{1.711645in}{0.499444in}}%
\pgfpathlineto{\pgfqpoint{1.773031in}{0.499444in}}%
\pgfpathlineto{\pgfqpoint{1.773031in}{0.552230in}}%
\pgfpathlineto{\pgfqpoint{1.711645in}{0.552230in}}%
\pgfpathlineto{\pgfqpoint{1.711645in}{0.499444in}}%
\pgfpathclose%
\pgfusepath{fill}%
\end{pgfscope}%
\begin{pgfscope}%
\pgfpathrectangle{\pgfqpoint{0.445556in}{0.499444in}}{\pgfqpoint{3.875000in}{1.155000in}}%
\pgfusepath{clip}%
\pgfsetbuttcap%
\pgfsetmiterjoin%
\definecolor{currentfill}{rgb}{0.000000,0.000000,0.000000}%
\pgfsetfillcolor{currentfill}%
\pgfsetlinewidth{0.000000pt}%
\definecolor{currentstroke}{rgb}{0.000000,0.000000,0.000000}%
\pgfsetstrokecolor{currentstroke}%
\pgfsetstrokeopacity{0.000000}%
\pgfsetdash{}{0pt}%
\pgfpathmoveto{\pgfqpoint{1.865110in}{0.499444in}}%
\pgfpathlineto{\pgfqpoint{1.926496in}{0.499444in}}%
\pgfpathlineto{\pgfqpoint{1.926496in}{0.567463in}}%
\pgfpathlineto{\pgfqpoint{1.865110in}{0.567463in}}%
\pgfpathlineto{\pgfqpoint{1.865110in}{0.499444in}}%
\pgfpathclose%
\pgfusepath{fill}%
\end{pgfscope}%
\begin{pgfscope}%
\pgfpathrectangle{\pgfqpoint{0.445556in}{0.499444in}}{\pgfqpoint{3.875000in}{1.155000in}}%
\pgfusepath{clip}%
\pgfsetbuttcap%
\pgfsetmiterjoin%
\definecolor{currentfill}{rgb}{0.000000,0.000000,0.000000}%
\pgfsetfillcolor{currentfill}%
\pgfsetlinewidth{0.000000pt}%
\definecolor{currentstroke}{rgb}{0.000000,0.000000,0.000000}%
\pgfsetstrokecolor{currentstroke}%
\pgfsetstrokeopacity{0.000000}%
\pgfsetdash{}{0pt}%
\pgfpathmoveto{\pgfqpoint{2.018575in}{0.499444in}}%
\pgfpathlineto{\pgfqpoint{2.079962in}{0.499444in}}%
\pgfpathlineto{\pgfqpoint{2.079962in}{0.581162in}}%
\pgfpathlineto{\pgfqpoint{2.018575in}{0.581162in}}%
\pgfpathlineto{\pgfqpoint{2.018575in}{0.499444in}}%
\pgfpathclose%
\pgfusepath{fill}%
\end{pgfscope}%
\begin{pgfscope}%
\pgfpathrectangle{\pgfqpoint{0.445556in}{0.499444in}}{\pgfqpoint{3.875000in}{1.155000in}}%
\pgfusepath{clip}%
\pgfsetbuttcap%
\pgfsetmiterjoin%
\definecolor{currentfill}{rgb}{0.000000,0.000000,0.000000}%
\pgfsetfillcolor{currentfill}%
\pgfsetlinewidth{0.000000pt}%
\definecolor{currentstroke}{rgb}{0.000000,0.000000,0.000000}%
\pgfsetstrokecolor{currentstroke}%
\pgfsetstrokeopacity{0.000000}%
\pgfsetdash{}{0pt}%
\pgfpathmoveto{\pgfqpoint{2.172041in}{0.499444in}}%
\pgfpathlineto{\pgfqpoint{2.233427in}{0.499444in}}%
\pgfpathlineto{\pgfqpoint{2.233427in}{0.590255in}}%
\pgfpathlineto{\pgfqpoint{2.172041in}{0.590255in}}%
\pgfpathlineto{\pgfqpoint{2.172041in}{0.499444in}}%
\pgfpathclose%
\pgfusepath{fill}%
\end{pgfscope}%
\begin{pgfscope}%
\pgfpathrectangle{\pgfqpoint{0.445556in}{0.499444in}}{\pgfqpoint{3.875000in}{1.155000in}}%
\pgfusepath{clip}%
\pgfsetbuttcap%
\pgfsetmiterjoin%
\definecolor{currentfill}{rgb}{0.000000,0.000000,0.000000}%
\pgfsetfillcolor{currentfill}%
\pgfsetlinewidth{0.000000pt}%
\definecolor{currentstroke}{rgb}{0.000000,0.000000,0.000000}%
\pgfsetstrokecolor{currentstroke}%
\pgfsetstrokeopacity{0.000000}%
\pgfsetdash{}{0pt}%
\pgfpathmoveto{\pgfqpoint{2.325506in}{0.499444in}}%
\pgfpathlineto{\pgfqpoint{2.386892in}{0.499444in}}%
\pgfpathlineto{\pgfqpoint{2.386892in}{0.604189in}}%
\pgfpathlineto{\pgfqpoint{2.325506in}{0.604189in}}%
\pgfpathlineto{\pgfqpoint{2.325506in}{0.499444in}}%
\pgfpathclose%
\pgfusepath{fill}%
\end{pgfscope}%
\begin{pgfscope}%
\pgfpathrectangle{\pgfqpoint{0.445556in}{0.499444in}}{\pgfqpoint{3.875000in}{1.155000in}}%
\pgfusepath{clip}%
\pgfsetbuttcap%
\pgfsetmiterjoin%
\definecolor{currentfill}{rgb}{0.000000,0.000000,0.000000}%
\pgfsetfillcolor{currentfill}%
\pgfsetlinewidth{0.000000pt}%
\definecolor{currentstroke}{rgb}{0.000000,0.000000,0.000000}%
\pgfsetstrokecolor{currentstroke}%
\pgfsetstrokeopacity{0.000000}%
\pgfsetdash{}{0pt}%
\pgfpathmoveto{\pgfqpoint{2.478972in}{0.499444in}}%
\pgfpathlineto{\pgfqpoint{2.540358in}{0.499444in}}%
\pgfpathlineto{\pgfqpoint{2.540358in}{0.622965in}}%
\pgfpathlineto{\pgfqpoint{2.478972in}{0.622965in}}%
\pgfpathlineto{\pgfqpoint{2.478972in}{0.499444in}}%
\pgfpathclose%
\pgfusepath{fill}%
\end{pgfscope}%
\begin{pgfscope}%
\pgfpathrectangle{\pgfqpoint{0.445556in}{0.499444in}}{\pgfqpoint{3.875000in}{1.155000in}}%
\pgfusepath{clip}%
\pgfsetbuttcap%
\pgfsetmiterjoin%
\definecolor{currentfill}{rgb}{0.000000,0.000000,0.000000}%
\pgfsetfillcolor{currentfill}%
\pgfsetlinewidth{0.000000pt}%
\definecolor{currentstroke}{rgb}{0.000000,0.000000,0.000000}%
\pgfsetstrokecolor{currentstroke}%
\pgfsetstrokeopacity{0.000000}%
\pgfsetdash{}{0pt}%
\pgfpathmoveto{\pgfqpoint{2.632437in}{0.499444in}}%
\pgfpathlineto{\pgfqpoint{2.693823in}{0.499444in}}%
\pgfpathlineto{\pgfqpoint{2.693823in}{0.647292in}}%
\pgfpathlineto{\pgfqpoint{2.632437in}{0.647292in}}%
\pgfpathlineto{\pgfqpoint{2.632437in}{0.499444in}}%
\pgfpathclose%
\pgfusepath{fill}%
\end{pgfscope}%
\begin{pgfscope}%
\pgfpathrectangle{\pgfqpoint{0.445556in}{0.499444in}}{\pgfqpoint{3.875000in}{1.155000in}}%
\pgfusepath{clip}%
\pgfsetbuttcap%
\pgfsetmiterjoin%
\definecolor{currentfill}{rgb}{0.000000,0.000000,0.000000}%
\pgfsetfillcolor{currentfill}%
\pgfsetlinewidth{0.000000pt}%
\definecolor{currentstroke}{rgb}{0.000000,0.000000,0.000000}%
\pgfsetstrokecolor{currentstroke}%
\pgfsetstrokeopacity{0.000000}%
\pgfsetdash{}{0pt}%
\pgfpathmoveto{\pgfqpoint{2.785902in}{0.499444in}}%
\pgfpathlineto{\pgfqpoint{2.847288in}{0.499444in}}%
\pgfpathlineto{\pgfqpoint{2.847288in}{0.671264in}}%
\pgfpathlineto{\pgfqpoint{2.785902in}{0.671264in}}%
\pgfpathlineto{\pgfqpoint{2.785902in}{0.499444in}}%
\pgfpathclose%
\pgfusepath{fill}%
\end{pgfscope}%
\begin{pgfscope}%
\pgfpathrectangle{\pgfqpoint{0.445556in}{0.499444in}}{\pgfqpoint{3.875000in}{1.155000in}}%
\pgfusepath{clip}%
\pgfsetbuttcap%
\pgfsetmiterjoin%
\definecolor{currentfill}{rgb}{0.000000,0.000000,0.000000}%
\pgfsetfillcolor{currentfill}%
\pgfsetlinewidth{0.000000pt}%
\definecolor{currentstroke}{rgb}{0.000000,0.000000,0.000000}%
\pgfsetstrokecolor{currentstroke}%
\pgfsetstrokeopacity{0.000000}%
\pgfsetdash{}{0pt}%
\pgfpathmoveto{\pgfqpoint{2.939368in}{0.499444in}}%
\pgfpathlineto{\pgfqpoint{3.000754in}{0.499444in}}%
\pgfpathlineto{\pgfqpoint{3.000754in}{0.687442in}}%
\pgfpathlineto{\pgfqpoint{2.939368in}{0.687442in}}%
\pgfpathlineto{\pgfqpoint{2.939368in}{0.499444in}}%
\pgfpathclose%
\pgfusepath{fill}%
\end{pgfscope}%
\begin{pgfscope}%
\pgfpathrectangle{\pgfqpoint{0.445556in}{0.499444in}}{\pgfqpoint{3.875000in}{1.155000in}}%
\pgfusepath{clip}%
\pgfsetbuttcap%
\pgfsetmiterjoin%
\definecolor{currentfill}{rgb}{0.000000,0.000000,0.000000}%
\pgfsetfillcolor{currentfill}%
\pgfsetlinewidth{0.000000pt}%
\definecolor{currentstroke}{rgb}{0.000000,0.000000,0.000000}%
\pgfsetstrokecolor{currentstroke}%
\pgfsetstrokeopacity{0.000000}%
\pgfsetdash{}{0pt}%
\pgfpathmoveto{\pgfqpoint{3.092833in}{0.499444in}}%
\pgfpathlineto{\pgfqpoint{3.154219in}{0.499444in}}%
\pgfpathlineto{\pgfqpoint{3.154219in}{0.722869in}}%
\pgfpathlineto{\pgfqpoint{3.092833in}{0.722869in}}%
\pgfpathlineto{\pgfqpoint{3.092833in}{0.499444in}}%
\pgfpathclose%
\pgfusepath{fill}%
\end{pgfscope}%
\begin{pgfscope}%
\pgfpathrectangle{\pgfqpoint{0.445556in}{0.499444in}}{\pgfqpoint{3.875000in}{1.155000in}}%
\pgfusepath{clip}%
\pgfsetbuttcap%
\pgfsetmiterjoin%
\definecolor{currentfill}{rgb}{0.000000,0.000000,0.000000}%
\pgfsetfillcolor{currentfill}%
\pgfsetlinewidth{0.000000pt}%
\definecolor{currentstroke}{rgb}{0.000000,0.000000,0.000000}%
\pgfsetstrokecolor{currentstroke}%
\pgfsetstrokeopacity{0.000000}%
\pgfsetdash{}{0pt}%
\pgfpathmoveto{\pgfqpoint{3.246298in}{0.499444in}}%
\pgfpathlineto{\pgfqpoint{3.307684in}{0.499444in}}%
\pgfpathlineto{\pgfqpoint{3.307684in}{0.758059in}}%
\pgfpathlineto{\pgfqpoint{3.246298in}{0.758059in}}%
\pgfpathlineto{\pgfqpoint{3.246298in}{0.499444in}}%
\pgfpathclose%
\pgfusepath{fill}%
\end{pgfscope}%
\begin{pgfscope}%
\pgfpathrectangle{\pgfqpoint{0.445556in}{0.499444in}}{\pgfqpoint{3.875000in}{1.155000in}}%
\pgfusepath{clip}%
\pgfsetbuttcap%
\pgfsetmiterjoin%
\definecolor{currentfill}{rgb}{0.000000,0.000000,0.000000}%
\pgfsetfillcolor{currentfill}%
\pgfsetlinewidth{0.000000pt}%
\definecolor{currentstroke}{rgb}{0.000000,0.000000,0.000000}%
\pgfsetstrokecolor{currentstroke}%
\pgfsetstrokeopacity{0.000000}%
\pgfsetdash{}{0pt}%
\pgfpathmoveto{\pgfqpoint{3.399764in}{0.499444in}}%
\pgfpathlineto{\pgfqpoint{3.461150in}{0.499444in}}%
\pgfpathlineto{\pgfqpoint{3.461150in}{0.806476in}}%
\pgfpathlineto{\pgfqpoint{3.399764in}{0.806476in}}%
\pgfpathlineto{\pgfqpoint{3.399764in}{0.499444in}}%
\pgfpathclose%
\pgfusepath{fill}%
\end{pgfscope}%
\begin{pgfscope}%
\pgfpathrectangle{\pgfqpoint{0.445556in}{0.499444in}}{\pgfqpoint{3.875000in}{1.155000in}}%
\pgfusepath{clip}%
\pgfsetbuttcap%
\pgfsetmiterjoin%
\definecolor{currentfill}{rgb}{0.000000,0.000000,0.000000}%
\pgfsetfillcolor{currentfill}%
\pgfsetlinewidth{0.000000pt}%
\definecolor{currentstroke}{rgb}{0.000000,0.000000,0.000000}%
\pgfsetstrokecolor{currentstroke}%
\pgfsetstrokeopacity{0.000000}%
\pgfsetdash{}{0pt}%
\pgfpathmoveto{\pgfqpoint{3.553229in}{0.499444in}}%
\pgfpathlineto{\pgfqpoint{3.614615in}{0.499444in}}%
\pgfpathlineto{\pgfqpoint{3.614615in}{0.846862in}}%
\pgfpathlineto{\pgfqpoint{3.553229in}{0.846862in}}%
\pgfpathlineto{\pgfqpoint{3.553229in}{0.499444in}}%
\pgfpathclose%
\pgfusepath{fill}%
\end{pgfscope}%
\begin{pgfscope}%
\pgfpathrectangle{\pgfqpoint{0.445556in}{0.499444in}}{\pgfqpoint{3.875000in}{1.155000in}}%
\pgfusepath{clip}%
\pgfsetbuttcap%
\pgfsetmiterjoin%
\definecolor{currentfill}{rgb}{0.000000,0.000000,0.000000}%
\pgfsetfillcolor{currentfill}%
\pgfsetlinewidth{0.000000pt}%
\definecolor{currentstroke}{rgb}{0.000000,0.000000,0.000000}%
\pgfsetstrokecolor{currentstroke}%
\pgfsetstrokeopacity{0.000000}%
\pgfsetdash{}{0pt}%
\pgfpathmoveto{\pgfqpoint{3.706694in}{0.499444in}}%
\pgfpathlineto{\pgfqpoint{3.768080in}{0.499444in}}%
\pgfpathlineto{\pgfqpoint{3.768080in}{0.900947in}}%
\pgfpathlineto{\pgfqpoint{3.706694in}{0.900947in}}%
\pgfpathlineto{\pgfqpoint{3.706694in}{0.499444in}}%
\pgfpathclose%
\pgfusepath{fill}%
\end{pgfscope}%
\begin{pgfscope}%
\pgfpathrectangle{\pgfqpoint{0.445556in}{0.499444in}}{\pgfqpoint{3.875000in}{1.155000in}}%
\pgfusepath{clip}%
\pgfsetbuttcap%
\pgfsetmiterjoin%
\definecolor{currentfill}{rgb}{0.000000,0.000000,0.000000}%
\pgfsetfillcolor{currentfill}%
\pgfsetlinewidth{0.000000pt}%
\definecolor{currentstroke}{rgb}{0.000000,0.000000,0.000000}%
\pgfsetstrokecolor{currentstroke}%
\pgfsetstrokeopacity{0.000000}%
\pgfsetdash{}{0pt}%
\pgfpathmoveto{\pgfqpoint{3.860160in}{0.499444in}}%
\pgfpathlineto{\pgfqpoint{3.921546in}{0.499444in}}%
\pgfpathlineto{\pgfqpoint{3.921546in}{0.937200in}}%
\pgfpathlineto{\pgfqpoint{3.860160in}{0.937200in}}%
\pgfpathlineto{\pgfqpoint{3.860160in}{0.499444in}}%
\pgfpathclose%
\pgfusepath{fill}%
\end{pgfscope}%
\begin{pgfscope}%
\pgfpathrectangle{\pgfqpoint{0.445556in}{0.499444in}}{\pgfqpoint{3.875000in}{1.155000in}}%
\pgfusepath{clip}%
\pgfsetbuttcap%
\pgfsetmiterjoin%
\definecolor{currentfill}{rgb}{0.000000,0.000000,0.000000}%
\pgfsetfillcolor{currentfill}%
\pgfsetlinewidth{0.000000pt}%
\definecolor{currentstroke}{rgb}{0.000000,0.000000,0.000000}%
\pgfsetstrokecolor{currentstroke}%
\pgfsetstrokeopacity{0.000000}%
\pgfsetdash{}{0pt}%
\pgfpathmoveto{\pgfqpoint{4.013625in}{0.499444in}}%
\pgfpathlineto{\pgfqpoint{4.075011in}{0.499444in}}%
\pgfpathlineto{\pgfqpoint{4.075011in}{0.945112in}}%
\pgfpathlineto{\pgfqpoint{4.013625in}{0.945112in}}%
\pgfpathlineto{\pgfqpoint{4.013625in}{0.499444in}}%
\pgfpathclose%
\pgfusepath{fill}%
\end{pgfscope}%
\begin{pgfscope}%
\pgfpathrectangle{\pgfqpoint{0.445556in}{0.499444in}}{\pgfqpoint{3.875000in}{1.155000in}}%
\pgfusepath{clip}%
\pgfsetbuttcap%
\pgfsetmiterjoin%
\definecolor{currentfill}{rgb}{0.000000,0.000000,0.000000}%
\pgfsetfillcolor{currentfill}%
\pgfsetlinewidth{0.000000pt}%
\definecolor{currentstroke}{rgb}{0.000000,0.000000,0.000000}%
\pgfsetstrokecolor{currentstroke}%
\pgfsetstrokeopacity{0.000000}%
\pgfsetdash{}{0pt}%
\pgfpathmoveto{\pgfqpoint{4.167090in}{0.499444in}}%
\pgfpathlineto{\pgfqpoint{4.228476in}{0.499444in}}%
\pgfpathlineto{\pgfqpoint{4.228476in}{0.863985in}}%
\pgfpathlineto{\pgfqpoint{4.167090in}{0.863985in}}%
\pgfpathlineto{\pgfqpoint{4.167090in}{0.499444in}}%
\pgfpathclose%
\pgfusepath{fill}%
\end{pgfscope}%
\begin{pgfscope}%
\pgfsetbuttcap%
\pgfsetroundjoin%
\definecolor{currentfill}{rgb}{0.000000,0.000000,0.000000}%
\pgfsetfillcolor{currentfill}%
\pgfsetlinewidth{0.803000pt}%
\definecolor{currentstroke}{rgb}{0.000000,0.000000,0.000000}%
\pgfsetstrokecolor{currentstroke}%
\pgfsetdash{}{0pt}%
\pgfsys@defobject{currentmarker}{\pgfqpoint{0.000000in}{-0.048611in}}{\pgfqpoint{0.000000in}{0.000000in}}{%
\pgfpathmoveto{\pgfqpoint{0.000000in}{0.000000in}}%
\pgfpathlineto{\pgfqpoint{0.000000in}{-0.048611in}}%
\pgfusepath{stroke,fill}%
}%
\begin{pgfscope}%
\pgfsys@transformshift{0.483922in}{0.499444in}%
\pgfsys@useobject{currentmarker}{}%
\end{pgfscope}%
\end{pgfscope}%
\begin{pgfscope}%
\definecolor{textcolor}{rgb}{0.000000,0.000000,0.000000}%
\pgfsetstrokecolor{textcolor}%
\pgfsetfillcolor{textcolor}%
\pgftext[x=0.483922in,y=0.402222in,,top]{\color{textcolor}\rmfamily\fontsize{10.000000}{12.000000}\selectfont 0.0}%
\end{pgfscope}%
\begin{pgfscope}%
\pgfsetbuttcap%
\pgfsetroundjoin%
\definecolor{currentfill}{rgb}{0.000000,0.000000,0.000000}%
\pgfsetfillcolor{currentfill}%
\pgfsetlinewidth{0.803000pt}%
\definecolor{currentstroke}{rgb}{0.000000,0.000000,0.000000}%
\pgfsetstrokecolor{currentstroke}%
\pgfsetdash{}{0pt}%
\pgfsys@defobject{currentmarker}{\pgfqpoint{0.000000in}{-0.048611in}}{\pgfqpoint{0.000000in}{0.000000in}}{%
\pgfpathmoveto{\pgfqpoint{0.000000in}{0.000000in}}%
\pgfpathlineto{\pgfqpoint{0.000000in}{-0.048611in}}%
\pgfusepath{stroke,fill}%
}%
\begin{pgfscope}%
\pgfsys@transformshift{0.867585in}{0.499444in}%
\pgfsys@useobject{currentmarker}{}%
\end{pgfscope}%
\end{pgfscope}%
\begin{pgfscope}%
\definecolor{textcolor}{rgb}{0.000000,0.000000,0.000000}%
\pgfsetstrokecolor{textcolor}%
\pgfsetfillcolor{textcolor}%
\pgftext[x=0.867585in,y=0.402222in,,top]{\color{textcolor}\rmfamily\fontsize{10.000000}{12.000000}\selectfont 0.1}%
\end{pgfscope}%
\begin{pgfscope}%
\pgfsetbuttcap%
\pgfsetroundjoin%
\definecolor{currentfill}{rgb}{0.000000,0.000000,0.000000}%
\pgfsetfillcolor{currentfill}%
\pgfsetlinewidth{0.803000pt}%
\definecolor{currentstroke}{rgb}{0.000000,0.000000,0.000000}%
\pgfsetstrokecolor{currentstroke}%
\pgfsetdash{}{0pt}%
\pgfsys@defobject{currentmarker}{\pgfqpoint{0.000000in}{-0.048611in}}{\pgfqpoint{0.000000in}{0.000000in}}{%
\pgfpathmoveto{\pgfqpoint{0.000000in}{0.000000in}}%
\pgfpathlineto{\pgfqpoint{0.000000in}{-0.048611in}}%
\pgfusepath{stroke,fill}%
}%
\begin{pgfscope}%
\pgfsys@transformshift{1.251249in}{0.499444in}%
\pgfsys@useobject{currentmarker}{}%
\end{pgfscope}%
\end{pgfscope}%
\begin{pgfscope}%
\definecolor{textcolor}{rgb}{0.000000,0.000000,0.000000}%
\pgfsetstrokecolor{textcolor}%
\pgfsetfillcolor{textcolor}%
\pgftext[x=1.251249in,y=0.402222in,,top]{\color{textcolor}\rmfamily\fontsize{10.000000}{12.000000}\selectfont 0.2}%
\end{pgfscope}%
\begin{pgfscope}%
\pgfsetbuttcap%
\pgfsetroundjoin%
\definecolor{currentfill}{rgb}{0.000000,0.000000,0.000000}%
\pgfsetfillcolor{currentfill}%
\pgfsetlinewidth{0.803000pt}%
\definecolor{currentstroke}{rgb}{0.000000,0.000000,0.000000}%
\pgfsetstrokecolor{currentstroke}%
\pgfsetdash{}{0pt}%
\pgfsys@defobject{currentmarker}{\pgfqpoint{0.000000in}{-0.048611in}}{\pgfqpoint{0.000000in}{0.000000in}}{%
\pgfpathmoveto{\pgfqpoint{0.000000in}{0.000000in}}%
\pgfpathlineto{\pgfqpoint{0.000000in}{-0.048611in}}%
\pgfusepath{stroke,fill}%
}%
\begin{pgfscope}%
\pgfsys@transformshift{1.634912in}{0.499444in}%
\pgfsys@useobject{currentmarker}{}%
\end{pgfscope}%
\end{pgfscope}%
\begin{pgfscope}%
\definecolor{textcolor}{rgb}{0.000000,0.000000,0.000000}%
\pgfsetstrokecolor{textcolor}%
\pgfsetfillcolor{textcolor}%
\pgftext[x=1.634912in,y=0.402222in,,top]{\color{textcolor}\rmfamily\fontsize{10.000000}{12.000000}\selectfont 0.3}%
\end{pgfscope}%
\begin{pgfscope}%
\pgfsetbuttcap%
\pgfsetroundjoin%
\definecolor{currentfill}{rgb}{0.000000,0.000000,0.000000}%
\pgfsetfillcolor{currentfill}%
\pgfsetlinewidth{0.803000pt}%
\definecolor{currentstroke}{rgb}{0.000000,0.000000,0.000000}%
\pgfsetstrokecolor{currentstroke}%
\pgfsetdash{}{0pt}%
\pgfsys@defobject{currentmarker}{\pgfqpoint{0.000000in}{-0.048611in}}{\pgfqpoint{0.000000in}{0.000000in}}{%
\pgfpathmoveto{\pgfqpoint{0.000000in}{0.000000in}}%
\pgfpathlineto{\pgfqpoint{0.000000in}{-0.048611in}}%
\pgfusepath{stroke,fill}%
}%
\begin{pgfscope}%
\pgfsys@transformshift{2.018575in}{0.499444in}%
\pgfsys@useobject{currentmarker}{}%
\end{pgfscope}%
\end{pgfscope}%
\begin{pgfscope}%
\definecolor{textcolor}{rgb}{0.000000,0.000000,0.000000}%
\pgfsetstrokecolor{textcolor}%
\pgfsetfillcolor{textcolor}%
\pgftext[x=2.018575in,y=0.402222in,,top]{\color{textcolor}\rmfamily\fontsize{10.000000}{12.000000}\selectfont 0.4}%
\end{pgfscope}%
\begin{pgfscope}%
\pgfsetbuttcap%
\pgfsetroundjoin%
\definecolor{currentfill}{rgb}{0.000000,0.000000,0.000000}%
\pgfsetfillcolor{currentfill}%
\pgfsetlinewidth{0.803000pt}%
\definecolor{currentstroke}{rgb}{0.000000,0.000000,0.000000}%
\pgfsetstrokecolor{currentstroke}%
\pgfsetdash{}{0pt}%
\pgfsys@defobject{currentmarker}{\pgfqpoint{0.000000in}{-0.048611in}}{\pgfqpoint{0.000000in}{0.000000in}}{%
\pgfpathmoveto{\pgfqpoint{0.000000in}{0.000000in}}%
\pgfpathlineto{\pgfqpoint{0.000000in}{-0.048611in}}%
\pgfusepath{stroke,fill}%
}%
\begin{pgfscope}%
\pgfsys@transformshift{2.402239in}{0.499444in}%
\pgfsys@useobject{currentmarker}{}%
\end{pgfscope}%
\end{pgfscope}%
\begin{pgfscope}%
\definecolor{textcolor}{rgb}{0.000000,0.000000,0.000000}%
\pgfsetstrokecolor{textcolor}%
\pgfsetfillcolor{textcolor}%
\pgftext[x=2.402239in,y=0.402222in,,top]{\color{textcolor}\rmfamily\fontsize{10.000000}{12.000000}\selectfont 0.5}%
\end{pgfscope}%
\begin{pgfscope}%
\pgfsetbuttcap%
\pgfsetroundjoin%
\definecolor{currentfill}{rgb}{0.000000,0.000000,0.000000}%
\pgfsetfillcolor{currentfill}%
\pgfsetlinewidth{0.803000pt}%
\definecolor{currentstroke}{rgb}{0.000000,0.000000,0.000000}%
\pgfsetstrokecolor{currentstroke}%
\pgfsetdash{}{0pt}%
\pgfsys@defobject{currentmarker}{\pgfqpoint{0.000000in}{-0.048611in}}{\pgfqpoint{0.000000in}{0.000000in}}{%
\pgfpathmoveto{\pgfqpoint{0.000000in}{0.000000in}}%
\pgfpathlineto{\pgfqpoint{0.000000in}{-0.048611in}}%
\pgfusepath{stroke,fill}%
}%
\begin{pgfscope}%
\pgfsys@transformshift{2.785902in}{0.499444in}%
\pgfsys@useobject{currentmarker}{}%
\end{pgfscope}%
\end{pgfscope}%
\begin{pgfscope}%
\definecolor{textcolor}{rgb}{0.000000,0.000000,0.000000}%
\pgfsetstrokecolor{textcolor}%
\pgfsetfillcolor{textcolor}%
\pgftext[x=2.785902in,y=0.402222in,,top]{\color{textcolor}\rmfamily\fontsize{10.000000}{12.000000}\selectfont 0.6}%
\end{pgfscope}%
\begin{pgfscope}%
\pgfsetbuttcap%
\pgfsetroundjoin%
\definecolor{currentfill}{rgb}{0.000000,0.000000,0.000000}%
\pgfsetfillcolor{currentfill}%
\pgfsetlinewidth{0.803000pt}%
\definecolor{currentstroke}{rgb}{0.000000,0.000000,0.000000}%
\pgfsetstrokecolor{currentstroke}%
\pgfsetdash{}{0pt}%
\pgfsys@defobject{currentmarker}{\pgfqpoint{0.000000in}{-0.048611in}}{\pgfqpoint{0.000000in}{0.000000in}}{%
\pgfpathmoveto{\pgfqpoint{0.000000in}{0.000000in}}%
\pgfpathlineto{\pgfqpoint{0.000000in}{-0.048611in}}%
\pgfusepath{stroke,fill}%
}%
\begin{pgfscope}%
\pgfsys@transformshift{3.169566in}{0.499444in}%
\pgfsys@useobject{currentmarker}{}%
\end{pgfscope}%
\end{pgfscope}%
\begin{pgfscope}%
\definecolor{textcolor}{rgb}{0.000000,0.000000,0.000000}%
\pgfsetstrokecolor{textcolor}%
\pgfsetfillcolor{textcolor}%
\pgftext[x=3.169566in,y=0.402222in,,top]{\color{textcolor}\rmfamily\fontsize{10.000000}{12.000000}\selectfont 0.7}%
\end{pgfscope}%
\begin{pgfscope}%
\pgfsetbuttcap%
\pgfsetroundjoin%
\definecolor{currentfill}{rgb}{0.000000,0.000000,0.000000}%
\pgfsetfillcolor{currentfill}%
\pgfsetlinewidth{0.803000pt}%
\definecolor{currentstroke}{rgb}{0.000000,0.000000,0.000000}%
\pgfsetstrokecolor{currentstroke}%
\pgfsetdash{}{0pt}%
\pgfsys@defobject{currentmarker}{\pgfqpoint{0.000000in}{-0.048611in}}{\pgfqpoint{0.000000in}{0.000000in}}{%
\pgfpathmoveto{\pgfqpoint{0.000000in}{0.000000in}}%
\pgfpathlineto{\pgfqpoint{0.000000in}{-0.048611in}}%
\pgfusepath{stroke,fill}%
}%
\begin{pgfscope}%
\pgfsys@transformshift{3.553229in}{0.499444in}%
\pgfsys@useobject{currentmarker}{}%
\end{pgfscope}%
\end{pgfscope}%
\begin{pgfscope}%
\definecolor{textcolor}{rgb}{0.000000,0.000000,0.000000}%
\pgfsetstrokecolor{textcolor}%
\pgfsetfillcolor{textcolor}%
\pgftext[x=3.553229in,y=0.402222in,,top]{\color{textcolor}\rmfamily\fontsize{10.000000}{12.000000}\selectfont 0.8}%
\end{pgfscope}%
\begin{pgfscope}%
\pgfsetbuttcap%
\pgfsetroundjoin%
\definecolor{currentfill}{rgb}{0.000000,0.000000,0.000000}%
\pgfsetfillcolor{currentfill}%
\pgfsetlinewidth{0.803000pt}%
\definecolor{currentstroke}{rgb}{0.000000,0.000000,0.000000}%
\pgfsetstrokecolor{currentstroke}%
\pgfsetdash{}{0pt}%
\pgfsys@defobject{currentmarker}{\pgfqpoint{0.000000in}{-0.048611in}}{\pgfqpoint{0.000000in}{0.000000in}}{%
\pgfpathmoveto{\pgfqpoint{0.000000in}{0.000000in}}%
\pgfpathlineto{\pgfqpoint{0.000000in}{-0.048611in}}%
\pgfusepath{stroke,fill}%
}%
\begin{pgfscope}%
\pgfsys@transformshift{3.936892in}{0.499444in}%
\pgfsys@useobject{currentmarker}{}%
\end{pgfscope}%
\end{pgfscope}%
\begin{pgfscope}%
\definecolor{textcolor}{rgb}{0.000000,0.000000,0.000000}%
\pgfsetstrokecolor{textcolor}%
\pgfsetfillcolor{textcolor}%
\pgftext[x=3.936892in,y=0.402222in,,top]{\color{textcolor}\rmfamily\fontsize{10.000000}{12.000000}\selectfont 0.9}%
\end{pgfscope}%
\begin{pgfscope}%
\pgfsetbuttcap%
\pgfsetroundjoin%
\definecolor{currentfill}{rgb}{0.000000,0.000000,0.000000}%
\pgfsetfillcolor{currentfill}%
\pgfsetlinewidth{0.803000pt}%
\definecolor{currentstroke}{rgb}{0.000000,0.000000,0.000000}%
\pgfsetstrokecolor{currentstroke}%
\pgfsetdash{}{0pt}%
\pgfsys@defobject{currentmarker}{\pgfqpoint{0.000000in}{-0.048611in}}{\pgfqpoint{0.000000in}{0.000000in}}{%
\pgfpathmoveto{\pgfqpoint{0.000000in}{0.000000in}}%
\pgfpathlineto{\pgfqpoint{0.000000in}{-0.048611in}}%
\pgfusepath{stroke,fill}%
}%
\begin{pgfscope}%
\pgfsys@transformshift{4.320556in}{0.499444in}%
\pgfsys@useobject{currentmarker}{}%
\end{pgfscope}%
\end{pgfscope}%
\begin{pgfscope}%
\definecolor{textcolor}{rgb}{0.000000,0.000000,0.000000}%
\pgfsetstrokecolor{textcolor}%
\pgfsetfillcolor{textcolor}%
\pgftext[x=4.320556in,y=0.402222in,,top]{\color{textcolor}\rmfamily\fontsize{10.000000}{12.000000}\selectfont 1.0}%
\end{pgfscope}%
\begin{pgfscope}%
\definecolor{textcolor}{rgb}{0.000000,0.000000,0.000000}%
\pgfsetstrokecolor{textcolor}%
\pgfsetfillcolor{textcolor}%
\pgftext[x=2.383056in,y=0.223333in,,top]{\color{textcolor}\rmfamily\fontsize{10.000000}{12.000000}\selectfont \(\displaystyle p\)}%
\end{pgfscope}%
\begin{pgfscope}%
\pgfsetbuttcap%
\pgfsetroundjoin%
\definecolor{currentfill}{rgb}{0.000000,0.000000,0.000000}%
\pgfsetfillcolor{currentfill}%
\pgfsetlinewidth{0.803000pt}%
\definecolor{currentstroke}{rgb}{0.000000,0.000000,0.000000}%
\pgfsetstrokecolor{currentstroke}%
\pgfsetdash{}{0pt}%
\pgfsys@defobject{currentmarker}{\pgfqpoint{-0.048611in}{0.000000in}}{\pgfqpoint{-0.000000in}{0.000000in}}{%
\pgfpathmoveto{\pgfqpoint{-0.000000in}{0.000000in}}%
\pgfpathlineto{\pgfqpoint{-0.048611in}{0.000000in}}%
\pgfusepath{stroke,fill}%
}%
\begin{pgfscope}%
\pgfsys@transformshift{0.445556in}{0.499444in}%
\pgfsys@useobject{currentmarker}{}%
\end{pgfscope}%
\end{pgfscope}%
\begin{pgfscope}%
\definecolor{textcolor}{rgb}{0.000000,0.000000,0.000000}%
\pgfsetstrokecolor{textcolor}%
\pgfsetfillcolor{textcolor}%
\pgftext[x=0.278889in, y=0.451250in, left, base]{\color{textcolor}\rmfamily\fontsize{10.000000}{12.000000}\selectfont \(\displaystyle {0}\)}%
\end{pgfscope}%
\begin{pgfscope}%
\pgfsetbuttcap%
\pgfsetroundjoin%
\definecolor{currentfill}{rgb}{0.000000,0.000000,0.000000}%
\pgfsetfillcolor{currentfill}%
\pgfsetlinewidth{0.803000pt}%
\definecolor{currentstroke}{rgb}{0.000000,0.000000,0.000000}%
\pgfsetstrokecolor{currentstroke}%
\pgfsetdash{}{0pt}%
\pgfsys@defobject{currentmarker}{\pgfqpoint{-0.048611in}{0.000000in}}{\pgfqpoint{-0.000000in}{0.000000in}}{%
\pgfpathmoveto{\pgfqpoint{-0.000000in}{0.000000in}}%
\pgfpathlineto{\pgfqpoint{-0.048611in}{0.000000in}}%
\pgfusepath{stroke,fill}%
}%
\begin{pgfscope}%
\pgfsys@transformshift{0.445556in}{1.005031in}%
\pgfsys@useobject{currentmarker}{}%
\end{pgfscope}%
\end{pgfscope}%
\begin{pgfscope}%
\definecolor{textcolor}{rgb}{0.000000,0.000000,0.000000}%
\pgfsetstrokecolor{textcolor}%
\pgfsetfillcolor{textcolor}%
\pgftext[x=0.278889in, y=0.956836in, left, base]{\color{textcolor}\rmfamily\fontsize{10.000000}{12.000000}\selectfont \(\displaystyle {2}\)}%
\end{pgfscope}%
\begin{pgfscope}%
\pgfsetbuttcap%
\pgfsetroundjoin%
\definecolor{currentfill}{rgb}{0.000000,0.000000,0.000000}%
\pgfsetfillcolor{currentfill}%
\pgfsetlinewidth{0.803000pt}%
\definecolor{currentstroke}{rgb}{0.000000,0.000000,0.000000}%
\pgfsetstrokecolor{currentstroke}%
\pgfsetdash{}{0pt}%
\pgfsys@defobject{currentmarker}{\pgfqpoint{-0.048611in}{0.000000in}}{\pgfqpoint{-0.000000in}{0.000000in}}{%
\pgfpathmoveto{\pgfqpoint{-0.000000in}{0.000000in}}%
\pgfpathlineto{\pgfqpoint{-0.048611in}{0.000000in}}%
\pgfusepath{stroke,fill}%
}%
\begin{pgfscope}%
\pgfsys@transformshift{0.445556in}{1.510618in}%
\pgfsys@useobject{currentmarker}{}%
\end{pgfscope}%
\end{pgfscope}%
\begin{pgfscope}%
\definecolor{textcolor}{rgb}{0.000000,0.000000,0.000000}%
\pgfsetstrokecolor{textcolor}%
\pgfsetfillcolor{textcolor}%
\pgftext[x=0.278889in, y=1.462423in, left, base]{\color{textcolor}\rmfamily\fontsize{10.000000}{12.000000}\selectfont \(\displaystyle {4}\)}%
\end{pgfscope}%
\begin{pgfscope}%
\definecolor{textcolor}{rgb}{0.000000,0.000000,0.000000}%
\pgfsetstrokecolor{textcolor}%
\pgfsetfillcolor{textcolor}%
\pgftext[x=0.223333in,y=1.076944in,,bottom,rotate=90.000000]{\color{textcolor}\rmfamily\fontsize{10.000000}{12.000000}\selectfont Percent of Data Set}%
\end{pgfscope}%
\begin{pgfscope}%
\pgfsetrectcap%
\pgfsetmiterjoin%
\pgfsetlinewidth{0.803000pt}%
\definecolor{currentstroke}{rgb}{0.000000,0.000000,0.000000}%
\pgfsetstrokecolor{currentstroke}%
\pgfsetdash{}{0pt}%
\pgfpathmoveto{\pgfqpoint{0.445556in}{0.499444in}}%
\pgfpathlineto{\pgfqpoint{0.445556in}{1.654444in}}%
\pgfusepath{stroke}%
\end{pgfscope}%
\begin{pgfscope}%
\pgfsetrectcap%
\pgfsetmiterjoin%
\pgfsetlinewidth{0.803000pt}%
\definecolor{currentstroke}{rgb}{0.000000,0.000000,0.000000}%
\pgfsetstrokecolor{currentstroke}%
\pgfsetdash{}{0pt}%
\pgfpathmoveto{\pgfqpoint{4.320556in}{0.499444in}}%
\pgfpathlineto{\pgfqpoint{4.320556in}{1.654444in}}%
\pgfusepath{stroke}%
\end{pgfscope}%
\begin{pgfscope}%
\pgfsetrectcap%
\pgfsetmiterjoin%
\pgfsetlinewidth{0.803000pt}%
\definecolor{currentstroke}{rgb}{0.000000,0.000000,0.000000}%
\pgfsetstrokecolor{currentstroke}%
\pgfsetdash{}{0pt}%
\pgfpathmoveto{\pgfqpoint{0.445556in}{0.499444in}}%
\pgfpathlineto{\pgfqpoint{4.320556in}{0.499444in}}%
\pgfusepath{stroke}%
\end{pgfscope}%
\begin{pgfscope}%
\pgfsetrectcap%
\pgfsetmiterjoin%
\pgfsetlinewidth{0.803000pt}%
\definecolor{currentstroke}{rgb}{0.000000,0.000000,0.000000}%
\pgfsetstrokecolor{currentstroke}%
\pgfsetdash{}{0pt}%
\pgfpathmoveto{\pgfqpoint{0.445556in}{1.654444in}}%
\pgfpathlineto{\pgfqpoint{4.320556in}{1.654444in}}%
\pgfusepath{stroke}%
\end{pgfscope}%
\begin{pgfscope}%
\pgfsetbuttcap%
\pgfsetmiterjoin%
\definecolor{currentfill}{rgb}{1.000000,1.000000,1.000000}%
\pgfsetfillcolor{currentfill}%
\pgfsetfillopacity{0.800000}%
\pgfsetlinewidth{1.003750pt}%
\definecolor{currentstroke}{rgb}{0.800000,0.800000,0.800000}%
\pgfsetstrokecolor{currentstroke}%
\pgfsetstrokeopacity{0.800000}%
\pgfsetdash{}{0pt}%
\pgfpathmoveto{\pgfqpoint{3.543611in}{1.154445in}}%
\pgfpathlineto{\pgfqpoint{4.223333in}{1.154445in}}%
\pgfpathquadraticcurveto{\pgfqpoint{4.251111in}{1.154445in}}{\pgfqpoint{4.251111in}{1.182222in}}%
\pgfpathlineto{\pgfqpoint{4.251111in}{1.557222in}}%
\pgfpathquadraticcurveto{\pgfqpoint{4.251111in}{1.585000in}}{\pgfqpoint{4.223333in}{1.585000in}}%
\pgfpathlineto{\pgfqpoint{3.543611in}{1.585000in}}%
\pgfpathquadraticcurveto{\pgfqpoint{3.515833in}{1.585000in}}{\pgfqpoint{3.515833in}{1.557222in}}%
\pgfpathlineto{\pgfqpoint{3.515833in}{1.182222in}}%
\pgfpathquadraticcurveto{\pgfqpoint{3.515833in}{1.154445in}}{\pgfqpoint{3.543611in}{1.154445in}}%
\pgfpathlineto{\pgfqpoint{3.543611in}{1.154445in}}%
\pgfpathclose%
\pgfusepath{stroke,fill}%
\end{pgfscope}%
\begin{pgfscope}%
\pgfsetbuttcap%
\pgfsetmiterjoin%
\pgfsetlinewidth{1.003750pt}%
\definecolor{currentstroke}{rgb}{0.000000,0.000000,0.000000}%
\pgfsetstrokecolor{currentstroke}%
\pgfsetdash{}{0pt}%
\pgfpathmoveto{\pgfqpoint{3.571389in}{1.432222in}}%
\pgfpathlineto{\pgfqpoint{3.849167in}{1.432222in}}%
\pgfpathlineto{\pgfqpoint{3.849167in}{1.529444in}}%
\pgfpathlineto{\pgfqpoint{3.571389in}{1.529444in}}%
\pgfpathlineto{\pgfqpoint{3.571389in}{1.432222in}}%
\pgfpathclose%
\pgfusepath{stroke}%
\end{pgfscope}%
\begin{pgfscope}%
\definecolor{textcolor}{rgb}{0.000000,0.000000,0.000000}%
\pgfsetstrokecolor{textcolor}%
\pgfsetfillcolor{textcolor}%
\pgftext[x=3.960278in,y=1.432222in,left,base]{\color{textcolor}\rmfamily\fontsize{10.000000}{12.000000}\selectfont Neg}%
\end{pgfscope}%
\begin{pgfscope}%
\pgfsetbuttcap%
\pgfsetmiterjoin%
\definecolor{currentfill}{rgb}{0.000000,0.000000,0.000000}%
\pgfsetfillcolor{currentfill}%
\pgfsetlinewidth{0.000000pt}%
\definecolor{currentstroke}{rgb}{0.000000,0.000000,0.000000}%
\pgfsetstrokecolor{currentstroke}%
\pgfsetstrokeopacity{0.000000}%
\pgfsetdash{}{0pt}%
\pgfpathmoveto{\pgfqpoint{3.571389in}{1.236944in}}%
\pgfpathlineto{\pgfqpoint{3.849167in}{1.236944in}}%
\pgfpathlineto{\pgfqpoint{3.849167in}{1.334167in}}%
\pgfpathlineto{\pgfqpoint{3.571389in}{1.334167in}}%
\pgfpathlineto{\pgfqpoint{3.571389in}{1.236944in}}%
\pgfpathclose%
\pgfusepath{fill}%
\end{pgfscope}%
\begin{pgfscope}%
\definecolor{textcolor}{rgb}{0.000000,0.000000,0.000000}%
\pgfsetstrokecolor{textcolor}%
\pgfsetfillcolor{textcolor}%
\pgftext[x=3.960278in,y=1.236944in,left,base]{\color{textcolor}\rmfamily\fontsize{10.000000}{12.000000}\selectfont Pos}%
\end{pgfscope}%
\end{pgfpicture}%
\makeatother%
\endgroup%

&
	\vskip 0pt
	\qquad \qquad ROC Curve
	
	%% Creator: Matplotlib, PGF backend
%%
%% To include the figure in your LaTeX document, write
%%   \input{<filename>.pgf}
%%
%% Make sure the required packages are loaded in your preamble
%%   \usepackage{pgf}
%%
%% Also ensure that all the required font packages are loaded; for instance,
%% the lmodern package is sometimes necessary when using math font.
%%   \usepackage{lmodern}
%%
%% Figures using additional raster images can only be included by \input if
%% they are in the same directory as the main LaTeX file. For loading figures
%% from other directories you can use the `import` package
%%   \usepackage{import}
%%
%% and then include the figures with
%%   \import{<path to file>}{<filename>.pgf}
%%
%% Matplotlib used the following preamble
%%   
%%   \usepackage{fontspec}
%%   \makeatletter\@ifpackageloaded{underscore}{}{\usepackage[strings]{underscore}}\makeatother
%%
\begingroup%
\makeatletter%
\begin{pgfpicture}%
\pgfpathrectangle{\pgfpointorigin}{\pgfqpoint{2.221861in}{1.754444in}}%
\pgfusepath{use as bounding box, clip}%
\begin{pgfscope}%
\pgfsetbuttcap%
\pgfsetmiterjoin%
\definecolor{currentfill}{rgb}{1.000000,1.000000,1.000000}%
\pgfsetfillcolor{currentfill}%
\pgfsetlinewidth{0.000000pt}%
\definecolor{currentstroke}{rgb}{1.000000,1.000000,1.000000}%
\pgfsetstrokecolor{currentstroke}%
\pgfsetdash{}{0pt}%
\pgfpathmoveto{\pgfqpoint{0.000000in}{0.000000in}}%
\pgfpathlineto{\pgfqpoint{2.221861in}{0.000000in}}%
\pgfpathlineto{\pgfqpoint{2.221861in}{1.754444in}}%
\pgfpathlineto{\pgfqpoint{0.000000in}{1.754444in}}%
\pgfpathlineto{\pgfqpoint{0.000000in}{0.000000in}}%
\pgfpathclose%
\pgfusepath{fill}%
\end{pgfscope}%
\begin{pgfscope}%
\pgfsetbuttcap%
\pgfsetmiterjoin%
\definecolor{currentfill}{rgb}{1.000000,1.000000,1.000000}%
\pgfsetfillcolor{currentfill}%
\pgfsetlinewidth{0.000000pt}%
\definecolor{currentstroke}{rgb}{0.000000,0.000000,0.000000}%
\pgfsetstrokecolor{currentstroke}%
\pgfsetstrokeopacity{0.000000}%
\pgfsetdash{}{0pt}%
\pgfpathmoveto{\pgfqpoint{0.553581in}{0.499444in}}%
\pgfpathlineto{\pgfqpoint{2.103581in}{0.499444in}}%
\pgfpathlineto{\pgfqpoint{2.103581in}{1.654444in}}%
\pgfpathlineto{\pgfqpoint{0.553581in}{1.654444in}}%
\pgfpathlineto{\pgfqpoint{0.553581in}{0.499444in}}%
\pgfpathclose%
\pgfusepath{fill}%
\end{pgfscope}%
\begin{pgfscope}%
\pgfsetbuttcap%
\pgfsetroundjoin%
\definecolor{currentfill}{rgb}{0.000000,0.000000,0.000000}%
\pgfsetfillcolor{currentfill}%
\pgfsetlinewidth{0.803000pt}%
\definecolor{currentstroke}{rgb}{0.000000,0.000000,0.000000}%
\pgfsetstrokecolor{currentstroke}%
\pgfsetdash{}{0pt}%
\pgfsys@defobject{currentmarker}{\pgfqpoint{0.000000in}{-0.048611in}}{\pgfqpoint{0.000000in}{0.000000in}}{%
\pgfpathmoveto{\pgfqpoint{0.000000in}{0.000000in}}%
\pgfpathlineto{\pgfqpoint{0.000000in}{-0.048611in}}%
\pgfusepath{stroke,fill}%
}%
\begin{pgfscope}%
\pgfsys@transformshift{0.624035in}{0.499444in}%
\pgfsys@useobject{currentmarker}{}%
\end{pgfscope}%
\end{pgfscope}%
\begin{pgfscope}%
\definecolor{textcolor}{rgb}{0.000000,0.000000,0.000000}%
\pgfsetstrokecolor{textcolor}%
\pgfsetfillcolor{textcolor}%
\pgftext[x=0.624035in,y=0.402222in,,top]{\color{textcolor}\rmfamily\fontsize{10.000000}{12.000000}\selectfont \(\displaystyle {0.0}\)}%
\end{pgfscope}%
\begin{pgfscope}%
\pgfsetbuttcap%
\pgfsetroundjoin%
\definecolor{currentfill}{rgb}{0.000000,0.000000,0.000000}%
\pgfsetfillcolor{currentfill}%
\pgfsetlinewidth{0.803000pt}%
\definecolor{currentstroke}{rgb}{0.000000,0.000000,0.000000}%
\pgfsetstrokecolor{currentstroke}%
\pgfsetdash{}{0pt}%
\pgfsys@defobject{currentmarker}{\pgfqpoint{0.000000in}{-0.048611in}}{\pgfqpoint{0.000000in}{0.000000in}}{%
\pgfpathmoveto{\pgfqpoint{0.000000in}{0.000000in}}%
\pgfpathlineto{\pgfqpoint{0.000000in}{-0.048611in}}%
\pgfusepath{stroke,fill}%
}%
\begin{pgfscope}%
\pgfsys@transformshift{1.328581in}{0.499444in}%
\pgfsys@useobject{currentmarker}{}%
\end{pgfscope}%
\end{pgfscope}%
\begin{pgfscope}%
\definecolor{textcolor}{rgb}{0.000000,0.000000,0.000000}%
\pgfsetstrokecolor{textcolor}%
\pgfsetfillcolor{textcolor}%
\pgftext[x=1.328581in,y=0.402222in,,top]{\color{textcolor}\rmfamily\fontsize{10.000000}{12.000000}\selectfont \(\displaystyle {0.5}\)}%
\end{pgfscope}%
\begin{pgfscope}%
\pgfsetbuttcap%
\pgfsetroundjoin%
\definecolor{currentfill}{rgb}{0.000000,0.000000,0.000000}%
\pgfsetfillcolor{currentfill}%
\pgfsetlinewidth{0.803000pt}%
\definecolor{currentstroke}{rgb}{0.000000,0.000000,0.000000}%
\pgfsetstrokecolor{currentstroke}%
\pgfsetdash{}{0pt}%
\pgfsys@defobject{currentmarker}{\pgfqpoint{0.000000in}{-0.048611in}}{\pgfqpoint{0.000000in}{0.000000in}}{%
\pgfpathmoveto{\pgfqpoint{0.000000in}{0.000000in}}%
\pgfpathlineto{\pgfqpoint{0.000000in}{-0.048611in}}%
\pgfusepath{stroke,fill}%
}%
\begin{pgfscope}%
\pgfsys@transformshift{2.033126in}{0.499444in}%
\pgfsys@useobject{currentmarker}{}%
\end{pgfscope}%
\end{pgfscope}%
\begin{pgfscope}%
\definecolor{textcolor}{rgb}{0.000000,0.000000,0.000000}%
\pgfsetstrokecolor{textcolor}%
\pgfsetfillcolor{textcolor}%
\pgftext[x=2.033126in,y=0.402222in,,top]{\color{textcolor}\rmfamily\fontsize{10.000000}{12.000000}\selectfont \(\displaystyle {1.0}\)}%
\end{pgfscope}%
\begin{pgfscope}%
\definecolor{textcolor}{rgb}{0.000000,0.000000,0.000000}%
\pgfsetstrokecolor{textcolor}%
\pgfsetfillcolor{textcolor}%
\pgftext[x=1.328581in,y=0.223333in,,top]{\color{textcolor}\rmfamily\fontsize{10.000000}{12.000000}\selectfont False positive rate}%
\end{pgfscope}%
\begin{pgfscope}%
\pgfsetbuttcap%
\pgfsetroundjoin%
\definecolor{currentfill}{rgb}{0.000000,0.000000,0.000000}%
\pgfsetfillcolor{currentfill}%
\pgfsetlinewidth{0.803000pt}%
\definecolor{currentstroke}{rgb}{0.000000,0.000000,0.000000}%
\pgfsetstrokecolor{currentstroke}%
\pgfsetdash{}{0pt}%
\pgfsys@defobject{currentmarker}{\pgfqpoint{-0.048611in}{0.000000in}}{\pgfqpoint{-0.000000in}{0.000000in}}{%
\pgfpathmoveto{\pgfqpoint{-0.000000in}{0.000000in}}%
\pgfpathlineto{\pgfqpoint{-0.048611in}{0.000000in}}%
\pgfusepath{stroke,fill}%
}%
\begin{pgfscope}%
\pgfsys@transformshift{0.553581in}{0.551944in}%
\pgfsys@useobject{currentmarker}{}%
\end{pgfscope}%
\end{pgfscope}%
\begin{pgfscope}%
\definecolor{textcolor}{rgb}{0.000000,0.000000,0.000000}%
\pgfsetstrokecolor{textcolor}%
\pgfsetfillcolor{textcolor}%
\pgftext[x=0.278889in, y=0.503750in, left, base]{\color{textcolor}\rmfamily\fontsize{10.000000}{12.000000}\selectfont \(\displaystyle {0.0}\)}%
\end{pgfscope}%
\begin{pgfscope}%
\pgfsetbuttcap%
\pgfsetroundjoin%
\definecolor{currentfill}{rgb}{0.000000,0.000000,0.000000}%
\pgfsetfillcolor{currentfill}%
\pgfsetlinewidth{0.803000pt}%
\definecolor{currentstroke}{rgb}{0.000000,0.000000,0.000000}%
\pgfsetstrokecolor{currentstroke}%
\pgfsetdash{}{0pt}%
\pgfsys@defobject{currentmarker}{\pgfqpoint{-0.048611in}{0.000000in}}{\pgfqpoint{-0.000000in}{0.000000in}}{%
\pgfpathmoveto{\pgfqpoint{-0.000000in}{0.000000in}}%
\pgfpathlineto{\pgfqpoint{-0.048611in}{0.000000in}}%
\pgfusepath{stroke,fill}%
}%
\begin{pgfscope}%
\pgfsys@transformshift{0.553581in}{1.076944in}%
\pgfsys@useobject{currentmarker}{}%
\end{pgfscope}%
\end{pgfscope}%
\begin{pgfscope}%
\definecolor{textcolor}{rgb}{0.000000,0.000000,0.000000}%
\pgfsetstrokecolor{textcolor}%
\pgfsetfillcolor{textcolor}%
\pgftext[x=0.278889in, y=1.028750in, left, base]{\color{textcolor}\rmfamily\fontsize{10.000000}{12.000000}\selectfont \(\displaystyle {0.5}\)}%
\end{pgfscope}%
\begin{pgfscope}%
\pgfsetbuttcap%
\pgfsetroundjoin%
\definecolor{currentfill}{rgb}{0.000000,0.000000,0.000000}%
\pgfsetfillcolor{currentfill}%
\pgfsetlinewidth{0.803000pt}%
\definecolor{currentstroke}{rgb}{0.000000,0.000000,0.000000}%
\pgfsetstrokecolor{currentstroke}%
\pgfsetdash{}{0pt}%
\pgfsys@defobject{currentmarker}{\pgfqpoint{-0.048611in}{0.000000in}}{\pgfqpoint{-0.000000in}{0.000000in}}{%
\pgfpathmoveto{\pgfqpoint{-0.000000in}{0.000000in}}%
\pgfpathlineto{\pgfqpoint{-0.048611in}{0.000000in}}%
\pgfusepath{stroke,fill}%
}%
\begin{pgfscope}%
\pgfsys@transformshift{0.553581in}{1.601944in}%
\pgfsys@useobject{currentmarker}{}%
\end{pgfscope}%
\end{pgfscope}%
\begin{pgfscope}%
\definecolor{textcolor}{rgb}{0.000000,0.000000,0.000000}%
\pgfsetstrokecolor{textcolor}%
\pgfsetfillcolor{textcolor}%
\pgftext[x=0.278889in, y=1.553750in, left, base]{\color{textcolor}\rmfamily\fontsize{10.000000}{12.000000}\selectfont \(\displaystyle {1.0}\)}%
\end{pgfscope}%
\begin{pgfscope}%
\definecolor{textcolor}{rgb}{0.000000,0.000000,0.000000}%
\pgfsetstrokecolor{textcolor}%
\pgfsetfillcolor{textcolor}%
\pgftext[x=0.223333in,y=1.076944in,,bottom,rotate=90.000000]{\color{textcolor}\rmfamily\fontsize{10.000000}{12.000000}\selectfont True positive rate}%
\end{pgfscope}%
\begin{pgfscope}%
\pgfpathrectangle{\pgfqpoint{0.553581in}{0.499444in}}{\pgfqpoint{1.550000in}{1.155000in}}%
\pgfusepath{clip}%
\pgfsetbuttcap%
\pgfsetroundjoin%
\pgfsetlinewidth{1.505625pt}%
\definecolor{currentstroke}{rgb}{0.000000,0.000000,0.000000}%
\pgfsetstrokecolor{currentstroke}%
\pgfsetdash{{5.550000pt}{2.400000pt}}{0.000000pt}%
\pgfpathmoveto{\pgfqpoint{0.624035in}{0.551944in}}%
\pgfpathlineto{\pgfqpoint{2.033126in}{1.601944in}}%
\pgfusepath{stroke}%
\end{pgfscope}%
\begin{pgfscope}%
\pgfpathrectangle{\pgfqpoint{0.553581in}{0.499444in}}{\pgfqpoint{1.550000in}{1.155000in}}%
\pgfusepath{clip}%
\pgfsetrectcap%
\pgfsetroundjoin%
\pgfsetlinewidth{1.505625pt}%
\definecolor{currentstroke}{rgb}{0.000000,0.000000,0.000000}%
\pgfsetstrokecolor{currentstroke}%
\pgfsetdash{}{0pt}%
\pgfpathmoveto{\pgfqpoint{0.624035in}{0.551944in}}%
\pgfpathlineto{\pgfqpoint{0.624098in}{0.552969in}}%
\pgfpathlineto{\pgfqpoint{0.625208in}{0.569887in}}%
\pgfpathlineto{\pgfqpoint{0.625310in}{0.570880in}}%
\pgfpathlineto{\pgfqpoint{0.625310in}{0.570973in}}%
\pgfpathlineto{\pgfqpoint{0.626420in}{0.585159in}}%
\pgfpathlineto{\pgfqpoint{0.626529in}{0.586091in}}%
\pgfpathlineto{\pgfqpoint{0.627639in}{0.597390in}}%
\pgfpathlineto{\pgfqpoint{0.627780in}{0.598383in}}%
\pgfpathlineto{\pgfqpoint{0.628882in}{0.607541in}}%
\pgfpathlineto{\pgfqpoint{0.629062in}{0.608627in}}%
\pgfpathlineto{\pgfqpoint{0.630172in}{0.617443in}}%
\pgfpathlineto{\pgfqpoint{0.630430in}{0.618530in}}%
\pgfpathlineto{\pgfqpoint{0.631540in}{0.627097in}}%
\pgfpathlineto{\pgfqpoint{0.631673in}{0.628122in}}%
\pgfpathlineto{\pgfqpoint{0.632783in}{0.636472in}}%
\pgfpathlineto{\pgfqpoint{0.632908in}{0.637403in}}%
\pgfpathlineto{\pgfqpoint{0.634011in}{0.643891in}}%
\pgfpathlineto{\pgfqpoint{0.634190in}{0.644853in}}%
\pgfpathlineto{\pgfqpoint{0.635277in}{0.651683in}}%
\pgfpathlineto{\pgfqpoint{0.635293in}{0.651683in}}%
\pgfpathlineto{\pgfqpoint{0.635480in}{0.652707in}}%
\pgfpathlineto{\pgfqpoint{0.636590in}{0.659040in}}%
\pgfpathlineto{\pgfqpoint{0.636825in}{0.660064in}}%
\pgfpathlineto{\pgfqpoint{0.637927in}{0.665962in}}%
\pgfpathlineto{\pgfqpoint{0.638201in}{0.667017in}}%
\pgfpathlineto{\pgfqpoint{0.639311in}{0.672419in}}%
\pgfpathlineto{\pgfqpoint{0.639553in}{0.673505in}}%
\pgfpathlineto{\pgfqpoint{0.640663in}{0.678782in}}%
\pgfpathlineto{\pgfqpoint{0.640976in}{0.679838in}}%
\pgfpathlineto{\pgfqpoint{0.642086in}{0.684339in}}%
\pgfpathlineto{\pgfqpoint{0.642297in}{0.685425in}}%
\pgfpathlineto{\pgfqpoint{0.643407in}{0.691510in}}%
\pgfpathlineto{\pgfqpoint{0.643720in}{0.692596in}}%
\pgfpathlineto{\pgfqpoint{0.644830in}{0.697687in}}%
\pgfpathlineto{\pgfqpoint{0.645026in}{0.698742in}}%
\pgfpathlineto{\pgfqpoint{0.646104in}{0.703616in}}%
\pgfpathlineto{\pgfqpoint{0.646542in}{0.704640in}}%
\pgfpathlineto{\pgfqpoint{0.647652in}{0.708862in}}%
\pgfpathlineto{\pgfqpoint{0.647871in}{0.709949in}}%
\pgfpathlineto{\pgfqpoint{0.648966in}{0.714574in}}%
\pgfpathlineto{\pgfqpoint{0.649200in}{0.715660in}}%
\pgfpathlineto{\pgfqpoint{0.650310in}{0.721403in}}%
\pgfpathlineto{\pgfqpoint{0.650654in}{0.722490in}}%
\pgfpathlineto{\pgfqpoint{0.651741in}{0.726463in}}%
\pgfpathlineto{\pgfqpoint{0.652069in}{0.727549in}}%
\pgfpathlineto{\pgfqpoint{0.653172in}{0.732423in}}%
\pgfpathlineto{\pgfqpoint{0.653453in}{0.733510in}}%
\pgfpathlineto{\pgfqpoint{0.654555in}{0.738073in}}%
\pgfpathlineto{\pgfqpoint{0.654845in}{0.739159in}}%
\pgfpathlineto{\pgfqpoint{0.655955in}{0.743008in}}%
\pgfpathlineto{\pgfqpoint{0.656338in}{0.744095in}}%
\pgfpathlineto{\pgfqpoint{0.657448in}{0.748037in}}%
\pgfpathlineto{\pgfqpoint{0.657792in}{0.749124in}}%
\pgfpathlineto{\pgfqpoint{0.658894in}{0.752694in}}%
\pgfpathlineto{\pgfqpoint{0.659222in}{0.753780in}}%
\pgfpathlineto{\pgfqpoint{0.660333in}{0.757381in}}%
\pgfpathlineto{\pgfqpoint{0.660700in}{0.758467in}}%
\pgfpathlineto{\pgfqpoint{0.661810in}{0.762255in}}%
\pgfpathlineto{\pgfqpoint{0.662092in}{0.763341in}}%
\pgfpathlineto{\pgfqpoint{0.667666in}{0.780973in}}%
\pgfpathlineto{\pgfqpoint{0.668017in}{0.782059in}}%
\pgfpathlineto{\pgfqpoint{0.669034in}{0.785847in}}%
\pgfpathlineto{\pgfqpoint{0.669479in}{0.786933in}}%
\pgfpathlineto{\pgfqpoint{0.670589in}{0.791372in}}%
\pgfpathlineto{\pgfqpoint{0.670855in}{0.792459in}}%
\pgfpathlineto{\pgfqpoint{0.671965in}{0.795780in}}%
\pgfpathlineto{\pgfqpoint{0.672387in}{0.796867in}}%
\pgfpathlineto{\pgfqpoint{0.673466in}{0.799629in}}%
\pgfpathlineto{\pgfqpoint{0.674021in}{0.800716in}}%
\pgfpathlineto{\pgfqpoint{0.675131in}{0.803975in}}%
\pgfpathlineto{\pgfqpoint{0.675811in}{0.805031in}}%
\pgfpathlineto{\pgfqpoint{0.676828in}{0.808228in}}%
\pgfpathlineto{\pgfqpoint{0.677297in}{0.809283in}}%
\pgfpathlineto{\pgfqpoint{0.678391in}{0.812201in}}%
\pgfpathlineto{\pgfqpoint{0.678939in}{0.813288in}}%
\pgfpathlineto{\pgfqpoint{0.680049in}{0.815957in}}%
\pgfpathlineto{\pgfqpoint{0.680744in}{0.817013in}}%
\pgfpathlineto{\pgfqpoint{0.681855in}{0.819683in}}%
\pgfpathlineto{\pgfqpoint{0.682433in}{0.820738in}}%
\pgfpathlineto{\pgfqpoint{0.683543in}{0.823377in}}%
\pgfpathlineto{\pgfqpoint{0.684012in}{0.824370in}}%
\pgfpathlineto{\pgfqpoint{0.685122in}{0.827412in}}%
\pgfpathlineto{\pgfqpoint{0.685560in}{0.828498in}}%
\pgfpathlineto{\pgfqpoint{0.686662in}{0.831385in}}%
\pgfpathlineto{\pgfqpoint{0.687045in}{0.832472in}}%
\pgfpathlineto{\pgfqpoint{0.688148in}{0.835731in}}%
\pgfpathlineto{\pgfqpoint{0.688859in}{0.836818in}}%
\pgfpathlineto{\pgfqpoint{0.689969in}{0.839456in}}%
\pgfpathlineto{\pgfqpoint{0.690454in}{0.840543in}}%
\pgfpathlineto{\pgfqpoint{0.691564in}{0.843430in}}%
\pgfpathlineto{\pgfqpoint{0.691838in}{0.844516in}}%
\pgfpathlineto{\pgfqpoint{0.692948in}{0.847900in}}%
\pgfpathlineto{\pgfqpoint{0.693253in}{0.848986in}}%
\pgfpathlineto{\pgfqpoint{0.694363in}{0.852308in}}%
\pgfpathlineto{\pgfqpoint{0.694847in}{0.853394in}}%
\pgfpathlineto{\pgfqpoint{0.695934in}{0.855785in}}%
\pgfpathlineto{\pgfqpoint{0.696481in}{0.856840in}}%
\pgfpathlineto{\pgfqpoint{0.697576in}{0.859789in}}%
\pgfpathlineto{\pgfqpoint{0.698186in}{0.860844in}}%
\pgfpathlineto{\pgfqpoint{0.699296in}{0.863887in}}%
\pgfpathlineto{\pgfqpoint{0.699859in}{0.864942in}}%
\pgfpathlineto{\pgfqpoint{0.700969in}{0.867922in}}%
\pgfpathlineto{\pgfqpoint{0.701578in}{0.868977in}}%
\pgfpathlineto{\pgfqpoint{0.702665in}{0.871119in}}%
\pgfpathlineto{\pgfqpoint{0.703111in}{0.872206in}}%
\pgfpathlineto{\pgfqpoint{0.704205in}{0.874037in}}%
\pgfpathlineto{\pgfqpoint{0.704690in}{0.875093in}}%
\pgfpathlineto{\pgfqpoint{0.705800in}{0.877855in}}%
\pgfpathlineto{\pgfqpoint{0.706574in}{0.878942in}}%
\pgfpathlineto{\pgfqpoint{0.707684in}{0.881425in}}%
\pgfpathlineto{\pgfqpoint{0.708294in}{0.882512in}}%
\pgfpathlineto{\pgfqpoint{0.709373in}{0.884933in}}%
\pgfpathlineto{\pgfqpoint{0.709396in}{0.884933in}}%
\pgfpathlineto{\pgfqpoint{0.710029in}{0.886020in}}%
\pgfpathlineto{\pgfqpoint{0.711116in}{0.888317in}}%
\pgfpathlineto{\pgfqpoint{0.711632in}{0.889403in}}%
\pgfpathlineto{\pgfqpoint{0.712719in}{0.892042in}}%
\pgfpathlineto{\pgfqpoint{0.713149in}{0.893128in}}%
\pgfpathlineto{\pgfqpoint{0.714235in}{0.895705in}}%
\pgfpathlineto{\pgfqpoint{0.714939in}{0.896791in}}%
\pgfpathlineto{\pgfqpoint{0.716041in}{0.899368in}}%
\pgfpathlineto{\pgfqpoint{0.716463in}{0.900454in}}%
\pgfpathlineto{\pgfqpoint{0.717573in}{0.902906in}}%
\pgfpathlineto{\pgfqpoint{0.718089in}{0.903993in}}%
\pgfpathlineto{\pgfqpoint{0.719168in}{0.905887in}}%
\pgfpathlineto{\pgfqpoint{0.719794in}{0.906942in}}%
\pgfpathlineto{\pgfqpoint{0.720896in}{0.909301in}}%
\pgfpathlineto{\pgfqpoint{0.721443in}{0.910388in}}%
\pgfpathlineto{\pgfqpoint{0.722553in}{0.912685in}}%
\pgfpathlineto{\pgfqpoint{0.723014in}{0.913771in}}%
\pgfpathlineto{\pgfqpoint{0.724117in}{0.916037in}}%
\pgfpathlineto{\pgfqpoint{0.724750in}{0.917124in}}%
\pgfpathlineto{\pgfqpoint{0.725852in}{0.918893in}}%
\pgfpathlineto{\pgfqpoint{0.726407in}{0.919918in}}%
\pgfpathlineto{\pgfqpoint{0.727517in}{0.922649in}}%
\pgfpathlineto{\pgfqpoint{0.728182in}{0.923736in}}%
\pgfpathlineto{\pgfqpoint{0.729292in}{0.926002in}}%
\pgfpathlineto{\pgfqpoint{0.729972in}{0.927057in}}%
\pgfpathlineto{\pgfqpoint{0.731082in}{0.928547in}}%
\pgfpathlineto{\pgfqpoint{0.731543in}{0.929634in}}%
\pgfpathlineto{\pgfqpoint{0.732646in}{0.932024in}}%
\pgfpathlineto{\pgfqpoint{0.733263in}{0.933110in}}%
\pgfpathlineto{\pgfqpoint{0.734366in}{0.934663in}}%
\pgfpathlineto{\pgfqpoint{0.735007in}{0.935749in}}%
\pgfpathlineto{\pgfqpoint{0.736117in}{0.937363in}}%
\pgfpathlineto{\pgfqpoint{0.736617in}{0.938450in}}%
\pgfpathlineto{\pgfqpoint{0.737688in}{0.940592in}}%
\pgfpathlineto{\pgfqpoint{0.738282in}{0.941585in}}%
\pgfpathlineto{\pgfqpoint{0.739392in}{0.943851in}}%
\pgfpathlineto{\pgfqpoint{0.740002in}{0.944938in}}%
\pgfpathlineto{\pgfqpoint{0.741089in}{0.947079in}}%
\pgfpathlineto{\pgfqpoint{0.741706in}{0.948166in}}%
\pgfpathlineto{\pgfqpoint{0.742816in}{0.949935in}}%
\pgfpathlineto{\pgfqpoint{0.743528in}{0.951022in}}%
\pgfpathlineto{\pgfqpoint{0.744583in}{0.953288in}}%
\pgfpathlineto{\pgfqpoint{0.745506in}{0.954374in}}%
\pgfpathlineto{\pgfqpoint{0.746616in}{0.956423in}}%
\pgfpathlineto{\pgfqpoint{0.747147in}{0.957510in}}%
\pgfpathlineto{\pgfqpoint{0.748187in}{0.959217in}}%
\pgfpathlineto{\pgfqpoint{0.748938in}{0.960303in}}%
\pgfpathlineto{\pgfqpoint{0.750048in}{0.962663in}}%
\pgfpathlineto{\pgfqpoint{0.750556in}{0.963749in}}%
\pgfpathlineto{\pgfqpoint{0.751666in}{0.965860in}}%
\pgfpathlineto{\pgfqpoint{0.752448in}{0.966946in}}%
\pgfpathlineto{\pgfqpoint{0.753550in}{0.968902in}}%
\pgfpathlineto{\pgfqpoint{0.754176in}{0.969926in}}%
\pgfpathlineto{\pgfqpoint{0.754176in}{0.969989in}}%
\pgfpathlineto{\pgfqpoint{0.755966in}{0.972658in}}%
\pgfpathlineto{\pgfqpoint{0.756763in}{0.973714in}}%
\pgfpathlineto{\pgfqpoint{0.757873in}{0.975793in}}%
\pgfpathlineto{\pgfqpoint{0.758421in}{0.976880in}}%
\pgfpathlineto{\pgfqpoint{0.759531in}{0.978773in}}%
\pgfpathlineto{\pgfqpoint{0.760140in}{0.979829in}}%
\pgfpathlineto{\pgfqpoint{0.761227in}{0.982033in}}%
\pgfpathlineto{\pgfqpoint{0.762017in}{0.983119in}}%
\pgfpathlineto{\pgfqpoint{0.763088in}{0.985106in}}%
\pgfpathlineto{\pgfqpoint{0.763729in}{0.986193in}}%
\pgfpathlineto{\pgfqpoint{0.764792in}{0.988210in}}%
\pgfpathlineto{\pgfqpoint{0.765636in}{0.989297in}}%
\pgfpathlineto{\pgfqpoint{0.766738in}{0.991594in}}%
\pgfpathlineto{\pgfqpoint{0.767466in}{0.992680in}}%
\pgfpathlineto{\pgfqpoint{0.768552in}{0.994636in}}%
\pgfpathlineto{\pgfqpoint{0.769021in}{0.995722in}}%
\pgfpathlineto{\pgfqpoint{0.770116in}{0.997088in}}%
\pgfpathlineto{\pgfqpoint{0.770475in}{0.998144in}}%
\pgfpathlineto{\pgfqpoint{0.771585in}{0.999696in}}%
\pgfpathlineto{\pgfqpoint{0.772391in}{1.000782in}}%
\pgfpathlineto{\pgfqpoint{0.773501in}{1.002800in}}%
\pgfpathlineto{\pgfqpoint{0.773993in}{1.003887in}}%
\pgfpathlineto{\pgfqpoint{0.775064in}{1.005159in}}%
\pgfpathlineto{\pgfqpoint{0.775823in}{1.006246in}}%
\pgfpathlineto{\pgfqpoint{0.776886in}{1.007581in}}%
\pgfpathlineto{\pgfqpoint{0.778004in}{1.008667in}}%
\pgfpathlineto{\pgfqpoint{0.779028in}{1.010405in}}%
\pgfpathlineto{\pgfqpoint{0.779114in}{1.010405in}}%
\pgfpathlineto{\pgfqpoint{0.779896in}{1.011492in}}%
\pgfpathlineto{\pgfqpoint{0.780990in}{1.012982in}}%
\pgfpathlineto{\pgfqpoint{0.781811in}{1.014037in}}%
\pgfpathlineto{\pgfqpoint{0.782890in}{1.015589in}}%
\pgfpathlineto{\pgfqpoint{0.782905in}{1.015589in}}%
\pgfpathlineto{\pgfqpoint{0.783757in}{1.016645in}}%
\pgfpathlineto{\pgfqpoint{0.784844in}{1.018880in}}%
\pgfpathlineto{\pgfqpoint{0.785900in}{1.019966in}}%
\pgfpathlineto{\pgfqpoint{0.787010in}{1.021705in}}%
\pgfpathlineto{\pgfqpoint{0.787963in}{1.022729in}}%
\pgfpathlineto{\pgfqpoint{0.787963in}{1.022760in}}%
\pgfpathlineto{\pgfqpoint{0.789019in}{1.024498in}}%
\pgfpathlineto{\pgfqpoint{0.789073in}{1.024498in}}%
\pgfpathlineto{\pgfqpoint{0.789801in}{1.025585in}}%
\pgfpathlineto{\pgfqpoint{0.790887in}{1.027106in}}%
\pgfpathlineto{\pgfqpoint{0.791591in}{1.028193in}}%
\pgfpathlineto{\pgfqpoint{0.792693in}{1.030303in}}%
\pgfpathlineto{\pgfqpoint{0.793404in}{1.031390in}}%
\pgfpathlineto{\pgfqpoint{0.794507in}{1.033035in}}%
\pgfpathlineto{\pgfqpoint{0.795226in}{1.034122in}}%
\pgfpathlineto{\pgfqpoint{0.796320in}{1.035798in}}%
\pgfpathlineto{\pgfqpoint{0.797024in}{1.036884in}}%
\pgfpathlineto{\pgfqpoint{0.798126in}{1.038467in}}%
\pgfpathlineto{\pgfqpoint{0.798939in}{1.039554in}}%
\pgfpathlineto{\pgfqpoint{0.800049in}{1.041230in}}%
\pgfpathlineto{\pgfqpoint{0.800808in}{1.042317in}}%
\pgfpathlineto{\pgfqpoint{0.801894in}{1.043993in}}%
\pgfpathlineto{\pgfqpoint{0.802457in}{1.045079in}}%
\pgfpathlineto{\pgfqpoint{0.803466in}{1.046507in}}%
\pgfpathlineto{\pgfqpoint{0.803536in}{1.046507in}}%
\pgfpathlineto{\pgfqpoint{0.804240in}{1.047594in}}%
\pgfpathlineto{\pgfqpoint{0.805303in}{1.049239in}}%
\pgfpathlineto{\pgfqpoint{0.806218in}{1.050294in}}%
\pgfpathlineto{\pgfqpoint{0.807312in}{1.051971in}}%
\pgfpathlineto{\pgfqpoint{0.808063in}{1.053057in}}%
\pgfpathlineto{\pgfqpoint{0.809165in}{1.054237in}}%
\pgfpathlineto{\pgfqpoint{0.809939in}{1.055323in}}%
\pgfpathlineto{\pgfqpoint{0.811494in}{1.057000in}}%
\pgfpathlineto{\pgfqpoint{0.812698in}{1.058086in}}%
\pgfpathlineto{\pgfqpoint{0.813746in}{1.059514in}}%
\pgfpathlineto{\pgfqpoint{0.814340in}{1.060600in}}%
\pgfpathlineto{\pgfqpoint{0.815419in}{1.062308in}}%
\pgfpathlineto{\pgfqpoint{0.816130in}{1.063394in}}%
\pgfpathlineto{\pgfqpoint{0.817240in}{1.064667in}}%
\pgfpathlineto{\pgfqpoint{0.818233in}{1.065753in}}%
\pgfpathlineto{\pgfqpoint{0.819281in}{1.066964in}}%
\pgfpathlineto{\pgfqpoint{0.820078in}{1.068051in}}%
\pgfpathlineto{\pgfqpoint{0.821165in}{1.069696in}}%
\pgfpathlineto{\pgfqpoint{0.821868in}{1.070782in}}%
\pgfpathlineto{\pgfqpoint{0.822947in}{1.072210in}}%
\pgfpathlineto{\pgfqpoint{0.822971in}{1.072210in}}%
\pgfpathlineto{\pgfqpoint{0.823745in}{1.073297in}}%
\pgfpathlineto{\pgfqpoint{0.824855in}{1.074787in}}%
\pgfpathlineto{\pgfqpoint{0.825379in}{1.075873in}}%
\pgfpathlineto{\pgfqpoint{0.826489in}{1.077456in}}%
\pgfpathlineto{\pgfqpoint{0.827098in}{1.078543in}}%
\pgfpathlineto{\pgfqpoint{0.828201in}{1.080126in}}%
\pgfpathlineto{\pgfqpoint{0.828990in}{1.081212in}}%
\pgfpathlineto{\pgfqpoint{0.830085in}{1.082454in}}%
\pgfpathlineto{\pgfqpoint{0.831046in}{1.083541in}}%
\pgfpathlineto{\pgfqpoint{0.832149in}{1.084534in}}%
\pgfpathlineto{\pgfqpoint{0.833321in}{1.085589in}}%
\pgfpathlineto{\pgfqpoint{0.834431in}{1.087204in}}%
\pgfpathlineto{\pgfqpoint{0.835229in}{1.088259in}}%
\pgfpathlineto{\pgfqpoint{0.836323in}{1.089718in}}%
\pgfpathlineto{\pgfqpoint{0.837410in}{1.090804in}}%
\pgfpathlineto{\pgfqpoint{0.838520in}{1.092574in}}%
\pgfpathlineto{\pgfqpoint{0.839231in}{1.093660in}}%
\pgfpathlineto{\pgfqpoint{0.840310in}{1.095275in}}%
\pgfpathlineto{\pgfqpoint{0.840342in}{1.095275in}}%
\pgfpathlineto{\pgfqpoint{0.841194in}{1.096361in}}%
\pgfpathlineto{\pgfqpoint{0.842280in}{1.097416in}}%
\pgfpathlineto{\pgfqpoint{0.843015in}{1.098503in}}%
\pgfpathlineto{\pgfqpoint{0.844071in}{1.100024in}}%
\pgfpathlineto{\pgfqpoint{0.844094in}{1.100024in}}%
\pgfpathlineto{\pgfqpoint{0.844899in}{1.101048in}}%
\pgfpathlineto{\pgfqpoint{0.845923in}{1.102383in}}%
\pgfpathlineto{\pgfqpoint{0.846971in}{1.103439in}}%
\pgfpathlineto{\pgfqpoint{0.848073in}{1.104773in}}%
\pgfpathlineto{\pgfqpoint{0.849105in}{1.105860in}}%
\pgfpathlineto{\pgfqpoint{0.850114in}{1.107381in}}%
\pgfpathlineto{\pgfqpoint{0.850974in}{1.108467in}}%
\pgfpathlineto{\pgfqpoint{0.852013in}{1.110237in}}%
\pgfpathlineto{\pgfqpoint{0.852967in}{1.111292in}}%
\pgfpathlineto{\pgfqpoint{0.854022in}{1.112379in}}%
\pgfpathlineto{\pgfqpoint{0.854781in}{1.113434in}}%
\pgfpathlineto{\pgfqpoint{0.855891in}{1.115017in}}%
\pgfpathlineto{\pgfqpoint{0.856892in}{1.116104in}}%
\pgfpathlineto{\pgfqpoint{0.857931in}{1.117128in}}%
\pgfpathlineto{\pgfqpoint{0.858002in}{1.117128in}}%
\pgfpathlineto{\pgfqpoint{0.858987in}{1.118215in}}%
\pgfpathlineto{\pgfqpoint{0.860073in}{1.119705in}}%
\pgfpathlineto{\pgfqpoint{0.860871in}{1.120760in}}%
\pgfpathlineto{\pgfqpoint{0.861942in}{1.122095in}}%
\pgfpathlineto{\pgfqpoint{0.861981in}{1.122095in}}%
\pgfpathlineto{\pgfqpoint{0.862880in}{1.123181in}}%
\pgfpathlineto{\pgfqpoint{0.863959in}{1.124578in}}%
\pgfpathlineto{\pgfqpoint{0.864834in}{1.125665in}}%
\pgfpathlineto{\pgfqpoint{0.865929in}{1.126720in}}%
\pgfpathlineto{\pgfqpoint{0.866781in}{1.127807in}}%
\pgfpathlineto{\pgfqpoint{0.867883in}{1.129110in}}%
\pgfpathlineto{\pgfqpoint{0.869001in}{1.130197in}}%
\pgfpathlineto{\pgfqpoint{0.870088in}{1.131687in}}%
\pgfpathlineto{\pgfqpoint{0.870103in}{1.131687in}}%
\pgfpathlineto{\pgfqpoint{0.870924in}{1.132773in}}%
\pgfpathlineto{\pgfqpoint{0.871980in}{1.133829in}}%
\pgfpathlineto{\pgfqpoint{0.873113in}{1.134915in}}%
\pgfpathlineto{\pgfqpoint{0.874176in}{1.136592in}}%
\pgfpathlineto{\pgfqpoint{0.875474in}{1.137678in}}%
\pgfpathlineto{\pgfqpoint{0.876576in}{1.139106in}}%
\pgfpathlineto{\pgfqpoint{0.877210in}{1.140193in}}%
\pgfpathlineto{\pgfqpoint{0.878296in}{1.141186in}}%
\pgfpathlineto{\pgfqpoint{0.879289in}{1.142272in}}%
\pgfpathlineto{\pgfqpoint{0.880368in}{1.143483in}}%
\pgfpathlineto{\pgfqpoint{0.881658in}{1.144569in}}%
\pgfpathlineto{\pgfqpoint{0.882729in}{1.146184in}}%
\pgfpathlineto{\pgfqpoint{0.884066in}{1.147239in}}%
\pgfpathlineto{\pgfqpoint{0.885113in}{1.148574in}}%
\pgfpathlineto{\pgfqpoint{0.886169in}{1.149660in}}%
\pgfpathlineto{\pgfqpoint{0.887263in}{1.150747in}}%
\pgfpathlineto{\pgfqpoint{0.888100in}{1.151833in}}%
\pgfpathlineto{\pgfqpoint{0.889124in}{1.152827in}}%
\pgfpathlineto{\pgfqpoint{0.890367in}{1.153913in}}%
\pgfpathlineto{\pgfqpoint{0.891438in}{1.155279in}}%
\pgfpathlineto{\pgfqpoint{0.892642in}{1.156365in}}%
\pgfpathlineto{\pgfqpoint{0.893744in}{1.157390in}}%
\pgfpathlineto{\pgfqpoint{0.894698in}{1.158476in}}%
\pgfpathlineto{\pgfqpoint{0.895792in}{1.159749in}}%
\pgfpathlineto{\pgfqpoint{0.896949in}{1.160836in}}%
\pgfpathlineto{\pgfqpoint{0.898044in}{1.161829in}}%
\pgfpathlineto{\pgfqpoint{0.899107in}{1.162915in}}%
\pgfpathlineto{\pgfqpoint{0.900193in}{1.164405in}}%
\pgfpathlineto{\pgfqpoint{0.900920in}{1.165461in}}%
\pgfpathlineto{\pgfqpoint{0.901984in}{1.166609in}}%
\pgfpathlineto{\pgfqpoint{0.903094in}{1.167603in}}%
\pgfpathlineto{\pgfqpoint{0.904102in}{1.168689in}}%
\pgfpathlineto{\pgfqpoint{0.904196in}{1.168689in}}%
\pgfpathlineto{\pgfqpoint{0.905267in}{1.169776in}}%
\pgfpathlineto{\pgfqpoint{0.906354in}{1.170893in}}%
\pgfpathlineto{\pgfqpoint{0.907558in}{1.171980in}}%
\pgfpathlineto{\pgfqpoint{0.908582in}{1.173066in}}%
\pgfpathlineto{\pgfqpoint{0.909778in}{1.174153in}}%
\pgfpathlineto{\pgfqpoint{0.910865in}{1.175053in}}%
\pgfpathlineto{\pgfqpoint{0.911834in}{1.176139in}}%
\pgfpathlineto{\pgfqpoint{0.912921in}{1.177164in}}%
\pgfpathlineto{\pgfqpoint{0.913843in}{1.178219in}}%
\pgfpathlineto{\pgfqpoint{0.914953in}{1.179150in}}%
\pgfpathlineto{\pgfqpoint{0.916071in}{1.180237in}}%
\pgfpathlineto{\pgfqpoint{0.917166in}{1.181510in}}%
\pgfpathlineto{\pgfqpoint{0.918237in}{1.182534in}}%
\pgfpathlineto{\pgfqpoint{0.919347in}{1.183683in}}%
\pgfpathlineto{\pgfqpoint{0.920128in}{1.184769in}}%
\pgfpathlineto{\pgfqpoint{0.921207in}{1.185887in}}%
\pgfpathlineto{\pgfqpoint{0.922489in}{1.186942in}}%
\pgfpathlineto{\pgfqpoint{0.923584in}{1.187966in}}%
\pgfpathlineto{\pgfqpoint{0.924631in}{1.189053in}}%
\pgfpathlineto{\pgfqpoint{0.925742in}{1.189922in}}%
\pgfpathlineto{\pgfqpoint{0.926781in}{1.191008in}}%
\pgfpathlineto{\pgfqpoint{0.927813in}{1.192374in}}%
\pgfpathlineto{\pgfqpoint{0.929150in}{1.193461in}}%
\pgfpathlineto{\pgfqpoint{0.930252in}{1.194609in}}%
\pgfpathlineto{\pgfqpoint{0.931097in}{1.195696in}}%
\pgfpathlineto{\pgfqpoint{0.932129in}{1.197248in}}%
\pgfpathlineto{\pgfqpoint{0.933192in}{1.198334in}}%
\pgfpathlineto{\pgfqpoint{0.934286in}{1.199204in}}%
\pgfpathlineto{\pgfqpoint{0.935482in}{1.200290in}}%
\pgfpathlineto{\pgfqpoint{0.936569in}{1.201345in}}%
\pgfpathlineto{\pgfqpoint{0.937820in}{1.202432in}}%
\pgfpathlineto{\pgfqpoint{0.938899in}{1.203674in}}%
\pgfpathlineto{\pgfqpoint{0.940517in}{1.204760in}}%
\pgfpathlineto{\pgfqpoint{0.941619in}{1.205722in}}%
\pgfpathlineto{\pgfqpoint{0.943042in}{1.206809in}}%
\pgfpathlineto{\pgfqpoint{0.944144in}{1.207771in}}%
\pgfpathlineto{\pgfqpoint{0.945489in}{1.208858in}}%
\pgfpathlineto{\pgfqpoint{0.946576in}{1.209913in}}%
\pgfpathlineto{\pgfqpoint{0.947662in}{1.211000in}}%
\pgfpathlineto{\pgfqpoint{0.948772in}{1.211807in}}%
\pgfpathlineto{\pgfqpoint{0.949984in}{1.212893in}}%
\pgfpathlineto{\pgfqpoint{0.951063in}{1.213514in}}%
\pgfpathlineto{\pgfqpoint{0.952736in}{1.214569in}}%
\pgfpathlineto{\pgfqpoint{0.953830in}{1.215594in}}%
\pgfpathlineto{\pgfqpoint{0.954893in}{1.216680in}}%
\pgfpathlineto{\pgfqpoint{0.956004in}{1.218077in}}%
\pgfpathlineto{\pgfqpoint{0.957270in}{1.219164in}}%
\pgfpathlineto{\pgfqpoint{0.958341in}{1.220281in}}%
\pgfpathlineto{\pgfqpoint{0.959600in}{1.221368in}}%
\pgfpathlineto{\pgfqpoint{0.960710in}{1.222547in}}%
\pgfpathlineto{\pgfqpoint{0.961750in}{1.223634in}}%
\pgfpathlineto{\pgfqpoint{0.962821in}{1.224472in}}%
\pgfpathlineto{\pgfqpoint{0.964079in}{1.225558in}}%
\pgfpathlineto{\pgfqpoint{0.965166in}{1.226769in}}%
\pgfpathlineto{\pgfqpoint{0.966432in}{1.227855in}}%
\pgfpathlineto{\pgfqpoint{0.967542in}{1.229128in}}%
\pgfpathlineto{\pgfqpoint{0.968355in}{1.230215in}}%
\pgfpathlineto{\pgfqpoint{0.969442in}{1.231239in}}%
\pgfpathlineto{\pgfqpoint{0.970669in}{1.232326in}}%
\pgfpathlineto{\pgfqpoint{0.971748in}{1.233350in}}%
\pgfpathlineto{\pgfqpoint{0.972991in}{1.234436in}}%
\pgfpathlineto{\pgfqpoint{0.974086in}{1.235430in}}%
\pgfpathlineto{\pgfqpoint{0.975290in}{1.236516in}}%
\pgfpathlineto{\pgfqpoint{0.976314in}{1.237447in}}%
\pgfpathlineto{\pgfqpoint{0.977666in}{1.238534in}}%
\pgfpathlineto{\pgfqpoint{0.978690in}{1.239589in}}%
\pgfpathlineto{\pgfqpoint{0.978706in}{1.239589in}}%
\pgfpathlineto{\pgfqpoint{0.980105in}{1.240676in}}%
\pgfpathlineto{\pgfqpoint{0.981200in}{1.241980in}}%
\pgfpathlineto{\pgfqpoint{0.982505in}{1.243066in}}%
\pgfpathlineto{\pgfqpoint{0.983569in}{1.243842in}}%
\pgfpathlineto{\pgfqpoint{0.983608in}{1.243842in}}%
\pgfpathlineto{\pgfqpoint{0.985328in}{1.244929in}}%
\pgfpathlineto{\pgfqpoint{0.986328in}{1.245860in}}%
\pgfpathlineto{\pgfqpoint{0.987626in}{1.246946in}}%
\pgfpathlineto{\pgfqpoint{0.988666in}{1.247722in}}%
\pgfpathlineto{\pgfqpoint{0.988728in}{1.247722in}}%
\pgfpathlineto{\pgfqpoint{0.989885in}{1.248809in}}%
\pgfpathlineto{\pgfqpoint{0.990956in}{1.249802in}}%
\pgfpathlineto{\pgfqpoint{0.992371in}{1.250889in}}%
\pgfpathlineto{\pgfqpoint{0.993474in}{1.252006in}}%
\pgfpathlineto{\pgfqpoint{0.994670in}{1.253093in}}%
\pgfpathlineto{\pgfqpoint{0.995764in}{1.254272in}}%
\pgfpathlineto{\pgfqpoint{0.997476in}{1.255359in}}%
\pgfpathlineto{\pgfqpoint{0.998586in}{1.256476in}}%
\pgfpathlineto{\pgfqpoint{1.000533in}{1.257563in}}%
\pgfpathlineto{\pgfqpoint{1.001643in}{1.258525in}}%
\pgfpathlineto{\pgfqpoint{1.003003in}{1.259581in}}%
\pgfpathlineto{\pgfqpoint{1.004027in}{1.260667in}}%
\pgfpathlineto{\pgfqpoint{1.005435in}{1.261753in}}%
\pgfpathlineto{\pgfqpoint{1.006529in}{1.262716in}}%
\pgfpathlineto{\pgfqpoint{1.007803in}{1.263802in}}%
\pgfpathlineto{\pgfqpoint{1.008906in}{1.264547in}}%
\pgfpathlineto{\pgfqpoint{1.010790in}{1.265634in}}%
\pgfpathlineto{\pgfqpoint{1.011822in}{1.266596in}}%
\pgfpathlineto{\pgfqpoint{1.013346in}{1.267683in}}%
\pgfpathlineto{\pgfqpoint{1.014394in}{1.268241in}}%
\pgfpathlineto{\pgfqpoint{1.015637in}{1.269328in}}%
\pgfpathlineto{\pgfqpoint{1.016747in}{1.270042in}}%
\pgfpathlineto{\pgfqpoint{1.018209in}{1.271097in}}%
\pgfpathlineto{\pgfqpoint{1.019311in}{1.271904in}}%
\pgfpathlineto{\pgfqpoint{1.020546in}{1.272991in}}%
\pgfpathlineto{\pgfqpoint{1.021648in}{1.274046in}}%
\pgfpathlineto{\pgfqpoint{1.022743in}{1.275133in}}%
\pgfpathlineto{\pgfqpoint{1.023853in}{1.276157in}}%
\pgfpathlineto{\pgfqpoint{1.025010in}{1.277212in}}%
\pgfpathlineto{\pgfqpoint{1.026089in}{1.278113in}}%
\pgfpathlineto{\pgfqpoint{1.027371in}{1.279199in}}%
\pgfpathlineto{\pgfqpoint{1.028309in}{1.279820in}}%
\pgfpathlineto{\pgfqpoint{1.029779in}{1.280906in}}%
\pgfpathlineto{\pgfqpoint{1.030889in}{1.281745in}}%
\pgfpathlineto{\pgfqpoint{1.032359in}{1.282831in}}%
\pgfpathlineto{\pgfqpoint{1.033461in}{1.283793in}}%
\pgfpathlineto{\pgfqpoint{1.034759in}{1.284849in}}%
\pgfpathlineto{\pgfqpoint{1.035822in}{1.285470in}}%
\pgfpathlineto{\pgfqpoint{1.037276in}{1.286556in}}%
\pgfpathlineto{\pgfqpoint{1.038378in}{1.287177in}}%
\pgfpathlineto{\pgfqpoint{1.040254in}{1.288263in}}%
\pgfpathlineto{\pgfqpoint{1.041318in}{1.288915in}}%
\pgfpathlineto{\pgfqpoint{1.042936in}{1.290002in}}%
\pgfpathlineto{\pgfqpoint{1.043968in}{1.290747in}}%
\pgfpathlineto{\pgfqpoint{1.045508in}{1.291833in}}%
\pgfpathlineto{\pgfqpoint{1.046540in}{1.292547in}}%
\pgfpathlineto{\pgfqpoint{1.048158in}{1.293634in}}%
\pgfpathlineto{\pgfqpoint{1.049206in}{1.294472in}}%
\pgfpathlineto{\pgfqpoint{1.050589in}{1.295558in}}%
\pgfpathlineto{\pgfqpoint{1.051692in}{1.296365in}}%
\pgfpathlineto{\pgfqpoint{1.053443in}{1.297452in}}%
\pgfpathlineto{\pgfqpoint{1.054506in}{1.298197in}}%
\pgfpathlineto{\pgfqpoint{1.056140in}{1.299283in}}%
\pgfpathlineto{\pgfqpoint{1.057234in}{1.299935in}}%
\pgfpathlineto{\pgfqpoint{1.058892in}{1.300991in}}%
\pgfpathlineto{\pgfqpoint{1.059994in}{1.301953in}}%
\pgfpathlineto{\pgfqpoint{1.061956in}{1.303040in}}%
\pgfpathlineto{\pgfqpoint{1.063019in}{1.303660in}}%
\pgfpathlineto{\pgfqpoint{1.064286in}{1.304747in}}%
\pgfpathlineto{\pgfqpoint{1.065279in}{1.305430in}}%
\pgfpathlineto{\pgfqpoint{1.065372in}{1.305430in}}%
\pgfpathlineto{\pgfqpoint{1.066944in}{1.306516in}}%
\pgfpathlineto{\pgfqpoint{1.068030in}{1.307385in}}%
\pgfpathlineto{\pgfqpoint{1.069750in}{1.308472in}}%
\pgfpathlineto{\pgfqpoint{1.070845in}{1.309031in}}%
\pgfpathlineto{\pgfqpoint{1.072479in}{1.310117in}}%
\pgfpathlineto{\pgfqpoint{1.073518in}{1.310831in}}%
\pgfpathlineto{\pgfqpoint{1.074949in}{1.311918in}}%
\pgfpathlineto{\pgfqpoint{1.076020in}{1.312756in}}%
\pgfpathlineto{\pgfqpoint{1.077490in}{1.313842in}}%
\pgfpathlineto{\pgfqpoint{1.078600in}{1.314867in}}%
\pgfpathlineto{\pgfqpoint{1.079835in}{1.315922in}}%
\pgfpathlineto{\pgfqpoint{1.080914in}{1.316729in}}%
\pgfpathlineto{\pgfqpoint{1.080945in}{1.316729in}}%
\pgfpathlineto{\pgfqpoint{1.082837in}{1.317816in}}%
\pgfpathlineto{\pgfqpoint{1.083908in}{1.318654in}}%
\pgfpathlineto{\pgfqpoint{1.085698in}{1.319740in}}%
\pgfpathlineto{\pgfqpoint{1.086761in}{1.320485in}}%
\pgfpathlineto{\pgfqpoint{1.088474in}{1.321572in}}%
\pgfpathlineto{\pgfqpoint{1.089560in}{1.322379in}}%
\pgfpathlineto{\pgfqpoint{1.091687in}{1.323434in}}%
\pgfpathlineto{\pgfqpoint{1.092797in}{1.324148in}}%
\pgfpathlineto{\pgfqpoint{1.094610in}{1.325235in}}%
\pgfpathlineto{\pgfqpoint{1.095619in}{1.326259in}}%
\pgfpathlineto{\pgfqpoint{1.097261in}{1.327345in}}%
\pgfpathlineto{\pgfqpoint{1.098316in}{1.328215in}}%
\pgfpathlineto{\pgfqpoint{1.100255in}{1.329301in}}%
\pgfpathlineto{\pgfqpoint{1.101341in}{1.330170in}}%
\pgfpathlineto{\pgfqpoint{1.103030in}{1.331257in}}%
\pgfpathlineto{\pgfqpoint{1.104031in}{1.332095in}}%
\pgfpathlineto{\pgfqpoint{1.106095in}{1.333181in}}%
\pgfpathlineto{\pgfqpoint{1.107158in}{1.333895in}}%
\pgfpathlineto{\pgfqpoint{1.108627in}{1.334982in}}%
\pgfpathlineto{\pgfqpoint{1.109706in}{1.335665in}}%
\pgfpathlineto{\pgfqpoint{1.110746in}{1.336751in}}%
\pgfpathlineto{\pgfqpoint{1.111840in}{1.337403in}}%
\pgfpathlineto{\pgfqpoint{1.113693in}{1.338490in}}%
\pgfpathlineto{\pgfqpoint{1.114733in}{1.339328in}}%
\pgfpathlineto{\pgfqpoint{1.117063in}{1.340414in}}%
\pgfpathlineto{\pgfqpoint{1.118040in}{1.341128in}}%
\pgfpathlineto{\pgfqpoint{1.118173in}{1.341128in}}%
\pgfpathlineto{\pgfqpoint{1.119947in}{1.342215in}}%
\pgfpathlineto{\pgfqpoint{1.121206in}{1.342804in}}%
\pgfpathlineto{\pgfqpoint{1.122543in}{1.343891in}}%
\pgfpathlineto{\pgfqpoint{1.123536in}{1.344481in}}%
\pgfpathlineto{\pgfqpoint{1.123606in}{1.344481in}}%
\pgfpathlineto{\pgfqpoint{1.125404in}{1.345567in}}%
\pgfpathlineto{\pgfqpoint{1.126491in}{1.346219in}}%
\pgfpathlineto{\pgfqpoint{1.128328in}{1.347306in}}%
\pgfpathlineto{\pgfqpoint{1.129422in}{1.347989in}}%
\pgfpathlineto{\pgfqpoint{1.131643in}{1.349075in}}%
\pgfpathlineto{\pgfqpoint{1.132682in}{1.349820in}}%
\pgfpathlineto{\pgfqpoint{1.134598in}{1.350875in}}%
\pgfpathlineto{\pgfqpoint{1.135684in}{1.351714in}}%
\pgfpathlineto{\pgfqpoint{1.137475in}{1.352800in}}%
\pgfpathlineto{\pgfqpoint{1.138561in}{1.353731in}}%
\pgfpathlineto{\pgfqpoint{1.140476in}{1.354818in}}%
\pgfpathlineto{\pgfqpoint{1.141469in}{1.355439in}}%
\pgfpathlineto{\pgfqpoint{1.143822in}{1.356525in}}%
\pgfpathlineto{\pgfqpoint{1.144925in}{1.357612in}}%
\pgfpathlineto{\pgfqpoint{1.146512in}{1.358667in}}%
\pgfpathlineto{\pgfqpoint{1.147606in}{1.359412in}}%
\pgfpathlineto{\pgfqpoint{1.149811in}{1.360498in}}%
\pgfpathlineto{\pgfqpoint{1.150843in}{1.361119in}}%
\pgfpathlineto{\pgfqpoint{1.152711in}{1.362206in}}%
\pgfpathlineto{\pgfqpoint{1.153821in}{1.363013in}}%
\pgfpathlineto{\pgfqpoint{1.155533in}{1.364099in}}%
\pgfpathlineto{\pgfqpoint{1.156589in}{1.364844in}}%
\pgfpathlineto{\pgfqpoint{1.158754in}{1.365931in}}%
\pgfpathlineto{\pgfqpoint{1.159559in}{1.366117in}}%
\pgfpathlineto{\pgfqpoint{1.159692in}{1.366117in}}%
\pgfpathlineto{\pgfqpoint{1.161287in}{1.367204in}}%
\pgfpathlineto{\pgfqpoint{1.162374in}{1.368259in}}%
\pgfpathlineto{\pgfqpoint{1.164336in}{1.369345in}}%
\pgfpathlineto{\pgfqpoint{1.165391in}{1.370339in}}%
\pgfpathlineto{\pgfqpoint{1.167103in}{1.371425in}}%
\pgfpathlineto{\pgfqpoint{1.168174in}{1.372419in}}%
\pgfpathlineto{\pgfqpoint{1.168198in}{1.372419in}}%
\pgfpathlineto{\pgfqpoint{1.169933in}{1.373474in}}%
\pgfpathlineto{\pgfqpoint{1.171028in}{1.374467in}}%
\pgfpathlineto{\pgfqpoint{1.173178in}{1.375554in}}%
\pgfpathlineto{\pgfqpoint{1.174280in}{1.376206in}}%
\pgfpathlineto{\pgfqpoint{1.176313in}{1.377292in}}%
\pgfpathlineto{\pgfqpoint{1.177360in}{1.377975in}}%
\pgfpathlineto{\pgfqpoint{1.177415in}{1.377975in}}%
\pgfpathlineto{\pgfqpoint{1.179393in}{1.379062in}}%
\pgfpathlineto{\pgfqpoint{1.180487in}{1.379745in}}%
\pgfpathlineto{\pgfqpoint{1.182426in}{1.380831in}}%
\pgfpathlineto{\pgfqpoint{1.183536in}{1.381421in}}%
\pgfpathlineto{\pgfqpoint{1.185850in}{1.382507in}}%
\pgfpathlineto{\pgfqpoint{1.186960in}{1.383221in}}%
\pgfpathlineto{\pgfqpoint{1.189032in}{1.384308in}}%
\pgfpathlineto{\pgfqpoint{1.190017in}{1.385177in}}%
\pgfpathlineto{\pgfqpoint{1.192104in}{1.386263in}}%
\pgfpathlineto{\pgfqpoint{1.193199in}{1.387164in}}%
\pgfpathlineto{\pgfqpoint{1.195200in}{1.388219in}}%
\pgfpathlineto{\pgfqpoint{1.196240in}{1.388964in}}%
\pgfpathlineto{\pgfqpoint{1.198343in}{1.390020in}}%
\pgfpathlineto{\pgfqpoint{1.199429in}{1.390765in}}%
\pgfpathlineto{\pgfqpoint{1.201947in}{1.391851in}}%
\pgfpathlineto{\pgfqpoint{1.203049in}{1.392534in}}%
\pgfpathlineto{\pgfqpoint{1.204487in}{1.393589in}}%
\pgfpathlineto{\pgfqpoint{1.205558in}{1.394148in}}%
\pgfpathlineto{\pgfqpoint{1.207583in}{1.395235in}}%
\pgfpathlineto{\pgfqpoint{1.208685in}{1.396011in}}%
\pgfpathlineto{\pgfqpoint{1.210796in}{1.397097in}}%
\pgfpathlineto{\pgfqpoint{1.211836in}{1.397935in}}%
\pgfpathlineto{\pgfqpoint{1.213853in}{1.399022in}}%
\pgfpathlineto{\pgfqpoint{1.214908in}{1.399798in}}%
\pgfpathlineto{\pgfqpoint{1.216902in}{1.400884in}}%
\pgfpathlineto{\pgfqpoint{1.218012in}{1.401350in}}%
\pgfpathlineto{\pgfqpoint{1.220076in}{1.402436in}}%
\pgfpathlineto{\pgfqpoint{1.221014in}{1.403057in}}%
\pgfpathlineto{\pgfqpoint{1.222929in}{1.404144in}}%
\pgfpathlineto{\pgfqpoint{1.224039in}{1.404796in}}%
\pgfpathlineto{\pgfqpoint{1.226799in}{1.405882in}}%
\pgfpathlineto{\pgfqpoint{1.227886in}{1.406689in}}%
\pgfpathlineto{\pgfqpoint{1.229926in}{1.407776in}}%
\pgfpathlineto{\pgfqpoint{1.230981in}{1.408583in}}%
\pgfpathlineto{\pgfqpoint{1.231036in}{1.408583in}}%
\pgfpathlineto{\pgfqpoint{1.233014in}{1.409638in}}%
\pgfpathlineto{\pgfqpoint{1.234108in}{1.410445in}}%
\pgfpathlineto{\pgfqpoint{1.236532in}{1.411532in}}%
\pgfpathlineto{\pgfqpoint{1.237603in}{1.412246in}}%
\pgfpathlineto{\pgfqpoint{1.239956in}{1.413332in}}%
\pgfpathlineto{\pgfqpoint{1.240925in}{1.413705in}}%
\pgfpathlineto{\pgfqpoint{1.241066in}{1.413705in}}%
\pgfpathlineto{\pgfqpoint{1.243013in}{1.414791in}}%
\pgfpathlineto{\pgfqpoint{1.244271in}{1.415412in}}%
\pgfpathlineto{\pgfqpoint{1.246726in}{1.416498in}}%
\pgfpathlineto{\pgfqpoint{1.247797in}{1.417026in}}%
\pgfpathlineto{\pgfqpoint{1.247836in}{1.417026in}}%
\pgfpathlineto{\pgfqpoint{1.250236in}{1.418113in}}%
\pgfpathlineto{\pgfqpoint{1.251276in}{1.418889in}}%
\pgfpathlineto{\pgfqpoint{1.253449in}{1.419975in}}%
\pgfpathlineto{\pgfqpoint{1.254544in}{1.420565in}}%
\pgfpathlineto{\pgfqpoint{1.257131in}{1.421651in}}%
\pgfpathlineto{\pgfqpoint{1.258241in}{1.422179in}}%
\pgfpathlineto{\pgfqpoint{1.260376in}{1.423266in}}%
\pgfpathlineto{\pgfqpoint{1.261369in}{1.423731in}}%
\pgfpathlineto{\pgfqpoint{1.263941in}{1.424818in}}%
\pgfpathlineto{\pgfqpoint{1.264941in}{1.425252in}}%
\pgfpathlineto{\pgfqpoint{1.267388in}{1.426339in}}%
\pgfpathlineto{\pgfqpoint{1.268373in}{1.426649in}}%
\pgfpathlineto{\pgfqpoint{1.271156in}{1.427705in}}%
\pgfpathlineto{\pgfqpoint{1.272258in}{1.428108in}}%
\pgfpathlineto{\pgfqpoint{1.274408in}{1.429195in}}%
\pgfpathlineto{\pgfqpoint{1.275393in}{1.429691in}}%
\pgfpathlineto{\pgfqpoint{1.278364in}{1.430778in}}%
\pgfpathlineto{\pgfqpoint{1.279412in}{1.431306in}}%
\pgfpathlineto{\pgfqpoint{1.279459in}{1.431306in}}%
\pgfpathlineto{\pgfqpoint{1.281155in}{1.432392in}}%
\pgfpathlineto{\pgfqpoint{1.282218in}{1.432951in}}%
\pgfpathlineto{\pgfqpoint{1.284047in}{1.434037in}}%
\pgfpathlineto{\pgfqpoint{1.285009in}{1.434379in}}%
\pgfpathlineto{\pgfqpoint{1.287464in}{1.435465in}}%
\pgfpathlineto{\pgfqpoint{1.288707in}{1.436055in}}%
\pgfpathlineto{\pgfqpoint{1.291459in}{1.437142in}}%
\pgfpathlineto{\pgfqpoint{1.292459in}{1.437793in}}%
\pgfpathlineto{\pgfqpoint{1.294664in}{1.438880in}}%
\pgfpathlineto{\pgfqpoint{1.295758in}{1.439439in}}%
\pgfpathlineto{\pgfqpoint{1.298377in}{1.440525in}}%
\pgfpathlineto{\pgfqpoint{1.299417in}{1.441084in}}%
\pgfpathlineto{\pgfqpoint{1.301059in}{1.442170in}}%
\pgfpathlineto{\pgfqpoint{1.302169in}{1.443102in}}%
\pgfpathlineto{\pgfqpoint{1.304975in}{1.444188in}}%
\pgfpathlineto{\pgfqpoint{1.306062in}{1.444995in}}%
\pgfpathlineto{\pgfqpoint{1.308697in}{1.446082in}}%
\pgfpathlineto{\pgfqpoint{1.309807in}{1.446734in}}%
\pgfpathlineto{\pgfqpoint{1.312566in}{1.447820in}}%
\pgfpathlineto{\pgfqpoint{1.313676in}{1.448348in}}%
\pgfpathlineto{\pgfqpoint{1.316147in}{1.449434in}}%
\pgfpathlineto{\pgfqpoint{1.317249in}{1.449900in}}%
\pgfpathlineto{\pgfqpoint{1.320407in}{1.450986in}}%
\pgfpathlineto{\pgfqpoint{1.321236in}{1.451421in}}%
\pgfpathlineto{\pgfqpoint{1.321431in}{1.451421in}}%
\pgfpathlineto{\pgfqpoint{1.324473in}{1.452507in}}%
\pgfpathlineto{\pgfqpoint{1.325379in}{1.453035in}}%
\pgfpathlineto{\pgfqpoint{1.328256in}{1.454091in}}%
\pgfpathlineto{\pgfqpoint{1.329171in}{1.454649in}}%
\pgfpathlineto{\pgfqpoint{1.331759in}{1.455736in}}%
\pgfpathlineto{\pgfqpoint{1.332845in}{1.456170in}}%
\pgfpathlineto{\pgfqpoint{1.335245in}{1.457257in}}%
\pgfpathlineto{\pgfqpoint{1.336293in}{1.457722in}}%
\pgfpathlineto{\pgfqpoint{1.339342in}{1.458809in}}%
\pgfpathlineto{\pgfqpoint{1.340436in}{1.459212in}}%
\pgfpathlineto{\pgfqpoint{1.342953in}{1.460268in}}%
\pgfpathlineto{\pgfqpoint{1.344048in}{1.460765in}}%
\pgfpathlineto{\pgfqpoint{1.347050in}{1.461851in}}%
\pgfpathlineto{\pgfqpoint{1.348082in}{1.462410in}}%
\pgfpathlineto{\pgfqpoint{1.351826in}{1.463496in}}%
\pgfpathlineto{\pgfqpoint{1.352905in}{1.463962in}}%
\pgfpathlineto{\pgfqpoint{1.354938in}{1.465048in}}%
\pgfpathlineto{\pgfqpoint{1.356032in}{1.465483in}}%
\pgfpathlineto{\pgfqpoint{1.358354in}{1.466569in}}%
\pgfpathlineto{\pgfqpoint{1.359386in}{1.466973in}}%
\pgfpathlineto{\pgfqpoint{1.359456in}{1.466973in}}%
\pgfpathlineto{\pgfqpoint{1.363717in}{1.468059in}}%
\pgfpathlineto{\pgfqpoint{1.364577in}{1.468308in}}%
\pgfpathlineto{\pgfqpoint{1.364827in}{1.468308in}}%
\pgfpathlineto{\pgfqpoint{1.367227in}{1.469394in}}%
\pgfpathlineto{\pgfqpoint{1.368283in}{1.470015in}}%
\pgfpathlineto{\pgfqpoint{1.371668in}{1.471102in}}%
\pgfpathlineto{\pgfqpoint{1.372762in}{1.471598in}}%
\pgfpathlineto{\pgfqpoint{1.375100in}{1.472685in}}%
\pgfpathlineto{\pgfqpoint{1.376053in}{1.473244in}}%
\pgfpathlineto{\pgfqpoint{1.378875in}{1.474330in}}%
\pgfpathlineto{\pgfqpoint{1.379853in}{1.474920in}}%
\pgfpathlineto{\pgfqpoint{1.382737in}{1.476006in}}%
\pgfpathlineto{\pgfqpoint{1.383793in}{1.476472in}}%
\pgfpathlineto{\pgfqpoint{1.386638in}{1.477527in}}%
\pgfpathlineto{\pgfqpoint{1.387756in}{1.478086in}}%
\pgfpathlineto{\pgfqpoint{1.389593in}{1.479173in}}%
\pgfpathlineto{\pgfqpoint{1.390704in}{1.479793in}}%
\pgfpathlineto{\pgfqpoint{1.393494in}{1.480880in}}%
\pgfpathlineto{\pgfqpoint{1.394597in}{1.481314in}}%
\pgfpathlineto{\pgfqpoint{1.397857in}{1.482401in}}%
\pgfpathlineto{\pgfqpoint{1.398732in}{1.482960in}}%
\pgfpathlineto{\pgfqpoint{1.402282in}{1.484046in}}%
\pgfpathlineto{\pgfqpoint{1.403368in}{1.484512in}}%
\pgfpathlineto{\pgfqpoint{1.407011in}{1.485598in}}%
\pgfpathlineto{\pgfqpoint{1.407957in}{1.486281in}}%
\pgfpathlineto{\pgfqpoint{1.411100in}{1.487368in}}%
\pgfpathlineto{\pgfqpoint{1.412108in}{1.487864in}}%
\pgfpathlineto{\pgfqpoint{1.415751in}{1.488858in}}%
\pgfpathlineto{\pgfqpoint{1.416736in}{1.489447in}}%
\pgfpathlineto{\pgfqpoint{1.416807in}{1.489447in}}%
\pgfpathlineto{\pgfqpoint{1.420473in}{1.490534in}}%
\pgfpathlineto{\pgfqpoint{1.421552in}{1.490938in}}%
\pgfpathlineto{\pgfqpoint{1.424359in}{1.492024in}}%
\pgfpathlineto{\pgfqpoint{1.425390in}{1.492521in}}%
\pgfpathlineto{\pgfqpoint{1.428533in}{1.493607in}}%
\pgfpathlineto{\pgfqpoint{1.429542in}{1.494135in}}%
\pgfpathlineto{\pgfqpoint{1.432747in}{1.495221in}}%
\pgfpathlineto{\pgfqpoint{1.433818in}{1.495780in}}%
\pgfpathlineto{\pgfqpoint{1.437273in}{1.496867in}}%
\pgfpathlineto{\pgfqpoint{1.438180in}{1.497270in}}%
\pgfpathlineto{\pgfqpoint{1.442378in}{1.498357in}}%
\pgfpathlineto{\pgfqpoint{1.443473in}{1.498636in}}%
\pgfpathlineto{\pgfqpoint{1.447905in}{1.499722in}}%
\pgfpathlineto{\pgfqpoint{1.448757in}{1.500157in}}%
\pgfpathlineto{\pgfqpoint{1.452111in}{1.501244in}}%
\pgfpathlineto{\pgfqpoint{1.453018in}{1.501554in}}%
\pgfpathlineto{\pgfqpoint{1.457154in}{1.502640in}}%
\pgfpathlineto{\pgfqpoint{1.458217in}{1.503106in}}%
\pgfpathlineto{\pgfqpoint{1.458264in}{1.503106in}}%
\pgfpathlineto{\pgfqpoint{1.460367in}{1.504193in}}%
\pgfpathlineto{\pgfqpoint{1.460984in}{1.504472in}}%
\pgfpathlineto{\pgfqpoint{1.461469in}{1.504472in}}%
\pgfpathlineto{\pgfqpoint{1.465956in}{1.505558in}}%
\pgfpathlineto{\pgfqpoint{1.466996in}{1.505900in}}%
\pgfpathlineto{\pgfqpoint{1.467027in}{1.505900in}}%
\pgfpathlineto{\pgfqpoint{1.470006in}{1.506986in}}%
\pgfpathlineto{\pgfqpoint{1.471092in}{1.507359in}}%
\pgfpathlineto{\pgfqpoint{1.474001in}{1.508414in}}%
\pgfpathlineto{\pgfqpoint{1.475040in}{1.508880in}}%
\pgfpathlineto{\pgfqpoint{1.475079in}{1.508880in}}%
\pgfpathlineto{\pgfqpoint{1.478496in}{1.509966in}}%
\pgfpathlineto{\pgfqpoint{1.479238in}{1.510246in}}%
\pgfpathlineto{\pgfqpoint{1.482475in}{1.511332in}}%
\pgfpathlineto{\pgfqpoint{1.483530in}{1.511829in}}%
\pgfpathlineto{\pgfqpoint{1.483585in}{1.511829in}}%
\pgfpathlineto{\pgfqpoint{1.486032in}{1.512915in}}%
\pgfpathlineto{\pgfqpoint{1.487040in}{1.513443in}}%
\pgfpathlineto{\pgfqpoint{1.490511in}{1.514530in}}%
\pgfpathlineto{\pgfqpoint{1.491582in}{1.514964in}}%
\pgfpathlineto{\pgfqpoint{1.495311in}{1.516051in}}%
\pgfpathlineto{\pgfqpoint{1.496367in}{1.516454in}}%
\pgfpathlineto{\pgfqpoint{1.500213in}{1.517541in}}%
\pgfpathlineto{\pgfqpoint{1.501151in}{1.517758in}}%
\pgfpathlineto{\pgfqpoint{1.501245in}{1.517758in}}%
\pgfpathlineto{\pgfqpoint{1.505639in}{1.518844in}}%
\pgfpathlineto{\pgfqpoint{1.506647in}{1.519279in}}%
\pgfpathlineto{\pgfqpoint{1.506717in}{1.519279in}}%
\pgfpathlineto{\pgfqpoint{1.510142in}{1.520334in}}%
\pgfpathlineto{\pgfqpoint{1.511173in}{1.520831in}}%
\pgfpathlineto{\pgfqpoint{1.515207in}{1.521918in}}%
\pgfpathlineto{\pgfqpoint{1.516317in}{1.522476in}}%
\pgfpathlineto{\pgfqpoint{1.520688in}{1.523563in}}%
\pgfpathlineto{\pgfqpoint{1.521790in}{1.523904in}}%
\pgfpathlineto{\pgfqpoint{1.525644in}{1.524991in}}%
\pgfpathlineto{\pgfqpoint{1.526754in}{1.525487in}}%
\pgfpathlineto{\pgfqpoint{1.531718in}{1.526574in}}%
\pgfpathlineto{\pgfqpoint{1.532609in}{1.526977in}}%
\pgfpathlineto{\pgfqpoint{1.536784in}{1.528064in}}%
\pgfpathlineto{\pgfqpoint{1.537839in}{1.528716in}}%
\pgfpathlineto{\pgfqpoint{1.537894in}{1.528716in}}%
\pgfpathlineto{\pgfqpoint{1.541482in}{1.529771in}}%
\pgfpathlineto{\pgfqpoint{1.542569in}{1.530051in}}%
\pgfpathlineto{\pgfqpoint{1.547400in}{1.531137in}}%
\pgfpathlineto{\pgfqpoint{1.548323in}{1.531510in}}%
\pgfpathlineto{\pgfqpoint{1.552912in}{1.532596in}}%
\pgfpathlineto{\pgfqpoint{1.553920in}{1.533000in}}%
\pgfpathlineto{\pgfqpoint{1.557516in}{1.534086in}}%
\pgfpathlineto{\pgfqpoint{1.558548in}{1.534428in}}%
\pgfpathlineto{\pgfqpoint{1.562770in}{1.535514in}}%
\pgfpathlineto{\pgfqpoint{1.563825in}{1.535855in}}%
\pgfpathlineto{\pgfqpoint{1.567930in}{1.536911in}}%
\pgfpathlineto{\pgfqpoint{1.568625in}{1.537252in}}%
\pgfpathlineto{\pgfqpoint{1.573253in}{1.538339in}}%
\pgfpathlineto{\pgfqpoint{1.574340in}{1.538804in}}%
\pgfpathlineto{\pgfqpoint{1.577881in}{1.539891in}}%
\pgfpathlineto{\pgfqpoint{1.578984in}{1.540326in}}%
\pgfpathlineto{\pgfqpoint{1.582197in}{1.541412in}}%
\pgfpathlineto{\pgfqpoint{1.583252in}{1.541660in}}%
\pgfpathlineto{\pgfqpoint{1.588482in}{1.542747in}}%
\pgfpathlineto{\pgfqpoint{1.588764in}{1.542809in}}%
\pgfpathlineto{\pgfqpoint{1.589592in}{1.542809in}}%
\pgfpathlineto{\pgfqpoint{1.594564in}{1.543895in}}%
\pgfpathlineto{\pgfqpoint{1.595604in}{1.544206in}}%
\pgfpathlineto{\pgfqpoint{1.601498in}{1.545292in}}%
\pgfpathlineto{\pgfqpoint{1.602585in}{1.545696in}}%
\pgfpathlineto{\pgfqpoint{1.607065in}{1.546782in}}%
\pgfpathlineto{\pgfqpoint{1.608151in}{1.547124in}}%
\pgfpathlineto{\pgfqpoint{1.613436in}{1.548210in}}%
\pgfpathlineto{\pgfqpoint{1.614546in}{1.548552in}}%
\pgfpathlineto{\pgfqpoint{1.619721in}{1.549638in}}%
\pgfpathlineto{\pgfqpoint{1.620753in}{1.550011in}}%
\pgfpathlineto{\pgfqpoint{1.625741in}{1.551097in}}%
\pgfpathlineto{\pgfqpoint{1.626945in}{1.551345in}}%
\pgfpathlineto{\pgfqpoint{1.632925in}{1.552432in}}%
\pgfpathlineto{\pgfqpoint{1.633926in}{1.552804in}}%
\pgfpathlineto{\pgfqpoint{1.638499in}{1.553891in}}%
\pgfpathlineto{\pgfqpoint{1.639258in}{1.554294in}}%
\pgfpathlineto{\pgfqpoint{1.645801in}{1.555381in}}%
\pgfpathlineto{\pgfqpoint{1.646817in}{1.555660in}}%
\pgfpathlineto{\pgfqpoint{1.649655in}{1.556747in}}%
\pgfpathlineto{\pgfqpoint{1.650687in}{1.556933in}}%
\pgfpathlineto{\pgfqpoint{1.654526in}{1.558020in}}%
\pgfpathlineto{\pgfqpoint{1.655557in}{1.558299in}}%
\pgfpathlineto{\pgfqpoint{1.661655in}{1.559385in}}%
\pgfpathlineto{\pgfqpoint{1.662726in}{1.559665in}}%
\pgfpathlineto{\pgfqpoint{1.669645in}{1.560751in}}%
\pgfpathlineto{\pgfqpoint{1.670114in}{1.560906in}}%
\pgfpathlineto{\pgfqpoint{1.670724in}{1.560906in}}%
\pgfpathlineto{\pgfqpoint{1.677392in}{1.561993in}}%
\pgfpathlineto{\pgfqpoint{1.678448in}{1.562241in}}%
\pgfpathlineto{\pgfqpoint{1.684092in}{1.563328in}}%
\pgfpathlineto{\pgfqpoint{1.685179in}{1.563700in}}%
\pgfpathlineto{\pgfqpoint{1.692418in}{1.564787in}}%
\pgfpathlineto{\pgfqpoint{1.693512in}{1.565035in}}%
\pgfpathlineto{\pgfqpoint{1.700134in}{1.566122in}}%
\pgfpathlineto{\pgfqpoint{1.700853in}{1.566308in}}%
\pgfpathlineto{\pgfqpoint{1.701158in}{1.566308in}}%
\pgfpathlineto{\pgfqpoint{1.706951in}{1.567394in}}%
\pgfpathlineto{\pgfqpoint{1.708061in}{1.567643in}}%
\pgfpathlineto{\pgfqpoint{1.713283in}{1.568729in}}%
\pgfpathlineto{\pgfqpoint{1.714284in}{1.569040in}}%
\pgfpathlineto{\pgfqpoint{1.719506in}{1.570126in}}%
\pgfpathlineto{\pgfqpoint{1.720366in}{1.570281in}}%
\pgfpathlineto{\pgfqpoint{1.726581in}{1.571368in}}%
\pgfpathlineto{\pgfqpoint{1.727409in}{1.571740in}}%
\pgfpathlineto{\pgfqpoint{1.727613in}{1.571740in}}%
\pgfpathlineto{\pgfqpoint{1.734375in}{1.572827in}}%
\pgfpathlineto{\pgfqpoint{1.735469in}{1.573075in}}%
\pgfpathlineto{\pgfqpoint{1.741309in}{1.574161in}}%
\pgfpathlineto{\pgfqpoint{1.742372in}{1.574534in}}%
\pgfpathlineto{\pgfqpoint{1.748752in}{1.575620in}}%
\pgfpathlineto{\pgfqpoint{1.749682in}{1.575931in}}%
\pgfpathlineto{\pgfqpoint{1.755858in}{1.577017in}}%
\pgfpathlineto{\pgfqpoint{1.756905in}{1.577173in}}%
\pgfpathlineto{\pgfqpoint{1.762933in}{1.578259in}}%
\pgfpathlineto{\pgfqpoint{1.763316in}{1.578383in}}%
\pgfpathlineto{\pgfqpoint{1.763973in}{1.578383in}}%
\pgfpathlineto{\pgfqpoint{1.769367in}{1.579470in}}%
\pgfpathlineto{\pgfqpoint{1.770446in}{1.579687in}}%
\pgfpathlineto{\pgfqpoint{1.777575in}{1.580773in}}%
\pgfpathlineto{\pgfqpoint{1.778529in}{1.580929in}}%
\pgfpathlineto{\pgfqpoint{1.786018in}{1.582015in}}%
\pgfpathlineto{\pgfqpoint{1.786425in}{1.582170in}}%
\pgfpathlineto{\pgfqpoint{1.787089in}{1.582170in}}%
\pgfpathlineto{\pgfqpoint{1.794805in}{1.583257in}}%
\pgfpathlineto{\pgfqpoint{1.795876in}{1.583412in}}%
\pgfpathlineto{\pgfqpoint{1.803561in}{1.584498in}}%
\pgfpathlineto{\pgfqpoint{1.804593in}{1.584716in}}%
\pgfpathlineto{\pgfqpoint{1.815585in}{1.585802in}}%
\pgfpathlineto{\pgfqpoint{1.816687in}{1.585989in}}%
\pgfpathlineto{\pgfqpoint{1.825529in}{1.587075in}}%
\pgfpathlineto{\pgfqpoint{1.826303in}{1.587168in}}%
\pgfpathlineto{\pgfqpoint{1.835981in}{1.588255in}}%
\pgfpathlineto{\pgfqpoint{1.837052in}{1.588410in}}%
\pgfpathlineto{\pgfqpoint{1.846105in}{1.589496in}}%
\pgfpathlineto{\pgfqpoint{1.847066in}{1.589589in}}%
\pgfpathlineto{\pgfqpoint{1.854626in}{1.590676in}}%
\pgfpathlineto{\pgfqpoint{1.855728in}{1.590769in}}%
\pgfpathlineto{\pgfqpoint{1.866782in}{1.591855in}}%
\pgfpathlineto{\pgfqpoint{1.867564in}{1.591980in}}%
\pgfpathlineto{\pgfqpoint{1.880190in}{1.593066in}}%
\pgfpathlineto{\pgfqpoint{1.881222in}{1.593159in}}%
\pgfpathlineto{\pgfqpoint{1.891940in}{1.594246in}}%
\pgfpathlineto{\pgfqpoint{1.892416in}{1.594339in}}%
\pgfpathlineto{\pgfqpoint{1.892823in}{1.594339in}}%
\pgfpathlineto{\pgfqpoint{1.907708in}{1.595425in}}%
\pgfpathlineto{\pgfqpoint{1.908787in}{1.595549in}}%
\pgfpathlineto{\pgfqpoint{1.924469in}{1.596636in}}%
\pgfpathlineto{\pgfqpoint{1.925305in}{1.596760in}}%
\pgfpathlineto{\pgfqpoint{1.925438in}{1.596760in}}%
\pgfpathlineto{\pgfqpoint{1.947414in}{1.597847in}}%
\pgfpathlineto{\pgfqpoint{1.947992in}{1.598033in}}%
\pgfpathlineto{\pgfqpoint{1.948375in}{1.598033in}}%
\pgfpathlineto{\pgfqpoint{1.964104in}{1.599119in}}%
\pgfpathlineto{\pgfqpoint{1.964221in}{1.599212in}}%
\pgfpathlineto{\pgfqpoint{1.965214in}{1.599212in}}%
\pgfpathlineto{\pgfqpoint{1.989472in}{1.600299in}}%
\pgfpathlineto{\pgfqpoint{1.990145in}{1.600423in}}%
\pgfpathlineto{\pgfqpoint{1.990285in}{1.600423in}}%
\pgfpathlineto{\pgfqpoint{2.016256in}{1.601510in}}%
\pgfpathlineto{\pgfqpoint{2.016357in}{1.601572in}}%
\pgfpathlineto{\pgfqpoint{2.016967in}{1.601572in}}%
\pgfpathlineto{\pgfqpoint{2.033126in}{1.601944in}}%
\pgfpathlineto{\pgfqpoint{2.033126in}{1.601944in}}%
\pgfusepath{stroke}%
\end{pgfscope}%
\begin{pgfscope}%
\pgfsetrectcap%
\pgfsetmiterjoin%
\pgfsetlinewidth{0.803000pt}%
\definecolor{currentstroke}{rgb}{0.000000,0.000000,0.000000}%
\pgfsetstrokecolor{currentstroke}%
\pgfsetdash{}{0pt}%
\pgfpathmoveto{\pgfqpoint{0.553581in}{0.499444in}}%
\pgfpathlineto{\pgfqpoint{0.553581in}{1.654444in}}%
\pgfusepath{stroke}%
\end{pgfscope}%
\begin{pgfscope}%
\pgfsetrectcap%
\pgfsetmiterjoin%
\pgfsetlinewidth{0.803000pt}%
\definecolor{currentstroke}{rgb}{0.000000,0.000000,0.000000}%
\pgfsetstrokecolor{currentstroke}%
\pgfsetdash{}{0pt}%
\pgfpathmoveto{\pgfqpoint{2.103581in}{0.499444in}}%
\pgfpathlineto{\pgfqpoint{2.103581in}{1.654444in}}%
\pgfusepath{stroke}%
\end{pgfscope}%
\begin{pgfscope}%
\pgfsetrectcap%
\pgfsetmiterjoin%
\pgfsetlinewidth{0.803000pt}%
\definecolor{currentstroke}{rgb}{0.000000,0.000000,0.000000}%
\pgfsetstrokecolor{currentstroke}%
\pgfsetdash{}{0pt}%
\pgfpathmoveto{\pgfqpoint{0.553581in}{0.499444in}}%
\pgfpathlineto{\pgfqpoint{2.103581in}{0.499444in}}%
\pgfusepath{stroke}%
\end{pgfscope}%
\begin{pgfscope}%
\pgfsetrectcap%
\pgfsetmiterjoin%
\pgfsetlinewidth{0.803000pt}%
\definecolor{currentstroke}{rgb}{0.000000,0.000000,0.000000}%
\pgfsetstrokecolor{currentstroke}%
\pgfsetdash{}{0pt}%
\pgfpathmoveto{\pgfqpoint{0.553581in}{1.654444in}}%
\pgfpathlineto{\pgfqpoint{2.103581in}{1.654444in}}%
\pgfusepath{stroke}%
\end{pgfscope}%
\begin{pgfscope}%
\pgfsetbuttcap%
\pgfsetmiterjoin%
\definecolor{currentfill}{rgb}{1.000000,1.000000,1.000000}%
\pgfsetfillcolor{currentfill}%
\pgfsetfillopacity{0.800000}%
\pgfsetlinewidth{1.003750pt}%
\definecolor{currentstroke}{rgb}{0.800000,0.800000,0.800000}%
\pgfsetstrokecolor{currentstroke}%
\pgfsetstrokeopacity{0.800000}%
\pgfsetdash{}{0pt}%
\pgfpathmoveto{\pgfqpoint{0.832747in}{0.568889in}}%
\pgfpathlineto{\pgfqpoint{2.006358in}{0.568889in}}%
\pgfpathquadraticcurveto{\pgfqpoint{2.034136in}{0.568889in}}{\pgfqpoint{2.034136in}{0.596666in}}%
\pgfpathlineto{\pgfqpoint{2.034136in}{0.776388in}}%
\pgfpathquadraticcurveto{\pgfqpoint{2.034136in}{0.804166in}}{\pgfqpoint{2.006358in}{0.804166in}}%
\pgfpathlineto{\pgfqpoint{0.832747in}{0.804166in}}%
\pgfpathquadraticcurveto{\pgfqpoint{0.804970in}{0.804166in}}{\pgfqpoint{0.804970in}{0.776388in}}%
\pgfpathlineto{\pgfqpoint{0.804970in}{0.596666in}}%
\pgfpathquadraticcurveto{\pgfqpoint{0.804970in}{0.568889in}}{\pgfqpoint{0.832747in}{0.568889in}}%
\pgfpathlineto{\pgfqpoint{0.832747in}{0.568889in}}%
\pgfpathclose%
\pgfusepath{stroke,fill}%
\end{pgfscope}%
\begin{pgfscope}%
\pgfsetrectcap%
\pgfsetroundjoin%
\pgfsetlinewidth{1.505625pt}%
\definecolor{currentstroke}{rgb}{0.000000,0.000000,0.000000}%
\pgfsetstrokecolor{currentstroke}%
\pgfsetdash{}{0pt}%
\pgfpathmoveto{\pgfqpoint{0.860525in}{0.700000in}}%
\pgfpathlineto{\pgfqpoint{0.999414in}{0.700000in}}%
\pgfpathlineto{\pgfqpoint{1.138303in}{0.700000in}}%
\pgfusepath{stroke}%
\end{pgfscope}%
\begin{pgfscope}%
\definecolor{textcolor}{rgb}{0.000000,0.000000,0.000000}%
\pgfsetstrokecolor{textcolor}%
\pgfsetfillcolor{textcolor}%
\pgftext[x=1.249414in,y=0.651388in,left,base]{\color{textcolor}\rmfamily\fontsize{10.000000}{12.000000}\selectfont AUC=0.776}%
\end{pgfscope}%
\end{pgfpicture}%
\makeatother%
\endgroup%

\end{tabular}


\newpage
	
Model 1:  $\alpha = 0.5$ for no class weights

\noindent\begin{tabular}{@{\hspace{-6pt}}p{4.5in} @{\hspace{-6pt}}p{2.0in}}
	\vskip 0pt
	\qquad \qquad Raw Model Output
	
	%% Creator: Matplotlib, PGF backend
%%
%% To include the figure in your LaTeX document, write
%%   \input{<filename>.pgf}
%%
%% Make sure the required packages are loaded in your preamble
%%   \usepackage{pgf}
%%
%% Also ensure that all the required font packages are loaded; for instance,
%% the lmodern package is sometimes necessary when using math font.
%%   \usepackage{lmodern}
%%
%% Figures using additional raster images can only be included by \input if
%% they are in the same directory as the main LaTeX file. For loading figures
%% from other directories you can use the `import` package
%%   \usepackage{import}
%%
%% and then include the figures with
%%   \import{<path to file>}{<filename>.pgf}
%%
%% Matplotlib used the following preamble
%%   
%%   \usepackage{fontspec}
%%   \makeatletter\@ifpackageloaded{underscore}{}{\usepackage[strings]{underscore}}\makeatother
%%
\begingroup%
\makeatletter%
\begin{pgfpicture}%
\pgfpathrectangle{\pgfpointorigin}{\pgfqpoint{4.102500in}{1.754444in}}%
\pgfusepath{use as bounding box, clip}%
\begin{pgfscope}%
\pgfsetbuttcap%
\pgfsetmiterjoin%
\definecolor{currentfill}{rgb}{1.000000,1.000000,1.000000}%
\pgfsetfillcolor{currentfill}%
\pgfsetlinewidth{0.000000pt}%
\definecolor{currentstroke}{rgb}{1.000000,1.000000,1.000000}%
\pgfsetstrokecolor{currentstroke}%
\pgfsetdash{}{0pt}%
\pgfpathmoveto{\pgfqpoint{0.000000in}{0.000000in}}%
\pgfpathlineto{\pgfqpoint{4.102500in}{0.000000in}}%
\pgfpathlineto{\pgfqpoint{4.102500in}{1.754444in}}%
\pgfpathlineto{\pgfqpoint{0.000000in}{1.754444in}}%
\pgfpathlineto{\pgfqpoint{0.000000in}{0.000000in}}%
\pgfpathclose%
\pgfusepath{fill}%
\end{pgfscope}%
\begin{pgfscope}%
\pgfsetbuttcap%
\pgfsetmiterjoin%
\definecolor{currentfill}{rgb}{1.000000,1.000000,1.000000}%
\pgfsetfillcolor{currentfill}%
\pgfsetlinewidth{0.000000pt}%
\definecolor{currentstroke}{rgb}{0.000000,0.000000,0.000000}%
\pgfsetstrokecolor{currentstroke}%
\pgfsetstrokeopacity{0.000000}%
\pgfsetdash{}{0pt}%
\pgfpathmoveto{\pgfqpoint{0.515000in}{0.499444in}}%
\pgfpathlineto{\pgfqpoint{4.002500in}{0.499444in}}%
\pgfpathlineto{\pgfqpoint{4.002500in}{1.654444in}}%
\pgfpathlineto{\pgfqpoint{0.515000in}{1.654444in}}%
\pgfpathlineto{\pgfqpoint{0.515000in}{0.499444in}}%
\pgfpathclose%
\pgfusepath{fill}%
\end{pgfscope}%
\begin{pgfscope}%
\pgfpathrectangle{\pgfqpoint{0.515000in}{0.499444in}}{\pgfqpoint{3.487500in}{1.155000in}}%
\pgfusepath{clip}%
\pgfsetbuttcap%
\pgfsetmiterjoin%
\pgfsetlinewidth{1.003750pt}%
\definecolor{currentstroke}{rgb}{0.000000,0.000000,0.000000}%
\pgfsetstrokecolor{currentstroke}%
\pgfsetdash{}{0pt}%
\pgfpathmoveto{\pgfqpoint{0.610114in}{0.499444in}}%
\pgfpathlineto{\pgfqpoint{0.673523in}{0.499444in}}%
\pgfpathlineto{\pgfqpoint{0.673523in}{0.499444in}}%
\pgfpathlineto{\pgfqpoint{0.610114in}{0.499444in}}%
\pgfpathlineto{\pgfqpoint{0.610114in}{0.499444in}}%
\pgfpathclose%
\pgfusepath{stroke}%
\end{pgfscope}%
\begin{pgfscope}%
\pgfpathrectangle{\pgfqpoint{0.515000in}{0.499444in}}{\pgfqpoint{3.487500in}{1.155000in}}%
\pgfusepath{clip}%
\pgfsetbuttcap%
\pgfsetmiterjoin%
\pgfsetlinewidth{1.003750pt}%
\definecolor{currentstroke}{rgb}{0.000000,0.000000,0.000000}%
\pgfsetstrokecolor{currentstroke}%
\pgfsetdash{}{0pt}%
\pgfpathmoveto{\pgfqpoint{0.768637in}{0.499444in}}%
\pgfpathlineto{\pgfqpoint{0.832046in}{0.499444in}}%
\pgfpathlineto{\pgfqpoint{0.832046in}{1.599444in}}%
\pgfpathlineto{\pgfqpoint{0.768637in}{1.599444in}}%
\pgfpathlineto{\pgfqpoint{0.768637in}{0.499444in}}%
\pgfpathclose%
\pgfusepath{stroke}%
\end{pgfscope}%
\begin{pgfscope}%
\pgfpathrectangle{\pgfqpoint{0.515000in}{0.499444in}}{\pgfqpoint{3.487500in}{1.155000in}}%
\pgfusepath{clip}%
\pgfsetbuttcap%
\pgfsetmiterjoin%
\pgfsetlinewidth{1.003750pt}%
\definecolor{currentstroke}{rgb}{0.000000,0.000000,0.000000}%
\pgfsetstrokecolor{currentstroke}%
\pgfsetdash{}{0pt}%
\pgfpathmoveto{\pgfqpoint{0.927159in}{0.499444in}}%
\pgfpathlineto{\pgfqpoint{0.990568in}{0.499444in}}%
\pgfpathlineto{\pgfqpoint{0.990568in}{1.346951in}}%
\pgfpathlineto{\pgfqpoint{0.927159in}{1.346951in}}%
\pgfpathlineto{\pgfqpoint{0.927159in}{0.499444in}}%
\pgfpathclose%
\pgfusepath{stroke}%
\end{pgfscope}%
\begin{pgfscope}%
\pgfpathrectangle{\pgfqpoint{0.515000in}{0.499444in}}{\pgfqpoint{3.487500in}{1.155000in}}%
\pgfusepath{clip}%
\pgfsetbuttcap%
\pgfsetmiterjoin%
\pgfsetlinewidth{1.003750pt}%
\definecolor{currentstroke}{rgb}{0.000000,0.000000,0.000000}%
\pgfsetstrokecolor{currentstroke}%
\pgfsetdash{}{0pt}%
\pgfpathmoveto{\pgfqpoint{1.085682in}{0.499444in}}%
\pgfpathlineto{\pgfqpoint{1.149091in}{0.499444in}}%
\pgfpathlineto{\pgfqpoint{1.149091in}{1.032251in}}%
\pgfpathlineto{\pgfqpoint{1.085682in}{1.032251in}}%
\pgfpathlineto{\pgfqpoint{1.085682in}{0.499444in}}%
\pgfpathclose%
\pgfusepath{stroke}%
\end{pgfscope}%
\begin{pgfscope}%
\pgfpathrectangle{\pgfqpoint{0.515000in}{0.499444in}}{\pgfqpoint{3.487500in}{1.155000in}}%
\pgfusepath{clip}%
\pgfsetbuttcap%
\pgfsetmiterjoin%
\pgfsetlinewidth{1.003750pt}%
\definecolor{currentstroke}{rgb}{0.000000,0.000000,0.000000}%
\pgfsetstrokecolor{currentstroke}%
\pgfsetdash{}{0pt}%
\pgfpathmoveto{\pgfqpoint{1.244205in}{0.499444in}}%
\pgfpathlineto{\pgfqpoint{1.307614in}{0.499444in}}%
\pgfpathlineto{\pgfqpoint{1.307614in}{0.848431in}}%
\pgfpathlineto{\pgfqpoint{1.244205in}{0.848431in}}%
\pgfpathlineto{\pgfqpoint{1.244205in}{0.499444in}}%
\pgfpathclose%
\pgfusepath{stroke}%
\end{pgfscope}%
\begin{pgfscope}%
\pgfpathrectangle{\pgfqpoint{0.515000in}{0.499444in}}{\pgfqpoint{3.487500in}{1.155000in}}%
\pgfusepath{clip}%
\pgfsetbuttcap%
\pgfsetmiterjoin%
\pgfsetlinewidth{1.003750pt}%
\definecolor{currentstroke}{rgb}{0.000000,0.000000,0.000000}%
\pgfsetstrokecolor{currentstroke}%
\pgfsetdash{}{0pt}%
\pgfpathmoveto{\pgfqpoint{1.402728in}{0.499444in}}%
\pgfpathlineto{\pgfqpoint{1.466137in}{0.499444in}}%
\pgfpathlineto{\pgfqpoint{1.466137in}{0.740608in}}%
\pgfpathlineto{\pgfqpoint{1.402728in}{0.740608in}}%
\pgfpathlineto{\pgfqpoint{1.402728in}{0.499444in}}%
\pgfpathclose%
\pgfusepath{stroke}%
\end{pgfscope}%
\begin{pgfscope}%
\pgfpathrectangle{\pgfqpoint{0.515000in}{0.499444in}}{\pgfqpoint{3.487500in}{1.155000in}}%
\pgfusepath{clip}%
\pgfsetbuttcap%
\pgfsetmiterjoin%
\pgfsetlinewidth{1.003750pt}%
\definecolor{currentstroke}{rgb}{0.000000,0.000000,0.000000}%
\pgfsetstrokecolor{currentstroke}%
\pgfsetdash{}{0pt}%
\pgfpathmoveto{\pgfqpoint{1.561250in}{0.499444in}}%
\pgfpathlineto{\pgfqpoint{1.624659in}{0.499444in}}%
\pgfpathlineto{\pgfqpoint{1.624659in}{0.660907in}}%
\pgfpathlineto{\pgfqpoint{1.561250in}{0.660907in}}%
\pgfpathlineto{\pgfqpoint{1.561250in}{0.499444in}}%
\pgfpathclose%
\pgfusepath{stroke}%
\end{pgfscope}%
\begin{pgfscope}%
\pgfpathrectangle{\pgfqpoint{0.515000in}{0.499444in}}{\pgfqpoint{3.487500in}{1.155000in}}%
\pgfusepath{clip}%
\pgfsetbuttcap%
\pgfsetmiterjoin%
\pgfsetlinewidth{1.003750pt}%
\definecolor{currentstroke}{rgb}{0.000000,0.000000,0.000000}%
\pgfsetstrokecolor{currentstroke}%
\pgfsetdash{}{0pt}%
\pgfpathmoveto{\pgfqpoint{1.719773in}{0.499444in}}%
\pgfpathlineto{\pgfqpoint{1.783182in}{0.499444in}}%
\pgfpathlineto{\pgfqpoint{1.783182in}{0.611530in}}%
\pgfpathlineto{\pgfqpoint{1.719773in}{0.611530in}}%
\pgfpathlineto{\pgfqpoint{1.719773in}{0.499444in}}%
\pgfpathclose%
\pgfusepath{stroke}%
\end{pgfscope}%
\begin{pgfscope}%
\pgfpathrectangle{\pgfqpoint{0.515000in}{0.499444in}}{\pgfqpoint{3.487500in}{1.155000in}}%
\pgfusepath{clip}%
\pgfsetbuttcap%
\pgfsetmiterjoin%
\pgfsetlinewidth{1.003750pt}%
\definecolor{currentstroke}{rgb}{0.000000,0.000000,0.000000}%
\pgfsetstrokecolor{currentstroke}%
\pgfsetdash{}{0pt}%
\pgfpathmoveto{\pgfqpoint{1.878296in}{0.499444in}}%
\pgfpathlineto{\pgfqpoint{1.941705in}{0.499444in}}%
\pgfpathlineto{\pgfqpoint{1.941705in}{0.577063in}}%
\pgfpathlineto{\pgfqpoint{1.878296in}{0.577063in}}%
\pgfpathlineto{\pgfqpoint{1.878296in}{0.499444in}}%
\pgfpathclose%
\pgfusepath{stroke}%
\end{pgfscope}%
\begin{pgfscope}%
\pgfpathrectangle{\pgfqpoint{0.515000in}{0.499444in}}{\pgfqpoint{3.487500in}{1.155000in}}%
\pgfusepath{clip}%
\pgfsetbuttcap%
\pgfsetmiterjoin%
\pgfsetlinewidth{1.003750pt}%
\definecolor{currentstroke}{rgb}{0.000000,0.000000,0.000000}%
\pgfsetstrokecolor{currentstroke}%
\pgfsetdash{}{0pt}%
\pgfpathmoveto{\pgfqpoint{2.036818in}{0.499444in}}%
\pgfpathlineto{\pgfqpoint{2.100228in}{0.499444in}}%
\pgfpathlineto{\pgfqpoint{2.100228in}{0.554326in}}%
\pgfpathlineto{\pgfqpoint{2.036818in}{0.554326in}}%
\pgfpathlineto{\pgfqpoint{2.036818in}{0.499444in}}%
\pgfpathclose%
\pgfusepath{stroke}%
\end{pgfscope}%
\begin{pgfscope}%
\pgfpathrectangle{\pgfqpoint{0.515000in}{0.499444in}}{\pgfqpoint{3.487500in}{1.155000in}}%
\pgfusepath{clip}%
\pgfsetbuttcap%
\pgfsetmiterjoin%
\pgfsetlinewidth{1.003750pt}%
\definecolor{currentstroke}{rgb}{0.000000,0.000000,0.000000}%
\pgfsetstrokecolor{currentstroke}%
\pgfsetdash{}{0pt}%
\pgfpathmoveto{\pgfqpoint{2.195341in}{0.499444in}}%
\pgfpathlineto{\pgfqpoint{2.258750in}{0.499444in}}%
\pgfpathlineto{\pgfqpoint{2.258750in}{0.536953in}}%
\pgfpathlineto{\pgfqpoint{2.195341in}{0.536953in}}%
\pgfpathlineto{\pgfqpoint{2.195341in}{0.499444in}}%
\pgfpathclose%
\pgfusepath{stroke}%
\end{pgfscope}%
\begin{pgfscope}%
\pgfpathrectangle{\pgfqpoint{0.515000in}{0.499444in}}{\pgfqpoint{3.487500in}{1.155000in}}%
\pgfusepath{clip}%
\pgfsetbuttcap%
\pgfsetmiterjoin%
\pgfsetlinewidth{1.003750pt}%
\definecolor{currentstroke}{rgb}{0.000000,0.000000,0.000000}%
\pgfsetstrokecolor{currentstroke}%
\pgfsetdash{}{0pt}%
\pgfpathmoveto{\pgfqpoint{2.353864in}{0.499444in}}%
\pgfpathlineto{\pgfqpoint{2.417273in}{0.499444in}}%
\pgfpathlineto{\pgfqpoint{2.417273in}{0.528186in}}%
\pgfpathlineto{\pgfqpoint{2.353864in}{0.528186in}}%
\pgfpathlineto{\pgfqpoint{2.353864in}{0.499444in}}%
\pgfpathclose%
\pgfusepath{stroke}%
\end{pgfscope}%
\begin{pgfscope}%
\pgfpathrectangle{\pgfqpoint{0.515000in}{0.499444in}}{\pgfqpoint{3.487500in}{1.155000in}}%
\pgfusepath{clip}%
\pgfsetbuttcap%
\pgfsetmiterjoin%
\pgfsetlinewidth{1.003750pt}%
\definecolor{currentstroke}{rgb}{0.000000,0.000000,0.000000}%
\pgfsetstrokecolor{currentstroke}%
\pgfsetdash{}{0pt}%
\pgfpathmoveto{\pgfqpoint{2.512387in}{0.499444in}}%
\pgfpathlineto{\pgfqpoint{2.575796in}{0.499444in}}%
\pgfpathlineto{\pgfqpoint{2.575796in}{0.519540in}}%
\pgfpathlineto{\pgfqpoint{2.512387in}{0.519540in}}%
\pgfpathlineto{\pgfqpoint{2.512387in}{0.499444in}}%
\pgfpathclose%
\pgfusepath{stroke}%
\end{pgfscope}%
\begin{pgfscope}%
\pgfpathrectangle{\pgfqpoint{0.515000in}{0.499444in}}{\pgfqpoint{3.487500in}{1.155000in}}%
\pgfusepath{clip}%
\pgfsetbuttcap%
\pgfsetmiterjoin%
\pgfsetlinewidth{1.003750pt}%
\definecolor{currentstroke}{rgb}{0.000000,0.000000,0.000000}%
\pgfsetstrokecolor{currentstroke}%
\pgfsetdash{}{0pt}%
\pgfpathmoveto{\pgfqpoint{2.670909in}{0.499444in}}%
\pgfpathlineto{\pgfqpoint{2.734318in}{0.499444in}}%
\pgfpathlineto{\pgfqpoint{2.734318in}{0.513475in}}%
\pgfpathlineto{\pgfqpoint{2.670909in}{0.513475in}}%
\pgfpathlineto{\pgfqpoint{2.670909in}{0.499444in}}%
\pgfpathclose%
\pgfusepath{stroke}%
\end{pgfscope}%
\begin{pgfscope}%
\pgfpathrectangle{\pgfqpoint{0.515000in}{0.499444in}}{\pgfqpoint{3.487500in}{1.155000in}}%
\pgfusepath{clip}%
\pgfsetbuttcap%
\pgfsetmiterjoin%
\pgfsetlinewidth{1.003750pt}%
\definecolor{currentstroke}{rgb}{0.000000,0.000000,0.000000}%
\pgfsetstrokecolor{currentstroke}%
\pgfsetdash{}{0pt}%
\pgfpathmoveto{\pgfqpoint{2.829432in}{0.499444in}}%
\pgfpathlineto{\pgfqpoint{2.892841in}{0.499444in}}%
\pgfpathlineto{\pgfqpoint{2.892841in}{0.509932in}}%
\pgfpathlineto{\pgfqpoint{2.829432in}{0.509932in}}%
\pgfpathlineto{\pgfqpoint{2.829432in}{0.499444in}}%
\pgfpathclose%
\pgfusepath{stroke}%
\end{pgfscope}%
\begin{pgfscope}%
\pgfpathrectangle{\pgfqpoint{0.515000in}{0.499444in}}{\pgfqpoint{3.487500in}{1.155000in}}%
\pgfusepath{clip}%
\pgfsetbuttcap%
\pgfsetmiterjoin%
\pgfsetlinewidth{1.003750pt}%
\definecolor{currentstroke}{rgb}{0.000000,0.000000,0.000000}%
\pgfsetstrokecolor{currentstroke}%
\pgfsetdash{}{0pt}%
\pgfpathmoveto{\pgfqpoint{2.987955in}{0.499444in}}%
\pgfpathlineto{\pgfqpoint{3.051364in}{0.499444in}}%
\pgfpathlineto{\pgfqpoint{3.051364in}{0.507630in}}%
\pgfpathlineto{\pgfqpoint{2.987955in}{0.507630in}}%
\pgfpathlineto{\pgfqpoint{2.987955in}{0.499444in}}%
\pgfpathclose%
\pgfusepath{stroke}%
\end{pgfscope}%
\begin{pgfscope}%
\pgfpathrectangle{\pgfqpoint{0.515000in}{0.499444in}}{\pgfqpoint{3.487500in}{1.155000in}}%
\pgfusepath{clip}%
\pgfsetbuttcap%
\pgfsetmiterjoin%
\pgfsetlinewidth{1.003750pt}%
\definecolor{currentstroke}{rgb}{0.000000,0.000000,0.000000}%
\pgfsetstrokecolor{currentstroke}%
\pgfsetdash{}{0pt}%
\pgfpathmoveto{\pgfqpoint{3.146478in}{0.499444in}}%
\pgfpathlineto{\pgfqpoint{3.209887in}{0.499444in}}%
\pgfpathlineto{\pgfqpoint{3.209887in}{0.505709in}}%
\pgfpathlineto{\pgfqpoint{3.146478in}{0.505709in}}%
\pgfpathlineto{\pgfqpoint{3.146478in}{0.499444in}}%
\pgfpathclose%
\pgfusepath{stroke}%
\end{pgfscope}%
\begin{pgfscope}%
\pgfpathrectangle{\pgfqpoint{0.515000in}{0.499444in}}{\pgfqpoint{3.487500in}{1.155000in}}%
\pgfusepath{clip}%
\pgfsetbuttcap%
\pgfsetmiterjoin%
\pgfsetlinewidth{1.003750pt}%
\definecolor{currentstroke}{rgb}{0.000000,0.000000,0.000000}%
\pgfsetstrokecolor{currentstroke}%
\pgfsetdash{}{0pt}%
\pgfpathmoveto{\pgfqpoint{3.305000in}{0.499444in}}%
\pgfpathlineto{\pgfqpoint{3.368409in}{0.499444in}}%
\pgfpathlineto{\pgfqpoint{3.368409in}{0.503147in}}%
\pgfpathlineto{\pgfqpoint{3.305000in}{0.503147in}}%
\pgfpathlineto{\pgfqpoint{3.305000in}{0.499444in}}%
\pgfpathclose%
\pgfusepath{stroke}%
\end{pgfscope}%
\begin{pgfscope}%
\pgfpathrectangle{\pgfqpoint{0.515000in}{0.499444in}}{\pgfqpoint{3.487500in}{1.155000in}}%
\pgfusepath{clip}%
\pgfsetbuttcap%
\pgfsetmiterjoin%
\pgfsetlinewidth{1.003750pt}%
\definecolor{currentstroke}{rgb}{0.000000,0.000000,0.000000}%
\pgfsetstrokecolor{currentstroke}%
\pgfsetdash{}{0pt}%
\pgfpathmoveto{\pgfqpoint{3.463523in}{0.499444in}}%
\pgfpathlineto{\pgfqpoint{3.526932in}{0.499444in}}%
\pgfpathlineto{\pgfqpoint{3.526932in}{0.501266in}}%
\pgfpathlineto{\pgfqpoint{3.463523in}{0.501266in}}%
\pgfpathlineto{\pgfqpoint{3.463523in}{0.499444in}}%
\pgfpathclose%
\pgfusepath{stroke}%
\end{pgfscope}%
\begin{pgfscope}%
\pgfpathrectangle{\pgfqpoint{0.515000in}{0.499444in}}{\pgfqpoint{3.487500in}{1.155000in}}%
\pgfusepath{clip}%
\pgfsetbuttcap%
\pgfsetmiterjoin%
\pgfsetlinewidth{1.003750pt}%
\definecolor{currentstroke}{rgb}{0.000000,0.000000,0.000000}%
\pgfsetstrokecolor{currentstroke}%
\pgfsetdash{}{0pt}%
\pgfpathmoveto{\pgfqpoint{3.622046in}{0.499444in}}%
\pgfpathlineto{\pgfqpoint{3.685455in}{0.499444in}}%
\pgfpathlineto{\pgfqpoint{3.685455in}{0.499744in}}%
\pgfpathlineto{\pgfqpoint{3.622046in}{0.499744in}}%
\pgfpathlineto{\pgfqpoint{3.622046in}{0.499444in}}%
\pgfpathclose%
\pgfusepath{stroke}%
\end{pgfscope}%
\begin{pgfscope}%
\pgfpathrectangle{\pgfqpoint{0.515000in}{0.499444in}}{\pgfqpoint{3.487500in}{1.155000in}}%
\pgfusepath{clip}%
\pgfsetbuttcap%
\pgfsetmiterjoin%
\pgfsetlinewidth{1.003750pt}%
\definecolor{currentstroke}{rgb}{0.000000,0.000000,0.000000}%
\pgfsetstrokecolor{currentstroke}%
\pgfsetdash{}{0pt}%
\pgfpathmoveto{\pgfqpoint{3.780568in}{0.499444in}}%
\pgfpathlineto{\pgfqpoint{3.843978in}{0.499444in}}%
\pgfpathlineto{\pgfqpoint{3.843978in}{0.499444in}}%
\pgfpathlineto{\pgfqpoint{3.780568in}{0.499444in}}%
\pgfpathlineto{\pgfqpoint{3.780568in}{0.499444in}}%
\pgfpathclose%
\pgfusepath{stroke}%
\end{pgfscope}%
\begin{pgfscope}%
\pgfpathrectangle{\pgfqpoint{0.515000in}{0.499444in}}{\pgfqpoint{3.487500in}{1.155000in}}%
\pgfusepath{clip}%
\pgfsetbuttcap%
\pgfsetmiterjoin%
\definecolor{currentfill}{rgb}{0.000000,0.000000,0.000000}%
\pgfsetfillcolor{currentfill}%
\pgfsetlinewidth{0.000000pt}%
\definecolor{currentstroke}{rgb}{0.000000,0.000000,0.000000}%
\pgfsetstrokecolor{currentstroke}%
\pgfsetstrokeopacity{0.000000}%
\pgfsetdash{}{0pt}%
\pgfpathmoveto{\pgfqpoint{0.673523in}{0.499444in}}%
\pgfpathlineto{\pgfqpoint{0.736932in}{0.499444in}}%
\pgfpathlineto{\pgfqpoint{0.736932in}{0.499444in}}%
\pgfpathlineto{\pgfqpoint{0.673523in}{0.499444in}}%
\pgfpathlineto{\pgfqpoint{0.673523in}{0.499444in}}%
\pgfpathclose%
\pgfusepath{fill}%
\end{pgfscope}%
\begin{pgfscope}%
\pgfpathrectangle{\pgfqpoint{0.515000in}{0.499444in}}{\pgfqpoint{3.487500in}{1.155000in}}%
\pgfusepath{clip}%
\pgfsetbuttcap%
\pgfsetmiterjoin%
\definecolor{currentfill}{rgb}{0.000000,0.000000,0.000000}%
\pgfsetfillcolor{currentfill}%
\pgfsetlinewidth{0.000000pt}%
\definecolor{currentstroke}{rgb}{0.000000,0.000000,0.000000}%
\pgfsetstrokecolor{currentstroke}%
\pgfsetstrokeopacity{0.000000}%
\pgfsetdash{}{0pt}%
\pgfpathmoveto{\pgfqpoint{0.832046in}{0.499444in}}%
\pgfpathlineto{\pgfqpoint{0.895455in}{0.499444in}}%
\pgfpathlineto{\pgfqpoint{0.895455in}{0.535532in}}%
\pgfpathlineto{\pgfqpoint{0.832046in}{0.535532in}}%
\pgfpathlineto{\pgfqpoint{0.832046in}{0.499444in}}%
\pgfpathclose%
\pgfusepath{fill}%
\end{pgfscope}%
\begin{pgfscope}%
\pgfpathrectangle{\pgfqpoint{0.515000in}{0.499444in}}{\pgfqpoint{3.487500in}{1.155000in}}%
\pgfusepath{clip}%
\pgfsetbuttcap%
\pgfsetmiterjoin%
\definecolor{currentfill}{rgb}{0.000000,0.000000,0.000000}%
\pgfsetfillcolor{currentfill}%
\pgfsetlinewidth{0.000000pt}%
\definecolor{currentstroke}{rgb}{0.000000,0.000000,0.000000}%
\pgfsetstrokecolor{currentstroke}%
\pgfsetstrokeopacity{0.000000}%
\pgfsetdash{}{0pt}%
\pgfpathmoveto{\pgfqpoint{0.990568in}{0.499444in}}%
\pgfpathlineto{\pgfqpoint{1.053978in}{0.499444in}}%
\pgfpathlineto{\pgfqpoint{1.053978in}{0.575602in}}%
\pgfpathlineto{\pgfqpoint{0.990568in}{0.575602in}}%
\pgfpathlineto{\pgfqpoint{0.990568in}{0.499444in}}%
\pgfpathclose%
\pgfusepath{fill}%
\end{pgfscope}%
\begin{pgfscope}%
\pgfpathrectangle{\pgfqpoint{0.515000in}{0.499444in}}{\pgfqpoint{3.487500in}{1.155000in}}%
\pgfusepath{clip}%
\pgfsetbuttcap%
\pgfsetmiterjoin%
\definecolor{currentfill}{rgb}{0.000000,0.000000,0.000000}%
\pgfsetfillcolor{currentfill}%
\pgfsetlinewidth{0.000000pt}%
\definecolor{currentstroke}{rgb}{0.000000,0.000000,0.000000}%
\pgfsetstrokecolor{currentstroke}%
\pgfsetstrokeopacity{0.000000}%
\pgfsetdash{}{0pt}%
\pgfpathmoveto{\pgfqpoint{1.149091in}{0.499444in}}%
\pgfpathlineto{\pgfqpoint{1.212500in}{0.499444in}}%
\pgfpathlineto{\pgfqpoint{1.212500in}{0.576243in}}%
\pgfpathlineto{\pgfqpoint{1.149091in}{0.576243in}}%
\pgfpathlineto{\pgfqpoint{1.149091in}{0.499444in}}%
\pgfpathclose%
\pgfusepath{fill}%
\end{pgfscope}%
\begin{pgfscope}%
\pgfpathrectangle{\pgfqpoint{0.515000in}{0.499444in}}{\pgfqpoint{3.487500in}{1.155000in}}%
\pgfusepath{clip}%
\pgfsetbuttcap%
\pgfsetmiterjoin%
\definecolor{currentfill}{rgb}{0.000000,0.000000,0.000000}%
\pgfsetfillcolor{currentfill}%
\pgfsetlinewidth{0.000000pt}%
\definecolor{currentstroke}{rgb}{0.000000,0.000000,0.000000}%
\pgfsetstrokecolor{currentstroke}%
\pgfsetstrokeopacity{0.000000}%
\pgfsetdash{}{0pt}%
\pgfpathmoveto{\pgfqpoint{1.307614in}{0.499444in}}%
\pgfpathlineto{\pgfqpoint{1.371023in}{0.499444in}}%
\pgfpathlineto{\pgfqpoint{1.371023in}{0.571940in}}%
\pgfpathlineto{\pgfqpoint{1.307614in}{0.571940in}}%
\pgfpathlineto{\pgfqpoint{1.307614in}{0.499444in}}%
\pgfpathclose%
\pgfusepath{fill}%
\end{pgfscope}%
\begin{pgfscope}%
\pgfpathrectangle{\pgfqpoint{0.515000in}{0.499444in}}{\pgfqpoint{3.487500in}{1.155000in}}%
\pgfusepath{clip}%
\pgfsetbuttcap%
\pgfsetmiterjoin%
\definecolor{currentfill}{rgb}{0.000000,0.000000,0.000000}%
\pgfsetfillcolor{currentfill}%
\pgfsetlinewidth{0.000000pt}%
\definecolor{currentstroke}{rgb}{0.000000,0.000000,0.000000}%
\pgfsetstrokecolor{currentstroke}%
\pgfsetstrokeopacity{0.000000}%
\pgfsetdash{}{0pt}%
\pgfpathmoveto{\pgfqpoint{1.466137in}{0.499444in}}%
\pgfpathlineto{\pgfqpoint{1.529546in}{0.499444in}}%
\pgfpathlineto{\pgfqpoint{1.529546in}{0.564794in}}%
\pgfpathlineto{\pgfqpoint{1.466137in}{0.564794in}}%
\pgfpathlineto{\pgfqpoint{1.466137in}{0.499444in}}%
\pgfpathclose%
\pgfusepath{fill}%
\end{pgfscope}%
\begin{pgfscope}%
\pgfpathrectangle{\pgfqpoint{0.515000in}{0.499444in}}{\pgfqpoint{3.487500in}{1.155000in}}%
\pgfusepath{clip}%
\pgfsetbuttcap%
\pgfsetmiterjoin%
\definecolor{currentfill}{rgb}{0.000000,0.000000,0.000000}%
\pgfsetfillcolor{currentfill}%
\pgfsetlinewidth{0.000000pt}%
\definecolor{currentstroke}{rgb}{0.000000,0.000000,0.000000}%
\pgfsetstrokecolor{currentstroke}%
\pgfsetstrokeopacity{0.000000}%
\pgfsetdash{}{0pt}%
\pgfpathmoveto{\pgfqpoint{1.624659in}{0.499444in}}%
\pgfpathlineto{\pgfqpoint{1.688068in}{0.499444in}}%
\pgfpathlineto{\pgfqpoint{1.688068in}{0.557128in}}%
\pgfpathlineto{\pgfqpoint{1.624659in}{0.557128in}}%
\pgfpathlineto{\pgfqpoint{1.624659in}{0.499444in}}%
\pgfpathclose%
\pgfusepath{fill}%
\end{pgfscope}%
\begin{pgfscope}%
\pgfpathrectangle{\pgfqpoint{0.515000in}{0.499444in}}{\pgfqpoint{3.487500in}{1.155000in}}%
\pgfusepath{clip}%
\pgfsetbuttcap%
\pgfsetmiterjoin%
\definecolor{currentfill}{rgb}{0.000000,0.000000,0.000000}%
\pgfsetfillcolor{currentfill}%
\pgfsetlinewidth{0.000000pt}%
\definecolor{currentstroke}{rgb}{0.000000,0.000000,0.000000}%
\pgfsetstrokecolor{currentstroke}%
\pgfsetstrokeopacity{0.000000}%
\pgfsetdash{}{0pt}%
\pgfpathmoveto{\pgfqpoint{1.783182in}{0.499444in}}%
\pgfpathlineto{\pgfqpoint{1.846591in}{0.499444in}}%
\pgfpathlineto{\pgfqpoint{1.846591in}{0.548922in}}%
\pgfpathlineto{\pgfqpoint{1.783182in}{0.548922in}}%
\pgfpathlineto{\pgfqpoint{1.783182in}{0.499444in}}%
\pgfpathclose%
\pgfusepath{fill}%
\end{pgfscope}%
\begin{pgfscope}%
\pgfpathrectangle{\pgfqpoint{0.515000in}{0.499444in}}{\pgfqpoint{3.487500in}{1.155000in}}%
\pgfusepath{clip}%
\pgfsetbuttcap%
\pgfsetmiterjoin%
\definecolor{currentfill}{rgb}{0.000000,0.000000,0.000000}%
\pgfsetfillcolor{currentfill}%
\pgfsetlinewidth{0.000000pt}%
\definecolor{currentstroke}{rgb}{0.000000,0.000000,0.000000}%
\pgfsetstrokecolor{currentstroke}%
\pgfsetstrokeopacity{0.000000}%
\pgfsetdash{}{0pt}%
\pgfpathmoveto{\pgfqpoint{1.941705in}{0.499444in}}%
\pgfpathlineto{\pgfqpoint{2.005114in}{0.499444in}}%
\pgfpathlineto{\pgfqpoint{2.005114in}{0.542597in}}%
\pgfpathlineto{\pgfqpoint{1.941705in}{0.542597in}}%
\pgfpathlineto{\pgfqpoint{1.941705in}{0.499444in}}%
\pgfpathclose%
\pgfusepath{fill}%
\end{pgfscope}%
\begin{pgfscope}%
\pgfpathrectangle{\pgfqpoint{0.515000in}{0.499444in}}{\pgfqpoint{3.487500in}{1.155000in}}%
\pgfusepath{clip}%
\pgfsetbuttcap%
\pgfsetmiterjoin%
\definecolor{currentfill}{rgb}{0.000000,0.000000,0.000000}%
\pgfsetfillcolor{currentfill}%
\pgfsetlinewidth{0.000000pt}%
\definecolor{currentstroke}{rgb}{0.000000,0.000000,0.000000}%
\pgfsetstrokecolor{currentstroke}%
\pgfsetstrokeopacity{0.000000}%
\pgfsetdash{}{0pt}%
\pgfpathmoveto{\pgfqpoint{2.100228in}{0.499444in}}%
\pgfpathlineto{\pgfqpoint{2.163637in}{0.499444in}}%
\pgfpathlineto{\pgfqpoint{2.163637in}{0.534451in}}%
\pgfpathlineto{\pgfqpoint{2.100228in}{0.534451in}}%
\pgfpathlineto{\pgfqpoint{2.100228in}{0.499444in}}%
\pgfpathclose%
\pgfusepath{fill}%
\end{pgfscope}%
\begin{pgfscope}%
\pgfpathrectangle{\pgfqpoint{0.515000in}{0.499444in}}{\pgfqpoint{3.487500in}{1.155000in}}%
\pgfusepath{clip}%
\pgfsetbuttcap%
\pgfsetmiterjoin%
\definecolor{currentfill}{rgb}{0.000000,0.000000,0.000000}%
\pgfsetfillcolor{currentfill}%
\pgfsetlinewidth{0.000000pt}%
\definecolor{currentstroke}{rgb}{0.000000,0.000000,0.000000}%
\pgfsetstrokecolor{currentstroke}%
\pgfsetstrokeopacity{0.000000}%
\pgfsetdash{}{0pt}%
\pgfpathmoveto{\pgfqpoint{2.258750in}{0.499444in}}%
\pgfpathlineto{\pgfqpoint{2.322159in}{0.499444in}}%
\pgfpathlineto{\pgfqpoint{2.322159in}{0.528907in}}%
\pgfpathlineto{\pgfqpoint{2.258750in}{0.528907in}}%
\pgfpathlineto{\pgfqpoint{2.258750in}{0.499444in}}%
\pgfpathclose%
\pgfusepath{fill}%
\end{pgfscope}%
\begin{pgfscope}%
\pgfpathrectangle{\pgfqpoint{0.515000in}{0.499444in}}{\pgfqpoint{3.487500in}{1.155000in}}%
\pgfusepath{clip}%
\pgfsetbuttcap%
\pgfsetmiterjoin%
\definecolor{currentfill}{rgb}{0.000000,0.000000,0.000000}%
\pgfsetfillcolor{currentfill}%
\pgfsetlinewidth{0.000000pt}%
\definecolor{currentstroke}{rgb}{0.000000,0.000000,0.000000}%
\pgfsetstrokecolor{currentstroke}%
\pgfsetstrokeopacity{0.000000}%
\pgfsetdash{}{0pt}%
\pgfpathmoveto{\pgfqpoint{2.417273in}{0.499444in}}%
\pgfpathlineto{\pgfqpoint{2.480682in}{0.499444in}}%
\pgfpathlineto{\pgfqpoint{2.480682in}{0.525304in}}%
\pgfpathlineto{\pgfqpoint{2.417273in}{0.525304in}}%
\pgfpathlineto{\pgfqpoint{2.417273in}{0.499444in}}%
\pgfpathclose%
\pgfusepath{fill}%
\end{pgfscope}%
\begin{pgfscope}%
\pgfpathrectangle{\pgfqpoint{0.515000in}{0.499444in}}{\pgfqpoint{3.487500in}{1.155000in}}%
\pgfusepath{clip}%
\pgfsetbuttcap%
\pgfsetmiterjoin%
\definecolor{currentfill}{rgb}{0.000000,0.000000,0.000000}%
\pgfsetfillcolor{currentfill}%
\pgfsetlinewidth{0.000000pt}%
\definecolor{currentstroke}{rgb}{0.000000,0.000000,0.000000}%
\pgfsetstrokecolor{currentstroke}%
\pgfsetstrokeopacity{0.000000}%
\pgfsetdash{}{0pt}%
\pgfpathmoveto{\pgfqpoint{2.575796in}{0.499444in}}%
\pgfpathlineto{\pgfqpoint{2.639205in}{0.499444in}}%
\pgfpathlineto{\pgfqpoint{2.639205in}{0.521501in}}%
\pgfpathlineto{\pgfqpoint{2.575796in}{0.521501in}}%
\pgfpathlineto{\pgfqpoint{2.575796in}{0.499444in}}%
\pgfpathclose%
\pgfusepath{fill}%
\end{pgfscope}%
\begin{pgfscope}%
\pgfpathrectangle{\pgfqpoint{0.515000in}{0.499444in}}{\pgfqpoint{3.487500in}{1.155000in}}%
\pgfusepath{clip}%
\pgfsetbuttcap%
\pgfsetmiterjoin%
\definecolor{currentfill}{rgb}{0.000000,0.000000,0.000000}%
\pgfsetfillcolor{currentfill}%
\pgfsetlinewidth{0.000000pt}%
\definecolor{currentstroke}{rgb}{0.000000,0.000000,0.000000}%
\pgfsetstrokecolor{currentstroke}%
\pgfsetstrokeopacity{0.000000}%
\pgfsetdash{}{0pt}%
\pgfpathmoveto{\pgfqpoint{2.734318in}{0.499444in}}%
\pgfpathlineto{\pgfqpoint{2.797728in}{0.499444in}}%
\pgfpathlineto{\pgfqpoint{2.797728in}{0.518299in}}%
\pgfpathlineto{\pgfqpoint{2.734318in}{0.518299in}}%
\pgfpathlineto{\pgfqpoint{2.734318in}{0.499444in}}%
\pgfpathclose%
\pgfusepath{fill}%
\end{pgfscope}%
\begin{pgfscope}%
\pgfpathrectangle{\pgfqpoint{0.515000in}{0.499444in}}{\pgfqpoint{3.487500in}{1.155000in}}%
\pgfusepath{clip}%
\pgfsetbuttcap%
\pgfsetmiterjoin%
\definecolor{currentfill}{rgb}{0.000000,0.000000,0.000000}%
\pgfsetfillcolor{currentfill}%
\pgfsetlinewidth{0.000000pt}%
\definecolor{currentstroke}{rgb}{0.000000,0.000000,0.000000}%
\pgfsetstrokecolor{currentstroke}%
\pgfsetstrokeopacity{0.000000}%
\pgfsetdash{}{0pt}%
\pgfpathmoveto{\pgfqpoint{2.892841in}{0.499444in}}%
\pgfpathlineto{\pgfqpoint{2.956250in}{0.499444in}}%
\pgfpathlineto{\pgfqpoint{2.956250in}{0.517418in}}%
\pgfpathlineto{\pgfqpoint{2.892841in}{0.517418in}}%
\pgfpathlineto{\pgfqpoint{2.892841in}{0.499444in}}%
\pgfpathclose%
\pgfusepath{fill}%
\end{pgfscope}%
\begin{pgfscope}%
\pgfpathrectangle{\pgfqpoint{0.515000in}{0.499444in}}{\pgfqpoint{3.487500in}{1.155000in}}%
\pgfusepath{clip}%
\pgfsetbuttcap%
\pgfsetmiterjoin%
\definecolor{currentfill}{rgb}{0.000000,0.000000,0.000000}%
\pgfsetfillcolor{currentfill}%
\pgfsetlinewidth{0.000000pt}%
\definecolor{currentstroke}{rgb}{0.000000,0.000000,0.000000}%
\pgfsetstrokecolor{currentstroke}%
\pgfsetstrokeopacity{0.000000}%
\pgfsetdash{}{0pt}%
\pgfpathmoveto{\pgfqpoint{3.051364in}{0.499444in}}%
\pgfpathlineto{\pgfqpoint{3.114773in}{0.499444in}}%
\pgfpathlineto{\pgfqpoint{3.114773in}{0.515116in}}%
\pgfpathlineto{\pgfqpoint{3.051364in}{0.515116in}}%
\pgfpathlineto{\pgfqpoint{3.051364in}{0.499444in}}%
\pgfpathclose%
\pgfusepath{fill}%
\end{pgfscope}%
\begin{pgfscope}%
\pgfpathrectangle{\pgfqpoint{0.515000in}{0.499444in}}{\pgfqpoint{3.487500in}{1.155000in}}%
\pgfusepath{clip}%
\pgfsetbuttcap%
\pgfsetmiterjoin%
\definecolor{currentfill}{rgb}{0.000000,0.000000,0.000000}%
\pgfsetfillcolor{currentfill}%
\pgfsetlinewidth{0.000000pt}%
\definecolor{currentstroke}{rgb}{0.000000,0.000000,0.000000}%
\pgfsetstrokecolor{currentstroke}%
\pgfsetstrokeopacity{0.000000}%
\pgfsetdash{}{0pt}%
\pgfpathmoveto{\pgfqpoint{3.209887in}{0.499444in}}%
\pgfpathlineto{\pgfqpoint{3.273296in}{0.499444in}}%
\pgfpathlineto{\pgfqpoint{3.273296in}{0.513695in}}%
\pgfpathlineto{\pgfqpoint{3.209887in}{0.513695in}}%
\pgfpathlineto{\pgfqpoint{3.209887in}{0.499444in}}%
\pgfpathclose%
\pgfusepath{fill}%
\end{pgfscope}%
\begin{pgfscope}%
\pgfpathrectangle{\pgfqpoint{0.515000in}{0.499444in}}{\pgfqpoint{3.487500in}{1.155000in}}%
\pgfusepath{clip}%
\pgfsetbuttcap%
\pgfsetmiterjoin%
\definecolor{currentfill}{rgb}{0.000000,0.000000,0.000000}%
\pgfsetfillcolor{currentfill}%
\pgfsetlinewidth{0.000000pt}%
\definecolor{currentstroke}{rgb}{0.000000,0.000000,0.000000}%
\pgfsetstrokecolor{currentstroke}%
\pgfsetstrokeopacity{0.000000}%
\pgfsetdash{}{0pt}%
\pgfpathmoveto{\pgfqpoint{3.368409in}{0.499444in}}%
\pgfpathlineto{\pgfqpoint{3.431818in}{0.499444in}}%
\pgfpathlineto{\pgfqpoint{3.431818in}{0.511653in}}%
\pgfpathlineto{\pgfqpoint{3.368409in}{0.511653in}}%
\pgfpathlineto{\pgfqpoint{3.368409in}{0.499444in}}%
\pgfpathclose%
\pgfusepath{fill}%
\end{pgfscope}%
\begin{pgfscope}%
\pgfpathrectangle{\pgfqpoint{0.515000in}{0.499444in}}{\pgfqpoint{3.487500in}{1.155000in}}%
\pgfusepath{clip}%
\pgfsetbuttcap%
\pgfsetmiterjoin%
\definecolor{currentfill}{rgb}{0.000000,0.000000,0.000000}%
\pgfsetfillcolor{currentfill}%
\pgfsetlinewidth{0.000000pt}%
\definecolor{currentstroke}{rgb}{0.000000,0.000000,0.000000}%
\pgfsetstrokecolor{currentstroke}%
\pgfsetstrokeopacity{0.000000}%
\pgfsetdash{}{0pt}%
\pgfpathmoveto{\pgfqpoint{3.526932in}{0.499444in}}%
\pgfpathlineto{\pgfqpoint{3.590341in}{0.499444in}}%
\pgfpathlineto{\pgfqpoint{3.590341in}{0.505949in}}%
\pgfpathlineto{\pgfqpoint{3.526932in}{0.505949in}}%
\pgfpathlineto{\pgfqpoint{3.526932in}{0.499444in}}%
\pgfpathclose%
\pgfusepath{fill}%
\end{pgfscope}%
\begin{pgfscope}%
\pgfpathrectangle{\pgfqpoint{0.515000in}{0.499444in}}{\pgfqpoint{3.487500in}{1.155000in}}%
\pgfusepath{clip}%
\pgfsetbuttcap%
\pgfsetmiterjoin%
\definecolor{currentfill}{rgb}{0.000000,0.000000,0.000000}%
\pgfsetfillcolor{currentfill}%
\pgfsetlinewidth{0.000000pt}%
\definecolor{currentstroke}{rgb}{0.000000,0.000000,0.000000}%
\pgfsetstrokecolor{currentstroke}%
\pgfsetstrokeopacity{0.000000}%
\pgfsetdash{}{0pt}%
\pgfpathmoveto{\pgfqpoint{3.685455in}{0.499444in}}%
\pgfpathlineto{\pgfqpoint{3.748864in}{0.499444in}}%
\pgfpathlineto{\pgfqpoint{3.748864in}{0.501386in}}%
\pgfpathlineto{\pgfqpoint{3.685455in}{0.501386in}}%
\pgfpathlineto{\pgfqpoint{3.685455in}{0.499444in}}%
\pgfpathclose%
\pgfusepath{fill}%
\end{pgfscope}%
\begin{pgfscope}%
\pgfpathrectangle{\pgfqpoint{0.515000in}{0.499444in}}{\pgfqpoint{3.487500in}{1.155000in}}%
\pgfusepath{clip}%
\pgfsetbuttcap%
\pgfsetmiterjoin%
\definecolor{currentfill}{rgb}{0.000000,0.000000,0.000000}%
\pgfsetfillcolor{currentfill}%
\pgfsetlinewidth{0.000000pt}%
\definecolor{currentstroke}{rgb}{0.000000,0.000000,0.000000}%
\pgfsetstrokecolor{currentstroke}%
\pgfsetstrokeopacity{0.000000}%
\pgfsetdash{}{0pt}%
\pgfpathmoveto{\pgfqpoint{3.843978in}{0.499444in}}%
\pgfpathlineto{\pgfqpoint{3.907387in}{0.499444in}}%
\pgfpathlineto{\pgfqpoint{3.907387in}{0.499464in}}%
\pgfpathlineto{\pgfqpoint{3.843978in}{0.499464in}}%
\pgfpathlineto{\pgfqpoint{3.843978in}{0.499444in}}%
\pgfpathclose%
\pgfusepath{fill}%
\end{pgfscope}%
\begin{pgfscope}%
\pgfsetbuttcap%
\pgfsetroundjoin%
\definecolor{currentfill}{rgb}{0.000000,0.000000,0.000000}%
\pgfsetfillcolor{currentfill}%
\pgfsetlinewidth{0.803000pt}%
\definecolor{currentstroke}{rgb}{0.000000,0.000000,0.000000}%
\pgfsetstrokecolor{currentstroke}%
\pgfsetdash{}{0pt}%
\pgfsys@defobject{currentmarker}{\pgfqpoint{0.000000in}{-0.048611in}}{\pgfqpoint{0.000000in}{0.000000in}}{%
\pgfpathmoveto{\pgfqpoint{0.000000in}{0.000000in}}%
\pgfpathlineto{\pgfqpoint{0.000000in}{-0.048611in}}%
\pgfusepath{stroke,fill}%
}%
\begin{pgfscope}%
\pgfsys@transformshift{0.515000in}{0.499444in}%
\pgfsys@useobject{currentmarker}{}%
\end{pgfscope}%
\end{pgfscope}%
\begin{pgfscope}%
\pgfsetbuttcap%
\pgfsetroundjoin%
\definecolor{currentfill}{rgb}{0.000000,0.000000,0.000000}%
\pgfsetfillcolor{currentfill}%
\pgfsetlinewidth{0.803000pt}%
\definecolor{currentstroke}{rgb}{0.000000,0.000000,0.000000}%
\pgfsetstrokecolor{currentstroke}%
\pgfsetdash{}{0pt}%
\pgfsys@defobject{currentmarker}{\pgfqpoint{0.000000in}{-0.048611in}}{\pgfqpoint{0.000000in}{0.000000in}}{%
\pgfpathmoveto{\pgfqpoint{0.000000in}{0.000000in}}%
\pgfpathlineto{\pgfqpoint{0.000000in}{-0.048611in}}%
\pgfusepath{stroke,fill}%
}%
\begin{pgfscope}%
\pgfsys@transformshift{0.673523in}{0.499444in}%
\pgfsys@useobject{currentmarker}{}%
\end{pgfscope}%
\end{pgfscope}%
\begin{pgfscope}%
\definecolor{textcolor}{rgb}{0.000000,0.000000,0.000000}%
\pgfsetstrokecolor{textcolor}%
\pgfsetfillcolor{textcolor}%
\pgftext[x=0.673523in,y=0.402222in,,top]{\color{textcolor}\rmfamily\fontsize{10.000000}{12.000000}\selectfont 0.0}%
\end{pgfscope}%
\begin{pgfscope}%
\pgfsetbuttcap%
\pgfsetroundjoin%
\definecolor{currentfill}{rgb}{0.000000,0.000000,0.000000}%
\pgfsetfillcolor{currentfill}%
\pgfsetlinewidth{0.803000pt}%
\definecolor{currentstroke}{rgb}{0.000000,0.000000,0.000000}%
\pgfsetstrokecolor{currentstroke}%
\pgfsetdash{}{0pt}%
\pgfsys@defobject{currentmarker}{\pgfqpoint{0.000000in}{-0.048611in}}{\pgfqpoint{0.000000in}{0.000000in}}{%
\pgfpathmoveto{\pgfqpoint{0.000000in}{0.000000in}}%
\pgfpathlineto{\pgfqpoint{0.000000in}{-0.048611in}}%
\pgfusepath{stroke,fill}%
}%
\begin{pgfscope}%
\pgfsys@transformshift{0.832046in}{0.499444in}%
\pgfsys@useobject{currentmarker}{}%
\end{pgfscope}%
\end{pgfscope}%
\begin{pgfscope}%
\pgfsetbuttcap%
\pgfsetroundjoin%
\definecolor{currentfill}{rgb}{0.000000,0.000000,0.000000}%
\pgfsetfillcolor{currentfill}%
\pgfsetlinewidth{0.803000pt}%
\definecolor{currentstroke}{rgb}{0.000000,0.000000,0.000000}%
\pgfsetstrokecolor{currentstroke}%
\pgfsetdash{}{0pt}%
\pgfsys@defobject{currentmarker}{\pgfqpoint{0.000000in}{-0.048611in}}{\pgfqpoint{0.000000in}{0.000000in}}{%
\pgfpathmoveto{\pgfqpoint{0.000000in}{0.000000in}}%
\pgfpathlineto{\pgfqpoint{0.000000in}{-0.048611in}}%
\pgfusepath{stroke,fill}%
}%
\begin{pgfscope}%
\pgfsys@transformshift{0.990568in}{0.499444in}%
\pgfsys@useobject{currentmarker}{}%
\end{pgfscope}%
\end{pgfscope}%
\begin{pgfscope}%
\definecolor{textcolor}{rgb}{0.000000,0.000000,0.000000}%
\pgfsetstrokecolor{textcolor}%
\pgfsetfillcolor{textcolor}%
\pgftext[x=0.990568in,y=0.402222in,,top]{\color{textcolor}\rmfamily\fontsize{10.000000}{12.000000}\selectfont 0.1}%
\end{pgfscope}%
\begin{pgfscope}%
\pgfsetbuttcap%
\pgfsetroundjoin%
\definecolor{currentfill}{rgb}{0.000000,0.000000,0.000000}%
\pgfsetfillcolor{currentfill}%
\pgfsetlinewidth{0.803000pt}%
\definecolor{currentstroke}{rgb}{0.000000,0.000000,0.000000}%
\pgfsetstrokecolor{currentstroke}%
\pgfsetdash{}{0pt}%
\pgfsys@defobject{currentmarker}{\pgfqpoint{0.000000in}{-0.048611in}}{\pgfqpoint{0.000000in}{0.000000in}}{%
\pgfpathmoveto{\pgfqpoint{0.000000in}{0.000000in}}%
\pgfpathlineto{\pgfqpoint{0.000000in}{-0.048611in}}%
\pgfusepath{stroke,fill}%
}%
\begin{pgfscope}%
\pgfsys@transformshift{1.149091in}{0.499444in}%
\pgfsys@useobject{currentmarker}{}%
\end{pgfscope}%
\end{pgfscope}%
\begin{pgfscope}%
\pgfsetbuttcap%
\pgfsetroundjoin%
\definecolor{currentfill}{rgb}{0.000000,0.000000,0.000000}%
\pgfsetfillcolor{currentfill}%
\pgfsetlinewidth{0.803000pt}%
\definecolor{currentstroke}{rgb}{0.000000,0.000000,0.000000}%
\pgfsetstrokecolor{currentstroke}%
\pgfsetdash{}{0pt}%
\pgfsys@defobject{currentmarker}{\pgfqpoint{0.000000in}{-0.048611in}}{\pgfqpoint{0.000000in}{0.000000in}}{%
\pgfpathmoveto{\pgfqpoint{0.000000in}{0.000000in}}%
\pgfpathlineto{\pgfqpoint{0.000000in}{-0.048611in}}%
\pgfusepath{stroke,fill}%
}%
\begin{pgfscope}%
\pgfsys@transformshift{1.307614in}{0.499444in}%
\pgfsys@useobject{currentmarker}{}%
\end{pgfscope}%
\end{pgfscope}%
\begin{pgfscope}%
\definecolor{textcolor}{rgb}{0.000000,0.000000,0.000000}%
\pgfsetstrokecolor{textcolor}%
\pgfsetfillcolor{textcolor}%
\pgftext[x=1.307614in,y=0.402222in,,top]{\color{textcolor}\rmfamily\fontsize{10.000000}{12.000000}\selectfont 0.2}%
\end{pgfscope}%
\begin{pgfscope}%
\pgfsetbuttcap%
\pgfsetroundjoin%
\definecolor{currentfill}{rgb}{0.000000,0.000000,0.000000}%
\pgfsetfillcolor{currentfill}%
\pgfsetlinewidth{0.803000pt}%
\definecolor{currentstroke}{rgb}{0.000000,0.000000,0.000000}%
\pgfsetstrokecolor{currentstroke}%
\pgfsetdash{}{0pt}%
\pgfsys@defobject{currentmarker}{\pgfqpoint{0.000000in}{-0.048611in}}{\pgfqpoint{0.000000in}{0.000000in}}{%
\pgfpathmoveto{\pgfqpoint{0.000000in}{0.000000in}}%
\pgfpathlineto{\pgfqpoint{0.000000in}{-0.048611in}}%
\pgfusepath{stroke,fill}%
}%
\begin{pgfscope}%
\pgfsys@transformshift{1.466137in}{0.499444in}%
\pgfsys@useobject{currentmarker}{}%
\end{pgfscope}%
\end{pgfscope}%
\begin{pgfscope}%
\pgfsetbuttcap%
\pgfsetroundjoin%
\definecolor{currentfill}{rgb}{0.000000,0.000000,0.000000}%
\pgfsetfillcolor{currentfill}%
\pgfsetlinewidth{0.803000pt}%
\definecolor{currentstroke}{rgb}{0.000000,0.000000,0.000000}%
\pgfsetstrokecolor{currentstroke}%
\pgfsetdash{}{0pt}%
\pgfsys@defobject{currentmarker}{\pgfqpoint{0.000000in}{-0.048611in}}{\pgfqpoint{0.000000in}{0.000000in}}{%
\pgfpathmoveto{\pgfqpoint{0.000000in}{0.000000in}}%
\pgfpathlineto{\pgfqpoint{0.000000in}{-0.048611in}}%
\pgfusepath{stroke,fill}%
}%
\begin{pgfscope}%
\pgfsys@transformshift{1.624659in}{0.499444in}%
\pgfsys@useobject{currentmarker}{}%
\end{pgfscope}%
\end{pgfscope}%
\begin{pgfscope}%
\definecolor{textcolor}{rgb}{0.000000,0.000000,0.000000}%
\pgfsetstrokecolor{textcolor}%
\pgfsetfillcolor{textcolor}%
\pgftext[x=1.624659in,y=0.402222in,,top]{\color{textcolor}\rmfamily\fontsize{10.000000}{12.000000}\selectfont 0.3}%
\end{pgfscope}%
\begin{pgfscope}%
\pgfsetbuttcap%
\pgfsetroundjoin%
\definecolor{currentfill}{rgb}{0.000000,0.000000,0.000000}%
\pgfsetfillcolor{currentfill}%
\pgfsetlinewidth{0.803000pt}%
\definecolor{currentstroke}{rgb}{0.000000,0.000000,0.000000}%
\pgfsetstrokecolor{currentstroke}%
\pgfsetdash{}{0pt}%
\pgfsys@defobject{currentmarker}{\pgfqpoint{0.000000in}{-0.048611in}}{\pgfqpoint{0.000000in}{0.000000in}}{%
\pgfpathmoveto{\pgfqpoint{0.000000in}{0.000000in}}%
\pgfpathlineto{\pgfqpoint{0.000000in}{-0.048611in}}%
\pgfusepath{stroke,fill}%
}%
\begin{pgfscope}%
\pgfsys@transformshift{1.783182in}{0.499444in}%
\pgfsys@useobject{currentmarker}{}%
\end{pgfscope}%
\end{pgfscope}%
\begin{pgfscope}%
\pgfsetbuttcap%
\pgfsetroundjoin%
\definecolor{currentfill}{rgb}{0.000000,0.000000,0.000000}%
\pgfsetfillcolor{currentfill}%
\pgfsetlinewidth{0.803000pt}%
\definecolor{currentstroke}{rgb}{0.000000,0.000000,0.000000}%
\pgfsetstrokecolor{currentstroke}%
\pgfsetdash{}{0pt}%
\pgfsys@defobject{currentmarker}{\pgfqpoint{0.000000in}{-0.048611in}}{\pgfqpoint{0.000000in}{0.000000in}}{%
\pgfpathmoveto{\pgfqpoint{0.000000in}{0.000000in}}%
\pgfpathlineto{\pgfqpoint{0.000000in}{-0.048611in}}%
\pgfusepath{stroke,fill}%
}%
\begin{pgfscope}%
\pgfsys@transformshift{1.941705in}{0.499444in}%
\pgfsys@useobject{currentmarker}{}%
\end{pgfscope}%
\end{pgfscope}%
\begin{pgfscope}%
\definecolor{textcolor}{rgb}{0.000000,0.000000,0.000000}%
\pgfsetstrokecolor{textcolor}%
\pgfsetfillcolor{textcolor}%
\pgftext[x=1.941705in,y=0.402222in,,top]{\color{textcolor}\rmfamily\fontsize{10.000000}{12.000000}\selectfont 0.4}%
\end{pgfscope}%
\begin{pgfscope}%
\pgfsetbuttcap%
\pgfsetroundjoin%
\definecolor{currentfill}{rgb}{0.000000,0.000000,0.000000}%
\pgfsetfillcolor{currentfill}%
\pgfsetlinewidth{0.803000pt}%
\definecolor{currentstroke}{rgb}{0.000000,0.000000,0.000000}%
\pgfsetstrokecolor{currentstroke}%
\pgfsetdash{}{0pt}%
\pgfsys@defobject{currentmarker}{\pgfqpoint{0.000000in}{-0.048611in}}{\pgfqpoint{0.000000in}{0.000000in}}{%
\pgfpathmoveto{\pgfqpoint{0.000000in}{0.000000in}}%
\pgfpathlineto{\pgfqpoint{0.000000in}{-0.048611in}}%
\pgfusepath{stroke,fill}%
}%
\begin{pgfscope}%
\pgfsys@transformshift{2.100228in}{0.499444in}%
\pgfsys@useobject{currentmarker}{}%
\end{pgfscope}%
\end{pgfscope}%
\begin{pgfscope}%
\pgfsetbuttcap%
\pgfsetroundjoin%
\definecolor{currentfill}{rgb}{0.000000,0.000000,0.000000}%
\pgfsetfillcolor{currentfill}%
\pgfsetlinewidth{0.803000pt}%
\definecolor{currentstroke}{rgb}{0.000000,0.000000,0.000000}%
\pgfsetstrokecolor{currentstroke}%
\pgfsetdash{}{0pt}%
\pgfsys@defobject{currentmarker}{\pgfqpoint{0.000000in}{-0.048611in}}{\pgfqpoint{0.000000in}{0.000000in}}{%
\pgfpathmoveto{\pgfqpoint{0.000000in}{0.000000in}}%
\pgfpathlineto{\pgfqpoint{0.000000in}{-0.048611in}}%
\pgfusepath{stroke,fill}%
}%
\begin{pgfscope}%
\pgfsys@transformshift{2.258750in}{0.499444in}%
\pgfsys@useobject{currentmarker}{}%
\end{pgfscope}%
\end{pgfscope}%
\begin{pgfscope}%
\definecolor{textcolor}{rgb}{0.000000,0.000000,0.000000}%
\pgfsetstrokecolor{textcolor}%
\pgfsetfillcolor{textcolor}%
\pgftext[x=2.258750in,y=0.402222in,,top]{\color{textcolor}\rmfamily\fontsize{10.000000}{12.000000}\selectfont 0.5}%
\end{pgfscope}%
\begin{pgfscope}%
\pgfsetbuttcap%
\pgfsetroundjoin%
\definecolor{currentfill}{rgb}{0.000000,0.000000,0.000000}%
\pgfsetfillcolor{currentfill}%
\pgfsetlinewidth{0.803000pt}%
\definecolor{currentstroke}{rgb}{0.000000,0.000000,0.000000}%
\pgfsetstrokecolor{currentstroke}%
\pgfsetdash{}{0pt}%
\pgfsys@defobject{currentmarker}{\pgfqpoint{0.000000in}{-0.048611in}}{\pgfqpoint{0.000000in}{0.000000in}}{%
\pgfpathmoveto{\pgfqpoint{0.000000in}{0.000000in}}%
\pgfpathlineto{\pgfqpoint{0.000000in}{-0.048611in}}%
\pgfusepath{stroke,fill}%
}%
\begin{pgfscope}%
\pgfsys@transformshift{2.417273in}{0.499444in}%
\pgfsys@useobject{currentmarker}{}%
\end{pgfscope}%
\end{pgfscope}%
\begin{pgfscope}%
\pgfsetbuttcap%
\pgfsetroundjoin%
\definecolor{currentfill}{rgb}{0.000000,0.000000,0.000000}%
\pgfsetfillcolor{currentfill}%
\pgfsetlinewidth{0.803000pt}%
\definecolor{currentstroke}{rgb}{0.000000,0.000000,0.000000}%
\pgfsetstrokecolor{currentstroke}%
\pgfsetdash{}{0pt}%
\pgfsys@defobject{currentmarker}{\pgfqpoint{0.000000in}{-0.048611in}}{\pgfqpoint{0.000000in}{0.000000in}}{%
\pgfpathmoveto{\pgfqpoint{0.000000in}{0.000000in}}%
\pgfpathlineto{\pgfqpoint{0.000000in}{-0.048611in}}%
\pgfusepath{stroke,fill}%
}%
\begin{pgfscope}%
\pgfsys@transformshift{2.575796in}{0.499444in}%
\pgfsys@useobject{currentmarker}{}%
\end{pgfscope}%
\end{pgfscope}%
\begin{pgfscope}%
\definecolor{textcolor}{rgb}{0.000000,0.000000,0.000000}%
\pgfsetstrokecolor{textcolor}%
\pgfsetfillcolor{textcolor}%
\pgftext[x=2.575796in,y=0.402222in,,top]{\color{textcolor}\rmfamily\fontsize{10.000000}{12.000000}\selectfont 0.6}%
\end{pgfscope}%
\begin{pgfscope}%
\pgfsetbuttcap%
\pgfsetroundjoin%
\definecolor{currentfill}{rgb}{0.000000,0.000000,0.000000}%
\pgfsetfillcolor{currentfill}%
\pgfsetlinewidth{0.803000pt}%
\definecolor{currentstroke}{rgb}{0.000000,0.000000,0.000000}%
\pgfsetstrokecolor{currentstroke}%
\pgfsetdash{}{0pt}%
\pgfsys@defobject{currentmarker}{\pgfqpoint{0.000000in}{-0.048611in}}{\pgfqpoint{0.000000in}{0.000000in}}{%
\pgfpathmoveto{\pgfqpoint{0.000000in}{0.000000in}}%
\pgfpathlineto{\pgfqpoint{0.000000in}{-0.048611in}}%
\pgfusepath{stroke,fill}%
}%
\begin{pgfscope}%
\pgfsys@transformshift{2.734318in}{0.499444in}%
\pgfsys@useobject{currentmarker}{}%
\end{pgfscope}%
\end{pgfscope}%
\begin{pgfscope}%
\pgfsetbuttcap%
\pgfsetroundjoin%
\definecolor{currentfill}{rgb}{0.000000,0.000000,0.000000}%
\pgfsetfillcolor{currentfill}%
\pgfsetlinewidth{0.803000pt}%
\definecolor{currentstroke}{rgb}{0.000000,0.000000,0.000000}%
\pgfsetstrokecolor{currentstroke}%
\pgfsetdash{}{0pt}%
\pgfsys@defobject{currentmarker}{\pgfqpoint{0.000000in}{-0.048611in}}{\pgfqpoint{0.000000in}{0.000000in}}{%
\pgfpathmoveto{\pgfqpoint{0.000000in}{0.000000in}}%
\pgfpathlineto{\pgfqpoint{0.000000in}{-0.048611in}}%
\pgfusepath{stroke,fill}%
}%
\begin{pgfscope}%
\pgfsys@transformshift{2.892841in}{0.499444in}%
\pgfsys@useobject{currentmarker}{}%
\end{pgfscope}%
\end{pgfscope}%
\begin{pgfscope}%
\definecolor{textcolor}{rgb}{0.000000,0.000000,0.000000}%
\pgfsetstrokecolor{textcolor}%
\pgfsetfillcolor{textcolor}%
\pgftext[x=2.892841in,y=0.402222in,,top]{\color{textcolor}\rmfamily\fontsize{10.000000}{12.000000}\selectfont 0.7}%
\end{pgfscope}%
\begin{pgfscope}%
\pgfsetbuttcap%
\pgfsetroundjoin%
\definecolor{currentfill}{rgb}{0.000000,0.000000,0.000000}%
\pgfsetfillcolor{currentfill}%
\pgfsetlinewidth{0.803000pt}%
\definecolor{currentstroke}{rgb}{0.000000,0.000000,0.000000}%
\pgfsetstrokecolor{currentstroke}%
\pgfsetdash{}{0pt}%
\pgfsys@defobject{currentmarker}{\pgfqpoint{0.000000in}{-0.048611in}}{\pgfqpoint{0.000000in}{0.000000in}}{%
\pgfpathmoveto{\pgfqpoint{0.000000in}{0.000000in}}%
\pgfpathlineto{\pgfqpoint{0.000000in}{-0.048611in}}%
\pgfusepath{stroke,fill}%
}%
\begin{pgfscope}%
\pgfsys@transformshift{3.051364in}{0.499444in}%
\pgfsys@useobject{currentmarker}{}%
\end{pgfscope}%
\end{pgfscope}%
\begin{pgfscope}%
\pgfsetbuttcap%
\pgfsetroundjoin%
\definecolor{currentfill}{rgb}{0.000000,0.000000,0.000000}%
\pgfsetfillcolor{currentfill}%
\pgfsetlinewidth{0.803000pt}%
\definecolor{currentstroke}{rgb}{0.000000,0.000000,0.000000}%
\pgfsetstrokecolor{currentstroke}%
\pgfsetdash{}{0pt}%
\pgfsys@defobject{currentmarker}{\pgfqpoint{0.000000in}{-0.048611in}}{\pgfqpoint{0.000000in}{0.000000in}}{%
\pgfpathmoveto{\pgfqpoint{0.000000in}{0.000000in}}%
\pgfpathlineto{\pgfqpoint{0.000000in}{-0.048611in}}%
\pgfusepath{stroke,fill}%
}%
\begin{pgfscope}%
\pgfsys@transformshift{3.209887in}{0.499444in}%
\pgfsys@useobject{currentmarker}{}%
\end{pgfscope}%
\end{pgfscope}%
\begin{pgfscope}%
\definecolor{textcolor}{rgb}{0.000000,0.000000,0.000000}%
\pgfsetstrokecolor{textcolor}%
\pgfsetfillcolor{textcolor}%
\pgftext[x=3.209887in,y=0.402222in,,top]{\color{textcolor}\rmfamily\fontsize{10.000000}{12.000000}\selectfont 0.8}%
\end{pgfscope}%
\begin{pgfscope}%
\pgfsetbuttcap%
\pgfsetroundjoin%
\definecolor{currentfill}{rgb}{0.000000,0.000000,0.000000}%
\pgfsetfillcolor{currentfill}%
\pgfsetlinewidth{0.803000pt}%
\definecolor{currentstroke}{rgb}{0.000000,0.000000,0.000000}%
\pgfsetstrokecolor{currentstroke}%
\pgfsetdash{}{0pt}%
\pgfsys@defobject{currentmarker}{\pgfqpoint{0.000000in}{-0.048611in}}{\pgfqpoint{0.000000in}{0.000000in}}{%
\pgfpathmoveto{\pgfqpoint{0.000000in}{0.000000in}}%
\pgfpathlineto{\pgfqpoint{0.000000in}{-0.048611in}}%
\pgfusepath{stroke,fill}%
}%
\begin{pgfscope}%
\pgfsys@transformshift{3.368409in}{0.499444in}%
\pgfsys@useobject{currentmarker}{}%
\end{pgfscope}%
\end{pgfscope}%
\begin{pgfscope}%
\pgfsetbuttcap%
\pgfsetroundjoin%
\definecolor{currentfill}{rgb}{0.000000,0.000000,0.000000}%
\pgfsetfillcolor{currentfill}%
\pgfsetlinewidth{0.803000pt}%
\definecolor{currentstroke}{rgb}{0.000000,0.000000,0.000000}%
\pgfsetstrokecolor{currentstroke}%
\pgfsetdash{}{0pt}%
\pgfsys@defobject{currentmarker}{\pgfqpoint{0.000000in}{-0.048611in}}{\pgfqpoint{0.000000in}{0.000000in}}{%
\pgfpathmoveto{\pgfqpoint{0.000000in}{0.000000in}}%
\pgfpathlineto{\pgfqpoint{0.000000in}{-0.048611in}}%
\pgfusepath{stroke,fill}%
}%
\begin{pgfscope}%
\pgfsys@transformshift{3.526932in}{0.499444in}%
\pgfsys@useobject{currentmarker}{}%
\end{pgfscope}%
\end{pgfscope}%
\begin{pgfscope}%
\definecolor{textcolor}{rgb}{0.000000,0.000000,0.000000}%
\pgfsetstrokecolor{textcolor}%
\pgfsetfillcolor{textcolor}%
\pgftext[x=3.526932in,y=0.402222in,,top]{\color{textcolor}\rmfamily\fontsize{10.000000}{12.000000}\selectfont 0.9}%
\end{pgfscope}%
\begin{pgfscope}%
\pgfsetbuttcap%
\pgfsetroundjoin%
\definecolor{currentfill}{rgb}{0.000000,0.000000,0.000000}%
\pgfsetfillcolor{currentfill}%
\pgfsetlinewidth{0.803000pt}%
\definecolor{currentstroke}{rgb}{0.000000,0.000000,0.000000}%
\pgfsetstrokecolor{currentstroke}%
\pgfsetdash{}{0pt}%
\pgfsys@defobject{currentmarker}{\pgfqpoint{0.000000in}{-0.048611in}}{\pgfqpoint{0.000000in}{0.000000in}}{%
\pgfpathmoveto{\pgfqpoint{0.000000in}{0.000000in}}%
\pgfpathlineto{\pgfqpoint{0.000000in}{-0.048611in}}%
\pgfusepath{stroke,fill}%
}%
\begin{pgfscope}%
\pgfsys@transformshift{3.685455in}{0.499444in}%
\pgfsys@useobject{currentmarker}{}%
\end{pgfscope}%
\end{pgfscope}%
\begin{pgfscope}%
\pgfsetbuttcap%
\pgfsetroundjoin%
\definecolor{currentfill}{rgb}{0.000000,0.000000,0.000000}%
\pgfsetfillcolor{currentfill}%
\pgfsetlinewidth{0.803000pt}%
\definecolor{currentstroke}{rgb}{0.000000,0.000000,0.000000}%
\pgfsetstrokecolor{currentstroke}%
\pgfsetdash{}{0pt}%
\pgfsys@defobject{currentmarker}{\pgfqpoint{0.000000in}{-0.048611in}}{\pgfqpoint{0.000000in}{0.000000in}}{%
\pgfpathmoveto{\pgfqpoint{0.000000in}{0.000000in}}%
\pgfpathlineto{\pgfqpoint{0.000000in}{-0.048611in}}%
\pgfusepath{stroke,fill}%
}%
\begin{pgfscope}%
\pgfsys@transformshift{3.843978in}{0.499444in}%
\pgfsys@useobject{currentmarker}{}%
\end{pgfscope}%
\end{pgfscope}%
\begin{pgfscope}%
\definecolor{textcolor}{rgb}{0.000000,0.000000,0.000000}%
\pgfsetstrokecolor{textcolor}%
\pgfsetfillcolor{textcolor}%
\pgftext[x=3.843978in,y=0.402222in,,top]{\color{textcolor}\rmfamily\fontsize{10.000000}{12.000000}\selectfont 1.0}%
\end{pgfscope}%
\begin{pgfscope}%
\pgfsetbuttcap%
\pgfsetroundjoin%
\definecolor{currentfill}{rgb}{0.000000,0.000000,0.000000}%
\pgfsetfillcolor{currentfill}%
\pgfsetlinewidth{0.803000pt}%
\definecolor{currentstroke}{rgb}{0.000000,0.000000,0.000000}%
\pgfsetstrokecolor{currentstroke}%
\pgfsetdash{}{0pt}%
\pgfsys@defobject{currentmarker}{\pgfqpoint{0.000000in}{-0.048611in}}{\pgfqpoint{0.000000in}{0.000000in}}{%
\pgfpathmoveto{\pgfqpoint{0.000000in}{0.000000in}}%
\pgfpathlineto{\pgfqpoint{0.000000in}{-0.048611in}}%
\pgfusepath{stroke,fill}%
}%
\begin{pgfscope}%
\pgfsys@transformshift{4.002500in}{0.499444in}%
\pgfsys@useobject{currentmarker}{}%
\end{pgfscope}%
\end{pgfscope}%
\begin{pgfscope}%
\definecolor{textcolor}{rgb}{0.000000,0.000000,0.000000}%
\pgfsetstrokecolor{textcolor}%
\pgfsetfillcolor{textcolor}%
\pgftext[x=2.258750in,y=0.223333in,,top]{\color{textcolor}\rmfamily\fontsize{10.000000}{12.000000}\selectfont \(\displaystyle p\)}%
\end{pgfscope}%
\begin{pgfscope}%
\pgfsetbuttcap%
\pgfsetroundjoin%
\definecolor{currentfill}{rgb}{0.000000,0.000000,0.000000}%
\pgfsetfillcolor{currentfill}%
\pgfsetlinewidth{0.803000pt}%
\definecolor{currentstroke}{rgb}{0.000000,0.000000,0.000000}%
\pgfsetstrokecolor{currentstroke}%
\pgfsetdash{}{0pt}%
\pgfsys@defobject{currentmarker}{\pgfqpoint{-0.048611in}{0.000000in}}{\pgfqpoint{-0.000000in}{0.000000in}}{%
\pgfpathmoveto{\pgfqpoint{-0.000000in}{0.000000in}}%
\pgfpathlineto{\pgfqpoint{-0.048611in}{0.000000in}}%
\pgfusepath{stroke,fill}%
}%
\begin{pgfscope}%
\pgfsys@transformshift{0.515000in}{0.499444in}%
\pgfsys@useobject{currentmarker}{}%
\end{pgfscope}%
\end{pgfscope}%
\begin{pgfscope}%
\definecolor{textcolor}{rgb}{0.000000,0.000000,0.000000}%
\pgfsetstrokecolor{textcolor}%
\pgfsetfillcolor{textcolor}%
\pgftext[x=0.348333in, y=0.451250in, left, base]{\color{textcolor}\rmfamily\fontsize{10.000000}{12.000000}\selectfont \(\displaystyle {0}\)}%
\end{pgfscope}%
\begin{pgfscope}%
\pgfsetbuttcap%
\pgfsetroundjoin%
\definecolor{currentfill}{rgb}{0.000000,0.000000,0.000000}%
\pgfsetfillcolor{currentfill}%
\pgfsetlinewidth{0.803000pt}%
\definecolor{currentstroke}{rgb}{0.000000,0.000000,0.000000}%
\pgfsetstrokecolor{currentstroke}%
\pgfsetdash{}{0pt}%
\pgfsys@defobject{currentmarker}{\pgfqpoint{-0.048611in}{0.000000in}}{\pgfqpoint{-0.000000in}{0.000000in}}{%
\pgfpathmoveto{\pgfqpoint{-0.000000in}{0.000000in}}%
\pgfpathlineto{\pgfqpoint{-0.048611in}{0.000000in}}%
\pgfusepath{stroke,fill}%
}%
\begin{pgfscope}%
\pgfsys@transformshift{0.515000in}{0.927911in}%
\pgfsys@useobject{currentmarker}{}%
\end{pgfscope}%
\end{pgfscope}%
\begin{pgfscope}%
\definecolor{textcolor}{rgb}{0.000000,0.000000,0.000000}%
\pgfsetstrokecolor{textcolor}%
\pgfsetfillcolor{textcolor}%
\pgftext[x=0.278889in, y=0.879717in, left, base]{\color{textcolor}\rmfamily\fontsize{10.000000}{12.000000}\selectfont \(\displaystyle {10}\)}%
\end{pgfscope}%
\begin{pgfscope}%
\pgfsetbuttcap%
\pgfsetroundjoin%
\definecolor{currentfill}{rgb}{0.000000,0.000000,0.000000}%
\pgfsetfillcolor{currentfill}%
\pgfsetlinewidth{0.803000pt}%
\definecolor{currentstroke}{rgb}{0.000000,0.000000,0.000000}%
\pgfsetstrokecolor{currentstroke}%
\pgfsetdash{}{0pt}%
\pgfsys@defobject{currentmarker}{\pgfqpoint{-0.048611in}{0.000000in}}{\pgfqpoint{-0.000000in}{0.000000in}}{%
\pgfpathmoveto{\pgfqpoint{-0.000000in}{0.000000in}}%
\pgfpathlineto{\pgfqpoint{-0.048611in}{0.000000in}}%
\pgfusepath{stroke,fill}%
}%
\begin{pgfscope}%
\pgfsys@transformshift{0.515000in}{1.356379in}%
\pgfsys@useobject{currentmarker}{}%
\end{pgfscope}%
\end{pgfscope}%
\begin{pgfscope}%
\definecolor{textcolor}{rgb}{0.000000,0.000000,0.000000}%
\pgfsetstrokecolor{textcolor}%
\pgfsetfillcolor{textcolor}%
\pgftext[x=0.278889in, y=1.308184in, left, base]{\color{textcolor}\rmfamily\fontsize{10.000000}{12.000000}\selectfont \(\displaystyle {20}\)}%
\end{pgfscope}%
\begin{pgfscope}%
\definecolor{textcolor}{rgb}{0.000000,0.000000,0.000000}%
\pgfsetstrokecolor{textcolor}%
\pgfsetfillcolor{textcolor}%
\pgftext[x=0.223333in,y=1.076944in,,bottom,rotate=90.000000]{\color{textcolor}\rmfamily\fontsize{10.000000}{12.000000}\selectfont Percent of Data Set}%
\end{pgfscope}%
\begin{pgfscope}%
\pgfsetrectcap%
\pgfsetmiterjoin%
\pgfsetlinewidth{0.803000pt}%
\definecolor{currentstroke}{rgb}{0.000000,0.000000,0.000000}%
\pgfsetstrokecolor{currentstroke}%
\pgfsetdash{}{0pt}%
\pgfpathmoveto{\pgfqpoint{0.515000in}{0.499444in}}%
\pgfpathlineto{\pgfqpoint{0.515000in}{1.654444in}}%
\pgfusepath{stroke}%
\end{pgfscope}%
\begin{pgfscope}%
\pgfsetrectcap%
\pgfsetmiterjoin%
\pgfsetlinewidth{0.803000pt}%
\definecolor{currentstroke}{rgb}{0.000000,0.000000,0.000000}%
\pgfsetstrokecolor{currentstroke}%
\pgfsetdash{}{0pt}%
\pgfpathmoveto{\pgfqpoint{4.002500in}{0.499444in}}%
\pgfpathlineto{\pgfqpoint{4.002500in}{1.654444in}}%
\pgfusepath{stroke}%
\end{pgfscope}%
\begin{pgfscope}%
\pgfsetrectcap%
\pgfsetmiterjoin%
\pgfsetlinewidth{0.803000pt}%
\definecolor{currentstroke}{rgb}{0.000000,0.000000,0.000000}%
\pgfsetstrokecolor{currentstroke}%
\pgfsetdash{}{0pt}%
\pgfpathmoveto{\pgfqpoint{0.515000in}{0.499444in}}%
\pgfpathlineto{\pgfqpoint{4.002500in}{0.499444in}}%
\pgfusepath{stroke}%
\end{pgfscope}%
\begin{pgfscope}%
\pgfsetrectcap%
\pgfsetmiterjoin%
\pgfsetlinewidth{0.803000pt}%
\definecolor{currentstroke}{rgb}{0.000000,0.000000,0.000000}%
\pgfsetstrokecolor{currentstroke}%
\pgfsetdash{}{0pt}%
\pgfpathmoveto{\pgfqpoint{0.515000in}{1.654444in}}%
\pgfpathlineto{\pgfqpoint{4.002500in}{1.654444in}}%
\pgfusepath{stroke}%
\end{pgfscope}%
\begin{pgfscope}%
\pgfsetbuttcap%
\pgfsetmiterjoin%
\definecolor{currentfill}{rgb}{1.000000,1.000000,1.000000}%
\pgfsetfillcolor{currentfill}%
\pgfsetfillopacity{0.800000}%
\pgfsetlinewidth{1.003750pt}%
\definecolor{currentstroke}{rgb}{0.800000,0.800000,0.800000}%
\pgfsetstrokecolor{currentstroke}%
\pgfsetstrokeopacity{0.800000}%
\pgfsetdash{}{0pt}%
\pgfpathmoveto{\pgfqpoint{3.225556in}{1.154445in}}%
\pgfpathlineto{\pgfqpoint{3.905278in}{1.154445in}}%
\pgfpathquadraticcurveto{\pgfqpoint{3.933056in}{1.154445in}}{\pgfqpoint{3.933056in}{1.182222in}}%
\pgfpathlineto{\pgfqpoint{3.933056in}{1.557222in}}%
\pgfpathquadraticcurveto{\pgfqpoint{3.933056in}{1.585000in}}{\pgfqpoint{3.905278in}{1.585000in}}%
\pgfpathlineto{\pgfqpoint{3.225556in}{1.585000in}}%
\pgfpathquadraticcurveto{\pgfqpoint{3.197778in}{1.585000in}}{\pgfqpoint{3.197778in}{1.557222in}}%
\pgfpathlineto{\pgfqpoint{3.197778in}{1.182222in}}%
\pgfpathquadraticcurveto{\pgfqpoint{3.197778in}{1.154445in}}{\pgfqpoint{3.225556in}{1.154445in}}%
\pgfpathlineto{\pgfqpoint{3.225556in}{1.154445in}}%
\pgfpathclose%
\pgfusepath{stroke,fill}%
\end{pgfscope}%
\begin{pgfscope}%
\pgfsetbuttcap%
\pgfsetmiterjoin%
\pgfsetlinewidth{1.003750pt}%
\definecolor{currentstroke}{rgb}{0.000000,0.000000,0.000000}%
\pgfsetstrokecolor{currentstroke}%
\pgfsetdash{}{0pt}%
\pgfpathmoveto{\pgfqpoint{3.253334in}{1.432222in}}%
\pgfpathlineto{\pgfqpoint{3.531111in}{1.432222in}}%
\pgfpathlineto{\pgfqpoint{3.531111in}{1.529444in}}%
\pgfpathlineto{\pgfqpoint{3.253334in}{1.529444in}}%
\pgfpathlineto{\pgfqpoint{3.253334in}{1.432222in}}%
\pgfpathclose%
\pgfusepath{stroke}%
\end{pgfscope}%
\begin{pgfscope}%
\definecolor{textcolor}{rgb}{0.000000,0.000000,0.000000}%
\pgfsetstrokecolor{textcolor}%
\pgfsetfillcolor{textcolor}%
\pgftext[x=3.642223in,y=1.432222in,left,base]{\color{textcolor}\rmfamily\fontsize{10.000000}{12.000000}\selectfont Neg}%
\end{pgfscope}%
\begin{pgfscope}%
\pgfsetbuttcap%
\pgfsetmiterjoin%
\definecolor{currentfill}{rgb}{0.000000,0.000000,0.000000}%
\pgfsetfillcolor{currentfill}%
\pgfsetlinewidth{0.000000pt}%
\definecolor{currentstroke}{rgb}{0.000000,0.000000,0.000000}%
\pgfsetstrokecolor{currentstroke}%
\pgfsetstrokeopacity{0.000000}%
\pgfsetdash{}{0pt}%
\pgfpathmoveto{\pgfqpoint{3.253334in}{1.236944in}}%
\pgfpathlineto{\pgfqpoint{3.531111in}{1.236944in}}%
\pgfpathlineto{\pgfqpoint{3.531111in}{1.334167in}}%
\pgfpathlineto{\pgfqpoint{3.253334in}{1.334167in}}%
\pgfpathlineto{\pgfqpoint{3.253334in}{1.236944in}}%
\pgfpathclose%
\pgfusepath{fill}%
\end{pgfscope}%
\begin{pgfscope}%
\definecolor{textcolor}{rgb}{0.000000,0.000000,0.000000}%
\pgfsetstrokecolor{textcolor}%
\pgfsetfillcolor{textcolor}%
\pgftext[x=3.642223in,y=1.236944in,left,base]{\color{textcolor}\rmfamily\fontsize{10.000000}{12.000000}\selectfont Pos}%
\end{pgfscope}%
\end{pgfpicture}%
\makeatother%
\endgroup%

&
	\vskip 0pt
	\qquad \qquad FP/TP
	
	%% Creator: Matplotlib, PGF backend
%%
%% To include the figure in your LaTeX document, write
%%   \input{<filename>.pgf}
%%
%% Make sure the required packages are loaded in your preamble
%%   \usepackage{pgf}
%%
%% Also ensure that all the required font packages are loaded; for instance,
%% the lmodern package is sometimes necessary when using math font.
%%   \usepackage{lmodern}
%%
%% Figures using additional raster images can only be included by \input if
%% they are in the same directory as the main LaTeX file. For loading figures
%% from other directories you can use the `import` package
%%   \usepackage{import}
%%
%% and then include the figures with
%%   \import{<path to file>}{<filename>.pgf}
%%
%% Matplotlib used the following preamble
%%   
%%   \usepackage{fontspec}
%%   \makeatletter\@ifpackageloaded{underscore}{}{\usepackage[strings]{underscore}}\makeatother
%%
\begingroup%
\makeatletter%
\begin{pgfpicture}%
\pgfpathrectangle{\pgfpointorigin}{\pgfqpoint{2.282529in}{1.759278in}}%
\pgfusepath{use as bounding box, clip}%
\begin{pgfscope}%
\pgfsetbuttcap%
\pgfsetmiterjoin%
\definecolor{currentfill}{rgb}{1.000000,1.000000,1.000000}%
\pgfsetfillcolor{currentfill}%
\pgfsetlinewidth{0.000000pt}%
\definecolor{currentstroke}{rgb}{1.000000,1.000000,1.000000}%
\pgfsetstrokecolor{currentstroke}%
\pgfsetdash{}{0pt}%
\pgfpathmoveto{\pgfqpoint{0.000000in}{0.000000in}}%
\pgfpathlineto{\pgfqpoint{2.282529in}{0.000000in}}%
\pgfpathlineto{\pgfqpoint{2.282529in}{1.759278in}}%
\pgfpathlineto{\pgfqpoint{0.000000in}{1.759278in}}%
\pgfpathlineto{\pgfqpoint{0.000000in}{0.000000in}}%
\pgfpathclose%
\pgfusepath{fill}%
\end{pgfscope}%
\begin{pgfscope}%
\pgfsetbuttcap%
\pgfsetmiterjoin%
\definecolor{currentfill}{rgb}{1.000000,1.000000,1.000000}%
\pgfsetfillcolor{currentfill}%
\pgfsetlinewidth{0.000000pt}%
\definecolor{currentstroke}{rgb}{0.000000,0.000000,0.000000}%
\pgfsetstrokecolor{currentstroke}%
\pgfsetstrokeopacity{0.000000}%
\pgfsetdash{}{0pt}%
\pgfpathmoveto{\pgfqpoint{0.530556in}{0.499444in}}%
\pgfpathlineto{\pgfqpoint{2.080556in}{0.499444in}}%
\pgfpathlineto{\pgfqpoint{2.080556in}{1.654444in}}%
\pgfpathlineto{\pgfqpoint{0.530556in}{1.654444in}}%
\pgfpathlineto{\pgfqpoint{0.530556in}{0.499444in}}%
\pgfpathclose%
\pgfusepath{fill}%
\end{pgfscope}%
\begin{pgfscope}%
\pgfsetbuttcap%
\pgfsetroundjoin%
\definecolor{currentfill}{rgb}{0.000000,0.000000,0.000000}%
\pgfsetfillcolor{currentfill}%
\pgfsetlinewidth{0.803000pt}%
\definecolor{currentstroke}{rgb}{0.000000,0.000000,0.000000}%
\pgfsetstrokecolor{currentstroke}%
\pgfsetdash{}{0pt}%
\pgfsys@defobject{currentmarker}{\pgfqpoint{0.000000in}{-0.048611in}}{\pgfqpoint{0.000000in}{0.000000in}}{%
\pgfpathmoveto{\pgfqpoint{0.000000in}{0.000000in}}%
\pgfpathlineto{\pgfqpoint{0.000000in}{-0.048611in}}%
\pgfusepath{stroke,fill}%
}%
\begin{pgfscope}%
\pgfsys@transformshift{0.601010in}{0.499444in}%
\pgfsys@useobject{currentmarker}{}%
\end{pgfscope}%
\end{pgfscope}%
\begin{pgfscope}%
\definecolor{textcolor}{rgb}{0.000000,0.000000,0.000000}%
\pgfsetstrokecolor{textcolor}%
\pgfsetfillcolor{textcolor}%
\pgftext[x=0.601010in,y=0.402222in,,top]{\color{textcolor}\rmfamily\fontsize{10.000000}{12.000000}\selectfont 0.005}%
\end{pgfscope}%
\begin{pgfscope}%
\pgfsetbuttcap%
\pgfsetroundjoin%
\definecolor{currentfill}{rgb}{0.000000,0.000000,0.000000}%
\pgfsetfillcolor{currentfill}%
\pgfsetlinewidth{0.803000pt}%
\definecolor{currentstroke}{rgb}{0.000000,0.000000,0.000000}%
\pgfsetstrokecolor{currentstroke}%
\pgfsetdash{}{0pt}%
\pgfsys@defobject{currentmarker}{\pgfqpoint{0.000000in}{-0.048611in}}{\pgfqpoint{0.000000in}{0.000000in}}{%
\pgfpathmoveto{\pgfqpoint{0.000000in}{0.000000in}}%
\pgfpathlineto{\pgfqpoint{0.000000in}{-0.048611in}}%
\pgfusepath{stroke,fill}%
}%
\begin{pgfscope}%
\pgfsys@transformshift{2.024334in}{0.499444in}%
\pgfsys@useobject{currentmarker}{}%
\end{pgfscope}%
\end{pgfscope}%
\begin{pgfscope}%
\definecolor{textcolor}{rgb}{0.000000,0.000000,0.000000}%
\pgfsetstrokecolor{textcolor}%
\pgfsetfillcolor{textcolor}%
\pgftext[x=2.024334in,y=0.402222in,,top]{\color{textcolor}\rmfamily\fontsize{10.000000}{12.000000}\selectfont 0.943}%
\end{pgfscope}%
\begin{pgfscope}%
\definecolor{textcolor}{rgb}{0.000000,0.000000,0.000000}%
\pgfsetstrokecolor{textcolor}%
\pgfsetfillcolor{textcolor}%
\pgftext[x=1.305556in,y=0.223333in,,top]{\color{textcolor}\rmfamily\fontsize{10.000000}{12.000000}\selectfont \(\displaystyle p\)}%
\end{pgfscope}%
\begin{pgfscope}%
\pgfsetbuttcap%
\pgfsetroundjoin%
\definecolor{currentfill}{rgb}{0.000000,0.000000,0.000000}%
\pgfsetfillcolor{currentfill}%
\pgfsetlinewidth{0.803000pt}%
\definecolor{currentstroke}{rgb}{0.000000,0.000000,0.000000}%
\pgfsetstrokecolor{currentstroke}%
\pgfsetdash{}{0pt}%
\pgfsys@defobject{currentmarker}{\pgfqpoint{-0.048611in}{0.000000in}}{\pgfqpoint{-0.000000in}{0.000000in}}{%
\pgfpathmoveto{\pgfqpoint{-0.000000in}{0.000000in}}%
\pgfpathlineto{\pgfqpoint{-0.048611in}{0.000000in}}%
\pgfusepath{stroke,fill}%
}%
\begin{pgfscope}%
\pgfsys@transformshift{0.530556in}{0.542126in}%
\pgfsys@useobject{currentmarker}{}%
\end{pgfscope}%
\end{pgfscope}%
\begin{pgfscope}%
\definecolor{textcolor}{rgb}{0.000000,0.000000,0.000000}%
\pgfsetstrokecolor{textcolor}%
\pgfsetfillcolor{textcolor}%
\pgftext[x=0.363889in, y=0.493932in, left, base]{\color{textcolor}\rmfamily\fontsize{10.000000}{12.000000}\selectfont \(\displaystyle {0}\)}%
\end{pgfscope}%
\begin{pgfscope}%
\pgfsetbuttcap%
\pgfsetroundjoin%
\definecolor{currentfill}{rgb}{0.000000,0.000000,0.000000}%
\pgfsetfillcolor{currentfill}%
\pgfsetlinewidth{0.803000pt}%
\definecolor{currentstroke}{rgb}{0.000000,0.000000,0.000000}%
\pgfsetstrokecolor{currentstroke}%
\pgfsetdash{}{0pt}%
\pgfsys@defobject{currentmarker}{\pgfqpoint{-0.048611in}{0.000000in}}{\pgfqpoint{-0.000000in}{0.000000in}}{%
\pgfpathmoveto{\pgfqpoint{-0.000000in}{0.000000in}}%
\pgfpathlineto{\pgfqpoint{-0.048611in}{0.000000in}}%
\pgfusepath{stroke,fill}%
}%
\begin{pgfscope}%
\pgfsys@transformshift{0.530556in}{1.076605in}%
\pgfsys@useobject{currentmarker}{}%
\end{pgfscope}%
\end{pgfscope}%
\begin{pgfscope}%
\definecolor{textcolor}{rgb}{0.000000,0.000000,0.000000}%
\pgfsetstrokecolor{textcolor}%
\pgfsetfillcolor{textcolor}%
\pgftext[x=0.294444in, y=1.028411in, left, base]{\color{textcolor}\rmfamily\fontsize{10.000000}{12.000000}\selectfont \(\displaystyle {20}\)}%
\end{pgfscope}%
\begin{pgfscope}%
\pgfsetbuttcap%
\pgfsetroundjoin%
\definecolor{currentfill}{rgb}{0.000000,0.000000,0.000000}%
\pgfsetfillcolor{currentfill}%
\pgfsetlinewidth{0.803000pt}%
\definecolor{currentstroke}{rgb}{0.000000,0.000000,0.000000}%
\pgfsetstrokecolor{currentstroke}%
\pgfsetdash{}{0pt}%
\pgfsys@defobject{currentmarker}{\pgfqpoint{-0.048611in}{0.000000in}}{\pgfqpoint{-0.000000in}{0.000000in}}{%
\pgfpathmoveto{\pgfqpoint{-0.000000in}{0.000000in}}%
\pgfpathlineto{\pgfqpoint{-0.048611in}{0.000000in}}%
\pgfusepath{stroke,fill}%
}%
\begin{pgfscope}%
\pgfsys@transformshift{0.530556in}{1.611084in}%
\pgfsys@useobject{currentmarker}{}%
\end{pgfscope}%
\end{pgfscope}%
\begin{pgfscope}%
\definecolor{textcolor}{rgb}{0.000000,0.000000,0.000000}%
\pgfsetstrokecolor{textcolor}%
\pgfsetfillcolor{textcolor}%
\pgftext[x=0.294444in, y=1.562889in, left, base]{\color{textcolor}\rmfamily\fontsize{10.000000}{12.000000}\selectfont \(\displaystyle {40}\)}%
\end{pgfscope}%
\begin{pgfscope}%
\definecolor{textcolor}{rgb}{0.000000,0.000000,0.000000}%
\pgfsetstrokecolor{textcolor}%
\pgfsetfillcolor{textcolor}%
\pgftext[x=0.238889in,y=1.076944in,,bottom,rotate=90.000000]{\color{textcolor}\rmfamily\fontsize{10.000000}{12.000000}\selectfont \(\displaystyle \Delta\)FP/\(\displaystyle \Delta\)TP}%
\end{pgfscope}%
\begin{pgfscope}%
\pgfpathrectangle{\pgfqpoint{0.530556in}{0.499444in}}{\pgfqpoint{1.550000in}{1.155000in}}%
\pgfusepath{clip}%
\pgfsetrectcap%
\pgfsetroundjoin%
\pgfsetlinewidth{1.505625pt}%
\definecolor{currentstroke}{rgb}{0.000000,0.000000,0.000000}%
\pgfsetstrokecolor{currentstroke}%
\pgfsetdash{}{0pt}%
\pgfpathmoveto{\pgfqpoint{0.601010in}{1.601944in}}%
\pgfpathlineto{\pgfqpoint{0.615243in}{1.553622in}}%
\pgfpathlineto{\pgfqpoint{0.629477in}{1.506071in}}%
\pgfpathlineto{\pgfqpoint{0.643710in}{1.459420in}}%
\pgfpathlineto{\pgfqpoint{0.657943in}{1.414751in}}%
\pgfpathlineto{\pgfqpoint{0.672176in}{1.373683in}}%
\pgfpathlineto{\pgfqpoint{0.686410in}{1.282677in}}%
\pgfpathlineto{\pgfqpoint{0.700643in}{1.203512in}}%
\pgfpathlineto{\pgfqpoint{0.714876in}{1.135789in}}%
\pgfpathlineto{\pgfqpoint{0.729109in}{1.076953in}}%
\pgfpathlineto{\pgfqpoint{0.743343in}{1.023089in}}%
\pgfpathlineto{\pgfqpoint{0.757576in}{0.974517in}}%
\pgfpathlineto{\pgfqpoint{0.771809in}{0.933499in}}%
\pgfpathlineto{\pgfqpoint{0.786042in}{0.899946in}}%
\pgfpathlineto{\pgfqpoint{0.800276in}{0.872749in}}%
\pgfpathlineto{\pgfqpoint{0.814509in}{0.849419in}}%
\pgfpathlineto{\pgfqpoint{0.828742in}{0.829028in}}%
\pgfpathlineto{\pgfqpoint{0.842975in}{0.810628in}}%
\pgfpathlineto{\pgfqpoint{0.857209in}{0.794256in}}%
\pgfpathlineto{\pgfqpoint{0.871442in}{0.779942in}}%
\pgfpathlineto{\pgfqpoint{0.885675in}{0.766886in}}%
\pgfpathlineto{\pgfqpoint{0.899908in}{0.755071in}}%
\pgfpathlineto{\pgfqpoint{0.914142in}{0.743957in}}%
\pgfpathlineto{\pgfqpoint{0.928375in}{0.733440in}}%
\pgfpathlineto{\pgfqpoint{0.942608in}{0.723397in}}%
\pgfpathlineto{\pgfqpoint{0.956841in}{0.714417in}}%
\pgfpathlineto{\pgfqpoint{0.971075in}{0.706150in}}%
\pgfpathlineto{\pgfqpoint{0.985308in}{0.698759in}}%
\pgfpathlineto{\pgfqpoint{0.999541in}{0.692145in}}%
\pgfpathlineto{\pgfqpoint{1.013774in}{0.685573in}}%
\pgfpathlineto{\pgfqpoint{1.028007in}{0.679717in}}%
\pgfpathlineto{\pgfqpoint{1.042241in}{0.674177in}}%
\pgfpathlineto{\pgfqpoint{1.056474in}{0.669182in}}%
\pgfpathlineto{\pgfqpoint{1.070707in}{0.664531in}}%
\pgfpathlineto{\pgfqpoint{1.084940in}{0.660281in}}%
\pgfpathlineto{\pgfqpoint{1.099174in}{0.656045in}}%
\pgfpathlineto{\pgfqpoint{1.113407in}{0.651912in}}%
\pgfpathlineto{\pgfqpoint{1.127640in}{0.648150in}}%
\pgfpathlineto{\pgfqpoint{1.141873in}{0.644376in}}%
\pgfpathlineto{\pgfqpoint{1.156107in}{0.641007in}}%
\pgfpathlineto{\pgfqpoint{1.170340in}{0.637824in}}%
\pgfpathlineto{\pgfqpoint{1.184573in}{0.634830in}}%
\pgfpathlineto{\pgfqpoint{1.198806in}{0.631917in}}%
\pgfpathlineto{\pgfqpoint{1.213040in}{0.629128in}}%
\pgfpathlineto{\pgfqpoint{1.227273in}{0.626350in}}%
\pgfpathlineto{\pgfqpoint{1.241506in}{0.623701in}}%
\pgfpathlineto{\pgfqpoint{1.255739in}{0.621265in}}%
\pgfpathlineto{\pgfqpoint{1.269973in}{0.618756in}}%
\pgfpathlineto{\pgfqpoint{1.284206in}{0.616392in}}%
\pgfpathlineto{\pgfqpoint{1.298439in}{0.613935in}}%
\pgfpathlineto{\pgfqpoint{1.312672in}{0.611447in}}%
\pgfpathlineto{\pgfqpoint{1.326906in}{0.609196in}}%
\pgfpathlineto{\pgfqpoint{1.341139in}{0.606986in}}%
\pgfpathlineto{\pgfqpoint{1.355372in}{0.604812in}}%
\pgfpathlineto{\pgfqpoint{1.369605in}{0.602820in}}%
\pgfpathlineto{\pgfqpoint{1.383839in}{0.600906in}}%
\pgfpathlineto{\pgfqpoint{1.398072in}{0.598948in}}%
\pgfpathlineto{\pgfqpoint{1.412305in}{0.597101in}}%
\pgfpathlineto{\pgfqpoint{1.426538in}{0.595268in}}%
\pgfpathlineto{\pgfqpoint{1.440771in}{0.593603in}}%
\pgfpathlineto{\pgfqpoint{1.455005in}{0.591944in}}%
\pgfpathlineto{\pgfqpoint{1.469238in}{0.590089in}}%
\pgfpathlineto{\pgfqpoint{1.483471in}{0.588243in}}%
\pgfpathlineto{\pgfqpoint{1.497704in}{0.586510in}}%
\pgfpathlineto{\pgfqpoint{1.511938in}{0.584718in}}%
\pgfpathlineto{\pgfqpoint{1.526171in}{0.582989in}}%
\pgfpathlineto{\pgfqpoint{1.540404in}{0.581378in}}%
\pgfpathlineto{\pgfqpoint{1.554637in}{0.579794in}}%
\pgfpathlineto{\pgfqpoint{1.568871in}{0.578302in}}%
\pgfpathlineto{\pgfqpoint{1.583104in}{0.576812in}}%
\pgfpathlineto{\pgfqpoint{1.597337in}{0.575435in}}%
\pgfpathlineto{\pgfqpoint{1.611570in}{0.574148in}}%
\pgfpathlineto{\pgfqpoint{1.625804in}{0.572960in}}%
\pgfpathlineto{\pgfqpoint{1.640037in}{0.571823in}}%
\pgfpathlineto{\pgfqpoint{1.654270in}{0.570744in}}%
\pgfpathlineto{\pgfqpoint{1.668503in}{0.569696in}}%
\pgfpathlineto{\pgfqpoint{1.682737in}{0.568645in}}%
\pgfpathlineto{\pgfqpoint{1.696970in}{0.567658in}}%
\pgfpathlineto{\pgfqpoint{1.711203in}{0.566691in}}%
\pgfpathlineto{\pgfqpoint{1.725436in}{0.565775in}}%
\pgfpathlineto{\pgfqpoint{1.739670in}{0.564790in}}%
\pgfpathlineto{\pgfqpoint{1.753903in}{0.563869in}}%
\pgfpathlineto{\pgfqpoint{1.768136in}{0.562959in}}%
\pgfpathlineto{\pgfqpoint{1.782369in}{0.561999in}}%
\pgfpathlineto{\pgfqpoint{1.796603in}{0.561117in}}%
\pgfpathlineto{\pgfqpoint{1.810836in}{0.560275in}}%
\pgfpathlineto{\pgfqpoint{1.825069in}{0.559489in}}%
\pgfpathlineto{\pgfqpoint{1.839302in}{0.558654in}}%
\pgfpathlineto{\pgfqpoint{1.853535in}{0.557821in}}%
\pgfpathlineto{\pgfqpoint{1.867769in}{0.556960in}}%
\pgfpathlineto{\pgfqpoint{1.882002in}{0.556220in}}%
\pgfpathlineto{\pgfqpoint{1.896235in}{0.555516in}}%
\pgfpathlineto{\pgfqpoint{1.910468in}{0.554853in}}%
\pgfpathlineto{\pgfqpoint{1.924702in}{0.554266in}}%
\pgfpathlineto{\pgfqpoint{1.938935in}{0.553682in}}%
\pgfpathlineto{\pgfqpoint{1.953168in}{0.553061in}}%
\pgfpathlineto{\pgfqpoint{1.967401in}{0.552710in}}%
\pgfpathlineto{\pgfqpoint{1.981635in}{0.552414in}}%
\pgfpathlineto{\pgfqpoint{1.995868in}{0.552144in}}%
\pgfpathlineto{\pgfqpoint{2.010101in}{0.551944in}}%
\pgfusepath{stroke}%
\end{pgfscope}%
\begin{pgfscope}%
\pgfpathrectangle{\pgfqpoint{0.530556in}{0.499444in}}{\pgfqpoint{1.550000in}{1.155000in}}%
\pgfusepath{clip}%
\pgfsetbuttcap%
\pgfsetroundjoin%
\pgfsetlinewidth{1.505625pt}%
\definecolor{currentstroke}{rgb}{0.000000,0.000000,0.000000}%
\pgfsetstrokecolor{currentstroke}%
\pgfsetdash{{5.550000pt}{2.400000pt}}{0.000000pt}%
\pgfpathmoveto{\pgfqpoint{0.530556in}{0.595574in}}%
\pgfpathlineto{\pgfqpoint{2.080556in}{0.595574in}}%
\pgfusepath{stroke}%
\end{pgfscope}%
\begin{pgfscope}%
\pgfpathrectangle{\pgfqpoint{0.530556in}{0.499444in}}{\pgfqpoint{1.550000in}{1.155000in}}%
\pgfusepath{clip}%
\pgfsetrectcap%
\pgfsetroundjoin%
\pgfsetlinewidth{1.505625pt}%
\definecolor{currentstroke}{rgb}{0.121569,0.466667,0.705882}%
\pgfsetstrokecolor{currentstroke}%
\pgfsetdash{}{0pt}%
\pgfpathmoveto{\pgfqpoint{1.426538in}{0.595574in}}%
\pgfusepath{stroke}%
\end{pgfscope}%
\begin{pgfscope}%
\pgfpathrectangle{\pgfqpoint{0.530556in}{0.499444in}}{\pgfqpoint{1.550000in}{1.155000in}}%
\pgfusepath{clip}%
\pgfsetbuttcap%
\pgfsetroundjoin%
\definecolor{currentfill}{rgb}{0.000000,0.000000,0.000000}%
\pgfsetfillcolor{currentfill}%
\pgfsetlinewidth{1.003750pt}%
\definecolor{currentstroke}{rgb}{0.000000,0.000000,0.000000}%
\pgfsetstrokecolor{currentstroke}%
\pgfsetdash{}{0pt}%
\pgfsys@defobject{currentmarker}{\pgfqpoint{-0.041667in}{-0.041667in}}{\pgfqpoint{0.041667in}{0.041667in}}{%
\pgfpathmoveto{\pgfqpoint{0.000000in}{-0.041667in}}%
\pgfpathcurveto{\pgfqpoint{0.011050in}{-0.041667in}}{\pgfqpoint{0.021649in}{-0.037276in}}{\pgfqpoint{0.029463in}{-0.029463in}}%
\pgfpathcurveto{\pgfqpoint{0.037276in}{-0.021649in}}{\pgfqpoint{0.041667in}{-0.011050in}}{\pgfqpoint{0.041667in}{0.000000in}}%
\pgfpathcurveto{\pgfqpoint{0.041667in}{0.011050in}}{\pgfqpoint{0.037276in}{0.021649in}}{\pgfqpoint{0.029463in}{0.029463in}}%
\pgfpathcurveto{\pgfqpoint{0.021649in}{0.037276in}}{\pgfqpoint{0.011050in}{0.041667in}}{\pgfqpoint{0.000000in}{0.041667in}}%
\pgfpathcurveto{\pgfqpoint{-0.011050in}{0.041667in}}{\pgfqpoint{-0.021649in}{0.037276in}}{\pgfqpoint{-0.029463in}{0.029463in}}%
\pgfpathcurveto{\pgfqpoint{-0.037276in}{0.021649in}}{\pgfqpoint{-0.041667in}{0.011050in}}{\pgfqpoint{-0.041667in}{0.000000in}}%
\pgfpathcurveto{\pgfqpoint{-0.041667in}{-0.011050in}}{\pgfqpoint{-0.037276in}{-0.021649in}}{\pgfqpoint{-0.029463in}{-0.029463in}}%
\pgfpathcurveto{\pgfqpoint{-0.021649in}{-0.037276in}}{\pgfqpoint{-0.011050in}{-0.041667in}}{\pgfqpoint{0.000000in}{-0.041667in}}%
\pgfpathlineto{\pgfqpoint{0.000000in}{-0.041667in}}%
\pgfpathclose%
\pgfusepath{stroke,fill}%
}%
\begin{pgfscope}%
\pgfsys@transformshift{1.426538in}{0.595574in}%
\pgfsys@useobject{currentmarker}{}%
\end{pgfscope}%
\end{pgfscope}%
\begin{pgfscope}%
\pgfsetrectcap%
\pgfsetmiterjoin%
\pgfsetlinewidth{0.803000pt}%
\definecolor{currentstroke}{rgb}{0.000000,0.000000,0.000000}%
\pgfsetstrokecolor{currentstroke}%
\pgfsetdash{}{0pt}%
\pgfpathmoveto{\pgfqpoint{0.530556in}{0.499444in}}%
\pgfpathlineto{\pgfqpoint{0.530556in}{1.654444in}}%
\pgfusepath{stroke}%
\end{pgfscope}%
\begin{pgfscope}%
\pgfsetrectcap%
\pgfsetmiterjoin%
\pgfsetlinewidth{0.803000pt}%
\definecolor{currentstroke}{rgb}{0.000000,0.000000,0.000000}%
\pgfsetstrokecolor{currentstroke}%
\pgfsetdash{}{0pt}%
\pgfpathmoveto{\pgfqpoint{2.080556in}{0.499444in}}%
\pgfpathlineto{\pgfqpoint{2.080556in}{1.654444in}}%
\pgfusepath{stroke}%
\end{pgfscope}%
\begin{pgfscope}%
\pgfsetrectcap%
\pgfsetmiterjoin%
\pgfsetlinewidth{0.803000pt}%
\definecolor{currentstroke}{rgb}{0.000000,0.000000,0.000000}%
\pgfsetstrokecolor{currentstroke}%
\pgfsetdash{}{0pt}%
\pgfpathmoveto{\pgfqpoint{0.530556in}{0.499444in}}%
\pgfpathlineto{\pgfqpoint{2.080556in}{0.499444in}}%
\pgfusepath{stroke}%
\end{pgfscope}%
\begin{pgfscope}%
\pgfsetrectcap%
\pgfsetmiterjoin%
\pgfsetlinewidth{0.803000pt}%
\definecolor{currentstroke}{rgb}{0.000000,0.000000,0.000000}%
\pgfsetstrokecolor{currentstroke}%
\pgfsetdash{}{0pt}%
\pgfpathmoveto{\pgfqpoint{0.530556in}{1.654444in}}%
\pgfpathlineto{\pgfqpoint{2.080556in}{1.654444in}}%
\pgfusepath{stroke}%
\end{pgfscope}%
\begin{pgfscope}%
\pgfsetbuttcap%
\pgfsetmiterjoin%
\definecolor{currentfill}{rgb}{1.000000,1.000000,1.000000}%
\pgfsetfillcolor{currentfill}%
\pgfsetfillopacity{0.800000}%
\pgfsetlinewidth{1.003750pt}%
\definecolor{currentstroke}{rgb}{0.800000,0.800000,0.800000}%
\pgfsetstrokecolor{currentstroke}%
\pgfsetstrokeopacity{0.800000}%
\pgfsetdash{}{0pt}%
\pgfpathmoveto{\pgfqpoint{0.811987in}{1.126667in}}%
\pgfpathlineto{\pgfqpoint{1.983333in}{1.126667in}}%
\pgfpathquadraticcurveto{\pgfqpoint{2.011111in}{1.126667in}}{\pgfqpoint{2.011111in}{1.154444in}}%
\pgfpathlineto{\pgfqpoint{2.011111in}{1.557222in}}%
\pgfpathquadraticcurveto{\pgfqpoint{2.011111in}{1.585000in}}{\pgfqpoint{1.983333in}{1.585000in}}%
\pgfpathlineto{\pgfqpoint{0.811987in}{1.585000in}}%
\pgfpathquadraticcurveto{\pgfqpoint{0.784210in}{1.585000in}}{\pgfqpoint{0.784210in}{1.557222in}}%
\pgfpathlineto{\pgfqpoint{0.784210in}{1.154444in}}%
\pgfpathquadraticcurveto{\pgfqpoint{0.784210in}{1.126667in}}{\pgfqpoint{0.811987in}{1.126667in}}%
\pgfpathlineto{\pgfqpoint{0.811987in}{1.126667in}}%
\pgfpathclose%
\pgfusepath{stroke,fill}%
\end{pgfscope}%
\begin{pgfscope}%
\pgfsetrectcap%
\pgfsetroundjoin%
\pgfsetlinewidth{1.505625pt}%
\definecolor{currentstroke}{rgb}{0.000000,0.000000,0.000000}%
\pgfsetstrokecolor{currentstroke}%
\pgfsetdash{}{0pt}%
\pgfpathmoveto{\pgfqpoint{0.839765in}{1.473889in}}%
\pgfpathlineto{\pgfqpoint{0.978654in}{1.473889in}}%
\pgfpathlineto{\pgfqpoint{1.117543in}{1.473889in}}%
\pgfusepath{stroke}%
\end{pgfscope}%
\begin{pgfscope}%
\definecolor{textcolor}{rgb}{0.000000,0.000000,0.000000}%
\pgfsetstrokecolor{textcolor}%
\pgfsetfillcolor{textcolor}%
\pgftext[x=1.228654in,y=1.425277in,left,base]{\color{textcolor}\rmfamily\fontsize{10.000000}{12.000000}\selectfont \(\displaystyle \Delta FP/\Delta TP\)}%
\end{pgfscope}%
\begin{pgfscope}%
\pgfsetrectcap%
\pgfsetroundjoin%
\pgfsetlinewidth{1.505625pt}%
\definecolor{currentstroke}{rgb}{0.121569,0.466667,0.705882}%
\pgfsetstrokecolor{currentstroke}%
\pgfsetdash{}{0pt}%
\pgfpathmoveto{\pgfqpoint{0.839765in}{1.265555in}}%
\pgfpathlineto{\pgfqpoint{0.978654in}{1.265555in}}%
\pgfpathlineto{\pgfqpoint{1.117543in}{1.265555in}}%
\pgfusepath{stroke}%
\end{pgfscope}%
\begin{pgfscope}%
\pgfsetbuttcap%
\pgfsetroundjoin%
\definecolor{currentfill}{rgb}{0.000000,0.000000,0.000000}%
\pgfsetfillcolor{currentfill}%
\pgfsetlinewidth{1.003750pt}%
\definecolor{currentstroke}{rgb}{0.000000,0.000000,0.000000}%
\pgfsetstrokecolor{currentstroke}%
\pgfsetdash{}{0pt}%
\pgfsys@defobject{currentmarker}{\pgfqpoint{-0.041667in}{-0.041667in}}{\pgfqpoint{0.041667in}{0.041667in}}{%
\pgfpathmoveto{\pgfqpoint{0.000000in}{-0.041667in}}%
\pgfpathcurveto{\pgfqpoint{0.011050in}{-0.041667in}}{\pgfqpoint{0.021649in}{-0.037276in}}{\pgfqpoint{0.029463in}{-0.029463in}}%
\pgfpathcurveto{\pgfqpoint{0.037276in}{-0.021649in}}{\pgfqpoint{0.041667in}{-0.011050in}}{\pgfqpoint{0.041667in}{0.000000in}}%
\pgfpathcurveto{\pgfqpoint{0.041667in}{0.011050in}}{\pgfqpoint{0.037276in}{0.021649in}}{\pgfqpoint{0.029463in}{0.029463in}}%
\pgfpathcurveto{\pgfqpoint{0.021649in}{0.037276in}}{\pgfqpoint{0.011050in}{0.041667in}}{\pgfqpoint{0.000000in}{0.041667in}}%
\pgfpathcurveto{\pgfqpoint{-0.011050in}{0.041667in}}{\pgfqpoint{-0.021649in}{0.037276in}}{\pgfqpoint{-0.029463in}{0.029463in}}%
\pgfpathcurveto{\pgfqpoint{-0.037276in}{0.021649in}}{\pgfqpoint{-0.041667in}{0.011050in}}{\pgfqpoint{-0.041667in}{0.000000in}}%
\pgfpathcurveto{\pgfqpoint{-0.041667in}{-0.011050in}}{\pgfqpoint{-0.037276in}{-0.021649in}}{\pgfqpoint{-0.029463in}{-0.029463in}}%
\pgfpathcurveto{\pgfqpoint{-0.021649in}{-0.037276in}}{\pgfqpoint{-0.011050in}{-0.041667in}}{\pgfqpoint{0.000000in}{-0.041667in}}%
\pgfpathlineto{\pgfqpoint{0.000000in}{-0.041667in}}%
\pgfpathclose%
\pgfusepath{stroke,fill}%
}%
\begin{pgfscope}%
\pgfsys@transformshift{0.978654in}{1.265555in}%
\pgfsys@useobject{currentmarker}{}%
\end{pgfscope}%
\end{pgfscope}%
\begin{pgfscope}%
\definecolor{textcolor}{rgb}{0.000000,0.000000,0.000000}%
\pgfsetstrokecolor{textcolor}%
\pgfsetfillcolor{textcolor}%
\pgftext[x=1.228654in,y=1.216944in,left,base]{\color{textcolor}\rmfamily\fontsize{10.000000}{12.000000}\selectfont (0.549,2)}%
\end{pgfscope}%
\end{pgfpicture}%
\makeatother%
\endgroup%

\end{tabular}


\noindent\begin{tabular}{@{\hspace{-6pt}}p{4.5in} @{\hspace{6pt}}p{2.0in}}
	\vskip 0pt
	\qquad \qquad Transformed Model Output:  Map $0.549$ to 0.5 and 0 to 0.
	
	%% Creator: Matplotlib, PGF backend
%%
%% To include the figure in your LaTeX document, write
%%   \input{<filename>.pgf}
%%
%% Make sure the required packages are loaded in your preamble
%%   \usepackage{pgf}
%%
%% Also ensure that all the required font packages are loaded; for instance,
%% the lmodern package is sometimes necessary when using math font.
%%   \usepackage{lmodern}
%%
%% Figures using additional raster images can only be included by \input if
%% they are in the same directory as the main LaTeX file. For loading figures
%% from other directories you can use the `import` package
%%   \usepackage{import}
%%
%% and then include the figures with
%%   \import{<path to file>}{<filename>.pgf}
%%
%% Matplotlib used the following preamble
%%   
%%   \usepackage{fontspec}
%%   \makeatletter\@ifpackageloaded{underscore}{}{\usepackage[strings]{underscore}}\makeatother
%%
\begingroup%
\makeatletter%
\begin{pgfpicture}%
\pgfpathrectangle{\pgfpointorigin}{\pgfqpoint{4.578750in}{1.754444in}}%
\pgfusepath{use as bounding box, clip}%
\begin{pgfscope}%
\pgfsetbuttcap%
\pgfsetmiterjoin%
\definecolor{currentfill}{rgb}{1.000000,1.000000,1.000000}%
\pgfsetfillcolor{currentfill}%
\pgfsetlinewidth{0.000000pt}%
\definecolor{currentstroke}{rgb}{1.000000,1.000000,1.000000}%
\pgfsetstrokecolor{currentstroke}%
\pgfsetdash{}{0pt}%
\pgfpathmoveto{\pgfqpoint{0.000000in}{0.000000in}}%
\pgfpathlineto{\pgfqpoint{4.578750in}{0.000000in}}%
\pgfpathlineto{\pgfqpoint{4.578750in}{1.754444in}}%
\pgfpathlineto{\pgfqpoint{0.000000in}{1.754444in}}%
\pgfpathlineto{\pgfqpoint{0.000000in}{0.000000in}}%
\pgfpathclose%
\pgfusepath{fill}%
\end{pgfscope}%
\begin{pgfscope}%
\pgfsetbuttcap%
\pgfsetmiterjoin%
\definecolor{currentfill}{rgb}{1.000000,1.000000,1.000000}%
\pgfsetfillcolor{currentfill}%
\pgfsetlinewidth{0.000000pt}%
\definecolor{currentstroke}{rgb}{0.000000,0.000000,0.000000}%
\pgfsetstrokecolor{currentstroke}%
\pgfsetstrokeopacity{0.000000}%
\pgfsetdash{}{0pt}%
\pgfpathmoveto{\pgfqpoint{0.515000in}{0.499444in}}%
\pgfpathlineto{\pgfqpoint{4.390000in}{0.499444in}}%
\pgfpathlineto{\pgfqpoint{4.390000in}{1.654444in}}%
\pgfpathlineto{\pgfqpoint{0.515000in}{1.654444in}}%
\pgfpathlineto{\pgfqpoint{0.515000in}{0.499444in}}%
\pgfpathclose%
\pgfusepath{fill}%
\end{pgfscope}%
\begin{pgfscope}%
\pgfpathrectangle{\pgfqpoint{0.515000in}{0.499444in}}{\pgfqpoint{3.875000in}{1.155000in}}%
\pgfusepath{clip}%
\pgfsetbuttcap%
\pgfsetmiterjoin%
\pgfsetlinewidth{1.003750pt}%
\definecolor{currentstroke}{rgb}{0.000000,0.000000,0.000000}%
\pgfsetstrokecolor{currentstroke}%
\pgfsetdash{}{0pt}%
\pgfpathmoveto{\pgfqpoint{0.505000in}{0.499444in}}%
\pgfpathlineto{\pgfqpoint{0.553367in}{0.499444in}}%
\pgfpathlineto{\pgfqpoint{0.553367in}{1.563807in}}%
\pgfpathlineto{\pgfqpoint{0.505000in}{1.563807in}}%
\pgfusepath{stroke}%
\end{pgfscope}%
\begin{pgfscope}%
\pgfpathrectangle{\pgfqpoint{0.515000in}{0.499444in}}{\pgfqpoint{3.875000in}{1.155000in}}%
\pgfusepath{clip}%
\pgfsetbuttcap%
\pgfsetmiterjoin%
\pgfsetlinewidth{1.003750pt}%
\definecolor{currentstroke}{rgb}{0.000000,0.000000,0.000000}%
\pgfsetstrokecolor{currentstroke}%
\pgfsetdash{}{0pt}%
\pgfpathmoveto{\pgfqpoint{0.645446in}{0.499444in}}%
\pgfpathlineto{\pgfqpoint{0.706832in}{0.499444in}}%
\pgfpathlineto{\pgfqpoint{0.706832in}{1.599444in}}%
\pgfpathlineto{\pgfqpoint{0.645446in}{1.599444in}}%
\pgfpathlineto{\pgfqpoint{0.645446in}{0.499444in}}%
\pgfpathclose%
\pgfusepath{stroke}%
\end{pgfscope}%
\begin{pgfscope}%
\pgfpathrectangle{\pgfqpoint{0.515000in}{0.499444in}}{\pgfqpoint{3.875000in}{1.155000in}}%
\pgfusepath{clip}%
\pgfsetbuttcap%
\pgfsetmiterjoin%
\pgfsetlinewidth{1.003750pt}%
\definecolor{currentstroke}{rgb}{0.000000,0.000000,0.000000}%
\pgfsetstrokecolor{currentstroke}%
\pgfsetdash{}{0pt}%
\pgfpathmoveto{\pgfqpoint{0.798911in}{0.499444in}}%
\pgfpathlineto{\pgfqpoint{0.860297in}{0.499444in}}%
\pgfpathlineto{\pgfqpoint{0.860297in}{1.437679in}}%
\pgfpathlineto{\pgfqpoint{0.798911in}{1.437679in}}%
\pgfpathlineto{\pgfqpoint{0.798911in}{0.499444in}}%
\pgfpathclose%
\pgfusepath{stroke}%
\end{pgfscope}%
\begin{pgfscope}%
\pgfpathrectangle{\pgfqpoint{0.515000in}{0.499444in}}{\pgfqpoint{3.875000in}{1.155000in}}%
\pgfusepath{clip}%
\pgfsetbuttcap%
\pgfsetmiterjoin%
\pgfsetlinewidth{1.003750pt}%
\definecolor{currentstroke}{rgb}{0.000000,0.000000,0.000000}%
\pgfsetstrokecolor{currentstroke}%
\pgfsetdash{}{0pt}%
\pgfpathmoveto{\pgfqpoint{0.952377in}{0.499444in}}%
\pgfpathlineto{\pgfqpoint{1.013763in}{0.499444in}}%
\pgfpathlineto{\pgfqpoint{1.013763in}{1.268414in}}%
\pgfpathlineto{\pgfqpoint{0.952377in}{1.268414in}}%
\pgfpathlineto{\pgfqpoint{0.952377in}{0.499444in}}%
\pgfpathclose%
\pgfusepath{stroke}%
\end{pgfscope}%
\begin{pgfscope}%
\pgfpathrectangle{\pgfqpoint{0.515000in}{0.499444in}}{\pgfqpoint{3.875000in}{1.155000in}}%
\pgfusepath{clip}%
\pgfsetbuttcap%
\pgfsetmiterjoin%
\pgfsetlinewidth{1.003750pt}%
\definecolor{currentstroke}{rgb}{0.000000,0.000000,0.000000}%
\pgfsetstrokecolor{currentstroke}%
\pgfsetdash{}{0pt}%
\pgfpathmoveto{\pgfqpoint{1.105842in}{0.499444in}}%
\pgfpathlineto{\pgfqpoint{1.167228in}{0.499444in}}%
\pgfpathlineto{\pgfqpoint{1.167228in}{1.135151in}}%
\pgfpathlineto{\pgfqpoint{1.105842in}{1.135151in}}%
\pgfpathlineto{\pgfqpoint{1.105842in}{0.499444in}}%
\pgfpathclose%
\pgfusepath{stroke}%
\end{pgfscope}%
\begin{pgfscope}%
\pgfpathrectangle{\pgfqpoint{0.515000in}{0.499444in}}{\pgfqpoint{3.875000in}{1.155000in}}%
\pgfusepath{clip}%
\pgfsetbuttcap%
\pgfsetmiterjoin%
\pgfsetlinewidth{1.003750pt}%
\definecolor{currentstroke}{rgb}{0.000000,0.000000,0.000000}%
\pgfsetstrokecolor{currentstroke}%
\pgfsetdash{}{0pt}%
\pgfpathmoveto{\pgfqpoint{1.259307in}{0.499444in}}%
\pgfpathlineto{\pgfqpoint{1.320693in}{0.499444in}}%
\pgfpathlineto{\pgfqpoint{1.320693in}{1.020983in}}%
\pgfpathlineto{\pgfqpoint{1.259307in}{1.020983in}}%
\pgfpathlineto{\pgfqpoint{1.259307in}{0.499444in}}%
\pgfpathclose%
\pgfusepath{stroke}%
\end{pgfscope}%
\begin{pgfscope}%
\pgfpathrectangle{\pgfqpoint{0.515000in}{0.499444in}}{\pgfqpoint{3.875000in}{1.155000in}}%
\pgfusepath{clip}%
\pgfsetbuttcap%
\pgfsetmiterjoin%
\pgfsetlinewidth{1.003750pt}%
\definecolor{currentstroke}{rgb}{0.000000,0.000000,0.000000}%
\pgfsetstrokecolor{currentstroke}%
\pgfsetdash{}{0pt}%
\pgfpathmoveto{\pgfqpoint{1.412773in}{0.499444in}}%
\pgfpathlineto{\pgfqpoint{1.474159in}{0.499444in}}%
\pgfpathlineto{\pgfqpoint{1.474159in}{0.928871in}}%
\pgfpathlineto{\pgfqpoint{1.412773in}{0.928871in}}%
\pgfpathlineto{\pgfqpoint{1.412773in}{0.499444in}}%
\pgfpathclose%
\pgfusepath{stroke}%
\end{pgfscope}%
\begin{pgfscope}%
\pgfpathrectangle{\pgfqpoint{0.515000in}{0.499444in}}{\pgfqpoint{3.875000in}{1.155000in}}%
\pgfusepath{clip}%
\pgfsetbuttcap%
\pgfsetmiterjoin%
\pgfsetlinewidth{1.003750pt}%
\definecolor{currentstroke}{rgb}{0.000000,0.000000,0.000000}%
\pgfsetstrokecolor{currentstroke}%
\pgfsetdash{}{0pt}%
\pgfpathmoveto{\pgfqpoint{1.566238in}{0.499444in}}%
\pgfpathlineto{\pgfqpoint{1.627624in}{0.499444in}}%
\pgfpathlineto{\pgfqpoint{1.627624in}{0.851110in}}%
\pgfpathlineto{\pgfqpoint{1.566238in}{0.851110in}}%
\pgfpathlineto{\pgfqpoint{1.566238in}{0.499444in}}%
\pgfpathclose%
\pgfusepath{stroke}%
\end{pgfscope}%
\begin{pgfscope}%
\pgfpathrectangle{\pgfqpoint{0.515000in}{0.499444in}}{\pgfqpoint{3.875000in}{1.155000in}}%
\pgfusepath{clip}%
\pgfsetbuttcap%
\pgfsetmiterjoin%
\pgfsetlinewidth{1.003750pt}%
\definecolor{currentstroke}{rgb}{0.000000,0.000000,0.000000}%
\pgfsetstrokecolor{currentstroke}%
\pgfsetdash{}{0pt}%
\pgfpathmoveto{\pgfqpoint{1.719703in}{0.499444in}}%
\pgfpathlineto{\pgfqpoint{1.781089in}{0.499444in}}%
\pgfpathlineto{\pgfqpoint{1.781089in}{0.793945in}}%
\pgfpathlineto{\pgfqpoint{1.719703in}{0.793945in}}%
\pgfpathlineto{\pgfqpoint{1.719703in}{0.499444in}}%
\pgfpathclose%
\pgfusepath{stroke}%
\end{pgfscope}%
\begin{pgfscope}%
\pgfpathrectangle{\pgfqpoint{0.515000in}{0.499444in}}{\pgfqpoint{3.875000in}{1.155000in}}%
\pgfusepath{clip}%
\pgfsetbuttcap%
\pgfsetmiterjoin%
\pgfsetlinewidth{1.003750pt}%
\definecolor{currentstroke}{rgb}{0.000000,0.000000,0.000000}%
\pgfsetstrokecolor{currentstroke}%
\pgfsetdash{}{0pt}%
\pgfpathmoveto{\pgfqpoint{1.873169in}{0.499444in}}%
\pgfpathlineto{\pgfqpoint{1.934555in}{0.499444in}}%
\pgfpathlineto{\pgfqpoint{1.934555in}{0.750240in}}%
\pgfpathlineto{\pgfqpoint{1.873169in}{0.750240in}}%
\pgfpathlineto{\pgfqpoint{1.873169in}{0.499444in}}%
\pgfpathclose%
\pgfusepath{stroke}%
\end{pgfscope}%
\begin{pgfscope}%
\pgfpathrectangle{\pgfqpoint{0.515000in}{0.499444in}}{\pgfqpoint{3.875000in}{1.155000in}}%
\pgfusepath{clip}%
\pgfsetbuttcap%
\pgfsetmiterjoin%
\pgfsetlinewidth{1.003750pt}%
\definecolor{currentstroke}{rgb}{0.000000,0.000000,0.000000}%
\pgfsetstrokecolor{currentstroke}%
\pgfsetdash{}{0pt}%
\pgfpathmoveto{\pgfqpoint{2.026634in}{0.499444in}}%
\pgfpathlineto{\pgfqpoint{2.088020in}{0.499444in}}%
\pgfpathlineto{\pgfqpoint{2.088020in}{0.707590in}}%
\pgfpathlineto{\pgfqpoint{2.026634in}{0.707590in}}%
\pgfpathlineto{\pgfqpoint{2.026634in}{0.499444in}}%
\pgfpathclose%
\pgfusepath{stroke}%
\end{pgfscope}%
\begin{pgfscope}%
\pgfpathrectangle{\pgfqpoint{0.515000in}{0.499444in}}{\pgfqpoint{3.875000in}{1.155000in}}%
\pgfusepath{clip}%
\pgfsetbuttcap%
\pgfsetmiterjoin%
\pgfsetlinewidth{1.003750pt}%
\definecolor{currentstroke}{rgb}{0.000000,0.000000,0.000000}%
\pgfsetstrokecolor{currentstroke}%
\pgfsetdash{}{0pt}%
\pgfpathmoveto{\pgfqpoint{2.180099in}{0.499444in}}%
\pgfpathlineto{\pgfqpoint{2.241485in}{0.499444in}}%
\pgfpathlineto{\pgfqpoint{2.241485in}{0.667290in}}%
\pgfpathlineto{\pgfqpoint{2.180099in}{0.667290in}}%
\pgfpathlineto{\pgfqpoint{2.180099in}{0.499444in}}%
\pgfpathclose%
\pgfusepath{stroke}%
\end{pgfscope}%
\begin{pgfscope}%
\pgfpathrectangle{\pgfqpoint{0.515000in}{0.499444in}}{\pgfqpoint{3.875000in}{1.155000in}}%
\pgfusepath{clip}%
\pgfsetbuttcap%
\pgfsetmiterjoin%
\pgfsetlinewidth{1.003750pt}%
\definecolor{currentstroke}{rgb}{0.000000,0.000000,0.000000}%
\pgfsetstrokecolor{currentstroke}%
\pgfsetdash{}{0pt}%
\pgfpathmoveto{\pgfqpoint{2.333565in}{0.499444in}}%
\pgfpathlineto{\pgfqpoint{2.394951in}{0.499444in}}%
\pgfpathlineto{\pgfqpoint{2.394951in}{0.634329in}}%
\pgfpathlineto{\pgfqpoint{2.333565in}{0.634329in}}%
\pgfpathlineto{\pgfqpoint{2.333565in}{0.499444in}}%
\pgfpathclose%
\pgfusepath{stroke}%
\end{pgfscope}%
\begin{pgfscope}%
\pgfpathrectangle{\pgfqpoint{0.515000in}{0.499444in}}{\pgfqpoint{3.875000in}{1.155000in}}%
\pgfusepath{clip}%
\pgfsetbuttcap%
\pgfsetmiterjoin%
\pgfsetlinewidth{1.003750pt}%
\definecolor{currentstroke}{rgb}{0.000000,0.000000,0.000000}%
\pgfsetstrokecolor{currentstroke}%
\pgfsetdash{}{0pt}%
\pgfpathmoveto{\pgfqpoint{2.487030in}{0.499444in}}%
\pgfpathlineto{\pgfqpoint{2.548416in}{0.499444in}}%
\pgfpathlineto{\pgfqpoint{2.548416in}{0.606030in}}%
\pgfpathlineto{\pgfqpoint{2.487030in}{0.606030in}}%
\pgfpathlineto{\pgfqpoint{2.487030in}{0.499444in}}%
\pgfpathclose%
\pgfusepath{stroke}%
\end{pgfscope}%
\begin{pgfscope}%
\pgfpathrectangle{\pgfqpoint{0.515000in}{0.499444in}}{\pgfqpoint{3.875000in}{1.155000in}}%
\pgfusepath{clip}%
\pgfsetbuttcap%
\pgfsetmiterjoin%
\pgfsetlinewidth{1.003750pt}%
\definecolor{currentstroke}{rgb}{0.000000,0.000000,0.000000}%
\pgfsetstrokecolor{currentstroke}%
\pgfsetdash{}{0pt}%
\pgfpathmoveto{\pgfqpoint{2.640495in}{0.499444in}}%
\pgfpathlineto{\pgfqpoint{2.701881in}{0.499444in}}%
\pgfpathlineto{\pgfqpoint{2.701881in}{0.581381in}}%
\pgfpathlineto{\pgfqpoint{2.640495in}{0.581381in}}%
\pgfpathlineto{\pgfqpoint{2.640495in}{0.499444in}}%
\pgfpathclose%
\pgfusepath{stroke}%
\end{pgfscope}%
\begin{pgfscope}%
\pgfpathrectangle{\pgfqpoint{0.515000in}{0.499444in}}{\pgfqpoint{3.875000in}{1.155000in}}%
\pgfusepath{clip}%
\pgfsetbuttcap%
\pgfsetmiterjoin%
\pgfsetlinewidth{1.003750pt}%
\definecolor{currentstroke}{rgb}{0.000000,0.000000,0.000000}%
\pgfsetstrokecolor{currentstroke}%
\pgfsetdash{}{0pt}%
\pgfpathmoveto{\pgfqpoint{2.793961in}{0.499444in}}%
\pgfpathlineto{\pgfqpoint{2.855347in}{0.499444in}}%
\pgfpathlineto{\pgfqpoint{2.855347in}{0.561434in}}%
\pgfpathlineto{\pgfqpoint{2.793961in}{0.561434in}}%
\pgfpathlineto{\pgfqpoint{2.793961in}{0.499444in}}%
\pgfpathclose%
\pgfusepath{stroke}%
\end{pgfscope}%
\begin{pgfscope}%
\pgfpathrectangle{\pgfqpoint{0.515000in}{0.499444in}}{\pgfqpoint{3.875000in}{1.155000in}}%
\pgfusepath{clip}%
\pgfsetbuttcap%
\pgfsetmiterjoin%
\pgfsetlinewidth{1.003750pt}%
\definecolor{currentstroke}{rgb}{0.000000,0.000000,0.000000}%
\pgfsetstrokecolor{currentstroke}%
\pgfsetdash{}{0pt}%
\pgfpathmoveto{\pgfqpoint{2.947426in}{0.499444in}}%
\pgfpathlineto{\pgfqpoint{3.008812in}{0.499444in}}%
\pgfpathlineto{\pgfqpoint{3.008812in}{0.547244in}}%
\pgfpathlineto{\pgfqpoint{2.947426in}{0.547244in}}%
\pgfpathlineto{\pgfqpoint{2.947426in}{0.499444in}}%
\pgfpathclose%
\pgfusepath{stroke}%
\end{pgfscope}%
\begin{pgfscope}%
\pgfpathrectangle{\pgfqpoint{0.515000in}{0.499444in}}{\pgfqpoint{3.875000in}{1.155000in}}%
\pgfusepath{clip}%
\pgfsetbuttcap%
\pgfsetmiterjoin%
\pgfsetlinewidth{1.003750pt}%
\definecolor{currentstroke}{rgb}{0.000000,0.000000,0.000000}%
\pgfsetstrokecolor{currentstroke}%
\pgfsetdash{}{0pt}%
\pgfpathmoveto{\pgfqpoint{3.100891in}{0.499444in}}%
\pgfpathlineto{\pgfqpoint{3.162278in}{0.499444in}}%
\pgfpathlineto{\pgfqpoint{3.162278in}{0.533419in}}%
\pgfpathlineto{\pgfqpoint{3.100891in}{0.533419in}}%
\pgfpathlineto{\pgfqpoint{3.100891in}{0.499444in}}%
\pgfpathclose%
\pgfusepath{stroke}%
\end{pgfscope}%
\begin{pgfscope}%
\pgfpathrectangle{\pgfqpoint{0.515000in}{0.499444in}}{\pgfqpoint{3.875000in}{1.155000in}}%
\pgfusepath{clip}%
\pgfsetbuttcap%
\pgfsetmiterjoin%
\pgfsetlinewidth{1.003750pt}%
\definecolor{currentstroke}{rgb}{0.000000,0.000000,0.000000}%
\pgfsetstrokecolor{currentstroke}%
\pgfsetdash{}{0pt}%
\pgfpathmoveto{\pgfqpoint{3.254357in}{0.499444in}}%
\pgfpathlineto{\pgfqpoint{3.315743in}{0.499444in}}%
\pgfpathlineto{\pgfqpoint{3.315743in}{0.525229in}}%
\pgfpathlineto{\pgfqpoint{3.254357in}{0.525229in}}%
\pgfpathlineto{\pgfqpoint{3.254357in}{0.499444in}}%
\pgfpathclose%
\pgfusepath{stroke}%
\end{pgfscope}%
\begin{pgfscope}%
\pgfpathrectangle{\pgfqpoint{0.515000in}{0.499444in}}{\pgfqpoint{3.875000in}{1.155000in}}%
\pgfusepath{clip}%
\pgfsetbuttcap%
\pgfsetmiterjoin%
\pgfsetlinewidth{1.003750pt}%
\definecolor{currentstroke}{rgb}{0.000000,0.000000,0.000000}%
\pgfsetstrokecolor{currentstroke}%
\pgfsetdash{}{0pt}%
\pgfpathmoveto{\pgfqpoint{3.407822in}{0.499444in}}%
\pgfpathlineto{\pgfqpoint{3.469208in}{0.499444in}}%
\pgfpathlineto{\pgfqpoint{3.469208in}{0.518702in}}%
\pgfpathlineto{\pgfqpoint{3.407822in}{0.518702in}}%
\pgfpathlineto{\pgfqpoint{3.407822in}{0.499444in}}%
\pgfpathclose%
\pgfusepath{stroke}%
\end{pgfscope}%
\begin{pgfscope}%
\pgfpathrectangle{\pgfqpoint{0.515000in}{0.499444in}}{\pgfqpoint{3.875000in}{1.155000in}}%
\pgfusepath{clip}%
\pgfsetbuttcap%
\pgfsetmiterjoin%
\pgfsetlinewidth{1.003750pt}%
\definecolor{currentstroke}{rgb}{0.000000,0.000000,0.000000}%
\pgfsetstrokecolor{currentstroke}%
\pgfsetdash{}{0pt}%
\pgfpathmoveto{\pgfqpoint{3.561287in}{0.499444in}}%
\pgfpathlineto{\pgfqpoint{3.622674in}{0.499444in}}%
\pgfpathlineto{\pgfqpoint{3.622674in}{0.515094in}}%
\pgfpathlineto{\pgfqpoint{3.561287in}{0.515094in}}%
\pgfpathlineto{\pgfqpoint{3.561287in}{0.499444in}}%
\pgfpathclose%
\pgfusepath{stroke}%
\end{pgfscope}%
\begin{pgfscope}%
\pgfpathrectangle{\pgfqpoint{0.515000in}{0.499444in}}{\pgfqpoint{3.875000in}{1.155000in}}%
\pgfusepath{clip}%
\pgfsetbuttcap%
\pgfsetmiterjoin%
\pgfsetlinewidth{1.003750pt}%
\definecolor{currentstroke}{rgb}{0.000000,0.000000,0.000000}%
\pgfsetstrokecolor{currentstroke}%
\pgfsetdash{}{0pt}%
\pgfpathmoveto{\pgfqpoint{3.714753in}{0.499444in}}%
\pgfpathlineto{\pgfqpoint{3.776139in}{0.499444in}}%
\pgfpathlineto{\pgfqpoint{3.776139in}{0.504431in}}%
\pgfpathlineto{\pgfqpoint{3.714753in}{0.504431in}}%
\pgfpathlineto{\pgfqpoint{3.714753in}{0.499444in}}%
\pgfpathclose%
\pgfusepath{stroke}%
\end{pgfscope}%
\begin{pgfscope}%
\pgfpathrectangle{\pgfqpoint{0.515000in}{0.499444in}}{\pgfqpoint{3.875000in}{1.155000in}}%
\pgfusepath{clip}%
\pgfsetbuttcap%
\pgfsetmiterjoin%
\pgfsetlinewidth{1.003750pt}%
\definecolor{currentstroke}{rgb}{0.000000,0.000000,0.000000}%
\pgfsetstrokecolor{currentstroke}%
\pgfsetdash{}{0pt}%
\pgfpathmoveto{\pgfqpoint{3.868218in}{0.499444in}}%
\pgfpathlineto{\pgfqpoint{3.929604in}{0.499444in}}%
\pgfpathlineto{\pgfqpoint{3.929604in}{0.499728in}}%
\pgfpathlineto{\pgfqpoint{3.868218in}{0.499728in}}%
\pgfpathlineto{\pgfqpoint{3.868218in}{0.499444in}}%
\pgfpathclose%
\pgfusepath{stroke}%
\end{pgfscope}%
\begin{pgfscope}%
\pgfpathrectangle{\pgfqpoint{0.515000in}{0.499444in}}{\pgfqpoint{3.875000in}{1.155000in}}%
\pgfusepath{clip}%
\pgfsetbuttcap%
\pgfsetmiterjoin%
\pgfsetlinewidth{1.003750pt}%
\definecolor{currentstroke}{rgb}{0.000000,0.000000,0.000000}%
\pgfsetstrokecolor{currentstroke}%
\pgfsetdash{}{0pt}%
\pgfpathmoveto{\pgfqpoint{4.021683in}{0.499444in}}%
\pgfpathlineto{\pgfqpoint{4.083070in}{0.499444in}}%
\pgfpathlineto{\pgfqpoint{4.083070in}{0.499444in}}%
\pgfpathlineto{\pgfqpoint{4.021683in}{0.499444in}}%
\pgfpathlineto{\pgfqpoint{4.021683in}{0.499444in}}%
\pgfpathclose%
\pgfusepath{stroke}%
\end{pgfscope}%
\begin{pgfscope}%
\pgfpathrectangle{\pgfqpoint{0.515000in}{0.499444in}}{\pgfqpoint{3.875000in}{1.155000in}}%
\pgfusepath{clip}%
\pgfsetbuttcap%
\pgfsetmiterjoin%
\pgfsetlinewidth{1.003750pt}%
\definecolor{currentstroke}{rgb}{0.000000,0.000000,0.000000}%
\pgfsetstrokecolor{currentstroke}%
\pgfsetdash{}{0pt}%
\pgfpathmoveto{\pgfqpoint{4.175149in}{0.499444in}}%
\pgfpathlineto{\pgfqpoint{4.236535in}{0.499444in}}%
\pgfpathlineto{\pgfqpoint{4.236535in}{0.499444in}}%
\pgfpathlineto{\pgfqpoint{4.175149in}{0.499444in}}%
\pgfpathlineto{\pgfqpoint{4.175149in}{0.499444in}}%
\pgfpathclose%
\pgfusepath{stroke}%
\end{pgfscope}%
\begin{pgfscope}%
\pgfpathrectangle{\pgfqpoint{0.515000in}{0.499444in}}{\pgfqpoint{3.875000in}{1.155000in}}%
\pgfusepath{clip}%
\pgfsetbuttcap%
\pgfsetmiterjoin%
\definecolor{currentfill}{rgb}{0.000000,0.000000,0.000000}%
\pgfsetfillcolor{currentfill}%
\pgfsetlinewidth{0.000000pt}%
\definecolor{currentstroke}{rgb}{0.000000,0.000000,0.000000}%
\pgfsetstrokecolor{currentstroke}%
\pgfsetstrokeopacity{0.000000}%
\pgfsetdash{}{0pt}%
\pgfpathmoveto{\pgfqpoint{0.553367in}{0.499444in}}%
\pgfpathlineto{\pgfqpoint{0.614753in}{0.499444in}}%
\pgfpathlineto{\pgfqpoint{0.614753in}{0.519594in}}%
\pgfpathlineto{\pgfqpoint{0.553367in}{0.519594in}}%
\pgfpathlineto{\pgfqpoint{0.553367in}{0.499444in}}%
\pgfpathclose%
\pgfusepath{fill}%
\end{pgfscope}%
\begin{pgfscope}%
\pgfpathrectangle{\pgfqpoint{0.515000in}{0.499444in}}{\pgfqpoint{3.875000in}{1.155000in}}%
\pgfusepath{clip}%
\pgfsetbuttcap%
\pgfsetmiterjoin%
\definecolor{currentfill}{rgb}{0.000000,0.000000,0.000000}%
\pgfsetfillcolor{currentfill}%
\pgfsetlinewidth{0.000000pt}%
\definecolor{currentstroke}{rgb}{0.000000,0.000000,0.000000}%
\pgfsetstrokecolor{currentstroke}%
\pgfsetstrokeopacity{0.000000}%
\pgfsetdash{}{0pt}%
\pgfpathmoveto{\pgfqpoint{0.706832in}{0.499444in}}%
\pgfpathlineto{\pgfqpoint{0.768218in}{0.499444in}}%
\pgfpathlineto{\pgfqpoint{0.768218in}{0.550609in}}%
\pgfpathlineto{\pgfqpoint{0.706832in}{0.550609in}}%
\pgfpathlineto{\pgfqpoint{0.706832in}{0.499444in}}%
\pgfpathclose%
\pgfusepath{fill}%
\end{pgfscope}%
\begin{pgfscope}%
\pgfpathrectangle{\pgfqpoint{0.515000in}{0.499444in}}{\pgfqpoint{3.875000in}{1.155000in}}%
\pgfusepath{clip}%
\pgfsetbuttcap%
\pgfsetmiterjoin%
\definecolor{currentfill}{rgb}{0.000000,0.000000,0.000000}%
\pgfsetfillcolor{currentfill}%
\pgfsetlinewidth{0.000000pt}%
\definecolor{currentstroke}{rgb}{0.000000,0.000000,0.000000}%
\pgfsetstrokecolor{currentstroke}%
\pgfsetstrokeopacity{0.000000}%
\pgfsetdash{}{0pt}%
\pgfpathmoveto{\pgfqpoint{0.860297in}{0.499444in}}%
\pgfpathlineto{\pgfqpoint{0.921683in}{0.499444in}}%
\pgfpathlineto{\pgfqpoint{0.921683in}{0.571042in}}%
\pgfpathlineto{\pgfqpoint{0.860297in}{0.571042in}}%
\pgfpathlineto{\pgfqpoint{0.860297in}{0.499444in}}%
\pgfpathclose%
\pgfusepath{fill}%
\end{pgfscope}%
\begin{pgfscope}%
\pgfpathrectangle{\pgfqpoint{0.515000in}{0.499444in}}{\pgfqpoint{3.875000in}{1.155000in}}%
\pgfusepath{clip}%
\pgfsetbuttcap%
\pgfsetmiterjoin%
\definecolor{currentfill}{rgb}{0.000000,0.000000,0.000000}%
\pgfsetfillcolor{currentfill}%
\pgfsetlinewidth{0.000000pt}%
\definecolor{currentstroke}{rgb}{0.000000,0.000000,0.000000}%
\pgfsetstrokecolor{currentstroke}%
\pgfsetstrokeopacity{0.000000}%
\pgfsetdash{}{0pt}%
\pgfpathmoveto{\pgfqpoint{1.013763in}{0.499444in}}%
\pgfpathlineto{\pgfqpoint{1.075149in}{0.499444in}}%
\pgfpathlineto{\pgfqpoint{1.075149in}{0.577083in}}%
\pgfpathlineto{\pgfqpoint{1.013763in}{0.577083in}}%
\pgfpathlineto{\pgfqpoint{1.013763in}{0.499444in}}%
\pgfpathclose%
\pgfusepath{fill}%
\end{pgfscope}%
\begin{pgfscope}%
\pgfpathrectangle{\pgfqpoint{0.515000in}{0.499444in}}{\pgfqpoint{3.875000in}{1.155000in}}%
\pgfusepath{clip}%
\pgfsetbuttcap%
\pgfsetmiterjoin%
\definecolor{currentfill}{rgb}{0.000000,0.000000,0.000000}%
\pgfsetfillcolor{currentfill}%
\pgfsetlinewidth{0.000000pt}%
\definecolor{currentstroke}{rgb}{0.000000,0.000000,0.000000}%
\pgfsetstrokecolor{currentstroke}%
\pgfsetstrokeopacity{0.000000}%
\pgfsetdash{}{0pt}%
\pgfpathmoveto{\pgfqpoint{1.167228in}{0.499444in}}%
\pgfpathlineto{\pgfqpoint{1.228614in}{0.499444in}}%
\pgfpathlineto{\pgfqpoint{1.228614in}{0.581867in}}%
\pgfpathlineto{\pgfqpoint{1.167228in}{0.581867in}}%
\pgfpathlineto{\pgfqpoint{1.167228in}{0.499444in}}%
\pgfpathclose%
\pgfusepath{fill}%
\end{pgfscope}%
\begin{pgfscope}%
\pgfpathrectangle{\pgfqpoint{0.515000in}{0.499444in}}{\pgfqpoint{3.875000in}{1.155000in}}%
\pgfusepath{clip}%
\pgfsetbuttcap%
\pgfsetmiterjoin%
\definecolor{currentfill}{rgb}{0.000000,0.000000,0.000000}%
\pgfsetfillcolor{currentfill}%
\pgfsetlinewidth{0.000000pt}%
\definecolor{currentstroke}{rgb}{0.000000,0.000000,0.000000}%
\pgfsetstrokecolor{currentstroke}%
\pgfsetstrokeopacity{0.000000}%
\pgfsetdash{}{0pt}%
\pgfpathmoveto{\pgfqpoint{1.320693in}{0.499444in}}%
\pgfpathlineto{\pgfqpoint{1.382079in}{0.499444in}}%
\pgfpathlineto{\pgfqpoint{1.382079in}{0.587665in}}%
\pgfpathlineto{\pgfqpoint{1.320693in}{0.587665in}}%
\pgfpathlineto{\pgfqpoint{1.320693in}{0.499444in}}%
\pgfpathclose%
\pgfusepath{fill}%
\end{pgfscope}%
\begin{pgfscope}%
\pgfpathrectangle{\pgfqpoint{0.515000in}{0.499444in}}{\pgfqpoint{3.875000in}{1.155000in}}%
\pgfusepath{clip}%
\pgfsetbuttcap%
\pgfsetmiterjoin%
\definecolor{currentfill}{rgb}{0.000000,0.000000,0.000000}%
\pgfsetfillcolor{currentfill}%
\pgfsetlinewidth{0.000000pt}%
\definecolor{currentstroke}{rgb}{0.000000,0.000000,0.000000}%
\pgfsetstrokecolor{currentstroke}%
\pgfsetstrokeopacity{0.000000}%
\pgfsetdash{}{0pt}%
\pgfpathmoveto{\pgfqpoint{1.474159in}{0.499444in}}%
\pgfpathlineto{\pgfqpoint{1.535545in}{0.499444in}}%
\pgfpathlineto{\pgfqpoint{1.535545in}{0.587949in}}%
\pgfpathlineto{\pgfqpoint{1.474159in}{0.587949in}}%
\pgfpathlineto{\pgfqpoint{1.474159in}{0.499444in}}%
\pgfpathclose%
\pgfusepath{fill}%
\end{pgfscope}%
\begin{pgfscope}%
\pgfpathrectangle{\pgfqpoint{0.515000in}{0.499444in}}{\pgfqpoint{3.875000in}{1.155000in}}%
\pgfusepath{clip}%
\pgfsetbuttcap%
\pgfsetmiterjoin%
\definecolor{currentfill}{rgb}{0.000000,0.000000,0.000000}%
\pgfsetfillcolor{currentfill}%
\pgfsetlinewidth{0.000000pt}%
\definecolor{currentstroke}{rgb}{0.000000,0.000000,0.000000}%
\pgfsetstrokecolor{currentstroke}%
\pgfsetstrokeopacity{0.000000}%
\pgfsetdash{}{0pt}%
\pgfpathmoveto{\pgfqpoint{1.627624in}{0.499444in}}%
\pgfpathlineto{\pgfqpoint{1.689010in}{0.499444in}}%
\pgfpathlineto{\pgfqpoint{1.689010in}{0.583529in}}%
\pgfpathlineto{\pgfqpoint{1.627624in}{0.583529in}}%
\pgfpathlineto{\pgfqpoint{1.627624in}{0.499444in}}%
\pgfpathclose%
\pgfusepath{fill}%
\end{pgfscope}%
\begin{pgfscope}%
\pgfpathrectangle{\pgfqpoint{0.515000in}{0.499444in}}{\pgfqpoint{3.875000in}{1.155000in}}%
\pgfusepath{clip}%
\pgfsetbuttcap%
\pgfsetmiterjoin%
\definecolor{currentfill}{rgb}{0.000000,0.000000,0.000000}%
\pgfsetfillcolor{currentfill}%
\pgfsetlinewidth{0.000000pt}%
\definecolor{currentstroke}{rgb}{0.000000,0.000000,0.000000}%
\pgfsetstrokecolor{currentstroke}%
\pgfsetstrokeopacity{0.000000}%
\pgfsetdash{}{0pt}%
\pgfpathmoveto{\pgfqpoint{1.781089in}{0.499444in}}%
\pgfpathlineto{\pgfqpoint{1.842476in}{0.499444in}}%
\pgfpathlineto{\pgfqpoint{1.842476in}{0.583732in}}%
\pgfpathlineto{\pgfqpoint{1.781089in}{0.583732in}}%
\pgfpathlineto{\pgfqpoint{1.781089in}{0.499444in}}%
\pgfpathclose%
\pgfusepath{fill}%
\end{pgfscope}%
\begin{pgfscope}%
\pgfpathrectangle{\pgfqpoint{0.515000in}{0.499444in}}{\pgfqpoint{3.875000in}{1.155000in}}%
\pgfusepath{clip}%
\pgfsetbuttcap%
\pgfsetmiterjoin%
\definecolor{currentfill}{rgb}{0.000000,0.000000,0.000000}%
\pgfsetfillcolor{currentfill}%
\pgfsetlinewidth{0.000000pt}%
\definecolor{currentstroke}{rgb}{0.000000,0.000000,0.000000}%
\pgfsetstrokecolor{currentstroke}%
\pgfsetstrokeopacity{0.000000}%
\pgfsetdash{}{0pt}%
\pgfpathmoveto{\pgfqpoint{1.934555in}{0.499444in}}%
\pgfpathlineto{\pgfqpoint{1.995941in}{0.499444in}}%
\pgfpathlineto{\pgfqpoint{1.995941in}{0.580772in}}%
\pgfpathlineto{\pgfqpoint{1.934555in}{0.580772in}}%
\pgfpathlineto{\pgfqpoint{1.934555in}{0.499444in}}%
\pgfpathclose%
\pgfusepath{fill}%
\end{pgfscope}%
\begin{pgfscope}%
\pgfpathrectangle{\pgfqpoint{0.515000in}{0.499444in}}{\pgfqpoint{3.875000in}{1.155000in}}%
\pgfusepath{clip}%
\pgfsetbuttcap%
\pgfsetmiterjoin%
\definecolor{currentfill}{rgb}{0.000000,0.000000,0.000000}%
\pgfsetfillcolor{currentfill}%
\pgfsetlinewidth{0.000000pt}%
\definecolor{currentstroke}{rgb}{0.000000,0.000000,0.000000}%
\pgfsetstrokecolor{currentstroke}%
\pgfsetstrokeopacity{0.000000}%
\pgfsetdash{}{0pt}%
\pgfpathmoveto{\pgfqpoint{2.088020in}{0.499444in}}%
\pgfpathlineto{\pgfqpoint{2.149406in}{0.499444in}}%
\pgfpathlineto{\pgfqpoint{2.149406in}{0.577853in}}%
\pgfpathlineto{\pgfqpoint{2.088020in}{0.577853in}}%
\pgfpathlineto{\pgfqpoint{2.088020in}{0.499444in}}%
\pgfpathclose%
\pgfusepath{fill}%
\end{pgfscope}%
\begin{pgfscope}%
\pgfpathrectangle{\pgfqpoint{0.515000in}{0.499444in}}{\pgfqpoint{3.875000in}{1.155000in}}%
\pgfusepath{clip}%
\pgfsetbuttcap%
\pgfsetmiterjoin%
\definecolor{currentfill}{rgb}{0.000000,0.000000,0.000000}%
\pgfsetfillcolor{currentfill}%
\pgfsetlinewidth{0.000000pt}%
\definecolor{currentstroke}{rgb}{0.000000,0.000000,0.000000}%
\pgfsetstrokecolor{currentstroke}%
\pgfsetstrokeopacity{0.000000}%
\pgfsetdash{}{0pt}%
\pgfpathmoveto{\pgfqpoint{2.241485in}{0.499444in}}%
\pgfpathlineto{\pgfqpoint{2.302872in}{0.499444in}}%
\pgfpathlineto{\pgfqpoint{2.302872in}{0.573799in}}%
\pgfpathlineto{\pgfqpoint{2.241485in}{0.573799in}}%
\pgfpathlineto{\pgfqpoint{2.241485in}{0.499444in}}%
\pgfpathclose%
\pgfusepath{fill}%
\end{pgfscope}%
\begin{pgfscope}%
\pgfpathrectangle{\pgfqpoint{0.515000in}{0.499444in}}{\pgfqpoint{3.875000in}{1.155000in}}%
\pgfusepath{clip}%
\pgfsetbuttcap%
\pgfsetmiterjoin%
\definecolor{currentfill}{rgb}{0.000000,0.000000,0.000000}%
\pgfsetfillcolor{currentfill}%
\pgfsetlinewidth{0.000000pt}%
\definecolor{currentstroke}{rgb}{0.000000,0.000000,0.000000}%
\pgfsetstrokecolor{currentstroke}%
\pgfsetstrokeopacity{0.000000}%
\pgfsetdash{}{0pt}%
\pgfpathmoveto{\pgfqpoint{2.394951in}{0.499444in}}%
\pgfpathlineto{\pgfqpoint{2.456337in}{0.499444in}}%
\pgfpathlineto{\pgfqpoint{2.456337in}{0.567596in}}%
\pgfpathlineto{\pgfqpoint{2.394951in}{0.567596in}}%
\pgfpathlineto{\pgfqpoint{2.394951in}{0.499444in}}%
\pgfpathclose%
\pgfusepath{fill}%
\end{pgfscope}%
\begin{pgfscope}%
\pgfpathrectangle{\pgfqpoint{0.515000in}{0.499444in}}{\pgfqpoint{3.875000in}{1.155000in}}%
\pgfusepath{clip}%
\pgfsetbuttcap%
\pgfsetmiterjoin%
\definecolor{currentfill}{rgb}{0.000000,0.000000,0.000000}%
\pgfsetfillcolor{currentfill}%
\pgfsetlinewidth{0.000000pt}%
\definecolor{currentstroke}{rgb}{0.000000,0.000000,0.000000}%
\pgfsetstrokecolor{currentstroke}%
\pgfsetstrokeopacity{0.000000}%
\pgfsetdash{}{0pt}%
\pgfpathmoveto{\pgfqpoint{2.548416in}{0.499444in}}%
\pgfpathlineto{\pgfqpoint{2.609802in}{0.499444in}}%
\pgfpathlineto{\pgfqpoint{2.609802in}{0.566015in}}%
\pgfpathlineto{\pgfqpoint{2.548416in}{0.566015in}}%
\pgfpathlineto{\pgfqpoint{2.548416in}{0.499444in}}%
\pgfpathclose%
\pgfusepath{fill}%
\end{pgfscope}%
\begin{pgfscope}%
\pgfpathrectangle{\pgfqpoint{0.515000in}{0.499444in}}{\pgfqpoint{3.875000in}{1.155000in}}%
\pgfusepath{clip}%
\pgfsetbuttcap%
\pgfsetmiterjoin%
\definecolor{currentfill}{rgb}{0.000000,0.000000,0.000000}%
\pgfsetfillcolor{currentfill}%
\pgfsetlinewidth{0.000000pt}%
\definecolor{currentstroke}{rgb}{0.000000,0.000000,0.000000}%
\pgfsetstrokecolor{currentstroke}%
\pgfsetstrokeopacity{0.000000}%
\pgfsetdash{}{0pt}%
\pgfpathmoveto{\pgfqpoint{2.701881in}{0.499444in}}%
\pgfpathlineto{\pgfqpoint{2.763268in}{0.499444in}}%
\pgfpathlineto{\pgfqpoint{2.763268in}{0.561190in}}%
\pgfpathlineto{\pgfqpoint{2.701881in}{0.561190in}}%
\pgfpathlineto{\pgfqpoint{2.701881in}{0.499444in}}%
\pgfpathclose%
\pgfusepath{fill}%
\end{pgfscope}%
\begin{pgfscope}%
\pgfpathrectangle{\pgfqpoint{0.515000in}{0.499444in}}{\pgfqpoint{3.875000in}{1.155000in}}%
\pgfusepath{clip}%
\pgfsetbuttcap%
\pgfsetmiterjoin%
\definecolor{currentfill}{rgb}{0.000000,0.000000,0.000000}%
\pgfsetfillcolor{currentfill}%
\pgfsetlinewidth{0.000000pt}%
\definecolor{currentstroke}{rgb}{0.000000,0.000000,0.000000}%
\pgfsetstrokecolor{currentstroke}%
\pgfsetstrokeopacity{0.000000}%
\pgfsetdash{}{0pt}%
\pgfpathmoveto{\pgfqpoint{2.855347in}{0.499444in}}%
\pgfpathlineto{\pgfqpoint{2.916733in}{0.499444in}}%
\pgfpathlineto{\pgfqpoint{2.916733in}{0.555271in}}%
\pgfpathlineto{\pgfqpoint{2.855347in}{0.555271in}}%
\pgfpathlineto{\pgfqpoint{2.855347in}{0.499444in}}%
\pgfpathclose%
\pgfusepath{fill}%
\end{pgfscope}%
\begin{pgfscope}%
\pgfpathrectangle{\pgfqpoint{0.515000in}{0.499444in}}{\pgfqpoint{3.875000in}{1.155000in}}%
\pgfusepath{clip}%
\pgfsetbuttcap%
\pgfsetmiterjoin%
\definecolor{currentfill}{rgb}{0.000000,0.000000,0.000000}%
\pgfsetfillcolor{currentfill}%
\pgfsetlinewidth{0.000000pt}%
\definecolor{currentstroke}{rgb}{0.000000,0.000000,0.000000}%
\pgfsetstrokecolor{currentstroke}%
\pgfsetstrokeopacity{0.000000}%
\pgfsetdash{}{0pt}%
\pgfpathmoveto{\pgfqpoint{3.008812in}{0.499444in}}%
\pgfpathlineto{\pgfqpoint{3.070198in}{0.499444in}}%
\pgfpathlineto{\pgfqpoint{3.070198in}{0.549514in}}%
\pgfpathlineto{\pgfqpoint{3.008812in}{0.549514in}}%
\pgfpathlineto{\pgfqpoint{3.008812in}{0.499444in}}%
\pgfpathclose%
\pgfusepath{fill}%
\end{pgfscope}%
\begin{pgfscope}%
\pgfpathrectangle{\pgfqpoint{0.515000in}{0.499444in}}{\pgfqpoint{3.875000in}{1.155000in}}%
\pgfusepath{clip}%
\pgfsetbuttcap%
\pgfsetmiterjoin%
\definecolor{currentfill}{rgb}{0.000000,0.000000,0.000000}%
\pgfsetfillcolor{currentfill}%
\pgfsetlinewidth{0.000000pt}%
\definecolor{currentstroke}{rgb}{0.000000,0.000000,0.000000}%
\pgfsetstrokecolor{currentstroke}%
\pgfsetstrokeopacity{0.000000}%
\pgfsetdash{}{0pt}%
\pgfpathmoveto{\pgfqpoint{3.162278in}{0.499444in}}%
\pgfpathlineto{\pgfqpoint{3.223664in}{0.499444in}}%
\pgfpathlineto{\pgfqpoint{3.223664in}{0.543554in}}%
\pgfpathlineto{\pgfqpoint{3.162278in}{0.543554in}}%
\pgfpathlineto{\pgfqpoint{3.162278in}{0.499444in}}%
\pgfpathclose%
\pgfusepath{fill}%
\end{pgfscope}%
\begin{pgfscope}%
\pgfpathrectangle{\pgfqpoint{0.515000in}{0.499444in}}{\pgfqpoint{3.875000in}{1.155000in}}%
\pgfusepath{clip}%
\pgfsetbuttcap%
\pgfsetmiterjoin%
\definecolor{currentfill}{rgb}{0.000000,0.000000,0.000000}%
\pgfsetfillcolor{currentfill}%
\pgfsetlinewidth{0.000000pt}%
\definecolor{currentstroke}{rgb}{0.000000,0.000000,0.000000}%
\pgfsetstrokecolor{currentstroke}%
\pgfsetstrokeopacity{0.000000}%
\pgfsetdash{}{0pt}%
\pgfpathmoveto{\pgfqpoint{3.315743in}{0.499444in}}%
\pgfpathlineto{\pgfqpoint{3.377129in}{0.499444in}}%
\pgfpathlineto{\pgfqpoint{3.377129in}{0.539014in}}%
\pgfpathlineto{\pgfqpoint{3.315743in}{0.539014in}}%
\pgfpathlineto{\pgfqpoint{3.315743in}{0.499444in}}%
\pgfpathclose%
\pgfusepath{fill}%
\end{pgfscope}%
\begin{pgfscope}%
\pgfpathrectangle{\pgfqpoint{0.515000in}{0.499444in}}{\pgfqpoint{3.875000in}{1.155000in}}%
\pgfusepath{clip}%
\pgfsetbuttcap%
\pgfsetmiterjoin%
\definecolor{currentfill}{rgb}{0.000000,0.000000,0.000000}%
\pgfsetfillcolor{currentfill}%
\pgfsetlinewidth{0.000000pt}%
\definecolor{currentstroke}{rgb}{0.000000,0.000000,0.000000}%
\pgfsetstrokecolor{currentstroke}%
\pgfsetstrokeopacity{0.000000}%
\pgfsetdash{}{0pt}%
\pgfpathmoveto{\pgfqpoint{3.469208in}{0.499444in}}%
\pgfpathlineto{\pgfqpoint{3.530594in}{0.499444in}}%
\pgfpathlineto{\pgfqpoint{3.530594in}{0.540635in}}%
\pgfpathlineto{\pgfqpoint{3.469208in}{0.540635in}}%
\pgfpathlineto{\pgfqpoint{3.469208in}{0.499444in}}%
\pgfpathclose%
\pgfusepath{fill}%
\end{pgfscope}%
\begin{pgfscope}%
\pgfpathrectangle{\pgfqpoint{0.515000in}{0.499444in}}{\pgfqpoint{3.875000in}{1.155000in}}%
\pgfusepath{clip}%
\pgfsetbuttcap%
\pgfsetmiterjoin%
\definecolor{currentfill}{rgb}{0.000000,0.000000,0.000000}%
\pgfsetfillcolor{currentfill}%
\pgfsetlinewidth{0.000000pt}%
\definecolor{currentstroke}{rgb}{0.000000,0.000000,0.000000}%
\pgfsetstrokecolor{currentstroke}%
\pgfsetstrokeopacity{0.000000}%
\pgfsetdash{}{0pt}%
\pgfpathmoveto{\pgfqpoint{3.622674in}{0.499444in}}%
\pgfpathlineto{\pgfqpoint{3.684060in}{0.499444in}}%
\pgfpathlineto{\pgfqpoint{3.684060in}{0.539743in}}%
\pgfpathlineto{\pgfqpoint{3.622674in}{0.539743in}}%
\pgfpathlineto{\pgfqpoint{3.622674in}{0.499444in}}%
\pgfpathclose%
\pgfusepath{fill}%
\end{pgfscope}%
\begin{pgfscope}%
\pgfpathrectangle{\pgfqpoint{0.515000in}{0.499444in}}{\pgfqpoint{3.875000in}{1.155000in}}%
\pgfusepath{clip}%
\pgfsetbuttcap%
\pgfsetmiterjoin%
\definecolor{currentfill}{rgb}{0.000000,0.000000,0.000000}%
\pgfsetfillcolor{currentfill}%
\pgfsetlinewidth{0.000000pt}%
\definecolor{currentstroke}{rgb}{0.000000,0.000000,0.000000}%
\pgfsetstrokecolor{currentstroke}%
\pgfsetstrokeopacity{0.000000}%
\pgfsetdash{}{0pt}%
\pgfpathmoveto{\pgfqpoint{3.776139in}{0.499444in}}%
\pgfpathlineto{\pgfqpoint{3.837525in}{0.499444in}}%
\pgfpathlineto{\pgfqpoint{3.837525in}{0.519432in}}%
\pgfpathlineto{\pgfqpoint{3.776139in}{0.519432in}}%
\pgfpathlineto{\pgfqpoint{3.776139in}{0.499444in}}%
\pgfpathclose%
\pgfusepath{fill}%
\end{pgfscope}%
\begin{pgfscope}%
\pgfpathrectangle{\pgfqpoint{0.515000in}{0.499444in}}{\pgfqpoint{3.875000in}{1.155000in}}%
\pgfusepath{clip}%
\pgfsetbuttcap%
\pgfsetmiterjoin%
\definecolor{currentfill}{rgb}{0.000000,0.000000,0.000000}%
\pgfsetfillcolor{currentfill}%
\pgfsetlinewidth{0.000000pt}%
\definecolor{currentstroke}{rgb}{0.000000,0.000000,0.000000}%
\pgfsetstrokecolor{currentstroke}%
\pgfsetstrokeopacity{0.000000}%
\pgfsetdash{}{0pt}%
\pgfpathmoveto{\pgfqpoint{3.929604in}{0.499444in}}%
\pgfpathlineto{\pgfqpoint{3.990990in}{0.499444in}}%
\pgfpathlineto{\pgfqpoint{3.990990in}{0.500944in}}%
\pgfpathlineto{\pgfqpoint{3.929604in}{0.500944in}}%
\pgfpathlineto{\pgfqpoint{3.929604in}{0.499444in}}%
\pgfpathclose%
\pgfusepath{fill}%
\end{pgfscope}%
\begin{pgfscope}%
\pgfpathrectangle{\pgfqpoint{0.515000in}{0.499444in}}{\pgfqpoint{3.875000in}{1.155000in}}%
\pgfusepath{clip}%
\pgfsetbuttcap%
\pgfsetmiterjoin%
\definecolor{currentfill}{rgb}{0.000000,0.000000,0.000000}%
\pgfsetfillcolor{currentfill}%
\pgfsetlinewidth{0.000000pt}%
\definecolor{currentstroke}{rgb}{0.000000,0.000000,0.000000}%
\pgfsetstrokecolor{currentstroke}%
\pgfsetstrokeopacity{0.000000}%
\pgfsetdash{}{0pt}%
\pgfpathmoveto{\pgfqpoint{4.083070in}{0.499444in}}%
\pgfpathlineto{\pgfqpoint{4.144456in}{0.499444in}}%
\pgfpathlineto{\pgfqpoint{4.144456in}{0.499444in}}%
\pgfpathlineto{\pgfqpoint{4.083070in}{0.499444in}}%
\pgfpathlineto{\pgfqpoint{4.083070in}{0.499444in}}%
\pgfpathclose%
\pgfusepath{fill}%
\end{pgfscope}%
\begin{pgfscope}%
\pgfpathrectangle{\pgfqpoint{0.515000in}{0.499444in}}{\pgfqpoint{3.875000in}{1.155000in}}%
\pgfusepath{clip}%
\pgfsetbuttcap%
\pgfsetmiterjoin%
\definecolor{currentfill}{rgb}{0.000000,0.000000,0.000000}%
\pgfsetfillcolor{currentfill}%
\pgfsetlinewidth{0.000000pt}%
\definecolor{currentstroke}{rgb}{0.000000,0.000000,0.000000}%
\pgfsetstrokecolor{currentstroke}%
\pgfsetstrokeopacity{0.000000}%
\pgfsetdash{}{0pt}%
\pgfpathmoveto{\pgfqpoint{4.236535in}{0.499444in}}%
\pgfpathlineto{\pgfqpoint{4.297921in}{0.499444in}}%
\pgfpathlineto{\pgfqpoint{4.297921in}{0.499444in}}%
\pgfpathlineto{\pgfqpoint{4.236535in}{0.499444in}}%
\pgfpathlineto{\pgfqpoint{4.236535in}{0.499444in}}%
\pgfpathclose%
\pgfusepath{fill}%
\end{pgfscope}%
\begin{pgfscope}%
\pgfsetbuttcap%
\pgfsetroundjoin%
\definecolor{currentfill}{rgb}{0.000000,0.000000,0.000000}%
\pgfsetfillcolor{currentfill}%
\pgfsetlinewidth{0.803000pt}%
\definecolor{currentstroke}{rgb}{0.000000,0.000000,0.000000}%
\pgfsetstrokecolor{currentstroke}%
\pgfsetdash{}{0pt}%
\pgfsys@defobject{currentmarker}{\pgfqpoint{0.000000in}{-0.048611in}}{\pgfqpoint{0.000000in}{0.000000in}}{%
\pgfpathmoveto{\pgfqpoint{0.000000in}{0.000000in}}%
\pgfpathlineto{\pgfqpoint{0.000000in}{-0.048611in}}%
\pgfusepath{stroke,fill}%
}%
\begin{pgfscope}%
\pgfsys@transformshift{0.553367in}{0.499444in}%
\pgfsys@useobject{currentmarker}{}%
\end{pgfscope}%
\end{pgfscope}%
\begin{pgfscope}%
\definecolor{textcolor}{rgb}{0.000000,0.000000,0.000000}%
\pgfsetstrokecolor{textcolor}%
\pgfsetfillcolor{textcolor}%
\pgftext[x=0.553367in,y=0.402222in,,top]{\color{textcolor}\rmfamily\fontsize{10.000000}{12.000000}\selectfont 0.0}%
\end{pgfscope}%
\begin{pgfscope}%
\pgfsetbuttcap%
\pgfsetroundjoin%
\definecolor{currentfill}{rgb}{0.000000,0.000000,0.000000}%
\pgfsetfillcolor{currentfill}%
\pgfsetlinewidth{0.803000pt}%
\definecolor{currentstroke}{rgb}{0.000000,0.000000,0.000000}%
\pgfsetstrokecolor{currentstroke}%
\pgfsetdash{}{0pt}%
\pgfsys@defobject{currentmarker}{\pgfqpoint{0.000000in}{-0.048611in}}{\pgfqpoint{0.000000in}{0.000000in}}{%
\pgfpathmoveto{\pgfqpoint{0.000000in}{0.000000in}}%
\pgfpathlineto{\pgfqpoint{0.000000in}{-0.048611in}}%
\pgfusepath{stroke,fill}%
}%
\begin{pgfscope}%
\pgfsys@transformshift{0.937030in}{0.499444in}%
\pgfsys@useobject{currentmarker}{}%
\end{pgfscope}%
\end{pgfscope}%
\begin{pgfscope}%
\definecolor{textcolor}{rgb}{0.000000,0.000000,0.000000}%
\pgfsetstrokecolor{textcolor}%
\pgfsetfillcolor{textcolor}%
\pgftext[x=0.937030in,y=0.402222in,,top]{\color{textcolor}\rmfamily\fontsize{10.000000}{12.000000}\selectfont 0.1}%
\end{pgfscope}%
\begin{pgfscope}%
\pgfsetbuttcap%
\pgfsetroundjoin%
\definecolor{currentfill}{rgb}{0.000000,0.000000,0.000000}%
\pgfsetfillcolor{currentfill}%
\pgfsetlinewidth{0.803000pt}%
\definecolor{currentstroke}{rgb}{0.000000,0.000000,0.000000}%
\pgfsetstrokecolor{currentstroke}%
\pgfsetdash{}{0pt}%
\pgfsys@defobject{currentmarker}{\pgfqpoint{0.000000in}{-0.048611in}}{\pgfqpoint{0.000000in}{0.000000in}}{%
\pgfpathmoveto{\pgfqpoint{0.000000in}{0.000000in}}%
\pgfpathlineto{\pgfqpoint{0.000000in}{-0.048611in}}%
\pgfusepath{stroke,fill}%
}%
\begin{pgfscope}%
\pgfsys@transformshift{1.320693in}{0.499444in}%
\pgfsys@useobject{currentmarker}{}%
\end{pgfscope}%
\end{pgfscope}%
\begin{pgfscope}%
\definecolor{textcolor}{rgb}{0.000000,0.000000,0.000000}%
\pgfsetstrokecolor{textcolor}%
\pgfsetfillcolor{textcolor}%
\pgftext[x=1.320693in,y=0.402222in,,top]{\color{textcolor}\rmfamily\fontsize{10.000000}{12.000000}\selectfont 0.2}%
\end{pgfscope}%
\begin{pgfscope}%
\pgfsetbuttcap%
\pgfsetroundjoin%
\definecolor{currentfill}{rgb}{0.000000,0.000000,0.000000}%
\pgfsetfillcolor{currentfill}%
\pgfsetlinewidth{0.803000pt}%
\definecolor{currentstroke}{rgb}{0.000000,0.000000,0.000000}%
\pgfsetstrokecolor{currentstroke}%
\pgfsetdash{}{0pt}%
\pgfsys@defobject{currentmarker}{\pgfqpoint{0.000000in}{-0.048611in}}{\pgfqpoint{0.000000in}{0.000000in}}{%
\pgfpathmoveto{\pgfqpoint{0.000000in}{0.000000in}}%
\pgfpathlineto{\pgfqpoint{0.000000in}{-0.048611in}}%
\pgfusepath{stroke,fill}%
}%
\begin{pgfscope}%
\pgfsys@transformshift{1.704357in}{0.499444in}%
\pgfsys@useobject{currentmarker}{}%
\end{pgfscope}%
\end{pgfscope}%
\begin{pgfscope}%
\definecolor{textcolor}{rgb}{0.000000,0.000000,0.000000}%
\pgfsetstrokecolor{textcolor}%
\pgfsetfillcolor{textcolor}%
\pgftext[x=1.704357in,y=0.402222in,,top]{\color{textcolor}\rmfamily\fontsize{10.000000}{12.000000}\selectfont 0.3}%
\end{pgfscope}%
\begin{pgfscope}%
\pgfsetbuttcap%
\pgfsetroundjoin%
\definecolor{currentfill}{rgb}{0.000000,0.000000,0.000000}%
\pgfsetfillcolor{currentfill}%
\pgfsetlinewidth{0.803000pt}%
\definecolor{currentstroke}{rgb}{0.000000,0.000000,0.000000}%
\pgfsetstrokecolor{currentstroke}%
\pgfsetdash{}{0pt}%
\pgfsys@defobject{currentmarker}{\pgfqpoint{0.000000in}{-0.048611in}}{\pgfqpoint{0.000000in}{0.000000in}}{%
\pgfpathmoveto{\pgfqpoint{0.000000in}{0.000000in}}%
\pgfpathlineto{\pgfqpoint{0.000000in}{-0.048611in}}%
\pgfusepath{stroke,fill}%
}%
\begin{pgfscope}%
\pgfsys@transformshift{2.088020in}{0.499444in}%
\pgfsys@useobject{currentmarker}{}%
\end{pgfscope}%
\end{pgfscope}%
\begin{pgfscope}%
\definecolor{textcolor}{rgb}{0.000000,0.000000,0.000000}%
\pgfsetstrokecolor{textcolor}%
\pgfsetfillcolor{textcolor}%
\pgftext[x=2.088020in,y=0.402222in,,top]{\color{textcolor}\rmfamily\fontsize{10.000000}{12.000000}\selectfont 0.4}%
\end{pgfscope}%
\begin{pgfscope}%
\pgfsetbuttcap%
\pgfsetroundjoin%
\definecolor{currentfill}{rgb}{0.000000,0.000000,0.000000}%
\pgfsetfillcolor{currentfill}%
\pgfsetlinewidth{0.803000pt}%
\definecolor{currentstroke}{rgb}{0.000000,0.000000,0.000000}%
\pgfsetstrokecolor{currentstroke}%
\pgfsetdash{}{0pt}%
\pgfsys@defobject{currentmarker}{\pgfqpoint{0.000000in}{-0.048611in}}{\pgfqpoint{0.000000in}{0.000000in}}{%
\pgfpathmoveto{\pgfqpoint{0.000000in}{0.000000in}}%
\pgfpathlineto{\pgfqpoint{0.000000in}{-0.048611in}}%
\pgfusepath{stroke,fill}%
}%
\begin{pgfscope}%
\pgfsys@transformshift{2.471683in}{0.499444in}%
\pgfsys@useobject{currentmarker}{}%
\end{pgfscope}%
\end{pgfscope}%
\begin{pgfscope}%
\definecolor{textcolor}{rgb}{0.000000,0.000000,0.000000}%
\pgfsetstrokecolor{textcolor}%
\pgfsetfillcolor{textcolor}%
\pgftext[x=2.471683in,y=0.402222in,,top]{\color{textcolor}\rmfamily\fontsize{10.000000}{12.000000}\selectfont 0.5}%
\end{pgfscope}%
\begin{pgfscope}%
\pgfsetbuttcap%
\pgfsetroundjoin%
\definecolor{currentfill}{rgb}{0.000000,0.000000,0.000000}%
\pgfsetfillcolor{currentfill}%
\pgfsetlinewidth{0.803000pt}%
\definecolor{currentstroke}{rgb}{0.000000,0.000000,0.000000}%
\pgfsetstrokecolor{currentstroke}%
\pgfsetdash{}{0pt}%
\pgfsys@defobject{currentmarker}{\pgfqpoint{0.000000in}{-0.048611in}}{\pgfqpoint{0.000000in}{0.000000in}}{%
\pgfpathmoveto{\pgfqpoint{0.000000in}{0.000000in}}%
\pgfpathlineto{\pgfqpoint{0.000000in}{-0.048611in}}%
\pgfusepath{stroke,fill}%
}%
\begin{pgfscope}%
\pgfsys@transformshift{2.855347in}{0.499444in}%
\pgfsys@useobject{currentmarker}{}%
\end{pgfscope}%
\end{pgfscope}%
\begin{pgfscope}%
\definecolor{textcolor}{rgb}{0.000000,0.000000,0.000000}%
\pgfsetstrokecolor{textcolor}%
\pgfsetfillcolor{textcolor}%
\pgftext[x=2.855347in,y=0.402222in,,top]{\color{textcolor}\rmfamily\fontsize{10.000000}{12.000000}\selectfont 0.6}%
\end{pgfscope}%
\begin{pgfscope}%
\pgfsetbuttcap%
\pgfsetroundjoin%
\definecolor{currentfill}{rgb}{0.000000,0.000000,0.000000}%
\pgfsetfillcolor{currentfill}%
\pgfsetlinewidth{0.803000pt}%
\definecolor{currentstroke}{rgb}{0.000000,0.000000,0.000000}%
\pgfsetstrokecolor{currentstroke}%
\pgfsetdash{}{0pt}%
\pgfsys@defobject{currentmarker}{\pgfqpoint{0.000000in}{-0.048611in}}{\pgfqpoint{0.000000in}{0.000000in}}{%
\pgfpathmoveto{\pgfqpoint{0.000000in}{0.000000in}}%
\pgfpathlineto{\pgfqpoint{0.000000in}{-0.048611in}}%
\pgfusepath{stroke,fill}%
}%
\begin{pgfscope}%
\pgfsys@transformshift{3.239010in}{0.499444in}%
\pgfsys@useobject{currentmarker}{}%
\end{pgfscope}%
\end{pgfscope}%
\begin{pgfscope}%
\definecolor{textcolor}{rgb}{0.000000,0.000000,0.000000}%
\pgfsetstrokecolor{textcolor}%
\pgfsetfillcolor{textcolor}%
\pgftext[x=3.239010in,y=0.402222in,,top]{\color{textcolor}\rmfamily\fontsize{10.000000}{12.000000}\selectfont 0.7}%
\end{pgfscope}%
\begin{pgfscope}%
\pgfsetbuttcap%
\pgfsetroundjoin%
\definecolor{currentfill}{rgb}{0.000000,0.000000,0.000000}%
\pgfsetfillcolor{currentfill}%
\pgfsetlinewidth{0.803000pt}%
\definecolor{currentstroke}{rgb}{0.000000,0.000000,0.000000}%
\pgfsetstrokecolor{currentstroke}%
\pgfsetdash{}{0pt}%
\pgfsys@defobject{currentmarker}{\pgfqpoint{0.000000in}{-0.048611in}}{\pgfqpoint{0.000000in}{0.000000in}}{%
\pgfpathmoveto{\pgfqpoint{0.000000in}{0.000000in}}%
\pgfpathlineto{\pgfqpoint{0.000000in}{-0.048611in}}%
\pgfusepath{stroke,fill}%
}%
\begin{pgfscope}%
\pgfsys@transformshift{3.622674in}{0.499444in}%
\pgfsys@useobject{currentmarker}{}%
\end{pgfscope}%
\end{pgfscope}%
\begin{pgfscope}%
\definecolor{textcolor}{rgb}{0.000000,0.000000,0.000000}%
\pgfsetstrokecolor{textcolor}%
\pgfsetfillcolor{textcolor}%
\pgftext[x=3.622674in,y=0.402222in,,top]{\color{textcolor}\rmfamily\fontsize{10.000000}{12.000000}\selectfont 0.8}%
\end{pgfscope}%
\begin{pgfscope}%
\pgfsetbuttcap%
\pgfsetroundjoin%
\definecolor{currentfill}{rgb}{0.000000,0.000000,0.000000}%
\pgfsetfillcolor{currentfill}%
\pgfsetlinewidth{0.803000pt}%
\definecolor{currentstroke}{rgb}{0.000000,0.000000,0.000000}%
\pgfsetstrokecolor{currentstroke}%
\pgfsetdash{}{0pt}%
\pgfsys@defobject{currentmarker}{\pgfqpoint{0.000000in}{-0.048611in}}{\pgfqpoint{0.000000in}{0.000000in}}{%
\pgfpathmoveto{\pgfqpoint{0.000000in}{0.000000in}}%
\pgfpathlineto{\pgfqpoint{0.000000in}{-0.048611in}}%
\pgfusepath{stroke,fill}%
}%
\begin{pgfscope}%
\pgfsys@transformshift{4.006337in}{0.499444in}%
\pgfsys@useobject{currentmarker}{}%
\end{pgfscope}%
\end{pgfscope}%
\begin{pgfscope}%
\definecolor{textcolor}{rgb}{0.000000,0.000000,0.000000}%
\pgfsetstrokecolor{textcolor}%
\pgfsetfillcolor{textcolor}%
\pgftext[x=4.006337in,y=0.402222in,,top]{\color{textcolor}\rmfamily\fontsize{10.000000}{12.000000}\selectfont 0.9}%
\end{pgfscope}%
\begin{pgfscope}%
\pgfsetbuttcap%
\pgfsetroundjoin%
\definecolor{currentfill}{rgb}{0.000000,0.000000,0.000000}%
\pgfsetfillcolor{currentfill}%
\pgfsetlinewidth{0.803000pt}%
\definecolor{currentstroke}{rgb}{0.000000,0.000000,0.000000}%
\pgfsetstrokecolor{currentstroke}%
\pgfsetdash{}{0pt}%
\pgfsys@defobject{currentmarker}{\pgfqpoint{0.000000in}{-0.048611in}}{\pgfqpoint{0.000000in}{0.000000in}}{%
\pgfpathmoveto{\pgfqpoint{0.000000in}{0.000000in}}%
\pgfpathlineto{\pgfqpoint{0.000000in}{-0.048611in}}%
\pgfusepath{stroke,fill}%
}%
\begin{pgfscope}%
\pgfsys@transformshift{4.390000in}{0.499444in}%
\pgfsys@useobject{currentmarker}{}%
\end{pgfscope}%
\end{pgfscope}%
\begin{pgfscope}%
\definecolor{textcolor}{rgb}{0.000000,0.000000,0.000000}%
\pgfsetstrokecolor{textcolor}%
\pgfsetfillcolor{textcolor}%
\pgftext[x=4.390000in,y=0.402222in,,top]{\color{textcolor}\rmfamily\fontsize{10.000000}{12.000000}\selectfont 1.0}%
\end{pgfscope}%
\begin{pgfscope}%
\definecolor{textcolor}{rgb}{0.000000,0.000000,0.000000}%
\pgfsetstrokecolor{textcolor}%
\pgfsetfillcolor{textcolor}%
\pgftext[x=2.452500in,y=0.223333in,,top]{\color{textcolor}\rmfamily\fontsize{10.000000}{12.000000}\selectfont \(\displaystyle p\)}%
\end{pgfscope}%
\begin{pgfscope}%
\pgfsetbuttcap%
\pgfsetroundjoin%
\definecolor{currentfill}{rgb}{0.000000,0.000000,0.000000}%
\pgfsetfillcolor{currentfill}%
\pgfsetlinewidth{0.803000pt}%
\definecolor{currentstroke}{rgb}{0.000000,0.000000,0.000000}%
\pgfsetstrokecolor{currentstroke}%
\pgfsetdash{}{0pt}%
\pgfsys@defobject{currentmarker}{\pgfqpoint{-0.048611in}{0.000000in}}{\pgfqpoint{-0.000000in}{0.000000in}}{%
\pgfpathmoveto{\pgfqpoint{-0.000000in}{0.000000in}}%
\pgfpathlineto{\pgfqpoint{-0.048611in}{0.000000in}}%
\pgfusepath{stroke,fill}%
}%
\begin{pgfscope}%
\pgfsys@transformshift{0.515000in}{0.499444in}%
\pgfsys@useobject{currentmarker}{}%
\end{pgfscope}%
\end{pgfscope}%
\begin{pgfscope}%
\definecolor{textcolor}{rgb}{0.000000,0.000000,0.000000}%
\pgfsetstrokecolor{textcolor}%
\pgfsetfillcolor{textcolor}%
\pgftext[x=0.348333in, y=0.451250in, left, base]{\color{textcolor}\rmfamily\fontsize{10.000000}{12.000000}\selectfont \(\displaystyle {0}\)}%
\end{pgfscope}%
\begin{pgfscope}%
\pgfsetbuttcap%
\pgfsetroundjoin%
\definecolor{currentfill}{rgb}{0.000000,0.000000,0.000000}%
\pgfsetfillcolor{currentfill}%
\pgfsetlinewidth{0.803000pt}%
\definecolor{currentstroke}{rgb}{0.000000,0.000000,0.000000}%
\pgfsetstrokecolor{currentstroke}%
\pgfsetdash{}{0pt}%
\pgfsys@defobject{currentmarker}{\pgfqpoint{-0.048611in}{0.000000in}}{\pgfqpoint{-0.000000in}{0.000000in}}{%
\pgfpathmoveto{\pgfqpoint{-0.000000in}{0.000000in}}%
\pgfpathlineto{\pgfqpoint{-0.048611in}{0.000000in}}%
\pgfusepath{stroke,fill}%
}%
\begin{pgfscope}%
\pgfsys@transformshift{0.515000in}{0.933391in}%
\pgfsys@useobject{currentmarker}{}%
\end{pgfscope}%
\end{pgfscope}%
\begin{pgfscope}%
\definecolor{textcolor}{rgb}{0.000000,0.000000,0.000000}%
\pgfsetstrokecolor{textcolor}%
\pgfsetfillcolor{textcolor}%
\pgftext[x=0.348333in, y=0.885197in, left, base]{\color{textcolor}\rmfamily\fontsize{10.000000}{12.000000}\selectfont \(\displaystyle {5}\)}%
\end{pgfscope}%
\begin{pgfscope}%
\pgfsetbuttcap%
\pgfsetroundjoin%
\definecolor{currentfill}{rgb}{0.000000,0.000000,0.000000}%
\pgfsetfillcolor{currentfill}%
\pgfsetlinewidth{0.803000pt}%
\definecolor{currentstroke}{rgb}{0.000000,0.000000,0.000000}%
\pgfsetstrokecolor{currentstroke}%
\pgfsetdash{}{0pt}%
\pgfsys@defobject{currentmarker}{\pgfqpoint{-0.048611in}{0.000000in}}{\pgfqpoint{-0.000000in}{0.000000in}}{%
\pgfpathmoveto{\pgfqpoint{-0.000000in}{0.000000in}}%
\pgfpathlineto{\pgfqpoint{-0.048611in}{0.000000in}}%
\pgfusepath{stroke,fill}%
}%
\begin{pgfscope}%
\pgfsys@transformshift{0.515000in}{1.367338in}%
\pgfsys@useobject{currentmarker}{}%
\end{pgfscope}%
\end{pgfscope}%
\begin{pgfscope}%
\definecolor{textcolor}{rgb}{0.000000,0.000000,0.000000}%
\pgfsetstrokecolor{textcolor}%
\pgfsetfillcolor{textcolor}%
\pgftext[x=0.278889in, y=1.319144in, left, base]{\color{textcolor}\rmfamily\fontsize{10.000000}{12.000000}\selectfont \(\displaystyle {10}\)}%
\end{pgfscope}%
\begin{pgfscope}%
\definecolor{textcolor}{rgb}{0.000000,0.000000,0.000000}%
\pgfsetstrokecolor{textcolor}%
\pgfsetfillcolor{textcolor}%
\pgftext[x=0.223333in,y=1.076944in,,bottom,rotate=90.000000]{\color{textcolor}\rmfamily\fontsize{10.000000}{12.000000}\selectfont Percent of Data Set}%
\end{pgfscope}%
\begin{pgfscope}%
\pgfsetrectcap%
\pgfsetmiterjoin%
\pgfsetlinewidth{0.803000pt}%
\definecolor{currentstroke}{rgb}{0.000000,0.000000,0.000000}%
\pgfsetstrokecolor{currentstroke}%
\pgfsetdash{}{0pt}%
\pgfpathmoveto{\pgfqpoint{0.515000in}{0.499444in}}%
\pgfpathlineto{\pgfqpoint{0.515000in}{1.654444in}}%
\pgfusepath{stroke}%
\end{pgfscope}%
\begin{pgfscope}%
\pgfsetrectcap%
\pgfsetmiterjoin%
\pgfsetlinewidth{0.803000pt}%
\definecolor{currentstroke}{rgb}{0.000000,0.000000,0.000000}%
\pgfsetstrokecolor{currentstroke}%
\pgfsetdash{}{0pt}%
\pgfpathmoveto{\pgfqpoint{4.390000in}{0.499444in}}%
\pgfpathlineto{\pgfqpoint{4.390000in}{1.654444in}}%
\pgfusepath{stroke}%
\end{pgfscope}%
\begin{pgfscope}%
\pgfsetrectcap%
\pgfsetmiterjoin%
\pgfsetlinewidth{0.803000pt}%
\definecolor{currentstroke}{rgb}{0.000000,0.000000,0.000000}%
\pgfsetstrokecolor{currentstroke}%
\pgfsetdash{}{0pt}%
\pgfpathmoveto{\pgfqpoint{0.515000in}{0.499444in}}%
\pgfpathlineto{\pgfqpoint{4.390000in}{0.499444in}}%
\pgfusepath{stroke}%
\end{pgfscope}%
\begin{pgfscope}%
\pgfsetrectcap%
\pgfsetmiterjoin%
\pgfsetlinewidth{0.803000pt}%
\definecolor{currentstroke}{rgb}{0.000000,0.000000,0.000000}%
\pgfsetstrokecolor{currentstroke}%
\pgfsetdash{}{0pt}%
\pgfpathmoveto{\pgfqpoint{0.515000in}{1.654444in}}%
\pgfpathlineto{\pgfqpoint{4.390000in}{1.654444in}}%
\pgfusepath{stroke}%
\end{pgfscope}%
\begin{pgfscope}%
\pgfsetbuttcap%
\pgfsetmiterjoin%
\definecolor{currentfill}{rgb}{1.000000,1.000000,1.000000}%
\pgfsetfillcolor{currentfill}%
\pgfsetfillopacity{0.800000}%
\pgfsetlinewidth{1.003750pt}%
\definecolor{currentstroke}{rgb}{0.800000,0.800000,0.800000}%
\pgfsetstrokecolor{currentstroke}%
\pgfsetstrokeopacity{0.800000}%
\pgfsetdash{}{0pt}%
\pgfpathmoveto{\pgfqpoint{3.613056in}{1.154445in}}%
\pgfpathlineto{\pgfqpoint{4.292778in}{1.154445in}}%
\pgfpathquadraticcurveto{\pgfqpoint{4.320556in}{1.154445in}}{\pgfqpoint{4.320556in}{1.182222in}}%
\pgfpathlineto{\pgfqpoint{4.320556in}{1.557222in}}%
\pgfpathquadraticcurveto{\pgfqpoint{4.320556in}{1.585000in}}{\pgfqpoint{4.292778in}{1.585000in}}%
\pgfpathlineto{\pgfqpoint{3.613056in}{1.585000in}}%
\pgfpathquadraticcurveto{\pgfqpoint{3.585278in}{1.585000in}}{\pgfqpoint{3.585278in}{1.557222in}}%
\pgfpathlineto{\pgfqpoint{3.585278in}{1.182222in}}%
\pgfpathquadraticcurveto{\pgfqpoint{3.585278in}{1.154445in}}{\pgfqpoint{3.613056in}{1.154445in}}%
\pgfpathlineto{\pgfqpoint{3.613056in}{1.154445in}}%
\pgfpathclose%
\pgfusepath{stroke,fill}%
\end{pgfscope}%
\begin{pgfscope}%
\pgfsetbuttcap%
\pgfsetmiterjoin%
\pgfsetlinewidth{1.003750pt}%
\definecolor{currentstroke}{rgb}{0.000000,0.000000,0.000000}%
\pgfsetstrokecolor{currentstroke}%
\pgfsetdash{}{0pt}%
\pgfpathmoveto{\pgfqpoint{3.640834in}{1.432222in}}%
\pgfpathlineto{\pgfqpoint{3.918611in}{1.432222in}}%
\pgfpathlineto{\pgfqpoint{3.918611in}{1.529444in}}%
\pgfpathlineto{\pgfqpoint{3.640834in}{1.529444in}}%
\pgfpathlineto{\pgfqpoint{3.640834in}{1.432222in}}%
\pgfpathclose%
\pgfusepath{stroke}%
\end{pgfscope}%
\begin{pgfscope}%
\definecolor{textcolor}{rgb}{0.000000,0.000000,0.000000}%
\pgfsetstrokecolor{textcolor}%
\pgfsetfillcolor{textcolor}%
\pgftext[x=4.029723in,y=1.432222in,left,base]{\color{textcolor}\rmfamily\fontsize{10.000000}{12.000000}\selectfont Neg}%
\end{pgfscope}%
\begin{pgfscope}%
\pgfsetbuttcap%
\pgfsetmiterjoin%
\definecolor{currentfill}{rgb}{0.000000,0.000000,0.000000}%
\pgfsetfillcolor{currentfill}%
\pgfsetlinewidth{0.000000pt}%
\definecolor{currentstroke}{rgb}{0.000000,0.000000,0.000000}%
\pgfsetstrokecolor{currentstroke}%
\pgfsetstrokeopacity{0.000000}%
\pgfsetdash{}{0pt}%
\pgfpathmoveto{\pgfqpoint{3.640834in}{1.236944in}}%
\pgfpathlineto{\pgfqpoint{3.918611in}{1.236944in}}%
\pgfpathlineto{\pgfqpoint{3.918611in}{1.334167in}}%
\pgfpathlineto{\pgfqpoint{3.640834in}{1.334167in}}%
\pgfpathlineto{\pgfqpoint{3.640834in}{1.236944in}}%
\pgfpathclose%
\pgfusepath{fill}%
\end{pgfscope}%
\begin{pgfscope}%
\definecolor{textcolor}{rgb}{0.000000,0.000000,0.000000}%
\pgfsetstrokecolor{textcolor}%
\pgfsetfillcolor{textcolor}%
\pgftext[x=4.029723in,y=1.236944in,left,base]{\color{textcolor}\rmfamily\fontsize{10.000000}{12.000000}\selectfont Pos}%
\end{pgfscope}%
\end{pgfpicture}%
\makeatother%
\endgroup%

&
	\vskip 0pt
	\begin{tabular}{cc|c|c|}
	&\multicolumn{1}{c}{}& \multicolumn{2}{c}{Prediction} \cr
	&\multicolumn{1}{c}{} & \multicolumn{1}{c}{N} & \multicolumn{1}{c}{P} \cr\cline{3-4}
	\multirow{2}{*}{\rotatebox[origin=c]{90}{Actual}}&N &
168,762 & 11,483
	\vrule width 0pt height 10pt depth 2pt \cr\cline{3-4}
	&P & 
22,661 & 11,164
	\vrule width 0pt height 10pt depth 2pt \cr\cline{3-4}
	\end{tabular}

	\hfil\begin{tabular}{ll}
	\cr
0.493 & Precision \cr	0.330 & Recall \cr	0.395 & F1 \cr
\end{tabular}
\end{tabular}

\

Model 2:  $\alpha = 0.67$ for 33\% chance an ambulance is needed

\noindent\begin{tabular}{@{\hspace{-6pt}}p{4.5in} @{\hspace{-3pt}}p{2.0in}}
	\vskip 0pt
	\qquad \qquad Raw Model Output
	
	%% Creator: Matplotlib, PGF backend
%%
%% To include the figure in your LaTeX document, write
%%   \input{<filename>.pgf}
%%
%% Make sure the required packages are loaded in your preamble
%%   \usepackage{pgf}
%%
%% Also ensure that all the required font packages are loaded; for instance,
%% the lmodern package is sometimes necessary when using math font.
%%   \usepackage{lmodern}
%%
%% Figures using additional raster images can only be included by \input if
%% they are in the same directory as the main LaTeX file. For loading figures
%% from other directories you can use the `import` package
%%   \usepackage{import}
%%
%% and then include the figures with
%%   \import{<path to file>}{<filename>.pgf}
%%
%% Matplotlib used the following preamble
%%   
%%   \usepackage{fontspec}
%%   \makeatletter\@ifpackageloaded{underscore}{}{\usepackage[strings]{underscore}}\makeatother
%%
\begingroup%
\makeatletter%
\begin{pgfpicture}%
\pgfpathrectangle{\pgfpointorigin}{\pgfqpoint{4.509306in}{1.754444in}}%
\pgfusepath{use as bounding box, clip}%
\begin{pgfscope}%
\pgfsetbuttcap%
\pgfsetmiterjoin%
\definecolor{currentfill}{rgb}{1.000000,1.000000,1.000000}%
\pgfsetfillcolor{currentfill}%
\pgfsetlinewidth{0.000000pt}%
\definecolor{currentstroke}{rgb}{1.000000,1.000000,1.000000}%
\pgfsetstrokecolor{currentstroke}%
\pgfsetdash{}{0pt}%
\pgfpathmoveto{\pgfqpoint{0.000000in}{0.000000in}}%
\pgfpathlineto{\pgfqpoint{4.509306in}{0.000000in}}%
\pgfpathlineto{\pgfqpoint{4.509306in}{1.754444in}}%
\pgfpathlineto{\pgfqpoint{0.000000in}{1.754444in}}%
\pgfpathlineto{\pgfqpoint{0.000000in}{0.000000in}}%
\pgfpathclose%
\pgfusepath{fill}%
\end{pgfscope}%
\begin{pgfscope}%
\pgfsetbuttcap%
\pgfsetmiterjoin%
\definecolor{currentfill}{rgb}{1.000000,1.000000,1.000000}%
\pgfsetfillcolor{currentfill}%
\pgfsetlinewidth{0.000000pt}%
\definecolor{currentstroke}{rgb}{0.000000,0.000000,0.000000}%
\pgfsetstrokecolor{currentstroke}%
\pgfsetstrokeopacity{0.000000}%
\pgfsetdash{}{0pt}%
\pgfpathmoveto{\pgfqpoint{0.445556in}{0.499444in}}%
\pgfpathlineto{\pgfqpoint{4.320556in}{0.499444in}}%
\pgfpathlineto{\pgfqpoint{4.320556in}{1.654444in}}%
\pgfpathlineto{\pgfqpoint{0.445556in}{1.654444in}}%
\pgfpathlineto{\pgfqpoint{0.445556in}{0.499444in}}%
\pgfpathclose%
\pgfusepath{fill}%
\end{pgfscope}%
\begin{pgfscope}%
\pgfpathrectangle{\pgfqpoint{0.445556in}{0.499444in}}{\pgfqpoint{3.875000in}{1.155000in}}%
\pgfusepath{clip}%
\pgfsetbuttcap%
\pgfsetmiterjoin%
\pgfsetlinewidth{1.003750pt}%
\definecolor{currentstroke}{rgb}{0.000000,0.000000,0.000000}%
\pgfsetstrokecolor{currentstroke}%
\pgfsetdash{}{0pt}%
\pgfpathmoveto{\pgfqpoint{0.435556in}{0.499444in}}%
\pgfpathlineto{\pgfqpoint{0.483922in}{0.499444in}}%
\pgfpathlineto{\pgfqpoint{0.483922in}{1.196060in}}%
\pgfpathlineto{\pgfqpoint{0.435556in}{1.196060in}}%
\pgfusepath{stroke}%
\end{pgfscope}%
\begin{pgfscope}%
\pgfpathrectangle{\pgfqpoint{0.445556in}{0.499444in}}{\pgfqpoint{3.875000in}{1.155000in}}%
\pgfusepath{clip}%
\pgfsetbuttcap%
\pgfsetmiterjoin%
\pgfsetlinewidth{1.003750pt}%
\definecolor{currentstroke}{rgb}{0.000000,0.000000,0.000000}%
\pgfsetstrokecolor{currentstroke}%
\pgfsetdash{}{0pt}%
\pgfpathmoveto{\pgfqpoint{0.576001in}{0.499444in}}%
\pgfpathlineto{\pgfqpoint{0.637387in}{0.499444in}}%
\pgfpathlineto{\pgfqpoint{0.637387in}{1.565132in}}%
\pgfpathlineto{\pgfqpoint{0.576001in}{1.565132in}}%
\pgfpathlineto{\pgfqpoint{0.576001in}{0.499444in}}%
\pgfpathclose%
\pgfusepath{stroke}%
\end{pgfscope}%
\begin{pgfscope}%
\pgfpathrectangle{\pgfqpoint{0.445556in}{0.499444in}}{\pgfqpoint{3.875000in}{1.155000in}}%
\pgfusepath{clip}%
\pgfsetbuttcap%
\pgfsetmiterjoin%
\pgfsetlinewidth{1.003750pt}%
\definecolor{currentstroke}{rgb}{0.000000,0.000000,0.000000}%
\pgfsetstrokecolor{currentstroke}%
\pgfsetdash{}{0pt}%
\pgfpathmoveto{\pgfqpoint{0.729467in}{0.499444in}}%
\pgfpathlineto{\pgfqpoint{0.790853in}{0.499444in}}%
\pgfpathlineto{\pgfqpoint{0.790853in}{1.599444in}}%
\pgfpathlineto{\pgfqpoint{0.729467in}{1.599444in}}%
\pgfpathlineto{\pgfqpoint{0.729467in}{0.499444in}}%
\pgfpathclose%
\pgfusepath{stroke}%
\end{pgfscope}%
\begin{pgfscope}%
\pgfpathrectangle{\pgfqpoint{0.445556in}{0.499444in}}{\pgfqpoint{3.875000in}{1.155000in}}%
\pgfusepath{clip}%
\pgfsetbuttcap%
\pgfsetmiterjoin%
\pgfsetlinewidth{1.003750pt}%
\definecolor{currentstroke}{rgb}{0.000000,0.000000,0.000000}%
\pgfsetstrokecolor{currentstroke}%
\pgfsetdash{}{0pt}%
\pgfpathmoveto{\pgfqpoint{0.882932in}{0.499444in}}%
\pgfpathlineto{\pgfqpoint{0.944318in}{0.499444in}}%
\pgfpathlineto{\pgfqpoint{0.944318in}{1.555920in}}%
\pgfpathlineto{\pgfqpoint{0.882932in}{1.555920in}}%
\pgfpathlineto{\pgfqpoint{0.882932in}{0.499444in}}%
\pgfpathclose%
\pgfusepath{stroke}%
\end{pgfscope}%
\begin{pgfscope}%
\pgfpathrectangle{\pgfqpoint{0.445556in}{0.499444in}}{\pgfqpoint{3.875000in}{1.155000in}}%
\pgfusepath{clip}%
\pgfsetbuttcap%
\pgfsetmiterjoin%
\pgfsetlinewidth{1.003750pt}%
\definecolor{currentstroke}{rgb}{0.000000,0.000000,0.000000}%
\pgfsetstrokecolor{currentstroke}%
\pgfsetdash{}{0pt}%
\pgfpathmoveto{\pgfqpoint{1.036397in}{0.499444in}}%
\pgfpathlineto{\pgfqpoint{1.097783in}{0.499444in}}%
\pgfpathlineto{\pgfqpoint{1.097783in}{1.498579in}}%
\pgfpathlineto{\pgfqpoint{1.036397in}{1.498579in}}%
\pgfpathlineto{\pgfqpoint{1.036397in}{0.499444in}}%
\pgfpathclose%
\pgfusepath{stroke}%
\end{pgfscope}%
\begin{pgfscope}%
\pgfpathrectangle{\pgfqpoint{0.445556in}{0.499444in}}{\pgfqpoint{3.875000in}{1.155000in}}%
\pgfusepath{clip}%
\pgfsetbuttcap%
\pgfsetmiterjoin%
\pgfsetlinewidth{1.003750pt}%
\definecolor{currentstroke}{rgb}{0.000000,0.000000,0.000000}%
\pgfsetstrokecolor{currentstroke}%
\pgfsetdash{}{0pt}%
\pgfpathmoveto{\pgfqpoint{1.189863in}{0.499444in}}%
\pgfpathlineto{\pgfqpoint{1.251249in}{0.499444in}}%
\pgfpathlineto{\pgfqpoint{1.251249in}{1.466723in}}%
\pgfpathlineto{\pgfqpoint{1.189863in}{1.466723in}}%
\pgfpathlineto{\pgfqpoint{1.189863in}{0.499444in}}%
\pgfpathclose%
\pgfusepath{stroke}%
\end{pgfscope}%
\begin{pgfscope}%
\pgfpathrectangle{\pgfqpoint{0.445556in}{0.499444in}}{\pgfqpoint{3.875000in}{1.155000in}}%
\pgfusepath{clip}%
\pgfsetbuttcap%
\pgfsetmiterjoin%
\pgfsetlinewidth{1.003750pt}%
\definecolor{currentstroke}{rgb}{0.000000,0.000000,0.000000}%
\pgfsetstrokecolor{currentstroke}%
\pgfsetdash{}{0pt}%
\pgfpathmoveto{\pgfqpoint{1.343328in}{0.499444in}}%
\pgfpathlineto{\pgfqpoint{1.404714in}{0.499444in}}%
\pgfpathlineto{\pgfqpoint{1.404714in}{1.406695in}}%
\pgfpathlineto{\pgfqpoint{1.343328in}{1.406695in}}%
\pgfpathlineto{\pgfqpoint{1.343328in}{0.499444in}}%
\pgfpathclose%
\pgfusepath{stroke}%
\end{pgfscope}%
\begin{pgfscope}%
\pgfpathrectangle{\pgfqpoint{0.445556in}{0.499444in}}{\pgfqpoint{3.875000in}{1.155000in}}%
\pgfusepath{clip}%
\pgfsetbuttcap%
\pgfsetmiterjoin%
\pgfsetlinewidth{1.003750pt}%
\definecolor{currentstroke}{rgb}{0.000000,0.000000,0.000000}%
\pgfsetstrokecolor{currentstroke}%
\pgfsetdash{}{0pt}%
\pgfpathmoveto{\pgfqpoint{1.496793in}{0.499444in}}%
\pgfpathlineto{\pgfqpoint{1.558179in}{0.499444in}}%
\pgfpathlineto{\pgfqpoint{1.558179in}{1.343289in}}%
\pgfpathlineto{\pgfqpoint{1.496793in}{1.343289in}}%
\pgfpathlineto{\pgfqpoint{1.496793in}{0.499444in}}%
\pgfpathclose%
\pgfusepath{stroke}%
\end{pgfscope}%
\begin{pgfscope}%
\pgfpathrectangle{\pgfqpoint{0.445556in}{0.499444in}}{\pgfqpoint{3.875000in}{1.155000in}}%
\pgfusepath{clip}%
\pgfsetbuttcap%
\pgfsetmiterjoin%
\pgfsetlinewidth{1.003750pt}%
\definecolor{currentstroke}{rgb}{0.000000,0.000000,0.000000}%
\pgfsetstrokecolor{currentstroke}%
\pgfsetdash{}{0pt}%
\pgfpathmoveto{\pgfqpoint{1.650259in}{0.499444in}}%
\pgfpathlineto{\pgfqpoint{1.711645in}{0.499444in}}%
\pgfpathlineto{\pgfqpoint{1.711645in}{1.288097in}}%
\pgfpathlineto{\pgfqpoint{1.650259in}{1.288097in}}%
\pgfpathlineto{\pgfqpoint{1.650259in}{0.499444in}}%
\pgfpathclose%
\pgfusepath{stroke}%
\end{pgfscope}%
\begin{pgfscope}%
\pgfpathrectangle{\pgfqpoint{0.445556in}{0.499444in}}{\pgfqpoint{3.875000in}{1.155000in}}%
\pgfusepath{clip}%
\pgfsetbuttcap%
\pgfsetmiterjoin%
\pgfsetlinewidth{1.003750pt}%
\definecolor{currentstroke}{rgb}{0.000000,0.000000,0.000000}%
\pgfsetstrokecolor{currentstroke}%
\pgfsetdash{}{0pt}%
\pgfpathmoveto{\pgfqpoint{1.803724in}{0.499444in}}%
\pgfpathlineto{\pgfqpoint{1.865110in}{0.499444in}}%
\pgfpathlineto{\pgfqpoint{1.865110in}{1.219549in}}%
\pgfpathlineto{\pgfqpoint{1.803724in}{1.219549in}}%
\pgfpathlineto{\pgfqpoint{1.803724in}{0.499444in}}%
\pgfpathclose%
\pgfusepath{stroke}%
\end{pgfscope}%
\begin{pgfscope}%
\pgfpathrectangle{\pgfqpoint{0.445556in}{0.499444in}}{\pgfqpoint{3.875000in}{1.155000in}}%
\pgfusepath{clip}%
\pgfsetbuttcap%
\pgfsetmiterjoin%
\pgfsetlinewidth{1.003750pt}%
\definecolor{currentstroke}{rgb}{0.000000,0.000000,0.000000}%
\pgfsetstrokecolor{currentstroke}%
\pgfsetdash{}{0pt}%
\pgfpathmoveto{\pgfqpoint{1.957189in}{0.499444in}}%
\pgfpathlineto{\pgfqpoint{2.018575in}{0.499444in}}%
\pgfpathlineto{\pgfqpoint{2.018575in}{1.159751in}}%
\pgfpathlineto{\pgfqpoint{1.957189in}{1.159751in}}%
\pgfpathlineto{\pgfqpoint{1.957189in}{0.499444in}}%
\pgfpathclose%
\pgfusepath{stroke}%
\end{pgfscope}%
\begin{pgfscope}%
\pgfpathrectangle{\pgfqpoint{0.445556in}{0.499444in}}{\pgfqpoint{3.875000in}{1.155000in}}%
\pgfusepath{clip}%
\pgfsetbuttcap%
\pgfsetmiterjoin%
\pgfsetlinewidth{1.003750pt}%
\definecolor{currentstroke}{rgb}{0.000000,0.000000,0.000000}%
\pgfsetstrokecolor{currentstroke}%
\pgfsetdash{}{0pt}%
\pgfpathmoveto{\pgfqpoint{2.110655in}{0.499444in}}%
\pgfpathlineto{\pgfqpoint{2.172041in}{0.499444in}}%
\pgfpathlineto{\pgfqpoint{2.172041in}{1.097420in}}%
\pgfpathlineto{\pgfqpoint{2.110655in}{1.097420in}}%
\pgfpathlineto{\pgfqpoint{2.110655in}{0.499444in}}%
\pgfpathclose%
\pgfusepath{stroke}%
\end{pgfscope}%
\begin{pgfscope}%
\pgfpathrectangle{\pgfqpoint{0.445556in}{0.499444in}}{\pgfqpoint{3.875000in}{1.155000in}}%
\pgfusepath{clip}%
\pgfsetbuttcap%
\pgfsetmiterjoin%
\pgfsetlinewidth{1.003750pt}%
\definecolor{currentstroke}{rgb}{0.000000,0.000000,0.000000}%
\pgfsetstrokecolor{currentstroke}%
\pgfsetdash{}{0pt}%
\pgfpathmoveto{\pgfqpoint{2.264120in}{0.499444in}}%
\pgfpathlineto{\pgfqpoint{2.325506in}{0.499444in}}%
\pgfpathlineto{\pgfqpoint{2.325506in}{1.048523in}}%
\pgfpathlineto{\pgfqpoint{2.264120in}{1.048523in}}%
\pgfpathlineto{\pgfqpoint{2.264120in}{0.499444in}}%
\pgfpathclose%
\pgfusepath{stroke}%
\end{pgfscope}%
\begin{pgfscope}%
\pgfpathrectangle{\pgfqpoint{0.445556in}{0.499444in}}{\pgfqpoint{3.875000in}{1.155000in}}%
\pgfusepath{clip}%
\pgfsetbuttcap%
\pgfsetmiterjoin%
\pgfsetlinewidth{1.003750pt}%
\definecolor{currentstroke}{rgb}{0.000000,0.000000,0.000000}%
\pgfsetstrokecolor{currentstroke}%
\pgfsetdash{}{0pt}%
\pgfpathmoveto{\pgfqpoint{2.417585in}{0.499444in}}%
\pgfpathlineto{\pgfqpoint{2.478972in}{0.499444in}}%
\pgfpathlineto{\pgfqpoint{2.478972in}{0.993178in}}%
\pgfpathlineto{\pgfqpoint{2.417585in}{0.993178in}}%
\pgfpathlineto{\pgfqpoint{2.417585in}{0.499444in}}%
\pgfpathclose%
\pgfusepath{stroke}%
\end{pgfscope}%
\begin{pgfscope}%
\pgfpathrectangle{\pgfqpoint{0.445556in}{0.499444in}}{\pgfqpoint{3.875000in}{1.155000in}}%
\pgfusepath{clip}%
\pgfsetbuttcap%
\pgfsetmiterjoin%
\pgfsetlinewidth{1.003750pt}%
\definecolor{currentstroke}{rgb}{0.000000,0.000000,0.000000}%
\pgfsetstrokecolor{currentstroke}%
\pgfsetdash{}{0pt}%
\pgfpathmoveto{\pgfqpoint{2.571051in}{0.499444in}}%
\pgfpathlineto{\pgfqpoint{2.632437in}{0.499444in}}%
\pgfpathlineto{\pgfqpoint{2.632437in}{0.958097in}}%
\pgfpathlineto{\pgfqpoint{2.571051in}{0.958097in}}%
\pgfpathlineto{\pgfqpoint{2.571051in}{0.499444in}}%
\pgfpathclose%
\pgfusepath{stroke}%
\end{pgfscope}%
\begin{pgfscope}%
\pgfpathrectangle{\pgfqpoint{0.445556in}{0.499444in}}{\pgfqpoint{3.875000in}{1.155000in}}%
\pgfusepath{clip}%
\pgfsetbuttcap%
\pgfsetmiterjoin%
\pgfsetlinewidth{1.003750pt}%
\definecolor{currentstroke}{rgb}{0.000000,0.000000,0.000000}%
\pgfsetstrokecolor{currentstroke}%
\pgfsetdash{}{0pt}%
\pgfpathmoveto{\pgfqpoint{2.724516in}{0.499444in}}%
\pgfpathlineto{\pgfqpoint{2.785902in}{0.499444in}}%
\pgfpathlineto{\pgfqpoint{2.785902in}{0.906513in}}%
\pgfpathlineto{\pgfqpoint{2.724516in}{0.906513in}}%
\pgfpathlineto{\pgfqpoint{2.724516in}{0.499444in}}%
\pgfpathclose%
\pgfusepath{stroke}%
\end{pgfscope}%
\begin{pgfscope}%
\pgfpathrectangle{\pgfqpoint{0.445556in}{0.499444in}}{\pgfqpoint{3.875000in}{1.155000in}}%
\pgfusepath{clip}%
\pgfsetbuttcap%
\pgfsetmiterjoin%
\pgfsetlinewidth{1.003750pt}%
\definecolor{currentstroke}{rgb}{0.000000,0.000000,0.000000}%
\pgfsetstrokecolor{currentstroke}%
\pgfsetdash{}{0pt}%
\pgfpathmoveto{\pgfqpoint{2.877981in}{0.499444in}}%
\pgfpathlineto{\pgfqpoint{2.939368in}{0.499444in}}%
\pgfpathlineto{\pgfqpoint{2.939368in}{0.860379in}}%
\pgfpathlineto{\pgfqpoint{2.877981in}{0.860379in}}%
\pgfpathlineto{\pgfqpoint{2.877981in}{0.499444in}}%
\pgfpathclose%
\pgfusepath{stroke}%
\end{pgfscope}%
\begin{pgfscope}%
\pgfpathrectangle{\pgfqpoint{0.445556in}{0.499444in}}{\pgfqpoint{3.875000in}{1.155000in}}%
\pgfusepath{clip}%
\pgfsetbuttcap%
\pgfsetmiterjoin%
\pgfsetlinewidth{1.003750pt}%
\definecolor{currentstroke}{rgb}{0.000000,0.000000,0.000000}%
\pgfsetstrokecolor{currentstroke}%
\pgfsetdash{}{0pt}%
\pgfpathmoveto{\pgfqpoint{3.031447in}{0.499444in}}%
\pgfpathlineto{\pgfqpoint{3.092833in}{0.499444in}}%
\pgfpathlineto{\pgfqpoint{3.092833in}{0.810791in}}%
\pgfpathlineto{\pgfqpoint{3.031447in}{0.810791in}}%
\pgfpathlineto{\pgfqpoint{3.031447in}{0.499444in}}%
\pgfpathclose%
\pgfusepath{stroke}%
\end{pgfscope}%
\begin{pgfscope}%
\pgfpathrectangle{\pgfqpoint{0.445556in}{0.499444in}}{\pgfqpoint{3.875000in}{1.155000in}}%
\pgfusepath{clip}%
\pgfsetbuttcap%
\pgfsetmiterjoin%
\pgfsetlinewidth{1.003750pt}%
\definecolor{currentstroke}{rgb}{0.000000,0.000000,0.000000}%
\pgfsetstrokecolor{currentstroke}%
\pgfsetdash{}{0pt}%
\pgfpathmoveto{\pgfqpoint{3.184912in}{0.499444in}}%
\pgfpathlineto{\pgfqpoint{3.246298in}{0.499444in}}%
\pgfpathlineto{\pgfqpoint{3.246298in}{0.746234in}}%
\pgfpathlineto{\pgfqpoint{3.184912in}{0.746234in}}%
\pgfpathlineto{\pgfqpoint{3.184912in}{0.499444in}}%
\pgfpathclose%
\pgfusepath{stroke}%
\end{pgfscope}%
\begin{pgfscope}%
\pgfpathrectangle{\pgfqpoint{0.445556in}{0.499444in}}{\pgfqpoint{3.875000in}{1.155000in}}%
\pgfusepath{clip}%
\pgfsetbuttcap%
\pgfsetmiterjoin%
\pgfsetlinewidth{1.003750pt}%
\definecolor{currentstroke}{rgb}{0.000000,0.000000,0.000000}%
\pgfsetstrokecolor{currentstroke}%
\pgfsetdash{}{0pt}%
\pgfpathmoveto{\pgfqpoint{3.338377in}{0.499444in}}%
\pgfpathlineto{\pgfqpoint{3.399764in}{0.499444in}}%
\pgfpathlineto{\pgfqpoint{3.399764in}{0.706318in}}%
\pgfpathlineto{\pgfqpoint{3.338377in}{0.706318in}}%
\pgfpathlineto{\pgfqpoint{3.338377in}{0.499444in}}%
\pgfpathclose%
\pgfusepath{stroke}%
\end{pgfscope}%
\begin{pgfscope}%
\pgfpathrectangle{\pgfqpoint{0.445556in}{0.499444in}}{\pgfqpoint{3.875000in}{1.155000in}}%
\pgfusepath{clip}%
\pgfsetbuttcap%
\pgfsetmiterjoin%
\pgfsetlinewidth{1.003750pt}%
\definecolor{currentstroke}{rgb}{0.000000,0.000000,0.000000}%
\pgfsetstrokecolor{currentstroke}%
\pgfsetdash{}{0pt}%
\pgfpathmoveto{\pgfqpoint{3.491843in}{0.499444in}}%
\pgfpathlineto{\pgfqpoint{3.553229in}{0.499444in}}%
\pgfpathlineto{\pgfqpoint{3.553229in}{0.652661in}}%
\pgfpathlineto{\pgfqpoint{3.491843in}{0.652661in}}%
\pgfpathlineto{\pgfqpoint{3.491843in}{0.499444in}}%
\pgfpathclose%
\pgfusepath{stroke}%
\end{pgfscope}%
\begin{pgfscope}%
\pgfpathrectangle{\pgfqpoint{0.445556in}{0.499444in}}{\pgfqpoint{3.875000in}{1.155000in}}%
\pgfusepath{clip}%
\pgfsetbuttcap%
\pgfsetmiterjoin%
\pgfsetlinewidth{1.003750pt}%
\definecolor{currentstroke}{rgb}{0.000000,0.000000,0.000000}%
\pgfsetstrokecolor{currentstroke}%
\pgfsetdash{}{0pt}%
\pgfpathmoveto{\pgfqpoint{3.645308in}{0.499444in}}%
\pgfpathlineto{\pgfqpoint{3.706694in}{0.499444in}}%
\pgfpathlineto{\pgfqpoint{3.706694in}{0.605222in}}%
\pgfpathlineto{\pgfqpoint{3.645308in}{0.605222in}}%
\pgfpathlineto{\pgfqpoint{3.645308in}{0.499444in}}%
\pgfpathclose%
\pgfusepath{stroke}%
\end{pgfscope}%
\begin{pgfscope}%
\pgfpathrectangle{\pgfqpoint{0.445556in}{0.499444in}}{\pgfqpoint{3.875000in}{1.155000in}}%
\pgfusepath{clip}%
\pgfsetbuttcap%
\pgfsetmiterjoin%
\pgfsetlinewidth{1.003750pt}%
\definecolor{currentstroke}{rgb}{0.000000,0.000000,0.000000}%
\pgfsetstrokecolor{currentstroke}%
\pgfsetdash{}{0pt}%
\pgfpathmoveto{\pgfqpoint{3.798774in}{0.499444in}}%
\pgfpathlineto{\pgfqpoint{3.860160in}{0.499444in}}%
\pgfpathlineto{\pgfqpoint{3.860160in}{0.570142in}}%
\pgfpathlineto{\pgfqpoint{3.798774in}{0.570142in}}%
\pgfpathlineto{\pgfqpoint{3.798774in}{0.499444in}}%
\pgfpathclose%
\pgfusepath{stroke}%
\end{pgfscope}%
\begin{pgfscope}%
\pgfpathrectangle{\pgfqpoint{0.445556in}{0.499444in}}{\pgfqpoint{3.875000in}{1.155000in}}%
\pgfusepath{clip}%
\pgfsetbuttcap%
\pgfsetmiterjoin%
\pgfsetlinewidth{1.003750pt}%
\definecolor{currentstroke}{rgb}{0.000000,0.000000,0.000000}%
\pgfsetstrokecolor{currentstroke}%
\pgfsetdash{}{0pt}%
\pgfpathmoveto{\pgfqpoint{3.952239in}{0.499444in}}%
\pgfpathlineto{\pgfqpoint{4.013625in}{0.499444in}}%
\pgfpathlineto{\pgfqpoint{4.013625in}{0.554943in}}%
\pgfpathlineto{\pgfqpoint{3.952239in}{0.554943in}}%
\pgfpathlineto{\pgfqpoint{3.952239in}{0.499444in}}%
\pgfpathclose%
\pgfusepath{stroke}%
\end{pgfscope}%
\begin{pgfscope}%
\pgfpathrectangle{\pgfqpoint{0.445556in}{0.499444in}}{\pgfqpoint{3.875000in}{1.155000in}}%
\pgfusepath{clip}%
\pgfsetbuttcap%
\pgfsetmiterjoin%
\pgfsetlinewidth{1.003750pt}%
\definecolor{currentstroke}{rgb}{0.000000,0.000000,0.000000}%
\pgfsetstrokecolor{currentstroke}%
\pgfsetdash{}{0pt}%
\pgfpathmoveto{\pgfqpoint{4.105704in}{0.499444in}}%
\pgfpathlineto{\pgfqpoint{4.167090in}{0.499444in}}%
\pgfpathlineto{\pgfqpoint{4.167090in}{0.512417in}}%
\pgfpathlineto{\pgfqpoint{4.105704in}{0.512417in}}%
\pgfpathlineto{\pgfqpoint{4.105704in}{0.499444in}}%
\pgfpathclose%
\pgfusepath{stroke}%
\end{pgfscope}%
\begin{pgfscope}%
\pgfpathrectangle{\pgfqpoint{0.445556in}{0.499444in}}{\pgfqpoint{3.875000in}{1.155000in}}%
\pgfusepath{clip}%
\pgfsetbuttcap%
\pgfsetmiterjoin%
\definecolor{currentfill}{rgb}{0.000000,0.000000,0.000000}%
\pgfsetfillcolor{currentfill}%
\pgfsetlinewidth{0.000000pt}%
\definecolor{currentstroke}{rgb}{0.000000,0.000000,0.000000}%
\pgfsetstrokecolor{currentstroke}%
\pgfsetstrokeopacity{0.000000}%
\pgfsetdash{}{0pt}%
\pgfpathmoveto{\pgfqpoint{0.483922in}{0.499444in}}%
\pgfpathlineto{\pgfqpoint{0.545308in}{0.499444in}}%
\pgfpathlineto{\pgfqpoint{0.545308in}{0.506506in}}%
\pgfpathlineto{\pgfqpoint{0.483922in}{0.506506in}}%
\pgfpathlineto{\pgfqpoint{0.483922in}{0.499444in}}%
\pgfpathclose%
\pgfusepath{fill}%
\end{pgfscope}%
\begin{pgfscope}%
\pgfpathrectangle{\pgfqpoint{0.445556in}{0.499444in}}{\pgfqpoint{3.875000in}{1.155000in}}%
\pgfusepath{clip}%
\pgfsetbuttcap%
\pgfsetmiterjoin%
\definecolor{currentfill}{rgb}{0.000000,0.000000,0.000000}%
\pgfsetfillcolor{currentfill}%
\pgfsetlinewidth{0.000000pt}%
\definecolor{currentstroke}{rgb}{0.000000,0.000000,0.000000}%
\pgfsetstrokecolor{currentstroke}%
\pgfsetstrokeopacity{0.000000}%
\pgfsetdash{}{0pt}%
\pgfpathmoveto{\pgfqpoint{0.637387in}{0.499444in}}%
\pgfpathlineto{\pgfqpoint{0.698774in}{0.499444in}}%
\pgfpathlineto{\pgfqpoint{0.698774in}{0.518865in}}%
\pgfpathlineto{\pgfqpoint{0.637387in}{0.518865in}}%
\pgfpathlineto{\pgfqpoint{0.637387in}{0.499444in}}%
\pgfpathclose%
\pgfusepath{fill}%
\end{pgfscope}%
\begin{pgfscope}%
\pgfpathrectangle{\pgfqpoint{0.445556in}{0.499444in}}{\pgfqpoint{3.875000in}{1.155000in}}%
\pgfusepath{clip}%
\pgfsetbuttcap%
\pgfsetmiterjoin%
\definecolor{currentfill}{rgb}{0.000000,0.000000,0.000000}%
\pgfsetfillcolor{currentfill}%
\pgfsetlinewidth{0.000000pt}%
\definecolor{currentstroke}{rgb}{0.000000,0.000000,0.000000}%
\pgfsetstrokecolor{currentstroke}%
\pgfsetstrokeopacity{0.000000}%
\pgfsetdash{}{0pt}%
\pgfpathmoveto{\pgfqpoint{0.790853in}{0.499444in}}%
\pgfpathlineto{\pgfqpoint{0.852239in}{0.499444in}}%
\pgfpathlineto{\pgfqpoint{0.852239in}{0.538593in}}%
\pgfpathlineto{\pgfqpoint{0.790853in}{0.538593in}}%
\pgfpathlineto{\pgfqpoint{0.790853in}{0.499444in}}%
\pgfpathclose%
\pgfusepath{fill}%
\end{pgfscope}%
\begin{pgfscope}%
\pgfpathrectangle{\pgfqpoint{0.445556in}{0.499444in}}{\pgfqpoint{3.875000in}{1.155000in}}%
\pgfusepath{clip}%
\pgfsetbuttcap%
\pgfsetmiterjoin%
\definecolor{currentfill}{rgb}{0.000000,0.000000,0.000000}%
\pgfsetfillcolor{currentfill}%
\pgfsetlinewidth{0.000000pt}%
\definecolor{currentstroke}{rgb}{0.000000,0.000000,0.000000}%
\pgfsetstrokecolor{currentstroke}%
\pgfsetstrokeopacity{0.000000}%
\pgfsetdash{}{0pt}%
\pgfpathmoveto{\pgfqpoint{0.944318in}{0.499444in}}%
\pgfpathlineto{\pgfqpoint{1.005704in}{0.499444in}}%
\pgfpathlineto{\pgfqpoint{1.005704in}{0.551105in}}%
\pgfpathlineto{\pgfqpoint{0.944318in}{0.551105in}}%
\pgfpathlineto{\pgfqpoint{0.944318in}{0.499444in}}%
\pgfpathclose%
\pgfusepath{fill}%
\end{pgfscope}%
\begin{pgfscope}%
\pgfpathrectangle{\pgfqpoint{0.445556in}{0.499444in}}{\pgfqpoint{3.875000in}{1.155000in}}%
\pgfusepath{clip}%
\pgfsetbuttcap%
\pgfsetmiterjoin%
\definecolor{currentfill}{rgb}{0.000000,0.000000,0.000000}%
\pgfsetfillcolor{currentfill}%
\pgfsetlinewidth{0.000000pt}%
\definecolor{currentstroke}{rgb}{0.000000,0.000000,0.000000}%
\pgfsetstrokecolor{currentstroke}%
\pgfsetstrokeopacity{0.000000}%
\pgfsetdash{}{0pt}%
\pgfpathmoveto{\pgfqpoint{1.097783in}{0.499444in}}%
\pgfpathlineto{\pgfqpoint{1.159170in}{0.499444in}}%
\pgfpathlineto{\pgfqpoint{1.159170in}{0.562082in}}%
\pgfpathlineto{\pgfqpoint{1.097783in}{0.562082in}}%
\pgfpathlineto{\pgfqpoint{1.097783in}{0.499444in}}%
\pgfpathclose%
\pgfusepath{fill}%
\end{pgfscope}%
\begin{pgfscope}%
\pgfpathrectangle{\pgfqpoint{0.445556in}{0.499444in}}{\pgfqpoint{3.875000in}{1.155000in}}%
\pgfusepath{clip}%
\pgfsetbuttcap%
\pgfsetmiterjoin%
\definecolor{currentfill}{rgb}{0.000000,0.000000,0.000000}%
\pgfsetfillcolor{currentfill}%
\pgfsetlinewidth{0.000000pt}%
\definecolor{currentstroke}{rgb}{0.000000,0.000000,0.000000}%
\pgfsetstrokecolor{currentstroke}%
\pgfsetstrokeopacity{0.000000}%
\pgfsetdash{}{0pt}%
\pgfpathmoveto{\pgfqpoint{1.251249in}{0.499444in}}%
\pgfpathlineto{\pgfqpoint{1.312635in}{0.499444in}}%
\pgfpathlineto{\pgfqpoint{1.312635in}{0.576590in}}%
\pgfpathlineto{\pgfqpoint{1.251249in}{0.576590in}}%
\pgfpathlineto{\pgfqpoint{1.251249in}{0.499444in}}%
\pgfpathclose%
\pgfusepath{fill}%
\end{pgfscope}%
\begin{pgfscope}%
\pgfpathrectangle{\pgfqpoint{0.445556in}{0.499444in}}{\pgfqpoint{3.875000in}{1.155000in}}%
\pgfusepath{clip}%
\pgfsetbuttcap%
\pgfsetmiterjoin%
\definecolor{currentfill}{rgb}{0.000000,0.000000,0.000000}%
\pgfsetfillcolor{currentfill}%
\pgfsetlinewidth{0.000000pt}%
\definecolor{currentstroke}{rgb}{0.000000,0.000000,0.000000}%
\pgfsetstrokecolor{currentstroke}%
\pgfsetstrokeopacity{0.000000}%
\pgfsetdash{}{0pt}%
\pgfpathmoveto{\pgfqpoint{1.404714in}{0.499444in}}%
\pgfpathlineto{\pgfqpoint{1.466100in}{0.499444in}}%
\pgfpathlineto{\pgfqpoint{1.466100in}{0.587874in}}%
\pgfpathlineto{\pgfqpoint{1.404714in}{0.587874in}}%
\pgfpathlineto{\pgfqpoint{1.404714in}{0.499444in}}%
\pgfpathclose%
\pgfusepath{fill}%
\end{pgfscope}%
\begin{pgfscope}%
\pgfpathrectangle{\pgfqpoint{0.445556in}{0.499444in}}{\pgfqpoint{3.875000in}{1.155000in}}%
\pgfusepath{clip}%
\pgfsetbuttcap%
\pgfsetmiterjoin%
\definecolor{currentfill}{rgb}{0.000000,0.000000,0.000000}%
\pgfsetfillcolor{currentfill}%
\pgfsetlinewidth{0.000000pt}%
\definecolor{currentstroke}{rgb}{0.000000,0.000000,0.000000}%
\pgfsetstrokecolor{currentstroke}%
\pgfsetstrokeopacity{0.000000}%
\pgfsetdash{}{0pt}%
\pgfpathmoveto{\pgfqpoint{1.558179in}{0.499444in}}%
\pgfpathlineto{\pgfqpoint{1.619566in}{0.499444in}}%
\pgfpathlineto{\pgfqpoint{1.619566in}{0.598621in}}%
\pgfpathlineto{\pgfqpoint{1.558179in}{0.598621in}}%
\pgfpathlineto{\pgfqpoint{1.558179in}{0.499444in}}%
\pgfpathclose%
\pgfusepath{fill}%
\end{pgfscope}%
\begin{pgfscope}%
\pgfpathrectangle{\pgfqpoint{0.445556in}{0.499444in}}{\pgfqpoint{3.875000in}{1.155000in}}%
\pgfusepath{clip}%
\pgfsetbuttcap%
\pgfsetmiterjoin%
\definecolor{currentfill}{rgb}{0.000000,0.000000,0.000000}%
\pgfsetfillcolor{currentfill}%
\pgfsetlinewidth{0.000000pt}%
\definecolor{currentstroke}{rgb}{0.000000,0.000000,0.000000}%
\pgfsetstrokecolor{currentstroke}%
\pgfsetstrokeopacity{0.000000}%
\pgfsetdash{}{0pt}%
\pgfpathmoveto{\pgfqpoint{1.711645in}{0.499444in}}%
\pgfpathlineto{\pgfqpoint{1.773031in}{0.499444in}}%
\pgfpathlineto{\pgfqpoint{1.773031in}{0.601307in}}%
\pgfpathlineto{\pgfqpoint{1.711645in}{0.601307in}}%
\pgfpathlineto{\pgfqpoint{1.711645in}{0.499444in}}%
\pgfpathclose%
\pgfusepath{fill}%
\end{pgfscope}%
\begin{pgfscope}%
\pgfpathrectangle{\pgfqpoint{0.445556in}{0.499444in}}{\pgfqpoint{3.875000in}{1.155000in}}%
\pgfusepath{clip}%
\pgfsetbuttcap%
\pgfsetmiterjoin%
\definecolor{currentfill}{rgb}{0.000000,0.000000,0.000000}%
\pgfsetfillcolor{currentfill}%
\pgfsetlinewidth{0.000000pt}%
\definecolor{currentstroke}{rgb}{0.000000,0.000000,0.000000}%
\pgfsetstrokecolor{currentstroke}%
\pgfsetstrokeopacity{0.000000}%
\pgfsetdash{}{0pt}%
\pgfpathmoveto{\pgfqpoint{1.865110in}{0.499444in}}%
\pgfpathlineto{\pgfqpoint{1.926496in}{0.499444in}}%
\pgfpathlineto{\pgfqpoint{1.926496in}{0.612668in}}%
\pgfpathlineto{\pgfqpoint{1.865110in}{0.612668in}}%
\pgfpathlineto{\pgfqpoint{1.865110in}{0.499444in}}%
\pgfpathclose%
\pgfusepath{fill}%
\end{pgfscope}%
\begin{pgfscope}%
\pgfpathrectangle{\pgfqpoint{0.445556in}{0.499444in}}{\pgfqpoint{3.875000in}{1.155000in}}%
\pgfusepath{clip}%
\pgfsetbuttcap%
\pgfsetmiterjoin%
\definecolor{currentfill}{rgb}{0.000000,0.000000,0.000000}%
\pgfsetfillcolor{currentfill}%
\pgfsetlinewidth{0.000000pt}%
\definecolor{currentstroke}{rgb}{0.000000,0.000000,0.000000}%
\pgfsetstrokecolor{currentstroke}%
\pgfsetstrokeopacity{0.000000}%
\pgfsetdash{}{0pt}%
\pgfpathmoveto{\pgfqpoint{2.018575in}{0.499444in}}%
\pgfpathlineto{\pgfqpoint{2.079962in}{0.499444in}}%
\pgfpathlineto{\pgfqpoint{2.079962in}{0.620037in}}%
\pgfpathlineto{\pgfqpoint{2.018575in}{0.620037in}}%
\pgfpathlineto{\pgfqpoint{2.018575in}{0.499444in}}%
\pgfpathclose%
\pgfusepath{fill}%
\end{pgfscope}%
\begin{pgfscope}%
\pgfpathrectangle{\pgfqpoint{0.445556in}{0.499444in}}{\pgfqpoint{3.875000in}{1.155000in}}%
\pgfusepath{clip}%
\pgfsetbuttcap%
\pgfsetmiterjoin%
\definecolor{currentfill}{rgb}{0.000000,0.000000,0.000000}%
\pgfsetfillcolor{currentfill}%
\pgfsetlinewidth{0.000000pt}%
\definecolor{currentstroke}{rgb}{0.000000,0.000000,0.000000}%
\pgfsetstrokecolor{currentstroke}%
\pgfsetstrokeopacity{0.000000}%
\pgfsetdash{}{0pt}%
\pgfpathmoveto{\pgfqpoint{2.172041in}{0.499444in}}%
\pgfpathlineto{\pgfqpoint{2.233427in}{0.499444in}}%
\pgfpathlineto{\pgfqpoint{2.233427in}{0.625180in}}%
\pgfpathlineto{\pgfqpoint{2.172041in}{0.625180in}}%
\pgfpathlineto{\pgfqpoint{2.172041in}{0.499444in}}%
\pgfpathclose%
\pgfusepath{fill}%
\end{pgfscope}%
\begin{pgfscope}%
\pgfpathrectangle{\pgfqpoint{0.445556in}{0.499444in}}{\pgfqpoint{3.875000in}{1.155000in}}%
\pgfusepath{clip}%
\pgfsetbuttcap%
\pgfsetmiterjoin%
\definecolor{currentfill}{rgb}{0.000000,0.000000,0.000000}%
\pgfsetfillcolor{currentfill}%
\pgfsetlinewidth{0.000000pt}%
\definecolor{currentstroke}{rgb}{0.000000,0.000000,0.000000}%
\pgfsetstrokecolor{currentstroke}%
\pgfsetstrokeopacity{0.000000}%
\pgfsetdash{}{0pt}%
\pgfpathmoveto{\pgfqpoint{2.325506in}{0.499444in}}%
\pgfpathlineto{\pgfqpoint{2.386892in}{0.499444in}}%
\pgfpathlineto{\pgfqpoint{2.386892in}{0.628251in}}%
\pgfpathlineto{\pgfqpoint{2.325506in}{0.628251in}}%
\pgfpathlineto{\pgfqpoint{2.325506in}{0.499444in}}%
\pgfpathclose%
\pgfusepath{fill}%
\end{pgfscope}%
\begin{pgfscope}%
\pgfpathrectangle{\pgfqpoint{0.445556in}{0.499444in}}{\pgfqpoint{3.875000in}{1.155000in}}%
\pgfusepath{clip}%
\pgfsetbuttcap%
\pgfsetmiterjoin%
\definecolor{currentfill}{rgb}{0.000000,0.000000,0.000000}%
\pgfsetfillcolor{currentfill}%
\pgfsetlinewidth{0.000000pt}%
\definecolor{currentstroke}{rgb}{0.000000,0.000000,0.000000}%
\pgfsetstrokecolor{currentstroke}%
\pgfsetstrokeopacity{0.000000}%
\pgfsetdash{}{0pt}%
\pgfpathmoveto{\pgfqpoint{2.478972in}{0.499444in}}%
\pgfpathlineto{\pgfqpoint{2.540358in}{0.499444in}}%
\pgfpathlineto{\pgfqpoint{2.540358in}{0.637616in}}%
\pgfpathlineto{\pgfqpoint{2.478972in}{0.637616in}}%
\pgfpathlineto{\pgfqpoint{2.478972in}{0.499444in}}%
\pgfpathclose%
\pgfusepath{fill}%
\end{pgfscope}%
\begin{pgfscope}%
\pgfpathrectangle{\pgfqpoint{0.445556in}{0.499444in}}{\pgfqpoint{3.875000in}{1.155000in}}%
\pgfusepath{clip}%
\pgfsetbuttcap%
\pgfsetmiterjoin%
\definecolor{currentfill}{rgb}{0.000000,0.000000,0.000000}%
\pgfsetfillcolor{currentfill}%
\pgfsetlinewidth{0.000000pt}%
\definecolor{currentstroke}{rgb}{0.000000,0.000000,0.000000}%
\pgfsetstrokecolor{currentstroke}%
\pgfsetstrokeopacity{0.000000}%
\pgfsetdash{}{0pt}%
\pgfpathmoveto{\pgfqpoint{2.632437in}{0.499444in}}%
\pgfpathlineto{\pgfqpoint{2.693823in}{0.499444in}}%
\pgfpathlineto{\pgfqpoint{2.693823in}{0.638537in}}%
\pgfpathlineto{\pgfqpoint{2.632437in}{0.638537in}}%
\pgfpathlineto{\pgfqpoint{2.632437in}{0.499444in}}%
\pgfpathclose%
\pgfusepath{fill}%
\end{pgfscope}%
\begin{pgfscope}%
\pgfpathrectangle{\pgfqpoint{0.445556in}{0.499444in}}{\pgfqpoint{3.875000in}{1.155000in}}%
\pgfusepath{clip}%
\pgfsetbuttcap%
\pgfsetmiterjoin%
\definecolor{currentfill}{rgb}{0.000000,0.000000,0.000000}%
\pgfsetfillcolor{currentfill}%
\pgfsetlinewidth{0.000000pt}%
\definecolor{currentstroke}{rgb}{0.000000,0.000000,0.000000}%
\pgfsetstrokecolor{currentstroke}%
\pgfsetstrokeopacity{0.000000}%
\pgfsetdash{}{0pt}%
\pgfpathmoveto{\pgfqpoint{2.785902in}{0.499444in}}%
\pgfpathlineto{\pgfqpoint{2.847288in}{0.499444in}}%
\pgfpathlineto{\pgfqpoint{2.847288in}{0.640840in}}%
\pgfpathlineto{\pgfqpoint{2.785902in}{0.640840in}}%
\pgfpathlineto{\pgfqpoint{2.785902in}{0.499444in}}%
\pgfpathclose%
\pgfusepath{fill}%
\end{pgfscope}%
\begin{pgfscope}%
\pgfpathrectangle{\pgfqpoint{0.445556in}{0.499444in}}{\pgfqpoint{3.875000in}{1.155000in}}%
\pgfusepath{clip}%
\pgfsetbuttcap%
\pgfsetmiterjoin%
\definecolor{currentfill}{rgb}{0.000000,0.000000,0.000000}%
\pgfsetfillcolor{currentfill}%
\pgfsetlinewidth{0.000000pt}%
\definecolor{currentstroke}{rgb}{0.000000,0.000000,0.000000}%
\pgfsetstrokecolor{currentstroke}%
\pgfsetstrokeopacity{0.000000}%
\pgfsetdash{}{0pt}%
\pgfpathmoveto{\pgfqpoint{2.939368in}{0.499444in}}%
\pgfpathlineto{\pgfqpoint{3.000754in}{0.499444in}}%
\pgfpathlineto{\pgfqpoint{3.000754in}{0.650589in}}%
\pgfpathlineto{\pgfqpoint{2.939368in}{0.650589in}}%
\pgfpathlineto{\pgfqpoint{2.939368in}{0.499444in}}%
\pgfpathclose%
\pgfusepath{fill}%
\end{pgfscope}%
\begin{pgfscope}%
\pgfpathrectangle{\pgfqpoint{0.445556in}{0.499444in}}{\pgfqpoint{3.875000in}{1.155000in}}%
\pgfusepath{clip}%
\pgfsetbuttcap%
\pgfsetmiterjoin%
\definecolor{currentfill}{rgb}{0.000000,0.000000,0.000000}%
\pgfsetfillcolor{currentfill}%
\pgfsetlinewidth{0.000000pt}%
\definecolor{currentstroke}{rgb}{0.000000,0.000000,0.000000}%
\pgfsetstrokecolor{currentstroke}%
\pgfsetstrokeopacity{0.000000}%
\pgfsetdash{}{0pt}%
\pgfpathmoveto{\pgfqpoint{3.092833in}{0.499444in}}%
\pgfpathlineto{\pgfqpoint{3.154219in}{0.499444in}}%
\pgfpathlineto{\pgfqpoint{3.154219in}{0.646367in}}%
\pgfpathlineto{\pgfqpoint{3.092833in}{0.646367in}}%
\pgfpathlineto{\pgfqpoint{3.092833in}{0.499444in}}%
\pgfpathclose%
\pgfusepath{fill}%
\end{pgfscope}%
\begin{pgfscope}%
\pgfpathrectangle{\pgfqpoint{0.445556in}{0.499444in}}{\pgfqpoint{3.875000in}{1.155000in}}%
\pgfusepath{clip}%
\pgfsetbuttcap%
\pgfsetmiterjoin%
\definecolor{currentfill}{rgb}{0.000000,0.000000,0.000000}%
\pgfsetfillcolor{currentfill}%
\pgfsetlinewidth{0.000000pt}%
\definecolor{currentstroke}{rgb}{0.000000,0.000000,0.000000}%
\pgfsetstrokecolor{currentstroke}%
\pgfsetstrokeopacity{0.000000}%
\pgfsetdash{}{0pt}%
\pgfpathmoveto{\pgfqpoint{3.246298in}{0.499444in}}%
\pgfpathlineto{\pgfqpoint{3.307684in}{0.499444in}}%
\pgfpathlineto{\pgfqpoint{3.307684in}{0.650051in}}%
\pgfpathlineto{\pgfqpoint{3.246298in}{0.650051in}}%
\pgfpathlineto{\pgfqpoint{3.246298in}{0.499444in}}%
\pgfpathclose%
\pgfusepath{fill}%
\end{pgfscope}%
\begin{pgfscope}%
\pgfpathrectangle{\pgfqpoint{0.445556in}{0.499444in}}{\pgfqpoint{3.875000in}{1.155000in}}%
\pgfusepath{clip}%
\pgfsetbuttcap%
\pgfsetmiterjoin%
\definecolor{currentfill}{rgb}{0.000000,0.000000,0.000000}%
\pgfsetfillcolor{currentfill}%
\pgfsetlinewidth{0.000000pt}%
\definecolor{currentstroke}{rgb}{0.000000,0.000000,0.000000}%
\pgfsetstrokecolor{currentstroke}%
\pgfsetstrokeopacity{0.000000}%
\pgfsetdash{}{0pt}%
\pgfpathmoveto{\pgfqpoint{3.399764in}{0.499444in}}%
\pgfpathlineto{\pgfqpoint{3.461150in}{0.499444in}}%
\pgfpathlineto{\pgfqpoint{3.461150in}{0.649975in}}%
\pgfpathlineto{\pgfqpoint{3.399764in}{0.649975in}}%
\pgfpathlineto{\pgfqpoint{3.399764in}{0.499444in}}%
\pgfpathclose%
\pgfusepath{fill}%
\end{pgfscope}%
\begin{pgfscope}%
\pgfpathrectangle{\pgfqpoint{0.445556in}{0.499444in}}{\pgfqpoint{3.875000in}{1.155000in}}%
\pgfusepath{clip}%
\pgfsetbuttcap%
\pgfsetmiterjoin%
\definecolor{currentfill}{rgb}{0.000000,0.000000,0.000000}%
\pgfsetfillcolor{currentfill}%
\pgfsetlinewidth{0.000000pt}%
\definecolor{currentstroke}{rgb}{0.000000,0.000000,0.000000}%
\pgfsetstrokecolor{currentstroke}%
\pgfsetstrokeopacity{0.000000}%
\pgfsetdash{}{0pt}%
\pgfpathmoveto{\pgfqpoint{3.553229in}{0.499444in}}%
\pgfpathlineto{\pgfqpoint{3.614615in}{0.499444in}}%
\pgfpathlineto{\pgfqpoint{3.614615in}{0.626409in}}%
\pgfpathlineto{\pgfqpoint{3.553229in}{0.626409in}}%
\pgfpathlineto{\pgfqpoint{3.553229in}{0.499444in}}%
\pgfpathclose%
\pgfusepath{fill}%
\end{pgfscope}%
\begin{pgfscope}%
\pgfpathrectangle{\pgfqpoint{0.445556in}{0.499444in}}{\pgfqpoint{3.875000in}{1.155000in}}%
\pgfusepath{clip}%
\pgfsetbuttcap%
\pgfsetmiterjoin%
\definecolor{currentfill}{rgb}{0.000000,0.000000,0.000000}%
\pgfsetfillcolor{currentfill}%
\pgfsetlinewidth{0.000000pt}%
\definecolor{currentstroke}{rgb}{0.000000,0.000000,0.000000}%
\pgfsetstrokecolor{currentstroke}%
\pgfsetstrokeopacity{0.000000}%
\pgfsetdash{}{0pt}%
\pgfpathmoveto{\pgfqpoint{3.706694in}{0.499444in}}%
\pgfpathlineto{\pgfqpoint{3.768080in}{0.499444in}}%
\pgfpathlineto{\pgfqpoint{3.768080in}{0.625718in}}%
\pgfpathlineto{\pgfqpoint{3.706694in}{0.625718in}}%
\pgfpathlineto{\pgfqpoint{3.706694in}{0.499444in}}%
\pgfpathclose%
\pgfusepath{fill}%
\end{pgfscope}%
\begin{pgfscope}%
\pgfpathrectangle{\pgfqpoint{0.445556in}{0.499444in}}{\pgfqpoint{3.875000in}{1.155000in}}%
\pgfusepath{clip}%
\pgfsetbuttcap%
\pgfsetmiterjoin%
\definecolor{currentfill}{rgb}{0.000000,0.000000,0.000000}%
\pgfsetfillcolor{currentfill}%
\pgfsetlinewidth{0.000000pt}%
\definecolor{currentstroke}{rgb}{0.000000,0.000000,0.000000}%
\pgfsetstrokecolor{currentstroke}%
\pgfsetstrokeopacity{0.000000}%
\pgfsetdash{}{0pt}%
\pgfpathmoveto{\pgfqpoint{3.860160in}{0.499444in}}%
\pgfpathlineto{\pgfqpoint{3.921546in}{0.499444in}}%
\pgfpathlineto{\pgfqpoint{3.921546in}{0.608984in}}%
\pgfpathlineto{\pgfqpoint{3.860160in}{0.608984in}}%
\pgfpathlineto{\pgfqpoint{3.860160in}{0.499444in}}%
\pgfpathclose%
\pgfusepath{fill}%
\end{pgfscope}%
\begin{pgfscope}%
\pgfpathrectangle{\pgfqpoint{0.445556in}{0.499444in}}{\pgfqpoint{3.875000in}{1.155000in}}%
\pgfusepath{clip}%
\pgfsetbuttcap%
\pgfsetmiterjoin%
\definecolor{currentfill}{rgb}{0.000000,0.000000,0.000000}%
\pgfsetfillcolor{currentfill}%
\pgfsetlinewidth{0.000000pt}%
\definecolor{currentstroke}{rgb}{0.000000,0.000000,0.000000}%
\pgfsetstrokecolor{currentstroke}%
\pgfsetstrokeopacity{0.000000}%
\pgfsetdash{}{0pt}%
\pgfpathmoveto{\pgfqpoint{4.013625in}{0.499444in}}%
\pgfpathlineto{\pgfqpoint{4.075011in}{0.499444in}}%
\pgfpathlineto{\pgfqpoint{4.075011in}{0.629402in}}%
\pgfpathlineto{\pgfqpoint{4.013625in}{0.629402in}}%
\pgfpathlineto{\pgfqpoint{4.013625in}{0.499444in}}%
\pgfpathclose%
\pgfusepath{fill}%
\end{pgfscope}%
\begin{pgfscope}%
\pgfpathrectangle{\pgfqpoint{0.445556in}{0.499444in}}{\pgfqpoint{3.875000in}{1.155000in}}%
\pgfusepath{clip}%
\pgfsetbuttcap%
\pgfsetmiterjoin%
\definecolor{currentfill}{rgb}{0.000000,0.000000,0.000000}%
\pgfsetfillcolor{currentfill}%
\pgfsetlinewidth{0.000000pt}%
\definecolor{currentstroke}{rgb}{0.000000,0.000000,0.000000}%
\pgfsetstrokecolor{currentstroke}%
\pgfsetstrokeopacity{0.000000}%
\pgfsetdash{}{0pt}%
\pgfpathmoveto{\pgfqpoint{4.167090in}{0.499444in}}%
\pgfpathlineto{\pgfqpoint{4.228476in}{0.499444in}}%
\pgfpathlineto{\pgfqpoint{4.228476in}{0.550414in}}%
\pgfpathlineto{\pgfqpoint{4.167090in}{0.550414in}}%
\pgfpathlineto{\pgfqpoint{4.167090in}{0.499444in}}%
\pgfpathclose%
\pgfusepath{fill}%
\end{pgfscope}%
\begin{pgfscope}%
\pgfsetbuttcap%
\pgfsetroundjoin%
\definecolor{currentfill}{rgb}{0.000000,0.000000,0.000000}%
\pgfsetfillcolor{currentfill}%
\pgfsetlinewidth{0.803000pt}%
\definecolor{currentstroke}{rgb}{0.000000,0.000000,0.000000}%
\pgfsetstrokecolor{currentstroke}%
\pgfsetdash{}{0pt}%
\pgfsys@defobject{currentmarker}{\pgfqpoint{0.000000in}{-0.048611in}}{\pgfqpoint{0.000000in}{0.000000in}}{%
\pgfpathmoveto{\pgfqpoint{0.000000in}{0.000000in}}%
\pgfpathlineto{\pgfqpoint{0.000000in}{-0.048611in}}%
\pgfusepath{stroke,fill}%
}%
\begin{pgfscope}%
\pgfsys@transformshift{0.483922in}{0.499444in}%
\pgfsys@useobject{currentmarker}{}%
\end{pgfscope}%
\end{pgfscope}%
\begin{pgfscope}%
\definecolor{textcolor}{rgb}{0.000000,0.000000,0.000000}%
\pgfsetstrokecolor{textcolor}%
\pgfsetfillcolor{textcolor}%
\pgftext[x=0.483922in,y=0.402222in,,top]{\color{textcolor}\rmfamily\fontsize{10.000000}{12.000000}\selectfont 0.0}%
\end{pgfscope}%
\begin{pgfscope}%
\pgfsetbuttcap%
\pgfsetroundjoin%
\definecolor{currentfill}{rgb}{0.000000,0.000000,0.000000}%
\pgfsetfillcolor{currentfill}%
\pgfsetlinewidth{0.803000pt}%
\definecolor{currentstroke}{rgb}{0.000000,0.000000,0.000000}%
\pgfsetstrokecolor{currentstroke}%
\pgfsetdash{}{0pt}%
\pgfsys@defobject{currentmarker}{\pgfqpoint{0.000000in}{-0.048611in}}{\pgfqpoint{0.000000in}{0.000000in}}{%
\pgfpathmoveto{\pgfqpoint{0.000000in}{0.000000in}}%
\pgfpathlineto{\pgfqpoint{0.000000in}{-0.048611in}}%
\pgfusepath{stroke,fill}%
}%
\begin{pgfscope}%
\pgfsys@transformshift{0.867585in}{0.499444in}%
\pgfsys@useobject{currentmarker}{}%
\end{pgfscope}%
\end{pgfscope}%
\begin{pgfscope}%
\definecolor{textcolor}{rgb}{0.000000,0.000000,0.000000}%
\pgfsetstrokecolor{textcolor}%
\pgfsetfillcolor{textcolor}%
\pgftext[x=0.867585in,y=0.402222in,,top]{\color{textcolor}\rmfamily\fontsize{10.000000}{12.000000}\selectfont 0.1}%
\end{pgfscope}%
\begin{pgfscope}%
\pgfsetbuttcap%
\pgfsetroundjoin%
\definecolor{currentfill}{rgb}{0.000000,0.000000,0.000000}%
\pgfsetfillcolor{currentfill}%
\pgfsetlinewidth{0.803000pt}%
\definecolor{currentstroke}{rgb}{0.000000,0.000000,0.000000}%
\pgfsetstrokecolor{currentstroke}%
\pgfsetdash{}{0pt}%
\pgfsys@defobject{currentmarker}{\pgfqpoint{0.000000in}{-0.048611in}}{\pgfqpoint{0.000000in}{0.000000in}}{%
\pgfpathmoveto{\pgfqpoint{0.000000in}{0.000000in}}%
\pgfpathlineto{\pgfqpoint{0.000000in}{-0.048611in}}%
\pgfusepath{stroke,fill}%
}%
\begin{pgfscope}%
\pgfsys@transformshift{1.251249in}{0.499444in}%
\pgfsys@useobject{currentmarker}{}%
\end{pgfscope}%
\end{pgfscope}%
\begin{pgfscope}%
\definecolor{textcolor}{rgb}{0.000000,0.000000,0.000000}%
\pgfsetstrokecolor{textcolor}%
\pgfsetfillcolor{textcolor}%
\pgftext[x=1.251249in,y=0.402222in,,top]{\color{textcolor}\rmfamily\fontsize{10.000000}{12.000000}\selectfont 0.2}%
\end{pgfscope}%
\begin{pgfscope}%
\pgfsetbuttcap%
\pgfsetroundjoin%
\definecolor{currentfill}{rgb}{0.000000,0.000000,0.000000}%
\pgfsetfillcolor{currentfill}%
\pgfsetlinewidth{0.803000pt}%
\definecolor{currentstroke}{rgb}{0.000000,0.000000,0.000000}%
\pgfsetstrokecolor{currentstroke}%
\pgfsetdash{}{0pt}%
\pgfsys@defobject{currentmarker}{\pgfqpoint{0.000000in}{-0.048611in}}{\pgfqpoint{0.000000in}{0.000000in}}{%
\pgfpathmoveto{\pgfqpoint{0.000000in}{0.000000in}}%
\pgfpathlineto{\pgfqpoint{0.000000in}{-0.048611in}}%
\pgfusepath{stroke,fill}%
}%
\begin{pgfscope}%
\pgfsys@transformshift{1.634912in}{0.499444in}%
\pgfsys@useobject{currentmarker}{}%
\end{pgfscope}%
\end{pgfscope}%
\begin{pgfscope}%
\definecolor{textcolor}{rgb}{0.000000,0.000000,0.000000}%
\pgfsetstrokecolor{textcolor}%
\pgfsetfillcolor{textcolor}%
\pgftext[x=1.634912in,y=0.402222in,,top]{\color{textcolor}\rmfamily\fontsize{10.000000}{12.000000}\selectfont 0.3}%
\end{pgfscope}%
\begin{pgfscope}%
\pgfsetbuttcap%
\pgfsetroundjoin%
\definecolor{currentfill}{rgb}{0.000000,0.000000,0.000000}%
\pgfsetfillcolor{currentfill}%
\pgfsetlinewidth{0.803000pt}%
\definecolor{currentstroke}{rgb}{0.000000,0.000000,0.000000}%
\pgfsetstrokecolor{currentstroke}%
\pgfsetdash{}{0pt}%
\pgfsys@defobject{currentmarker}{\pgfqpoint{0.000000in}{-0.048611in}}{\pgfqpoint{0.000000in}{0.000000in}}{%
\pgfpathmoveto{\pgfqpoint{0.000000in}{0.000000in}}%
\pgfpathlineto{\pgfqpoint{0.000000in}{-0.048611in}}%
\pgfusepath{stroke,fill}%
}%
\begin{pgfscope}%
\pgfsys@transformshift{2.018575in}{0.499444in}%
\pgfsys@useobject{currentmarker}{}%
\end{pgfscope}%
\end{pgfscope}%
\begin{pgfscope}%
\definecolor{textcolor}{rgb}{0.000000,0.000000,0.000000}%
\pgfsetstrokecolor{textcolor}%
\pgfsetfillcolor{textcolor}%
\pgftext[x=2.018575in,y=0.402222in,,top]{\color{textcolor}\rmfamily\fontsize{10.000000}{12.000000}\selectfont 0.4}%
\end{pgfscope}%
\begin{pgfscope}%
\pgfsetbuttcap%
\pgfsetroundjoin%
\definecolor{currentfill}{rgb}{0.000000,0.000000,0.000000}%
\pgfsetfillcolor{currentfill}%
\pgfsetlinewidth{0.803000pt}%
\definecolor{currentstroke}{rgb}{0.000000,0.000000,0.000000}%
\pgfsetstrokecolor{currentstroke}%
\pgfsetdash{}{0pt}%
\pgfsys@defobject{currentmarker}{\pgfqpoint{0.000000in}{-0.048611in}}{\pgfqpoint{0.000000in}{0.000000in}}{%
\pgfpathmoveto{\pgfqpoint{0.000000in}{0.000000in}}%
\pgfpathlineto{\pgfqpoint{0.000000in}{-0.048611in}}%
\pgfusepath{stroke,fill}%
}%
\begin{pgfscope}%
\pgfsys@transformshift{2.402239in}{0.499444in}%
\pgfsys@useobject{currentmarker}{}%
\end{pgfscope}%
\end{pgfscope}%
\begin{pgfscope}%
\definecolor{textcolor}{rgb}{0.000000,0.000000,0.000000}%
\pgfsetstrokecolor{textcolor}%
\pgfsetfillcolor{textcolor}%
\pgftext[x=2.402239in,y=0.402222in,,top]{\color{textcolor}\rmfamily\fontsize{10.000000}{12.000000}\selectfont 0.5}%
\end{pgfscope}%
\begin{pgfscope}%
\pgfsetbuttcap%
\pgfsetroundjoin%
\definecolor{currentfill}{rgb}{0.000000,0.000000,0.000000}%
\pgfsetfillcolor{currentfill}%
\pgfsetlinewidth{0.803000pt}%
\definecolor{currentstroke}{rgb}{0.000000,0.000000,0.000000}%
\pgfsetstrokecolor{currentstroke}%
\pgfsetdash{}{0pt}%
\pgfsys@defobject{currentmarker}{\pgfqpoint{0.000000in}{-0.048611in}}{\pgfqpoint{0.000000in}{0.000000in}}{%
\pgfpathmoveto{\pgfqpoint{0.000000in}{0.000000in}}%
\pgfpathlineto{\pgfqpoint{0.000000in}{-0.048611in}}%
\pgfusepath{stroke,fill}%
}%
\begin{pgfscope}%
\pgfsys@transformshift{2.785902in}{0.499444in}%
\pgfsys@useobject{currentmarker}{}%
\end{pgfscope}%
\end{pgfscope}%
\begin{pgfscope}%
\definecolor{textcolor}{rgb}{0.000000,0.000000,0.000000}%
\pgfsetstrokecolor{textcolor}%
\pgfsetfillcolor{textcolor}%
\pgftext[x=2.785902in,y=0.402222in,,top]{\color{textcolor}\rmfamily\fontsize{10.000000}{12.000000}\selectfont 0.6}%
\end{pgfscope}%
\begin{pgfscope}%
\pgfsetbuttcap%
\pgfsetroundjoin%
\definecolor{currentfill}{rgb}{0.000000,0.000000,0.000000}%
\pgfsetfillcolor{currentfill}%
\pgfsetlinewidth{0.803000pt}%
\definecolor{currentstroke}{rgb}{0.000000,0.000000,0.000000}%
\pgfsetstrokecolor{currentstroke}%
\pgfsetdash{}{0pt}%
\pgfsys@defobject{currentmarker}{\pgfqpoint{0.000000in}{-0.048611in}}{\pgfqpoint{0.000000in}{0.000000in}}{%
\pgfpathmoveto{\pgfqpoint{0.000000in}{0.000000in}}%
\pgfpathlineto{\pgfqpoint{0.000000in}{-0.048611in}}%
\pgfusepath{stroke,fill}%
}%
\begin{pgfscope}%
\pgfsys@transformshift{3.169566in}{0.499444in}%
\pgfsys@useobject{currentmarker}{}%
\end{pgfscope}%
\end{pgfscope}%
\begin{pgfscope}%
\definecolor{textcolor}{rgb}{0.000000,0.000000,0.000000}%
\pgfsetstrokecolor{textcolor}%
\pgfsetfillcolor{textcolor}%
\pgftext[x=3.169566in,y=0.402222in,,top]{\color{textcolor}\rmfamily\fontsize{10.000000}{12.000000}\selectfont 0.7}%
\end{pgfscope}%
\begin{pgfscope}%
\pgfsetbuttcap%
\pgfsetroundjoin%
\definecolor{currentfill}{rgb}{0.000000,0.000000,0.000000}%
\pgfsetfillcolor{currentfill}%
\pgfsetlinewidth{0.803000pt}%
\definecolor{currentstroke}{rgb}{0.000000,0.000000,0.000000}%
\pgfsetstrokecolor{currentstroke}%
\pgfsetdash{}{0pt}%
\pgfsys@defobject{currentmarker}{\pgfqpoint{0.000000in}{-0.048611in}}{\pgfqpoint{0.000000in}{0.000000in}}{%
\pgfpathmoveto{\pgfqpoint{0.000000in}{0.000000in}}%
\pgfpathlineto{\pgfqpoint{0.000000in}{-0.048611in}}%
\pgfusepath{stroke,fill}%
}%
\begin{pgfscope}%
\pgfsys@transformshift{3.553229in}{0.499444in}%
\pgfsys@useobject{currentmarker}{}%
\end{pgfscope}%
\end{pgfscope}%
\begin{pgfscope}%
\definecolor{textcolor}{rgb}{0.000000,0.000000,0.000000}%
\pgfsetstrokecolor{textcolor}%
\pgfsetfillcolor{textcolor}%
\pgftext[x=3.553229in,y=0.402222in,,top]{\color{textcolor}\rmfamily\fontsize{10.000000}{12.000000}\selectfont 0.8}%
\end{pgfscope}%
\begin{pgfscope}%
\pgfsetbuttcap%
\pgfsetroundjoin%
\definecolor{currentfill}{rgb}{0.000000,0.000000,0.000000}%
\pgfsetfillcolor{currentfill}%
\pgfsetlinewidth{0.803000pt}%
\definecolor{currentstroke}{rgb}{0.000000,0.000000,0.000000}%
\pgfsetstrokecolor{currentstroke}%
\pgfsetdash{}{0pt}%
\pgfsys@defobject{currentmarker}{\pgfqpoint{0.000000in}{-0.048611in}}{\pgfqpoint{0.000000in}{0.000000in}}{%
\pgfpathmoveto{\pgfqpoint{0.000000in}{0.000000in}}%
\pgfpathlineto{\pgfqpoint{0.000000in}{-0.048611in}}%
\pgfusepath{stroke,fill}%
}%
\begin{pgfscope}%
\pgfsys@transformshift{3.936892in}{0.499444in}%
\pgfsys@useobject{currentmarker}{}%
\end{pgfscope}%
\end{pgfscope}%
\begin{pgfscope}%
\definecolor{textcolor}{rgb}{0.000000,0.000000,0.000000}%
\pgfsetstrokecolor{textcolor}%
\pgfsetfillcolor{textcolor}%
\pgftext[x=3.936892in,y=0.402222in,,top]{\color{textcolor}\rmfamily\fontsize{10.000000}{12.000000}\selectfont 0.9}%
\end{pgfscope}%
\begin{pgfscope}%
\pgfsetbuttcap%
\pgfsetroundjoin%
\definecolor{currentfill}{rgb}{0.000000,0.000000,0.000000}%
\pgfsetfillcolor{currentfill}%
\pgfsetlinewidth{0.803000pt}%
\definecolor{currentstroke}{rgb}{0.000000,0.000000,0.000000}%
\pgfsetstrokecolor{currentstroke}%
\pgfsetdash{}{0pt}%
\pgfsys@defobject{currentmarker}{\pgfqpoint{0.000000in}{-0.048611in}}{\pgfqpoint{0.000000in}{0.000000in}}{%
\pgfpathmoveto{\pgfqpoint{0.000000in}{0.000000in}}%
\pgfpathlineto{\pgfqpoint{0.000000in}{-0.048611in}}%
\pgfusepath{stroke,fill}%
}%
\begin{pgfscope}%
\pgfsys@transformshift{4.320556in}{0.499444in}%
\pgfsys@useobject{currentmarker}{}%
\end{pgfscope}%
\end{pgfscope}%
\begin{pgfscope}%
\definecolor{textcolor}{rgb}{0.000000,0.000000,0.000000}%
\pgfsetstrokecolor{textcolor}%
\pgfsetfillcolor{textcolor}%
\pgftext[x=4.320556in,y=0.402222in,,top]{\color{textcolor}\rmfamily\fontsize{10.000000}{12.000000}\selectfont 1.0}%
\end{pgfscope}%
\begin{pgfscope}%
\definecolor{textcolor}{rgb}{0.000000,0.000000,0.000000}%
\pgfsetstrokecolor{textcolor}%
\pgfsetfillcolor{textcolor}%
\pgftext[x=2.383056in,y=0.223333in,,top]{\color{textcolor}\rmfamily\fontsize{10.000000}{12.000000}\selectfont \(\displaystyle p\)}%
\end{pgfscope}%
\begin{pgfscope}%
\pgfsetbuttcap%
\pgfsetroundjoin%
\definecolor{currentfill}{rgb}{0.000000,0.000000,0.000000}%
\pgfsetfillcolor{currentfill}%
\pgfsetlinewidth{0.803000pt}%
\definecolor{currentstroke}{rgb}{0.000000,0.000000,0.000000}%
\pgfsetstrokecolor{currentstroke}%
\pgfsetdash{}{0pt}%
\pgfsys@defobject{currentmarker}{\pgfqpoint{-0.048611in}{0.000000in}}{\pgfqpoint{-0.000000in}{0.000000in}}{%
\pgfpathmoveto{\pgfqpoint{-0.000000in}{0.000000in}}%
\pgfpathlineto{\pgfqpoint{-0.048611in}{0.000000in}}%
\pgfusepath{stroke,fill}%
}%
\begin{pgfscope}%
\pgfsys@transformshift{0.445556in}{0.499444in}%
\pgfsys@useobject{currentmarker}{}%
\end{pgfscope}%
\end{pgfscope}%
\begin{pgfscope}%
\definecolor{textcolor}{rgb}{0.000000,0.000000,0.000000}%
\pgfsetstrokecolor{textcolor}%
\pgfsetfillcolor{textcolor}%
\pgftext[x=0.278889in, y=0.451250in, left, base]{\color{textcolor}\rmfamily\fontsize{10.000000}{12.000000}\selectfont \(\displaystyle {0}\)}%
\end{pgfscope}%
\begin{pgfscope}%
\pgfsetbuttcap%
\pgfsetroundjoin%
\definecolor{currentfill}{rgb}{0.000000,0.000000,0.000000}%
\pgfsetfillcolor{currentfill}%
\pgfsetlinewidth{0.803000pt}%
\definecolor{currentstroke}{rgb}{0.000000,0.000000,0.000000}%
\pgfsetstrokecolor{currentstroke}%
\pgfsetdash{}{0pt}%
\pgfsys@defobject{currentmarker}{\pgfqpoint{-0.048611in}{0.000000in}}{\pgfqpoint{-0.000000in}{0.000000in}}{%
\pgfpathmoveto{\pgfqpoint{-0.000000in}{0.000000in}}%
\pgfpathlineto{\pgfqpoint{-0.048611in}{0.000000in}}%
\pgfusepath{stroke,fill}%
}%
\begin{pgfscope}%
\pgfsys@transformshift{0.445556in}{0.828093in}%
\pgfsys@useobject{currentmarker}{}%
\end{pgfscope}%
\end{pgfscope}%
\begin{pgfscope}%
\definecolor{textcolor}{rgb}{0.000000,0.000000,0.000000}%
\pgfsetstrokecolor{textcolor}%
\pgfsetfillcolor{textcolor}%
\pgftext[x=0.278889in, y=0.779899in, left, base]{\color{textcolor}\rmfamily\fontsize{10.000000}{12.000000}\selectfont \(\displaystyle {2}\)}%
\end{pgfscope}%
\begin{pgfscope}%
\pgfsetbuttcap%
\pgfsetroundjoin%
\definecolor{currentfill}{rgb}{0.000000,0.000000,0.000000}%
\pgfsetfillcolor{currentfill}%
\pgfsetlinewidth{0.803000pt}%
\definecolor{currentstroke}{rgb}{0.000000,0.000000,0.000000}%
\pgfsetstrokecolor{currentstroke}%
\pgfsetdash{}{0pt}%
\pgfsys@defobject{currentmarker}{\pgfqpoint{-0.048611in}{0.000000in}}{\pgfqpoint{-0.000000in}{0.000000in}}{%
\pgfpathmoveto{\pgfqpoint{-0.000000in}{0.000000in}}%
\pgfpathlineto{\pgfqpoint{-0.048611in}{0.000000in}}%
\pgfusepath{stroke,fill}%
}%
\begin{pgfscope}%
\pgfsys@transformshift{0.445556in}{1.156742in}%
\pgfsys@useobject{currentmarker}{}%
\end{pgfscope}%
\end{pgfscope}%
\begin{pgfscope}%
\definecolor{textcolor}{rgb}{0.000000,0.000000,0.000000}%
\pgfsetstrokecolor{textcolor}%
\pgfsetfillcolor{textcolor}%
\pgftext[x=0.278889in, y=1.108548in, left, base]{\color{textcolor}\rmfamily\fontsize{10.000000}{12.000000}\selectfont \(\displaystyle {4}\)}%
\end{pgfscope}%
\begin{pgfscope}%
\pgfsetbuttcap%
\pgfsetroundjoin%
\definecolor{currentfill}{rgb}{0.000000,0.000000,0.000000}%
\pgfsetfillcolor{currentfill}%
\pgfsetlinewidth{0.803000pt}%
\definecolor{currentstroke}{rgb}{0.000000,0.000000,0.000000}%
\pgfsetstrokecolor{currentstroke}%
\pgfsetdash{}{0pt}%
\pgfsys@defobject{currentmarker}{\pgfqpoint{-0.048611in}{0.000000in}}{\pgfqpoint{-0.000000in}{0.000000in}}{%
\pgfpathmoveto{\pgfqpoint{-0.000000in}{0.000000in}}%
\pgfpathlineto{\pgfqpoint{-0.048611in}{0.000000in}}%
\pgfusepath{stroke,fill}%
}%
\begin{pgfscope}%
\pgfsys@transformshift{0.445556in}{1.485391in}%
\pgfsys@useobject{currentmarker}{}%
\end{pgfscope}%
\end{pgfscope}%
\begin{pgfscope}%
\definecolor{textcolor}{rgb}{0.000000,0.000000,0.000000}%
\pgfsetstrokecolor{textcolor}%
\pgfsetfillcolor{textcolor}%
\pgftext[x=0.278889in, y=1.437197in, left, base]{\color{textcolor}\rmfamily\fontsize{10.000000}{12.000000}\selectfont \(\displaystyle {6}\)}%
\end{pgfscope}%
\begin{pgfscope}%
\definecolor{textcolor}{rgb}{0.000000,0.000000,0.000000}%
\pgfsetstrokecolor{textcolor}%
\pgfsetfillcolor{textcolor}%
\pgftext[x=0.223333in,y=1.076944in,,bottom,rotate=90.000000]{\color{textcolor}\rmfamily\fontsize{10.000000}{12.000000}\selectfont Percent of Data Set}%
\end{pgfscope}%
\begin{pgfscope}%
\pgfsetrectcap%
\pgfsetmiterjoin%
\pgfsetlinewidth{0.803000pt}%
\definecolor{currentstroke}{rgb}{0.000000,0.000000,0.000000}%
\pgfsetstrokecolor{currentstroke}%
\pgfsetdash{}{0pt}%
\pgfpathmoveto{\pgfqpoint{0.445556in}{0.499444in}}%
\pgfpathlineto{\pgfqpoint{0.445556in}{1.654444in}}%
\pgfusepath{stroke}%
\end{pgfscope}%
\begin{pgfscope}%
\pgfsetrectcap%
\pgfsetmiterjoin%
\pgfsetlinewidth{0.803000pt}%
\definecolor{currentstroke}{rgb}{0.000000,0.000000,0.000000}%
\pgfsetstrokecolor{currentstroke}%
\pgfsetdash{}{0pt}%
\pgfpathmoveto{\pgfqpoint{4.320556in}{0.499444in}}%
\pgfpathlineto{\pgfqpoint{4.320556in}{1.654444in}}%
\pgfusepath{stroke}%
\end{pgfscope}%
\begin{pgfscope}%
\pgfsetrectcap%
\pgfsetmiterjoin%
\pgfsetlinewidth{0.803000pt}%
\definecolor{currentstroke}{rgb}{0.000000,0.000000,0.000000}%
\pgfsetstrokecolor{currentstroke}%
\pgfsetdash{}{0pt}%
\pgfpathmoveto{\pgfqpoint{0.445556in}{0.499444in}}%
\pgfpathlineto{\pgfqpoint{4.320556in}{0.499444in}}%
\pgfusepath{stroke}%
\end{pgfscope}%
\begin{pgfscope}%
\pgfsetrectcap%
\pgfsetmiterjoin%
\pgfsetlinewidth{0.803000pt}%
\definecolor{currentstroke}{rgb}{0.000000,0.000000,0.000000}%
\pgfsetstrokecolor{currentstroke}%
\pgfsetdash{}{0pt}%
\pgfpathmoveto{\pgfqpoint{0.445556in}{1.654444in}}%
\pgfpathlineto{\pgfqpoint{4.320556in}{1.654444in}}%
\pgfusepath{stroke}%
\end{pgfscope}%
\begin{pgfscope}%
\pgfsetbuttcap%
\pgfsetmiterjoin%
\definecolor{currentfill}{rgb}{1.000000,1.000000,1.000000}%
\pgfsetfillcolor{currentfill}%
\pgfsetfillopacity{0.800000}%
\pgfsetlinewidth{1.003750pt}%
\definecolor{currentstroke}{rgb}{0.800000,0.800000,0.800000}%
\pgfsetstrokecolor{currentstroke}%
\pgfsetstrokeopacity{0.800000}%
\pgfsetdash{}{0pt}%
\pgfpathmoveto{\pgfqpoint{3.543611in}{1.154445in}}%
\pgfpathlineto{\pgfqpoint{4.223333in}{1.154445in}}%
\pgfpathquadraticcurveto{\pgfqpoint{4.251111in}{1.154445in}}{\pgfqpoint{4.251111in}{1.182222in}}%
\pgfpathlineto{\pgfqpoint{4.251111in}{1.557222in}}%
\pgfpathquadraticcurveto{\pgfqpoint{4.251111in}{1.585000in}}{\pgfqpoint{4.223333in}{1.585000in}}%
\pgfpathlineto{\pgfqpoint{3.543611in}{1.585000in}}%
\pgfpathquadraticcurveto{\pgfqpoint{3.515833in}{1.585000in}}{\pgfqpoint{3.515833in}{1.557222in}}%
\pgfpathlineto{\pgfqpoint{3.515833in}{1.182222in}}%
\pgfpathquadraticcurveto{\pgfqpoint{3.515833in}{1.154445in}}{\pgfqpoint{3.543611in}{1.154445in}}%
\pgfpathlineto{\pgfqpoint{3.543611in}{1.154445in}}%
\pgfpathclose%
\pgfusepath{stroke,fill}%
\end{pgfscope}%
\begin{pgfscope}%
\pgfsetbuttcap%
\pgfsetmiterjoin%
\pgfsetlinewidth{1.003750pt}%
\definecolor{currentstroke}{rgb}{0.000000,0.000000,0.000000}%
\pgfsetstrokecolor{currentstroke}%
\pgfsetdash{}{0pt}%
\pgfpathmoveto{\pgfqpoint{3.571389in}{1.432222in}}%
\pgfpathlineto{\pgfqpoint{3.849167in}{1.432222in}}%
\pgfpathlineto{\pgfqpoint{3.849167in}{1.529444in}}%
\pgfpathlineto{\pgfqpoint{3.571389in}{1.529444in}}%
\pgfpathlineto{\pgfqpoint{3.571389in}{1.432222in}}%
\pgfpathclose%
\pgfusepath{stroke}%
\end{pgfscope}%
\begin{pgfscope}%
\definecolor{textcolor}{rgb}{0.000000,0.000000,0.000000}%
\pgfsetstrokecolor{textcolor}%
\pgfsetfillcolor{textcolor}%
\pgftext[x=3.960278in,y=1.432222in,left,base]{\color{textcolor}\rmfamily\fontsize{10.000000}{12.000000}\selectfont Neg}%
\end{pgfscope}%
\begin{pgfscope}%
\pgfsetbuttcap%
\pgfsetmiterjoin%
\definecolor{currentfill}{rgb}{0.000000,0.000000,0.000000}%
\pgfsetfillcolor{currentfill}%
\pgfsetlinewidth{0.000000pt}%
\definecolor{currentstroke}{rgb}{0.000000,0.000000,0.000000}%
\pgfsetstrokecolor{currentstroke}%
\pgfsetstrokeopacity{0.000000}%
\pgfsetdash{}{0pt}%
\pgfpathmoveto{\pgfqpoint{3.571389in}{1.236944in}}%
\pgfpathlineto{\pgfqpoint{3.849167in}{1.236944in}}%
\pgfpathlineto{\pgfqpoint{3.849167in}{1.334167in}}%
\pgfpathlineto{\pgfqpoint{3.571389in}{1.334167in}}%
\pgfpathlineto{\pgfqpoint{3.571389in}{1.236944in}}%
\pgfpathclose%
\pgfusepath{fill}%
\end{pgfscope}%
\begin{pgfscope}%
\definecolor{textcolor}{rgb}{0.000000,0.000000,0.000000}%
\pgfsetstrokecolor{textcolor}%
\pgfsetfillcolor{textcolor}%
\pgftext[x=3.960278in,y=1.236944in,left,base]{\color{textcolor}\rmfamily\fontsize{10.000000}{12.000000}\selectfont Pos}%
\end{pgfscope}%
\end{pgfpicture}%
\makeatother%
\endgroup%

&
	\vskip 0pt
	\qquad \qquad FP/TP
	
	%% Creator: Matplotlib, PGF backend
%%
%% To include the figure in your LaTeX document, write
%%   \input{<filename>.pgf}
%%
%% Make sure the required packages are loaded in your preamble
%%   \usepackage{pgf}
%%
%% Also ensure that all the required font packages are loaded; for instance,
%% the lmodern package is sometimes necessary when using math font.
%%   \usepackage{lmodern}
%%
%% Figures using additional raster images can only be included by \input if
%% they are in the same directory as the main LaTeX file. For loading figures
%% from other directories you can use the `import` package
%%   \usepackage{import}
%%
%% and then include the figures with
%%   \import{<path to file>}{<filename>.pgf}
%%
%% Matplotlib used the following preamble
%%   
%%   \usepackage{fontspec}
%%   \makeatletter\@ifpackageloaded{underscore}{}{\usepackage[strings]{underscore}}\makeatother
%%
\begingroup%
\makeatletter%
\begin{pgfpicture}%
\pgfpathrectangle{\pgfpointorigin}{\pgfqpoint{2.282529in}{1.754444in}}%
\pgfusepath{use as bounding box, clip}%
\begin{pgfscope}%
\pgfsetbuttcap%
\pgfsetmiterjoin%
\definecolor{currentfill}{rgb}{1.000000,1.000000,1.000000}%
\pgfsetfillcolor{currentfill}%
\pgfsetlinewidth{0.000000pt}%
\definecolor{currentstroke}{rgb}{1.000000,1.000000,1.000000}%
\pgfsetstrokecolor{currentstroke}%
\pgfsetdash{}{0pt}%
\pgfpathmoveto{\pgfqpoint{0.000000in}{0.000000in}}%
\pgfpathlineto{\pgfqpoint{2.282529in}{0.000000in}}%
\pgfpathlineto{\pgfqpoint{2.282529in}{1.754444in}}%
\pgfpathlineto{\pgfqpoint{0.000000in}{1.754444in}}%
\pgfpathlineto{\pgfqpoint{0.000000in}{0.000000in}}%
\pgfpathclose%
\pgfusepath{fill}%
\end{pgfscope}%
\begin{pgfscope}%
\pgfsetbuttcap%
\pgfsetmiterjoin%
\definecolor{currentfill}{rgb}{1.000000,1.000000,1.000000}%
\pgfsetfillcolor{currentfill}%
\pgfsetlinewidth{0.000000pt}%
\definecolor{currentstroke}{rgb}{0.000000,0.000000,0.000000}%
\pgfsetstrokecolor{currentstroke}%
\pgfsetstrokeopacity{0.000000}%
\pgfsetdash{}{0pt}%
\pgfpathmoveto{\pgfqpoint{0.530556in}{0.499444in}}%
\pgfpathlineto{\pgfqpoint{2.080556in}{0.499444in}}%
\pgfpathlineto{\pgfqpoint{2.080556in}{1.654444in}}%
\pgfpathlineto{\pgfqpoint{0.530556in}{1.654444in}}%
\pgfpathlineto{\pgfqpoint{0.530556in}{0.499444in}}%
\pgfpathclose%
\pgfusepath{fill}%
\end{pgfscope}%
\begin{pgfscope}%
\pgfsetbuttcap%
\pgfsetroundjoin%
\definecolor{currentfill}{rgb}{0.000000,0.000000,0.000000}%
\pgfsetfillcolor{currentfill}%
\pgfsetlinewidth{0.803000pt}%
\definecolor{currentstroke}{rgb}{0.000000,0.000000,0.000000}%
\pgfsetstrokecolor{currentstroke}%
\pgfsetdash{}{0pt}%
\pgfsys@defobject{currentmarker}{\pgfqpoint{0.000000in}{-0.048611in}}{\pgfqpoint{0.000000in}{0.000000in}}{%
\pgfpathmoveto{\pgfqpoint{0.000000in}{0.000000in}}%
\pgfpathlineto{\pgfqpoint{0.000000in}{-0.048611in}}%
\pgfusepath{stroke,fill}%
}%
\begin{pgfscope}%
\pgfsys@transformshift{0.601010in}{0.499444in}%
\pgfsys@useobject{currentmarker}{}%
\end{pgfscope}%
\end{pgfscope}%
\begin{pgfscope}%
\definecolor{textcolor}{rgb}{0.000000,0.000000,0.000000}%
\pgfsetstrokecolor{textcolor}%
\pgfsetfillcolor{textcolor}%
\pgftext[x=0.601010in,y=0.402222in,,top]{\color{textcolor}\rmfamily\fontsize{10.000000}{12.000000}\selectfont 0.009}%
\end{pgfscope}%
\begin{pgfscope}%
\pgfsetbuttcap%
\pgfsetroundjoin%
\definecolor{currentfill}{rgb}{0.000000,0.000000,0.000000}%
\pgfsetfillcolor{currentfill}%
\pgfsetlinewidth{0.803000pt}%
\definecolor{currentstroke}{rgb}{0.000000,0.000000,0.000000}%
\pgfsetstrokecolor{currentstroke}%
\pgfsetdash{}{0pt}%
\pgfsys@defobject{currentmarker}{\pgfqpoint{0.000000in}{-0.048611in}}{\pgfqpoint{0.000000in}{0.000000in}}{%
\pgfpathmoveto{\pgfqpoint{0.000000in}{0.000000in}}%
\pgfpathlineto{\pgfqpoint{0.000000in}{-0.048611in}}%
\pgfusepath{stroke,fill}%
}%
\begin{pgfscope}%
\pgfsys@transformshift{2.024334in}{0.499444in}%
\pgfsys@useobject{currentmarker}{}%
\end{pgfscope}%
\end{pgfscope}%
\begin{pgfscope}%
\definecolor{textcolor}{rgb}{0.000000,0.000000,0.000000}%
\pgfsetstrokecolor{textcolor}%
\pgfsetfillcolor{textcolor}%
\pgftext[x=2.024334in,y=0.402222in,,top]{\color{textcolor}\rmfamily\fontsize{10.000000}{12.000000}\selectfont 0.975}%
\end{pgfscope}%
\begin{pgfscope}%
\definecolor{textcolor}{rgb}{0.000000,0.000000,0.000000}%
\pgfsetstrokecolor{textcolor}%
\pgfsetfillcolor{textcolor}%
\pgftext[x=1.305556in,y=0.223333in,,top]{\color{textcolor}\rmfamily\fontsize{10.000000}{12.000000}\selectfont \(\displaystyle p\)}%
\end{pgfscope}%
\begin{pgfscope}%
\pgfsetbuttcap%
\pgfsetroundjoin%
\definecolor{currentfill}{rgb}{0.000000,0.000000,0.000000}%
\pgfsetfillcolor{currentfill}%
\pgfsetlinewidth{0.803000pt}%
\definecolor{currentstroke}{rgb}{0.000000,0.000000,0.000000}%
\pgfsetstrokecolor{currentstroke}%
\pgfsetdash{}{0pt}%
\pgfsys@defobject{currentmarker}{\pgfqpoint{-0.048611in}{0.000000in}}{\pgfqpoint{-0.000000in}{0.000000in}}{%
\pgfpathmoveto{\pgfqpoint{-0.000000in}{0.000000in}}%
\pgfpathlineto{\pgfqpoint{-0.048611in}{0.000000in}}%
\pgfusepath{stroke,fill}%
}%
\begin{pgfscope}%
\pgfsys@transformshift{0.530556in}{0.544522in}%
\pgfsys@useobject{currentmarker}{}%
\end{pgfscope}%
\end{pgfscope}%
\begin{pgfscope}%
\definecolor{textcolor}{rgb}{0.000000,0.000000,0.000000}%
\pgfsetstrokecolor{textcolor}%
\pgfsetfillcolor{textcolor}%
\pgftext[x=0.363889in, y=0.496328in, left, base]{\color{textcolor}\rmfamily\fontsize{10.000000}{12.000000}\selectfont \(\displaystyle {0}\)}%
\end{pgfscope}%
\begin{pgfscope}%
\pgfsetbuttcap%
\pgfsetroundjoin%
\definecolor{currentfill}{rgb}{0.000000,0.000000,0.000000}%
\pgfsetfillcolor{currentfill}%
\pgfsetlinewidth{0.803000pt}%
\definecolor{currentstroke}{rgb}{0.000000,0.000000,0.000000}%
\pgfsetstrokecolor{currentstroke}%
\pgfsetdash{}{0pt}%
\pgfsys@defobject{currentmarker}{\pgfqpoint{-0.048611in}{0.000000in}}{\pgfqpoint{-0.000000in}{0.000000in}}{%
\pgfpathmoveto{\pgfqpoint{-0.000000in}{0.000000in}}%
\pgfpathlineto{\pgfqpoint{-0.048611in}{0.000000in}}%
\pgfusepath{stroke,fill}%
}%
\begin{pgfscope}%
\pgfsys@transformshift{0.530556in}{0.864911in}%
\pgfsys@useobject{currentmarker}{}%
\end{pgfscope}%
\end{pgfscope}%
\begin{pgfscope}%
\definecolor{textcolor}{rgb}{0.000000,0.000000,0.000000}%
\pgfsetstrokecolor{textcolor}%
\pgfsetfillcolor{textcolor}%
\pgftext[x=0.294444in, y=0.816717in, left, base]{\color{textcolor}\rmfamily\fontsize{10.000000}{12.000000}\selectfont \(\displaystyle {20}\)}%
\end{pgfscope}%
\begin{pgfscope}%
\pgfsetbuttcap%
\pgfsetroundjoin%
\definecolor{currentfill}{rgb}{0.000000,0.000000,0.000000}%
\pgfsetfillcolor{currentfill}%
\pgfsetlinewidth{0.803000pt}%
\definecolor{currentstroke}{rgb}{0.000000,0.000000,0.000000}%
\pgfsetstrokecolor{currentstroke}%
\pgfsetdash{}{0pt}%
\pgfsys@defobject{currentmarker}{\pgfqpoint{-0.048611in}{0.000000in}}{\pgfqpoint{-0.000000in}{0.000000in}}{%
\pgfpathmoveto{\pgfqpoint{-0.000000in}{0.000000in}}%
\pgfpathlineto{\pgfqpoint{-0.048611in}{0.000000in}}%
\pgfusepath{stroke,fill}%
}%
\begin{pgfscope}%
\pgfsys@transformshift{0.530556in}{1.185300in}%
\pgfsys@useobject{currentmarker}{}%
\end{pgfscope}%
\end{pgfscope}%
\begin{pgfscope}%
\definecolor{textcolor}{rgb}{0.000000,0.000000,0.000000}%
\pgfsetstrokecolor{textcolor}%
\pgfsetfillcolor{textcolor}%
\pgftext[x=0.294444in, y=1.137105in, left, base]{\color{textcolor}\rmfamily\fontsize{10.000000}{12.000000}\selectfont \(\displaystyle {40}\)}%
\end{pgfscope}%
\begin{pgfscope}%
\pgfsetbuttcap%
\pgfsetroundjoin%
\definecolor{currentfill}{rgb}{0.000000,0.000000,0.000000}%
\pgfsetfillcolor{currentfill}%
\pgfsetlinewidth{0.803000pt}%
\definecolor{currentstroke}{rgb}{0.000000,0.000000,0.000000}%
\pgfsetstrokecolor{currentstroke}%
\pgfsetdash{}{0pt}%
\pgfsys@defobject{currentmarker}{\pgfqpoint{-0.048611in}{0.000000in}}{\pgfqpoint{-0.000000in}{0.000000in}}{%
\pgfpathmoveto{\pgfqpoint{-0.000000in}{0.000000in}}%
\pgfpathlineto{\pgfqpoint{-0.048611in}{0.000000in}}%
\pgfusepath{stroke,fill}%
}%
\begin{pgfscope}%
\pgfsys@transformshift{0.530556in}{1.505689in}%
\pgfsys@useobject{currentmarker}{}%
\end{pgfscope}%
\end{pgfscope}%
\begin{pgfscope}%
\definecolor{textcolor}{rgb}{0.000000,0.000000,0.000000}%
\pgfsetstrokecolor{textcolor}%
\pgfsetfillcolor{textcolor}%
\pgftext[x=0.294444in, y=1.457494in, left, base]{\color{textcolor}\rmfamily\fontsize{10.000000}{12.000000}\selectfont \(\displaystyle {60}\)}%
\end{pgfscope}%
\begin{pgfscope}%
\definecolor{textcolor}{rgb}{0.000000,0.000000,0.000000}%
\pgfsetstrokecolor{textcolor}%
\pgfsetfillcolor{textcolor}%
\pgftext[x=0.238889in,y=1.076944in,,bottom,rotate=90.000000]{\color{textcolor}\rmfamily\fontsize{10.000000}{12.000000}\selectfont \(\displaystyle \Delta\)FP/\(\displaystyle \Delta\)TP}%
\end{pgfscope}%
\begin{pgfscope}%
\pgfpathrectangle{\pgfqpoint{0.530556in}{0.499444in}}{\pgfqpoint{1.550000in}{1.155000in}}%
\pgfusepath{clip}%
\pgfsetrectcap%
\pgfsetroundjoin%
\pgfsetlinewidth{1.505625pt}%
\definecolor{currentstroke}{rgb}{0.000000,0.000000,0.000000}%
\pgfsetstrokecolor{currentstroke}%
\pgfsetdash{}{0pt}%
\pgfpathmoveto{\pgfqpoint{0.601010in}{1.601944in}}%
\pgfpathlineto{\pgfqpoint{0.615243in}{1.558029in}}%
\pgfpathlineto{\pgfqpoint{0.629477in}{1.511746in}}%
\pgfpathlineto{\pgfqpoint{0.643710in}{1.466949in}}%
\pgfpathlineto{\pgfqpoint{0.657943in}{1.424226in}}%
\pgfpathlineto{\pgfqpoint{0.672176in}{1.385282in}}%
\pgfpathlineto{\pgfqpoint{0.686410in}{1.307537in}}%
\pgfpathlineto{\pgfqpoint{0.700643in}{1.235607in}}%
\pgfpathlineto{\pgfqpoint{0.714876in}{1.165339in}}%
\pgfpathlineto{\pgfqpoint{0.729109in}{1.100661in}}%
\pgfpathlineto{\pgfqpoint{0.743343in}{1.044250in}}%
\pgfpathlineto{\pgfqpoint{0.757576in}{0.994618in}}%
\pgfpathlineto{\pgfqpoint{0.771809in}{0.953512in}}%
\pgfpathlineto{\pgfqpoint{0.786042in}{0.918731in}}%
\pgfpathlineto{\pgfqpoint{0.800276in}{0.889563in}}%
\pgfpathlineto{\pgfqpoint{0.814509in}{0.864071in}}%
\pgfpathlineto{\pgfqpoint{0.828742in}{0.841335in}}%
\pgfpathlineto{\pgfqpoint{0.842975in}{0.821618in}}%
\pgfpathlineto{\pgfqpoint{0.857209in}{0.804996in}}%
\pgfpathlineto{\pgfqpoint{0.871442in}{0.790241in}}%
\pgfpathlineto{\pgfqpoint{0.885675in}{0.776738in}}%
\pgfpathlineto{\pgfqpoint{0.899908in}{0.764300in}}%
\pgfpathlineto{\pgfqpoint{0.914142in}{0.753115in}}%
\pgfpathlineto{\pgfqpoint{0.928375in}{0.742856in}}%
\pgfpathlineto{\pgfqpoint{0.942608in}{0.733651in}}%
\pgfpathlineto{\pgfqpoint{0.956841in}{0.724839in}}%
\pgfpathlineto{\pgfqpoint{0.971075in}{0.716927in}}%
\pgfpathlineto{\pgfqpoint{0.985308in}{0.709599in}}%
\pgfpathlineto{\pgfqpoint{0.999541in}{0.702536in}}%
\pgfpathlineto{\pgfqpoint{1.013774in}{0.696235in}}%
\pgfpathlineto{\pgfqpoint{1.028007in}{0.690244in}}%
\pgfpathlineto{\pgfqpoint{1.042241in}{0.684621in}}%
\pgfpathlineto{\pgfqpoint{1.056474in}{0.679235in}}%
\pgfpathlineto{\pgfqpoint{1.070707in}{0.674315in}}%
\pgfpathlineto{\pgfqpoint{1.084940in}{0.669331in}}%
\pgfpathlineto{\pgfqpoint{1.099174in}{0.664617in}}%
\pgfpathlineto{\pgfqpoint{1.113407in}{0.660187in}}%
\pgfpathlineto{\pgfqpoint{1.127640in}{0.655902in}}%
\pgfpathlineto{\pgfqpoint{1.141873in}{0.651796in}}%
\pgfpathlineto{\pgfqpoint{1.156107in}{0.647652in}}%
\pgfpathlineto{\pgfqpoint{1.170340in}{0.643818in}}%
\pgfpathlineto{\pgfqpoint{1.184573in}{0.640181in}}%
\pgfpathlineto{\pgfqpoint{1.198806in}{0.636727in}}%
\pgfpathlineto{\pgfqpoint{1.213040in}{0.633418in}}%
\pgfpathlineto{\pgfqpoint{1.227273in}{0.630392in}}%
\pgfpathlineto{\pgfqpoint{1.241506in}{0.627543in}}%
\pgfpathlineto{\pgfqpoint{1.255739in}{0.624726in}}%
\pgfpathlineto{\pgfqpoint{1.269973in}{0.622100in}}%
\pgfpathlineto{\pgfqpoint{1.284206in}{0.619547in}}%
\pgfpathlineto{\pgfqpoint{1.298439in}{0.617252in}}%
\pgfpathlineto{\pgfqpoint{1.312672in}{0.615028in}}%
\pgfpathlineto{\pgfqpoint{1.326906in}{0.612875in}}%
\pgfpathlineto{\pgfqpoint{1.341139in}{0.610777in}}%
\pgfpathlineto{\pgfqpoint{1.355372in}{0.608700in}}%
\pgfpathlineto{\pgfqpoint{1.369605in}{0.606638in}}%
\pgfpathlineto{\pgfqpoint{1.383839in}{0.604700in}}%
\pgfpathlineto{\pgfqpoint{1.398072in}{0.602863in}}%
\pgfpathlineto{\pgfqpoint{1.412305in}{0.601070in}}%
\pgfpathlineto{\pgfqpoint{1.426538in}{0.599433in}}%
\pgfpathlineto{\pgfqpoint{1.440771in}{0.597761in}}%
\pgfpathlineto{\pgfqpoint{1.455005in}{0.596057in}}%
\pgfpathlineto{\pgfqpoint{1.469238in}{0.594429in}}%
\pgfpathlineto{\pgfqpoint{1.483471in}{0.592840in}}%
\pgfpathlineto{\pgfqpoint{1.497704in}{0.591324in}}%
\pgfpathlineto{\pgfqpoint{1.511938in}{0.589852in}}%
\pgfpathlineto{\pgfqpoint{1.526171in}{0.588314in}}%
\pgfpathlineto{\pgfqpoint{1.540404in}{0.586769in}}%
\pgfpathlineto{\pgfqpoint{1.554637in}{0.585211in}}%
\pgfpathlineto{\pgfqpoint{1.568871in}{0.583631in}}%
\pgfpathlineto{\pgfqpoint{1.583104in}{0.582056in}}%
\pgfpathlineto{\pgfqpoint{1.597337in}{0.580574in}}%
\pgfpathlineto{\pgfqpoint{1.611570in}{0.579125in}}%
\pgfpathlineto{\pgfqpoint{1.625804in}{0.577668in}}%
\pgfpathlineto{\pgfqpoint{1.640037in}{0.576165in}}%
\pgfpathlineto{\pgfqpoint{1.654270in}{0.574707in}}%
\pgfpathlineto{\pgfqpoint{1.668503in}{0.573371in}}%
\pgfpathlineto{\pgfqpoint{1.682737in}{0.572128in}}%
\pgfpathlineto{\pgfqpoint{1.696970in}{0.570963in}}%
\pgfpathlineto{\pgfqpoint{1.711203in}{0.569815in}}%
\pgfpathlineto{\pgfqpoint{1.725436in}{0.568719in}}%
\pgfpathlineto{\pgfqpoint{1.739670in}{0.567602in}}%
\pgfpathlineto{\pgfqpoint{1.753903in}{0.566505in}}%
\pgfpathlineto{\pgfqpoint{1.768136in}{0.565464in}}%
\pgfpathlineto{\pgfqpoint{1.782369in}{0.564490in}}%
\pgfpathlineto{\pgfqpoint{1.796603in}{0.563516in}}%
\pgfpathlineto{\pgfqpoint{1.810836in}{0.562495in}}%
\pgfpathlineto{\pgfqpoint{1.825069in}{0.561465in}}%
\pgfpathlineto{\pgfqpoint{1.839302in}{0.560425in}}%
\pgfpathlineto{\pgfqpoint{1.853535in}{0.559410in}}%
\pgfpathlineto{\pgfqpoint{1.867769in}{0.558428in}}%
\pgfpathlineto{\pgfqpoint{1.882002in}{0.557478in}}%
\pgfpathlineto{\pgfqpoint{1.896235in}{0.556535in}}%
\pgfpathlineto{\pgfqpoint{1.910468in}{0.555655in}}%
\pgfpathlineto{\pgfqpoint{1.924702in}{0.554823in}}%
\pgfpathlineto{\pgfqpoint{1.938935in}{0.554062in}}%
\pgfpathlineto{\pgfqpoint{1.953168in}{0.553369in}}%
\pgfpathlineto{\pgfqpoint{1.967401in}{0.552977in}}%
\pgfpathlineto{\pgfqpoint{1.981635in}{0.552608in}}%
\pgfpathlineto{\pgfqpoint{1.995868in}{0.552268in}}%
\pgfpathlineto{\pgfqpoint{2.010101in}{0.551944in}}%
\pgfusepath{stroke}%
\end{pgfscope}%
\begin{pgfscope}%
\pgfpathrectangle{\pgfqpoint{0.530556in}{0.499444in}}{\pgfqpoint{1.550000in}{1.155000in}}%
\pgfusepath{clip}%
\pgfsetbuttcap%
\pgfsetroundjoin%
\pgfsetlinewidth{1.505625pt}%
\definecolor{currentstroke}{rgb}{0.000000,0.000000,0.000000}%
\pgfsetstrokecolor{currentstroke}%
\pgfsetdash{{5.550000pt}{2.400000pt}}{0.000000pt}%
\pgfpathmoveto{\pgfqpoint{0.530556in}{0.576561in}}%
\pgfpathlineto{\pgfqpoint{2.080556in}{0.576561in}}%
\pgfusepath{stroke}%
\end{pgfscope}%
\begin{pgfscope}%
\pgfpathrectangle{\pgfqpoint{0.530556in}{0.499444in}}{\pgfqpoint{1.550000in}{1.155000in}}%
\pgfusepath{clip}%
\pgfsetrectcap%
\pgfsetroundjoin%
\pgfsetlinewidth{1.505625pt}%
\definecolor{currentstroke}{rgb}{0.121569,0.466667,0.705882}%
\pgfsetstrokecolor{currentstroke}%
\pgfsetdash{}{0pt}%
\pgfpathmoveto{\pgfqpoint{1.640037in}{0.576561in}}%
\pgfusepath{stroke}%
\end{pgfscope}%
\begin{pgfscope}%
\pgfpathrectangle{\pgfqpoint{0.530556in}{0.499444in}}{\pgfqpoint{1.550000in}{1.155000in}}%
\pgfusepath{clip}%
\pgfsetbuttcap%
\pgfsetroundjoin%
\definecolor{currentfill}{rgb}{0.000000,0.000000,0.000000}%
\pgfsetfillcolor{currentfill}%
\pgfsetlinewidth{1.003750pt}%
\definecolor{currentstroke}{rgb}{0.000000,0.000000,0.000000}%
\pgfsetstrokecolor{currentstroke}%
\pgfsetdash{}{0pt}%
\pgfsys@defobject{currentmarker}{\pgfqpoint{-0.041667in}{-0.041667in}}{\pgfqpoint{0.041667in}{0.041667in}}{%
\pgfpathmoveto{\pgfqpoint{0.000000in}{-0.041667in}}%
\pgfpathcurveto{\pgfqpoint{0.011050in}{-0.041667in}}{\pgfqpoint{0.021649in}{-0.037276in}}{\pgfqpoint{0.029463in}{-0.029463in}}%
\pgfpathcurveto{\pgfqpoint{0.037276in}{-0.021649in}}{\pgfqpoint{0.041667in}{-0.011050in}}{\pgfqpoint{0.041667in}{0.000000in}}%
\pgfpathcurveto{\pgfqpoint{0.041667in}{0.011050in}}{\pgfqpoint{0.037276in}{0.021649in}}{\pgfqpoint{0.029463in}{0.029463in}}%
\pgfpathcurveto{\pgfqpoint{0.021649in}{0.037276in}}{\pgfqpoint{0.011050in}{0.041667in}}{\pgfqpoint{0.000000in}{0.041667in}}%
\pgfpathcurveto{\pgfqpoint{-0.011050in}{0.041667in}}{\pgfqpoint{-0.021649in}{0.037276in}}{\pgfqpoint{-0.029463in}{0.029463in}}%
\pgfpathcurveto{\pgfqpoint{-0.037276in}{0.021649in}}{\pgfqpoint{-0.041667in}{0.011050in}}{\pgfqpoint{-0.041667in}{0.000000in}}%
\pgfpathcurveto{\pgfqpoint{-0.041667in}{-0.011050in}}{\pgfqpoint{-0.037276in}{-0.021649in}}{\pgfqpoint{-0.029463in}{-0.029463in}}%
\pgfpathcurveto{\pgfqpoint{-0.021649in}{-0.037276in}}{\pgfqpoint{-0.011050in}{-0.041667in}}{\pgfqpoint{0.000000in}{-0.041667in}}%
\pgfpathlineto{\pgfqpoint{0.000000in}{-0.041667in}}%
\pgfpathclose%
\pgfusepath{stroke,fill}%
}%
\begin{pgfscope}%
\pgfsys@transformshift{1.640037in}{0.576561in}%
\pgfsys@useobject{currentmarker}{}%
\end{pgfscope}%
\end{pgfscope}%
\begin{pgfscope}%
\pgfsetrectcap%
\pgfsetmiterjoin%
\pgfsetlinewidth{0.803000pt}%
\definecolor{currentstroke}{rgb}{0.000000,0.000000,0.000000}%
\pgfsetstrokecolor{currentstroke}%
\pgfsetdash{}{0pt}%
\pgfpathmoveto{\pgfqpoint{0.530556in}{0.499444in}}%
\pgfpathlineto{\pgfqpoint{0.530556in}{1.654444in}}%
\pgfusepath{stroke}%
\end{pgfscope}%
\begin{pgfscope}%
\pgfsetrectcap%
\pgfsetmiterjoin%
\pgfsetlinewidth{0.803000pt}%
\definecolor{currentstroke}{rgb}{0.000000,0.000000,0.000000}%
\pgfsetstrokecolor{currentstroke}%
\pgfsetdash{}{0pt}%
\pgfpathmoveto{\pgfqpoint{2.080556in}{0.499444in}}%
\pgfpathlineto{\pgfqpoint{2.080556in}{1.654444in}}%
\pgfusepath{stroke}%
\end{pgfscope}%
\begin{pgfscope}%
\pgfsetrectcap%
\pgfsetmiterjoin%
\pgfsetlinewidth{0.803000pt}%
\definecolor{currentstroke}{rgb}{0.000000,0.000000,0.000000}%
\pgfsetstrokecolor{currentstroke}%
\pgfsetdash{}{0pt}%
\pgfpathmoveto{\pgfqpoint{0.530556in}{0.499444in}}%
\pgfpathlineto{\pgfqpoint{2.080556in}{0.499444in}}%
\pgfusepath{stroke}%
\end{pgfscope}%
\begin{pgfscope}%
\pgfsetrectcap%
\pgfsetmiterjoin%
\pgfsetlinewidth{0.803000pt}%
\definecolor{currentstroke}{rgb}{0.000000,0.000000,0.000000}%
\pgfsetstrokecolor{currentstroke}%
\pgfsetdash{}{0pt}%
\pgfpathmoveto{\pgfqpoint{0.530556in}{1.654444in}}%
\pgfpathlineto{\pgfqpoint{2.080556in}{1.654444in}}%
\pgfusepath{stroke}%
\end{pgfscope}%
\begin{pgfscope}%
\pgfsetbuttcap%
\pgfsetmiterjoin%
\definecolor{currentfill}{rgb}{1.000000,1.000000,1.000000}%
\pgfsetfillcolor{currentfill}%
\pgfsetfillopacity{0.800000}%
\pgfsetlinewidth{1.003750pt}%
\definecolor{currentstroke}{rgb}{0.800000,0.800000,0.800000}%
\pgfsetstrokecolor{currentstroke}%
\pgfsetstrokeopacity{0.800000}%
\pgfsetdash{}{0pt}%
\pgfpathmoveto{\pgfqpoint{0.811987in}{1.126667in}}%
\pgfpathlineto{\pgfqpoint{1.983333in}{1.126667in}}%
\pgfpathquadraticcurveto{\pgfqpoint{2.011111in}{1.126667in}}{\pgfqpoint{2.011111in}{1.154444in}}%
\pgfpathlineto{\pgfqpoint{2.011111in}{1.557222in}}%
\pgfpathquadraticcurveto{\pgfqpoint{2.011111in}{1.585000in}}{\pgfqpoint{1.983333in}{1.585000in}}%
\pgfpathlineto{\pgfqpoint{0.811987in}{1.585000in}}%
\pgfpathquadraticcurveto{\pgfqpoint{0.784210in}{1.585000in}}{\pgfqpoint{0.784210in}{1.557222in}}%
\pgfpathlineto{\pgfqpoint{0.784210in}{1.154444in}}%
\pgfpathquadraticcurveto{\pgfqpoint{0.784210in}{1.126667in}}{\pgfqpoint{0.811987in}{1.126667in}}%
\pgfpathlineto{\pgfqpoint{0.811987in}{1.126667in}}%
\pgfpathclose%
\pgfusepath{stroke,fill}%
\end{pgfscope}%
\begin{pgfscope}%
\pgfsetrectcap%
\pgfsetroundjoin%
\pgfsetlinewidth{1.505625pt}%
\definecolor{currentstroke}{rgb}{0.000000,0.000000,0.000000}%
\pgfsetstrokecolor{currentstroke}%
\pgfsetdash{}{0pt}%
\pgfpathmoveto{\pgfqpoint{0.839765in}{1.473889in}}%
\pgfpathlineto{\pgfqpoint{0.978654in}{1.473889in}}%
\pgfpathlineto{\pgfqpoint{1.117543in}{1.473889in}}%
\pgfusepath{stroke}%
\end{pgfscope}%
\begin{pgfscope}%
\definecolor{textcolor}{rgb}{0.000000,0.000000,0.000000}%
\pgfsetstrokecolor{textcolor}%
\pgfsetfillcolor{textcolor}%
\pgftext[x=1.228654in,y=1.425277in,left,base]{\color{textcolor}\rmfamily\fontsize{10.000000}{12.000000}\selectfont \(\displaystyle \Delta FP/\Delta TP\)}%
\end{pgfscope}%
\begin{pgfscope}%
\pgfsetrectcap%
\pgfsetroundjoin%
\pgfsetlinewidth{1.505625pt}%
\definecolor{currentstroke}{rgb}{0.121569,0.466667,0.705882}%
\pgfsetstrokecolor{currentstroke}%
\pgfsetdash{}{0pt}%
\pgfpathmoveto{\pgfqpoint{0.839765in}{1.265555in}}%
\pgfpathlineto{\pgfqpoint{0.978654in}{1.265555in}}%
\pgfpathlineto{\pgfqpoint{1.117543in}{1.265555in}}%
\pgfusepath{stroke}%
\end{pgfscope}%
\begin{pgfscope}%
\pgfsetbuttcap%
\pgfsetroundjoin%
\definecolor{currentfill}{rgb}{0.000000,0.000000,0.000000}%
\pgfsetfillcolor{currentfill}%
\pgfsetlinewidth{1.003750pt}%
\definecolor{currentstroke}{rgb}{0.000000,0.000000,0.000000}%
\pgfsetstrokecolor{currentstroke}%
\pgfsetdash{}{0pt}%
\pgfsys@defobject{currentmarker}{\pgfqpoint{-0.041667in}{-0.041667in}}{\pgfqpoint{0.041667in}{0.041667in}}{%
\pgfpathmoveto{\pgfqpoint{0.000000in}{-0.041667in}}%
\pgfpathcurveto{\pgfqpoint{0.011050in}{-0.041667in}}{\pgfqpoint{0.021649in}{-0.037276in}}{\pgfqpoint{0.029463in}{-0.029463in}}%
\pgfpathcurveto{\pgfqpoint{0.037276in}{-0.021649in}}{\pgfqpoint{0.041667in}{-0.011050in}}{\pgfqpoint{0.041667in}{0.000000in}}%
\pgfpathcurveto{\pgfqpoint{0.041667in}{0.011050in}}{\pgfqpoint{0.037276in}{0.021649in}}{\pgfqpoint{0.029463in}{0.029463in}}%
\pgfpathcurveto{\pgfqpoint{0.021649in}{0.037276in}}{\pgfqpoint{0.011050in}{0.041667in}}{\pgfqpoint{0.000000in}{0.041667in}}%
\pgfpathcurveto{\pgfqpoint{-0.011050in}{0.041667in}}{\pgfqpoint{-0.021649in}{0.037276in}}{\pgfqpoint{-0.029463in}{0.029463in}}%
\pgfpathcurveto{\pgfqpoint{-0.037276in}{0.021649in}}{\pgfqpoint{-0.041667in}{0.011050in}}{\pgfqpoint{-0.041667in}{0.000000in}}%
\pgfpathcurveto{\pgfqpoint{-0.041667in}{-0.011050in}}{\pgfqpoint{-0.037276in}{-0.021649in}}{\pgfqpoint{-0.029463in}{-0.029463in}}%
\pgfpathcurveto{\pgfqpoint{-0.021649in}{-0.037276in}}{\pgfqpoint{-0.011050in}{-0.041667in}}{\pgfqpoint{0.000000in}{-0.041667in}}%
\pgfpathlineto{\pgfqpoint{0.000000in}{-0.041667in}}%
\pgfpathclose%
\pgfusepath{stroke,fill}%
}%
\begin{pgfscope}%
\pgfsys@transformshift{0.978654in}{1.265555in}%
\pgfsys@useobject{currentmarker}{}%
\end{pgfscope}%
\end{pgfscope}%
\begin{pgfscope}%
\definecolor{textcolor}{rgb}{0.000000,0.000000,0.000000}%
\pgfsetstrokecolor{textcolor}%
\pgfsetfillcolor{textcolor}%
\pgftext[x=1.228654in,y=1.216944in,left,base]{\color{textcolor}\rmfamily\fontsize{10.000000}{12.000000}\selectfont (0.714,2)}%
\end{pgfscope}%
\end{pgfpicture}%
\makeatother%
\endgroup%

\end{tabular}


\noindent\begin{tabular}{@{\hspace{-6pt}}p{4.5in} @{\hspace{6pt}}p{2.0in}}
	\vskip 0pt
	\qquad \qquad Transformed Model Output:  Map $0.714$ to 0.5 and 0 to 0.
	
	%% Creator: Matplotlib, PGF backend
%%
%% To include the figure in your LaTeX document, write
%%   \input{<filename>.pgf}
%%
%% Make sure the required packages are loaded in your preamble
%%   \usepackage{pgf}
%%
%% Also ensure that all the required font packages are loaded; for instance,
%% the lmodern package is sometimes necessary when using math font.
%%   \usepackage{lmodern}
%%
%% Figures using additional raster images can only be included by \input if
%% they are in the same directory as the main LaTeX file. For loading figures
%% from other directories you can use the `import` package
%%   \usepackage{import}
%%
%% and then include the figures with
%%   \import{<path to file>}{<filename>.pgf}
%%
%% Matplotlib used the following preamble
%%   
%%   \usepackage{fontspec}
%%   \makeatletter\@ifpackageloaded{underscore}{}{\usepackage[strings]{underscore}}\makeatother
%%
\begingroup%
\makeatletter%
\begin{pgfpicture}%
\pgfpathrectangle{\pgfpointorigin}{\pgfqpoint{4.617331in}{1.754444in}}%
\pgfusepath{use as bounding box, clip}%
\begin{pgfscope}%
\pgfsetbuttcap%
\pgfsetmiterjoin%
\definecolor{currentfill}{rgb}{1.000000,1.000000,1.000000}%
\pgfsetfillcolor{currentfill}%
\pgfsetlinewidth{0.000000pt}%
\definecolor{currentstroke}{rgb}{1.000000,1.000000,1.000000}%
\pgfsetstrokecolor{currentstroke}%
\pgfsetdash{}{0pt}%
\pgfpathmoveto{\pgfqpoint{0.000000in}{0.000000in}}%
\pgfpathlineto{\pgfqpoint{4.617331in}{0.000000in}}%
\pgfpathlineto{\pgfqpoint{4.617331in}{1.754444in}}%
\pgfpathlineto{\pgfqpoint{0.000000in}{1.754444in}}%
\pgfpathlineto{\pgfqpoint{0.000000in}{0.000000in}}%
\pgfpathclose%
\pgfusepath{fill}%
\end{pgfscope}%
\begin{pgfscope}%
\pgfsetbuttcap%
\pgfsetmiterjoin%
\definecolor{currentfill}{rgb}{1.000000,1.000000,1.000000}%
\pgfsetfillcolor{currentfill}%
\pgfsetlinewidth{0.000000pt}%
\definecolor{currentstroke}{rgb}{0.000000,0.000000,0.000000}%
\pgfsetstrokecolor{currentstroke}%
\pgfsetstrokeopacity{0.000000}%
\pgfsetdash{}{0pt}%
\pgfpathmoveto{\pgfqpoint{0.553581in}{0.499444in}}%
\pgfpathlineto{\pgfqpoint{4.428581in}{0.499444in}}%
\pgfpathlineto{\pgfqpoint{4.428581in}{1.654444in}}%
\pgfpathlineto{\pgfqpoint{0.553581in}{1.654444in}}%
\pgfpathlineto{\pgfqpoint{0.553581in}{0.499444in}}%
\pgfpathclose%
\pgfusepath{fill}%
\end{pgfscope}%
\begin{pgfscope}%
\pgfpathrectangle{\pgfqpoint{0.553581in}{0.499444in}}{\pgfqpoint{3.875000in}{1.155000in}}%
\pgfusepath{clip}%
\pgfsetbuttcap%
\pgfsetmiterjoin%
\pgfsetlinewidth{1.003750pt}%
\definecolor{currentstroke}{rgb}{0.000000,0.000000,0.000000}%
\pgfsetstrokecolor{currentstroke}%
\pgfsetdash{}{0pt}%
\pgfpathmoveto{\pgfqpoint{0.543581in}{0.499444in}}%
\pgfpathlineto{\pgfqpoint{0.591947in}{0.499444in}}%
\pgfpathlineto{\pgfqpoint{0.591947in}{1.420226in}}%
\pgfpathlineto{\pgfqpoint{0.543581in}{1.420226in}}%
\pgfusepath{stroke}%
\end{pgfscope}%
\begin{pgfscope}%
\pgfpathrectangle{\pgfqpoint{0.553581in}{0.499444in}}{\pgfqpoint{3.875000in}{1.155000in}}%
\pgfusepath{clip}%
\pgfsetbuttcap%
\pgfsetmiterjoin%
\pgfsetlinewidth{1.003750pt}%
\definecolor{currentstroke}{rgb}{0.000000,0.000000,0.000000}%
\pgfsetstrokecolor{currentstroke}%
\pgfsetdash{}{0pt}%
\pgfpathmoveto{\pgfqpoint{0.684026in}{0.499444in}}%
\pgfpathlineto{\pgfqpoint{0.745412in}{0.499444in}}%
\pgfpathlineto{\pgfqpoint{0.745412in}{1.599444in}}%
\pgfpathlineto{\pgfqpoint{0.684026in}{1.599444in}}%
\pgfpathlineto{\pgfqpoint{0.684026in}{0.499444in}}%
\pgfpathclose%
\pgfusepath{stroke}%
\end{pgfscope}%
\begin{pgfscope}%
\pgfpathrectangle{\pgfqpoint{0.553581in}{0.499444in}}{\pgfqpoint{3.875000in}{1.155000in}}%
\pgfusepath{clip}%
\pgfsetbuttcap%
\pgfsetmiterjoin%
\pgfsetlinewidth{1.003750pt}%
\definecolor{currentstroke}{rgb}{0.000000,0.000000,0.000000}%
\pgfsetstrokecolor{currentstroke}%
\pgfsetdash{}{0pt}%
\pgfpathmoveto{\pgfqpoint{0.837492in}{0.499444in}}%
\pgfpathlineto{\pgfqpoint{0.898878in}{0.499444in}}%
\pgfpathlineto{\pgfqpoint{0.898878in}{1.544937in}}%
\pgfpathlineto{\pgfqpoint{0.837492in}{1.544937in}}%
\pgfpathlineto{\pgfqpoint{0.837492in}{0.499444in}}%
\pgfpathclose%
\pgfusepath{stroke}%
\end{pgfscope}%
\begin{pgfscope}%
\pgfpathrectangle{\pgfqpoint{0.553581in}{0.499444in}}{\pgfqpoint{3.875000in}{1.155000in}}%
\pgfusepath{clip}%
\pgfsetbuttcap%
\pgfsetmiterjoin%
\pgfsetlinewidth{1.003750pt}%
\definecolor{currentstroke}{rgb}{0.000000,0.000000,0.000000}%
\pgfsetstrokecolor{currentstroke}%
\pgfsetdash{}{0pt}%
\pgfpathmoveto{\pgfqpoint{0.990957in}{0.499444in}}%
\pgfpathlineto{\pgfqpoint{1.052343in}{0.499444in}}%
\pgfpathlineto{\pgfqpoint{1.052343in}{1.472553in}}%
\pgfpathlineto{\pgfqpoint{0.990957in}{1.472553in}}%
\pgfpathlineto{\pgfqpoint{0.990957in}{0.499444in}}%
\pgfpathclose%
\pgfusepath{stroke}%
\end{pgfscope}%
\begin{pgfscope}%
\pgfpathrectangle{\pgfqpoint{0.553581in}{0.499444in}}{\pgfqpoint{3.875000in}{1.155000in}}%
\pgfusepath{clip}%
\pgfsetbuttcap%
\pgfsetmiterjoin%
\pgfsetlinewidth{1.003750pt}%
\definecolor{currentstroke}{rgb}{0.000000,0.000000,0.000000}%
\pgfsetstrokecolor{currentstroke}%
\pgfsetdash{}{0pt}%
\pgfpathmoveto{\pgfqpoint{1.144422in}{0.499444in}}%
\pgfpathlineto{\pgfqpoint{1.205808in}{0.499444in}}%
\pgfpathlineto{\pgfqpoint{1.205808in}{1.403002in}}%
\pgfpathlineto{\pgfqpoint{1.144422in}{1.403002in}}%
\pgfpathlineto{\pgfqpoint{1.144422in}{0.499444in}}%
\pgfpathclose%
\pgfusepath{stroke}%
\end{pgfscope}%
\begin{pgfscope}%
\pgfpathrectangle{\pgfqpoint{0.553581in}{0.499444in}}{\pgfqpoint{3.875000in}{1.155000in}}%
\pgfusepath{clip}%
\pgfsetbuttcap%
\pgfsetmiterjoin%
\pgfsetlinewidth{1.003750pt}%
\definecolor{currentstroke}{rgb}{0.000000,0.000000,0.000000}%
\pgfsetstrokecolor{currentstroke}%
\pgfsetdash{}{0pt}%
\pgfpathmoveto{\pgfqpoint{1.297888in}{0.499444in}}%
\pgfpathlineto{\pgfqpoint{1.359274in}{0.499444in}}%
\pgfpathlineto{\pgfqpoint{1.359274in}{1.311867in}}%
\pgfpathlineto{\pgfqpoint{1.297888in}{1.311867in}}%
\pgfpathlineto{\pgfqpoint{1.297888in}{0.499444in}}%
\pgfpathclose%
\pgfusepath{stroke}%
\end{pgfscope}%
\begin{pgfscope}%
\pgfpathrectangle{\pgfqpoint{0.553581in}{0.499444in}}{\pgfqpoint{3.875000in}{1.155000in}}%
\pgfusepath{clip}%
\pgfsetbuttcap%
\pgfsetmiterjoin%
\pgfsetlinewidth{1.003750pt}%
\definecolor{currentstroke}{rgb}{0.000000,0.000000,0.000000}%
\pgfsetstrokecolor{currentstroke}%
\pgfsetdash{}{0pt}%
\pgfpathmoveto{\pgfqpoint{1.451353in}{0.499444in}}%
\pgfpathlineto{\pgfqpoint{1.512739in}{0.499444in}}%
\pgfpathlineto{\pgfqpoint{1.512739in}{1.231469in}}%
\pgfpathlineto{\pgfqpoint{1.451353in}{1.231469in}}%
\pgfpathlineto{\pgfqpoint{1.451353in}{0.499444in}}%
\pgfpathclose%
\pgfusepath{stroke}%
\end{pgfscope}%
\begin{pgfscope}%
\pgfpathrectangle{\pgfqpoint{0.553581in}{0.499444in}}{\pgfqpoint{3.875000in}{1.155000in}}%
\pgfusepath{clip}%
\pgfsetbuttcap%
\pgfsetmiterjoin%
\pgfsetlinewidth{1.003750pt}%
\definecolor{currentstroke}{rgb}{0.000000,0.000000,0.000000}%
\pgfsetstrokecolor{currentstroke}%
\pgfsetdash{}{0pt}%
\pgfpathmoveto{\pgfqpoint{1.604818in}{0.499444in}}%
\pgfpathlineto{\pgfqpoint{1.666204in}{0.499444in}}%
\pgfpathlineto{\pgfqpoint{1.666204in}{1.138154in}}%
\pgfpathlineto{\pgfqpoint{1.604818in}{1.138154in}}%
\pgfpathlineto{\pgfqpoint{1.604818in}{0.499444in}}%
\pgfpathclose%
\pgfusepath{stroke}%
\end{pgfscope}%
\begin{pgfscope}%
\pgfpathrectangle{\pgfqpoint{0.553581in}{0.499444in}}{\pgfqpoint{3.875000in}{1.155000in}}%
\pgfusepath{clip}%
\pgfsetbuttcap%
\pgfsetmiterjoin%
\pgfsetlinewidth{1.003750pt}%
\definecolor{currentstroke}{rgb}{0.000000,0.000000,0.000000}%
\pgfsetstrokecolor{currentstroke}%
\pgfsetdash{}{0pt}%
\pgfpathmoveto{\pgfqpoint{1.758284in}{0.499444in}}%
\pgfpathlineto{\pgfqpoint{1.819670in}{0.499444in}}%
\pgfpathlineto{\pgfqpoint{1.819670in}{1.061463in}}%
\pgfpathlineto{\pgfqpoint{1.758284in}{1.061463in}}%
\pgfpathlineto{\pgfqpoint{1.758284in}{0.499444in}}%
\pgfpathclose%
\pgfusepath{stroke}%
\end{pgfscope}%
\begin{pgfscope}%
\pgfpathrectangle{\pgfqpoint{0.553581in}{0.499444in}}{\pgfqpoint{3.875000in}{1.155000in}}%
\pgfusepath{clip}%
\pgfsetbuttcap%
\pgfsetmiterjoin%
\pgfsetlinewidth{1.003750pt}%
\definecolor{currentstroke}{rgb}{0.000000,0.000000,0.000000}%
\pgfsetstrokecolor{currentstroke}%
\pgfsetdash{}{0pt}%
\pgfpathmoveto{\pgfqpoint{1.911749in}{0.499444in}}%
\pgfpathlineto{\pgfqpoint{1.973135in}{0.499444in}}%
\pgfpathlineto{\pgfqpoint{1.973135in}{0.989569in}}%
\pgfpathlineto{\pgfqpoint{1.911749in}{0.989569in}}%
\pgfpathlineto{\pgfqpoint{1.911749in}{0.499444in}}%
\pgfpathclose%
\pgfusepath{stroke}%
\end{pgfscope}%
\begin{pgfscope}%
\pgfpathrectangle{\pgfqpoint{0.553581in}{0.499444in}}{\pgfqpoint{3.875000in}{1.155000in}}%
\pgfusepath{clip}%
\pgfsetbuttcap%
\pgfsetmiterjoin%
\pgfsetlinewidth{1.003750pt}%
\definecolor{currentstroke}{rgb}{0.000000,0.000000,0.000000}%
\pgfsetstrokecolor{currentstroke}%
\pgfsetdash{}{0pt}%
\pgfpathmoveto{\pgfqpoint{2.065214in}{0.499444in}}%
\pgfpathlineto{\pgfqpoint{2.126600in}{0.499444in}}%
\pgfpathlineto{\pgfqpoint{2.126600in}{0.932064in}}%
\pgfpathlineto{\pgfqpoint{2.065214in}{0.932064in}}%
\pgfpathlineto{\pgfqpoint{2.065214in}{0.499444in}}%
\pgfpathclose%
\pgfusepath{stroke}%
\end{pgfscope}%
\begin{pgfscope}%
\pgfpathrectangle{\pgfqpoint{0.553581in}{0.499444in}}{\pgfqpoint{3.875000in}{1.155000in}}%
\pgfusepath{clip}%
\pgfsetbuttcap%
\pgfsetmiterjoin%
\pgfsetlinewidth{1.003750pt}%
\definecolor{currentstroke}{rgb}{0.000000,0.000000,0.000000}%
\pgfsetstrokecolor{currentstroke}%
\pgfsetdash{}{0pt}%
\pgfpathmoveto{\pgfqpoint{2.218680in}{0.499444in}}%
\pgfpathlineto{\pgfqpoint{2.280066in}{0.499444in}}%
\pgfpathlineto{\pgfqpoint{2.280066in}{0.864258in}}%
\pgfpathlineto{\pgfqpoint{2.218680in}{0.864258in}}%
\pgfpathlineto{\pgfqpoint{2.218680in}{0.499444in}}%
\pgfpathclose%
\pgfusepath{stroke}%
\end{pgfscope}%
\begin{pgfscope}%
\pgfpathrectangle{\pgfqpoint{0.553581in}{0.499444in}}{\pgfqpoint{3.875000in}{1.155000in}}%
\pgfusepath{clip}%
\pgfsetbuttcap%
\pgfsetmiterjoin%
\pgfsetlinewidth{1.003750pt}%
\definecolor{currentstroke}{rgb}{0.000000,0.000000,0.000000}%
\pgfsetstrokecolor{currentstroke}%
\pgfsetdash{}{0pt}%
\pgfpathmoveto{\pgfqpoint{2.372145in}{0.499444in}}%
\pgfpathlineto{\pgfqpoint{2.433531in}{0.499444in}}%
\pgfpathlineto{\pgfqpoint{2.433531in}{0.785877in}}%
\pgfpathlineto{\pgfqpoint{2.372145in}{0.785877in}}%
\pgfpathlineto{\pgfqpoint{2.372145in}{0.499444in}}%
\pgfpathclose%
\pgfusepath{stroke}%
\end{pgfscope}%
\begin{pgfscope}%
\pgfpathrectangle{\pgfqpoint{0.553581in}{0.499444in}}{\pgfqpoint{3.875000in}{1.155000in}}%
\pgfusepath{clip}%
\pgfsetbuttcap%
\pgfsetmiterjoin%
\pgfsetlinewidth{1.003750pt}%
\definecolor{currentstroke}{rgb}{0.000000,0.000000,0.000000}%
\pgfsetstrokecolor{currentstroke}%
\pgfsetdash{}{0pt}%
\pgfpathmoveto{\pgfqpoint{2.525610in}{0.499444in}}%
\pgfpathlineto{\pgfqpoint{2.586997in}{0.499444in}}%
\pgfpathlineto{\pgfqpoint{2.586997in}{0.716326in}}%
\pgfpathlineto{\pgfqpoint{2.525610in}{0.716326in}}%
\pgfpathlineto{\pgfqpoint{2.525610in}{0.499444in}}%
\pgfpathclose%
\pgfusepath{stroke}%
\end{pgfscope}%
\begin{pgfscope}%
\pgfpathrectangle{\pgfqpoint{0.553581in}{0.499444in}}{\pgfqpoint{3.875000in}{1.155000in}}%
\pgfusepath{clip}%
\pgfsetbuttcap%
\pgfsetmiterjoin%
\pgfsetlinewidth{1.003750pt}%
\definecolor{currentstroke}{rgb}{0.000000,0.000000,0.000000}%
\pgfsetstrokecolor{currentstroke}%
\pgfsetdash{}{0pt}%
\pgfpathmoveto{\pgfqpoint{2.679076in}{0.499444in}}%
\pgfpathlineto{\pgfqpoint{2.740462in}{0.499444in}}%
\pgfpathlineto{\pgfqpoint{2.740462in}{0.644650in}}%
\pgfpathlineto{\pgfqpoint{2.679076in}{0.644650in}}%
\pgfpathlineto{\pgfqpoint{2.679076in}{0.499444in}}%
\pgfpathclose%
\pgfusepath{stroke}%
\end{pgfscope}%
\begin{pgfscope}%
\pgfpathrectangle{\pgfqpoint{0.553581in}{0.499444in}}{\pgfqpoint{3.875000in}{1.155000in}}%
\pgfusepath{clip}%
\pgfsetbuttcap%
\pgfsetmiterjoin%
\pgfsetlinewidth{1.003750pt}%
\definecolor{currentstroke}{rgb}{0.000000,0.000000,0.000000}%
\pgfsetstrokecolor{currentstroke}%
\pgfsetdash{}{0pt}%
\pgfpathmoveto{\pgfqpoint{2.832541in}{0.499444in}}%
\pgfpathlineto{\pgfqpoint{2.893927in}{0.499444in}}%
\pgfpathlineto{\pgfqpoint{2.893927in}{0.582512in}}%
\pgfpathlineto{\pgfqpoint{2.832541in}{0.582512in}}%
\pgfpathlineto{\pgfqpoint{2.832541in}{0.499444in}}%
\pgfpathclose%
\pgfusepath{stroke}%
\end{pgfscope}%
\begin{pgfscope}%
\pgfpathrectangle{\pgfqpoint{0.553581in}{0.499444in}}{\pgfqpoint{3.875000in}{1.155000in}}%
\pgfusepath{clip}%
\pgfsetbuttcap%
\pgfsetmiterjoin%
\pgfsetlinewidth{1.003750pt}%
\definecolor{currentstroke}{rgb}{0.000000,0.000000,0.000000}%
\pgfsetstrokecolor{currentstroke}%
\pgfsetdash{}{0pt}%
\pgfpathmoveto{\pgfqpoint{2.986006in}{0.499444in}}%
\pgfpathlineto{\pgfqpoint{3.047393in}{0.499444in}}%
\pgfpathlineto{\pgfqpoint{3.047393in}{0.553842in}}%
\pgfpathlineto{\pgfqpoint{2.986006in}{0.553842in}}%
\pgfpathlineto{\pgfqpoint{2.986006in}{0.499444in}}%
\pgfpathclose%
\pgfusepath{stroke}%
\end{pgfscope}%
\begin{pgfscope}%
\pgfpathrectangle{\pgfqpoint{0.553581in}{0.499444in}}{\pgfqpoint{3.875000in}{1.155000in}}%
\pgfusepath{clip}%
\pgfsetbuttcap%
\pgfsetmiterjoin%
\pgfsetlinewidth{1.003750pt}%
\definecolor{currentstroke}{rgb}{0.000000,0.000000,0.000000}%
\pgfsetstrokecolor{currentstroke}%
\pgfsetdash{}{0pt}%
\pgfpathmoveto{\pgfqpoint{3.139472in}{0.499444in}}%
\pgfpathlineto{\pgfqpoint{3.200858in}{0.499444in}}%
\pgfpathlineto{\pgfqpoint{3.200858in}{0.504295in}}%
\pgfpathlineto{\pgfqpoint{3.139472in}{0.504295in}}%
\pgfpathlineto{\pgfqpoint{3.139472in}{0.499444in}}%
\pgfpathclose%
\pgfusepath{stroke}%
\end{pgfscope}%
\begin{pgfscope}%
\pgfpathrectangle{\pgfqpoint{0.553581in}{0.499444in}}{\pgfqpoint{3.875000in}{1.155000in}}%
\pgfusepath{clip}%
\pgfsetbuttcap%
\pgfsetmiterjoin%
\pgfsetlinewidth{1.003750pt}%
\definecolor{currentstroke}{rgb}{0.000000,0.000000,0.000000}%
\pgfsetstrokecolor{currentstroke}%
\pgfsetdash{}{0pt}%
\pgfpathmoveto{\pgfqpoint{3.292937in}{0.499444in}}%
\pgfpathlineto{\pgfqpoint{3.354323in}{0.499444in}}%
\pgfpathlineto{\pgfqpoint{3.354323in}{0.499444in}}%
\pgfpathlineto{\pgfqpoint{3.292937in}{0.499444in}}%
\pgfpathlineto{\pgfqpoint{3.292937in}{0.499444in}}%
\pgfpathclose%
\pgfusepath{stroke}%
\end{pgfscope}%
\begin{pgfscope}%
\pgfpathrectangle{\pgfqpoint{0.553581in}{0.499444in}}{\pgfqpoint{3.875000in}{1.155000in}}%
\pgfusepath{clip}%
\pgfsetbuttcap%
\pgfsetmiterjoin%
\pgfsetlinewidth{1.003750pt}%
\definecolor{currentstroke}{rgb}{0.000000,0.000000,0.000000}%
\pgfsetstrokecolor{currentstroke}%
\pgfsetdash{}{0pt}%
\pgfpathmoveto{\pgfqpoint{3.446402in}{0.499444in}}%
\pgfpathlineto{\pgfqpoint{3.507789in}{0.499444in}}%
\pgfpathlineto{\pgfqpoint{3.507789in}{0.499444in}}%
\pgfpathlineto{\pgfqpoint{3.446402in}{0.499444in}}%
\pgfpathlineto{\pgfqpoint{3.446402in}{0.499444in}}%
\pgfpathclose%
\pgfusepath{stroke}%
\end{pgfscope}%
\begin{pgfscope}%
\pgfpathrectangle{\pgfqpoint{0.553581in}{0.499444in}}{\pgfqpoint{3.875000in}{1.155000in}}%
\pgfusepath{clip}%
\pgfsetbuttcap%
\pgfsetmiterjoin%
\pgfsetlinewidth{1.003750pt}%
\definecolor{currentstroke}{rgb}{0.000000,0.000000,0.000000}%
\pgfsetstrokecolor{currentstroke}%
\pgfsetdash{}{0pt}%
\pgfpathmoveto{\pgfqpoint{3.599868in}{0.499444in}}%
\pgfpathlineto{\pgfqpoint{3.661254in}{0.499444in}}%
\pgfpathlineto{\pgfqpoint{3.661254in}{0.499444in}}%
\pgfpathlineto{\pgfqpoint{3.599868in}{0.499444in}}%
\pgfpathlineto{\pgfqpoint{3.599868in}{0.499444in}}%
\pgfpathclose%
\pgfusepath{stroke}%
\end{pgfscope}%
\begin{pgfscope}%
\pgfpathrectangle{\pgfqpoint{0.553581in}{0.499444in}}{\pgfqpoint{3.875000in}{1.155000in}}%
\pgfusepath{clip}%
\pgfsetbuttcap%
\pgfsetmiterjoin%
\pgfsetlinewidth{1.003750pt}%
\definecolor{currentstroke}{rgb}{0.000000,0.000000,0.000000}%
\pgfsetstrokecolor{currentstroke}%
\pgfsetdash{}{0pt}%
\pgfpathmoveto{\pgfqpoint{3.753333in}{0.499444in}}%
\pgfpathlineto{\pgfqpoint{3.814719in}{0.499444in}}%
\pgfpathlineto{\pgfqpoint{3.814719in}{0.499444in}}%
\pgfpathlineto{\pgfqpoint{3.753333in}{0.499444in}}%
\pgfpathlineto{\pgfqpoint{3.753333in}{0.499444in}}%
\pgfpathclose%
\pgfusepath{stroke}%
\end{pgfscope}%
\begin{pgfscope}%
\pgfpathrectangle{\pgfqpoint{0.553581in}{0.499444in}}{\pgfqpoint{3.875000in}{1.155000in}}%
\pgfusepath{clip}%
\pgfsetbuttcap%
\pgfsetmiterjoin%
\pgfsetlinewidth{1.003750pt}%
\definecolor{currentstroke}{rgb}{0.000000,0.000000,0.000000}%
\pgfsetstrokecolor{currentstroke}%
\pgfsetdash{}{0pt}%
\pgfpathmoveto{\pgfqpoint{3.906799in}{0.499444in}}%
\pgfpathlineto{\pgfqpoint{3.968185in}{0.499444in}}%
\pgfpathlineto{\pgfqpoint{3.968185in}{0.499444in}}%
\pgfpathlineto{\pgfqpoint{3.906799in}{0.499444in}}%
\pgfpathlineto{\pgfqpoint{3.906799in}{0.499444in}}%
\pgfpathclose%
\pgfusepath{stroke}%
\end{pgfscope}%
\begin{pgfscope}%
\pgfpathrectangle{\pgfqpoint{0.553581in}{0.499444in}}{\pgfqpoint{3.875000in}{1.155000in}}%
\pgfusepath{clip}%
\pgfsetbuttcap%
\pgfsetmiterjoin%
\pgfsetlinewidth{1.003750pt}%
\definecolor{currentstroke}{rgb}{0.000000,0.000000,0.000000}%
\pgfsetstrokecolor{currentstroke}%
\pgfsetdash{}{0pt}%
\pgfpathmoveto{\pgfqpoint{4.060264in}{0.499444in}}%
\pgfpathlineto{\pgfqpoint{4.121650in}{0.499444in}}%
\pgfpathlineto{\pgfqpoint{4.121650in}{0.499444in}}%
\pgfpathlineto{\pgfqpoint{4.060264in}{0.499444in}}%
\pgfpathlineto{\pgfqpoint{4.060264in}{0.499444in}}%
\pgfpathclose%
\pgfusepath{stroke}%
\end{pgfscope}%
\begin{pgfscope}%
\pgfpathrectangle{\pgfqpoint{0.553581in}{0.499444in}}{\pgfqpoint{3.875000in}{1.155000in}}%
\pgfusepath{clip}%
\pgfsetbuttcap%
\pgfsetmiterjoin%
\pgfsetlinewidth{1.003750pt}%
\definecolor{currentstroke}{rgb}{0.000000,0.000000,0.000000}%
\pgfsetstrokecolor{currentstroke}%
\pgfsetdash{}{0pt}%
\pgfpathmoveto{\pgfqpoint{4.213729in}{0.499444in}}%
\pgfpathlineto{\pgfqpoint{4.275115in}{0.499444in}}%
\pgfpathlineto{\pgfqpoint{4.275115in}{0.499444in}}%
\pgfpathlineto{\pgfqpoint{4.213729in}{0.499444in}}%
\pgfpathlineto{\pgfqpoint{4.213729in}{0.499444in}}%
\pgfpathclose%
\pgfusepath{stroke}%
\end{pgfscope}%
\begin{pgfscope}%
\pgfpathrectangle{\pgfqpoint{0.553581in}{0.499444in}}{\pgfqpoint{3.875000in}{1.155000in}}%
\pgfusepath{clip}%
\pgfsetbuttcap%
\pgfsetmiterjoin%
\definecolor{currentfill}{rgb}{0.000000,0.000000,0.000000}%
\pgfsetfillcolor{currentfill}%
\pgfsetlinewidth{0.000000pt}%
\definecolor{currentstroke}{rgb}{0.000000,0.000000,0.000000}%
\pgfsetstrokecolor{currentstroke}%
\pgfsetstrokeopacity{0.000000}%
\pgfsetdash{}{0pt}%
\pgfpathmoveto{\pgfqpoint{0.591947in}{0.499444in}}%
\pgfpathlineto{\pgfqpoint{0.653333in}{0.499444in}}%
\pgfpathlineto{\pgfqpoint{0.653333in}{0.512526in}}%
\pgfpathlineto{\pgfqpoint{0.591947in}{0.512526in}}%
\pgfpathlineto{\pgfqpoint{0.591947in}{0.499444in}}%
\pgfpathclose%
\pgfusepath{fill}%
\end{pgfscope}%
\begin{pgfscope}%
\pgfpathrectangle{\pgfqpoint{0.553581in}{0.499444in}}{\pgfqpoint{3.875000in}{1.155000in}}%
\pgfusepath{clip}%
\pgfsetbuttcap%
\pgfsetmiterjoin%
\definecolor{currentfill}{rgb}{0.000000,0.000000,0.000000}%
\pgfsetfillcolor{currentfill}%
\pgfsetlinewidth{0.000000pt}%
\definecolor{currentstroke}{rgb}{0.000000,0.000000,0.000000}%
\pgfsetstrokecolor{currentstroke}%
\pgfsetstrokeopacity{0.000000}%
\pgfsetdash{}{0pt}%
\pgfpathmoveto{\pgfqpoint{0.745412in}{0.499444in}}%
\pgfpathlineto{\pgfqpoint{0.806799in}{0.499444in}}%
\pgfpathlineto{\pgfqpoint{0.806799in}{0.534874in}}%
\pgfpathlineto{\pgfqpoint{0.745412in}{0.534874in}}%
\pgfpathlineto{\pgfqpoint{0.745412in}{0.499444in}}%
\pgfpathclose%
\pgfusepath{fill}%
\end{pgfscope}%
\begin{pgfscope}%
\pgfpathrectangle{\pgfqpoint{0.553581in}{0.499444in}}{\pgfqpoint{3.875000in}{1.155000in}}%
\pgfusepath{clip}%
\pgfsetbuttcap%
\pgfsetmiterjoin%
\definecolor{currentfill}{rgb}{0.000000,0.000000,0.000000}%
\pgfsetfillcolor{currentfill}%
\pgfsetlinewidth{0.000000pt}%
\definecolor{currentstroke}{rgb}{0.000000,0.000000,0.000000}%
\pgfsetstrokecolor{currentstroke}%
\pgfsetstrokeopacity{0.000000}%
\pgfsetdash{}{0pt}%
\pgfpathmoveto{\pgfqpoint{0.898878in}{0.499444in}}%
\pgfpathlineto{\pgfqpoint{0.960264in}{0.499444in}}%
\pgfpathlineto{\pgfqpoint{0.960264in}{0.553896in}}%
\pgfpathlineto{\pgfqpoint{0.898878in}{0.553896in}}%
\pgfpathlineto{\pgfqpoint{0.898878in}{0.499444in}}%
\pgfpathclose%
\pgfusepath{fill}%
\end{pgfscope}%
\begin{pgfscope}%
\pgfpathrectangle{\pgfqpoint{0.553581in}{0.499444in}}{\pgfqpoint{3.875000in}{1.155000in}}%
\pgfusepath{clip}%
\pgfsetbuttcap%
\pgfsetmiterjoin%
\definecolor{currentfill}{rgb}{0.000000,0.000000,0.000000}%
\pgfsetfillcolor{currentfill}%
\pgfsetlinewidth{0.000000pt}%
\definecolor{currentstroke}{rgb}{0.000000,0.000000,0.000000}%
\pgfsetstrokecolor{currentstroke}%
\pgfsetstrokeopacity{0.000000}%
\pgfsetdash{}{0pt}%
\pgfpathmoveto{\pgfqpoint{1.052343in}{0.499444in}}%
\pgfpathlineto{\pgfqpoint{1.113729in}{0.499444in}}%
\pgfpathlineto{\pgfqpoint{1.113729in}{0.572810in}}%
\pgfpathlineto{\pgfqpoint{1.052343in}{0.572810in}}%
\pgfpathlineto{\pgfqpoint{1.052343in}{0.499444in}}%
\pgfpathclose%
\pgfusepath{fill}%
\end{pgfscope}%
\begin{pgfscope}%
\pgfpathrectangle{\pgfqpoint{0.553581in}{0.499444in}}{\pgfqpoint{3.875000in}{1.155000in}}%
\pgfusepath{clip}%
\pgfsetbuttcap%
\pgfsetmiterjoin%
\definecolor{currentfill}{rgb}{0.000000,0.000000,0.000000}%
\pgfsetfillcolor{currentfill}%
\pgfsetlinewidth{0.000000pt}%
\definecolor{currentstroke}{rgb}{0.000000,0.000000,0.000000}%
\pgfsetstrokecolor{currentstroke}%
\pgfsetstrokeopacity{0.000000}%
\pgfsetdash{}{0pt}%
\pgfpathmoveto{\pgfqpoint{1.205808in}{0.499444in}}%
\pgfpathlineto{\pgfqpoint{1.267195in}{0.499444in}}%
\pgfpathlineto{\pgfqpoint{1.267195in}{0.587363in}}%
\pgfpathlineto{\pgfqpoint{1.205808in}{0.587363in}}%
\pgfpathlineto{\pgfqpoint{1.205808in}{0.499444in}}%
\pgfpathclose%
\pgfusepath{fill}%
\end{pgfscope}%
\begin{pgfscope}%
\pgfpathrectangle{\pgfqpoint{0.553581in}{0.499444in}}{\pgfqpoint{3.875000in}{1.155000in}}%
\pgfusepath{clip}%
\pgfsetbuttcap%
\pgfsetmiterjoin%
\definecolor{currentfill}{rgb}{0.000000,0.000000,0.000000}%
\pgfsetfillcolor{currentfill}%
\pgfsetlinewidth{0.000000pt}%
\definecolor{currentstroke}{rgb}{0.000000,0.000000,0.000000}%
\pgfsetstrokecolor{currentstroke}%
\pgfsetstrokeopacity{0.000000}%
\pgfsetdash{}{0pt}%
\pgfpathmoveto{\pgfqpoint{1.359274in}{0.499444in}}%
\pgfpathlineto{\pgfqpoint{1.420660in}{0.499444in}}%
\pgfpathlineto{\pgfqpoint{1.420660in}{0.599955in}}%
\pgfpathlineto{\pgfqpoint{1.359274in}{0.599955in}}%
\pgfpathlineto{\pgfqpoint{1.359274in}{0.499444in}}%
\pgfpathclose%
\pgfusepath{fill}%
\end{pgfscope}%
\begin{pgfscope}%
\pgfpathrectangle{\pgfqpoint{0.553581in}{0.499444in}}{\pgfqpoint{3.875000in}{1.155000in}}%
\pgfusepath{clip}%
\pgfsetbuttcap%
\pgfsetmiterjoin%
\definecolor{currentfill}{rgb}{0.000000,0.000000,0.000000}%
\pgfsetfillcolor{currentfill}%
\pgfsetlinewidth{0.000000pt}%
\definecolor{currentstroke}{rgb}{0.000000,0.000000,0.000000}%
\pgfsetstrokecolor{currentstroke}%
\pgfsetstrokeopacity{0.000000}%
\pgfsetdash{}{0pt}%
\pgfpathmoveto{\pgfqpoint{1.512739in}{0.499444in}}%
\pgfpathlineto{\pgfqpoint{1.574125in}{0.499444in}}%
\pgfpathlineto{\pgfqpoint{1.574125in}{0.610692in}}%
\pgfpathlineto{\pgfqpoint{1.512739in}{0.610692in}}%
\pgfpathlineto{\pgfqpoint{1.512739in}{0.499444in}}%
\pgfpathclose%
\pgfusepath{fill}%
\end{pgfscope}%
\begin{pgfscope}%
\pgfpathrectangle{\pgfqpoint{0.553581in}{0.499444in}}{\pgfqpoint{3.875000in}{1.155000in}}%
\pgfusepath{clip}%
\pgfsetbuttcap%
\pgfsetmiterjoin%
\definecolor{currentfill}{rgb}{0.000000,0.000000,0.000000}%
\pgfsetfillcolor{currentfill}%
\pgfsetlinewidth{0.000000pt}%
\definecolor{currentstroke}{rgb}{0.000000,0.000000,0.000000}%
\pgfsetstrokecolor{currentstroke}%
\pgfsetstrokeopacity{0.000000}%
\pgfsetdash{}{0pt}%
\pgfpathmoveto{\pgfqpoint{1.666204in}{0.499444in}}%
\pgfpathlineto{\pgfqpoint{1.727591in}{0.499444in}}%
\pgfpathlineto{\pgfqpoint{1.727591in}{0.622520in}}%
\pgfpathlineto{\pgfqpoint{1.666204in}{0.622520in}}%
\pgfpathlineto{\pgfqpoint{1.666204in}{0.499444in}}%
\pgfpathclose%
\pgfusepath{fill}%
\end{pgfscope}%
\begin{pgfscope}%
\pgfpathrectangle{\pgfqpoint{0.553581in}{0.499444in}}{\pgfqpoint{3.875000in}{1.155000in}}%
\pgfusepath{clip}%
\pgfsetbuttcap%
\pgfsetmiterjoin%
\definecolor{currentfill}{rgb}{0.000000,0.000000,0.000000}%
\pgfsetfillcolor{currentfill}%
\pgfsetlinewidth{0.000000pt}%
\definecolor{currentstroke}{rgb}{0.000000,0.000000,0.000000}%
\pgfsetstrokecolor{currentstroke}%
\pgfsetstrokeopacity{0.000000}%
\pgfsetdash{}{0pt}%
\pgfpathmoveto{\pgfqpoint{1.819670in}{0.499444in}}%
\pgfpathlineto{\pgfqpoint{1.881056in}{0.499444in}}%
\pgfpathlineto{\pgfqpoint{1.881056in}{0.626935in}}%
\pgfpathlineto{\pgfqpoint{1.819670in}{0.626935in}}%
\pgfpathlineto{\pgfqpoint{1.819670in}{0.499444in}}%
\pgfpathclose%
\pgfusepath{fill}%
\end{pgfscope}%
\begin{pgfscope}%
\pgfpathrectangle{\pgfqpoint{0.553581in}{0.499444in}}{\pgfqpoint{3.875000in}{1.155000in}}%
\pgfusepath{clip}%
\pgfsetbuttcap%
\pgfsetmiterjoin%
\definecolor{currentfill}{rgb}{0.000000,0.000000,0.000000}%
\pgfsetfillcolor{currentfill}%
\pgfsetlinewidth{0.000000pt}%
\definecolor{currentstroke}{rgb}{0.000000,0.000000,0.000000}%
\pgfsetstrokecolor{currentstroke}%
\pgfsetstrokeopacity{0.000000}%
\pgfsetdash{}{0pt}%
\pgfpathmoveto{\pgfqpoint{1.973135in}{0.499444in}}%
\pgfpathlineto{\pgfqpoint{2.034521in}{0.499444in}}%
\pgfpathlineto{\pgfqpoint{2.034521in}{0.635057in}}%
\pgfpathlineto{\pgfqpoint{1.973135in}{0.635057in}}%
\pgfpathlineto{\pgfqpoint{1.973135in}{0.499444in}}%
\pgfpathclose%
\pgfusepath{fill}%
\end{pgfscope}%
\begin{pgfscope}%
\pgfpathrectangle{\pgfqpoint{0.553581in}{0.499444in}}{\pgfqpoint{3.875000in}{1.155000in}}%
\pgfusepath{clip}%
\pgfsetbuttcap%
\pgfsetmiterjoin%
\definecolor{currentfill}{rgb}{0.000000,0.000000,0.000000}%
\pgfsetfillcolor{currentfill}%
\pgfsetlinewidth{0.000000pt}%
\definecolor{currentstroke}{rgb}{0.000000,0.000000,0.000000}%
\pgfsetstrokecolor{currentstroke}%
\pgfsetstrokeopacity{0.000000}%
\pgfsetdash{}{0pt}%
\pgfpathmoveto{\pgfqpoint{2.126600in}{0.499444in}}%
\pgfpathlineto{\pgfqpoint{2.187987in}{0.499444in}}%
\pgfpathlineto{\pgfqpoint{2.187987in}{0.643069in}}%
\pgfpathlineto{\pgfqpoint{2.126600in}{0.643069in}}%
\pgfpathlineto{\pgfqpoint{2.126600in}{0.499444in}}%
\pgfpathclose%
\pgfusepath{fill}%
\end{pgfscope}%
\begin{pgfscope}%
\pgfpathrectangle{\pgfqpoint{0.553581in}{0.499444in}}{\pgfqpoint{3.875000in}{1.155000in}}%
\pgfusepath{clip}%
\pgfsetbuttcap%
\pgfsetmiterjoin%
\definecolor{currentfill}{rgb}{0.000000,0.000000,0.000000}%
\pgfsetfillcolor{currentfill}%
\pgfsetlinewidth{0.000000pt}%
\definecolor{currentstroke}{rgb}{0.000000,0.000000,0.000000}%
\pgfsetstrokecolor{currentstroke}%
\pgfsetstrokeopacity{0.000000}%
\pgfsetdash{}{0pt}%
\pgfpathmoveto{\pgfqpoint{2.280066in}{0.499444in}}%
\pgfpathlineto{\pgfqpoint{2.341452in}{0.499444in}}%
\pgfpathlineto{\pgfqpoint{2.341452in}{0.649229in}}%
\pgfpathlineto{\pgfqpoint{2.280066in}{0.649229in}}%
\pgfpathlineto{\pgfqpoint{2.280066in}{0.499444in}}%
\pgfpathclose%
\pgfusepath{fill}%
\end{pgfscope}%
\begin{pgfscope}%
\pgfpathrectangle{\pgfqpoint{0.553581in}{0.499444in}}{\pgfqpoint{3.875000in}{1.155000in}}%
\pgfusepath{clip}%
\pgfsetbuttcap%
\pgfsetmiterjoin%
\definecolor{currentfill}{rgb}{0.000000,0.000000,0.000000}%
\pgfsetfillcolor{currentfill}%
\pgfsetlinewidth{0.000000pt}%
\definecolor{currentstroke}{rgb}{0.000000,0.000000,0.000000}%
\pgfsetstrokecolor{currentstroke}%
\pgfsetstrokeopacity{0.000000}%
\pgfsetdash{}{0pt}%
\pgfpathmoveto{\pgfqpoint{2.433531in}{0.499444in}}%
\pgfpathlineto{\pgfqpoint{2.494917in}{0.499444in}}%
\pgfpathlineto{\pgfqpoint{2.494917in}{0.648847in}}%
\pgfpathlineto{\pgfqpoint{2.433531in}{0.648847in}}%
\pgfpathlineto{\pgfqpoint{2.433531in}{0.499444in}}%
\pgfpathclose%
\pgfusepath{fill}%
\end{pgfscope}%
\begin{pgfscope}%
\pgfpathrectangle{\pgfqpoint{0.553581in}{0.499444in}}{\pgfqpoint{3.875000in}{1.155000in}}%
\pgfusepath{clip}%
\pgfsetbuttcap%
\pgfsetmiterjoin%
\definecolor{currentfill}{rgb}{0.000000,0.000000,0.000000}%
\pgfsetfillcolor{currentfill}%
\pgfsetlinewidth{0.000000pt}%
\definecolor{currentstroke}{rgb}{0.000000,0.000000,0.000000}%
\pgfsetstrokecolor{currentstroke}%
\pgfsetstrokeopacity{0.000000}%
\pgfsetdash{}{0pt}%
\pgfpathmoveto{\pgfqpoint{2.586997in}{0.499444in}}%
\pgfpathlineto{\pgfqpoint{2.648383in}{0.499444in}}%
\pgfpathlineto{\pgfqpoint{2.648383in}{0.646939in}}%
\pgfpathlineto{\pgfqpoint{2.586997in}{0.646939in}}%
\pgfpathlineto{\pgfqpoint{2.586997in}{0.499444in}}%
\pgfpathclose%
\pgfusepath{fill}%
\end{pgfscope}%
\begin{pgfscope}%
\pgfpathrectangle{\pgfqpoint{0.553581in}{0.499444in}}{\pgfqpoint{3.875000in}{1.155000in}}%
\pgfusepath{clip}%
\pgfsetbuttcap%
\pgfsetmiterjoin%
\definecolor{currentfill}{rgb}{0.000000,0.000000,0.000000}%
\pgfsetfillcolor{currentfill}%
\pgfsetlinewidth{0.000000pt}%
\definecolor{currentstroke}{rgb}{0.000000,0.000000,0.000000}%
\pgfsetstrokecolor{currentstroke}%
\pgfsetstrokeopacity{0.000000}%
\pgfsetdash{}{0pt}%
\pgfpathmoveto{\pgfqpoint{2.740462in}{0.499444in}}%
\pgfpathlineto{\pgfqpoint{2.801848in}{0.499444in}}%
\pgfpathlineto{\pgfqpoint{2.801848in}{0.629988in}}%
\pgfpathlineto{\pgfqpoint{2.740462in}{0.629988in}}%
\pgfpathlineto{\pgfqpoint{2.740462in}{0.499444in}}%
\pgfpathclose%
\pgfusepath{fill}%
\end{pgfscope}%
\begin{pgfscope}%
\pgfpathrectangle{\pgfqpoint{0.553581in}{0.499444in}}{\pgfqpoint{3.875000in}{1.155000in}}%
\pgfusepath{clip}%
\pgfsetbuttcap%
\pgfsetmiterjoin%
\definecolor{currentfill}{rgb}{0.000000,0.000000,0.000000}%
\pgfsetfillcolor{currentfill}%
\pgfsetlinewidth{0.000000pt}%
\definecolor{currentstroke}{rgb}{0.000000,0.000000,0.000000}%
\pgfsetstrokecolor{currentstroke}%
\pgfsetstrokeopacity{0.000000}%
\pgfsetdash{}{0pt}%
\pgfpathmoveto{\pgfqpoint{2.893927in}{0.499444in}}%
\pgfpathlineto{\pgfqpoint{2.955313in}{0.499444in}}%
\pgfpathlineto{\pgfqpoint{2.955313in}{0.614235in}}%
\pgfpathlineto{\pgfqpoint{2.893927in}{0.614235in}}%
\pgfpathlineto{\pgfqpoint{2.893927in}{0.499444in}}%
\pgfpathclose%
\pgfusepath{fill}%
\end{pgfscope}%
\begin{pgfscope}%
\pgfpathrectangle{\pgfqpoint{0.553581in}{0.499444in}}{\pgfqpoint{3.875000in}{1.155000in}}%
\pgfusepath{clip}%
\pgfsetbuttcap%
\pgfsetmiterjoin%
\definecolor{currentfill}{rgb}{0.000000,0.000000,0.000000}%
\pgfsetfillcolor{currentfill}%
\pgfsetlinewidth{0.000000pt}%
\definecolor{currentstroke}{rgb}{0.000000,0.000000,0.000000}%
\pgfsetstrokecolor{currentstroke}%
\pgfsetstrokeopacity{0.000000}%
\pgfsetdash{}{0pt}%
\pgfpathmoveto{\pgfqpoint{3.047393in}{0.499444in}}%
\pgfpathlineto{\pgfqpoint{3.108779in}{0.499444in}}%
\pgfpathlineto{\pgfqpoint{3.108779in}{0.624537in}}%
\pgfpathlineto{\pgfqpoint{3.047393in}{0.624537in}}%
\pgfpathlineto{\pgfqpoint{3.047393in}{0.499444in}}%
\pgfpathclose%
\pgfusepath{fill}%
\end{pgfscope}%
\begin{pgfscope}%
\pgfpathrectangle{\pgfqpoint{0.553581in}{0.499444in}}{\pgfqpoint{3.875000in}{1.155000in}}%
\pgfusepath{clip}%
\pgfsetbuttcap%
\pgfsetmiterjoin%
\definecolor{currentfill}{rgb}{0.000000,0.000000,0.000000}%
\pgfsetfillcolor{currentfill}%
\pgfsetlinewidth{0.000000pt}%
\definecolor{currentstroke}{rgb}{0.000000,0.000000,0.000000}%
\pgfsetstrokecolor{currentstroke}%
\pgfsetstrokeopacity{0.000000}%
\pgfsetdash{}{0pt}%
\pgfpathmoveto{\pgfqpoint{3.200858in}{0.499444in}}%
\pgfpathlineto{\pgfqpoint{3.262244in}{0.499444in}}%
\pgfpathlineto{\pgfqpoint{3.262244in}{0.519939in}}%
\pgfpathlineto{\pgfqpoint{3.200858in}{0.519939in}}%
\pgfpathlineto{\pgfqpoint{3.200858in}{0.499444in}}%
\pgfpathclose%
\pgfusepath{fill}%
\end{pgfscope}%
\begin{pgfscope}%
\pgfpathrectangle{\pgfqpoint{0.553581in}{0.499444in}}{\pgfqpoint{3.875000in}{1.155000in}}%
\pgfusepath{clip}%
\pgfsetbuttcap%
\pgfsetmiterjoin%
\definecolor{currentfill}{rgb}{0.000000,0.000000,0.000000}%
\pgfsetfillcolor{currentfill}%
\pgfsetlinewidth{0.000000pt}%
\definecolor{currentstroke}{rgb}{0.000000,0.000000,0.000000}%
\pgfsetstrokecolor{currentstroke}%
\pgfsetstrokeopacity{0.000000}%
\pgfsetdash{}{0pt}%
\pgfpathmoveto{\pgfqpoint{3.354323in}{0.499444in}}%
\pgfpathlineto{\pgfqpoint{3.415709in}{0.499444in}}%
\pgfpathlineto{\pgfqpoint{3.415709in}{0.499444in}}%
\pgfpathlineto{\pgfqpoint{3.354323in}{0.499444in}}%
\pgfpathlineto{\pgfqpoint{3.354323in}{0.499444in}}%
\pgfpathclose%
\pgfusepath{fill}%
\end{pgfscope}%
\begin{pgfscope}%
\pgfpathrectangle{\pgfqpoint{0.553581in}{0.499444in}}{\pgfqpoint{3.875000in}{1.155000in}}%
\pgfusepath{clip}%
\pgfsetbuttcap%
\pgfsetmiterjoin%
\definecolor{currentfill}{rgb}{0.000000,0.000000,0.000000}%
\pgfsetfillcolor{currentfill}%
\pgfsetlinewidth{0.000000pt}%
\definecolor{currentstroke}{rgb}{0.000000,0.000000,0.000000}%
\pgfsetstrokecolor{currentstroke}%
\pgfsetstrokeopacity{0.000000}%
\pgfsetdash{}{0pt}%
\pgfpathmoveto{\pgfqpoint{3.507789in}{0.499444in}}%
\pgfpathlineto{\pgfqpoint{3.569175in}{0.499444in}}%
\pgfpathlineto{\pgfqpoint{3.569175in}{0.499444in}}%
\pgfpathlineto{\pgfqpoint{3.507789in}{0.499444in}}%
\pgfpathlineto{\pgfqpoint{3.507789in}{0.499444in}}%
\pgfpathclose%
\pgfusepath{fill}%
\end{pgfscope}%
\begin{pgfscope}%
\pgfpathrectangle{\pgfqpoint{0.553581in}{0.499444in}}{\pgfqpoint{3.875000in}{1.155000in}}%
\pgfusepath{clip}%
\pgfsetbuttcap%
\pgfsetmiterjoin%
\definecolor{currentfill}{rgb}{0.000000,0.000000,0.000000}%
\pgfsetfillcolor{currentfill}%
\pgfsetlinewidth{0.000000pt}%
\definecolor{currentstroke}{rgb}{0.000000,0.000000,0.000000}%
\pgfsetstrokecolor{currentstroke}%
\pgfsetstrokeopacity{0.000000}%
\pgfsetdash{}{0pt}%
\pgfpathmoveto{\pgfqpoint{3.661254in}{0.499444in}}%
\pgfpathlineto{\pgfqpoint{3.722640in}{0.499444in}}%
\pgfpathlineto{\pgfqpoint{3.722640in}{0.499444in}}%
\pgfpathlineto{\pgfqpoint{3.661254in}{0.499444in}}%
\pgfpathlineto{\pgfqpoint{3.661254in}{0.499444in}}%
\pgfpathclose%
\pgfusepath{fill}%
\end{pgfscope}%
\begin{pgfscope}%
\pgfpathrectangle{\pgfqpoint{0.553581in}{0.499444in}}{\pgfqpoint{3.875000in}{1.155000in}}%
\pgfusepath{clip}%
\pgfsetbuttcap%
\pgfsetmiterjoin%
\definecolor{currentfill}{rgb}{0.000000,0.000000,0.000000}%
\pgfsetfillcolor{currentfill}%
\pgfsetlinewidth{0.000000pt}%
\definecolor{currentstroke}{rgb}{0.000000,0.000000,0.000000}%
\pgfsetstrokecolor{currentstroke}%
\pgfsetstrokeopacity{0.000000}%
\pgfsetdash{}{0pt}%
\pgfpathmoveto{\pgfqpoint{3.814719in}{0.499444in}}%
\pgfpathlineto{\pgfqpoint{3.876105in}{0.499444in}}%
\pgfpathlineto{\pgfqpoint{3.876105in}{0.499444in}}%
\pgfpathlineto{\pgfqpoint{3.814719in}{0.499444in}}%
\pgfpathlineto{\pgfqpoint{3.814719in}{0.499444in}}%
\pgfpathclose%
\pgfusepath{fill}%
\end{pgfscope}%
\begin{pgfscope}%
\pgfpathrectangle{\pgfqpoint{0.553581in}{0.499444in}}{\pgfqpoint{3.875000in}{1.155000in}}%
\pgfusepath{clip}%
\pgfsetbuttcap%
\pgfsetmiterjoin%
\definecolor{currentfill}{rgb}{0.000000,0.000000,0.000000}%
\pgfsetfillcolor{currentfill}%
\pgfsetlinewidth{0.000000pt}%
\definecolor{currentstroke}{rgb}{0.000000,0.000000,0.000000}%
\pgfsetstrokecolor{currentstroke}%
\pgfsetstrokeopacity{0.000000}%
\pgfsetdash{}{0pt}%
\pgfpathmoveto{\pgfqpoint{3.968185in}{0.499444in}}%
\pgfpathlineto{\pgfqpoint{4.029571in}{0.499444in}}%
\pgfpathlineto{\pgfqpoint{4.029571in}{0.499444in}}%
\pgfpathlineto{\pgfqpoint{3.968185in}{0.499444in}}%
\pgfpathlineto{\pgfqpoint{3.968185in}{0.499444in}}%
\pgfpathclose%
\pgfusepath{fill}%
\end{pgfscope}%
\begin{pgfscope}%
\pgfpathrectangle{\pgfqpoint{0.553581in}{0.499444in}}{\pgfqpoint{3.875000in}{1.155000in}}%
\pgfusepath{clip}%
\pgfsetbuttcap%
\pgfsetmiterjoin%
\definecolor{currentfill}{rgb}{0.000000,0.000000,0.000000}%
\pgfsetfillcolor{currentfill}%
\pgfsetlinewidth{0.000000pt}%
\definecolor{currentstroke}{rgb}{0.000000,0.000000,0.000000}%
\pgfsetstrokecolor{currentstroke}%
\pgfsetstrokeopacity{0.000000}%
\pgfsetdash{}{0pt}%
\pgfpathmoveto{\pgfqpoint{4.121650in}{0.499444in}}%
\pgfpathlineto{\pgfqpoint{4.183036in}{0.499444in}}%
\pgfpathlineto{\pgfqpoint{4.183036in}{0.499444in}}%
\pgfpathlineto{\pgfqpoint{4.121650in}{0.499444in}}%
\pgfpathlineto{\pgfqpoint{4.121650in}{0.499444in}}%
\pgfpathclose%
\pgfusepath{fill}%
\end{pgfscope}%
\begin{pgfscope}%
\pgfpathrectangle{\pgfqpoint{0.553581in}{0.499444in}}{\pgfqpoint{3.875000in}{1.155000in}}%
\pgfusepath{clip}%
\pgfsetbuttcap%
\pgfsetmiterjoin%
\definecolor{currentfill}{rgb}{0.000000,0.000000,0.000000}%
\pgfsetfillcolor{currentfill}%
\pgfsetlinewidth{0.000000pt}%
\definecolor{currentstroke}{rgb}{0.000000,0.000000,0.000000}%
\pgfsetstrokecolor{currentstroke}%
\pgfsetstrokeopacity{0.000000}%
\pgfsetdash{}{0pt}%
\pgfpathmoveto{\pgfqpoint{4.275115in}{0.499444in}}%
\pgfpathlineto{\pgfqpoint{4.336501in}{0.499444in}}%
\pgfpathlineto{\pgfqpoint{4.336501in}{0.499444in}}%
\pgfpathlineto{\pgfqpoint{4.275115in}{0.499444in}}%
\pgfpathlineto{\pgfqpoint{4.275115in}{0.499444in}}%
\pgfpathclose%
\pgfusepath{fill}%
\end{pgfscope}%
\begin{pgfscope}%
\pgfsetbuttcap%
\pgfsetroundjoin%
\definecolor{currentfill}{rgb}{0.000000,0.000000,0.000000}%
\pgfsetfillcolor{currentfill}%
\pgfsetlinewidth{0.803000pt}%
\definecolor{currentstroke}{rgb}{0.000000,0.000000,0.000000}%
\pgfsetstrokecolor{currentstroke}%
\pgfsetdash{}{0pt}%
\pgfsys@defobject{currentmarker}{\pgfqpoint{0.000000in}{-0.048611in}}{\pgfqpoint{0.000000in}{0.000000in}}{%
\pgfpathmoveto{\pgfqpoint{0.000000in}{0.000000in}}%
\pgfpathlineto{\pgfqpoint{0.000000in}{-0.048611in}}%
\pgfusepath{stroke,fill}%
}%
\begin{pgfscope}%
\pgfsys@transformshift{0.591947in}{0.499444in}%
\pgfsys@useobject{currentmarker}{}%
\end{pgfscope}%
\end{pgfscope}%
\begin{pgfscope}%
\definecolor{textcolor}{rgb}{0.000000,0.000000,0.000000}%
\pgfsetstrokecolor{textcolor}%
\pgfsetfillcolor{textcolor}%
\pgftext[x=0.591947in,y=0.402222in,,top]{\color{textcolor}\rmfamily\fontsize{10.000000}{12.000000}\selectfont 0.0}%
\end{pgfscope}%
\begin{pgfscope}%
\pgfsetbuttcap%
\pgfsetroundjoin%
\definecolor{currentfill}{rgb}{0.000000,0.000000,0.000000}%
\pgfsetfillcolor{currentfill}%
\pgfsetlinewidth{0.803000pt}%
\definecolor{currentstroke}{rgb}{0.000000,0.000000,0.000000}%
\pgfsetstrokecolor{currentstroke}%
\pgfsetdash{}{0pt}%
\pgfsys@defobject{currentmarker}{\pgfqpoint{0.000000in}{-0.048611in}}{\pgfqpoint{0.000000in}{0.000000in}}{%
\pgfpathmoveto{\pgfqpoint{0.000000in}{0.000000in}}%
\pgfpathlineto{\pgfqpoint{0.000000in}{-0.048611in}}%
\pgfusepath{stroke,fill}%
}%
\begin{pgfscope}%
\pgfsys@transformshift{0.975610in}{0.499444in}%
\pgfsys@useobject{currentmarker}{}%
\end{pgfscope}%
\end{pgfscope}%
\begin{pgfscope}%
\definecolor{textcolor}{rgb}{0.000000,0.000000,0.000000}%
\pgfsetstrokecolor{textcolor}%
\pgfsetfillcolor{textcolor}%
\pgftext[x=0.975610in,y=0.402222in,,top]{\color{textcolor}\rmfamily\fontsize{10.000000}{12.000000}\selectfont 0.1}%
\end{pgfscope}%
\begin{pgfscope}%
\pgfsetbuttcap%
\pgfsetroundjoin%
\definecolor{currentfill}{rgb}{0.000000,0.000000,0.000000}%
\pgfsetfillcolor{currentfill}%
\pgfsetlinewidth{0.803000pt}%
\definecolor{currentstroke}{rgb}{0.000000,0.000000,0.000000}%
\pgfsetstrokecolor{currentstroke}%
\pgfsetdash{}{0pt}%
\pgfsys@defobject{currentmarker}{\pgfqpoint{0.000000in}{-0.048611in}}{\pgfqpoint{0.000000in}{0.000000in}}{%
\pgfpathmoveto{\pgfqpoint{0.000000in}{0.000000in}}%
\pgfpathlineto{\pgfqpoint{0.000000in}{-0.048611in}}%
\pgfusepath{stroke,fill}%
}%
\begin{pgfscope}%
\pgfsys@transformshift{1.359274in}{0.499444in}%
\pgfsys@useobject{currentmarker}{}%
\end{pgfscope}%
\end{pgfscope}%
\begin{pgfscope}%
\definecolor{textcolor}{rgb}{0.000000,0.000000,0.000000}%
\pgfsetstrokecolor{textcolor}%
\pgfsetfillcolor{textcolor}%
\pgftext[x=1.359274in,y=0.402222in,,top]{\color{textcolor}\rmfamily\fontsize{10.000000}{12.000000}\selectfont 0.2}%
\end{pgfscope}%
\begin{pgfscope}%
\pgfsetbuttcap%
\pgfsetroundjoin%
\definecolor{currentfill}{rgb}{0.000000,0.000000,0.000000}%
\pgfsetfillcolor{currentfill}%
\pgfsetlinewidth{0.803000pt}%
\definecolor{currentstroke}{rgb}{0.000000,0.000000,0.000000}%
\pgfsetstrokecolor{currentstroke}%
\pgfsetdash{}{0pt}%
\pgfsys@defobject{currentmarker}{\pgfqpoint{0.000000in}{-0.048611in}}{\pgfqpoint{0.000000in}{0.000000in}}{%
\pgfpathmoveto{\pgfqpoint{0.000000in}{0.000000in}}%
\pgfpathlineto{\pgfqpoint{0.000000in}{-0.048611in}}%
\pgfusepath{stroke,fill}%
}%
\begin{pgfscope}%
\pgfsys@transformshift{1.742937in}{0.499444in}%
\pgfsys@useobject{currentmarker}{}%
\end{pgfscope}%
\end{pgfscope}%
\begin{pgfscope}%
\definecolor{textcolor}{rgb}{0.000000,0.000000,0.000000}%
\pgfsetstrokecolor{textcolor}%
\pgfsetfillcolor{textcolor}%
\pgftext[x=1.742937in,y=0.402222in,,top]{\color{textcolor}\rmfamily\fontsize{10.000000}{12.000000}\selectfont 0.3}%
\end{pgfscope}%
\begin{pgfscope}%
\pgfsetbuttcap%
\pgfsetroundjoin%
\definecolor{currentfill}{rgb}{0.000000,0.000000,0.000000}%
\pgfsetfillcolor{currentfill}%
\pgfsetlinewidth{0.803000pt}%
\definecolor{currentstroke}{rgb}{0.000000,0.000000,0.000000}%
\pgfsetstrokecolor{currentstroke}%
\pgfsetdash{}{0pt}%
\pgfsys@defobject{currentmarker}{\pgfqpoint{0.000000in}{-0.048611in}}{\pgfqpoint{0.000000in}{0.000000in}}{%
\pgfpathmoveto{\pgfqpoint{0.000000in}{0.000000in}}%
\pgfpathlineto{\pgfqpoint{0.000000in}{-0.048611in}}%
\pgfusepath{stroke,fill}%
}%
\begin{pgfscope}%
\pgfsys@transformshift{2.126600in}{0.499444in}%
\pgfsys@useobject{currentmarker}{}%
\end{pgfscope}%
\end{pgfscope}%
\begin{pgfscope}%
\definecolor{textcolor}{rgb}{0.000000,0.000000,0.000000}%
\pgfsetstrokecolor{textcolor}%
\pgfsetfillcolor{textcolor}%
\pgftext[x=2.126600in,y=0.402222in,,top]{\color{textcolor}\rmfamily\fontsize{10.000000}{12.000000}\selectfont 0.4}%
\end{pgfscope}%
\begin{pgfscope}%
\pgfsetbuttcap%
\pgfsetroundjoin%
\definecolor{currentfill}{rgb}{0.000000,0.000000,0.000000}%
\pgfsetfillcolor{currentfill}%
\pgfsetlinewidth{0.803000pt}%
\definecolor{currentstroke}{rgb}{0.000000,0.000000,0.000000}%
\pgfsetstrokecolor{currentstroke}%
\pgfsetdash{}{0pt}%
\pgfsys@defobject{currentmarker}{\pgfqpoint{0.000000in}{-0.048611in}}{\pgfqpoint{0.000000in}{0.000000in}}{%
\pgfpathmoveto{\pgfqpoint{0.000000in}{0.000000in}}%
\pgfpathlineto{\pgfqpoint{0.000000in}{-0.048611in}}%
\pgfusepath{stroke,fill}%
}%
\begin{pgfscope}%
\pgfsys@transformshift{2.510264in}{0.499444in}%
\pgfsys@useobject{currentmarker}{}%
\end{pgfscope}%
\end{pgfscope}%
\begin{pgfscope}%
\definecolor{textcolor}{rgb}{0.000000,0.000000,0.000000}%
\pgfsetstrokecolor{textcolor}%
\pgfsetfillcolor{textcolor}%
\pgftext[x=2.510264in,y=0.402222in,,top]{\color{textcolor}\rmfamily\fontsize{10.000000}{12.000000}\selectfont 0.5}%
\end{pgfscope}%
\begin{pgfscope}%
\pgfsetbuttcap%
\pgfsetroundjoin%
\definecolor{currentfill}{rgb}{0.000000,0.000000,0.000000}%
\pgfsetfillcolor{currentfill}%
\pgfsetlinewidth{0.803000pt}%
\definecolor{currentstroke}{rgb}{0.000000,0.000000,0.000000}%
\pgfsetstrokecolor{currentstroke}%
\pgfsetdash{}{0pt}%
\pgfsys@defobject{currentmarker}{\pgfqpoint{0.000000in}{-0.048611in}}{\pgfqpoint{0.000000in}{0.000000in}}{%
\pgfpathmoveto{\pgfqpoint{0.000000in}{0.000000in}}%
\pgfpathlineto{\pgfqpoint{0.000000in}{-0.048611in}}%
\pgfusepath{stroke,fill}%
}%
\begin{pgfscope}%
\pgfsys@transformshift{2.893927in}{0.499444in}%
\pgfsys@useobject{currentmarker}{}%
\end{pgfscope}%
\end{pgfscope}%
\begin{pgfscope}%
\definecolor{textcolor}{rgb}{0.000000,0.000000,0.000000}%
\pgfsetstrokecolor{textcolor}%
\pgfsetfillcolor{textcolor}%
\pgftext[x=2.893927in,y=0.402222in,,top]{\color{textcolor}\rmfamily\fontsize{10.000000}{12.000000}\selectfont 0.6}%
\end{pgfscope}%
\begin{pgfscope}%
\pgfsetbuttcap%
\pgfsetroundjoin%
\definecolor{currentfill}{rgb}{0.000000,0.000000,0.000000}%
\pgfsetfillcolor{currentfill}%
\pgfsetlinewidth{0.803000pt}%
\definecolor{currentstroke}{rgb}{0.000000,0.000000,0.000000}%
\pgfsetstrokecolor{currentstroke}%
\pgfsetdash{}{0pt}%
\pgfsys@defobject{currentmarker}{\pgfqpoint{0.000000in}{-0.048611in}}{\pgfqpoint{0.000000in}{0.000000in}}{%
\pgfpathmoveto{\pgfqpoint{0.000000in}{0.000000in}}%
\pgfpathlineto{\pgfqpoint{0.000000in}{-0.048611in}}%
\pgfusepath{stroke,fill}%
}%
\begin{pgfscope}%
\pgfsys@transformshift{3.277591in}{0.499444in}%
\pgfsys@useobject{currentmarker}{}%
\end{pgfscope}%
\end{pgfscope}%
\begin{pgfscope}%
\definecolor{textcolor}{rgb}{0.000000,0.000000,0.000000}%
\pgfsetstrokecolor{textcolor}%
\pgfsetfillcolor{textcolor}%
\pgftext[x=3.277591in,y=0.402222in,,top]{\color{textcolor}\rmfamily\fontsize{10.000000}{12.000000}\selectfont 0.7}%
\end{pgfscope}%
\begin{pgfscope}%
\pgfsetbuttcap%
\pgfsetroundjoin%
\definecolor{currentfill}{rgb}{0.000000,0.000000,0.000000}%
\pgfsetfillcolor{currentfill}%
\pgfsetlinewidth{0.803000pt}%
\definecolor{currentstroke}{rgb}{0.000000,0.000000,0.000000}%
\pgfsetstrokecolor{currentstroke}%
\pgfsetdash{}{0pt}%
\pgfsys@defobject{currentmarker}{\pgfqpoint{0.000000in}{-0.048611in}}{\pgfqpoint{0.000000in}{0.000000in}}{%
\pgfpathmoveto{\pgfqpoint{0.000000in}{0.000000in}}%
\pgfpathlineto{\pgfqpoint{0.000000in}{-0.048611in}}%
\pgfusepath{stroke,fill}%
}%
\begin{pgfscope}%
\pgfsys@transformshift{3.661254in}{0.499444in}%
\pgfsys@useobject{currentmarker}{}%
\end{pgfscope}%
\end{pgfscope}%
\begin{pgfscope}%
\definecolor{textcolor}{rgb}{0.000000,0.000000,0.000000}%
\pgfsetstrokecolor{textcolor}%
\pgfsetfillcolor{textcolor}%
\pgftext[x=3.661254in,y=0.402222in,,top]{\color{textcolor}\rmfamily\fontsize{10.000000}{12.000000}\selectfont 0.8}%
\end{pgfscope}%
\begin{pgfscope}%
\pgfsetbuttcap%
\pgfsetroundjoin%
\definecolor{currentfill}{rgb}{0.000000,0.000000,0.000000}%
\pgfsetfillcolor{currentfill}%
\pgfsetlinewidth{0.803000pt}%
\definecolor{currentstroke}{rgb}{0.000000,0.000000,0.000000}%
\pgfsetstrokecolor{currentstroke}%
\pgfsetdash{}{0pt}%
\pgfsys@defobject{currentmarker}{\pgfqpoint{0.000000in}{-0.048611in}}{\pgfqpoint{0.000000in}{0.000000in}}{%
\pgfpathmoveto{\pgfqpoint{0.000000in}{0.000000in}}%
\pgfpathlineto{\pgfqpoint{0.000000in}{-0.048611in}}%
\pgfusepath{stroke,fill}%
}%
\begin{pgfscope}%
\pgfsys@transformshift{4.044917in}{0.499444in}%
\pgfsys@useobject{currentmarker}{}%
\end{pgfscope}%
\end{pgfscope}%
\begin{pgfscope}%
\definecolor{textcolor}{rgb}{0.000000,0.000000,0.000000}%
\pgfsetstrokecolor{textcolor}%
\pgfsetfillcolor{textcolor}%
\pgftext[x=4.044917in,y=0.402222in,,top]{\color{textcolor}\rmfamily\fontsize{10.000000}{12.000000}\selectfont 0.9}%
\end{pgfscope}%
\begin{pgfscope}%
\pgfsetbuttcap%
\pgfsetroundjoin%
\definecolor{currentfill}{rgb}{0.000000,0.000000,0.000000}%
\pgfsetfillcolor{currentfill}%
\pgfsetlinewidth{0.803000pt}%
\definecolor{currentstroke}{rgb}{0.000000,0.000000,0.000000}%
\pgfsetstrokecolor{currentstroke}%
\pgfsetdash{}{0pt}%
\pgfsys@defobject{currentmarker}{\pgfqpoint{0.000000in}{-0.048611in}}{\pgfqpoint{0.000000in}{0.000000in}}{%
\pgfpathmoveto{\pgfqpoint{0.000000in}{0.000000in}}%
\pgfpathlineto{\pgfqpoint{0.000000in}{-0.048611in}}%
\pgfusepath{stroke,fill}%
}%
\begin{pgfscope}%
\pgfsys@transformshift{4.428581in}{0.499444in}%
\pgfsys@useobject{currentmarker}{}%
\end{pgfscope}%
\end{pgfscope}%
\begin{pgfscope}%
\definecolor{textcolor}{rgb}{0.000000,0.000000,0.000000}%
\pgfsetstrokecolor{textcolor}%
\pgfsetfillcolor{textcolor}%
\pgftext[x=4.428581in,y=0.402222in,,top]{\color{textcolor}\rmfamily\fontsize{10.000000}{12.000000}\selectfont 1.0}%
\end{pgfscope}%
\begin{pgfscope}%
\definecolor{textcolor}{rgb}{0.000000,0.000000,0.000000}%
\pgfsetstrokecolor{textcolor}%
\pgfsetfillcolor{textcolor}%
\pgftext[x=2.491081in,y=0.223333in,,top]{\color{textcolor}\rmfamily\fontsize{10.000000}{12.000000}\selectfont \(\displaystyle p\)}%
\end{pgfscope}%
\begin{pgfscope}%
\pgfsetbuttcap%
\pgfsetroundjoin%
\definecolor{currentfill}{rgb}{0.000000,0.000000,0.000000}%
\pgfsetfillcolor{currentfill}%
\pgfsetlinewidth{0.803000pt}%
\definecolor{currentstroke}{rgb}{0.000000,0.000000,0.000000}%
\pgfsetstrokecolor{currentstroke}%
\pgfsetdash{}{0pt}%
\pgfsys@defobject{currentmarker}{\pgfqpoint{-0.048611in}{0.000000in}}{\pgfqpoint{-0.000000in}{0.000000in}}{%
\pgfpathmoveto{\pgfqpoint{-0.000000in}{0.000000in}}%
\pgfpathlineto{\pgfqpoint{-0.048611in}{0.000000in}}%
\pgfusepath{stroke,fill}%
}%
\begin{pgfscope}%
\pgfsys@transformshift{0.553581in}{0.499444in}%
\pgfsys@useobject{currentmarker}{}%
\end{pgfscope}%
\end{pgfscope}%
\begin{pgfscope}%
\definecolor{textcolor}{rgb}{0.000000,0.000000,0.000000}%
\pgfsetstrokecolor{textcolor}%
\pgfsetfillcolor{textcolor}%
\pgftext[x=0.278889in, y=0.451250in, left, base]{\color{textcolor}\rmfamily\fontsize{10.000000}{12.000000}\selectfont \(\displaystyle {0.0}\)}%
\end{pgfscope}%
\begin{pgfscope}%
\pgfsetbuttcap%
\pgfsetroundjoin%
\definecolor{currentfill}{rgb}{0.000000,0.000000,0.000000}%
\pgfsetfillcolor{currentfill}%
\pgfsetlinewidth{0.803000pt}%
\definecolor{currentstroke}{rgb}{0.000000,0.000000,0.000000}%
\pgfsetstrokecolor{currentstroke}%
\pgfsetdash{}{0pt}%
\pgfsys@defobject{currentmarker}{\pgfqpoint{-0.048611in}{0.000000in}}{\pgfqpoint{-0.000000in}{0.000000in}}{%
\pgfpathmoveto{\pgfqpoint{-0.000000in}{0.000000in}}%
\pgfpathlineto{\pgfqpoint{-0.048611in}{0.000000in}}%
\pgfusepath{stroke,fill}%
}%
\begin{pgfscope}%
\pgfsys@transformshift{0.553581in}{0.791150in}%
\pgfsys@useobject{currentmarker}{}%
\end{pgfscope}%
\end{pgfscope}%
\begin{pgfscope}%
\definecolor{textcolor}{rgb}{0.000000,0.000000,0.000000}%
\pgfsetstrokecolor{textcolor}%
\pgfsetfillcolor{textcolor}%
\pgftext[x=0.278889in, y=0.742956in, left, base]{\color{textcolor}\rmfamily\fontsize{10.000000}{12.000000}\selectfont \(\displaystyle {2.5}\)}%
\end{pgfscope}%
\begin{pgfscope}%
\pgfsetbuttcap%
\pgfsetroundjoin%
\definecolor{currentfill}{rgb}{0.000000,0.000000,0.000000}%
\pgfsetfillcolor{currentfill}%
\pgfsetlinewidth{0.803000pt}%
\definecolor{currentstroke}{rgb}{0.000000,0.000000,0.000000}%
\pgfsetstrokecolor{currentstroke}%
\pgfsetdash{}{0pt}%
\pgfsys@defobject{currentmarker}{\pgfqpoint{-0.048611in}{0.000000in}}{\pgfqpoint{-0.000000in}{0.000000in}}{%
\pgfpathmoveto{\pgfqpoint{-0.000000in}{0.000000in}}%
\pgfpathlineto{\pgfqpoint{-0.048611in}{0.000000in}}%
\pgfusepath{stroke,fill}%
}%
\begin{pgfscope}%
\pgfsys@transformshift{0.553581in}{1.082857in}%
\pgfsys@useobject{currentmarker}{}%
\end{pgfscope}%
\end{pgfscope}%
\begin{pgfscope}%
\definecolor{textcolor}{rgb}{0.000000,0.000000,0.000000}%
\pgfsetstrokecolor{textcolor}%
\pgfsetfillcolor{textcolor}%
\pgftext[x=0.278889in, y=1.034662in, left, base]{\color{textcolor}\rmfamily\fontsize{10.000000}{12.000000}\selectfont \(\displaystyle {5.0}\)}%
\end{pgfscope}%
\begin{pgfscope}%
\pgfsetbuttcap%
\pgfsetroundjoin%
\definecolor{currentfill}{rgb}{0.000000,0.000000,0.000000}%
\pgfsetfillcolor{currentfill}%
\pgfsetlinewidth{0.803000pt}%
\definecolor{currentstroke}{rgb}{0.000000,0.000000,0.000000}%
\pgfsetstrokecolor{currentstroke}%
\pgfsetdash{}{0pt}%
\pgfsys@defobject{currentmarker}{\pgfqpoint{-0.048611in}{0.000000in}}{\pgfqpoint{-0.000000in}{0.000000in}}{%
\pgfpathmoveto{\pgfqpoint{-0.000000in}{0.000000in}}%
\pgfpathlineto{\pgfqpoint{-0.048611in}{0.000000in}}%
\pgfusepath{stroke,fill}%
}%
\begin{pgfscope}%
\pgfsys@transformshift{0.553581in}{1.374563in}%
\pgfsys@useobject{currentmarker}{}%
\end{pgfscope}%
\end{pgfscope}%
\begin{pgfscope}%
\definecolor{textcolor}{rgb}{0.000000,0.000000,0.000000}%
\pgfsetstrokecolor{textcolor}%
\pgfsetfillcolor{textcolor}%
\pgftext[x=0.278889in, y=1.326369in, left, base]{\color{textcolor}\rmfamily\fontsize{10.000000}{12.000000}\selectfont \(\displaystyle {7.5}\)}%
\end{pgfscope}%
\begin{pgfscope}%
\definecolor{textcolor}{rgb}{0.000000,0.000000,0.000000}%
\pgfsetstrokecolor{textcolor}%
\pgfsetfillcolor{textcolor}%
\pgftext[x=0.223333in,y=1.076944in,,bottom,rotate=90.000000]{\color{textcolor}\rmfamily\fontsize{10.000000}{12.000000}\selectfont Percent of Data Set}%
\end{pgfscope}%
\begin{pgfscope}%
\pgfsetrectcap%
\pgfsetmiterjoin%
\pgfsetlinewidth{0.803000pt}%
\definecolor{currentstroke}{rgb}{0.000000,0.000000,0.000000}%
\pgfsetstrokecolor{currentstroke}%
\pgfsetdash{}{0pt}%
\pgfpathmoveto{\pgfqpoint{0.553581in}{0.499444in}}%
\pgfpathlineto{\pgfqpoint{0.553581in}{1.654444in}}%
\pgfusepath{stroke}%
\end{pgfscope}%
\begin{pgfscope}%
\pgfsetrectcap%
\pgfsetmiterjoin%
\pgfsetlinewidth{0.803000pt}%
\definecolor{currentstroke}{rgb}{0.000000,0.000000,0.000000}%
\pgfsetstrokecolor{currentstroke}%
\pgfsetdash{}{0pt}%
\pgfpathmoveto{\pgfqpoint{4.428581in}{0.499444in}}%
\pgfpathlineto{\pgfqpoint{4.428581in}{1.654444in}}%
\pgfusepath{stroke}%
\end{pgfscope}%
\begin{pgfscope}%
\pgfsetrectcap%
\pgfsetmiterjoin%
\pgfsetlinewidth{0.803000pt}%
\definecolor{currentstroke}{rgb}{0.000000,0.000000,0.000000}%
\pgfsetstrokecolor{currentstroke}%
\pgfsetdash{}{0pt}%
\pgfpathmoveto{\pgfqpoint{0.553581in}{0.499444in}}%
\pgfpathlineto{\pgfqpoint{4.428581in}{0.499444in}}%
\pgfusepath{stroke}%
\end{pgfscope}%
\begin{pgfscope}%
\pgfsetrectcap%
\pgfsetmiterjoin%
\pgfsetlinewidth{0.803000pt}%
\definecolor{currentstroke}{rgb}{0.000000,0.000000,0.000000}%
\pgfsetstrokecolor{currentstroke}%
\pgfsetdash{}{0pt}%
\pgfpathmoveto{\pgfqpoint{0.553581in}{1.654444in}}%
\pgfpathlineto{\pgfqpoint{4.428581in}{1.654444in}}%
\pgfusepath{stroke}%
\end{pgfscope}%
\begin{pgfscope}%
\pgfsetbuttcap%
\pgfsetmiterjoin%
\definecolor{currentfill}{rgb}{1.000000,1.000000,1.000000}%
\pgfsetfillcolor{currentfill}%
\pgfsetfillopacity{0.800000}%
\pgfsetlinewidth{1.003750pt}%
\definecolor{currentstroke}{rgb}{0.800000,0.800000,0.800000}%
\pgfsetstrokecolor{currentstroke}%
\pgfsetstrokeopacity{0.800000}%
\pgfsetdash{}{0pt}%
\pgfpathmoveto{\pgfqpoint{3.651636in}{1.154445in}}%
\pgfpathlineto{\pgfqpoint{4.331358in}{1.154445in}}%
\pgfpathquadraticcurveto{\pgfqpoint{4.359136in}{1.154445in}}{\pgfqpoint{4.359136in}{1.182222in}}%
\pgfpathlineto{\pgfqpoint{4.359136in}{1.557222in}}%
\pgfpathquadraticcurveto{\pgfqpoint{4.359136in}{1.585000in}}{\pgfqpoint{4.331358in}{1.585000in}}%
\pgfpathlineto{\pgfqpoint{3.651636in}{1.585000in}}%
\pgfpathquadraticcurveto{\pgfqpoint{3.623858in}{1.585000in}}{\pgfqpoint{3.623858in}{1.557222in}}%
\pgfpathlineto{\pgfqpoint{3.623858in}{1.182222in}}%
\pgfpathquadraticcurveto{\pgfqpoint{3.623858in}{1.154445in}}{\pgfqpoint{3.651636in}{1.154445in}}%
\pgfpathlineto{\pgfqpoint{3.651636in}{1.154445in}}%
\pgfpathclose%
\pgfusepath{stroke,fill}%
\end{pgfscope}%
\begin{pgfscope}%
\pgfsetbuttcap%
\pgfsetmiterjoin%
\pgfsetlinewidth{1.003750pt}%
\definecolor{currentstroke}{rgb}{0.000000,0.000000,0.000000}%
\pgfsetstrokecolor{currentstroke}%
\pgfsetdash{}{0pt}%
\pgfpathmoveto{\pgfqpoint{3.679414in}{1.432222in}}%
\pgfpathlineto{\pgfqpoint{3.957192in}{1.432222in}}%
\pgfpathlineto{\pgfqpoint{3.957192in}{1.529444in}}%
\pgfpathlineto{\pgfqpoint{3.679414in}{1.529444in}}%
\pgfpathlineto{\pgfqpoint{3.679414in}{1.432222in}}%
\pgfpathclose%
\pgfusepath{stroke}%
\end{pgfscope}%
\begin{pgfscope}%
\definecolor{textcolor}{rgb}{0.000000,0.000000,0.000000}%
\pgfsetstrokecolor{textcolor}%
\pgfsetfillcolor{textcolor}%
\pgftext[x=4.068303in,y=1.432222in,left,base]{\color{textcolor}\rmfamily\fontsize{10.000000}{12.000000}\selectfont Neg}%
\end{pgfscope}%
\begin{pgfscope}%
\pgfsetbuttcap%
\pgfsetmiterjoin%
\definecolor{currentfill}{rgb}{0.000000,0.000000,0.000000}%
\pgfsetfillcolor{currentfill}%
\pgfsetlinewidth{0.000000pt}%
\definecolor{currentstroke}{rgb}{0.000000,0.000000,0.000000}%
\pgfsetstrokecolor{currentstroke}%
\pgfsetstrokeopacity{0.000000}%
\pgfsetdash{}{0pt}%
\pgfpathmoveto{\pgfqpoint{3.679414in}{1.236944in}}%
\pgfpathlineto{\pgfqpoint{3.957192in}{1.236944in}}%
\pgfpathlineto{\pgfqpoint{3.957192in}{1.334167in}}%
\pgfpathlineto{\pgfqpoint{3.679414in}{1.334167in}}%
\pgfpathlineto{\pgfqpoint{3.679414in}{1.236944in}}%
\pgfpathclose%
\pgfusepath{fill}%
\end{pgfscope}%
\begin{pgfscope}%
\definecolor{textcolor}{rgb}{0.000000,0.000000,0.000000}%
\pgfsetstrokecolor{textcolor}%
\pgfsetfillcolor{textcolor}%
\pgftext[x=4.068303in,y=1.236944in,left,base]{\color{textcolor}\rmfamily\fontsize{10.000000}{12.000000}\selectfont Pos}%
\end{pgfscope}%
\end{pgfpicture}%
\makeatother%
\endgroup%

&
	\vskip 0pt
	\begin{tabular}{cc|c|c|}
	&\multicolumn{1}{c}{}& \multicolumn{2}{c}{Prediction} \cr
	&\multicolumn{1}{c}{} & \multicolumn{1}{c}{N} & \multicolumn{1}{c}{P} \cr\cline{3-4}
	\multirow{2}{*}{\rotatebox[origin=c]{90}{Actual}}&N &
168,598 & 11,647
	\vrule width 0pt height 10pt depth 2pt \cr\cline{3-4}
	&P & 
22,513 & 11,312
	\vrule width 0pt height 10pt depth 2pt \cr\cline{3-4}
	\end{tabular}

	\hfil\begin{tabular}{ll}
	\cr
0.493 & Precision \cr	0.334 & Recall \cr	0.398 & F1 \cr
\end{tabular}
\end{tabular}


\

Model 3:  $\alpha = \pi_0 = 0.84$ for class balance

\noindent\begin{tabular}{@{\hspace{-6pt}}p{4.5in} @{\hspace{-3pt}}p{2.0in}}
	\vskip 0pt
	\qquad \qquad Raw Model Output
	
	%% Creator: Matplotlib, PGF backend
%%
%% To include the figure in your LaTeX document, write
%%   \input{<filename>.pgf}
%%
%% Make sure the required packages are loaded in your preamble
%%   \usepackage{pgf}
%%
%% Also ensure that all the required font packages are loaded; for instance,
%% the lmodern package is sometimes necessary when using math font.
%%   \usepackage{lmodern}
%%
%% Figures using additional raster images can only be included by \input if
%% they are in the same directory as the main LaTeX file. For loading figures
%% from other directories you can use the `import` package
%%   \usepackage{import}
%%
%% and then include the figures with
%%   \import{<path to file>}{<filename>.pgf}
%%
%% Matplotlib used the following preamble
%%   
%%   \usepackage{fontspec}
%%   \makeatletter\@ifpackageloaded{underscore}{}{\usepackage[strings]{underscore}}\makeatother
%%
\begingroup%
\makeatletter%
\begin{pgfpicture}%
\pgfpathrectangle{\pgfpointorigin}{\pgfqpoint{4.509306in}{1.754444in}}%
\pgfusepath{use as bounding box, clip}%
\begin{pgfscope}%
\pgfsetbuttcap%
\pgfsetmiterjoin%
\definecolor{currentfill}{rgb}{1.000000,1.000000,1.000000}%
\pgfsetfillcolor{currentfill}%
\pgfsetlinewidth{0.000000pt}%
\definecolor{currentstroke}{rgb}{1.000000,1.000000,1.000000}%
\pgfsetstrokecolor{currentstroke}%
\pgfsetdash{}{0pt}%
\pgfpathmoveto{\pgfqpoint{0.000000in}{0.000000in}}%
\pgfpathlineto{\pgfqpoint{4.509306in}{0.000000in}}%
\pgfpathlineto{\pgfqpoint{4.509306in}{1.754444in}}%
\pgfpathlineto{\pgfqpoint{0.000000in}{1.754444in}}%
\pgfpathlineto{\pgfqpoint{0.000000in}{0.000000in}}%
\pgfpathclose%
\pgfusepath{fill}%
\end{pgfscope}%
\begin{pgfscope}%
\pgfsetbuttcap%
\pgfsetmiterjoin%
\definecolor{currentfill}{rgb}{1.000000,1.000000,1.000000}%
\pgfsetfillcolor{currentfill}%
\pgfsetlinewidth{0.000000pt}%
\definecolor{currentstroke}{rgb}{0.000000,0.000000,0.000000}%
\pgfsetstrokecolor{currentstroke}%
\pgfsetstrokeopacity{0.000000}%
\pgfsetdash{}{0pt}%
\pgfpathmoveto{\pgfqpoint{0.445556in}{0.499444in}}%
\pgfpathlineto{\pgfqpoint{4.320556in}{0.499444in}}%
\pgfpathlineto{\pgfqpoint{4.320556in}{1.654444in}}%
\pgfpathlineto{\pgfqpoint{0.445556in}{1.654444in}}%
\pgfpathlineto{\pgfqpoint{0.445556in}{0.499444in}}%
\pgfpathclose%
\pgfusepath{fill}%
\end{pgfscope}%
\begin{pgfscope}%
\pgfpathrectangle{\pgfqpoint{0.445556in}{0.499444in}}{\pgfqpoint{3.875000in}{1.155000in}}%
\pgfusepath{clip}%
\pgfsetbuttcap%
\pgfsetmiterjoin%
\pgfsetlinewidth{1.003750pt}%
\definecolor{currentstroke}{rgb}{0.000000,0.000000,0.000000}%
\pgfsetstrokecolor{currentstroke}%
\pgfsetdash{}{0pt}%
\pgfpathmoveto{\pgfqpoint{0.435556in}{0.499444in}}%
\pgfpathlineto{\pgfqpoint{0.483922in}{0.499444in}}%
\pgfpathlineto{\pgfqpoint{0.483922in}{0.905316in}}%
\pgfpathlineto{\pgfqpoint{0.435556in}{0.905316in}}%
\pgfusepath{stroke}%
\end{pgfscope}%
\begin{pgfscope}%
\pgfpathrectangle{\pgfqpoint{0.445556in}{0.499444in}}{\pgfqpoint{3.875000in}{1.155000in}}%
\pgfusepath{clip}%
\pgfsetbuttcap%
\pgfsetmiterjoin%
\pgfsetlinewidth{1.003750pt}%
\definecolor{currentstroke}{rgb}{0.000000,0.000000,0.000000}%
\pgfsetstrokecolor{currentstroke}%
\pgfsetdash{}{0pt}%
\pgfpathmoveto{\pgfqpoint{0.576001in}{0.499444in}}%
\pgfpathlineto{\pgfqpoint{0.637387in}{0.499444in}}%
\pgfpathlineto{\pgfqpoint{0.637387in}{1.095204in}}%
\pgfpathlineto{\pgfqpoint{0.576001in}{1.095204in}}%
\pgfpathlineto{\pgfqpoint{0.576001in}{0.499444in}}%
\pgfpathclose%
\pgfusepath{stroke}%
\end{pgfscope}%
\begin{pgfscope}%
\pgfpathrectangle{\pgfqpoint{0.445556in}{0.499444in}}{\pgfqpoint{3.875000in}{1.155000in}}%
\pgfusepath{clip}%
\pgfsetbuttcap%
\pgfsetmiterjoin%
\pgfsetlinewidth{1.003750pt}%
\definecolor{currentstroke}{rgb}{0.000000,0.000000,0.000000}%
\pgfsetstrokecolor{currentstroke}%
\pgfsetdash{}{0pt}%
\pgfpathmoveto{\pgfqpoint{0.729467in}{0.499444in}}%
\pgfpathlineto{\pgfqpoint{0.790853in}{0.499444in}}%
\pgfpathlineto{\pgfqpoint{0.790853in}{1.209750in}}%
\pgfpathlineto{\pgfqpoint{0.729467in}{1.209750in}}%
\pgfpathlineto{\pgfqpoint{0.729467in}{0.499444in}}%
\pgfpathclose%
\pgfusepath{stroke}%
\end{pgfscope}%
\begin{pgfscope}%
\pgfpathrectangle{\pgfqpoint{0.445556in}{0.499444in}}{\pgfqpoint{3.875000in}{1.155000in}}%
\pgfusepath{clip}%
\pgfsetbuttcap%
\pgfsetmiterjoin%
\pgfsetlinewidth{1.003750pt}%
\definecolor{currentstroke}{rgb}{0.000000,0.000000,0.000000}%
\pgfsetstrokecolor{currentstroke}%
\pgfsetdash{}{0pt}%
\pgfpathmoveto{\pgfqpoint{0.882932in}{0.499444in}}%
\pgfpathlineto{\pgfqpoint{0.944318in}{0.499444in}}%
\pgfpathlineto{\pgfqpoint{0.944318in}{1.285327in}}%
\pgfpathlineto{\pgfqpoint{0.882932in}{1.285327in}}%
\pgfpathlineto{\pgfqpoint{0.882932in}{0.499444in}}%
\pgfpathclose%
\pgfusepath{stroke}%
\end{pgfscope}%
\begin{pgfscope}%
\pgfpathrectangle{\pgfqpoint{0.445556in}{0.499444in}}{\pgfqpoint{3.875000in}{1.155000in}}%
\pgfusepath{clip}%
\pgfsetbuttcap%
\pgfsetmiterjoin%
\pgfsetlinewidth{1.003750pt}%
\definecolor{currentstroke}{rgb}{0.000000,0.000000,0.000000}%
\pgfsetstrokecolor{currentstroke}%
\pgfsetdash{}{0pt}%
\pgfpathmoveto{\pgfqpoint{1.036397in}{0.499444in}}%
\pgfpathlineto{\pgfqpoint{1.097783in}{0.499444in}}%
\pgfpathlineto{\pgfqpoint{1.097783in}{1.316621in}}%
\pgfpathlineto{\pgfqpoint{1.036397in}{1.316621in}}%
\pgfpathlineto{\pgfqpoint{1.036397in}{0.499444in}}%
\pgfpathclose%
\pgfusepath{stroke}%
\end{pgfscope}%
\begin{pgfscope}%
\pgfpathrectangle{\pgfqpoint{0.445556in}{0.499444in}}{\pgfqpoint{3.875000in}{1.155000in}}%
\pgfusepath{clip}%
\pgfsetbuttcap%
\pgfsetmiterjoin%
\pgfsetlinewidth{1.003750pt}%
\definecolor{currentstroke}{rgb}{0.000000,0.000000,0.000000}%
\pgfsetstrokecolor{currentstroke}%
\pgfsetdash{}{0pt}%
\pgfpathmoveto{\pgfqpoint{1.189863in}{0.499444in}}%
\pgfpathlineto{\pgfqpoint{1.251249in}{0.499444in}}%
\pgfpathlineto{\pgfqpoint{1.251249in}{1.329256in}}%
\pgfpathlineto{\pgfqpoint{1.189863in}{1.329256in}}%
\pgfpathlineto{\pgfqpoint{1.189863in}{0.499444in}}%
\pgfpathclose%
\pgfusepath{stroke}%
\end{pgfscope}%
\begin{pgfscope}%
\pgfpathrectangle{\pgfqpoint{0.445556in}{0.499444in}}{\pgfqpoint{3.875000in}{1.155000in}}%
\pgfusepath{clip}%
\pgfsetbuttcap%
\pgfsetmiterjoin%
\pgfsetlinewidth{1.003750pt}%
\definecolor{currentstroke}{rgb}{0.000000,0.000000,0.000000}%
\pgfsetstrokecolor{currentstroke}%
\pgfsetdash{}{0pt}%
\pgfpathmoveto{\pgfqpoint{1.343328in}{0.499444in}}%
\pgfpathlineto{\pgfqpoint{1.404714in}{0.499444in}}%
\pgfpathlineto{\pgfqpoint{1.404714in}{1.369643in}}%
\pgfpathlineto{\pgfqpoint{1.343328in}{1.369643in}}%
\pgfpathlineto{\pgfqpoint{1.343328in}{0.499444in}}%
\pgfpathclose%
\pgfusepath{stroke}%
\end{pgfscope}%
\begin{pgfscope}%
\pgfpathrectangle{\pgfqpoint{0.445556in}{0.499444in}}{\pgfqpoint{3.875000in}{1.155000in}}%
\pgfusepath{clip}%
\pgfsetbuttcap%
\pgfsetmiterjoin%
\pgfsetlinewidth{1.003750pt}%
\definecolor{currentstroke}{rgb}{0.000000,0.000000,0.000000}%
\pgfsetstrokecolor{currentstroke}%
\pgfsetdash{}{0pt}%
\pgfpathmoveto{\pgfqpoint{1.496793in}{0.499444in}}%
\pgfpathlineto{\pgfqpoint{1.558179in}{0.499444in}}%
\pgfpathlineto{\pgfqpoint{1.558179in}{1.389836in}}%
\pgfpathlineto{\pgfqpoint{1.496793in}{1.389836in}}%
\pgfpathlineto{\pgfqpoint{1.496793in}{0.499444in}}%
\pgfpathclose%
\pgfusepath{stroke}%
\end{pgfscope}%
\begin{pgfscope}%
\pgfpathrectangle{\pgfqpoint{0.445556in}{0.499444in}}{\pgfqpoint{3.875000in}{1.155000in}}%
\pgfusepath{clip}%
\pgfsetbuttcap%
\pgfsetmiterjoin%
\pgfsetlinewidth{1.003750pt}%
\definecolor{currentstroke}{rgb}{0.000000,0.000000,0.000000}%
\pgfsetstrokecolor{currentstroke}%
\pgfsetdash{}{0pt}%
\pgfpathmoveto{\pgfqpoint{1.650259in}{0.499444in}}%
\pgfpathlineto{\pgfqpoint{1.711645in}{0.499444in}}%
\pgfpathlineto{\pgfqpoint{1.711645in}{1.423728in}}%
\pgfpathlineto{\pgfqpoint{1.650259in}{1.423728in}}%
\pgfpathlineto{\pgfqpoint{1.650259in}{0.499444in}}%
\pgfpathclose%
\pgfusepath{stroke}%
\end{pgfscope}%
\begin{pgfscope}%
\pgfpathrectangle{\pgfqpoint{0.445556in}{0.499444in}}{\pgfqpoint{3.875000in}{1.155000in}}%
\pgfusepath{clip}%
\pgfsetbuttcap%
\pgfsetmiterjoin%
\pgfsetlinewidth{1.003750pt}%
\definecolor{currentstroke}{rgb}{0.000000,0.000000,0.000000}%
\pgfsetstrokecolor{currentstroke}%
\pgfsetdash{}{0pt}%
\pgfpathmoveto{\pgfqpoint{1.803724in}{0.499444in}}%
\pgfpathlineto{\pgfqpoint{1.865110in}{0.499444in}}%
\pgfpathlineto{\pgfqpoint{1.865110in}{1.450534in}}%
\pgfpathlineto{\pgfqpoint{1.803724in}{1.450534in}}%
\pgfpathlineto{\pgfqpoint{1.803724in}{0.499444in}}%
\pgfpathclose%
\pgfusepath{stroke}%
\end{pgfscope}%
\begin{pgfscope}%
\pgfpathrectangle{\pgfqpoint{0.445556in}{0.499444in}}{\pgfqpoint{3.875000in}{1.155000in}}%
\pgfusepath{clip}%
\pgfsetbuttcap%
\pgfsetmiterjoin%
\pgfsetlinewidth{1.003750pt}%
\definecolor{currentstroke}{rgb}{0.000000,0.000000,0.000000}%
\pgfsetstrokecolor{currentstroke}%
\pgfsetdash{}{0pt}%
\pgfpathmoveto{\pgfqpoint{1.957189in}{0.499444in}}%
\pgfpathlineto{\pgfqpoint{2.018575in}{0.499444in}}%
\pgfpathlineto{\pgfqpoint{2.018575in}{1.477104in}}%
\pgfpathlineto{\pgfqpoint{1.957189in}{1.477104in}}%
\pgfpathlineto{\pgfqpoint{1.957189in}{0.499444in}}%
\pgfpathclose%
\pgfusepath{stroke}%
\end{pgfscope}%
\begin{pgfscope}%
\pgfpathrectangle{\pgfqpoint{0.445556in}{0.499444in}}{\pgfqpoint{3.875000in}{1.155000in}}%
\pgfusepath{clip}%
\pgfsetbuttcap%
\pgfsetmiterjoin%
\pgfsetlinewidth{1.003750pt}%
\definecolor{currentstroke}{rgb}{0.000000,0.000000,0.000000}%
\pgfsetstrokecolor{currentstroke}%
\pgfsetdash{}{0pt}%
\pgfpathmoveto{\pgfqpoint{2.110655in}{0.499444in}}%
\pgfpathlineto{\pgfqpoint{2.172041in}{0.499444in}}%
\pgfpathlineto{\pgfqpoint{2.172041in}{1.501312in}}%
\pgfpathlineto{\pgfqpoint{2.110655in}{1.501312in}}%
\pgfpathlineto{\pgfqpoint{2.110655in}{0.499444in}}%
\pgfpathclose%
\pgfusepath{stroke}%
\end{pgfscope}%
\begin{pgfscope}%
\pgfpathrectangle{\pgfqpoint{0.445556in}{0.499444in}}{\pgfqpoint{3.875000in}{1.155000in}}%
\pgfusepath{clip}%
\pgfsetbuttcap%
\pgfsetmiterjoin%
\pgfsetlinewidth{1.003750pt}%
\definecolor{currentstroke}{rgb}{0.000000,0.000000,0.000000}%
\pgfsetstrokecolor{currentstroke}%
\pgfsetdash{}{0pt}%
\pgfpathmoveto{\pgfqpoint{2.264120in}{0.499444in}}%
\pgfpathlineto{\pgfqpoint{2.325506in}{0.499444in}}%
\pgfpathlineto{\pgfqpoint{2.325506in}{1.532960in}}%
\pgfpathlineto{\pgfqpoint{2.264120in}{1.532960in}}%
\pgfpathlineto{\pgfqpoint{2.264120in}{0.499444in}}%
\pgfpathclose%
\pgfusepath{stroke}%
\end{pgfscope}%
\begin{pgfscope}%
\pgfpathrectangle{\pgfqpoint{0.445556in}{0.499444in}}{\pgfqpoint{3.875000in}{1.155000in}}%
\pgfusepath{clip}%
\pgfsetbuttcap%
\pgfsetmiterjoin%
\pgfsetlinewidth{1.003750pt}%
\definecolor{currentstroke}{rgb}{0.000000,0.000000,0.000000}%
\pgfsetstrokecolor{currentstroke}%
\pgfsetdash{}{0pt}%
\pgfpathmoveto{\pgfqpoint{2.417585in}{0.499444in}}%
\pgfpathlineto{\pgfqpoint{2.478972in}{0.499444in}}%
\pgfpathlineto{\pgfqpoint{2.478972in}{1.543352in}}%
\pgfpathlineto{\pgfqpoint{2.417585in}{1.543352in}}%
\pgfpathlineto{\pgfqpoint{2.417585in}{0.499444in}}%
\pgfpathclose%
\pgfusepath{stroke}%
\end{pgfscope}%
\begin{pgfscope}%
\pgfpathrectangle{\pgfqpoint{0.445556in}{0.499444in}}{\pgfqpoint{3.875000in}{1.155000in}}%
\pgfusepath{clip}%
\pgfsetbuttcap%
\pgfsetmiterjoin%
\pgfsetlinewidth{1.003750pt}%
\definecolor{currentstroke}{rgb}{0.000000,0.000000,0.000000}%
\pgfsetstrokecolor{currentstroke}%
\pgfsetdash{}{0pt}%
\pgfpathmoveto{\pgfqpoint{2.571051in}{0.499444in}}%
\pgfpathlineto{\pgfqpoint{2.632437in}{0.499444in}}%
\pgfpathlineto{\pgfqpoint{2.632437in}{1.562718in}}%
\pgfpathlineto{\pgfqpoint{2.571051in}{1.562718in}}%
\pgfpathlineto{\pgfqpoint{2.571051in}{0.499444in}}%
\pgfpathclose%
\pgfusepath{stroke}%
\end{pgfscope}%
\begin{pgfscope}%
\pgfpathrectangle{\pgfqpoint{0.445556in}{0.499444in}}{\pgfqpoint{3.875000in}{1.155000in}}%
\pgfusepath{clip}%
\pgfsetbuttcap%
\pgfsetmiterjoin%
\pgfsetlinewidth{1.003750pt}%
\definecolor{currentstroke}{rgb}{0.000000,0.000000,0.000000}%
\pgfsetstrokecolor{currentstroke}%
\pgfsetdash{}{0pt}%
\pgfpathmoveto{\pgfqpoint{2.724516in}{0.499444in}}%
\pgfpathlineto{\pgfqpoint{2.785902in}{0.499444in}}%
\pgfpathlineto{\pgfqpoint{2.785902in}{1.589052in}}%
\pgfpathlineto{\pgfqpoint{2.724516in}{1.589052in}}%
\pgfpathlineto{\pgfqpoint{2.724516in}{0.499444in}}%
\pgfpathclose%
\pgfusepath{stroke}%
\end{pgfscope}%
\begin{pgfscope}%
\pgfpathrectangle{\pgfqpoint{0.445556in}{0.499444in}}{\pgfqpoint{3.875000in}{1.155000in}}%
\pgfusepath{clip}%
\pgfsetbuttcap%
\pgfsetmiterjoin%
\pgfsetlinewidth{1.003750pt}%
\definecolor{currentstroke}{rgb}{0.000000,0.000000,0.000000}%
\pgfsetstrokecolor{currentstroke}%
\pgfsetdash{}{0pt}%
\pgfpathmoveto{\pgfqpoint{2.877981in}{0.499444in}}%
\pgfpathlineto{\pgfqpoint{2.939368in}{0.499444in}}%
\pgfpathlineto{\pgfqpoint{2.939368in}{1.599444in}}%
\pgfpathlineto{\pgfqpoint{2.877981in}{1.599444in}}%
\pgfpathlineto{\pgfqpoint{2.877981in}{0.499444in}}%
\pgfpathclose%
\pgfusepath{stroke}%
\end{pgfscope}%
\begin{pgfscope}%
\pgfpathrectangle{\pgfqpoint{0.445556in}{0.499444in}}{\pgfqpoint{3.875000in}{1.155000in}}%
\pgfusepath{clip}%
\pgfsetbuttcap%
\pgfsetmiterjoin%
\pgfsetlinewidth{1.003750pt}%
\definecolor{currentstroke}{rgb}{0.000000,0.000000,0.000000}%
\pgfsetstrokecolor{currentstroke}%
\pgfsetdash{}{0pt}%
\pgfpathmoveto{\pgfqpoint{3.031447in}{0.499444in}}%
\pgfpathlineto{\pgfqpoint{3.092833in}{0.499444in}}%
\pgfpathlineto{\pgfqpoint{3.092833in}{1.557759in}}%
\pgfpathlineto{\pgfqpoint{3.031447in}{1.557759in}}%
\pgfpathlineto{\pgfqpoint{3.031447in}{0.499444in}}%
\pgfpathclose%
\pgfusepath{stroke}%
\end{pgfscope}%
\begin{pgfscope}%
\pgfpathrectangle{\pgfqpoint{0.445556in}{0.499444in}}{\pgfqpoint{3.875000in}{1.155000in}}%
\pgfusepath{clip}%
\pgfsetbuttcap%
\pgfsetmiterjoin%
\pgfsetlinewidth{1.003750pt}%
\definecolor{currentstroke}{rgb}{0.000000,0.000000,0.000000}%
\pgfsetstrokecolor{currentstroke}%
\pgfsetdash{}{0pt}%
\pgfpathmoveto{\pgfqpoint{3.184912in}{0.499444in}}%
\pgfpathlineto{\pgfqpoint{3.246298in}{0.499444in}}%
\pgfpathlineto{\pgfqpoint{3.246298in}{1.555043in}}%
\pgfpathlineto{\pgfqpoint{3.184912in}{1.555043in}}%
\pgfpathlineto{\pgfqpoint{3.184912in}{0.499444in}}%
\pgfpathclose%
\pgfusepath{stroke}%
\end{pgfscope}%
\begin{pgfscope}%
\pgfpathrectangle{\pgfqpoint{0.445556in}{0.499444in}}{\pgfqpoint{3.875000in}{1.155000in}}%
\pgfusepath{clip}%
\pgfsetbuttcap%
\pgfsetmiterjoin%
\pgfsetlinewidth{1.003750pt}%
\definecolor{currentstroke}{rgb}{0.000000,0.000000,0.000000}%
\pgfsetstrokecolor{currentstroke}%
\pgfsetdash{}{0pt}%
\pgfpathmoveto{\pgfqpoint{3.338377in}{0.499444in}}%
\pgfpathlineto{\pgfqpoint{3.399764in}{0.499444in}}%
\pgfpathlineto{\pgfqpoint{3.399764in}{1.526111in}}%
\pgfpathlineto{\pgfqpoint{3.338377in}{1.526111in}}%
\pgfpathlineto{\pgfqpoint{3.338377in}{0.499444in}}%
\pgfpathclose%
\pgfusepath{stroke}%
\end{pgfscope}%
\begin{pgfscope}%
\pgfpathrectangle{\pgfqpoint{0.445556in}{0.499444in}}{\pgfqpoint{3.875000in}{1.155000in}}%
\pgfusepath{clip}%
\pgfsetbuttcap%
\pgfsetmiterjoin%
\pgfsetlinewidth{1.003750pt}%
\definecolor{currentstroke}{rgb}{0.000000,0.000000,0.000000}%
\pgfsetstrokecolor{currentstroke}%
\pgfsetdash{}{0pt}%
\pgfpathmoveto{\pgfqpoint{3.491843in}{0.499444in}}%
\pgfpathlineto{\pgfqpoint{3.553229in}{0.499444in}}%
\pgfpathlineto{\pgfqpoint{3.553229in}{1.448999in}}%
\pgfpathlineto{\pgfqpoint{3.491843in}{1.448999in}}%
\pgfpathlineto{\pgfqpoint{3.491843in}{0.499444in}}%
\pgfpathclose%
\pgfusepath{stroke}%
\end{pgfscope}%
\begin{pgfscope}%
\pgfpathrectangle{\pgfqpoint{0.445556in}{0.499444in}}{\pgfqpoint{3.875000in}{1.155000in}}%
\pgfusepath{clip}%
\pgfsetbuttcap%
\pgfsetmiterjoin%
\pgfsetlinewidth{1.003750pt}%
\definecolor{currentstroke}{rgb}{0.000000,0.000000,0.000000}%
\pgfsetstrokecolor{currentstroke}%
\pgfsetdash{}{0pt}%
\pgfpathmoveto{\pgfqpoint{3.645308in}{0.499444in}}%
\pgfpathlineto{\pgfqpoint{3.706694in}{0.499444in}}%
\pgfpathlineto{\pgfqpoint{3.706694in}{1.348032in}}%
\pgfpathlineto{\pgfqpoint{3.645308in}{1.348032in}}%
\pgfpathlineto{\pgfqpoint{3.645308in}{0.499444in}}%
\pgfpathclose%
\pgfusepath{stroke}%
\end{pgfscope}%
\begin{pgfscope}%
\pgfpathrectangle{\pgfqpoint{0.445556in}{0.499444in}}{\pgfqpoint{3.875000in}{1.155000in}}%
\pgfusepath{clip}%
\pgfsetbuttcap%
\pgfsetmiterjoin%
\pgfsetlinewidth{1.003750pt}%
\definecolor{currentstroke}{rgb}{0.000000,0.000000,0.000000}%
\pgfsetstrokecolor{currentstroke}%
\pgfsetdash{}{0pt}%
\pgfpathmoveto{\pgfqpoint{3.798774in}{0.499444in}}%
\pgfpathlineto{\pgfqpoint{3.860160in}{0.499444in}}%
\pgfpathlineto{\pgfqpoint{3.860160in}{1.173143in}}%
\pgfpathlineto{\pgfqpoint{3.798774in}{1.173143in}}%
\pgfpathlineto{\pgfqpoint{3.798774in}{0.499444in}}%
\pgfpathclose%
\pgfusepath{stroke}%
\end{pgfscope}%
\begin{pgfscope}%
\pgfpathrectangle{\pgfqpoint{0.445556in}{0.499444in}}{\pgfqpoint{3.875000in}{1.155000in}}%
\pgfusepath{clip}%
\pgfsetbuttcap%
\pgfsetmiterjoin%
\pgfsetlinewidth{1.003750pt}%
\definecolor{currentstroke}{rgb}{0.000000,0.000000,0.000000}%
\pgfsetstrokecolor{currentstroke}%
\pgfsetdash{}{0pt}%
\pgfpathmoveto{\pgfqpoint{3.952239in}{0.499444in}}%
\pgfpathlineto{\pgfqpoint{4.013625in}{0.499444in}}%
\pgfpathlineto{\pgfqpoint{4.013625in}{0.920432in}}%
\pgfpathlineto{\pgfqpoint{3.952239in}{0.920432in}}%
\pgfpathlineto{\pgfqpoint{3.952239in}{0.499444in}}%
\pgfpathclose%
\pgfusepath{stroke}%
\end{pgfscope}%
\begin{pgfscope}%
\pgfpathrectangle{\pgfqpoint{0.445556in}{0.499444in}}{\pgfqpoint{3.875000in}{1.155000in}}%
\pgfusepath{clip}%
\pgfsetbuttcap%
\pgfsetmiterjoin%
\pgfsetlinewidth{1.003750pt}%
\definecolor{currentstroke}{rgb}{0.000000,0.000000,0.000000}%
\pgfsetstrokecolor{currentstroke}%
\pgfsetdash{}{0pt}%
\pgfpathmoveto{\pgfqpoint{4.105704in}{0.499444in}}%
\pgfpathlineto{\pgfqpoint{4.167090in}{0.499444in}}%
\pgfpathlineto{\pgfqpoint{4.167090in}{0.660400in}}%
\pgfpathlineto{\pgfqpoint{4.105704in}{0.660400in}}%
\pgfpathlineto{\pgfqpoint{4.105704in}{0.499444in}}%
\pgfpathclose%
\pgfusepath{stroke}%
\end{pgfscope}%
\begin{pgfscope}%
\pgfpathrectangle{\pgfqpoint{0.445556in}{0.499444in}}{\pgfqpoint{3.875000in}{1.155000in}}%
\pgfusepath{clip}%
\pgfsetbuttcap%
\pgfsetmiterjoin%
\definecolor{currentfill}{rgb}{0.000000,0.000000,0.000000}%
\pgfsetfillcolor{currentfill}%
\pgfsetlinewidth{0.000000pt}%
\definecolor{currentstroke}{rgb}{0.000000,0.000000,0.000000}%
\pgfsetstrokecolor{currentstroke}%
\pgfsetstrokeopacity{0.000000}%
\pgfsetdash{}{0pt}%
\pgfpathmoveto{\pgfqpoint{0.483922in}{0.499444in}}%
\pgfpathlineto{\pgfqpoint{0.545308in}{0.499444in}}%
\pgfpathlineto{\pgfqpoint{0.545308in}{0.502514in}}%
\pgfpathlineto{\pgfqpoint{0.483922in}{0.502514in}}%
\pgfpathlineto{\pgfqpoint{0.483922in}{0.499444in}}%
\pgfpathclose%
\pgfusepath{fill}%
\end{pgfscope}%
\begin{pgfscope}%
\pgfpathrectangle{\pgfqpoint{0.445556in}{0.499444in}}{\pgfqpoint{3.875000in}{1.155000in}}%
\pgfusepath{clip}%
\pgfsetbuttcap%
\pgfsetmiterjoin%
\definecolor{currentfill}{rgb}{0.000000,0.000000,0.000000}%
\pgfsetfillcolor{currentfill}%
\pgfsetlinewidth{0.000000pt}%
\definecolor{currentstroke}{rgb}{0.000000,0.000000,0.000000}%
\pgfsetstrokecolor{currentstroke}%
\pgfsetstrokeopacity{0.000000}%
\pgfsetdash{}{0pt}%
\pgfpathmoveto{\pgfqpoint{0.637387in}{0.499444in}}%
\pgfpathlineto{\pgfqpoint{0.698774in}{0.499444in}}%
\pgfpathlineto{\pgfqpoint{0.698774in}{0.506411in}}%
\pgfpathlineto{\pgfqpoint{0.637387in}{0.506411in}}%
\pgfpathlineto{\pgfqpoint{0.637387in}{0.499444in}}%
\pgfpathclose%
\pgfusepath{fill}%
\end{pgfscope}%
\begin{pgfscope}%
\pgfpathrectangle{\pgfqpoint{0.445556in}{0.499444in}}{\pgfqpoint{3.875000in}{1.155000in}}%
\pgfusepath{clip}%
\pgfsetbuttcap%
\pgfsetmiterjoin%
\definecolor{currentfill}{rgb}{0.000000,0.000000,0.000000}%
\pgfsetfillcolor{currentfill}%
\pgfsetlinewidth{0.000000pt}%
\definecolor{currentstroke}{rgb}{0.000000,0.000000,0.000000}%
\pgfsetstrokecolor{currentstroke}%
\pgfsetstrokeopacity{0.000000}%
\pgfsetdash{}{0pt}%
\pgfpathmoveto{\pgfqpoint{0.790853in}{0.499444in}}%
\pgfpathlineto{\pgfqpoint{0.852239in}{0.499444in}}%
\pgfpathlineto{\pgfqpoint{0.852239in}{0.511489in}}%
\pgfpathlineto{\pgfqpoint{0.790853in}{0.511489in}}%
\pgfpathlineto{\pgfqpoint{0.790853in}{0.499444in}}%
\pgfpathclose%
\pgfusepath{fill}%
\end{pgfscope}%
\begin{pgfscope}%
\pgfpathrectangle{\pgfqpoint{0.445556in}{0.499444in}}{\pgfqpoint{3.875000in}{1.155000in}}%
\pgfusepath{clip}%
\pgfsetbuttcap%
\pgfsetmiterjoin%
\definecolor{currentfill}{rgb}{0.000000,0.000000,0.000000}%
\pgfsetfillcolor{currentfill}%
\pgfsetlinewidth{0.000000pt}%
\definecolor{currentstroke}{rgb}{0.000000,0.000000,0.000000}%
\pgfsetstrokecolor{currentstroke}%
\pgfsetstrokeopacity{0.000000}%
\pgfsetdash{}{0pt}%
\pgfpathmoveto{\pgfqpoint{0.944318in}{0.499444in}}%
\pgfpathlineto{\pgfqpoint{1.005704in}{0.499444in}}%
\pgfpathlineto{\pgfqpoint{1.005704in}{0.515268in}}%
\pgfpathlineto{\pgfqpoint{0.944318in}{0.515268in}}%
\pgfpathlineto{\pgfqpoint{0.944318in}{0.499444in}}%
\pgfpathclose%
\pgfusepath{fill}%
\end{pgfscope}%
\begin{pgfscope}%
\pgfpathrectangle{\pgfqpoint{0.445556in}{0.499444in}}{\pgfqpoint{3.875000in}{1.155000in}}%
\pgfusepath{clip}%
\pgfsetbuttcap%
\pgfsetmiterjoin%
\definecolor{currentfill}{rgb}{0.000000,0.000000,0.000000}%
\pgfsetfillcolor{currentfill}%
\pgfsetlinewidth{0.000000pt}%
\definecolor{currentstroke}{rgb}{0.000000,0.000000,0.000000}%
\pgfsetstrokecolor{currentstroke}%
\pgfsetstrokeopacity{0.000000}%
\pgfsetdash{}{0pt}%
\pgfpathmoveto{\pgfqpoint{1.097783in}{0.499444in}}%
\pgfpathlineto{\pgfqpoint{1.159170in}{0.499444in}}%
\pgfpathlineto{\pgfqpoint{1.159170in}{0.523416in}}%
\pgfpathlineto{\pgfqpoint{1.097783in}{0.523416in}}%
\pgfpathlineto{\pgfqpoint{1.097783in}{0.499444in}}%
\pgfpathclose%
\pgfusepath{fill}%
\end{pgfscope}%
\begin{pgfscope}%
\pgfpathrectangle{\pgfqpoint{0.445556in}{0.499444in}}{\pgfqpoint{3.875000in}{1.155000in}}%
\pgfusepath{clip}%
\pgfsetbuttcap%
\pgfsetmiterjoin%
\definecolor{currentfill}{rgb}{0.000000,0.000000,0.000000}%
\pgfsetfillcolor{currentfill}%
\pgfsetlinewidth{0.000000pt}%
\definecolor{currentstroke}{rgb}{0.000000,0.000000,0.000000}%
\pgfsetstrokecolor{currentstroke}%
\pgfsetstrokeopacity{0.000000}%
\pgfsetdash{}{0pt}%
\pgfpathmoveto{\pgfqpoint{1.251249in}{0.499444in}}%
\pgfpathlineto{\pgfqpoint{1.312635in}{0.499444in}}%
\pgfpathlineto{\pgfqpoint{1.312635in}{0.530029in}}%
\pgfpathlineto{\pgfqpoint{1.251249in}{0.530029in}}%
\pgfpathlineto{\pgfqpoint{1.251249in}{0.499444in}}%
\pgfpathclose%
\pgfusepath{fill}%
\end{pgfscope}%
\begin{pgfscope}%
\pgfpathrectangle{\pgfqpoint{0.445556in}{0.499444in}}{\pgfqpoint{3.875000in}{1.155000in}}%
\pgfusepath{clip}%
\pgfsetbuttcap%
\pgfsetmiterjoin%
\definecolor{currentfill}{rgb}{0.000000,0.000000,0.000000}%
\pgfsetfillcolor{currentfill}%
\pgfsetlinewidth{0.000000pt}%
\definecolor{currentstroke}{rgb}{0.000000,0.000000,0.000000}%
\pgfsetstrokecolor{currentstroke}%
\pgfsetstrokeopacity{0.000000}%
\pgfsetdash{}{0pt}%
\pgfpathmoveto{\pgfqpoint{1.404714in}{0.499444in}}%
\pgfpathlineto{\pgfqpoint{1.466100in}{0.499444in}}%
\pgfpathlineto{\pgfqpoint{1.466100in}{0.542547in}}%
\pgfpathlineto{\pgfqpoint{1.404714in}{0.542547in}}%
\pgfpathlineto{\pgfqpoint{1.404714in}{0.499444in}}%
\pgfpathclose%
\pgfusepath{fill}%
\end{pgfscope}%
\begin{pgfscope}%
\pgfpathrectangle{\pgfqpoint{0.445556in}{0.499444in}}{\pgfqpoint{3.875000in}{1.155000in}}%
\pgfusepath{clip}%
\pgfsetbuttcap%
\pgfsetmiterjoin%
\definecolor{currentfill}{rgb}{0.000000,0.000000,0.000000}%
\pgfsetfillcolor{currentfill}%
\pgfsetlinewidth{0.000000pt}%
\definecolor{currentstroke}{rgb}{0.000000,0.000000,0.000000}%
\pgfsetstrokecolor{currentstroke}%
\pgfsetstrokeopacity{0.000000}%
\pgfsetdash{}{0pt}%
\pgfpathmoveto{\pgfqpoint{1.558179in}{0.499444in}}%
\pgfpathlineto{\pgfqpoint{1.619566in}{0.499444in}}%
\pgfpathlineto{\pgfqpoint{1.619566in}{0.543019in}}%
\pgfpathlineto{\pgfqpoint{1.558179in}{0.543019in}}%
\pgfpathlineto{\pgfqpoint{1.558179in}{0.499444in}}%
\pgfpathclose%
\pgfusepath{fill}%
\end{pgfscope}%
\begin{pgfscope}%
\pgfpathrectangle{\pgfqpoint{0.445556in}{0.499444in}}{\pgfqpoint{3.875000in}{1.155000in}}%
\pgfusepath{clip}%
\pgfsetbuttcap%
\pgfsetmiterjoin%
\definecolor{currentfill}{rgb}{0.000000,0.000000,0.000000}%
\pgfsetfillcolor{currentfill}%
\pgfsetlinewidth{0.000000pt}%
\definecolor{currentstroke}{rgb}{0.000000,0.000000,0.000000}%
\pgfsetstrokecolor{currentstroke}%
\pgfsetstrokeopacity{0.000000}%
\pgfsetdash{}{0pt}%
\pgfpathmoveto{\pgfqpoint{1.711645in}{0.499444in}}%
\pgfpathlineto{\pgfqpoint{1.773031in}{0.499444in}}%
\pgfpathlineto{\pgfqpoint{1.773031in}{0.552230in}}%
\pgfpathlineto{\pgfqpoint{1.711645in}{0.552230in}}%
\pgfpathlineto{\pgfqpoint{1.711645in}{0.499444in}}%
\pgfpathclose%
\pgfusepath{fill}%
\end{pgfscope}%
\begin{pgfscope}%
\pgfpathrectangle{\pgfqpoint{0.445556in}{0.499444in}}{\pgfqpoint{3.875000in}{1.155000in}}%
\pgfusepath{clip}%
\pgfsetbuttcap%
\pgfsetmiterjoin%
\definecolor{currentfill}{rgb}{0.000000,0.000000,0.000000}%
\pgfsetfillcolor{currentfill}%
\pgfsetlinewidth{0.000000pt}%
\definecolor{currentstroke}{rgb}{0.000000,0.000000,0.000000}%
\pgfsetstrokecolor{currentstroke}%
\pgfsetstrokeopacity{0.000000}%
\pgfsetdash{}{0pt}%
\pgfpathmoveto{\pgfqpoint{1.865110in}{0.499444in}}%
\pgfpathlineto{\pgfqpoint{1.926496in}{0.499444in}}%
\pgfpathlineto{\pgfqpoint{1.926496in}{0.567463in}}%
\pgfpathlineto{\pgfqpoint{1.865110in}{0.567463in}}%
\pgfpathlineto{\pgfqpoint{1.865110in}{0.499444in}}%
\pgfpathclose%
\pgfusepath{fill}%
\end{pgfscope}%
\begin{pgfscope}%
\pgfpathrectangle{\pgfqpoint{0.445556in}{0.499444in}}{\pgfqpoint{3.875000in}{1.155000in}}%
\pgfusepath{clip}%
\pgfsetbuttcap%
\pgfsetmiterjoin%
\definecolor{currentfill}{rgb}{0.000000,0.000000,0.000000}%
\pgfsetfillcolor{currentfill}%
\pgfsetlinewidth{0.000000pt}%
\definecolor{currentstroke}{rgb}{0.000000,0.000000,0.000000}%
\pgfsetstrokecolor{currentstroke}%
\pgfsetstrokeopacity{0.000000}%
\pgfsetdash{}{0pt}%
\pgfpathmoveto{\pgfqpoint{2.018575in}{0.499444in}}%
\pgfpathlineto{\pgfqpoint{2.079962in}{0.499444in}}%
\pgfpathlineto{\pgfqpoint{2.079962in}{0.581162in}}%
\pgfpathlineto{\pgfqpoint{2.018575in}{0.581162in}}%
\pgfpathlineto{\pgfqpoint{2.018575in}{0.499444in}}%
\pgfpathclose%
\pgfusepath{fill}%
\end{pgfscope}%
\begin{pgfscope}%
\pgfpathrectangle{\pgfqpoint{0.445556in}{0.499444in}}{\pgfqpoint{3.875000in}{1.155000in}}%
\pgfusepath{clip}%
\pgfsetbuttcap%
\pgfsetmiterjoin%
\definecolor{currentfill}{rgb}{0.000000,0.000000,0.000000}%
\pgfsetfillcolor{currentfill}%
\pgfsetlinewidth{0.000000pt}%
\definecolor{currentstroke}{rgb}{0.000000,0.000000,0.000000}%
\pgfsetstrokecolor{currentstroke}%
\pgfsetstrokeopacity{0.000000}%
\pgfsetdash{}{0pt}%
\pgfpathmoveto{\pgfqpoint{2.172041in}{0.499444in}}%
\pgfpathlineto{\pgfqpoint{2.233427in}{0.499444in}}%
\pgfpathlineto{\pgfqpoint{2.233427in}{0.590255in}}%
\pgfpathlineto{\pgfqpoint{2.172041in}{0.590255in}}%
\pgfpathlineto{\pgfqpoint{2.172041in}{0.499444in}}%
\pgfpathclose%
\pgfusepath{fill}%
\end{pgfscope}%
\begin{pgfscope}%
\pgfpathrectangle{\pgfqpoint{0.445556in}{0.499444in}}{\pgfqpoint{3.875000in}{1.155000in}}%
\pgfusepath{clip}%
\pgfsetbuttcap%
\pgfsetmiterjoin%
\definecolor{currentfill}{rgb}{0.000000,0.000000,0.000000}%
\pgfsetfillcolor{currentfill}%
\pgfsetlinewidth{0.000000pt}%
\definecolor{currentstroke}{rgb}{0.000000,0.000000,0.000000}%
\pgfsetstrokecolor{currentstroke}%
\pgfsetstrokeopacity{0.000000}%
\pgfsetdash{}{0pt}%
\pgfpathmoveto{\pgfqpoint{2.325506in}{0.499444in}}%
\pgfpathlineto{\pgfqpoint{2.386892in}{0.499444in}}%
\pgfpathlineto{\pgfqpoint{2.386892in}{0.604189in}}%
\pgfpathlineto{\pgfqpoint{2.325506in}{0.604189in}}%
\pgfpathlineto{\pgfqpoint{2.325506in}{0.499444in}}%
\pgfpathclose%
\pgfusepath{fill}%
\end{pgfscope}%
\begin{pgfscope}%
\pgfpathrectangle{\pgfqpoint{0.445556in}{0.499444in}}{\pgfqpoint{3.875000in}{1.155000in}}%
\pgfusepath{clip}%
\pgfsetbuttcap%
\pgfsetmiterjoin%
\definecolor{currentfill}{rgb}{0.000000,0.000000,0.000000}%
\pgfsetfillcolor{currentfill}%
\pgfsetlinewidth{0.000000pt}%
\definecolor{currentstroke}{rgb}{0.000000,0.000000,0.000000}%
\pgfsetstrokecolor{currentstroke}%
\pgfsetstrokeopacity{0.000000}%
\pgfsetdash{}{0pt}%
\pgfpathmoveto{\pgfqpoint{2.478972in}{0.499444in}}%
\pgfpathlineto{\pgfqpoint{2.540358in}{0.499444in}}%
\pgfpathlineto{\pgfqpoint{2.540358in}{0.622965in}}%
\pgfpathlineto{\pgfqpoint{2.478972in}{0.622965in}}%
\pgfpathlineto{\pgfqpoint{2.478972in}{0.499444in}}%
\pgfpathclose%
\pgfusepath{fill}%
\end{pgfscope}%
\begin{pgfscope}%
\pgfpathrectangle{\pgfqpoint{0.445556in}{0.499444in}}{\pgfqpoint{3.875000in}{1.155000in}}%
\pgfusepath{clip}%
\pgfsetbuttcap%
\pgfsetmiterjoin%
\definecolor{currentfill}{rgb}{0.000000,0.000000,0.000000}%
\pgfsetfillcolor{currentfill}%
\pgfsetlinewidth{0.000000pt}%
\definecolor{currentstroke}{rgb}{0.000000,0.000000,0.000000}%
\pgfsetstrokecolor{currentstroke}%
\pgfsetstrokeopacity{0.000000}%
\pgfsetdash{}{0pt}%
\pgfpathmoveto{\pgfqpoint{2.632437in}{0.499444in}}%
\pgfpathlineto{\pgfqpoint{2.693823in}{0.499444in}}%
\pgfpathlineto{\pgfqpoint{2.693823in}{0.647292in}}%
\pgfpathlineto{\pgfqpoint{2.632437in}{0.647292in}}%
\pgfpathlineto{\pgfqpoint{2.632437in}{0.499444in}}%
\pgfpathclose%
\pgfusepath{fill}%
\end{pgfscope}%
\begin{pgfscope}%
\pgfpathrectangle{\pgfqpoint{0.445556in}{0.499444in}}{\pgfqpoint{3.875000in}{1.155000in}}%
\pgfusepath{clip}%
\pgfsetbuttcap%
\pgfsetmiterjoin%
\definecolor{currentfill}{rgb}{0.000000,0.000000,0.000000}%
\pgfsetfillcolor{currentfill}%
\pgfsetlinewidth{0.000000pt}%
\definecolor{currentstroke}{rgb}{0.000000,0.000000,0.000000}%
\pgfsetstrokecolor{currentstroke}%
\pgfsetstrokeopacity{0.000000}%
\pgfsetdash{}{0pt}%
\pgfpathmoveto{\pgfqpoint{2.785902in}{0.499444in}}%
\pgfpathlineto{\pgfqpoint{2.847288in}{0.499444in}}%
\pgfpathlineto{\pgfqpoint{2.847288in}{0.671264in}}%
\pgfpathlineto{\pgfqpoint{2.785902in}{0.671264in}}%
\pgfpathlineto{\pgfqpoint{2.785902in}{0.499444in}}%
\pgfpathclose%
\pgfusepath{fill}%
\end{pgfscope}%
\begin{pgfscope}%
\pgfpathrectangle{\pgfqpoint{0.445556in}{0.499444in}}{\pgfqpoint{3.875000in}{1.155000in}}%
\pgfusepath{clip}%
\pgfsetbuttcap%
\pgfsetmiterjoin%
\definecolor{currentfill}{rgb}{0.000000,0.000000,0.000000}%
\pgfsetfillcolor{currentfill}%
\pgfsetlinewidth{0.000000pt}%
\definecolor{currentstroke}{rgb}{0.000000,0.000000,0.000000}%
\pgfsetstrokecolor{currentstroke}%
\pgfsetstrokeopacity{0.000000}%
\pgfsetdash{}{0pt}%
\pgfpathmoveto{\pgfqpoint{2.939368in}{0.499444in}}%
\pgfpathlineto{\pgfqpoint{3.000754in}{0.499444in}}%
\pgfpathlineto{\pgfqpoint{3.000754in}{0.687442in}}%
\pgfpathlineto{\pgfqpoint{2.939368in}{0.687442in}}%
\pgfpathlineto{\pgfqpoint{2.939368in}{0.499444in}}%
\pgfpathclose%
\pgfusepath{fill}%
\end{pgfscope}%
\begin{pgfscope}%
\pgfpathrectangle{\pgfqpoint{0.445556in}{0.499444in}}{\pgfqpoint{3.875000in}{1.155000in}}%
\pgfusepath{clip}%
\pgfsetbuttcap%
\pgfsetmiterjoin%
\definecolor{currentfill}{rgb}{0.000000,0.000000,0.000000}%
\pgfsetfillcolor{currentfill}%
\pgfsetlinewidth{0.000000pt}%
\definecolor{currentstroke}{rgb}{0.000000,0.000000,0.000000}%
\pgfsetstrokecolor{currentstroke}%
\pgfsetstrokeopacity{0.000000}%
\pgfsetdash{}{0pt}%
\pgfpathmoveto{\pgfqpoint{3.092833in}{0.499444in}}%
\pgfpathlineto{\pgfqpoint{3.154219in}{0.499444in}}%
\pgfpathlineto{\pgfqpoint{3.154219in}{0.722869in}}%
\pgfpathlineto{\pgfqpoint{3.092833in}{0.722869in}}%
\pgfpathlineto{\pgfqpoint{3.092833in}{0.499444in}}%
\pgfpathclose%
\pgfusepath{fill}%
\end{pgfscope}%
\begin{pgfscope}%
\pgfpathrectangle{\pgfqpoint{0.445556in}{0.499444in}}{\pgfqpoint{3.875000in}{1.155000in}}%
\pgfusepath{clip}%
\pgfsetbuttcap%
\pgfsetmiterjoin%
\definecolor{currentfill}{rgb}{0.000000,0.000000,0.000000}%
\pgfsetfillcolor{currentfill}%
\pgfsetlinewidth{0.000000pt}%
\definecolor{currentstroke}{rgb}{0.000000,0.000000,0.000000}%
\pgfsetstrokecolor{currentstroke}%
\pgfsetstrokeopacity{0.000000}%
\pgfsetdash{}{0pt}%
\pgfpathmoveto{\pgfqpoint{3.246298in}{0.499444in}}%
\pgfpathlineto{\pgfqpoint{3.307684in}{0.499444in}}%
\pgfpathlineto{\pgfqpoint{3.307684in}{0.758059in}}%
\pgfpathlineto{\pgfqpoint{3.246298in}{0.758059in}}%
\pgfpathlineto{\pgfqpoint{3.246298in}{0.499444in}}%
\pgfpathclose%
\pgfusepath{fill}%
\end{pgfscope}%
\begin{pgfscope}%
\pgfpathrectangle{\pgfqpoint{0.445556in}{0.499444in}}{\pgfqpoint{3.875000in}{1.155000in}}%
\pgfusepath{clip}%
\pgfsetbuttcap%
\pgfsetmiterjoin%
\definecolor{currentfill}{rgb}{0.000000,0.000000,0.000000}%
\pgfsetfillcolor{currentfill}%
\pgfsetlinewidth{0.000000pt}%
\definecolor{currentstroke}{rgb}{0.000000,0.000000,0.000000}%
\pgfsetstrokecolor{currentstroke}%
\pgfsetstrokeopacity{0.000000}%
\pgfsetdash{}{0pt}%
\pgfpathmoveto{\pgfqpoint{3.399764in}{0.499444in}}%
\pgfpathlineto{\pgfqpoint{3.461150in}{0.499444in}}%
\pgfpathlineto{\pgfqpoint{3.461150in}{0.806476in}}%
\pgfpathlineto{\pgfqpoint{3.399764in}{0.806476in}}%
\pgfpathlineto{\pgfqpoint{3.399764in}{0.499444in}}%
\pgfpathclose%
\pgfusepath{fill}%
\end{pgfscope}%
\begin{pgfscope}%
\pgfpathrectangle{\pgfqpoint{0.445556in}{0.499444in}}{\pgfqpoint{3.875000in}{1.155000in}}%
\pgfusepath{clip}%
\pgfsetbuttcap%
\pgfsetmiterjoin%
\definecolor{currentfill}{rgb}{0.000000,0.000000,0.000000}%
\pgfsetfillcolor{currentfill}%
\pgfsetlinewidth{0.000000pt}%
\definecolor{currentstroke}{rgb}{0.000000,0.000000,0.000000}%
\pgfsetstrokecolor{currentstroke}%
\pgfsetstrokeopacity{0.000000}%
\pgfsetdash{}{0pt}%
\pgfpathmoveto{\pgfqpoint{3.553229in}{0.499444in}}%
\pgfpathlineto{\pgfqpoint{3.614615in}{0.499444in}}%
\pgfpathlineto{\pgfqpoint{3.614615in}{0.846862in}}%
\pgfpathlineto{\pgfqpoint{3.553229in}{0.846862in}}%
\pgfpathlineto{\pgfqpoint{3.553229in}{0.499444in}}%
\pgfpathclose%
\pgfusepath{fill}%
\end{pgfscope}%
\begin{pgfscope}%
\pgfpathrectangle{\pgfqpoint{0.445556in}{0.499444in}}{\pgfqpoint{3.875000in}{1.155000in}}%
\pgfusepath{clip}%
\pgfsetbuttcap%
\pgfsetmiterjoin%
\definecolor{currentfill}{rgb}{0.000000,0.000000,0.000000}%
\pgfsetfillcolor{currentfill}%
\pgfsetlinewidth{0.000000pt}%
\definecolor{currentstroke}{rgb}{0.000000,0.000000,0.000000}%
\pgfsetstrokecolor{currentstroke}%
\pgfsetstrokeopacity{0.000000}%
\pgfsetdash{}{0pt}%
\pgfpathmoveto{\pgfqpoint{3.706694in}{0.499444in}}%
\pgfpathlineto{\pgfqpoint{3.768080in}{0.499444in}}%
\pgfpathlineto{\pgfqpoint{3.768080in}{0.900947in}}%
\pgfpathlineto{\pgfqpoint{3.706694in}{0.900947in}}%
\pgfpathlineto{\pgfqpoint{3.706694in}{0.499444in}}%
\pgfpathclose%
\pgfusepath{fill}%
\end{pgfscope}%
\begin{pgfscope}%
\pgfpathrectangle{\pgfqpoint{0.445556in}{0.499444in}}{\pgfqpoint{3.875000in}{1.155000in}}%
\pgfusepath{clip}%
\pgfsetbuttcap%
\pgfsetmiterjoin%
\definecolor{currentfill}{rgb}{0.000000,0.000000,0.000000}%
\pgfsetfillcolor{currentfill}%
\pgfsetlinewidth{0.000000pt}%
\definecolor{currentstroke}{rgb}{0.000000,0.000000,0.000000}%
\pgfsetstrokecolor{currentstroke}%
\pgfsetstrokeopacity{0.000000}%
\pgfsetdash{}{0pt}%
\pgfpathmoveto{\pgfqpoint{3.860160in}{0.499444in}}%
\pgfpathlineto{\pgfqpoint{3.921546in}{0.499444in}}%
\pgfpathlineto{\pgfqpoint{3.921546in}{0.937200in}}%
\pgfpathlineto{\pgfqpoint{3.860160in}{0.937200in}}%
\pgfpathlineto{\pgfqpoint{3.860160in}{0.499444in}}%
\pgfpathclose%
\pgfusepath{fill}%
\end{pgfscope}%
\begin{pgfscope}%
\pgfpathrectangle{\pgfqpoint{0.445556in}{0.499444in}}{\pgfqpoint{3.875000in}{1.155000in}}%
\pgfusepath{clip}%
\pgfsetbuttcap%
\pgfsetmiterjoin%
\definecolor{currentfill}{rgb}{0.000000,0.000000,0.000000}%
\pgfsetfillcolor{currentfill}%
\pgfsetlinewidth{0.000000pt}%
\definecolor{currentstroke}{rgb}{0.000000,0.000000,0.000000}%
\pgfsetstrokecolor{currentstroke}%
\pgfsetstrokeopacity{0.000000}%
\pgfsetdash{}{0pt}%
\pgfpathmoveto{\pgfqpoint{4.013625in}{0.499444in}}%
\pgfpathlineto{\pgfqpoint{4.075011in}{0.499444in}}%
\pgfpathlineto{\pgfqpoint{4.075011in}{0.945112in}}%
\pgfpathlineto{\pgfqpoint{4.013625in}{0.945112in}}%
\pgfpathlineto{\pgfqpoint{4.013625in}{0.499444in}}%
\pgfpathclose%
\pgfusepath{fill}%
\end{pgfscope}%
\begin{pgfscope}%
\pgfpathrectangle{\pgfqpoint{0.445556in}{0.499444in}}{\pgfqpoint{3.875000in}{1.155000in}}%
\pgfusepath{clip}%
\pgfsetbuttcap%
\pgfsetmiterjoin%
\definecolor{currentfill}{rgb}{0.000000,0.000000,0.000000}%
\pgfsetfillcolor{currentfill}%
\pgfsetlinewidth{0.000000pt}%
\definecolor{currentstroke}{rgb}{0.000000,0.000000,0.000000}%
\pgfsetstrokecolor{currentstroke}%
\pgfsetstrokeopacity{0.000000}%
\pgfsetdash{}{0pt}%
\pgfpathmoveto{\pgfqpoint{4.167090in}{0.499444in}}%
\pgfpathlineto{\pgfqpoint{4.228476in}{0.499444in}}%
\pgfpathlineto{\pgfqpoint{4.228476in}{0.863985in}}%
\pgfpathlineto{\pgfqpoint{4.167090in}{0.863985in}}%
\pgfpathlineto{\pgfqpoint{4.167090in}{0.499444in}}%
\pgfpathclose%
\pgfusepath{fill}%
\end{pgfscope}%
\begin{pgfscope}%
\pgfsetbuttcap%
\pgfsetroundjoin%
\definecolor{currentfill}{rgb}{0.000000,0.000000,0.000000}%
\pgfsetfillcolor{currentfill}%
\pgfsetlinewidth{0.803000pt}%
\definecolor{currentstroke}{rgb}{0.000000,0.000000,0.000000}%
\pgfsetstrokecolor{currentstroke}%
\pgfsetdash{}{0pt}%
\pgfsys@defobject{currentmarker}{\pgfqpoint{0.000000in}{-0.048611in}}{\pgfqpoint{0.000000in}{0.000000in}}{%
\pgfpathmoveto{\pgfqpoint{0.000000in}{0.000000in}}%
\pgfpathlineto{\pgfqpoint{0.000000in}{-0.048611in}}%
\pgfusepath{stroke,fill}%
}%
\begin{pgfscope}%
\pgfsys@transformshift{0.483922in}{0.499444in}%
\pgfsys@useobject{currentmarker}{}%
\end{pgfscope}%
\end{pgfscope}%
\begin{pgfscope}%
\definecolor{textcolor}{rgb}{0.000000,0.000000,0.000000}%
\pgfsetstrokecolor{textcolor}%
\pgfsetfillcolor{textcolor}%
\pgftext[x=0.483922in,y=0.402222in,,top]{\color{textcolor}\rmfamily\fontsize{10.000000}{12.000000}\selectfont 0.0}%
\end{pgfscope}%
\begin{pgfscope}%
\pgfsetbuttcap%
\pgfsetroundjoin%
\definecolor{currentfill}{rgb}{0.000000,0.000000,0.000000}%
\pgfsetfillcolor{currentfill}%
\pgfsetlinewidth{0.803000pt}%
\definecolor{currentstroke}{rgb}{0.000000,0.000000,0.000000}%
\pgfsetstrokecolor{currentstroke}%
\pgfsetdash{}{0pt}%
\pgfsys@defobject{currentmarker}{\pgfqpoint{0.000000in}{-0.048611in}}{\pgfqpoint{0.000000in}{0.000000in}}{%
\pgfpathmoveto{\pgfqpoint{0.000000in}{0.000000in}}%
\pgfpathlineto{\pgfqpoint{0.000000in}{-0.048611in}}%
\pgfusepath{stroke,fill}%
}%
\begin{pgfscope}%
\pgfsys@transformshift{0.867585in}{0.499444in}%
\pgfsys@useobject{currentmarker}{}%
\end{pgfscope}%
\end{pgfscope}%
\begin{pgfscope}%
\definecolor{textcolor}{rgb}{0.000000,0.000000,0.000000}%
\pgfsetstrokecolor{textcolor}%
\pgfsetfillcolor{textcolor}%
\pgftext[x=0.867585in,y=0.402222in,,top]{\color{textcolor}\rmfamily\fontsize{10.000000}{12.000000}\selectfont 0.1}%
\end{pgfscope}%
\begin{pgfscope}%
\pgfsetbuttcap%
\pgfsetroundjoin%
\definecolor{currentfill}{rgb}{0.000000,0.000000,0.000000}%
\pgfsetfillcolor{currentfill}%
\pgfsetlinewidth{0.803000pt}%
\definecolor{currentstroke}{rgb}{0.000000,0.000000,0.000000}%
\pgfsetstrokecolor{currentstroke}%
\pgfsetdash{}{0pt}%
\pgfsys@defobject{currentmarker}{\pgfqpoint{0.000000in}{-0.048611in}}{\pgfqpoint{0.000000in}{0.000000in}}{%
\pgfpathmoveto{\pgfqpoint{0.000000in}{0.000000in}}%
\pgfpathlineto{\pgfqpoint{0.000000in}{-0.048611in}}%
\pgfusepath{stroke,fill}%
}%
\begin{pgfscope}%
\pgfsys@transformshift{1.251249in}{0.499444in}%
\pgfsys@useobject{currentmarker}{}%
\end{pgfscope}%
\end{pgfscope}%
\begin{pgfscope}%
\definecolor{textcolor}{rgb}{0.000000,0.000000,0.000000}%
\pgfsetstrokecolor{textcolor}%
\pgfsetfillcolor{textcolor}%
\pgftext[x=1.251249in,y=0.402222in,,top]{\color{textcolor}\rmfamily\fontsize{10.000000}{12.000000}\selectfont 0.2}%
\end{pgfscope}%
\begin{pgfscope}%
\pgfsetbuttcap%
\pgfsetroundjoin%
\definecolor{currentfill}{rgb}{0.000000,0.000000,0.000000}%
\pgfsetfillcolor{currentfill}%
\pgfsetlinewidth{0.803000pt}%
\definecolor{currentstroke}{rgb}{0.000000,0.000000,0.000000}%
\pgfsetstrokecolor{currentstroke}%
\pgfsetdash{}{0pt}%
\pgfsys@defobject{currentmarker}{\pgfqpoint{0.000000in}{-0.048611in}}{\pgfqpoint{0.000000in}{0.000000in}}{%
\pgfpathmoveto{\pgfqpoint{0.000000in}{0.000000in}}%
\pgfpathlineto{\pgfqpoint{0.000000in}{-0.048611in}}%
\pgfusepath{stroke,fill}%
}%
\begin{pgfscope}%
\pgfsys@transformshift{1.634912in}{0.499444in}%
\pgfsys@useobject{currentmarker}{}%
\end{pgfscope}%
\end{pgfscope}%
\begin{pgfscope}%
\definecolor{textcolor}{rgb}{0.000000,0.000000,0.000000}%
\pgfsetstrokecolor{textcolor}%
\pgfsetfillcolor{textcolor}%
\pgftext[x=1.634912in,y=0.402222in,,top]{\color{textcolor}\rmfamily\fontsize{10.000000}{12.000000}\selectfont 0.3}%
\end{pgfscope}%
\begin{pgfscope}%
\pgfsetbuttcap%
\pgfsetroundjoin%
\definecolor{currentfill}{rgb}{0.000000,0.000000,0.000000}%
\pgfsetfillcolor{currentfill}%
\pgfsetlinewidth{0.803000pt}%
\definecolor{currentstroke}{rgb}{0.000000,0.000000,0.000000}%
\pgfsetstrokecolor{currentstroke}%
\pgfsetdash{}{0pt}%
\pgfsys@defobject{currentmarker}{\pgfqpoint{0.000000in}{-0.048611in}}{\pgfqpoint{0.000000in}{0.000000in}}{%
\pgfpathmoveto{\pgfqpoint{0.000000in}{0.000000in}}%
\pgfpathlineto{\pgfqpoint{0.000000in}{-0.048611in}}%
\pgfusepath{stroke,fill}%
}%
\begin{pgfscope}%
\pgfsys@transformshift{2.018575in}{0.499444in}%
\pgfsys@useobject{currentmarker}{}%
\end{pgfscope}%
\end{pgfscope}%
\begin{pgfscope}%
\definecolor{textcolor}{rgb}{0.000000,0.000000,0.000000}%
\pgfsetstrokecolor{textcolor}%
\pgfsetfillcolor{textcolor}%
\pgftext[x=2.018575in,y=0.402222in,,top]{\color{textcolor}\rmfamily\fontsize{10.000000}{12.000000}\selectfont 0.4}%
\end{pgfscope}%
\begin{pgfscope}%
\pgfsetbuttcap%
\pgfsetroundjoin%
\definecolor{currentfill}{rgb}{0.000000,0.000000,0.000000}%
\pgfsetfillcolor{currentfill}%
\pgfsetlinewidth{0.803000pt}%
\definecolor{currentstroke}{rgb}{0.000000,0.000000,0.000000}%
\pgfsetstrokecolor{currentstroke}%
\pgfsetdash{}{0pt}%
\pgfsys@defobject{currentmarker}{\pgfqpoint{0.000000in}{-0.048611in}}{\pgfqpoint{0.000000in}{0.000000in}}{%
\pgfpathmoveto{\pgfqpoint{0.000000in}{0.000000in}}%
\pgfpathlineto{\pgfqpoint{0.000000in}{-0.048611in}}%
\pgfusepath{stroke,fill}%
}%
\begin{pgfscope}%
\pgfsys@transformshift{2.402239in}{0.499444in}%
\pgfsys@useobject{currentmarker}{}%
\end{pgfscope}%
\end{pgfscope}%
\begin{pgfscope}%
\definecolor{textcolor}{rgb}{0.000000,0.000000,0.000000}%
\pgfsetstrokecolor{textcolor}%
\pgfsetfillcolor{textcolor}%
\pgftext[x=2.402239in,y=0.402222in,,top]{\color{textcolor}\rmfamily\fontsize{10.000000}{12.000000}\selectfont 0.5}%
\end{pgfscope}%
\begin{pgfscope}%
\pgfsetbuttcap%
\pgfsetroundjoin%
\definecolor{currentfill}{rgb}{0.000000,0.000000,0.000000}%
\pgfsetfillcolor{currentfill}%
\pgfsetlinewidth{0.803000pt}%
\definecolor{currentstroke}{rgb}{0.000000,0.000000,0.000000}%
\pgfsetstrokecolor{currentstroke}%
\pgfsetdash{}{0pt}%
\pgfsys@defobject{currentmarker}{\pgfqpoint{0.000000in}{-0.048611in}}{\pgfqpoint{0.000000in}{0.000000in}}{%
\pgfpathmoveto{\pgfqpoint{0.000000in}{0.000000in}}%
\pgfpathlineto{\pgfqpoint{0.000000in}{-0.048611in}}%
\pgfusepath{stroke,fill}%
}%
\begin{pgfscope}%
\pgfsys@transformshift{2.785902in}{0.499444in}%
\pgfsys@useobject{currentmarker}{}%
\end{pgfscope}%
\end{pgfscope}%
\begin{pgfscope}%
\definecolor{textcolor}{rgb}{0.000000,0.000000,0.000000}%
\pgfsetstrokecolor{textcolor}%
\pgfsetfillcolor{textcolor}%
\pgftext[x=2.785902in,y=0.402222in,,top]{\color{textcolor}\rmfamily\fontsize{10.000000}{12.000000}\selectfont 0.6}%
\end{pgfscope}%
\begin{pgfscope}%
\pgfsetbuttcap%
\pgfsetroundjoin%
\definecolor{currentfill}{rgb}{0.000000,0.000000,0.000000}%
\pgfsetfillcolor{currentfill}%
\pgfsetlinewidth{0.803000pt}%
\definecolor{currentstroke}{rgb}{0.000000,0.000000,0.000000}%
\pgfsetstrokecolor{currentstroke}%
\pgfsetdash{}{0pt}%
\pgfsys@defobject{currentmarker}{\pgfqpoint{0.000000in}{-0.048611in}}{\pgfqpoint{0.000000in}{0.000000in}}{%
\pgfpathmoveto{\pgfqpoint{0.000000in}{0.000000in}}%
\pgfpathlineto{\pgfqpoint{0.000000in}{-0.048611in}}%
\pgfusepath{stroke,fill}%
}%
\begin{pgfscope}%
\pgfsys@transformshift{3.169566in}{0.499444in}%
\pgfsys@useobject{currentmarker}{}%
\end{pgfscope}%
\end{pgfscope}%
\begin{pgfscope}%
\definecolor{textcolor}{rgb}{0.000000,0.000000,0.000000}%
\pgfsetstrokecolor{textcolor}%
\pgfsetfillcolor{textcolor}%
\pgftext[x=3.169566in,y=0.402222in,,top]{\color{textcolor}\rmfamily\fontsize{10.000000}{12.000000}\selectfont 0.7}%
\end{pgfscope}%
\begin{pgfscope}%
\pgfsetbuttcap%
\pgfsetroundjoin%
\definecolor{currentfill}{rgb}{0.000000,0.000000,0.000000}%
\pgfsetfillcolor{currentfill}%
\pgfsetlinewidth{0.803000pt}%
\definecolor{currentstroke}{rgb}{0.000000,0.000000,0.000000}%
\pgfsetstrokecolor{currentstroke}%
\pgfsetdash{}{0pt}%
\pgfsys@defobject{currentmarker}{\pgfqpoint{0.000000in}{-0.048611in}}{\pgfqpoint{0.000000in}{0.000000in}}{%
\pgfpathmoveto{\pgfqpoint{0.000000in}{0.000000in}}%
\pgfpathlineto{\pgfqpoint{0.000000in}{-0.048611in}}%
\pgfusepath{stroke,fill}%
}%
\begin{pgfscope}%
\pgfsys@transformshift{3.553229in}{0.499444in}%
\pgfsys@useobject{currentmarker}{}%
\end{pgfscope}%
\end{pgfscope}%
\begin{pgfscope}%
\definecolor{textcolor}{rgb}{0.000000,0.000000,0.000000}%
\pgfsetstrokecolor{textcolor}%
\pgfsetfillcolor{textcolor}%
\pgftext[x=3.553229in,y=0.402222in,,top]{\color{textcolor}\rmfamily\fontsize{10.000000}{12.000000}\selectfont 0.8}%
\end{pgfscope}%
\begin{pgfscope}%
\pgfsetbuttcap%
\pgfsetroundjoin%
\definecolor{currentfill}{rgb}{0.000000,0.000000,0.000000}%
\pgfsetfillcolor{currentfill}%
\pgfsetlinewidth{0.803000pt}%
\definecolor{currentstroke}{rgb}{0.000000,0.000000,0.000000}%
\pgfsetstrokecolor{currentstroke}%
\pgfsetdash{}{0pt}%
\pgfsys@defobject{currentmarker}{\pgfqpoint{0.000000in}{-0.048611in}}{\pgfqpoint{0.000000in}{0.000000in}}{%
\pgfpathmoveto{\pgfqpoint{0.000000in}{0.000000in}}%
\pgfpathlineto{\pgfqpoint{0.000000in}{-0.048611in}}%
\pgfusepath{stroke,fill}%
}%
\begin{pgfscope}%
\pgfsys@transformshift{3.936892in}{0.499444in}%
\pgfsys@useobject{currentmarker}{}%
\end{pgfscope}%
\end{pgfscope}%
\begin{pgfscope}%
\definecolor{textcolor}{rgb}{0.000000,0.000000,0.000000}%
\pgfsetstrokecolor{textcolor}%
\pgfsetfillcolor{textcolor}%
\pgftext[x=3.936892in,y=0.402222in,,top]{\color{textcolor}\rmfamily\fontsize{10.000000}{12.000000}\selectfont 0.9}%
\end{pgfscope}%
\begin{pgfscope}%
\pgfsetbuttcap%
\pgfsetroundjoin%
\definecolor{currentfill}{rgb}{0.000000,0.000000,0.000000}%
\pgfsetfillcolor{currentfill}%
\pgfsetlinewidth{0.803000pt}%
\definecolor{currentstroke}{rgb}{0.000000,0.000000,0.000000}%
\pgfsetstrokecolor{currentstroke}%
\pgfsetdash{}{0pt}%
\pgfsys@defobject{currentmarker}{\pgfqpoint{0.000000in}{-0.048611in}}{\pgfqpoint{0.000000in}{0.000000in}}{%
\pgfpathmoveto{\pgfqpoint{0.000000in}{0.000000in}}%
\pgfpathlineto{\pgfqpoint{0.000000in}{-0.048611in}}%
\pgfusepath{stroke,fill}%
}%
\begin{pgfscope}%
\pgfsys@transformshift{4.320556in}{0.499444in}%
\pgfsys@useobject{currentmarker}{}%
\end{pgfscope}%
\end{pgfscope}%
\begin{pgfscope}%
\definecolor{textcolor}{rgb}{0.000000,0.000000,0.000000}%
\pgfsetstrokecolor{textcolor}%
\pgfsetfillcolor{textcolor}%
\pgftext[x=4.320556in,y=0.402222in,,top]{\color{textcolor}\rmfamily\fontsize{10.000000}{12.000000}\selectfont 1.0}%
\end{pgfscope}%
\begin{pgfscope}%
\definecolor{textcolor}{rgb}{0.000000,0.000000,0.000000}%
\pgfsetstrokecolor{textcolor}%
\pgfsetfillcolor{textcolor}%
\pgftext[x=2.383056in,y=0.223333in,,top]{\color{textcolor}\rmfamily\fontsize{10.000000}{12.000000}\selectfont \(\displaystyle p\)}%
\end{pgfscope}%
\begin{pgfscope}%
\pgfsetbuttcap%
\pgfsetroundjoin%
\definecolor{currentfill}{rgb}{0.000000,0.000000,0.000000}%
\pgfsetfillcolor{currentfill}%
\pgfsetlinewidth{0.803000pt}%
\definecolor{currentstroke}{rgb}{0.000000,0.000000,0.000000}%
\pgfsetstrokecolor{currentstroke}%
\pgfsetdash{}{0pt}%
\pgfsys@defobject{currentmarker}{\pgfqpoint{-0.048611in}{0.000000in}}{\pgfqpoint{-0.000000in}{0.000000in}}{%
\pgfpathmoveto{\pgfqpoint{-0.000000in}{0.000000in}}%
\pgfpathlineto{\pgfqpoint{-0.048611in}{0.000000in}}%
\pgfusepath{stroke,fill}%
}%
\begin{pgfscope}%
\pgfsys@transformshift{0.445556in}{0.499444in}%
\pgfsys@useobject{currentmarker}{}%
\end{pgfscope}%
\end{pgfscope}%
\begin{pgfscope}%
\definecolor{textcolor}{rgb}{0.000000,0.000000,0.000000}%
\pgfsetstrokecolor{textcolor}%
\pgfsetfillcolor{textcolor}%
\pgftext[x=0.278889in, y=0.451250in, left, base]{\color{textcolor}\rmfamily\fontsize{10.000000}{12.000000}\selectfont \(\displaystyle {0}\)}%
\end{pgfscope}%
\begin{pgfscope}%
\pgfsetbuttcap%
\pgfsetroundjoin%
\definecolor{currentfill}{rgb}{0.000000,0.000000,0.000000}%
\pgfsetfillcolor{currentfill}%
\pgfsetlinewidth{0.803000pt}%
\definecolor{currentstroke}{rgb}{0.000000,0.000000,0.000000}%
\pgfsetstrokecolor{currentstroke}%
\pgfsetdash{}{0pt}%
\pgfsys@defobject{currentmarker}{\pgfqpoint{-0.048611in}{0.000000in}}{\pgfqpoint{-0.000000in}{0.000000in}}{%
\pgfpathmoveto{\pgfqpoint{-0.000000in}{0.000000in}}%
\pgfpathlineto{\pgfqpoint{-0.048611in}{0.000000in}}%
\pgfusepath{stroke,fill}%
}%
\begin{pgfscope}%
\pgfsys@transformshift{0.445556in}{1.005031in}%
\pgfsys@useobject{currentmarker}{}%
\end{pgfscope}%
\end{pgfscope}%
\begin{pgfscope}%
\definecolor{textcolor}{rgb}{0.000000,0.000000,0.000000}%
\pgfsetstrokecolor{textcolor}%
\pgfsetfillcolor{textcolor}%
\pgftext[x=0.278889in, y=0.956836in, left, base]{\color{textcolor}\rmfamily\fontsize{10.000000}{12.000000}\selectfont \(\displaystyle {2}\)}%
\end{pgfscope}%
\begin{pgfscope}%
\pgfsetbuttcap%
\pgfsetroundjoin%
\definecolor{currentfill}{rgb}{0.000000,0.000000,0.000000}%
\pgfsetfillcolor{currentfill}%
\pgfsetlinewidth{0.803000pt}%
\definecolor{currentstroke}{rgb}{0.000000,0.000000,0.000000}%
\pgfsetstrokecolor{currentstroke}%
\pgfsetdash{}{0pt}%
\pgfsys@defobject{currentmarker}{\pgfqpoint{-0.048611in}{0.000000in}}{\pgfqpoint{-0.000000in}{0.000000in}}{%
\pgfpathmoveto{\pgfqpoint{-0.000000in}{0.000000in}}%
\pgfpathlineto{\pgfqpoint{-0.048611in}{0.000000in}}%
\pgfusepath{stroke,fill}%
}%
\begin{pgfscope}%
\pgfsys@transformshift{0.445556in}{1.510618in}%
\pgfsys@useobject{currentmarker}{}%
\end{pgfscope}%
\end{pgfscope}%
\begin{pgfscope}%
\definecolor{textcolor}{rgb}{0.000000,0.000000,0.000000}%
\pgfsetstrokecolor{textcolor}%
\pgfsetfillcolor{textcolor}%
\pgftext[x=0.278889in, y=1.462423in, left, base]{\color{textcolor}\rmfamily\fontsize{10.000000}{12.000000}\selectfont \(\displaystyle {4}\)}%
\end{pgfscope}%
\begin{pgfscope}%
\definecolor{textcolor}{rgb}{0.000000,0.000000,0.000000}%
\pgfsetstrokecolor{textcolor}%
\pgfsetfillcolor{textcolor}%
\pgftext[x=0.223333in,y=1.076944in,,bottom,rotate=90.000000]{\color{textcolor}\rmfamily\fontsize{10.000000}{12.000000}\selectfont Percent of Data Set}%
\end{pgfscope}%
\begin{pgfscope}%
\pgfsetrectcap%
\pgfsetmiterjoin%
\pgfsetlinewidth{0.803000pt}%
\definecolor{currentstroke}{rgb}{0.000000,0.000000,0.000000}%
\pgfsetstrokecolor{currentstroke}%
\pgfsetdash{}{0pt}%
\pgfpathmoveto{\pgfqpoint{0.445556in}{0.499444in}}%
\pgfpathlineto{\pgfqpoint{0.445556in}{1.654444in}}%
\pgfusepath{stroke}%
\end{pgfscope}%
\begin{pgfscope}%
\pgfsetrectcap%
\pgfsetmiterjoin%
\pgfsetlinewidth{0.803000pt}%
\definecolor{currentstroke}{rgb}{0.000000,0.000000,0.000000}%
\pgfsetstrokecolor{currentstroke}%
\pgfsetdash{}{0pt}%
\pgfpathmoveto{\pgfqpoint{4.320556in}{0.499444in}}%
\pgfpathlineto{\pgfqpoint{4.320556in}{1.654444in}}%
\pgfusepath{stroke}%
\end{pgfscope}%
\begin{pgfscope}%
\pgfsetrectcap%
\pgfsetmiterjoin%
\pgfsetlinewidth{0.803000pt}%
\definecolor{currentstroke}{rgb}{0.000000,0.000000,0.000000}%
\pgfsetstrokecolor{currentstroke}%
\pgfsetdash{}{0pt}%
\pgfpathmoveto{\pgfqpoint{0.445556in}{0.499444in}}%
\pgfpathlineto{\pgfqpoint{4.320556in}{0.499444in}}%
\pgfusepath{stroke}%
\end{pgfscope}%
\begin{pgfscope}%
\pgfsetrectcap%
\pgfsetmiterjoin%
\pgfsetlinewidth{0.803000pt}%
\definecolor{currentstroke}{rgb}{0.000000,0.000000,0.000000}%
\pgfsetstrokecolor{currentstroke}%
\pgfsetdash{}{0pt}%
\pgfpathmoveto{\pgfqpoint{0.445556in}{1.654444in}}%
\pgfpathlineto{\pgfqpoint{4.320556in}{1.654444in}}%
\pgfusepath{stroke}%
\end{pgfscope}%
\begin{pgfscope}%
\pgfsetbuttcap%
\pgfsetmiterjoin%
\definecolor{currentfill}{rgb}{1.000000,1.000000,1.000000}%
\pgfsetfillcolor{currentfill}%
\pgfsetfillopacity{0.800000}%
\pgfsetlinewidth{1.003750pt}%
\definecolor{currentstroke}{rgb}{0.800000,0.800000,0.800000}%
\pgfsetstrokecolor{currentstroke}%
\pgfsetstrokeopacity{0.800000}%
\pgfsetdash{}{0pt}%
\pgfpathmoveto{\pgfqpoint{3.543611in}{1.154445in}}%
\pgfpathlineto{\pgfqpoint{4.223333in}{1.154445in}}%
\pgfpathquadraticcurveto{\pgfqpoint{4.251111in}{1.154445in}}{\pgfqpoint{4.251111in}{1.182222in}}%
\pgfpathlineto{\pgfqpoint{4.251111in}{1.557222in}}%
\pgfpathquadraticcurveto{\pgfqpoint{4.251111in}{1.585000in}}{\pgfqpoint{4.223333in}{1.585000in}}%
\pgfpathlineto{\pgfqpoint{3.543611in}{1.585000in}}%
\pgfpathquadraticcurveto{\pgfqpoint{3.515833in}{1.585000in}}{\pgfqpoint{3.515833in}{1.557222in}}%
\pgfpathlineto{\pgfqpoint{3.515833in}{1.182222in}}%
\pgfpathquadraticcurveto{\pgfqpoint{3.515833in}{1.154445in}}{\pgfqpoint{3.543611in}{1.154445in}}%
\pgfpathlineto{\pgfqpoint{3.543611in}{1.154445in}}%
\pgfpathclose%
\pgfusepath{stroke,fill}%
\end{pgfscope}%
\begin{pgfscope}%
\pgfsetbuttcap%
\pgfsetmiterjoin%
\pgfsetlinewidth{1.003750pt}%
\definecolor{currentstroke}{rgb}{0.000000,0.000000,0.000000}%
\pgfsetstrokecolor{currentstroke}%
\pgfsetdash{}{0pt}%
\pgfpathmoveto{\pgfqpoint{3.571389in}{1.432222in}}%
\pgfpathlineto{\pgfqpoint{3.849167in}{1.432222in}}%
\pgfpathlineto{\pgfqpoint{3.849167in}{1.529444in}}%
\pgfpathlineto{\pgfqpoint{3.571389in}{1.529444in}}%
\pgfpathlineto{\pgfqpoint{3.571389in}{1.432222in}}%
\pgfpathclose%
\pgfusepath{stroke}%
\end{pgfscope}%
\begin{pgfscope}%
\definecolor{textcolor}{rgb}{0.000000,0.000000,0.000000}%
\pgfsetstrokecolor{textcolor}%
\pgfsetfillcolor{textcolor}%
\pgftext[x=3.960278in,y=1.432222in,left,base]{\color{textcolor}\rmfamily\fontsize{10.000000}{12.000000}\selectfont Neg}%
\end{pgfscope}%
\begin{pgfscope}%
\pgfsetbuttcap%
\pgfsetmiterjoin%
\definecolor{currentfill}{rgb}{0.000000,0.000000,0.000000}%
\pgfsetfillcolor{currentfill}%
\pgfsetlinewidth{0.000000pt}%
\definecolor{currentstroke}{rgb}{0.000000,0.000000,0.000000}%
\pgfsetstrokecolor{currentstroke}%
\pgfsetstrokeopacity{0.000000}%
\pgfsetdash{}{0pt}%
\pgfpathmoveto{\pgfqpoint{3.571389in}{1.236944in}}%
\pgfpathlineto{\pgfqpoint{3.849167in}{1.236944in}}%
\pgfpathlineto{\pgfqpoint{3.849167in}{1.334167in}}%
\pgfpathlineto{\pgfqpoint{3.571389in}{1.334167in}}%
\pgfpathlineto{\pgfqpoint{3.571389in}{1.236944in}}%
\pgfpathclose%
\pgfusepath{fill}%
\end{pgfscope}%
\begin{pgfscope}%
\definecolor{textcolor}{rgb}{0.000000,0.000000,0.000000}%
\pgfsetstrokecolor{textcolor}%
\pgfsetfillcolor{textcolor}%
\pgftext[x=3.960278in,y=1.236944in,left,base]{\color{textcolor}\rmfamily\fontsize{10.000000}{12.000000}\selectfont Pos}%
\end{pgfscope}%
\end{pgfpicture}%
\makeatother%
\endgroup%

&
	\vskip 0pt
	\qquad \qquad FP/TP
	
	%% Creator: Matplotlib, PGF backend
%%
%% To include the figure in your LaTeX document, write
%%   \input{<filename>.pgf}
%%
%% Make sure the required packages are loaded in your preamble
%%   \usepackage{pgf}
%%
%% Also ensure that all the required font packages are loaded; for instance,
%% the lmodern package is sometimes necessary when using math font.
%%   \usepackage{lmodern}
%%
%% Figures using additional raster images can only be included by \input if
%% they are in the same directory as the main LaTeX file. For loading figures
%% from other directories you can use the `import` package
%%   \usepackage{import}
%%
%% and then include the figures with
%%   \import{<path to file>}{<filename>.pgf}
%%
%% Matplotlib used the following preamble
%%   
%%   \usepackage{fontspec}
%%   \makeatletter\@ifpackageloaded{underscore}{}{\usepackage[strings]{underscore}}\makeatother
%%
\begingroup%
\makeatletter%
\begin{pgfpicture}%
\pgfpathrectangle{\pgfpointorigin}{\pgfqpoint{2.247807in}{1.754444in}}%
\pgfusepath{use as bounding box, clip}%
\begin{pgfscope}%
\pgfsetbuttcap%
\pgfsetmiterjoin%
\definecolor{currentfill}{rgb}{1.000000,1.000000,1.000000}%
\pgfsetfillcolor{currentfill}%
\pgfsetlinewidth{0.000000pt}%
\definecolor{currentstroke}{rgb}{1.000000,1.000000,1.000000}%
\pgfsetstrokecolor{currentstroke}%
\pgfsetdash{}{0pt}%
\pgfpathmoveto{\pgfqpoint{0.000000in}{0.000000in}}%
\pgfpathlineto{\pgfqpoint{2.247807in}{0.000000in}}%
\pgfpathlineto{\pgfqpoint{2.247807in}{1.754444in}}%
\pgfpathlineto{\pgfqpoint{0.000000in}{1.754444in}}%
\pgfpathlineto{\pgfqpoint{0.000000in}{0.000000in}}%
\pgfpathclose%
\pgfusepath{fill}%
\end{pgfscope}%
\begin{pgfscope}%
\pgfsetbuttcap%
\pgfsetmiterjoin%
\definecolor{currentfill}{rgb}{1.000000,1.000000,1.000000}%
\pgfsetfillcolor{currentfill}%
\pgfsetlinewidth{0.000000pt}%
\definecolor{currentstroke}{rgb}{0.000000,0.000000,0.000000}%
\pgfsetstrokecolor{currentstroke}%
\pgfsetstrokeopacity{0.000000}%
\pgfsetdash{}{0pt}%
\pgfpathmoveto{\pgfqpoint{0.530556in}{0.499444in}}%
\pgfpathlineto{\pgfqpoint{2.080556in}{0.499444in}}%
\pgfpathlineto{\pgfqpoint{2.080556in}{1.654444in}}%
\pgfpathlineto{\pgfqpoint{0.530556in}{1.654444in}}%
\pgfpathlineto{\pgfqpoint{0.530556in}{0.499444in}}%
\pgfpathclose%
\pgfusepath{fill}%
\end{pgfscope}%
\begin{pgfscope}%
\pgfsetbuttcap%
\pgfsetroundjoin%
\definecolor{currentfill}{rgb}{0.000000,0.000000,0.000000}%
\pgfsetfillcolor{currentfill}%
\pgfsetlinewidth{0.803000pt}%
\definecolor{currentstroke}{rgb}{0.000000,0.000000,0.000000}%
\pgfsetstrokecolor{currentstroke}%
\pgfsetdash{}{0pt}%
\pgfsys@defobject{currentmarker}{\pgfqpoint{0.000000in}{-0.048611in}}{\pgfqpoint{0.000000in}{0.000000in}}{%
\pgfpathmoveto{\pgfqpoint{0.000000in}{0.000000in}}%
\pgfpathlineto{\pgfqpoint{0.000000in}{-0.048611in}}%
\pgfusepath{stroke,fill}%
}%
\begin{pgfscope}%
\pgfsys@transformshift{0.601010in}{0.499444in}%
\pgfsys@useobject{currentmarker}{}%
\end{pgfscope}%
\end{pgfscope}%
\begin{pgfscope}%
\definecolor{textcolor}{rgb}{0.000000,0.000000,0.000000}%
\pgfsetstrokecolor{textcolor}%
\pgfsetfillcolor{textcolor}%
\pgftext[x=0.601010in,y=0.402222in,,top]{\color{textcolor}\rmfamily\fontsize{10.000000}{12.000000}\selectfont 0.014}%
\end{pgfscope}%
\begin{pgfscope}%
\pgfsetbuttcap%
\pgfsetroundjoin%
\definecolor{currentfill}{rgb}{0.000000,0.000000,0.000000}%
\pgfsetfillcolor{currentfill}%
\pgfsetlinewidth{0.803000pt}%
\definecolor{currentstroke}{rgb}{0.000000,0.000000,0.000000}%
\pgfsetstrokecolor{currentstroke}%
\pgfsetdash{}{0pt}%
\pgfsys@defobject{currentmarker}{\pgfqpoint{0.000000in}{-0.048611in}}{\pgfqpoint{0.000000in}{0.000000in}}{%
\pgfpathmoveto{\pgfqpoint{0.000000in}{0.000000in}}%
\pgfpathlineto{\pgfqpoint{0.000000in}{-0.048611in}}%
\pgfusepath{stroke,fill}%
}%
\begin{pgfscope}%
\pgfsys@transformshift{2.024334in}{0.499444in}%
\pgfsys@useobject{currentmarker}{}%
\end{pgfscope}%
\end{pgfscope}%
\begin{pgfscope}%
\definecolor{textcolor}{rgb}{0.000000,0.000000,0.000000}%
\pgfsetstrokecolor{textcolor}%
\pgfsetfillcolor{textcolor}%
\pgftext[x=2.024334in,y=0.402222in,,top]{\color{textcolor}\rmfamily\fontsize{10.000000}{12.000000}\selectfont 0.99}%
\end{pgfscope}%
\begin{pgfscope}%
\definecolor{textcolor}{rgb}{0.000000,0.000000,0.000000}%
\pgfsetstrokecolor{textcolor}%
\pgfsetfillcolor{textcolor}%
\pgftext[x=1.305556in,y=0.223333in,,top]{\color{textcolor}\rmfamily\fontsize{10.000000}{12.000000}\selectfont \(\displaystyle p\)}%
\end{pgfscope}%
\begin{pgfscope}%
\pgfsetbuttcap%
\pgfsetroundjoin%
\definecolor{currentfill}{rgb}{0.000000,0.000000,0.000000}%
\pgfsetfillcolor{currentfill}%
\pgfsetlinewidth{0.803000pt}%
\definecolor{currentstroke}{rgb}{0.000000,0.000000,0.000000}%
\pgfsetstrokecolor{currentstroke}%
\pgfsetdash{}{0pt}%
\pgfsys@defobject{currentmarker}{\pgfqpoint{-0.048611in}{0.000000in}}{\pgfqpoint{-0.000000in}{0.000000in}}{%
\pgfpathmoveto{\pgfqpoint{-0.000000in}{0.000000in}}%
\pgfpathlineto{\pgfqpoint{-0.048611in}{0.000000in}}%
\pgfusepath{stroke,fill}%
}%
\begin{pgfscope}%
\pgfsys@transformshift{0.530556in}{0.542456in}%
\pgfsys@useobject{currentmarker}{}%
\end{pgfscope}%
\end{pgfscope}%
\begin{pgfscope}%
\definecolor{textcolor}{rgb}{0.000000,0.000000,0.000000}%
\pgfsetstrokecolor{textcolor}%
\pgfsetfillcolor{textcolor}%
\pgftext[x=0.363889in, y=0.494261in, left, base]{\color{textcolor}\rmfamily\fontsize{10.000000}{12.000000}\selectfont \(\displaystyle {0}\)}%
\end{pgfscope}%
\begin{pgfscope}%
\pgfsetbuttcap%
\pgfsetroundjoin%
\definecolor{currentfill}{rgb}{0.000000,0.000000,0.000000}%
\pgfsetfillcolor{currentfill}%
\pgfsetlinewidth{0.803000pt}%
\definecolor{currentstroke}{rgb}{0.000000,0.000000,0.000000}%
\pgfsetstrokecolor{currentstroke}%
\pgfsetdash{}{0pt}%
\pgfsys@defobject{currentmarker}{\pgfqpoint{-0.048611in}{0.000000in}}{\pgfqpoint{-0.000000in}{0.000000in}}{%
\pgfpathmoveto{\pgfqpoint{-0.000000in}{0.000000in}}%
\pgfpathlineto{\pgfqpoint{-0.048611in}{0.000000in}}%
\pgfusepath{stroke,fill}%
}%
\begin{pgfscope}%
\pgfsys@transformshift{0.530556in}{1.108345in}%
\pgfsys@useobject{currentmarker}{}%
\end{pgfscope}%
\end{pgfscope}%
\begin{pgfscope}%
\definecolor{textcolor}{rgb}{0.000000,0.000000,0.000000}%
\pgfsetstrokecolor{textcolor}%
\pgfsetfillcolor{textcolor}%
\pgftext[x=0.294444in, y=1.060150in, left, base]{\color{textcolor}\rmfamily\fontsize{10.000000}{12.000000}\selectfont \(\displaystyle {50}\)}%
\end{pgfscope}%
\begin{pgfscope}%
\definecolor{textcolor}{rgb}{0.000000,0.000000,0.000000}%
\pgfsetstrokecolor{textcolor}%
\pgfsetfillcolor{textcolor}%
\pgftext[x=0.238889in,y=1.076944in,,bottom,rotate=90.000000]{\color{textcolor}\rmfamily\fontsize{10.000000}{12.000000}\selectfont \(\displaystyle \Delta\)FP/\(\displaystyle \Delta\)TP}%
\end{pgfscope}%
\begin{pgfscope}%
\pgfpathrectangle{\pgfqpoint{0.530556in}{0.499444in}}{\pgfqpoint{1.550000in}{1.155000in}}%
\pgfusepath{clip}%
\pgfsetrectcap%
\pgfsetroundjoin%
\pgfsetlinewidth{1.505625pt}%
\definecolor{currentstroke}{rgb}{0.000000,0.000000,0.000000}%
\pgfsetstrokecolor{currentstroke}%
\pgfsetdash{}{0pt}%
\pgfpathmoveto{\pgfqpoint{0.601010in}{1.601944in}}%
\pgfpathlineto{\pgfqpoint{0.615243in}{1.571216in}}%
\pgfpathlineto{\pgfqpoint{0.629477in}{1.540689in}}%
\pgfpathlineto{\pgfqpoint{0.643710in}{1.512843in}}%
\pgfpathlineto{\pgfqpoint{0.657943in}{1.487516in}}%
\pgfpathlineto{\pgfqpoint{0.672176in}{1.460543in}}%
\pgfpathlineto{\pgfqpoint{0.686410in}{1.409324in}}%
\pgfpathlineto{\pgfqpoint{0.700643in}{1.347499in}}%
\pgfpathlineto{\pgfqpoint{0.714876in}{1.293259in}}%
\pgfpathlineto{\pgfqpoint{0.729109in}{1.250832in}}%
\pgfpathlineto{\pgfqpoint{0.743343in}{1.208957in}}%
\pgfpathlineto{\pgfqpoint{0.757576in}{1.167186in}}%
\pgfpathlineto{\pgfqpoint{0.771809in}{1.128181in}}%
\pgfpathlineto{\pgfqpoint{0.786042in}{1.089923in}}%
\pgfpathlineto{\pgfqpoint{0.800276in}{1.051927in}}%
\pgfpathlineto{\pgfqpoint{0.814509in}{1.018069in}}%
\pgfpathlineto{\pgfqpoint{0.828742in}{0.984843in}}%
\pgfpathlineto{\pgfqpoint{0.842975in}{0.955283in}}%
\pgfpathlineto{\pgfqpoint{0.857209in}{0.929223in}}%
\pgfpathlineto{\pgfqpoint{0.871442in}{0.904269in}}%
\pgfpathlineto{\pgfqpoint{0.885675in}{0.881558in}}%
\pgfpathlineto{\pgfqpoint{0.899908in}{0.861177in}}%
\pgfpathlineto{\pgfqpoint{0.914142in}{0.842616in}}%
\pgfpathlineto{\pgfqpoint{0.928375in}{0.826237in}}%
\pgfpathlineto{\pgfqpoint{0.942608in}{0.812265in}}%
\pgfpathlineto{\pgfqpoint{0.956841in}{0.800593in}}%
\pgfpathlineto{\pgfqpoint{0.971075in}{0.791081in}}%
\pgfpathlineto{\pgfqpoint{0.985308in}{0.782225in}}%
\pgfpathlineto{\pgfqpoint{0.999541in}{0.774418in}}%
\pgfpathlineto{\pgfqpoint{1.013774in}{0.766603in}}%
\pgfpathlineto{\pgfqpoint{1.028007in}{0.759292in}}%
\pgfpathlineto{\pgfqpoint{1.042241in}{0.751462in}}%
\pgfpathlineto{\pgfqpoint{1.056474in}{0.744731in}}%
\pgfpathlineto{\pgfqpoint{1.070707in}{0.738173in}}%
\pgfpathlineto{\pgfqpoint{1.084940in}{0.731648in}}%
\pgfpathlineto{\pgfqpoint{1.099174in}{0.724845in}}%
\pgfpathlineto{\pgfqpoint{1.113407in}{0.717816in}}%
\pgfpathlineto{\pgfqpoint{1.127640in}{0.711493in}}%
\pgfpathlineto{\pgfqpoint{1.141873in}{0.705342in}}%
\pgfpathlineto{\pgfqpoint{1.156107in}{0.699413in}}%
\pgfpathlineto{\pgfqpoint{1.170340in}{0.693678in}}%
\pgfpathlineto{\pgfqpoint{1.184573in}{0.688676in}}%
\pgfpathlineto{\pgfqpoint{1.198806in}{0.683630in}}%
\pgfpathlineto{\pgfqpoint{1.213040in}{0.679369in}}%
\pgfpathlineto{\pgfqpoint{1.227273in}{0.675182in}}%
\pgfpathlineto{\pgfqpoint{1.241506in}{0.671138in}}%
\pgfpathlineto{\pgfqpoint{1.255739in}{0.667145in}}%
\pgfpathlineto{\pgfqpoint{1.269973in}{0.663314in}}%
\pgfpathlineto{\pgfqpoint{1.284206in}{0.659440in}}%
\pgfpathlineto{\pgfqpoint{1.298439in}{0.656039in}}%
\pgfpathlineto{\pgfqpoint{1.312672in}{0.652758in}}%
\pgfpathlineto{\pgfqpoint{1.326906in}{0.649363in}}%
\pgfpathlineto{\pgfqpoint{1.341139in}{0.645935in}}%
\pgfpathlineto{\pgfqpoint{1.355372in}{0.642223in}}%
\pgfpathlineto{\pgfqpoint{1.369605in}{0.638856in}}%
\pgfpathlineto{\pgfqpoint{1.383839in}{0.635669in}}%
\pgfpathlineto{\pgfqpoint{1.398072in}{0.632578in}}%
\pgfpathlineto{\pgfqpoint{1.412305in}{0.629479in}}%
\pgfpathlineto{\pgfqpoint{1.426538in}{0.626568in}}%
\pgfpathlineto{\pgfqpoint{1.440771in}{0.623697in}}%
\pgfpathlineto{\pgfqpoint{1.455005in}{0.621017in}}%
\pgfpathlineto{\pgfqpoint{1.469238in}{0.618543in}}%
\pgfpathlineto{\pgfqpoint{1.483471in}{0.616274in}}%
\pgfpathlineto{\pgfqpoint{1.497704in}{0.614143in}}%
\pgfpathlineto{\pgfqpoint{1.511938in}{0.611880in}}%
\pgfpathlineto{\pgfqpoint{1.526171in}{0.609695in}}%
\pgfpathlineto{\pgfqpoint{1.540404in}{0.607577in}}%
\pgfpathlineto{\pgfqpoint{1.554637in}{0.605516in}}%
\pgfpathlineto{\pgfqpoint{1.568871in}{0.603411in}}%
\pgfpathlineto{\pgfqpoint{1.583104in}{0.601235in}}%
\pgfpathlineto{\pgfqpoint{1.597337in}{0.599033in}}%
\pgfpathlineto{\pgfqpoint{1.611570in}{0.596775in}}%
\pgfpathlineto{\pgfqpoint{1.625804in}{0.594509in}}%
\pgfpathlineto{\pgfqpoint{1.640037in}{0.592298in}}%
\pgfpathlineto{\pgfqpoint{1.654270in}{0.590226in}}%
\pgfpathlineto{\pgfqpoint{1.668503in}{0.588117in}}%
\pgfpathlineto{\pgfqpoint{1.682737in}{0.586065in}}%
\pgfpathlineto{\pgfqpoint{1.696970in}{0.584032in}}%
\pgfpathlineto{\pgfqpoint{1.711203in}{0.582076in}}%
\pgfpathlineto{\pgfqpoint{1.725436in}{0.580196in}}%
\pgfpathlineto{\pgfqpoint{1.739670in}{0.578360in}}%
\pgfpathlineto{\pgfqpoint{1.753903in}{0.576609in}}%
\pgfpathlineto{\pgfqpoint{1.768136in}{0.574904in}}%
\pgfpathlineto{\pgfqpoint{1.782369in}{0.573200in}}%
\pgfpathlineto{\pgfqpoint{1.796603in}{0.571505in}}%
\pgfpathlineto{\pgfqpoint{1.810836in}{0.569826in}}%
\pgfpathlineto{\pgfqpoint{1.825069in}{0.568175in}}%
\pgfpathlineto{\pgfqpoint{1.839302in}{0.566552in}}%
\pgfpathlineto{\pgfqpoint{1.853535in}{0.564935in}}%
\pgfpathlineto{\pgfqpoint{1.867769in}{0.563335in}}%
\pgfpathlineto{\pgfqpoint{1.882002in}{0.561737in}}%
\pgfpathlineto{\pgfqpoint{1.896235in}{0.560146in}}%
\pgfpathlineto{\pgfqpoint{1.910468in}{0.558627in}}%
\pgfpathlineto{\pgfqpoint{1.924702in}{0.557183in}}%
\pgfpathlineto{\pgfqpoint{1.938935in}{0.555831in}}%
\pgfpathlineto{\pgfqpoint{1.953168in}{0.554577in}}%
\pgfpathlineto{\pgfqpoint{1.967401in}{0.553880in}}%
\pgfpathlineto{\pgfqpoint{1.981635in}{0.553202in}}%
\pgfpathlineto{\pgfqpoint{1.995868in}{0.552552in}}%
\pgfpathlineto{\pgfqpoint{2.010101in}{0.551944in}}%
\pgfusepath{stroke}%
\end{pgfscope}%
\begin{pgfscope}%
\pgfpathrectangle{\pgfqpoint{0.530556in}{0.499444in}}{\pgfqpoint{1.550000in}{1.155000in}}%
\pgfusepath{clip}%
\pgfsetbuttcap%
\pgfsetroundjoin%
\pgfsetlinewidth{1.505625pt}%
\definecolor{currentstroke}{rgb}{0.000000,0.000000,0.000000}%
\pgfsetstrokecolor{currentstroke}%
\pgfsetdash{{5.550000pt}{2.400000pt}}{0.000000pt}%
\pgfpathmoveto{\pgfqpoint{0.530556in}{0.565091in}}%
\pgfpathlineto{\pgfqpoint{2.080556in}{0.565091in}}%
\pgfusepath{stroke}%
\end{pgfscope}%
\begin{pgfscope}%
\pgfpathrectangle{\pgfqpoint{0.530556in}{0.499444in}}{\pgfqpoint{1.550000in}{1.155000in}}%
\pgfusepath{clip}%
\pgfsetrectcap%
\pgfsetroundjoin%
\pgfsetlinewidth{1.505625pt}%
\definecolor{currentstroke}{rgb}{0.121569,0.466667,0.705882}%
\pgfsetstrokecolor{currentstroke}%
\pgfsetdash{}{0pt}%
\pgfpathmoveto{\pgfqpoint{1.853535in}{0.565091in}}%
\pgfusepath{stroke}%
\end{pgfscope}%
\begin{pgfscope}%
\pgfpathrectangle{\pgfqpoint{0.530556in}{0.499444in}}{\pgfqpoint{1.550000in}{1.155000in}}%
\pgfusepath{clip}%
\pgfsetbuttcap%
\pgfsetroundjoin%
\definecolor{currentfill}{rgb}{0.000000,0.000000,0.000000}%
\pgfsetfillcolor{currentfill}%
\pgfsetlinewidth{1.003750pt}%
\definecolor{currentstroke}{rgb}{0.000000,0.000000,0.000000}%
\pgfsetstrokecolor{currentstroke}%
\pgfsetdash{}{0pt}%
\pgfsys@defobject{currentmarker}{\pgfqpoint{-0.041667in}{-0.041667in}}{\pgfqpoint{0.041667in}{0.041667in}}{%
\pgfpathmoveto{\pgfqpoint{0.000000in}{-0.041667in}}%
\pgfpathcurveto{\pgfqpoint{0.011050in}{-0.041667in}}{\pgfqpoint{0.021649in}{-0.037276in}}{\pgfqpoint{0.029463in}{-0.029463in}}%
\pgfpathcurveto{\pgfqpoint{0.037276in}{-0.021649in}}{\pgfqpoint{0.041667in}{-0.011050in}}{\pgfqpoint{0.041667in}{0.000000in}}%
\pgfpathcurveto{\pgfqpoint{0.041667in}{0.011050in}}{\pgfqpoint{0.037276in}{0.021649in}}{\pgfqpoint{0.029463in}{0.029463in}}%
\pgfpathcurveto{\pgfqpoint{0.021649in}{0.037276in}}{\pgfqpoint{0.011050in}{0.041667in}}{\pgfqpoint{0.000000in}{0.041667in}}%
\pgfpathcurveto{\pgfqpoint{-0.011050in}{0.041667in}}{\pgfqpoint{-0.021649in}{0.037276in}}{\pgfqpoint{-0.029463in}{0.029463in}}%
\pgfpathcurveto{\pgfqpoint{-0.037276in}{0.021649in}}{\pgfqpoint{-0.041667in}{0.011050in}}{\pgfqpoint{-0.041667in}{0.000000in}}%
\pgfpathcurveto{\pgfqpoint{-0.041667in}{-0.011050in}}{\pgfqpoint{-0.037276in}{-0.021649in}}{\pgfqpoint{-0.029463in}{-0.029463in}}%
\pgfpathcurveto{\pgfqpoint{-0.021649in}{-0.037276in}}{\pgfqpoint{-0.011050in}{-0.041667in}}{\pgfqpoint{0.000000in}{-0.041667in}}%
\pgfpathlineto{\pgfqpoint{0.000000in}{-0.041667in}}%
\pgfpathclose%
\pgfusepath{stroke,fill}%
}%
\begin{pgfscope}%
\pgfsys@transformshift{1.853535in}{0.565091in}%
\pgfsys@useobject{currentmarker}{}%
\end{pgfscope}%
\end{pgfscope}%
\begin{pgfscope}%
\pgfsetrectcap%
\pgfsetmiterjoin%
\pgfsetlinewidth{0.803000pt}%
\definecolor{currentstroke}{rgb}{0.000000,0.000000,0.000000}%
\pgfsetstrokecolor{currentstroke}%
\pgfsetdash{}{0pt}%
\pgfpathmoveto{\pgfqpoint{0.530556in}{0.499444in}}%
\pgfpathlineto{\pgfqpoint{0.530556in}{1.654444in}}%
\pgfusepath{stroke}%
\end{pgfscope}%
\begin{pgfscope}%
\pgfsetrectcap%
\pgfsetmiterjoin%
\pgfsetlinewidth{0.803000pt}%
\definecolor{currentstroke}{rgb}{0.000000,0.000000,0.000000}%
\pgfsetstrokecolor{currentstroke}%
\pgfsetdash{}{0pt}%
\pgfpathmoveto{\pgfqpoint{2.080556in}{0.499444in}}%
\pgfpathlineto{\pgfqpoint{2.080556in}{1.654444in}}%
\pgfusepath{stroke}%
\end{pgfscope}%
\begin{pgfscope}%
\pgfsetrectcap%
\pgfsetmiterjoin%
\pgfsetlinewidth{0.803000pt}%
\definecolor{currentstroke}{rgb}{0.000000,0.000000,0.000000}%
\pgfsetstrokecolor{currentstroke}%
\pgfsetdash{}{0pt}%
\pgfpathmoveto{\pgfqpoint{0.530556in}{0.499444in}}%
\pgfpathlineto{\pgfqpoint{2.080556in}{0.499444in}}%
\pgfusepath{stroke}%
\end{pgfscope}%
\begin{pgfscope}%
\pgfsetrectcap%
\pgfsetmiterjoin%
\pgfsetlinewidth{0.803000pt}%
\definecolor{currentstroke}{rgb}{0.000000,0.000000,0.000000}%
\pgfsetstrokecolor{currentstroke}%
\pgfsetdash{}{0pt}%
\pgfpathmoveto{\pgfqpoint{0.530556in}{1.654444in}}%
\pgfpathlineto{\pgfqpoint{2.080556in}{1.654444in}}%
\pgfusepath{stroke}%
\end{pgfscope}%
\begin{pgfscope}%
\pgfsetbuttcap%
\pgfsetmiterjoin%
\definecolor{currentfill}{rgb}{1.000000,1.000000,1.000000}%
\pgfsetfillcolor{currentfill}%
\pgfsetfillopacity{0.800000}%
\pgfsetlinewidth{1.003750pt}%
\definecolor{currentstroke}{rgb}{0.800000,0.800000,0.800000}%
\pgfsetstrokecolor{currentstroke}%
\pgfsetstrokeopacity{0.800000}%
\pgfsetdash{}{0pt}%
\pgfpathmoveto{\pgfqpoint{0.811987in}{1.126667in}}%
\pgfpathlineto{\pgfqpoint{1.983333in}{1.126667in}}%
\pgfpathquadraticcurveto{\pgfqpoint{2.011111in}{1.126667in}}{\pgfqpoint{2.011111in}{1.154444in}}%
\pgfpathlineto{\pgfqpoint{2.011111in}{1.557222in}}%
\pgfpathquadraticcurveto{\pgfqpoint{2.011111in}{1.585000in}}{\pgfqpoint{1.983333in}{1.585000in}}%
\pgfpathlineto{\pgfqpoint{0.811987in}{1.585000in}}%
\pgfpathquadraticcurveto{\pgfqpoint{0.784210in}{1.585000in}}{\pgfqpoint{0.784210in}{1.557222in}}%
\pgfpathlineto{\pgfqpoint{0.784210in}{1.154444in}}%
\pgfpathquadraticcurveto{\pgfqpoint{0.784210in}{1.126667in}}{\pgfqpoint{0.811987in}{1.126667in}}%
\pgfpathlineto{\pgfqpoint{0.811987in}{1.126667in}}%
\pgfpathclose%
\pgfusepath{stroke,fill}%
\end{pgfscope}%
\begin{pgfscope}%
\pgfsetrectcap%
\pgfsetroundjoin%
\pgfsetlinewidth{1.505625pt}%
\definecolor{currentstroke}{rgb}{0.000000,0.000000,0.000000}%
\pgfsetstrokecolor{currentstroke}%
\pgfsetdash{}{0pt}%
\pgfpathmoveto{\pgfqpoint{0.839765in}{1.473889in}}%
\pgfpathlineto{\pgfqpoint{0.978654in}{1.473889in}}%
\pgfpathlineto{\pgfqpoint{1.117543in}{1.473889in}}%
\pgfusepath{stroke}%
\end{pgfscope}%
\begin{pgfscope}%
\definecolor{textcolor}{rgb}{0.000000,0.000000,0.000000}%
\pgfsetstrokecolor{textcolor}%
\pgfsetfillcolor{textcolor}%
\pgftext[x=1.228654in,y=1.425277in,left,base]{\color{textcolor}\rmfamily\fontsize{10.000000}{12.000000}\selectfont \(\displaystyle \Delta FP/\Delta TP\)}%
\end{pgfscope}%
\begin{pgfscope}%
\pgfsetrectcap%
\pgfsetroundjoin%
\pgfsetlinewidth{1.505625pt}%
\definecolor{currentstroke}{rgb}{0.121569,0.466667,0.705882}%
\pgfsetstrokecolor{currentstroke}%
\pgfsetdash{}{0pt}%
\pgfpathmoveto{\pgfqpoint{0.839765in}{1.265555in}}%
\pgfpathlineto{\pgfqpoint{0.978654in}{1.265555in}}%
\pgfpathlineto{\pgfqpoint{1.117543in}{1.265555in}}%
\pgfusepath{stroke}%
\end{pgfscope}%
\begin{pgfscope}%
\pgfsetbuttcap%
\pgfsetroundjoin%
\definecolor{currentfill}{rgb}{0.000000,0.000000,0.000000}%
\pgfsetfillcolor{currentfill}%
\pgfsetlinewidth{1.003750pt}%
\definecolor{currentstroke}{rgb}{0.000000,0.000000,0.000000}%
\pgfsetstrokecolor{currentstroke}%
\pgfsetdash{}{0pt}%
\pgfsys@defobject{currentmarker}{\pgfqpoint{-0.041667in}{-0.041667in}}{\pgfqpoint{0.041667in}{0.041667in}}{%
\pgfpathmoveto{\pgfqpoint{0.000000in}{-0.041667in}}%
\pgfpathcurveto{\pgfqpoint{0.011050in}{-0.041667in}}{\pgfqpoint{0.021649in}{-0.037276in}}{\pgfqpoint{0.029463in}{-0.029463in}}%
\pgfpathcurveto{\pgfqpoint{0.037276in}{-0.021649in}}{\pgfqpoint{0.041667in}{-0.011050in}}{\pgfqpoint{0.041667in}{0.000000in}}%
\pgfpathcurveto{\pgfqpoint{0.041667in}{0.011050in}}{\pgfqpoint{0.037276in}{0.021649in}}{\pgfqpoint{0.029463in}{0.029463in}}%
\pgfpathcurveto{\pgfqpoint{0.021649in}{0.037276in}}{\pgfqpoint{0.011050in}{0.041667in}}{\pgfqpoint{0.000000in}{0.041667in}}%
\pgfpathcurveto{\pgfqpoint{-0.011050in}{0.041667in}}{\pgfqpoint{-0.021649in}{0.037276in}}{\pgfqpoint{-0.029463in}{0.029463in}}%
\pgfpathcurveto{\pgfqpoint{-0.037276in}{0.021649in}}{\pgfqpoint{-0.041667in}{0.011050in}}{\pgfqpoint{-0.041667in}{0.000000in}}%
\pgfpathcurveto{\pgfqpoint{-0.041667in}{-0.011050in}}{\pgfqpoint{-0.037276in}{-0.021649in}}{\pgfqpoint{-0.029463in}{-0.029463in}}%
\pgfpathcurveto{\pgfqpoint{-0.021649in}{-0.037276in}}{\pgfqpoint{-0.011050in}{-0.041667in}}{\pgfqpoint{0.000000in}{-0.041667in}}%
\pgfpathlineto{\pgfqpoint{0.000000in}{-0.041667in}}%
\pgfpathclose%
\pgfusepath{stroke,fill}%
}%
\begin{pgfscope}%
\pgfsys@transformshift{0.978654in}{1.265555in}%
\pgfsys@useobject{currentmarker}{}%
\end{pgfscope}%
\end{pgfscope}%
\begin{pgfscope}%
\definecolor{textcolor}{rgb}{0.000000,0.000000,0.000000}%
\pgfsetstrokecolor{textcolor}%
\pgfsetfillcolor{textcolor}%
\pgftext[x=1.228654in,y=1.216944in,left,base]{\color{textcolor}\rmfamily\fontsize{10.000000}{12.000000}\selectfont (0.873,2)}%
\end{pgfscope}%
\end{pgfpicture}%
\makeatother%
\endgroup%

\end{tabular}


\noindent\begin{tabular}{@{\hspace{-6pt}}p{4.5in} @{\hspace{6pt}}p{2.0in}}
	\vskip 0pt
	\qquad \qquad Transformed Model Output:  Map $0.873$ to 0.5 and 0 to 0.
	
	%% Creator: Matplotlib, PGF backend
%%
%% To include the figure in your LaTeX document, write
%%   \input{<filename>.pgf}
%%
%% Make sure the required packages are loaded in your preamble
%%   \usepackage{pgf}
%%
%% Also ensure that all the required font packages are loaded; for instance,
%% the lmodern package is sometimes necessary when using math font.
%%   \usepackage{lmodern}
%%
%% Figures using additional raster images can only be included by \input if
%% they are in the same directory as the main LaTeX file. For loading figures
%% from other directories you can use the `import` package
%%   \usepackage{import}
%%
%% and then include the figures with
%%   \import{<path to file>}{<filename>.pgf}
%%
%% Matplotlib used the following preamble
%%   
%%   \usepackage{fontspec}
%%   \makeatletter\@ifpackageloaded{underscore}{}{\usepackage[strings]{underscore}}\makeatother
%%
\begingroup%
\makeatletter%
\begin{pgfpicture}%
\pgfpathrectangle{\pgfpointorigin}{\pgfqpoint{4.509306in}{1.754444in}}%
\pgfusepath{use as bounding box, clip}%
\begin{pgfscope}%
\pgfsetbuttcap%
\pgfsetmiterjoin%
\definecolor{currentfill}{rgb}{1.000000,1.000000,1.000000}%
\pgfsetfillcolor{currentfill}%
\pgfsetlinewidth{0.000000pt}%
\definecolor{currentstroke}{rgb}{1.000000,1.000000,1.000000}%
\pgfsetstrokecolor{currentstroke}%
\pgfsetdash{}{0pt}%
\pgfpathmoveto{\pgfqpoint{0.000000in}{0.000000in}}%
\pgfpathlineto{\pgfqpoint{4.509306in}{0.000000in}}%
\pgfpathlineto{\pgfqpoint{4.509306in}{1.754444in}}%
\pgfpathlineto{\pgfqpoint{0.000000in}{1.754444in}}%
\pgfpathlineto{\pgfqpoint{0.000000in}{0.000000in}}%
\pgfpathclose%
\pgfusepath{fill}%
\end{pgfscope}%
\begin{pgfscope}%
\pgfsetbuttcap%
\pgfsetmiterjoin%
\definecolor{currentfill}{rgb}{1.000000,1.000000,1.000000}%
\pgfsetfillcolor{currentfill}%
\pgfsetlinewidth{0.000000pt}%
\definecolor{currentstroke}{rgb}{0.000000,0.000000,0.000000}%
\pgfsetstrokecolor{currentstroke}%
\pgfsetstrokeopacity{0.000000}%
\pgfsetdash{}{0pt}%
\pgfpathmoveto{\pgfqpoint{0.445556in}{0.499444in}}%
\pgfpathlineto{\pgfqpoint{4.320556in}{0.499444in}}%
\pgfpathlineto{\pgfqpoint{4.320556in}{1.654444in}}%
\pgfpathlineto{\pgfqpoint{0.445556in}{1.654444in}}%
\pgfpathlineto{\pgfqpoint{0.445556in}{0.499444in}}%
\pgfpathclose%
\pgfusepath{fill}%
\end{pgfscope}%
\begin{pgfscope}%
\pgfpathrectangle{\pgfqpoint{0.445556in}{0.499444in}}{\pgfqpoint{3.875000in}{1.155000in}}%
\pgfusepath{clip}%
\pgfsetbuttcap%
\pgfsetmiterjoin%
\pgfsetlinewidth{1.003750pt}%
\definecolor{currentstroke}{rgb}{0.000000,0.000000,0.000000}%
\pgfsetstrokecolor{currentstroke}%
\pgfsetdash{}{0pt}%
\pgfpathmoveto{\pgfqpoint{0.435556in}{0.499444in}}%
\pgfpathlineto{\pgfqpoint{0.483922in}{0.499444in}}%
\pgfpathlineto{\pgfqpoint{0.483922in}{1.057277in}}%
\pgfpathlineto{\pgfqpoint{0.435556in}{1.057277in}}%
\pgfusepath{stroke}%
\end{pgfscope}%
\begin{pgfscope}%
\pgfpathrectangle{\pgfqpoint{0.445556in}{0.499444in}}{\pgfqpoint{3.875000in}{1.155000in}}%
\pgfusepath{clip}%
\pgfsetbuttcap%
\pgfsetmiterjoin%
\pgfsetlinewidth{1.003750pt}%
\definecolor{currentstroke}{rgb}{0.000000,0.000000,0.000000}%
\pgfsetstrokecolor{currentstroke}%
\pgfsetdash{}{0pt}%
\pgfpathmoveto{\pgfqpoint{0.576001in}{0.499444in}}%
\pgfpathlineto{\pgfqpoint{0.637387in}{0.499444in}}%
\pgfpathlineto{\pgfqpoint{0.637387in}{1.251353in}}%
\pgfpathlineto{\pgfqpoint{0.576001in}{1.251353in}}%
\pgfpathlineto{\pgfqpoint{0.576001in}{0.499444in}}%
\pgfpathclose%
\pgfusepath{stroke}%
\end{pgfscope}%
\begin{pgfscope}%
\pgfpathrectangle{\pgfqpoint{0.445556in}{0.499444in}}{\pgfqpoint{3.875000in}{1.155000in}}%
\pgfusepath{clip}%
\pgfsetbuttcap%
\pgfsetmiterjoin%
\pgfsetlinewidth{1.003750pt}%
\definecolor{currentstroke}{rgb}{0.000000,0.000000,0.000000}%
\pgfsetstrokecolor{currentstroke}%
\pgfsetdash{}{0pt}%
\pgfpathmoveto{\pgfqpoint{0.729467in}{0.499444in}}%
\pgfpathlineto{\pgfqpoint{0.790853in}{0.499444in}}%
\pgfpathlineto{\pgfqpoint{0.790853in}{1.323578in}}%
\pgfpathlineto{\pgfqpoint{0.729467in}{1.323578in}}%
\pgfpathlineto{\pgfqpoint{0.729467in}{0.499444in}}%
\pgfpathclose%
\pgfusepath{stroke}%
\end{pgfscope}%
\begin{pgfscope}%
\pgfpathrectangle{\pgfqpoint{0.445556in}{0.499444in}}{\pgfqpoint{3.875000in}{1.155000in}}%
\pgfusepath{clip}%
\pgfsetbuttcap%
\pgfsetmiterjoin%
\pgfsetlinewidth{1.003750pt}%
\definecolor{currentstroke}{rgb}{0.000000,0.000000,0.000000}%
\pgfsetstrokecolor{currentstroke}%
\pgfsetdash{}{0pt}%
\pgfpathmoveto{\pgfqpoint{0.882932in}{0.499444in}}%
\pgfpathlineto{\pgfqpoint{0.944318in}{0.499444in}}%
\pgfpathlineto{\pgfqpoint{0.944318in}{1.367037in}}%
\pgfpathlineto{\pgfqpoint{0.882932in}{1.367037in}}%
\pgfpathlineto{\pgfqpoint{0.882932in}{0.499444in}}%
\pgfpathclose%
\pgfusepath{stroke}%
\end{pgfscope}%
\begin{pgfscope}%
\pgfpathrectangle{\pgfqpoint{0.445556in}{0.499444in}}{\pgfqpoint{3.875000in}{1.155000in}}%
\pgfusepath{clip}%
\pgfsetbuttcap%
\pgfsetmiterjoin%
\pgfsetlinewidth{1.003750pt}%
\definecolor{currentstroke}{rgb}{0.000000,0.000000,0.000000}%
\pgfsetstrokecolor{currentstroke}%
\pgfsetdash{}{0pt}%
\pgfpathmoveto{\pgfqpoint{1.036397in}{0.499444in}}%
\pgfpathlineto{\pgfqpoint{1.097783in}{0.499444in}}%
\pgfpathlineto{\pgfqpoint{1.097783in}{1.415210in}}%
\pgfpathlineto{\pgfqpoint{1.036397in}{1.415210in}}%
\pgfpathlineto{\pgfqpoint{1.036397in}{0.499444in}}%
\pgfpathclose%
\pgfusepath{stroke}%
\end{pgfscope}%
\begin{pgfscope}%
\pgfpathrectangle{\pgfqpoint{0.445556in}{0.499444in}}{\pgfqpoint{3.875000in}{1.155000in}}%
\pgfusepath{clip}%
\pgfsetbuttcap%
\pgfsetmiterjoin%
\pgfsetlinewidth{1.003750pt}%
\definecolor{currentstroke}{rgb}{0.000000,0.000000,0.000000}%
\pgfsetstrokecolor{currentstroke}%
\pgfsetdash{}{0pt}%
\pgfpathmoveto{\pgfqpoint{1.189863in}{0.499444in}}%
\pgfpathlineto{\pgfqpoint{1.251249in}{0.499444in}}%
\pgfpathlineto{\pgfqpoint{1.251249in}{1.461234in}}%
\pgfpathlineto{\pgfqpoint{1.189863in}{1.461234in}}%
\pgfpathlineto{\pgfqpoint{1.189863in}{0.499444in}}%
\pgfpathclose%
\pgfusepath{stroke}%
\end{pgfscope}%
\begin{pgfscope}%
\pgfpathrectangle{\pgfqpoint{0.445556in}{0.499444in}}{\pgfqpoint{3.875000in}{1.155000in}}%
\pgfusepath{clip}%
\pgfsetbuttcap%
\pgfsetmiterjoin%
\pgfsetlinewidth{1.003750pt}%
\definecolor{currentstroke}{rgb}{0.000000,0.000000,0.000000}%
\pgfsetstrokecolor{currentstroke}%
\pgfsetdash{}{0pt}%
\pgfpathmoveto{\pgfqpoint{1.343328in}{0.499444in}}%
\pgfpathlineto{\pgfqpoint{1.404714in}{0.499444in}}%
\pgfpathlineto{\pgfqpoint{1.404714in}{1.507812in}}%
\pgfpathlineto{\pgfqpoint{1.343328in}{1.507812in}}%
\pgfpathlineto{\pgfqpoint{1.343328in}{0.499444in}}%
\pgfpathclose%
\pgfusepath{stroke}%
\end{pgfscope}%
\begin{pgfscope}%
\pgfpathrectangle{\pgfqpoint{0.445556in}{0.499444in}}{\pgfqpoint{3.875000in}{1.155000in}}%
\pgfusepath{clip}%
\pgfsetbuttcap%
\pgfsetmiterjoin%
\pgfsetlinewidth{1.003750pt}%
\definecolor{currentstroke}{rgb}{0.000000,0.000000,0.000000}%
\pgfsetstrokecolor{currentstroke}%
\pgfsetdash{}{0pt}%
\pgfpathmoveto{\pgfqpoint{1.496793in}{0.499444in}}%
\pgfpathlineto{\pgfqpoint{1.558179in}{0.499444in}}%
\pgfpathlineto{\pgfqpoint{1.558179in}{1.555153in}}%
\pgfpathlineto{\pgfqpoint{1.496793in}{1.555153in}}%
\pgfpathlineto{\pgfqpoint{1.496793in}{0.499444in}}%
\pgfpathclose%
\pgfusepath{stroke}%
\end{pgfscope}%
\begin{pgfscope}%
\pgfpathrectangle{\pgfqpoint{0.445556in}{0.499444in}}{\pgfqpoint{3.875000in}{1.155000in}}%
\pgfusepath{clip}%
\pgfsetbuttcap%
\pgfsetmiterjoin%
\pgfsetlinewidth{1.003750pt}%
\definecolor{currentstroke}{rgb}{0.000000,0.000000,0.000000}%
\pgfsetstrokecolor{currentstroke}%
\pgfsetdash{}{0pt}%
\pgfpathmoveto{\pgfqpoint{1.650259in}{0.499444in}}%
\pgfpathlineto{\pgfqpoint{1.711645in}{0.499444in}}%
\pgfpathlineto{\pgfqpoint{1.711645in}{1.577680in}}%
\pgfpathlineto{\pgfqpoint{1.650259in}{1.577680in}}%
\pgfpathlineto{\pgfqpoint{1.650259in}{0.499444in}}%
\pgfpathclose%
\pgfusepath{stroke}%
\end{pgfscope}%
\begin{pgfscope}%
\pgfpathrectangle{\pgfqpoint{0.445556in}{0.499444in}}{\pgfqpoint{3.875000in}{1.155000in}}%
\pgfusepath{clip}%
\pgfsetbuttcap%
\pgfsetmiterjoin%
\pgfsetlinewidth{1.003750pt}%
\definecolor{currentstroke}{rgb}{0.000000,0.000000,0.000000}%
\pgfsetstrokecolor{currentstroke}%
\pgfsetdash{}{0pt}%
\pgfpathmoveto{\pgfqpoint{1.803724in}{0.499444in}}%
\pgfpathlineto{\pgfqpoint{1.865110in}{0.499444in}}%
\pgfpathlineto{\pgfqpoint{1.865110in}{1.599444in}}%
\pgfpathlineto{\pgfqpoint{1.803724in}{1.599444in}}%
\pgfpathlineto{\pgfqpoint{1.803724in}{0.499444in}}%
\pgfpathclose%
\pgfusepath{stroke}%
\end{pgfscope}%
\begin{pgfscope}%
\pgfpathrectangle{\pgfqpoint{0.445556in}{0.499444in}}{\pgfqpoint{3.875000in}{1.155000in}}%
\pgfusepath{clip}%
\pgfsetbuttcap%
\pgfsetmiterjoin%
\pgfsetlinewidth{1.003750pt}%
\definecolor{currentstroke}{rgb}{0.000000,0.000000,0.000000}%
\pgfsetstrokecolor{currentstroke}%
\pgfsetdash{}{0pt}%
\pgfpathmoveto{\pgfqpoint{1.957189in}{0.499444in}}%
\pgfpathlineto{\pgfqpoint{2.018575in}{0.499444in}}%
\pgfpathlineto{\pgfqpoint{2.018575in}{1.556678in}}%
\pgfpathlineto{\pgfqpoint{1.957189in}{1.556678in}}%
\pgfpathlineto{\pgfqpoint{1.957189in}{0.499444in}}%
\pgfpathclose%
\pgfusepath{stroke}%
\end{pgfscope}%
\begin{pgfscope}%
\pgfpathrectangle{\pgfqpoint{0.445556in}{0.499444in}}{\pgfqpoint{3.875000in}{1.155000in}}%
\pgfusepath{clip}%
\pgfsetbuttcap%
\pgfsetmiterjoin%
\pgfsetlinewidth{1.003750pt}%
\definecolor{currentstroke}{rgb}{0.000000,0.000000,0.000000}%
\pgfsetstrokecolor{currentstroke}%
\pgfsetdash{}{0pt}%
\pgfpathmoveto{\pgfqpoint{2.110655in}{0.499444in}}%
\pgfpathlineto{\pgfqpoint{2.172041in}{0.499444in}}%
\pgfpathlineto{\pgfqpoint{2.172041in}{1.485909in}}%
\pgfpathlineto{\pgfqpoint{2.110655in}{1.485909in}}%
\pgfpathlineto{\pgfqpoint{2.110655in}{0.499444in}}%
\pgfpathclose%
\pgfusepath{stroke}%
\end{pgfscope}%
\begin{pgfscope}%
\pgfpathrectangle{\pgfqpoint{0.445556in}{0.499444in}}{\pgfqpoint{3.875000in}{1.155000in}}%
\pgfusepath{clip}%
\pgfsetbuttcap%
\pgfsetmiterjoin%
\pgfsetlinewidth{1.003750pt}%
\definecolor{currentstroke}{rgb}{0.000000,0.000000,0.000000}%
\pgfsetstrokecolor{currentstroke}%
\pgfsetdash{}{0pt}%
\pgfpathmoveto{\pgfqpoint{2.264120in}{0.499444in}}%
\pgfpathlineto{\pgfqpoint{2.325506in}{0.499444in}}%
\pgfpathlineto{\pgfqpoint{2.325506in}{1.300982in}}%
\pgfpathlineto{\pgfqpoint{2.264120in}{1.300982in}}%
\pgfpathlineto{\pgfqpoint{2.264120in}{0.499444in}}%
\pgfpathclose%
\pgfusepath{stroke}%
\end{pgfscope}%
\begin{pgfscope}%
\pgfpathrectangle{\pgfqpoint{0.445556in}{0.499444in}}{\pgfqpoint{3.875000in}{1.155000in}}%
\pgfusepath{clip}%
\pgfsetbuttcap%
\pgfsetmiterjoin%
\pgfsetlinewidth{1.003750pt}%
\definecolor{currentstroke}{rgb}{0.000000,0.000000,0.000000}%
\pgfsetstrokecolor{currentstroke}%
\pgfsetdash{}{0pt}%
\pgfpathmoveto{\pgfqpoint{2.417585in}{0.499444in}}%
\pgfpathlineto{\pgfqpoint{2.478972in}{0.499444in}}%
\pgfpathlineto{\pgfqpoint{2.478972in}{0.915808in}}%
\pgfpathlineto{\pgfqpoint{2.417585in}{0.915808in}}%
\pgfpathlineto{\pgfqpoint{2.417585in}{0.499444in}}%
\pgfpathclose%
\pgfusepath{stroke}%
\end{pgfscope}%
\begin{pgfscope}%
\pgfpathrectangle{\pgfqpoint{0.445556in}{0.499444in}}{\pgfqpoint{3.875000in}{1.155000in}}%
\pgfusepath{clip}%
\pgfsetbuttcap%
\pgfsetmiterjoin%
\pgfsetlinewidth{1.003750pt}%
\definecolor{currentstroke}{rgb}{0.000000,0.000000,0.000000}%
\pgfsetstrokecolor{currentstroke}%
\pgfsetdash{}{0pt}%
\pgfpathmoveto{\pgfqpoint{2.571051in}{0.499444in}}%
\pgfpathlineto{\pgfqpoint{2.632437in}{0.499444in}}%
\pgfpathlineto{\pgfqpoint{2.632437in}{0.536111in}}%
\pgfpathlineto{\pgfqpoint{2.571051in}{0.536111in}}%
\pgfpathlineto{\pgfqpoint{2.571051in}{0.499444in}}%
\pgfpathclose%
\pgfusepath{stroke}%
\end{pgfscope}%
\begin{pgfscope}%
\pgfpathrectangle{\pgfqpoint{0.445556in}{0.499444in}}{\pgfqpoint{3.875000in}{1.155000in}}%
\pgfusepath{clip}%
\pgfsetbuttcap%
\pgfsetmiterjoin%
\pgfsetlinewidth{1.003750pt}%
\definecolor{currentstroke}{rgb}{0.000000,0.000000,0.000000}%
\pgfsetstrokecolor{currentstroke}%
\pgfsetdash{}{0pt}%
\pgfpathmoveto{\pgfqpoint{2.724516in}{0.499444in}}%
\pgfpathlineto{\pgfqpoint{2.785902in}{0.499444in}}%
\pgfpathlineto{\pgfqpoint{2.785902in}{0.499444in}}%
\pgfpathlineto{\pgfqpoint{2.724516in}{0.499444in}}%
\pgfpathlineto{\pgfqpoint{2.724516in}{0.499444in}}%
\pgfpathclose%
\pgfusepath{stroke}%
\end{pgfscope}%
\begin{pgfscope}%
\pgfpathrectangle{\pgfqpoint{0.445556in}{0.499444in}}{\pgfqpoint{3.875000in}{1.155000in}}%
\pgfusepath{clip}%
\pgfsetbuttcap%
\pgfsetmiterjoin%
\pgfsetlinewidth{1.003750pt}%
\definecolor{currentstroke}{rgb}{0.000000,0.000000,0.000000}%
\pgfsetstrokecolor{currentstroke}%
\pgfsetdash{}{0pt}%
\pgfpathmoveto{\pgfqpoint{2.877981in}{0.499444in}}%
\pgfpathlineto{\pgfqpoint{2.939368in}{0.499444in}}%
\pgfpathlineto{\pgfqpoint{2.939368in}{0.499444in}}%
\pgfpathlineto{\pgfqpoint{2.877981in}{0.499444in}}%
\pgfpathlineto{\pgfqpoint{2.877981in}{0.499444in}}%
\pgfpathclose%
\pgfusepath{stroke}%
\end{pgfscope}%
\begin{pgfscope}%
\pgfpathrectangle{\pgfqpoint{0.445556in}{0.499444in}}{\pgfqpoint{3.875000in}{1.155000in}}%
\pgfusepath{clip}%
\pgfsetbuttcap%
\pgfsetmiterjoin%
\pgfsetlinewidth{1.003750pt}%
\definecolor{currentstroke}{rgb}{0.000000,0.000000,0.000000}%
\pgfsetstrokecolor{currentstroke}%
\pgfsetdash{}{0pt}%
\pgfpathmoveto{\pgfqpoint{3.031447in}{0.499444in}}%
\pgfpathlineto{\pgfqpoint{3.092833in}{0.499444in}}%
\pgfpathlineto{\pgfqpoint{3.092833in}{0.499444in}}%
\pgfpathlineto{\pgfqpoint{3.031447in}{0.499444in}}%
\pgfpathlineto{\pgfqpoint{3.031447in}{0.499444in}}%
\pgfpathclose%
\pgfusepath{stroke}%
\end{pgfscope}%
\begin{pgfscope}%
\pgfpathrectangle{\pgfqpoint{0.445556in}{0.499444in}}{\pgfqpoint{3.875000in}{1.155000in}}%
\pgfusepath{clip}%
\pgfsetbuttcap%
\pgfsetmiterjoin%
\pgfsetlinewidth{1.003750pt}%
\definecolor{currentstroke}{rgb}{0.000000,0.000000,0.000000}%
\pgfsetstrokecolor{currentstroke}%
\pgfsetdash{}{0pt}%
\pgfpathmoveto{\pgfqpoint{3.184912in}{0.499444in}}%
\pgfpathlineto{\pgfqpoint{3.246298in}{0.499444in}}%
\pgfpathlineto{\pgfqpoint{3.246298in}{0.499444in}}%
\pgfpathlineto{\pgfqpoint{3.184912in}{0.499444in}}%
\pgfpathlineto{\pgfqpoint{3.184912in}{0.499444in}}%
\pgfpathclose%
\pgfusepath{stroke}%
\end{pgfscope}%
\begin{pgfscope}%
\pgfpathrectangle{\pgfqpoint{0.445556in}{0.499444in}}{\pgfqpoint{3.875000in}{1.155000in}}%
\pgfusepath{clip}%
\pgfsetbuttcap%
\pgfsetmiterjoin%
\pgfsetlinewidth{1.003750pt}%
\definecolor{currentstroke}{rgb}{0.000000,0.000000,0.000000}%
\pgfsetstrokecolor{currentstroke}%
\pgfsetdash{}{0pt}%
\pgfpathmoveto{\pgfqpoint{3.338377in}{0.499444in}}%
\pgfpathlineto{\pgfqpoint{3.399764in}{0.499444in}}%
\pgfpathlineto{\pgfqpoint{3.399764in}{0.499444in}}%
\pgfpathlineto{\pgfqpoint{3.338377in}{0.499444in}}%
\pgfpathlineto{\pgfqpoint{3.338377in}{0.499444in}}%
\pgfpathclose%
\pgfusepath{stroke}%
\end{pgfscope}%
\begin{pgfscope}%
\pgfpathrectangle{\pgfqpoint{0.445556in}{0.499444in}}{\pgfqpoint{3.875000in}{1.155000in}}%
\pgfusepath{clip}%
\pgfsetbuttcap%
\pgfsetmiterjoin%
\pgfsetlinewidth{1.003750pt}%
\definecolor{currentstroke}{rgb}{0.000000,0.000000,0.000000}%
\pgfsetstrokecolor{currentstroke}%
\pgfsetdash{}{0pt}%
\pgfpathmoveto{\pgfqpoint{3.491843in}{0.499444in}}%
\pgfpathlineto{\pgfqpoint{3.553229in}{0.499444in}}%
\pgfpathlineto{\pgfqpoint{3.553229in}{0.499444in}}%
\pgfpathlineto{\pgfqpoint{3.491843in}{0.499444in}}%
\pgfpathlineto{\pgfqpoint{3.491843in}{0.499444in}}%
\pgfpathclose%
\pgfusepath{stroke}%
\end{pgfscope}%
\begin{pgfscope}%
\pgfpathrectangle{\pgfqpoint{0.445556in}{0.499444in}}{\pgfqpoint{3.875000in}{1.155000in}}%
\pgfusepath{clip}%
\pgfsetbuttcap%
\pgfsetmiterjoin%
\pgfsetlinewidth{1.003750pt}%
\definecolor{currentstroke}{rgb}{0.000000,0.000000,0.000000}%
\pgfsetstrokecolor{currentstroke}%
\pgfsetdash{}{0pt}%
\pgfpathmoveto{\pgfqpoint{3.645308in}{0.499444in}}%
\pgfpathlineto{\pgfqpoint{3.706694in}{0.499444in}}%
\pgfpathlineto{\pgfqpoint{3.706694in}{0.499444in}}%
\pgfpathlineto{\pgfqpoint{3.645308in}{0.499444in}}%
\pgfpathlineto{\pgfqpoint{3.645308in}{0.499444in}}%
\pgfpathclose%
\pgfusepath{stroke}%
\end{pgfscope}%
\begin{pgfscope}%
\pgfpathrectangle{\pgfqpoint{0.445556in}{0.499444in}}{\pgfqpoint{3.875000in}{1.155000in}}%
\pgfusepath{clip}%
\pgfsetbuttcap%
\pgfsetmiterjoin%
\pgfsetlinewidth{1.003750pt}%
\definecolor{currentstroke}{rgb}{0.000000,0.000000,0.000000}%
\pgfsetstrokecolor{currentstroke}%
\pgfsetdash{}{0pt}%
\pgfpathmoveto{\pgfqpoint{3.798774in}{0.499444in}}%
\pgfpathlineto{\pgfqpoint{3.860160in}{0.499444in}}%
\pgfpathlineto{\pgfqpoint{3.860160in}{0.499444in}}%
\pgfpathlineto{\pgfqpoint{3.798774in}{0.499444in}}%
\pgfpathlineto{\pgfqpoint{3.798774in}{0.499444in}}%
\pgfpathclose%
\pgfusepath{stroke}%
\end{pgfscope}%
\begin{pgfscope}%
\pgfpathrectangle{\pgfqpoint{0.445556in}{0.499444in}}{\pgfqpoint{3.875000in}{1.155000in}}%
\pgfusepath{clip}%
\pgfsetbuttcap%
\pgfsetmiterjoin%
\pgfsetlinewidth{1.003750pt}%
\definecolor{currentstroke}{rgb}{0.000000,0.000000,0.000000}%
\pgfsetstrokecolor{currentstroke}%
\pgfsetdash{}{0pt}%
\pgfpathmoveto{\pgfqpoint{3.952239in}{0.499444in}}%
\pgfpathlineto{\pgfqpoint{4.013625in}{0.499444in}}%
\pgfpathlineto{\pgfqpoint{4.013625in}{0.499444in}}%
\pgfpathlineto{\pgfqpoint{3.952239in}{0.499444in}}%
\pgfpathlineto{\pgfqpoint{3.952239in}{0.499444in}}%
\pgfpathclose%
\pgfusepath{stroke}%
\end{pgfscope}%
\begin{pgfscope}%
\pgfpathrectangle{\pgfqpoint{0.445556in}{0.499444in}}{\pgfqpoint{3.875000in}{1.155000in}}%
\pgfusepath{clip}%
\pgfsetbuttcap%
\pgfsetmiterjoin%
\pgfsetlinewidth{1.003750pt}%
\definecolor{currentstroke}{rgb}{0.000000,0.000000,0.000000}%
\pgfsetstrokecolor{currentstroke}%
\pgfsetdash{}{0pt}%
\pgfpathmoveto{\pgfqpoint{4.105704in}{0.499444in}}%
\pgfpathlineto{\pgfqpoint{4.167090in}{0.499444in}}%
\pgfpathlineto{\pgfqpoint{4.167090in}{0.499444in}}%
\pgfpathlineto{\pgfqpoint{4.105704in}{0.499444in}}%
\pgfpathlineto{\pgfqpoint{4.105704in}{0.499444in}}%
\pgfpathclose%
\pgfusepath{stroke}%
\end{pgfscope}%
\begin{pgfscope}%
\pgfpathrectangle{\pgfqpoint{0.445556in}{0.499444in}}{\pgfqpoint{3.875000in}{1.155000in}}%
\pgfusepath{clip}%
\pgfsetbuttcap%
\pgfsetmiterjoin%
\definecolor{currentfill}{rgb}{0.000000,0.000000,0.000000}%
\pgfsetfillcolor{currentfill}%
\pgfsetlinewidth{0.000000pt}%
\definecolor{currentstroke}{rgb}{0.000000,0.000000,0.000000}%
\pgfsetstrokecolor{currentstroke}%
\pgfsetstrokeopacity{0.000000}%
\pgfsetdash{}{0pt}%
\pgfpathmoveto{\pgfqpoint{0.483922in}{0.499444in}}%
\pgfpathlineto{\pgfqpoint{0.545308in}{0.499444in}}%
\pgfpathlineto{\pgfqpoint{0.545308in}{0.505682in}}%
\pgfpathlineto{\pgfqpoint{0.483922in}{0.505682in}}%
\pgfpathlineto{\pgfqpoint{0.483922in}{0.499444in}}%
\pgfpathclose%
\pgfusepath{fill}%
\end{pgfscope}%
\begin{pgfscope}%
\pgfpathrectangle{\pgfqpoint{0.445556in}{0.499444in}}{\pgfqpoint{3.875000in}{1.155000in}}%
\pgfusepath{clip}%
\pgfsetbuttcap%
\pgfsetmiterjoin%
\definecolor{currentfill}{rgb}{0.000000,0.000000,0.000000}%
\pgfsetfillcolor{currentfill}%
\pgfsetlinewidth{0.000000pt}%
\definecolor{currentstroke}{rgb}{0.000000,0.000000,0.000000}%
\pgfsetstrokecolor{currentstroke}%
\pgfsetstrokeopacity{0.000000}%
\pgfsetdash{}{0pt}%
\pgfpathmoveto{\pgfqpoint{0.637387in}{0.499444in}}%
\pgfpathlineto{\pgfqpoint{0.698774in}{0.499444in}}%
\pgfpathlineto{\pgfqpoint{0.698774in}{0.513376in}}%
\pgfpathlineto{\pgfqpoint{0.637387in}{0.513376in}}%
\pgfpathlineto{\pgfqpoint{0.637387in}{0.499444in}}%
\pgfpathclose%
\pgfusepath{fill}%
\end{pgfscope}%
\begin{pgfscope}%
\pgfpathrectangle{\pgfqpoint{0.445556in}{0.499444in}}{\pgfqpoint{3.875000in}{1.155000in}}%
\pgfusepath{clip}%
\pgfsetbuttcap%
\pgfsetmiterjoin%
\definecolor{currentfill}{rgb}{0.000000,0.000000,0.000000}%
\pgfsetfillcolor{currentfill}%
\pgfsetlinewidth{0.000000pt}%
\definecolor{currentstroke}{rgb}{0.000000,0.000000,0.000000}%
\pgfsetstrokecolor{currentstroke}%
\pgfsetstrokeopacity{0.000000}%
\pgfsetdash{}{0pt}%
\pgfpathmoveto{\pgfqpoint{0.790853in}{0.499444in}}%
\pgfpathlineto{\pgfqpoint{0.852239in}{0.499444in}}%
\pgfpathlineto{\pgfqpoint{0.852239in}{0.524258in}}%
\pgfpathlineto{\pgfqpoint{0.790853in}{0.524258in}}%
\pgfpathlineto{\pgfqpoint{0.790853in}{0.499444in}}%
\pgfpathclose%
\pgfusepath{fill}%
\end{pgfscope}%
\begin{pgfscope}%
\pgfpathrectangle{\pgfqpoint{0.445556in}{0.499444in}}{\pgfqpoint{3.875000in}{1.155000in}}%
\pgfusepath{clip}%
\pgfsetbuttcap%
\pgfsetmiterjoin%
\definecolor{currentfill}{rgb}{0.000000,0.000000,0.000000}%
\pgfsetfillcolor{currentfill}%
\pgfsetlinewidth{0.000000pt}%
\definecolor{currentstroke}{rgb}{0.000000,0.000000,0.000000}%
\pgfsetstrokecolor{currentstroke}%
\pgfsetstrokeopacity{0.000000}%
\pgfsetdash{}{0pt}%
\pgfpathmoveto{\pgfqpoint{0.944318in}{0.499444in}}%
\pgfpathlineto{\pgfqpoint{1.005704in}{0.499444in}}%
\pgfpathlineto{\pgfqpoint{1.005704in}{0.538606in}}%
\pgfpathlineto{\pgfqpoint{0.944318in}{0.538606in}}%
\pgfpathlineto{\pgfqpoint{0.944318in}{0.499444in}}%
\pgfpathclose%
\pgfusepath{fill}%
\end{pgfscope}%
\begin{pgfscope}%
\pgfpathrectangle{\pgfqpoint{0.445556in}{0.499444in}}{\pgfqpoint{3.875000in}{1.155000in}}%
\pgfusepath{clip}%
\pgfsetbuttcap%
\pgfsetmiterjoin%
\definecolor{currentfill}{rgb}{0.000000,0.000000,0.000000}%
\pgfsetfillcolor{currentfill}%
\pgfsetlinewidth{0.000000pt}%
\definecolor{currentstroke}{rgb}{0.000000,0.000000,0.000000}%
\pgfsetstrokecolor{currentstroke}%
\pgfsetstrokeopacity{0.000000}%
\pgfsetdash{}{0pt}%
\pgfpathmoveto{\pgfqpoint{1.097783in}{0.499444in}}%
\pgfpathlineto{\pgfqpoint{1.159170in}{0.499444in}}%
\pgfpathlineto{\pgfqpoint{1.159170in}{0.549558in}}%
\pgfpathlineto{\pgfqpoint{1.097783in}{0.549558in}}%
\pgfpathlineto{\pgfqpoint{1.097783in}{0.499444in}}%
\pgfpathclose%
\pgfusepath{fill}%
\end{pgfscope}%
\begin{pgfscope}%
\pgfpathrectangle{\pgfqpoint{0.445556in}{0.499444in}}{\pgfqpoint{3.875000in}{1.155000in}}%
\pgfusepath{clip}%
\pgfsetbuttcap%
\pgfsetmiterjoin%
\definecolor{currentfill}{rgb}{0.000000,0.000000,0.000000}%
\pgfsetfillcolor{currentfill}%
\pgfsetlinewidth{0.000000pt}%
\definecolor{currentstroke}{rgb}{0.000000,0.000000,0.000000}%
\pgfsetstrokecolor{currentstroke}%
\pgfsetstrokeopacity{0.000000}%
\pgfsetdash{}{0pt}%
\pgfpathmoveto{\pgfqpoint{1.251249in}{0.499444in}}%
\pgfpathlineto{\pgfqpoint{1.312635in}{0.499444in}}%
\pgfpathlineto{\pgfqpoint{1.312635in}{0.572847in}}%
\pgfpathlineto{\pgfqpoint{1.251249in}{0.572847in}}%
\pgfpathlineto{\pgfqpoint{1.251249in}{0.499444in}}%
\pgfpathclose%
\pgfusepath{fill}%
\end{pgfscope}%
\begin{pgfscope}%
\pgfpathrectangle{\pgfqpoint{0.445556in}{0.499444in}}{\pgfqpoint{3.875000in}{1.155000in}}%
\pgfusepath{clip}%
\pgfsetbuttcap%
\pgfsetmiterjoin%
\definecolor{currentfill}{rgb}{0.000000,0.000000,0.000000}%
\pgfsetfillcolor{currentfill}%
\pgfsetlinewidth{0.000000pt}%
\definecolor{currentstroke}{rgb}{0.000000,0.000000,0.000000}%
\pgfsetstrokecolor{currentstroke}%
\pgfsetstrokeopacity{0.000000}%
\pgfsetdash{}{0pt}%
\pgfpathmoveto{\pgfqpoint{1.404714in}{0.499444in}}%
\pgfpathlineto{\pgfqpoint{1.466100in}{0.499444in}}%
\pgfpathlineto{\pgfqpoint{1.466100in}{0.591977in}}%
\pgfpathlineto{\pgfqpoint{1.404714in}{0.591977in}}%
\pgfpathlineto{\pgfqpoint{1.404714in}{0.499444in}}%
\pgfpathclose%
\pgfusepath{fill}%
\end{pgfscope}%
\begin{pgfscope}%
\pgfpathrectangle{\pgfqpoint{0.445556in}{0.499444in}}{\pgfqpoint{3.875000in}{1.155000in}}%
\pgfusepath{clip}%
\pgfsetbuttcap%
\pgfsetmiterjoin%
\definecolor{currentfill}{rgb}{0.000000,0.000000,0.000000}%
\pgfsetfillcolor{currentfill}%
\pgfsetlinewidth{0.000000pt}%
\definecolor{currentstroke}{rgb}{0.000000,0.000000,0.000000}%
\pgfsetstrokecolor{currentstroke}%
\pgfsetstrokeopacity{0.000000}%
\pgfsetdash{}{0pt}%
\pgfpathmoveto{\pgfqpoint{1.558179in}{0.499444in}}%
\pgfpathlineto{\pgfqpoint{1.619566in}{0.499444in}}%
\pgfpathlineto{\pgfqpoint{1.619566in}{0.618663in}}%
\pgfpathlineto{\pgfqpoint{1.558179in}{0.618663in}}%
\pgfpathlineto{\pgfqpoint{1.558179in}{0.499444in}}%
\pgfpathclose%
\pgfusepath{fill}%
\end{pgfscope}%
\begin{pgfscope}%
\pgfpathrectangle{\pgfqpoint{0.445556in}{0.499444in}}{\pgfqpoint{3.875000in}{1.155000in}}%
\pgfusepath{clip}%
\pgfsetbuttcap%
\pgfsetmiterjoin%
\definecolor{currentfill}{rgb}{0.000000,0.000000,0.000000}%
\pgfsetfillcolor{currentfill}%
\pgfsetlinewidth{0.000000pt}%
\definecolor{currentstroke}{rgb}{0.000000,0.000000,0.000000}%
\pgfsetstrokecolor{currentstroke}%
\pgfsetstrokeopacity{0.000000}%
\pgfsetdash{}{0pt}%
\pgfpathmoveto{\pgfqpoint{1.711645in}{0.499444in}}%
\pgfpathlineto{\pgfqpoint{1.773031in}{0.499444in}}%
\pgfpathlineto{\pgfqpoint{1.773031in}{0.659142in}}%
\pgfpathlineto{\pgfqpoint{1.711645in}{0.659142in}}%
\pgfpathlineto{\pgfqpoint{1.711645in}{0.499444in}}%
\pgfpathclose%
\pgfusepath{fill}%
\end{pgfscope}%
\begin{pgfscope}%
\pgfpathrectangle{\pgfqpoint{0.445556in}{0.499444in}}{\pgfqpoint{3.875000in}{1.155000in}}%
\pgfusepath{clip}%
\pgfsetbuttcap%
\pgfsetmiterjoin%
\definecolor{currentfill}{rgb}{0.000000,0.000000,0.000000}%
\pgfsetfillcolor{currentfill}%
\pgfsetlinewidth{0.000000pt}%
\definecolor{currentstroke}{rgb}{0.000000,0.000000,0.000000}%
\pgfsetstrokecolor{currentstroke}%
\pgfsetstrokeopacity{0.000000}%
\pgfsetdash{}{0pt}%
\pgfpathmoveto{\pgfqpoint{1.865110in}{0.499444in}}%
\pgfpathlineto{\pgfqpoint{1.926496in}{0.499444in}}%
\pgfpathlineto{\pgfqpoint{1.926496in}{0.695808in}}%
\pgfpathlineto{\pgfqpoint{1.865110in}{0.695808in}}%
\pgfpathlineto{\pgfqpoint{1.865110in}{0.499444in}}%
\pgfpathclose%
\pgfusepath{fill}%
\end{pgfscope}%
\begin{pgfscope}%
\pgfpathrectangle{\pgfqpoint{0.445556in}{0.499444in}}{\pgfqpoint{3.875000in}{1.155000in}}%
\pgfusepath{clip}%
\pgfsetbuttcap%
\pgfsetmiterjoin%
\definecolor{currentfill}{rgb}{0.000000,0.000000,0.000000}%
\pgfsetfillcolor{currentfill}%
\pgfsetlinewidth{0.000000pt}%
\definecolor{currentstroke}{rgb}{0.000000,0.000000,0.000000}%
\pgfsetstrokecolor{currentstroke}%
\pgfsetstrokeopacity{0.000000}%
\pgfsetdash{}{0pt}%
\pgfpathmoveto{\pgfqpoint{2.018575in}{0.499444in}}%
\pgfpathlineto{\pgfqpoint{2.079962in}{0.499444in}}%
\pgfpathlineto{\pgfqpoint{2.079962in}{0.756249in}}%
\pgfpathlineto{\pgfqpoint{2.018575in}{0.756249in}}%
\pgfpathlineto{\pgfqpoint{2.018575in}{0.499444in}}%
\pgfpathclose%
\pgfusepath{fill}%
\end{pgfscope}%
\begin{pgfscope}%
\pgfpathrectangle{\pgfqpoint{0.445556in}{0.499444in}}{\pgfqpoint{3.875000in}{1.155000in}}%
\pgfusepath{clip}%
\pgfsetbuttcap%
\pgfsetmiterjoin%
\definecolor{currentfill}{rgb}{0.000000,0.000000,0.000000}%
\pgfsetfillcolor{currentfill}%
\pgfsetlinewidth{0.000000pt}%
\definecolor{currentstroke}{rgb}{0.000000,0.000000,0.000000}%
\pgfsetstrokecolor{currentstroke}%
\pgfsetstrokeopacity{0.000000}%
\pgfsetdash{}{0pt}%
\pgfpathmoveto{\pgfqpoint{2.172041in}{0.499444in}}%
\pgfpathlineto{\pgfqpoint{2.233427in}{0.499444in}}%
\pgfpathlineto{\pgfqpoint{2.233427in}{0.834989in}}%
\pgfpathlineto{\pgfqpoint{2.172041in}{0.834989in}}%
\pgfpathlineto{\pgfqpoint{2.172041in}{0.499444in}}%
\pgfpathclose%
\pgfusepath{fill}%
\end{pgfscope}%
\begin{pgfscope}%
\pgfpathrectangle{\pgfqpoint{0.445556in}{0.499444in}}{\pgfqpoint{3.875000in}{1.155000in}}%
\pgfusepath{clip}%
\pgfsetbuttcap%
\pgfsetmiterjoin%
\definecolor{currentfill}{rgb}{0.000000,0.000000,0.000000}%
\pgfsetfillcolor{currentfill}%
\pgfsetlinewidth{0.000000pt}%
\definecolor{currentstroke}{rgb}{0.000000,0.000000,0.000000}%
\pgfsetstrokecolor{currentstroke}%
\pgfsetstrokeopacity{0.000000}%
\pgfsetdash{}{0pt}%
\pgfpathmoveto{\pgfqpoint{2.325506in}{0.499444in}}%
\pgfpathlineto{\pgfqpoint{2.386892in}{0.499444in}}%
\pgfpathlineto{\pgfqpoint{2.386892in}{0.913036in}}%
\pgfpathlineto{\pgfqpoint{2.325506in}{0.913036in}}%
\pgfpathlineto{\pgfqpoint{2.325506in}{0.499444in}}%
\pgfpathclose%
\pgfusepath{fill}%
\end{pgfscope}%
\begin{pgfscope}%
\pgfpathrectangle{\pgfqpoint{0.445556in}{0.499444in}}{\pgfqpoint{3.875000in}{1.155000in}}%
\pgfusepath{clip}%
\pgfsetbuttcap%
\pgfsetmiterjoin%
\definecolor{currentfill}{rgb}{0.000000,0.000000,0.000000}%
\pgfsetfillcolor{currentfill}%
\pgfsetlinewidth{0.000000pt}%
\definecolor{currentstroke}{rgb}{0.000000,0.000000,0.000000}%
\pgfsetstrokecolor{currentstroke}%
\pgfsetstrokeopacity{0.000000}%
\pgfsetdash{}{0pt}%
\pgfpathmoveto{\pgfqpoint{2.478972in}{0.499444in}}%
\pgfpathlineto{\pgfqpoint{2.540358in}{0.499444in}}%
\pgfpathlineto{\pgfqpoint{2.540358in}{0.950534in}}%
\pgfpathlineto{\pgfqpoint{2.478972in}{0.950534in}}%
\pgfpathlineto{\pgfqpoint{2.478972in}{0.499444in}}%
\pgfpathclose%
\pgfusepath{fill}%
\end{pgfscope}%
\begin{pgfscope}%
\pgfpathrectangle{\pgfqpoint{0.445556in}{0.499444in}}{\pgfqpoint{3.875000in}{1.155000in}}%
\pgfusepath{clip}%
\pgfsetbuttcap%
\pgfsetmiterjoin%
\definecolor{currentfill}{rgb}{0.000000,0.000000,0.000000}%
\pgfsetfillcolor{currentfill}%
\pgfsetlinewidth{0.000000pt}%
\definecolor{currentstroke}{rgb}{0.000000,0.000000,0.000000}%
\pgfsetstrokecolor{currentstroke}%
\pgfsetstrokeopacity{0.000000}%
\pgfsetdash{}{0pt}%
\pgfpathmoveto{\pgfqpoint{2.632437in}{0.499444in}}%
\pgfpathlineto{\pgfqpoint{2.693823in}{0.499444in}}%
\pgfpathlineto{\pgfqpoint{2.693823in}{0.610969in}}%
\pgfpathlineto{\pgfqpoint{2.632437in}{0.610969in}}%
\pgfpathlineto{\pgfqpoint{2.632437in}{0.499444in}}%
\pgfpathclose%
\pgfusepath{fill}%
\end{pgfscope}%
\begin{pgfscope}%
\pgfpathrectangle{\pgfqpoint{0.445556in}{0.499444in}}{\pgfqpoint{3.875000in}{1.155000in}}%
\pgfusepath{clip}%
\pgfsetbuttcap%
\pgfsetmiterjoin%
\definecolor{currentfill}{rgb}{0.000000,0.000000,0.000000}%
\pgfsetfillcolor{currentfill}%
\pgfsetlinewidth{0.000000pt}%
\definecolor{currentstroke}{rgb}{0.000000,0.000000,0.000000}%
\pgfsetstrokecolor{currentstroke}%
\pgfsetstrokeopacity{0.000000}%
\pgfsetdash{}{0pt}%
\pgfpathmoveto{\pgfqpoint{2.785902in}{0.499444in}}%
\pgfpathlineto{\pgfqpoint{2.847288in}{0.499444in}}%
\pgfpathlineto{\pgfqpoint{2.847288in}{0.499444in}}%
\pgfpathlineto{\pgfqpoint{2.785902in}{0.499444in}}%
\pgfpathlineto{\pgfqpoint{2.785902in}{0.499444in}}%
\pgfpathclose%
\pgfusepath{fill}%
\end{pgfscope}%
\begin{pgfscope}%
\pgfpathrectangle{\pgfqpoint{0.445556in}{0.499444in}}{\pgfqpoint{3.875000in}{1.155000in}}%
\pgfusepath{clip}%
\pgfsetbuttcap%
\pgfsetmiterjoin%
\definecolor{currentfill}{rgb}{0.000000,0.000000,0.000000}%
\pgfsetfillcolor{currentfill}%
\pgfsetlinewidth{0.000000pt}%
\definecolor{currentstroke}{rgb}{0.000000,0.000000,0.000000}%
\pgfsetstrokecolor{currentstroke}%
\pgfsetstrokeopacity{0.000000}%
\pgfsetdash{}{0pt}%
\pgfpathmoveto{\pgfqpoint{2.939368in}{0.499444in}}%
\pgfpathlineto{\pgfqpoint{3.000754in}{0.499444in}}%
\pgfpathlineto{\pgfqpoint{3.000754in}{0.499444in}}%
\pgfpathlineto{\pgfqpoint{2.939368in}{0.499444in}}%
\pgfpathlineto{\pgfqpoint{2.939368in}{0.499444in}}%
\pgfpathclose%
\pgfusepath{fill}%
\end{pgfscope}%
\begin{pgfscope}%
\pgfpathrectangle{\pgfqpoint{0.445556in}{0.499444in}}{\pgfqpoint{3.875000in}{1.155000in}}%
\pgfusepath{clip}%
\pgfsetbuttcap%
\pgfsetmiterjoin%
\definecolor{currentfill}{rgb}{0.000000,0.000000,0.000000}%
\pgfsetfillcolor{currentfill}%
\pgfsetlinewidth{0.000000pt}%
\definecolor{currentstroke}{rgb}{0.000000,0.000000,0.000000}%
\pgfsetstrokecolor{currentstroke}%
\pgfsetstrokeopacity{0.000000}%
\pgfsetdash{}{0pt}%
\pgfpathmoveto{\pgfqpoint{3.092833in}{0.499444in}}%
\pgfpathlineto{\pgfqpoint{3.154219in}{0.499444in}}%
\pgfpathlineto{\pgfqpoint{3.154219in}{0.499444in}}%
\pgfpathlineto{\pgfqpoint{3.092833in}{0.499444in}}%
\pgfpathlineto{\pgfqpoint{3.092833in}{0.499444in}}%
\pgfpathclose%
\pgfusepath{fill}%
\end{pgfscope}%
\begin{pgfscope}%
\pgfpathrectangle{\pgfqpoint{0.445556in}{0.499444in}}{\pgfqpoint{3.875000in}{1.155000in}}%
\pgfusepath{clip}%
\pgfsetbuttcap%
\pgfsetmiterjoin%
\definecolor{currentfill}{rgb}{0.000000,0.000000,0.000000}%
\pgfsetfillcolor{currentfill}%
\pgfsetlinewidth{0.000000pt}%
\definecolor{currentstroke}{rgb}{0.000000,0.000000,0.000000}%
\pgfsetstrokecolor{currentstroke}%
\pgfsetstrokeopacity{0.000000}%
\pgfsetdash{}{0pt}%
\pgfpathmoveto{\pgfqpoint{3.246298in}{0.499444in}}%
\pgfpathlineto{\pgfqpoint{3.307684in}{0.499444in}}%
\pgfpathlineto{\pgfqpoint{3.307684in}{0.499444in}}%
\pgfpathlineto{\pgfqpoint{3.246298in}{0.499444in}}%
\pgfpathlineto{\pgfqpoint{3.246298in}{0.499444in}}%
\pgfpathclose%
\pgfusepath{fill}%
\end{pgfscope}%
\begin{pgfscope}%
\pgfpathrectangle{\pgfqpoint{0.445556in}{0.499444in}}{\pgfqpoint{3.875000in}{1.155000in}}%
\pgfusepath{clip}%
\pgfsetbuttcap%
\pgfsetmiterjoin%
\definecolor{currentfill}{rgb}{0.000000,0.000000,0.000000}%
\pgfsetfillcolor{currentfill}%
\pgfsetlinewidth{0.000000pt}%
\definecolor{currentstroke}{rgb}{0.000000,0.000000,0.000000}%
\pgfsetstrokecolor{currentstroke}%
\pgfsetstrokeopacity{0.000000}%
\pgfsetdash{}{0pt}%
\pgfpathmoveto{\pgfqpoint{3.399764in}{0.499444in}}%
\pgfpathlineto{\pgfqpoint{3.461150in}{0.499444in}}%
\pgfpathlineto{\pgfqpoint{3.461150in}{0.499444in}}%
\pgfpathlineto{\pgfqpoint{3.399764in}{0.499444in}}%
\pgfpathlineto{\pgfqpoint{3.399764in}{0.499444in}}%
\pgfpathclose%
\pgfusepath{fill}%
\end{pgfscope}%
\begin{pgfscope}%
\pgfpathrectangle{\pgfqpoint{0.445556in}{0.499444in}}{\pgfqpoint{3.875000in}{1.155000in}}%
\pgfusepath{clip}%
\pgfsetbuttcap%
\pgfsetmiterjoin%
\definecolor{currentfill}{rgb}{0.000000,0.000000,0.000000}%
\pgfsetfillcolor{currentfill}%
\pgfsetlinewidth{0.000000pt}%
\definecolor{currentstroke}{rgb}{0.000000,0.000000,0.000000}%
\pgfsetstrokecolor{currentstroke}%
\pgfsetstrokeopacity{0.000000}%
\pgfsetdash{}{0pt}%
\pgfpathmoveto{\pgfqpoint{3.553229in}{0.499444in}}%
\pgfpathlineto{\pgfqpoint{3.614615in}{0.499444in}}%
\pgfpathlineto{\pgfqpoint{3.614615in}{0.499444in}}%
\pgfpathlineto{\pgfqpoint{3.553229in}{0.499444in}}%
\pgfpathlineto{\pgfqpoint{3.553229in}{0.499444in}}%
\pgfpathclose%
\pgfusepath{fill}%
\end{pgfscope}%
\begin{pgfscope}%
\pgfpathrectangle{\pgfqpoint{0.445556in}{0.499444in}}{\pgfqpoint{3.875000in}{1.155000in}}%
\pgfusepath{clip}%
\pgfsetbuttcap%
\pgfsetmiterjoin%
\definecolor{currentfill}{rgb}{0.000000,0.000000,0.000000}%
\pgfsetfillcolor{currentfill}%
\pgfsetlinewidth{0.000000pt}%
\definecolor{currentstroke}{rgb}{0.000000,0.000000,0.000000}%
\pgfsetstrokecolor{currentstroke}%
\pgfsetstrokeopacity{0.000000}%
\pgfsetdash{}{0pt}%
\pgfpathmoveto{\pgfqpoint{3.706694in}{0.499444in}}%
\pgfpathlineto{\pgfqpoint{3.768080in}{0.499444in}}%
\pgfpathlineto{\pgfqpoint{3.768080in}{0.499444in}}%
\pgfpathlineto{\pgfqpoint{3.706694in}{0.499444in}}%
\pgfpathlineto{\pgfqpoint{3.706694in}{0.499444in}}%
\pgfpathclose%
\pgfusepath{fill}%
\end{pgfscope}%
\begin{pgfscope}%
\pgfpathrectangle{\pgfqpoint{0.445556in}{0.499444in}}{\pgfqpoint{3.875000in}{1.155000in}}%
\pgfusepath{clip}%
\pgfsetbuttcap%
\pgfsetmiterjoin%
\definecolor{currentfill}{rgb}{0.000000,0.000000,0.000000}%
\pgfsetfillcolor{currentfill}%
\pgfsetlinewidth{0.000000pt}%
\definecolor{currentstroke}{rgb}{0.000000,0.000000,0.000000}%
\pgfsetstrokecolor{currentstroke}%
\pgfsetstrokeopacity{0.000000}%
\pgfsetdash{}{0pt}%
\pgfpathmoveto{\pgfqpoint{3.860160in}{0.499444in}}%
\pgfpathlineto{\pgfqpoint{3.921546in}{0.499444in}}%
\pgfpathlineto{\pgfqpoint{3.921546in}{0.499444in}}%
\pgfpathlineto{\pgfqpoint{3.860160in}{0.499444in}}%
\pgfpathlineto{\pgfqpoint{3.860160in}{0.499444in}}%
\pgfpathclose%
\pgfusepath{fill}%
\end{pgfscope}%
\begin{pgfscope}%
\pgfpathrectangle{\pgfqpoint{0.445556in}{0.499444in}}{\pgfqpoint{3.875000in}{1.155000in}}%
\pgfusepath{clip}%
\pgfsetbuttcap%
\pgfsetmiterjoin%
\definecolor{currentfill}{rgb}{0.000000,0.000000,0.000000}%
\pgfsetfillcolor{currentfill}%
\pgfsetlinewidth{0.000000pt}%
\definecolor{currentstroke}{rgb}{0.000000,0.000000,0.000000}%
\pgfsetstrokecolor{currentstroke}%
\pgfsetstrokeopacity{0.000000}%
\pgfsetdash{}{0pt}%
\pgfpathmoveto{\pgfqpoint{4.013625in}{0.499444in}}%
\pgfpathlineto{\pgfqpoint{4.075011in}{0.499444in}}%
\pgfpathlineto{\pgfqpoint{4.075011in}{0.499444in}}%
\pgfpathlineto{\pgfqpoint{4.013625in}{0.499444in}}%
\pgfpathlineto{\pgfqpoint{4.013625in}{0.499444in}}%
\pgfpathclose%
\pgfusepath{fill}%
\end{pgfscope}%
\begin{pgfscope}%
\pgfpathrectangle{\pgfqpoint{0.445556in}{0.499444in}}{\pgfqpoint{3.875000in}{1.155000in}}%
\pgfusepath{clip}%
\pgfsetbuttcap%
\pgfsetmiterjoin%
\definecolor{currentfill}{rgb}{0.000000,0.000000,0.000000}%
\pgfsetfillcolor{currentfill}%
\pgfsetlinewidth{0.000000pt}%
\definecolor{currentstroke}{rgb}{0.000000,0.000000,0.000000}%
\pgfsetstrokecolor{currentstroke}%
\pgfsetstrokeopacity{0.000000}%
\pgfsetdash{}{0pt}%
\pgfpathmoveto{\pgfqpoint{4.167090in}{0.499444in}}%
\pgfpathlineto{\pgfqpoint{4.228476in}{0.499444in}}%
\pgfpathlineto{\pgfqpoint{4.228476in}{0.499444in}}%
\pgfpathlineto{\pgfqpoint{4.167090in}{0.499444in}}%
\pgfpathlineto{\pgfqpoint{4.167090in}{0.499444in}}%
\pgfpathclose%
\pgfusepath{fill}%
\end{pgfscope}%
\begin{pgfscope}%
\pgfsetbuttcap%
\pgfsetroundjoin%
\definecolor{currentfill}{rgb}{0.000000,0.000000,0.000000}%
\pgfsetfillcolor{currentfill}%
\pgfsetlinewidth{0.803000pt}%
\definecolor{currentstroke}{rgb}{0.000000,0.000000,0.000000}%
\pgfsetstrokecolor{currentstroke}%
\pgfsetdash{}{0pt}%
\pgfsys@defobject{currentmarker}{\pgfqpoint{0.000000in}{-0.048611in}}{\pgfqpoint{0.000000in}{0.000000in}}{%
\pgfpathmoveto{\pgfqpoint{0.000000in}{0.000000in}}%
\pgfpathlineto{\pgfqpoint{0.000000in}{-0.048611in}}%
\pgfusepath{stroke,fill}%
}%
\begin{pgfscope}%
\pgfsys@transformshift{0.483922in}{0.499444in}%
\pgfsys@useobject{currentmarker}{}%
\end{pgfscope}%
\end{pgfscope}%
\begin{pgfscope}%
\definecolor{textcolor}{rgb}{0.000000,0.000000,0.000000}%
\pgfsetstrokecolor{textcolor}%
\pgfsetfillcolor{textcolor}%
\pgftext[x=0.483922in,y=0.402222in,,top]{\color{textcolor}\rmfamily\fontsize{10.000000}{12.000000}\selectfont 0.0}%
\end{pgfscope}%
\begin{pgfscope}%
\pgfsetbuttcap%
\pgfsetroundjoin%
\definecolor{currentfill}{rgb}{0.000000,0.000000,0.000000}%
\pgfsetfillcolor{currentfill}%
\pgfsetlinewidth{0.803000pt}%
\definecolor{currentstroke}{rgb}{0.000000,0.000000,0.000000}%
\pgfsetstrokecolor{currentstroke}%
\pgfsetdash{}{0pt}%
\pgfsys@defobject{currentmarker}{\pgfqpoint{0.000000in}{-0.048611in}}{\pgfqpoint{0.000000in}{0.000000in}}{%
\pgfpathmoveto{\pgfqpoint{0.000000in}{0.000000in}}%
\pgfpathlineto{\pgfqpoint{0.000000in}{-0.048611in}}%
\pgfusepath{stroke,fill}%
}%
\begin{pgfscope}%
\pgfsys@transformshift{0.867585in}{0.499444in}%
\pgfsys@useobject{currentmarker}{}%
\end{pgfscope}%
\end{pgfscope}%
\begin{pgfscope}%
\definecolor{textcolor}{rgb}{0.000000,0.000000,0.000000}%
\pgfsetstrokecolor{textcolor}%
\pgfsetfillcolor{textcolor}%
\pgftext[x=0.867585in,y=0.402222in,,top]{\color{textcolor}\rmfamily\fontsize{10.000000}{12.000000}\selectfont 0.1}%
\end{pgfscope}%
\begin{pgfscope}%
\pgfsetbuttcap%
\pgfsetroundjoin%
\definecolor{currentfill}{rgb}{0.000000,0.000000,0.000000}%
\pgfsetfillcolor{currentfill}%
\pgfsetlinewidth{0.803000pt}%
\definecolor{currentstroke}{rgb}{0.000000,0.000000,0.000000}%
\pgfsetstrokecolor{currentstroke}%
\pgfsetdash{}{0pt}%
\pgfsys@defobject{currentmarker}{\pgfqpoint{0.000000in}{-0.048611in}}{\pgfqpoint{0.000000in}{0.000000in}}{%
\pgfpathmoveto{\pgfqpoint{0.000000in}{0.000000in}}%
\pgfpathlineto{\pgfqpoint{0.000000in}{-0.048611in}}%
\pgfusepath{stroke,fill}%
}%
\begin{pgfscope}%
\pgfsys@transformshift{1.251249in}{0.499444in}%
\pgfsys@useobject{currentmarker}{}%
\end{pgfscope}%
\end{pgfscope}%
\begin{pgfscope}%
\definecolor{textcolor}{rgb}{0.000000,0.000000,0.000000}%
\pgfsetstrokecolor{textcolor}%
\pgfsetfillcolor{textcolor}%
\pgftext[x=1.251249in,y=0.402222in,,top]{\color{textcolor}\rmfamily\fontsize{10.000000}{12.000000}\selectfont 0.2}%
\end{pgfscope}%
\begin{pgfscope}%
\pgfsetbuttcap%
\pgfsetroundjoin%
\definecolor{currentfill}{rgb}{0.000000,0.000000,0.000000}%
\pgfsetfillcolor{currentfill}%
\pgfsetlinewidth{0.803000pt}%
\definecolor{currentstroke}{rgb}{0.000000,0.000000,0.000000}%
\pgfsetstrokecolor{currentstroke}%
\pgfsetdash{}{0pt}%
\pgfsys@defobject{currentmarker}{\pgfqpoint{0.000000in}{-0.048611in}}{\pgfqpoint{0.000000in}{0.000000in}}{%
\pgfpathmoveto{\pgfqpoint{0.000000in}{0.000000in}}%
\pgfpathlineto{\pgfqpoint{0.000000in}{-0.048611in}}%
\pgfusepath{stroke,fill}%
}%
\begin{pgfscope}%
\pgfsys@transformshift{1.634912in}{0.499444in}%
\pgfsys@useobject{currentmarker}{}%
\end{pgfscope}%
\end{pgfscope}%
\begin{pgfscope}%
\definecolor{textcolor}{rgb}{0.000000,0.000000,0.000000}%
\pgfsetstrokecolor{textcolor}%
\pgfsetfillcolor{textcolor}%
\pgftext[x=1.634912in,y=0.402222in,,top]{\color{textcolor}\rmfamily\fontsize{10.000000}{12.000000}\selectfont 0.3}%
\end{pgfscope}%
\begin{pgfscope}%
\pgfsetbuttcap%
\pgfsetroundjoin%
\definecolor{currentfill}{rgb}{0.000000,0.000000,0.000000}%
\pgfsetfillcolor{currentfill}%
\pgfsetlinewidth{0.803000pt}%
\definecolor{currentstroke}{rgb}{0.000000,0.000000,0.000000}%
\pgfsetstrokecolor{currentstroke}%
\pgfsetdash{}{0pt}%
\pgfsys@defobject{currentmarker}{\pgfqpoint{0.000000in}{-0.048611in}}{\pgfqpoint{0.000000in}{0.000000in}}{%
\pgfpathmoveto{\pgfqpoint{0.000000in}{0.000000in}}%
\pgfpathlineto{\pgfqpoint{0.000000in}{-0.048611in}}%
\pgfusepath{stroke,fill}%
}%
\begin{pgfscope}%
\pgfsys@transformshift{2.018575in}{0.499444in}%
\pgfsys@useobject{currentmarker}{}%
\end{pgfscope}%
\end{pgfscope}%
\begin{pgfscope}%
\definecolor{textcolor}{rgb}{0.000000,0.000000,0.000000}%
\pgfsetstrokecolor{textcolor}%
\pgfsetfillcolor{textcolor}%
\pgftext[x=2.018575in,y=0.402222in,,top]{\color{textcolor}\rmfamily\fontsize{10.000000}{12.000000}\selectfont 0.4}%
\end{pgfscope}%
\begin{pgfscope}%
\pgfsetbuttcap%
\pgfsetroundjoin%
\definecolor{currentfill}{rgb}{0.000000,0.000000,0.000000}%
\pgfsetfillcolor{currentfill}%
\pgfsetlinewidth{0.803000pt}%
\definecolor{currentstroke}{rgb}{0.000000,0.000000,0.000000}%
\pgfsetstrokecolor{currentstroke}%
\pgfsetdash{}{0pt}%
\pgfsys@defobject{currentmarker}{\pgfqpoint{0.000000in}{-0.048611in}}{\pgfqpoint{0.000000in}{0.000000in}}{%
\pgfpathmoveto{\pgfqpoint{0.000000in}{0.000000in}}%
\pgfpathlineto{\pgfqpoint{0.000000in}{-0.048611in}}%
\pgfusepath{stroke,fill}%
}%
\begin{pgfscope}%
\pgfsys@transformshift{2.402239in}{0.499444in}%
\pgfsys@useobject{currentmarker}{}%
\end{pgfscope}%
\end{pgfscope}%
\begin{pgfscope}%
\definecolor{textcolor}{rgb}{0.000000,0.000000,0.000000}%
\pgfsetstrokecolor{textcolor}%
\pgfsetfillcolor{textcolor}%
\pgftext[x=2.402239in,y=0.402222in,,top]{\color{textcolor}\rmfamily\fontsize{10.000000}{12.000000}\selectfont 0.5}%
\end{pgfscope}%
\begin{pgfscope}%
\pgfsetbuttcap%
\pgfsetroundjoin%
\definecolor{currentfill}{rgb}{0.000000,0.000000,0.000000}%
\pgfsetfillcolor{currentfill}%
\pgfsetlinewidth{0.803000pt}%
\definecolor{currentstroke}{rgb}{0.000000,0.000000,0.000000}%
\pgfsetstrokecolor{currentstroke}%
\pgfsetdash{}{0pt}%
\pgfsys@defobject{currentmarker}{\pgfqpoint{0.000000in}{-0.048611in}}{\pgfqpoint{0.000000in}{0.000000in}}{%
\pgfpathmoveto{\pgfqpoint{0.000000in}{0.000000in}}%
\pgfpathlineto{\pgfqpoint{0.000000in}{-0.048611in}}%
\pgfusepath{stroke,fill}%
}%
\begin{pgfscope}%
\pgfsys@transformshift{2.785902in}{0.499444in}%
\pgfsys@useobject{currentmarker}{}%
\end{pgfscope}%
\end{pgfscope}%
\begin{pgfscope}%
\definecolor{textcolor}{rgb}{0.000000,0.000000,0.000000}%
\pgfsetstrokecolor{textcolor}%
\pgfsetfillcolor{textcolor}%
\pgftext[x=2.785902in,y=0.402222in,,top]{\color{textcolor}\rmfamily\fontsize{10.000000}{12.000000}\selectfont 0.6}%
\end{pgfscope}%
\begin{pgfscope}%
\pgfsetbuttcap%
\pgfsetroundjoin%
\definecolor{currentfill}{rgb}{0.000000,0.000000,0.000000}%
\pgfsetfillcolor{currentfill}%
\pgfsetlinewidth{0.803000pt}%
\definecolor{currentstroke}{rgb}{0.000000,0.000000,0.000000}%
\pgfsetstrokecolor{currentstroke}%
\pgfsetdash{}{0pt}%
\pgfsys@defobject{currentmarker}{\pgfqpoint{0.000000in}{-0.048611in}}{\pgfqpoint{0.000000in}{0.000000in}}{%
\pgfpathmoveto{\pgfqpoint{0.000000in}{0.000000in}}%
\pgfpathlineto{\pgfqpoint{0.000000in}{-0.048611in}}%
\pgfusepath{stroke,fill}%
}%
\begin{pgfscope}%
\pgfsys@transformshift{3.169566in}{0.499444in}%
\pgfsys@useobject{currentmarker}{}%
\end{pgfscope}%
\end{pgfscope}%
\begin{pgfscope}%
\definecolor{textcolor}{rgb}{0.000000,0.000000,0.000000}%
\pgfsetstrokecolor{textcolor}%
\pgfsetfillcolor{textcolor}%
\pgftext[x=3.169566in,y=0.402222in,,top]{\color{textcolor}\rmfamily\fontsize{10.000000}{12.000000}\selectfont 0.7}%
\end{pgfscope}%
\begin{pgfscope}%
\pgfsetbuttcap%
\pgfsetroundjoin%
\definecolor{currentfill}{rgb}{0.000000,0.000000,0.000000}%
\pgfsetfillcolor{currentfill}%
\pgfsetlinewidth{0.803000pt}%
\definecolor{currentstroke}{rgb}{0.000000,0.000000,0.000000}%
\pgfsetstrokecolor{currentstroke}%
\pgfsetdash{}{0pt}%
\pgfsys@defobject{currentmarker}{\pgfqpoint{0.000000in}{-0.048611in}}{\pgfqpoint{0.000000in}{0.000000in}}{%
\pgfpathmoveto{\pgfqpoint{0.000000in}{0.000000in}}%
\pgfpathlineto{\pgfqpoint{0.000000in}{-0.048611in}}%
\pgfusepath{stroke,fill}%
}%
\begin{pgfscope}%
\pgfsys@transformshift{3.553229in}{0.499444in}%
\pgfsys@useobject{currentmarker}{}%
\end{pgfscope}%
\end{pgfscope}%
\begin{pgfscope}%
\definecolor{textcolor}{rgb}{0.000000,0.000000,0.000000}%
\pgfsetstrokecolor{textcolor}%
\pgfsetfillcolor{textcolor}%
\pgftext[x=3.553229in,y=0.402222in,,top]{\color{textcolor}\rmfamily\fontsize{10.000000}{12.000000}\selectfont 0.8}%
\end{pgfscope}%
\begin{pgfscope}%
\pgfsetbuttcap%
\pgfsetroundjoin%
\definecolor{currentfill}{rgb}{0.000000,0.000000,0.000000}%
\pgfsetfillcolor{currentfill}%
\pgfsetlinewidth{0.803000pt}%
\definecolor{currentstroke}{rgb}{0.000000,0.000000,0.000000}%
\pgfsetstrokecolor{currentstroke}%
\pgfsetdash{}{0pt}%
\pgfsys@defobject{currentmarker}{\pgfqpoint{0.000000in}{-0.048611in}}{\pgfqpoint{0.000000in}{0.000000in}}{%
\pgfpathmoveto{\pgfqpoint{0.000000in}{0.000000in}}%
\pgfpathlineto{\pgfqpoint{0.000000in}{-0.048611in}}%
\pgfusepath{stroke,fill}%
}%
\begin{pgfscope}%
\pgfsys@transformshift{3.936892in}{0.499444in}%
\pgfsys@useobject{currentmarker}{}%
\end{pgfscope}%
\end{pgfscope}%
\begin{pgfscope}%
\definecolor{textcolor}{rgb}{0.000000,0.000000,0.000000}%
\pgfsetstrokecolor{textcolor}%
\pgfsetfillcolor{textcolor}%
\pgftext[x=3.936892in,y=0.402222in,,top]{\color{textcolor}\rmfamily\fontsize{10.000000}{12.000000}\selectfont 0.9}%
\end{pgfscope}%
\begin{pgfscope}%
\pgfsetbuttcap%
\pgfsetroundjoin%
\definecolor{currentfill}{rgb}{0.000000,0.000000,0.000000}%
\pgfsetfillcolor{currentfill}%
\pgfsetlinewidth{0.803000pt}%
\definecolor{currentstroke}{rgb}{0.000000,0.000000,0.000000}%
\pgfsetstrokecolor{currentstroke}%
\pgfsetdash{}{0pt}%
\pgfsys@defobject{currentmarker}{\pgfqpoint{0.000000in}{-0.048611in}}{\pgfqpoint{0.000000in}{0.000000in}}{%
\pgfpathmoveto{\pgfqpoint{0.000000in}{0.000000in}}%
\pgfpathlineto{\pgfqpoint{0.000000in}{-0.048611in}}%
\pgfusepath{stroke,fill}%
}%
\begin{pgfscope}%
\pgfsys@transformshift{4.320556in}{0.499444in}%
\pgfsys@useobject{currentmarker}{}%
\end{pgfscope}%
\end{pgfscope}%
\begin{pgfscope}%
\definecolor{textcolor}{rgb}{0.000000,0.000000,0.000000}%
\pgfsetstrokecolor{textcolor}%
\pgfsetfillcolor{textcolor}%
\pgftext[x=4.320556in,y=0.402222in,,top]{\color{textcolor}\rmfamily\fontsize{10.000000}{12.000000}\selectfont 1.0}%
\end{pgfscope}%
\begin{pgfscope}%
\definecolor{textcolor}{rgb}{0.000000,0.000000,0.000000}%
\pgfsetstrokecolor{textcolor}%
\pgfsetfillcolor{textcolor}%
\pgftext[x=2.383056in,y=0.223333in,,top]{\color{textcolor}\rmfamily\fontsize{10.000000}{12.000000}\selectfont \(\displaystyle p\)}%
\end{pgfscope}%
\begin{pgfscope}%
\pgfsetbuttcap%
\pgfsetroundjoin%
\definecolor{currentfill}{rgb}{0.000000,0.000000,0.000000}%
\pgfsetfillcolor{currentfill}%
\pgfsetlinewidth{0.803000pt}%
\definecolor{currentstroke}{rgb}{0.000000,0.000000,0.000000}%
\pgfsetstrokecolor{currentstroke}%
\pgfsetdash{}{0pt}%
\pgfsys@defobject{currentmarker}{\pgfqpoint{-0.048611in}{0.000000in}}{\pgfqpoint{-0.000000in}{0.000000in}}{%
\pgfpathmoveto{\pgfqpoint{-0.000000in}{0.000000in}}%
\pgfpathlineto{\pgfqpoint{-0.048611in}{0.000000in}}%
\pgfusepath{stroke,fill}%
}%
\begin{pgfscope}%
\pgfsys@transformshift{0.445556in}{0.499444in}%
\pgfsys@useobject{currentmarker}{}%
\end{pgfscope}%
\end{pgfscope}%
\begin{pgfscope}%
\definecolor{textcolor}{rgb}{0.000000,0.000000,0.000000}%
\pgfsetstrokecolor{textcolor}%
\pgfsetfillcolor{textcolor}%
\pgftext[x=0.278889in, y=0.451250in, left, base]{\color{textcolor}\rmfamily\fontsize{10.000000}{12.000000}\selectfont \(\displaystyle {0}\)}%
\end{pgfscope}%
\begin{pgfscope}%
\pgfsetbuttcap%
\pgfsetroundjoin%
\definecolor{currentfill}{rgb}{0.000000,0.000000,0.000000}%
\pgfsetfillcolor{currentfill}%
\pgfsetlinewidth{0.803000pt}%
\definecolor{currentstroke}{rgb}{0.000000,0.000000,0.000000}%
\pgfsetstrokecolor{currentstroke}%
\pgfsetdash{}{0pt}%
\pgfsys@defobject{currentmarker}{\pgfqpoint{-0.048611in}{0.000000in}}{\pgfqpoint{-0.000000in}{0.000000in}}{%
\pgfpathmoveto{\pgfqpoint{-0.000000in}{0.000000in}}%
\pgfpathlineto{\pgfqpoint{-0.048611in}{0.000000in}}%
\pgfusepath{stroke,fill}%
}%
\begin{pgfscope}%
\pgfsys@transformshift{0.445556in}{0.796202in}%
\pgfsys@useobject{currentmarker}{}%
\end{pgfscope}%
\end{pgfscope}%
\begin{pgfscope}%
\definecolor{textcolor}{rgb}{0.000000,0.000000,0.000000}%
\pgfsetstrokecolor{textcolor}%
\pgfsetfillcolor{textcolor}%
\pgftext[x=0.278889in, y=0.748007in, left, base]{\color{textcolor}\rmfamily\fontsize{10.000000}{12.000000}\selectfont \(\displaystyle {2}\)}%
\end{pgfscope}%
\begin{pgfscope}%
\pgfsetbuttcap%
\pgfsetroundjoin%
\definecolor{currentfill}{rgb}{0.000000,0.000000,0.000000}%
\pgfsetfillcolor{currentfill}%
\pgfsetlinewidth{0.803000pt}%
\definecolor{currentstroke}{rgb}{0.000000,0.000000,0.000000}%
\pgfsetstrokecolor{currentstroke}%
\pgfsetdash{}{0pt}%
\pgfsys@defobject{currentmarker}{\pgfqpoint{-0.048611in}{0.000000in}}{\pgfqpoint{-0.000000in}{0.000000in}}{%
\pgfpathmoveto{\pgfqpoint{-0.000000in}{0.000000in}}%
\pgfpathlineto{\pgfqpoint{-0.048611in}{0.000000in}}%
\pgfusepath{stroke,fill}%
}%
\begin{pgfscope}%
\pgfsys@transformshift{0.445556in}{1.092959in}%
\pgfsys@useobject{currentmarker}{}%
\end{pgfscope}%
\end{pgfscope}%
\begin{pgfscope}%
\definecolor{textcolor}{rgb}{0.000000,0.000000,0.000000}%
\pgfsetstrokecolor{textcolor}%
\pgfsetfillcolor{textcolor}%
\pgftext[x=0.278889in, y=1.044765in, left, base]{\color{textcolor}\rmfamily\fontsize{10.000000}{12.000000}\selectfont \(\displaystyle {4}\)}%
\end{pgfscope}%
\begin{pgfscope}%
\pgfsetbuttcap%
\pgfsetroundjoin%
\definecolor{currentfill}{rgb}{0.000000,0.000000,0.000000}%
\pgfsetfillcolor{currentfill}%
\pgfsetlinewidth{0.803000pt}%
\definecolor{currentstroke}{rgb}{0.000000,0.000000,0.000000}%
\pgfsetstrokecolor{currentstroke}%
\pgfsetdash{}{0pt}%
\pgfsys@defobject{currentmarker}{\pgfqpoint{-0.048611in}{0.000000in}}{\pgfqpoint{-0.000000in}{0.000000in}}{%
\pgfpathmoveto{\pgfqpoint{-0.000000in}{0.000000in}}%
\pgfpathlineto{\pgfqpoint{-0.048611in}{0.000000in}}%
\pgfusepath{stroke,fill}%
}%
\begin{pgfscope}%
\pgfsys@transformshift{0.445556in}{1.389716in}%
\pgfsys@useobject{currentmarker}{}%
\end{pgfscope}%
\end{pgfscope}%
\begin{pgfscope}%
\definecolor{textcolor}{rgb}{0.000000,0.000000,0.000000}%
\pgfsetstrokecolor{textcolor}%
\pgfsetfillcolor{textcolor}%
\pgftext[x=0.278889in, y=1.341522in, left, base]{\color{textcolor}\rmfamily\fontsize{10.000000}{12.000000}\selectfont \(\displaystyle {6}\)}%
\end{pgfscope}%
\begin{pgfscope}%
\definecolor{textcolor}{rgb}{0.000000,0.000000,0.000000}%
\pgfsetstrokecolor{textcolor}%
\pgfsetfillcolor{textcolor}%
\pgftext[x=0.223333in,y=1.076944in,,bottom,rotate=90.000000]{\color{textcolor}\rmfamily\fontsize{10.000000}{12.000000}\selectfont Percent of Data Set}%
\end{pgfscope}%
\begin{pgfscope}%
\pgfsetrectcap%
\pgfsetmiterjoin%
\pgfsetlinewidth{0.803000pt}%
\definecolor{currentstroke}{rgb}{0.000000,0.000000,0.000000}%
\pgfsetstrokecolor{currentstroke}%
\pgfsetdash{}{0pt}%
\pgfpathmoveto{\pgfqpoint{0.445556in}{0.499444in}}%
\pgfpathlineto{\pgfqpoint{0.445556in}{1.654444in}}%
\pgfusepath{stroke}%
\end{pgfscope}%
\begin{pgfscope}%
\pgfsetrectcap%
\pgfsetmiterjoin%
\pgfsetlinewidth{0.803000pt}%
\definecolor{currentstroke}{rgb}{0.000000,0.000000,0.000000}%
\pgfsetstrokecolor{currentstroke}%
\pgfsetdash{}{0pt}%
\pgfpathmoveto{\pgfqpoint{4.320556in}{0.499444in}}%
\pgfpathlineto{\pgfqpoint{4.320556in}{1.654444in}}%
\pgfusepath{stroke}%
\end{pgfscope}%
\begin{pgfscope}%
\pgfsetrectcap%
\pgfsetmiterjoin%
\pgfsetlinewidth{0.803000pt}%
\definecolor{currentstroke}{rgb}{0.000000,0.000000,0.000000}%
\pgfsetstrokecolor{currentstroke}%
\pgfsetdash{}{0pt}%
\pgfpathmoveto{\pgfqpoint{0.445556in}{0.499444in}}%
\pgfpathlineto{\pgfqpoint{4.320556in}{0.499444in}}%
\pgfusepath{stroke}%
\end{pgfscope}%
\begin{pgfscope}%
\pgfsetrectcap%
\pgfsetmiterjoin%
\pgfsetlinewidth{0.803000pt}%
\definecolor{currentstroke}{rgb}{0.000000,0.000000,0.000000}%
\pgfsetstrokecolor{currentstroke}%
\pgfsetdash{}{0pt}%
\pgfpathmoveto{\pgfqpoint{0.445556in}{1.654444in}}%
\pgfpathlineto{\pgfqpoint{4.320556in}{1.654444in}}%
\pgfusepath{stroke}%
\end{pgfscope}%
\begin{pgfscope}%
\pgfsetbuttcap%
\pgfsetmiterjoin%
\definecolor{currentfill}{rgb}{1.000000,1.000000,1.000000}%
\pgfsetfillcolor{currentfill}%
\pgfsetfillopacity{0.800000}%
\pgfsetlinewidth{1.003750pt}%
\definecolor{currentstroke}{rgb}{0.800000,0.800000,0.800000}%
\pgfsetstrokecolor{currentstroke}%
\pgfsetstrokeopacity{0.800000}%
\pgfsetdash{}{0pt}%
\pgfpathmoveto{\pgfqpoint{3.543611in}{1.154445in}}%
\pgfpathlineto{\pgfqpoint{4.223333in}{1.154445in}}%
\pgfpathquadraticcurveto{\pgfqpoint{4.251111in}{1.154445in}}{\pgfqpoint{4.251111in}{1.182222in}}%
\pgfpathlineto{\pgfqpoint{4.251111in}{1.557222in}}%
\pgfpathquadraticcurveto{\pgfqpoint{4.251111in}{1.585000in}}{\pgfqpoint{4.223333in}{1.585000in}}%
\pgfpathlineto{\pgfqpoint{3.543611in}{1.585000in}}%
\pgfpathquadraticcurveto{\pgfqpoint{3.515833in}{1.585000in}}{\pgfqpoint{3.515833in}{1.557222in}}%
\pgfpathlineto{\pgfqpoint{3.515833in}{1.182222in}}%
\pgfpathquadraticcurveto{\pgfqpoint{3.515833in}{1.154445in}}{\pgfqpoint{3.543611in}{1.154445in}}%
\pgfpathlineto{\pgfqpoint{3.543611in}{1.154445in}}%
\pgfpathclose%
\pgfusepath{stroke,fill}%
\end{pgfscope}%
\begin{pgfscope}%
\pgfsetbuttcap%
\pgfsetmiterjoin%
\pgfsetlinewidth{1.003750pt}%
\definecolor{currentstroke}{rgb}{0.000000,0.000000,0.000000}%
\pgfsetstrokecolor{currentstroke}%
\pgfsetdash{}{0pt}%
\pgfpathmoveto{\pgfqpoint{3.571389in}{1.432222in}}%
\pgfpathlineto{\pgfqpoint{3.849167in}{1.432222in}}%
\pgfpathlineto{\pgfqpoint{3.849167in}{1.529444in}}%
\pgfpathlineto{\pgfqpoint{3.571389in}{1.529444in}}%
\pgfpathlineto{\pgfqpoint{3.571389in}{1.432222in}}%
\pgfpathclose%
\pgfusepath{stroke}%
\end{pgfscope}%
\begin{pgfscope}%
\definecolor{textcolor}{rgb}{0.000000,0.000000,0.000000}%
\pgfsetstrokecolor{textcolor}%
\pgfsetfillcolor{textcolor}%
\pgftext[x=3.960278in,y=1.432222in,left,base]{\color{textcolor}\rmfamily\fontsize{10.000000}{12.000000}\selectfont Neg}%
\end{pgfscope}%
\begin{pgfscope}%
\pgfsetbuttcap%
\pgfsetmiterjoin%
\definecolor{currentfill}{rgb}{0.000000,0.000000,0.000000}%
\pgfsetfillcolor{currentfill}%
\pgfsetlinewidth{0.000000pt}%
\definecolor{currentstroke}{rgb}{0.000000,0.000000,0.000000}%
\pgfsetstrokecolor{currentstroke}%
\pgfsetstrokeopacity{0.000000}%
\pgfsetdash{}{0pt}%
\pgfpathmoveto{\pgfqpoint{3.571389in}{1.236944in}}%
\pgfpathlineto{\pgfqpoint{3.849167in}{1.236944in}}%
\pgfpathlineto{\pgfqpoint{3.849167in}{1.334167in}}%
\pgfpathlineto{\pgfqpoint{3.571389in}{1.334167in}}%
\pgfpathlineto{\pgfqpoint{3.571389in}{1.236944in}}%
\pgfpathclose%
\pgfusepath{fill}%
\end{pgfscope}%
\begin{pgfscope}%
\definecolor{textcolor}{rgb}{0.000000,0.000000,0.000000}%
\pgfsetstrokecolor{textcolor}%
\pgfsetfillcolor{textcolor}%
\pgftext[x=3.960278in,y=1.236944in,left,base]{\color{textcolor}\rmfamily\fontsize{10.000000}{12.000000}\selectfont Pos}%
\end{pgfscope}%
\end{pgfpicture}%
\makeatother%
\endgroup%

&
	\vskip 0pt
	\begin{tabular}{cc|c|c|}
	&\multicolumn{1}{c}{}& \multicolumn{2}{c}{Prediction} \cr
	&\multicolumn{1}{c}{} & \multicolumn{1}{c}{N} & \multicolumn{1}{c}{P} \cr\cline{3-4}
	\multirow{2}{*}{\rotatebox[origin=c]{90}{Actual}}&N &
168,381 & 11,864
	\vrule width 0pt height 10pt depth 2pt \cr\cline{3-4}
	&P & 
22,587 & 11,238
	\vrule width 0pt height 10pt depth 2pt \cr\cline{3-4}
	\end{tabular}

	\hfil\begin{tabular}{ll}
	\cr
0.487 & Precision \cr	0.332 & Recall \cr	0.395 & F1 \cr
\end{tabular}
\end{tabular}

I ran the same Keras three with the same three class weights, but using $\Delta FP/\Delta TP = 1$, and got similar results, that when you ``normalize'' the output to center where $\Delta FP/\Delta TP = 1$, you get basically the same results.  One had better recall (0.146 instead of 0.156), but otherwise the same.  Then I tried it with $\Delta FP/\Delta TP = 3$, and also basically the same.  

Any differences may be attributable to the values of $p$ not really being continuous, as most of them only have two decimal places, like $0.52$, with very few like $0.5246$, so the differences in the metrics may be attributable to rounding errors.  



%%%%%
\newpage
\subsection{Other Questions:  Overfitting?}

When I use the Keras Binary Crossentropy Classifier and test for overfitting by running the classifier on both the training and test sets, I get basically the same thing, which (I think) means it's not overfitting.  

\

\verb|y_proba = estimator.predict_proba(X_train)|

\noindent\begin{tabular}{@{\hspace{-6pt}}p{4.5in} @{\hspace{-6pt}}p{2.0in}}
	\vskip 0pt
	\qquad \qquad Raw Model Output on Training Set
	
	%% Creator: Matplotlib, PGF backend
%%
%% To include the figure in your LaTeX document, write
%%   \input{<filename>.pgf}
%%
%% Make sure the required packages are loaded in your preamble
%%   \usepackage{pgf}
%%
%% Also ensure that all the required font packages are loaded; for instance,
%% the lmodern package is sometimes necessary when using math font.
%%   \usepackage{lmodern}
%%
%% Figures using additional raster images can only be included by \input if
%% they are in the same directory as the main LaTeX file. For loading figures
%% from other directories you can use the `import` package
%%   \usepackage{import}
%%
%% and then include the figures with
%%   \import{<path to file>}{<filename>.pgf}
%%
%% Matplotlib used the following preamble
%%   
%%   \usepackage{fontspec}
%%   \makeatletter\@ifpackageloaded{underscore}{}{\usepackage[strings]{underscore}}\makeatother
%%
\begingroup%
\makeatletter%
\begin{pgfpicture}%
\pgfpathrectangle{\pgfpointorigin}{\pgfqpoint{4.102500in}{1.754444in}}%
\pgfusepath{use as bounding box, clip}%
\begin{pgfscope}%
\pgfsetbuttcap%
\pgfsetmiterjoin%
\definecolor{currentfill}{rgb}{1.000000,1.000000,1.000000}%
\pgfsetfillcolor{currentfill}%
\pgfsetlinewidth{0.000000pt}%
\definecolor{currentstroke}{rgb}{1.000000,1.000000,1.000000}%
\pgfsetstrokecolor{currentstroke}%
\pgfsetdash{}{0pt}%
\pgfpathmoveto{\pgfqpoint{0.000000in}{0.000000in}}%
\pgfpathlineto{\pgfqpoint{4.102500in}{0.000000in}}%
\pgfpathlineto{\pgfqpoint{4.102500in}{1.754444in}}%
\pgfpathlineto{\pgfqpoint{0.000000in}{1.754444in}}%
\pgfpathlineto{\pgfqpoint{0.000000in}{0.000000in}}%
\pgfpathclose%
\pgfusepath{fill}%
\end{pgfscope}%
\begin{pgfscope}%
\pgfsetbuttcap%
\pgfsetmiterjoin%
\definecolor{currentfill}{rgb}{1.000000,1.000000,1.000000}%
\pgfsetfillcolor{currentfill}%
\pgfsetlinewidth{0.000000pt}%
\definecolor{currentstroke}{rgb}{0.000000,0.000000,0.000000}%
\pgfsetstrokecolor{currentstroke}%
\pgfsetstrokeopacity{0.000000}%
\pgfsetdash{}{0pt}%
\pgfpathmoveto{\pgfqpoint{0.515000in}{0.499444in}}%
\pgfpathlineto{\pgfqpoint{4.002500in}{0.499444in}}%
\pgfpathlineto{\pgfqpoint{4.002500in}{1.654444in}}%
\pgfpathlineto{\pgfqpoint{0.515000in}{1.654444in}}%
\pgfpathlineto{\pgfqpoint{0.515000in}{0.499444in}}%
\pgfpathclose%
\pgfusepath{fill}%
\end{pgfscope}%
\begin{pgfscope}%
\pgfpathrectangle{\pgfqpoint{0.515000in}{0.499444in}}{\pgfqpoint{3.487500in}{1.155000in}}%
\pgfusepath{clip}%
\pgfsetbuttcap%
\pgfsetmiterjoin%
\pgfsetlinewidth{1.003750pt}%
\definecolor{currentstroke}{rgb}{0.000000,0.000000,0.000000}%
\pgfsetstrokecolor{currentstroke}%
\pgfsetdash{}{0pt}%
\pgfpathmoveto{\pgfqpoint{0.610114in}{0.499444in}}%
\pgfpathlineto{\pgfqpoint{0.673523in}{0.499444in}}%
\pgfpathlineto{\pgfqpoint{0.673523in}{0.499444in}}%
\pgfpathlineto{\pgfqpoint{0.610114in}{0.499444in}}%
\pgfpathlineto{\pgfqpoint{0.610114in}{0.499444in}}%
\pgfpathclose%
\pgfusepath{stroke}%
\end{pgfscope}%
\begin{pgfscope}%
\pgfpathrectangle{\pgfqpoint{0.515000in}{0.499444in}}{\pgfqpoint{3.487500in}{1.155000in}}%
\pgfusepath{clip}%
\pgfsetbuttcap%
\pgfsetmiterjoin%
\pgfsetlinewidth{1.003750pt}%
\definecolor{currentstroke}{rgb}{0.000000,0.000000,0.000000}%
\pgfsetstrokecolor{currentstroke}%
\pgfsetdash{}{0pt}%
\pgfpathmoveto{\pgfqpoint{0.768637in}{0.499444in}}%
\pgfpathlineto{\pgfqpoint{0.832046in}{0.499444in}}%
\pgfpathlineto{\pgfqpoint{0.832046in}{1.599444in}}%
\pgfpathlineto{\pgfqpoint{0.768637in}{1.599444in}}%
\pgfpathlineto{\pgfqpoint{0.768637in}{0.499444in}}%
\pgfpathclose%
\pgfusepath{stroke}%
\end{pgfscope}%
\begin{pgfscope}%
\pgfpathrectangle{\pgfqpoint{0.515000in}{0.499444in}}{\pgfqpoint{3.487500in}{1.155000in}}%
\pgfusepath{clip}%
\pgfsetbuttcap%
\pgfsetmiterjoin%
\pgfsetlinewidth{1.003750pt}%
\definecolor{currentstroke}{rgb}{0.000000,0.000000,0.000000}%
\pgfsetstrokecolor{currentstroke}%
\pgfsetdash{}{0pt}%
\pgfpathmoveto{\pgfqpoint{0.927159in}{0.499444in}}%
\pgfpathlineto{\pgfqpoint{0.990568in}{0.499444in}}%
\pgfpathlineto{\pgfqpoint{0.990568in}{1.342040in}}%
\pgfpathlineto{\pgfqpoint{0.927159in}{1.342040in}}%
\pgfpathlineto{\pgfqpoint{0.927159in}{0.499444in}}%
\pgfpathclose%
\pgfusepath{stroke}%
\end{pgfscope}%
\begin{pgfscope}%
\pgfpathrectangle{\pgfqpoint{0.515000in}{0.499444in}}{\pgfqpoint{3.487500in}{1.155000in}}%
\pgfusepath{clip}%
\pgfsetbuttcap%
\pgfsetmiterjoin%
\pgfsetlinewidth{1.003750pt}%
\definecolor{currentstroke}{rgb}{0.000000,0.000000,0.000000}%
\pgfsetstrokecolor{currentstroke}%
\pgfsetdash{}{0pt}%
\pgfpathmoveto{\pgfqpoint{1.085682in}{0.499444in}}%
\pgfpathlineto{\pgfqpoint{1.149091in}{0.499444in}}%
\pgfpathlineto{\pgfqpoint{1.149091in}{1.036613in}}%
\pgfpathlineto{\pgfqpoint{1.085682in}{1.036613in}}%
\pgfpathlineto{\pgfqpoint{1.085682in}{0.499444in}}%
\pgfpathclose%
\pgfusepath{stroke}%
\end{pgfscope}%
\begin{pgfscope}%
\pgfpathrectangle{\pgfqpoint{0.515000in}{0.499444in}}{\pgfqpoint{3.487500in}{1.155000in}}%
\pgfusepath{clip}%
\pgfsetbuttcap%
\pgfsetmiterjoin%
\pgfsetlinewidth{1.003750pt}%
\definecolor{currentstroke}{rgb}{0.000000,0.000000,0.000000}%
\pgfsetstrokecolor{currentstroke}%
\pgfsetdash{}{0pt}%
\pgfpathmoveto{\pgfqpoint{1.244205in}{0.499444in}}%
\pgfpathlineto{\pgfqpoint{1.307614in}{0.499444in}}%
\pgfpathlineto{\pgfqpoint{1.307614in}{0.850803in}}%
\pgfpathlineto{\pgfqpoint{1.244205in}{0.850803in}}%
\pgfpathlineto{\pgfqpoint{1.244205in}{0.499444in}}%
\pgfpathclose%
\pgfusepath{stroke}%
\end{pgfscope}%
\begin{pgfscope}%
\pgfpathrectangle{\pgfqpoint{0.515000in}{0.499444in}}{\pgfqpoint{3.487500in}{1.155000in}}%
\pgfusepath{clip}%
\pgfsetbuttcap%
\pgfsetmiterjoin%
\pgfsetlinewidth{1.003750pt}%
\definecolor{currentstroke}{rgb}{0.000000,0.000000,0.000000}%
\pgfsetstrokecolor{currentstroke}%
\pgfsetdash{}{0pt}%
\pgfpathmoveto{\pgfqpoint{1.402728in}{0.499444in}}%
\pgfpathlineto{\pgfqpoint{1.466137in}{0.499444in}}%
\pgfpathlineto{\pgfqpoint{1.466137in}{0.740648in}}%
\pgfpathlineto{\pgfqpoint{1.402728in}{0.740648in}}%
\pgfpathlineto{\pgfqpoint{1.402728in}{0.499444in}}%
\pgfpathclose%
\pgfusepath{stroke}%
\end{pgfscope}%
\begin{pgfscope}%
\pgfpathrectangle{\pgfqpoint{0.515000in}{0.499444in}}{\pgfqpoint{3.487500in}{1.155000in}}%
\pgfusepath{clip}%
\pgfsetbuttcap%
\pgfsetmiterjoin%
\pgfsetlinewidth{1.003750pt}%
\definecolor{currentstroke}{rgb}{0.000000,0.000000,0.000000}%
\pgfsetstrokecolor{currentstroke}%
\pgfsetdash{}{0pt}%
\pgfpathmoveto{\pgfqpoint{1.561250in}{0.499444in}}%
\pgfpathlineto{\pgfqpoint{1.624659in}{0.499444in}}%
\pgfpathlineto{\pgfqpoint{1.624659in}{0.661469in}}%
\pgfpathlineto{\pgfqpoint{1.561250in}{0.661469in}}%
\pgfpathlineto{\pgfqpoint{1.561250in}{0.499444in}}%
\pgfpathclose%
\pgfusepath{stroke}%
\end{pgfscope}%
\begin{pgfscope}%
\pgfpathrectangle{\pgfqpoint{0.515000in}{0.499444in}}{\pgfqpoint{3.487500in}{1.155000in}}%
\pgfusepath{clip}%
\pgfsetbuttcap%
\pgfsetmiterjoin%
\pgfsetlinewidth{1.003750pt}%
\definecolor{currentstroke}{rgb}{0.000000,0.000000,0.000000}%
\pgfsetstrokecolor{currentstroke}%
\pgfsetdash{}{0pt}%
\pgfpathmoveto{\pgfqpoint{1.719773in}{0.499444in}}%
\pgfpathlineto{\pgfqpoint{1.783182in}{0.499444in}}%
\pgfpathlineto{\pgfqpoint{1.783182in}{0.607339in}}%
\pgfpathlineto{\pgfqpoint{1.719773in}{0.607339in}}%
\pgfpathlineto{\pgfqpoint{1.719773in}{0.499444in}}%
\pgfpathclose%
\pgfusepath{stroke}%
\end{pgfscope}%
\begin{pgfscope}%
\pgfpathrectangle{\pgfqpoint{0.515000in}{0.499444in}}{\pgfqpoint{3.487500in}{1.155000in}}%
\pgfusepath{clip}%
\pgfsetbuttcap%
\pgfsetmiterjoin%
\pgfsetlinewidth{1.003750pt}%
\definecolor{currentstroke}{rgb}{0.000000,0.000000,0.000000}%
\pgfsetstrokecolor{currentstroke}%
\pgfsetdash{}{0pt}%
\pgfpathmoveto{\pgfqpoint{1.878296in}{0.499444in}}%
\pgfpathlineto{\pgfqpoint{1.941705in}{0.499444in}}%
\pgfpathlineto{\pgfqpoint{1.941705in}{0.574340in}}%
\pgfpathlineto{\pgfqpoint{1.878296in}{0.574340in}}%
\pgfpathlineto{\pgfqpoint{1.878296in}{0.499444in}}%
\pgfpathclose%
\pgfusepath{stroke}%
\end{pgfscope}%
\begin{pgfscope}%
\pgfpathrectangle{\pgfqpoint{0.515000in}{0.499444in}}{\pgfqpoint{3.487500in}{1.155000in}}%
\pgfusepath{clip}%
\pgfsetbuttcap%
\pgfsetmiterjoin%
\pgfsetlinewidth{1.003750pt}%
\definecolor{currentstroke}{rgb}{0.000000,0.000000,0.000000}%
\pgfsetstrokecolor{currentstroke}%
\pgfsetdash{}{0pt}%
\pgfpathmoveto{\pgfqpoint{2.036818in}{0.499444in}}%
\pgfpathlineto{\pgfqpoint{2.100228in}{0.499444in}}%
\pgfpathlineto{\pgfqpoint{2.100228in}{0.551433in}}%
\pgfpathlineto{\pgfqpoint{2.036818in}{0.551433in}}%
\pgfpathlineto{\pgfqpoint{2.036818in}{0.499444in}}%
\pgfpathclose%
\pgfusepath{stroke}%
\end{pgfscope}%
\begin{pgfscope}%
\pgfpathrectangle{\pgfqpoint{0.515000in}{0.499444in}}{\pgfqpoint{3.487500in}{1.155000in}}%
\pgfusepath{clip}%
\pgfsetbuttcap%
\pgfsetmiterjoin%
\pgfsetlinewidth{1.003750pt}%
\definecolor{currentstroke}{rgb}{0.000000,0.000000,0.000000}%
\pgfsetstrokecolor{currentstroke}%
\pgfsetdash{}{0pt}%
\pgfpathmoveto{\pgfqpoint{2.195341in}{0.499444in}}%
\pgfpathlineto{\pgfqpoint{2.258750in}{0.499444in}}%
\pgfpathlineto{\pgfqpoint{2.258750in}{0.534951in}}%
\pgfpathlineto{\pgfqpoint{2.195341in}{0.534951in}}%
\pgfpathlineto{\pgfqpoint{2.195341in}{0.499444in}}%
\pgfpathclose%
\pgfusepath{stroke}%
\end{pgfscope}%
\begin{pgfscope}%
\pgfpathrectangle{\pgfqpoint{0.515000in}{0.499444in}}{\pgfqpoint{3.487500in}{1.155000in}}%
\pgfusepath{clip}%
\pgfsetbuttcap%
\pgfsetmiterjoin%
\pgfsetlinewidth{1.003750pt}%
\definecolor{currentstroke}{rgb}{0.000000,0.000000,0.000000}%
\pgfsetstrokecolor{currentstroke}%
\pgfsetdash{}{0pt}%
\pgfpathmoveto{\pgfqpoint{2.353864in}{0.499444in}}%
\pgfpathlineto{\pgfqpoint{2.417273in}{0.499444in}}%
\pgfpathlineto{\pgfqpoint{2.417273in}{0.525447in}}%
\pgfpathlineto{\pgfqpoint{2.353864in}{0.525447in}}%
\pgfpathlineto{\pgfqpoint{2.353864in}{0.499444in}}%
\pgfpathclose%
\pgfusepath{stroke}%
\end{pgfscope}%
\begin{pgfscope}%
\pgfpathrectangle{\pgfqpoint{0.515000in}{0.499444in}}{\pgfqpoint{3.487500in}{1.155000in}}%
\pgfusepath{clip}%
\pgfsetbuttcap%
\pgfsetmiterjoin%
\pgfsetlinewidth{1.003750pt}%
\definecolor{currentstroke}{rgb}{0.000000,0.000000,0.000000}%
\pgfsetstrokecolor{currentstroke}%
\pgfsetdash{}{0pt}%
\pgfpathmoveto{\pgfqpoint{2.512387in}{0.499444in}}%
\pgfpathlineto{\pgfqpoint{2.575796in}{0.499444in}}%
\pgfpathlineto{\pgfqpoint{2.575796in}{0.517522in}}%
\pgfpathlineto{\pgfqpoint{2.512387in}{0.517522in}}%
\pgfpathlineto{\pgfqpoint{2.512387in}{0.499444in}}%
\pgfpathclose%
\pgfusepath{stroke}%
\end{pgfscope}%
\begin{pgfscope}%
\pgfpathrectangle{\pgfqpoint{0.515000in}{0.499444in}}{\pgfqpoint{3.487500in}{1.155000in}}%
\pgfusepath{clip}%
\pgfsetbuttcap%
\pgfsetmiterjoin%
\pgfsetlinewidth{1.003750pt}%
\definecolor{currentstroke}{rgb}{0.000000,0.000000,0.000000}%
\pgfsetstrokecolor{currentstroke}%
\pgfsetdash{}{0pt}%
\pgfpathmoveto{\pgfqpoint{2.670909in}{0.499444in}}%
\pgfpathlineto{\pgfqpoint{2.734318in}{0.499444in}}%
\pgfpathlineto{\pgfqpoint{2.734318in}{0.512190in}}%
\pgfpathlineto{\pgfqpoint{2.670909in}{0.512190in}}%
\pgfpathlineto{\pgfqpoint{2.670909in}{0.499444in}}%
\pgfpathclose%
\pgfusepath{stroke}%
\end{pgfscope}%
\begin{pgfscope}%
\pgfpathrectangle{\pgfqpoint{0.515000in}{0.499444in}}{\pgfqpoint{3.487500in}{1.155000in}}%
\pgfusepath{clip}%
\pgfsetbuttcap%
\pgfsetmiterjoin%
\pgfsetlinewidth{1.003750pt}%
\definecolor{currentstroke}{rgb}{0.000000,0.000000,0.000000}%
\pgfsetstrokecolor{currentstroke}%
\pgfsetdash{}{0pt}%
\pgfpathmoveto{\pgfqpoint{2.829432in}{0.499444in}}%
\pgfpathlineto{\pgfqpoint{2.892841in}{0.499444in}}%
\pgfpathlineto{\pgfqpoint{2.892841in}{0.508803in}}%
\pgfpathlineto{\pgfqpoint{2.829432in}{0.508803in}}%
\pgfpathlineto{\pgfqpoint{2.829432in}{0.499444in}}%
\pgfpathclose%
\pgfusepath{stroke}%
\end{pgfscope}%
\begin{pgfscope}%
\pgfpathrectangle{\pgfqpoint{0.515000in}{0.499444in}}{\pgfqpoint{3.487500in}{1.155000in}}%
\pgfusepath{clip}%
\pgfsetbuttcap%
\pgfsetmiterjoin%
\pgfsetlinewidth{1.003750pt}%
\definecolor{currentstroke}{rgb}{0.000000,0.000000,0.000000}%
\pgfsetstrokecolor{currentstroke}%
\pgfsetdash{}{0pt}%
\pgfpathmoveto{\pgfqpoint{2.987955in}{0.499444in}}%
\pgfpathlineto{\pgfqpoint{3.051364in}{0.499444in}}%
\pgfpathlineto{\pgfqpoint{3.051364in}{0.506662in}}%
\pgfpathlineto{\pgfqpoint{2.987955in}{0.506662in}}%
\pgfpathlineto{\pgfqpoint{2.987955in}{0.499444in}}%
\pgfpathclose%
\pgfusepath{stroke}%
\end{pgfscope}%
\begin{pgfscope}%
\pgfpathrectangle{\pgfqpoint{0.515000in}{0.499444in}}{\pgfqpoint{3.487500in}{1.155000in}}%
\pgfusepath{clip}%
\pgfsetbuttcap%
\pgfsetmiterjoin%
\pgfsetlinewidth{1.003750pt}%
\definecolor{currentstroke}{rgb}{0.000000,0.000000,0.000000}%
\pgfsetstrokecolor{currentstroke}%
\pgfsetdash{}{0pt}%
\pgfpathmoveto{\pgfqpoint{3.146478in}{0.499444in}}%
\pgfpathlineto{\pgfqpoint{3.209887in}{0.499444in}}%
\pgfpathlineto{\pgfqpoint{3.209887in}{0.504606in}}%
\pgfpathlineto{\pgfqpoint{3.146478in}{0.504606in}}%
\pgfpathlineto{\pgfqpoint{3.146478in}{0.499444in}}%
\pgfpathclose%
\pgfusepath{stroke}%
\end{pgfscope}%
\begin{pgfscope}%
\pgfpathrectangle{\pgfqpoint{0.515000in}{0.499444in}}{\pgfqpoint{3.487500in}{1.155000in}}%
\pgfusepath{clip}%
\pgfsetbuttcap%
\pgfsetmiterjoin%
\pgfsetlinewidth{1.003750pt}%
\definecolor{currentstroke}{rgb}{0.000000,0.000000,0.000000}%
\pgfsetstrokecolor{currentstroke}%
\pgfsetdash{}{0pt}%
\pgfpathmoveto{\pgfqpoint{3.305000in}{0.499444in}}%
\pgfpathlineto{\pgfqpoint{3.368409in}{0.499444in}}%
\pgfpathlineto{\pgfqpoint{3.368409in}{0.502678in}}%
\pgfpathlineto{\pgfqpoint{3.305000in}{0.502678in}}%
\pgfpathlineto{\pgfqpoint{3.305000in}{0.499444in}}%
\pgfpathclose%
\pgfusepath{stroke}%
\end{pgfscope}%
\begin{pgfscope}%
\pgfpathrectangle{\pgfqpoint{0.515000in}{0.499444in}}{\pgfqpoint{3.487500in}{1.155000in}}%
\pgfusepath{clip}%
\pgfsetbuttcap%
\pgfsetmiterjoin%
\pgfsetlinewidth{1.003750pt}%
\definecolor{currentstroke}{rgb}{0.000000,0.000000,0.000000}%
\pgfsetstrokecolor{currentstroke}%
\pgfsetdash{}{0pt}%
\pgfpathmoveto{\pgfqpoint{3.463523in}{0.499444in}}%
\pgfpathlineto{\pgfqpoint{3.526932in}{0.499444in}}%
\pgfpathlineto{\pgfqpoint{3.526932in}{0.500826in}}%
\pgfpathlineto{\pgfqpoint{3.463523in}{0.500826in}}%
\pgfpathlineto{\pgfqpoint{3.463523in}{0.499444in}}%
\pgfpathclose%
\pgfusepath{stroke}%
\end{pgfscope}%
\begin{pgfscope}%
\pgfpathrectangle{\pgfqpoint{0.515000in}{0.499444in}}{\pgfqpoint{3.487500in}{1.155000in}}%
\pgfusepath{clip}%
\pgfsetbuttcap%
\pgfsetmiterjoin%
\pgfsetlinewidth{1.003750pt}%
\definecolor{currentstroke}{rgb}{0.000000,0.000000,0.000000}%
\pgfsetstrokecolor{currentstroke}%
\pgfsetdash{}{0pt}%
\pgfpathmoveto{\pgfqpoint{3.622046in}{0.499444in}}%
\pgfpathlineto{\pgfqpoint{3.685455in}{0.499444in}}%
\pgfpathlineto{\pgfqpoint{3.685455in}{0.499598in}}%
\pgfpathlineto{\pgfqpoint{3.622046in}{0.499598in}}%
\pgfpathlineto{\pgfqpoint{3.622046in}{0.499444in}}%
\pgfpathclose%
\pgfusepath{stroke}%
\end{pgfscope}%
\begin{pgfscope}%
\pgfpathrectangle{\pgfqpoint{0.515000in}{0.499444in}}{\pgfqpoint{3.487500in}{1.155000in}}%
\pgfusepath{clip}%
\pgfsetbuttcap%
\pgfsetmiterjoin%
\pgfsetlinewidth{1.003750pt}%
\definecolor{currentstroke}{rgb}{0.000000,0.000000,0.000000}%
\pgfsetstrokecolor{currentstroke}%
\pgfsetdash{}{0pt}%
\pgfpathmoveto{\pgfqpoint{3.780568in}{0.499444in}}%
\pgfpathlineto{\pgfqpoint{3.843978in}{0.499444in}}%
\pgfpathlineto{\pgfqpoint{3.843978in}{0.499453in}}%
\pgfpathlineto{\pgfqpoint{3.780568in}{0.499453in}}%
\pgfpathlineto{\pgfqpoint{3.780568in}{0.499444in}}%
\pgfpathclose%
\pgfusepath{stroke}%
\end{pgfscope}%
\begin{pgfscope}%
\pgfpathrectangle{\pgfqpoint{0.515000in}{0.499444in}}{\pgfqpoint{3.487500in}{1.155000in}}%
\pgfusepath{clip}%
\pgfsetbuttcap%
\pgfsetmiterjoin%
\definecolor{currentfill}{rgb}{0.000000,0.000000,0.000000}%
\pgfsetfillcolor{currentfill}%
\pgfsetlinewidth{0.000000pt}%
\definecolor{currentstroke}{rgb}{0.000000,0.000000,0.000000}%
\pgfsetstrokecolor{currentstroke}%
\pgfsetstrokeopacity{0.000000}%
\pgfsetdash{}{0pt}%
\pgfpathmoveto{\pgfqpoint{0.673523in}{0.499444in}}%
\pgfpathlineto{\pgfqpoint{0.736932in}{0.499444in}}%
\pgfpathlineto{\pgfqpoint{0.736932in}{0.499444in}}%
\pgfpathlineto{\pgfqpoint{0.673523in}{0.499444in}}%
\pgfpathlineto{\pgfqpoint{0.673523in}{0.499444in}}%
\pgfpathclose%
\pgfusepath{fill}%
\end{pgfscope}%
\begin{pgfscope}%
\pgfpathrectangle{\pgfqpoint{0.515000in}{0.499444in}}{\pgfqpoint{3.487500in}{1.155000in}}%
\pgfusepath{clip}%
\pgfsetbuttcap%
\pgfsetmiterjoin%
\definecolor{currentfill}{rgb}{0.000000,0.000000,0.000000}%
\pgfsetfillcolor{currentfill}%
\pgfsetlinewidth{0.000000pt}%
\definecolor{currentstroke}{rgb}{0.000000,0.000000,0.000000}%
\pgfsetstrokecolor{currentstroke}%
\pgfsetstrokeopacity{0.000000}%
\pgfsetdash{}{0pt}%
\pgfpathmoveto{\pgfqpoint{0.832046in}{0.499444in}}%
\pgfpathlineto{\pgfqpoint{0.895455in}{0.499444in}}%
\pgfpathlineto{\pgfqpoint{0.895455in}{0.527802in}}%
\pgfpathlineto{\pgfqpoint{0.832046in}{0.527802in}}%
\pgfpathlineto{\pgfqpoint{0.832046in}{0.499444in}}%
\pgfpathclose%
\pgfusepath{fill}%
\end{pgfscope}%
\begin{pgfscope}%
\pgfpathrectangle{\pgfqpoint{0.515000in}{0.499444in}}{\pgfqpoint{3.487500in}{1.155000in}}%
\pgfusepath{clip}%
\pgfsetbuttcap%
\pgfsetmiterjoin%
\definecolor{currentfill}{rgb}{0.000000,0.000000,0.000000}%
\pgfsetfillcolor{currentfill}%
\pgfsetlinewidth{0.000000pt}%
\definecolor{currentstroke}{rgb}{0.000000,0.000000,0.000000}%
\pgfsetstrokecolor{currentstroke}%
\pgfsetstrokeopacity{0.000000}%
\pgfsetdash{}{0pt}%
\pgfpathmoveto{\pgfqpoint{0.990568in}{0.499444in}}%
\pgfpathlineto{\pgfqpoint{1.053978in}{0.499444in}}%
\pgfpathlineto{\pgfqpoint{1.053978in}{0.564392in}}%
\pgfpathlineto{\pgfqpoint{0.990568in}{0.564392in}}%
\pgfpathlineto{\pgfqpoint{0.990568in}{0.499444in}}%
\pgfpathclose%
\pgfusepath{fill}%
\end{pgfscope}%
\begin{pgfscope}%
\pgfpathrectangle{\pgfqpoint{0.515000in}{0.499444in}}{\pgfqpoint{3.487500in}{1.155000in}}%
\pgfusepath{clip}%
\pgfsetbuttcap%
\pgfsetmiterjoin%
\definecolor{currentfill}{rgb}{0.000000,0.000000,0.000000}%
\pgfsetfillcolor{currentfill}%
\pgfsetlinewidth{0.000000pt}%
\definecolor{currentstroke}{rgb}{0.000000,0.000000,0.000000}%
\pgfsetstrokecolor{currentstroke}%
\pgfsetstrokeopacity{0.000000}%
\pgfsetdash{}{0pt}%
\pgfpathmoveto{\pgfqpoint{1.149091in}{0.499444in}}%
\pgfpathlineto{\pgfqpoint{1.212500in}{0.499444in}}%
\pgfpathlineto{\pgfqpoint{1.212500in}{0.571934in}}%
\pgfpathlineto{\pgfqpoint{1.149091in}{0.571934in}}%
\pgfpathlineto{\pgfqpoint{1.149091in}{0.499444in}}%
\pgfpathclose%
\pgfusepath{fill}%
\end{pgfscope}%
\begin{pgfscope}%
\pgfpathrectangle{\pgfqpoint{0.515000in}{0.499444in}}{\pgfqpoint{3.487500in}{1.155000in}}%
\pgfusepath{clip}%
\pgfsetbuttcap%
\pgfsetmiterjoin%
\definecolor{currentfill}{rgb}{0.000000,0.000000,0.000000}%
\pgfsetfillcolor{currentfill}%
\pgfsetlinewidth{0.000000pt}%
\definecolor{currentstroke}{rgb}{0.000000,0.000000,0.000000}%
\pgfsetstrokecolor{currentstroke}%
\pgfsetstrokeopacity{0.000000}%
\pgfsetdash{}{0pt}%
\pgfpathmoveto{\pgfqpoint{1.307614in}{0.499444in}}%
\pgfpathlineto{\pgfqpoint{1.371023in}{0.499444in}}%
\pgfpathlineto{\pgfqpoint{1.371023in}{0.570552in}}%
\pgfpathlineto{\pgfqpoint{1.307614in}{0.570552in}}%
\pgfpathlineto{\pgfqpoint{1.307614in}{0.499444in}}%
\pgfpathclose%
\pgfusepath{fill}%
\end{pgfscope}%
\begin{pgfscope}%
\pgfpathrectangle{\pgfqpoint{0.515000in}{0.499444in}}{\pgfqpoint{3.487500in}{1.155000in}}%
\pgfusepath{clip}%
\pgfsetbuttcap%
\pgfsetmiterjoin%
\definecolor{currentfill}{rgb}{0.000000,0.000000,0.000000}%
\pgfsetfillcolor{currentfill}%
\pgfsetlinewidth{0.000000pt}%
\definecolor{currentstroke}{rgb}{0.000000,0.000000,0.000000}%
\pgfsetstrokecolor{currentstroke}%
\pgfsetstrokeopacity{0.000000}%
\pgfsetdash{}{0pt}%
\pgfpathmoveto{\pgfqpoint{1.466137in}{0.499444in}}%
\pgfpathlineto{\pgfqpoint{1.529546in}{0.499444in}}%
\pgfpathlineto{\pgfqpoint{1.529546in}{0.568266in}}%
\pgfpathlineto{\pgfqpoint{1.466137in}{0.568266in}}%
\pgfpathlineto{\pgfqpoint{1.466137in}{0.499444in}}%
\pgfpathclose%
\pgfusepath{fill}%
\end{pgfscope}%
\begin{pgfscope}%
\pgfpathrectangle{\pgfqpoint{0.515000in}{0.499444in}}{\pgfqpoint{3.487500in}{1.155000in}}%
\pgfusepath{clip}%
\pgfsetbuttcap%
\pgfsetmiterjoin%
\definecolor{currentfill}{rgb}{0.000000,0.000000,0.000000}%
\pgfsetfillcolor{currentfill}%
\pgfsetlinewidth{0.000000pt}%
\definecolor{currentstroke}{rgb}{0.000000,0.000000,0.000000}%
\pgfsetstrokecolor{currentstroke}%
\pgfsetstrokeopacity{0.000000}%
\pgfsetdash{}{0pt}%
\pgfpathmoveto{\pgfqpoint{1.624659in}{0.499444in}}%
\pgfpathlineto{\pgfqpoint{1.688068in}{0.499444in}}%
\pgfpathlineto{\pgfqpoint{1.688068in}{0.560263in}}%
\pgfpathlineto{\pgfqpoint{1.624659in}{0.560263in}}%
\pgfpathlineto{\pgfqpoint{1.624659in}{0.499444in}}%
\pgfpathclose%
\pgfusepath{fill}%
\end{pgfscope}%
\begin{pgfscope}%
\pgfpathrectangle{\pgfqpoint{0.515000in}{0.499444in}}{\pgfqpoint{3.487500in}{1.155000in}}%
\pgfusepath{clip}%
\pgfsetbuttcap%
\pgfsetmiterjoin%
\definecolor{currentfill}{rgb}{0.000000,0.000000,0.000000}%
\pgfsetfillcolor{currentfill}%
\pgfsetlinewidth{0.000000pt}%
\definecolor{currentstroke}{rgb}{0.000000,0.000000,0.000000}%
\pgfsetstrokecolor{currentstroke}%
\pgfsetstrokeopacity{0.000000}%
\pgfsetdash{}{0pt}%
\pgfpathmoveto{\pgfqpoint{1.783182in}{0.499444in}}%
\pgfpathlineto{\pgfqpoint{1.846591in}{0.499444in}}%
\pgfpathlineto{\pgfqpoint{1.846591in}{0.551391in}}%
\pgfpathlineto{\pgfqpoint{1.783182in}{0.551391in}}%
\pgfpathlineto{\pgfqpoint{1.783182in}{0.499444in}}%
\pgfpathclose%
\pgfusepath{fill}%
\end{pgfscope}%
\begin{pgfscope}%
\pgfpathrectangle{\pgfqpoint{0.515000in}{0.499444in}}{\pgfqpoint{3.487500in}{1.155000in}}%
\pgfusepath{clip}%
\pgfsetbuttcap%
\pgfsetmiterjoin%
\definecolor{currentfill}{rgb}{0.000000,0.000000,0.000000}%
\pgfsetfillcolor{currentfill}%
\pgfsetlinewidth{0.000000pt}%
\definecolor{currentstroke}{rgb}{0.000000,0.000000,0.000000}%
\pgfsetstrokecolor{currentstroke}%
\pgfsetstrokeopacity{0.000000}%
\pgfsetdash{}{0pt}%
\pgfpathmoveto{\pgfqpoint{1.941705in}{0.499444in}}%
\pgfpathlineto{\pgfqpoint{2.005114in}{0.499444in}}%
\pgfpathlineto{\pgfqpoint{2.005114in}{0.541776in}}%
\pgfpathlineto{\pgfqpoint{1.941705in}{0.541776in}}%
\pgfpathlineto{\pgfqpoint{1.941705in}{0.499444in}}%
\pgfpathclose%
\pgfusepath{fill}%
\end{pgfscope}%
\begin{pgfscope}%
\pgfpathrectangle{\pgfqpoint{0.515000in}{0.499444in}}{\pgfqpoint{3.487500in}{1.155000in}}%
\pgfusepath{clip}%
\pgfsetbuttcap%
\pgfsetmiterjoin%
\definecolor{currentfill}{rgb}{0.000000,0.000000,0.000000}%
\pgfsetfillcolor{currentfill}%
\pgfsetlinewidth{0.000000pt}%
\definecolor{currentstroke}{rgb}{0.000000,0.000000,0.000000}%
\pgfsetstrokecolor{currentstroke}%
\pgfsetstrokeopacity{0.000000}%
\pgfsetdash{}{0pt}%
\pgfpathmoveto{\pgfqpoint{2.100228in}{0.499444in}}%
\pgfpathlineto{\pgfqpoint{2.163637in}{0.499444in}}%
\pgfpathlineto{\pgfqpoint{2.163637in}{0.536564in}}%
\pgfpathlineto{\pgfqpoint{2.100228in}{0.536564in}}%
\pgfpathlineto{\pgfqpoint{2.100228in}{0.499444in}}%
\pgfpathclose%
\pgfusepath{fill}%
\end{pgfscope}%
\begin{pgfscope}%
\pgfpathrectangle{\pgfqpoint{0.515000in}{0.499444in}}{\pgfqpoint{3.487500in}{1.155000in}}%
\pgfusepath{clip}%
\pgfsetbuttcap%
\pgfsetmiterjoin%
\definecolor{currentfill}{rgb}{0.000000,0.000000,0.000000}%
\pgfsetfillcolor{currentfill}%
\pgfsetlinewidth{0.000000pt}%
\definecolor{currentstroke}{rgb}{0.000000,0.000000,0.000000}%
\pgfsetstrokecolor{currentstroke}%
\pgfsetstrokeopacity{0.000000}%
\pgfsetdash{}{0pt}%
\pgfpathmoveto{\pgfqpoint{2.258750in}{0.499444in}}%
\pgfpathlineto{\pgfqpoint{2.322159in}{0.499444in}}%
\pgfpathlineto{\pgfqpoint{2.322159in}{0.530660in}}%
\pgfpathlineto{\pgfqpoint{2.258750in}{0.530660in}}%
\pgfpathlineto{\pgfqpoint{2.258750in}{0.499444in}}%
\pgfpathclose%
\pgfusepath{fill}%
\end{pgfscope}%
\begin{pgfscope}%
\pgfpathrectangle{\pgfqpoint{0.515000in}{0.499444in}}{\pgfqpoint{3.487500in}{1.155000in}}%
\pgfusepath{clip}%
\pgfsetbuttcap%
\pgfsetmiterjoin%
\definecolor{currentfill}{rgb}{0.000000,0.000000,0.000000}%
\pgfsetfillcolor{currentfill}%
\pgfsetlinewidth{0.000000pt}%
\definecolor{currentstroke}{rgb}{0.000000,0.000000,0.000000}%
\pgfsetstrokecolor{currentstroke}%
\pgfsetstrokeopacity{0.000000}%
\pgfsetdash{}{0pt}%
\pgfpathmoveto{\pgfqpoint{2.417273in}{0.499444in}}%
\pgfpathlineto{\pgfqpoint{2.480682in}{0.499444in}}%
\pgfpathlineto{\pgfqpoint{2.480682in}{0.527333in}}%
\pgfpathlineto{\pgfqpoint{2.417273in}{0.527333in}}%
\pgfpathlineto{\pgfqpoint{2.417273in}{0.499444in}}%
\pgfpathclose%
\pgfusepath{fill}%
\end{pgfscope}%
\begin{pgfscope}%
\pgfpathrectangle{\pgfqpoint{0.515000in}{0.499444in}}{\pgfqpoint{3.487500in}{1.155000in}}%
\pgfusepath{clip}%
\pgfsetbuttcap%
\pgfsetmiterjoin%
\definecolor{currentfill}{rgb}{0.000000,0.000000,0.000000}%
\pgfsetfillcolor{currentfill}%
\pgfsetlinewidth{0.000000pt}%
\definecolor{currentstroke}{rgb}{0.000000,0.000000,0.000000}%
\pgfsetstrokecolor{currentstroke}%
\pgfsetstrokeopacity{0.000000}%
\pgfsetdash{}{0pt}%
\pgfpathmoveto{\pgfqpoint{2.575796in}{0.499444in}}%
\pgfpathlineto{\pgfqpoint{2.639205in}{0.499444in}}%
\pgfpathlineto{\pgfqpoint{2.639205in}{0.523340in}}%
\pgfpathlineto{\pgfqpoint{2.575796in}{0.523340in}}%
\pgfpathlineto{\pgfqpoint{2.575796in}{0.499444in}}%
\pgfpathclose%
\pgfusepath{fill}%
\end{pgfscope}%
\begin{pgfscope}%
\pgfpathrectangle{\pgfqpoint{0.515000in}{0.499444in}}{\pgfqpoint{3.487500in}{1.155000in}}%
\pgfusepath{clip}%
\pgfsetbuttcap%
\pgfsetmiterjoin%
\definecolor{currentfill}{rgb}{0.000000,0.000000,0.000000}%
\pgfsetfillcolor{currentfill}%
\pgfsetlinewidth{0.000000pt}%
\definecolor{currentstroke}{rgb}{0.000000,0.000000,0.000000}%
\pgfsetstrokecolor{currentstroke}%
\pgfsetstrokeopacity{0.000000}%
\pgfsetdash{}{0pt}%
\pgfpathmoveto{\pgfqpoint{2.734318in}{0.499444in}}%
\pgfpathlineto{\pgfqpoint{2.797728in}{0.499444in}}%
\pgfpathlineto{\pgfqpoint{2.797728in}{0.519876in}}%
\pgfpathlineto{\pgfqpoint{2.734318in}{0.519876in}}%
\pgfpathlineto{\pgfqpoint{2.734318in}{0.499444in}}%
\pgfpathclose%
\pgfusepath{fill}%
\end{pgfscope}%
\begin{pgfscope}%
\pgfpathrectangle{\pgfqpoint{0.515000in}{0.499444in}}{\pgfqpoint{3.487500in}{1.155000in}}%
\pgfusepath{clip}%
\pgfsetbuttcap%
\pgfsetmiterjoin%
\definecolor{currentfill}{rgb}{0.000000,0.000000,0.000000}%
\pgfsetfillcolor{currentfill}%
\pgfsetlinewidth{0.000000pt}%
\definecolor{currentstroke}{rgb}{0.000000,0.000000,0.000000}%
\pgfsetstrokecolor{currentstroke}%
\pgfsetstrokeopacity{0.000000}%
\pgfsetdash{}{0pt}%
\pgfpathmoveto{\pgfqpoint{2.892841in}{0.499444in}}%
\pgfpathlineto{\pgfqpoint{2.956250in}{0.499444in}}%
\pgfpathlineto{\pgfqpoint{2.956250in}{0.517257in}}%
\pgfpathlineto{\pgfqpoint{2.892841in}{0.517257in}}%
\pgfpathlineto{\pgfqpoint{2.892841in}{0.499444in}}%
\pgfpathclose%
\pgfusepath{fill}%
\end{pgfscope}%
\begin{pgfscope}%
\pgfpathrectangle{\pgfqpoint{0.515000in}{0.499444in}}{\pgfqpoint{3.487500in}{1.155000in}}%
\pgfusepath{clip}%
\pgfsetbuttcap%
\pgfsetmiterjoin%
\definecolor{currentfill}{rgb}{0.000000,0.000000,0.000000}%
\pgfsetfillcolor{currentfill}%
\pgfsetlinewidth{0.000000pt}%
\definecolor{currentstroke}{rgb}{0.000000,0.000000,0.000000}%
\pgfsetstrokecolor{currentstroke}%
\pgfsetstrokeopacity{0.000000}%
\pgfsetdash{}{0pt}%
\pgfpathmoveto{\pgfqpoint{3.051364in}{0.499444in}}%
\pgfpathlineto{\pgfqpoint{3.114773in}{0.499444in}}%
\pgfpathlineto{\pgfqpoint{3.114773in}{0.516379in}}%
\pgfpathlineto{\pgfqpoint{3.051364in}{0.516379in}}%
\pgfpathlineto{\pgfqpoint{3.051364in}{0.499444in}}%
\pgfpathclose%
\pgfusepath{fill}%
\end{pgfscope}%
\begin{pgfscope}%
\pgfpathrectangle{\pgfqpoint{0.515000in}{0.499444in}}{\pgfqpoint{3.487500in}{1.155000in}}%
\pgfusepath{clip}%
\pgfsetbuttcap%
\pgfsetmiterjoin%
\definecolor{currentfill}{rgb}{0.000000,0.000000,0.000000}%
\pgfsetfillcolor{currentfill}%
\pgfsetlinewidth{0.000000pt}%
\definecolor{currentstroke}{rgb}{0.000000,0.000000,0.000000}%
\pgfsetstrokecolor{currentstroke}%
\pgfsetstrokeopacity{0.000000}%
\pgfsetdash{}{0pt}%
\pgfpathmoveto{\pgfqpoint{3.209887in}{0.499444in}}%
\pgfpathlineto{\pgfqpoint{3.273296in}{0.499444in}}%
\pgfpathlineto{\pgfqpoint{3.273296in}{0.515415in}}%
\pgfpathlineto{\pgfqpoint{3.209887in}{0.515415in}}%
\pgfpathlineto{\pgfqpoint{3.209887in}{0.499444in}}%
\pgfpathclose%
\pgfusepath{fill}%
\end{pgfscope}%
\begin{pgfscope}%
\pgfpathrectangle{\pgfqpoint{0.515000in}{0.499444in}}{\pgfqpoint{3.487500in}{1.155000in}}%
\pgfusepath{clip}%
\pgfsetbuttcap%
\pgfsetmiterjoin%
\definecolor{currentfill}{rgb}{0.000000,0.000000,0.000000}%
\pgfsetfillcolor{currentfill}%
\pgfsetlinewidth{0.000000pt}%
\definecolor{currentstroke}{rgb}{0.000000,0.000000,0.000000}%
\pgfsetstrokecolor{currentstroke}%
\pgfsetstrokeopacity{0.000000}%
\pgfsetdash{}{0pt}%
\pgfpathmoveto{\pgfqpoint{3.368409in}{0.499444in}}%
\pgfpathlineto{\pgfqpoint{3.431818in}{0.499444in}}%
\pgfpathlineto{\pgfqpoint{3.431818in}{0.512429in}}%
\pgfpathlineto{\pgfqpoint{3.368409in}{0.512429in}}%
\pgfpathlineto{\pgfqpoint{3.368409in}{0.499444in}}%
\pgfpathclose%
\pgfusepath{fill}%
\end{pgfscope}%
\begin{pgfscope}%
\pgfpathrectangle{\pgfqpoint{0.515000in}{0.499444in}}{\pgfqpoint{3.487500in}{1.155000in}}%
\pgfusepath{clip}%
\pgfsetbuttcap%
\pgfsetmiterjoin%
\definecolor{currentfill}{rgb}{0.000000,0.000000,0.000000}%
\pgfsetfillcolor{currentfill}%
\pgfsetlinewidth{0.000000pt}%
\definecolor{currentstroke}{rgb}{0.000000,0.000000,0.000000}%
\pgfsetstrokecolor{currentstroke}%
\pgfsetstrokeopacity{0.000000}%
\pgfsetdash{}{0pt}%
\pgfpathmoveto{\pgfqpoint{3.526932in}{0.499444in}}%
\pgfpathlineto{\pgfqpoint{3.590341in}{0.499444in}}%
\pgfpathlineto{\pgfqpoint{3.590341in}{0.505740in}}%
\pgfpathlineto{\pgfqpoint{3.526932in}{0.505740in}}%
\pgfpathlineto{\pgfqpoint{3.526932in}{0.499444in}}%
\pgfpathclose%
\pgfusepath{fill}%
\end{pgfscope}%
\begin{pgfscope}%
\pgfpathrectangle{\pgfqpoint{0.515000in}{0.499444in}}{\pgfqpoint{3.487500in}{1.155000in}}%
\pgfusepath{clip}%
\pgfsetbuttcap%
\pgfsetmiterjoin%
\definecolor{currentfill}{rgb}{0.000000,0.000000,0.000000}%
\pgfsetfillcolor{currentfill}%
\pgfsetlinewidth{0.000000pt}%
\definecolor{currentstroke}{rgb}{0.000000,0.000000,0.000000}%
\pgfsetstrokecolor{currentstroke}%
\pgfsetstrokeopacity{0.000000}%
\pgfsetdash{}{0pt}%
\pgfpathmoveto{\pgfqpoint{3.685455in}{0.499444in}}%
\pgfpathlineto{\pgfqpoint{3.748864in}{0.499444in}}%
\pgfpathlineto{\pgfqpoint{3.748864in}{0.501295in}}%
\pgfpathlineto{\pgfqpoint{3.685455in}{0.501295in}}%
\pgfpathlineto{\pgfqpoint{3.685455in}{0.499444in}}%
\pgfpathclose%
\pgfusepath{fill}%
\end{pgfscope}%
\begin{pgfscope}%
\pgfpathrectangle{\pgfqpoint{0.515000in}{0.499444in}}{\pgfqpoint{3.487500in}{1.155000in}}%
\pgfusepath{clip}%
\pgfsetbuttcap%
\pgfsetmiterjoin%
\definecolor{currentfill}{rgb}{0.000000,0.000000,0.000000}%
\pgfsetfillcolor{currentfill}%
\pgfsetlinewidth{0.000000pt}%
\definecolor{currentstroke}{rgb}{0.000000,0.000000,0.000000}%
\pgfsetstrokecolor{currentstroke}%
\pgfsetstrokeopacity{0.000000}%
\pgfsetdash{}{0pt}%
\pgfpathmoveto{\pgfqpoint{3.843978in}{0.499444in}}%
\pgfpathlineto{\pgfqpoint{3.907387in}{0.499444in}}%
\pgfpathlineto{\pgfqpoint{3.907387in}{0.499555in}}%
\pgfpathlineto{\pgfqpoint{3.843978in}{0.499555in}}%
\pgfpathlineto{\pgfqpoint{3.843978in}{0.499444in}}%
\pgfpathclose%
\pgfusepath{fill}%
\end{pgfscope}%
\begin{pgfscope}%
\pgfsetbuttcap%
\pgfsetroundjoin%
\definecolor{currentfill}{rgb}{0.000000,0.000000,0.000000}%
\pgfsetfillcolor{currentfill}%
\pgfsetlinewidth{0.803000pt}%
\definecolor{currentstroke}{rgb}{0.000000,0.000000,0.000000}%
\pgfsetstrokecolor{currentstroke}%
\pgfsetdash{}{0pt}%
\pgfsys@defobject{currentmarker}{\pgfqpoint{0.000000in}{-0.048611in}}{\pgfqpoint{0.000000in}{0.000000in}}{%
\pgfpathmoveto{\pgfqpoint{0.000000in}{0.000000in}}%
\pgfpathlineto{\pgfqpoint{0.000000in}{-0.048611in}}%
\pgfusepath{stroke,fill}%
}%
\begin{pgfscope}%
\pgfsys@transformshift{0.515000in}{0.499444in}%
\pgfsys@useobject{currentmarker}{}%
\end{pgfscope}%
\end{pgfscope}%
\begin{pgfscope}%
\pgfsetbuttcap%
\pgfsetroundjoin%
\definecolor{currentfill}{rgb}{0.000000,0.000000,0.000000}%
\pgfsetfillcolor{currentfill}%
\pgfsetlinewidth{0.803000pt}%
\definecolor{currentstroke}{rgb}{0.000000,0.000000,0.000000}%
\pgfsetstrokecolor{currentstroke}%
\pgfsetdash{}{0pt}%
\pgfsys@defobject{currentmarker}{\pgfqpoint{0.000000in}{-0.048611in}}{\pgfqpoint{0.000000in}{0.000000in}}{%
\pgfpathmoveto{\pgfqpoint{0.000000in}{0.000000in}}%
\pgfpathlineto{\pgfqpoint{0.000000in}{-0.048611in}}%
\pgfusepath{stroke,fill}%
}%
\begin{pgfscope}%
\pgfsys@transformshift{0.673523in}{0.499444in}%
\pgfsys@useobject{currentmarker}{}%
\end{pgfscope}%
\end{pgfscope}%
\begin{pgfscope}%
\definecolor{textcolor}{rgb}{0.000000,0.000000,0.000000}%
\pgfsetstrokecolor{textcolor}%
\pgfsetfillcolor{textcolor}%
\pgftext[x=0.673523in,y=0.402222in,,top]{\color{textcolor}\rmfamily\fontsize{10.000000}{12.000000}\selectfont 0.0}%
\end{pgfscope}%
\begin{pgfscope}%
\pgfsetbuttcap%
\pgfsetroundjoin%
\definecolor{currentfill}{rgb}{0.000000,0.000000,0.000000}%
\pgfsetfillcolor{currentfill}%
\pgfsetlinewidth{0.803000pt}%
\definecolor{currentstroke}{rgb}{0.000000,0.000000,0.000000}%
\pgfsetstrokecolor{currentstroke}%
\pgfsetdash{}{0pt}%
\pgfsys@defobject{currentmarker}{\pgfqpoint{0.000000in}{-0.048611in}}{\pgfqpoint{0.000000in}{0.000000in}}{%
\pgfpathmoveto{\pgfqpoint{0.000000in}{0.000000in}}%
\pgfpathlineto{\pgfqpoint{0.000000in}{-0.048611in}}%
\pgfusepath{stroke,fill}%
}%
\begin{pgfscope}%
\pgfsys@transformshift{0.832046in}{0.499444in}%
\pgfsys@useobject{currentmarker}{}%
\end{pgfscope}%
\end{pgfscope}%
\begin{pgfscope}%
\pgfsetbuttcap%
\pgfsetroundjoin%
\definecolor{currentfill}{rgb}{0.000000,0.000000,0.000000}%
\pgfsetfillcolor{currentfill}%
\pgfsetlinewidth{0.803000pt}%
\definecolor{currentstroke}{rgb}{0.000000,0.000000,0.000000}%
\pgfsetstrokecolor{currentstroke}%
\pgfsetdash{}{0pt}%
\pgfsys@defobject{currentmarker}{\pgfqpoint{0.000000in}{-0.048611in}}{\pgfqpoint{0.000000in}{0.000000in}}{%
\pgfpathmoveto{\pgfqpoint{0.000000in}{0.000000in}}%
\pgfpathlineto{\pgfqpoint{0.000000in}{-0.048611in}}%
\pgfusepath{stroke,fill}%
}%
\begin{pgfscope}%
\pgfsys@transformshift{0.990568in}{0.499444in}%
\pgfsys@useobject{currentmarker}{}%
\end{pgfscope}%
\end{pgfscope}%
\begin{pgfscope}%
\definecolor{textcolor}{rgb}{0.000000,0.000000,0.000000}%
\pgfsetstrokecolor{textcolor}%
\pgfsetfillcolor{textcolor}%
\pgftext[x=0.990568in,y=0.402222in,,top]{\color{textcolor}\rmfamily\fontsize{10.000000}{12.000000}\selectfont 0.1}%
\end{pgfscope}%
\begin{pgfscope}%
\pgfsetbuttcap%
\pgfsetroundjoin%
\definecolor{currentfill}{rgb}{0.000000,0.000000,0.000000}%
\pgfsetfillcolor{currentfill}%
\pgfsetlinewidth{0.803000pt}%
\definecolor{currentstroke}{rgb}{0.000000,0.000000,0.000000}%
\pgfsetstrokecolor{currentstroke}%
\pgfsetdash{}{0pt}%
\pgfsys@defobject{currentmarker}{\pgfqpoint{0.000000in}{-0.048611in}}{\pgfqpoint{0.000000in}{0.000000in}}{%
\pgfpathmoveto{\pgfqpoint{0.000000in}{0.000000in}}%
\pgfpathlineto{\pgfqpoint{0.000000in}{-0.048611in}}%
\pgfusepath{stroke,fill}%
}%
\begin{pgfscope}%
\pgfsys@transformshift{1.149091in}{0.499444in}%
\pgfsys@useobject{currentmarker}{}%
\end{pgfscope}%
\end{pgfscope}%
\begin{pgfscope}%
\pgfsetbuttcap%
\pgfsetroundjoin%
\definecolor{currentfill}{rgb}{0.000000,0.000000,0.000000}%
\pgfsetfillcolor{currentfill}%
\pgfsetlinewidth{0.803000pt}%
\definecolor{currentstroke}{rgb}{0.000000,0.000000,0.000000}%
\pgfsetstrokecolor{currentstroke}%
\pgfsetdash{}{0pt}%
\pgfsys@defobject{currentmarker}{\pgfqpoint{0.000000in}{-0.048611in}}{\pgfqpoint{0.000000in}{0.000000in}}{%
\pgfpathmoveto{\pgfqpoint{0.000000in}{0.000000in}}%
\pgfpathlineto{\pgfqpoint{0.000000in}{-0.048611in}}%
\pgfusepath{stroke,fill}%
}%
\begin{pgfscope}%
\pgfsys@transformshift{1.307614in}{0.499444in}%
\pgfsys@useobject{currentmarker}{}%
\end{pgfscope}%
\end{pgfscope}%
\begin{pgfscope}%
\definecolor{textcolor}{rgb}{0.000000,0.000000,0.000000}%
\pgfsetstrokecolor{textcolor}%
\pgfsetfillcolor{textcolor}%
\pgftext[x=1.307614in,y=0.402222in,,top]{\color{textcolor}\rmfamily\fontsize{10.000000}{12.000000}\selectfont 0.2}%
\end{pgfscope}%
\begin{pgfscope}%
\pgfsetbuttcap%
\pgfsetroundjoin%
\definecolor{currentfill}{rgb}{0.000000,0.000000,0.000000}%
\pgfsetfillcolor{currentfill}%
\pgfsetlinewidth{0.803000pt}%
\definecolor{currentstroke}{rgb}{0.000000,0.000000,0.000000}%
\pgfsetstrokecolor{currentstroke}%
\pgfsetdash{}{0pt}%
\pgfsys@defobject{currentmarker}{\pgfqpoint{0.000000in}{-0.048611in}}{\pgfqpoint{0.000000in}{0.000000in}}{%
\pgfpathmoveto{\pgfqpoint{0.000000in}{0.000000in}}%
\pgfpathlineto{\pgfqpoint{0.000000in}{-0.048611in}}%
\pgfusepath{stroke,fill}%
}%
\begin{pgfscope}%
\pgfsys@transformshift{1.466137in}{0.499444in}%
\pgfsys@useobject{currentmarker}{}%
\end{pgfscope}%
\end{pgfscope}%
\begin{pgfscope}%
\pgfsetbuttcap%
\pgfsetroundjoin%
\definecolor{currentfill}{rgb}{0.000000,0.000000,0.000000}%
\pgfsetfillcolor{currentfill}%
\pgfsetlinewidth{0.803000pt}%
\definecolor{currentstroke}{rgb}{0.000000,0.000000,0.000000}%
\pgfsetstrokecolor{currentstroke}%
\pgfsetdash{}{0pt}%
\pgfsys@defobject{currentmarker}{\pgfqpoint{0.000000in}{-0.048611in}}{\pgfqpoint{0.000000in}{0.000000in}}{%
\pgfpathmoveto{\pgfqpoint{0.000000in}{0.000000in}}%
\pgfpathlineto{\pgfqpoint{0.000000in}{-0.048611in}}%
\pgfusepath{stroke,fill}%
}%
\begin{pgfscope}%
\pgfsys@transformshift{1.624659in}{0.499444in}%
\pgfsys@useobject{currentmarker}{}%
\end{pgfscope}%
\end{pgfscope}%
\begin{pgfscope}%
\definecolor{textcolor}{rgb}{0.000000,0.000000,0.000000}%
\pgfsetstrokecolor{textcolor}%
\pgfsetfillcolor{textcolor}%
\pgftext[x=1.624659in,y=0.402222in,,top]{\color{textcolor}\rmfamily\fontsize{10.000000}{12.000000}\selectfont 0.3}%
\end{pgfscope}%
\begin{pgfscope}%
\pgfsetbuttcap%
\pgfsetroundjoin%
\definecolor{currentfill}{rgb}{0.000000,0.000000,0.000000}%
\pgfsetfillcolor{currentfill}%
\pgfsetlinewidth{0.803000pt}%
\definecolor{currentstroke}{rgb}{0.000000,0.000000,0.000000}%
\pgfsetstrokecolor{currentstroke}%
\pgfsetdash{}{0pt}%
\pgfsys@defobject{currentmarker}{\pgfqpoint{0.000000in}{-0.048611in}}{\pgfqpoint{0.000000in}{0.000000in}}{%
\pgfpathmoveto{\pgfqpoint{0.000000in}{0.000000in}}%
\pgfpathlineto{\pgfqpoint{0.000000in}{-0.048611in}}%
\pgfusepath{stroke,fill}%
}%
\begin{pgfscope}%
\pgfsys@transformshift{1.783182in}{0.499444in}%
\pgfsys@useobject{currentmarker}{}%
\end{pgfscope}%
\end{pgfscope}%
\begin{pgfscope}%
\pgfsetbuttcap%
\pgfsetroundjoin%
\definecolor{currentfill}{rgb}{0.000000,0.000000,0.000000}%
\pgfsetfillcolor{currentfill}%
\pgfsetlinewidth{0.803000pt}%
\definecolor{currentstroke}{rgb}{0.000000,0.000000,0.000000}%
\pgfsetstrokecolor{currentstroke}%
\pgfsetdash{}{0pt}%
\pgfsys@defobject{currentmarker}{\pgfqpoint{0.000000in}{-0.048611in}}{\pgfqpoint{0.000000in}{0.000000in}}{%
\pgfpathmoveto{\pgfqpoint{0.000000in}{0.000000in}}%
\pgfpathlineto{\pgfqpoint{0.000000in}{-0.048611in}}%
\pgfusepath{stroke,fill}%
}%
\begin{pgfscope}%
\pgfsys@transformshift{1.941705in}{0.499444in}%
\pgfsys@useobject{currentmarker}{}%
\end{pgfscope}%
\end{pgfscope}%
\begin{pgfscope}%
\definecolor{textcolor}{rgb}{0.000000,0.000000,0.000000}%
\pgfsetstrokecolor{textcolor}%
\pgfsetfillcolor{textcolor}%
\pgftext[x=1.941705in,y=0.402222in,,top]{\color{textcolor}\rmfamily\fontsize{10.000000}{12.000000}\selectfont 0.4}%
\end{pgfscope}%
\begin{pgfscope}%
\pgfsetbuttcap%
\pgfsetroundjoin%
\definecolor{currentfill}{rgb}{0.000000,0.000000,0.000000}%
\pgfsetfillcolor{currentfill}%
\pgfsetlinewidth{0.803000pt}%
\definecolor{currentstroke}{rgb}{0.000000,0.000000,0.000000}%
\pgfsetstrokecolor{currentstroke}%
\pgfsetdash{}{0pt}%
\pgfsys@defobject{currentmarker}{\pgfqpoint{0.000000in}{-0.048611in}}{\pgfqpoint{0.000000in}{0.000000in}}{%
\pgfpathmoveto{\pgfqpoint{0.000000in}{0.000000in}}%
\pgfpathlineto{\pgfqpoint{0.000000in}{-0.048611in}}%
\pgfusepath{stroke,fill}%
}%
\begin{pgfscope}%
\pgfsys@transformshift{2.100228in}{0.499444in}%
\pgfsys@useobject{currentmarker}{}%
\end{pgfscope}%
\end{pgfscope}%
\begin{pgfscope}%
\pgfsetbuttcap%
\pgfsetroundjoin%
\definecolor{currentfill}{rgb}{0.000000,0.000000,0.000000}%
\pgfsetfillcolor{currentfill}%
\pgfsetlinewidth{0.803000pt}%
\definecolor{currentstroke}{rgb}{0.000000,0.000000,0.000000}%
\pgfsetstrokecolor{currentstroke}%
\pgfsetdash{}{0pt}%
\pgfsys@defobject{currentmarker}{\pgfqpoint{0.000000in}{-0.048611in}}{\pgfqpoint{0.000000in}{0.000000in}}{%
\pgfpathmoveto{\pgfqpoint{0.000000in}{0.000000in}}%
\pgfpathlineto{\pgfqpoint{0.000000in}{-0.048611in}}%
\pgfusepath{stroke,fill}%
}%
\begin{pgfscope}%
\pgfsys@transformshift{2.258750in}{0.499444in}%
\pgfsys@useobject{currentmarker}{}%
\end{pgfscope}%
\end{pgfscope}%
\begin{pgfscope}%
\definecolor{textcolor}{rgb}{0.000000,0.000000,0.000000}%
\pgfsetstrokecolor{textcolor}%
\pgfsetfillcolor{textcolor}%
\pgftext[x=2.258750in,y=0.402222in,,top]{\color{textcolor}\rmfamily\fontsize{10.000000}{12.000000}\selectfont 0.5}%
\end{pgfscope}%
\begin{pgfscope}%
\pgfsetbuttcap%
\pgfsetroundjoin%
\definecolor{currentfill}{rgb}{0.000000,0.000000,0.000000}%
\pgfsetfillcolor{currentfill}%
\pgfsetlinewidth{0.803000pt}%
\definecolor{currentstroke}{rgb}{0.000000,0.000000,0.000000}%
\pgfsetstrokecolor{currentstroke}%
\pgfsetdash{}{0pt}%
\pgfsys@defobject{currentmarker}{\pgfqpoint{0.000000in}{-0.048611in}}{\pgfqpoint{0.000000in}{0.000000in}}{%
\pgfpathmoveto{\pgfqpoint{0.000000in}{0.000000in}}%
\pgfpathlineto{\pgfqpoint{0.000000in}{-0.048611in}}%
\pgfusepath{stroke,fill}%
}%
\begin{pgfscope}%
\pgfsys@transformshift{2.417273in}{0.499444in}%
\pgfsys@useobject{currentmarker}{}%
\end{pgfscope}%
\end{pgfscope}%
\begin{pgfscope}%
\pgfsetbuttcap%
\pgfsetroundjoin%
\definecolor{currentfill}{rgb}{0.000000,0.000000,0.000000}%
\pgfsetfillcolor{currentfill}%
\pgfsetlinewidth{0.803000pt}%
\definecolor{currentstroke}{rgb}{0.000000,0.000000,0.000000}%
\pgfsetstrokecolor{currentstroke}%
\pgfsetdash{}{0pt}%
\pgfsys@defobject{currentmarker}{\pgfqpoint{0.000000in}{-0.048611in}}{\pgfqpoint{0.000000in}{0.000000in}}{%
\pgfpathmoveto{\pgfqpoint{0.000000in}{0.000000in}}%
\pgfpathlineto{\pgfqpoint{0.000000in}{-0.048611in}}%
\pgfusepath{stroke,fill}%
}%
\begin{pgfscope}%
\pgfsys@transformshift{2.575796in}{0.499444in}%
\pgfsys@useobject{currentmarker}{}%
\end{pgfscope}%
\end{pgfscope}%
\begin{pgfscope}%
\definecolor{textcolor}{rgb}{0.000000,0.000000,0.000000}%
\pgfsetstrokecolor{textcolor}%
\pgfsetfillcolor{textcolor}%
\pgftext[x=2.575796in,y=0.402222in,,top]{\color{textcolor}\rmfamily\fontsize{10.000000}{12.000000}\selectfont 0.6}%
\end{pgfscope}%
\begin{pgfscope}%
\pgfsetbuttcap%
\pgfsetroundjoin%
\definecolor{currentfill}{rgb}{0.000000,0.000000,0.000000}%
\pgfsetfillcolor{currentfill}%
\pgfsetlinewidth{0.803000pt}%
\definecolor{currentstroke}{rgb}{0.000000,0.000000,0.000000}%
\pgfsetstrokecolor{currentstroke}%
\pgfsetdash{}{0pt}%
\pgfsys@defobject{currentmarker}{\pgfqpoint{0.000000in}{-0.048611in}}{\pgfqpoint{0.000000in}{0.000000in}}{%
\pgfpathmoveto{\pgfqpoint{0.000000in}{0.000000in}}%
\pgfpathlineto{\pgfqpoint{0.000000in}{-0.048611in}}%
\pgfusepath{stroke,fill}%
}%
\begin{pgfscope}%
\pgfsys@transformshift{2.734318in}{0.499444in}%
\pgfsys@useobject{currentmarker}{}%
\end{pgfscope}%
\end{pgfscope}%
\begin{pgfscope}%
\pgfsetbuttcap%
\pgfsetroundjoin%
\definecolor{currentfill}{rgb}{0.000000,0.000000,0.000000}%
\pgfsetfillcolor{currentfill}%
\pgfsetlinewidth{0.803000pt}%
\definecolor{currentstroke}{rgb}{0.000000,0.000000,0.000000}%
\pgfsetstrokecolor{currentstroke}%
\pgfsetdash{}{0pt}%
\pgfsys@defobject{currentmarker}{\pgfqpoint{0.000000in}{-0.048611in}}{\pgfqpoint{0.000000in}{0.000000in}}{%
\pgfpathmoveto{\pgfqpoint{0.000000in}{0.000000in}}%
\pgfpathlineto{\pgfqpoint{0.000000in}{-0.048611in}}%
\pgfusepath{stroke,fill}%
}%
\begin{pgfscope}%
\pgfsys@transformshift{2.892841in}{0.499444in}%
\pgfsys@useobject{currentmarker}{}%
\end{pgfscope}%
\end{pgfscope}%
\begin{pgfscope}%
\definecolor{textcolor}{rgb}{0.000000,0.000000,0.000000}%
\pgfsetstrokecolor{textcolor}%
\pgfsetfillcolor{textcolor}%
\pgftext[x=2.892841in,y=0.402222in,,top]{\color{textcolor}\rmfamily\fontsize{10.000000}{12.000000}\selectfont 0.7}%
\end{pgfscope}%
\begin{pgfscope}%
\pgfsetbuttcap%
\pgfsetroundjoin%
\definecolor{currentfill}{rgb}{0.000000,0.000000,0.000000}%
\pgfsetfillcolor{currentfill}%
\pgfsetlinewidth{0.803000pt}%
\definecolor{currentstroke}{rgb}{0.000000,0.000000,0.000000}%
\pgfsetstrokecolor{currentstroke}%
\pgfsetdash{}{0pt}%
\pgfsys@defobject{currentmarker}{\pgfqpoint{0.000000in}{-0.048611in}}{\pgfqpoint{0.000000in}{0.000000in}}{%
\pgfpathmoveto{\pgfqpoint{0.000000in}{0.000000in}}%
\pgfpathlineto{\pgfqpoint{0.000000in}{-0.048611in}}%
\pgfusepath{stroke,fill}%
}%
\begin{pgfscope}%
\pgfsys@transformshift{3.051364in}{0.499444in}%
\pgfsys@useobject{currentmarker}{}%
\end{pgfscope}%
\end{pgfscope}%
\begin{pgfscope}%
\pgfsetbuttcap%
\pgfsetroundjoin%
\definecolor{currentfill}{rgb}{0.000000,0.000000,0.000000}%
\pgfsetfillcolor{currentfill}%
\pgfsetlinewidth{0.803000pt}%
\definecolor{currentstroke}{rgb}{0.000000,0.000000,0.000000}%
\pgfsetstrokecolor{currentstroke}%
\pgfsetdash{}{0pt}%
\pgfsys@defobject{currentmarker}{\pgfqpoint{0.000000in}{-0.048611in}}{\pgfqpoint{0.000000in}{0.000000in}}{%
\pgfpathmoveto{\pgfqpoint{0.000000in}{0.000000in}}%
\pgfpathlineto{\pgfqpoint{0.000000in}{-0.048611in}}%
\pgfusepath{stroke,fill}%
}%
\begin{pgfscope}%
\pgfsys@transformshift{3.209887in}{0.499444in}%
\pgfsys@useobject{currentmarker}{}%
\end{pgfscope}%
\end{pgfscope}%
\begin{pgfscope}%
\definecolor{textcolor}{rgb}{0.000000,0.000000,0.000000}%
\pgfsetstrokecolor{textcolor}%
\pgfsetfillcolor{textcolor}%
\pgftext[x=3.209887in,y=0.402222in,,top]{\color{textcolor}\rmfamily\fontsize{10.000000}{12.000000}\selectfont 0.8}%
\end{pgfscope}%
\begin{pgfscope}%
\pgfsetbuttcap%
\pgfsetroundjoin%
\definecolor{currentfill}{rgb}{0.000000,0.000000,0.000000}%
\pgfsetfillcolor{currentfill}%
\pgfsetlinewidth{0.803000pt}%
\definecolor{currentstroke}{rgb}{0.000000,0.000000,0.000000}%
\pgfsetstrokecolor{currentstroke}%
\pgfsetdash{}{0pt}%
\pgfsys@defobject{currentmarker}{\pgfqpoint{0.000000in}{-0.048611in}}{\pgfqpoint{0.000000in}{0.000000in}}{%
\pgfpathmoveto{\pgfqpoint{0.000000in}{0.000000in}}%
\pgfpathlineto{\pgfqpoint{0.000000in}{-0.048611in}}%
\pgfusepath{stroke,fill}%
}%
\begin{pgfscope}%
\pgfsys@transformshift{3.368409in}{0.499444in}%
\pgfsys@useobject{currentmarker}{}%
\end{pgfscope}%
\end{pgfscope}%
\begin{pgfscope}%
\pgfsetbuttcap%
\pgfsetroundjoin%
\definecolor{currentfill}{rgb}{0.000000,0.000000,0.000000}%
\pgfsetfillcolor{currentfill}%
\pgfsetlinewidth{0.803000pt}%
\definecolor{currentstroke}{rgb}{0.000000,0.000000,0.000000}%
\pgfsetstrokecolor{currentstroke}%
\pgfsetdash{}{0pt}%
\pgfsys@defobject{currentmarker}{\pgfqpoint{0.000000in}{-0.048611in}}{\pgfqpoint{0.000000in}{0.000000in}}{%
\pgfpathmoveto{\pgfqpoint{0.000000in}{0.000000in}}%
\pgfpathlineto{\pgfqpoint{0.000000in}{-0.048611in}}%
\pgfusepath{stroke,fill}%
}%
\begin{pgfscope}%
\pgfsys@transformshift{3.526932in}{0.499444in}%
\pgfsys@useobject{currentmarker}{}%
\end{pgfscope}%
\end{pgfscope}%
\begin{pgfscope}%
\definecolor{textcolor}{rgb}{0.000000,0.000000,0.000000}%
\pgfsetstrokecolor{textcolor}%
\pgfsetfillcolor{textcolor}%
\pgftext[x=3.526932in,y=0.402222in,,top]{\color{textcolor}\rmfamily\fontsize{10.000000}{12.000000}\selectfont 0.9}%
\end{pgfscope}%
\begin{pgfscope}%
\pgfsetbuttcap%
\pgfsetroundjoin%
\definecolor{currentfill}{rgb}{0.000000,0.000000,0.000000}%
\pgfsetfillcolor{currentfill}%
\pgfsetlinewidth{0.803000pt}%
\definecolor{currentstroke}{rgb}{0.000000,0.000000,0.000000}%
\pgfsetstrokecolor{currentstroke}%
\pgfsetdash{}{0pt}%
\pgfsys@defobject{currentmarker}{\pgfqpoint{0.000000in}{-0.048611in}}{\pgfqpoint{0.000000in}{0.000000in}}{%
\pgfpathmoveto{\pgfqpoint{0.000000in}{0.000000in}}%
\pgfpathlineto{\pgfqpoint{0.000000in}{-0.048611in}}%
\pgfusepath{stroke,fill}%
}%
\begin{pgfscope}%
\pgfsys@transformshift{3.685455in}{0.499444in}%
\pgfsys@useobject{currentmarker}{}%
\end{pgfscope}%
\end{pgfscope}%
\begin{pgfscope}%
\pgfsetbuttcap%
\pgfsetroundjoin%
\definecolor{currentfill}{rgb}{0.000000,0.000000,0.000000}%
\pgfsetfillcolor{currentfill}%
\pgfsetlinewidth{0.803000pt}%
\definecolor{currentstroke}{rgb}{0.000000,0.000000,0.000000}%
\pgfsetstrokecolor{currentstroke}%
\pgfsetdash{}{0pt}%
\pgfsys@defobject{currentmarker}{\pgfqpoint{0.000000in}{-0.048611in}}{\pgfqpoint{0.000000in}{0.000000in}}{%
\pgfpathmoveto{\pgfqpoint{0.000000in}{0.000000in}}%
\pgfpathlineto{\pgfqpoint{0.000000in}{-0.048611in}}%
\pgfusepath{stroke,fill}%
}%
\begin{pgfscope}%
\pgfsys@transformshift{3.843978in}{0.499444in}%
\pgfsys@useobject{currentmarker}{}%
\end{pgfscope}%
\end{pgfscope}%
\begin{pgfscope}%
\definecolor{textcolor}{rgb}{0.000000,0.000000,0.000000}%
\pgfsetstrokecolor{textcolor}%
\pgfsetfillcolor{textcolor}%
\pgftext[x=3.843978in,y=0.402222in,,top]{\color{textcolor}\rmfamily\fontsize{10.000000}{12.000000}\selectfont 1.0}%
\end{pgfscope}%
\begin{pgfscope}%
\pgfsetbuttcap%
\pgfsetroundjoin%
\definecolor{currentfill}{rgb}{0.000000,0.000000,0.000000}%
\pgfsetfillcolor{currentfill}%
\pgfsetlinewidth{0.803000pt}%
\definecolor{currentstroke}{rgb}{0.000000,0.000000,0.000000}%
\pgfsetstrokecolor{currentstroke}%
\pgfsetdash{}{0pt}%
\pgfsys@defobject{currentmarker}{\pgfqpoint{0.000000in}{-0.048611in}}{\pgfqpoint{0.000000in}{0.000000in}}{%
\pgfpathmoveto{\pgfqpoint{0.000000in}{0.000000in}}%
\pgfpathlineto{\pgfqpoint{0.000000in}{-0.048611in}}%
\pgfusepath{stroke,fill}%
}%
\begin{pgfscope}%
\pgfsys@transformshift{4.002500in}{0.499444in}%
\pgfsys@useobject{currentmarker}{}%
\end{pgfscope}%
\end{pgfscope}%
\begin{pgfscope}%
\definecolor{textcolor}{rgb}{0.000000,0.000000,0.000000}%
\pgfsetstrokecolor{textcolor}%
\pgfsetfillcolor{textcolor}%
\pgftext[x=2.258750in,y=0.223333in,,top]{\color{textcolor}\rmfamily\fontsize{10.000000}{12.000000}\selectfont \(\displaystyle p\)}%
\end{pgfscope}%
\begin{pgfscope}%
\pgfsetbuttcap%
\pgfsetroundjoin%
\definecolor{currentfill}{rgb}{0.000000,0.000000,0.000000}%
\pgfsetfillcolor{currentfill}%
\pgfsetlinewidth{0.803000pt}%
\definecolor{currentstroke}{rgb}{0.000000,0.000000,0.000000}%
\pgfsetstrokecolor{currentstroke}%
\pgfsetdash{}{0pt}%
\pgfsys@defobject{currentmarker}{\pgfqpoint{-0.048611in}{0.000000in}}{\pgfqpoint{-0.000000in}{0.000000in}}{%
\pgfpathmoveto{\pgfqpoint{-0.000000in}{0.000000in}}%
\pgfpathlineto{\pgfqpoint{-0.048611in}{0.000000in}}%
\pgfusepath{stroke,fill}%
}%
\begin{pgfscope}%
\pgfsys@transformshift{0.515000in}{0.499444in}%
\pgfsys@useobject{currentmarker}{}%
\end{pgfscope}%
\end{pgfscope}%
\begin{pgfscope}%
\definecolor{textcolor}{rgb}{0.000000,0.000000,0.000000}%
\pgfsetstrokecolor{textcolor}%
\pgfsetfillcolor{textcolor}%
\pgftext[x=0.348333in, y=0.451250in, left, base]{\color{textcolor}\rmfamily\fontsize{10.000000}{12.000000}\selectfont \(\displaystyle {0}\)}%
\end{pgfscope}%
\begin{pgfscope}%
\pgfsetbuttcap%
\pgfsetroundjoin%
\definecolor{currentfill}{rgb}{0.000000,0.000000,0.000000}%
\pgfsetfillcolor{currentfill}%
\pgfsetlinewidth{0.803000pt}%
\definecolor{currentstroke}{rgb}{0.000000,0.000000,0.000000}%
\pgfsetstrokecolor{currentstroke}%
\pgfsetdash{}{0pt}%
\pgfsys@defobject{currentmarker}{\pgfqpoint{-0.048611in}{0.000000in}}{\pgfqpoint{-0.000000in}{0.000000in}}{%
\pgfpathmoveto{\pgfqpoint{-0.000000in}{0.000000in}}%
\pgfpathlineto{\pgfqpoint{-0.048611in}{0.000000in}}%
\pgfusepath{stroke,fill}%
}%
\begin{pgfscope}%
\pgfsys@transformshift{0.515000in}{0.925576in}%
\pgfsys@useobject{currentmarker}{}%
\end{pgfscope}%
\end{pgfscope}%
\begin{pgfscope}%
\definecolor{textcolor}{rgb}{0.000000,0.000000,0.000000}%
\pgfsetstrokecolor{textcolor}%
\pgfsetfillcolor{textcolor}%
\pgftext[x=0.278889in, y=0.877381in, left, base]{\color{textcolor}\rmfamily\fontsize{10.000000}{12.000000}\selectfont \(\displaystyle {10}\)}%
\end{pgfscope}%
\begin{pgfscope}%
\pgfsetbuttcap%
\pgfsetroundjoin%
\definecolor{currentfill}{rgb}{0.000000,0.000000,0.000000}%
\pgfsetfillcolor{currentfill}%
\pgfsetlinewidth{0.803000pt}%
\definecolor{currentstroke}{rgb}{0.000000,0.000000,0.000000}%
\pgfsetstrokecolor{currentstroke}%
\pgfsetdash{}{0pt}%
\pgfsys@defobject{currentmarker}{\pgfqpoint{-0.048611in}{0.000000in}}{\pgfqpoint{-0.000000in}{0.000000in}}{%
\pgfpathmoveto{\pgfqpoint{-0.000000in}{0.000000in}}%
\pgfpathlineto{\pgfqpoint{-0.048611in}{0.000000in}}%
\pgfusepath{stroke,fill}%
}%
\begin{pgfscope}%
\pgfsys@transformshift{0.515000in}{1.351707in}%
\pgfsys@useobject{currentmarker}{}%
\end{pgfscope}%
\end{pgfscope}%
\begin{pgfscope}%
\definecolor{textcolor}{rgb}{0.000000,0.000000,0.000000}%
\pgfsetstrokecolor{textcolor}%
\pgfsetfillcolor{textcolor}%
\pgftext[x=0.278889in, y=1.303513in, left, base]{\color{textcolor}\rmfamily\fontsize{10.000000}{12.000000}\selectfont \(\displaystyle {20}\)}%
\end{pgfscope}%
\begin{pgfscope}%
\definecolor{textcolor}{rgb}{0.000000,0.000000,0.000000}%
\pgfsetstrokecolor{textcolor}%
\pgfsetfillcolor{textcolor}%
\pgftext[x=0.223333in,y=1.076944in,,bottom,rotate=90.000000]{\color{textcolor}\rmfamily\fontsize{10.000000}{12.000000}\selectfont Percent of Data Set}%
\end{pgfscope}%
\begin{pgfscope}%
\pgfsetrectcap%
\pgfsetmiterjoin%
\pgfsetlinewidth{0.803000pt}%
\definecolor{currentstroke}{rgb}{0.000000,0.000000,0.000000}%
\pgfsetstrokecolor{currentstroke}%
\pgfsetdash{}{0pt}%
\pgfpathmoveto{\pgfqpoint{0.515000in}{0.499444in}}%
\pgfpathlineto{\pgfqpoint{0.515000in}{1.654444in}}%
\pgfusepath{stroke}%
\end{pgfscope}%
\begin{pgfscope}%
\pgfsetrectcap%
\pgfsetmiterjoin%
\pgfsetlinewidth{0.803000pt}%
\definecolor{currentstroke}{rgb}{0.000000,0.000000,0.000000}%
\pgfsetstrokecolor{currentstroke}%
\pgfsetdash{}{0pt}%
\pgfpathmoveto{\pgfqpoint{4.002500in}{0.499444in}}%
\pgfpathlineto{\pgfqpoint{4.002500in}{1.654444in}}%
\pgfusepath{stroke}%
\end{pgfscope}%
\begin{pgfscope}%
\pgfsetrectcap%
\pgfsetmiterjoin%
\pgfsetlinewidth{0.803000pt}%
\definecolor{currentstroke}{rgb}{0.000000,0.000000,0.000000}%
\pgfsetstrokecolor{currentstroke}%
\pgfsetdash{}{0pt}%
\pgfpathmoveto{\pgfqpoint{0.515000in}{0.499444in}}%
\pgfpathlineto{\pgfqpoint{4.002500in}{0.499444in}}%
\pgfusepath{stroke}%
\end{pgfscope}%
\begin{pgfscope}%
\pgfsetrectcap%
\pgfsetmiterjoin%
\pgfsetlinewidth{0.803000pt}%
\definecolor{currentstroke}{rgb}{0.000000,0.000000,0.000000}%
\pgfsetstrokecolor{currentstroke}%
\pgfsetdash{}{0pt}%
\pgfpathmoveto{\pgfqpoint{0.515000in}{1.654444in}}%
\pgfpathlineto{\pgfqpoint{4.002500in}{1.654444in}}%
\pgfusepath{stroke}%
\end{pgfscope}%
\begin{pgfscope}%
\pgfsetbuttcap%
\pgfsetmiterjoin%
\definecolor{currentfill}{rgb}{1.000000,1.000000,1.000000}%
\pgfsetfillcolor{currentfill}%
\pgfsetfillopacity{0.800000}%
\pgfsetlinewidth{1.003750pt}%
\definecolor{currentstroke}{rgb}{0.800000,0.800000,0.800000}%
\pgfsetstrokecolor{currentstroke}%
\pgfsetstrokeopacity{0.800000}%
\pgfsetdash{}{0pt}%
\pgfpathmoveto{\pgfqpoint{3.225556in}{1.154445in}}%
\pgfpathlineto{\pgfqpoint{3.905278in}{1.154445in}}%
\pgfpathquadraticcurveto{\pgfqpoint{3.933056in}{1.154445in}}{\pgfqpoint{3.933056in}{1.182222in}}%
\pgfpathlineto{\pgfqpoint{3.933056in}{1.557222in}}%
\pgfpathquadraticcurveto{\pgfqpoint{3.933056in}{1.585000in}}{\pgfqpoint{3.905278in}{1.585000in}}%
\pgfpathlineto{\pgfqpoint{3.225556in}{1.585000in}}%
\pgfpathquadraticcurveto{\pgfqpoint{3.197778in}{1.585000in}}{\pgfqpoint{3.197778in}{1.557222in}}%
\pgfpathlineto{\pgfqpoint{3.197778in}{1.182222in}}%
\pgfpathquadraticcurveto{\pgfqpoint{3.197778in}{1.154445in}}{\pgfqpoint{3.225556in}{1.154445in}}%
\pgfpathlineto{\pgfqpoint{3.225556in}{1.154445in}}%
\pgfpathclose%
\pgfusepath{stroke,fill}%
\end{pgfscope}%
\begin{pgfscope}%
\pgfsetbuttcap%
\pgfsetmiterjoin%
\pgfsetlinewidth{1.003750pt}%
\definecolor{currentstroke}{rgb}{0.000000,0.000000,0.000000}%
\pgfsetstrokecolor{currentstroke}%
\pgfsetdash{}{0pt}%
\pgfpathmoveto{\pgfqpoint{3.253334in}{1.432222in}}%
\pgfpathlineto{\pgfqpoint{3.531111in}{1.432222in}}%
\pgfpathlineto{\pgfqpoint{3.531111in}{1.529444in}}%
\pgfpathlineto{\pgfqpoint{3.253334in}{1.529444in}}%
\pgfpathlineto{\pgfqpoint{3.253334in}{1.432222in}}%
\pgfpathclose%
\pgfusepath{stroke}%
\end{pgfscope}%
\begin{pgfscope}%
\definecolor{textcolor}{rgb}{0.000000,0.000000,0.000000}%
\pgfsetstrokecolor{textcolor}%
\pgfsetfillcolor{textcolor}%
\pgftext[x=3.642223in,y=1.432222in,left,base]{\color{textcolor}\rmfamily\fontsize{10.000000}{12.000000}\selectfont Neg}%
\end{pgfscope}%
\begin{pgfscope}%
\pgfsetbuttcap%
\pgfsetmiterjoin%
\definecolor{currentfill}{rgb}{0.000000,0.000000,0.000000}%
\pgfsetfillcolor{currentfill}%
\pgfsetlinewidth{0.000000pt}%
\definecolor{currentstroke}{rgb}{0.000000,0.000000,0.000000}%
\pgfsetstrokecolor{currentstroke}%
\pgfsetstrokeopacity{0.000000}%
\pgfsetdash{}{0pt}%
\pgfpathmoveto{\pgfqpoint{3.253334in}{1.236944in}}%
\pgfpathlineto{\pgfqpoint{3.531111in}{1.236944in}}%
\pgfpathlineto{\pgfqpoint{3.531111in}{1.334167in}}%
\pgfpathlineto{\pgfqpoint{3.253334in}{1.334167in}}%
\pgfpathlineto{\pgfqpoint{3.253334in}{1.236944in}}%
\pgfpathclose%
\pgfusepath{fill}%
\end{pgfscope}%
\begin{pgfscope}%
\definecolor{textcolor}{rgb}{0.000000,0.000000,0.000000}%
\pgfsetstrokecolor{textcolor}%
\pgfsetfillcolor{textcolor}%
\pgftext[x=3.642223in,y=1.236944in,left,base]{\color{textcolor}\rmfamily\fontsize{10.000000}{12.000000}\selectfont Pos}%
\end{pgfscope}%
\end{pgfpicture}%
\makeatother%
\endgroup%

&
	\vskip 0pt
	\qquad \qquad ROC Curve
	
	%% Creator: Matplotlib, PGF backend
%%
%% To include the figure in your LaTeX document, write
%%   \input{<filename>.pgf}
%%
%% Make sure the required packages are loaded in your preamble
%%   \usepackage{pgf}
%%
%% Also ensure that all the required font packages are loaded; for instance,
%% the lmodern package is sometimes necessary when using math font.
%%   \usepackage{lmodern}
%%
%% Figures using additional raster images can only be included by \input if
%% they are in the same directory as the main LaTeX file. For loading figures
%% from other directories you can use the `import` package
%%   \usepackage{import}
%%
%% and then include the figures with
%%   \import{<path to file>}{<filename>.pgf}
%%
%% Matplotlib used the following preamble
%%   
%%   \usepackage{fontspec}
%%   \makeatletter\@ifpackageloaded{underscore}{}{\usepackage[strings]{underscore}}\makeatother
%%
\begingroup%
\makeatletter%
\begin{pgfpicture}%
\pgfpathrectangle{\pgfpointorigin}{\pgfqpoint{2.221861in}{1.754444in}}%
\pgfusepath{use as bounding box, clip}%
\begin{pgfscope}%
\pgfsetbuttcap%
\pgfsetmiterjoin%
\definecolor{currentfill}{rgb}{1.000000,1.000000,1.000000}%
\pgfsetfillcolor{currentfill}%
\pgfsetlinewidth{0.000000pt}%
\definecolor{currentstroke}{rgb}{1.000000,1.000000,1.000000}%
\pgfsetstrokecolor{currentstroke}%
\pgfsetdash{}{0pt}%
\pgfpathmoveto{\pgfqpoint{0.000000in}{0.000000in}}%
\pgfpathlineto{\pgfqpoint{2.221861in}{0.000000in}}%
\pgfpathlineto{\pgfqpoint{2.221861in}{1.754444in}}%
\pgfpathlineto{\pgfqpoint{0.000000in}{1.754444in}}%
\pgfpathlineto{\pgfqpoint{0.000000in}{0.000000in}}%
\pgfpathclose%
\pgfusepath{fill}%
\end{pgfscope}%
\begin{pgfscope}%
\pgfsetbuttcap%
\pgfsetmiterjoin%
\definecolor{currentfill}{rgb}{1.000000,1.000000,1.000000}%
\pgfsetfillcolor{currentfill}%
\pgfsetlinewidth{0.000000pt}%
\definecolor{currentstroke}{rgb}{0.000000,0.000000,0.000000}%
\pgfsetstrokecolor{currentstroke}%
\pgfsetstrokeopacity{0.000000}%
\pgfsetdash{}{0pt}%
\pgfpathmoveto{\pgfqpoint{0.553581in}{0.499444in}}%
\pgfpathlineto{\pgfqpoint{2.103581in}{0.499444in}}%
\pgfpathlineto{\pgfqpoint{2.103581in}{1.654444in}}%
\pgfpathlineto{\pgfqpoint{0.553581in}{1.654444in}}%
\pgfpathlineto{\pgfqpoint{0.553581in}{0.499444in}}%
\pgfpathclose%
\pgfusepath{fill}%
\end{pgfscope}%
\begin{pgfscope}%
\pgfsetbuttcap%
\pgfsetroundjoin%
\definecolor{currentfill}{rgb}{0.000000,0.000000,0.000000}%
\pgfsetfillcolor{currentfill}%
\pgfsetlinewidth{0.803000pt}%
\definecolor{currentstroke}{rgb}{0.000000,0.000000,0.000000}%
\pgfsetstrokecolor{currentstroke}%
\pgfsetdash{}{0pt}%
\pgfsys@defobject{currentmarker}{\pgfqpoint{0.000000in}{-0.048611in}}{\pgfqpoint{0.000000in}{0.000000in}}{%
\pgfpathmoveto{\pgfqpoint{0.000000in}{0.000000in}}%
\pgfpathlineto{\pgfqpoint{0.000000in}{-0.048611in}}%
\pgfusepath{stroke,fill}%
}%
\begin{pgfscope}%
\pgfsys@transformshift{0.624035in}{0.499444in}%
\pgfsys@useobject{currentmarker}{}%
\end{pgfscope}%
\end{pgfscope}%
\begin{pgfscope}%
\definecolor{textcolor}{rgb}{0.000000,0.000000,0.000000}%
\pgfsetstrokecolor{textcolor}%
\pgfsetfillcolor{textcolor}%
\pgftext[x=0.624035in,y=0.402222in,,top]{\color{textcolor}\rmfamily\fontsize{10.000000}{12.000000}\selectfont \(\displaystyle {0.0}\)}%
\end{pgfscope}%
\begin{pgfscope}%
\pgfsetbuttcap%
\pgfsetroundjoin%
\definecolor{currentfill}{rgb}{0.000000,0.000000,0.000000}%
\pgfsetfillcolor{currentfill}%
\pgfsetlinewidth{0.803000pt}%
\definecolor{currentstroke}{rgb}{0.000000,0.000000,0.000000}%
\pgfsetstrokecolor{currentstroke}%
\pgfsetdash{}{0pt}%
\pgfsys@defobject{currentmarker}{\pgfqpoint{0.000000in}{-0.048611in}}{\pgfqpoint{0.000000in}{0.000000in}}{%
\pgfpathmoveto{\pgfqpoint{0.000000in}{0.000000in}}%
\pgfpathlineto{\pgfqpoint{0.000000in}{-0.048611in}}%
\pgfusepath{stroke,fill}%
}%
\begin{pgfscope}%
\pgfsys@transformshift{1.328581in}{0.499444in}%
\pgfsys@useobject{currentmarker}{}%
\end{pgfscope}%
\end{pgfscope}%
\begin{pgfscope}%
\definecolor{textcolor}{rgb}{0.000000,0.000000,0.000000}%
\pgfsetstrokecolor{textcolor}%
\pgfsetfillcolor{textcolor}%
\pgftext[x=1.328581in,y=0.402222in,,top]{\color{textcolor}\rmfamily\fontsize{10.000000}{12.000000}\selectfont \(\displaystyle {0.5}\)}%
\end{pgfscope}%
\begin{pgfscope}%
\pgfsetbuttcap%
\pgfsetroundjoin%
\definecolor{currentfill}{rgb}{0.000000,0.000000,0.000000}%
\pgfsetfillcolor{currentfill}%
\pgfsetlinewidth{0.803000pt}%
\definecolor{currentstroke}{rgb}{0.000000,0.000000,0.000000}%
\pgfsetstrokecolor{currentstroke}%
\pgfsetdash{}{0pt}%
\pgfsys@defobject{currentmarker}{\pgfqpoint{0.000000in}{-0.048611in}}{\pgfqpoint{0.000000in}{0.000000in}}{%
\pgfpathmoveto{\pgfqpoint{0.000000in}{0.000000in}}%
\pgfpathlineto{\pgfqpoint{0.000000in}{-0.048611in}}%
\pgfusepath{stroke,fill}%
}%
\begin{pgfscope}%
\pgfsys@transformshift{2.033126in}{0.499444in}%
\pgfsys@useobject{currentmarker}{}%
\end{pgfscope}%
\end{pgfscope}%
\begin{pgfscope}%
\definecolor{textcolor}{rgb}{0.000000,0.000000,0.000000}%
\pgfsetstrokecolor{textcolor}%
\pgfsetfillcolor{textcolor}%
\pgftext[x=2.033126in,y=0.402222in,,top]{\color{textcolor}\rmfamily\fontsize{10.000000}{12.000000}\selectfont \(\displaystyle {1.0}\)}%
\end{pgfscope}%
\begin{pgfscope}%
\definecolor{textcolor}{rgb}{0.000000,0.000000,0.000000}%
\pgfsetstrokecolor{textcolor}%
\pgfsetfillcolor{textcolor}%
\pgftext[x=1.328581in,y=0.223333in,,top]{\color{textcolor}\rmfamily\fontsize{10.000000}{12.000000}\selectfont False positive rate}%
\end{pgfscope}%
\begin{pgfscope}%
\pgfsetbuttcap%
\pgfsetroundjoin%
\definecolor{currentfill}{rgb}{0.000000,0.000000,0.000000}%
\pgfsetfillcolor{currentfill}%
\pgfsetlinewidth{0.803000pt}%
\definecolor{currentstroke}{rgb}{0.000000,0.000000,0.000000}%
\pgfsetstrokecolor{currentstroke}%
\pgfsetdash{}{0pt}%
\pgfsys@defobject{currentmarker}{\pgfqpoint{-0.048611in}{0.000000in}}{\pgfqpoint{-0.000000in}{0.000000in}}{%
\pgfpathmoveto{\pgfqpoint{-0.000000in}{0.000000in}}%
\pgfpathlineto{\pgfqpoint{-0.048611in}{0.000000in}}%
\pgfusepath{stroke,fill}%
}%
\begin{pgfscope}%
\pgfsys@transformshift{0.553581in}{0.551944in}%
\pgfsys@useobject{currentmarker}{}%
\end{pgfscope}%
\end{pgfscope}%
\begin{pgfscope}%
\definecolor{textcolor}{rgb}{0.000000,0.000000,0.000000}%
\pgfsetstrokecolor{textcolor}%
\pgfsetfillcolor{textcolor}%
\pgftext[x=0.278889in, y=0.503750in, left, base]{\color{textcolor}\rmfamily\fontsize{10.000000}{12.000000}\selectfont \(\displaystyle {0.0}\)}%
\end{pgfscope}%
\begin{pgfscope}%
\pgfsetbuttcap%
\pgfsetroundjoin%
\definecolor{currentfill}{rgb}{0.000000,0.000000,0.000000}%
\pgfsetfillcolor{currentfill}%
\pgfsetlinewidth{0.803000pt}%
\definecolor{currentstroke}{rgb}{0.000000,0.000000,0.000000}%
\pgfsetstrokecolor{currentstroke}%
\pgfsetdash{}{0pt}%
\pgfsys@defobject{currentmarker}{\pgfqpoint{-0.048611in}{0.000000in}}{\pgfqpoint{-0.000000in}{0.000000in}}{%
\pgfpathmoveto{\pgfqpoint{-0.000000in}{0.000000in}}%
\pgfpathlineto{\pgfqpoint{-0.048611in}{0.000000in}}%
\pgfusepath{stroke,fill}%
}%
\begin{pgfscope}%
\pgfsys@transformshift{0.553581in}{1.076944in}%
\pgfsys@useobject{currentmarker}{}%
\end{pgfscope}%
\end{pgfscope}%
\begin{pgfscope}%
\definecolor{textcolor}{rgb}{0.000000,0.000000,0.000000}%
\pgfsetstrokecolor{textcolor}%
\pgfsetfillcolor{textcolor}%
\pgftext[x=0.278889in, y=1.028750in, left, base]{\color{textcolor}\rmfamily\fontsize{10.000000}{12.000000}\selectfont \(\displaystyle {0.5}\)}%
\end{pgfscope}%
\begin{pgfscope}%
\pgfsetbuttcap%
\pgfsetroundjoin%
\definecolor{currentfill}{rgb}{0.000000,0.000000,0.000000}%
\pgfsetfillcolor{currentfill}%
\pgfsetlinewidth{0.803000pt}%
\definecolor{currentstroke}{rgb}{0.000000,0.000000,0.000000}%
\pgfsetstrokecolor{currentstroke}%
\pgfsetdash{}{0pt}%
\pgfsys@defobject{currentmarker}{\pgfqpoint{-0.048611in}{0.000000in}}{\pgfqpoint{-0.000000in}{0.000000in}}{%
\pgfpathmoveto{\pgfqpoint{-0.000000in}{0.000000in}}%
\pgfpathlineto{\pgfqpoint{-0.048611in}{0.000000in}}%
\pgfusepath{stroke,fill}%
}%
\begin{pgfscope}%
\pgfsys@transformshift{0.553581in}{1.601944in}%
\pgfsys@useobject{currentmarker}{}%
\end{pgfscope}%
\end{pgfscope}%
\begin{pgfscope}%
\definecolor{textcolor}{rgb}{0.000000,0.000000,0.000000}%
\pgfsetstrokecolor{textcolor}%
\pgfsetfillcolor{textcolor}%
\pgftext[x=0.278889in, y=1.553750in, left, base]{\color{textcolor}\rmfamily\fontsize{10.000000}{12.000000}\selectfont \(\displaystyle {1.0}\)}%
\end{pgfscope}%
\begin{pgfscope}%
\definecolor{textcolor}{rgb}{0.000000,0.000000,0.000000}%
\pgfsetstrokecolor{textcolor}%
\pgfsetfillcolor{textcolor}%
\pgftext[x=0.223333in,y=1.076944in,,bottom,rotate=90.000000]{\color{textcolor}\rmfamily\fontsize{10.000000}{12.000000}\selectfont True positive rate}%
\end{pgfscope}%
\begin{pgfscope}%
\pgfpathrectangle{\pgfqpoint{0.553581in}{0.499444in}}{\pgfqpoint{1.550000in}{1.155000in}}%
\pgfusepath{clip}%
\pgfsetbuttcap%
\pgfsetroundjoin%
\pgfsetlinewidth{1.505625pt}%
\definecolor{currentstroke}{rgb}{0.000000,0.000000,0.000000}%
\pgfsetstrokecolor{currentstroke}%
\pgfsetdash{{5.550000pt}{2.400000pt}}{0.000000pt}%
\pgfpathmoveto{\pgfqpoint{0.624035in}{0.551944in}}%
\pgfpathlineto{\pgfqpoint{2.033126in}{1.601944in}}%
\pgfusepath{stroke}%
\end{pgfscope}%
\begin{pgfscope}%
\pgfpathrectangle{\pgfqpoint{0.553581in}{0.499444in}}{\pgfqpoint{1.550000in}{1.155000in}}%
\pgfusepath{clip}%
\pgfsetrectcap%
\pgfsetroundjoin%
\pgfsetlinewidth{1.505625pt}%
\definecolor{currentstroke}{rgb}{0.000000,0.000000,0.000000}%
\pgfsetstrokecolor{currentstroke}%
\pgfsetdash{}{0pt}%
\pgfpathmoveto{\pgfqpoint{0.624035in}{0.551944in}}%
\pgfpathlineto{\pgfqpoint{0.625144in}{0.573390in}}%
\pgfpathlineto{\pgfqpoint{0.625288in}{0.574494in}}%
\pgfpathlineto{\pgfqpoint{0.626397in}{0.588808in}}%
\pgfpathlineto{\pgfqpoint{0.626468in}{0.589899in}}%
\pgfpathlineto{\pgfqpoint{0.627577in}{0.602950in}}%
\pgfpathlineto{\pgfqpoint{0.627660in}{0.604054in}}%
\pgfpathlineto{\pgfqpoint{0.628766in}{0.615362in}}%
\pgfpathlineto{\pgfqpoint{0.628937in}{0.616453in}}%
\pgfpathlineto{\pgfqpoint{0.630039in}{0.626218in}}%
\pgfpathlineto{\pgfqpoint{0.630200in}{0.627282in}}%
\pgfpathlineto{\pgfqpoint{0.631309in}{0.635916in}}%
\pgfpathlineto{\pgfqpoint{0.631426in}{0.636994in}}%
\pgfpathlineto{\pgfqpoint{0.632532in}{0.644817in}}%
\pgfpathlineto{\pgfqpoint{0.632730in}{0.645894in}}%
\pgfpathlineto{\pgfqpoint{0.632730in}{0.645921in}}%
\pgfpathlineto{\pgfqpoint{0.639028in}{0.687095in}}%
\pgfpathlineto{\pgfqpoint{0.639223in}{0.688186in}}%
\pgfpathlineto{\pgfqpoint{0.640322in}{0.694572in}}%
\pgfpathlineto{\pgfqpoint{0.640332in}{0.694572in}}%
\pgfpathlineto{\pgfqpoint{0.640600in}{0.695676in}}%
\pgfpathlineto{\pgfqpoint{0.641699in}{0.701184in}}%
\pgfpathlineto{\pgfqpoint{0.641970in}{0.702275in}}%
\pgfpathlineto{\pgfqpoint{0.643079in}{0.707889in}}%
\pgfpathlineto{\pgfqpoint{0.643344in}{0.708967in}}%
\pgfpathlineto{\pgfqpoint{0.644449in}{0.714049in}}%
\pgfpathlineto{\pgfqpoint{0.644711in}{0.715139in}}%
\pgfpathlineto{\pgfqpoint{0.645813in}{0.720873in}}%
\pgfpathlineto{\pgfqpoint{0.646058in}{0.721977in}}%
\pgfpathlineto{\pgfqpoint{0.647163in}{0.726980in}}%
\pgfpathlineto{\pgfqpoint{0.647324in}{0.728071in}}%
\pgfpathlineto{\pgfqpoint{0.648433in}{0.733099in}}%
\pgfpathlineto{\pgfqpoint{0.648587in}{0.734177in}}%
\pgfpathlineto{\pgfqpoint{0.648587in}{0.734203in}}%
\pgfpathlineto{\pgfqpoint{0.649689in}{0.738820in}}%
\pgfpathlineto{\pgfqpoint{0.649941in}{0.739924in}}%
\pgfpathlineto{\pgfqpoint{0.651046in}{0.745166in}}%
\pgfpathlineto{\pgfqpoint{0.651304in}{0.746257in}}%
\pgfpathlineto{\pgfqpoint{0.652413in}{0.750647in}}%
\pgfpathlineto{\pgfqpoint{0.652691in}{0.751751in}}%
\pgfpathlineto{\pgfqpoint{0.653797in}{0.756593in}}%
\pgfpathlineto{\pgfqpoint{0.654048in}{0.757658in}}%
\pgfpathlineto{\pgfqpoint{0.655154in}{0.761729in}}%
\pgfpathlineto{\pgfqpoint{0.655442in}{0.762820in}}%
\pgfpathlineto{\pgfqpoint{0.656551in}{0.766930in}}%
\pgfpathlineto{\pgfqpoint{0.656843in}{0.768021in}}%
\pgfpathlineto{\pgfqpoint{0.657952in}{0.772398in}}%
\pgfpathlineto{\pgfqpoint{0.658186in}{0.773489in}}%
\pgfpathlineto{\pgfqpoint{0.658186in}{0.773502in}}%
\pgfpathlineto{\pgfqpoint{0.666690in}{0.802757in}}%
\pgfpathlineto{\pgfqpoint{0.667102in}{0.803848in}}%
\pgfpathlineto{\pgfqpoint{0.668207in}{0.807506in}}%
\pgfpathlineto{\pgfqpoint{0.668613in}{0.808611in}}%
\pgfpathlineto{\pgfqpoint{0.669715in}{0.812203in}}%
\pgfpathlineto{\pgfqpoint{0.669966in}{0.813293in}}%
\pgfpathlineto{\pgfqpoint{0.671075in}{0.817511in}}%
\pgfpathlineto{\pgfqpoint{0.671508in}{0.818615in}}%
\pgfpathlineto{\pgfqpoint{0.672613in}{0.822100in}}%
\pgfpathlineto{\pgfqpoint{0.672992in}{0.823205in}}%
\pgfpathlineto{\pgfqpoint{0.674087in}{0.826823in}}%
\pgfpathlineto{\pgfqpoint{0.674469in}{0.827927in}}%
\pgfpathlineto{\pgfqpoint{0.675578in}{0.831040in}}%
\pgfpathlineto{\pgfqpoint{0.675990in}{0.832131in}}%
\pgfpathlineto{\pgfqpoint{0.677089in}{0.835404in}}%
\pgfpathlineto{\pgfqpoint{0.677444in}{0.836468in}}%
\pgfpathlineto{\pgfqpoint{0.678543in}{0.840180in}}%
\pgfpathlineto{\pgfqpoint{0.678875in}{0.841218in}}%
\pgfpathlineto{\pgfqpoint{0.679984in}{0.844490in}}%
\pgfpathlineto{\pgfqpoint{0.680497in}{0.845595in}}%
\pgfpathlineto{\pgfqpoint{0.681599in}{0.848774in}}%
\pgfpathlineto{\pgfqpoint{0.681947in}{0.849878in}}%
\pgfpathlineto{\pgfqpoint{0.683046in}{0.852818in}}%
\pgfpathlineto{\pgfqpoint{0.683546in}{0.853896in}}%
\pgfpathlineto{\pgfqpoint{0.684645in}{0.857448in}}%
\pgfpathlineto{\pgfqpoint{0.685117in}{0.858539in}}%
\pgfpathlineto{\pgfqpoint{0.686223in}{0.861333in}}%
\pgfpathlineto{\pgfqpoint{0.686655in}{0.862424in}}%
\pgfpathlineto{\pgfqpoint{0.687764in}{0.865643in}}%
\pgfpathlineto{\pgfqpoint{0.688273in}{0.866734in}}%
\pgfpathlineto{\pgfqpoint{0.688273in}{0.866747in}}%
\pgfpathlineto{\pgfqpoint{0.691955in}{0.876858in}}%
\pgfpathlineto{\pgfqpoint{0.692454in}{0.877962in}}%
\pgfpathlineto{\pgfqpoint{0.693557in}{0.880836in}}%
\pgfpathlineto{\pgfqpoint{0.694002in}{0.881940in}}%
\pgfpathlineto{\pgfqpoint{0.695111in}{0.884814in}}%
\pgfpathlineto{\pgfqpoint{0.695493in}{0.885904in}}%
\pgfpathlineto{\pgfqpoint{0.696599in}{0.888672in}}%
\pgfpathlineto{\pgfqpoint{0.697038in}{0.889776in}}%
\pgfpathlineto{\pgfqpoint{0.698137in}{0.892583in}}%
\pgfpathlineto{\pgfqpoint{0.698663in}{0.893687in}}%
\pgfpathlineto{\pgfqpoint{0.699768in}{0.896268in}}%
\pgfpathlineto{\pgfqpoint{0.700268in}{0.897359in}}%
\pgfpathlineto{\pgfqpoint{0.701367in}{0.899674in}}%
\pgfpathlineto{\pgfqpoint{0.701966in}{0.900778in}}%
\pgfpathlineto{\pgfqpoint{0.703059in}{0.903439in}}%
\pgfpathlineto{\pgfqpoint{0.703069in}{0.903439in}}%
\pgfpathlineto{\pgfqpoint{0.703574in}{0.904529in}}%
\pgfpathlineto{\pgfqpoint{0.704680in}{0.907190in}}%
\pgfpathlineto{\pgfqpoint{0.705186in}{0.908294in}}%
\pgfpathlineto{\pgfqpoint{0.706292in}{0.911407in}}%
\pgfpathlineto{\pgfqpoint{0.706697in}{0.912512in}}%
\pgfpathlineto{\pgfqpoint{0.707806in}{0.914999in}}%
\pgfpathlineto{\pgfqpoint{0.708268in}{0.916090in}}%
\pgfpathlineto{\pgfqpoint{0.709377in}{0.918645in}}%
\pgfpathlineto{\pgfqpoint{0.709850in}{0.919749in}}%
\pgfpathlineto{\pgfqpoint{0.710955in}{0.922343in}}%
\pgfpathlineto{\pgfqpoint{0.711488in}{0.923421in}}%
\pgfpathlineto{\pgfqpoint{0.711488in}{0.923447in}}%
\pgfpathlineto{\pgfqpoint{0.714095in}{0.929288in}}%
\pgfpathlineto{\pgfqpoint{0.714668in}{0.930392in}}%
\pgfpathlineto{\pgfqpoint{0.715773in}{0.932613in}}%
\pgfpathlineto{\pgfqpoint{0.716259in}{0.933718in}}%
\pgfpathlineto{\pgfqpoint{0.717348in}{0.936192in}}%
\pgfpathlineto{\pgfqpoint{0.717924in}{0.937270in}}%
\pgfpathlineto{\pgfqpoint{0.719023in}{0.939651in}}%
\pgfpathlineto{\pgfqpoint{0.719606in}{0.940755in}}%
\pgfpathlineto{\pgfqpoint{0.720715in}{0.942698in}}%
\pgfpathlineto{\pgfqpoint{0.721221in}{0.943788in}}%
\pgfpathlineto{\pgfqpoint{0.722317in}{0.946103in}}%
\pgfpathlineto{\pgfqpoint{0.722803in}{0.947207in}}%
\pgfpathlineto{\pgfqpoint{0.723908in}{0.949482in}}%
\pgfpathlineto{\pgfqpoint{0.724428in}{0.950573in}}%
\pgfpathlineto{\pgfqpoint{0.725527in}{0.952981in}}%
\pgfpathlineto{\pgfqpoint{0.726113in}{0.954085in}}%
\pgfpathlineto{\pgfqpoint{0.727215in}{0.955948in}}%
\pgfpathlineto{\pgfqpoint{0.727744in}{0.957052in}}%
\pgfpathlineto{\pgfqpoint{0.728853in}{0.959048in}}%
\pgfpathlineto{\pgfqpoint{0.729597in}{0.960152in}}%
\pgfpathlineto{\pgfqpoint{0.730706in}{0.962520in}}%
\pgfpathlineto{\pgfqpoint{0.731340in}{0.963624in}}%
\pgfpathlineto{\pgfqpoint{0.732438in}{0.966165in}}%
\pgfpathlineto{\pgfqpoint{0.732448in}{0.966165in}}%
\pgfpathlineto{\pgfqpoint{0.732961in}{0.967189in}}%
\pgfpathlineto{\pgfqpoint{0.732961in}{0.967256in}}%
\pgfpathlineto{\pgfqpoint{0.734070in}{0.969491in}}%
\pgfpathlineto{\pgfqpoint{0.734807in}{0.970595in}}%
\pgfpathlineto{\pgfqpoint{0.735913in}{0.972657in}}%
\pgfpathlineto{\pgfqpoint{0.736469in}{0.973761in}}%
\pgfpathlineto{\pgfqpoint{0.737578in}{0.975983in}}%
\pgfpathlineto{\pgfqpoint{0.738111in}{0.977087in}}%
\pgfpathlineto{\pgfqpoint{0.739213in}{0.979149in}}%
\pgfpathlineto{\pgfqpoint{0.739220in}{0.979149in}}%
\pgfpathlineto{\pgfqpoint{0.739732in}{0.980254in}}%
\pgfpathlineto{\pgfqpoint{0.740835in}{0.982129in}}%
\pgfpathlineto{\pgfqpoint{0.741418in}{0.983234in}}%
\pgfpathlineto{\pgfqpoint{0.742527in}{0.985775in}}%
\pgfpathlineto{\pgfqpoint{0.743116in}{0.986879in}}%
\pgfpathlineto{\pgfqpoint{0.744219in}{0.989220in}}%
\pgfpathlineto{\pgfqpoint{0.744745in}{0.990324in}}%
\pgfpathlineto{\pgfqpoint{0.745840in}{0.992400in}}%
\pgfpathlineto{\pgfqpoint{0.745854in}{0.992400in}}%
\pgfpathlineto{\pgfqpoint{0.746400in}{0.993504in}}%
\pgfpathlineto{\pgfqpoint{0.747505in}{0.995606in}}%
\pgfpathlineto{\pgfqpoint{0.748269in}{0.996697in}}%
\pgfpathlineto{\pgfqpoint{0.749378in}{0.998945in}}%
\pgfpathlineto{\pgfqpoint{0.749968in}{1.000049in}}%
\pgfpathlineto{\pgfqpoint{0.751053in}{1.001845in}}%
\pgfpathlineto{\pgfqpoint{0.751573in}{1.002950in}}%
\pgfpathlineto{\pgfqpoint{0.752675in}{1.004852in}}%
\pgfpathlineto{\pgfqpoint{0.753278in}{1.005956in}}%
\pgfpathlineto{\pgfqpoint{0.754357in}{1.007632in}}%
\pgfpathlineto{\pgfqpoint{0.754983in}{1.008737in}}%
\pgfpathlineto{\pgfqpoint{0.756092in}{1.010426in}}%
\pgfpathlineto{\pgfqpoint{0.756675in}{1.011530in}}%
\pgfpathlineto{\pgfqpoint{0.757748in}{1.013592in}}%
\pgfpathlineto{\pgfqpoint{0.757764in}{1.013592in}}%
\pgfpathlineto{\pgfqpoint{0.758398in}{1.014697in}}%
\pgfpathlineto{\pgfqpoint{0.760498in}{1.018515in}}%
\pgfpathlineto{\pgfqpoint{0.761245in}{1.019619in}}%
\pgfpathlineto{\pgfqpoint{0.762354in}{1.021242in}}%
\pgfpathlineto{\pgfqpoint{0.762934in}{1.022346in}}%
\pgfpathlineto{\pgfqpoint{0.764040in}{1.024169in}}%
\pgfpathlineto{\pgfqpoint{0.764680in}{1.025273in}}%
\pgfpathlineto{\pgfqpoint{0.765785in}{1.027269in}}%
\pgfpathlineto{\pgfqpoint{0.766341in}{1.028373in}}%
\pgfpathlineto{\pgfqpoint{0.767444in}{1.030488in}}%
\pgfpathlineto{\pgfqpoint{0.768127in}{1.031579in}}%
\pgfpathlineto{\pgfqpoint{0.769233in}{1.033362in}}%
\pgfpathlineto{\pgfqpoint{0.769823in}{1.034386in}}%
\pgfpathlineto{\pgfqpoint{0.770925in}{1.036182in}}%
\pgfpathlineto{\pgfqpoint{0.771638in}{1.037286in}}%
\pgfpathlineto{\pgfqpoint{0.772731in}{1.039335in}}%
\pgfpathlineto{\pgfqpoint{0.773304in}{1.040439in}}%
\pgfpathlineto{\pgfqpoint{0.774399in}{1.042488in}}%
\pgfpathlineto{\pgfqpoint{0.774982in}{1.043579in}}%
\pgfpathlineto{\pgfqpoint{0.776071in}{1.045188in}}%
\pgfpathlineto{\pgfqpoint{0.776822in}{1.046293in}}%
\pgfpathlineto{\pgfqpoint{0.777914in}{1.047982in}}%
\pgfpathlineto{\pgfqpoint{0.778554in}{1.049086in}}%
\pgfpathlineto{\pgfqpoint{0.779659in}{1.050962in}}%
\pgfpathlineto{\pgfqpoint{0.779663in}{1.050962in}}%
\pgfpathlineto{\pgfqpoint{0.780356in}{1.052053in}}%
\pgfpathlineto{\pgfqpoint{0.781449in}{1.053916in}}%
\pgfpathlineto{\pgfqpoint{0.782192in}{1.055020in}}%
\pgfpathlineto{\pgfqpoint{0.783298in}{1.056536in}}%
\pgfpathlineto{\pgfqpoint{0.783988in}{1.057641in}}%
\pgfpathlineto{\pgfqpoint{0.785090in}{1.059317in}}%
\pgfpathlineto{\pgfqpoint{0.785804in}{1.060421in}}%
\pgfpathlineto{\pgfqpoint{0.786906in}{1.062111in}}%
\pgfpathlineto{\pgfqpoint{0.787798in}{1.063215in}}%
\pgfpathlineto{\pgfqpoint{0.788903in}{1.064998in}}%
\pgfpathlineto{\pgfqpoint{0.789681in}{1.066102in}}%
\pgfpathlineto{\pgfqpoint{0.790776in}{1.067871in}}%
\pgfpathlineto{\pgfqpoint{0.791255in}{1.068975in}}%
\pgfpathlineto{\pgfqpoint{0.792341in}{1.070545in}}%
\pgfpathlineto{\pgfqpoint{0.793192in}{1.071649in}}%
\pgfpathlineto{\pgfqpoint{0.794291in}{1.073365in}}%
\pgfpathlineto{\pgfqpoint{0.795105in}{1.074470in}}%
\pgfpathlineto{\pgfqpoint{0.796197in}{1.076186in}}%
\pgfpathlineto{\pgfqpoint{0.796887in}{1.077290in}}%
\pgfpathlineto{\pgfqpoint{0.797983in}{1.079392in}}%
\pgfpathlineto{\pgfqpoint{0.798549in}{1.080496in}}%
\pgfpathlineto{\pgfqpoint{0.799618in}{1.082133in}}%
\pgfpathlineto{\pgfqpoint{0.799655in}{1.082133in}}%
\pgfpathlineto{\pgfqpoint{0.800295in}{1.083237in}}%
\pgfpathlineto{\pgfqpoint{0.801384in}{1.084966in}}%
\pgfpathlineto{\pgfqpoint{0.802047in}{1.086070in}}%
\pgfpathlineto{\pgfqpoint{0.803149in}{1.087760in}}%
\pgfpathlineto{\pgfqpoint{0.804134in}{1.088851in}}%
\pgfpathlineto{\pgfqpoint{0.805240in}{1.090607in}}%
\pgfpathlineto{\pgfqpoint{0.805937in}{1.091711in}}%
\pgfpathlineto{\pgfqpoint{0.807019in}{1.093175in}}%
\pgfpathlineto{\pgfqpoint{0.807830in}{1.094279in}}%
\pgfpathlineto{\pgfqpoint{0.808935in}{1.096128in}}%
\pgfpathlineto{\pgfqpoint{0.809931in}{1.097232in}}%
\pgfpathlineto{\pgfqpoint{0.811040in}{1.099201in}}%
\pgfpathlineto{\pgfqpoint{0.811750in}{1.100265in}}%
\pgfpathlineto{\pgfqpoint{0.812859in}{1.102021in}}%
\pgfpathlineto{\pgfqpoint{0.813720in}{1.103126in}}%
\pgfpathlineto{\pgfqpoint{0.814812in}{1.104549in}}%
\pgfpathlineto{\pgfqpoint{0.815697in}{1.105640in}}%
\pgfpathlineto{\pgfqpoint{0.816792in}{1.107197in}}%
\pgfpathlineto{\pgfqpoint{0.817469in}{1.108301in}}%
\pgfpathlineto{\pgfqpoint{0.818558in}{1.109804in}}%
\pgfpathlineto{\pgfqpoint{0.819456in}{1.110908in}}%
\pgfpathlineto{\pgfqpoint{0.820565in}{1.112598in}}%
\pgfpathlineto{\pgfqpoint{0.821389in}{1.113702in}}%
\pgfpathlineto{\pgfqpoint{0.822478in}{1.115152in}}%
\pgfpathlineto{\pgfqpoint{0.822488in}{1.115152in}}%
\pgfpathlineto{\pgfqpoint{0.823316in}{1.116243in}}%
\pgfpathlineto{\pgfqpoint{0.824414in}{1.117706in}}%
\pgfpathlineto{\pgfqpoint{0.825188in}{1.118811in}}%
\pgfpathlineto{\pgfqpoint{0.826297in}{1.120181in}}%
\pgfpathlineto{\pgfqpoint{0.827071in}{1.121258in}}%
\pgfpathlineto{\pgfqpoint{0.827071in}{1.121272in}}%
\pgfpathlineto{\pgfqpoint{0.828164in}{1.122775in}}%
\pgfpathlineto{\pgfqpoint{0.828864in}{1.123879in}}%
\pgfpathlineto{\pgfqpoint{0.829973in}{1.125396in}}%
\pgfpathlineto{\pgfqpoint{0.830693in}{1.126500in}}%
\pgfpathlineto{\pgfqpoint{0.831802in}{1.127724in}}%
\pgfpathlineto{\pgfqpoint{0.832827in}{1.128828in}}%
\pgfpathlineto{\pgfqpoint{0.833923in}{1.130518in}}%
\pgfpathlineto{\pgfqpoint{0.834710in}{1.131622in}}%
\pgfpathlineto{\pgfqpoint{0.835803in}{1.132819in}}%
\pgfpathlineto{\pgfqpoint{0.835819in}{1.132819in}}%
\pgfpathlineto{\pgfqpoint{0.836781in}{1.133923in}}%
\pgfpathlineto{\pgfqpoint{0.837887in}{1.135267in}}%
\pgfpathlineto{\pgfqpoint{0.838614in}{1.136371in}}%
\pgfpathlineto{\pgfqpoint{0.839723in}{1.137821in}}%
\pgfpathlineto{\pgfqpoint{0.840708in}{1.138886in}}%
\pgfpathlineto{\pgfqpoint{0.840708in}{1.138912in}}%
\pgfpathlineto{\pgfqpoint{0.841813in}{1.140456in}}%
\pgfpathlineto{\pgfqpoint{0.842708in}{1.141560in}}%
\pgfpathlineto{\pgfqpoint{0.843797in}{1.143036in}}%
\pgfpathlineto{\pgfqpoint{0.843817in}{1.143036in}}%
\pgfpathlineto{\pgfqpoint{0.844668in}{1.144141in}}%
\pgfpathlineto{\pgfqpoint{0.845757in}{1.145378in}}%
\pgfpathlineto{\pgfqpoint{0.846698in}{1.146469in}}%
\pgfpathlineto{\pgfqpoint{0.847797in}{1.148079in}}%
\pgfpathlineto{\pgfqpoint{0.847807in}{1.148079in}}%
\pgfpathlineto{\pgfqpoint{0.848765in}{1.149156in}}%
\pgfpathlineto{\pgfqpoint{0.849851in}{1.150872in}}%
\pgfpathlineto{\pgfqpoint{0.849864in}{1.150872in}}%
\pgfpathlineto{\pgfqpoint{0.850990in}{1.151976in}}%
\pgfpathlineto{\pgfqpoint{0.852082in}{1.153427in}}%
\pgfpathlineto{\pgfqpoint{0.853195in}{1.154531in}}%
\pgfpathlineto{\pgfqpoint{0.854304in}{1.156047in}}%
\pgfpathlineto{\pgfqpoint{0.855262in}{1.157152in}}%
\pgfpathlineto{\pgfqpoint{0.856371in}{1.158575in}}%
\pgfpathlineto{\pgfqpoint{0.857108in}{1.159679in}}%
\pgfpathlineto{\pgfqpoint{0.858217in}{1.160717in}}%
\pgfpathlineto{\pgfqpoint{0.858998in}{1.161821in}}%
\pgfpathlineto{\pgfqpoint{0.860107in}{1.163085in}}%
\pgfpathlineto{\pgfqpoint{0.861058in}{1.164189in}}%
\pgfpathlineto{\pgfqpoint{0.862137in}{1.165719in}}%
\pgfpathlineto{\pgfqpoint{0.863249in}{1.166810in}}%
\pgfpathlineto{\pgfqpoint{0.864345in}{1.168167in}}%
\pgfpathlineto{\pgfqpoint{0.865172in}{1.169231in}}%
\pgfpathlineto{\pgfqpoint{0.866235in}{1.170695in}}%
\pgfpathlineto{\pgfqpoint{0.867280in}{1.171786in}}%
\pgfpathlineto{\pgfqpoint{0.868386in}{1.173329in}}%
\pgfpathlineto{\pgfqpoint{0.869384in}{1.174420in}}%
\pgfpathlineto{\pgfqpoint{0.870483in}{1.175843in}}%
\pgfpathlineto{\pgfqpoint{0.871344in}{1.176947in}}%
\pgfpathlineto{\pgfqpoint{0.872450in}{1.178344in}}%
\pgfpathlineto{\pgfqpoint{0.873247in}{1.179448in}}%
\pgfpathlineto{\pgfqpoint{0.874333in}{1.180805in}}%
\pgfpathlineto{\pgfqpoint{0.875217in}{1.181910in}}%
\pgfpathlineto{\pgfqpoint{0.876319in}{1.183094in}}%
\pgfpathlineto{\pgfqpoint{0.877311in}{1.184198in}}%
\pgfpathlineto{\pgfqpoint{0.878420in}{1.185448in}}%
\pgfpathlineto{\pgfqpoint{0.879429in}{1.186553in}}%
\pgfpathlineto{\pgfqpoint{0.880538in}{1.187750in}}%
\pgfpathlineto{\pgfqpoint{0.881305in}{1.188854in}}%
\pgfpathlineto{\pgfqpoint{0.882397in}{1.190145in}}%
\pgfpathlineto{\pgfqpoint{0.883302in}{1.191209in}}%
\pgfpathlineto{\pgfqpoint{0.883302in}{1.191249in}}%
\pgfpathlineto{\pgfqpoint{0.884394in}{1.192579in}}%
\pgfpathlineto{\pgfqpoint{0.885617in}{1.193670in}}%
\pgfpathlineto{\pgfqpoint{0.886699in}{1.194814in}}%
\pgfpathlineto{\pgfqpoint{0.887724in}{1.195918in}}%
\pgfpathlineto{\pgfqpoint{0.888833in}{1.197062in}}%
\pgfpathlineto{\pgfqpoint{0.889862in}{1.198167in}}%
\pgfpathlineto{\pgfqpoint{0.890964in}{1.199444in}}%
\pgfpathlineto{\pgfqpoint{0.892251in}{1.200548in}}%
\pgfpathlineto{\pgfqpoint{0.893360in}{1.201732in}}%
\pgfpathlineto{\pgfqpoint{0.894341in}{1.202836in}}%
\pgfpathlineto{\pgfqpoint{0.895437in}{1.204060in}}%
\pgfpathlineto{\pgfqpoint{0.896435in}{1.205164in}}%
\pgfpathlineto{\pgfqpoint{0.897541in}{1.206255in}}%
\pgfpathlineto{\pgfqpoint{0.898389in}{1.207359in}}%
\pgfpathlineto{\pgfqpoint{0.899488in}{1.208889in}}%
\pgfpathlineto{\pgfqpoint{0.900258in}{1.209993in}}%
\pgfpathlineto{\pgfqpoint{0.901367in}{1.211044in}}%
\pgfpathlineto{\pgfqpoint{0.902547in}{1.212149in}}%
\pgfpathlineto{\pgfqpoint{0.903649in}{1.213293in}}%
\pgfpathlineto{\pgfqpoint{0.904808in}{1.214397in}}%
\pgfpathlineto{\pgfqpoint{0.905907in}{1.215501in}}%
\pgfpathlineto{\pgfqpoint{0.907096in}{1.216605in}}%
\pgfpathlineto{\pgfqpoint{0.908185in}{1.217577in}}%
\pgfpathlineto{\pgfqpoint{0.909626in}{1.218681in}}%
\pgfpathlineto{\pgfqpoint{0.910712in}{1.219665in}}%
\pgfpathlineto{\pgfqpoint{0.910732in}{1.219665in}}%
\pgfpathlineto{\pgfqpoint{0.911921in}{1.220769in}}%
\pgfpathlineto{\pgfqpoint{0.913017in}{1.221874in}}%
\pgfpathlineto{\pgfqpoint{0.913027in}{1.221874in}}%
\pgfpathlineto{\pgfqpoint{0.914554in}{1.222978in}}%
\pgfpathlineto{\pgfqpoint{0.915660in}{1.224308in}}%
\pgfpathlineto{\pgfqpoint{0.916736in}{1.225386in}}%
\pgfpathlineto{\pgfqpoint{0.917845in}{1.226543in}}%
\pgfpathlineto{\pgfqpoint{0.918947in}{1.227634in}}%
\pgfpathlineto{\pgfqpoint{0.918947in}{1.227647in}}%
\pgfpathlineto{\pgfqpoint{0.920424in}{1.229004in}}%
\pgfpathlineto{\pgfqpoint{0.921674in}{1.230109in}}%
\pgfpathlineto{\pgfqpoint{0.922783in}{1.231266in}}%
\pgfpathlineto{\pgfqpoint{0.923895in}{1.232370in}}%
\pgfpathlineto{\pgfqpoint{0.925004in}{1.233235in}}%
\pgfpathlineto{\pgfqpoint{0.926294in}{1.234339in}}%
\pgfpathlineto{\pgfqpoint{0.927390in}{1.235496in}}%
\pgfpathlineto{\pgfqpoint{0.928496in}{1.236601in}}%
\pgfpathlineto{\pgfqpoint{0.929605in}{1.237665in}}%
\pgfpathlineto{\pgfqpoint{0.930791in}{1.238756in}}%
\pgfpathlineto{\pgfqpoint{0.931893in}{1.239900in}}%
\pgfpathlineto{\pgfqpoint{0.932922in}{1.241004in}}%
\pgfpathlineto{\pgfqpoint{0.934000in}{1.242228in}}%
\pgfpathlineto{\pgfqpoint{0.935384in}{1.243332in}}%
\pgfpathlineto{\pgfqpoint{0.936490in}{1.244676in}}%
\pgfpathlineto{\pgfqpoint{0.937703in}{1.245780in}}%
\pgfpathlineto{\pgfqpoint{0.938741in}{1.246898in}}%
\pgfpathlineto{\pgfqpoint{0.938808in}{1.246898in}}%
\pgfpathlineto{\pgfqpoint{0.939927in}{1.248002in}}%
\pgfpathlineto{\pgfqpoint{0.941013in}{1.249159in}}%
\pgfpathlineto{\pgfqpoint{0.942289in}{1.250263in}}%
\pgfpathlineto{\pgfqpoint{0.943395in}{1.251514in}}%
\pgfpathlineto{\pgfqpoint{0.944812in}{1.252618in}}%
\pgfpathlineto{\pgfqpoint{0.945915in}{1.253762in}}%
\pgfpathlineto{\pgfqpoint{0.945921in}{1.253762in}}%
\pgfpathlineto{\pgfqpoint{0.946960in}{1.254867in}}%
\pgfpathlineto{\pgfqpoint{0.948052in}{1.255851in}}%
\pgfpathlineto{\pgfqpoint{0.949443in}{1.256955in}}%
\pgfpathlineto{\pgfqpoint{0.950491in}{1.257833in}}%
\pgfpathlineto{\pgfqpoint{0.950538in}{1.257833in}}%
\pgfpathlineto{\pgfqpoint{0.952029in}{1.258937in}}%
\pgfpathlineto{\pgfqpoint{0.953101in}{1.259802in}}%
\pgfpathlineto{\pgfqpoint{0.954405in}{1.260906in}}%
\pgfpathlineto{\pgfqpoint{0.955467in}{1.261838in}}%
\pgfpathlineto{\pgfqpoint{0.956696in}{1.262942in}}%
\pgfpathlineto{\pgfqpoint{0.957799in}{1.263940in}}%
\pgfpathlineto{\pgfqpoint{0.959042in}{1.265044in}}%
\pgfpathlineto{\pgfqpoint{0.960134in}{1.266002in}}%
\pgfpathlineto{\pgfqpoint{0.960140in}{1.266002in}}%
\pgfpathlineto{\pgfqpoint{0.961621in}{1.267093in}}%
\pgfpathlineto{\pgfqpoint{0.962720in}{1.268050in}}%
\pgfpathlineto{\pgfqpoint{0.964091in}{1.269155in}}%
\pgfpathlineto{\pgfqpoint{0.965186in}{1.270299in}}%
\pgfpathlineto{\pgfqpoint{0.966597in}{1.271403in}}%
\pgfpathlineto{\pgfqpoint{0.967679in}{1.272454in}}%
\pgfpathlineto{\pgfqpoint{0.969012in}{1.273558in}}%
\pgfpathlineto{\pgfqpoint{0.971237in}{1.275607in}}%
\pgfpathlineto{\pgfqpoint{0.972557in}{1.276711in}}%
\pgfpathlineto{\pgfqpoint{0.973643in}{1.277602in}}%
\pgfpathlineto{\pgfqpoint{0.975050in}{1.278693in}}%
\pgfpathlineto{\pgfqpoint{0.976159in}{1.279771in}}%
\pgfpathlineto{\pgfqpoint{0.977422in}{1.280875in}}%
\pgfpathlineto{\pgfqpoint{0.978531in}{1.281993in}}%
\pgfpathlineto{\pgfqpoint{0.979670in}{1.283097in}}%
\pgfpathlineto{\pgfqpoint{0.980742in}{1.283935in}}%
\pgfpathlineto{\pgfqpoint{0.980749in}{1.283935in}}%
\pgfpathlineto{\pgfqpoint{0.981972in}{1.285039in}}%
\pgfpathlineto{\pgfqpoint{0.983081in}{1.285930in}}%
\pgfpathlineto{\pgfqpoint{0.984478in}{1.287035in}}%
\pgfpathlineto{\pgfqpoint{0.985584in}{1.288072in}}%
\pgfpathlineto{\pgfqpoint{0.986843in}{1.289177in}}%
\pgfpathlineto{\pgfqpoint{0.987946in}{1.290068in}}%
\pgfpathlineto{\pgfqpoint{0.989088in}{1.291159in}}%
\pgfpathlineto{\pgfqpoint{0.990160in}{1.291984in}}%
\pgfpathlineto{\pgfqpoint{0.990177in}{1.291984in}}%
\pgfpathlineto{\pgfqpoint{0.991568in}{1.293088in}}%
\pgfpathlineto{\pgfqpoint{0.992586in}{1.294179in}}%
\pgfpathlineto{\pgfqpoint{0.992673in}{1.294179in}}%
\pgfpathlineto{\pgfqpoint{0.994060in}{1.295283in}}%
\pgfpathlineto{\pgfqpoint{0.995169in}{1.296374in}}%
\pgfpathlineto{\pgfqpoint{0.996349in}{1.297478in}}%
\pgfpathlineto{\pgfqpoint{0.997448in}{1.298329in}}%
\pgfpathlineto{\pgfqpoint{0.998932in}{1.299420in}}%
\pgfpathlineto{\pgfqpoint{1.000034in}{1.300298in}}%
\pgfpathlineto{\pgfqpoint{1.001290in}{1.301403in}}%
\pgfpathlineto{\pgfqpoint{1.002399in}{1.302241in}}%
\pgfpathlineto{\pgfqpoint{1.003833in}{1.303345in}}%
\pgfpathlineto{\pgfqpoint{1.004942in}{1.304130in}}%
\pgfpathlineto{\pgfqpoint{1.006705in}{1.305234in}}%
\pgfpathlineto{\pgfqpoint{1.007800in}{1.305939in}}%
\pgfpathlineto{\pgfqpoint{1.008973in}{1.307043in}}%
\pgfpathlineto{\pgfqpoint{1.010069in}{1.307868in}}%
\pgfpathlineto{\pgfqpoint{1.011402in}{1.308972in}}%
\pgfpathlineto{\pgfqpoint{1.012491in}{1.309797in}}%
\pgfpathlineto{\pgfqpoint{1.012511in}{1.309797in}}%
\pgfpathlineto{\pgfqpoint{1.013778in}{1.310901in}}%
\pgfpathlineto{\pgfqpoint{1.014883in}{1.311739in}}%
\pgfpathlineto{\pgfqpoint{1.016277in}{1.312844in}}%
\pgfpathlineto{\pgfqpoint{1.017349in}{1.313708in}}%
\pgfpathlineto{\pgfqpoint{1.017362in}{1.313708in}}%
\pgfpathlineto{\pgfqpoint{1.018726in}{1.314813in}}%
\pgfpathlineto{\pgfqpoint{1.019828in}{1.315651in}}%
\pgfpathlineto{\pgfqpoint{1.021336in}{1.316755in}}%
\pgfpathlineto{\pgfqpoint{1.022412in}{1.317620in}}%
\pgfpathlineto{\pgfqpoint{1.024063in}{1.318724in}}%
\pgfpathlineto{\pgfqpoint{1.025142in}{1.319456in}}%
\pgfpathlineto{\pgfqpoint{1.026703in}{1.320560in}}%
\pgfpathlineto{\pgfqpoint{1.027796in}{1.321331in}}%
\pgfpathlineto{\pgfqpoint{1.029156in}{1.322436in}}%
\pgfpathlineto{\pgfqpoint{1.030265in}{1.323314in}}%
\pgfpathlineto{\pgfqpoint{1.031853in}{1.324418in}}%
\pgfpathlineto{\pgfqpoint{1.032929in}{1.325282in}}%
\pgfpathlineto{\pgfqpoint{1.034798in}{1.326387in}}%
\pgfpathlineto{\pgfqpoint{1.035887in}{1.327251in}}%
\pgfpathlineto{\pgfqpoint{1.037485in}{1.328356in}}%
\pgfpathlineto{\pgfqpoint{1.038574in}{1.329313in}}%
\pgfpathlineto{\pgfqpoint{1.040356in}{1.330418in}}%
\pgfpathlineto{\pgfqpoint{1.041549in}{1.331242in}}%
\pgfpathlineto{\pgfqpoint{1.043680in}{1.332347in}}%
\pgfpathlineto{\pgfqpoint{1.044766in}{1.333105in}}%
\pgfpathlineto{\pgfqpoint{1.046012in}{1.334209in}}%
\pgfpathlineto{\pgfqpoint{1.047111in}{1.335114in}}%
\pgfpathlineto{\pgfqpoint{1.048545in}{1.336205in}}%
\pgfpathlineto{\pgfqpoint{1.049637in}{1.336950in}}%
\pgfpathlineto{\pgfqpoint{1.049651in}{1.336950in}}%
\pgfpathlineto{\pgfqpoint{1.051312in}{1.338054in}}%
\pgfpathlineto{\pgfqpoint{1.052411in}{1.338679in}}%
\pgfpathlineto{\pgfqpoint{1.053825in}{1.339783in}}%
\pgfpathlineto{\pgfqpoint{1.055212in}{1.340794in}}%
\pgfpathlineto{\pgfqpoint{1.056877in}{1.341899in}}%
\pgfpathlineto{\pgfqpoint{1.057936in}{1.342444in}}%
\pgfpathlineto{\pgfqpoint{1.059618in}{1.343548in}}%
\pgfpathlineto{\pgfqpoint{1.060684in}{1.344214in}}%
\pgfpathlineto{\pgfqpoint{1.062443in}{1.345318in}}%
\pgfpathlineto{\pgfqpoint{1.063525in}{1.346023in}}%
\pgfpathlineto{\pgfqpoint{1.064996in}{1.347127in}}%
\pgfpathlineto{\pgfqpoint{1.066058in}{1.347912in}}%
\pgfpathlineto{\pgfqpoint{1.066101in}{1.347912in}}%
\pgfpathlineto{\pgfqpoint{1.067746in}{1.349016in}}%
\pgfpathlineto{\pgfqpoint{1.068855in}{1.349735in}}%
\pgfpathlineto{\pgfqpoint{1.070792in}{1.350839in}}%
\pgfpathlineto{\pgfqpoint{1.071871in}{1.351597in}}%
\pgfpathlineto{\pgfqpoint{1.071884in}{1.351597in}}%
\pgfpathlineto{\pgfqpoint{1.073455in}{1.352701in}}%
\pgfpathlineto{\pgfqpoint{1.074558in}{1.353380in}}%
\pgfpathlineto{\pgfqpoint{1.076491in}{1.354484in}}%
\pgfpathlineto{\pgfqpoint{1.077580in}{1.355109in}}%
\pgfpathlineto{\pgfqpoint{1.079037in}{1.356213in}}%
\pgfpathlineto{\pgfqpoint{1.080126in}{1.356905in}}%
\pgfpathlineto{\pgfqpoint{1.082130in}{1.358009in}}%
\pgfpathlineto{\pgfqpoint{1.083202in}{1.358768in}}%
\pgfpathlineto{\pgfqpoint{1.084931in}{1.359859in}}%
\pgfpathlineto{\pgfqpoint{1.086033in}{1.360697in}}%
\pgfpathlineto{\pgfqpoint{1.087742in}{1.361801in}}%
\pgfpathlineto{\pgfqpoint{1.088844in}{1.362572in}}%
\pgfpathlineto{\pgfqpoint{1.090978in}{1.363663in}}%
\pgfpathlineto{\pgfqpoint{1.092087in}{1.364302in}}%
\pgfpathlineto{\pgfqpoint{1.093876in}{1.365406in}}%
\pgfpathlineto{\pgfqpoint{1.094975in}{1.366284in}}%
\pgfpathlineto{\pgfqpoint{1.097009in}{1.367388in}}%
\pgfpathlineto{\pgfqpoint{1.098115in}{1.367974in}}%
\pgfpathlineto{\pgfqpoint{1.098118in}{1.367974in}}%
\pgfpathlineto{\pgfqpoint{1.099478in}{1.369078in}}%
\pgfpathlineto{\pgfqpoint{1.100570in}{1.369823in}}%
\pgfpathlineto{\pgfqpoint{1.100587in}{1.369823in}}%
\pgfpathlineto{\pgfqpoint{1.102132in}{1.370927in}}%
\pgfpathlineto{\pgfqpoint{1.103241in}{1.371632in}}%
\pgfpathlineto{\pgfqpoint{1.105097in}{1.372736in}}%
\pgfpathlineto{\pgfqpoint{1.106206in}{1.373428in}}%
\pgfpathlineto{\pgfqpoint{1.107935in}{1.374532in}}%
\pgfpathlineto{\pgfqpoint{1.109040in}{1.375251in}}%
\pgfpathlineto{\pgfqpoint{1.110890in}{1.376355in}}%
\pgfpathlineto{\pgfqpoint{1.111995in}{1.377100in}}%
\pgfpathlineto{\pgfqpoint{1.113734in}{1.378204in}}%
\pgfpathlineto{\pgfqpoint{1.114840in}{1.378869in}}%
\pgfpathlineto{\pgfqpoint{1.116284in}{1.379974in}}%
\pgfpathlineto{\pgfqpoint{1.117393in}{1.380546in}}%
\pgfpathlineto{\pgfqpoint{1.119098in}{1.381650in}}%
\pgfpathlineto{\pgfqpoint{1.120187in}{1.382475in}}%
\pgfpathlineto{\pgfqpoint{1.121923in}{1.383579in}}%
\pgfpathlineto{\pgfqpoint{1.123032in}{1.384430in}}%
\pgfpathlineto{\pgfqpoint{1.124549in}{1.385534in}}%
\pgfpathlineto{\pgfqpoint{1.125565in}{1.386120in}}%
\pgfpathlineto{\pgfqpoint{1.125638in}{1.386120in}}%
\pgfpathlineto{\pgfqpoint{1.127622in}{1.387224in}}%
\pgfpathlineto{\pgfqpoint{1.128707in}{1.387743in}}%
\pgfpathlineto{\pgfqpoint{1.128717in}{1.387743in}}%
\pgfpathlineto{\pgfqpoint{1.130731in}{1.388834in}}%
\pgfpathlineto{\pgfqpoint{1.131800in}{1.389499in}}%
\pgfpathlineto{\pgfqpoint{1.133740in}{1.390603in}}%
\pgfpathlineto{\pgfqpoint{1.134832in}{1.391375in}}%
\pgfpathlineto{\pgfqpoint{1.136400in}{1.392479in}}%
\pgfpathlineto{\pgfqpoint{1.137502in}{1.393144in}}%
\pgfpathlineto{\pgfqpoint{1.139456in}{1.394248in}}%
\pgfpathlineto{\pgfqpoint{1.140511in}{1.394914in}}%
\pgfpathlineto{\pgfqpoint{1.142246in}{1.396018in}}%
\pgfpathlineto{\pgfqpoint{1.143345in}{1.396736in}}%
\pgfpathlineto{\pgfqpoint{1.145650in}{1.397840in}}%
\pgfpathlineto{\pgfqpoint{1.146756in}{1.398426in}}%
\pgfpathlineto{\pgfqpoint{1.148683in}{1.399530in}}%
\pgfpathlineto{\pgfqpoint{1.149792in}{1.400208in}}%
\pgfpathlineto{\pgfqpoint{1.151859in}{1.401313in}}%
\pgfpathlineto{\pgfqpoint{1.152964in}{1.401964in}}%
\pgfpathlineto{\pgfqpoint{1.155122in}{1.403069in}}%
\pgfpathlineto{\pgfqpoint{1.156218in}{1.403814in}}%
\pgfpathlineto{\pgfqpoint{1.158389in}{1.404918in}}%
\pgfpathlineto{\pgfqpoint{1.159454in}{1.405410in}}%
\pgfpathlineto{\pgfqpoint{1.159461in}{1.405410in}}%
\pgfpathlineto{\pgfqpoint{1.161518in}{1.406514in}}%
\pgfpathlineto{\pgfqpoint{1.162617in}{1.407153in}}%
\pgfpathlineto{\pgfqpoint{1.164929in}{1.408244in}}%
\pgfpathlineto{\pgfqpoint{1.166018in}{1.408802in}}%
\pgfpathlineto{\pgfqpoint{1.167760in}{1.409907in}}%
\pgfpathlineto{\pgfqpoint{1.168869in}{1.410559in}}%
\pgfpathlineto{\pgfqpoint{1.170682in}{1.411663in}}%
\pgfpathlineto{\pgfqpoint{1.171784in}{1.412275in}}%
\pgfpathlineto{\pgfqpoint{1.173898in}{1.413379in}}%
\pgfpathlineto{\pgfqpoint{1.174997in}{1.413805in}}%
\pgfpathlineto{\pgfqpoint{1.177714in}{1.414909in}}%
\pgfpathlineto{\pgfqpoint{1.178736in}{1.415481in}}%
\pgfpathlineto{\pgfqpoint{1.178813in}{1.415481in}}%
\pgfpathlineto{\pgfqpoint{1.180850in}{1.416585in}}%
\pgfpathlineto{\pgfqpoint{1.181942in}{1.417277in}}%
\pgfpathlineto{\pgfqpoint{1.183825in}{1.418381in}}%
\pgfpathlineto{\pgfqpoint{1.184837in}{1.418953in}}%
\pgfpathlineto{\pgfqpoint{1.184861in}{1.418953in}}%
\pgfpathlineto{\pgfqpoint{1.187136in}{1.420057in}}%
\pgfpathlineto{\pgfqpoint{1.188174in}{1.420683in}}%
\pgfpathlineto{\pgfqpoint{1.188228in}{1.420683in}}%
\pgfpathlineto{\pgfqpoint{1.189997in}{1.421787in}}%
\pgfpathlineto{\pgfqpoint{1.191079in}{1.422572in}}%
\pgfpathlineto{\pgfqpoint{1.193210in}{1.423676in}}%
\pgfpathlineto{\pgfqpoint{1.194265in}{1.424115in}}%
\pgfpathlineto{\pgfqpoint{1.196634in}{1.425219in}}%
\pgfpathlineto{\pgfqpoint{1.197730in}{1.425911in}}%
\pgfpathlineto{\pgfqpoint{1.199877in}{1.427015in}}%
\pgfpathlineto{\pgfqpoint{1.200980in}{1.427428in}}%
\pgfpathlineto{\pgfqpoint{1.203204in}{1.428532in}}%
\pgfpathlineto{\pgfqpoint{1.204246in}{1.429157in}}%
\pgfpathlineto{\pgfqpoint{1.204280in}{1.429157in}}%
\pgfpathlineto{\pgfqpoint{1.206705in}{1.430261in}}%
\pgfpathlineto{\pgfqpoint{1.207764in}{1.430833in}}%
\pgfpathlineto{\pgfqpoint{1.210605in}{1.431924in}}%
\pgfpathlineto{\pgfqpoint{1.211708in}{1.432589in}}%
\pgfpathlineto{\pgfqpoint{1.213986in}{1.433640in}}%
\pgfpathlineto{\pgfqpoint{1.213986in}{1.433694in}}%
\pgfpathlineto{\pgfqpoint{1.215081in}{1.434172in}}%
\pgfpathlineto{\pgfqpoint{1.217805in}{1.435277in}}%
\pgfpathlineto{\pgfqpoint{1.218884in}{1.435716in}}%
\pgfpathlineto{\pgfqpoint{1.221226in}{1.436807in}}%
\pgfpathlineto{\pgfqpoint{1.222265in}{1.437259in}}%
\pgfpathlineto{\pgfqpoint{1.222288in}{1.437259in}}%
\pgfpathlineto{\pgfqpoint{1.224603in}{1.438363in}}%
\pgfpathlineto{\pgfqpoint{1.225712in}{1.438802in}}%
\pgfpathlineto{\pgfqpoint{1.228145in}{1.439906in}}%
\pgfpathlineto{\pgfqpoint{1.229116in}{1.440412in}}%
\pgfpathlineto{\pgfqpoint{1.229250in}{1.440412in}}%
\pgfpathlineto{\pgfqpoint{1.231837in}{1.441516in}}%
\pgfpathlineto{\pgfqpoint{1.232946in}{1.441968in}}%
\pgfpathlineto{\pgfqpoint{1.234842in}{1.443059in}}%
\pgfpathlineto{\pgfqpoint{1.234842in}{1.443073in}}%
\pgfpathlineto{\pgfqpoint{1.236956in}{1.444044in}}%
\pgfpathlineto{\pgfqpoint{1.238886in}{1.445135in}}%
\pgfpathlineto{\pgfqpoint{1.239952in}{1.445707in}}%
\pgfpathlineto{\pgfqpoint{1.239985in}{1.445707in}}%
\pgfpathlineto{\pgfqpoint{1.242042in}{1.446811in}}%
\pgfpathlineto{\pgfqpoint{1.243145in}{1.447463in}}%
\pgfpathlineto{\pgfqpoint{1.245235in}{1.448567in}}%
\pgfpathlineto{\pgfqpoint{1.246257in}{1.449059in}}%
\pgfpathlineto{\pgfqpoint{1.246274in}{1.449059in}}%
\pgfpathlineto{\pgfqpoint{1.248706in}{1.450163in}}%
\pgfpathlineto{\pgfqpoint{1.249789in}{1.450709in}}%
\pgfpathlineto{\pgfqpoint{1.252797in}{1.451800in}}%
\pgfpathlineto{\pgfqpoint{1.253906in}{1.452305in}}%
\pgfpathlineto{\pgfqpoint{1.256124in}{1.453409in}}%
\pgfpathlineto{\pgfqpoint{1.257217in}{1.453968in}}%
\pgfpathlineto{\pgfqpoint{1.259632in}{1.455072in}}%
\pgfpathlineto{\pgfqpoint{1.260684in}{1.455432in}}%
\pgfpathlineto{\pgfqpoint{1.260708in}{1.455432in}}%
\pgfpathlineto{\pgfqpoint{1.263344in}{1.456536in}}%
\pgfpathlineto{\pgfqpoint{1.264443in}{1.457148in}}%
\pgfpathlineto{\pgfqpoint{1.267365in}{1.458239in}}%
\pgfpathlineto{\pgfqpoint{1.268447in}{1.458651in}}%
\pgfpathlineto{\pgfqpoint{1.268461in}{1.458651in}}%
\pgfpathlineto{\pgfqpoint{1.270957in}{1.459755in}}%
\pgfpathlineto{\pgfqpoint{1.272042in}{1.460181in}}%
\pgfpathlineto{\pgfqpoint{1.274669in}{1.461285in}}%
\pgfpathlineto{\pgfqpoint{1.275738in}{1.461871in}}%
\pgfpathlineto{\pgfqpoint{1.278405in}{1.462975in}}%
\pgfpathlineto{\pgfqpoint{1.279510in}{1.463321in}}%
\pgfpathlineto{\pgfqpoint{1.282016in}{1.464425in}}%
\pgfpathlineto{\pgfqpoint{1.283243in}{1.464797in}}%
\pgfpathlineto{\pgfqpoint{1.286070in}{1.465902in}}%
\pgfpathlineto{\pgfqpoint{1.287166in}{1.466367in}}%
\pgfpathlineto{\pgfqpoint{1.290171in}{1.467471in}}%
\pgfpathlineto{\pgfqpoint{1.291243in}{1.468057in}}%
\pgfpathlineto{\pgfqpoint{1.293746in}{1.469161in}}%
\pgfpathlineto{\pgfqpoint{1.294852in}{1.469613in}}%
\pgfpathlineto{\pgfqpoint{1.297392in}{1.470717in}}%
\pgfpathlineto{\pgfqpoint{1.298444in}{1.471223in}}%
\pgfpathlineto{\pgfqpoint{1.298494in}{1.471223in}}%
\pgfpathlineto{\pgfqpoint{1.300990in}{1.472314in}}%
\pgfpathlineto{\pgfqpoint{1.302072in}{1.472700in}}%
\pgfpathlineto{\pgfqpoint{1.304943in}{1.473804in}}%
\pgfpathlineto{\pgfqpoint{1.306046in}{1.474070in}}%
\pgfpathlineto{\pgfqpoint{1.308870in}{1.475174in}}%
\pgfpathlineto{\pgfqpoint{1.309925in}{1.475613in}}%
\pgfpathlineto{\pgfqpoint{1.312241in}{1.476717in}}%
\pgfpathlineto{\pgfqpoint{1.313340in}{1.477236in}}%
\pgfpathlineto{\pgfqpoint{1.313350in}{1.477236in}}%
\pgfpathlineto{\pgfqpoint{1.316600in}{1.478340in}}%
\pgfpathlineto{\pgfqpoint{1.317615in}{1.478753in}}%
\pgfpathlineto{\pgfqpoint{1.320888in}{1.479857in}}%
\pgfpathlineto{\pgfqpoint{1.321933in}{1.480349in}}%
\pgfpathlineto{\pgfqpoint{1.321947in}{1.480349in}}%
\pgfpathlineto{\pgfqpoint{1.325009in}{1.481440in}}%
\pgfpathlineto{\pgfqpoint{1.326058in}{1.481986in}}%
\pgfpathlineto{\pgfqpoint{1.326088in}{1.481986in}}%
\pgfpathlineto{\pgfqpoint{1.329177in}{1.483090in}}%
\pgfpathlineto{\pgfqpoint{1.330370in}{1.483329in}}%
\pgfpathlineto{\pgfqpoint{1.333010in}{1.484433in}}%
\pgfpathlineto{\pgfqpoint{1.334102in}{1.484806in}}%
\pgfpathlineto{\pgfqpoint{1.337108in}{1.485910in}}%
\pgfpathlineto{\pgfqpoint{1.338186in}{1.486349in}}%
\pgfpathlineto{\pgfqpoint{1.341895in}{1.487453in}}%
\pgfpathlineto{\pgfqpoint{1.342994in}{1.487852in}}%
\pgfpathlineto{\pgfqpoint{1.346127in}{1.488957in}}%
\pgfpathlineto{\pgfqpoint{1.347209in}{1.489342in}}%
\pgfpathlineto{\pgfqpoint{1.350694in}{1.490447in}}%
\pgfpathlineto{\pgfqpoint{1.351792in}{1.491059in}}%
\pgfpathlineto{\pgfqpoint{1.355384in}{1.492163in}}%
\pgfpathlineto{\pgfqpoint{1.356329in}{1.492549in}}%
\pgfpathlineto{\pgfqpoint{1.356406in}{1.492549in}}%
\pgfpathlineto{\pgfqpoint{1.359448in}{1.493653in}}%
\pgfpathlineto{\pgfqpoint{1.360547in}{1.494252in}}%
\pgfpathlineto{\pgfqpoint{1.363482in}{1.495356in}}%
\pgfpathlineto{\pgfqpoint{1.364544in}{1.495835in}}%
\pgfpathlineto{\pgfqpoint{1.364561in}{1.495835in}}%
\pgfpathlineto{\pgfqpoint{1.367831in}{1.496939in}}%
\pgfpathlineto{\pgfqpoint{1.368927in}{1.497365in}}%
\pgfpathlineto{\pgfqpoint{1.371845in}{1.498469in}}%
\pgfpathlineto{\pgfqpoint{1.372930in}{1.498775in}}%
\pgfpathlineto{\pgfqpoint{1.375651in}{1.499879in}}%
\pgfpathlineto{\pgfqpoint{1.376753in}{1.500371in}}%
\pgfpathlineto{\pgfqpoint{1.376760in}{1.500371in}}%
\pgfpathlineto{\pgfqpoint{1.380522in}{1.501475in}}%
\pgfpathlineto{\pgfqpoint{1.381735in}{1.501781in}}%
\pgfpathlineto{\pgfqpoint{1.384422in}{1.502859in}}%
\pgfpathlineto{\pgfqpoint{1.385518in}{1.503311in}}%
\pgfpathlineto{\pgfqpoint{1.389495in}{1.504415in}}%
\pgfpathlineto{\pgfqpoint{1.390570in}{1.504708in}}%
\pgfpathlineto{\pgfqpoint{1.390601in}{1.504708in}}%
\pgfpathlineto{\pgfqpoint{1.394634in}{1.505812in}}%
\pgfpathlineto{\pgfqpoint{1.395737in}{1.506238in}}%
\pgfpathlineto{\pgfqpoint{1.399171in}{1.507342in}}%
\pgfpathlineto{\pgfqpoint{1.400267in}{1.507755in}}%
\pgfpathlineto{\pgfqpoint{1.400280in}{1.507755in}}%
\pgfpathlineto{\pgfqpoint{1.403610in}{1.508859in}}%
\pgfpathlineto{\pgfqpoint{1.404575in}{1.509205in}}%
\pgfpathlineto{\pgfqpoint{1.404679in}{1.509205in}}%
\pgfpathlineto{\pgfqpoint{1.407453in}{1.510309in}}%
\pgfpathlineto{\pgfqpoint{1.408539in}{1.510655in}}%
\pgfpathlineto{\pgfqpoint{1.412707in}{1.511759in}}%
\pgfpathlineto{\pgfqpoint{1.413812in}{1.512052in}}%
\pgfpathlineto{\pgfqpoint{1.413816in}{1.512052in}}%
\pgfpathlineto{\pgfqpoint{1.416848in}{1.513156in}}%
\pgfpathlineto{\pgfqpoint{1.417943in}{1.513515in}}%
\pgfpathlineto{\pgfqpoint{1.421126in}{1.514619in}}%
\pgfpathlineto{\pgfqpoint{1.422155in}{1.514912in}}%
\pgfpathlineto{\pgfqpoint{1.422225in}{1.514912in}}%
\pgfpathlineto{\pgfqpoint{1.426162in}{1.516016in}}%
\pgfpathlineto{\pgfqpoint{1.427214in}{1.516309in}}%
\pgfpathlineto{\pgfqpoint{1.431094in}{1.517413in}}%
\pgfpathlineto{\pgfqpoint{1.431992in}{1.517639in}}%
\pgfpathlineto{\pgfqpoint{1.435483in}{1.518743in}}%
\pgfpathlineto{\pgfqpoint{1.436475in}{1.518970in}}%
\pgfpathlineto{\pgfqpoint{1.436485in}{1.518970in}}%
\pgfpathlineto{\pgfqpoint{1.440720in}{1.520074in}}%
\pgfpathlineto{\pgfqpoint{1.441802in}{1.520460in}}%
\pgfpathlineto{\pgfqpoint{1.441825in}{1.520460in}}%
\pgfpathlineto{\pgfqpoint{1.446194in}{1.521564in}}%
\pgfpathlineto{\pgfqpoint{1.447266in}{1.522029in}}%
\pgfpathlineto{\pgfqpoint{1.447303in}{1.522029in}}%
\pgfpathlineto{\pgfqpoint{1.451039in}{1.523134in}}%
\pgfpathlineto{\pgfqpoint{1.452148in}{1.523466in}}%
\pgfpathlineto{\pgfqpoint{1.455304in}{1.524570in}}%
\pgfpathlineto{\pgfqpoint{1.456343in}{1.524930in}}%
\pgfpathlineto{\pgfqpoint{1.459589in}{1.526034in}}%
\pgfpathlineto{\pgfqpoint{1.460695in}{1.526366in}}%
\pgfpathlineto{\pgfqpoint{1.466133in}{1.527471in}}%
\pgfpathlineto{\pgfqpoint{1.467195in}{1.527843in}}%
\pgfpathlineto{\pgfqpoint{1.467211in}{1.527843in}}%
\pgfpathlineto{\pgfqpoint{1.470910in}{1.528947in}}%
\pgfpathlineto{\pgfqpoint{1.472006in}{1.529280in}}%
\pgfpathlineto{\pgfqpoint{1.476898in}{1.530384in}}%
\pgfpathlineto{\pgfqpoint{1.477956in}{1.530690in}}%
\pgfpathlineto{\pgfqpoint{1.478003in}{1.530690in}}%
\pgfpathlineto{\pgfqpoint{1.482379in}{1.531794in}}%
\pgfpathlineto{\pgfqpoint{1.483485in}{1.531994in}}%
\pgfpathlineto{\pgfqpoint{1.489679in}{1.533098in}}%
\pgfpathlineto{\pgfqpoint{1.490768in}{1.533351in}}%
\pgfpathlineto{\pgfqpoint{1.490775in}{1.533351in}}%
\pgfpathlineto{\pgfqpoint{1.495241in}{1.534455in}}%
\pgfpathlineto{\pgfqpoint{1.496146in}{1.534788in}}%
\pgfpathlineto{\pgfqpoint{1.496320in}{1.534788in}}%
\pgfpathlineto{\pgfqpoint{1.500116in}{1.535892in}}%
\pgfpathlineto{\pgfqpoint{1.501148in}{1.536171in}}%
\pgfpathlineto{\pgfqpoint{1.501188in}{1.536171in}}%
\pgfpathlineto{\pgfqpoint{1.506468in}{1.537275in}}%
\pgfpathlineto{\pgfqpoint{1.507521in}{1.537581in}}%
\pgfpathlineto{\pgfqpoint{1.507534in}{1.537581in}}%
\pgfpathlineto{\pgfqpoint{1.512027in}{1.538672in}}%
\pgfpathlineto{\pgfqpoint{1.512915in}{1.538965in}}%
\pgfpathlineto{\pgfqpoint{1.513112in}{1.538965in}}%
\pgfpathlineto{\pgfqpoint{1.517515in}{1.540069in}}%
\pgfpathlineto{\pgfqpoint{1.518708in}{1.540388in}}%
\pgfpathlineto{\pgfqpoint{1.523428in}{1.541493in}}%
\pgfpathlineto{\pgfqpoint{1.524524in}{1.541892in}}%
\pgfpathlineto{\pgfqpoint{1.524531in}{1.541892in}}%
\pgfpathlineto{\pgfqpoint{1.529084in}{1.542996in}}%
\pgfpathlineto{\pgfqpoint{1.530136in}{1.543195in}}%
\pgfpathlineto{\pgfqpoint{1.530163in}{1.543195in}}%
\pgfpathlineto{\pgfqpoint{1.533962in}{1.544300in}}%
\pgfpathlineto{\pgfqpoint{1.535041in}{1.544606in}}%
\pgfpathlineto{\pgfqpoint{1.540060in}{1.545710in}}%
\pgfpathlineto{\pgfqpoint{1.541112in}{1.545830in}}%
\pgfpathlineto{\pgfqpoint{1.546057in}{1.546934in}}%
\pgfpathlineto{\pgfqpoint{1.547143in}{1.547226in}}%
\pgfpathlineto{\pgfqpoint{1.547166in}{1.547226in}}%
\pgfpathlineto{\pgfqpoint{1.551756in}{1.548331in}}%
\pgfpathlineto{\pgfqpoint{1.552634in}{1.548557in}}%
\pgfpathlineto{\pgfqpoint{1.552849in}{1.548557in}}%
\pgfpathlineto{\pgfqpoint{1.558377in}{1.549661in}}%
\pgfpathlineto{\pgfqpoint{1.559466in}{1.549967in}}%
\pgfpathlineto{\pgfqpoint{1.564307in}{1.551071in}}%
\pgfpathlineto{\pgfqpoint{1.565393in}{1.551324in}}%
\pgfpathlineto{\pgfqpoint{1.570666in}{1.552428in}}%
\pgfpathlineto{\pgfqpoint{1.571651in}{1.552654in}}%
\pgfpathlineto{\pgfqpoint{1.571775in}{1.552654in}}%
\pgfpathlineto{\pgfqpoint{1.577287in}{1.553759in}}%
\pgfpathlineto{\pgfqpoint{1.578355in}{1.554011in}}%
\pgfpathlineto{\pgfqpoint{1.578396in}{1.554011in}}%
\pgfpathlineto{\pgfqpoint{1.584148in}{1.555115in}}%
\pgfpathlineto{\pgfqpoint{1.585140in}{1.555275in}}%
\pgfpathlineto{\pgfqpoint{1.585237in}{1.555275in}}%
\pgfpathlineto{\pgfqpoint{1.591000in}{1.556379in}}%
\pgfpathlineto{\pgfqpoint{1.591992in}{1.556579in}}%
\pgfpathlineto{\pgfqpoint{1.592085in}{1.556579in}}%
\pgfpathlineto{\pgfqpoint{1.598123in}{1.557683in}}%
\pgfpathlineto{\pgfqpoint{1.599212in}{1.558016in}}%
\pgfpathlineto{\pgfqpoint{1.604606in}{1.559120in}}%
\pgfpathlineto{\pgfqpoint{1.605537in}{1.559373in}}%
\pgfpathlineto{\pgfqpoint{1.605702in}{1.559373in}}%
\pgfpathlineto{\pgfqpoint{1.611401in}{1.560477in}}%
\pgfpathlineto{\pgfqpoint{1.612382in}{1.560743in}}%
\pgfpathlineto{\pgfqpoint{1.618976in}{1.561847in}}%
\pgfpathlineto{\pgfqpoint{1.619904in}{1.562126in}}%
\pgfpathlineto{\pgfqpoint{1.620075in}{1.562126in}}%
\pgfpathlineto{\pgfqpoint{1.626454in}{1.563231in}}%
\pgfpathlineto{\pgfqpoint{1.627499in}{1.563457in}}%
\pgfpathlineto{\pgfqpoint{1.634951in}{1.564561in}}%
\pgfpathlineto{\pgfqpoint{1.636050in}{1.564787in}}%
\pgfpathlineto{\pgfqpoint{1.642580in}{1.565891in}}%
\pgfpathlineto{\pgfqpoint{1.643669in}{1.566131in}}%
\pgfpathlineto{\pgfqpoint{1.648936in}{1.567235in}}%
\pgfpathlineto{\pgfqpoint{1.649954in}{1.567488in}}%
\pgfpathlineto{\pgfqpoint{1.650038in}{1.567488in}}%
\pgfpathlineto{\pgfqpoint{1.657285in}{1.568592in}}%
\pgfpathlineto{\pgfqpoint{1.658384in}{1.568698in}}%
\pgfpathlineto{\pgfqpoint{1.665822in}{1.569803in}}%
\pgfpathlineto{\pgfqpoint{1.666767in}{1.569936in}}%
\pgfpathlineto{\pgfqpoint{1.666870in}{1.569936in}}%
\pgfpathlineto{\pgfqpoint{1.673752in}{1.571040in}}%
\pgfpathlineto{\pgfqpoint{1.674529in}{1.571200in}}%
\pgfpathlineto{\pgfqpoint{1.674556in}{1.571200in}}%
\pgfpathlineto{\pgfqpoint{1.681177in}{1.572304in}}%
\pgfpathlineto{\pgfqpoint{1.682162in}{1.572543in}}%
\pgfpathlineto{\pgfqpoint{1.682286in}{1.572543in}}%
\pgfpathlineto{\pgfqpoint{1.688621in}{1.573647in}}%
\pgfpathlineto{\pgfqpoint{1.689583in}{1.573794in}}%
\pgfpathlineto{\pgfqpoint{1.689677in}{1.573794in}}%
\pgfpathlineto{\pgfqpoint{1.696378in}{1.574898in}}%
\pgfpathlineto{\pgfqpoint{1.697426in}{1.575058in}}%
\pgfpathlineto{\pgfqpoint{1.697487in}{1.575058in}}%
\pgfpathlineto{\pgfqpoint{1.705444in}{1.576162in}}%
\pgfpathlineto{\pgfqpoint{1.706553in}{1.576335in}}%
\pgfpathlineto{\pgfqpoint{1.714711in}{1.577439in}}%
\pgfpathlineto{\pgfqpoint{1.715817in}{1.577599in}}%
\pgfpathlineto{\pgfqpoint{1.722779in}{1.578703in}}%
\pgfpathlineto{\pgfqpoint{1.723741in}{1.578889in}}%
\pgfpathlineto{\pgfqpoint{1.723871in}{1.578889in}}%
\pgfpathlineto{\pgfqpoint{1.731554in}{1.579993in}}%
\pgfpathlineto{\pgfqpoint{1.732552in}{1.580166in}}%
\pgfpathlineto{\pgfqpoint{1.732656in}{1.580166in}}%
\pgfpathlineto{\pgfqpoint{1.740613in}{1.581257in}}%
\pgfpathlineto{\pgfqpoint{1.741706in}{1.581497in}}%
\pgfpathlineto{\pgfqpoint{1.749981in}{1.582601in}}%
\pgfpathlineto{\pgfqpoint{1.750842in}{1.582694in}}%
\pgfpathlineto{\pgfqpoint{1.760274in}{1.583798in}}%
\pgfpathlineto{\pgfqpoint{1.761363in}{1.584038in}}%
\pgfpathlineto{\pgfqpoint{1.770482in}{1.585142in}}%
\pgfpathlineto{\pgfqpoint{1.771283in}{1.585248in}}%
\pgfpathlineto{\pgfqpoint{1.771457in}{1.585248in}}%
\pgfpathlineto{\pgfqpoint{1.782008in}{1.586339in}}%
\pgfpathlineto{\pgfqpoint{1.783083in}{1.586539in}}%
\pgfpathlineto{\pgfqpoint{1.793986in}{1.587643in}}%
\pgfpathlineto{\pgfqpoint{1.795085in}{1.587816in}}%
\pgfpathlineto{\pgfqpoint{1.807749in}{1.588920in}}%
\pgfpathlineto{\pgfqpoint{1.808835in}{1.589146in}}%
\pgfpathlineto{\pgfqpoint{1.808848in}{1.589146in}}%
\pgfpathlineto{\pgfqpoint{1.820263in}{1.590250in}}%
\pgfpathlineto{\pgfqpoint{1.821416in}{1.590330in}}%
\pgfpathlineto{\pgfqpoint{1.834134in}{1.591434in}}%
\pgfpathlineto{\pgfqpoint{1.834787in}{1.591514in}}%
\pgfpathlineto{\pgfqpoint{1.835230in}{1.591514in}}%
\pgfpathlineto{\pgfqpoint{1.847264in}{1.592618in}}%
\pgfpathlineto{\pgfqpoint{1.848176in}{1.592725in}}%
\pgfpathlineto{\pgfqpoint{1.848233in}{1.592725in}}%
\pgfpathlineto{\pgfqpoint{1.861363in}{1.593829in}}%
\pgfpathlineto{\pgfqpoint{1.862170in}{1.593909in}}%
\pgfpathlineto{\pgfqpoint{1.862432in}{1.593909in}}%
\pgfpathlineto{\pgfqpoint{1.878879in}{1.595013in}}%
\pgfpathlineto{\pgfqpoint{1.879954in}{1.595173in}}%
\pgfpathlineto{\pgfqpoint{1.894706in}{1.596277in}}%
\pgfpathlineto{\pgfqpoint{1.895802in}{1.596397in}}%
\pgfpathlineto{\pgfqpoint{1.911864in}{1.597501in}}%
\pgfpathlineto{\pgfqpoint{1.912933in}{1.597607in}}%
\pgfpathlineto{\pgfqpoint{1.912943in}{1.597607in}}%
\pgfpathlineto{\pgfqpoint{1.934446in}{1.598711in}}%
\pgfpathlineto{\pgfqpoint{1.935451in}{1.598765in}}%
\pgfpathlineto{\pgfqpoint{1.935491in}{1.598765in}}%
\pgfpathlineto{\pgfqpoint{1.959497in}{1.599869in}}%
\pgfpathlineto{\pgfqpoint{1.960462in}{1.599949in}}%
\pgfpathlineto{\pgfqpoint{1.960599in}{1.599949in}}%
\pgfpathlineto{\pgfqpoint{1.990599in}{1.601053in}}%
\pgfpathlineto{\pgfqpoint{1.991460in}{1.601079in}}%
\pgfpathlineto{\pgfqpoint{1.991708in}{1.601079in}}%
\pgfpathlineto{\pgfqpoint{2.033126in}{1.601944in}}%
\pgfpathlineto{\pgfqpoint{2.033126in}{1.601944in}}%
\pgfusepath{stroke}%
\end{pgfscope}%
\begin{pgfscope}%
\pgfsetrectcap%
\pgfsetmiterjoin%
\pgfsetlinewidth{0.803000pt}%
\definecolor{currentstroke}{rgb}{0.000000,0.000000,0.000000}%
\pgfsetstrokecolor{currentstroke}%
\pgfsetdash{}{0pt}%
\pgfpathmoveto{\pgfqpoint{0.553581in}{0.499444in}}%
\pgfpathlineto{\pgfqpoint{0.553581in}{1.654444in}}%
\pgfusepath{stroke}%
\end{pgfscope}%
\begin{pgfscope}%
\pgfsetrectcap%
\pgfsetmiterjoin%
\pgfsetlinewidth{0.803000pt}%
\definecolor{currentstroke}{rgb}{0.000000,0.000000,0.000000}%
\pgfsetstrokecolor{currentstroke}%
\pgfsetdash{}{0pt}%
\pgfpathmoveto{\pgfqpoint{2.103581in}{0.499444in}}%
\pgfpathlineto{\pgfqpoint{2.103581in}{1.654444in}}%
\pgfusepath{stroke}%
\end{pgfscope}%
\begin{pgfscope}%
\pgfsetrectcap%
\pgfsetmiterjoin%
\pgfsetlinewidth{0.803000pt}%
\definecolor{currentstroke}{rgb}{0.000000,0.000000,0.000000}%
\pgfsetstrokecolor{currentstroke}%
\pgfsetdash{}{0pt}%
\pgfpathmoveto{\pgfqpoint{0.553581in}{0.499444in}}%
\pgfpathlineto{\pgfqpoint{2.103581in}{0.499444in}}%
\pgfusepath{stroke}%
\end{pgfscope}%
\begin{pgfscope}%
\pgfsetrectcap%
\pgfsetmiterjoin%
\pgfsetlinewidth{0.803000pt}%
\definecolor{currentstroke}{rgb}{0.000000,0.000000,0.000000}%
\pgfsetstrokecolor{currentstroke}%
\pgfsetdash{}{0pt}%
\pgfpathmoveto{\pgfqpoint{0.553581in}{1.654444in}}%
\pgfpathlineto{\pgfqpoint{2.103581in}{1.654444in}}%
\pgfusepath{stroke}%
\end{pgfscope}%
\begin{pgfscope}%
\pgfsetbuttcap%
\pgfsetmiterjoin%
\definecolor{currentfill}{rgb}{1.000000,1.000000,1.000000}%
\pgfsetfillcolor{currentfill}%
\pgfsetfillopacity{0.800000}%
\pgfsetlinewidth{1.003750pt}%
\definecolor{currentstroke}{rgb}{0.800000,0.800000,0.800000}%
\pgfsetstrokecolor{currentstroke}%
\pgfsetstrokeopacity{0.800000}%
\pgfsetdash{}{0pt}%
\pgfpathmoveto{\pgfqpoint{0.832747in}{0.568889in}}%
\pgfpathlineto{\pgfqpoint{2.006358in}{0.568889in}}%
\pgfpathquadraticcurveto{\pgfqpoint{2.034136in}{0.568889in}}{\pgfqpoint{2.034136in}{0.596666in}}%
\pgfpathlineto{\pgfqpoint{2.034136in}{0.776388in}}%
\pgfpathquadraticcurveto{\pgfqpoint{2.034136in}{0.804166in}}{\pgfqpoint{2.006358in}{0.804166in}}%
\pgfpathlineto{\pgfqpoint{0.832747in}{0.804166in}}%
\pgfpathquadraticcurveto{\pgfqpoint{0.804970in}{0.804166in}}{\pgfqpoint{0.804970in}{0.776388in}}%
\pgfpathlineto{\pgfqpoint{0.804970in}{0.596666in}}%
\pgfpathquadraticcurveto{\pgfqpoint{0.804970in}{0.568889in}}{\pgfqpoint{0.832747in}{0.568889in}}%
\pgfpathlineto{\pgfqpoint{0.832747in}{0.568889in}}%
\pgfpathclose%
\pgfusepath{stroke,fill}%
\end{pgfscope}%
\begin{pgfscope}%
\pgfsetrectcap%
\pgfsetroundjoin%
\pgfsetlinewidth{1.505625pt}%
\definecolor{currentstroke}{rgb}{0.000000,0.000000,0.000000}%
\pgfsetstrokecolor{currentstroke}%
\pgfsetdash{}{0pt}%
\pgfpathmoveto{\pgfqpoint{0.860525in}{0.700000in}}%
\pgfpathlineto{\pgfqpoint{0.999414in}{0.700000in}}%
\pgfpathlineto{\pgfqpoint{1.138303in}{0.700000in}}%
\pgfusepath{stroke}%
\end{pgfscope}%
\begin{pgfscope}%
\definecolor{textcolor}{rgb}{0.000000,0.000000,0.000000}%
\pgfsetstrokecolor{textcolor}%
\pgfsetfillcolor{textcolor}%
\pgftext[x=1.249414in,y=0.651388in,left,base]{\color{textcolor}\rmfamily\fontsize{10.000000}{12.000000}\selectfont AUC=0.800}%
\end{pgfscope}%
\end{pgfpicture}%
\makeatother%
\endgroup%

\end{tabular}

\verb|y_proba = estimator.predict_proba(X_test)|


\noindent\begin{tabular}{@{\hspace{-6pt}}p{4.5in} @{\hspace{-6pt}}p{2.0in}}
	\vskip 0pt
	\qquad \qquad Raw Model Output on Test Set
	
	%% Creator: Matplotlib, PGF backend
%%
%% To include the figure in your LaTeX document, write
%%   \input{<filename>.pgf}
%%
%% Make sure the required packages are loaded in your preamble
%%   \usepackage{pgf}
%%
%% Also ensure that all the required font packages are loaded; for instance,
%% the lmodern package is sometimes necessary when using math font.
%%   \usepackage{lmodern}
%%
%% Figures using additional raster images can only be included by \input if
%% they are in the same directory as the main LaTeX file. For loading figures
%% from other directories you can use the `import` package
%%   \usepackage{import}
%%
%% and then include the figures with
%%   \import{<path to file>}{<filename>.pgf}
%%
%% Matplotlib used the following preamble
%%   
%%   \usepackage{fontspec}
%%   \makeatletter\@ifpackageloaded{underscore}{}{\usepackage[strings]{underscore}}\makeatother
%%
\begingroup%
\makeatletter%
\begin{pgfpicture}%
\pgfpathrectangle{\pgfpointorigin}{\pgfqpoint{4.102500in}{1.754444in}}%
\pgfusepath{use as bounding box, clip}%
\begin{pgfscope}%
\pgfsetbuttcap%
\pgfsetmiterjoin%
\definecolor{currentfill}{rgb}{1.000000,1.000000,1.000000}%
\pgfsetfillcolor{currentfill}%
\pgfsetlinewidth{0.000000pt}%
\definecolor{currentstroke}{rgb}{1.000000,1.000000,1.000000}%
\pgfsetstrokecolor{currentstroke}%
\pgfsetdash{}{0pt}%
\pgfpathmoveto{\pgfqpoint{0.000000in}{0.000000in}}%
\pgfpathlineto{\pgfqpoint{4.102500in}{0.000000in}}%
\pgfpathlineto{\pgfqpoint{4.102500in}{1.754444in}}%
\pgfpathlineto{\pgfqpoint{0.000000in}{1.754444in}}%
\pgfpathlineto{\pgfqpoint{0.000000in}{0.000000in}}%
\pgfpathclose%
\pgfusepath{fill}%
\end{pgfscope}%
\begin{pgfscope}%
\pgfsetbuttcap%
\pgfsetmiterjoin%
\definecolor{currentfill}{rgb}{1.000000,1.000000,1.000000}%
\pgfsetfillcolor{currentfill}%
\pgfsetlinewidth{0.000000pt}%
\definecolor{currentstroke}{rgb}{0.000000,0.000000,0.000000}%
\pgfsetstrokecolor{currentstroke}%
\pgfsetstrokeopacity{0.000000}%
\pgfsetdash{}{0pt}%
\pgfpathmoveto{\pgfqpoint{0.515000in}{0.499444in}}%
\pgfpathlineto{\pgfqpoint{4.002500in}{0.499444in}}%
\pgfpathlineto{\pgfqpoint{4.002500in}{1.654444in}}%
\pgfpathlineto{\pgfqpoint{0.515000in}{1.654444in}}%
\pgfpathlineto{\pgfqpoint{0.515000in}{0.499444in}}%
\pgfpathclose%
\pgfusepath{fill}%
\end{pgfscope}%
\begin{pgfscope}%
\pgfpathrectangle{\pgfqpoint{0.515000in}{0.499444in}}{\pgfqpoint{3.487500in}{1.155000in}}%
\pgfusepath{clip}%
\pgfsetbuttcap%
\pgfsetmiterjoin%
\pgfsetlinewidth{1.003750pt}%
\definecolor{currentstroke}{rgb}{0.000000,0.000000,0.000000}%
\pgfsetstrokecolor{currentstroke}%
\pgfsetdash{}{0pt}%
\pgfpathmoveto{\pgfqpoint{0.610114in}{0.499444in}}%
\pgfpathlineto{\pgfqpoint{0.673523in}{0.499444in}}%
\pgfpathlineto{\pgfqpoint{0.673523in}{0.499444in}}%
\pgfpathlineto{\pgfqpoint{0.610114in}{0.499444in}}%
\pgfpathlineto{\pgfqpoint{0.610114in}{0.499444in}}%
\pgfpathclose%
\pgfusepath{stroke}%
\end{pgfscope}%
\begin{pgfscope}%
\pgfpathrectangle{\pgfqpoint{0.515000in}{0.499444in}}{\pgfqpoint{3.487500in}{1.155000in}}%
\pgfusepath{clip}%
\pgfsetbuttcap%
\pgfsetmiterjoin%
\pgfsetlinewidth{1.003750pt}%
\definecolor{currentstroke}{rgb}{0.000000,0.000000,0.000000}%
\pgfsetstrokecolor{currentstroke}%
\pgfsetdash{}{0pt}%
\pgfpathmoveto{\pgfqpoint{0.768637in}{0.499444in}}%
\pgfpathlineto{\pgfqpoint{0.832046in}{0.499444in}}%
\pgfpathlineto{\pgfqpoint{0.832046in}{1.599444in}}%
\pgfpathlineto{\pgfqpoint{0.768637in}{1.599444in}}%
\pgfpathlineto{\pgfqpoint{0.768637in}{0.499444in}}%
\pgfpathclose%
\pgfusepath{stroke}%
\end{pgfscope}%
\begin{pgfscope}%
\pgfpathrectangle{\pgfqpoint{0.515000in}{0.499444in}}{\pgfqpoint{3.487500in}{1.155000in}}%
\pgfusepath{clip}%
\pgfsetbuttcap%
\pgfsetmiterjoin%
\pgfsetlinewidth{1.003750pt}%
\definecolor{currentstroke}{rgb}{0.000000,0.000000,0.000000}%
\pgfsetstrokecolor{currentstroke}%
\pgfsetdash{}{0pt}%
\pgfpathmoveto{\pgfqpoint{0.927159in}{0.499444in}}%
\pgfpathlineto{\pgfqpoint{0.990568in}{0.499444in}}%
\pgfpathlineto{\pgfqpoint{0.990568in}{1.346951in}}%
\pgfpathlineto{\pgfqpoint{0.927159in}{1.346951in}}%
\pgfpathlineto{\pgfqpoint{0.927159in}{0.499444in}}%
\pgfpathclose%
\pgfusepath{stroke}%
\end{pgfscope}%
\begin{pgfscope}%
\pgfpathrectangle{\pgfqpoint{0.515000in}{0.499444in}}{\pgfqpoint{3.487500in}{1.155000in}}%
\pgfusepath{clip}%
\pgfsetbuttcap%
\pgfsetmiterjoin%
\pgfsetlinewidth{1.003750pt}%
\definecolor{currentstroke}{rgb}{0.000000,0.000000,0.000000}%
\pgfsetstrokecolor{currentstroke}%
\pgfsetdash{}{0pt}%
\pgfpathmoveto{\pgfqpoint{1.085682in}{0.499444in}}%
\pgfpathlineto{\pgfqpoint{1.149091in}{0.499444in}}%
\pgfpathlineto{\pgfqpoint{1.149091in}{1.032251in}}%
\pgfpathlineto{\pgfqpoint{1.085682in}{1.032251in}}%
\pgfpathlineto{\pgfqpoint{1.085682in}{0.499444in}}%
\pgfpathclose%
\pgfusepath{stroke}%
\end{pgfscope}%
\begin{pgfscope}%
\pgfpathrectangle{\pgfqpoint{0.515000in}{0.499444in}}{\pgfqpoint{3.487500in}{1.155000in}}%
\pgfusepath{clip}%
\pgfsetbuttcap%
\pgfsetmiterjoin%
\pgfsetlinewidth{1.003750pt}%
\definecolor{currentstroke}{rgb}{0.000000,0.000000,0.000000}%
\pgfsetstrokecolor{currentstroke}%
\pgfsetdash{}{0pt}%
\pgfpathmoveto{\pgfqpoint{1.244205in}{0.499444in}}%
\pgfpathlineto{\pgfqpoint{1.307614in}{0.499444in}}%
\pgfpathlineto{\pgfqpoint{1.307614in}{0.848431in}}%
\pgfpathlineto{\pgfqpoint{1.244205in}{0.848431in}}%
\pgfpathlineto{\pgfqpoint{1.244205in}{0.499444in}}%
\pgfpathclose%
\pgfusepath{stroke}%
\end{pgfscope}%
\begin{pgfscope}%
\pgfpathrectangle{\pgfqpoint{0.515000in}{0.499444in}}{\pgfqpoint{3.487500in}{1.155000in}}%
\pgfusepath{clip}%
\pgfsetbuttcap%
\pgfsetmiterjoin%
\pgfsetlinewidth{1.003750pt}%
\definecolor{currentstroke}{rgb}{0.000000,0.000000,0.000000}%
\pgfsetstrokecolor{currentstroke}%
\pgfsetdash{}{0pt}%
\pgfpathmoveto{\pgfqpoint{1.402728in}{0.499444in}}%
\pgfpathlineto{\pgfqpoint{1.466137in}{0.499444in}}%
\pgfpathlineto{\pgfqpoint{1.466137in}{0.740608in}}%
\pgfpathlineto{\pgfqpoint{1.402728in}{0.740608in}}%
\pgfpathlineto{\pgfqpoint{1.402728in}{0.499444in}}%
\pgfpathclose%
\pgfusepath{stroke}%
\end{pgfscope}%
\begin{pgfscope}%
\pgfpathrectangle{\pgfqpoint{0.515000in}{0.499444in}}{\pgfqpoint{3.487500in}{1.155000in}}%
\pgfusepath{clip}%
\pgfsetbuttcap%
\pgfsetmiterjoin%
\pgfsetlinewidth{1.003750pt}%
\definecolor{currentstroke}{rgb}{0.000000,0.000000,0.000000}%
\pgfsetstrokecolor{currentstroke}%
\pgfsetdash{}{0pt}%
\pgfpathmoveto{\pgfqpoint{1.561250in}{0.499444in}}%
\pgfpathlineto{\pgfqpoint{1.624659in}{0.499444in}}%
\pgfpathlineto{\pgfqpoint{1.624659in}{0.660907in}}%
\pgfpathlineto{\pgfqpoint{1.561250in}{0.660907in}}%
\pgfpathlineto{\pgfqpoint{1.561250in}{0.499444in}}%
\pgfpathclose%
\pgfusepath{stroke}%
\end{pgfscope}%
\begin{pgfscope}%
\pgfpathrectangle{\pgfqpoint{0.515000in}{0.499444in}}{\pgfqpoint{3.487500in}{1.155000in}}%
\pgfusepath{clip}%
\pgfsetbuttcap%
\pgfsetmiterjoin%
\pgfsetlinewidth{1.003750pt}%
\definecolor{currentstroke}{rgb}{0.000000,0.000000,0.000000}%
\pgfsetstrokecolor{currentstroke}%
\pgfsetdash{}{0pt}%
\pgfpathmoveto{\pgfqpoint{1.719773in}{0.499444in}}%
\pgfpathlineto{\pgfqpoint{1.783182in}{0.499444in}}%
\pgfpathlineto{\pgfqpoint{1.783182in}{0.611530in}}%
\pgfpathlineto{\pgfqpoint{1.719773in}{0.611530in}}%
\pgfpathlineto{\pgfqpoint{1.719773in}{0.499444in}}%
\pgfpathclose%
\pgfusepath{stroke}%
\end{pgfscope}%
\begin{pgfscope}%
\pgfpathrectangle{\pgfqpoint{0.515000in}{0.499444in}}{\pgfqpoint{3.487500in}{1.155000in}}%
\pgfusepath{clip}%
\pgfsetbuttcap%
\pgfsetmiterjoin%
\pgfsetlinewidth{1.003750pt}%
\definecolor{currentstroke}{rgb}{0.000000,0.000000,0.000000}%
\pgfsetstrokecolor{currentstroke}%
\pgfsetdash{}{0pt}%
\pgfpathmoveto{\pgfqpoint{1.878296in}{0.499444in}}%
\pgfpathlineto{\pgfqpoint{1.941705in}{0.499444in}}%
\pgfpathlineto{\pgfqpoint{1.941705in}{0.577063in}}%
\pgfpathlineto{\pgfqpoint{1.878296in}{0.577063in}}%
\pgfpathlineto{\pgfqpoint{1.878296in}{0.499444in}}%
\pgfpathclose%
\pgfusepath{stroke}%
\end{pgfscope}%
\begin{pgfscope}%
\pgfpathrectangle{\pgfqpoint{0.515000in}{0.499444in}}{\pgfqpoint{3.487500in}{1.155000in}}%
\pgfusepath{clip}%
\pgfsetbuttcap%
\pgfsetmiterjoin%
\pgfsetlinewidth{1.003750pt}%
\definecolor{currentstroke}{rgb}{0.000000,0.000000,0.000000}%
\pgfsetstrokecolor{currentstroke}%
\pgfsetdash{}{0pt}%
\pgfpathmoveto{\pgfqpoint{2.036818in}{0.499444in}}%
\pgfpathlineto{\pgfqpoint{2.100228in}{0.499444in}}%
\pgfpathlineto{\pgfqpoint{2.100228in}{0.554326in}}%
\pgfpathlineto{\pgfqpoint{2.036818in}{0.554326in}}%
\pgfpathlineto{\pgfqpoint{2.036818in}{0.499444in}}%
\pgfpathclose%
\pgfusepath{stroke}%
\end{pgfscope}%
\begin{pgfscope}%
\pgfpathrectangle{\pgfqpoint{0.515000in}{0.499444in}}{\pgfqpoint{3.487500in}{1.155000in}}%
\pgfusepath{clip}%
\pgfsetbuttcap%
\pgfsetmiterjoin%
\pgfsetlinewidth{1.003750pt}%
\definecolor{currentstroke}{rgb}{0.000000,0.000000,0.000000}%
\pgfsetstrokecolor{currentstroke}%
\pgfsetdash{}{0pt}%
\pgfpathmoveto{\pgfqpoint{2.195341in}{0.499444in}}%
\pgfpathlineto{\pgfqpoint{2.258750in}{0.499444in}}%
\pgfpathlineto{\pgfqpoint{2.258750in}{0.536953in}}%
\pgfpathlineto{\pgfqpoint{2.195341in}{0.536953in}}%
\pgfpathlineto{\pgfqpoint{2.195341in}{0.499444in}}%
\pgfpathclose%
\pgfusepath{stroke}%
\end{pgfscope}%
\begin{pgfscope}%
\pgfpathrectangle{\pgfqpoint{0.515000in}{0.499444in}}{\pgfqpoint{3.487500in}{1.155000in}}%
\pgfusepath{clip}%
\pgfsetbuttcap%
\pgfsetmiterjoin%
\pgfsetlinewidth{1.003750pt}%
\definecolor{currentstroke}{rgb}{0.000000,0.000000,0.000000}%
\pgfsetstrokecolor{currentstroke}%
\pgfsetdash{}{0pt}%
\pgfpathmoveto{\pgfqpoint{2.353864in}{0.499444in}}%
\pgfpathlineto{\pgfqpoint{2.417273in}{0.499444in}}%
\pgfpathlineto{\pgfqpoint{2.417273in}{0.528186in}}%
\pgfpathlineto{\pgfqpoint{2.353864in}{0.528186in}}%
\pgfpathlineto{\pgfqpoint{2.353864in}{0.499444in}}%
\pgfpathclose%
\pgfusepath{stroke}%
\end{pgfscope}%
\begin{pgfscope}%
\pgfpathrectangle{\pgfqpoint{0.515000in}{0.499444in}}{\pgfqpoint{3.487500in}{1.155000in}}%
\pgfusepath{clip}%
\pgfsetbuttcap%
\pgfsetmiterjoin%
\pgfsetlinewidth{1.003750pt}%
\definecolor{currentstroke}{rgb}{0.000000,0.000000,0.000000}%
\pgfsetstrokecolor{currentstroke}%
\pgfsetdash{}{0pt}%
\pgfpathmoveto{\pgfqpoint{2.512387in}{0.499444in}}%
\pgfpathlineto{\pgfqpoint{2.575796in}{0.499444in}}%
\pgfpathlineto{\pgfqpoint{2.575796in}{0.519540in}}%
\pgfpathlineto{\pgfqpoint{2.512387in}{0.519540in}}%
\pgfpathlineto{\pgfqpoint{2.512387in}{0.499444in}}%
\pgfpathclose%
\pgfusepath{stroke}%
\end{pgfscope}%
\begin{pgfscope}%
\pgfpathrectangle{\pgfqpoint{0.515000in}{0.499444in}}{\pgfqpoint{3.487500in}{1.155000in}}%
\pgfusepath{clip}%
\pgfsetbuttcap%
\pgfsetmiterjoin%
\pgfsetlinewidth{1.003750pt}%
\definecolor{currentstroke}{rgb}{0.000000,0.000000,0.000000}%
\pgfsetstrokecolor{currentstroke}%
\pgfsetdash{}{0pt}%
\pgfpathmoveto{\pgfqpoint{2.670909in}{0.499444in}}%
\pgfpathlineto{\pgfqpoint{2.734318in}{0.499444in}}%
\pgfpathlineto{\pgfqpoint{2.734318in}{0.513475in}}%
\pgfpathlineto{\pgfqpoint{2.670909in}{0.513475in}}%
\pgfpathlineto{\pgfqpoint{2.670909in}{0.499444in}}%
\pgfpathclose%
\pgfusepath{stroke}%
\end{pgfscope}%
\begin{pgfscope}%
\pgfpathrectangle{\pgfqpoint{0.515000in}{0.499444in}}{\pgfqpoint{3.487500in}{1.155000in}}%
\pgfusepath{clip}%
\pgfsetbuttcap%
\pgfsetmiterjoin%
\pgfsetlinewidth{1.003750pt}%
\definecolor{currentstroke}{rgb}{0.000000,0.000000,0.000000}%
\pgfsetstrokecolor{currentstroke}%
\pgfsetdash{}{0pt}%
\pgfpathmoveto{\pgfqpoint{2.829432in}{0.499444in}}%
\pgfpathlineto{\pgfqpoint{2.892841in}{0.499444in}}%
\pgfpathlineto{\pgfqpoint{2.892841in}{0.509932in}}%
\pgfpathlineto{\pgfqpoint{2.829432in}{0.509932in}}%
\pgfpathlineto{\pgfqpoint{2.829432in}{0.499444in}}%
\pgfpathclose%
\pgfusepath{stroke}%
\end{pgfscope}%
\begin{pgfscope}%
\pgfpathrectangle{\pgfqpoint{0.515000in}{0.499444in}}{\pgfqpoint{3.487500in}{1.155000in}}%
\pgfusepath{clip}%
\pgfsetbuttcap%
\pgfsetmiterjoin%
\pgfsetlinewidth{1.003750pt}%
\definecolor{currentstroke}{rgb}{0.000000,0.000000,0.000000}%
\pgfsetstrokecolor{currentstroke}%
\pgfsetdash{}{0pt}%
\pgfpathmoveto{\pgfqpoint{2.987955in}{0.499444in}}%
\pgfpathlineto{\pgfqpoint{3.051364in}{0.499444in}}%
\pgfpathlineto{\pgfqpoint{3.051364in}{0.507630in}}%
\pgfpathlineto{\pgfqpoint{2.987955in}{0.507630in}}%
\pgfpathlineto{\pgfqpoint{2.987955in}{0.499444in}}%
\pgfpathclose%
\pgfusepath{stroke}%
\end{pgfscope}%
\begin{pgfscope}%
\pgfpathrectangle{\pgfqpoint{0.515000in}{0.499444in}}{\pgfqpoint{3.487500in}{1.155000in}}%
\pgfusepath{clip}%
\pgfsetbuttcap%
\pgfsetmiterjoin%
\pgfsetlinewidth{1.003750pt}%
\definecolor{currentstroke}{rgb}{0.000000,0.000000,0.000000}%
\pgfsetstrokecolor{currentstroke}%
\pgfsetdash{}{0pt}%
\pgfpathmoveto{\pgfqpoint{3.146478in}{0.499444in}}%
\pgfpathlineto{\pgfqpoint{3.209887in}{0.499444in}}%
\pgfpathlineto{\pgfqpoint{3.209887in}{0.505709in}}%
\pgfpathlineto{\pgfqpoint{3.146478in}{0.505709in}}%
\pgfpathlineto{\pgfqpoint{3.146478in}{0.499444in}}%
\pgfpathclose%
\pgfusepath{stroke}%
\end{pgfscope}%
\begin{pgfscope}%
\pgfpathrectangle{\pgfqpoint{0.515000in}{0.499444in}}{\pgfqpoint{3.487500in}{1.155000in}}%
\pgfusepath{clip}%
\pgfsetbuttcap%
\pgfsetmiterjoin%
\pgfsetlinewidth{1.003750pt}%
\definecolor{currentstroke}{rgb}{0.000000,0.000000,0.000000}%
\pgfsetstrokecolor{currentstroke}%
\pgfsetdash{}{0pt}%
\pgfpathmoveto{\pgfqpoint{3.305000in}{0.499444in}}%
\pgfpathlineto{\pgfqpoint{3.368409in}{0.499444in}}%
\pgfpathlineto{\pgfqpoint{3.368409in}{0.503147in}}%
\pgfpathlineto{\pgfqpoint{3.305000in}{0.503147in}}%
\pgfpathlineto{\pgfqpoint{3.305000in}{0.499444in}}%
\pgfpathclose%
\pgfusepath{stroke}%
\end{pgfscope}%
\begin{pgfscope}%
\pgfpathrectangle{\pgfqpoint{0.515000in}{0.499444in}}{\pgfqpoint{3.487500in}{1.155000in}}%
\pgfusepath{clip}%
\pgfsetbuttcap%
\pgfsetmiterjoin%
\pgfsetlinewidth{1.003750pt}%
\definecolor{currentstroke}{rgb}{0.000000,0.000000,0.000000}%
\pgfsetstrokecolor{currentstroke}%
\pgfsetdash{}{0pt}%
\pgfpathmoveto{\pgfqpoint{3.463523in}{0.499444in}}%
\pgfpathlineto{\pgfqpoint{3.526932in}{0.499444in}}%
\pgfpathlineto{\pgfqpoint{3.526932in}{0.501266in}}%
\pgfpathlineto{\pgfqpoint{3.463523in}{0.501266in}}%
\pgfpathlineto{\pgfqpoint{3.463523in}{0.499444in}}%
\pgfpathclose%
\pgfusepath{stroke}%
\end{pgfscope}%
\begin{pgfscope}%
\pgfpathrectangle{\pgfqpoint{0.515000in}{0.499444in}}{\pgfqpoint{3.487500in}{1.155000in}}%
\pgfusepath{clip}%
\pgfsetbuttcap%
\pgfsetmiterjoin%
\pgfsetlinewidth{1.003750pt}%
\definecolor{currentstroke}{rgb}{0.000000,0.000000,0.000000}%
\pgfsetstrokecolor{currentstroke}%
\pgfsetdash{}{0pt}%
\pgfpathmoveto{\pgfqpoint{3.622046in}{0.499444in}}%
\pgfpathlineto{\pgfqpoint{3.685455in}{0.499444in}}%
\pgfpathlineto{\pgfqpoint{3.685455in}{0.499744in}}%
\pgfpathlineto{\pgfqpoint{3.622046in}{0.499744in}}%
\pgfpathlineto{\pgfqpoint{3.622046in}{0.499444in}}%
\pgfpathclose%
\pgfusepath{stroke}%
\end{pgfscope}%
\begin{pgfscope}%
\pgfpathrectangle{\pgfqpoint{0.515000in}{0.499444in}}{\pgfqpoint{3.487500in}{1.155000in}}%
\pgfusepath{clip}%
\pgfsetbuttcap%
\pgfsetmiterjoin%
\pgfsetlinewidth{1.003750pt}%
\definecolor{currentstroke}{rgb}{0.000000,0.000000,0.000000}%
\pgfsetstrokecolor{currentstroke}%
\pgfsetdash{}{0pt}%
\pgfpathmoveto{\pgfqpoint{3.780568in}{0.499444in}}%
\pgfpathlineto{\pgfqpoint{3.843978in}{0.499444in}}%
\pgfpathlineto{\pgfqpoint{3.843978in}{0.499444in}}%
\pgfpathlineto{\pgfqpoint{3.780568in}{0.499444in}}%
\pgfpathlineto{\pgfqpoint{3.780568in}{0.499444in}}%
\pgfpathclose%
\pgfusepath{stroke}%
\end{pgfscope}%
\begin{pgfscope}%
\pgfpathrectangle{\pgfqpoint{0.515000in}{0.499444in}}{\pgfqpoint{3.487500in}{1.155000in}}%
\pgfusepath{clip}%
\pgfsetbuttcap%
\pgfsetmiterjoin%
\definecolor{currentfill}{rgb}{0.000000,0.000000,0.000000}%
\pgfsetfillcolor{currentfill}%
\pgfsetlinewidth{0.000000pt}%
\definecolor{currentstroke}{rgb}{0.000000,0.000000,0.000000}%
\pgfsetstrokecolor{currentstroke}%
\pgfsetstrokeopacity{0.000000}%
\pgfsetdash{}{0pt}%
\pgfpathmoveto{\pgfqpoint{0.673523in}{0.499444in}}%
\pgfpathlineto{\pgfqpoint{0.736932in}{0.499444in}}%
\pgfpathlineto{\pgfqpoint{0.736932in}{0.499444in}}%
\pgfpathlineto{\pgfqpoint{0.673523in}{0.499444in}}%
\pgfpathlineto{\pgfqpoint{0.673523in}{0.499444in}}%
\pgfpathclose%
\pgfusepath{fill}%
\end{pgfscope}%
\begin{pgfscope}%
\pgfpathrectangle{\pgfqpoint{0.515000in}{0.499444in}}{\pgfqpoint{3.487500in}{1.155000in}}%
\pgfusepath{clip}%
\pgfsetbuttcap%
\pgfsetmiterjoin%
\definecolor{currentfill}{rgb}{0.000000,0.000000,0.000000}%
\pgfsetfillcolor{currentfill}%
\pgfsetlinewidth{0.000000pt}%
\definecolor{currentstroke}{rgb}{0.000000,0.000000,0.000000}%
\pgfsetstrokecolor{currentstroke}%
\pgfsetstrokeopacity{0.000000}%
\pgfsetdash{}{0pt}%
\pgfpathmoveto{\pgfqpoint{0.832046in}{0.499444in}}%
\pgfpathlineto{\pgfqpoint{0.895455in}{0.499444in}}%
\pgfpathlineto{\pgfqpoint{0.895455in}{0.535532in}}%
\pgfpathlineto{\pgfqpoint{0.832046in}{0.535532in}}%
\pgfpathlineto{\pgfqpoint{0.832046in}{0.499444in}}%
\pgfpathclose%
\pgfusepath{fill}%
\end{pgfscope}%
\begin{pgfscope}%
\pgfpathrectangle{\pgfqpoint{0.515000in}{0.499444in}}{\pgfqpoint{3.487500in}{1.155000in}}%
\pgfusepath{clip}%
\pgfsetbuttcap%
\pgfsetmiterjoin%
\definecolor{currentfill}{rgb}{0.000000,0.000000,0.000000}%
\pgfsetfillcolor{currentfill}%
\pgfsetlinewidth{0.000000pt}%
\definecolor{currentstroke}{rgb}{0.000000,0.000000,0.000000}%
\pgfsetstrokecolor{currentstroke}%
\pgfsetstrokeopacity{0.000000}%
\pgfsetdash{}{0pt}%
\pgfpathmoveto{\pgfqpoint{0.990568in}{0.499444in}}%
\pgfpathlineto{\pgfqpoint{1.053978in}{0.499444in}}%
\pgfpathlineto{\pgfqpoint{1.053978in}{0.575602in}}%
\pgfpathlineto{\pgfqpoint{0.990568in}{0.575602in}}%
\pgfpathlineto{\pgfqpoint{0.990568in}{0.499444in}}%
\pgfpathclose%
\pgfusepath{fill}%
\end{pgfscope}%
\begin{pgfscope}%
\pgfpathrectangle{\pgfqpoint{0.515000in}{0.499444in}}{\pgfqpoint{3.487500in}{1.155000in}}%
\pgfusepath{clip}%
\pgfsetbuttcap%
\pgfsetmiterjoin%
\definecolor{currentfill}{rgb}{0.000000,0.000000,0.000000}%
\pgfsetfillcolor{currentfill}%
\pgfsetlinewidth{0.000000pt}%
\definecolor{currentstroke}{rgb}{0.000000,0.000000,0.000000}%
\pgfsetstrokecolor{currentstroke}%
\pgfsetstrokeopacity{0.000000}%
\pgfsetdash{}{0pt}%
\pgfpathmoveto{\pgfqpoint{1.149091in}{0.499444in}}%
\pgfpathlineto{\pgfqpoint{1.212500in}{0.499444in}}%
\pgfpathlineto{\pgfqpoint{1.212500in}{0.576243in}}%
\pgfpathlineto{\pgfqpoint{1.149091in}{0.576243in}}%
\pgfpathlineto{\pgfqpoint{1.149091in}{0.499444in}}%
\pgfpathclose%
\pgfusepath{fill}%
\end{pgfscope}%
\begin{pgfscope}%
\pgfpathrectangle{\pgfqpoint{0.515000in}{0.499444in}}{\pgfqpoint{3.487500in}{1.155000in}}%
\pgfusepath{clip}%
\pgfsetbuttcap%
\pgfsetmiterjoin%
\definecolor{currentfill}{rgb}{0.000000,0.000000,0.000000}%
\pgfsetfillcolor{currentfill}%
\pgfsetlinewidth{0.000000pt}%
\definecolor{currentstroke}{rgb}{0.000000,0.000000,0.000000}%
\pgfsetstrokecolor{currentstroke}%
\pgfsetstrokeopacity{0.000000}%
\pgfsetdash{}{0pt}%
\pgfpathmoveto{\pgfqpoint{1.307614in}{0.499444in}}%
\pgfpathlineto{\pgfqpoint{1.371023in}{0.499444in}}%
\pgfpathlineto{\pgfqpoint{1.371023in}{0.571940in}}%
\pgfpathlineto{\pgfqpoint{1.307614in}{0.571940in}}%
\pgfpathlineto{\pgfqpoint{1.307614in}{0.499444in}}%
\pgfpathclose%
\pgfusepath{fill}%
\end{pgfscope}%
\begin{pgfscope}%
\pgfpathrectangle{\pgfqpoint{0.515000in}{0.499444in}}{\pgfqpoint{3.487500in}{1.155000in}}%
\pgfusepath{clip}%
\pgfsetbuttcap%
\pgfsetmiterjoin%
\definecolor{currentfill}{rgb}{0.000000,0.000000,0.000000}%
\pgfsetfillcolor{currentfill}%
\pgfsetlinewidth{0.000000pt}%
\definecolor{currentstroke}{rgb}{0.000000,0.000000,0.000000}%
\pgfsetstrokecolor{currentstroke}%
\pgfsetstrokeopacity{0.000000}%
\pgfsetdash{}{0pt}%
\pgfpathmoveto{\pgfqpoint{1.466137in}{0.499444in}}%
\pgfpathlineto{\pgfqpoint{1.529546in}{0.499444in}}%
\pgfpathlineto{\pgfqpoint{1.529546in}{0.564794in}}%
\pgfpathlineto{\pgfqpoint{1.466137in}{0.564794in}}%
\pgfpathlineto{\pgfqpoint{1.466137in}{0.499444in}}%
\pgfpathclose%
\pgfusepath{fill}%
\end{pgfscope}%
\begin{pgfscope}%
\pgfpathrectangle{\pgfqpoint{0.515000in}{0.499444in}}{\pgfqpoint{3.487500in}{1.155000in}}%
\pgfusepath{clip}%
\pgfsetbuttcap%
\pgfsetmiterjoin%
\definecolor{currentfill}{rgb}{0.000000,0.000000,0.000000}%
\pgfsetfillcolor{currentfill}%
\pgfsetlinewidth{0.000000pt}%
\definecolor{currentstroke}{rgb}{0.000000,0.000000,0.000000}%
\pgfsetstrokecolor{currentstroke}%
\pgfsetstrokeopacity{0.000000}%
\pgfsetdash{}{0pt}%
\pgfpathmoveto{\pgfqpoint{1.624659in}{0.499444in}}%
\pgfpathlineto{\pgfqpoint{1.688068in}{0.499444in}}%
\pgfpathlineto{\pgfqpoint{1.688068in}{0.557128in}}%
\pgfpathlineto{\pgfqpoint{1.624659in}{0.557128in}}%
\pgfpathlineto{\pgfqpoint{1.624659in}{0.499444in}}%
\pgfpathclose%
\pgfusepath{fill}%
\end{pgfscope}%
\begin{pgfscope}%
\pgfpathrectangle{\pgfqpoint{0.515000in}{0.499444in}}{\pgfqpoint{3.487500in}{1.155000in}}%
\pgfusepath{clip}%
\pgfsetbuttcap%
\pgfsetmiterjoin%
\definecolor{currentfill}{rgb}{0.000000,0.000000,0.000000}%
\pgfsetfillcolor{currentfill}%
\pgfsetlinewidth{0.000000pt}%
\definecolor{currentstroke}{rgb}{0.000000,0.000000,0.000000}%
\pgfsetstrokecolor{currentstroke}%
\pgfsetstrokeopacity{0.000000}%
\pgfsetdash{}{0pt}%
\pgfpathmoveto{\pgfqpoint{1.783182in}{0.499444in}}%
\pgfpathlineto{\pgfqpoint{1.846591in}{0.499444in}}%
\pgfpathlineto{\pgfqpoint{1.846591in}{0.548922in}}%
\pgfpathlineto{\pgfqpoint{1.783182in}{0.548922in}}%
\pgfpathlineto{\pgfqpoint{1.783182in}{0.499444in}}%
\pgfpathclose%
\pgfusepath{fill}%
\end{pgfscope}%
\begin{pgfscope}%
\pgfpathrectangle{\pgfqpoint{0.515000in}{0.499444in}}{\pgfqpoint{3.487500in}{1.155000in}}%
\pgfusepath{clip}%
\pgfsetbuttcap%
\pgfsetmiterjoin%
\definecolor{currentfill}{rgb}{0.000000,0.000000,0.000000}%
\pgfsetfillcolor{currentfill}%
\pgfsetlinewidth{0.000000pt}%
\definecolor{currentstroke}{rgb}{0.000000,0.000000,0.000000}%
\pgfsetstrokecolor{currentstroke}%
\pgfsetstrokeopacity{0.000000}%
\pgfsetdash{}{0pt}%
\pgfpathmoveto{\pgfqpoint{1.941705in}{0.499444in}}%
\pgfpathlineto{\pgfqpoint{2.005114in}{0.499444in}}%
\pgfpathlineto{\pgfqpoint{2.005114in}{0.542597in}}%
\pgfpathlineto{\pgfqpoint{1.941705in}{0.542597in}}%
\pgfpathlineto{\pgfqpoint{1.941705in}{0.499444in}}%
\pgfpathclose%
\pgfusepath{fill}%
\end{pgfscope}%
\begin{pgfscope}%
\pgfpathrectangle{\pgfqpoint{0.515000in}{0.499444in}}{\pgfqpoint{3.487500in}{1.155000in}}%
\pgfusepath{clip}%
\pgfsetbuttcap%
\pgfsetmiterjoin%
\definecolor{currentfill}{rgb}{0.000000,0.000000,0.000000}%
\pgfsetfillcolor{currentfill}%
\pgfsetlinewidth{0.000000pt}%
\definecolor{currentstroke}{rgb}{0.000000,0.000000,0.000000}%
\pgfsetstrokecolor{currentstroke}%
\pgfsetstrokeopacity{0.000000}%
\pgfsetdash{}{0pt}%
\pgfpathmoveto{\pgfqpoint{2.100228in}{0.499444in}}%
\pgfpathlineto{\pgfqpoint{2.163637in}{0.499444in}}%
\pgfpathlineto{\pgfqpoint{2.163637in}{0.534451in}}%
\pgfpathlineto{\pgfqpoint{2.100228in}{0.534451in}}%
\pgfpathlineto{\pgfqpoint{2.100228in}{0.499444in}}%
\pgfpathclose%
\pgfusepath{fill}%
\end{pgfscope}%
\begin{pgfscope}%
\pgfpathrectangle{\pgfqpoint{0.515000in}{0.499444in}}{\pgfqpoint{3.487500in}{1.155000in}}%
\pgfusepath{clip}%
\pgfsetbuttcap%
\pgfsetmiterjoin%
\definecolor{currentfill}{rgb}{0.000000,0.000000,0.000000}%
\pgfsetfillcolor{currentfill}%
\pgfsetlinewidth{0.000000pt}%
\definecolor{currentstroke}{rgb}{0.000000,0.000000,0.000000}%
\pgfsetstrokecolor{currentstroke}%
\pgfsetstrokeopacity{0.000000}%
\pgfsetdash{}{0pt}%
\pgfpathmoveto{\pgfqpoint{2.258750in}{0.499444in}}%
\pgfpathlineto{\pgfqpoint{2.322159in}{0.499444in}}%
\pgfpathlineto{\pgfqpoint{2.322159in}{0.528907in}}%
\pgfpathlineto{\pgfqpoint{2.258750in}{0.528907in}}%
\pgfpathlineto{\pgfqpoint{2.258750in}{0.499444in}}%
\pgfpathclose%
\pgfusepath{fill}%
\end{pgfscope}%
\begin{pgfscope}%
\pgfpathrectangle{\pgfqpoint{0.515000in}{0.499444in}}{\pgfqpoint{3.487500in}{1.155000in}}%
\pgfusepath{clip}%
\pgfsetbuttcap%
\pgfsetmiterjoin%
\definecolor{currentfill}{rgb}{0.000000,0.000000,0.000000}%
\pgfsetfillcolor{currentfill}%
\pgfsetlinewidth{0.000000pt}%
\definecolor{currentstroke}{rgb}{0.000000,0.000000,0.000000}%
\pgfsetstrokecolor{currentstroke}%
\pgfsetstrokeopacity{0.000000}%
\pgfsetdash{}{0pt}%
\pgfpathmoveto{\pgfqpoint{2.417273in}{0.499444in}}%
\pgfpathlineto{\pgfqpoint{2.480682in}{0.499444in}}%
\pgfpathlineto{\pgfqpoint{2.480682in}{0.525304in}}%
\pgfpathlineto{\pgfqpoint{2.417273in}{0.525304in}}%
\pgfpathlineto{\pgfqpoint{2.417273in}{0.499444in}}%
\pgfpathclose%
\pgfusepath{fill}%
\end{pgfscope}%
\begin{pgfscope}%
\pgfpathrectangle{\pgfqpoint{0.515000in}{0.499444in}}{\pgfqpoint{3.487500in}{1.155000in}}%
\pgfusepath{clip}%
\pgfsetbuttcap%
\pgfsetmiterjoin%
\definecolor{currentfill}{rgb}{0.000000,0.000000,0.000000}%
\pgfsetfillcolor{currentfill}%
\pgfsetlinewidth{0.000000pt}%
\definecolor{currentstroke}{rgb}{0.000000,0.000000,0.000000}%
\pgfsetstrokecolor{currentstroke}%
\pgfsetstrokeopacity{0.000000}%
\pgfsetdash{}{0pt}%
\pgfpathmoveto{\pgfqpoint{2.575796in}{0.499444in}}%
\pgfpathlineto{\pgfqpoint{2.639205in}{0.499444in}}%
\pgfpathlineto{\pgfqpoint{2.639205in}{0.521501in}}%
\pgfpathlineto{\pgfqpoint{2.575796in}{0.521501in}}%
\pgfpathlineto{\pgfqpoint{2.575796in}{0.499444in}}%
\pgfpathclose%
\pgfusepath{fill}%
\end{pgfscope}%
\begin{pgfscope}%
\pgfpathrectangle{\pgfqpoint{0.515000in}{0.499444in}}{\pgfqpoint{3.487500in}{1.155000in}}%
\pgfusepath{clip}%
\pgfsetbuttcap%
\pgfsetmiterjoin%
\definecolor{currentfill}{rgb}{0.000000,0.000000,0.000000}%
\pgfsetfillcolor{currentfill}%
\pgfsetlinewidth{0.000000pt}%
\definecolor{currentstroke}{rgb}{0.000000,0.000000,0.000000}%
\pgfsetstrokecolor{currentstroke}%
\pgfsetstrokeopacity{0.000000}%
\pgfsetdash{}{0pt}%
\pgfpathmoveto{\pgfqpoint{2.734318in}{0.499444in}}%
\pgfpathlineto{\pgfqpoint{2.797728in}{0.499444in}}%
\pgfpathlineto{\pgfqpoint{2.797728in}{0.518299in}}%
\pgfpathlineto{\pgfqpoint{2.734318in}{0.518299in}}%
\pgfpathlineto{\pgfqpoint{2.734318in}{0.499444in}}%
\pgfpathclose%
\pgfusepath{fill}%
\end{pgfscope}%
\begin{pgfscope}%
\pgfpathrectangle{\pgfqpoint{0.515000in}{0.499444in}}{\pgfqpoint{3.487500in}{1.155000in}}%
\pgfusepath{clip}%
\pgfsetbuttcap%
\pgfsetmiterjoin%
\definecolor{currentfill}{rgb}{0.000000,0.000000,0.000000}%
\pgfsetfillcolor{currentfill}%
\pgfsetlinewidth{0.000000pt}%
\definecolor{currentstroke}{rgb}{0.000000,0.000000,0.000000}%
\pgfsetstrokecolor{currentstroke}%
\pgfsetstrokeopacity{0.000000}%
\pgfsetdash{}{0pt}%
\pgfpathmoveto{\pgfqpoint{2.892841in}{0.499444in}}%
\pgfpathlineto{\pgfqpoint{2.956250in}{0.499444in}}%
\pgfpathlineto{\pgfqpoint{2.956250in}{0.517418in}}%
\pgfpathlineto{\pgfqpoint{2.892841in}{0.517418in}}%
\pgfpathlineto{\pgfqpoint{2.892841in}{0.499444in}}%
\pgfpathclose%
\pgfusepath{fill}%
\end{pgfscope}%
\begin{pgfscope}%
\pgfpathrectangle{\pgfqpoint{0.515000in}{0.499444in}}{\pgfqpoint{3.487500in}{1.155000in}}%
\pgfusepath{clip}%
\pgfsetbuttcap%
\pgfsetmiterjoin%
\definecolor{currentfill}{rgb}{0.000000,0.000000,0.000000}%
\pgfsetfillcolor{currentfill}%
\pgfsetlinewidth{0.000000pt}%
\definecolor{currentstroke}{rgb}{0.000000,0.000000,0.000000}%
\pgfsetstrokecolor{currentstroke}%
\pgfsetstrokeopacity{0.000000}%
\pgfsetdash{}{0pt}%
\pgfpathmoveto{\pgfqpoint{3.051364in}{0.499444in}}%
\pgfpathlineto{\pgfqpoint{3.114773in}{0.499444in}}%
\pgfpathlineto{\pgfqpoint{3.114773in}{0.515116in}}%
\pgfpathlineto{\pgfqpoint{3.051364in}{0.515116in}}%
\pgfpathlineto{\pgfqpoint{3.051364in}{0.499444in}}%
\pgfpathclose%
\pgfusepath{fill}%
\end{pgfscope}%
\begin{pgfscope}%
\pgfpathrectangle{\pgfqpoint{0.515000in}{0.499444in}}{\pgfqpoint{3.487500in}{1.155000in}}%
\pgfusepath{clip}%
\pgfsetbuttcap%
\pgfsetmiterjoin%
\definecolor{currentfill}{rgb}{0.000000,0.000000,0.000000}%
\pgfsetfillcolor{currentfill}%
\pgfsetlinewidth{0.000000pt}%
\definecolor{currentstroke}{rgb}{0.000000,0.000000,0.000000}%
\pgfsetstrokecolor{currentstroke}%
\pgfsetstrokeopacity{0.000000}%
\pgfsetdash{}{0pt}%
\pgfpathmoveto{\pgfqpoint{3.209887in}{0.499444in}}%
\pgfpathlineto{\pgfqpoint{3.273296in}{0.499444in}}%
\pgfpathlineto{\pgfqpoint{3.273296in}{0.513695in}}%
\pgfpathlineto{\pgfqpoint{3.209887in}{0.513695in}}%
\pgfpathlineto{\pgfqpoint{3.209887in}{0.499444in}}%
\pgfpathclose%
\pgfusepath{fill}%
\end{pgfscope}%
\begin{pgfscope}%
\pgfpathrectangle{\pgfqpoint{0.515000in}{0.499444in}}{\pgfqpoint{3.487500in}{1.155000in}}%
\pgfusepath{clip}%
\pgfsetbuttcap%
\pgfsetmiterjoin%
\definecolor{currentfill}{rgb}{0.000000,0.000000,0.000000}%
\pgfsetfillcolor{currentfill}%
\pgfsetlinewidth{0.000000pt}%
\definecolor{currentstroke}{rgb}{0.000000,0.000000,0.000000}%
\pgfsetstrokecolor{currentstroke}%
\pgfsetstrokeopacity{0.000000}%
\pgfsetdash{}{0pt}%
\pgfpathmoveto{\pgfqpoint{3.368409in}{0.499444in}}%
\pgfpathlineto{\pgfqpoint{3.431818in}{0.499444in}}%
\pgfpathlineto{\pgfqpoint{3.431818in}{0.511653in}}%
\pgfpathlineto{\pgfqpoint{3.368409in}{0.511653in}}%
\pgfpathlineto{\pgfqpoint{3.368409in}{0.499444in}}%
\pgfpathclose%
\pgfusepath{fill}%
\end{pgfscope}%
\begin{pgfscope}%
\pgfpathrectangle{\pgfqpoint{0.515000in}{0.499444in}}{\pgfqpoint{3.487500in}{1.155000in}}%
\pgfusepath{clip}%
\pgfsetbuttcap%
\pgfsetmiterjoin%
\definecolor{currentfill}{rgb}{0.000000,0.000000,0.000000}%
\pgfsetfillcolor{currentfill}%
\pgfsetlinewidth{0.000000pt}%
\definecolor{currentstroke}{rgb}{0.000000,0.000000,0.000000}%
\pgfsetstrokecolor{currentstroke}%
\pgfsetstrokeopacity{0.000000}%
\pgfsetdash{}{0pt}%
\pgfpathmoveto{\pgfqpoint{3.526932in}{0.499444in}}%
\pgfpathlineto{\pgfqpoint{3.590341in}{0.499444in}}%
\pgfpathlineto{\pgfqpoint{3.590341in}{0.505949in}}%
\pgfpathlineto{\pgfqpoint{3.526932in}{0.505949in}}%
\pgfpathlineto{\pgfqpoint{3.526932in}{0.499444in}}%
\pgfpathclose%
\pgfusepath{fill}%
\end{pgfscope}%
\begin{pgfscope}%
\pgfpathrectangle{\pgfqpoint{0.515000in}{0.499444in}}{\pgfqpoint{3.487500in}{1.155000in}}%
\pgfusepath{clip}%
\pgfsetbuttcap%
\pgfsetmiterjoin%
\definecolor{currentfill}{rgb}{0.000000,0.000000,0.000000}%
\pgfsetfillcolor{currentfill}%
\pgfsetlinewidth{0.000000pt}%
\definecolor{currentstroke}{rgb}{0.000000,0.000000,0.000000}%
\pgfsetstrokecolor{currentstroke}%
\pgfsetstrokeopacity{0.000000}%
\pgfsetdash{}{0pt}%
\pgfpathmoveto{\pgfqpoint{3.685455in}{0.499444in}}%
\pgfpathlineto{\pgfqpoint{3.748864in}{0.499444in}}%
\pgfpathlineto{\pgfqpoint{3.748864in}{0.501386in}}%
\pgfpathlineto{\pgfqpoint{3.685455in}{0.501386in}}%
\pgfpathlineto{\pgfqpoint{3.685455in}{0.499444in}}%
\pgfpathclose%
\pgfusepath{fill}%
\end{pgfscope}%
\begin{pgfscope}%
\pgfpathrectangle{\pgfqpoint{0.515000in}{0.499444in}}{\pgfqpoint{3.487500in}{1.155000in}}%
\pgfusepath{clip}%
\pgfsetbuttcap%
\pgfsetmiterjoin%
\definecolor{currentfill}{rgb}{0.000000,0.000000,0.000000}%
\pgfsetfillcolor{currentfill}%
\pgfsetlinewidth{0.000000pt}%
\definecolor{currentstroke}{rgb}{0.000000,0.000000,0.000000}%
\pgfsetstrokecolor{currentstroke}%
\pgfsetstrokeopacity{0.000000}%
\pgfsetdash{}{0pt}%
\pgfpathmoveto{\pgfqpoint{3.843978in}{0.499444in}}%
\pgfpathlineto{\pgfqpoint{3.907387in}{0.499444in}}%
\pgfpathlineto{\pgfqpoint{3.907387in}{0.499464in}}%
\pgfpathlineto{\pgfqpoint{3.843978in}{0.499464in}}%
\pgfpathlineto{\pgfqpoint{3.843978in}{0.499444in}}%
\pgfpathclose%
\pgfusepath{fill}%
\end{pgfscope}%
\begin{pgfscope}%
\pgfsetbuttcap%
\pgfsetroundjoin%
\definecolor{currentfill}{rgb}{0.000000,0.000000,0.000000}%
\pgfsetfillcolor{currentfill}%
\pgfsetlinewidth{0.803000pt}%
\definecolor{currentstroke}{rgb}{0.000000,0.000000,0.000000}%
\pgfsetstrokecolor{currentstroke}%
\pgfsetdash{}{0pt}%
\pgfsys@defobject{currentmarker}{\pgfqpoint{0.000000in}{-0.048611in}}{\pgfqpoint{0.000000in}{0.000000in}}{%
\pgfpathmoveto{\pgfqpoint{0.000000in}{0.000000in}}%
\pgfpathlineto{\pgfqpoint{0.000000in}{-0.048611in}}%
\pgfusepath{stroke,fill}%
}%
\begin{pgfscope}%
\pgfsys@transformshift{0.515000in}{0.499444in}%
\pgfsys@useobject{currentmarker}{}%
\end{pgfscope}%
\end{pgfscope}%
\begin{pgfscope}%
\pgfsetbuttcap%
\pgfsetroundjoin%
\definecolor{currentfill}{rgb}{0.000000,0.000000,0.000000}%
\pgfsetfillcolor{currentfill}%
\pgfsetlinewidth{0.803000pt}%
\definecolor{currentstroke}{rgb}{0.000000,0.000000,0.000000}%
\pgfsetstrokecolor{currentstroke}%
\pgfsetdash{}{0pt}%
\pgfsys@defobject{currentmarker}{\pgfqpoint{0.000000in}{-0.048611in}}{\pgfqpoint{0.000000in}{0.000000in}}{%
\pgfpathmoveto{\pgfqpoint{0.000000in}{0.000000in}}%
\pgfpathlineto{\pgfqpoint{0.000000in}{-0.048611in}}%
\pgfusepath{stroke,fill}%
}%
\begin{pgfscope}%
\pgfsys@transformshift{0.673523in}{0.499444in}%
\pgfsys@useobject{currentmarker}{}%
\end{pgfscope}%
\end{pgfscope}%
\begin{pgfscope}%
\definecolor{textcolor}{rgb}{0.000000,0.000000,0.000000}%
\pgfsetstrokecolor{textcolor}%
\pgfsetfillcolor{textcolor}%
\pgftext[x=0.673523in,y=0.402222in,,top]{\color{textcolor}\rmfamily\fontsize{10.000000}{12.000000}\selectfont 0.0}%
\end{pgfscope}%
\begin{pgfscope}%
\pgfsetbuttcap%
\pgfsetroundjoin%
\definecolor{currentfill}{rgb}{0.000000,0.000000,0.000000}%
\pgfsetfillcolor{currentfill}%
\pgfsetlinewidth{0.803000pt}%
\definecolor{currentstroke}{rgb}{0.000000,0.000000,0.000000}%
\pgfsetstrokecolor{currentstroke}%
\pgfsetdash{}{0pt}%
\pgfsys@defobject{currentmarker}{\pgfqpoint{0.000000in}{-0.048611in}}{\pgfqpoint{0.000000in}{0.000000in}}{%
\pgfpathmoveto{\pgfqpoint{0.000000in}{0.000000in}}%
\pgfpathlineto{\pgfqpoint{0.000000in}{-0.048611in}}%
\pgfusepath{stroke,fill}%
}%
\begin{pgfscope}%
\pgfsys@transformshift{0.832046in}{0.499444in}%
\pgfsys@useobject{currentmarker}{}%
\end{pgfscope}%
\end{pgfscope}%
\begin{pgfscope}%
\pgfsetbuttcap%
\pgfsetroundjoin%
\definecolor{currentfill}{rgb}{0.000000,0.000000,0.000000}%
\pgfsetfillcolor{currentfill}%
\pgfsetlinewidth{0.803000pt}%
\definecolor{currentstroke}{rgb}{0.000000,0.000000,0.000000}%
\pgfsetstrokecolor{currentstroke}%
\pgfsetdash{}{0pt}%
\pgfsys@defobject{currentmarker}{\pgfqpoint{0.000000in}{-0.048611in}}{\pgfqpoint{0.000000in}{0.000000in}}{%
\pgfpathmoveto{\pgfqpoint{0.000000in}{0.000000in}}%
\pgfpathlineto{\pgfqpoint{0.000000in}{-0.048611in}}%
\pgfusepath{stroke,fill}%
}%
\begin{pgfscope}%
\pgfsys@transformshift{0.990568in}{0.499444in}%
\pgfsys@useobject{currentmarker}{}%
\end{pgfscope}%
\end{pgfscope}%
\begin{pgfscope}%
\definecolor{textcolor}{rgb}{0.000000,0.000000,0.000000}%
\pgfsetstrokecolor{textcolor}%
\pgfsetfillcolor{textcolor}%
\pgftext[x=0.990568in,y=0.402222in,,top]{\color{textcolor}\rmfamily\fontsize{10.000000}{12.000000}\selectfont 0.1}%
\end{pgfscope}%
\begin{pgfscope}%
\pgfsetbuttcap%
\pgfsetroundjoin%
\definecolor{currentfill}{rgb}{0.000000,0.000000,0.000000}%
\pgfsetfillcolor{currentfill}%
\pgfsetlinewidth{0.803000pt}%
\definecolor{currentstroke}{rgb}{0.000000,0.000000,0.000000}%
\pgfsetstrokecolor{currentstroke}%
\pgfsetdash{}{0pt}%
\pgfsys@defobject{currentmarker}{\pgfqpoint{0.000000in}{-0.048611in}}{\pgfqpoint{0.000000in}{0.000000in}}{%
\pgfpathmoveto{\pgfqpoint{0.000000in}{0.000000in}}%
\pgfpathlineto{\pgfqpoint{0.000000in}{-0.048611in}}%
\pgfusepath{stroke,fill}%
}%
\begin{pgfscope}%
\pgfsys@transformshift{1.149091in}{0.499444in}%
\pgfsys@useobject{currentmarker}{}%
\end{pgfscope}%
\end{pgfscope}%
\begin{pgfscope}%
\pgfsetbuttcap%
\pgfsetroundjoin%
\definecolor{currentfill}{rgb}{0.000000,0.000000,0.000000}%
\pgfsetfillcolor{currentfill}%
\pgfsetlinewidth{0.803000pt}%
\definecolor{currentstroke}{rgb}{0.000000,0.000000,0.000000}%
\pgfsetstrokecolor{currentstroke}%
\pgfsetdash{}{0pt}%
\pgfsys@defobject{currentmarker}{\pgfqpoint{0.000000in}{-0.048611in}}{\pgfqpoint{0.000000in}{0.000000in}}{%
\pgfpathmoveto{\pgfqpoint{0.000000in}{0.000000in}}%
\pgfpathlineto{\pgfqpoint{0.000000in}{-0.048611in}}%
\pgfusepath{stroke,fill}%
}%
\begin{pgfscope}%
\pgfsys@transformshift{1.307614in}{0.499444in}%
\pgfsys@useobject{currentmarker}{}%
\end{pgfscope}%
\end{pgfscope}%
\begin{pgfscope}%
\definecolor{textcolor}{rgb}{0.000000,0.000000,0.000000}%
\pgfsetstrokecolor{textcolor}%
\pgfsetfillcolor{textcolor}%
\pgftext[x=1.307614in,y=0.402222in,,top]{\color{textcolor}\rmfamily\fontsize{10.000000}{12.000000}\selectfont 0.2}%
\end{pgfscope}%
\begin{pgfscope}%
\pgfsetbuttcap%
\pgfsetroundjoin%
\definecolor{currentfill}{rgb}{0.000000,0.000000,0.000000}%
\pgfsetfillcolor{currentfill}%
\pgfsetlinewidth{0.803000pt}%
\definecolor{currentstroke}{rgb}{0.000000,0.000000,0.000000}%
\pgfsetstrokecolor{currentstroke}%
\pgfsetdash{}{0pt}%
\pgfsys@defobject{currentmarker}{\pgfqpoint{0.000000in}{-0.048611in}}{\pgfqpoint{0.000000in}{0.000000in}}{%
\pgfpathmoveto{\pgfqpoint{0.000000in}{0.000000in}}%
\pgfpathlineto{\pgfqpoint{0.000000in}{-0.048611in}}%
\pgfusepath{stroke,fill}%
}%
\begin{pgfscope}%
\pgfsys@transformshift{1.466137in}{0.499444in}%
\pgfsys@useobject{currentmarker}{}%
\end{pgfscope}%
\end{pgfscope}%
\begin{pgfscope}%
\pgfsetbuttcap%
\pgfsetroundjoin%
\definecolor{currentfill}{rgb}{0.000000,0.000000,0.000000}%
\pgfsetfillcolor{currentfill}%
\pgfsetlinewidth{0.803000pt}%
\definecolor{currentstroke}{rgb}{0.000000,0.000000,0.000000}%
\pgfsetstrokecolor{currentstroke}%
\pgfsetdash{}{0pt}%
\pgfsys@defobject{currentmarker}{\pgfqpoint{0.000000in}{-0.048611in}}{\pgfqpoint{0.000000in}{0.000000in}}{%
\pgfpathmoveto{\pgfqpoint{0.000000in}{0.000000in}}%
\pgfpathlineto{\pgfqpoint{0.000000in}{-0.048611in}}%
\pgfusepath{stroke,fill}%
}%
\begin{pgfscope}%
\pgfsys@transformshift{1.624659in}{0.499444in}%
\pgfsys@useobject{currentmarker}{}%
\end{pgfscope}%
\end{pgfscope}%
\begin{pgfscope}%
\definecolor{textcolor}{rgb}{0.000000,0.000000,0.000000}%
\pgfsetstrokecolor{textcolor}%
\pgfsetfillcolor{textcolor}%
\pgftext[x=1.624659in,y=0.402222in,,top]{\color{textcolor}\rmfamily\fontsize{10.000000}{12.000000}\selectfont 0.3}%
\end{pgfscope}%
\begin{pgfscope}%
\pgfsetbuttcap%
\pgfsetroundjoin%
\definecolor{currentfill}{rgb}{0.000000,0.000000,0.000000}%
\pgfsetfillcolor{currentfill}%
\pgfsetlinewidth{0.803000pt}%
\definecolor{currentstroke}{rgb}{0.000000,0.000000,0.000000}%
\pgfsetstrokecolor{currentstroke}%
\pgfsetdash{}{0pt}%
\pgfsys@defobject{currentmarker}{\pgfqpoint{0.000000in}{-0.048611in}}{\pgfqpoint{0.000000in}{0.000000in}}{%
\pgfpathmoveto{\pgfqpoint{0.000000in}{0.000000in}}%
\pgfpathlineto{\pgfqpoint{0.000000in}{-0.048611in}}%
\pgfusepath{stroke,fill}%
}%
\begin{pgfscope}%
\pgfsys@transformshift{1.783182in}{0.499444in}%
\pgfsys@useobject{currentmarker}{}%
\end{pgfscope}%
\end{pgfscope}%
\begin{pgfscope}%
\pgfsetbuttcap%
\pgfsetroundjoin%
\definecolor{currentfill}{rgb}{0.000000,0.000000,0.000000}%
\pgfsetfillcolor{currentfill}%
\pgfsetlinewidth{0.803000pt}%
\definecolor{currentstroke}{rgb}{0.000000,0.000000,0.000000}%
\pgfsetstrokecolor{currentstroke}%
\pgfsetdash{}{0pt}%
\pgfsys@defobject{currentmarker}{\pgfqpoint{0.000000in}{-0.048611in}}{\pgfqpoint{0.000000in}{0.000000in}}{%
\pgfpathmoveto{\pgfqpoint{0.000000in}{0.000000in}}%
\pgfpathlineto{\pgfqpoint{0.000000in}{-0.048611in}}%
\pgfusepath{stroke,fill}%
}%
\begin{pgfscope}%
\pgfsys@transformshift{1.941705in}{0.499444in}%
\pgfsys@useobject{currentmarker}{}%
\end{pgfscope}%
\end{pgfscope}%
\begin{pgfscope}%
\definecolor{textcolor}{rgb}{0.000000,0.000000,0.000000}%
\pgfsetstrokecolor{textcolor}%
\pgfsetfillcolor{textcolor}%
\pgftext[x=1.941705in,y=0.402222in,,top]{\color{textcolor}\rmfamily\fontsize{10.000000}{12.000000}\selectfont 0.4}%
\end{pgfscope}%
\begin{pgfscope}%
\pgfsetbuttcap%
\pgfsetroundjoin%
\definecolor{currentfill}{rgb}{0.000000,0.000000,0.000000}%
\pgfsetfillcolor{currentfill}%
\pgfsetlinewidth{0.803000pt}%
\definecolor{currentstroke}{rgb}{0.000000,0.000000,0.000000}%
\pgfsetstrokecolor{currentstroke}%
\pgfsetdash{}{0pt}%
\pgfsys@defobject{currentmarker}{\pgfqpoint{0.000000in}{-0.048611in}}{\pgfqpoint{0.000000in}{0.000000in}}{%
\pgfpathmoveto{\pgfqpoint{0.000000in}{0.000000in}}%
\pgfpathlineto{\pgfqpoint{0.000000in}{-0.048611in}}%
\pgfusepath{stroke,fill}%
}%
\begin{pgfscope}%
\pgfsys@transformshift{2.100228in}{0.499444in}%
\pgfsys@useobject{currentmarker}{}%
\end{pgfscope}%
\end{pgfscope}%
\begin{pgfscope}%
\pgfsetbuttcap%
\pgfsetroundjoin%
\definecolor{currentfill}{rgb}{0.000000,0.000000,0.000000}%
\pgfsetfillcolor{currentfill}%
\pgfsetlinewidth{0.803000pt}%
\definecolor{currentstroke}{rgb}{0.000000,0.000000,0.000000}%
\pgfsetstrokecolor{currentstroke}%
\pgfsetdash{}{0pt}%
\pgfsys@defobject{currentmarker}{\pgfqpoint{0.000000in}{-0.048611in}}{\pgfqpoint{0.000000in}{0.000000in}}{%
\pgfpathmoveto{\pgfqpoint{0.000000in}{0.000000in}}%
\pgfpathlineto{\pgfqpoint{0.000000in}{-0.048611in}}%
\pgfusepath{stroke,fill}%
}%
\begin{pgfscope}%
\pgfsys@transformshift{2.258750in}{0.499444in}%
\pgfsys@useobject{currentmarker}{}%
\end{pgfscope}%
\end{pgfscope}%
\begin{pgfscope}%
\definecolor{textcolor}{rgb}{0.000000,0.000000,0.000000}%
\pgfsetstrokecolor{textcolor}%
\pgfsetfillcolor{textcolor}%
\pgftext[x=2.258750in,y=0.402222in,,top]{\color{textcolor}\rmfamily\fontsize{10.000000}{12.000000}\selectfont 0.5}%
\end{pgfscope}%
\begin{pgfscope}%
\pgfsetbuttcap%
\pgfsetroundjoin%
\definecolor{currentfill}{rgb}{0.000000,0.000000,0.000000}%
\pgfsetfillcolor{currentfill}%
\pgfsetlinewidth{0.803000pt}%
\definecolor{currentstroke}{rgb}{0.000000,0.000000,0.000000}%
\pgfsetstrokecolor{currentstroke}%
\pgfsetdash{}{0pt}%
\pgfsys@defobject{currentmarker}{\pgfqpoint{0.000000in}{-0.048611in}}{\pgfqpoint{0.000000in}{0.000000in}}{%
\pgfpathmoveto{\pgfqpoint{0.000000in}{0.000000in}}%
\pgfpathlineto{\pgfqpoint{0.000000in}{-0.048611in}}%
\pgfusepath{stroke,fill}%
}%
\begin{pgfscope}%
\pgfsys@transformshift{2.417273in}{0.499444in}%
\pgfsys@useobject{currentmarker}{}%
\end{pgfscope}%
\end{pgfscope}%
\begin{pgfscope}%
\pgfsetbuttcap%
\pgfsetroundjoin%
\definecolor{currentfill}{rgb}{0.000000,0.000000,0.000000}%
\pgfsetfillcolor{currentfill}%
\pgfsetlinewidth{0.803000pt}%
\definecolor{currentstroke}{rgb}{0.000000,0.000000,0.000000}%
\pgfsetstrokecolor{currentstroke}%
\pgfsetdash{}{0pt}%
\pgfsys@defobject{currentmarker}{\pgfqpoint{0.000000in}{-0.048611in}}{\pgfqpoint{0.000000in}{0.000000in}}{%
\pgfpathmoveto{\pgfqpoint{0.000000in}{0.000000in}}%
\pgfpathlineto{\pgfqpoint{0.000000in}{-0.048611in}}%
\pgfusepath{stroke,fill}%
}%
\begin{pgfscope}%
\pgfsys@transformshift{2.575796in}{0.499444in}%
\pgfsys@useobject{currentmarker}{}%
\end{pgfscope}%
\end{pgfscope}%
\begin{pgfscope}%
\definecolor{textcolor}{rgb}{0.000000,0.000000,0.000000}%
\pgfsetstrokecolor{textcolor}%
\pgfsetfillcolor{textcolor}%
\pgftext[x=2.575796in,y=0.402222in,,top]{\color{textcolor}\rmfamily\fontsize{10.000000}{12.000000}\selectfont 0.6}%
\end{pgfscope}%
\begin{pgfscope}%
\pgfsetbuttcap%
\pgfsetroundjoin%
\definecolor{currentfill}{rgb}{0.000000,0.000000,0.000000}%
\pgfsetfillcolor{currentfill}%
\pgfsetlinewidth{0.803000pt}%
\definecolor{currentstroke}{rgb}{0.000000,0.000000,0.000000}%
\pgfsetstrokecolor{currentstroke}%
\pgfsetdash{}{0pt}%
\pgfsys@defobject{currentmarker}{\pgfqpoint{0.000000in}{-0.048611in}}{\pgfqpoint{0.000000in}{0.000000in}}{%
\pgfpathmoveto{\pgfqpoint{0.000000in}{0.000000in}}%
\pgfpathlineto{\pgfqpoint{0.000000in}{-0.048611in}}%
\pgfusepath{stroke,fill}%
}%
\begin{pgfscope}%
\pgfsys@transformshift{2.734318in}{0.499444in}%
\pgfsys@useobject{currentmarker}{}%
\end{pgfscope}%
\end{pgfscope}%
\begin{pgfscope}%
\pgfsetbuttcap%
\pgfsetroundjoin%
\definecolor{currentfill}{rgb}{0.000000,0.000000,0.000000}%
\pgfsetfillcolor{currentfill}%
\pgfsetlinewidth{0.803000pt}%
\definecolor{currentstroke}{rgb}{0.000000,0.000000,0.000000}%
\pgfsetstrokecolor{currentstroke}%
\pgfsetdash{}{0pt}%
\pgfsys@defobject{currentmarker}{\pgfqpoint{0.000000in}{-0.048611in}}{\pgfqpoint{0.000000in}{0.000000in}}{%
\pgfpathmoveto{\pgfqpoint{0.000000in}{0.000000in}}%
\pgfpathlineto{\pgfqpoint{0.000000in}{-0.048611in}}%
\pgfusepath{stroke,fill}%
}%
\begin{pgfscope}%
\pgfsys@transformshift{2.892841in}{0.499444in}%
\pgfsys@useobject{currentmarker}{}%
\end{pgfscope}%
\end{pgfscope}%
\begin{pgfscope}%
\definecolor{textcolor}{rgb}{0.000000,0.000000,0.000000}%
\pgfsetstrokecolor{textcolor}%
\pgfsetfillcolor{textcolor}%
\pgftext[x=2.892841in,y=0.402222in,,top]{\color{textcolor}\rmfamily\fontsize{10.000000}{12.000000}\selectfont 0.7}%
\end{pgfscope}%
\begin{pgfscope}%
\pgfsetbuttcap%
\pgfsetroundjoin%
\definecolor{currentfill}{rgb}{0.000000,0.000000,0.000000}%
\pgfsetfillcolor{currentfill}%
\pgfsetlinewidth{0.803000pt}%
\definecolor{currentstroke}{rgb}{0.000000,0.000000,0.000000}%
\pgfsetstrokecolor{currentstroke}%
\pgfsetdash{}{0pt}%
\pgfsys@defobject{currentmarker}{\pgfqpoint{0.000000in}{-0.048611in}}{\pgfqpoint{0.000000in}{0.000000in}}{%
\pgfpathmoveto{\pgfqpoint{0.000000in}{0.000000in}}%
\pgfpathlineto{\pgfqpoint{0.000000in}{-0.048611in}}%
\pgfusepath{stroke,fill}%
}%
\begin{pgfscope}%
\pgfsys@transformshift{3.051364in}{0.499444in}%
\pgfsys@useobject{currentmarker}{}%
\end{pgfscope}%
\end{pgfscope}%
\begin{pgfscope}%
\pgfsetbuttcap%
\pgfsetroundjoin%
\definecolor{currentfill}{rgb}{0.000000,0.000000,0.000000}%
\pgfsetfillcolor{currentfill}%
\pgfsetlinewidth{0.803000pt}%
\definecolor{currentstroke}{rgb}{0.000000,0.000000,0.000000}%
\pgfsetstrokecolor{currentstroke}%
\pgfsetdash{}{0pt}%
\pgfsys@defobject{currentmarker}{\pgfqpoint{0.000000in}{-0.048611in}}{\pgfqpoint{0.000000in}{0.000000in}}{%
\pgfpathmoveto{\pgfqpoint{0.000000in}{0.000000in}}%
\pgfpathlineto{\pgfqpoint{0.000000in}{-0.048611in}}%
\pgfusepath{stroke,fill}%
}%
\begin{pgfscope}%
\pgfsys@transformshift{3.209887in}{0.499444in}%
\pgfsys@useobject{currentmarker}{}%
\end{pgfscope}%
\end{pgfscope}%
\begin{pgfscope}%
\definecolor{textcolor}{rgb}{0.000000,0.000000,0.000000}%
\pgfsetstrokecolor{textcolor}%
\pgfsetfillcolor{textcolor}%
\pgftext[x=3.209887in,y=0.402222in,,top]{\color{textcolor}\rmfamily\fontsize{10.000000}{12.000000}\selectfont 0.8}%
\end{pgfscope}%
\begin{pgfscope}%
\pgfsetbuttcap%
\pgfsetroundjoin%
\definecolor{currentfill}{rgb}{0.000000,0.000000,0.000000}%
\pgfsetfillcolor{currentfill}%
\pgfsetlinewidth{0.803000pt}%
\definecolor{currentstroke}{rgb}{0.000000,0.000000,0.000000}%
\pgfsetstrokecolor{currentstroke}%
\pgfsetdash{}{0pt}%
\pgfsys@defobject{currentmarker}{\pgfqpoint{0.000000in}{-0.048611in}}{\pgfqpoint{0.000000in}{0.000000in}}{%
\pgfpathmoveto{\pgfqpoint{0.000000in}{0.000000in}}%
\pgfpathlineto{\pgfqpoint{0.000000in}{-0.048611in}}%
\pgfusepath{stroke,fill}%
}%
\begin{pgfscope}%
\pgfsys@transformshift{3.368409in}{0.499444in}%
\pgfsys@useobject{currentmarker}{}%
\end{pgfscope}%
\end{pgfscope}%
\begin{pgfscope}%
\pgfsetbuttcap%
\pgfsetroundjoin%
\definecolor{currentfill}{rgb}{0.000000,0.000000,0.000000}%
\pgfsetfillcolor{currentfill}%
\pgfsetlinewidth{0.803000pt}%
\definecolor{currentstroke}{rgb}{0.000000,0.000000,0.000000}%
\pgfsetstrokecolor{currentstroke}%
\pgfsetdash{}{0pt}%
\pgfsys@defobject{currentmarker}{\pgfqpoint{0.000000in}{-0.048611in}}{\pgfqpoint{0.000000in}{0.000000in}}{%
\pgfpathmoveto{\pgfqpoint{0.000000in}{0.000000in}}%
\pgfpathlineto{\pgfqpoint{0.000000in}{-0.048611in}}%
\pgfusepath{stroke,fill}%
}%
\begin{pgfscope}%
\pgfsys@transformshift{3.526932in}{0.499444in}%
\pgfsys@useobject{currentmarker}{}%
\end{pgfscope}%
\end{pgfscope}%
\begin{pgfscope}%
\definecolor{textcolor}{rgb}{0.000000,0.000000,0.000000}%
\pgfsetstrokecolor{textcolor}%
\pgfsetfillcolor{textcolor}%
\pgftext[x=3.526932in,y=0.402222in,,top]{\color{textcolor}\rmfamily\fontsize{10.000000}{12.000000}\selectfont 0.9}%
\end{pgfscope}%
\begin{pgfscope}%
\pgfsetbuttcap%
\pgfsetroundjoin%
\definecolor{currentfill}{rgb}{0.000000,0.000000,0.000000}%
\pgfsetfillcolor{currentfill}%
\pgfsetlinewidth{0.803000pt}%
\definecolor{currentstroke}{rgb}{0.000000,0.000000,0.000000}%
\pgfsetstrokecolor{currentstroke}%
\pgfsetdash{}{0pt}%
\pgfsys@defobject{currentmarker}{\pgfqpoint{0.000000in}{-0.048611in}}{\pgfqpoint{0.000000in}{0.000000in}}{%
\pgfpathmoveto{\pgfqpoint{0.000000in}{0.000000in}}%
\pgfpathlineto{\pgfqpoint{0.000000in}{-0.048611in}}%
\pgfusepath{stroke,fill}%
}%
\begin{pgfscope}%
\pgfsys@transformshift{3.685455in}{0.499444in}%
\pgfsys@useobject{currentmarker}{}%
\end{pgfscope}%
\end{pgfscope}%
\begin{pgfscope}%
\pgfsetbuttcap%
\pgfsetroundjoin%
\definecolor{currentfill}{rgb}{0.000000,0.000000,0.000000}%
\pgfsetfillcolor{currentfill}%
\pgfsetlinewidth{0.803000pt}%
\definecolor{currentstroke}{rgb}{0.000000,0.000000,0.000000}%
\pgfsetstrokecolor{currentstroke}%
\pgfsetdash{}{0pt}%
\pgfsys@defobject{currentmarker}{\pgfqpoint{0.000000in}{-0.048611in}}{\pgfqpoint{0.000000in}{0.000000in}}{%
\pgfpathmoveto{\pgfqpoint{0.000000in}{0.000000in}}%
\pgfpathlineto{\pgfqpoint{0.000000in}{-0.048611in}}%
\pgfusepath{stroke,fill}%
}%
\begin{pgfscope}%
\pgfsys@transformshift{3.843978in}{0.499444in}%
\pgfsys@useobject{currentmarker}{}%
\end{pgfscope}%
\end{pgfscope}%
\begin{pgfscope}%
\definecolor{textcolor}{rgb}{0.000000,0.000000,0.000000}%
\pgfsetstrokecolor{textcolor}%
\pgfsetfillcolor{textcolor}%
\pgftext[x=3.843978in,y=0.402222in,,top]{\color{textcolor}\rmfamily\fontsize{10.000000}{12.000000}\selectfont 1.0}%
\end{pgfscope}%
\begin{pgfscope}%
\pgfsetbuttcap%
\pgfsetroundjoin%
\definecolor{currentfill}{rgb}{0.000000,0.000000,0.000000}%
\pgfsetfillcolor{currentfill}%
\pgfsetlinewidth{0.803000pt}%
\definecolor{currentstroke}{rgb}{0.000000,0.000000,0.000000}%
\pgfsetstrokecolor{currentstroke}%
\pgfsetdash{}{0pt}%
\pgfsys@defobject{currentmarker}{\pgfqpoint{0.000000in}{-0.048611in}}{\pgfqpoint{0.000000in}{0.000000in}}{%
\pgfpathmoveto{\pgfqpoint{0.000000in}{0.000000in}}%
\pgfpathlineto{\pgfqpoint{0.000000in}{-0.048611in}}%
\pgfusepath{stroke,fill}%
}%
\begin{pgfscope}%
\pgfsys@transformshift{4.002500in}{0.499444in}%
\pgfsys@useobject{currentmarker}{}%
\end{pgfscope}%
\end{pgfscope}%
\begin{pgfscope}%
\definecolor{textcolor}{rgb}{0.000000,0.000000,0.000000}%
\pgfsetstrokecolor{textcolor}%
\pgfsetfillcolor{textcolor}%
\pgftext[x=2.258750in,y=0.223333in,,top]{\color{textcolor}\rmfamily\fontsize{10.000000}{12.000000}\selectfont \(\displaystyle p\)}%
\end{pgfscope}%
\begin{pgfscope}%
\pgfsetbuttcap%
\pgfsetroundjoin%
\definecolor{currentfill}{rgb}{0.000000,0.000000,0.000000}%
\pgfsetfillcolor{currentfill}%
\pgfsetlinewidth{0.803000pt}%
\definecolor{currentstroke}{rgb}{0.000000,0.000000,0.000000}%
\pgfsetstrokecolor{currentstroke}%
\pgfsetdash{}{0pt}%
\pgfsys@defobject{currentmarker}{\pgfqpoint{-0.048611in}{0.000000in}}{\pgfqpoint{-0.000000in}{0.000000in}}{%
\pgfpathmoveto{\pgfqpoint{-0.000000in}{0.000000in}}%
\pgfpathlineto{\pgfqpoint{-0.048611in}{0.000000in}}%
\pgfusepath{stroke,fill}%
}%
\begin{pgfscope}%
\pgfsys@transformshift{0.515000in}{0.499444in}%
\pgfsys@useobject{currentmarker}{}%
\end{pgfscope}%
\end{pgfscope}%
\begin{pgfscope}%
\definecolor{textcolor}{rgb}{0.000000,0.000000,0.000000}%
\pgfsetstrokecolor{textcolor}%
\pgfsetfillcolor{textcolor}%
\pgftext[x=0.348333in, y=0.451250in, left, base]{\color{textcolor}\rmfamily\fontsize{10.000000}{12.000000}\selectfont \(\displaystyle {0}\)}%
\end{pgfscope}%
\begin{pgfscope}%
\pgfsetbuttcap%
\pgfsetroundjoin%
\definecolor{currentfill}{rgb}{0.000000,0.000000,0.000000}%
\pgfsetfillcolor{currentfill}%
\pgfsetlinewidth{0.803000pt}%
\definecolor{currentstroke}{rgb}{0.000000,0.000000,0.000000}%
\pgfsetstrokecolor{currentstroke}%
\pgfsetdash{}{0pt}%
\pgfsys@defobject{currentmarker}{\pgfqpoint{-0.048611in}{0.000000in}}{\pgfqpoint{-0.000000in}{0.000000in}}{%
\pgfpathmoveto{\pgfqpoint{-0.000000in}{0.000000in}}%
\pgfpathlineto{\pgfqpoint{-0.048611in}{0.000000in}}%
\pgfusepath{stroke,fill}%
}%
\begin{pgfscope}%
\pgfsys@transformshift{0.515000in}{0.927911in}%
\pgfsys@useobject{currentmarker}{}%
\end{pgfscope}%
\end{pgfscope}%
\begin{pgfscope}%
\definecolor{textcolor}{rgb}{0.000000,0.000000,0.000000}%
\pgfsetstrokecolor{textcolor}%
\pgfsetfillcolor{textcolor}%
\pgftext[x=0.278889in, y=0.879717in, left, base]{\color{textcolor}\rmfamily\fontsize{10.000000}{12.000000}\selectfont \(\displaystyle {10}\)}%
\end{pgfscope}%
\begin{pgfscope}%
\pgfsetbuttcap%
\pgfsetroundjoin%
\definecolor{currentfill}{rgb}{0.000000,0.000000,0.000000}%
\pgfsetfillcolor{currentfill}%
\pgfsetlinewidth{0.803000pt}%
\definecolor{currentstroke}{rgb}{0.000000,0.000000,0.000000}%
\pgfsetstrokecolor{currentstroke}%
\pgfsetdash{}{0pt}%
\pgfsys@defobject{currentmarker}{\pgfqpoint{-0.048611in}{0.000000in}}{\pgfqpoint{-0.000000in}{0.000000in}}{%
\pgfpathmoveto{\pgfqpoint{-0.000000in}{0.000000in}}%
\pgfpathlineto{\pgfqpoint{-0.048611in}{0.000000in}}%
\pgfusepath{stroke,fill}%
}%
\begin{pgfscope}%
\pgfsys@transformshift{0.515000in}{1.356379in}%
\pgfsys@useobject{currentmarker}{}%
\end{pgfscope}%
\end{pgfscope}%
\begin{pgfscope}%
\definecolor{textcolor}{rgb}{0.000000,0.000000,0.000000}%
\pgfsetstrokecolor{textcolor}%
\pgfsetfillcolor{textcolor}%
\pgftext[x=0.278889in, y=1.308184in, left, base]{\color{textcolor}\rmfamily\fontsize{10.000000}{12.000000}\selectfont \(\displaystyle {20}\)}%
\end{pgfscope}%
\begin{pgfscope}%
\definecolor{textcolor}{rgb}{0.000000,0.000000,0.000000}%
\pgfsetstrokecolor{textcolor}%
\pgfsetfillcolor{textcolor}%
\pgftext[x=0.223333in,y=1.076944in,,bottom,rotate=90.000000]{\color{textcolor}\rmfamily\fontsize{10.000000}{12.000000}\selectfont Percent of Data Set}%
\end{pgfscope}%
\begin{pgfscope}%
\pgfsetrectcap%
\pgfsetmiterjoin%
\pgfsetlinewidth{0.803000pt}%
\definecolor{currentstroke}{rgb}{0.000000,0.000000,0.000000}%
\pgfsetstrokecolor{currentstroke}%
\pgfsetdash{}{0pt}%
\pgfpathmoveto{\pgfqpoint{0.515000in}{0.499444in}}%
\pgfpathlineto{\pgfqpoint{0.515000in}{1.654444in}}%
\pgfusepath{stroke}%
\end{pgfscope}%
\begin{pgfscope}%
\pgfsetrectcap%
\pgfsetmiterjoin%
\pgfsetlinewidth{0.803000pt}%
\definecolor{currentstroke}{rgb}{0.000000,0.000000,0.000000}%
\pgfsetstrokecolor{currentstroke}%
\pgfsetdash{}{0pt}%
\pgfpathmoveto{\pgfqpoint{4.002500in}{0.499444in}}%
\pgfpathlineto{\pgfqpoint{4.002500in}{1.654444in}}%
\pgfusepath{stroke}%
\end{pgfscope}%
\begin{pgfscope}%
\pgfsetrectcap%
\pgfsetmiterjoin%
\pgfsetlinewidth{0.803000pt}%
\definecolor{currentstroke}{rgb}{0.000000,0.000000,0.000000}%
\pgfsetstrokecolor{currentstroke}%
\pgfsetdash{}{0pt}%
\pgfpathmoveto{\pgfqpoint{0.515000in}{0.499444in}}%
\pgfpathlineto{\pgfqpoint{4.002500in}{0.499444in}}%
\pgfusepath{stroke}%
\end{pgfscope}%
\begin{pgfscope}%
\pgfsetrectcap%
\pgfsetmiterjoin%
\pgfsetlinewidth{0.803000pt}%
\definecolor{currentstroke}{rgb}{0.000000,0.000000,0.000000}%
\pgfsetstrokecolor{currentstroke}%
\pgfsetdash{}{0pt}%
\pgfpathmoveto{\pgfqpoint{0.515000in}{1.654444in}}%
\pgfpathlineto{\pgfqpoint{4.002500in}{1.654444in}}%
\pgfusepath{stroke}%
\end{pgfscope}%
\begin{pgfscope}%
\pgfsetbuttcap%
\pgfsetmiterjoin%
\definecolor{currentfill}{rgb}{1.000000,1.000000,1.000000}%
\pgfsetfillcolor{currentfill}%
\pgfsetfillopacity{0.800000}%
\pgfsetlinewidth{1.003750pt}%
\definecolor{currentstroke}{rgb}{0.800000,0.800000,0.800000}%
\pgfsetstrokecolor{currentstroke}%
\pgfsetstrokeopacity{0.800000}%
\pgfsetdash{}{0pt}%
\pgfpathmoveto{\pgfqpoint{3.225556in}{1.154445in}}%
\pgfpathlineto{\pgfqpoint{3.905278in}{1.154445in}}%
\pgfpathquadraticcurveto{\pgfqpoint{3.933056in}{1.154445in}}{\pgfqpoint{3.933056in}{1.182222in}}%
\pgfpathlineto{\pgfqpoint{3.933056in}{1.557222in}}%
\pgfpathquadraticcurveto{\pgfqpoint{3.933056in}{1.585000in}}{\pgfqpoint{3.905278in}{1.585000in}}%
\pgfpathlineto{\pgfqpoint{3.225556in}{1.585000in}}%
\pgfpathquadraticcurveto{\pgfqpoint{3.197778in}{1.585000in}}{\pgfqpoint{3.197778in}{1.557222in}}%
\pgfpathlineto{\pgfqpoint{3.197778in}{1.182222in}}%
\pgfpathquadraticcurveto{\pgfqpoint{3.197778in}{1.154445in}}{\pgfqpoint{3.225556in}{1.154445in}}%
\pgfpathlineto{\pgfqpoint{3.225556in}{1.154445in}}%
\pgfpathclose%
\pgfusepath{stroke,fill}%
\end{pgfscope}%
\begin{pgfscope}%
\pgfsetbuttcap%
\pgfsetmiterjoin%
\pgfsetlinewidth{1.003750pt}%
\definecolor{currentstroke}{rgb}{0.000000,0.000000,0.000000}%
\pgfsetstrokecolor{currentstroke}%
\pgfsetdash{}{0pt}%
\pgfpathmoveto{\pgfqpoint{3.253334in}{1.432222in}}%
\pgfpathlineto{\pgfqpoint{3.531111in}{1.432222in}}%
\pgfpathlineto{\pgfqpoint{3.531111in}{1.529444in}}%
\pgfpathlineto{\pgfqpoint{3.253334in}{1.529444in}}%
\pgfpathlineto{\pgfqpoint{3.253334in}{1.432222in}}%
\pgfpathclose%
\pgfusepath{stroke}%
\end{pgfscope}%
\begin{pgfscope}%
\definecolor{textcolor}{rgb}{0.000000,0.000000,0.000000}%
\pgfsetstrokecolor{textcolor}%
\pgfsetfillcolor{textcolor}%
\pgftext[x=3.642223in,y=1.432222in,left,base]{\color{textcolor}\rmfamily\fontsize{10.000000}{12.000000}\selectfont Neg}%
\end{pgfscope}%
\begin{pgfscope}%
\pgfsetbuttcap%
\pgfsetmiterjoin%
\definecolor{currentfill}{rgb}{0.000000,0.000000,0.000000}%
\pgfsetfillcolor{currentfill}%
\pgfsetlinewidth{0.000000pt}%
\definecolor{currentstroke}{rgb}{0.000000,0.000000,0.000000}%
\pgfsetstrokecolor{currentstroke}%
\pgfsetstrokeopacity{0.000000}%
\pgfsetdash{}{0pt}%
\pgfpathmoveto{\pgfqpoint{3.253334in}{1.236944in}}%
\pgfpathlineto{\pgfqpoint{3.531111in}{1.236944in}}%
\pgfpathlineto{\pgfqpoint{3.531111in}{1.334167in}}%
\pgfpathlineto{\pgfqpoint{3.253334in}{1.334167in}}%
\pgfpathlineto{\pgfqpoint{3.253334in}{1.236944in}}%
\pgfpathclose%
\pgfusepath{fill}%
\end{pgfscope}%
\begin{pgfscope}%
\definecolor{textcolor}{rgb}{0.000000,0.000000,0.000000}%
\pgfsetstrokecolor{textcolor}%
\pgfsetfillcolor{textcolor}%
\pgftext[x=3.642223in,y=1.236944in,left,base]{\color{textcolor}\rmfamily\fontsize{10.000000}{12.000000}\selectfont Pos}%
\end{pgfscope}%
\end{pgfpicture}%
\makeatother%
\endgroup%

&
	\vskip 0pt
	\qquad \qquad ROC Curve
	
	%% Creator: Matplotlib, PGF backend
%%
%% To include the figure in your LaTeX document, write
%%   \input{<filename>.pgf}
%%
%% Make sure the required packages are loaded in your preamble
%%   \usepackage{pgf}
%%
%% Also ensure that all the required font packages are loaded; for instance,
%% the lmodern package is sometimes necessary when using math font.
%%   \usepackage{lmodern}
%%
%% Figures using additional raster images can only be included by \input if
%% they are in the same directory as the main LaTeX file. For loading figures
%% from other directories you can use the `import` package
%%   \usepackage{import}
%%
%% and then include the figures with
%%   \import{<path to file>}{<filename>.pgf}
%%
%% Matplotlib used the following preamble
%%   
%%   \usepackage{fontspec}
%%   \makeatletter\@ifpackageloaded{underscore}{}{\usepackage[strings]{underscore}}\makeatother
%%
\begingroup%
\makeatletter%
\begin{pgfpicture}%
\pgfpathrectangle{\pgfpointorigin}{\pgfqpoint{2.221861in}{1.754444in}}%
\pgfusepath{use as bounding box, clip}%
\begin{pgfscope}%
\pgfsetbuttcap%
\pgfsetmiterjoin%
\definecolor{currentfill}{rgb}{1.000000,1.000000,1.000000}%
\pgfsetfillcolor{currentfill}%
\pgfsetlinewidth{0.000000pt}%
\definecolor{currentstroke}{rgb}{1.000000,1.000000,1.000000}%
\pgfsetstrokecolor{currentstroke}%
\pgfsetdash{}{0pt}%
\pgfpathmoveto{\pgfqpoint{0.000000in}{0.000000in}}%
\pgfpathlineto{\pgfqpoint{2.221861in}{0.000000in}}%
\pgfpathlineto{\pgfqpoint{2.221861in}{1.754444in}}%
\pgfpathlineto{\pgfqpoint{0.000000in}{1.754444in}}%
\pgfpathlineto{\pgfqpoint{0.000000in}{0.000000in}}%
\pgfpathclose%
\pgfusepath{fill}%
\end{pgfscope}%
\begin{pgfscope}%
\pgfsetbuttcap%
\pgfsetmiterjoin%
\definecolor{currentfill}{rgb}{1.000000,1.000000,1.000000}%
\pgfsetfillcolor{currentfill}%
\pgfsetlinewidth{0.000000pt}%
\definecolor{currentstroke}{rgb}{0.000000,0.000000,0.000000}%
\pgfsetstrokecolor{currentstroke}%
\pgfsetstrokeopacity{0.000000}%
\pgfsetdash{}{0pt}%
\pgfpathmoveto{\pgfqpoint{0.553581in}{0.499444in}}%
\pgfpathlineto{\pgfqpoint{2.103581in}{0.499444in}}%
\pgfpathlineto{\pgfqpoint{2.103581in}{1.654444in}}%
\pgfpathlineto{\pgfqpoint{0.553581in}{1.654444in}}%
\pgfpathlineto{\pgfqpoint{0.553581in}{0.499444in}}%
\pgfpathclose%
\pgfusepath{fill}%
\end{pgfscope}%
\begin{pgfscope}%
\pgfsetbuttcap%
\pgfsetroundjoin%
\definecolor{currentfill}{rgb}{0.000000,0.000000,0.000000}%
\pgfsetfillcolor{currentfill}%
\pgfsetlinewidth{0.803000pt}%
\definecolor{currentstroke}{rgb}{0.000000,0.000000,0.000000}%
\pgfsetstrokecolor{currentstroke}%
\pgfsetdash{}{0pt}%
\pgfsys@defobject{currentmarker}{\pgfqpoint{0.000000in}{-0.048611in}}{\pgfqpoint{0.000000in}{0.000000in}}{%
\pgfpathmoveto{\pgfqpoint{0.000000in}{0.000000in}}%
\pgfpathlineto{\pgfqpoint{0.000000in}{-0.048611in}}%
\pgfusepath{stroke,fill}%
}%
\begin{pgfscope}%
\pgfsys@transformshift{0.624035in}{0.499444in}%
\pgfsys@useobject{currentmarker}{}%
\end{pgfscope}%
\end{pgfscope}%
\begin{pgfscope}%
\definecolor{textcolor}{rgb}{0.000000,0.000000,0.000000}%
\pgfsetstrokecolor{textcolor}%
\pgfsetfillcolor{textcolor}%
\pgftext[x=0.624035in,y=0.402222in,,top]{\color{textcolor}\rmfamily\fontsize{10.000000}{12.000000}\selectfont \(\displaystyle {0.0}\)}%
\end{pgfscope}%
\begin{pgfscope}%
\pgfsetbuttcap%
\pgfsetroundjoin%
\definecolor{currentfill}{rgb}{0.000000,0.000000,0.000000}%
\pgfsetfillcolor{currentfill}%
\pgfsetlinewidth{0.803000pt}%
\definecolor{currentstroke}{rgb}{0.000000,0.000000,0.000000}%
\pgfsetstrokecolor{currentstroke}%
\pgfsetdash{}{0pt}%
\pgfsys@defobject{currentmarker}{\pgfqpoint{0.000000in}{-0.048611in}}{\pgfqpoint{0.000000in}{0.000000in}}{%
\pgfpathmoveto{\pgfqpoint{0.000000in}{0.000000in}}%
\pgfpathlineto{\pgfqpoint{0.000000in}{-0.048611in}}%
\pgfusepath{stroke,fill}%
}%
\begin{pgfscope}%
\pgfsys@transformshift{1.328581in}{0.499444in}%
\pgfsys@useobject{currentmarker}{}%
\end{pgfscope}%
\end{pgfscope}%
\begin{pgfscope}%
\definecolor{textcolor}{rgb}{0.000000,0.000000,0.000000}%
\pgfsetstrokecolor{textcolor}%
\pgfsetfillcolor{textcolor}%
\pgftext[x=1.328581in,y=0.402222in,,top]{\color{textcolor}\rmfamily\fontsize{10.000000}{12.000000}\selectfont \(\displaystyle {0.5}\)}%
\end{pgfscope}%
\begin{pgfscope}%
\pgfsetbuttcap%
\pgfsetroundjoin%
\definecolor{currentfill}{rgb}{0.000000,0.000000,0.000000}%
\pgfsetfillcolor{currentfill}%
\pgfsetlinewidth{0.803000pt}%
\definecolor{currentstroke}{rgb}{0.000000,0.000000,0.000000}%
\pgfsetstrokecolor{currentstroke}%
\pgfsetdash{}{0pt}%
\pgfsys@defobject{currentmarker}{\pgfqpoint{0.000000in}{-0.048611in}}{\pgfqpoint{0.000000in}{0.000000in}}{%
\pgfpathmoveto{\pgfqpoint{0.000000in}{0.000000in}}%
\pgfpathlineto{\pgfqpoint{0.000000in}{-0.048611in}}%
\pgfusepath{stroke,fill}%
}%
\begin{pgfscope}%
\pgfsys@transformshift{2.033126in}{0.499444in}%
\pgfsys@useobject{currentmarker}{}%
\end{pgfscope}%
\end{pgfscope}%
\begin{pgfscope}%
\definecolor{textcolor}{rgb}{0.000000,0.000000,0.000000}%
\pgfsetstrokecolor{textcolor}%
\pgfsetfillcolor{textcolor}%
\pgftext[x=2.033126in,y=0.402222in,,top]{\color{textcolor}\rmfamily\fontsize{10.000000}{12.000000}\selectfont \(\displaystyle {1.0}\)}%
\end{pgfscope}%
\begin{pgfscope}%
\definecolor{textcolor}{rgb}{0.000000,0.000000,0.000000}%
\pgfsetstrokecolor{textcolor}%
\pgfsetfillcolor{textcolor}%
\pgftext[x=1.328581in,y=0.223333in,,top]{\color{textcolor}\rmfamily\fontsize{10.000000}{12.000000}\selectfont False positive rate}%
\end{pgfscope}%
\begin{pgfscope}%
\pgfsetbuttcap%
\pgfsetroundjoin%
\definecolor{currentfill}{rgb}{0.000000,0.000000,0.000000}%
\pgfsetfillcolor{currentfill}%
\pgfsetlinewidth{0.803000pt}%
\definecolor{currentstroke}{rgb}{0.000000,0.000000,0.000000}%
\pgfsetstrokecolor{currentstroke}%
\pgfsetdash{}{0pt}%
\pgfsys@defobject{currentmarker}{\pgfqpoint{-0.048611in}{0.000000in}}{\pgfqpoint{-0.000000in}{0.000000in}}{%
\pgfpathmoveto{\pgfqpoint{-0.000000in}{0.000000in}}%
\pgfpathlineto{\pgfqpoint{-0.048611in}{0.000000in}}%
\pgfusepath{stroke,fill}%
}%
\begin{pgfscope}%
\pgfsys@transformshift{0.553581in}{0.551944in}%
\pgfsys@useobject{currentmarker}{}%
\end{pgfscope}%
\end{pgfscope}%
\begin{pgfscope}%
\definecolor{textcolor}{rgb}{0.000000,0.000000,0.000000}%
\pgfsetstrokecolor{textcolor}%
\pgfsetfillcolor{textcolor}%
\pgftext[x=0.278889in, y=0.503750in, left, base]{\color{textcolor}\rmfamily\fontsize{10.000000}{12.000000}\selectfont \(\displaystyle {0.0}\)}%
\end{pgfscope}%
\begin{pgfscope}%
\pgfsetbuttcap%
\pgfsetroundjoin%
\definecolor{currentfill}{rgb}{0.000000,0.000000,0.000000}%
\pgfsetfillcolor{currentfill}%
\pgfsetlinewidth{0.803000pt}%
\definecolor{currentstroke}{rgb}{0.000000,0.000000,0.000000}%
\pgfsetstrokecolor{currentstroke}%
\pgfsetdash{}{0pt}%
\pgfsys@defobject{currentmarker}{\pgfqpoint{-0.048611in}{0.000000in}}{\pgfqpoint{-0.000000in}{0.000000in}}{%
\pgfpathmoveto{\pgfqpoint{-0.000000in}{0.000000in}}%
\pgfpathlineto{\pgfqpoint{-0.048611in}{0.000000in}}%
\pgfusepath{stroke,fill}%
}%
\begin{pgfscope}%
\pgfsys@transformshift{0.553581in}{1.076944in}%
\pgfsys@useobject{currentmarker}{}%
\end{pgfscope}%
\end{pgfscope}%
\begin{pgfscope}%
\definecolor{textcolor}{rgb}{0.000000,0.000000,0.000000}%
\pgfsetstrokecolor{textcolor}%
\pgfsetfillcolor{textcolor}%
\pgftext[x=0.278889in, y=1.028750in, left, base]{\color{textcolor}\rmfamily\fontsize{10.000000}{12.000000}\selectfont \(\displaystyle {0.5}\)}%
\end{pgfscope}%
\begin{pgfscope}%
\pgfsetbuttcap%
\pgfsetroundjoin%
\definecolor{currentfill}{rgb}{0.000000,0.000000,0.000000}%
\pgfsetfillcolor{currentfill}%
\pgfsetlinewidth{0.803000pt}%
\definecolor{currentstroke}{rgb}{0.000000,0.000000,0.000000}%
\pgfsetstrokecolor{currentstroke}%
\pgfsetdash{}{0pt}%
\pgfsys@defobject{currentmarker}{\pgfqpoint{-0.048611in}{0.000000in}}{\pgfqpoint{-0.000000in}{0.000000in}}{%
\pgfpathmoveto{\pgfqpoint{-0.000000in}{0.000000in}}%
\pgfpathlineto{\pgfqpoint{-0.048611in}{0.000000in}}%
\pgfusepath{stroke,fill}%
}%
\begin{pgfscope}%
\pgfsys@transformshift{0.553581in}{1.601944in}%
\pgfsys@useobject{currentmarker}{}%
\end{pgfscope}%
\end{pgfscope}%
\begin{pgfscope}%
\definecolor{textcolor}{rgb}{0.000000,0.000000,0.000000}%
\pgfsetstrokecolor{textcolor}%
\pgfsetfillcolor{textcolor}%
\pgftext[x=0.278889in, y=1.553750in, left, base]{\color{textcolor}\rmfamily\fontsize{10.000000}{12.000000}\selectfont \(\displaystyle {1.0}\)}%
\end{pgfscope}%
\begin{pgfscope}%
\definecolor{textcolor}{rgb}{0.000000,0.000000,0.000000}%
\pgfsetstrokecolor{textcolor}%
\pgfsetfillcolor{textcolor}%
\pgftext[x=0.223333in,y=1.076944in,,bottom,rotate=90.000000]{\color{textcolor}\rmfamily\fontsize{10.000000}{12.000000}\selectfont True positive rate}%
\end{pgfscope}%
\begin{pgfscope}%
\pgfpathrectangle{\pgfqpoint{0.553581in}{0.499444in}}{\pgfqpoint{1.550000in}{1.155000in}}%
\pgfusepath{clip}%
\pgfsetbuttcap%
\pgfsetroundjoin%
\pgfsetlinewidth{1.505625pt}%
\definecolor{currentstroke}{rgb}{0.000000,0.000000,0.000000}%
\pgfsetstrokecolor{currentstroke}%
\pgfsetdash{{5.550000pt}{2.400000pt}}{0.000000pt}%
\pgfpathmoveto{\pgfqpoint{0.624035in}{0.551944in}}%
\pgfpathlineto{\pgfqpoint{2.033126in}{1.601944in}}%
\pgfusepath{stroke}%
\end{pgfscope}%
\begin{pgfscope}%
\pgfpathrectangle{\pgfqpoint{0.553581in}{0.499444in}}{\pgfqpoint{1.550000in}{1.155000in}}%
\pgfusepath{clip}%
\pgfsetrectcap%
\pgfsetroundjoin%
\pgfsetlinewidth{1.505625pt}%
\definecolor{currentstroke}{rgb}{0.000000,0.000000,0.000000}%
\pgfsetstrokecolor{currentstroke}%
\pgfsetdash{}{0pt}%
\pgfpathmoveto{\pgfqpoint{0.624035in}{0.551944in}}%
\pgfpathlineto{\pgfqpoint{0.625145in}{0.568893in}}%
\pgfpathlineto{\pgfqpoint{0.625239in}{0.569949in}}%
\pgfpathlineto{\pgfqpoint{0.626341in}{0.584414in}}%
\pgfpathlineto{\pgfqpoint{0.626388in}{0.585345in}}%
\pgfpathlineto{\pgfqpoint{0.627498in}{0.595310in}}%
\pgfpathlineto{\pgfqpoint{0.627616in}{0.596396in}}%
\pgfpathlineto{\pgfqpoint{0.628726in}{0.605306in}}%
\pgfpathlineto{\pgfqpoint{0.628781in}{0.606237in}}%
\pgfpathlineto{\pgfqpoint{0.629891in}{0.616698in}}%
\pgfpathlineto{\pgfqpoint{0.629992in}{0.617785in}}%
\pgfpathlineto{\pgfqpoint{0.631102in}{0.624986in}}%
\pgfpathlineto{\pgfqpoint{0.631235in}{0.625887in}}%
\pgfpathlineto{\pgfqpoint{0.632330in}{0.633150in}}%
\pgfpathlineto{\pgfqpoint{0.632533in}{0.634175in}}%
\pgfpathlineto{\pgfqpoint{0.633643in}{0.643208in}}%
\pgfpathlineto{\pgfqpoint{0.633885in}{0.644263in}}%
\pgfpathlineto{\pgfqpoint{0.634996in}{0.651869in}}%
\pgfpathlineto{\pgfqpoint{0.635128in}{0.652924in}}%
\pgfpathlineto{\pgfqpoint{0.636239in}{0.659785in}}%
\pgfpathlineto{\pgfqpoint{0.636442in}{0.660840in}}%
\pgfpathlineto{\pgfqpoint{0.637552in}{0.666862in}}%
\pgfpathlineto{\pgfqpoint{0.637833in}{0.667949in}}%
\pgfpathlineto{\pgfqpoint{0.638943in}{0.674219in}}%
\pgfpathlineto{\pgfqpoint{0.639084in}{0.675306in}}%
\pgfpathlineto{\pgfqpoint{0.640194in}{0.680955in}}%
\pgfpathlineto{\pgfqpoint{0.640444in}{0.682042in}}%
\pgfpathlineto{\pgfqpoint{0.641547in}{0.687567in}}%
\pgfpathlineto{\pgfqpoint{0.641766in}{0.688654in}}%
\pgfpathlineto{\pgfqpoint{0.642868in}{0.693124in}}%
\pgfpathlineto{\pgfqpoint{0.643040in}{0.694210in}}%
\pgfpathlineto{\pgfqpoint{0.644111in}{0.699549in}}%
\pgfpathlineto{\pgfqpoint{0.644431in}{0.700574in}}%
\pgfpathlineto{\pgfqpoint{0.645542in}{0.705696in}}%
\pgfpathlineto{\pgfqpoint{0.645901in}{0.706782in}}%
\pgfpathlineto{\pgfqpoint{0.647003in}{0.711687in}}%
\pgfpathlineto{\pgfqpoint{0.647308in}{0.712773in}}%
\pgfpathlineto{\pgfqpoint{0.648418in}{0.717585in}}%
\pgfpathlineto{\pgfqpoint{0.648661in}{0.718609in}}%
\pgfpathlineto{\pgfqpoint{0.649771in}{0.723390in}}%
\pgfpathlineto{\pgfqpoint{0.650076in}{0.724476in}}%
\pgfpathlineto{\pgfqpoint{0.651186in}{0.727953in}}%
\pgfpathlineto{\pgfqpoint{0.651483in}{0.729040in}}%
\pgfpathlineto{\pgfqpoint{0.652593in}{0.733820in}}%
\pgfpathlineto{\pgfqpoint{0.652921in}{0.734813in}}%
\pgfpathlineto{\pgfqpoint{0.654016in}{0.740059in}}%
\pgfpathlineto{\pgfqpoint{0.654462in}{0.741115in}}%
\pgfpathlineto{\pgfqpoint{0.655556in}{0.745740in}}%
\pgfpathlineto{\pgfqpoint{0.655861in}{0.746827in}}%
\pgfpathlineto{\pgfqpoint{0.656971in}{0.750334in}}%
\pgfpathlineto{\pgfqpoint{0.657323in}{0.751390in}}%
\pgfpathlineto{\pgfqpoint{0.658425in}{0.754711in}}%
\pgfpathlineto{\pgfqpoint{0.658800in}{0.755736in}}%
\pgfpathlineto{\pgfqpoint{0.659895in}{0.759492in}}%
\pgfpathlineto{\pgfqpoint{0.660161in}{0.760516in}}%
\pgfpathlineto{\pgfqpoint{0.661271in}{0.765173in}}%
\pgfpathlineto{\pgfqpoint{0.661544in}{0.766228in}}%
\pgfpathlineto{\pgfqpoint{0.662647in}{0.769922in}}%
\pgfpathlineto{\pgfqpoint{0.663037in}{0.770853in}}%
\pgfpathlineto{\pgfqpoint{0.664148in}{0.774144in}}%
\pgfpathlineto{\pgfqpoint{0.664531in}{0.775199in}}%
\pgfpathlineto{\pgfqpoint{0.665641in}{0.778769in}}%
\pgfpathlineto{\pgfqpoint{0.666094in}{0.779855in}}%
\pgfpathlineto{\pgfqpoint{0.667204in}{0.783239in}}%
\pgfpathlineto{\pgfqpoint{0.667462in}{0.784294in}}%
\pgfpathlineto{\pgfqpoint{0.668565in}{0.787802in}}%
\pgfpathlineto{\pgfqpoint{0.669010in}{0.788889in}}%
\pgfpathlineto{\pgfqpoint{0.670120in}{0.792521in}}%
\pgfpathlineto{\pgfqpoint{0.670511in}{0.793607in}}%
\pgfpathlineto{\pgfqpoint{0.671582in}{0.797860in}}%
\pgfpathlineto{\pgfqpoint{0.671981in}{0.798915in}}%
\pgfpathlineto{\pgfqpoint{0.671981in}{0.798946in}}%
\pgfpathlineto{\pgfqpoint{0.673091in}{0.802113in}}%
\pgfpathlineto{\pgfqpoint{0.673568in}{0.803199in}}%
\pgfpathlineto{\pgfqpoint{0.674670in}{0.806241in}}%
\pgfpathlineto{\pgfqpoint{0.675163in}{0.807328in}}%
\pgfpathlineto{\pgfqpoint{0.676265in}{0.810370in}}%
\pgfpathlineto{\pgfqpoint{0.676562in}{0.811425in}}%
\pgfpathlineto{\pgfqpoint{0.677672in}{0.814467in}}%
\pgfpathlineto{\pgfqpoint{0.678079in}{0.815461in}}%
\pgfpathlineto{\pgfqpoint{0.679181in}{0.819093in}}%
\pgfpathlineto{\pgfqpoint{0.679689in}{0.820148in}}%
\pgfpathlineto{\pgfqpoint{0.680776in}{0.823066in}}%
\pgfpathlineto{\pgfqpoint{0.681104in}{0.824091in}}%
\pgfpathlineto{\pgfqpoint{0.682214in}{0.826760in}}%
\pgfpathlineto{\pgfqpoint{0.682668in}{0.827847in}}%
\pgfpathlineto{\pgfqpoint{0.683778in}{0.831013in}}%
\pgfpathlineto{\pgfqpoint{0.684239in}{0.832099in}}%
\pgfpathlineto{\pgfqpoint{0.685302in}{0.834831in}}%
\pgfpathlineto{\pgfqpoint{0.685888in}{0.835918in}}%
\pgfpathlineto{\pgfqpoint{0.686991in}{0.839332in}}%
\pgfpathlineto{\pgfqpoint{0.687475in}{0.840419in}}%
\pgfpathlineto{\pgfqpoint{0.688578in}{0.843523in}}%
\pgfpathlineto{\pgfqpoint{0.689039in}{0.844578in}}%
\pgfpathlineto{\pgfqpoint{0.690141in}{0.846844in}}%
\pgfpathlineto{\pgfqpoint{0.690790in}{0.847931in}}%
\pgfpathlineto{\pgfqpoint{0.691900in}{0.850414in}}%
\pgfpathlineto{\pgfqpoint{0.692393in}{0.851439in}}%
\pgfpathlineto{\pgfqpoint{0.693495in}{0.853922in}}%
\pgfpathlineto{\pgfqpoint{0.693792in}{0.854915in}}%
\pgfpathlineto{\pgfqpoint{0.694871in}{0.857926in}}%
\pgfpathlineto{\pgfqpoint{0.695301in}{0.859013in}}%
\pgfpathlineto{\pgfqpoint{0.696395in}{0.861372in}}%
\pgfpathlineto{\pgfqpoint{0.696880in}{0.862459in}}%
\pgfpathlineto{\pgfqpoint{0.697990in}{0.865345in}}%
\pgfpathlineto{\pgfqpoint{0.698475in}{0.866370in}}%
\pgfpathlineto{\pgfqpoint{0.699538in}{0.869381in}}%
\pgfpathlineto{\pgfqpoint{0.700023in}{0.870405in}}%
\pgfpathlineto{\pgfqpoint{0.701133in}{0.873013in}}%
\pgfpathlineto{\pgfqpoint{0.701672in}{0.874037in}}%
\pgfpathlineto{\pgfqpoint{0.702751in}{0.876521in}}%
\pgfpathlineto{\pgfqpoint{0.703181in}{0.877607in}}%
\pgfpathlineto{\pgfqpoint{0.704291in}{0.880494in}}%
\pgfpathlineto{\pgfqpoint{0.704838in}{0.881581in}}%
\pgfpathlineto{\pgfqpoint{0.705941in}{0.884561in}}%
\pgfpathlineto{\pgfqpoint{0.706543in}{0.885585in}}%
\pgfpathlineto{\pgfqpoint{0.707653in}{0.887975in}}%
\pgfpathlineto{\pgfqpoint{0.708075in}{0.889062in}}%
\pgfpathlineto{\pgfqpoint{0.709185in}{0.891762in}}%
\pgfpathlineto{\pgfqpoint{0.709842in}{0.892849in}}%
\pgfpathlineto{\pgfqpoint{0.710928in}{0.895860in}}%
\pgfpathlineto{\pgfqpoint{0.711476in}{0.896946in}}%
\pgfpathlineto{\pgfqpoint{0.712547in}{0.899554in}}%
\pgfpathlineto{\pgfqpoint{0.713297in}{0.900640in}}%
\pgfpathlineto{\pgfqpoint{0.714392in}{0.902844in}}%
\pgfpathlineto{\pgfqpoint{0.714853in}{0.903745in}}%
\pgfpathlineto{\pgfqpoint{0.714853in}{0.903900in}}%
\pgfpathlineto{\pgfqpoint{0.715955in}{0.906290in}}%
\pgfpathlineto{\pgfqpoint{0.716643in}{0.907377in}}%
\pgfpathlineto{\pgfqpoint{0.717753in}{0.909363in}}%
\pgfpathlineto{\pgfqpoint{0.718128in}{0.910450in}}%
\pgfpathlineto{\pgfqpoint{0.719238in}{0.912964in}}%
\pgfpathlineto{\pgfqpoint{0.719723in}{0.914051in}}%
\pgfpathlineto{\pgfqpoint{0.720825in}{0.916161in}}%
\pgfpathlineto{\pgfqpoint{0.721631in}{0.917248in}}%
\pgfpathlineto{\pgfqpoint{0.722725in}{0.919142in}}%
\pgfpathlineto{\pgfqpoint{0.723225in}{0.920228in}}%
\pgfpathlineto{\pgfqpoint{0.724320in}{0.923115in}}%
\pgfpathlineto{\pgfqpoint{0.724875in}{0.924201in}}%
\pgfpathlineto{\pgfqpoint{0.725954in}{0.926312in}}%
\pgfpathlineto{\pgfqpoint{0.726525in}{0.927399in}}%
\pgfpathlineto{\pgfqpoint{0.727627in}{0.929789in}}%
\pgfpathlineto{\pgfqpoint{0.728221in}{0.930875in}}%
\pgfpathlineto{\pgfqpoint{0.729323in}{0.933079in}}%
\pgfpathlineto{\pgfqpoint{0.729824in}{0.934166in}}%
\pgfpathlineto{\pgfqpoint{0.730887in}{0.936184in}}%
\pgfpathlineto{\pgfqpoint{0.731637in}{0.937270in}}%
\pgfpathlineto{\pgfqpoint{0.732732in}{0.940033in}}%
\pgfpathlineto{\pgfqpoint{0.733240in}{0.941119in}}%
\pgfpathlineto{\pgfqpoint{0.734334in}{0.943447in}}%
\pgfpathlineto{\pgfqpoint{0.734975in}{0.944534in}}%
\pgfpathlineto{\pgfqpoint{0.736078in}{0.947017in}}%
\pgfpathlineto{\pgfqpoint{0.736953in}{0.948104in}}%
\pgfpathlineto{\pgfqpoint{0.738040in}{0.949997in}}%
\pgfpathlineto{\pgfqpoint{0.738470in}{0.951084in}}%
\pgfpathlineto{\pgfqpoint{0.739549in}{0.953381in}}%
\pgfpathlineto{\pgfqpoint{0.739572in}{0.953381in}}%
\pgfpathlineto{\pgfqpoint{0.740049in}{0.954467in}}%
\pgfpathlineto{\pgfqpoint{0.741159in}{0.955957in}}%
\pgfpathlineto{\pgfqpoint{0.741706in}{0.957044in}}%
\pgfpathlineto{\pgfqpoint{0.742770in}{0.959031in}}%
\pgfpathlineto{\pgfqpoint{0.742801in}{0.959031in}}%
\pgfpathlineto{\pgfqpoint{0.743340in}{0.960117in}}%
\pgfpathlineto{\pgfqpoint{0.744450in}{0.961980in}}%
\pgfpathlineto{\pgfqpoint{0.745334in}{0.963066in}}%
\pgfpathlineto{\pgfqpoint{0.746413in}{0.965953in}}%
\pgfpathlineto{\pgfqpoint{0.746944in}{0.967040in}}%
\pgfpathlineto{\pgfqpoint{0.748046in}{0.969181in}}%
\pgfpathlineto{\pgfqpoint{0.748688in}{0.970268in}}%
\pgfpathlineto{\pgfqpoint{0.749790in}{0.971882in}}%
\pgfpathlineto{\pgfqpoint{0.750533in}{0.972969in}}%
\pgfpathlineto{\pgfqpoint{0.751627in}{0.974924in}}%
\pgfpathlineto{\pgfqpoint{0.752714in}{0.976011in}}%
\pgfpathlineto{\pgfqpoint{0.753808in}{0.977501in}}%
\pgfpathlineto{\pgfqpoint{0.754535in}{0.978587in}}%
\pgfpathlineto{\pgfqpoint{0.755630in}{0.980481in}}%
\pgfpathlineto{\pgfqpoint{0.756318in}{0.981567in}}%
\pgfpathlineto{\pgfqpoint{0.757412in}{0.983275in}}%
\pgfpathlineto{\pgfqpoint{0.757428in}{0.983275in}}%
\pgfpathlineto{\pgfqpoint{0.758092in}{0.984330in}}%
\pgfpathlineto{\pgfqpoint{0.759179in}{0.985975in}}%
\pgfpathlineto{\pgfqpoint{0.759945in}{0.987062in}}%
\pgfpathlineto{\pgfqpoint{0.761024in}{0.988738in}}%
\pgfpathlineto{\pgfqpoint{0.761884in}{0.989824in}}%
\pgfpathlineto{\pgfqpoint{0.762963in}{0.991749in}}%
\pgfpathlineto{\pgfqpoint{0.763705in}{0.992836in}}%
\pgfpathlineto{\pgfqpoint{0.764815in}{0.994543in}}%
\pgfpathlineto{\pgfqpoint{0.765433in}{0.995629in}}%
\pgfpathlineto{\pgfqpoint{0.766512in}{0.997647in}}%
\pgfpathlineto{\pgfqpoint{0.767270in}{0.998702in}}%
\pgfpathlineto{\pgfqpoint{0.768380in}{1.001062in}}%
\pgfpathlineto{\pgfqpoint{0.768998in}{1.002117in}}%
\pgfpathlineto{\pgfqpoint{0.770069in}{1.004197in}}%
\pgfpathlineto{\pgfqpoint{0.770600in}{1.005221in}}%
\pgfpathlineto{\pgfqpoint{0.771679in}{1.006804in}}%
\pgfpathlineto{\pgfqpoint{0.771695in}{1.006804in}}%
\pgfpathlineto{\pgfqpoint{0.772289in}{1.007891in}}%
\pgfpathlineto{\pgfqpoint{0.773383in}{1.009412in}}%
\pgfpathlineto{\pgfqpoint{0.774032in}{1.010498in}}%
\pgfpathlineto{\pgfqpoint{0.775142in}{1.011895in}}%
\pgfpathlineto{\pgfqpoint{0.775987in}{1.012951in}}%
\pgfpathlineto{\pgfqpoint{0.777097in}{1.014844in}}%
\pgfpathlineto{\pgfqpoint{0.777754in}{1.015931in}}%
\pgfpathlineto{\pgfqpoint{0.778840in}{1.017762in}}%
\pgfpathlineto{\pgfqpoint{0.779599in}{1.018849in}}%
\pgfpathlineto{\pgfqpoint{0.780709in}{1.021053in}}%
\pgfpathlineto{\pgfqpoint{0.781608in}{1.022139in}}%
\pgfpathlineto{\pgfqpoint{0.782640in}{1.023691in}}%
\pgfpathlineto{\pgfqpoint{0.783437in}{1.024778in}}%
\pgfpathlineto{\pgfqpoint{0.784508in}{1.026485in}}%
\pgfpathlineto{\pgfqpoint{0.785274in}{1.027572in}}%
\pgfpathlineto{\pgfqpoint{0.786376in}{1.029434in}}%
\pgfpathlineto{\pgfqpoint{0.786853in}{1.030490in}}%
\pgfpathlineto{\pgfqpoint{0.787948in}{1.032166in}}%
\pgfpathlineto{\pgfqpoint{0.788659in}{1.033252in}}%
\pgfpathlineto{\pgfqpoint{0.789769in}{1.035208in}}%
\pgfpathlineto{\pgfqpoint{0.790434in}{1.036294in}}%
\pgfpathlineto{\pgfqpoint{0.791520in}{1.038343in}}%
\pgfpathlineto{\pgfqpoint{0.791544in}{1.038343in}}%
\pgfpathlineto{\pgfqpoint{0.792068in}{1.039430in}}%
\pgfpathlineto{\pgfqpoint{0.793131in}{1.040951in}}%
\pgfpathlineto{\pgfqpoint{0.793960in}{1.042037in}}%
\pgfpathlineto{\pgfqpoint{0.795070in}{1.043527in}}%
\pgfpathlineto{\pgfqpoint{0.795914in}{1.044583in}}%
\pgfpathlineto{\pgfqpoint{0.797016in}{1.046166in}}%
\pgfpathlineto{\pgfqpoint{0.798236in}{1.047252in}}%
\pgfpathlineto{\pgfqpoint{0.799276in}{1.048711in}}%
\pgfpathlineto{\pgfqpoint{0.799322in}{1.048711in}}%
\pgfpathlineto{\pgfqpoint{0.800120in}{1.049798in}}%
\pgfpathlineto{\pgfqpoint{0.801167in}{1.051567in}}%
\pgfpathlineto{\pgfqpoint{0.801918in}{1.052654in}}%
\pgfpathlineto{\pgfqpoint{0.802997in}{1.054237in}}%
\pgfpathlineto{\pgfqpoint{0.803896in}{1.055323in}}%
\pgfpathlineto{\pgfqpoint{0.804935in}{1.056906in}}%
\pgfpathlineto{\pgfqpoint{0.805670in}{1.057962in}}%
\pgfpathlineto{\pgfqpoint{0.806749in}{1.059390in}}%
\pgfpathlineto{\pgfqpoint{0.808055in}{1.060445in}}%
\pgfpathlineto{\pgfqpoint{0.809165in}{1.061997in}}%
\pgfpathlineto{\pgfqpoint{0.810126in}{1.063084in}}%
\pgfpathlineto{\pgfqpoint{0.811236in}{1.065040in}}%
\pgfpathlineto{\pgfqpoint{0.812057in}{1.066064in}}%
\pgfpathlineto{\pgfqpoint{0.813152in}{1.067368in}}%
\pgfpathlineto{\pgfqpoint{0.814184in}{1.068454in}}%
\pgfpathlineto{\pgfqpoint{0.815270in}{1.070099in}}%
\pgfpathlineto{\pgfqpoint{0.816044in}{1.071155in}}%
\pgfpathlineto{\pgfqpoint{0.817154in}{1.072210in}}%
\pgfpathlineto{\pgfqpoint{0.817999in}{1.073266in}}%
\pgfpathlineto{\pgfqpoint{0.819101in}{1.074973in}}%
\pgfpathlineto{\pgfqpoint{0.819883in}{1.075997in}}%
\pgfpathlineto{\pgfqpoint{0.820962in}{1.078046in}}%
\pgfpathlineto{\pgfqpoint{0.821814in}{1.079102in}}%
\pgfpathlineto{\pgfqpoint{0.822924in}{1.080902in}}%
\pgfpathlineto{\pgfqpoint{0.823909in}{1.081989in}}%
\pgfpathlineto{\pgfqpoint{0.825011in}{1.083323in}}%
\pgfpathlineto{\pgfqpoint{0.825613in}{1.084410in}}%
\pgfpathlineto{\pgfqpoint{0.826692in}{1.086210in}}%
\pgfpathlineto{\pgfqpoint{0.827833in}{1.087297in}}%
\pgfpathlineto{\pgfqpoint{0.828936in}{1.088569in}}%
\pgfpathlineto{\pgfqpoint{0.830030in}{1.089656in}}%
\pgfpathlineto{\pgfqpoint{0.831132in}{1.091332in}}%
\pgfpathlineto{\pgfqpoint{0.832188in}{1.092419in}}%
\pgfpathlineto{\pgfqpoint{0.833235in}{1.094064in}}%
\pgfpathlineto{\pgfqpoint{0.833962in}{1.095119in}}%
\pgfpathlineto{\pgfqpoint{0.835065in}{1.096423in}}%
\pgfpathlineto{\pgfqpoint{0.836034in}{1.097479in}}%
\pgfpathlineto{\pgfqpoint{0.837136in}{1.099341in}}%
\pgfpathlineto{\pgfqpoint{0.838082in}{1.100428in}}%
\pgfpathlineto{\pgfqpoint{0.839192in}{1.102135in}}%
\pgfpathlineto{\pgfqpoint{0.839701in}{1.103221in}}%
\pgfpathlineto{\pgfqpoint{0.840748in}{1.104246in}}%
\pgfpathlineto{\pgfqpoint{0.841639in}{1.105332in}}%
\pgfpathlineto{\pgfqpoint{0.842726in}{1.106729in}}%
\pgfpathlineto{\pgfqpoint{0.843688in}{1.107753in}}%
\pgfpathlineto{\pgfqpoint{0.844782in}{1.108809in}}%
\pgfpathlineto{\pgfqpoint{0.845658in}{1.109895in}}%
\pgfpathlineto{\pgfqpoint{0.846768in}{1.111541in}}%
\pgfpathlineto{\pgfqpoint{0.847534in}{1.112627in}}%
\pgfpathlineto{\pgfqpoint{0.848636in}{1.113900in}}%
\pgfpathlineto{\pgfqpoint{0.849519in}{1.114986in}}%
\pgfpathlineto{\pgfqpoint{0.850630in}{1.116042in}}%
\pgfpathlineto{\pgfqpoint{0.851615in}{1.117128in}}%
\pgfpathlineto{\pgfqpoint{0.852748in}{1.118649in}}%
\pgfpathlineto{\pgfqpoint{0.853796in}{1.119705in}}%
\pgfpathlineto{\pgfqpoint{0.854875in}{1.120822in}}%
\pgfpathlineto{\pgfqpoint{0.855867in}{1.121909in}}%
\pgfpathlineto{\pgfqpoint{0.856962in}{1.123306in}}%
\pgfpathlineto{\pgfqpoint{0.857900in}{1.124392in}}%
\pgfpathlineto{\pgfqpoint{0.859002in}{1.126317in}}%
\pgfpathlineto{\pgfqpoint{0.860339in}{1.127372in}}%
\pgfpathlineto{\pgfqpoint{0.861441in}{1.128583in}}%
\pgfpathlineto{\pgfqpoint{0.862684in}{1.129669in}}%
\pgfpathlineto{\pgfqpoint{0.863771in}{1.130911in}}%
\pgfpathlineto{\pgfqpoint{0.864803in}{1.131966in}}%
\pgfpathlineto{\pgfqpoint{0.865874in}{1.133239in}}%
\pgfpathlineto{\pgfqpoint{0.867187in}{1.134326in}}%
\pgfpathlineto{\pgfqpoint{0.868282in}{1.135785in}}%
\pgfpathlineto{\pgfqpoint{0.869759in}{1.136871in}}%
\pgfpathlineto{\pgfqpoint{0.870869in}{1.138082in}}%
\pgfpathlineto{\pgfqpoint{0.871987in}{1.139137in}}%
\pgfpathlineto{\pgfqpoint{0.873098in}{1.140348in}}%
\pgfpathlineto{\pgfqpoint{0.874200in}{1.141434in}}%
\pgfpathlineto{\pgfqpoint{0.875310in}{1.142521in}}%
\pgfpathlineto{\pgfqpoint{0.876506in}{1.143607in}}%
\pgfpathlineto{\pgfqpoint{0.877616in}{1.144787in}}%
\pgfpathlineto{\pgfqpoint{0.878523in}{1.145842in}}%
\pgfpathlineto{\pgfqpoint{0.879633in}{1.147332in}}%
\pgfpathlineto{\pgfqpoint{0.880696in}{1.148419in}}%
\pgfpathlineto{\pgfqpoint{0.881720in}{1.149226in}}%
\pgfpathlineto{\pgfqpoint{0.883143in}{1.150312in}}%
\pgfpathlineto{\pgfqpoint{0.884253in}{1.151616in}}%
\pgfpathlineto{\pgfqpoint{0.885418in}{1.152702in}}%
\pgfpathlineto{\pgfqpoint{0.886481in}{1.153944in}}%
\pgfpathlineto{\pgfqpoint{0.886520in}{1.153944in}}%
\pgfpathlineto{\pgfqpoint{0.887818in}{1.155031in}}%
\pgfpathlineto{\pgfqpoint{0.888928in}{1.156521in}}%
\pgfpathlineto{\pgfqpoint{0.889929in}{1.157607in}}%
\pgfpathlineto{\pgfqpoint{0.891039in}{1.158600in}}%
\pgfpathlineto{\pgfqpoint{0.891891in}{1.159687in}}%
\pgfpathlineto{\pgfqpoint{0.892962in}{1.160680in}}%
\pgfpathlineto{\pgfqpoint{0.894002in}{1.161767in}}%
\pgfpathlineto{\pgfqpoint{0.895089in}{1.162698in}}%
\pgfpathlineto{\pgfqpoint{0.896034in}{1.163785in}}%
\pgfpathlineto{\pgfqpoint{0.897113in}{1.165119in}}%
\pgfpathlineto{\pgfqpoint{0.898067in}{1.166206in}}%
\pgfpathlineto{\pgfqpoint{0.899130in}{1.167572in}}%
\pgfpathlineto{\pgfqpoint{0.900279in}{1.168658in}}%
\pgfpathlineto{\pgfqpoint{0.901390in}{1.169838in}}%
\pgfpathlineto{\pgfqpoint{0.902672in}{1.170924in}}%
\pgfpathlineto{\pgfqpoint{0.903774in}{1.172042in}}%
\pgfpathlineto{\pgfqpoint{0.904853in}{1.173128in}}%
\pgfpathlineto{\pgfqpoint{0.905900in}{1.174122in}}%
\pgfpathlineto{\pgfqpoint{0.906721in}{1.175208in}}%
\pgfpathlineto{\pgfqpoint{0.907800in}{1.176388in}}%
\pgfpathlineto{\pgfqpoint{0.909090in}{1.177474in}}%
\pgfpathlineto{\pgfqpoint{0.910169in}{1.178064in}}%
\pgfpathlineto{\pgfqpoint{0.911310in}{1.179119in}}%
\pgfpathlineto{\pgfqpoint{0.912412in}{1.180082in}}%
\pgfpathlineto{\pgfqpoint{0.913499in}{1.181106in}}%
\pgfpathlineto{\pgfqpoint{0.914609in}{1.182503in}}%
\pgfpathlineto{\pgfqpoint{0.916079in}{1.183558in}}%
\pgfpathlineto{\pgfqpoint{0.917189in}{1.185017in}}%
\pgfpathlineto{\pgfqpoint{0.918080in}{1.186104in}}%
\pgfpathlineto{\pgfqpoint{0.919175in}{1.187625in}}%
\pgfpathlineto{\pgfqpoint{0.920183in}{1.188711in}}%
\pgfpathlineto{\pgfqpoint{0.921207in}{1.189643in}}%
\pgfpathlineto{\pgfqpoint{0.922450in}{1.190698in}}%
\pgfpathlineto{\pgfqpoint{0.923545in}{1.191722in}}%
\pgfpathlineto{\pgfqpoint{0.924889in}{1.192809in}}%
\pgfpathlineto{\pgfqpoint{0.925992in}{1.194299in}}%
\pgfpathlineto{\pgfqpoint{0.927149in}{1.195354in}}%
\pgfpathlineto{\pgfqpoint{0.928259in}{1.196534in}}%
\pgfpathlineto{\pgfqpoint{0.929486in}{1.197620in}}%
\pgfpathlineto{\pgfqpoint{0.930463in}{1.198428in}}%
\pgfpathlineto{\pgfqpoint{0.931878in}{1.199514in}}%
\pgfpathlineto{\pgfqpoint{0.932942in}{1.200414in}}%
\pgfpathlineto{\pgfqpoint{0.932981in}{1.200414in}}%
\pgfpathlineto{\pgfqpoint{0.934646in}{1.201439in}}%
\pgfpathlineto{\pgfqpoint{0.935646in}{1.202370in}}%
\pgfpathlineto{\pgfqpoint{0.937194in}{1.203456in}}%
\pgfpathlineto{\pgfqpoint{0.938289in}{1.204605in}}%
\pgfpathlineto{\pgfqpoint{0.939375in}{1.205691in}}%
\pgfpathlineto{\pgfqpoint{0.940478in}{1.206654in}}%
\pgfpathlineto{\pgfqpoint{0.941588in}{1.207709in}}%
\pgfpathlineto{\pgfqpoint{0.942682in}{1.208858in}}%
\pgfpathlineto{\pgfqpoint{0.944347in}{1.209944in}}%
\pgfpathlineto{\pgfqpoint{0.945426in}{1.210875in}}%
\pgfpathlineto{\pgfqpoint{0.946372in}{1.211962in}}%
\pgfpathlineto{\pgfqpoint{0.947467in}{1.213048in}}%
\pgfpathlineto{\pgfqpoint{0.948600in}{1.214135in}}%
\pgfpathlineto{\pgfqpoint{0.949695in}{1.215190in}}%
\pgfpathlineto{\pgfqpoint{0.950891in}{1.216277in}}%
\pgfpathlineto{\pgfqpoint{0.952001in}{1.217487in}}%
\pgfpathlineto{\pgfqpoint{0.953588in}{1.218574in}}%
\pgfpathlineto{\pgfqpoint{0.954659in}{1.219505in}}%
\pgfpathlineto{\pgfqpoint{0.954690in}{1.219505in}}%
\pgfpathlineto{\pgfqpoint{0.955417in}{1.220561in}}%
\pgfpathlineto{\pgfqpoint{0.956520in}{1.222020in}}%
\pgfpathlineto{\pgfqpoint{0.957778in}{1.223106in}}%
\pgfpathlineto{\pgfqpoint{0.958865in}{1.224037in}}%
\pgfpathlineto{\pgfqpoint{0.960014in}{1.225093in}}%
\pgfpathlineto{\pgfqpoint{0.961069in}{1.226086in}}%
\pgfpathlineto{\pgfqpoint{0.962461in}{1.227173in}}%
\pgfpathlineto{\pgfqpoint{0.963548in}{1.228073in}}%
\pgfpathlineto{\pgfqpoint{0.965111in}{1.229128in}}%
\pgfpathlineto{\pgfqpoint{0.966198in}{1.230153in}}%
\pgfpathlineto{\pgfqpoint{0.967347in}{1.231239in}}%
\pgfpathlineto{\pgfqpoint{0.968387in}{1.232294in}}%
\pgfpathlineto{\pgfqpoint{0.969880in}{1.233381in}}%
\pgfpathlineto{\pgfqpoint{0.970935in}{1.234467in}}%
\pgfpathlineto{\pgfqpoint{0.971952in}{1.235554in}}%
\pgfpathlineto{\pgfqpoint{0.973062in}{1.236392in}}%
\pgfpathlineto{\pgfqpoint{0.974438in}{1.237479in}}%
\pgfpathlineto{\pgfqpoint{0.975430in}{1.238255in}}%
\pgfpathlineto{\pgfqpoint{0.976908in}{1.239341in}}%
\pgfpathlineto{\pgfqpoint{0.978018in}{1.240396in}}%
\pgfpathlineto{\pgfqpoint{0.979644in}{1.241483in}}%
\pgfpathlineto{\pgfqpoint{0.980746in}{1.242694in}}%
\pgfpathlineto{\pgfqpoint{0.982193in}{1.243780in}}%
\pgfpathlineto{\pgfqpoint{0.983295in}{1.244960in}}%
\pgfpathlineto{\pgfqpoint{0.984444in}{1.246046in}}%
\pgfpathlineto{\pgfqpoint{0.985546in}{1.246946in}}%
\pgfpathlineto{\pgfqpoint{0.987352in}{1.248002in}}%
\pgfpathlineto{\pgfqpoint{0.988392in}{1.249026in}}%
\pgfpathlineto{\pgfqpoint{0.989971in}{1.250113in}}%
\pgfpathlineto{\pgfqpoint{0.991066in}{1.250982in}}%
\pgfpathlineto{\pgfqpoint{0.992184in}{1.252068in}}%
\pgfpathlineto{\pgfqpoint{0.993098in}{1.252751in}}%
\pgfpathlineto{\pgfqpoint{0.993176in}{1.252751in}}%
\pgfpathlineto{\pgfqpoint{0.994889in}{1.253838in}}%
\pgfpathlineto{\pgfqpoint{0.995967in}{1.254645in}}%
\pgfpathlineto{\pgfqpoint{0.997203in}{1.255731in}}%
\pgfpathlineto{\pgfqpoint{0.998258in}{1.256414in}}%
\pgfpathlineto{\pgfqpoint{0.999407in}{1.257501in}}%
\pgfpathlineto{\pgfqpoint{1.000455in}{1.258401in}}%
\pgfpathlineto{\pgfqpoint{1.000509in}{1.258401in}}%
\pgfpathlineto{\pgfqpoint{1.002143in}{1.259456in}}%
\pgfpathlineto{\pgfqpoint{1.003214in}{1.260481in}}%
\pgfpathlineto{\pgfqpoint{1.005020in}{1.261536in}}%
\pgfpathlineto{\pgfqpoint{1.006130in}{1.262498in}}%
\pgfpathlineto{\pgfqpoint{1.007451in}{1.263585in}}%
\pgfpathlineto{\pgfqpoint{1.008554in}{1.264392in}}%
\pgfpathlineto{\pgfqpoint{1.010188in}{1.265447in}}%
\pgfpathlineto{\pgfqpoint{1.011274in}{1.266317in}}%
\pgfpathlineto{\pgfqpoint{1.012861in}{1.267403in}}%
\pgfpathlineto{\pgfqpoint{1.013948in}{1.268490in}}%
\pgfpathlineto{\pgfqpoint{1.015379in}{1.269576in}}%
\pgfpathlineto{\pgfqpoint{1.016489in}{1.271035in}}%
\pgfpathlineto{\pgfqpoint{1.017505in}{1.272122in}}%
\pgfpathlineto{\pgfqpoint{1.018615in}{1.273208in}}%
\pgfpathlineto{\pgfqpoint{1.020116in}{1.274263in}}%
\pgfpathlineto{\pgfqpoint{1.021171in}{1.274915in}}%
\pgfpathlineto{\pgfqpoint{1.022719in}{1.276002in}}%
\pgfpathlineto{\pgfqpoint{1.023822in}{1.277306in}}%
\pgfpathlineto{\pgfqpoint{1.025096in}{1.278392in}}%
\pgfpathlineto{\pgfqpoint{1.026206in}{1.279572in}}%
\pgfpathlineto{\pgfqpoint{1.027472in}{1.280658in}}%
\pgfpathlineto{\pgfqpoint{1.028567in}{1.281745in}}%
\pgfpathlineto{\pgfqpoint{1.029974in}{1.282831in}}%
\pgfpathlineto{\pgfqpoint{1.031084in}{1.283980in}}%
\pgfpathlineto{\pgfqpoint{1.032812in}{1.285066in}}%
\pgfpathlineto{\pgfqpoint{1.033867in}{1.286401in}}%
\pgfpathlineto{\pgfqpoint{1.035384in}{1.287487in}}%
\pgfpathlineto{\pgfqpoint{1.036478in}{1.288388in}}%
\pgfpathlineto{\pgfqpoint{1.037901in}{1.289474in}}%
\pgfpathlineto{\pgfqpoint{1.038925in}{1.290095in}}%
\pgfpathlineto{\pgfqpoint{1.040137in}{1.291181in}}%
\pgfpathlineto{\pgfqpoint{1.041153in}{1.291833in}}%
\pgfpathlineto{\pgfqpoint{1.042686in}{1.292920in}}%
\pgfpathlineto{\pgfqpoint{1.043780in}{1.294161in}}%
\pgfpathlineto{\pgfqpoint{1.044953in}{1.295217in}}%
\pgfpathlineto{\pgfqpoint{1.045985in}{1.296086in}}%
\pgfpathlineto{\pgfqpoint{1.046008in}{1.296086in}}%
\pgfpathlineto{\pgfqpoint{1.047462in}{1.297142in}}%
\pgfpathlineto{\pgfqpoint{1.048400in}{1.297762in}}%
\pgfpathlineto{\pgfqpoint{1.050613in}{1.298849in}}%
\pgfpathlineto{\pgfqpoint{1.051684in}{1.299842in}}%
\pgfpathlineto{\pgfqpoint{1.053380in}{1.300929in}}%
\pgfpathlineto{\pgfqpoint{1.054412in}{1.301674in}}%
\pgfpathlineto{\pgfqpoint{1.055796in}{1.302760in}}%
\pgfpathlineto{\pgfqpoint{1.056804in}{1.303474in}}%
\pgfpathlineto{\pgfqpoint{1.058282in}{1.304561in}}%
\pgfpathlineto{\pgfqpoint{1.059314in}{1.305554in}}%
\pgfpathlineto{\pgfqpoint{1.061143in}{1.306640in}}%
\pgfpathlineto{\pgfqpoint{1.062245in}{1.307354in}}%
\pgfpathlineto{\pgfqpoint{1.064012in}{1.308441in}}%
\pgfpathlineto{\pgfqpoint{1.065107in}{1.309062in}}%
\pgfpathlineto{\pgfqpoint{1.066350in}{1.310117in}}%
\pgfpathlineto{\pgfqpoint{1.067421in}{1.311079in}}%
\pgfpathlineto{\pgfqpoint{1.068718in}{1.312166in}}%
\pgfpathlineto{\pgfqpoint{1.069797in}{1.313035in}}%
\pgfpathlineto{\pgfqpoint{1.071165in}{1.314122in}}%
\pgfpathlineto{\pgfqpoint{1.072197in}{1.314804in}}%
\pgfpathlineto{\pgfqpoint{1.073823in}{1.315891in}}%
\pgfpathlineto{\pgfqpoint{1.074910in}{1.316760in}}%
\pgfpathlineto{\pgfqpoint{1.076872in}{1.317816in}}%
\pgfpathlineto{\pgfqpoint{1.077982in}{1.318871in}}%
\pgfpathlineto{\pgfqpoint{1.079585in}{1.319957in}}%
\pgfpathlineto{\pgfqpoint{1.080625in}{1.320671in}}%
\pgfpathlineto{\pgfqpoint{1.080679in}{1.320671in}}%
\pgfpathlineto{\pgfqpoint{1.082133in}{1.321758in}}%
\pgfpathlineto{\pgfqpoint{1.083204in}{1.322627in}}%
\pgfpathlineto{\pgfqpoint{1.084823in}{1.323714in}}%
\pgfpathlineto{\pgfqpoint{1.085909in}{1.324490in}}%
\pgfpathlineto{\pgfqpoint{1.088427in}{1.325576in}}%
\pgfpathlineto{\pgfqpoint{1.089380in}{1.326197in}}%
\pgfpathlineto{\pgfqpoint{1.091468in}{1.327283in}}%
\pgfpathlineto{\pgfqpoint{1.092570in}{1.328277in}}%
\pgfpathlineto{\pgfqpoint{1.094454in}{1.329363in}}%
\pgfpathlineto{\pgfqpoint{1.095564in}{1.330015in}}%
\pgfpathlineto{\pgfqpoint{1.097135in}{1.331102in}}%
\pgfpathlineto{\pgfqpoint{1.098222in}{1.332033in}}%
\pgfpathlineto{\pgfqpoint{1.099762in}{1.333119in}}%
\pgfpathlineto{\pgfqpoint{1.100810in}{1.333926in}}%
\pgfpathlineto{\pgfqpoint{1.102663in}{1.334982in}}%
\pgfpathlineto{\pgfqpoint{1.103757in}{1.335851in}}%
\pgfpathlineto{\pgfqpoint{1.105352in}{1.336938in}}%
\pgfpathlineto{\pgfqpoint{1.106462in}{1.337496in}}%
\pgfpathlineto{\pgfqpoint{1.107642in}{1.338583in}}%
\pgfpathlineto{\pgfqpoint{1.108667in}{1.339576in}}%
\pgfpathlineto{\pgfqpoint{1.108690in}{1.339576in}}%
\pgfpathlineto{\pgfqpoint{1.110644in}{1.340663in}}%
\pgfpathlineto{\pgfqpoint{1.111747in}{1.341221in}}%
\pgfpathlineto{\pgfqpoint{1.113138in}{1.342308in}}%
\pgfpathlineto{\pgfqpoint{1.114233in}{1.342867in}}%
\pgfpathlineto{\pgfqpoint{1.115765in}{1.343953in}}%
\pgfpathlineto{\pgfqpoint{1.116805in}{1.344698in}}%
\pgfpathlineto{\pgfqpoint{1.118321in}{1.345753in}}%
\pgfpathlineto{\pgfqpoint{1.119338in}{1.346685in}}%
\pgfpathlineto{\pgfqpoint{1.122535in}{1.347771in}}%
\pgfpathlineto{\pgfqpoint{1.123622in}{1.348765in}}%
\pgfpathlineto{\pgfqpoint{1.125529in}{1.349820in}}%
\pgfpathlineto{\pgfqpoint{1.126553in}{1.350503in}}%
\pgfpathlineto{\pgfqpoint{1.128484in}{1.351589in}}%
\pgfpathlineto{\pgfqpoint{1.129469in}{1.352148in}}%
\pgfpathlineto{\pgfqpoint{1.132377in}{1.353235in}}%
\pgfpathlineto{\pgfqpoint{1.133441in}{1.354073in}}%
\pgfpathlineto{\pgfqpoint{1.135418in}{1.355159in}}%
\pgfpathlineto{\pgfqpoint{1.136466in}{1.355594in}}%
\pgfpathlineto{\pgfqpoint{1.138413in}{1.356680in}}%
\pgfpathlineto{\pgfqpoint{1.139515in}{1.357456in}}%
\pgfpathlineto{\pgfqpoint{1.141790in}{1.358543in}}%
\pgfpathlineto{\pgfqpoint{1.142869in}{1.359288in}}%
\pgfpathlineto{\pgfqpoint{1.144581in}{1.360374in}}%
\pgfpathlineto{\pgfqpoint{1.145660in}{1.361088in}}%
\pgfpathlineto{\pgfqpoint{1.147403in}{1.362144in}}%
\pgfpathlineto{\pgfqpoint{1.148458in}{1.363075in}}%
\pgfpathlineto{\pgfqpoint{1.148474in}{1.363075in}}%
\pgfpathlineto{\pgfqpoint{1.149647in}{1.364161in}}%
\pgfpathlineto{\pgfqpoint{1.150655in}{1.364813in}}%
\pgfpathlineto{\pgfqpoint{1.152508in}{1.365900in}}%
\pgfpathlineto{\pgfqpoint{1.153579in}{1.366614in}}%
\pgfpathlineto{\pgfqpoint{1.155103in}{1.367669in}}%
\pgfpathlineto{\pgfqpoint{1.156057in}{1.368259in}}%
\pgfpathlineto{\pgfqpoint{1.156143in}{1.368259in}}%
\pgfpathlineto{\pgfqpoint{1.157363in}{1.369345in}}%
\pgfpathlineto{\pgfqpoint{1.158426in}{1.369966in}}%
\pgfpathlineto{\pgfqpoint{1.158449in}{1.369966in}}%
\pgfpathlineto{\pgfqpoint{1.160239in}{1.371053in}}%
\pgfpathlineto{\pgfqpoint{1.161326in}{1.371798in}}%
\pgfpathlineto{\pgfqpoint{1.163023in}{1.372853in}}%
\pgfpathlineto{\pgfqpoint{1.164047in}{1.373722in}}%
\pgfpathlineto{\pgfqpoint{1.166048in}{1.374809in}}%
\pgfpathlineto{\pgfqpoint{1.167135in}{1.375461in}}%
\pgfpathlineto{\pgfqpoint{1.169480in}{1.376547in}}%
\pgfpathlineto{\pgfqpoint{1.170528in}{1.377230in}}%
\pgfpathlineto{\pgfqpoint{1.173272in}{1.378317in}}%
\pgfpathlineto{\pgfqpoint{1.174335in}{1.378969in}}%
\pgfpathlineto{\pgfqpoint{1.176375in}{1.380055in}}%
\pgfpathlineto{\pgfqpoint{1.177462in}{1.380893in}}%
\pgfpathlineto{\pgfqpoint{1.180221in}{1.381980in}}%
\pgfpathlineto{\pgfqpoint{1.181316in}{1.382476in}}%
\pgfpathlineto{\pgfqpoint{1.183497in}{1.383563in}}%
\pgfpathlineto{\pgfqpoint{1.184591in}{1.384091in}}%
\pgfpathlineto{\pgfqpoint{1.184599in}{1.384091in}}%
\pgfpathlineto{\pgfqpoint{1.186929in}{1.385177in}}%
\pgfpathlineto{\pgfqpoint{1.188039in}{1.385984in}}%
\pgfpathlineto{\pgfqpoint{1.190408in}{1.387071in}}%
\pgfpathlineto{\pgfqpoint{1.191494in}{1.387722in}}%
\pgfpathlineto{\pgfqpoint{1.194106in}{1.388809in}}%
\pgfpathlineto{\pgfqpoint{1.195177in}{1.389523in}}%
\pgfpathlineto{\pgfqpoint{1.197420in}{1.390609in}}%
\pgfpathlineto{\pgfqpoint{1.198436in}{1.391416in}}%
\pgfpathlineto{\pgfqpoint{1.200485in}{1.392472in}}%
\pgfpathlineto{\pgfqpoint{1.201587in}{1.393341in}}%
\pgfpathlineto{\pgfqpoint{1.203752in}{1.394428in}}%
\pgfpathlineto{\pgfqpoint{1.204777in}{1.394893in}}%
\pgfpathlineto{\pgfqpoint{1.206942in}{1.395980in}}%
\pgfpathlineto{\pgfqpoint{1.208044in}{1.396818in}}%
\pgfpathlineto{\pgfqpoint{1.210405in}{1.397904in}}%
\pgfpathlineto{\pgfqpoint{1.211422in}{1.398339in}}%
\pgfpathlineto{\pgfqpoint{1.213869in}{1.399394in}}%
\pgfpathlineto{\pgfqpoint{1.214971in}{1.400077in}}%
\pgfpathlineto{\pgfqpoint{1.217183in}{1.401164in}}%
\pgfpathlineto{\pgfqpoint{1.218285in}{1.401722in}}%
\pgfpathlineto{\pgfqpoint{1.220482in}{1.402809in}}%
\pgfpathlineto{\pgfqpoint{1.221444in}{1.403244in}}%
\pgfpathlineto{\pgfqpoint{1.223703in}{1.404330in}}%
\pgfpathlineto{\pgfqpoint{1.224329in}{1.404796in}}%
\pgfpathlineto{\pgfqpoint{1.224563in}{1.404796in}}%
\pgfpathlineto{\pgfqpoint{1.227800in}{1.405882in}}%
\pgfpathlineto{\pgfqpoint{1.228886in}{1.406534in}}%
\pgfpathlineto{\pgfqpoint{1.231357in}{1.407620in}}%
\pgfpathlineto{\pgfqpoint{1.232459in}{1.408241in}}%
\pgfpathlineto{\pgfqpoint{1.235117in}{1.409328in}}%
\pgfpathlineto{\pgfqpoint{1.236102in}{1.410011in}}%
\pgfpathlineto{\pgfqpoint{1.238385in}{1.411097in}}%
\pgfpathlineto{\pgfqpoint{1.239487in}{1.411532in}}%
\pgfpathlineto{\pgfqpoint{1.242028in}{1.412618in}}%
\pgfpathlineto{\pgfqpoint{1.243091in}{1.413301in}}%
\pgfpathlineto{\pgfqpoint{1.244826in}{1.414388in}}%
\pgfpathlineto{\pgfqpoint{1.245929in}{1.415102in}}%
\pgfpathlineto{\pgfqpoint{1.248032in}{1.416188in}}%
\pgfpathlineto{\pgfqpoint{1.249142in}{1.416871in}}%
\pgfpathlineto{\pgfqpoint{1.251088in}{1.417957in}}%
\pgfpathlineto{\pgfqpoint{1.252183in}{1.418454in}}%
\pgfpathlineto{\pgfqpoint{1.254810in}{1.419541in}}%
\pgfpathlineto{\pgfqpoint{1.255912in}{1.420068in}}%
\pgfpathlineto{\pgfqpoint{1.257593in}{1.421155in}}%
\pgfpathlineto{\pgfqpoint{1.258648in}{1.421714in}}%
\pgfpathlineto{\pgfqpoint{1.261064in}{1.422800in}}%
\pgfpathlineto{\pgfqpoint{1.262119in}{1.423204in}}%
\pgfpathlineto{\pgfqpoint{1.264988in}{1.424290in}}%
\pgfpathlineto{\pgfqpoint{1.266059in}{1.424756in}}%
\pgfpathlineto{\pgfqpoint{1.268529in}{1.425842in}}%
\pgfpathlineto{\pgfqpoint{1.269624in}{1.426401in}}%
\pgfpathlineto{\pgfqpoint{1.271649in}{1.427487in}}%
\pgfpathlineto{\pgfqpoint{1.272509in}{1.427891in}}%
\pgfpathlineto{\pgfqpoint{1.275112in}{1.428977in}}%
\pgfpathlineto{\pgfqpoint{1.276183in}{1.429722in}}%
\pgfpathlineto{\pgfqpoint{1.278497in}{1.430809in}}%
\pgfpathlineto{\pgfqpoint{1.279584in}{1.431368in}}%
\pgfpathlineto{\pgfqpoint{1.282070in}{1.432454in}}%
\pgfpathlineto{\pgfqpoint{1.283141in}{1.432951in}}%
\pgfpathlineto{\pgfqpoint{1.284603in}{1.434037in}}%
\pgfpathlineto{\pgfqpoint{1.285681in}{1.434813in}}%
\pgfpathlineto{\pgfqpoint{1.288550in}{1.435900in}}%
\pgfpathlineto{\pgfqpoint{1.289410in}{1.436272in}}%
\pgfpathlineto{\pgfqpoint{1.291466in}{1.437359in}}%
\pgfpathlineto{\pgfqpoint{1.292514in}{1.437855in}}%
\pgfpathlineto{\pgfqpoint{1.295203in}{1.438942in}}%
\pgfpathlineto{\pgfqpoint{1.296251in}{1.439314in}}%
\pgfpathlineto{\pgfqpoint{1.298987in}{1.440401in}}%
\pgfpathlineto{\pgfqpoint{1.299988in}{1.440960in}}%
\pgfpathlineto{\pgfqpoint{1.302216in}{1.442046in}}%
\pgfpathlineto{\pgfqpoint{1.303326in}{1.442543in}}%
\pgfpathlineto{\pgfqpoint{1.306164in}{1.443629in}}%
\pgfpathlineto{\pgfqpoint{1.307227in}{1.444064in}}%
\pgfpathlineto{\pgfqpoint{1.310878in}{1.445150in}}%
\pgfpathlineto{\pgfqpoint{1.311784in}{1.445492in}}%
\pgfpathlineto{\pgfqpoint{1.315084in}{1.446578in}}%
\pgfpathlineto{\pgfqpoint{1.316061in}{1.447137in}}%
\pgfpathlineto{\pgfqpoint{1.317820in}{1.448224in}}%
\pgfpathlineto{\pgfqpoint{1.318883in}{1.448938in}}%
\pgfpathlineto{\pgfqpoint{1.320751in}{1.450024in}}%
\pgfpathlineto{\pgfqpoint{1.321854in}{1.450334in}}%
\pgfpathlineto{\pgfqpoint{1.324332in}{1.451421in}}%
\pgfpathlineto{\pgfqpoint{1.325411in}{1.452011in}}%
\pgfpathlineto{\pgfqpoint{1.327850in}{1.453097in}}%
\pgfpathlineto{\pgfqpoint{1.328788in}{1.453563in}}%
\pgfpathlineto{\pgfqpoint{1.331813in}{1.454649in}}%
\pgfpathlineto{\pgfqpoint{1.332837in}{1.455053in}}%
\pgfpathlineto{\pgfqpoint{1.336043in}{1.456139in}}%
\pgfpathlineto{\pgfqpoint{1.337106in}{1.457040in}}%
\pgfpathlineto{\pgfqpoint{1.339240in}{1.458126in}}%
\pgfpathlineto{\pgfqpoint{1.340295in}{1.458716in}}%
\pgfpathlineto{\pgfqpoint{1.340311in}{1.458716in}}%
\pgfpathlineto{\pgfqpoint{1.342086in}{1.459802in}}%
\pgfpathlineto{\pgfqpoint{1.343188in}{1.460299in}}%
\pgfpathlineto{\pgfqpoint{1.345776in}{1.461385in}}%
\pgfpathlineto{\pgfqpoint{1.346800in}{1.461944in}}%
\pgfpathlineto{\pgfqpoint{1.350513in}{1.463031in}}%
\pgfpathlineto{\pgfqpoint{1.351576in}{1.463496in}}%
\pgfpathlineto{\pgfqpoint{1.354054in}{1.464552in}}%
\pgfpathlineto{\pgfqpoint{1.355094in}{1.465017in}}%
\pgfpathlineto{\pgfqpoint{1.358401in}{1.466104in}}%
\pgfpathlineto{\pgfqpoint{1.359308in}{1.466663in}}%
\pgfpathlineto{\pgfqpoint{1.359355in}{1.466663in}}%
\pgfpathlineto{\pgfqpoint{1.363068in}{1.467749in}}%
\pgfpathlineto{\pgfqpoint{1.363975in}{1.468153in}}%
\pgfpathlineto{\pgfqpoint{1.364085in}{1.468153in}}%
\pgfpathlineto{\pgfqpoint{1.366453in}{1.469239in}}%
\pgfpathlineto{\pgfqpoint{1.367438in}{1.469829in}}%
\pgfpathlineto{\pgfqpoint{1.370518in}{1.470915in}}%
\pgfpathlineto{\pgfqpoint{1.371402in}{1.471505in}}%
\pgfpathlineto{\pgfqpoint{1.371621in}{1.471505in}}%
\pgfpathlineto{\pgfqpoint{1.374107in}{1.472592in}}%
\pgfpathlineto{\pgfqpoint{1.375092in}{1.473026in}}%
\pgfpathlineto{\pgfqpoint{1.375107in}{1.473026in}}%
\pgfpathlineto{\pgfqpoint{1.377148in}{1.474113in}}%
\pgfpathlineto{\pgfqpoint{1.378172in}{1.474702in}}%
\pgfpathlineto{\pgfqpoint{1.380822in}{1.475789in}}%
\pgfpathlineto{\pgfqpoint{1.381877in}{1.476255in}}%
\pgfpathlineto{\pgfqpoint{1.384754in}{1.477341in}}%
\pgfpathlineto{\pgfqpoint{1.385841in}{1.477776in}}%
\pgfpathlineto{\pgfqpoint{1.385857in}{1.477776in}}%
\pgfpathlineto{\pgfqpoint{1.390047in}{1.478862in}}%
\pgfpathlineto{\pgfqpoint{1.391016in}{1.479142in}}%
\pgfpathlineto{\pgfqpoint{1.391141in}{1.479142in}}%
\pgfpathlineto{\pgfqpoint{1.394902in}{1.480228in}}%
\pgfpathlineto{\pgfqpoint{1.395965in}{1.480632in}}%
\pgfpathlineto{\pgfqpoint{1.399092in}{1.481718in}}%
\pgfpathlineto{\pgfqpoint{1.400179in}{1.482122in}}%
\pgfpathlineto{\pgfqpoint{1.400194in}{1.482122in}}%
\pgfpathlineto{\pgfqpoint{1.403251in}{1.483208in}}%
\pgfpathlineto{\pgfqpoint{1.404361in}{1.483767in}}%
\pgfpathlineto{\pgfqpoint{1.406988in}{1.484853in}}%
\pgfpathlineto{\pgfqpoint{1.408012in}{1.485474in}}%
\pgfpathlineto{\pgfqpoint{1.411741in}{1.486561in}}%
\pgfpathlineto{\pgfqpoint{1.412804in}{1.486933in}}%
\pgfpathlineto{\pgfqpoint{1.415579in}{1.488020in}}%
\pgfpathlineto{\pgfqpoint{1.416682in}{1.488578in}}%
\pgfpathlineto{\pgfqpoint{1.420544in}{1.489665in}}%
\pgfpathlineto{\pgfqpoint{1.421591in}{1.490130in}}%
\pgfpathlineto{\pgfqpoint{1.424468in}{1.491217in}}%
\pgfpathlineto{\pgfqpoint{1.425461in}{1.491714in}}%
\pgfpathlineto{\pgfqpoint{1.425531in}{1.491714in}}%
\pgfpathlineto{\pgfqpoint{1.428353in}{1.492800in}}%
\pgfpathlineto{\pgfqpoint{1.429370in}{1.493173in}}%
\pgfpathlineto{\pgfqpoint{1.429463in}{1.493173in}}%
\pgfpathlineto{\pgfqpoint{1.432442in}{1.494259in}}%
\pgfpathlineto{\pgfqpoint{1.433380in}{1.494600in}}%
\pgfpathlineto{\pgfqpoint{1.433466in}{1.494600in}}%
\pgfpathlineto{\pgfqpoint{1.437179in}{1.495687in}}%
\pgfpathlineto{\pgfqpoint{1.438282in}{1.496059in}}%
\pgfpathlineto{\pgfqpoint{1.441213in}{1.497146in}}%
\pgfpathlineto{\pgfqpoint{1.442300in}{1.497581in}}%
\pgfpathlineto{\pgfqpoint{1.445286in}{1.498667in}}%
\pgfpathlineto{\pgfqpoint{1.446310in}{1.499226in}}%
\pgfpathlineto{\pgfqpoint{1.449281in}{1.500312in}}%
\pgfpathlineto{\pgfqpoint{1.450368in}{1.500902in}}%
\pgfpathlineto{\pgfqpoint{1.453354in}{1.501989in}}%
\pgfpathlineto{\pgfqpoint{1.454433in}{1.502330in}}%
\pgfpathlineto{\pgfqpoint{1.458561in}{1.503385in}}%
\pgfpathlineto{\pgfqpoint{1.459632in}{1.503789in}}%
\pgfpathlineto{\pgfqpoint{1.462923in}{1.504875in}}%
\pgfpathlineto{\pgfqpoint{1.463822in}{1.505124in}}%
\pgfpathlineto{\pgfqpoint{1.463900in}{1.505124in}}%
\pgfpathlineto{\pgfqpoint{1.467301in}{1.506210in}}%
\pgfpathlineto{\pgfqpoint{1.468395in}{1.506645in}}%
\pgfpathlineto{\pgfqpoint{1.470874in}{1.507731in}}%
\pgfpathlineto{\pgfqpoint{1.471819in}{1.508042in}}%
\pgfpathlineto{\pgfqpoint{1.471945in}{1.508042in}}%
\pgfpathlineto{\pgfqpoint{1.474915in}{1.509097in}}%
\pgfpathlineto{\pgfqpoint{1.476018in}{1.509408in}}%
\pgfpathlineto{\pgfqpoint{1.476025in}{1.509408in}}%
\pgfpathlineto{\pgfqpoint{1.478425in}{1.510494in}}%
\pgfpathlineto{\pgfqpoint{1.479629in}{1.510898in}}%
\pgfpathlineto{\pgfqpoint{1.484038in}{1.511984in}}%
\pgfpathlineto{\pgfqpoint{1.485070in}{1.512481in}}%
\pgfpathlineto{\pgfqpoint{1.488815in}{1.513567in}}%
\pgfpathlineto{\pgfqpoint{1.489823in}{1.514033in}}%
\pgfpathlineto{\pgfqpoint{1.492786in}{1.515119in}}%
\pgfpathlineto{\pgfqpoint{1.493826in}{1.515399in}}%
\pgfpathlineto{\pgfqpoint{1.497696in}{1.516485in}}%
\pgfpathlineto{\pgfqpoint{1.498790in}{1.516796in}}%
\pgfpathlineto{\pgfqpoint{1.501761in}{1.517820in}}%
\pgfpathlineto{\pgfqpoint{1.501761in}{1.517851in}}%
\pgfpathlineto{\pgfqpoint{1.502801in}{1.518286in}}%
\pgfpathlineto{\pgfqpoint{1.507366in}{1.519372in}}%
\pgfpathlineto{\pgfqpoint{1.508429in}{1.519807in}}%
\pgfpathlineto{\pgfqpoint{1.511729in}{1.520893in}}%
\pgfpathlineto{\pgfqpoint{1.512768in}{1.521328in}}%
\pgfpathlineto{\pgfqpoint{1.516411in}{1.522383in}}%
\pgfpathlineto{\pgfqpoint{1.517467in}{1.522694in}}%
\pgfpathlineto{\pgfqpoint{1.521196in}{1.523749in}}%
\pgfpathlineto{\pgfqpoint{1.522228in}{1.524122in}}%
\pgfpathlineto{\pgfqpoint{1.527309in}{1.525208in}}%
\pgfpathlineto{\pgfqpoint{1.528396in}{1.525612in}}%
\pgfpathlineto{\pgfqpoint{1.532265in}{1.526698in}}%
\pgfpathlineto{\pgfqpoint{1.533250in}{1.526977in}}%
\pgfpathlineto{\pgfqpoint{1.536847in}{1.528064in}}%
\pgfpathlineto{\pgfqpoint{1.537957in}{1.528281in}}%
\pgfpathlineto{\pgfqpoint{1.541350in}{1.529337in}}%
\pgfpathlineto{\pgfqpoint{1.542374in}{1.529678in}}%
\pgfpathlineto{\pgfqpoint{1.546095in}{1.530765in}}%
\pgfpathlineto{\pgfqpoint{1.546994in}{1.531137in}}%
\pgfpathlineto{\pgfqpoint{1.551747in}{1.532224in}}%
\pgfpathlineto{\pgfqpoint{1.552787in}{1.532410in}}%
\pgfpathlineto{\pgfqpoint{1.556883in}{1.533496in}}%
\pgfpathlineto{\pgfqpoint{1.557962in}{1.533931in}}%
\pgfpathlineto{\pgfqpoint{1.557978in}{1.533931in}}%
\pgfpathlineto{\pgfqpoint{1.561902in}{1.535017in}}%
\pgfpathlineto{\pgfqpoint{1.562911in}{1.535452in}}%
\pgfpathlineto{\pgfqpoint{1.567234in}{1.536538in}}%
\pgfpathlineto{\pgfqpoint{1.568297in}{1.536911in}}%
\pgfpathlineto{\pgfqpoint{1.573019in}{1.537966in}}%
\pgfpathlineto{\pgfqpoint{1.574090in}{1.538308in}}%
\pgfpathlineto{\pgfqpoint{1.578655in}{1.539394in}}%
\pgfpathlineto{\pgfqpoint{1.579601in}{1.539643in}}%
\pgfpathlineto{\pgfqpoint{1.583651in}{1.540729in}}%
\pgfpathlineto{\pgfqpoint{1.584737in}{1.541008in}}%
\pgfpathlineto{\pgfqpoint{1.588779in}{1.542095in}}%
\pgfpathlineto{\pgfqpoint{1.589874in}{1.542312in}}%
\pgfpathlineto{\pgfqpoint{1.593048in}{1.543399in}}%
\pgfpathlineto{\pgfqpoint{1.594119in}{1.543492in}}%
\pgfpathlineto{\pgfqpoint{1.594134in}{1.543492in}}%
\pgfpathlineto{\pgfqpoint{1.597996in}{1.544547in}}%
\pgfpathlineto{\pgfqpoint{1.599012in}{1.544951in}}%
\pgfpathlineto{\pgfqpoint{1.603789in}{1.546037in}}%
\pgfpathlineto{\pgfqpoint{1.604852in}{1.546317in}}%
\pgfpathlineto{\pgfqpoint{1.610841in}{1.547403in}}%
\pgfpathlineto{\pgfqpoint{1.611388in}{1.547651in}}%
\pgfpathlineto{\pgfqpoint{1.611708in}{1.547651in}}%
\pgfpathlineto{\pgfqpoint{1.617306in}{1.548738in}}%
\pgfpathlineto{\pgfqpoint{1.618361in}{1.549110in}}%
\pgfpathlineto{\pgfqpoint{1.623091in}{1.550197in}}%
\pgfpathlineto{\pgfqpoint{1.623966in}{1.550569in}}%
\pgfpathlineto{\pgfqpoint{1.624193in}{1.550569in}}%
\pgfpathlineto{\pgfqpoint{1.630181in}{1.551656in}}%
\pgfpathlineto{\pgfqpoint{1.631135in}{1.551780in}}%
\pgfpathlineto{\pgfqpoint{1.631206in}{1.551780in}}%
\pgfpathlineto{\pgfqpoint{1.636084in}{1.552836in}}%
\pgfpathlineto{\pgfqpoint{1.637053in}{1.553208in}}%
\pgfpathlineto{\pgfqpoint{1.637170in}{1.553208in}}%
\pgfpathlineto{\pgfqpoint{1.641157in}{1.554294in}}%
\pgfpathlineto{\pgfqpoint{1.642252in}{1.554481in}}%
\pgfpathlineto{\pgfqpoint{1.646919in}{1.555567in}}%
\pgfpathlineto{\pgfqpoint{1.648013in}{1.555878in}}%
\pgfpathlineto{\pgfqpoint{1.652524in}{1.556964in}}%
\pgfpathlineto{\pgfqpoint{1.653603in}{1.557150in}}%
\pgfpathlineto{\pgfqpoint{1.660655in}{1.558237in}}%
\pgfpathlineto{\pgfqpoint{1.661710in}{1.558454in}}%
\pgfpathlineto{\pgfqpoint{1.661726in}{1.558454in}}%
\pgfpathlineto{\pgfqpoint{1.667956in}{1.559541in}}%
\pgfpathlineto{\pgfqpoint{1.668980in}{1.559727in}}%
\pgfpathlineto{\pgfqpoint{1.674851in}{1.560813in}}%
\pgfpathlineto{\pgfqpoint{1.675883in}{1.560938in}}%
\pgfpathlineto{\pgfqpoint{1.675946in}{1.560938in}}%
\pgfpathlineto{\pgfqpoint{1.681020in}{1.562024in}}%
\pgfpathlineto{\pgfqpoint{1.682020in}{1.562334in}}%
\pgfpathlineto{\pgfqpoint{1.686554in}{1.563421in}}%
\pgfpathlineto{\pgfqpoint{1.687625in}{1.563669in}}%
\pgfpathlineto{\pgfqpoint{1.694622in}{1.564756in}}%
\pgfpathlineto{\pgfqpoint{1.695513in}{1.565035in}}%
\pgfpathlineto{\pgfqpoint{1.700657in}{1.566122in}}%
\pgfpathlineto{\pgfqpoint{1.701760in}{1.566277in}}%
\pgfpathlineto{\pgfqpoint{1.706583in}{1.567363in}}%
\pgfpathlineto{\pgfqpoint{1.707428in}{1.567674in}}%
\pgfpathlineto{\pgfqpoint{1.713815in}{1.568760in}}%
\pgfpathlineto{\pgfqpoint{1.714917in}{1.569040in}}%
\pgfpathlineto{\pgfqpoint{1.720194in}{1.570126in}}%
\pgfpathlineto{\pgfqpoint{1.721241in}{1.570374in}}%
\pgfpathlineto{\pgfqpoint{1.728645in}{1.571461in}}%
\pgfpathlineto{\pgfqpoint{1.729614in}{1.571616in}}%
\pgfpathlineto{\pgfqpoint{1.735477in}{1.572702in}}%
\pgfpathlineto{\pgfqpoint{1.736501in}{1.573199in}}%
\pgfpathlineto{\pgfqpoint{1.742685in}{1.574286in}}%
\pgfpathlineto{\pgfqpoint{1.743709in}{1.574565in}}%
\pgfpathlineto{\pgfqpoint{1.750706in}{1.575651in}}%
\pgfpathlineto{\pgfqpoint{1.751761in}{1.575838in}}%
\pgfpathlineto{\pgfqpoint{1.756249in}{1.576924in}}%
\pgfpathlineto{\pgfqpoint{1.756819in}{1.577079in}}%
\pgfpathlineto{\pgfqpoint{1.763832in}{1.578166in}}%
\pgfpathlineto{\pgfqpoint{1.764848in}{1.578445in}}%
\pgfpathlineto{\pgfqpoint{1.764934in}{1.578445in}}%
\pgfpathlineto{\pgfqpoint{1.772150in}{1.579532in}}%
\pgfpathlineto{\pgfqpoint{1.773197in}{1.579780in}}%
\pgfpathlineto{\pgfqpoint{1.773260in}{1.579780in}}%
\pgfpathlineto{\pgfqpoint{1.780155in}{1.580867in}}%
\pgfpathlineto{\pgfqpoint{1.781031in}{1.581146in}}%
\pgfpathlineto{\pgfqpoint{1.790834in}{1.582232in}}%
\pgfpathlineto{\pgfqpoint{1.791897in}{1.582388in}}%
\pgfpathlineto{\pgfqpoint{1.800966in}{1.583474in}}%
\pgfpathlineto{\pgfqpoint{1.802013in}{1.583660in}}%
\pgfpathlineto{\pgfqpoint{1.812215in}{1.584747in}}%
\pgfpathlineto{\pgfqpoint{1.813192in}{1.585057in}}%
\pgfpathlineto{\pgfqpoint{1.822409in}{1.586144in}}%
\pgfpathlineto{\pgfqpoint{1.822409in}{1.586175in}}%
\pgfpathlineto{\pgfqpoint{1.823113in}{1.586175in}}%
\pgfpathlineto{\pgfqpoint{1.834066in}{1.587261in}}%
\pgfpathlineto{\pgfqpoint{1.834511in}{1.587447in}}%
\pgfpathlineto{\pgfqpoint{1.834965in}{1.587447in}}%
\pgfpathlineto{\pgfqpoint{1.842524in}{1.588534in}}%
\pgfpathlineto{\pgfqpoint{1.843580in}{1.588751in}}%
\pgfpathlineto{\pgfqpoint{1.851952in}{1.589838in}}%
\pgfpathlineto{\pgfqpoint{1.852937in}{1.589993in}}%
\pgfpathlineto{\pgfqpoint{1.853055in}{1.589993in}}%
\pgfpathlineto{\pgfqpoint{1.865453in}{1.591079in}}%
\pgfpathlineto{\pgfqpoint{1.866462in}{1.591266in}}%
\pgfpathlineto{\pgfqpoint{1.876656in}{1.592352in}}%
\pgfpathlineto{\pgfqpoint{1.877680in}{1.592507in}}%
\pgfpathlineto{\pgfqpoint{1.892565in}{1.593594in}}%
\pgfpathlineto{\pgfqpoint{1.893284in}{1.593687in}}%
\pgfpathlineto{\pgfqpoint{1.893628in}{1.593687in}}%
\pgfpathlineto{\pgfqpoint{1.909091in}{1.594773in}}%
\pgfpathlineto{\pgfqpoint{1.909865in}{1.594898in}}%
\pgfpathlineto{\pgfqpoint{1.910108in}{1.594898in}}%
\pgfpathlineto{\pgfqpoint{1.920310in}{1.595984in}}%
\pgfpathlineto{\pgfqpoint{1.920427in}{1.596046in}}%
\pgfpathlineto{\pgfqpoint{1.920497in}{1.596046in}}%
\pgfpathlineto{\pgfqpoint{1.935578in}{1.597133in}}%
\pgfpathlineto{\pgfqpoint{1.936484in}{1.597350in}}%
\pgfpathlineto{\pgfqpoint{1.951127in}{1.598436in}}%
\pgfpathlineto{\pgfqpoint{1.951987in}{1.598498in}}%
\pgfpathlineto{\pgfqpoint{1.968388in}{1.599585in}}%
\pgfpathlineto{\pgfqpoint{1.969451in}{1.599709in}}%
\pgfpathlineto{\pgfqpoint{1.991599in}{1.600796in}}%
\pgfpathlineto{\pgfqpoint{1.992334in}{1.600889in}}%
\pgfpathlineto{\pgfqpoint{1.992654in}{1.600889in}}%
\pgfpathlineto{\pgfqpoint{2.033126in}{1.601944in}}%
\pgfpathlineto{\pgfqpoint{2.033126in}{1.601944in}}%
\pgfusepath{stroke}%
\end{pgfscope}%
\begin{pgfscope}%
\pgfsetrectcap%
\pgfsetmiterjoin%
\pgfsetlinewidth{0.803000pt}%
\definecolor{currentstroke}{rgb}{0.000000,0.000000,0.000000}%
\pgfsetstrokecolor{currentstroke}%
\pgfsetdash{}{0pt}%
\pgfpathmoveto{\pgfqpoint{0.553581in}{0.499444in}}%
\pgfpathlineto{\pgfqpoint{0.553581in}{1.654444in}}%
\pgfusepath{stroke}%
\end{pgfscope}%
\begin{pgfscope}%
\pgfsetrectcap%
\pgfsetmiterjoin%
\pgfsetlinewidth{0.803000pt}%
\definecolor{currentstroke}{rgb}{0.000000,0.000000,0.000000}%
\pgfsetstrokecolor{currentstroke}%
\pgfsetdash{}{0pt}%
\pgfpathmoveto{\pgfqpoint{2.103581in}{0.499444in}}%
\pgfpathlineto{\pgfqpoint{2.103581in}{1.654444in}}%
\pgfusepath{stroke}%
\end{pgfscope}%
\begin{pgfscope}%
\pgfsetrectcap%
\pgfsetmiterjoin%
\pgfsetlinewidth{0.803000pt}%
\definecolor{currentstroke}{rgb}{0.000000,0.000000,0.000000}%
\pgfsetstrokecolor{currentstroke}%
\pgfsetdash{}{0pt}%
\pgfpathmoveto{\pgfqpoint{0.553581in}{0.499444in}}%
\pgfpathlineto{\pgfqpoint{2.103581in}{0.499444in}}%
\pgfusepath{stroke}%
\end{pgfscope}%
\begin{pgfscope}%
\pgfsetrectcap%
\pgfsetmiterjoin%
\pgfsetlinewidth{0.803000pt}%
\definecolor{currentstroke}{rgb}{0.000000,0.000000,0.000000}%
\pgfsetstrokecolor{currentstroke}%
\pgfsetdash{}{0pt}%
\pgfpathmoveto{\pgfqpoint{0.553581in}{1.654444in}}%
\pgfpathlineto{\pgfqpoint{2.103581in}{1.654444in}}%
\pgfusepath{stroke}%
\end{pgfscope}%
\begin{pgfscope}%
\pgfsetbuttcap%
\pgfsetmiterjoin%
\definecolor{currentfill}{rgb}{1.000000,1.000000,1.000000}%
\pgfsetfillcolor{currentfill}%
\pgfsetfillopacity{0.800000}%
\pgfsetlinewidth{1.003750pt}%
\definecolor{currentstroke}{rgb}{0.800000,0.800000,0.800000}%
\pgfsetstrokecolor{currentstroke}%
\pgfsetstrokeopacity{0.800000}%
\pgfsetdash{}{0pt}%
\pgfpathmoveto{\pgfqpoint{0.832747in}{0.568889in}}%
\pgfpathlineto{\pgfqpoint{2.006358in}{0.568889in}}%
\pgfpathquadraticcurveto{\pgfqpoint{2.034136in}{0.568889in}}{\pgfqpoint{2.034136in}{0.596666in}}%
\pgfpathlineto{\pgfqpoint{2.034136in}{0.776388in}}%
\pgfpathquadraticcurveto{\pgfqpoint{2.034136in}{0.804166in}}{\pgfqpoint{2.006358in}{0.804166in}}%
\pgfpathlineto{\pgfqpoint{0.832747in}{0.804166in}}%
\pgfpathquadraticcurveto{\pgfqpoint{0.804970in}{0.804166in}}{\pgfqpoint{0.804970in}{0.776388in}}%
\pgfpathlineto{\pgfqpoint{0.804970in}{0.596666in}}%
\pgfpathquadraticcurveto{\pgfqpoint{0.804970in}{0.568889in}}{\pgfqpoint{0.832747in}{0.568889in}}%
\pgfpathlineto{\pgfqpoint{0.832747in}{0.568889in}}%
\pgfpathclose%
\pgfusepath{stroke,fill}%
\end{pgfscope}%
\begin{pgfscope}%
\pgfsetrectcap%
\pgfsetroundjoin%
\pgfsetlinewidth{1.505625pt}%
\definecolor{currentstroke}{rgb}{0.000000,0.000000,0.000000}%
\pgfsetstrokecolor{currentstroke}%
\pgfsetdash{}{0pt}%
\pgfpathmoveto{\pgfqpoint{0.860525in}{0.700000in}}%
\pgfpathlineto{\pgfqpoint{0.999414in}{0.700000in}}%
\pgfpathlineto{\pgfqpoint{1.138303in}{0.700000in}}%
\pgfusepath{stroke}%
\end{pgfscope}%
\begin{pgfscope}%
\definecolor{textcolor}{rgb}{0.000000,0.000000,0.000000}%
\pgfsetstrokecolor{textcolor}%
\pgfsetfillcolor{textcolor}%
\pgftext[x=1.249414in,y=0.651388in,left,base]{\color{textcolor}\rmfamily\fontsize{10.000000}{12.000000}\selectfont AUC=0.778}%
\end{pgfscope}%
\end{pgfpicture}%
\makeatother%
\endgroup%

\end{tabular}

[continued on next page]
\newpage


When I use the Balanced Random Forest  Classifier, however, I get really different results, which is a sign that it's overfitting.  

\ 

\verb|y_proba = estimator.predict_proba(X_train)|

\noindent\begin{tabular}{@{\hspace{-6pt}}p{4.5in} @{\hspace{-6pt}}p{2.0in}}
	\vskip 0pt
	\qquad \qquad Raw Model Output on Training Set
	
	%% Creator: Matplotlib, PGF backend
%%
%% To include the figure in your LaTeX document, write
%%   \input{<filename>.pgf}
%%
%% Make sure the required packages are loaded in your preamble
%%   \usepackage{pgf}
%%
%% Also ensure that all the required font packages are loaded; for instance,
%% the lmodern package is sometimes necessary when using math font.
%%   \usepackage{lmodern}
%%
%% Figures using additional raster images can only be included by \input if
%% they are in the same directory as the main LaTeX file. For loading figures
%% from other directories you can use the `import` package
%%   \usepackage{import}
%%
%% and then include the figures with
%%   \import{<path to file>}{<filename>.pgf}
%%
%% Matplotlib used the following preamble
%%   
%%   \usepackage{fontspec}
%%   \makeatletter\@ifpackageloaded{underscore}{}{\usepackage[strings]{underscore}}\makeatother
%%
\begingroup%
\makeatletter%
\begin{pgfpicture}%
\pgfpathrectangle{\pgfpointorigin}{\pgfqpoint{4.509306in}{1.754444in}}%
\pgfusepath{use as bounding box, clip}%
\begin{pgfscope}%
\pgfsetbuttcap%
\pgfsetmiterjoin%
\definecolor{currentfill}{rgb}{1.000000,1.000000,1.000000}%
\pgfsetfillcolor{currentfill}%
\pgfsetlinewidth{0.000000pt}%
\definecolor{currentstroke}{rgb}{1.000000,1.000000,1.000000}%
\pgfsetstrokecolor{currentstroke}%
\pgfsetdash{}{0pt}%
\pgfpathmoveto{\pgfqpoint{0.000000in}{0.000000in}}%
\pgfpathlineto{\pgfqpoint{4.509306in}{0.000000in}}%
\pgfpathlineto{\pgfqpoint{4.509306in}{1.754444in}}%
\pgfpathlineto{\pgfqpoint{0.000000in}{1.754444in}}%
\pgfpathlineto{\pgfqpoint{0.000000in}{0.000000in}}%
\pgfpathclose%
\pgfusepath{fill}%
\end{pgfscope}%
\begin{pgfscope}%
\pgfsetbuttcap%
\pgfsetmiterjoin%
\definecolor{currentfill}{rgb}{1.000000,1.000000,1.000000}%
\pgfsetfillcolor{currentfill}%
\pgfsetlinewidth{0.000000pt}%
\definecolor{currentstroke}{rgb}{0.000000,0.000000,0.000000}%
\pgfsetstrokecolor{currentstroke}%
\pgfsetstrokeopacity{0.000000}%
\pgfsetdash{}{0pt}%
\pgfpathmoveto{\pgfqpoint{0.445556in}{0.499444in}}%
\pgfpathlineto{\pgfqpoint{4.320556in}{0.499444in}}%
\pgfpathlineto{\pgfqpoint{4.320556in}{1.654444in}}%
\pgfpathlineto{\pgfqpoint{0.445556in}{1.654444in}}%
\pgfpathlineto{\pgfqpoint{0.445556in}{0.499444in}}%
\pgfpathclose%
\pgfusepath{fill}%
\end{pgfscope}%
\begin{pgfscope}%
\pgfpathrectangle{\pgfqpoint{0.445556in}{0.499444in}}{\pgfqpoint{3.875000in}{1.155000in}}%
\pgfusepath{clip}%
\pgfsetbuttcap%
\pgfsetmiterjoin%
\pgfsetlinewidth{1.003750pt}%
\definecolor{currentstroke}{rgb}{0.000000,0.000000,0.000000}%
\pgfsetstrokecolor{currentstroke}%
\pgfsetdash{}{0pt}%
\pgfpathmoveto{\pgfqpoint{0.435556in}{0.499444in}}%
\pgfpathlineto{\pgfqpoint{0.483922in}{0.499444in}}%
\pgfpathlineto{\pgfqpoint{0.483922in}{0.632510in}}%
\pgfpathlineto{\pgfqpoint{0.435556in}{0.632510in}}%
\pgfusepath{stroke}%
\end{pgfscope}%
\begin{pgfscope}%
\pgfpathrectangle{\pgfqpoint{0.445556in}{0.499444in}}{\pgfqpoint{3.875000in}{1.155000in}}%
\pgfusepath{clip}%
\pgfsetbuttcap%
\pgfsetmiterjoin%
\pgfsetlinewidth{1.003750pt}%
\definecolor{currentstroke}{rgb}{0.000000,0.000000,0.000000}%
\pgfsetstrokecolor{currentstroke}%
\pgfsetdash{}{0pt}%
\pgfpathmoveto{\pgfqpoint{0.576001in}{0.499444in}}%
\pgfpathlineto{\pgfqpoint{0.637387in}{0.499444in}}%
\pgfpathlineto{\pgfqpoint{0.637387in}{0.855505in}}%
\pgfpathlineto{\pgfqpoint{0.576001in}{0.855505in}}%
\pgfpathlineto{\pgfqpoint{0.576001in}{0.499444in}}%
\pgfpathclose%
\pgfusepath{stroke}%
\end{pgfscope}%
\begin{pgfscope}%
\pgfpathrectangle{\pgfqpoint{0.445556in}{0.499444in}}{\pgfqpoint{3.875000in}{1.155000in}}%
\pgfusepath{clip}%
\pgfsetbuttcap%
\pgfsetmiterjoin%
\pgfsetlinewidth{1.003750pt}%
\definecolor{currentstroke}{rgb}{0.000000,0.000000,0.000000}%
\pgfsetstrokecolor{currentstroke}%
\pgfsetdash{}{0pt}%
\pgfpathmoveto{\pgfqpoint{0.729467in}{0.499444in}}%
\pgfpathlineto{\pgfqpoint{0.790853in}{0.499444in}}%
\pgfpathlineto{\pgfqpoint{0.790853in}{1.087595in}}%
\pgfpathlineto{\pgfqpoint{0.729467in}{1.087595in}}%
\pgfpathlineto{\pgfqpoint{0.729467in}{0.499444in}}%
\pgfpathclose%
\pgfusepath{stroke}%
\end{pgfscope}%
\begin{pgfscope}%
\pgfpathrectangle{\pgfqpoint{0.445556in}{0.499444in}}{\pgfqpoint{3.875000in}{1.155000in}}%
\pgfusepath{clip}%
\pgfsetbuttcap%
\pgfsetmiterjoin%
\pgfsetlinewidth{1.003750pt}%
\definecolor{currentstroke}{rgb}{0.000000,0.000000,0.000000}%
\pgfsetstrokecolor{currentstroke}%
\pgfsetdash{}{0pt}%
\pgfpathmoveto{\pgfqpoint{0.882932in}{0.499444in}}%
\pgfpathlineto{\pgfqpoint{0.944318in}{0.499444in}}%
\pgfpathlineto{\pgfqpoint{0.944318in}{1.269324in}}%
\pgfpathlineto{\pgfqpoint{0.882932in}{1.269324in}}%
\pgfpathlineto{\pgfqpoint{0.882932in}{0.499444in}}%
\pgfpathclose%
\pgfusepath{stroke}%
\end{pgfscope}%
\begin{pgfscope}%
\pgfpathrectangle{\pgfqpoint{0.445556in}{0.499444in}}{\pgfqpoint{3.875000in}{1.155000in}}%
\pgfusepath{clip}%
\pgfsetbuttcap%
\pgfsetmiterjoin%
\pgfsetlinewidth{1.003750pt}%
\definecolor{currentstroke}{rgb}{0.000000,0.000000,0.000000}%
\pgfsetstrokecolor{currentstroke}%
\pgfsetdash{}{0pt}%
\pgfpathmoveto{\pgfqpoint{1.036397in}{0.499444in}}%
\pgfpathlineto{\pgfqpoint{1.097783in}{0.499444in}}%
\pgfpathlineto{\pgfqpoint{1.097783in}{1.428301in}}%
\pgfpathlineto{\pgfqpoint{1.036397in}{1.428301in}}%
\pgfpathlineto{\pgfqpoint{1.036397in}{0.499444in}}%
\pgfpathclose%
\pgfusepath{stroke}%
\end{pgfscope}%
\begin{pgfscope}%
\pgfpathrectangle{\pgfqpoint{0.445556in}{0.499444in}}{\pgfqpoint{3.875000in}{1.155000in}}%
\pgfusepath{clip}%
\pgfsetbuttcap%
\pgfsetmiterjoin%
\pgfsetlinewidth{1.003750pt}%
\definecolor{currentstroke}{rgb}{0.000000,0.000000,0.000000}%
\pgfsetstrokecolor{currentstroke}%
\pgfsetdash{}{0pt}%
\pgfpathmoveto{\pgfqpoint{1.189863in}{0.499444in}}%
\pgfpathlineto{\pgfqpoint{1.251249in}{0.499444in}}%
\pgfpathlineto{\pgfqpoint{1.251249in}{1.521593in}}%
\pgfpathlineto{\pgfqpoint{1.189863in}{1.521593in}}%
\pgfpathlineto{\pgfqpoint{1.189863in}{0.499444in}}%
\pgfpathclose%
\pgfusepath{stroke}%
\end{pgfscope}%
\begin{pgfscope}%
\pgfpathrectangle{\pgfqpoint{0.445556in}{0.499444in}}{\pgfqpoint{3.875000in}{1.155000in}}%
\pgfusepath{clip}%
\pgfsetbuttcap%
\pgfsetmiterjoin%
\pgfsetlinewidth{1.003750pt}%
\definecolor{currentstroke}{rgb}{0.000000,0.000000,0.000000}%
\pgfsetstrokecolor{currentstroke}%
\pgfsetdash{}{0pt}%
\pgfpathmoveto{\pgfqpoint{1.343328in}{0.499444in}}%
\pgfpathlineto{\pgfqpoint{1.404714in}{0.499444in}}%
\pgfpathlineto{\pgfqpoint{1.404714in}{1.590875in}}%
\pgfpathlineto{\pgfqpoint{1.343328in}{1.590875in}}%
\pgfpathlineto{\pgfqpoint{1.343328in}{0.499444in}}%
\pgfpathclose%
\pgfusepath{stroke}%
\end{pgfscope}%
\begin{pgfscope}%
\pgfpathrectangle{\pgfqpoint{0.445556in}{0.499444in}}{\pgfqpoint{3.875000in}{1.155000in}}%
\pgfusepath{clip}%
\pgfsetbuttcap%
\pgfsetmiterjoin%
\pgfsetlinewidth{1.003750pt}%
\definecolor{currentstroke}{rgb}{0.000000,0.000000,0.000000}%
\pgfsetstrokecolor{currentstroke}%
\pgfsetdash{}{0pt}%
\pgfpathmoveto{\pgfqpoint{1.496793in}{0.499444in}}%
\pgfpathlineto{\pgfqpoint{1.558179in}{0.499444in}}%
\pgfpathlineto{\pgfqpoint{1.558179in}{1.599444in}}%
\pgfpathlineto{\pgfqpoint{1.496793in}{1.599444in}}%
\pgfpathlineto{\pgfqpoint{1.496793in}{0.499444in}}%
\pgfpathclose%
\pgfusepath{stroke}%
\end{pgfscope}%
\begin{pgfscope}%
\pgfpathrectangle{\pgfqpoint{0.445556in}{0.499444in}}{\pgfqpoint{3.875000in}{1.155000in}}%
\pgfusepath{clip}%
\pgfsetbuttcap%
\pgfsetmiterjoin%
\pgfsetlinewidth{1.003750pt}%
\definecolor{currentstroke}{rgb}{0.000000,0.000000,0.000000}%
\pgfsetstrokecolor{currentstroke}%
\pgfsetdash{}{0pt}%
\pgfpathmoveto{\pgfqpoint{1.650259in}{0.499444in}}%
\pgfpathlineto{\pgfqpoint{1.711645in}{0.499444in}}%
\pgfpathlineto{\pgfqpoint{1.711645in}{1.566953in}}%
\pgfpathlineto{\pgfqpoint{1.650259in}{1.566953in}}%
\pgfpathlineto{\pgfqpoint{1.650259in}{0.499444in}}%
\pgfpathclose%
\pgfusepath{stroke}%
\end{pgfscope}%
\begin{pgfscope}%
\pgfpathrectangle{\pgfqpoint{0.445556in}{0.499444in}}{\pgfqpoint{3.875000in}{1.155000in}}%
\pgfusepath{clip}%
\pgfsetbuttcap%
\pgfsetmiterjoin%
\pgfsetlinewidth{1.003750pt}%
\definecolor{currentstroke}{rgb}{0.000000,0.000000,0.000000}%
\pgfsetstrokecolor{currentstroke}%
\pgfsetdash{}{0pt}%
\pgfpathmoveto{\pgfqpoint{1.803724in}{0.499444in}}%
\pgfpathlineto{\pgfqpoint{1.865110in}{0.499444in}}%
\pgfpathlineto{\pgfqpoint{1.865110in}{1.506942in}}%
\pgfpathlineto{\pgfqpoint{1.803724in}{1.506942in}}%
\pgfpathlineto{\pgfqpoint{1.803724in}{0.499444in}}%
\pgfpathclose%
\pgfusepath{stroke}%
\end{pgfscope}%
\begin{pgfscope}%
\pgfpathrectangle{\pgfqpoint{0.445556in}{0.499444in}}{\pgfqpoint{3.875000in}{1.155000in}}%
\pgfusepath{clip}%
\pgfsetbuttcap%
\pgfsetmiterjoin%
\pgfsetlinewidth{1.003750pt}%
\definecolor{currentstroke}{rgb}{0.000000,0.000000,0.000000}%
\pgfsetstrokecolor{currentstroke}%
\pgfsetdash{}{0pt}%
\pgfpathmoveto{\pgfqpoint{1.957189in}{0.499444in}}%
\pgfpathlineto{\pgfqpoint{2.018575in}{0.499444in}}%
\pgfpathlineto{\pgfqpoint{2.018575in}{1.416545in}}%
\pgfpathlineto{\pgfqpoint{1.957189in}{1.416545in}}%
\pgfpathlineto{\pgfqpoint{1.957189in}{0.499444in}}%
\pgfpathclose%
\pgfusepath{stroke}%
\end{pgfscope}%
\begin{pgfscope}%
\pgfpathrectangle{\pgfqpoint{0.445556in}{0.499444in}}{\pgfqpoint{3.875000in}{1.155000in}}%
\pgfusepath{clip}%
\pgfsetbuttcap%
\pgfsetmiterjoin%
\pgfsetlinewidth{1.003750pt}%
\definecolor{currentstroke}{rgb}{0.000000,0.000000,0.000000}%
\pgfsetstrokecolor{currentstroke}%
\pgfsetdash{}{0pt}%
\pgfpathmoveto{\pgfqpoint{2.110655in}{0.499444in}}%
\pgfpathlineto{\pgfqpoint{2.172041in}{0.499444in}}%
\pgfpathlineto{\pgfqpoint{2.172041in}{1.298482in}}%
\pgfpathlineto{\pgfqpoint{2.110655in}{1.298482in}}%
\pgfpathlineto{\pgfqpoint{2.110655in}{0.499444in}}%
\pgfpathclose%
\pgfusepath{stroke}%
\end{pgfscope}%
\begin{pgfscope}%
\pgfpathrectangle{\pgfqpoint{0.445556in}{0.499444in}}{\pgfqpoint{3.875000in}{1.155000in}}%
\pgfusepath{clip}%
\pgfsetbuttcap%
\pgfsetmiterjoin%
\pgfsetlinewidth{1.003750pt}%
\definecolor{currentstroke}{rgb}{0.000000,0.000000,0.000000}%
\pgfsetstrokecolor{currentstroke}%
\pgfsetdash{}{0pt}%
\pgfpathmoveto{\pgfqpoint{2.264120in}{0.499444in}}%
\pgfpathlineto{\pgfqpoint{2.325506in}{0.499444in}}%
\pgfpathlineto{\pgfqpoint{2.325506in}{1.165884in}}%
\pgfpathlineto{\pgfqpoint{2.264120in}{1.165884in}}%
\pgfpathlineto{\pgfqpoint{2.264120in}{0.499444in}}%
\pgfpathclose%
\pgfusepath{stroke}%
\end{pgfscope}%
\begin{pgfscope}%
\pgfpathrectangle{\pgfqpoint{0.445556in}{0.499444in}}{\pgfqpoint{3.875000in}{1.155000in}}%
\pgfusepath{clip}%
\pgfsetbuttcap%
\pgfsetmiterjoin%
\pgfsetlinewidth{1.003750pt}%
\definecolor{currentstroke}{rgb}{0.000000,0.000000,0.000000}%
\pgfsetstrokecolor{currentstroke}%
\pgfsetdash{}{0pt}%
\pgfpathmoveto{\pgfqpoint{2.417585in}{0.499444in}}%
\pgfpathlineto{\pgfqpoint{2.478972in}{0.499444in}}%
\pgfpathlineto{\pgfqpoint{2.478972in}{1.041650in}}%
\pgfpathlineto{\pgfqpoint{2.417585in}{1.041650in}}%
\pgfpathlineto{\pgfqpoint{2.417585in}{0.499444in}}%
\pgfpathclose%
\pgfusepath{stroke}%
\end{pgfscope}%
\begin{pgfscope}%
\pgfpathrectangle{\pgfqpoint{0.445556in}{0.499444in}}{\pgfqpoint{3.875000in}{1.155000in}}%
\pgfusepath{clip}%
\pgfsetbuttcap%
\pgfsetmiterjoin%
\pgfsetlinewidth{1.003750pt}%
\definecolor{currentstroke}{rgb}{0.000000,0.000000,0.000000}%
\pgfsetstrokecolor{currentstroke}%
\pgfsetdash{}{0pt}%
\pgfpathmoveto{\pgfqpoint{2.571051in}{0.499444in}}%
\pgfpathlineto{\pgfqpoint{2.632437in}{0.499444in}}%
\pgfpathlineto{\pgfqpoint{2.632437in}{0.916218in}}%
\pgfpathlineto{\pgfqpoint{2.571051in}{0.916218in}}%
\pgfpathlineto{\pgfqpoint{2.571051in}{0.499444in}}%
\pgfpathclose%
\pgfusepath{stroke}%
\end{pgfscope}%
\begin{pgfscope}%
\pgfpathrectangle{\pgfqpoint{0.445556in}{0.499444in}}{\pgfqpoint{3.875000in}{1.155000in}}%
\pgfusepath{clip}%
\pgfsetbuttcap%
\pgfsetmiterjoin%
\pgfsetlinewidth{1.003750pt}%
\definecolor{currentstroke}{rgb}{0.000000,0.000000,0.000000}%
\pgfsetstrokecolor{currentstroke}%
\pgfsetdash{}{0pt}%
\pgfpathmoveto{\pgfqpoint{2.724516in}{0.499444in}}%
\pgfpathlineto{\pgfqpoint{2.785902in}{0.499444in}}%
\pgfpathlineto{\pgfqpoint{2.785902in}{0.806958in}}%
\pgfpathlineto{\pgfqpoint{2.724516in}{0.806958in}}%
\pgfpathlineto{\pgfqpoint{2.724516in}{0.499444in}}%
\pgfpathclose%
\pgfusepath{stroke}%
\end{pgfscope}%
\begin{pgfscope}%
\pgfpathrectangle{\pgfqpoint{0.445556in}{0.499444in}}{\pgfqpoint{3.875000in}{1.155000in}}%
\pgfusepath{clip}%
\pgfsetbuttcap%
\pgfsetmiterjoin%
\pgfsetlinewidth{1.003750pt}%
\definecolor{currentstroke}{rgb}{0.000000,0.000000,0.000000}%
\pgfsetstrokecolor{currentstroke}%
\pgfsetdash{}{0pt}%
\pgfpathmoveto{\pgfqpoint{2.877981in}{0.499444in}}%
\pgfpathlineto{\pgfqpoint{2.939368in}{0.499444in}}%
\pgfpathlineto{\pgfqpoint{2.939368in}{0.716063in}}%
\pgfpathlineto{\pgfqpoint{2.877981in}{0.716063in}}%
\pgfpathlineto{\pgfqpoint{2.877981in}{0.499444in}}%
\pgfpathclose%
\pgfusepath{stroke}%
\end{pgfscope}%
\begin{pgfscope}%
\pgfpathrectangle{\pgfqpoint{0.445556in}{0.499444in}}{\pgfqpoint{3.875000in}{1.155000in}}%
\pgfusepath{clip}%
\pgfsetbuttcap%
\pgfsetmiterjoin%
\pgfsetlinewidth{1.003750pt}%
\definecolor{currentstroke}{rgb}{0.000000,0.000000,0.000000}%
\pgfsetstrokecolor{currentstroke}%
\pgfsetdash{}{0pt}%
\pgfpathmoveto{\pgfqpoint{3.031447in}{0.499444in}}%
\pgfpathlineto{\pgfqpoint{3.092833in}{0.499444in}}%
\pgfpathlineto{\pgfqpoint{3.092833in}{0.645904in}}%
\pgfpathlineto{\pgfqpoint{3.031447in}{0.645904in}}%
\pgfpathlineto{\pgfqpoint{3.031447in}{0.499444in}}%
\pgfpathclose%
\pgfusepath{stroke}%
\end{pgfscope}%
\begin{pgfscope}%
\pgfpathrectangle{\pgfqpoint{0.445556in}{0.499444in}}{\pgfqpoint{3.875000in}{1.155000in}}%
\pgfusepath{clip}%
\pgfsetbuttcap%
\pgfsetmiterjoin%
\pgfsetlinewidth{1.003750pt}%
\definecolor{currentstroke}{rgb}{0.000000,0.000000,0.000000}%
\pgfsetstrokecolor{currentstroke}%
\pgfsetdash{}{0pt}%
\pgfpathmoveto{\pgfqpoint{3.184912in}{0.499444in}}%
\pgfpathlineto{\pgfqpoint{3.246298in}{0.499444in}}%
\pgfpathlineto{\pgfqpoint{3.246298in}{0.598936in}}%
\pgfpathlineto{\pgfqpoint{3.184912in}{0.598936in}}%
\pgfpathlineto{\pgfqpoint{3.184912in}{0.499444in}}%
\pgfpathclose%
\pgfusepath{stroke}%
\end{pgfscope}%
\begin{pgfscope}%
\pgfpathrectangle{\pgfqpoint{0.445556in}{0.499444in}}{\pgfqpoint{3.875000in}{1.155000in}}%
\pgfusepath{clip}%
\pgfsetbuttcap%
\pgfsetmiterjoin%
\pgfsetlinewidth{1.003750pt}%
\definecolor{currentstroke}{rgb}{0.000000,0.000000,0.000000}%
\pgfsetstrokecolor{currentstroke}%
\pgfsetdash{}{0pt}%
\pgfpathmoveto{\pgfqpoint{3.338377in}{0.499444in}}%
\pgfpathlineto{\pgfqpoint{3.399764in}{0.499444in}}%
\pgfpathlineto{\pgfqpoint{3.399764in}{0.560040in}}%
\pgfpathlineto{\pgfqpoint{3.338377in}{0.560040in}}%
\pgfpathlineto{\pgfqpoint{3.338377in}{0.499444in}}%
\pgfpathclose%
\pgfusepath{stroke}%
\end{pgfscope}%
\begin{pgfscope}%
\pgfpathrectangle{\pgfqpoint{0.445556in}{0.499444in}}{\pgfqpoint{3.875000in}{1.155000in}}%
\pgfusepath{clip}%
\pgfsetbuttcap%
\pgfsetmiterjoin%
\pgfsetlinewidth{1.003750pt}%
\definecolor{currentstroke}{rgb}{0.000000,0.000000,0.000000}%
\pgfsetstrokecolor{currentstroke}%
\pgfsetdash{}{0pt}%
\pgfpathmoveto{\pgfqpoint{3.491843in}{0.499444in}}%
\pgfpathlineto{\pgfqpoint{3.553229in}{0.499444in}}%
\pgfpathlineto{\pgfqpoint{3.553229in}{0.535591in}}%
\pgfpathlineto{\pgfqpoint{3.491843in}{0.535591in}}%
\pgfpathlineto{\pgfqpoint{3.491843in}{0.499444in}}%
\pgfpathclose%
\pgfusepath{stroke}%
\end{pgfscope}%
\begin{pgfscope}%
\pgfpathrectangle{\pgfqpoint{0.445556in}{0.499444in}}{\pgfqpoint{3.875000in}{1.155000in}}%
\pgfusepath{clip}%
\pgfsetbuttcap%
\pgfsetmiterjoin%
\pgfsetlinewidth{1.003750pt}%
\definecolor{currentstroke}{rgb}{0.000000,0.000000,0.000000}%
\pgfsetstrokecolor{currentstroke}%
\pgfsetdash{}{0pt}%
\pgfpathmoveto{\pgfqpoint{3.645308in}{0.499444in}}%
\pgfpathlineto{\pgfqpoint{3.706694in}{0.499444in}}%
\pgfpathlineto{\pgfqpoint{3.706694in}{0.512137in}}%
\pgfpathlineto{\pgfqpoint{3.645308in}{0.512137in}}%
\pgfpathlineto{\pgfqpoint{3.645308in}{0.499444in}}%
\pgfpathclose%
\pgfusepath{stroke}%
\end{pgfscope}%
\begin{pgfscope}%
\pgfpathrectangle{\pgfqpoint{0.445556in}{0.499444in}}{\pgfqpoint{3.875000in}{1.155000in}}%
\pgfusepath{clip}%
\pgfsetbuttcap%
\pgfsetmiterjoin%
\pgfsetlinewidth{1.003750pt}%
\definecolor{currentstroke}{rgb}{0.000000,0.000000,0.000000}%
\pgfsetstrokecolor{currentstroke}%
\pgfsetdash{}{0pt}%
\pgfpathmoveto{\pgfqpoint{3.798774in}{0.499444in}}%
\pgfpathlineto{\pgfqpoint{3.860160in}{0.499444in}}%
\pgfpathlineto{\pgfqpoint{3.860160in}{0.503305in}}%
\pgfpathlineto{\pgfqpoint{3.798774in}{0.503305in}}%
\pgfpathlineto{\pgfqpoint{3.798774in}{0.499444in}}%
\pgfpathclose%
\pgfusepath{stroke}%
\end{pgfscope}%
\begin{pgfscope}%
\pgfpathrectangle{\pgfqpoint{0.445556in}{0.499444in}}{\pgfqpoint{3.875000in}{1.155000in}}%
\pgfusepath{clip}%
\pgfsetbuttcap%
\pgfsetmiterjoin%
\pgfsetlinewidth{1.003750pt}%
\definecolor{currentstroke}{rgb}{0.000000,0.000000,0.000000}%
\pgfsetstrokecolor{currentstroke}%
\pgfsetdash{}{0pt}%
\pgfpathmoveto{\pgfqpoint{3.952239in}{0.499444in}}%
\pgfpathlineto{\pgfqpoint{4.013625in}{0.499444in}}%
\pgfpathlineto{\pgfqpoint{4.013625in}{0.500234in}}%
\pgfpathlineto{\pgfqpoint{3.952239in}{0.500234in}}%
\pgfpathlineto{\pgfqpoint{3.952239in}{0.499444in}}%
\pgfpathclose%
\pgfusepath{stroke}%
\end{pgfscope}%
\begin{pgfscope}%
\pgfpathrectangle{\pgfqpoint{0.445556in}{0.499444in}}{\pgfqpoint{3.875000in}{1.155000in}}%
\pgfusepath{clip}%
\pgfsetbuttcap%
\pgfsetmiterjoin%
\pgfsetlinewidth{1.003750pt}%
\definecolor{currentstroke}{rgb}{0.000000,0.000000,0.000000}%
\pgfsetstrokecolor{currentstroke}%
\pgfsetdash{}{0pt}%
\pgfpathmoveto{\pgfqpoint{4.105704in}{0.499444in}}%
\pgfpathlineto{\pgfqpoint{4.167090in}{0.499444in}}%
\pgfpathlineto{\pgfqpoint{4.167090in}{0.499503in}}%
\pgfpathlineto{\pgfqpoint{4.105704in}{0.499503in}}%
\pgfpathlineto{\pgfqpoint{4.105704in}{0.499444in}}%
\pgfpathclose%
\pgfusepath{stroke}%
\end{pgfscope}%
\begin{pgfscope}%
\pgfpathrectangle{\pgfqpoint{0.445556in}{0.499444in}}{\pgfqpoint{3.875000in}{1.155000in}}%
\pgfusepath{clip}%
\pgfsetbuttcap%
\pgfsetmiterjoin%
\definecolor{currentfill}{rgb}{0.000000,0.000000,0.000000}%
\pgfsetfillcolor{currentfill}%
\pgfsetlinewidth{0.000000pt}%
\definecolor{currentstroke}{rgb}{0.000000,0.000000,0.000000}%
\pgfsetstrokecolor{currentstroke}%
\pgfsetstrokeopacity{0.000000}%
\pgfsetdash{}{0pt}%
\pgfpathmoveto{\pgfqpoint{0.483922in}{0.499444in}}%
\pgfpathlineto{\pgfqpoint{0.545308in}{0.499444in}}%
\pgfpathlineto{\pgfqpoint{0.545308in}{0.499444in}}%
\pgfpathlineto{\pgfqpoint{0.483922in}{0.499444in}}%
\pgfpathlineto{\pgfqpoint{0.483922in}{0.499444in}}%
\pgfpathclose%
\pgfusepath{fill}%
\end{pgfscope}%
\begin{pgfscope}%
\pgfpathrectangle{\pgfqpoint{0.445556in}{0.499444in}}{\pgfqpoint{3.875000in}{1.155000in}}%
\pgfusepath{clip}%
\pgfsetbuttcap%
\pgfsetmiterjoin%
\definecolor{currentfill}{rgb}{0.000000,0.000000,0.000000}%
\pgfsetfillcolor{currentfill}%
\pgfsetlinewidth{0.000000pt}%
\definecolor{currentstroke}{rgb}{0.000000,0.000000,0.000000}%
\pgfsetstrokecolor{currentstroke}%
\pgfsetstrokeopacity{0.000000}%
\pgfsetdash{}{0pt}%
\pgfpathmoveto{\pgfqpoint{0.637387in}{0.499444in}}%
\pgfpathlineto{\pgfqpoint{0.698774in}{0.499444in}}%
\pgfpathlineto{\pgfqpoint{0.698774in}{0.499444in}}%
\pgfpathlineto{\pgfqpoint{0.637387in}{0.499444in}}%
\pgfpathlineto{\pgfqpoint{0.637387in}{0.499444in}}%
\pgfpathclose%
\pgfusepath{fill}%
\end{pgfscope}%
\begin{pgfscope}%
\pgfpathrectangle{\pgfqpoint{0.445556in}{0.499444in}}{\pgfqpoint{3.875000in}{1.155000in}}%
\pgfusepath{clip}%
\pgfsetbuttcap%
\pgfsetmiterjoin%
\definecolor{currentfill}{rgb}{0.000000,0.000000,0.000000}%
\pgfsetfillcolor{currentfill}%
\pgfsetlinewidth{0.000000pt}%
\definecolor{currentstroke}{rgb}{0.000000,0.000000,0.000000}%
\pgfsetstrokecolor{currentstroke}%
\pgfsetstrokeopacity{0.000000}%
\pgfsetdash{}{0pt}%
\pgfpathmoveto{\pgfqpoint{0.790853in}{0.499444in}}%
\pgfpathlineto{\pgfqpoint{0.852239in}{0.499444in}}%
\pgfpathlineto{\pgfqpoint{0.852239in}{0.499444in}}%
\pgfpathlineto{\pgfqpoint{0.790853in}{0.499444in}}%
\pgfpathlineto{\pgfqpoint{0.790853in}{0.499444in}}%
\pgfpathclose%
\pgfusepath{fill}%
\end{pgfscope}%
\begin{pgfscope}%
\pgfpathrectangle{\pgfqpoint{0.445556in}{0.499444in}}{\pgfqpoint{3.875000in}{1.155000in}}%
\pgfusepath{clip}%
\pgfsetbuttcap%
\pgfsetmiterjoin%
\definecolor{currentfill}{rgb}{0.000000,0.000000,0.000000}%
\pgfsetfillcolor{currentfill}%
\pgfsetlinewidth{0.000000pt}%
\definecolor{currentstroke}{rgb}{0.000000,0.000000,0.000000}%
\pgfsetstrokecolor{currentstroke}%
\pgfsetstrokeopacity{0.000000}%
\pgfsetdash{}{0pt}%
\pgfpathmoveto{\pgfqpoint{0.944318in}{0.499444in}}%
\pgfpathlineto{\pgfqpoint{1.005704in}{0.499444in}}%
\pgfpathlineto{\pgfqpoint{1.005704in}{0.499444in}}%
\pgfpathlineto{\pgfqpoint{0.944318in}{0.499444in}}%
\pgfpathlineto{\pgfqpoint{0.944318in}{0.499444in}}%
\pgfpathclose%
\pgfusepath{fill}%
\end{pgfscope}%
\begin{pgfscope}%
\pgfpathrectangle{\pgfqpoint{0.445556in}{0.499444in}}{\pgfqpoint{3.875000in}{1.155000in}}%
\pgfusepath{clip}%
\pgfsetbuttcap%
\pgfsetmiterjoin%
\definecolor{currentfill}{rgb}{0.000000,0.000000,0.000000}%
\pgfsetfillcolor{currentfill}%
\pgfsetlinewidth{0.000000pt}%
\definecolor{currentstroke}{rgb}{0.000000,0.000000,0.000000}%
\pgfsetstrokecolor{currentstroke}%
\pgfsetstrokeopacity{0.000000}%
\pgfsetdash{}{0pt}%
\pgfpathmoveto{\pgfqpoint{1.097783in}{0.499444in}}%
\pgfpathlineto{\pgfqpoint{1.159170in}{0.499444in}}%
\pgfpathlineto{\pgfqpoint{1.159170in}{0.499444in}}%
\pgfpathlineto{\pgfqpoint{1.097783in}{0.499444in}}%
\pgfpathlineto{\pgfqpoint{1.097783in}{0.499444in}}%
\pgfpathclose%
\pgfusepath{fill}%
\end{pgfscope}%
\begin{pgfscope}%
\pgfpathrectangle{\pgfqpoint{0.445556in}{0.499444in}}{\pgfqpoint{3.875000in}{1.155000in}}%
\pgfusepath{clip}%
\pgfsetbuttcap%
\pgfsetmiterjoin%
\definecolor{currentfill}{rgb}{0.000000,0.000000,0.000000}%
\pgfsetfillcolor{currentfill}%
\pgfsetlinewidth{0.000000pt}%
\definecolor{currentstroke}{rgb}{0.000000,0.000000,0.000000}%
\pgfsetstrokecolor{currentstroke}%
\pgfsetstrokeopacity{0.000000}%
\pgfsetdash{}{0pt}%
\pgfpathmoveto{\pgfqpoint{1.251249in}{0.499444in}}%
\pgfpathlineto{\pgfqpoint{1.312635in}{0.499444in}}%
\pgfpathlineto{\pgfqpoint{1.312635in}{0.499444in}}%
\pgfpathlineto{\pgfqpoint{1.251249in}{0.499444in}}%
\pgfpathlineto{\pgfqpoint{1.251249in}{0.499444in}}%
\pgfpathclose%
\pgfusepath{fill}%
\end{pgfscope}%
\begin{pgfscope}%
\pgfpathrectangle{\pgfqpoint{0.445556in}{0.499444in}}{\pgfqpoint{3.875000in}{1.155000in}}%
\pgfusepath{clip}%
\pgfsetbuttcap%
\pgfsetmiterjoin%
\definecolor{currentfill}{rgb}{0.000000,0.000000,0.000000}%
\pgfsetfillcolor{currentfill}%
\pgfsetlinewidth{0.000000pt}%
\definecolor{currentstroke}{rgb}{0.000000,0.000000,0.000000}%
\pgfsetstrokecolor{currentstroke}%
\pgfsetstrokeopacity{0.000000}%
\pgfsetdash{}{0pt}%
\pgfpathmoveto{\pgfqpoint{1.404714in}{0.499444in}}%
\pgfpathlineto{\pgfqpoint{1.466100in}{0.499444in}}%
\pgfpathlineto{\pgfqpoint{1.466100in}{0.499444in}}%
\pgfpathlineto{\pgfqpoint{1.404714in}{0.499444in}}%
\pgfpathlineto{\pgfqpoint{1.404714in}{0.499444in}}%
\pgfpathclose%
\pgfusepath{fill}%
\end{pgfscope}%
\begin{pgfscope}%
\pgfpathrectangle{\pgfqpoint{0.445556in}{0.499444in}}{\pgfqpoint{3.875000in}{1.155000in}}%
\pgfusepath{clip}%
\pgfsetbuttcap%
\pgfsetmiterjoin%
\definecolor{currentfill}{rgb}{0.000000,0.000000,0.000000}%
\pgfsetfillcolor{currentfill}%
\pgfsetlinewidth{0.000000pt}%
\definecolor{currentstroke}{rgb}{0.000000,0.000000,0.000000}%
\pgfsetstrokecolor{currentstroke}%
\pgfsetstrokeopacity{0.000000}%
\pgfsetdash{}{0pt}%
\pgfpathmoveto{\pgfqpoint{1.558179in}{0.499444in}}%
\pgfpathlineto{\pgfqpoint{1.619566in}{0.499444in}}%
\pgfpathlineto{\pgfqpoint{1.619566in}{0.499444in}}%
\pgfpathlineto{\pgfqpoint{1.558179in}{0.499444in}}%
\pgfpathlineto{\pgfqpoint{1.558179in}{0.499444in}}%
\pgfpathclose%
\pgfusepath{fill}%
\end{pgfscope}%
\begin{pgfscope}%
\pgfpathrectangle{\pgfqpoint{0.445556in}{0.499444in}}{\pgfqpoint{3.875000in}{1.155000in}}%
\pgfusepath{clip}%
\pgfsetbuttcap%
\pgfsetmiterjoin%
\definecolor{currentfill}{rgb}{0.000000,0.000000,0.000000}%
\pgfsetfillcolor{currentfill}%
\pgfsetlinewidth{0.000000pt}%
\definecolor{currentstroke}{rgb}{0.000000,0.000000,0.000000}%
\pgfsetstrokecolor{currentstroke}%
\pgfsetstrokeopacity{0.000000}%
\pgfsetdash{}{0pt}%
\pgfpathmoveto{\pgfqpoint{1.711645in}{0.499444in}}%
\pgfpathlineto{\pgfqpoint{1.773031in}{0.499444in}}%
\pgfpathlineto{\pgfqpoint{1.773031in}{0.499444in}}%
\pgfpathlineto{\pgfqpoint{1.711645in}{0.499444in}}%
\pgfpathlineto{\pgfqpoint{1.711645in}{0.499444in}}%
\pgfpathclose%
\pgfusepath{fill}%
\end{pgfscope}%
\begin{pgfscope}%
\pgfpathrectangle{\pgfqpoint{0.445556in}{0.499444in}}{\pgfqpoint{3.875000in}{1.155000in}}%
\pgfusepath{clip}%
\pgfsetbuttcap%
\pgfsetmiterjoin%
\definecolor{currentfill}{rgb}{0.000000,0.000000,0.000000}%
\pgfsetfillcolor{currentfill}%
\pgfsetlinewidth{0.000000pt}%
\definecolor{currentstroke}{rgb}{0.000000,0.000000,0.000000}%
\pgfsetstrokecolor{currentstroke}%
\pgfsetstrokeopacity{0.000000}%
\pgfsetdash{}{0pt}%
\pgfpathmoveto{\pgfqpoint{1.865110in}{0.499444in}}%
\pgfpathlineto{\pgfqpoint{1.926496in}{0.499444in}}%
\pgfpathlineto{\pgfqpoint{1.926496in}{0.499444in}}%
\pgfpathlineto{\pgfqpoint{1.865110in}{0.499444in}}%
\pgfpathlineto{\pgfqpoint{1.865110in}{0.499444in}}%
\pgfpathclose%
\pgfusepath{fill}%
\end{pgfscope}%
\begin{pgfscope}%
\pgfpathrectangle{\pgfqpoint{0.445556in}{0.499444in}}{\pgfqpoint{3.875000in}{1.155000in}}%
\pgfusepath{clip}%
\pgfsetbuttcap%
\pgfsetmiterjoin%
\definecolor{currentfill}{rgb}{0.000000,0.000000,0.000000}%
\pgfsetfillcolor{currentfill}%
\pgfsetlinewidth{0.000000pt}%
\definecolor{currentstroke}{rgb}{0.000000,0.000000,0.000000}%
\pgfsetstrokecolor{currentstroke}%
\pgfsetstrokeopacity{0.000000}%
\pgfsetdash{}{0pt}%
\pgfpathmoveto{\pgfqpoint{2.018575in}{0.499444in}}%
\pgfpathlineto{\pgfqpoint{2.079962in}{0.499444in}}%
\pgfpathlineto{\pgfqpoint{2.079962in}{0.499444in}}%
\pgfpathlineto{\pgfqpoint{2.018575in}{0.499444in}}%
\pgfpathlineto{\pgfqpoint{2.018575in}{0.499444in}}%
\pgfpathclose%
\pgfusepath{fill}%
\end{pgfscope}%
\begin{pgfscope}%
\pgfpathrectangle{\pgfqpoint{0.445556in}{0.499444in}}{\pgfqpoint{3.875000in}{1.155000in}}%
\pgfusepath{clip}%
\pgfsetbuttcap%
\pgfsetmiterjoin%
\definecolor{currentfill}{rgb}{0.000000,0.000000,0.000000}%
\pgfsetfillcolor{currentfill}%
\pgfsetlinewidth{0.000000pt}%
\definecolor{currentstroke}{rgb}{0.000000,0.000000,0.000000}%
\pgfsetstrokecolor{currentstroke}%
\pgfsetstrokeopacity{0.000000}%
\pgfsetdash{}{0pt}%
\pgfpathmoveto{\pgfqpoint{2.172041in}{0.499444in}}%
\pgfpathlineto{\pgfqpoint{2.233427in}{0.499444in}}%
\pgfpathlineto{\pgfqpoint{2.233427in}{0.499444in}}%
\pgfpathlineto{\pgfqpoint{2.172041in}{0.499444in}}%
\pgfpathlineto{\pgfqpoint{2.172041in}{0.499444in}}%
\pgfpathclose%
\pgfusepath{fill}%
\end{pgfscope}%
\begin{pgfscope}%
\pgfpathrectangle{\pgfqpoint{0.445556in}{0.499444in}}{\pgfqpoint{3.875000in}{1.155000in}}%
\pgfusepath{clip}%
\pgfsetbuttcap%
\pgfsetmiterjoin%
\definecolor{currentfill}{rgb}{0.000000,0.000000,0.000000}%
\pgfsetfillcolor{currentfill}%
\pgfsetlinewidth{0.000000pt}%
\definecolor{currentstroke}{rgb}{0.000000,0.000000,0.000000}%
\pgfsetstrokecolor{currentstroke}%
\pgfsetstrokeopacity{0.000000}%
\pgfsetdash{}{0pt}%
\pgfpathmoveto{\pgfqpoint{2.325506in}{0.499444in}}%
\pgfpathlineto{\pgfqpoint{2.386892in}{0.499444in}}%
\pgfpathlineto{\pgfqpoint{2.386892in}{0.499561in}}%
\pgfpathlineto{\pgfqpoint{2.325506in}{0.499561in}}%
\pgfpathlineto{\pgfqpoint{2.325506in}{0.499444in}}%
\pgfpathclose%
\pgfusepath{fill}%
\end{pgfscope}%
\begin{pgfscope}%
\pgfpathrectangle{\pgfqpoint{0.445556in}{0.499444in}}{\pgfqpoint{3.875000in}{1.155000in}}%
\pgfusepath{clip}%
\pgfsetbuttcap%
\pgfsetmiterjoin%
\definecolor{currentfill}{rgb}{0.000000,0.000000,0.000000}%
\pgfsetfillcolor{currentfill}%
\pgfsetlinewidth{0.000000pt}%
\definecolor{currentstroke}{rgb}{0.000000,0.000000,0.000000}%
\pgfsetstrokecolor{currentstroke}%
\pgfsetstrokeopacity{0.000000}%
\pgfsetdash{}{0pt}%
\pgfpathmoveto{\pgfqpoint{2.478972in}{0.499444in}}%
\pgfpathlineto{\pgfqpoint{2.540358in}{0.499444in}}%
\pgfpathlineto{\pgfqpoint{2.540358in}{0.499854in}}%
\pgfpathlineto{\pgfqpoint{2.478972in}{0.499854in}}%
\pgfpathlineto{\pgfqpoint{2.478972in}{0.499444in}}%
\pgfpathclose%
\pgfusepath{fill}%
\end{pgfscope}%
\begin{pgfscope}%
\pgfpathrectangle{\pgfqpoint{0.445556in}{0.499444in}}{\pgfqpoint{3.875000in}{1.155000in}}%
\pgfusepath{clip}%
\pgfsetbuttcap%
\pgfsetmiterjoin%
\definecolor{currentfill}{rgb}{0.000000,0.000000,0.000000}%
\pgfsetfillcolor{currentfill}%
\pgfsetlinewidth{0.000000pt}%
\definecolor{currentstroke}{rgb}{0.000000,0.000000,0.000000}%
\pgfsetstrokecolor{currentstroke}%
\pgfsetstrokeopacity{0.000000}%
\pgfsetdash{}{0pt}%
\pgfpathmoveto{\pgfqpoint{2.632437in}{0.499444in}}%
\pgfpathlineto{\pgfqpoint{2.693823in}{0.499444in}}%
\pgfpathlineto{\pgfqpoint{2.693823in}{0.501521in}}%
\pgfpathlineto{\pgfqpoint{2.632437in}{0.501521in}}%
\pgfpathlineto{\pgfqpoint{2.632437in}{0.499444in}}%
\pgfpathclose%
\pgfusepath{fill}%
\end{pgfscope}%
\begin{pgfscope}%
\pgfpathrectangle{\pgfqpoint{0.445556in}{0.499444in}}{\pgfqpoint{3.875000in}{1.155000in}}%
\pgfusepath{clip}%
\pgfsetbuttcap%
\pgfsetmiterjoin%
\definecolor{currentfill}{rgb}{0.000000,0.000000,0.000000}%
\pgfsetfillcolor{currentfill}%
\pgfsetlinewidth{0.000000pt}%
\definecolor{currentstroke}{rgb}{0.000000,0.000000,0.000000}%
\pgfsetstrokecolor{currentstroke}%
\pgfsetstrokeopacity{0.000000}%
\pgfsetdash{}{0pt}%
\pgfpathmoveto{\pgfqpoint{2.785902in}{0.499444in}}%
\pgfpathlineto{\pgfqpoint{2.847288in}{0.499444in}}%
\pgfpathlineto{\pgfqpoint{2.847288in}{0.511171in}}%
\pgfpathlineto{\pgfqpoint{2.785902in}{0.511171in}}%
\pgfpathlineto{\pgfqpoint{2.785902in}{0.499444in}}%
\pgfpathclose%
\pgfusepath{fill}%
\end{pgfscope}%
\begin{pgfscope}%
\pgfpathrectangle{\pgfqpoint{0.445556in}{0.499444in}}{\pgfqpoint{3.875000in}{1.155000in}}%
\pgfusepath{clip}%
\pgfsetbuttcap%
\pgfsetmiterjoin%
\definecolor{currentfill}{rgb}{0.000000,0.000000,0.000000}%
\pgfsetfillcolor{currentfill}%
\pgfsetlinewidth{0.000000pt}%
\definecolor{currentstroke}{rgb}{0.000000,0.000000,0.000000}%
\pgfsetstrokecolor{currentstroke}%
\pgfsetstrokeopacity{0.000000}%
\pgfsetdash{}{0pt}%
\pgfpathmoveto{\pgfqpoint{2.939368in}{0.499444in}}%
\pgfpathlineto{\pgfqpoint{3.000754in}{0.499444in}}%
\pgfpathlineto{\pgfqpoint{3.000754in}{0.534889in}}%
\pgfpathlineto{\pgfqpoint{2.939368in}{0.534889in}}%
\pgfpathlineto{\pgfqpoint{2.939368in}{0.499444in}}%
\pgfpathclose%
\pgfusepath{fill}%
\end{pgfscope}%
\begin{pgfscope}%
\pgfpathrectangle{\pgfqpoint{0.445556in}{0.499444in}}{\pgfqpoint{3.875000in}{1.155000in}}%
\pgfusepath{clip}%
\pgfsetbuttcap%
\pgfsetmiterjoin%
\definecolor{currentfill}{rgb}{0.000000,0.000000,0.000000}%
\pgfsetfillcolor{currentfill}%
\pgfsetlinewidth{0.000000pt}%
\definecolor{currentstroke}{rgb}{0.000000,0.000000,0.000000}%
\pgfsetstrokecolor{currentstroke}%
\pgfsetstrokeopacity{0.000000}%
\pgfsetdash{}{0pt}%
\pgfpathmoveto{\pgfqpoint{3.092833in}{0.499444in}}%
\pgfpathlineto{\pgfqpoint{3.154219in}{0.499444in}}%
\pgfpathlineto{\pgfqpoint{3.154219in}{0.582998in}}%
\pgfpathlineto{\pgfqpoint{3.092833in}{0.582998in}}%
\pgfpathlineto{\pgfqpoint{3.092833in}{0.499444in}}%
\pgfpathclose%
\pgfusepath{fill}%
\end{pgfscope}%
\begin{pgfscope}%
\pgfpathrectangle{\pgfqpoint{0.445556in}{0.499444in}}{\pgfqpoint{3.875000in}{1.155000in}}%
\pgfusepath{clip}%
\pgfsetbuttcap%
\pgfsetmiterjoin%
\definecolor{currentfill}{rgb}{0.000000,0.000000,0.000000}%
\pgfsetfillcolor{currentfill}%
\pgfsetlinewidth{0.000000pt}%
\definecolor{currentstroke}{rgb}{0.000000,0.000000,0.000000}%
\pgfsetstrokecolor{currentstroke}%
\pgfsetstrokeopacity{0.000000}%
\pgfsetdash{}{0pt}%
\pgfpathmoveto{\pgfqpoint{3.246298in}{0.499444in}}%
\pgfpathlineto{\pgfqpoint{3.307684in}{0.499444in}}%
\pgfpathlineto{\pgfqpoint{3.307684in}{0.662340in}}%
\pgfpathlineto{\pgfqpoint{3.246298in}{0.662340in}}%
\pgfpathlineto{\pgfqpoint{3.246298in}{0.499444in}}%
\pgfpathclose%
\pgfusepath{fill}%
\end{pgfscope}%
\begin{pgfscope}%
\pgfpathrectangle{\pgfqpoint{0.445556in}{0.499444in}}{\pgfqpoint{3.875000in}{1.155000in}}%
\pgfusepath{clip}%
\pgfsetbuttcap%
\pgfsetmiterjoin%
\definecolor{currentfill}{rgb}{0.000000,0.000000,0.000000}%
\pgfsetfillcolor{currentfill}%
\pgfsetlinewidth{0.000000pt}%
\definecolor{currentstroke}{rgb}{0.000000,0.000000,0.000000}%
\pgfsetstrokecolor{currentstroke}%
\pgfsetstrokeopacity{0.000000}%
\pgfsetdash{}{0pt}%
\pgfpathmoveto{\pgfqpoint{3.399764in}{0.499444in}}%
\pgfpathlineto{\pgfqpoint{3.461150in}{0.499444in}}%
\pgfpathlineto{\pgfqpoint{3.461150in}{0.763967in}}%
\pgfpathlineto{\pgfqpoint{3.399764in}{0.763967in}}%
\pgfpathlineto{\pgfqpoint{3.399764in}{0.499444in}}%
\pgfpathclose%
\pgfusepath{fill}%
\end{pgfscope}%
\begin{pgfscope}%
\pgfpathrectangle{\pgfqpoint{0.445556in}{0.499444in}}{\pgfqpoint{3.875000in}{1.155000in}}%
\pgfusepath{clip}%
\pgfsetbuttcap%
\pgfsetmiterjoin%
\definecolor{currentfill}{rgb}{0.000000,0.000000,0.000000}%
\pgfsetfillcolor{currentfill}%
\pgfsetlinewidth{0.000000pt}%
\definecolor{currentstroke}{rgb}{0.000000,0.000000,0.000000}%
\pgfsetstrokecolor{currentstroke}%
\pgfsetstrokeopacity{0.000000}%
\pgfsetdash{}{0pt}%
\pgfpathmoveto{\pgfqpoint{3.553229in}{0.499444in}}%
\pgfpathlineto{\pgfqpoint{3.614615in}{0.499444in}}%
\pgfpathlineto{\pgfqpoint{3.614615in}{0.871794in}}%
\pgfpathlineto{\pgfqpoint{3.553229in}{0.871794in}}%
\pgfpathlineto{\pgfqpoint{3.553229in}{0.499444in}}%
\pgfpathclose%
\pgfusepath{fill}%
\end{pgfscope}%
\begin{pgfscope}%
\pgfpathrectangle{\pgfqpoint{0.445556in}{0.499444in}}{\pgfqpoint{3.875000in}{1.155000in}}%
\pgfusepath{clip}%
\pgfsetbuttcap%
\pgfsetmiterjoin%
\definecolor{currentfill}{rgb}{0.000000,0.000000,0.000000}%
\pgfsetfillcolor{currentfill}%
\pgfsetlinewidth{0.000000pt}%
\definecolor{currentstroke}{rgb}{0.000000,0.000000,0.000000}%
\pgfsetstrokecolor{currentstroke}%
\pgfsetstrokeopacity{0.000000}%
\pgfsetdash{}{0pt}%
\pgfpathmoveto{\pgfqpoint{3.706694in}{0.499444in}}%
\pgfpathlineto{\pgfqpoint{3.768080in}{0.499444in}}%
\pgfpathlineto{\pgfqpoint{3.768080in}{0.925810in}}%
\pgfpathlineto{\pgfqpoint{3.706694in}{0.925810in}}%
\pgfpathlineto{\pgfqpoint{3.706694in}{0.499444in}}%
\pgfpathclose%
\pgfusepath{fill}%
\end{pgfscope}%
\begin{pgfscope}%
\pgfpathrectangle{\pgfqpoint{0.445556in}{0.499444in}}{\pgfqpoint{3.875000in}{1.155000in}}%
\pgfusepath{clip}%
\pgfsetbuttcap%
\pgfsetmiterjoin%
\definecolor{currentfill}{rgb}{0.000000,0.000000,0.000000}%
\pgfsetfillcolor{currentfill}%
\pgfsetlinewidth{0.000000pt}%
\definecolor{currentstroke}{rgb}{0.000000,0.000000,0.000000}%
\pgfsetstrokecolor{currentstroke}%
\pgfsetstrokeopacity{0.000000}%
\pgfsetdash{}{0pt}%
\pgfpathmoveto{\pgfqpoint{3.860160in}{0.499444in}}%
\pgfpathlineto{\pgfqpoint{3.921546in}{0.499444in}}%
\pgfpathlineto{\pgfqpoint{3.921546in}{0.913586in}}%
\pgfpathlineto{\pgfqpoint{3.860160in}{0.913586in}}%
\pgfpathlineto{\pgfqpoint{3.860160in}{0.499444in}}%
\pgfpathclose%
\pgfusepath{fill}%
\end{pgfscope}%
\begin{pgfscope}%
\pgfpathrectangle{\pgfqpoint{0.445556in}{0.499444in}}{\pgfqpoint{3.875000in}{1.155000in}}%
\pgfusepath{clip}%
\pgfsetbuttcap%
\pgfsetmiterjoin%
\definecolor{currentfill}{rgb}{0.000000,0.000000,0.000000}%
\pgfsetfillcolor{currentfill}%
\pgfsetlinewidth{0.000000pt}%
\definecolor{currentstroke}{rgb}{0.000000,0.000000,0.000000}%
\pgfsetstrokecolor{currentstroke}%
\pgfsetstrokeopacity{0.000000}%
\pgfsetdash{}{0pt}%
\pgfpathmoveto{\pgfqpoint{4.013625in}{0.499444in}}%
\pgfpathlineto{\pgfqpoint{4.075011in}{0.499444in}}%
\pgfpathlineto{\pgfqpoint{4.075011in}{0.851907in}}%
\pgfpathlineto{\pgfqpoint{4.013625in}{0.851907in}}%
\pgfpathlineto{\pgfqpoint{4.013625in}{0.499444in}}%
\pgfpathclose%
\pgfusepath{fill}%
\end{pgfscope}%
\begin{pgfscope}%
\pgfpathrectangle{\pgfqpoint{0.445556in}{0.499444in}}{\pgfqpoint{3.875000in}{1.155000in}}%
\pgfusepath{clip}%
\pgfsetbuttcap%
\pgfsetmiterjoin%
\definecolor{currentfill}{rgb}{0.000000,0.000000,0.000000}%
\pgfsetfillcolor{currentfill}%
\pgfsetlinewidth{0.000000pt}%
\definecolor{currentstroke}{rgb}{0.000000,0.000000,0.000000}%
\pgfsetstrokecolor{currentstroke}%
\pgfsetstrokeopacity{0.000000}%
\pgfsetdash{}{0pt}%
\pgfpathmoveto{\pgfqpoint{4.167090in}{0.499444in}}%
\pgfpathlineto{\pgfqpoint{4.228476in}{0.499444in}}%
\pgfpathlineto{\pgfqpoint{4.228476in}{0.681583in}}%
\pgfpathlineto{\pgfqpoint{4.167090in}{0.681583in}}%
\pgfpathlineto{\pgfqpoint{4.167090in}{0.499444in}}%
\pgfpathclose%
\pgfusepath{fill}%
\end{pgfscope}%
\begin{pgfscope}%
\pgfsetbuttcap%
\pgfsetroundjoin%
\definecolor{currentfill}{rgb}{0.000000,0.000000,0.000000}%
\pgfsetfillcolor{currentfill}%
\pgfsetlinewidth{0.803000pt}%
\definecolor{currentstroke}{rgb}{0.000000,0.000000,0.000000}%
\pgfsetstrokecolor{currentstroke}%
\pgfsetdash{}{0pt}%
\pgfsys@defobject{currentmarker}{\pgfqpoint{0.000000in}{-0.048611in}}{\pgfqpoint{0.000000in}{0.000000in}}{%
\pgfpathmoveto{\pgfqpoint{0.000000in}{0.000000in}}%
\pgfpathlineto{\pgfqpoint{0.000000in}{-0.048611in}}%
\pgfusepath{stroke,fill}%
}%
\begin{pgfscope}%
\pgfsys@transformshift{0.483922in}{0.499444in}%
\pgfsys@useobject{currentmarker}{}%
\end{pgfscope}%
\end{pgfscope}%
\begin{pgfscope}%
\definecolor{textcolor}{rgb}{0.000000,0.000000,0.000000}%
\pgfsetstrokecolor{textcolor}%
\pgfsetfillcolor{textcolor}%
\pgftext[x=0.483922in,y=0.402222in,,top]{\color{textcolor}\rmfamily\fontsize{10.000000}{12.000000}\selectfont 0.0}%
\end{pgfscope}%
\begin{pgfscope}%
\pgfsetbuttcap%
\pgfsetroundjoin%
\definecolor{currentfill}{rgb}{0.000000,0.000000,0.000000}%
\pgfsetfillcolor{currentfill}%
\pgfsetlinewidth{0.803000pt}%
\definecolor{currentstroke}{rgb}{0.000000,0.000000,0.000000}%
\pgfsetstrokecolor{currentstroke}%
\pgfsetdash{}{0pt}%
\pgfsys@defobject{currentmarker}{\pgfqpoint{0.000000in}{-0.048611in}}{\pgfqpoint{0.000000in}{0.000000in}}{%
\pgfpathmoveto{\pgfqpoint{0.000000in}{0.000000in}}%
\pgfpathlineto{\pgfqpoint{0.000000in}{-0.048611in}}%
\pgfusepath{stroke,fill}%
}%
\begin{pgfscope}%
\pgfsys@transformshift{0.867585in}{0.499444in}%
\pgfsys@useobject{currentmarker}{}%
\end{pgfscope}%
\end{pgfscope}%
\begin{pgfscope}%
\definecolor{textcolor}{rgb}{0.000000,0.000000,0.000000}%
\pgfsetstrokecolor{textcolor}%
\pgfsetfillcolor{textcolor}%
\pgftext[x=0.867585in,y=0.402222in,,top]{\color{textcolor}\rmfamily\fontsize{10.000000}{12.000000}\selectfont 0.1}%
\end{pgfscope}%
\begin{pgfscope}%
\pgfsetbuttcap%
\pgfsetroundjoin%
\definecolor{currentfill}{rgb}{0.000000,0.000000,0.000000}%
\pgfsetfillcolor{currentfill}%
\pgfsetlinewidth{0.803000pt}%
\definecolor{currentstroke}{rgb}{0.000000,0.000000,0.000000}%
\pgfsetstrokecolor{currentstroke}%
\pgfsetdash{}{0pt}%
\pgfsys@defobject{currentmarker}{\pgfqpoint{0.000000in}{-0.048611in}}{\pgfqpoint{0.000000in}{0.000000in}}{%
\pgfpathmoveto{\pgfqpoint{0.000000in}{0.000000in}}%
\pgfpathlineto{\pgfqpoint{0.000000in}{-0.048611in}}%
\pgfusepath{stroke,fill}%
}%
\begin{pgfscope}%
\pgfsys@transformshift{1.251249in}{0.499444in}%
\pgfsys@useobject{currentmarker}{}%
\end{pgfscope}%
\end{pgfscope}%
\begin{pgfscope}%
\definecolor{textcolor}{rgb}{0.000000,0.000000,0.000000}%
\pgfsetstrokecolor{textcolor}%
\pgfsetfillcolor{textcolor}%
\pgftext[x=1.251249in,y=0.402222in,,top]{\color{textcolor}\rmfamily\fontsize{10.000000}{12.000000}\selectfont 0.2}%
\end{pgfscope}%
\begin{pgfscope}%
\pgfsetbuttcap%
\pgfsetroundjoin%
\definecolor{currentfill}{rgb}{0.000000,0.000000,0.000000}%
\pgfsetfillcolor{currentfill}%
\pgfsetlinewidth{0.803000pt}%
\definecolor{currentstroke}{rgb}{0.000000,0.000000,0.000000}%
\pgfsetstrokecolor{currentstroke}%
\pgfsetdash{}{0pt}%
\pgfsys@defobject{currentmarker}{\pgfqpoint{0.000000in}{-0.048611in}}{\pgfqpoint{0.000000in}{0.000000in}}{%
\pgfpathmoveto{\pgfqpoint{0.000000in}{0.000000in}}%
\pgfpathlineto{\pgfqpoint{0.000000in}{-0.048611in}}%
\pgfusepath{stroke,fill}%
}%
\begin{pgfscope}%
\pgfsys@transformshift{1.634912in}{0.499444in}%
\pgfsys@useobject{currentmarker}{}%
\end{pgfscope}%
\end{pgfscope}%
\begin{pgfscope}%
\definecolor{textcolor}{rgb}{0.000000,0.000000,0.000000}%
\pgfsetstrokecolor{textcolor}%
\pgfsetfillcolor{textcolor}%
\pgftext[x=1.634912in,y=0.402222in,,top]{\color{textcolor}\rmfamily\fontsize{10.000000}{12.000000}\selectfont 0.3}%
\end{pgfscope}%
\begin{pgfscope}%
\pgfsetbuttcap%
\pgfsetroundjoin%
\definecolor{currentfill}{rgb}{0.000000,0.000000,0.000000}%
\pgfsetfillcolor{currentfill}%
\pgfsetlinewidth{0.803000pt}%
\definecolor{currentstroke}{rgb}{0.000000,0.000000,0.000000}%
\pgfsetstrokecolor{currentstroke}%
\pgfsetdash{}{0pt}%
\pgfsys@defobject{currentmarker}{\pgfqpoint{0.000000in}{-0.048611in}}{\pgfqpoint{0.000000in}{0.000000in}}{%
\pgfpathmoveto{\pgfqpoint{0.000000in}{0.000000in}}%
\pgfpathlineto{\pgfqpoint{0.000000in}{-0.048611in}}%
\pgfusepath{stroke,fill}%
}%
\begin{pgfscope}%
\pgfsys@transformshift{2.018575in}{0.499444in}%
\pgfsys@useobject{currentmarker}{}%
\end{pgfscope}%
\end{pgfscope}%
\begin{pgfscope}%
\definecolor{textcolor}{rgb}{0.000000,0.000000,0.000000}%
\pgfsetstrokecolor{textcolor}%
\pgfsetfillcolor{textcolor}%
\pgftext[x=2.018575in,y=0.402222in,,top]{\color{textcolor}\rmfamily\fontsize{10.000000}{12.000000}\selectfont 0.4}%
\end{pgfscope}%
\begin{pgfscope}%
\pgfsetbuttcap%
\pgfsetroundjoin%
\definecolor{currentfill}{rgb}{0.000000,0.000000,0.000000}%
\pgfsetfillcolor{currentfill}%
\pgfsetlinewidth{0.803000pt}%
\definecolor{currentstroke}{rgb}{0.000000,0.000000,0.000000}%
\pgfsetstrokecolor{currentstroke}%
\pgfsetdash{}{0pt}%
\pgfsys@defobject{currentmarker}{\pgfqpoint{0.000000in}{-0.048611in}}{\pgfqpoint{0.000000in}{0.000000in}}{%
\pgfpathmoveto{\pgfqpoint{0.000000in}{0.000000in}}%
\pgfpathlineto{\pgfqpoint{0.000000in}{-0.048611in}}%
\pgfusepath{stroke,fill}%
}%
\begin{pgfscope}%
\pgfsys@transformshift{2.402239in}{0.499444in}%
\pgfsys@useobject{currentmarker}{}%
\end{pgfscope}%
\end{pgfscope}%
\begin{pgfscope}%
\definecolor{textcolor}{rgb}{0.000000,0.000000,0.000000}%
\pgfsetstrokecolor{textcolor}%
\pgfsetfillcolor{textcolor}%
\pgftext[x=2.402239in,y=0.402222in,,top]{\color{textcolor}\rmfamily\fontsize{10.000000}{12.000000}\selectfont 0.5}%
\end{pgfscope}%
\begin{pgfscope}%
\pgfsetbuttcap%
\pgfsetroundjoin%
\definecolor{currentfill}{rgb}{0.000000,0.000000,0.000000}%
\pgfsetfillcolor{currentfill}%
\pgfsetlinewidth{0.803000pt}%
\definecolor{currentstroke}{rgb}{0.000000,0.000000,0.000000}%
\pgfsetstrokecolor{currentstroke}%
\pgfsetdash{}{0pt}%
\pgfsys@defobject{currentmarker}{\pgfqpoint{0.000000in}{-0.048611in}}{\pgfqpoint{0.000000in}{0.000000in}}{%
\pgfpathmoveto{\pgfqpoint{0.000000in}{0.000000in}}%
\pgfpathlineto{\pgfqpoint{0.000000in}{-0.048611in}}%
\pgfusepath{stroke,fill}%
}%
\begin{pgfscope}%
\pgfsys@transformshift{2.785902in}{0.499444in}%
\pgfsys@useobject{currentmarker}{}%
\end{pgfscope}%
\end{pgfscope}%
\begin{pgfscope}%
\definecolor{textcolor}{rgb}{0.000000,0.000000,0.000000}%
\pgfsetstrokecolor{textcolor}%
\pgfsetfillcolor{textcolor}%
\pgftext[x=2.785902in,y=0.402222in,,top]{\color{textcolor}\rmfamily\fontsize{10.000000}{12.000000}\selectfont 0.6}%
\end{pgfscope}%
\begin{pgfscope}%
\pgfsetbuttcap%
\pgfsetroundjoin%
\definecolor{currentfill}{rgb}{0.000000,0.000000,0.000000}%
\pgfsetfillcolor{currentfill}%
\pgfsetlinewidth{0.803000pt}%
\definecolor{currentstroke}{rgb}{0.000000,0.000000,0.000000}%
\pgfsetstrokecolor{currentstroke}%
\pgfsetdash{}{0pt}%
\pgfsys@defobject{currentmarker}{\pgfqpoint{0.000000in}{-0.048611in}}{\pgfqpoint{0.000000in}{0.000000in}}{%
\pgfpathmoveto{\pgfqpoint{0.000000in}{0.000000in}}%
\pgfpathlineto{\pgfqpoint{0.000000in}{-0.048611in}}%
\pgfusepath{stroke,fill}%
}%
\begin{pgfscope}%
\pgfsys@transformshift{3.169566in}{0.499444in}%
\pgfsys@useobject{currentmarker}{}%
\end{pgfscope}%
\end{pgfscope}%
\begin{pgfscope}%
\definecolor{textcolor}{rgb}{0.000000,0.000000,0.000000}%
\pgfsetstrokecolor{textcolor}%
\pgfsetfillcolor{textcolor}%
\pgftext[x=3.169566in,y=0.402222in,,top]{\color{textcolor}\rmfamily\fontsize{10.000000}{12.000000}\selectfont 0.7}%
\end{pgfscope}%
\begin{pgfscope}%
\pgfsetbuttcap%
\pgfsetroundjoin%
\definecolor{currentfill}{rgb}{0.000000,0.000000,0.000000}%
\pgfsetfillcolor{currentfill}%
\pgfsetlinewidth{0.803000pt}%
\definecolor{currentstroke}{rgb}{0.000000,0.000000,0.000000}%
\pgfsetstrokecolor{currentstroke}%
\pgfsetdash{}{0pt}%
\pgfsys@defobject{currentmarker}{\pgfqpoint{0.000000in}{-0.048611in}}{\pgfqpoint{0.000000in}{0.000000in}}{%
\pgfpathmoveto{\pgfqpoint{0.000000in}{0.000000in}}%
\pgfpathlineto{\pgfqpoint{0.000000in}{-0.048611in}}%
\pgfusepath{stroke,fill}%
}%
\begin{pgfscope}%
\pgfsys@transformshift{3.553229in}{0.499444in}%
\pgfsys@useobject{currentmarker}{}%
\end{pgfscope}%
\end{pgfscope}%
\begin{pgfscope}%
\definecolor{textcolor}{rgb}{0.000000,0.000000,0.000000}%
\pgfsetstrokecolor{textcolor}%
\pgfsetfillcolor{textcolor}%
\pgftext[x=3.553229in,y=0.402222in,,top]{\color{textcolor}\rmfamily\fontsize{10.000000}{12.000000}\selectfont 0.8}%
\end{pgfscope}%
\begin{pgfscope}%
\pgfsetbuttcap%
\pgfsetroundjoin%
\definecolor{currentfill}{rgb}{0.000000,0.000000,0.000000}%
\pgfsetfillcolor{currentfill}%
\pgfsetlinewidth{0.803000pt}%
\definecolor{currentstroke}{rgb}{0.000000,0.000000,0.000000}%
\pgfsetstrokecolor{currentstroke}%
\pgfsetdash{}{0pt}%
\pgfsys@defobject{currentmarker}{\pgfqpoint{0.000000in}{-0.048611in}}{\pgfqpoint{0.000000in}{0.000000in}}{%
\pgfpathmoveto{\pgfqpoint{0.000000in}{0.000000in}}%
\pgfpathlineto{\pgfqpoint{0.000000in}{-0.048611in}}%
\pgfusepath{stroke,fill}%
}%
\begin{pgfscope}%
\pgfsys@transformshift{3.936892in}{0.499444in}%
\pgfsys@useobject{currentmarker}{}%
\end{pgfscope}%
\end{pgfscope}%
\begin{pgfscope}%
\definecolor{textcolor}{rgb}{0.000000,0.000000,0.000000}%
\pgfsetstrokecolor{textcolor}%
\pgfsetfillcolor{textcolor}%
\pgftext[x=3.936892in,y=0.402222in,,top]{\color{textcolor}\rmfamily\fontsize{10.000000}{12.000000}\selectfont 0.9}%
\end{pgfscope}%
\begin{pgfscope}%
\pgfsetbuttcap%
\pgfsetroundjoin%
\definecolor{currentfill}{rgb}{0.000000,0.000000,0.000000}%
\pgfsetfillcolor{currentfill}%
\pgfsetlinewidth{0.803000pt}%
\definecolor{currentstroke}{rgb}{0.000000,0.000000,0.000000}%
\pgfsetstrokecolor{currentstroke}%
\pgfsetdash{}{0pt}%
\pgfsys@defobject{currentmarker}{\pgfqpoint{0.000000in}{-0.048611in}}{\pgfqpoint{0.000000in}{0.000000in}}{%
\pgfpathmoveto{\pgfqpoint{0.000000in}{0.000000in}}%
\pgfpathlineto{\pgfqpoint{0.000000in}{-0.048611in}}%
\pgfusepath{stroke,fill}%
}%
\begin{pgfscope}%
\pgfsys@transformshift{4.320556in}{0.499444in}%
\pgfsys@useobject{currentmarker}{}%
\end{pgfscope}%
\end{pgfscope}%
\begin{pgfscope}%
\definecolor{textcolor}{rgb}{0.000000,0.000000,0.000000}%
\pgfsetstrokecolor{textcolor}%
\pgfsetfillcolor{textcolor}%
\pgftext[x=4.320556in,y=0.402222in,,top]{\color{textcolor}\rmfamily\fontsize{10.000000}{12.000000}\selectfont 1.0}%
\end{pgfscope}%
\begin{pgfscope}%
\definecolor{textcolor}{rgb}{0.000000,0.000000,0.000000}%
\pgfsetstrokecolor{textcolor}%
\pgfsetfillcolor{textcolor}%
\pgftext[x=2.383056in,y=0.223333in,,top]{\color{textcolor}\rmfamily\fontsize{10.000000}{12.000000}\selectfont \(\displaystyle p\)}%
\end{pgfscope}%
\begin{pgfscope}%
\pgfsetbuttcap%
\pgfsetroundjoin%
\definecolor{currentfill}{rgb}{0.000000,0.000000,0.000000}%
\pgfsetfillcolor{currentfill}%
\pgfsetlinewidth{0.803000pt}%
\definecolor{currentstroke}{rgb}{0.000000,0.000000,0.000000}%
\pgfsetstrokecolor{currentstroke}%
\pgfsetdash{}{0pt}%
\pgfsys@defobject{currentmarker}{\pgfqpoint{-0.048611in}{0.000000in}}{\pgfqpoint{-0.000000in}{0.000000in}}{%
\pgfpathmoveto{\pgfqpoint{-0.000000in}{0.000000in}}%
\pgfpathlineto{\pgfqpoint{-0.048611in}{0.000000in}}%
\pgfusepath{stroke,fill}%
}%
\begin{pgfscope}%
\pgfsys@transformshift{0.445556in}{0.499444in}%
\pgfsys@useobject{currentmarker}{}%
\end{pgfscope}%
\end{pgfscope}%
\begin{pgfscope}%
\definecolor{textcolor}{rgb}{0.000000,0.000000,0.000000}%
\pgfsetstrokecolor{textcolor}%
\pgfsetfillcolor{textcolor}%
\pgftext[x=0.278889in, y=0.451250in, left, base]{\color{textcolor}\rmfamily\fontsize{10.000000}{12.000000}\selectfont \(\displaystyle {0}\)}%
\end{pgfscope}%
\begin{pgfscope}%
\pgfsetbuttcap%
\pgfsetroundjoin%
\definecolor{currentfill}{rgb}{0.000000,0.000000,0.000000}%
\pgfsetfillcolor{currentfill}%
\pgfsetlinewidth{0.803000pt}%
\definecolor{currentstroke}{rgb}{0.000000,0.000000,0.000000}%
\pgfsetstrokecolor{currentstroke}%
\pgfsetdash{}{0pt}%
\pgfsys@defobject{currentmarker}{\pgfqpoint{-0.048611in}{0.000000in}}{\pgfqpoint{-0.000000in}{0.000000in}}{%
\pgfpathmoveto{\pgfqpoint{-0.000000in}{0.000000in}}%
\pgfpathlineto{\pgfqpoint{-0.048611in}{0.000000in}}%
\pgfusepath{stroke,fill}%
}%
\begin{pgfscope}%
\pgfsys@transformshift{0.445556in}{0.791601in}%
\pgfsys@useobject{currentmarker}{}%
\end{pgfscope}%
\end{pgfscope}%
\begin{pgfscope}%
\definecolor{textcolor}{rgb}{0.000000,0.000000,0.000000}%
\pgfsetstrokecolor{textcolor}%
\pgfsetfillcolor{textcolor}%
\pgftext[x=0.278889in, y=0.743407in, left, base]{\color{textcolor}\rmfamily\fontsize{10.000000}{12.000000}\selectfont \(\displaystyle {2}\)}%
\end{pgfscope}%
\begin{pgfscope}%
\pgfsetbuttcap%
\pgfsetroundjoin%
\definecolor{currentfill}{rgb}{0.000000,0.000000,0.000000}%
\pgfsetfillcolor{currentfill}%
\pgfsetlinewidth{0.803000pt}%
\definecolor{currentstroke}{rgb}{0.000000,0.000000,0.000000}%
\pgfsetstrokecolor{currentstroke}%
\pgfsetdash{}{0pt}%
\pgfsys@defobject{currentmarker}{\pgfqpoint{-0.048611in}{0.000000in}}{\pgfqpoint{-0.000000in}{0.000000in}}{%
\pgfpathmoveto{\pgfqpoint{-0.000000in}{0.000000in}}%
\pgfpathlineto{\pgfqpoint{-0.048611in}{0.000000in}}%
\pgfusepath{stroke,fill}%
}%
\begin{pgfscope}%
\pgfsys@transformshift{0.445556in}{1.083759in}%
\pgfsys@useobject{currentmarker}{}%
\end{pgfscope}%
\end{pgfscope}%
\begin{pgfscope}%
\definecolor{textcolor}{rgb}{0.000000,0.000000,0.000000}%
\pgfsetstrokecolor{textcolor}%
\pgfsetfillcolor{textcolor}%
\pgftext[x=0.278889in, y=1.035564in, left, base]{\color{textcolor}\rmfamily\fontsize{10.000000}{12.000000}\selectfont \(\displaystyle {4}\)}%
\end{pgfscope}%
\begin{pgfscope}%
\pgfsetbuttcap%
\pgfsetroundjoin%
\definecolor{currentfill}{rgb}{0.000000,0.000000,0.000000}%
\pgfsetfillcolor{currentfill}%
\pgfsetlinewidth{0.803000pt}%
\definecolor{currentstroke}{rgb}{0.000000,0.000000,0.000000}%
\pgfsetstrokecolor{currentstroke}%
\pgfsetdash{}{0pt}%
\pgfsys@defobject{currentmarker}{\pgfqpoint{-0.048611in}{0.000000in}}{\pgfqpoint{-0.000000in}{0.000000in}}{%
\pgfpathmoveto{\pgfqpoint{-0.000000in}{0.000000in}}%
\pgfpathlineto{\pgfqpoint{-0.048611in}{0.000000in}}%
\pgfusepath{stroke,fill}%
}%
\begin{pgfscope}%
\pgfsys@transformshift{0.445556in}{1.375916in}%
\pgfsys@useobject{currentmarker}{}%
\end{pgfscope}%
\end{pgfscope}%
\begin{pgfscope}%
\definecolor{textcolor}{rgb}{0.000000,0.000000,0.000000}%
\pgfsetstrokecolor{textcolor}%
\pgfsetfillcolor{textcolor}%
\pgftext[x=0.278889in, y=1.327722in, left, base]{\color{textcolor}\rmfamily\fontsize{10.000000}{12.000000}\selectfont \(\displaystyle {6}\)}%
\end{pgfscope}%
\begin{pgfscope}%
\definecolor{textcolor}{rgb}{0.000000,0.000000,0.000000}%
\pgfsetstrokecolor{textcolor}%
\pgfsetfillcolor{textcolor}%
\pgftext[x=0.223333in,y=1.076944in,,bottom,rotate=90.000000]{\color{textcolor}\rmfamily\fontsize{10.000000}{12.000000}\selectfont Percent of Data Set}%
\end{pgfscope}%
\begin{pgfscope}%
\pgfsetrectcap%
\pgfsetmiterjoin%
\pgfsetlinewidth{0.803000pt}%
\definecolor{currentstroke}{rgb}{0.000000,0.000000,0.000000}%
\pgfsetstrokecolor{currentstroke}%
\pgfsetdash{}{0pt}%
\pgfpathmoveto{\pgfqpoint{0.445556in}{0.499444in}}%
\pgfpathlineto{\pgfqpoint{0.445556in}{1.654444in}}%
\pgfusepath{stroke}%
\end{pgfscope}%
\begin{pgfscope}%
\pgfsetrectcap%
\pgfsetmiterjoin%
\pgfsetlinewidth{0.803000pt}%
\definecolor{currentstroke}{rgb}{0.000000,0.000000,0.000000}%
\pgfsetstrokecolor{currentstroke}%
\pgfsetdash{}{0pt}%
\pgfpathmoveto{\pgfqpoint{4.320556in}{0.499444in}}%
\pgfpathlineto{\pgfqpoint{4.320556in}{1.654444in}}%
\pgfusepath{stroke}%
\end{pgfscope}%
\begin{pgfscope}%
\pgfsetrectcap%
\pgfsetmiterjoin%
\pgfsetlinewidth{0.803000pt}%
\definecolor{currentstroke}{rgb}{0.000000,0.000000,0.000000}%
\pgfsetstrokecolor{currentstroke}%
\pgfsetdash{}{0pt}%
\pgfpathmoveto{\pgfqpoint{0.445556in}{0.499444in}}%
\pgfpathlineto{\pgfqpoint{4.320556in}{0.499444in}}%
\pgfusepath{stroke}%
\end{pgfscope}%
\begin{pgfscope}%
\pgfsetrectcap%
\pgfsetmiterjoin%
\pgfsetlinewidth{0.803000pt}%
\definecolor{currentstroke}{rgb}{0.000000,0.000000,0.000000}%
\pgfsetstrokecolor{currentstroke}%
\pgfsetdash{}{0pt}%
\pgfpathmoveto{\pgfqpoint{0.445556in}{1.654444in}}%
\pgfpathlineto{\pgfqpoint{4.320556in}{1.654444in}}%
\pgfusepath{stroke}%
\end{pgfscope}%
\begin{pgfscope}%
\pgfsetbuttcap%
\pgfsetmiterjoin%
\definecolor{currentfill}{rgb}{1.000000,1.000000,1.000000}%
\pgfsetfillcolor{currentfill}%
\pgfsetfillopacity{0.800000}%
\pgfsetlinewidth{1.003750pt}%
\definecolor{currentstroke}{rgb}{0.800000,0.800000,0.800000}%
\pgfsetstrokecolor{currentstroke}%
\pgfsetstrokeopacity{0.800000}%
\pgfsetdash{}{0pt}%
\pgfpathmoveto{\pgfqpoint{3.543611in}{1.154445in}}%
\pgfpathlineto{\pgfqpoint{4.223333in}{1.154445in}}%
\pgfpathquadraticcurveto{\pgfqpoint{4.251111in}{1.154445in}}{\pgfqpoint{4.251111in}{1.182222in}}%
\pgfpathlineto{\pgfqpoint{4.251111in}{1.557222in}}%
\pgfpathquadraticcurveto{\pgfqpoint{4.251111in}{1.585000in}}{\pgfqpoint{4.223333in}{1.585000in}}%
\pgfpathlineto{\pgfqpoint{3.543611in}{1.585000in}}%
\pgfpathquadraticcurveto{\pgfqpoint{3.515833in}{1.585000in}}{\pgfqpoint{3.515833in}{1.557222in}}%
\pgfpathlineto{\pgfqpoint{3.515833in}{1.182222in}}%
\pgfpathquadraticcurveto{\pgfqpoint{3.515833in}{1.154445in}}{\pgfqpoint{3.543611in}{1.154445in}}%
\pgfpathlineto{\pgfqpoint{3.543611in}{1.154445in}}%
\pgfpathclose%
\pgfusepath{stroke,fill}%
\end{pgfscope}%
\begin{pgfscope}%
\pgfsetbuttcap%
\pgfsetmiterjoin%
\pgfsetlinewidth{1.003750pt}%
\definecolor{currentstroke}{rgb}{0.000000,0.000000,0.000000}%
\pgfsetstrokecolor{currentstroke}%
\pgfsetdash{}{0pt}%
\pgfpathmoveto{\pgfqpoint{3.571389in}{1.432222in}}%
\pgfpathlineto{\pgfqpoint{3.849167in}{1.432222in}}%
\pgfpathlineto{\pgfqpoint{3.849167in}{1.529444in}}%
\pgfpathlineto{\pgfqpoint{3.571389in}{1.529444in}}%
\pgfpathlineto{\pgfqpoint{3.571389in}{1.432222in}}%
\pgfpathclose%
\pgfusepath{stroke}%
\end{pgfscope}%
\begin{pgfscope}%
\definecolor{textcolor}{rgb}{0.000000,0.000000,0.000000}%
\pgfsetstrokecolor{textcolor}%
\pgfsetfillcolor{textcolor}%
\pgftext[x=3.960278in,y=1.432222in,left,base]{\color{textcolor}\rmfamily\fontsize{10.000000}{12.000000}\selectfont Neg}%
\end{pgfscope}%
\begin{pgfscope}%
\pgfsetbuttcap%
\pgfsetmiterjoin%
\definecolor{currentfill}{rgb}{0.000000,0.000000,0.000000}%
\pgfsetfillcolor{currentfill}%
\pgfsetlinewidth{0.000000pt}%
\definecolor{currentstroke}{rgb}{0.000000,0.000000,0.000000}%
\pgfsetstrokecolor{currentstroke}%
\pgfsetstrokeopacity{0.000000}%
\pgfsetdash{}{0pt}%
\pgfpathmoveto{\pgfqpoint{3.571389in}{1.236944in}}%
\pgfpathlineto{\pgfqpoint{3.849167in}{1.236944in}}%
\pgfpathlineto{\pgfqpoint{3.849167in}{1.334167in}}%
\pgfpathlineto{\pgfqpoint{3.571389in}{1.334167in}}%
\pgfpathlineto{\pgfqpoint{3.571389in}{1.236944in}}%
\pgfpathclose%
\pgfusepath{fill}%
\end{pgfscope}%
\begin{pgfscope}%
\definecolor{textcolor}{rgb}{0.000000,0.000000,0.000000}%
\pgfsetstrokecolor{textcolor}%
\pgfsetfillcolor{textcolor}%
\pgftext[x=3.960278in,y=1.236944in,left,base]{\color{textcolor}\rmfamily\fontsize{10.000000}{12.000000}\selectfont Pos}%
\end{pgfscope}%
\end{pgfpicture}%
\makeatother%
\endgroup%

&
	\vskip 0pt
	\qquad \qquad ROC Curve
	
	%% Creator: Matplotlib, PGF backend
%%
%% To include the figure in your LaTeX document, write
%%   \input{<filename>.pgf}
%%
%% Make sure the required packages are loaded in your preamble
%%   \usepackage{pgf}
%%
%% Also ensure that all the required font packages are loaded; for instance,
%% the lmodern package is sometimes necessary when using math font.
%%   \usepackage{lmodern}
%%
%% Figures using additional raster images can only be included by \input if
%% they are in the same directory as the main LaTeX file. For loading figures
%% from other directories you can use the `import` package
%%   \usepackage{import}
%%
%% and then include the figures with
%%   \import{<path to file>}{<filename>.pgf}
%%
%% Matplotlib used the following preamble
%%   
%%   \usepackage{fontspec}
%%   \makeatletter\@ifpackageloaded{underscore}{}{\usepackage[strings]{underscore}}\makeatother
%%
\begingroup%
\makeatletter%
\begin{pgfpicture}%
\pgfpathrectangle{\pgfpointorigin}{\pgfqpoint{2.221861in}{1.754444in}}%
\pgfusepath{use as bounding box, clip}%
\begin{pgfscope}%
\pgfsetbuttcap%
\pgfsetmiterjoin%
\definecolor{currentfill}{rgb}{1.000000,1.000000,1.000000}%
\pgfsetfillcolor{currentfill}%
\pgfsetlinewidth{0.000000pt}%
\definecolor{currentstroke}{rgb}{1.000000,1.000000,1.000000}%
\pgfsetstrokecolor{currentstroke}%
\pgfsetdash{}{0pt}%
\pgfpathmoveto{\pgfqpoint{0.000000in}{0.000000in}}%
\pgfpathlineto{\pgfqpoint{2.221861in}{0.000000in}}%
\pgfpathlineto{\pgfqpoint{2.221861in}{1.754444in}}%
\pgfpathlineto{\pgfqpoint{0.000000in}{1.754444in}}%
\pgfpathlineto{\pgfqpoint{0.000000in}{0.000000in}}%
\pgfpathclose%
\pgfusepath{fill}%
\end{pgfscope}%
\begin{pgfscope}%
\pgfsetbuttcap%
\pgfsetmiterjoin%
\definecolor{currentfill}{rgb}{1.000000,1.000000,1.000000}%
\pgfsetfillcolor{currentfill}%
\pgfsetlinewidth{0.000000pt}%
\definecolor{currentstroke}{rgb}{0.000000,0.000000,0.000000}%
\pgfsetstrokecolor{currentstroke}%
\pgfsetstrokeopacity{0.000000}%
\pgfsetdash{}{0pt}%
\pgfpathmoveto{\pgfqpoint{0.553581in}{0.499444in}}%
\pgfpathlineto{\pgfqpoint{2.103581in}{0.499444in}}%
\pgfpathlineto{\pgfqpoint{2.103581in}{1.654444in}}%
\pgfpathlineto{\pgfqpoint{0.553581in}{1.654444in}}%
\pgfpathlineto{\pgfqpoint{0.553581in}{0.499444in}}%
\pgfpathclose%
\pgfusepath{fill}%
\end{pgfscope}%
\begin{pgfscope}%
\pgfsetbuttcap%
\pgfsetroundjoin%
\definecolor{currentfill}{rgb}{0.000000,0.000000,0.000000}%
\pgfsetfillcolor{currentfill}%
\pgfsetlinewidth{0.803000pt}%
\definecolor{currentstroke}{rgb}{0.000000,0.000000,0.000000}%
\pgfsetstrokecolor{currentstroke}%
\pgfsetdash{}{0pt}%
\pgfsys@defobject{currentmarker}{\pgfqpoint{0.000000in}{-0.048611in}}{\pgfqpoint{0.000000in}{0.000000in}}{%
\pgfpathmoveto{\pgfqpoint{0.000000in}{0.000000in}}%
\pgfpathlineto{\pgfqpoint{0.000000in}{-0.048611in}}%
\pgfusepath{stroke,fill}%
}%
\begin{pgfscope}%
\pgfsys@transformshift{0.624035in}{0.499444in}%
\pgfsys@useobject{currentmarker}{}%
\end{pgfscope}%
\end{pgfscope}%
\begin{pgfscope}%
\definecolor{textcolor}{rgb}{0.000000,0.000000,0.000000}%
\pgfsetstrokecolor{textcolor}%
\pgfsetfillcolor{textcolor}%
\pgftext[x=0.624035in,y=0.402222in,,top]{\color{textcolor}\rmfamily\fontsize{10.000000}{12.000000}\selectfont \(\displaystyle {0.0}\)}%
\end{pgfscope}%
\begin{pgfscope}%
\pgfsetbuttcap%
\pgfsetroundjoin%
\definecolor{currentfill}{rgb}{0.000000,0.000000,0.000000}%
\pgfsetfillcolor{currentfill}%
\pgfsetlinewidth{0.803000pt}%
\definecolor{currentstroke}{rgb}{0.000000,0.000000,0.000000}%
\pgfsetstrokecolor{currentstroke}%
\pgfsetdash{}{0pt}%
\pgfsys@defobject{currentmarker}{\pgfqpoint{0.000000in}{-0.048611in}}{\pgfqpoint{0.000000in}{0.000000in}}{%
\pgfpathmoveto{\pgfqpoint{0.000000in}{0.000000in}}%
\pgfpathlineto{\pgfqpoint{0.000000in}{-0.048611in}}%
\pgfusepath{stroke,fill}%
}%
\begin{pgfscope}%
\pgfsys@transformshift{1.328581in}{0.499444in}%
\pgfsys@useobject{currentmarker}{}%
\end{pgfscope}%
\end{pgfscope}%
\begin{pgfscope}%
\definecolor{textcolor}{rgb}{0.000000,0.000000,0.000000}%
\pgfsetstrokecolor{textcolor}%
\pgfsetfillcolor{textcolor}%
\pgftext[x=1.328581in,y=0.402222in,,top]{\color{textcolor}\rmfamily\fontsize{10.000000}{12.000000}\selectfont \(\displaystyle {0.5}\)}%
\end{pgfscope}%
\begin{pgfscope}%
\pgfsetbuttcap%
\pgfsetroundjoin%
\definecolor{currentfill}{rgb}{0.000000,0.000000,0.000000}%
\pgfsetfillcolor{currentfill}%
\pgfsetlinewidth{0.803000pt}%
\definecolor{currentstroke}{rgb}{0.000000,0.000000,0.000000}%
\pgfsetstrokecolor{currentstroke}%
\pgfsetdash{}{0pt}%
\pgfsys@defobject{currentmarker}{\pgfqpoint{0.000000in}{-0.048611in}}{\pgfqpoint{0.000000in}{0.000000in}}{%
\pgfpathmoveto{\pgfqpoint{0.000000in}{0.000000in}}%
\pgfpathlineto{\pgfqpoint{0.000000in}{-0.048611in}}%
\pgfusepath{stroke,fill}%
}%
\begin{pgfscope}%
\pgfsys@transformshift{2.033126in}{0.499444in}%
\pgfsys@useobject{currentmarker}{}%
\end{pgfscope}%
\end{pgfscope}%
\begin{pgfscope}%
\definecolor{textcolor}{rgb}{0.000000,0.000000,0.000000}%
\pgfsetstrokecolor{textcolor}%
\pgfsetfillcolor{textcolor}%
\pgftext[x=2.033126in,y=0.402222in,,top]{\color{textcolor}\rmfamily\fontsize{10.000000}{12.000000}\selectfont \(\displaystyle {1.0}\)}%
\end{pgfscope}%
\begin{pgfscope}%
\definecolor{textcolor}{rgb}{0.000000,0.000000,0.000000}%
\pgfsetstrokecolor{textcolor}%
\pgfsetfillcolor{textcolor}%
\pgftext[x=1.328581in,y=0.223333in,,top]{\color{textcolor}\rmfamily\fontsize{10.000000}{12.000000}\selectfont False positive rate}%
\end{pgfscope}%
\begin{pgfscope}%
\pgfsetbuttcap%
\pgfsetroundjoin%
\definecolor{currentfill}{rgb}{0.000000,0.000000,0.000000}%
\pgfsetfillcolor{currentfill}%
\pgfsetlinewidth{0.803000pt}%
\definecolor{currentstroke}{rgb}{0.000000,0.000000,0.000000}%
\pgfsetstrokecolor{currentstroke}%
\pgfsetdash{}{0pt}%
\pgfsys@defobject{currentmarker}{\pgfqpoint{-0.048611in}{0.000000in}}{\pgfqpoint{-0.000000in}{0.000000in}}{%
\pgfpathmoveto{\pgfqpoint{-0.000000in}{0.000000in}}%
\pgfpathlineto{\pgfqpoint{-0.048611in}{0.000000in}}%
\pgfusepath{stroke,fill}%
}%
\begin{pgfscope}%
\pgfsys@transformshift{0.553581in}{0.551944in}%
\pgfsys@useobject{currentmarker}{}%
\end{pgfscope}%
\end{pgfscope}%
\begin{pgfscope}%
\definecolor{textcolor}{rgb}{0.000000,0.000000,0.000000}%
\pgfsetstrokecolor{textcolor}%
\pgfsetfillcolor{textcolor}%
\pgftext[x=0.278889in, y=0.503750in, left, base]{\color{textcolor}\rmfamily\fontsize{10.000000}{12.000000}\selectfont \(\displaystyle {0.0}\)}%
\end{pgfscope}%
\begin{pgfscope}%
\pgfsetbuttcap%
\pgfsetroundjoin%
\definecolor{currentfill}{rgb}{0.000000,0.000000,0.000000}%
\pgfsetfillcolor{currentfill}%
\pgfsetlinewidth{0.803000pt}%
\definecolor{currentstroke}{rgb}{0.000000,0.000000,0.000000}%
\pgfsetstrokecolor{currentstroke}%
\pgfsetdash{}{0pt}%
\pgfsys@defobject{currentmarker}{\pgfqpoint{-0.048611in}{0.000000in}}{\pgfqpoint{-0.000000in}{0.000000in}}{%
\pgfpathmoveto{\pgfqpoint{-0.000000in}{0.000000in}}%
\pgfpathlineto{\pgfqpoint{-0.048611in}{0.000000in}}%
\pgfusepath{stroke,fill}%
}%
\begin{pgfscope}%
\pgfsys@transformshift{0.553581in}{1.076944in}%
\pgfsys@useobject{currentmarker}{}%
\end{pgfscope}%
\end{pgfscope}%
\begin{pgfscope}%
\definecolor{textcolor}{rgb}{0.000000,0.000000,0.000000}%
\pgfsetstrokecolor{textcolor}%
\pgfsetfillcolor{textcolor}%
\pgftext[x=0.278889in, y=1.028750in, left, base]{\color{textcolor}\rmfamily\fontsize{10.000000}{12.000000}\selectfont \(\displaystyle {0.5}\)}%
\end{pgfscope}%
\begin{pgfscope}%
\pgfsetbuttcap%
\pgfsetroundjoin%
\definecolor{currentfill}{rgb}{0.000000,0.000000,0.000000}%
\pgfsetfillcolor{currentfill}%
\pgfsetlinewidth{0.803000pt}%
\definecolor{currentstroke}{rgb}{0.000000,0.000000,0.000000}%
\pgfsetstrokecolor{currentstroke}%
\pgfsetdash{}{0pt}%
\pgfsys@defobject{currentmarker}{\pgfqpoint{-0.048611in}{0.000000in}}{\pgfqpoint{-0.000000in}{0.000000in}}{%
\pgfpathmoveto{\pgfqpoint{-0.000000in}{0.000000in}}%
\pgfpathlineto{\pgfqpoint{-0.048611in}{0.000000in}}%
\pgfusepath{stroke,fill}%
}%
\begin{pgfscope}%
\pgfsys@transformshift{0.553581in}{1.601944in}%
\pgfsys@useobject{currentmarker}{}%
\end{pgfscope}%
\end{pgfscope}%
\begin{pgfscope}%
\definecolor{textcolor}{rgb}{0.000000,0.000000,0.000000}%
\pgfsetstrokecolor{textcolor}%
\pgfsetfillcolor{textcolor}%
\pgftext[x=0.278889in, y=1.553750in, left, base]{\color{textcolor}\rmfamily\fontsize{10.000000}{12.000000}\selectfont \(\displaystyle {1.0}\)}%
\end{pgfscope}%
\begin{pgfscope}%
\definecolor{textcolor}{rgb}{0.000000,0.000000,0.000000}%
\pgfsetstrokecolor{textcolor}%
\pgfsetfillcolor{textcolor}%
\pgftext[x=0.223333in,y=1.076944in,,bottom,rotate=90.000000]{\color{textcolor}\rmfamily\fontsize{10.000000}{12.000000}\selectfont True positive rate}%
\end{pgfscope}%
\begin{pgfscope}%
\pgfpathrectangle{\pgfqpoint{0.553581in}{0.499444in}}{\pgfqpoint{1.550000in}{1.155000in}}%
\pgfusepath{clip}%
\pgfsetbuttcap%
\pgfsetroundjoin%
\pgfsetlinewidth{1.505625pt}%
\definecolor{currentstroke}{rgb}{0.000000,0.000000,0.000000}%
\pgfsetstrokecolor{currentstroke}%
\pgfsetdash{{5.550000pt}{2.400000pt}}{0.000000pt}%
\pgfpathmoveto{\pgfqpoint{0.624035in}{0.551944in}}%
\pgfpathlineto{\pgfqpoint{2.033126in}{1.601944in}}%
\pgfusepath{stroke}%
\end{pgfscope}%
\begin{pgfscope}%
\pgfpathrectangle{\pgfqpoint{0.553581in}{0.499444in}}{\pgfqpoint{1.550000in}{1.155000in}}%
\pgfusepath{clip}%
\pgfsetrectcap%
\pgfsetroundjoin%
\pgfsetlinewidth{1.505625pt}%
\definecolor{currentstroke}{rgb}{0.000000,0.000000,0.000000}%
\pgfsetstrokecolor{currentstroke}%
\pgfsetdash{}{0pt}%
\pgfpathmoveto{\pgfqpoint{0.624035in}{0.551944in}}%
\pgfpathlineto{\pgfqpoint{0.625087in}{1.081068in}}%
\pgfpathlineto{\pgfqpoint{0.626766in}{1.222951in}}%
\pgfpathlineto{\pgfqpoint{0.630063in}{1.346143in}}%
\pgfpathlineto{\pgfqpoint{0.630170in}{1.346861in}}%
\pgfpathlineto{\pgfqpoint{0.633316in}{1.414483in}}%
\pgfpathlineto{\pgfqpoint{0.637112in}{1.467192in}}%
\pgfpathlineto{\pgfqpoint{0.642134in}{1.509564in}}%
\pgfpathlineto{\pgfqpoint{0.648493in}{1.541240in}}%
\pgfpathlineto{\pgfqpoint{0.648510in}{1.541293in}}%
\pgfpathlineto{\pgfqpoint{0.655596in}{1.563550in}}%
\pgfpathlineto{\pgfqpoint{0.660310in}{1.572583in}}%
\pgfpathlineto{\pgfqpoint{0.670656in}{1.585341in}}%
\pgfpathlineto{\pgfqpoint{0.682708in}{1.593071in}}%
\pgfpathlineto{\pgfqpoint{0.683120in}{1.593230in}}%
\pgfpathlineto{\pgfqpoint{0.697859in}{1.597421in}}%
\pgfpathlineto{\pgfqpoint{0.715458in}{1.600108in}}%
\pgfpathlineto{\pgfqpoint{0.747388in}{1.601439in}}%
\pgfpathlineto{\pgfqpoint{0.871277in}{1.601931in}}%
\pgfpathlineto{\pgfqpoint{2.033126in}{1.601944in}}%
\pgfpathlineto{\pgfqpoint{2.033126in}{1.601944in}}%
\pgfusepath{stroke}%
\end{pgfscope}%
\begin{pgfscope}%
\pgfsetrectcap%
\pgfsetmiterjoin%
\pgfsetlinewidth{0.803000pt}%
\definecolor{currentstroke}{rgb}{0.000000,0.000000,0.000000}%
\pgfsetstrokecolor{currentstroke}%
\pgfsetdash{}{0pt}%
\pgfpathmoveto{\pgfqpoint{0.553581in}{0.499444in}}%
\pgfpathlineto{\pgfqpoint{0.553581in}{1.654444in}}%
\pgfusepath{stroke}%
\end{pgfscope}%
\begin{pgfscope}%
\pgfsetrectcap%
\pgfsetmiterjoin%
\pgfsetlinewidth{0.803000pt}%
\definecolor{currentstroke}{rgb}{0.000000,0.000000,0.000000}%
\pgfsetstrokecolor{currentstroke}%
\pgfsetdash{}{0pt}%
\pgfpathmoveto{\pgfqpoint{2.103581in}{0.499444in}}%
\pgfpathlineto{\pgfqpoint{2.103581in}{1.654444in}}%
\pgfusepath{stroke}%
\end{pgfscope}%
\begin{pgfscope}%
\pgfsetrectcap%
\pgfsetmiterjoin%
\pgfsetlinewidth{0.803000pt}%
\definecolor{currentstroke}{rgb}{0.000000,0.000000,0.000000}%
\pgfsetstrokecolor{currentstroke}%
\pgfsetdash{}{0pt}%
\pgfpathmoveto{\pgfqpoint{0.553581in}{0.499444in}}%
\pgfpathlineto{\pgfqpoint{2.103581in}{0.499444in}}%
\pgfusepath{stroke}%
\end{pgfscope}%
\begin{pgfscope}%
\pgfsetrectcap%
\pgfsetmiterjoin%
\pgfsetlinewidth{0.803000pt}%
\definecolor{currentstroke}{rgb}{0.000000,0.000000,0.000000}%
\pgfsetstrokecolor{currentstroke}%
\pgfsetdash{}{0pt}%
\pgfpathmoveto{\pgfqpoint{0.553581in}{1.654444in}}%
\pgfpathlineto{\pgfqpoint{2.103581in}{1.654444in}}%
\pgfusepath{stroke}%
\end{pgfscope}%
\begin{pgfscope}%
\pgfsetbuttcap%
\pgfsetmiterjoin%
\definecolor{currentfill}{rgb}{1.000000,1.000000,1.000000}%
\pgfsetfillcolor{currentfill}%
\pgfsetfillopacity{0.800000}%
\pgfsetlinewidth{1.003750pt}%
\definecolor{currentstroke}{rgb}{0.800000,0.800000,0.800000}%
\pgfsetstrokecolor{currentstroke}%
\pgfsetstrokeopacity{0.800000}%
\pgfsetdash{}{0pt}%
\pgfpathmoveto{\pgfqpoint{0.832747in}{1.349722in}}%
\pgfpathlineto{\pgfqpoint{2.006358in}{1.349722in}}%
\pgfpathquadraticcurveto{\pgfqpoint{2.034136in}{1.349722in}}{\pgfqpoint{2.034136in}{1.377500in}}%
\pgfpathlineto{\pgfqpoint{2.034136in}{1.557222in}}%
\pgfpathquadraticcurveto{\pgfqpoint{2.034136in}{1.585000in}}{\pgfqpoint{2.006358in}{1.585000in}}%
\pgfpathlineto{\pgfqpoint{0.832747in}{1.585000in}}%
\pgfpathquadraticcurveto{\pgfqpoint{0.804970in}{1.585000in}}{\pgfqpoint{0.804970in}{1.557222in}}%
\pgfpathlineto{\pgfqpoint{0.804970in}{1.377500in}}%
\pgfpathquadraticcurveto{\pgfqpoint{0.804970in}{1.349722in}}{\pgfqpoint{0.832747in}{1.349722in}}%
\pgfpathlineto{\pgfqpoint{0.832747in}{1.349722in}}%
\pgfpathclose%
\pgfusepath{stroke,fill}%
\end{pgfscope}%
\begin{pgfscope}%
\pgfsetrectcap%
\pgfsetroundjoin%
\pgfsetlinewidth{1.505625pt}%
\definecolor{currentstroke}{rgb}{0.000000,0.000000,0.000000}%
\pgfsetstrokecolor{currentstroke}%
\pgfsetdash{}{0pt}%
\pgfpathmoveto{\pgfqpoint{0.860525in}{1.480833in}}%
\pgfpathlineto{\pgfqpoint{0.999414in}{1.480833in}}%
\pgfpathlineto{\pgfqpoint{1.138303in}{1.480833in}}%
\pgfusepath{stroke}%
\end{pgfscope}%
\begin{pgfscope}%
\definecolor{textcolor}{rgb}{0.000000,0.000000,0.000000}%
\pgfsetstrokecolor{textcolor}%
\pgfsetfillcolor{textcolor}%
\pgftext[x=1.249414in,y=1.432222in,left,base]{\color{textcolor}\rmfamily\fontsize{10.000000}{12.000000}\selectfont AUC=0.996}%
\end{pgfscope}%
\end{pgfpicture}%
\makeatother%
\endgroup%

\end{tabular}

\verb|y_proba = estimator.predict_proba(X_test)|


\noindent\begin{tabular}{@{\hspace{-6pt}}p{4.5in} @{\hspace{-6pt}}p{2.0in}}
	\vskip 0pt
	\qquad \qquad Raw Model Output on Test Set
	
	%% Creator: Matplotlib, PGF backend
%%
%% To include the figure in your LaTeX document, write
%%   \input{<filename>.pgf}
%%
%% Make sure the required packages are loaded in your preamble
%%   \usepackage{pgf}
%%
%% Also ensure that all the required font packages are loaded; for instance,
%% the lmodern package is sometimes necessary when using math font.
%%   \usepackage{lmodern}
%%
%% Figures using additional raster images can only be included by \input if
%% they are in the same directory as the main LaTeX file. For loading figures
%% from other directories you can use the `import` package
%%   \usepackage{import}
%%
%% and then include the figures with
%%   \import{<path to file>}{<filename>.pgf}
%%
%% Matplotlib used the following preamble
%%   
%%   \usepackage{fontspec}
%%   \makeatletter\@ifpackageloaded{underscore}{}{\usepackage[strings]{underscore}}\makeatother
%%
\begingroup%
\makeatletter%
\begin{pgfpicture}%
\pgfpathrectangle{\pgfpointorigin}{\pgfqpoint{4.509306in}{1.754444in}}%
\pgfusepath{use as bounding box, clip}%
\begin{pgfscope}%
\pgfsetbuttcap%
\pgfsetmiterjoin%
\definecolor{currentfill}{rgb}{1.000000,1.000000,1.000000}%
\pgfsetfillcolor{currentfill}%
\pgfsetlinewidth{0.000000pt}%
\definecolor{currentstroke}{rgb}{1.000000,1.000000,1.000000}%
\pgfsetstrokecolor{currentstroke}%
\pgfsetdash{}{0pt}%
\pgfpathmoveto{\pgfqpoint{0.000000in}{0.000000in}}%
\pgfpathlineto{\pgfqpoint{4.509306in}{0.000000in}}%
\pgfpathlineto{\pgfqpoint{4.509306in}{1.754444in}}%
\pgfpathlineto{\pgfqpoint{0.000000in}{1.754444in}}%
\pgfpathlineto{\pgfqpoint{0.000000in}{0.000000in}}%
\pgfpathclose%
\pgfusepath{fill}%
\end{pgfscope}%
\begin{pgfscope}%
\pgfsetbuttcap%
\pgfsetmiterjoin%
\definecolor{currentfill}{rgb}{1.000000,1.000000,1.000000}%
\pgfsetfillcolor{currentfill}%
\pgfsetlinewidth{0.000000pt}%
\definecolor{currentstroke}{rgb}{0.000000,0.000000,0.000000}%
\pgfsetstrokecolor{currentstroke}%
\pgfsetstrokeopacity{0.000000}%
\pgfsetdash{}{0pt}%
\pgfpathmoveto{\pgfqpoint{0.445556in}{0.499444in}}%
\pgfpathlineto{\pgfqpoint{4.320556in}{0.499444in}}%
\pgfpathlineto{\pgfqpoint{4.320556in}{1.654444in}}%
\pgfpathlineto{\pgfqpoint{0.445556in}{1.654444in}}%
\pgfpathlineto{\pgfqpoint{0.445556in}{0.499444in}}%
\pgfpathclose%
\pgfusepath{fill}%
\end{pgfscope}%
\begin{pgfscope}%
\pgfpathrectangle{\pgfqpoint{0.445556in}{0.499444in}}{\pgfqpoint{3.875000in}{1.155000in}}%
\pgfusepath{clip}%
\pgfsetbuttcap%
\pgfsetmiterjoin%
\pgfsetlinewidth{1.003750pt}%
\definecolor{currentstroke}{rgb}{0.000000,0.000000,0.000000}%
\pgfsetstrokecolor{currentstroke}%
\pgfsetdash{}{0pt}%
\pgfpathmoveto{\pgfqpoint{0.435556in}{0.499444in}}%
\pgfpathlineto{\pgfqpoint{0.483922in}{0.499444in}}%
\pgfpathlineto{\pgfqpoint{0.483922in}{0.618288in}}%
\pgfpathlineto{\pgfqpoint{0.435556in}{0.618288in}}%
\pgfusepath{stroke}%
\end{pgfscope}%
\begin{pgfscope}%
\pgfpathrectangle{\pgfqpoint{0.445556in}{0.499444in}}{\pgfqpoint{3.875000in}{1.155000in}}%
\pgfusepath{clip}%
\pgfsetbuttcap%
\pgfsetmiterjoin%
\pgfsetlinewidth{1.003750pt}%
\definecolor{currentstroke}{rgb}{0.000000,0.000000,0.000000}%
\pgfsetstrokecolor{currentstroke}%
\pgfsetdash{}{0pt}%
\pgfpathmoveto{\pgfqpoint{0.576001in}{0.499444in}}%
\pgfpathlineto{\pgfqpoint{0.637387in}{0.499444in}}%
\pgfpathlineto{\pgfqpoint{0.637387in}{0.801196in}}%
\pgfpathlineto{\pgfqpoint{0.576001in}{0.801196in}}%
\pgfpathlineto{\pgfqpoint{0.576001in}{0.499444in}}%
\pgfpathclose%
\pgfusepath{stroke}%
\end{pgfscope}%
\begin{pgfscope}%
\pgfpathrectangle{\pgfqpoint{0.445556in}{0.499444in}}{\pgfqpoint{3.875000in}{1.155000in}}%
\pgfusepath{clip}%
\pgfsetbuttcap%
\pgfsetmiterjoin%
\pgfsetlinewidth{1.003750pt}%
\definecolor{currentstroke}{rgb}{0.000000,0.000000,0.000000}%
\pgfsetstrokecolor{currentstroke}%
\pgfsetdash{}{0pt}%
\pgfpathmoveto{\pgfqpoint{0.729467in}{0.499444in}}%
\pgfpathlineto{\pgfqpoint{0.790853in}{0.499444in}}%
\pgfpathlineto{\pgfqpoint{0.790853in}{1.026194in}}%
\pgfpathlineto{\pgfqpoint{0.729467in}{1.026194in}}%
\pgfpathlineto{\pgfqpoint{0.729467in}{0.499444in}}%
\pgfpathclose%
\pgfusepath{stroke}%
\end{pgfscope}%
\begin{pgfscope}%
\pgfpathrectangle{\pgfqpoint{0.445556in}{0.499444in}}{\pgfqpoint{3.875000in}{1.155000in}}%
\pgfusepath{clip}%
\pgfsetbuttcap%
\pgfsetmiterjoin%
\pgfsetlinewidth{1.003750pt}%
\definecolor{currentstroke}{rgb}{0.000000,0.000000,0.000000}%
\pgfsetstrokecolor{currentstroke}%
\pgfsetdash{}{0pt}%
\pgfpathmoveto{\pgfqpoint{0.882932in}{0.499444in}}%
\pgfpathlineto{\pgfqpoint{0.944318in}{0.499444in}}%
\pgfpathlineto{\pgfqpoint{0.944318in}{1.203686in}}%
\pgfpathlineto{\pgfqpoint{0.882932in}{1.203686in}}%
\pgfpathlineto{\pgfqpoint{0.882932in}{0.499444in}}%
\pgfpathclose%
\pgfusepath{stroke}%
\end{pgfscope}%
\begin{pgfscope}%
\pgfpathrectangle{\pgfqpoint{0.445556in}{0.499444in}}{\pgfqpoint{3.875000in}{1.155000in}}%
\pgfusepath{clip}%
\pgfsetbuttcap%
\pgfsetmiterjoin%
\pgfsetlinewidth{1.003750pt}%
\definecolor{currentstroke}{rgb}{0.000000,0.000000,0.000000}%
\pgfsetstrokecolor{currentstroke}%
\pgfsetdash{}{0pt}%
\pgfpathmoveto{\pgfqpoint{1.036397in}{0.499444in}}%
\pgfpathlineto{\pgfqpoint{1.097783in}{0.499444in}}%
\pgfpathlineto{\pgfqpoint{1.097783in}{1.348758in}}%
\pgfpathlineto{\pgfqpoint{1.036397in}{1.348758in}}%
\pgfpathlineto{\pgfqpoint{1.036397in}{0.499444in}}%
\pgfpathclose%
\pgfusepath{stroke}%
\end{pgfscope}%
\begin{pgfscope}%
\pgfpathrectangle{\pgfqpoint{0.445556in}{0.499444in}}{\pgfqpoint{3.875000in}{1.155000in}}%
\pgfusepath{clip}%
\pgfsetbuttcap%
\pgfsetmiterjoin%
\pgfsetlinewidth{1.003750pt}%
\definecolor{currentstroke}{rgb}{0.000000,0.000000,0.000000}%
\pgfsetstrokecolor{currentstroke}%
\pgfsetdash{}{0pt}%
\pgfpathmoveto{\pgfqpoint{1.189863in}{0.499444in}}%
\pgfpathlineto{\pgfqpoint{1.251249in}{0.499444in}}%
\pgfpathlineto{\pgfqpoint{1.251249in}{1.455609in}}%
\pgfpathlineto{\pgfqpoint{1.189863in}{1.455609in}}%
\pgfpathlineto{\pgfqpoint{1.189863in}{0.499444in}}%
\pgfpathclose%
\pgfusepath{stroke}%
\end{pgfscope}%
\begin{pgfscope}%
\pgfpathrectangle{\pgfqpoint{0.445556in}{0.499444in}}{\pgfqpoint{3.875000in}{1.155000in}}%
\pgfusepath{clip}%
\pgfsetbuttcap%
\pgfsetmiterjoin%
\pgfsetlinewidth{1.003750pt}%
\definecolor{currentstroke}{rgb}{0.000000,0.000000,0.000000}%
\pgfsetstrokecolor{currentstroke}%
\pgfsetdash{}{0pt}%
\pgfpathmoveto{\pgfqpoint{1.343328in}{0.499444in}}%
\pgfpathlineto{\pgfqpoint{1.404714in}{0.499444in}}%
\pgfpathlineto{\pgfqpoint{1.404714in}{1.549075in}}%
\pgfpathlineto{\pgfqpoint{1.343328in}{1.549075in}}%
\pgfpathlineto{\pgfqpoint{1.343328in}{0.499444in}}%
\pgfpathclose%
\pgfusepath{stroke}%
\end{pgfscope}%
\begin{pgfscope}%
\pgfpathrectangle{\pgfqpoint{0.445556in}{0.499444in}}{\pgfqpoint{3.875000in}{1.155000in}}%
\pgfusepath{clip}%
\pgfsetbuttcap%
\pgfsetmiterjoin%
\pgfsetlinewidth{1.003750pt}%
\definecolor{currentstroke}{rgb}{0.000000,0.000000,0.000000}%
\pgfsetstrokecolor{currentstroke}%
\pgfsetdash{}{0pt}%
\pgfpathmoveto{\pgfqpoint{1.496793in}{0.499444in}}%
\pgfpathlineto{\pgfqpoint{1.558179in}{0.499444in}}%
\pgfpathlineto{\pgfqpoint{1.558179in}{1.599444in}}%
\pgfpathlineto{\pgfqpoint{1.496793in}{1.599444in}}%
\pgfpathlineto{\pgfqpoint{1.496793in}{0.499444in}}%
\pgfpathclose%
\pgfusepath{stroke}%
\end{pgfscope}%
\begin{pgfscope}%
\pgfpathrectangle{\pgfqpoint{0.445556in}{0.499444in}}{\pgfqpoint{3.875000in}{1.155000in}}%
\pgfusepath{clip}%
\pgfsetbuttcap%
\pgfsetmiterjoin%
\pgfsetlinewidth{1.003750pt}%
\definecolor{currentstroke}{rgb}{0.000000,0.000000,0.000000}%
\pgfsetstrokecolor{currentstroke}%
\pgfsetdash{}{0pt}%
\pgfpathmoveto{\pgfqpoint{1.650259in}{0.499444in}}%
\pgfpathlineto{\pgfqpoint{1.711645in}{0.499444in}}%
\pgfpathlineto{\pgfqpoint{1.711645in}{1.597742in}}%
\pgfpathlineto{\pgfqpoint{1.650259in}{1.597742in}}%
\pgfpathlineto{\pgfqpoint{1.650259in}{0.499444in}}%
\pgfpathclose%
\pgfusepath{stroke}%
\end{pgfscope}%
\begin{pgfscope}%
\pgfpathrectangle{\pgfqpoint{0.445556in}{0.499444in}}{\pgfqpoint{3.875000in}{1.155000in}}%
\pgfusepath{clip}%
\pgfsetbuttcap%
\pgfsetmiterjoin%
\pgfsetlinewidth{1.003750pt}%
\definecolor{currentstroke}{rgb}{0.000000,0.000000,0.000000}%
\pgfsetstrokecolor{currentstroke}%
\pgfsetdash{}{0pt}%
\pgfpathmoveto{\pgfqpoint{1.803724in}{0.499444in}}%
\pgfpathlineto{\pgfqpoint{1.865110in}{0.499444in}}%
\pgfpathlineto{\pgfqpoint{1.865110in}{1.571977in}}%
\pgfpathlineto{\pgfqpoint{1.803724in}{1.571977in}}%
\pgfpathlineto{\pgfqpoint{1.803724in}{0.499444in}}%
\pgfpathclose%
\pgfusepath{stroke}%
\end{pgfscope}%
\begin{pgfscope}%
\pgfpathrectangle{\pgfqpoint{0.445556in}{0.499444in}}{\pgfqpoint{3.875000in}{1.155000in}}%
\pgfusepath{clip}%
\pgfsetbuttcap%
\pgfsetmiterjoin%
\pgfsetlinewidth{1.003750pt}%
\definecolor{currentstroke}{rgb}{0.000000,0.000000,0.000000}%
\pgfsetstrokecolor{currentstroke}%
\pgfsetdash{}{0pt}%
\pgfpathmoveto{\pgfqpoint{1.957189in}{0.499444in}}%
\pgfpathlineto{\pgfqpoint{2.018575in}{0.499444in}}%
\pgfpathlineto{\pgfqpoint{2.018575in}{1.535535in}}%
\pgfpathlineto{\pgfqpoint{1.957189in}{1.535535in}}%
\pgfpathlineto{\pgfqpoint{1.957189in}{0.499444in}}%
\pgfpathclose%
\pgfusepath{stroke}%
\end{pgfscope}%
\begin{pgfscope}%
\pgfpathrectangle{\pgfqpoint{0.445556in}{0.499444in}}{\pgfqpoint{3.875000in}{1.155000in}}%
\pgfusepath{clip}%
\pgfsetbuttcap%
\pgfsetmiterjoin%
\pgfsetlinewidth{1.003750pt}%
\definecolor{currentstroke}{rgb}{0.000000,0.000000,0.000000}%
\pgfsetstrokecolor{currentstroke}%
\pgfsetdash{}{0pt}%
\pgfpathmoveto{\pgfqpoint{2.110655in}{0.499444in}}%
\pgfpathlineto{\pgfqpoint{2.172041in}{0.499444in}}%
\pgfpathlineto{\pgfqpoint{2.172041in}{1.450812in}}%
\pgfpathlineto{\pgfqpoint{2.110655in}{1.450812in}}%
\pgfpathlineto{\pgfqpoint{2.110655in}{0.499444in}}%
\pgfpathclose%
\pgfusepath{stroke}%
\end{pgfscope}%
\begin{pgfscope}%
\pgfpathrectangle{\pgfqpoint{0.445556in}{0.499444in}}{\pgfqpoint{3.875000in}{1.155000in}}%
\pgfusepath{clip}%
\pgfsetbuttcap%
\pgfsetmiterjoin%
\pgfsetlinewidth{1.003750pt}%
\definecolor{currentstroke}{rgb}{0.000000,0.000000,0.000000}%
\pgfsetstrokecolor{currentstroke}%
\pgfsetdash{}{0pt}%
\pgfpathmoveto{\pgfqpoint{2.264120in}{0.499444in}}%
\pgfpathlineto{\pgfqpoint{2.325506in}{0.499444in}}%
\pgfpathlineto{\pgfqpoint{2.325506in}{1.379398in}}%
\pgfpathlineto{\pgfqpoint{2.264120in}{1.379398in}}%
\pgfpathlineto{\pgfqpoint{2.264120in}{0.499444in}}%
\pgfpathclose%
\pgfusepath{stroke}%
\end{pgfscope}%
\begin{pgfscope}%
\pgfpathrectangle{\pgfqpoint{0.445556in}{0.499444in}}{\pgfqpoint{3.875000in}{1.155000in}}%
\pgfusepath{clip}%
\pgfsetbuttcap%
\pgfsetmiterjoin%
\pgfsetlinewidth{1.003750pt}%
\definecolor{currentstroke}{rgb}{0.000000,0.000000,0.000000}%
\pgfsetstrokecolor{currentstroke}%
\pgfsetdash{}{0pt}%
\pgfpathmoveto{\pgfqpoint{2.417585in}{0.499444in}}%
\pgfpathlineto{\pgfqpoint{2.478972in}{0.499444in}}%
\pgfpathlineto{\pgfqpoint{2.478972in}{1.248329in}}%
\pgfpathlineto{\pgfqpoint{2.417585in}{1.248329in}}%
\pgfpathlineto{\pgfqpoint{2.417585in}{0.499444in}}%
\pgfpathclose%
\pgfusepath{stroke}%
\end{pgfscope}%
\begin{pgfscope}%
\pgfpathrectangle{\pgfqpoint{0.445556in}{0.499444in}}{\pgfqpoint{3.875000in}{1.155000in}}%
\pgfusepath{clip}%
\pgfsetbuttcap%
\pgfsetmiterjoin%
\pgfsetlinewidth{1.003750pt}%
\definecolor{currentstroke}{rgb}{0.000000,0.000000,0.000000}%
\pgfsetstrokecolor{currentstroke}%
\pgfsetdash{}{0pt}%
\pgfpathmoveto{\pgfqpoint{2.571051in}{0.499444in}}%
\pgfpathlineto{\pgfqpoint{2.632437in}{0.499444in}}%
\pgfpathlineto{\pgfqpoint{2.632437in}{1.129640in}}%
\pgfpathlineto{\pgfqpoint{2.571051in}{1.129640in}}%
\pgfpathlineto{\pgfqpoint{2.571051in}{0.499444in}}%
\pgfpathclose%
\pgfusepath{stroke}%
\end{pgfscope}%
\begin{pgfscope}%
\pgfpathrectangle{\pgfqpoint{0.445556in}{0.499444in}}{\pgfqpoint{3.875000in}{1.155000in}}%
\pgfusepath{clip}%
\pgfsetbuttcap%
\pgfsetmiterjoin%
\pgfsetlinewidth{1.003750pt}%
\definecolor{currentstroke}{rgb}{0.000000,0.000000,0.000000}%
\pgfsetstrokecolor{currentstroke}%
\pgfsetdash{}{0pt}%
\pgfpathmoveto{\pgfqpoint{2.724516in}{0.499444in}}%
\pgfpathlineto{\pgfqpoint{2.785902in}{0.499444in}}%
\pgfpathlineto{\pgfqpoint{2.785902in}{1.011880in}}%
\pgfpathlineto{\pgfqpoint{2.724516in}{1.011880in}}%
\pgfpathlineto{\pgfqpoint{2.724516in}{0.499444in}}%
\pgfpathclose%
\pgfusepath{stroke}%
\end{pgfscope}%
\begin{pgfscope}%
\pgfpathrectangle{\pgfqpoint{0.445556in}{0.499444in}}{\pgfqpoint{3.875000in}{1.155000in}}%
\pgfusepath{clip}%
\pgfsetbuttcap%
\pgfsetmiterjoin%
\pgfsetlinewidth{1.003750pt}%
\definecolor{currentstroke}{rgb}{0.000000,0.000000,0.000000}%
\pgfsetstrokecolor{currentstroke}%
\pgfsetdash{}{0pt}%
\pgfpathmoveto{\pgfqpoint{2.877981in}{0.499444in}}%
\pgfpathlineto{\pgfqpoint{2.939368in}{0.499444in}}%
\pgfpathlineto{\pgfqpoint{2.939368in}{0.912921in}}%
\pgfpathlineto{\pgfqpoint{2.877981in}{0.912921in}}%
\pgfpathlineto{\pgfqpoint{2.877981in}{0.499444in}}%
\pgfpathclose%
\pgfusepath{stroke}%
\end{pgfscope}%
\begin{pgfscope}%
\pgfpathrectangle{\pgfqpoint{0.445556in}{0.499444in}}{\pgfqpoint{3.875000in}{1.155000in}}%
\pgfusepath{clip}%
\pgfsetbuttcap%
\pgfsetmiterjoin%
\pgfsetlinewidth{1.003750pt}%
\definecolor{currentstroke}{rgb}{0.000000,0.000000,0.000000}%
\pgfsetstrokecolor{currentstroke}%
\pgfsetdash{}{0pt}%
\pgfpathmoveto{\pgfqpoint{3.031447in}{0.499444in}}%
\pgfpathlineto{\pgfqpoint{3.092833in}{0.499444in}}%
\pgfpathlineto{\pgfqpoint{3.092833in}{0.805374in}}%
\pgfpathlineto{\pgfqpoint{3.031447in}{0.805374in}}%
\pgfpathlineto{\pgfqpoint{3.031447in}{0.499444in}}%
\pgfpathclose%
\pgfusepath{stroke}%
\end{pgfscope}%
\begin{pgfscope}%
\pgfpathrectangle{\pgfqpoint{0.445556in}{0.499444in}}{\pgfqpoint{3.875000in}{1.155000in}}%
\pgfusepath{clip}%
\pgfsetbuttcap%
\pgfsetmiterjoin%
\pgfsetlinewidth{1.003750pt}%
\definecolor{currentstroke}{rgb}{0.000000,0.000000,0.000000}%
\pgfsetstrokecolor{currentstroke}%
\pgfsetdash{}{0pt}%
\pgfpathmoveto{\pgfqpoint{3.184912in}{0.499444in}}%
\pgfpathlineto{\pgfqpoint{3.246298in}{0.499444in}}%
\pgfpathlineto{\pgfqpoint{3.246298in}{0.730710in}}%
\pgfpathlineto{\pgfqpoint{3.184912in}{0.730710in}}%
\pgfpathlineto{\pgfqpoint{3.184912in}{0.499444in}}%
\pgfpathclose%
\pgfusepath{stroke}%
\end{pgfscope}%
\begin{pgfscope}%
\pgfpathrectangle{\pgfqpoint{0.445556in}{0.499444in}}{\pgfqpoint{3.875000in}{1.155000in}}%
\pgfusepath{clip}%
\pgfsetbuttcap%
\pgfsetmiterjoin%
\pgfsetlinewidth{1.003750pt}%
\definecolor{currentstroke}{rgb}{0.000000,0.000000,0.000000}%
\pgfsetstrokecolor{currentstroke}%
\pgfsetdash{}{0pt}%
\pgfpathmoveto{\pgfqpoint{3.338377in}{0.499444in}}%
\pgfpathlineto{\pgfqpoint{3.399764in}{0.499444in}}%
\pgfpathlineto{\pgfqpoint{3.399764in}{0.672603in}}%
\pgfpathlineto{\pgfqpoint{3.338377in}{0.672603in}}%
\pgfpathlineto{\pgfqpoint{3.338377in}{0.499444in}}%
\pgfpathclose%
\pgfusepath{stroke}%
\end{pgfscope}%
\begin{pgfscope}%
\pgfpathrectangle{\pgfqpoint{0.445556in}{0.499444in}}{\pgfqpoint{3.875000in}{1.155000in}}%
\pgfusepath{clip}%
\pgfsetbuttcap%
\pgfsetmiterjoin%
\pgfsetlinewidth{1.003750pt}%
\definecolor{currentstroke}{rgb}{0.000000,0.000000,0.000000}%
\pgfsetstrokecolor{currentstroke}%
\pgfsetdash{}{0pt}%
\pgfpathmoveto{\pgfqpoint{3.491843in}{0.499444in}}%
\pgfpathlineto{\pgfqpoint{3.553229in}{0.499444in}}%
\pgfpathlineto{\pgfqpoint{3.553229in}{0.616663in}}%
\pgfpathlineto{\pgfqpoint{3.491843in}{0.616663in}}%
\pgfpathlineto{\pgfqpoint{3.491843in}{0.499444in}}%
\pgfpathclose%
\pgfusepath{stroke}%
\end{pgfscope}%
\begin{pgfscope}%
\pgfpathrectangle{\pgfqpoint{0.445556in}{0.499444in}}{\pgfqpoint{3.875000in}{1.155000in}}%
\pgfusepath{clip}%
\pgfsetbuttcap%
\pgfsetmiterjoin%
\pgfsetlinewidth{1.003750pt}%
\definecolor{currentstroke}{rgb}{0.000000,0.000000,0.000000}%
\pgfsetstrokecolor{currentstroke}%
\pgfsetdash{}{0pt}%
\pgfpathmoveto{\pgfqpoint{3.645308in}{0.499444in}}%
\pgfpathlineto{\pgfqpoint{3.706694in}{0.499444in}}%
\pgfpathlineto{\pgfqpoint{3.706694in}{0.574031in}}%
\pgfpathlineto{\pgfqpoint{3.645308in}{0.574031in}}%
\pgfpathlineto{\pgfqpoint{3.645308in}{0.499444in}}%
\pgfpathclose%
\pgfusepath{stroke}%
\end{pgfscope}%
\begin{pgfscope}%
\pgfpathrectangle{\pgfqpoint{0.445556in}{0.499444in}}{\pgfqpoint{3.875000in}{1.155000in}}%
\pgfusepath{clip}%
\pgfsetbuttcap%
\pgfsetmiterjoin%
\pgfsetlinewidth{1.003750pt}%
\definecolor{currentstroke}{rgb}{0.000000,0.000000,0.000000}%
\pgfsetstrokecolor{currentstroke}%
\pgfsetdash{}{0pt}%
\pgfpathmoveto{\pgfqpoint{3.798774in}{0.499444in}}%
\pgfpathlineto{\pgfqpoint{3.860160in}{0.499444in}}%
\pgfpathlineto{\pgfqpoint{3.860160in}{0.550587in}}%
\pgfpathlineto{\pgfqpoint{3.798774in}{0.550587in}}%
\pgfpathlineto{\pgfqpoint{3.798774in}{0.499444in}}%
\pgfpathclose%
\pgfusepath{stroke}%
\end{pgfscope}%
\begin{pgfscope}%
\pgfpathrectangle{\pgfqpoint{0.445556in}{0.499444in}}{\pgfqpoint{3.875000in}{1.155000in}}%
\pgfusepath{clip}%
\pgfsetbuttcap%
\pgfsetmiterjoin%
\pgfsetlinewidth{1.003750pt}%
\definecolor{currentstroke}{rgb}{0.000000,0.000000,0.000000}%
\pgfsetstrokecolor{currentstroke}%
\pgfsetdash{}{0pt}%
\pgfpathmoveto{\pgfqpoint{3.952239in}{0.499444in}}%
\pgfpathlineto{\pgfqpoint{4.013625in}{0.499444in}}%
\pgfpathlineto{\pgfqpoint{4.013625in}{0.526679in}}%
\pgfpathlineto{\pgfqpoint{3.952239in}{0.526679in}}%
\pgfpathlineto{\pgfqpoint{3.952239in}{0.499444in}}%
\pgfpathclose%
\pgfusepath{stroke}%
\end{pgfscope}%
\begin{pgfscope}%
\pgfpathrectangle{\pgfqpoint{0.445556in}{0.499444in}}{\pgfqpoint{3.875000in}{1.155000in}}%
\pgfusepath{clip}%
\pgfsetbuttcap%
\pgfsetmiterjoin%
\pgfsetlinewidth{1.003750pt}%
\definecolor{currentstroke}{rgb}{0.000000,0.000000,0.000000}%
\pgfsetstrokecolor{currentstroke}%
\pgfsetdash{}{0pt}%
\pgfpathmoveto{\pgfqpoint{4.105704in}{0.499444in}}%
\pgfpathlineto{\pgfqpoint{4.167090in}{0.499444in}}%
\pgfpathlineto{\pgfqpoint{4.167090in}{0.506640in}}%
\pgfpathlineto{\pgfqpoint{4.105704in}{0.506640in}}%
\pgfpathlineto{\pgfqpoint{4.105704in}{0.499444in}}%
\pgfpathclose%
\pgfusepath{stroke}%
\end{pgfscope}%
\begin{pgfscope}%
\pgfpathrectangle{\pgfqpoint{0.445556in}{0.499444in}}{\pgfqpoint{3.875000in}{1.155000in}}%
\pgfusepath{clip}%
\pgfsetbuttcap%
\pgfsetmiterjoin%
\definecolor{currentfill}{rgb}{0.000000,0.000000,0.000000}%
\pgfsetfillcolor{currentfill}%
\pgfsetlinewidth{0.000000pt}%
\definecolor{currentstroke}{rgb}{0.000000,0.000000,0.000000}%
\pgfsetstrokecolor{currentstroke}%
\pgfsetstrokeopacity{0.000000}%
\pgfsetdash{}{0pt}%
\pgfpathmoveto{\pgfqpoint{0.483922in}{0.499444in}}%
\pgfpathlineto{\pgfqpoint{0.545308in}{0.499444in}}%
\pgfpathlineto{\pgfqpoint{0.545308in}{0.500527in}}%
\pgfpathlineto{\pgfqpoint{0.483922in}{0.500527in}}%
\pgfpathlineto{\pgfqpoint{0.483922in}{0.499444in}}%
\pgfpathclose%
\pgfusepath{fill}%
\end{pgfscope}%
\begin{pgfscope}%
\pgfpathrectangle{\pgfqpoint{0.445556in}{0.499444in}}{\pgfqpoint{3.875000in}{1.155000in}}%
\pgfusepath{clip}%
\pgfsetbuttcap%
\pgfsetmiterjoin%
\definecolor{currentfill}{rgb}{0.000000,0.000000,0.000000}%
\pgfsetfillcolor{currentfill}%
\pgfsetlinewidth{0.000000pt}%
\definecolor{currentstroke}{rgb}{0.000000,0.000000,0.000000}%
\pgfsetstrokecolor{currentstroke}%
\pgfsetstrokeopacity{0.000000}%
\pgfsetdash{}{0pt}%
\pgfpathmoveto{\pgfqpoint{0.637387in}{0.499444in}}%
\pgfpathlineto{\pgfqpoint{0.698774in}{0.499444in}}%
\pgfpathlineto{\pgfqpoint{0.698774in}{0.502230in}}%
\pgfpathlineto{\pgfqpoint{0.637387in}{0.502230in}}%
\pgfpathlineto{\pgfqpoint{0.637387in}{0.499444in}}%
\pgfpathclose%
\pgfusepath{fill}%
\end{pgfscope}%
\begin{pgfscope}%
\pgfpathrectangle{\pgfqpoint{0.445556in}{0.499444in}}{\pgfqpoint{3.875000in}{1.155000in}}%
\pgfusepath{clip}%
\pgfsetbuttcap%
\pgfsetmiterjoin%
\definecolor{currentfill}{rgb}{0.000000,0.000000,0.000000}%
\pgfsetfillcolor{currentfill}%
\pgfsetlinewidth{0.000000pt}%
\definecolor{currentstroke}{rgb}{0.000000,0.000000,0.000000}%
\pgfsetstrokecolor{currentstroke}%
\pgfsetstrokeopacity{0.000000}%
\pgfsetdash{}{0pt}%
\pgfpathmoveto{\pgfqpoint{0.790853in}{0.499444in}}%
\pgfpathlineto{\pgfqpoint{0.852239in}{0.499444in}}%
\pgfpathlineto{\pgfqpoint{0.852239in}{0.506330in}}%
\pgfpathlineto{\pgfqpoint{0.790853in}{0.506330in}}%
\pgfpathlineto{\pgfqpoint{0.790853in}{0.499444in}}%
\pgfpathclose%
\pgfusepath{fill}%
\end{pgfscope}%
\begin{pgfscope}%
\pgfpathrectangle{\pgfqpoint{0.445556in}{0.499444in}}{\pgfqpoint{3.875000in}{1.155000in}}%
\pgfusepath{clip}%
\pgfsetbuttcap%
\pgfsetmiterjoin%
\definecolor{currentfill}{rgb}{0.000000,0.000000,0.000000}%
\pgfsetfillcolor{currentfill}%
\pgfsetlinewidth{0.000000pt}%
\definecolor{currentstroke}{rgb}{0.000000,0.000000,0.000000}%
\pgfsetstrokecolor{currentstroke}%
\pgfsetstrokeopacity{0.000000}%
\pgfsetdash{}{0pt}%
\pgfpathmoveto{\pgfqpoint{0.944318in}{0.499444in}}%
\pgfpathlineto{\pgfqpoint{1.005704in}{0.499444in}}%
\pgfpathlineto{\pgfqpoint{1.005704in}{0.514068in}}%
\pgfpathlineto{\pgfqpoint{0.944318in}{0.514068in}}%
\pgfpathlineto{\pgfqpoint{0.944318in}{0.499444in}}%
\pgfpathclose%
\pgfusepath{fill}%
\end{pgfscope}%
\begin{pgfscope}%
\pgfpathrectangle{\pgfqpoint{0.445556in}{0.499444in}}{\pgfqpoint{3.875000in}{1.155000in}}%
\pgfusepath{clip}%
\pgfsetbuttcap%
\pgfsetmiterjoin%
\definecolor{currentfill}{rgb}{0.000000,0.000000,0.000000}%
\pgfsetfillcolor{currentfill}%
\pgfsetlinewidth{0.000000pt}%
\definecolor{currentstroke}{rgb}{0.000000,0.000000,0.000000}%
\pgfsetstrokecolor{currentstroke}%
\pgfsetstrokeopacity{0.000000}%
\pgfsetdash{}{0pt}%
\pgfpathmoveto{\pgfqpoint{1.097783in}{0.499444in}}%
\pgfpathlineto{\pgfqpoint{1.159170in}{0.499444in}}%
\pgfpathlineto{\pgfqpoint{1.159170in}{0.524977in}}%
\pgfpathlineto{\pgfqpoint{1.097783in}{0.524977in}}%
\pgfpathlineto{\pgfqpoint{1.097783in}{0.499444in}}%
\pgfpathclose%
\pgfusepath{fill}%
\end{pgfscope}%
\begin{pgfscope}%
\pgfpathrectangle{\pgfqpoint{0.445556in}{0.499444in}}{\pgfqpoint{3.875000in}{1.155000in}}%
\pgfusepath{clip}%
\pgfsetbuttcap%
\pgfsetmiterjoin%
\definecolor{currentfill}{rgb}{0.000000,0.000000,0.000000}%
\pgfsetfillcolor{currentfill}%
\pgfsetlinewidth{0.000000pt}%
\definecolor{currentstroke}{rgb}{0.000000,0.000000,0.000000}%
\pgfsetstrokecolor{currentstroke}%
\pgfsetstrokeopacity{0.000000}%
\pgfsetdash{}{0pt}%
\pgfpathmoveto{\pgfqpoint{1.251249in}{0.499444in}}%
\pgfpathlineto{\pgfqpoint{1.312635in}{0.499444in}}%
\pgfpathlineto{\pgfqpoint{1.312635in}{0.535267in}}%
\pgfpathlineto{\pgfqpoint{1.251249in}{0.535267in}}%
\pgfpathlineto{\pgfqpoint{1.251249in}{0.499444in}}%
\pgfpathclose%
\pgfusepath{fill}%
\end{pgfscope}%
\begin{pgfscope}%
\pgfpathrectangle{\pgfqpoint{0.445556in}{0.499444in}}{\pgfqpoint{3.875000in}{1.155000in}}%
\pgfusepath{clip}%
\pgfsetbuttcap%
\pgfsetmiterjoin%
\definecolor{currentfill}{rgb}{0.000000,0.000000,0.000000}%
\pgfsetfillcolor{currentfill}%
\pgfsetlinewidth{0.000000pt}%
\definecolor{currentstroke}{rgb}{0.000000,0.000000,0.000000}%
\pgfsetstrokecolor{currentstroke}%
\pgfsetstrokeopacity{0.000000}%
\pgfsetdash{}{0pt}%
\pgfpathmoveto{\pgfqpoint{1.404714in}{0.499444in}}%
\pgfpathlineto{\pgfqpoint{1.466100in}{0.499444in}}%
\pgfpathlineto{\pgfqpoint{1.466100in}{0.549659in}}%
\pgfpathlineto{\pgfqpoint{1.404714in}{0.549659in}}%
\pgfpathlineto{\pgfqpoint{1.404714in}{0.499444in}}%
\pgfpathclose%
\pgfusepath{fill}%
\end{pgfscope}%
\begin{pgfscope}%
\pgfpathrectangle{\pgfqpoint{0.445556in}{0.499444in}}{\pgfqpoint{3.875000in}{1.155000in}}%
\pgfusepath{clip}%
\pgfsetbuttcap%
\pgfsetmiterjoin%
\definecolor{currentfill}{rgb}{0.000000,0.000000,0.000000}%
\pgfsetfillcolor{currentfill}%
\pgfsetlinewidth{0.000000pt}%
\definecolor{currentstroke}{rgb}{0.000000,0.000000,0.000000}%
\pgfsetstrokecolor{currentstroke}%
\pgfsetstrokeopacity{0.000000}%
\pgfsetdash{}{0pt}%
\pgfpathmoveto{\pgfqpoint{1.558179in}{0.499444in}}%
\pgfpathlineto{\pgfqpoint{1.619566in}{0.499444in}}%
\pgfpathlineto{\pgfqpoint{1.619566in}{0.568228in}}%
\pgfpathlineto{\pgfqpoint{1.558179in}{0.568228in}}%
\pgfpathlineto{\pgfqpoint{1.558179in}{0.499444in}}%
\pgfpathclose%
\pgfusepath{fill}%
\end{pgfscope}%
\begin{pgfscope}%
\pgfpathrectangle{\pgfqpoint{0.445556in}{0.499444in}}{\pgfqpoint{3.875000in}{1.155000in}}%
\pgfusepath{clip}%
\pgfsetbuttcap%
\pgfsetmiterjoin%
\definecolor{currentfill}{rgb}{0.000000,0.000000,0.000000}%
\pgfsetfillcolor{currentfill}%
\pgfsetlinewidth{0.000000pt}%
\definecolor{currentstroke}{rgb}{0.000000,0.000000,0.000000}%
\pgfsetstrokecolor{currentstroke}%
\pgfsetstrokeopacity{0.000000}%
\pgfsetdash{}{0pt}%
\pgfpathmoveto{\pgfqpoint{1.711645in}{0.499444in}}%
\pgfpathlineto{\pgfqpoint{1.773031in}{0.499444in}}%
\pgfpathlineto{\pgfqpoint{1.773031in}{0.590743in}}%
\pgfpathlineto{\pgfqpoint{1.711645in}{0.590743in}}%
\pgfpathlineto{\pgfqpoint{1.711645in}{0.499444in}}%
\pgfpathclose%
\pgfusepath{fill}%
\end{pgfscope}%
\begin{pgfscope}%
\pgfpathrectangle{\pgfqpoint{0.445556in}{0.499444in}}{\pgfqpoint{3.875000in}{1.155000in}}%
\pgfusepath{clip}%
\pgfsetbuttcap%
\pgfsetmiterjoin%
\definecolor{currentfill}{rgb}{0.000000,0.000000,0.000000}%
\pgfsetfillcolor{currentfill}%
\pgfsetlinewidth{0.000000pt}%
\definecolor{currentstroke}{rgb}{0.000000,0.000000,0.000000}%
\pgfsetstrokecolor{currentstroke}%
\pgfsetstrokeopacity{0.000000}%
\pgfsetdash{}{0pt}%
\pgfpathmoveto{\pgfqpoint{1.865110in}{0.499444in}}%
\pgfpathlineto{\pgfqpoint{1.926496in}{0.499444in}}%
\pgfpathlineto{\pgfqpoint{1.926496in}{0.605521in}}%
\pgfpathlineto{\pgfqpoint{1.865110in}{0.605521in}}%
\pgfpathlineto{\pgfqpoint{1.865110in}{0.499444in}}%
\pgfpathclose%
\pgfusepath{fill}%
\end{pgfscope}%
\begin{pgfscope}%
\pgfpathrectangle{\pgfqpoint{0.445556in}{0.499444in}}{\pgfqpoint{3.875000in}{1.155000in}}%
\pgfusepath{clip}%
\pgfsetbuttcap%
\pgfsetmiterjoin%
\definecolor{currentfill}{rgb}{0.000000,0.000000,0.000000}%
\pgfsetfillcolor{currentfill}%
\pgfsetlinewidth{0.000000pt}%
\definecolor{currentstroke}{rgb}{0.000000,0.000000,0.000000}%
\pgfsetstrokecolor{currentstroke}%
\pgfsetstrokeopacity{0.000000}%
\pgfsetdash{}{0pt}%
\pgfpathmoveto{\pgfqpoint{2.018575in}{0.499444in}}%
\pgfpathlineto{\pgfqpoint{2.079962in}{0.499444in}}%
\pgfpathlineto{\pgfqpoint{2.079962in}{0.624632in}}%
\pgfpathlineto{\pgfqpoint{2.018575in}{0.624632in}}%
\pgfpathlineto{\pgfqpoint{2.018575in}{0.499444in}}%
\pgfpathclose%
\pgfusepath{fill}%
\end{pgfscope}%
\begin{pgfscope}%
\pgfpathrectangle{\pgfqpoint{0.445556in}{0.499444in}}{\pgfqpoint{3.875000in}{1.155000in}}%
\pgfusepath{clip}%
\pgfsetbuttcap%
\pgfsetmiterjoin%
\definecolor{currentfill}{rgb}{0.000000,0.000000,0.000000}%
\pgfsetfillcolor{currentfill}%
\pgfsetlinewidth{0.000000pt}%
\definecolor{currentstroke}{rgb}{0.000000,0.000000,0.000000}%
\pgfsetstrokecolor{currentstroke}%
\pgfsetstrokeopacity{0.000000}%
\pgfsetdash{}{0pt}%
\pgfpathmoveto{\pgfqpoint{2.172041in}{0.499444in}}%
\pgfpathlineto{\pgfqpoint{2.233427in}{0.499444in}}%
\pgfpathlineto{\pgfqpoint{2.233427in}{0.643356in}}%
\pgfpathlineto{\pgfqpoint{2.172041in}{0.643356in}}%
\pgfpathlineto{\pgfqpoint{2.172041in}{0.499444in}}%
\pgfpathclose%
\pgfusepath{fill}%
\end{pgfscope}%
\begin{pgfscope}%
\pgfpathrectangle{\pgfqpoint{0.445556in}{0.499444in}}{\pgfqpoint{3.875000in}{1.155000in}}%
\pgfusepath{clip}%
\pgfsetbuttcap%
\pgfsetmiterjoin%
\definecolor{currentfill}{rgb}{0.000000,0.000000,0.000000}%
\pgfsetfillcolor{currentfill}%
\pgfsetlinewidth{0.000000pt}%
\definecolor{currentstroke}{rgb}{0.000000,0.000000,0.000000}%
\pgfsetstrokecolor{currentstroke}%
\pgfsetstrokeopacity{0.000000}%
\pgfsetdash{}{0pt}%
\pgfpathmoveto{\pgfqpoint{2.325506in}{0.499444in}}%
\pgfpathlineto{\pgfqpoint{2.386892in}{0.499444in}}%
\pgfpathlineto{\pgfqpoint{2.386892in}{0.662622in}}%
\pgfpathlineto{\pgfqpoint{2.325506in}{0.662622in}}%
\pgfpathlineto{\pgfqpoint{2.325506in}{0.499444in}}%
\pgfpathclose%
\pgfusepath{fill}%
\end{pgfscope}%
\begin{pgfscope}%
\pgfpathrectangle{\pgfqpoint{0.445556in}{0.499444in}}{\pgfqpoint{3.875000in}{1.155000in}}%
\pgfusepath{clip}%
\pgfsetbuttcap%
\pgfsetmiterjoin%
\definecolor{currentfill}{rgb}{0.000000,0.000000,0.000000}%
\pgfsetfillcolor{currentfill}%
\pgfsetlinewidth{0.000000pt}%
\definecolor{currentstroke}{rgb}{0.000000,0.000000,0.000000}%
\pgfsetstrokecolor{currentstroke}%
\pgfsetstrokeopacity{0.000000}%
\pgfsetdash{}{0pt}%
\pgfpathmoveto{\pgfqpoint{2.478972in}{0.499444in}}%
\pgfpathlineto{\pgfqpoint{2.540358in}{0.499444in}}%
\pgfpathlineto{\pgfqpoint{2.540358in}{0.672216in}}%
\pgfpathlineto{\pgfqpoint{2.478972in}{0.672216in}}%
\pgfpathlineto{\pgfqpoint{2.478972in}{0.499444in}}%
\pgfpathclose%
\pgfusepath{fill}%
\end{pgfscope}%
\begin{pgfscope}%
\pgfpathrectangle{\pgfqpoint{0.445556in}{0.499444in}}{\pgfqpoint{3.875000in}{1.155000in}}%
\pgfusepath{clip}%
\pgfsetbuttcap%
\pgfsetmiterjoin%
\definecolor{currentfill}{rgb}{0.000000,0.000000,0.000000}%
\pgfsetfillcolor{currentfill}%
\pgfsetlinewidth{0.000000pt}%
\definecolor{currentstroke}{rgb}{0.000000,0.000000,0.000000}%
\pgfsetstrokecolor{currentstroke}%
\pgfsetstrokeopacity{0.000000}%
\pgfsetdash{}{0pt}%
\pgfpathmoveto{\pgfqpoint{2.632437in}{0.499444in}}%
\pgfpathlineto{\pgfqpoint{2.693823in}{0.499444in}}%
\pgfpathlineto{\pgfqpoint{2.693823in}{0.684286in}}%
\pgfpathlineto{\pgfqpoint{2.632437in}{0.684286in}}%
\pgfpathlineto{\pgfqpoint{2.632437in}{0.499444in}}%
\pgfpathclose%
\pgfusepath{fill}%
\end{pgfscope}%
\begin{pgfscope}%
\pgfpathrectangle{\pgfqpoint{0.445556in}{0.499444in}}{\pgfqpoint{3.875000in}{1.155000in}}%
\pgfusepath{clip}%
\pgfsetbuttcap%
\pgfsetmiterjoin%
\definecolor{currentfill}{rgb}{0.000000,0.000000,0.000000}%
\pgfsetfillcolor{currentfill}%
\pgfsetlinewidth{0.000000pt}%
\definecolor{currentstroke}{rgb}{0.000000,0.000000,0.000000}%
\pgfsetstrokecolor{currentstroke}%
\pgfsetstrokeopacity{0.000000}%
\pgfsetdash{}{0pt}%
\pgfpathmoveto{\pgfqpoint{2.785902in}{0.499444in}}%
\pgfpathlineto{\pgfqpoint{2.847288in}{0.499444in}}%
\pgfpathlineto{\pgfqpoint{2.847288in}{0.686607in}}%
\pgfpathlineto{\pgfqpoint{2.785902in}{0.686607in}}%
\pgfpathlineto{\pgfqpoint{2.785902in}{0.499444in}}%
\pgfpathclose%
\pgfusepath{fill}%
\end{pgfscope}%
\begin{pgfscope}%
\pgfpathrectangle{\pgfqpoint{0.445556in}{0.499444in}}{\pgfqpoint{3.875000in}{1.155000in}}%
\pgfusepath{clip}%
\pgfsetbuttcap%
\pgfsetmiterjoin%
\definecolor{currentfill}{rgb}{0.000000,0.000000,0.000000}%
\pgfsetfillcolor{currentfill}%
\pgfsetlinewidth{0.000000pt}%
\definecolor{currentstroke}{rgb}{0.000000,0.000000,0.000000}%
\pgfsetstrokecolor{currentstroke}%
\pgfsetstrokeopacity{0.000000}%
\pgfsetdash{}{0pt}%
\pgfpathmoveto{\pgfqpoint{2.939368in}{0.499444in}}%
\pgfpathlineto{\pgfqpoint{3.000754in}{0.499444in}}%
\pgfpathlineto{\pgfqpoint{3.000754in}{0.682816in}}%
\pgfpathlineto{\pgfqpoint{2.939368in}{0.682816in}}%
\pgfpathlineto{\pgfqpoint{2.939368in}{0.499444in}}%
\pgfpathclose%
\pgfusepath{fill}%
\end{pgfscope}%
\begin{pgfscope}%
\pgfpathrectangle{\pgfqpoint{0.445556in}{0.499444in}}{\pgfqpoint{3.875000in}{1.155000in}}%
\pgfusepath{clip}%
\pgfsetbuttcap%
\pgfsetmiterjoin%
\definecolor{currentfill}{rgb}{0.000000,0.000000,0.000000}%
\pgfsetfillcolor{currentfill}%
\pgfsetlinewidth{0.000000pt}%
\definecolor{currentstroke}{rgb}{0.000000,0.000000,0.000000}%
\pgfsetstrokecolor{currentstroke}%
\pgfsetstrokeopacity{0.000000}%
\pgfsetdash{}{0pt}%
\pgfpathmoveto{\pgfqpoint{3.092833in}{0.499444in}}%
\pgfpathlineto{\pgfqpoint{3.154219in}{0.499444in}}%
\pgfpathlineto{\pgfqpoint{3.154219in}{0.681037in}}%
\pgfpathlineto{\pgfqpoint{3.092833in}{0.681037in}}%
\pgfpathlineto{\pgfqpoint{3.092833in}{0.499444in}}%
\pgfpathclose%
\pgfusepath{fill}%
\end{pgfscope}%
\begin{pgfscope}%
\pgfpathrectangle{\pgfqpoint{0.445556in}{0.499444in}}{\pgfqpoint{3.875000in}{1.155000in}}%
\pgfusepath{clip}%
\pgfsetbuttcap%
\pgfsetmiterjoin%
\definecolor{currentfill}{rgb}{0.000000,0.000000,0.000000}%
\pgfsetfillcolor{currentfill}%
\pgfsetlinewidth{0.000000pt}%
\definecolor{currentstroke}{rgb}{0.000000,0.000000,0.000000}%
\pgfsetstrokecolor{currentstroke}%
\pgfsetstrokeopacity{0.000000}%
\pgfsetdash{}{0pt}%
\pgfpathmoveto{\pgfqpoint{3.246298in}{0.499444in}}%
\pgfpathlineto{\pgfqpoint{3.307684in}{0.499444in}}%
\pgfpathlineto{\pgfqpoint{3.307684in}{0.677787in}}%
\pgfpathlineto{\pgfqpoint{3.246298in}{0.677787in}}%
\pgfpathlineto{\pgfqpoint{3.246298in}{0.499444in}}%
\pgfpathclose%
\pgfusepath{fill}%
\end{pgfscope}%
\begin{pgfscope}%
\pgfpathrectangle{\pgfqpoint{0.445556in}{0.499444in}}{\pgfqpoint{3.875000in}{1.155000in}}%
\pgfusepath{clip}%
\pgfsetbuttcap%
\pgfsetmiterjoin%
\definecolor{currentfill}{rgb}{0.000000,0.000000,0.000000}%
\pgfsetfillcolor{currentfill}%
\pgfsetlinewidth{0.000000pt}%
\definecolor{currentstroke}{rgb}{0.000000,0.000000,0.000000}%
\pgfsetstrokecolor{currentstroke}%
\pgfsetstrokeopacity{0.000000}%
\pgfsetdash{}{0pt}%
\pgfpathmoveto{\pgfqpoint{3.399764in}{0.499444in}}%
\pgfpathlineto{\pgfqpoint{3.461150in}{0.499444in}}%
\pgfpathlineto{\pgfqpoint{3.461150in}{0.659527in}}%
\pgfpathlineto{\pgfqpoint{3.399764in}{0.659527in}}%
\pgfpathlineto{\pgfqpoint{3.399764in}{0.499444in}}%
\pgfpathclose%
\pgfusepath{fill}%
\end{pgfscope}%
\begin{pgfscope}%
\pgfpathrectangle{\pgfqpoint{0.445556in}{0.499444in}}{\pgfqpoint{3.875000in}{1.155000in}}%
\pgfusepath{clip}%
\pgfsetbuttcap%
\pgfsetmiterjoin%
\definecolor{currentfill}{rgb}{0.000000,0.000000,0.000000}%
\pgfsetfillcolor{currentfill}%
\pgfsetlinewidth{0.000000pt}%
\definecolor{currentstroke}{rgb}{0.000000,0.000000,0.000000}%
\pgfsetstrokecolor{currentstroke}%
\pgfsetstrokeopacity{0.000000}%
\pgfsetdash{}{0pt}%
\pgfpathmoveto{\pgfqpoint{3.553229in}{0.499444in}}%
\pgfpathlineto{\pgfqpoint{3.614615in}{0.499444in}}%
\pgfpathlineto{\pgfqpoint{3.614615in}{0.657206in}}%
\pgfpathlineto{\pgfqpoint{3.553229in}{0.657206in}}%
\pgfpathlineto{\pgfqpoint{3.553229in}{0.499444in}}%
\pgfpathclose%
\pgfusepath{fill}%
\end{pgfscope}%
\begin{pgfscope}%
\pgfpathrectangle{\pgfqpoint{0.445556in}{0.499444in}}{\pgfqpoint{3.875000in}{1.155000in}}%
\pgfusepath{clip}%
\pgfsetbuttcap%
\pgfsetmiterjoin%
\definecolor{currentfill}{rgb}{0.000000,0.000000,0.000000}%
\pgfsetfillcolor{currentfill}%
\pgfsetlinewidth{0.000000pt}%
\definecolor{currentstroke}{rgb}{0.000000,0.000000,0.000000}%
\pgfsetstrokecolor{currentstroke}%
\pgfsetstrokeopacity{0.000000}%
\pgfsetdash{}{0pt}%
\pgfpathmoveto{\pgfqpoint{3.706694in}{0.499444in}}%
\pgfpathlineto{\pgfqpoint{3.768080in}{0.499444in}}%
\pgfpathlineto{\pgfqpoint{3.768080in}{0.639333in}}%
\pgfpathlineto{\pgfqpoint{3.706694in}{0.639333in}}%
\pgfpathlineto{\pgfqpoint{3.706694in}{0.499444in}}%
\pgfpathclose%
\pgfusepath{fill}%
\end{pgfscope}%
\begin{pgfscope}%
\pgfpathrectangle{\pgfqpoint{0.445556in}{0.499444in}}{\pgfqpoint{3.875000in}{1.155000in}}%
\pgfusepath{clip}%
\pgfsetbuttcap%
\pgfsetmiterjoin%
\definecolor{currentfill}{rgb}{0.000000,0.000000,0.000000}%
\pgfsetfillcolor{currentfill}%
\pgfsetlinewidth{0.000000pt}%
\definecolor{currentstroke}{rgb}{0.000000,0.000000,0.000000}%
\pgfsetstrokecolor{currentstroke}%
\pgfsetstrokeopacity{0.000000}%
\pgfsetdash{}{0pt}%
\pgfpathmoveto{\pgfqpoint{3.860160in}{0.499444in}}%
\pgfpathlineto{\pgfqpoint{3.921546in}{0.499444in}}%
\pgfpathlineto{\pgfqpoint{3.921546in}{0.623626in}}%
\pgfpathlineto{\pgfqpoint{3.860160in}{0.623626in}}%
\pgfpathlineto{\pgfqpoint{3.860160in}{0.499444in}}%
\pgfpathclose%
\pgfusepath{fill}%
\end{pgfscope}%
\begin{pgfscope}%
\pgfpathrectangle{\pgfqpoint{0.445556in}{0.499444in}}{\pgfqpoint{3.875000in}{1.155000in}}%
\pgfusepath{clip}%
\pgfsetbuttcap%
\pgfsetmiterjoin%
\definecolor{currentfill}{rgb}{0.000000,0.000000,0.000000}%
\pgfsetfillcolor{currentfill}%
\pgfsetlinewidth{0.000000pt}%
\definecolor{currentstroke}{rgb}{0.000000,0.000000,0.000000}%
\pgfsetstrokecolor{currentstroke}%
\pgfsetstrokeopacity{0.000000}%
\pgfsetdash{}{0pt}%
\pgfpathmoveto{\pgfqpoint{4.013625in}{0.499444in}}%
\pgfpathlineto{\pgfqpoint{4.075011in}{0.499444in}}%
\pgfpathlineto{\pgfqpoint{4.075011in}{0.580840in}}%
\pgfpathlineto{\pgfqpoint{4.013625in}{0.580840in}}%
\pgfpathlineto{\pgfqpoint{4.013625in}{0.499444in}}%
\pgfpathclose%
\pgfusepath{fill}%
\end{pgfscope}%
\begin{pgfscope}%
\pgfpathrectangle{\pgfqpoint{0.445556in}{0.499444in}}{\pgfqpoint{3.875000in}{1.155000in}}%
\pgfusepath{clip}%
\pgfsetbuttcap%
\pgfsetmiterjoin%
\definecolor{currentfill}{rgb}{0.000000,0.000000,0.000000}%
\pgfsetfillcolor{currentfill}%
\pgfsetlinewidth{0.000000pt}%
\definecolor{currentstroke}{rgb}{0.000000,0.000000,0.000000}%
\pgfsetstrokecolor{currentstroke}%
\pgfsetstrokeopacity{0.000000}%
\pgfsetdash{}{0pt}%
\pgfpathmoveto{\pgfqpoint{4.167090in}{0.499444in}}%
\pgfpathlineto{\pgfqpoint{4.228476in}{0.499444in}}%
\pgfpathlineto{\pgfqpoint{4.228476in}{0.529774in}}%
\pgfpathlineto{\pgfqpoint{4.167090in}{0.529774in}}%
\pgfpathlineto{\pgfqpoint{4.167090in}{0.499444in}}%
\pgfpathclose%
\pgfusepath{fill}%
\end{pgfscope}%
\begin{pgfscope}%
\pgfsetbuttcap%
\pgfsetroundjoin%
\definecolor{currentfill}{rgb}{0.000000,0.000000,0.000000}%
\pgfsetfillcolor{currentfill}%
\pgfsetlinewidth{0.803000pt}%
\definecolor{currentstroke}{rgb}{0.000000,0.000000,0.000000}%
\pgfsetstrokecolor{currentstroke}%
\pgfsetdash{}{0pt}%
\pgfsys@defobject{currentmarker}{\pgfqpoint{0.000000in}{-0.048611in}}{\pgfqpoint{0.000000in}{0.000000in}}{%
\pgfpathmoveto{\pgfqpoint{0.000000in}{0.000000in}}%
\pgfpathlineto{\pgfqpoint{0.000000in}{-0.048611in}}%
\pgfusepath{stroke,fill}%
}%
\begin{pgfscope}%
\pgfsys@transformshift{0.483922in}{0.499444in}%
\pgfsys@useobject{currentmarker}{}%
\end{pgfscope}%
\end{pgfscope}%
\begin{pgfscope}%
\definecolor{textcolor}{rgb}{0.000000,0.000000,0.000000}%
\pgfsetstrokecolor{textcolor}%
\pgfsetfillcolor{textcolor}%
\pgftext[x=0.483922in,y=0.402222in,,top]{\color{textcolor}\rmfamily\fontsize{10.000000}{12.000000}\selectfont 0.0}%
\end{pgfscope}%
\begin{pgfscope}%
\pgfsetbuttcap%
\pgfsetroundjoin%
\definecolor{currentfill}{rgb}{0.000000,0.000000,0.000000}%
\pgfsetfillcolor{currentfill}%
\pgfsetlinewidth{0.803000pt}%
\definecolor{currentstroke}{rgb}{0.000000,0.000000,0.000000}%
\pgfsetstrokecolor{currentstroke}%
\pgfsetdash{}{0pt}%
\pgfsys@defobject{currentmarker}{\pgfqpoint{0.000000in}{-0.048611in}}{\pgfqpoint{0.000000in}{0.000000in}}{%
\pgfpathmoveto{\pgfqpoint{0.000000in}{0.000000in}}%
\pgfpathlineto{\pgfqpoint{0.000000in}{-0.048611in}}%
\pgfusepath{stroke,fill}%
}%
\begin{pgfscope}%
\pgfsys@transformshift{0.867585in}{0.499444in}%
\pgfsys@useobject{currentmarker}{}%
\end{pgfscope}%
\end{pgfscope}%
\begin{pgfscope}%
\definecolor{textcolor}{rgb}{0.000000,0.000000,0.000000}%
\pgfsetstrokecolor{textcolor}%
\pgfsetfillcolor{textcolor}%
\pgftext[x=0.867585in,y=0.402222in,,top]{\color{textcolor}\rmfamily\fontsize{10.000000}{12.000000}\selectfont 0.1}%
\end{pgfscope}%
\begin{pgfscope}%
\pgfsetbuttcap%
\pgfsetroundjoin%
\definecolor{currentfill}{rgb}{0.000000,0.000000,0.000000}%
\pgfsetfillcolor{currentfill}%
\pgfsetlinewidth{0.803000pt}%
\definecolor{currentstroke}{rgb}{0.000000,0.000000,0.000000}%
\pgfsetstrokecolor{currentstroke}%
\pgfsetdash{}{0pt}%
\pgfsys@defobject{currentmarker}{\pgfqpoint{0.000000in}{-0.048611in}}{\pgfqpoint{0.000000in}{0.000000in}}{%
\pgfpathmoveto{\pgfqpoint{0.000000in}{0.000000in}}%
\pgfpathlineto{\pgfqpoint{0.000000in}{-0.048611in}}%
\pgfusepath{stroke,fill}%
}%
\begin{pgfscope}%
\pgfsys@transformshift{1.251249in}{0.499444in}%
\pgfsys@useobject{currentmarker}{}%
\end{pgfscope}%
\end{pgfscope}%
\begin{pgfscope}%
\definecolor{textcolor}{rgb}{0.000000,0.000000,0.000000}%
\pgfsetstrokecolor{textcolor}%
\pgfsetfillcolor{textcolor}%
\pgftext[x=1.251249in,y=0.402222in,,top]{\color{textcolor}\rmfamily\fontsize{10.000000}{12.000000}\selectfont 0.2}%
\end{pgfscope}%
\begin{pgfscope}%
\pgfsetbuttcap%
\pgfsetroundjoin%
\definecolor{currentfill}{rgb}{0.000000,0.000000,0.000000}%
\pgfsetfillcolor{currentfill}%
\pgfsetlinewidth{0.803000pt}%
\definecolor{currentstroke}{rgb}{0.000000,0.000000,0.000000}%
\pgfsetstrokecolor{currentstroke}%
\pgfsetdash{}{0pt}%
\pgfsys@defobject{currentmarker}{\pgfqpoint{0.000000in}{-0.048611in}}{\pgfqpoint{0.000000in}{0.000000in}}{%
\pgfpathmoveto{\pgfqpoint{0.000000in}{0.000000in}}%
\pgfpathlineto{\pgfqpoint{0.000000in}{-0.048611in}}%
\pgfusepath{stroke,fill}%
}%
\begin{pgfscope}%
\pgfsys@transformshift{1.634912in}{0.499444in}%
\pgfsys@useobject{currentmarker}{}%
\end{pgfscope}%
\end{pgfscope}%
\begin{pgfscope}%
\definecolor{textcolor}{rgb}{0.000000,0.000000,0.000000}%
\pgfsetstrokecolor{textcolor}%
\pgfsetfillcolor{textcolor}%
\pgftext[x=1.634912in,y=0.402222in,,top]{\color{textcolor}\rmfamily\fontsize{10.000000}{12.000000}\selectfont 0.3}%
\end{pgfscope}%
\begin{pgfscope}%
\pgfsetbuttcap%
\pgfsetroundjoin%
\definecolor{currentfill}{rgb}{0.000000,0.000000,0.000000}%
\pgfsetfillcolor{currentfill}%
\pgfsetlinewidth{0.803000pt}%
\definecolor{currentstroke}{rgb}{0.000000,0.000000,0.000000}%
\pgfsetstrokecolor{currentstroke}%
\pgfsetdash{}{0pt}%
\pgfsys@defobject{currentmarker}{\pgfqpoint{0.000000in}{-0.048611in}}{\pgfqpoint{0.000000in}{0.000000in}}{%
\pgfpathmoveto{\pgfqpoint{0.000000in}{0.000000in}}%
\pgfpathlineto{\pgfqpoint{0.000000in}{-0.048611in}}%
\pgfusepath{stroke,fill}%
}%
\begin{pgfscope}%
\pgfsys@transformshift{2.018575in}{0.499444in}%
\pgfsys@useobject{currentmarker}{}%
\end{pgfscope}%
\end{pgfscope}%
\begin{pgfscope}%
\definecolor{textcolor}{rgb}{0.000000,0.000000,0.000000}%
\pgfsetstrokecolor{textcolor}%
\pgfsetfillcolor{textcolor}%
\pgftext[x=2.018575in,y=0.402222in,,top]{\color{textcolor}\rmfamily\fontsize{10.000000}{12.000000}\selectfont 0.4}%
\end{pgfscope}%
\begin{pgfscope}%
\pgfsetbuttcap%
\pgfsetroundjoin%
\definecolor{currentfill}{rgb}{0.000000,0.000000,0.000000}%
\pgfsetfillcolor{currentfill}%
\pgfsetlinewidth{0.803000pt}%
\definecolor{currentstroke}{rgb}{0.000000,0.000000,0.000000}%
\pgfsetstrokecolor{currentstroke}%
\pgfsetdash{}{0pt}%
\pgfsys@defobject{currentmarker}{\pgfqpoint{0.000000in}{-0.048611in}}{\pgfqpoint{0.000000in}{0.000000in}}{%
\pgfpathmoveto{\pgfqpoint{0.000000in}{0.000000in}}%
\pgfpathlineto{\pgfqpoint{0.000000in}{-0.048611in}}%
\pgfusepath{stroke,fill}%
}%
\begin{pgfscope}%
\pgfsys@transformshift{2.402239in}{0.499444in}%
\pgfsys@useobject{currentmarker}{}%
\end{pgfscope}%
\end{pgfscope}%
\begin{pgfscope}%
\definecolor{textcolor}{rgb}{0.000000,0.000000,0.000000}%
\pgfsetstrokecolor{textcolor}%
\pgfsetfillcolor{textcolor}%
\pgftext[x=2.402239in,y=0.402222in,,top]{\color{textcolor}\rmfamily\fontsize{10.000000}{12.000000}\selectfont 0.5}%
\end{pgfscope}%
\begin{pgfscope}%
\pgfsetbuttcap%
\pgfsetroundjoin%
\definecolor{currentfill}{rgb}{0.000000,0.000000,0.000000}%
\pgfsetfillcolor{currentfill}%
\pgfsetlinewidth{0.803000pt}%
\definecolor{currentstroke}{rgb}{0.000000,0.000000,0.000000}%
\pgfsetstrokecolor{currentstroke}%
\pgfsetdash{}{0pt}%
\pgfsys@defobject{currentmarker}{\pgfqpoint{0.000000in}{-0.048611in}}{\pgfqpoint{0.000000in}{0.000000in}}{%
\pgfpathmoveto{\pgfqpoint{0.000000in}{0.000000in}}%
\pgfpathlineto{\pgfqpoint{0.000000in}{-0.048611in}}%
\pgfusepath{stroke,fill}%
}%
\begin{pgfscope}%
\pgfsys@transformshift{2.785902in}{0.499444in}%
\pgfsys@useobject{currentmarker}{}%
\end{pgfscope}%
\end{pgfscope}%
\begin{pgfscope}%
\definecolor{textcolor}{rgb}{0.000000,0.000000,0.000000}%
\pgfsetstrokecolor{textcolor}%
\pgfsetfillcolor{textcolor}%
\pgftext[x=2.785902in,y=0.402222in,,top]{\color{textcolor}\rmfamily\fontsize{10.000000}{12.000000}\selectfont 0.6}%
\end{pgfscope}%
\begin{pgfscope}%
\pgfsetbuttcap%
\pgfsetroundjoin%
\definecolor{currentfill}{rgb}{0.000000,0.000000,0.000000}%
\pgfsetfillcolor{currentfill}%
\pgfsetlinewidth{0.803000pt}%
\definecolor{currentstroke}{rgb}{0.000000,0.000000,0.000000}%
\pgfsetstrokecolor{currentstroke}%
\pgfsetdash{}{0pt}%
\pgfsys@defobject{currentmarker}{\pgfqpoint{0.000000in}{-0.048611in}}{\pgfqpoint{0.000000in}{0.000000in}}{%
\pgfpathmoveto{\pgfqpoint{0.000000in}{0.000000in}}%
\pgfpathlineto{\pgfqpoint{0.000000in}{-0.048611in}}%
\pgfusepath{stroke,fill}%
}%
\begin{pgfscope}%
\pgfsys@transformshift{3.169566in}{0.499444in}%
\pgfsys@useobject{currentmarker}{}%
\end{pgfscope}%
\end{pgfscope}%
\begin{pgfscope}%
\definecolor{textcolor}{rgb}{0.000000,0.000000,0.000000}%
\pgfsetstrokecolor{textcolor}%
\pgfsetfillcolor{textcolor}%
\pgftext[x=3.169566in,y=0.402222in,,top]{\color{textcolor}\rmfamily\fontsize{10.000000}{12.000000}\selectfont 0.7}%
\end{pgfscope}%
\begin{pgfscope}%
\pgfsetbuttcap%
\pgfsetroundjoin%
\definecolor{currentfill}{rgb}{0.000000,0.000000,0.000000}%
\pgfsetfillcolor{currentfill}%
\pgfsetlinewidth{0.803000pt}%
\definecolor{currentstroke}{rgb}{0.000000,0.000000,0.000000}%
\pgfsetstrokecolor{currentstroke}%
\pgfsetdash{}{0pt}%
\pgfsys@defobject{currentmarker}{\pgfqpoint{0.000000in}{-0.048611in}}{\pgfqpoint{0.000000in}{0.000000in}}{%
\pgfpathmoveto{\pgfqpoint{0.000000in}{0.000000in}}%
\pgfpathlineto{\pgfqpoint{0.000000in}{-0.048611in}}%
\pgfusepath{stroke,fill}%
}%
\begin{pgfscope}%
\pgfsys@transformshift{3.553229in}{0.499444in}%
\pgfsys@useobject{currentmarker}{}%
\end{pgfscope}%
\end{pgfscope}%
\begin{pgfscope}%
\definecolor{textcolor}{rgb}{0.000000,0.000000,0.000000}%
\pgfsetstrokecolor{textcolor}%
\pgfsetfillcolor{textcolor}%
\pgftext[x=3.553229in,y=0.402222in,,top]{\color{textcolor}\rmfamily\fontsize{10.000000}{12.000000}\selectfont 0.8}%
\end{pgfscope}%
\begin{pgfscope}%
\pgfsetbuttcap%
\pgfsetroundjoin%
\definecolor{currentfill}{rgb}{0.000000,0.000000,0.000000}%
\pgfsetfillcolor{currentfill}%
\pgfsetlinewidth{0.803000pt}%
\definecolor{currentstroke}{rgb}{0.000000,0.000000,0.000000}%
\pgfsetstrokecolor{currentstroke}%
\pgfsetdash{}{0pt}%
\pgfsys@defobject{currentmarker}{\pgfqpoint{0.000000in}{-0.048611in}}{\pgfqpoint{0.000000in}{0.000000in}}{%
\pgfpathmoveto{\pgfqpoint{0.000000in}{0.000000in}}%
\pgfpathlineto{\pgfqpoint{0.000000in}{-0.048611in}}%
\pgfusepath{stroke,fill}%
}%
\begin{pgfscope}%
\pgfsys@transformshift{3.936892in}{0.499444in}%
\pgfsys@useobject{currentmarker}{}%
\end{pgfscope}%
\end{pgfscope}%
\begin{pgfscope}%
\definecolor{textcolor}{rgb}{0.000000,0.000000,0.000000}%
\pgfsetstrokecolor{textcolor}%
\pgfsetfillcolor{textcolor}%
\pgftext[x=3.936892in,y=0.402222in,,top]{\color{textcolor}\rmfamily\fontsize{10.000000}{12.000000}\selectfont 0.9}%
\end{pgfscope}%
\begin{pgfscope}%
\pgfsetbuttcap%
\pgfsetroundjoin%
\definecolor{currentfill}{rgb}{0.000000,0.000000,0.000000}%
\pgfsetfillcolor{currentfill}%
\pgfsetlinewidth{0.803000pt}%
\definecolor{currentstroke}{rgb}{0.000000,0.000000,0.000000}%
\pgfsetstrokecolor{currentstroke}%
\pgfsetdash{}{0pt}%
\pgfsys@defobject{currentmarker}{\pgfqpoint{0.000000in}{-0.048611in}}{\pgfqpoint{0.000000in}{0.000000in}}{%
\pgfpathmoveto{\pgfqpoint{0.000000in}{0.000000in}}%
\pgfpathlineto{\pgfqpoint{0.000000in}{-0.048611in}}%
\pgfusepath{stroke,fill}%
}%
\begin{pgfscope}%
\pgfsys@transformshift{4.320556in}{0.499444in}%
\pgfsys@useobject{currentmarker}{}%
\end{pgfscope}%
\end{pgfscope}%
\begin{pgfscope}%
\definecolor{textcolor}{rgb}{0.000000,0.000000,0.000000}%
\pgfsetstrokecolor{textcolor}%
\pgfsetfillcolor{textcolor}%
\pgftext[x=4.320556in,y=0.402222in,,top]{\color{textcolor}\rmfamily\fontsize{10.000000}{12.000000}\selectfont 1.0}%
\end{pgfscope}%
\begin{pgfscope}%
\definecolor{textcolor}{rgb}{0.000000,0.000000,0.000000}%
\pgfsetstrokecolor{textcolor}%
\pgfsetfillcolor{textcolor}%
\pgftext[x=2.383056in,y=0.223333in,,top]{\color{textcolor}\rmfamily\fontsize{10.000000}{12.000000}\selectfont \(\displaystyle p\)}%
\end{pgfscope}%
\begin{pgfscope}%
\pgfsetbuttcap%
\pgfsetroundjoin%
\definecolor{currentfill}{rgb}{0.000000,0.000000,0.000000}%
\pgfsetfillcolor{currentfill}%
\pgfsetlinewidth{0.803000pt}%
\definecolor{currentstroke}{rgb}{0.000000,0.000000,0.000000}%
\pgfsetstrokecolor{currentstroke}%
\pgfsetdash{}{0pt}%
\pgfsys@defobject{currentmarker}{\pgfqpoint{-0.048611in}{0.000000in}}{\pgfqpoint{-0.000000in}{0.000000in}}{%
\pgfpathmoveto{\pgfqpoint{-0.000000in}{0.000000in}}%
\pgfpathlineto{\pgfqpoint{-0.048611in}{0.000000in}}%
\pgfusepath{stroke,fill}%
}%
\begin{pgfscope}%
\pgfsys@transformshift{0.445556in}{0.499444in}%
\pgfsys@useobject{currentmarker}{}%
\end{pgfscope}%
\end{pgfscope}%
\begin{pgfscope}%
\definecolor{textcolor}{rgb}{0.000000,0.000000,0.000000}%
\pgfsetstrokecolor{textcolor}%
\pgfsetfillcolor{textcolor}%
\pgftext[x=0.278889in, y=0.451250in, left, base]{\color{textcolor}\rmfamily\fontsize{10.000000}{12.000000}\selectfont \(\displaystyle {0}\)}%
\end{pgfscope}%
\begin{pgfscope}%
\pgfsetbuttcap%
\pgfsetroundjoin%
\definecolor{currentfill}{rgb}{0.000000,0.000000,0.000000}%
\pgfsetfillcolor{currentfill}%
\pgfsetlinewidth{0.803000pt}%
\definecolor{currentstroke}{rgb}{0.000000,0.000000,0.000000}%
\pgfsetstrokecolor{currentstroke}%
\pgfsetdash{}{0pt}%
\pgfsys@defobject{currentmarker}{\pgfqpoint{-0.048611in}{0.000000in}}{\pgfqpoint{-0.000000in}{0.000000in}}{%
\pgfpathmoveto{\pgfqpoint{-0.000000in}{0.000000in}}%
\pgfpathlineto{\pgfqpoint{-0.048611in}{0.000000in}}%
\pgfusepath{stroke,fill}%
}%
\begin{pgfscope}%
\pgfsys@transformshift{0.445556in}{0.830705in}%
\pgfsys@useobject{currentmarker}{}%
\end{pgfscope}%
\end{pgfscope}%
\begin{pgfscope}%
\definecolor{textcolor}{rgb}{0.000000,0.000000,0.000000}%
\pgfsetstrokecolor{textcolor}%
\pgfsetfillcolor{textcolor}%
\pgftext[x=0.278889in, y=0.782511in, left, base]{\color{textcolor}\rmfamily\fontsize{10.000000}{12.000000}\selectfont \(\displaystyle {2}\)}%
\end{pgfscope}%
\begin{pgfscope}%
\pgfsetbuttcap%
\pgfsetroundjoin%
\definecolor{currentfill}{rgb}{0.000000,0.000000,0.000000}%
\pgfsetfillcolor{currentfill}%
\pgfsetlinewidth{0.803000pt}%
\definecolor{currentstroke}{rgb}{0.000000,0.000000,0.000000}%
\pgfsetstrokecolor{currentstroke}%
\pgfsetdash{}{0pt}%
\pgfsys@defobject{currentmarker}{\pgfqpoint{-0.048611in}{0.000000in}}{\pgfqpoint{-0.000000in}{0.000000in}}{%
\pgfpathmoveto{\pgfqpoint{-0.000000in}{0.000000in}}%
\pgfpathlineto{\pgfqpoint{-0.048611in}{0.000000in}}%
\pgfusepath{stroke,fill}%
}%
\begin{pgfscope}%
\pgfsys@transformshift{0.445556in}{1.161966in}%
\pgfsys@useobject{currentmarker}{}%
\end{pgfscope}%
\end{pgfscope}%
\begin{pgfscope}%
\definecolor{textcolor}{rgb}{0.000000,0.000000,0.000000}%
\pgfsetstrokecolor{textcolor}%
\pgfsetfillcolor{textcolor}%
\pgftext[x=0.278889in, y=1.113772in, left, base]{\color{textcolor}\rmfamily\fontsize{10.000000}{12.000000}\selectfont \(\displaystyle {4}\)}%
\end{pgfscope}%
\begin{pgfscope}%
\pgfsetbuttcap%
\pgfsetroundjoin%
\definecolor{currentfill}{rgb}{0.000000,0.000000,0.000000}%
\pgfsetfillcolor{currentfill}%
\pgfsetlinewidth{0.803000pt}%
\definecolor{currentstroke}{rgb}{0.000000,0.000000,0.000000}%
\pgfsetstrokecolor{currentstroke}%
\pgfsetdash{}{0pt}%
\pgfsys@defobject{currentmarker}{\pgfqpoint{-0.048611in}{0.000000in}}{\pgfqpoint{-0.000000in}{0.000000in}}{%
\pgfpathmoveto{\pgfqpoint{-0.000000in}{0.000000in}}%
\pgfpathlineto{\pgfqpoint{-0.048611in}{0.000000in}}%
\pgfusepath{stroke,fill}%
}%
\begin{pgfscope}%
\pgfsys@transformshift{0.445556in}{1.493228in}%
\pgfsys@useobject{currentmarker}{}%
\end{pgfscope}%
\end{pgfscope}%
\begin{pgfscope}%
\definecolor{textcolor}{rgb}{0.000000,0.000000,0.000000}%
\pgfsetstrokecolor{textcolor}%
\pgfsetfillcolor{textcolor}%
\pgftext[x=0.278889in, y=1.445033in, left, base]{\color{textcolor}\rmfamily\fontsize{10.000000}{12.000000}\selectfont \(\displaystyle {6}\)}%
\end{pgfscope}%
\begin{pgfscope}%
\definecolor{textcolor}{rgb}{0.000000,0.000000,0.000000}%
\pgfsetstrokecolor{textcolor}%
\pgfsetfillcolor{textcolor}%
\pgftext[x=0.223333in,y=1.076944in,,bottom,rotate=90.000000]{\color{textcolor}\rmfamily\fontsize{10.000000}{12.000000}\selectfont Percent of Data Set}%
\end{pgfscope}%
\begin{pgfscope}%
\pgfsetrectcap%
\pgfsetmiterjoin%
\pgfsetlinewidth{0.803000pt}%
\definecolor{currentstroke}{rgb}{0.000000,0.000000,0.000000}%
\pgfsetstrokecolor{currentstroke}%
\pgfsetdash{}{0pt}%
\pgfpathmoveto{\pgfqpoint{0.445556in}{0.499444in}}%
\pgfpathlineto{\pgfqpoint{0.445556in}{1.654444in}}%
\pgfusepath{stroke}%
\end{pgfscope}%
\begin{pgfscope}%
\pgfsetrectcap%
\pgfsetmiterjoin%
\pgfsetlinewidth{0.803000pt}%
\definecolor{currentstroke}{rgb}{0.000000,0.000000,0.000000}%
\pgfsetstrokecolor{currentstroke}%
\pgfsetdash{}{0pt}%
\pgfpathmoveto{\pgfqpoint{4.320556in}{0.499444in}}%
\pgfpathlineto{\pgfqpoint{4.320556in}{1.654444in}}%
\pgfusepath{stroke}%
\end{pgfscope}%
\begin{pgfscope}%
\pgfsetrectcap%
\pgfsetmiterjoin%
\pgfsetlinewidth{0.803000pt}%
\definecolor{currentstroke}{rgb}{0.000000,0.000000,0.000000}%
\pgfsetstrokecolor{currentstroke}%
\pgfsetdash{}{0pt}%
\pgfpathmoveto{\pgfqpoint{0.445556in}{0.499444in}}%
\pgfpathlineto{\pgfqpoint{4.320556in}{0.499444in}}%
\pgfusepath{stroke}%
\end{pgfscope}%
\begin{pgfscope}%
\pgfsetrectcap%
\pgfsetmiterjoin%
\pgfsetlinewidth{0.803000pt}%
\definecolor{currentstroke}{rgb}{0.000000,0.000000,0.000000}%
\pgfsetstrokecolor{currentstroke}%
\pgfsetdash{}{0pt}%
\pgfpathmoveto{\pgfqpoint{0.445556in}{1.654444in}}%
\pgfpathlineto{\pgfqpoint{4.320556in}{1.654444in}}%
\pgfusepath{stroke}%
\end{pgfscope}%
\begin{pgfscope}%
\pgfsetbuttcap%
\pgfsetmiterjoin%
\definecolor{currentfill}{rgb}{1.000000,1.000000,1.000000}%
\pgfsetfillcolor{currentfill}%
\pgfsetfillopacity{0.800000}%
\pgfsetlinewidth{1.003750pt}%
\definecolor{currentstroke}{rgb}{0.800000,0.800000,0.800000}%
\pgfsetstrokecolor{currentstroke}%
\pgfsetstrokeopacity{0.800000}%
\pgfsetdash{}{0pt}%
\pgfpathmoveto{\pgfqpoint{3.543611in}{1.154445in}}%
\pgfpathlineto{\pgfqpoint{4.223333in}{1.154445in}}%
\pgfpathquadraticcurveto{\pgfqpoint{4.251111in}{1.154445in}}{\pgfqpoint{4.251111in}{1.182222in}}%
\pgfpathlineto{\pgfqpoint{4.251111in}{1.557222in}}%
\pgfpathquadraticcurveto{\pgfqpoint{4.251111in}{1.585000in}}{\pgfqpoint{4.223333in}{1.585000in}}%
\pgfpathlineto{\pgfqpoint{3.543611in}{1.585000in}}%
\pgfpathquadraticcurveto{\pgfqpoint{3.515833in}{1.585000in}}{\pgfqpoint{3.515833in}{1.557222in}}%
\pgfpathlineto{\pgfqpoint{3.515833in}{1.182222in}}%
\pgfpathquadraticcurveto{\pgfqpoint{3.515833in}{1.154445in}}{\pgfqpoint{3.543611in}{1.154445in}}%
\pgfpathlineto{\pgfqpoint{3.543611in}{1.154445in}}%
\pgfpathclose%
\pgfusepath{stroke,fill}%
\end{pgfscope}%
\begin{pgfscope}%
\pgfsetbuttcap%
\pgfsetmiterjoin%
\pgfsetlinewidth{1.003750pt}%
\definecolor{currentstroke}{rgb}{0.000000,0.000000,0.000000}%
\pgfsetstrokecolor{currentstroke}%
\pgfsetdash{}{0pt}%
\pgfpathmoveto{\pgfqpoint{3.571389in}{1.432222in}}%
\pgfpathlineto{\pgfqpoint{3.849167in}{1.432222in}}%
\pgfpathlineto{\pgfqpoint{3.849167in}{1.529444in}}%
\pgfpathlineto{\pgfqpoint{3.571389in}{1.529444in}}%
\pgfpathlineto{\pgfqpoint{3.571389in}{1.432222in}}%
\pgfpathclose%
\pgfusepath{stroke}%
\end{pgfscope}%
\begin{pgfscope}%
\definecolor{textcolor}{rgb}{0.000000,0.000000,0.000000}%
\pgfsetstrokecolor{textcolor}%
\pgfsetfillcolor{textcolor}%
\pgftext[x=3.960278in,y=1.432222in,left,base]{\color{textcolor}\rmfamily\fontsize{10.000000}{12.000000}\selectfont Neg}%
\end{pgfscope}%
\begin{pgfscope}%
\pgfsetbuttcap%
\pgfsetmiterjoin%
\definecolor{currentfill}{rgb}{0.000000,0.000000,0.000000}%
\pgfsetfillcolor{currentfill}%
\pgfsetlinewidth{0.000000pt}%
\definecolor{currentstroke}{rgb}{0.000000,0.000000,0.000000}%
\pgfsetstrokecolor{currentstroke}%
\pgfsetstrokeopacity{0.000000}%
\pgfsetdash{}{0pt}%
\pgfpathmoveto{\pgfqpoint{3.571389in}{1.236944in}}%
\pgfpathlineto{\pgfqpoint{3.849167in}{1.236944in}}%
\pgfpathlineto{\pgfqpoint{3.849167in}{1.334167in}}%
\pgfpathlineto{\pgfqpoint{3.571389in}{1.334167in}}%
\pgfpathlineto{\pgfqpoint{3.571389in}{1.236944in}}%
\pgfpathclose%
\pgfusepath{fill}%
\end{pgfscope}%
\begin{pgfscope}%
\definecolor{textcolor}{rgb}{0.000000,0.000000,0.000000}%
\pgfsetstrokecolor{textcolor}%
\pgfsetfillcolor{textcolor}%
\pgftext[x=3.960278in,y=1.236944in,left,base]{\color{textcolor}\rmfamily\fontsize{10.000000}{12.000000}\selectfont Pos}%
\end{pgfscope}%
\end{pgfpicture}%
\makeatother%
\endgroup%

&
	\vskip 0pt
	\qquad \qquad ROC Curve
	
	%% Creator: Matplotlib, PGF backend
%%
%% To include the figure in your LaTeX document, write
%%   \input{<filename>.pgf}
%%
%% Make sure the required packages are loaded in your preamble
%%   \usepackage{pgf}
%%
%% Also ensure that all the required font packages are loaded; for instance,
%% the lmodern package is sometimes necessary when using math font.
%%   \usepackage{lmodern}
%%
%% Figures using additional raster images can only be included by \input if
%% they are in the same directory as the main LaTeX file. For loading figures
%% from other directories you can use the `import` package
%%   \usepackage{import}
%%
%% and then include the figures with
%%   \import{<path to file>}{<filename>.pgf}
%%
%% Matplotlib used the following preamble
%%   
%%   \usepackage{fontspec}
%%   \makeatletter\@ifpackageloaded{underscore}{}{\usepackage[strings]{underscore}}\makeatother
%%
\begingroup%
\makeatletter%
\begin{pgfpicture}%
\pgfpathrectangle{\pgfpointorigin}{\pgfqpoint{2.221861in}{1.754444in}}%
\pgfusepath{use as bounding box, clip}%
\begin{pgfscope}%
\pgfsetbuttcap%
\pgfsetmiterjoin%
\definecolor{currentfill}{rgb}{1.000000,1.000000,1.000000}%
\pgfsetfillcolor{currentfill}%
\pgfsetlinewidth{0.000000pt}%
\definecolor{currentstroke}{rgb}{1.000000,1.000000,1.000000}%
\pgfsetstrokecolor{currentstroke}%
\pgfsetdash{}{0pt}%
\pgfpathmoveto{\pgfqpoint{0.000000in}{0.000000in}}%
\pgfpathlineto{\pgfqpoint{2.221861in}{0.000000in}}%
\pgfpathlineto{\pgfqpoint{2.221861in}{1.754444in}}%
\pgfpathlineto{\pgfqpoint{0.000000in}{1.754444in}}%
\pgfpathlineto{\pgfqpoint{0.000000in}{0.000000in}}%
\pgfpathclose%
\pgfusepath{fill}%
\end{pgfscope}%
\begin{pgfscope}%
\pgfsetbuttcap%
\pgfsetmiterjoin%
\definecolor{currentfill}{rgb}{1.000000,1.000000,1.000000}%
\pgfsetfillcolor{currentfill}%
\pgfsetlinewidth{0.000000pt}%
\definecolor{currentstroke}{rgb}{0.000000,0.000000,0.000000}%
\pgfsetstrokecolor{currentstroke}%
\pgfsetstrokeopacity{0.000000}%
\pgfsetdash{}{0pt}%
\pgfpathmoveto{\pgfqpoint{0.553581in}{0.499444in}}%
\pgfpathlineto{\pgfqpoint{2.103581in}{0.499444in}}%
\pgfpathlineto{\pgfqpoint{2.103581in}{1.654444in}}%
\pgfpathlineto{\pgfqpoint{0.553581in}{1.654444in}}%
\pgfpathlineto{\pgfqpoint{0.553581in}{0.499444in}}%
\pgfpathclose%
\pgfusepath{fill}%
\end{pgfscope}%
\begin{pgfscope}%
\pgfsetbuttcap%
\pgfsetroundjoin%
\definecolor{currentfill}{rgb}{0.000000,0.000000,0.000000}%
\pgfsetfillcolor{currentfill}%
\pgfsetlinewidth{0.803000pt}%
\definecolor{currentstroke}{rgb}{0.000000,0.000000,0.000000}%
\pgfsetstrokecolor{currentstroke}%
\pgfsetdash{}{0pt}%
\pgfsys@defobject{currentmarker}{\pgfqpoint{0.000000in}{-0.048611in}}{\pgfqpoint{0.000000in}{0.000000in}}{%
\pgfpathmoveto{\pgfqpoint{0.000000in}{0.000000in}}%
\pgfpathlineto{\pgfqpoint{0.000000in}{-0.048611in}}%
\pgfusepath{stroke,fill}%
}%
\begin{pgfscope}%
\pgfsys@transformshift{0.624035in}{0.499444in}%
\pgfsys@useobject{currentmarker}{}%
\end{pgfscope}%
\end{pgfscope}%
\begin{pgfscope}%
\definecolor{textcolor}{rgb}{0.000000,0.000000,0.000000}%
\pgfsetstrokecolor{textcolor}%
\pgfsetfillcolor{textcolor}%
\pgftext[x=0.624035in,y=0.402222in,,top]{\color{textcolor}\rmfamily\fontsize{10.000000}{12.000000}\selectfont \(\displaystyle {0.0}\)}%
\end{pgfscope}%
\begin{pgfscope}%
\pgfsetbuttcap%
\pgfsetroundjoin%
\definecolor{currentfill}{rgb}{0.000000,0.000000,0.000000}%
\pgfsetfillcolor{currentfill}%
\pgfsetlinewidth{0.803000pt}%
\definecolor{currentstroke}{rgb}{0.000000,0.000000,0.000000}%
\pgfsetstrokecolor{currentstroke}%
\pgfsetdash{}{0pt}%
\pgfsys@defobject{currentmarker}{\pgfqpoint{0.000000in}{-0.048611in}}{\pgfqpoint{0.000000in}{0.000000in}}{%
\pgfpathmoveto{\pgfqpoint{0.000000in}{0.000000in}}%
\pgfpathlineto{\pgfqpoint{0.000000in}{-0.048611in}}%
\pgfusepath{stroke,fill}%
}%
\begin{pgfscope}%
\pgfsys@transformshift{1.328581in}{0.499444in}%
\pgfsys@useobject{currentmarker}{}%
\end{pgfscope}%
\end{pgfscope}%
\begin{pgfscope}%
\definecolor{textcolor}{rgb}{0.000000,0.000000,0.000000}%
\pgfsetstrokecolor{textcolor}%
\pgfsetfillcolor{textcolor}%
\pgftext[x=1.328581in,y=0.402222in,,top]{\color{textcolor}\rmfamily\fontsize{10.000000}{12.000000}\selectfont \(\displaystyle {0.5}\)}%
\end{pgfscope}%
\begin{pgfscope}%
\pgfsetbuttcap%
\pgfsetroundjoin%
\definecolor{currentfill}{rgb}{0.000000,0.000000,0.000000}%
\pgfsetfillcolor{currentfill}%
\pgfsetlinewidth{0.803000pt}%
\definecolor{currentstroke}{rgb}{0.000000,0.000000,0.000000}%
\pgfsetstrokecolor{currentstroke}%
\pgfsetdash{}{0pt}%
\pgfsys@defobject{currentmarker}{\pgfqpoint{0.000000in}{-0.048611in}}{\pgfqpoint{0.000000in}{0.000000in}}{%
\pgfpathmoveto{\pgfqpoint{0.000000in}{0.000000in}}%
\pgfpathlineto{\pgfqpoint{0.000000in}{-0.048611in}}%
\pgfusepath{stroke,fill}%
}%
\begin{pgfscope}%
\pgfsys@transformshift{2.033126in}{0.499444in}%
\pgfsys@useobject{currentmarker}{}%
\end{pgfscope}%
\end{pgfscope}%
\begin{pgfscope}%
\definecolor{textcolor}{rgb}{0.000000,0.000000,0.000000}%
\pgfsetstrokecolor{textcolor}%
\pgfsetfillcolor{textcolor}%
\pgftext[x=2.033126in,y=0.402222in,,top]{\color{textcolor}\rmfamily\fontsize{10.000000}{12.000000}\selectfont \(\displaystyle {1.0}\)}%
\end{pgfscope}%
\begin{pgfscope}%
\definecolor{textcolor}{rgb}{0.000000,0.000000,0.000000}%
\pgfsetstrokecolor{textcolor}%
\pgfsetfillcolor{textcolor}%
\pgftext[x=1.328581in,y=0.223333in,,top]{\color{textcolor}\rmfamily\fontsize{10.000000}{12.000000}\selectfont False positive rate}%
\end{pgfscope}%
\begin{pgfscope}%
\pgfsetbuttcap%
\pgfsetroundjoin%
\definecolor{currentfill}{rgb}{0.000000,0.000000,0.000000}%
\pgfsetfillcolor{currentfill}%
\pgfsetlinewidth{0.803000pt}%
\definecolor{currentstroke}{rgb}{0.000000,0.000000,0.000000}%
\pgfsetstrokecolor{currentstroke}%
\pgfsetdash{}{0pt}%
\pgfsys@defobject{currentmarker}{\pgfqpoint{-0.048611in}{0.000000in}}{\pgfqpoint{-0.000000in}{0.000000in}}{%
\pgfpathmoveto{\pgfqpoint{-0.000000in}{0.000000in}}%
\pgfpathlineto{\pgfqpoint{-0.048611in}{0.000000in}}%
\pgfusepath{stroke,fill}%
}%
\begin{pgfscope}%
\pgfsys@transformshift{0.553581in}{0.551944in}%
\pgfsys@useobject{currentmarker}{}%
\end{pgfscope}%
\end{pgfscope}%
\begin{pgfscope}%
\definecolor{textcolor}{rgb}{0.000000,0.000000,0.000000}%
\pgfsetstrokecolor{textcolor}%
\pgfsetfillcolor{textcolor}%
\pgftext[x=0.278889in, y=0.503750in, left, base]{\color{textcolor}\rmfamily\fontsize{10.000000}{12.000000}\selectfont \(\displaystyle {0.0}\)}%
\end{pgfscope}%
\begin{pgfscope}%
\pgfsetbuttcap%
\pgfsetroundjoin%
\definecolor{currentfill}{rgb}{0.000000,0.000000,0.000000}%
\pgfsetfillcolor{currentfill}%
\pgfsetlinewidth{0.803000pt}%
\definecolor{currentstroke}{rgb}{0.000000,0.000000,0.000000}%
\pgfsetstrokecolor{currentstroke}%
\pgfsetdash{}{0pt}%
\pgfsys@defobject{currentmarker}{\pgfqpoint{-0.048611in}{0.000000in}}{\pgfqpoint{-0.000000in}{0.000000in}}{%
\pgfpathmoveto{\pgfqpoint{-0.000000in}{0.000000in}}%
\pgfpathlineto{\pgfqpoint{-0.048611in}{0.000000in}}%
\pgfusepath{stroke,fill}%
}%
\begin{pgfscope}%
\pgfsys@transformshift{0.553581in}{1.076944in}%
\pgfsys@useobject{currentmarker}{}%
\end{pgfscope}%
\end{pgfscope}%
\begin{pgfscope}%
\definecolor{textcolor}{rgb}{0.000000,0.000000,0.000000}%
\pgfsetstrokecolor{textcolor}%
\pgfsetfillcolor{textcolor}%
\pgftext[x=0.278889in, y=1.028750in, left, base]{\color{textcolor}\rmfamily\fontsize{10.000000}{12.000000}\selectfont \(\displaystyle {0.5}\)}%
\end{pgfscope}%
\begin{pgfscope}%
\pgfsetbuttcap%
\pgfsetroundjoin%
\definecolor{currentfill}{rgb}{0.000000,0.000000,0.000000}%
\pgfsetfillcolor{currentfill}%
\pgfsetlinewidth{0.803000pt}%
\definecolor{currentstroke}{rgb}{0.000000,0.000000,0.000000}%
\pgfsetstrokecolor{currentstroke}%
\pgfsetdash{}{0pt}%
\pgfsys@defobject{currentmarker}{\pgfqpoint{-0.048611in}{0.000000in}}{\pgfqpoint{-0.000000in}{0.000000in}}{%
\pgfpathmoveto{\pgfqpoint{-0.000000in}{0.000000in}}%
\pgfpathlineto{\pgfqpoint{-0.048611in}{0.000000in}}%
\pgfusepath{stroke,fill}%
}%
\begin{pgfscope}%
\pgfsys@transformshift{0.553581in}{1.601944in}%
\pgfsys@useobject{currentmarker}{}%
\end{pgfscope}%
\end{pgfscope}%
\begin{pgfscope}%
\definecolor{textcolor}{rgb}{0.000000,0.000000,0.000000}%
\pgfsetstrokecolor{textcolor}%
\pgfsetfillcolor{textcolor}%
\pgftext[x=0.278889in, y=1.553750in, left, base]{\color{textcolor}\rmfamily\fontsize{10.000000}{12.000000}\selectfont \(\displaystyle {1.0}\)}%
\end{pgfscope}%
\begin{pgfscope}%
\definecolor{textcolor}{rgb}{0.000000,0.000000,0.000000}%
\pgfsetstrokecolor{textcolor}%
\pgfsetfillcolor{textcolor}%
\pgftext[x=0.223333in,y=1.076944in,,bottom,rotate=90.000000]{\color{textcolor}\rmfamily\fontsize{10.000000}{12.000000}\selectfont True positive rate}%
\end{pgfscope}%
\begin{pgfscope}%
\pgfpathrectangle{\pgfqpoint{0.553581in}{0.499444in}}{\pgfqpoint{1.550000in}{1.155000in}}%
\pgfusepath{clip}%
\pgfsetbuttcap%
\pgfsetroundjoin%
\pgfsetlinewidth{1.505625pt}%
\definecolor{currentstroke}{rgb}{0.000000,0.000000,0.000000}%
\pgfsetstrokecolor{currentstroke}%
\pgfsetdash{{5.550000pt}{2.400000pt}}{0.000000pt}%
\pgfpathmoveto{\pgfqpoint{0.624035in}{0.551944in}}%
\pgfpathlineto{\pgfqpoint{2.033126in}{1.601944in}}%
\pgfusepath{stroke}%
\end{pgfscope}%
\begin{pgfscope}%
\pgfpathrectangle{\pgfqpoint{0.553581in}{0.499444in}}{\pgfqpoint{1.550000in}{1.155000in}}%
\pgfusepath{clip}%
\pgfsetrectcap%
\pgfsetroundjoin%
\pgfsetlinewidth{1.505625pt}%
\definecolor{currentstroke}{rgb}{0.000000,0.000000,0.000000}%
\pgfsetstrokecolor{currentstroke}%
\pgfsetdash{}{0pt}%
\pgfpathmoveto{\pgfqpoint{0.624035in}{0.551944in}}%
\pgfpathlineto{\pgfqpoint{0.625818in}{0.578454in}}%
\pgfpathlineto{\pgfqpoint{0.635863in}{0.672977in}}%
\pgfpathlineto{\pgfqpoint{0.642954in}{0.717523in}}%
\pgfpathlineto{\pgfqpoint{0.652061in}{0.766011in}}%
\pgfpathlineto{\pgfqpoint{0.660082in}{0.797456in}}%
\pgfpathlineto{\pgfqpoint{0.680307in}{0.866246in}}%
\pgfpathlineto{\pgfqpoint{0.692854in}{0.901510in}}%
\pgfpathlineto{\pgfqpoint{0.692924in}{0.901789in}}%
\pgfpathlineto{\pgfqpoint{0.715377in}{0.955057in}}%
\pgfpathlineto{\pgfqpoint{0.723835in}{0.974645in}}%
\pgfpathlineto{\pgfqpoint{0.765613in}{1.048215in}}%
\pgfpathlineto{\pgfqpoint{0.817389in}{1.123306in}}%
\pgfpathlineto{\pgfqpoint{0.847119in}{1.160680in}}%
\pgfpathlineto{\pgfqpoint{0.879485in}{1.196068in}}%
\pgfpathlineto{\pgfqpoint{0.880837in}{1.197248in}}%
\pgfpathlineto{\pgfqpoint{0.881064in}{1.197465in}}%
\pgfpathlineto{\pgfqpoint{0.936850in}{1.250454in}}%
\pgfpathlineto{\pgfqpoint{0.977643in}{1.283607in}}%
\pgfpathlineto{\pgfqpoint{0.999563in}{1.300153in}}%
\pgfpathlineto{\pgfqpoint{1.092633in}{1.363292in}}%
\pgfpathlineto{\pgfqpoint{1.116703in}{1.376827in}}%
\pgfpathlineto{\pgfqpoint{1.141766in}{1.389989in}}%
\pgfpathlineto{\pgfqpoint{1.167416in}{1.404609in}}%
\pgfpathlineto{\pgfqpoint{1.246452in}{1.440215in}}%
\pgfpathlineto{\pgfqpoint{1.273478in}{1.452507in}}%
\pgfpathlineto{\pgfqpoint{1.299753in}{1.462689in}}%
\pgfpathlineto{\pgfqpoint{1.438524in}{1.510618in}}%
\pgfpathlineto{\pgfqpoint{1.465792in}{1.519403in}}%
\pgfpathlineto{\pgfqpoint{1.521602in}{1.533651in}}%
\pgfpathlineto{\pgfqpoint{1.576935in}{1.547000in}}%
\pgfpathlineto{\pgfqpoint{1.707928in}{1.571678in}}%
\pgfpathlineto{\pgfqpoint{1.802459in}{1.584933in}}%
\pgfpathlineto{\pgfqpoint{1.865414in}{1.591762in}}%
\pgfpathlineto{\pgfqpoint{1.936570in}{1.597629in}}%
\pgfpathlineto{\pgfqpoint{1.989793in}{1.600392in}}%
\pgfpathlineto{\pgfqpoint{2.033126in}{1.601944in}}%
\pgfpathlineto{\pgfqpoint{2.033126in}{1.601944in}}%
\pgfusepath{stroke}%
\end{pgfscope}%
\begin{pgfscope}%
\pgfsetrectcap%
\pgfsetmiterjoin%
\pgfsetlinewidth{0.803000pt}%
\definecolor{currentstroke}{rgb}{0.000000,0.000000,0.000000}%
\pgfsetstrokecolor{currentstroke}%
\pgfsetdash{}{0pt}%
\pgfpathmoveto{\pgfqpoint{0.553581in}{0.499444in}}%
\pgfpathlineto{\pgfqpoint{0.553581in}{1.654444in}}%
\pgfusepath{stroke}%
\end{pgfscope}%
\begin{pgfscope}%
\pgfsetrectcap%
\pgfsetmiterjoin%
\pgfsetlinewidth{0.803000pt}%
\definecolor{currentstroke}{rgb}{0.000000,0.000000,0.000000}%
\pgfsetstrokecolor{currentstroke}%
\pgfsetdash{}{0pt}%
\pgfpathmoveto{\pgfqpoint{2.103581in}{0.499444in}}%
\pgfpathlineto{\pgfqpoint{2.103581in}{1.654444in}}%
\pgfusepath{stroke}%
\end{pgfscope}%
\begin{pgfscope}%
\pgfsetrectcap%
\pgfsetmiterjoin%
\pgfsetlinewidth{0.803000pt}%
\definecolor{currentstroke}{rgb}{0.000000,0.000000,0.000000}%
\pgfsetstrokecolor{currentstroke}%
\pgfsetdash{}{0pt}%
\pgfpathmoveto{\pgfqpoint{0.553581in}{0.499444in}}%
\pgfpathlineto{\pgfqpoint{2.103581in}{0.499444in}}%
\pgfusepath{stroke}%
\end{pgfscope}%
\begin{pgfscope}%
\pgfsetrectcap%
\pgfsetmiterjoin%
\pgfsetlinewidth{0.803000pt}%
\definecolor{currentstroke}{rgb}{0.000000,0.000000,0.000000}%
\pgfsetstrokecolor{currentstroke}%
\pgfsetdash{}{0pt}%
\pgfpathmoveto{\pgfqpoint{0.553581in}{1.654444in}}%
\pgfpathlineto{\pgfqpoint{2.103581in}{1.654444in}}%
\pgfusepath{stroke}%
\end{pgfscope}%
\begin{pgfscope}%
\pgfsetbuttcap%
\pgfsetmiterjoin%
\definecolor{currentfill}{rgb}{1.000000,1.000000,1.000000}%
\pgfsetfillcolor{currentfill}%
\pgfsetfillopacity{0.800000}%
\pgfsetlinewidth{1.003750pt}%
\definecolor{currentstroke}{rgb}{0.800000,0.800000,0.800000}%
\pgfsetstrokecolor{currentstroke}%
\pgfsetstrokeopacity{0.800000}%
\pgfsetdash{}{0pt}%
\pgfpathmoveto{\pgfqpoint{0.832747in}{0.568889in}}%
\pgfpathlineto{\pgfqpoint{2.006358in}{0.568889in}}%
\pgfpathquadraticcurveto{\pgfqpoint{2.034136in}{0.568889in}}{\pgfqpoint{2.034136in}{0.596666in}}%
\pgfpathlineto{\pgfqpoint{2.034136in}{0.776388in}}%
\pgfpathquadraticcurveto{\pgfqpoint{2.034136in}{0.804166in}}{\pgfqpoint{2.006358in}{0.804166in}}%
\pgfpathlineto{\pgfqpoint{0.832747in}{0.804166in}}%
\pgfpathquadraticcurveto{\pgfqpoint{0.804970in}{0.804166in}}{\pgfqpoint{0.804970in}{0.776388in}}%
\pgfpathlineto{\pgfqpoint{0.804970in}{0.596666in}}%
\pgfpathquadraticcurveto{\pgfqpoint{0.804970in}{0.568889in}}{\pgfqpoint{0.832747in}{0.568889in}}%
\pgfpathlineto{\pgfqpoint{0.832747in}{0.568889in}}%
\pgfpathclose%
\pgfusepath{stroke,fill}%
\end{pgfscope}%
\begin{pgfscope}%
\pgfsetrectcap%
\pgfsetroundjoin%
\pgfsetlinewidth{1.505625pt}%
\definecolor{currentstroke}{rgb}{0.000000,0.000000,0.000000}%
\pgfsetstrokecolor{currentstroke}%
\pgfsetdash{}{0pt}%
\pgfpathmoveto{\pgfqpoint{0.860525in}{0.700000in}}%
\pgfpathlineto{\pgfqpoint{0.999414in}{0.700000in}}%
\pgfpathlineto{\pgfqpoint{1.138303in}{0.700000in}}%
\pgfusepath{stroke}%
\end{pgfscope}%
\begin{pgfscope}%
\definecolor{textcolor}{rgb}{0.000000,0.000000,0.000000}%
\pgfsetstrokecolor{textcolor}%
\pgfsetfillcolor{textcolor}%
\pgftext[x=1.249414in,y=0.651388in,left,base]{\color{textcolor}\rmfamily\fontsize{10.000000}{12.000000}\selectfont AUC=0.799}%
\end{pgfscope}%
\end{pgfpicture}%
\makeatother%
\endgroup%

\end{tabular}

I tried several of the usual tricks to get it to not overfit, like changing the number of trees, the maximum depth, and the maximum number of leaf nodes, and they gave me poorer results for both the training set and the test set, so I don't know that that's better.  The Balanced Random Forest Classifier is by far the best model algorithm I've used for giving the best AUC, precision, accuracy, and F1.  Is overfitting a problem if it gives the best results on the test set?







