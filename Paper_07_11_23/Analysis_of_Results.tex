% Analysis of Results

Our ML algorithms assign to each sample (feature vector, crash person) a probability $p \in [0,1]$ that the person needs an ambulance.  The histogram below left shows the percentage of the dataset in each range of $p$, showing the percentages for the negative class (``Does not need an ambulance'') and the positive class (``Needs an ambulance'').  On the right, the Receiver Operating Characteristic (ROC) curve, and particularly the area under the curve (AUC), is a metric for how well the model separates the two classes, with $AUC=1.0$ being perfect and $AUC=0.5$ (the dashed line) being just random assignment with no insight.  

We would love to have results like in the graphs below, where the machine learning (ML) algorithm nearly perfectly separates the two classes.  There is some overlap between $p=0.6$ and $p=0.8$ with some samples the algorithm misclassifies, but the model clearly separates most samples.  Having an AUC of 0.996 would be amazing.  

\noindent\begin{tabular}{@{\hspace{-6pt}}p{4.3in} @{\hspace{-6pt}}p{2.0in}}
	\vskip 0pt
	\hfil Raw Model Output
	
	%% Creator: Matplotlib, PGF backend
%%
%% To include the figure in your LaTeX document, write
%%   \input{<filename>.pgf}
%%
%% Make sure the required packages are loaded in your preamble
%%   \usepackage{pgf}
%%
%% Also ensure that all the required font packages are loaded; for instance,
%% the lmodern package is sometimes necessary when using math font.
%%   \usepackage{lmodern}
%%
%% Figures using additional raster images can only be included by \input if
%% they are in the same directory as the main LaTeX file. For loading figures
%% from other directories you can use the `import` package
%%   \usepackage{import}
%%
%% and then include the figures with
%%   \import{<path to file>}{<filename>.pgf}
%%
%% Matplotlib used the following preamble
%%   
%%   \usepackage{fontspec}
%%   \makeatletter\@ifpackageloaded{underscore}{}{\usepackage[strings]{underscore}}\makeatother
%%
\begingroup%
\makeatletter%
\begin{pgfpicture}%
\pgfpathrectangle{\pgfpointorigin}{\pgfqpoint{4.509306in}{1.754444in}}%
\pgfusepath{use as bounding box, clip}%
\begin{pgfscope}%
\pgfsetbuttcap%
\pgfsetmiterjoin%
\definecolor{currentfill}{rgb}{1.000000,1.000000,1.000000}%
\pgfsetfillcolor{currentfill}%
\pgfsetlinewidth{0.000000pt}%
\definecolor{currentstroke}{rgb}{1.000000,1.000000,1.000000}%
\pgfsetstrokecolor{currentstroke}%
\pgfsetdash{}{0pt}%
\pgfpathmoveto{\pgfqpoint{0.000000in}{0.000000in}}%
\pgfpathlineto{\pgfqpoint{4.509306in}{0.000000in}}%
\pgfpathlineto{\pgfqpoint{4.509306in}{1.754444in}}%
\pgfpathlineto{\pgfqpoint{0.000000in}{1.754444in}}%
\pgfpathlineto{\pgfqpoint{0.000000in}{0.000000in}}%
\pgfpathclose%
\pgfusepath{fill}%
\end{pgfscope}%
\begin{pgfscope}%
\pgfsetbuttcap%
\pgfsetmiterjoin%
\definecolor{currentfill}{rgb}{1.000000,1.000000,1.000000}%
\pgfsetfillcolor{currentfill}%
\pgfsetlinewidth{0.000000pt}%
\definecolor{currentstroke}{rgb}{0.000000,0.000000,0.000000}%
\pgfsetstrokecolor{currentstroke}%
\pgfsetstrokeopacity{0.000000}%
\pgfsetdash{}{0pt}%
\pgfpathmoveto{\pgfqpoint{0.445556in}{0.499444in}}%
\pgfpathlineto{\pgfqpoint{4.320556in}{0.499444in}}%
\pgfpathlineto{\pgfqpoint{4.320556in}{1.654444in}}%
\pgfpathlineto{\pgfqpoint{0.445556in}{1.654444in}}%
\pgfpathlineto{\pgfqpoint{0.445556in}{0.499444in}}%
\pgfpathclose%
\pgfusepath{fill}%
\end{pgfscope}%
\begin{pgfscope}%
\pgfpathrectangle{\pgfqpoint{0.445556in}{0.499444in}}{\pgfqpoint{3.875000in}{1.155000in}}%
\pgfusepath{clip}%
\pgfsetbuttcap%
\pgfsetmiterjoin%
\pgfsetlinewidth{1.003750pt}%
\definecolor{currentstroke}{rgb}{0.000000,0.000000,0.000000}%
\pgfsetstrokecolor{currentstroke}%
\pgfsetdash{}{0pt}%
\pgfpathmoveto{\pgfqpoint{0.435556in}{0.499444in}}%
\pgfpathlineto{\pgfqpoint{0.483922in}{0.499444in}}%
\pgfpathlineto{\pgfqpoint{0.483922in}{0.632510in}}%
\pgfpathlineto{\pgfqpoint{0.435556in}{0.632510in}}%
\pgfusepath{stroke}%
\end{pgfscope}%
\begin{pgfscope}%
\pgfpathrectangle{\pgfqpoint{0.445556in}{0.499444in}}{\pgfqpoint{3.875000in}{1.155000in}}%
\pgfusepath{clip}%
\pgfsetbuttcap%
\pgfsetmiterjoin%
\pgfsetlinewidth{1.003750pt}%
\definecolor{currentstroke}{rgb}{0.000000,0.000000,0.000000}%
\pgfsetstrokecolor{currentstroke}%
\pgfsetdash{}{0pt}%
\pgfpathmoveto{\pgfqpoint{0.576001in}{0.499444in}}%
\pgfpathlineto{\pgfqpoint{0.637387in}{0.499444in}}%
\pgfpathlineto{\pgfqpoint{0.637387in}{0.855505in}}%
\pgfpathlineto{\pgfqpoint{0.576001in}{0.855505in}}%
\pgfpathlineto{\pgfqpoint{0.576001in}{0.499444in}}%
\pgfpathclose%
\pgfusepath{stroke}%
\end{pgfscope}%
\begin{pgfscope}%
\pgfpathrectangle{\pgfqpoint{0.445556in}{0.499444in}}{\pgfqpoint{3.875000in}{1.155000in}}%
\pgfusepath{clip}%
\pgfsetbuttcap%
\pgfsetmiterjoin%
\pgfsetlinewidth{1.003750pt}%
\definecolor{currentstroke}{rgb}{0.000000,0.000000,0.000000}%
\pgfsetstrokecolor{currentstroke}%
\pgfsetdash{}{0pt}%
\pgfpathmoveto{\pgfqpoint{0.729467in}{0.499444in}}%
\pgfpathlineto{\pgfqpoint{0.790853in}{0.499444in}}%
\pgfpathlineto{\pgfqpoint{0.790853in}{1.087595in}}%
\pgfpathlineto{\pgfqpoint{0.729467in}{1.087595in}}%
\pgfpathlineto{\pgfqpoint{0.729467in}{0.499444in}}%
\pgfpathclose%
\pgfusepath{stroke}%
\end{pgfscope}%
\begin{pgfscope}%
\pgfpathrectangle{\pgfqpoint{0.445556in}{0.499444in}}{\pgfqpoint{3.875000in}{1.155000in}}%
\pgfusepath{clip}%
\pgfsetbuttcap%
\pgfsetmiterjoin%
\pgfsetlinewidth{1.003750pt}%
\definecolor{currentstroke}{rgb}{0.000000,0.000000,0.000000}%
\pgfsetstrokecolor{currentstroke}%
\pgfsetdash{}{0pt}%
\pgfpathmoveto{\pgfqpoint{0.882932in}{0.499444in}}%
\pgfpathlineto{\pgfqpoint{0.944318in}{0.499444in}}%
\pgfpathlineto{\pgfqpoint{0.944318in}{1.269324in}}%
\pgfpathlineto{\pgfqpoint{0.882932in}{1.269324in}}%
\pgfpathlineto{\pgfqpoint{0.882932in}{0.499444in}}%
\pgfpathclose%
\pgfusepath{stroke}%
\end{pgfscope}%
\begin{pgfscope}%
\pgfpathrectangle{\pgfqpoint{0.445556in}{0.499444in}}{\pgfqpoint{3.875000in}{1.155000in}}%
\pgfusepath{clip}%
\pgfsetbuttcap%
\pgfsetmiterjoin%
\pgfsetlinewidth{1.003750pt}%
\definecolor{currentstroke}{rgb}{0.000000,0.000000,0.000000}%
\pgfsetstrokecolor{currentstroke}%
\pgfsetdash{}{0pt}%
\pgfpathmoveto{\pgfqpoint{1.036397in}{0.499444in}}%
\pgfpathlineto{\pgfqpoint{1.097783in}{0.499444in}}%
\pgfpathlineto{\pgfqpoint{1.097783in}{1.428301in}}%
\pgfpathlineto{\pgfqpoint{1.036397in}{1.428301in}}%
\pgfpathlineto{\pgfqpoint{1.036397in}{0.499444in}}%
\pgfpathclose%
\pgfusepath{stroke}%
\end{pgfscope}%
\begin{pgfscope}%
\pgfpathrectangle{\pgfqpoint{0.445556in}{0.499444in}}{\pgfqpoint{3.875000in}{1.155000in}}%
\pgfusepath{clip}%
\pgfsetbuttcap%
\pgfsetmiterjoin%
\pgfsetlinewidth{1.003750pt}%
\definecolor{currentstroke}{rgb}{0.000000,0.000000,0.000000}%
\pgfsetstrokecolor{currentstroke}%
\pgfsetdash{}{0pt}%
\pgfpathmoveto{\pgfqpoint{1.189863in}{0.499444in}}%
\pgfpathlineto{\pgfqpoint{1.251249in}{0.499444in}}%
\pgfpathlineto{\pgfqpoint{1.251249in}{1.521593in}}%
\pgfpathlineto{\pgfqpoint{1.189863in}{1.521593in}}%
\pgfpathlineto{\pgfqpoint{1.189863in}{0.499444in}}%
\pgfpathclose%
\pgfusepath{stroke}%
\end{pgfscope}%
\begin{pgfscope}%
\pgfpathrectangle{\pgfqpoint{0.445556in}{0.499444in}}{\pgfqpoint{3.875000in}{1.155000in}}%
\pgfusepath{clip}%
\pgfsetbuttcap%
\pgfsetmiterjoin%
\pgfsetlinewidth{1.003750pt}%
\definecolor{currentstroke}{rgb}{0.000000,0.000000,0.000000}%
\pgfsetstrokecolor{currentstroke}%
\pgfsetdash{}{0pt}%
\pgfpathmoveto{\pgfqpoint{1.343328in}{0.499444in}}%
\pgfpathlineto{\pgfqpoint{1.404714in}{0.499444in}}%
\pgfpathlineto{\pgfqpoint{1.404714in}{1.590875in}}%
\pgfpathlineto{\pgfqpoint{1.343328in}{1.590875in}}%
\pgfpathlineto{\pgfqpoint{1.343328in}{0.499444in}}%
\pgfpathclose%
\pgfusepath{stroke}%
\end{pgfscope}%
\begin{pgfscope}%
\pgfpathrectangle{\pgfqpoint{0.445556in}{0.499444in}}{\pgfqpoint{3.875000in}{1.155000in}}%
\pgfusepath{clip}%
\pgfsetbuttcap%
\pgfsetmiterjoin%
\pgfsetlinewidth{1.003750pt}%
\definecolor{currentstroke}{rgb}{0.000000,0.000000,0.000000}%
\pgfsetstrokecolor{currentstroke}%
\pgfsetdash{}{0pt}%
\pgfpathmoveto{\pgfqpoint{1.496793in}{0.499444in}}%
\pgfpathlineto{\pgfqpoint{1.558179in}{0.499444in}}%
\pgfpathlineto{\pgfqpoint{1.558179in}{1.599444in}}%
\pgfpathlineto{\pgfqpoint{1.496793in}{1.599444in}}%
\pgfpathlineto{\pgfqpoint{1.496793in}{0.499444in}}%
\pgfpathclose%
\pgfusepath{stroke}%
\end{pgfscope}%
\begin{pgfscope}%
\pgfpathrectangle{\pgfqpoint{0.445556in}{0.499444in}}{\pgfqpoint{3.875000in}{1.155000in}}%
\pgfusepath{clip}%
\pgfsetbuttcap%
\pgfsetmiterjoin%
\pgfsetlinewidth{1.003750pt}%
\definecolor{currentstroke}{rgb}{0.000000,0.000000,0.000000}%
\pgfsetstrokecolor{currentstroke}%
\pgfsetdash{}{0pt}%
\pgfpathmoveto{\pgfqpoint{1.650259in}{0.499444in}}%
\pgfpathlineto{\pgfqpoint{1.711645in}{0.499444in}}%
\pgfpathlineto{\pgfqpoint{1.711645in}{1.566953in}}%
\pgfpathlineto{\pgfqpoint{1.650259in}{1.566953in}}%
\pgfpathlineto{\pgfqpoint{1.650259in}{0.499444in}}%
\pgfpathclose%
\pgfusepath{stroke}%
\end{pgfscope}%
\begin{pgfscope}%
\pgfpathrectangle{\pgfqpoint{0.445556in}{0.499444in}}{\pgfqpoint{3.875000in}{1.155000in}}%
\pgfusepath{clip}%
\pgfsetbuttcap%
\pgfsetmiterjoin%
\pgfsetlinewidth{1.003750pt}%
\definecolor{currentstroke}{rgb}{0.000000,0.000000,0.000000}%
\pgfsetstrokecolor{currentstroke}%
\pgfsetdash{}{0pt}%
\pgfpathmoveto{\pgfqpoint{1.803724in}{0.499444in}}%
\pgfpathlineto{\pgfqpoint{1.865110in}{0.499444in}}%
\pgfpathlineto{\pgfqpoint{1.865110in}{1.506942in}}%
\pgfpathlineto{\pgfqpoint{1.803724in}{1.506942in}}%
\pgfpathlineto{\pgfqpoint{1.803724in}{0.499444in}}%
\pgfpathclose%
\pgfusepath{stroke}%
\end{pgfscope}%
\begin{pgfscope}%
\pgfpathrectangle{\pgfqpoint{0.445556in}{0.499444in}}{\pgfqpoint{3.875000in}{1.155000in}}%
\pgfusepath{clip}%
\pgfsetbuttcap%
\pgfsetmiterjoin%
\pgfsetlinewidth{1.003750pt}%
\definecolor{currentstroke}{rgb}{0.000000,0.000000,0.000000}%
\pgfsetstrokecolor{currentstroke}%
\pgfsetdash{}{0pt}%
\pgfpathmoveto{\pgfqpoint{1.957189in}{0.499444in}}%
\pgfpathlineto{\pgfqpoint{2.018575in}{0.499444in}}%
\pgfpathlineto{\pgfqpoint{2.018575in}{1.416545in}}%
\pgfpathlineto{\pgfqpoint{1.957189in}{1.416545in}}%
\pgfpathlineto{\pgfqpoint{1.957189in}{0.499444in}}%
\pgfpathclose%
\pgfusepath{stroke}%
\end{pgfscope}%
\begin{pgfscope}%
\pgfpathrectangle{\pgfqpoint{0.445556in}{0.499444in}}{\pgfqpoint{3.875000in}{1.155000in}}%
\pgfusepath{clip}%
\pgfsetbuttcap%
\pgfsetmiterjoin%
\pgfsetlinewidth{1.003750pt}%
\definecolor{currentstroke}{rgb}{0.000000,0.000000,0.000000}%
\pgfsetstrokecolor{currentstroke}%
\pgfsetdash{}{0pt}%
\pgfpathmoveto{\pgfqpoint{2.110655in}{0.499444in}}%
\pgfpathlineto{\pgfqpoint{2.172041in}{0.499444in}}%
\pgfpathlineto{\pgfqpoint{2.172041in}{1.298482in}}%
\pgfpathlineto{\pgfqpoint{2.110655in}{1.298482in}}%
\pgfpathlineto{\pgfqpoint{2.110655in}{0.499444in}}%
\pgfpathclose%
\pgfusepath{stroke}%
\end{pgfscope}%
\begin{pgfscope}%
\pgfpathrectangle{\pgfqpoint{0.445556in}{0.499444in}}{\pgfqpoint{3.875000in}{1.155000in}}%
\pgfusepath{clip}%
\pgfsetbuttcap%
\pgfsetmiterjoin%
\pgfsetlinewidth{1.003750pt}%
\definecolor{currentstroke}{rgb}{0.000000,0.000000,0.000000}%
\pgfsetstrokecolor{currentstroke}%
\pgfsetdash{}{0pt}%
\pgfpathmoveto{\pgfqpoint{2.264120in}{0.499444in}}%
\pgfpathlineto{\pgfqpoint{2.325506in}{0.499444in}}%
\pgfpathlineto{\pgfqpoint{2.325506in}{1.165884in}}%
\pgfpathlineto{\pgfqpoint{2.264120in}{1.165884in}}%
\pgfpathlineto{\pgfqpoint{2.264120in}{0.499444in}}%
\pgfpathclose%
\pgfusepath{stroke}%
\end{pgfscope}%
\begin{pgfscope}%
\pgfpathrectangle{\pgfqpoint{0.445556in}{0.499444in}}{\pgfqpoint{3.875000in}{1.155000in}}%
\pgfusepath{clip}%
\pgfsetbuttcap%
\pgfsetmiterjoin%
\pgfsetlinewidth{1.003750pt}%
\definecolor{currentstroke}{rgb}{0.000000,0.000000,0.000000}%
\pgfsetstrokecolor{currentstroke}%
\pgfsetdash{}{0pt}%
\pgfpathmoveto{\pgfqpoint{2.417585in}{0.499444in}}%
\pgfpathlineto{\pgfqpoint{2.478972in}{0.499444in}}%
\pgfpathlineto{\pgfqpoint{2.478972in}{1.041650in}}%
\pgfpathlineto{\pgfqpoint{2.417585in}{1.041650in}}%
\pgfpathlineto{\pgfqpoint{2.417585in}{0.499444in}}%
\pgfpathclose%
\pgfusepath{stroke}%
\end{pgfscope}%
\begin{pgfscope}%
\pgfpathrectangle{\pgfqpoint{0.445556in}{0.499444in}}{\pgfqpoint{3.875000in}{1.155000in}}%
\pgfusepath{clip}%
\pgfsetbuttcap%
\pgfsetmiterjoin%
\pgfsetlinewidth{1.003750pt}%
\definecolor{currentstroke}{rgb}{0.000000,0.000000,0.000000}%
\pgfsetstrokecolor{currentstroke}%
\pgfsetdash{}{0pt}%
\pgfpathmoveto{\pgfqpoint{2.571051in}{0.499444in}}%
\pgfpathlineto{\pgfqpoint{2.632437in}{0.499444in}}%
\pgfpathlineto{\pgfqpoint{2.632437in}{0.916218in}}%
\pgfpathlineto{\pgfqpoint{2.571051in}{0.916218in}}%
\pgfpathlineto{\pgfqpoint{2.571051in}{0.499444in}}%
\pgfpathclose%
\pgfusepath{stroke}%
\end{pgfscope}%
\begin{pgfscope}%
\pgfpathrectangle{\pgfqpoint{0.445556in}{0.499444in}}{\pgfqpoint{3.875000in}{1.155000in}}%
\pgfusepath{clip}%
\pgfsetbuttcap%
\pgfsetmiterjoin%
\pgfsetlinewidth{1.003750pt}%
\definecolor{currentstroke}{rgb}{0.000000,0.000000,0.000000}%
\pgfsetstrokecolor{currentstroke}%
\pgfsetdash{}{0pt}%
\pgfpathmoveto{\pgfqpoint{2.724516in}{0.499444in}}%
\pgfpathlineto{\pgfqpoint{2.785902in}{0.499444in}}%
\pgfpathlineto{\pgfqpoint{2.785902in}{0.806958in}}%
\pgfpathlineto{\pgfqpoint{2.724516in}{0.806958in}}%
\pgfpathlineto{\pgfqpoint{2.724516in}{0.499444in}}%
\pgfpathclose%
\pgfusepath{stroke}%
\end{pgfscope}%
\begin{pgfscope}%
\pgfpathrectangle{\pgfqpoint{0.445556in}{0.499444in}}{\pgfqpoint{3.875000in}{1.155000in}}%
\pgfusepath{clip}%
\pgfsetbuttcap%
\pgfsetmiterjoin%
\pgfsetlinewidth{1.003750pt}%
\definecolor{currentstroke}{rgb}{0.000000,0.000000,0.000000}%
\pgfsetstrokecolor{currentstroke}%
\pgfsetdash{}{0pt}%
\pgfpathmoveto{\pgfqpoint{2.877981in}{0.499444in}}%
\pgfpathlineto{\pgfqpoint{2.939368in}{0.499444in}}%
\pgfpathlineto{\pgfqpoint{2.939368in}{0.716063in}}%
\pgfpathlineto{\pgfqpoint{2.877981in}{0.716063in}}%
\pgfpathlineto{\pgfqpoint{2.877981in}{0.499444in}}%
\pgfpathclose%
\pgfusepath{stroke}%
\end{pgfscope}%
\begin{pgfscope}%
\pgfpathrectangle{\pgfqpoint{0.445556in}{0.499444in}}{\pgfqpoint{3.875000in}{1.155000in}}%
\pgfusepath{clip}%
\pgfsetbuttcap%
\pgfsetmiterjoin%
\pgfsetlinewidth{1.003750pt}%
\definecolor{currentstroke}{rgb}{0.000000,0.000000,0.000000}%
\pgfsetstrokecolor{currentstroke}%
\pgfsetdash{}{0pt}%
\pgfpathmoveto{\pgfqpoint{3.031447in}{0.499444in}}%
\pgfpathlineto{\pgfqpoint{3.092833in}{0.499444in}}%
\pgfpathlineto{\pgfqpoint{3.092833in}{0.645904in}}%
\pgfpathlineto{\pgfqpoint{3.031447in}{0.645904in}}%
\pgfpathlineto{\pgfqpoint{3.031447in}{0.499444in}}%
\pgfpathclose%
\pgfusepath{stroke}%
\end{pgfscope}%
\begin{pgfscope}%
\pgfpathrectangle{\pgfqpoint{0.445556in}{0.499444in}}{\pgfqpoint{3.875000in}{1.155000in}}%
\pgfusepath{clip}%
\pgfsetbuttcap%
\pgfsetmiterjoin%
\pgfsetlinewidth{1.003750pt}%
\definecolor{currentstroke}{rgb}{0.000000,0.000000,0.000000}%
\pgfsetstrokecolor{currentstroke}%
\pgfsetdash{}{0pt}%
\pgfpathmoveto{\pgfqpoint{3.184912in}{0.499444in}}%
\pgfpathlineto{\pgfqpoint{3.246298in}{0.499444in}}%
\pgfpathlineto{\pgfqpoint{3.246298in}{0.598936in}}%
\pgfpathlineto{\pgfqpoint{3.184912in}{0.598936in}}%
\pgfpathlineto{\pgfqpoint{3.184912in}{0.499444in}}%
\pgfpathclose%
\pgfusepath{stroke}%
\end{pgfscope}%
\begin{pgfscope}%
\pgfpathrectangle{\pgfqpoint{0.445556in}{0.499444in}}{\pgfqpoint{3.875000in}{1.155000in}}%
\pgfusepath{clip}%
\pgfsetbuttcap%
\pgfsetmiterjoin%
\pgfsetlinewidth{1.003750pt}%
\definecolor{currentstroke}{rgb}{0.000000,0.000000,0.000000}%
\pgfsetstrokecolor{currentstroke}%
\pgfsetdash{}{0pt}%
\pgfpathmoveto{\pgfqpoint{3.338377in}{0.499444in}}%
\pgfpathlineto{\pgfqpoint{3.399764in}{0.499444in}}%
\pgfpathlineto{\pgfqpoint{3.399764in}{0.560040in}}%
\pgfpathlineto{\pgfqpoint{3.338377in}{0.560040in}}%
\pgfpathlineto{\pgfqpoint{3.338377in}{0.499444in}}%
\pgfpathclose%
\pgfusepath{stroke}%
\end{pgfscope}%
\begin{pgfscope}%
\pgfpathrectangle{\pgfqpoint{0.445556in}{0.499444in}}{\pgfqpoint{3.875000in}{1.155000in}}%
\pgfusepath{clip}%
\pgfsetbuttcap%
\pgfsetmiterjoin%
\pgfsetlinewidth{1.003750pt}%
\definecolor{currentstroke}{rgb}{0.000000,0.000000,0.000000}%
\pgfsetstrokecolor{currentstroke}%
\pgfsetdash{}{0pt}%
\pgfpathmoveto{\pgfqpoint{3.491843in}{0.499444in}}%
\pgfpathlineto{\pgfqpoint{3.553229in}{0.499444in}}%
\pgfpathlineto{\pgfqpoint{3.553229in}{0.535591in}}%
\pgfpathlineto{\pgfqpoint{3.491843in}{0.535591in}}%
\pgfpathlineto{\pgfqpoint{3.491843in}{0.499444in}}%
\pgfpathclose%
\pgfusepath{stroke}%
\end{pgfscope}%
\begin{pgfscope}%
\pgfpathrectangle{\pgfqpoint{0.445556in}{0.499444in}}{\pgfqpoint{3.875000in}{1.155000in}}%
\pgfusepath{clip}%
\pgfsetbuttcap%
\pgfsetmiterjoin%
\pgfsetlinewidth{1.003750pt}%
\definecolor{currentstroke}{rgb}{0.000000,0.000000,0.000000}%
\pgfsetstrokecolor{currentstroke}%
\pgfsetdash{}{0pt}%
\pgfpathmoveto{\pgfqpoint{3.645308in}{0.499444in}}%
\pgfpathlineto{\pgfqpoint{3.706694in}{0.499444in}}%
\pgfpathlineto{\pgfqpoint{3.706694in}{0.512137in}}%
\pgfpathlineto{\pgfqpoint{3.645308in}{0.512137in}}%
\pgfpathlineto{\pgfqpoint{3.645308in}{0.499444in}}%
\pgfpathclose%
\pgfusepath{stroke}%
\end{pgfscope}%
\begin{pgfscope}%
\pgfpathrectangle{\pgfqpoint{0.445556in}{0.499444in}}{\pgfqpoint{3.875000in}{1.155000in}}%
\pgfusepath{clip}%
\pgfsetbuttcap%
\pgfsetmiterjoin%
\pgfsetlinewidth{1.003750pt}%
\definecolor{currentstroke}{rgb}{0.000000,0.000000,0.000000}%
\pgfsetstrokecolor{currentstroke}%
\pgfsetdash{}{0pt}%
\pgfpathmoveto{\pgfqpoint{3.798774in}{0.499444in}}%
\pgfpathlineto{\pgfqpoint{3.860160in}{0.499444in}}%
\pgfpathlineto{\pgfqpoint{3.860160in}{0.503305in}}%
\pgfpathlineto{\pgfqpoint{3.798774in}{0.503305in}}%
\pgfpathlineto{\pgfqpoint{3.798774in}{0.499444in}}%
\pgfpathclose%
\pgfusepath{stroke}%
\end{pgfscope}%
\begin{pgfscope}%
\pgfpathrectangle{\pgfqpoint{0.445556in}{0.499444in}}{\pgfqpoint{3.875000in}{1.155000in}}%
\pgfusepath{clip}%
\pgfsetbuttcap%
\pgfsetmiterjoin%
\pgfsetlinewidth{1.003750pt}%
\definecolor{currentstroke}{rgb}{0.000000,0.000000,0.000000}%
\pgfsetstrokecolor{currentstroke}%
\pgfsetdash{}{0pt}%
\pgfpathmoveto{\pgfqpoint{3.952239in}{0.499444in}}%
\pgfpathlineto{\pgfqpoint{4.013625in}{0.499444in}}%
\pgfpathlineto{\pgfqpoint{4.013625in}{0.500234in}}%
\pgfpathlineto{\pgfqpoint{3.952239in}{0.500234in}}%
\pgfpathlineto{\pgfqpoint{3.952239in}{0.499444in}}%
\pgfpathclose%
\pgfusepath{stroke}%
\end{pgfscope}%
\begin{pgfscope}%
\pgfpathrectangle{\pgfqpoint{0.445556in}{0.499444in}}{\pgfqpoint{3.875000in}{1.155000in}}%
\pgfusepath{clip}%
\pgfsetbuttcap%
\pgfsetmiterjoin%
\pgfsetlinewidth{1.003750pt}%
\definecolor{currentstroke}{rgb}{0.000000,0.000000,0.000000}%
\pgfsetstrokecolor{currentstroke}%
\pgfsetdash{}{0pt}%
\pgfpathmoveto{\pgfqpoint{4.105704in}{0.499444in}}%
\pgfpathlineto{\pgfqpoint{4.167090in}{0.499444in}}%
\pgfpathlineto{\pgfqpoint{4.167090in}{0.499503in}}%
\pgfpathlineto{\pgfqpoint{4.105704in}{0.499503in}}%
\pgfpathlineto{\pgfqpoint{4.105704in}{0.499444in}}%
\pgfpathclose%
\pgfusepath{stroke}%
\end{pgfscope}%
\begin{pgfscope}%
\pgfpathrectangle{\pgfqpoint{0.445556in}{0.499444in}}{\pgfqpoint{3.875000in}{1.155000in}}%
\pgfusepath{clip}%
\pgfsetbuttcap%
\pgfsetmiterjoin%
\definecolor{currentfill}{rgb}{0.000000,0.000000,0.000000}%
\pgfsetfillcolor{currentfill}%
\pgfsetlinewidth{0.000000pt}%
\definecolor{currentstroke}{rgb}{0.000000,0.000000,0.000000}%
\pgfsetstrokecolor{currentstroke}%
\pgfsetstrokeopacity{0.000000}%
\pgfsetdash{}{0pt}%
\pgfpathmoveto{\pgfqpoint{0.483922in}{0.499444in}}%
\pgfpathlineto{\pgfqpoint{0.545308in}{0.499444in}}%
\pgfpathlineto{\pgfqpoint{0.545308in}{0.499444in}}%
\pgfpathlineto{\pgfqpoint{0.483922in}{0.499444in}}%
\pgfpathlineto{\pgfqpoint{0.483922in}{0.499444in}}%
\pgfpathclose%
\pgfusepath{fill}%
\end{pgfscope}%
\begin{pgfscope}%
\pgfpathrectangle{\pgfqpoint{0.445556in}{0.499444in}}{\pgfqpoint{3.875000in}{1.155000in}}%
\pgfusepath{clip}%
\pgfsetbuttcap%
\pgfsetmiterjoin%
\definecolor{currentfill}{rgb}{0.000000,0.000000,0.000000}%
\pgfsetfillcolor{currentfill}%
\pgfsetlinewidth{0.000000pt}%
\definecolor{currentstroke}{rgb}{0.000000,0.000000,0.000000}%
\pgfsetstrokecolor{currentstroke}%
\pgfsetstrokeopacity{0.000000}%
\pgfsetdash{}{0pt}%
\pgfpathmoveto{\pgfqpoint{0.637387in}{0.499444in}}%
\pgfpathlineto{\pgfqpoint{0.698774in}{0.499444in}}%
\pgfpathlineto{\pgfqpoint{0.698774in}{0.499444in}}%
\pgfpathlineto{\pgfqpoint{0.637387in}{0.499444in}}%
\pgfpathlineto{\pgfqpoint{0.637387in}{0.499444in}}%
\pgfpathclose%
\pgfusepath{fill}%
\end{pgfscope}%
\begin{pgfscope}%
\pgfpathrectangle{\pgfqpoint{0.445556in}{0.499444in}}{\pgfqpoint{3.875000in}{1.155000in}}%
\pgfusepath{clip}%
\pgfsetbuttcap%
\pgfsetmiterjoin%
\definecolor{currentfill}{rgb}{0.000000,0.000000,0.000000}%
\pgfsetfillcolor{currentfill}%
\pgfsetlinewidth{0.000000pt}%
\definecolor{currentstroke}{rgb}{0.000000,0.000000,0.000000}%
\pgfsetstrokecolor{currentstroke}%
\pgfsetstrokeopacity{0.000000}%
\pgfsetdash{}{0pt}%
\pgfpathmoveto{\pgfqpoint{0.790853in}{0.499444in}}%
\pgfpathlineto{\pgfqpoint{0.852239in}{0.499444in}}%
\pgfpathlineto{\pgfqpoint{0.852239in}{0.499444in}}%
\pgfpathlineto{\pgfqpoint{0.790853in}{0.499444in}}%
\pgfpathlineto{\pgfqpoint{0.790853in}{0.499444in}}%
\pgfpathclose%
\pgfusepath{fill}%
\end{pgfscope}%
\begin{pgfscope}%
\pgfpathrectangle{\pgfqpoint{0.445556in}{0.499444in}}{\pgfqpoint{3.875000in}{1.155000in}}%
\pgfusepath{clip}%
\pgfsetbuttcap%
\pgfsetmiterjoin%
\definecolor{currentfill}{rgb}{0.000000,0.000000,0.000000}%
\pgfsetfillcolor{currentfill}%
\pgfsetlinewidth{0.000000pt}%
\definecolor{currentstroke}{rgb}{0.000000,0.000000,0.000000}%
\pgfsetstrokecolor{currentstroke}%
\pgfsetstrokeopacity{0.000000}%
\pgfsetdash{}{0pt}%
\pgfpathmoveto{\pgfqpoint{0.944318in}{0.499444in}}%
\pgfpathlineto{\pgfqpoint{1.005704in}{0.499444in}}%
\pgfpathlineto{\pgfqpoint{1.005704in}{0.499444in}}%
\pgfpathlineto{\pgfqpoint{0.944318in}{0.499444in}}%
\pgfpathlineto{\pgfqpoint{0.944318in}{0.499444in}}%
\pgfpathclose%
\pgfusepath{fill}%
\end{pgfscope}%
\begin{pgfscope}%
\pgfpathrectangle{\pgfqpoint{0.445556in}{0.499444in}}{\pgfqpoint{3.875000in}{1.155000in}}%
\pgfusepath{clip}%
\pgfsetbuttcap%
\pgfsetmiterjoin%
\definecolor{currentfill}{rgb}{0.000000,0.000000,0.000000}%
\pgfsetfillcolor{currentfill}%
\pgfsetlinewidth{0.000000pt}%
\definecolor{currentstroke}{rgb}{0.000000,0.000000,0.000000}%
\pgfsetstrokecolor{currentstroke}%
\pgfsetstrokeopacity{0.000000}%
\pgfsetdash{}{0pt}%
\pgfpathmoveto{\pgfqpoint{1.097783in}{0.499444in}}%
\pgfpathlineto{\pgfqpoint{1.159170in}{0.499444in}}%
\pgfpathlineto{\pgfqpoint{1.159170in}{0.499444in}}%
\pgfpathlineto{\pgfqpoint{1.097783in}{0.499444in}}%
\pgfpathlineto{\pgfqpoint{1.097783in}{0.499444in}}%
\pgfpathclose%
\pgfusepath{fill}%
\end{pgfscope}%
\begin{pgfscope}%
\pgfpathrectangle{\pgfqpoint{0.445556in}{0.499444in}}{\pgfqpoint{3.875000in}{1.155000in}}%
\pgfusepath{clip}%
\pgfsetbuttcap%
\pgfsetmiterjoin%
\definecolor{currentfill}{rgb}{0.000000,0.000000,0.000000}%
\pgfsetfillcolor{currentfill}%
\pgfsetlinewidth{0.000000pt}%
\definecolor{currentstroke}{rgb}{0.000000,0.000000,0.000000}%
\pgfsetstrokecolor{currentstroke}%
\pgfsetstrokeopacity{0.000000}%
\pgfsetdash{}{0pt}%
\pgfpathmoveto{\pgfqpoint{1.251249in}{0.499444in}}%
\pgfpathlineto{\pgfqpoint{1.312635in}{0.499444in}}%
\pgfpathlineto{\pgfqpoint{1.312635in}{0.499444in}}%
\pgfpathlineto{\pgfqpoint{1.251249in}{0.499444in}}%
\pgfpathlineto{\pgfqpoint{1.251249in}{0.499444in}}%
\pgfpathclose%
\pgfusepath{fill}%
\end{pgfscope}%
\begin{pgfscope}%
\pgfpathrectangle{\pgfqpoint{0.445556in}{0.499444in}}{\pgfqpoint{3.875000in}{1.155000in}}%
\pgfusepath{clip}%
\pgfsetbuttcap%
\pgfsetmiterjoin%
\definecolor{currentfill}{rgb}{0.000000,0.000000,0.000000}%
\pgfsetfillcolor{currentfill}%
\pgfsetlinewidth{0.000000pt}%
\definecolor{currentstroke}{rgb}{0.000000,0.000000,0.000000}%
\pgfsetstrokecolor{currentstroke}%
\pgfsetstrokeopacity{0.000000}%
\pgfsetdash{}{0pt}%
\pgfpathmoveto{\pgfqpoint{1.404714in}{0.499444in}}%
\pgfpathlineto{\pgfqpoint{1.466100in}{0.499444in}}%
\pgfpathlineto{\pgfqpoint{1.466100in}{0.499444in}}%
\pgfpathlineto{\pgfqpoint{1.404714in}{0.499444in}}%
\pgfpathlineto{\pgfqpoint{1.404714in}{0.499444in}}%
\pgfpathclose%
\pgfusepath{fill}%
\end{pgfscope}%
\begin{pgfscope}%
\pgfpathrectangle{\pgfqpoint{0.445556in}{0.499444in}}{\pgfqpoint{3.875000in}{1.155000in}}%
\pgfusepath{clip}%
\pgfsetbuttcap%
\pgfsetmiterjoin%
\definecolor{currentfill}{rgb}{0.000000,0.000000,0.000000}%
\pgfsetfillcolor{currentfill}%
\pgfsetlinewidth{0.000000pt}%
\definecolor{currentstroke}{rgb}{0.000000,0.000000,0.000000}%
\pgfsetstrokecolor{currentstroke}%
\pgfsetstrokeopacity{0.000000}%
\pgfsetdash{}{0pt}%
\pgfpathmoveto{\pgfqpoint{1.558179in}{0.499444in}}%
\pgfpathlineto{\pgfqpoint{1.619566in}{0.499444in}}%
\pgfpathlineto{\pgfqpoint{1.619566in}{0.499444in}}%
\pgfpathlineto{\pgfqpoint{1.558179in}{0.499444in}}%
\pgfpathlineto{\pgfqpoint{1.558179in}{0.499444in}}%
\pgfpathclose%
\pgfusepath{fill}%
\end{pgfscope}%
\begin{pgfscope}%
\pgfpathrectangle{\pgfqpoint{0.445556in}{0.499444in}}{\pgfqpoint{3.875000in}{1.155000in}}%
\pgfusepath{clip}%
\pgfsetbuttcap%
\pgfsetmiterjoin%
\definecolor{currentfill}{rgb}{0.000000,0.000000,0.000000}%
\pgfsetfillcolor{currentfill}%
\pgfsetlinewidth{0.000000pt}%
\definecolor{currentstroke}{rgb}{0.000000,0.000000,0.000000}%
\pgfsetstrokecolor{currentstroke}%
\pgfsetstrokeopacity{0.000000}%
\pgfsetdash{}{0pt}%
\pgfpathmoveto{\pgfqpoint{1.711645in}{0.499444in}}%
\pgfpathlineto{\pgfqpoint{1.773031in}{0.499444in}}%
\pgfpathlineto{\pgfqpoint{1.773031in}{0.499444in}}%
\pgfpathlineto{\pgfqpoint{1.711645in}{0.499444in}}%
\pgfpathlineto{\pgfqpoint{1.711645in}{0.499444in}}%
\pgfpathclose%
\pgfusepath{fill}%
\end{pgfscope}%
\begin{pgfscope}%
\pgfpathrectangle{\pgfqpoint{0.445556in}{0.499444in}}{\pgfqpoint{3.875000in}{1.155000in}}%
\pgfusepath{clip}%
\pgfsetbuttcap%
\pgfsetmiterjoin%
\definecolor{currentfill}{rgb}{0.000000,0.000000,0.000000}%
\pgfsetfillcolor{currentfill}%
\pgfsetlinewidth{0.000000pt}%
\definecolor{currentstroke}{rgb}{0.000000,0.000000,0.000000}%
\pgfsetstrokecolor{currentstroke}%
\pgfsetstrokeopacity{0.000000}%
\pgfsetdash{}{0pt}%
\pgfpathmoveto{\pgfqpoint{1.865110in}{0.499444in}}%
\pgfpathlineto{\pgfqpoint{1.926496in}{0.499444in}}%
\pgfpathlineto{\pgfqpoint{1.926496in}{0.499444in}}%
\pgfpathlineto{\pgfqpoint{1.865110in}{0.499444in}}%
\pgfpathlineto{\pgfqpoint{1.865110in}{0.499444in}}%
\pgfpathclose%
\pgfusepath{fill}%
\end{pgfscope}%
\begin{pgfscope}%
\pgfpathrectangle{\pgfqpoint{0.445556in}{0.499444in}}{\pgfqpoint{3.875000in}{1.155000in}}%
\pgfusepath{clip}%
\pgfsetbuttcap%
\pgfsetmiterjoin%
\definecolor{currentfill}{rgb}{0.000000,0.000000,0.000000}%
\pgfsetfillcolor{currentfill}%
\pgfsetlinewidth{0.000000pt}%
\definecolor{currentstroke}{rgb}{0.000000,0.000000,0.000000}%
\pgfsetstrokecolor{currentstroke}%
\pgfsetstrokeopacity{0.000000}%
\pgfsetdash{}{0pt}%
\pgfpathmoveto{\pgfqpoint{2.018575in}{0.499444in}}%
\pgfpathlineto{\pgfqpoint{2.079962in}{0.499444in}}%
\pgfpathlineto{\pgfqpoint{2.079962in}{0.499444in}}%
\pgfpathlineto{\pgfqpoint{2.018575in}{0.499444in}}%
\pgfpathlineto{\pgfqpoint{2.018575in}{0.499444in}}%
\pgfpathclose%
\pgfusepath{fill}%
\end{pgfscope}%
\begin{pgfscope}%
\pgfpathrectangle{\pgfqpoint{0.445556in}{0.499444in}}{\pgfqpoint{3.875000in}{1.155000in}}%
\pgfusepath{clip}%
\pgfsetbuttcap%
\pgfsetmiterjoin%
\definecolor{currentfill}{rgb}{0.000000,0.000000,0.000000}%
\pgfsetfillcolor{currentfill}%
\pgfsetlinewidth{0.000000pt}%
\definecolor{currentstroke}{rgb}{0.000000,0.000000,0.000000}%
\pgfsetstrokecolor{currentstroke}%
\pgfsetstrokeopacity{0.000000}%
\pgfsetdash{}{0pt}%
\pgfpathmoveto{\pgfqpoint{2.172041in}{0.499444in}}%
\pgfpathlineto{\pgfqpoint{2.233427in}{0.499444in}}%
\pgfpathlineto{\pgfqpoint{2.233427in}{0.499444in}}%
\pgfpathlineto{\pgfqpoint{2.172041in}{0.499444in}}%
\pgfpathlineto{\pgfqpoint{2.172041in}{0.499444in}}%
\pgfpathclose%
\pgfusepath{fill}%
\end{pgfscope}%
\begin{pgfscope}%
\pgfpathrectangle{\pgfqpoint{0.445556in}{0.499444in}}{\pgfqpoint{3.875000in}{1.155000in}}%
\pgfusepath{clip}%
\pgfsetbuttcap%
\pgfsetmiterjoin%
\definecolor{currentfill}{rgb}{0.000000,0.000000,0.000000}%
\pgfsetfillcolor{currentfill}%
\pgfsetlinewidth{0.000000pt}%
\definecolor{currentstroke}{rgb}{0.000000,0.000000,0.000000}%
\pgfsetstrokecolor{currentstroke}%
\pgfsetstrokeopacity{0.000000}%
\pgfsetdash{}{0pt}%
\pgfpathmoveto{\pgfqpoint{2.325506in}{0.499444in}}%
\pgfpathlineto{\pgfqpoint{2.386892in}{0.499444in}}%
\pgfpathlineto{\pgfqpoint{2.386892in}{0.499561in}}%
\pgfpathlineto{\pgfqpoint{2.325506in}{0.499561in}}%
\pgfpathlineto{\pgfqpoint{2.325506in}{0.499444in}}%
\pgfpathclose%
\pgfusepath{fill}%
\end{pgfscope}%
\begin{pgfscope}%
\pgfpathrectangle{\pgfqpoint{0.445556in}{0.499444in}}{\pgfqpoint{3.875000in}{1.155000in}}%
\pgfusepath{clip}%
\pgfsetbuttcap%
\pgfsetmiterjoin%
\definecolor{currentfill}{rgb}{0.000000,0.000000,0.000000}%
\pgfsetfillcolor{currentfill}%
\pgfsetlinewidth{0.000000pt}%
\definecolor{currentstroke}{rgb}{0.000000,0.000000,0.000000}%
\pgfsetstrokecolor{currentstroke}%
\pgfsetstrokeopacity{0.000000}%
\pgfsetdash{}{0pt}%
\pgfpathmoveto{\pgfqpoint{2.478972in}{0.499444in}}%
\pgfpathlineto{\pgfqpoint{2.540358in}{0.499444in}}%
\pgfpathlineto{\pgfqpoint{2.540358in}{0.499854in}}%
\pgfpathlineto{\pgfqpoint{2.478972in}{0.499854in}}%
\pgfpathlineto{\pgfqpoint{2.478972in}{0.499444in}}%
\pgfpathclose%
\pgfusepath{fill}%
\end{pgfscope}%
\begin{pgfscope}%
\pgfpathrectangle{\pgfqpoint{0.445556in}{0.499444in}}{\pgfqpoint{3.875000in}{1.155000in}}%
\pgfusepath{clip}%
\pgfsetbuttcap%
\pgfsetmiterjoin%
\definecolor{currentfill}{rgb}{0.000000,0.000000,0.000000}%
\pgfsetfillcolor{currentfill}%
\pgfsetlinewidth{0.000000pt}%
\definecolor{currentstroke}{rgb}{0.000000,0.000000,0.000000}%
\pgfsetstrokecolor{currentstroke}%
\pgfsetstrokeopacity{0.000000}%
\pgfsetdash{}{0pt}%
\pgfpathmoveto{\pgfqpoint{2.632437in}{0.499444in}}%
\pgfpathlineto{\pgfqpoint{2.693823in}{0.499444in}}%
\pgfpathlineto{\pgfqpoint{2.693823in}{0.501521in}}%
\pgfpathlineto{\pgfqpoint{2.632437in}{0.501521in}}%
\pgfpathlineto{\pgfqpoint{2.632437in}{0.499444in}}%
\pgfpathclose%
\pgfusepath{fill}%
\end{pgfscope}%
\begin{pgfscope}%
\pgfpathrectangle{\pgfqpoint{0.445556in}{0.499444in}}{\pgfqpoint{3.875000in}{1.155000in}}%
\pgfusepath{clip}%
\pgfsetbuttcap%
\pgfsetmiterjoin%
\definecolor{currentfill}{rgb}{0.000000,0.000000,0.000000}%
\pgfsetfillcolor{currentfill}%
\pgfsetlinewidth{0.000000pt}%
\definecolor{currentstroke}{rgb}{0.000000,0.000000,0.000000}%
\pgfsetstrokecolor{currentstroke}%
\pgfsetstrokeopacity{0.000000}%
\pgfsetdash{}{0pt}%
\pgfpathmoveto{\pgfqpoint{2.785902in}{0.499444in}}%
\pgfpathlineto{\pgfqpoint{2.847288in}{0.499444in}}%
\pgfpathlineto{\pgfqpoint{2.847288in}{0.511171in}}%
\pgfpathlineto{\pgfqpoint{2.785902in}{0.511171in}}%
\pgfpathlineto{\pgfqpoint{2.785902in}{0.499444in}}%
\pgfpathclose%
\pgfusepath{fill}%
\end{pgfscope}%
\begin{pgfscope}%
\pgfpathrectangle{\pgfqpoint{0.445556in}{0.499444in}}{\pgfqpoint{3.875000in}{1.155000in}}%
\pgfusepath{clip}%
\pgfsetbuttcap%
\pgfsetmiterjoin%
\definecolor{currentfill}{rgb}{0.000000,0.000000,0.000000}%
\pgfsetfillcolor{currentfill}%
\pgfsetlinewidth{0.000000pt}%
\definecolor{currentstroke}{rgb}{0.000000,0.000000,0.000000}%
\pgfsetstrokecolor{currentstroke}%
\pgfsetstrokeopacity{0.000000}%
\pgfsetdash{}{0pt}%
\pgfpathmoveto{\pgfqpoint{2.939368in}{0.499444in}}%
\pgfpathlineto{\pgfqpoint{3.000754in}{0.499444in}}%
\pgfpathlineto{\pgfqpoint{3.000754in}{0.534889in}}%
\pgfpathlineto{\pgfqpoint{2.939368in}{0.534889in}}%
\pgfpathlineto{\pgfqpoint{2.939368in}{0.499444in}}%
\pgfpathclose%
\pgfusepath{fill}%
\end{pgfscope}%
\begin{pgfscope}%
\pgfpathrectangle{\pgfqpoint{0.445556in}{0.499444in}}{\pgfqpoint{3.875000in}{1.155000in}}%
\pgfusepath{clip}%
\pgfsetbuttcap%
\pgfsetmiterjoin%
\definecolor{currentfill}{rgb}{0.000000,0.000000,0.000000}%
\pgfsetfillcolor{currentfill}%
\pgfsetlinewidth{0.000000pt}%
\definecolor{currentstroke}{rgb}{0.000000,0.000000,0.000000}%
\pgfsetstrokecolor{currentstroke}%
\pgfsetstrokeopacity{0.000000}%
\pgfsetdash{}{0pt}%
\pgfpathmoveto{\pgfqpoint{3.092833in}{0.499444in}}%
\pgfpathlineto{\pgfqpoint{3.154219in}{0.499444in}}%
\pgfpathlineto{\pgfqpoint{3.154219in}{0.582998in}}%
\pgfpathlineto{\pgfqpoint{3.092833in}{0.582998in}}%
\pgfpathlineto{\pgfqpoint{3.092833in}{0.499444in}}%
\pgfpathclose%
\pgfusepath{fill}%
\end{pgfscope}%
\begin{pgfscope}%
\pgfpathrectangle{\pgfqpoint{0.445556in}{0.499444in}}{\pgfqpoint{3.875000in}{1.155000in}}%
\pgfusepath{clip}%
\pgfsetbuttcap%
\pgfsetmiterjoin%
\definecolor{currentfill}{rgb}{0.000000,0.000000,0.000000}%
\pgfsetfillcolor{currentfill}%
\pgfsetlinewidth{0.000000pt}%
\definecolor{currentstroke}{rgb}{0.000000,0.000000,0.000000}%
\pgfsetstrokecolor{currentstroke}%
\pgfsetstrokeopacity{0.000000}%
\pgfsetdash{}{0pt}%
\pgfpathmoveto{\pgfqpoint{3.246298in}{0.499444in}}%
\pgfpathlineto{\pgfqpoint{3.307684in}{0.499444in}}%
\pgfpathlineto{\pgfqpoint{3.307684in}{0.662340in}}%
\pgfpathlineto{\pgfqpoint{3.246298in}{0.662340in}}%
\pgfpathlineto{\pgfqpoint{3.246298in}{0.499444in}}%
\pgfpathclose%
\pgfusepath{fill}%
\end{pgfscope}%
\begin{pgfscope}%
\pgfpathrectangle{\pgfqpoint{0.445556in}{0.499444in}}{\pgfqpoint{3.875000in}{1.155000in}}%
\pgfusepath{clip}%
\pgfsetbuttcap%
\pgfsetmiterjoin%
\definecolor{currentfill}{rgb}{0.000000,0.000000,0.000000}%
\pgfsetfillcolor{currentfill}%
\pgfsetlinewidth{0.000000pt}%
\definecolor{currentstroke}{rgb}{0.000000,0.000000,0.000000}%
\pgfsetstrokecolor{currentstroke}%
\pgfsetstrokeopacity{0.000000}%
\pgfsetdash{}{0pt}%
\pgfpathmoveto{\pgfqpoint{3.399764in}{0.499444in}}%
\pgfpathlineto{\pgfqpoint{3.461150in}{0.499444in}}%
\pgfpathlineto{\pgfqpoint{3.461150in}{0.763967in}}%
\pgfpathlineto{\pgfqpoint{3.399764in}{0.763967in}}%
\pgfpathlineto{\pgfqpoint{3.399764in}{0.499444in}}%
\pgfpathclose%
\pgfusepath{fill}%
\end{pgfscope}%
\begin{pgfscope}%
\pgfpathrectangle{\pgfqpoint{0.445556in}{0.499444in}}{\pgfqpoint{3.875000in}{1.155000in}}%
\pgfusepath{clip}%
\pgfsetbuttcap%
\pgfsetmiterjoin%
\definecolor{currentfill}{rgb}{0.000000,0.000000,0.000000}%
\pgfsetfillcolor{currentfill}%
\pgfsetlinewidth{0.000000pt}%
\definecolor{currentstroke}{rgb}{0.000000,0.000000,0.000000}%
\pgfsetstrokecolor{currentstroke}%
\pgfsetstrokeopacity{0.000000}%
\pgfsetdash{}{0pt}%
\pgfpathmoveto{\pgfqpoint{3.553229in}{0.499444in}}%
\pgfpathlineto{\pgfqpoint{3.614615in}{0.499444in}}%
\pgfpathlineto{\pgfqpoint{3.614615in}{0.871794in}}%
\pgfpathlineto{\pgfqpoint{3.553229in}{0.871794in}}%
\pgfpathlineto{\pgfqpoint{3.553229in}{0.499444in}}%
\pgfpathclose%
\pgfusepath{fill}%
\end{pgfscope}%
\begin{pgfscope}%
\pgfpathrectangle{\pgfqpoint{0.445556in}{0.499444in}}{\pgfqpoint{3.875000in}{1.155000in}}%
\pgfusepath{clip}%
\pgfsetbuttcap%
\pgfsetmiterjoin%
\definecolor{currentfill}{rgb}{0.000000,0.000000,0.000000}%
\pgfsetfillcolor{currentfill}%
\pgfsetlinewidth{0.000000pt}%
\definecolor{currentstroke}{rgb}{0.000000,0.000000,0.000000}%
\pgfsetstrokecolor{currentstroke}%
\pgfsetstrokeopacity{0.000000}%
\pgfsetdash{}{0pt}%
\pgfpathmoveto{\pgfqpoint{3.706694in}{0.499444in}}%
\pgfpathlineto{\pgfqpoint{3.768080in}{0.499444in}}%
\pgfpathlineto{\pgfqpoint{3.768080in}{0.925810in}}%
\pgfpathlineto{\pgfqpoint{3.706694in}{0.925810in}}%
\pgfpathlineto{\pgfqpoint{3.706694in}{0.499444in}}%
\pgfpathclose%
\pgfusepath{fill}%
\end{pgfscope}%
\begin{pgfscope}%
\pgfpathrectangle{\pgfqpoint{0.445556in}{0.499444in}}{\pgfqpoint{3.875000in}{1.155000in}}%
\pgfusepath{clip}%
\pgfsetbuttcap%
\pgfsetmiterjoin%
\definecolor{currentfill}{rgb}{0.000000,0.000000,0.000000}%
\pgfsetfillcolor{currentfill}%
\pgfsetlinewidth{0.000000pt}%
\definecolor{currentstroke}{rgb}{0.000000,0.000000,0.000000}%
\pgfsetstrokecolor{currentstroke}%
\pgfsetstrokeopacity{0.000000}%
\pgfsetdash{}{0pt}%
\pgfpathmoveto{\pgfqpoint{3.860160in}{0.499444in}}%
\pgfpathlineto{\pgfqpoint{3.921546in}{0.499444in}}%
\pgfpathlineto{\pgfqpoint{3.921546in}{0.913586in}}%
\pgfpathlineto{\pgfqpoint{3.860160in}{0.913586in}}%
\pgfpathlineto{\pgfqpoint{3.860160in}{0.499444in}}%
\pgfpathclose%
\pgfusepath{fill}%
\end{pgfscope}%
\begin{pgfscope}%
\pgfpathrectangle{\pgfqpoint{0.445556in}{0.499444in}}{\pgfqpoint{3.875000in}{1.155000in}}%
\pgfusepath{clip}%
\pgfsetbuttcap%
\pgfsetmiterjoin%
\definecolor{currentfill}{rgb}{0.000000,0.000000,0.000000}%
\pgfsetfillcolor{currentfill}%
\pgfsetlinewidth{0.000000pt}%
\definecolor{currentstroke}{rgb}{0.000000,0.000000,0.000000}%
\pgfsetstrokecolor{currentstroke}%
\pgfsetstrokeopacity{0.000000}%
\pgfsetdash{}{0pt}%
\pgfpathmoveto{\pgfqpoint{4.013625in}{0.499444in}}%
\pgfpathlineto{\pgfqpoint{4.075011in}{0.499444in}}%
\pgfpathlineto{\pgfqpoint{4.075011in}{0.851907in}}%
\pgfpathlineto{\pgfqpoint{4.013625in}{0.851907in}}%
\pgfpathlineto{\pgfqpoint{4.013625in}{0.499444in}}%
\pgfpathclose%
\pgfusepath{fill}%
\end{pgfscope}%
\begin{pgfscope}%
\pgfpathrectangle{\pgfqpoint{0.445556in}{0.499444in}}{\pgfqpoint{3.875000in}{1.155000in}}%
\pgfusepath{clip}%
\pgfsetbuttcap%
\pgfsetmiterjoin%
\definecolor{currentfill}{rgb}{0.000000,0.000000,0.000000}%
\pgfsetfillcolor{currentfill}%
\pgfsetlinewidth{0.000000pt}%
\definecolor{currentstroke}{rgb}{0.000000,0.000000,0.000000}%
\pgfsetstrokecolor{currentstroke}%
\pgfsetstrokeopacity{0.000000}%
\pgfsetdash{}{0pt}%
\pgfpathmoveto{\pgfqpoint{4.167090in}{0.499444in}}%
\pgfpathlineto{\pgfqpoint{4.228476in}{0.499444in}}%
\pgfpathlineto{\pgfqpoint{4.228476in}{0.681583in}}%
\pgfpathlineto{\pgfqpoint{4.167090in}{0.681583in}}%
\pgfpathlineto{\pgfqpoint{4.167090in}{0.499444in}}%
\pgfpathclose%
\pgfusepath{fill}%
\end{pgfscope}%
\begin{pgfscope}%
\pgfsetbuttcap%
\pgfsetroundjoin%
\definecolor{currentfill}{rgb}{0.000000,0.000000,0.000000}%
\pgfsetfillcolor{currentfill}%
\pgfsetlinewidth{0.803000pt}%
\definecolor{currentstroke}{rgb}{0.000000,0.000000,0.000000}%
\pgfsetstrokecolor{currentstroke}%
\pgfsetdash{}{0pt}%
\pgfsys@defobject{currentmarker}{\pgfqpoint{0.000000in}{-0.048611in}}{\pgfqpoint{0.000000in}{0.000000in}}{%
\pgfpathmoveto{\pgfqpoint{0.000000in}{0.000000in}}%
\pgfpathlineto{\pgfqpoint{0.000000in}{-0.048611in}}%
\pgfusepath{stroke,fill}%
}%
\begin{pgfscope}%
\pgfsys@transformshift{0.483922in}{0.499444in}%
\pgfsys@useobject{currentmarker}{}%
\end{pgfscope}%
\end{pgfscope}%
\begin{pgfscope}%
\definecolor{textcolor}{rgb}{0.000000,0.000000,0.000000}%
\pgfsetstrokecolor{textcolor}%
\pgfsetfillcolor{textcolor}%
\pgftext[x=0.483922in,y=0.402222in,,top]{\color{textcolor}\rmfamily\fontsize{10.000000}{12.000000}\selectfont 0.0}%
\end{pgfscope}%
\begin{pgfscope}%
\pgfsetbuttcap%
\pgfsetroundjoin%
\definecolor{currentfill}{rgb}{0.000000,0.000000,0.000000}%
\pgfsetfillcolor{currentfill}%
\pgfsetlinewidth{0.803000pt}%
\definecolor{currentstroke}{rgb}{0.000000,0.000000,0.000000}%
\pgfsetstrokecolor{currentstroke}%
\pgfsetdash{}{0pt}%
\pgfsys@defobject{currentmarker}{\pgfqpoint{0.000000in}{-0.048611in}}{\pgfqpoint{0.000000in}{0.000000in}}{%
\pgfpathmoveto{\pgfqpoint{0.000000in}{0.000000in}}%
\pgfpathlineto{\pgfqpoint{0.000000in}{-0.048611in}}%
\pgfusepath{stroke,fill}%
}%
\begin{pgfscope}%
\pgfsys@transformshift{0.867585in}{0.499444in}%
\pgfsys@useobject{currentmarker}{}%
\end{pgfscope}%
\end{pgfscope}%
\begin{pgfscope}%
\definecolor{textcolor}{rgb}{0.000000,0.000000,0.000000}%
\pgfsetstrokecolor{textcolor}%
\pgfsetfillcolor{textcolor}%
\pgftext[x=0.867585in,y=0.402222in,,top]{\color{textcolor}\rmfamily\fontsize{10.000000}{12.000000}\selectfont 0.1}%
\end{pgfscope}%
\begin{pgfscope}%
\pgfsetbuttcap%
\pgfsetroundjoin%
\definecolor{currentfill}{rgb}{0.000000,0.000000,0.000000}%
\pgfsetfillcolor{currentfill}%
\pgfsetlinewidth{0.803000pt}%
\definecolor{currentstroke}{rgb}{0.000000,0.000000,0.000000}%
\pgfsetstrokecolor{currentstroke}%
\pgfsetdash{}{0pt}%
\pgfsys@defobject{currentmarker}{\pgfqpoint{0.000000in}{-0.048611in}}{\pgfqpoint{0.000000in}{0.000000in}}{%
\pgfpathmoveto{\pgfqpoint{0.000000in}{0.000000in}}%
\pgfpathlineto{\pgfqpoint{0.000000in}{-0.048611in}}%
\pgfusepath{stroke,fill}%
}%
\begin{pgfscope}%
\pgfsys@transformshift{1.251249in}{0.499444in}%
\pgfsys@useobject{currentmarker}{}%
\end{pgfscope}%
\end{pgfscope}%
\begin{pgfscope}%
\definecolor{textcolor}{rgb}{0.000000,0.000000,0.000000}%
\pgfsetstrokecolor{textcolor}%
\pgfsetfillcolor{textcolor}%
\pgftext[x=1.251249in,y=0.402222in,,top]{\color{textcolor}\rmfamily\fontsize{10.000000}{12.000000}\selectfont 0.2}%
\end{pgfscope}%
\begin{pgfscope}%
\pgfsetbuttcap%
\pgfsetroundjoin%
\definecolor{currentfill}{rgb}{0.000000,0.000000,0.000000}%
\pgfsetfillcolor{currentfill}%
\pgfsetlinewidth{0.803000pt}%
\definecolor{currentstroke}{rgb}{0.000000,0.000000,0.000000}%
\pgfsetstrokecolor{currentstroke}%
\pgfsetdash{}{0pt}%
\pgfsys@defobject{currentmarker}{\pgfqpoint{0.000000in}{-0.048611in}}{\pgfqpoint{0.000000in}{0.000000in}}{%
\pgfpathmoveto{\pgfqpoint{0.000000in}{0.000000in}}%
\pgfpathlineto{\pgfqpoint{0.000000in}{-0.048611in}}%
\pgfusepath{stroke,fill}%
}%
\begin{pgfscope}%
\pgfsys@transformshift{1.634912in}{0.499444in}%
\pgfsys@useobject{currentmarker}{}%
\end{pgfscope}%
\end{pgfscope}%
\begin{pgfscope}%
\definecolor{textcolor}{rgb}{0.000000,0.000000,0.000000}%
\pgfsetstrokecolor{textcolor}%
\pgfsetfillcolor{textcolor}%
\pgftext[x=1.634912in,y=0.402222in,,top]{\color{textcolor}\rmfamily\fontsize{10.000000}{12.000000}\selectfont 0.3}%
\end{pgfscope}%
\begin{pgfscope}%
\pgfsetbuttcap%
\pgfsetroundjoin%
\definecolor{currentfill}{rgb}{0.000000,0.000000,0.000000}%
\pgfsetfillcolor{currentfill}%
\pgfsetlinewidth{0.803000pt}%
\definecolor{currentstroke}{rgb}{0.000000,0.000000,0.000000}%
\pgfsetstrokecolor{currentstroke}%
\pgfsetdash{}{0pt}%
\pgfsys@defobject{currentmarker}{\pgfqpoint{0.000000in}{-0.048611in}}{\pgfqpoint{0.000000in}{0.000000in}}{%
\pgfpathmoveto{\pgfqpoint{0.000000in}{0.000000in}}%
\pgfpathlineto{\pgfqpoint{0.000000in}{-0.048611in}}%
\pgfusepath{stroke,fill}%
}%
\begin{pgfscope}%
\pgfsys@transformshift{2.018575in}{0.499444in}%
\pgfsys@useobject{currentmarker}{}%
\end{pgfscope}%
\end{pgfscope}%
\begin{pgfscope}%
\definecolor{textcolor}{rgb}{0.000000,0.000000,0.000000}%
\pgfsetstrokecolor{textcolor}%
\pgfsetfillcolor{textcolor}%
\pgftext[x=2.018575in,y=0.402222in,,top]{\color{textcolor}\rmfamily\fontsize{10.000000}{12.000000}\selectfont 0.4}%
\end{pgfscope}%
\begin{pgfscope}%
\pgfsetbuttcap%
\pgfsetroundjoin%
\definecolor{currentfill}{rgb}{0.000000,0.000000,0.000000}%
\pgfsetfillcolor{currentfill}%
\pgfsetlinewidth{0.803000pt}%
\definecolor{currentstroke}{rgb}{0.000000,0.000000,0.000000}%
\pgfsetstrokecolor{currentstroke}%
\pgfsetdash{}{0pt}%
\pgfsys@defobject{currentmarker}{\pgfqpoint{0.000000in}{-0.048611in}}{\pgfqpoint{0.000000in}{0.000000in}}{%
\pgfpathmoveto{\pgfqpoint{0.000000in}{0.000000in}}%
\pgfpathlineto{\pgfqpoint{0.000000in}{-0.048611in}}%
\pgfusepath{stroke,fill}%
}%
\begin{pgfscope}%
\pgfsys@transformshift{2.402239in}{0.499444in}%
\pgfsys@useobject{currentmarker}{}%
\end{pgfscope}%
\end{pgfscope}%
\begin{pgfscope}%
\definecolor{textcolor}{rgb}{0.000000,0.000000,0.000000}%
\pgfsetstrokecolor{textcolor}%
\pgfsetfillcolor{textcolor}%
\pgftext[x=2.402239in,y=0.402222in,,top]{\color{textcolor}\rmfamily\fontsize{10.000000}{12.000000}\selectfont 0.5}%
\end{pgfscope}%
\begin{pgfscope}%
\pgfsetbuttcap%
\pgfsetroundjoin%
\definecolor{currentfill}{rgb}{0.000000,0.000000,0.000000}%
\pgfsetfillcolor{currentfill}%
\pgfsetlinewidth{0.803000pt}%
\definecolor{currentstroke}{rgb}{0.000000,0.000000,0.000000}%
\pgfsetstrokecolor{currentstroke}%
\pgfsetdash{}{0pt}%
\pgfsys@defobject{currentmarker}{\pgfqpoint{0.000000in}{-0.048611in}}{\pgfqpoint{0.000000in}{0.000000in}}{%
\pgfpathmoveto{\pgfqpoint{0.000000in}{0.000000in}}%
\pgfpathlineto{\pgfqpoint{0.000000in}{-0.048611in}}%
\pgfusepath{stroke,fill}%
}%
\begin{pgfscope}%
\pgfsys@transformshift{2.785902in}{0.499444in}%
\pgfsys@useobject{currentmarker}{}%
\end{pgfscope}%
\end{pgfscope}%
\begin{pgfscope}%
\definecolor{textcolor}{rgb}{0.000000,0.000000,0.000000}%
\pgfsetstrokecolor{textcolor}%
\pgfsetfillcolor{textcolor}%
\pgftext[x=2.785902in,y=0.402222in,,top]{\color{textcolor}\rmfamily\fontsize{10.000000}{12.000000}\selectfont 0.6}%
\end{pgfscope}%
\begin{pgfscope}%
\pgfsetbuttcap%
\pgfsetroundjoin%
\definecolor{currentfill}{rgb}{0.000000,0.000000,0.000000}%
\pgfsetfillcolor{currentfill}%
\pgfsetlinewidth{0.803000pt}%
\definecolor{currentstroke}{rgb}{0.000000,0.000000,0.000000}%
\pgfsetstrokecolor{currentstroke}%
\pgfsetdash{}{0pt}%
\pgfsys@defobject{currentmarker}{\pgfqpoint{0.000000in}{-0.048611in}}{\pgfqpoint{0.000000in}{0.000000in}}{%
\pgfpathmoveto{\pgfqpoint{0.000000in}{0.000000in}}%
\pgfpathlineto{\pgfqpoint{0.000000in}{-0.048611in}}%
\pgfusepath{stroke,fill}%
}%
\begin{pgfscope}%
\pgfsys@transformshift{3.169566in}{0.499444in}%
\pgfsys@useobject{currentmarker}{}%
\end{pgfscope}%
\end{pgfscope}%
\begin{pgfscope}%
\definecolor{textcolor}{rgb}{0.000000,0.000000,0.000000}%
\pgfsetstrokecolor{textcolor}%
\pgfsetfillcolor{textcolor}%
\pgftext[x=3.169566in,y=0.402222in,,top]{\color{textcolor}\rmfamily\fontsize{10.000000}{12.000000}\selectfont 0.7}%
\end{pgfscope}%
\begin{pgfscope}%
\pgfsetbuttcap%
\pgfsetroundjoin%
\definecolor{currentfill}{rgb}{0.000000,0.000000,0.000000}%
\pgfsetfillcolor{currentfill}%
\pgfsetlinewidth{0.803000pt}%
\definecolor{currentstroke}{rgb}{0.000000,0.000000,0.000000}%
\pgfsetstrokecolor{currentstroke}%
\pgfsetdash{}{0pt}%
\pgfsys@defobject{currentmarker}{\pgfqpoint{0.000000in}{-0.048611in}}{\pgfqpoint{0.000000in}{0.000000in}}{%
\pgfpathmoveto{\pgfqpoint{0.000000in}{0.000000in}}%
\pgfpathlineto{\pgfqpoint{0.000000in}{-0.048611in}}%
\pgfusepath{stroke,fill}%
}%
\begin{pgfscope}%
\pgfsys@transformshift{3.553229in}{0.499444in}%
\pgfsys@useobject{currentmarker}{}%
\end{pgfscope}%
\end{pgfscope}%
\begin{pgfscope}%
\definecolor{textcolor}{rgb}{0.000000,0.000000,0.000000}%
\pgfsetstrokecolor{textcolor}%
\pgfsetfillcolor{textcolor}%
\pgftext[x=3.553229in,y=0.402222in,,top]{\color{textcolor}\rmfamily\fontsize{10.000000}{12.000000}\selectfont 0.8}%
\end{pgfscope}%
\begin{pgfscope}%
\pgfsetbuttcap%
\pgfsetroundjoin%
\definecolor{currentfill}{rgb}{0.000000,0.000000,0.000000}%
\pgfsetfillcolor{currentfill}%
\pgfsetlinewidth{0.803000pt}%
\definecolor{currentstroke}{rgb}{0.000000,0.000000,0.000000}%
\pgfsetstrokecolor{currentstroke}%
\pgfsetdash{}{0pt}%
\pgfsys@defobject{currentmarker}{\pgfqpoint{0.000000in}{-0.048611in}}{\pgfqpoint{0.000000in}{0.000000in}}{%
\pgfpathmoveto{\pgfqpoint{0.000000in}{0.000000in}}%
\pgfpathlineto{\pgfqpoint{0.000000in}{-0.048611in}}%
\pgfusepath{stroke,fill}%
}%
\begin{pgfscope}%
\pgfsys@transformshift{3.936892in}{0.499444in}%
\pgfsys@useobject{currentmarker}{}%
\end{pgfscope}%
\end{pgfscope}%
\begin{pgfscope}%
\definecolor{textcolor}{rgb}{0.000000,0.000000,0.000000}%
\pgfsetstrokecolor{textcolor}%
\pgfsetfillcolor{textcolor}%
\pgftext[x=3.936892in,y=0.402222in,,top]{\color{textcolor}\rmfamily\fontsize{10.000000}{12.000000}\selectfont 0.9}%
\end{pgfscope}%
\begin{pgfscope}%
\pgfsetbuttcap%
\pgfsetroundjoin%
\definecolor{currentfill}{rgb}{0.000000,0.000000,0.000000}%
\pgfsetfillcolor{currentfill}%
\pgfsetlinewidth{0.803000pt}%
\definecolor{currentstroke}{rgb}{0.000000,0.000000,0.000000}%
\pgfsetstrokecolor{currentstroke}%
\pgfsetdash{}{0pt}%
\pgfsys@defobject{currentmarker}{\pgfqpoint{0.000000in}{-0.048611in}}{\pgfqpoint{0.000000in}{0.000000in}}{%
\pgfpathmoveto{\pgfqpoint{0.000000in}{0.000000in}}%
\pgfpathlineto{\pgfqpoint{0.000000in}{-0.048611in}}%
\pgfusepath{stroke,fill}%
}%
\begin{pgfscope}%
\pgfsys@transformshift{4.320556in}{0.499444in}%
\pgfsys@useobject{currentmarker}{}%
\end{pgfscope}%
\end{pgfscope}%
\begin{pgfscope}%
\definecolor{textcolor}{rgb}{0.000000,0.000000,0.000000}%
\pgfsetstrokecolor{textcolor}%
\pgfsetfillcolor{textcolor}%
\pgftext[x=4.320556in,y=0.402222in,,top]{\color{textcolor}\rmfamily\fontsize{10.000000}{12.000000}\selectfont 1.0}%
\end{pgfscope}%
\begin{pgfscope}%
\definecolor{textcolor}{rgb}{0.000000,0.000000,0.000000}%
\pgfsetstrokecolor{textcolor}%
\pgfsetfillcolor{textcolor}%
\pgftext[x=2.383056in,y=0.223333in,,top]{\color{textcolor}\rmfamily\fontsize{10.000000}{12.000000}\selectfont \(\displaystyle p\)}%
\end{pgfscope}%
\begin{pgfscope}%
\pgfsetbuttcap%
\pgfsetroundjoin%
\definecolor{currentfill}{rgb}{0.000000,0.000000,0.000000}%
\pgfsetfillcolor{currentfill}%
\pgfsetlinewidth{0.803000pt}%
\definecolor{currentstroke}{rgb}{0.000000,0.000000,0.000000}%
\pgfsetstrokecolor{currentstroke}%
\pgfsetdash{}{0pt}%
\pgfsys@defobject{currentmarker}{\pgfqpoint{-0.048611in}{0.000000in}}{\pgfqpoint{-0.000000in}{0.000000in}}{%
\pgfpathmoveto{\pgfqpoint{-0.000000in}{0.000000in}}%
\pgfpathlineto{\pgfqpoint{-0.048611in}{0.000000in}}%
\pgfusepath{stroke,fill}%
}%
\begin{pgfscope}%
\pgfsys@transformshift{0.445556in}{0.499444in}%
\pgfsys@useobject{currentmarker}{}%
\end{pgfscope}%
\end{pgfscope}%
\begin{pgfscope}%
\definecolor{textcolor}{rgb}{0.000000,0.000000,0.000000}%
\pgfsetstrokecolor{textcolor}%
\pgfsetfillcolor{textcolor}%
\pgftext[x=0.278889in, y=0.451250in, left, base]{\color{textcolor}\rmfamily\fontsize{10.000000}{12.000000}\selectfont \(\displaystyle {0}\)}%
\end{pgfscope}%
\begin{pgfscope}%
\pgfsetbuttcap%
\pgfsetroundjoin%
\definecolor{currentfill}{rgb}{0.000000,0.000000,0.000000}%
\pgfsetfillcolor{currentfill}%
\pgfsetlinewidth{0.803000pt}%
\definecolor{currentstroke}{rgb}{0.000000,0.000000,0.000000}%
\pgfsetstrokecolor{currentstroke}%
\pgfsetdash{}{0pt}%
\pgfsys@defobject{currentmarker}{\pgfqpoint{-0.048611in}{0.000000in}}{\pgfqpoint{-0.000000in}{0.000000in}}{%
\pgfpathmoveto{\pgfqpoint{-0.000000in}{0.000000in}}%
\pgfpathlineto{\pgfqpoint{-0.048611in}{0.000000in}}%
\pgfusepath{stroke,fill}%
}%
\begin{pgfscope}%
\pgfsys@transformshift{0.445556in}{0.791601in}%
\pgfsys@useobject{currentmarker}{}%
\end{pgfscope}%
\end{pgfscope}%
\begin{pgfscope}%
\definecolor{textcolor}{rgb}{0.000000,0.000000,0.000000}%
\pgfsetstrokecolor{textcolor}%
\pgfsetfillcolor{textcolor}%
\pgftext[x=0.278889in, y=0.743407in, left, base]{\color{textcolor}\rmfamily\fontsize{10.000000}{12.000000}\selectfont \(\displaystyle {2}\)}%
\end{pgfscope}%
\begin{pgfscope}%
\pgfsetbuttcap%
\pgfsetroundjoin%
\definecolor{currentfill}{rgb}{0.000000,0.000000,0.000000}%
\pgfsetfillcolor{currentfill}%
\pgfsetlinewidth{0.803000pt}%
\definecolor{currentstroke}{rgb}{0.000000,0.000000,0.000000}%
\pgfsetstrokecolor{currentstroke}%
\pgfsetdash{}{0pt}%
\pgfsys@defobject{currentmarker}{\pgfqpoint{-0.048611in}{0.000000in}}{\pgfqpoint{-0.000000in}{0.000000in}}{%
\pgfpathmoveto{\pgfqpoint{-0.000000in}{0.000000in}}%
\pgfpathlineto{\pgfqpoint{-0.048611in}{0.000000in}}%
\pgfusepath{stroke,fill}%
}%
\begin{pgfscope}%
\pgfsys@transformshift{0.445556in}{1.083759in}%
\pgfsys@useobject{currentmarker}{}%
\end{pgfscope}%
\end{pgfscope}%
\begin{pgfscope}%
\definecolor{textcolor}{rgb}{0.000000,0.000000,0.000000}%
\pgfsetstrokecolor{textcolor}%
\pgfsetfillcolor{textcolor}%
\pgftext[x=0.278889in, y=1.035564in, left, base]{\color{textcolor}\rmfamily\fontsize{10.000000}{12.000000}\selectfont \(\displaystyle {4}\)}%
\end{pgfscope}%
\begin{pgfscope}%
\pgfsetbuttcap%
\pgfsetroundjoin%
\definecolor{currentfill}{rgb}{0.000000,0.000000,0.000000}%
\pgfsetfillcolor{currentfill}%
\pgfsetlinewidth{0.803000pt}%
\definecolor{currentstroke}{rgb}{0.000000,0.000000,0.000000}%
\pgfsetstrokecolor{currentstroke}%
\pgfsetdash{}{0pt}%
\pgfsys@defobject{currentmarker}{\pgfqpoint{-0.048611in}{0.000000in}}{\pgfqpoint{-0.000000in}{0.000000in}}{%
\pgfpathmoveto{\pgfqpoint{-0.000000in}{0.000000in}}%
\pgfpathlineto{\pgfqpoint{-0.048611in}{0.000000in}}%
\pgfusepath{stroke,fill}%
}%
\begin{pgfscope}%
\pgfsys@transformshift{0.445556in}{1.375916in}%
\pgfsys@useobject{currentmarker}{}%
\end{pgfscope}%
\end{pgfscope}%
\begin{pgfscope}%
\definecolor{textcolor}{rgb}{0.000000,0.000000,0.000000}%
\pgfsetstrokecolor{textcolor}%
\pgfsetfillcolor{textcolor}%
\pgftext[x=0.278889in, y=1.327722in, left, base]{\color{textcolor}\rmfamily\fontsize{10.000000}{12.000000}\selectfont \(\displaystyle {6}\)}%
\end{pgfscope}%
\begin{pgfscope}%
\definecolor{textcolor}{rgb}{0.000000,0.000000,0.000000}%
\pgfsetstrokecolor{textcolor}%
\pgfsetfillcolor{textcolor}%
\pgftext[x=0.223333in,y=1.076944in,,bottom,rotate=90.000000]{\color{textcolor}\rmfamily\fontsize{10.000000}{12.000000}\selectfont Percent of Data Set}%
\end{pgfscope}%
\begin{pgfscope}%
\pgfsetrectcap%
\pgfsetmiterjoin%
\pgfsetlinewidth{0.803000pt}%
\definecolor{currentstroke}{rgb}{0.000000,0.000000,0.000000}%
\pgfsetstrokecolor{currentstroke}%
\pgfsetdash{}{0pt}%
\pgfpathmoveto{\pgfqpoint{0.445556in}{0.499444in}}%
\pgfpathlineto{\pgfqpoint{0.445556in}{1.654444in}}%
\pgfusepath{stroke}%
\end{pgfscope}%
\begin{pgfscope}%
\pgfsetrectcap%
\pgfsetmiterjoin%
\pgfsetlinewidth{0.803000pt}%
\definecolor{currentstroke}{rgb}{0.000000,0.000000,0.000000}%
\pgfsetstrokecolor{currentstroke}%
\pgfsetdash{}{0pt}%
\pgfpathmoveto{\pgfqpoint{4.320556in}{0.499444in}}%
\pgfpathlineto{\pgfqpoint{4.320556in}{1.654444in}}%
\pgfusepath{stroke}%
\end{pgfscope}%
\begin{pgfscope}%
\pgfsetrectcap%
\pgfsetmiterjoin%
\pgfsetlinewidth{0.803000pt}%
\definecolor{currentstroke}{rgb}{0.000000,0.000000,0.000000}%
\pgfsetstrokecolor{currentstroke}%
\pgfsetdash{}{0pt}%
\pgfpathmoveto{\pgfqpoint{0.445556in}{0.499444in}}%
\pgfpathlineto{\pgfqpoint{4.320556in}{0.499444in}}%
\pgfusepath{stroke}%
\end{pgfscope}%
\begin{pgfscope}%
\pgfsetrectcap%
\pgfsetmiterjoin%
\pgfsetlinewidth{0.803000pt}%
\definecolor{currentstroke}{rgb}{0.000000,0.000000,0.000000}%
\pgfsetstrokecolor{currentstroke}%
\pgfsetdash{}{0pt}%
\pgfpathmoveto{\pgfqpoint{0.445556in}{1.654444in}}%
\pgfpathlineto{\pgfqpoint{4.320556in}{1.654444in}}%
\pgfusepath{stroke}%
\end{pgfscope}%
\begin{pgfscope}%
\pgfsetbuttcap%
\pgfsetmiterjoin%
\definecolor{currentfill}{rgb}{1.000000,1.000000,1.000000}%
\pgfsetfillcolor{currentfill}%
\pgfsetfillopacity{0.800000}%
\pgfsetlinewidth{1.003750pt}%
\definecolor{currentstroke}{rgb}{0.800000,0.800000,0.800000}%
\pgfsetstrokecolor{currentstroke}%
\pgfsetstrokeopacity{0.800000}%
\pgfsetdash{}{0pt}%
\pgfpathmoveto{\pgfqpoint{3.543611in}{1.154445in}}%
\pgfpathlineto{\pgfqpoint{4.223333in}{1.154445in}}%
\pgfpathquadraticcurveto{\pgfqpoint{4.251111in}{1.154445in}}{\pgfqpoint{4.251111in}{1.182222in}}%
\pgfpathlineto{\pgfqpoint{4.251111in}{1.557222in}}%
\pgfpathquadraticcurveto{\pgfqpoint{4.251111in}{1.585000in}}{\pgfqpoint{4.223333in}{1.585000in}}%
\pgfpathlineto{\pgfqpoint{3.543611in}{1.585000in}}%
\pgfpathquadraticcurveto{\pgfqpoint{3.515833in}{1.585000in}}{\pgfqpoint{3.515833in}{1.557222in}}%
\pgfpathlineto{\pgfqpoint{3.515833in}{1.182222in}}%
\pgfpathquadraticcurveto{\pgfqpoint{3.515833in}{1.154445in}}{\pgfqpoint{3.543611in}{1.154445in}}%
\pgfpathlineto{\pgfqpoint{3.543611in}{1.154445in}}%
\pgfpathclose%
\pgfusepath{stroke,fill}%
\end{pgfscope}%
\begin{pgfscope}%
\pgfsetbuttcap%
\pgfsetmiterjoin%
\pgfsetlinewidth{1.003750pt}%
\definecolor{currentstroke}{rgb}{0.000000,0.000000,0.000000}%
\pgfsetstrokecolor{currentstroke}%
\pgfsetdash{}{0pt}%
\pgfpathmoveto{\pgfqpoint{3.571389in}{1.432222in}}%
\pgfpathlineto{\pgfqpoint{3.849167in}{1.432222in}}%
\pgfpathlineto{\pgfqpoint{3.849167in}{1.529444in}}%
\pgfpathlineto{\pgfqpoint{3.571389in}{1.529444in}}%
\pgfpathlineto{\pgfqpoint{3.571389in}{1.432222in}}%
\pgfpathclose%
\pgfusepath{stroke}%
\end{pgfscope}%
\begin{pgfscope}%
\definecolor{textcolor}{rgb}{0.000000,0.000000,0.000000}%
\pgfsetstrokecolor{textcolor}%
\pgfsetfillcolor{textcolor}%
\pgftext[x=3.960278in,y=1.432222in,left,base]{\color{textcolor}\rmfamily\fontsize{10.000000}{12.000000}\selectfont Neg}%
\end{pgfscope}%
\begin{pgfscope}%
\pgfsetbuttcap%
\pgfsetmiterjoin%
\definecolor{currentfill}{rgb}{0.000000,0.000000,0.000000}%
\pgfsetfillcolor{currentfill}%
\pgfsetlinewidth{0.000000pt}%
\definecolor{currentstroke}{rgb}{0.000000,0.000000,0.000000}%
\pgfsetstrokecolor{currentstroke}%
\pgfsetstrokeopacity{0.000000}%
\pgfsetdash{}{0pt}%
\pgfpathmoveto{\pgfqpoint{3.571389in}{1.236944in}}%
\pgfpathlineto{\pgfqpoint{3.849167in}{1.236944in}}%
\pgfpathlineto{\pgfqpoint{3.849167in}{1.334167in}}%
\pgfpathlineto{\pgfqpoint{3.571389in}{1.334167in}}%
\pgfpathlineto{\pgfqpoint{3.571389in}{1.236944in}}%
\pgfpathclose%
\pgfusepath{fill}%
\end{pgfscope}%
\begin{pgfscope}%
\definecolor{textcolor}{rgb}{0.000000,0.000000,0.000000}%
\pgfsetstrokecolor{textcolor}%
\pgfsetfillcolor{textcolor}%
\pgftext[x=3.960278in,y=1.236944in,left,base]{\color{textcolor}\rmfamily\fontsize{10.000000}{12.000000}\selectfont Pos}%
\end{pgfscope}%
\end{pgfpicture}%
\makeatother%
\endgroup%
	
&
	\vskip 0pt
	\hfil ROC Curve
	
	%% Creator: Matplotlib, PGF backend
%%
%% To include the figure in your LaTeX document, write
%%   \input{<filename>.pgf}
%%
%% Make sure the required packages are loaded in your preamble
%%   \usepackage{pgf}
%%
%% Also ensure that all the required font packages are loaded; for instance,
%% the lmodern package is sometimes necessary when using math font.
%%   \usepackage{lmodern}
%%
%% Figures using additional raster images can only be included by \input if
%% they are in the same directory as the main LaTeX file. For loading figures
%% from other directories you can use the `import` package
%%   \usepackage{import}
%%
%% and then include the figures with
%%   \import{<path to file>}{<filename>.pgf}
%%
%% Matplotlib used the following preamble
%%   
%%   \usepackage{fontspec}
%%   \makeatletter\@ifpackageloaded{underscore}{}{\usepackage[strings]{underscore}}\makeatother
%%
\begingroup%
\makeatletter%
\begin{pgfpicture}%
\pgfpathrectangle{\pgfpointorigin}{\pgfqpoint{2.221861in}{1.754444in}}%
\pgfusepath{use as bounding box, clip}%
\begin{pgfscope}%
\pgfsetbuttcap%
\pgfsetmiterjoin%
\definecolor{currentfill}{rgb}{1.000000,1.000000,1.000000}%
\pgfsetfillcolor{currentfill}%
\pgfsetlinewidth{0.000000pt}%
\definecolor{currentstroke}{rgb}{1.000000,1.000000,1.000000}%
\pgfsetstrokecolor{currentstroke}%
\pgfsetdash{}{0pt}%
\pgfpathmoveto{\pgfqpoint{0.000000in}{0.000000in}}%
\pgfpathlineto{\pgfqpoint{2.221861in}{0.000000in}}%
\pgfpathlineto{\pgfqpoint{2.221861in}{1.754444in}}%
\pgfpathlineto{\pgfqpoint{0.000000in}{1.754444in}}%
\pgfpathlineto{\pgfqpoint{0.000000in}{0.000000in}}%
\pgfpathclose%
\pgfusepath{fill}%
\end{pgfscope}%
\begin{pgfscope}%
\pgfsetbuttcap%
\pgfsetmiterjoin%
\definecolor{currentfill}{rgb}{1.000000,1.000000,1.000000}%
\pgfsetfillcolor{currentfill}%
\pgfsetlinewidth{0.000000pt}%
\definecolor{currentstroke}{rgb}{0.000000,0.000000,0.000000}%
\pgfsetstrokecolor{currentstroke}%
\pgfsetstrokeopacity{0.000000}%
\pgfsetdash{}{0pt}%
\pgfpathmoveto{\pgfqpoint{0.553581in}{0.499444in}}%
\pgfpathlineto{\pgfqpoint{2.103581in}{0.499444in}}%
\pgfpathlineto{\pgfqpoint{2.103581in}{1.654444in}}%
\pgfpathlineto{\pgfqpoint{0.553581in}{1.654444in}}%
\pgfpathlineto{\pgfqpoint{0.553581in}{0.499444in}}%
\pgfpathclose%
\pgfusepath{fill}%
\end{pgfscope}%
\begin{pgfscope}%
\pgfsetbuttcap%
\pgfsetroundjoin%
\definecolor{currentfill}{rgb}{0.000000,0.000000,0.000000}%
\pgfsetfillcolor{currentfill}%
\pgfsetlinewidth{0.803000pt}%
\definecolor{currentstroke}{rgb}{0.000000,0.000000,0.000000}%
\pgfsetstrokecolor{currentstroke}%
\pgfsetdash{}{0pt}%
\pgfsys@defobject{currentmarker}{\pgfqpoint{0.000000in}{-0.048611in}}{\pgfqpoint{0.000000in}{0.000000in}}{%
\pgfpathmoveto{\pgfqpoint{0.000000in}{0.000000in}}%
\pgfpathlineto{\pgfqpoint{0.000000in}{-0.048611in}}%
\pgfusepath{stroke,fill}%
}%
\begin{pgfscope}%
\pgfsys@transformshift{0.624035in}{0.499444in}%
\pgfsys@useobject{currentmarker}{}%
\end{pgfscope}%
\end{pgfscope}%
\begin{pgfscope}%
\definecolor{textcolor}{rgb}{0.000000,0.000000,0.000000}%
\pgfsetstrokecolor{textcolor}%
\pgfsetfillcolor{textcolor}%
\pgftext[x=0.624035in,y=0.402222in,,top]{\color{textcolor}\rmfamily\fontsize{10.000000}{12.000000}\selectfont \(\displaystyle {0.0}\)}%
\end{pgfscope}%
\begin{pgfscope}%
\pgfsetbuttcap%
\pgfsetroundjoin%
\definecolor{currentfill}{rgb}{0.000000,0.000000,0.000000}%
\pgfsetfillcolor{currentfill}%
\pgfsetlinewidth{0.803000pt}%
\definecolor{currentstroke}{rgb}{0.000000,0.000000,0.000000}%
\pgfsetstrokecolor{currentstroke}%
\pgfsetdash{}{0pt}%
\pgfsys@defobject{currentmarker}{\pgfqpoint{0.000000in}{-0.048611in}}{\pgfqpoint{0.000000in}{0.000000in}}{%
\pgfpathmoveto{\pgfqpoint{0.000000in}{0.000000in}}%
\pgfpathlineto{\pgfqpoint{0.000000in}{-0.048611in}}%
\pgfusepath{stroke,fill}%
}%
\begin{pgfscope}%
\pgfsys@transformshift{1.328581in}{0.499444in}%
\pgfsys@useobject{currentmarker}{}%
\end{pgfscope}%
\end{pgfscope}%
\begin{pgfscope}%
\definecolor{textcolor}{rgb}{0.000000,0.000000,0.000000}%
\pgfsetstrokecolor{textcolor}%
\pgfsetfillcolor{textcolor}%
\pgftext[x=1.328581in,y=0.402222in,,top]{\color{textcolor}\rmfamily\fontsize{10.000000}{12.000000}\selectfont \(\displaystyle {0.5}\)}%
\end{pgfscope}%
\begin{pgfscope}%
\pgfsetbuttcap%
\pgfsetroundjoin%
\definecolor{currentfill}{rgb}{0.000000,0.000000,0.000000}%
\pgfsetfillcolor{currentfill}%
\pgfsetlinewidth{0.803000pt}%
\definecolor{currentstroke}{rgb}{0.000000,0.000000,0.000000}%
\pgfsetstrokecolor{currentstroke}%
\pgfsetdash{}{0pt}%
\pgfsys@defobject{currentmarker}{\pgfqpoint{0.000000in}{-0.048611in}}{\pgfqpoint{0.000000in}{0.000000in}}{%
\pgfpathmoveto{\pgfqpoint{0.000000in}{0.000000in}}%
\pgfpathlineto{\pgfqpoint{0.000000in}{-0.048611in}}%
\pgfusepath{stroke,fill}%
}%
\begin{pgfscope}%
\pgfsys@transformshift{2.033126in}{0.499444in}%
\pgfsys@useobject{currentmarker}{}%
\end{pgfscope}%
\end{pgfscope}%
\begin{pgfscope}%
\definecolor{textcolor}{rgb}{0.000000,0.000000,0.000000}%
\pgfsetstrokecolor{textcolor}%
\pgfsetfillcolor{textcolor}%
\pgftext[x=2.033126in,y=0.402222in,,top]{\color{textcolor}\rmfamily\fontsize{10.000000}{12.000000}\selectfont \(\displaystyle {1.0}\)}%
\end{pgfscope}%
\begin{pgfscope}%
\definecolor{textcolor}{rgb}{0.000000,0.000000,0.000000}%
\pgfsetstrokecolor{textcolor}%
\pgfsetfillcolor{textcolor}%
\pgftext[x=1.328581in,y=0.223333in,,top]{\color{textcolor}\rmfamily\fontsize{10.000000}{12.000000}\selectfont False positive rate}%
\end{pgfscope}%
\begin{pgfscope}%
\pgfsetbuttcap%
\pgfsetroundjoin%
\definecolor{currentfill}{rgb}{0.000000,0.000000,0.000000}%
\pgfsetfillcolor{currentfill}%
\pgfsetlinewidth{0.803000pt}%
\definecolor{currentstroke}{rgb}{0.000000,0.000000,0.000000}%
\pgfsetstrokecolor{currentstroke}%
\pgfsetdash{}{0pt}%
\pgfsys@defobject{currentmarker}{\pgfqpoint{-0.048611in}{0.000000in}}{\pgfqpoint{-0.000000in}{0.000000in}}{%
\pgfpathmoveto{\pgfqpoint{-0.000000in}{0.000000in}}%
\pgfpathlineto{\pgfqpoint{-0.048611in}{0.000000in}}%
\pgfusepath{stroke,fill}%
}%
\begin{pgfscope}%
\pgfsys@transformshift{0.553581in}{0.551944in}%
\pgfsys@useobject{currentmarker}{}%
\end{pgfscope}%
\end{pgfscope}%
\begin{pgfscope}%
\definecolor{textcolor}{rgb}{0.000000,0.000000,0.000000}%
\pgfsetstrokecolor{textcolor}%
\pgfsetfillcolor{textcolor}%
\pgftext[x=0.278889in, y=0.503750in, left, base]{\color{textcolor}\rmfamily\fontsize{10.000000}{12.000000}\selectfont \(\displaystyle {0.0}\)}%
\end{pgfscope}%
\begin{pgfscope}%
\pgfsetbuttcap%
\pgfsetroundjoin%
\definecolor{currentfill}{rgb}{0.000000,0.000000,0.000000}%
\pgfsetfillcolor{currentfill}%
\pgfsetlinewidth{0.803000pt}%
\definecolor{currentstroke}{rgb}{0.000000,0.000000,0.000000}%
\pgfsetstrokecolor{currentstroke}%
\pgfsetdash{}{0pt}%
\pgfsys@defobject{currentmarker}{\pgfqpoint{-0.048611in}{0.000000in}}{\pgfqpoint{-0.000000in}{0.000000in}}{%
\pgfpathmoveto{\pgfqpoint{-0.000000in}{0.000000in}}%
\pgfpathlineto{\pgfqpoint{-0.048611in}{0.000000in}}%
\pgfusepath{stroke,fill}%
}%
\begin{pgfscope}%
\pgfsys@transformshift{0.553581in}{1.076944in}%
\pgfsys@useobject{currentmarker}{}%
\end{pgfscope}%
\end{pgfscope}%
\begin{pgfscope}%
\definecolor{textcolor}{rgb}{0.000000,0.000000,0.000000}%
\pgfsetstrokecolor{textcolor}%
\pgfsetfillcolor{textcolor}%
\pgftext[x=0.278889in, y=1.028750in, left, base]{\color{textcolor}\rmfamily\fontsize{10.000000}{12.000000}\selectfont \(\displaystyle {0.5}\)}%
\end{pgfscope}%
\begin{pgfscope}%
\pgfsetbuttcap%
\pgfsetroundjoin%
\definecolor{currentfill}{rgb}{0.000000,0.000000,0.000000}%
\pgfsetfillcolor{currentfill}%
\pgfsetlinewidth{0.803000pt}%
\definecolor{currentstroke}{rgb}{0.000000,0.000000,0.000000}%
\pgfsetstrokecolor{currentstroke}%
\pgfsetdash{}{0pt}%
\pgfsys@defobject{currentmarker}{\pgfqpoint{-0.048611in}{0.000000in}}{\pgfqpoint{-0.000000in}{0.000000in}}{%
\pgfpathmoveto{\pgfqpoint{-0.000000in}{0.000000in}}%
\pgfpathlineto{\pgfqpoint{-0.048611in}{0.000000in}}%
\pgfusepath{stroke,fill}%
}%
\begin{pgfscope}%
\pgfsys@transformshift{0.553581in}{1.601944in}%
\pgfsys@useobject{currentmarker}{}%
\end{pgfscope}%
\end{pgfscope}%
\begin{pgfscope}%
\definecolor{textcolor}{rgb}{0.000000,0.000000,0.000000}%
\pgfsetstrokecolor{textcolor}%
\pgfsetfillcolor{textcolor}%
\pgftext[x=0.278889in, y=1.553750in, left, base]{\color{textcolor}\rmfamily\fontsize{10.000000}{12.000000}\selectfont \(\displaystyle {1.0}\)}%
\end{pgfscope}%
\begin{pgfscope}%
\definecolor{textcolor}{rgb}{0.000000,0.000000,0.000000}%
\pgfsetstrokecolor{textcolor}%
\pgfsetfillcolor{textcolor}%
\pgftext[x=0.223333in,y=1.076944in,,bottom,rotate=90.000000]{\color{textcolor}\rmfamily\fontsize{10.000000}{12.000000}\selectfont True positive rate}%
\end{pgfscope}%
\begin{pgfscope}%
\pgfpathrectangle{\pgfqpoint{0.553581in}{0.499444in}}{\pgfqpoint{1.550000in}{1.155000in}}%
\pgfusepath{clip}%
\pgfsetbuttcap%
\pgfsetroundjoin%
\pgfsetlinewidth{1.505625pt}%
\definecolor{currentstroke}{rgb}{0.000000,0.000000,0.000000}%
\pgfsetstrokecolor{currentstroke}%
\pgfsetdash{{5.550000pt}{2.400000pt}}{0.000000pt}%
\pgfpathmoveto{\pgfqpoint{0.624035in}{0.551944in}}%
\pgfpathlineto{\pgfqpoint{2.033126in}{1.601944in}}%
\pgfusepath{stroke}%
\end{pgfscope}%
\begin{pgfscope}%
\pgfpathrectangle{\pgfqpoint{0.553581in}{0.499444in}}{\pgfqpoint{1.550000in}{1.155000in}}%
\pgfusepath{clip}%
\pgfsetrectcap%
\pgfsetroundjoin%
\pgfsetlinewidth{1.505625pt}%
\definecolor{currentstroke}{rgb}{0.000000,0.000000,0.000000}%
\pgfsetstrokecolor{currentstroke}%
\pgfsetdash{}{0pt}%
\pgfpathmoveto{\pgfqpoint{0.624035in}{0.551944in}}%
\pgfpathlineto{\pgfqpoint{0.625087in}{1.081068in}}%
\pgfpathlineto{\pgfqpoint{0.626766in}{1.222951in}}%
\pgfpathlineto{\pgfqpoint{0.630063in}{1.346143in}}%
\pgfpathlineto{\pgfqpoint{0.630170in}{1.346861in}}%
\pgfpathlineto{\pgfqpoint{0.633316in}{1.414483in}}%
\pgfpathlineto{\pgfqpoint{0.637112in}{1.467192in}}%
\pgfpathlineto{\pgfqpoint{0.642134in}{1.509564in}}%
\pgfpathlineto{\pgfqpoint{0.648493in}{1.541240in}}%
\pgfpathlineto{\pgfqpoint{0.648510in}{1.541293in}}%
\pgfpathlineto{\pgfqpoint{0.655596in}{1.563550in}}%
\pgfpathlineto{\pgfqpoint{0.660310in}{1.572583in}}%
\pgfpathlineto{\pgfqpoint{0.670656in}{1.585341in}}%
\pgfpathlineto{\pgfqpoint{0.682708in}{1.593071in}}%
\pgfpathlineto{\pgfqpoint{0.683120in}{1.593230in}}%
\pgfpathlineto{\pgfqpoint{0.697859in}{1.597421in}}%
\pgfpathlineto{\pgfqpoint{0.715458in}{1.600108in}}%
\pgfpathlineto{\pgfqpoint{0.747388in}{1.601439in}}%
\pgfpathlineto{\pgfqpoint{0.871277in}{1.601931in}}%
\pgfpathlineto{\pgfqpoint{2.033126in}{1.601944in}}%
\pgfpathlineto{\pgfqpoint{2.033126in}{1.601944in}}%
\pgfusepath{stroke}%
\end{pgfscope}%
\begin{pgfscope}%
\pgfsetrectcap%
\pgfsetmiterjoin%
\pgfsetlinewidth{0.803000pt}%
\definecolor{currentstroke}{rgb}{0.000000,0.000000,0.000000}%
\pgfsetstrokecolor{currentstroke}%
\pgfsetdash{}{0pt}%
\pgfpathmoveto{\pgfqpoint{0.553581in}{0.499444in}}%
\pgfpathlineto{\pgfqpoint{0.553581in}{1.654444in}}%
\pgfusepath{stroke}%
\end{pgfscope}%
\begin{pgfscope}%
\pgfsetrectcap%
\pgfsetmiterjoin%
\pgfsetlinewidth{0.803000pt}%
\definecolor{currentstroke}{rgb}{0.000000,0.000000,0.000000}%
\pgfsetstrokecolor{currentstroke}%
\pgfsetdash{}{0pt}%
\pgfpathmoveto{\pgfqpoint{2.103581in}{0.499444in}}%
\pgfpathlineto{\pgfqpoint{2.103581in}{1.654444in}}%
\pgfusepath{stroke}%
\end{pgfscope}%
\begin{pgfscope}%
\pgfsetrectcap%
\pgfsetmiterjoin%
\pgfsetlinewidth{0.803000pt}%
\definecolor{currentstroke}{rgb}{0.000000,0.000000,0.000000}%
\pgfsetstrokecolor{currentstroke}%
\pgfsetdash{}{0pt}%
\pgfpathmoveto{\pgfqpoint{0.553581in}{0.499444in}}%
\pgfpathlineto{\pgfqpoint{2.103581in}{0.499444in}}%
\pgfusepath{stroke}%
\end{pgfscope}%
\begin{pgfscope}%
\pgfsetrectcap%
\pgfsetmiterjoin%
\pgfsetlinewidth{0.803000pt}%
\definecolor{currentstroke}{rgb}{0.000000,0.000000,0.000000}%
\pgfsetstrokecolor{currentstroke}%
\pgfsetdash{}{0pt}%
\pgfpathmoveto{\pgfqpoint{0.553581in}{1.654444in}}%
\pgfpathlineto{\pgfqpoint{2.103581in}{1.654444in}}%
\pgfusepath{stroke}%
\end{pgfscope}%
\begin{pgfscope}%
\pgfsetbuttcap%
\pgfsetmiterjoin%
\definecolor{currentfill}{rgb}{1.000000,1.000000,1.000000}%
\pgfsetfillcolor{currentfill}%
\pgfsetfillopacity{0.800000}%
\pgfsetlinewidth{1.003750pt}%
\definecolor{currentstroke}{rgb}{0.800000,0.800000,0.800000}%
\pgfsetstrokecolor{currentstroke}%
\pgfsetstrokeopacity{0.800000}%
\pgfsetdash{}{0pt}%
\pgfpathmoveto{\pgfqpoint{0.832747in}{1.349722in}}%
\pgfpathlineto{\pgfqpoint{2.006358in}{1.349722in}}%
\pgfpathquadraticcurveto{\pgfqpoint{2.034136in}{1.349722in}}{\pgfqpoint{2.034136in}{1.377500in}}%
\pgfpathlineto{\pgfqpoint{2.034136in}{1.557222in}}%
\pgfpathquadraticcurveto{\pgfqpoint{2.034136in}{1.585000in}}{\pgfqpoint{2.006358in}{1.585000in}}%
\pgfpathlineto{\pgfqpoint{0.832747in}{1.585000in}}%
\pgfpathquadraticcurveto{\pgfqpoint{0.804970in}{1.585000in}}{\pgfqpoint{0.804970in}{1.557222in}}%
\pgfpathlineto{\pgfqpoint{0.804970in}{1.377500in}}%
\pgfpathquadraticcurveto{\pgfqpoint{0.804970in}{1.349722in}}{\pgfqpoint{0.832747in}{1.349722in}}%
\pgfpathlineto{\pgfqpoint{0.832747in}{1.349722in}}%
\pgfpathclose%
\pgfusepath{stroke,fill}%
\end{pgfscope}%
\begin{pgfscope}%
\pgfsetrectcap%
\pgfsetroundjoin%
\pgfsetlinewidth{1.505625pt}%
\definecolor{currentstroke}{rgb}{0.000000,0.000000,0.000000}%
\pgfsetstrokecolor{currentstroke}%
\pgfsetdash{}{0pt}%
\pgfpathmoveto{\pgfqpoint{0.860525in}{1.480833in}}%
\pgfpathlineto{\pgfqpoint{0.999414in}{1.480833in}}%
\pgfpathlineto{\pgfqpoint{1.138303in}{1.480833in}}%
\pgfusepath{stroke}%
\end{pgfscope}%
\begin{pgfscope}%
\definecolor{textcolor}{rgb}{0.000000,0.000000,0.000000}%
\pgfsetstrokecolor{textcolor}%
\pgfsetfillcolor{textcolor}%
\pgftext[x=1.249414in,y=1.432222in,left,base]{\color{textcolor}\rmfamily\fontsize{10.000000}{12.000000}\selectfont AUC=0.996}%
\end{pgfscope}%
\end{pgfpicture}%
\makeatother%
\endgroup%

	
\end{tabular}

Unfortunately, our test results do not look quite that nice.  They do not separate the two classes as well.  Some distributions are clustered to one side or in the middle.  Some models give the results in $p \in [0,1]$ rounded to two decimal places so that we cannot hope for a level of detail beyond that, and one algorithm, Bagging, gives $p$ rounded to only one decimal place.  

Let us look at some examples.  In all of them, AUC is in the range $[0.7,0.8]$, so the various models separate the positive and negative classes about equally well overall, with none being dramatically better or worse.  We will later show how we investigated which models do a better job in the ranges of interest.  

\

%
\verb|BRFC_Hard_Tomek_0_alpha_0_5_v1_Test|

\

This model does not separate the negative and positive classes as well as the ideal, giving a much lower AUC (area under the ROC curve).  These results are actually from the same model as the ideal above, but the ideal are the results on the training set and below on the test set, showing overfitting.  

In these results, the 100 most frequent values comprised 93\% of the results, meaning that, while there is some noise making the distribution look continuous, it is mostly discrete to two decimal places, so we cannot hope for fine detail in tuning the decision threshold.  

\noindent\begin{tabular}{@{\hspace{-6pt}}p{4.3in} @{\hspace{-6pt}}p{2.0in}}
	\vskip 0pt
	\hfil Raw Model Output
	
	%% Creator: Matplotlib, PGF backend
%%
%% To include the figure in your LaTeX document, write
%%   \input{<filename>.pgf}
%%
%% Make sure the required packages are loaded in your preamble
%%   \usepackage{pgf}
%%
%% Also ensure that all the required font packages are loaded; for instance,
%% the lmodern package is sometimes necessary when using math font.
%%   \usepackage{lmodern}
%%
%% Figures using additional raster images can only be included by \input if
%% they are in the same directory as the main LaTeX file. For loading figures
%% from other directories you can use the `import` package
%%   \usepackage{import}
%%
%% and then include the figures with
%%   \import{<path to file>}{<filename>.pgf}
%%
%% Matplotlib used the following preamble
%%   
%%   \usepackage{fontspec}
%%   \makeatletter\@ifpackageloaded{underscore}{}{\usepackage[strings]{underscore}}\makeatother
%%
\begingroup%
\makeatletter%
\begin{pgfpicture}%
\pgfpathrectangle{\pgfpointorigin}{\pgfqpoint{4.509306in}{1.754444in}}%
\pgfusepath{use as bounding box, clip}%
\begin{pgfscope}%
\pgfsetbuttcap%
\pgfsetmiterjoin%
\definecolor{currentfill}{rgb}{1.000000,1.000000,1.000000}%
\pgfsetfillcolor{currentfill}%
\pgfsetlinewidth{0.000000pt}%
\definecolor{currentstroke}{rgb}{1.000000,1.000000,1.000000}%
\pgfsetstrokecolor{currentstroke}%
\pgfsetdash{}{0pt}%
\pgfpathmoveto{\pgfqpoint{0.000000in}{0.000000in}}%
\pgfpathlineto{\pgfqpoint{4.509306in}{0.000000in}}%
\pgfpathlineto{\pgfqpoint{4.509306in}{1.754444in}}%
\pgfpathlineto{\pgfqpoint{0.000000in}{1.754444in}}%
\pgfpathlineto{\pgfqpoint{0.000000in}{0.000000in}}%
\pgfpathclose%
\pgfusepath{fill}%
\end{pgfscope}%
\begin{pgfscope}%
\pgfsetbuttcap%
\pgfsetmiterjoin%
\definecolor{currentfill}{rgb}{1.000000,1.000000,1.000000}%
\pgfsetfillcolor{currentfill}%
\pgfsetlinewidth{0.000000pt}%
\definecolor{currentstroke}{rgb}{0.000000,0.000000,0.000000}%
\pgfsetstrokecolor{currentstroke}%
\pgfsetstrokeopacity{0.000000}%
\pgfsetdash{}{0pt}%
\pgfpathmoveto{\pgfqpoint{0.445556in}{0.499444in}}%
\pgfpathlineto{\pgfqpoint{4.320556in}{0.499444in}}%
\pgfpathlineto{\pgfqpoint{4.320556in}{1.654444in}}%
\pgfpathlineto{\pgfqpoint{0.445556in}{1.654444in}}%
\pgfpathlineto{\pgfqpoint{0.445556in}{0.499444in}}%
\pgfpathclose%
\pgfusepath{fill}%
\end{pgfscope}%
\begin{pgfscope}%
\pgfpathrectangle{\pgfqpoint{0.445556in}{0.499444in}}{\pgfqpoint{3.875000in}{1.155000in}}%
\pgfusepath{clip}%
\pgfsetbuttcap%
\pgfsetmiterjoin%
\pgfsetlinewidth{1.003750pt}%
\definecolor{currentstroke}{rgb}{0.000000,0.000000,0.000000}%
\pgfsetstrokecolor{currentstroke}%
\pgfsetdash{}{0pt}%
\pgfpathmoveto{\pgfqpoint{0.435556in}{0.499444in}}%
\pgfpathlineto{\pgfqpoint{0.483922in}{0.499444in}}%
\pgfpathlineto{\pgfqpoint{0.483922in}{0.618288in}}%
\pgfpathlineto{\pgfqpoint{0.435556in}{0.618288in}}%
\pgfusepath{stroke}%
\end{pgfscope}%
\begin{pgfscope}%
\pgfpathrectangle{\pgfqpoint{0.445556in}{0.499444in}}{\pgfqpoint{3.875000in}{1.155000in}}%
\pgfusepath{clip}%
\pgfsetbuttcap%
\pgfsetmiterjoin%
\pgfsetlinewidth{1.003750pt}%
\definecolor{currentstroke}{rgb}{0.000000,0.000000,0.000000}%
\pgfsetstrokecolor{currentstroke}%
\pgfsetdash{}{0pt}%
\pgfpathmoveto{\pgfqpoint{0.576001in}{0.499444in}}%
\pgfpathlineto{\pgfqpoint{0.637387in}{0.499444in}}%
\pgfpathlineto{\pgfqpoint{0.637387in}{0.801196in}}%
\pgfpathlineto{\pgfqpoint{0.576001in}{0.801196in}}%
\pgfpathlineto{\pgfqpoint{0.576001in}{0.499444in}}%
\pgfpathclose%
\pgfusepath{stroke}%
\end{pgfscope}%
\begin{pgfscope}%
\pgfpathrectangle{\pgfqpoint{0.445556in}{0.499444in}}{\pgfqpoint{3.875000in}{1.155000in}}%
\pgfusepath{clip}%
\pgfsetbuttcap%
\pgfsetmiterjoin%
\pgfsetlinewidth{1.003750pt}%
\definecolor{currentstroke}{rgb}{0.000000,0.000000,0.000000}%
\pgfsetstrokecolor{currentstroke}%
\pgfsetdash{}{0pt}%
\pgfpathmoveto{\pgfqpoint{0.729467in}{0.499444in}}%
\pgfpathlineto{\pgfqpoint{0.790853in}{0.499444in}}%
\pgfpathlineto{\pgfqpoint{0.790853in}{1.026194in}}%
\pgfpathlineto{\pgfqpoint{0.729467in}{1.026194in}}%
\pgfpathlineto{\pgfqpoint{0.729467in}{0.499444in}}%
\pgfpathclose%
\pgfusepath{stroke}%
\end{pgfscope}%
\begin{pgfscope}%
\pgfpathrectangle{\pgfqpoint{0.445556in}{0.499444in}}{\pgfqpoint{3.875000in}{1.155000in}}%
\pgfusepath{clip}%
\pgfsetbuttcap%
\pgfsetmiterjoin%
\pgfsetlinewidth{1.003750pt}%
\definecolor{currentstroke}{rgb}{0.000000,0.000000,0.000000}%
\pgfsetstrokecolor{currentstroke}%
\pgfsetdash{}{0pt}%
\pgfpathmoveto{\pgfqpoint{0.882932in}{0.499444in}}%
\pgfpathlineto{\pgfqpoint{0.944318in}{0.499444in}}%
\pgfpathlineto{\pgfqpoint{0.944318in}{1.203686in}}%
\pgfpathlineto{\pgfqpoint{0.882932in}{1.203686in}}%
\pgfpathlineto{\pgfqpoint{0.882932in}{0.499444in}}%
\pgfpathclose%
\pgfusepath{stroke}%
\end{pgfscope}%
\begin{pgfscope}%
\pgfpathrectangle{\pgfqpoint{0.445556in}{0.499444in}}{\pgfqpoint{3.875000in}{1.155000in}}%
\pgfusepath{clip}%
\pgfsetbuttcap%
\pgfsetmiterjoin%
\pgfsetlinewidth{1.003750pt}%
\definecolor{currentstroke}{rgb}{0.000000,0.000000,0.000000}%
\pgfsetstrokecolor{currentstroke}%
\pgfsetdash{}{0pt}%
\pgfpathmoveto{\pgfqpoint{1.036397in}{0.499444in}}%
\pgfpathlineto{\pgfqpoint{1.097783in}{0.499444in}}%
\pgfpathlineto{\pgfqpoint{1.097783in}{1.348758in}}%
\pgfpathlineto{\pgfqpoint{1.036397in}{1.348758in}}%
\pgfpathlineto{\pgfqpoint{1.036397in}{0.499444in}}%
\pgfpathclose%
\pgfusepath{stroke}%
\end{pgfscope}%
\begin{pgfscope}%
\pgfpathrectangle{\pgfqpoint{0.445556in}{0.499444in}}{\pgfqpoint{3.875000in}{1.155000in}}%
\pgfusepath{clip}%
\pgfsetbuttcap%
\pgfsetmiterjoin%
\pgfsetlinewidth{1.003750pt}%
\definecolor{currentstroke}{rgb}{0.000000,0.000000,0.000000}%
\pgfsetstrokecolor{currentstroke}%
\pgfsetdash{}{0pt}%
\pgfpathmoveto{\pgfqpoint{1.189863in}{0.499444in}}%
\pgfpathlineto{\pgfqpoint{1.251249in}{0.499444in}}%
\pgfpathlineto{\pgfqpoint{1.251249in}{1.455609in}}%
\pgfpathlineto{\pgfqpoint{1.189863in}{1.455609in}}%
\pgfpathlineto{\pgfqpoint{1.189863in}{0.499444in}}%
\pgfpathclose%
\pgfusepath{stroke}%
\end{pgfscope}%
\begin{pgfscope}%
\pgfpathrectangle{\pgfqpoint{0.445556in}{0.499444in}}{\pgfqpoint{3.875000in}{1.155000in}}%
\pgfusepath{clip}%
\pgfsetbuttcap%
\pgfsetmiterjoin%
\pgfsetlinewidth{1.003750pt}%
\definecolor{currentstroke}{rgb}{0.000000,0.000000,0.000000}%
\pgfsetstrokecolor{currentstroke}%
\pgfsetdash{}{0pt}%
\pgfpathmoveto{\pgfqpoint{1.343328in}{0.499444in}}%
\pgfpathlineto{\pgfqpoint{1.404714in}{0.499444in}}%
\pgfpathlineto{\pgfqpoint{1.404714in}{1.549075in}}%
\pgfpathlineto{\pgfqpoint{1.343328in}{1.549075in}}%
\pgfpathlineto{\pgfqpoint{1.343328in}{0.499444in}}%
\pgfpathclose%
\pgfusepath{stroke}%
\end{pgfscope}%
\begin{pgfscope}%
\pgfpathrectangle{\pgfqpoint{0.445556in}{0.499444in}}{\pgfqpoint{3.875000in}{1.155000in}}%
\pgfusepath{clip}%
\pgfsetbuttcap%
\pgfsetmiterjoin%
\pgfsetlinewidth{1.003750pt}%
\definecolor{currentstroke}{rgb}{0.000000,0.000000,0.000000}%
\pgfsetstrokecolor{currentstroke}%
\pgfsetdash{}{0pt}%
\pgfpathmoveto{\pgfqpoint{1.496793in}{0.499444in}}%
\pgfpathlineto{\pgfqpoint{1.558179in}{0.499444in}}%
\pgfpathlineto{\pgfqpoint{1.558179in}{1.599444in}}%
\pgfpathlineto{\pgfqpoint{1.496793in}{1.599444in}}%
\pgfpathlineto{\pgfqpoint{1.496793in}{0.499444in}}%
\pgfpathclose%
\pgfusepath{stroke}%
\end{pgfscope}%
\begin{pgfscope}%
\pgfpathrectangle{\pgfqpoint{0.445556in}{0.499444in}}{\pgfqpoint{3.875000in}{1.155000in}}%
\pgfusepath{clip}%
\pgfsetbuttcap%
\pgfsetmiterjoin%
\pgfsetlinewidth{1.003750pt}%
\definecolor{currentstroke}{rgb}{0.000000,0.000000,0.000000}%
\pgfsetstrokecolor{currentstroke}%
\pgfsetdash{}{0pt}%
\pgfpathmoveto{\pgfqpoint{1.650259in}{0.499444in}}%
\pgfpathlineto{\pgfqpoint{1.711645in}{0.499444in}}%
\pgfpathlineto{\pgfqpoint{1.711645in}{1.597742in}}%
\pgfpathlineto{\pgfqpoint{1.650259in}{1.597742in}}%
\pgfpathlineto{\pgfqpoint{1.650259in}{0.499444in}}%
\pgfpathclose%
\pgfusepath{stroke}%
\end{pgfscope}%
\begin{pgfscope}%
\pgfpathrectangle{\pgfqpoint{0.445556in}{0.499444in}}{\pgfqpoint{3.875000in}{1.155000in}}%
\pgfusepath{clip}%
\pgfsetbuttcap%
\pgfsetmiterjoin%
\pgfsetlinewidth{1.003750pt}%
\definecolor{currentstroke}{rgb}{0.000000,0.000000,0.000000}%
\pgfsetstrokecolor{currentstroke}%
\pgfsetdash{}{0pt}%
\pgfpathmoveto{\pgfqpoint{1.803724in}{0.499444in}}%
\pgfpathlineto{\pgfqpoint{1.865110in}{0.499444in}}%
\pgfpathlineto{\pgfqpoint{1.865110in}{1.571977in}}%
\pgfpathlineto{\pgfqpoint{1.803724in}{1.571977in}}%
\pgfpathlineto{\pgfqpoint{1.803724in}{0.499444in}}%
\pgfpathclose%
\pgfusepath{stroke}%
\end{pgfscope}%
\begin{pgfscope}%
\pgfpathrectangle{\pgfqpoint{0.445556in}{0.499444in}}{\pgfqpoint{3.875000in}{1.155000in}}%
\pgfusepath{clip}%
\pgfsetbuttcap%
\pgfsetmiterjoin%
\pgfsetlinewidth{1.003750pt}%
\definecolor{currentstroke}{rgb}{0.000000,0.000000,0.000000}%
\pgfsetstrokecolor{currentstroke}%
\pgfsetdash{}{0pt}%
\pgfpathmoveto{\pgfqpoint{1.957189in}{0.499444in}}%
\pgfpathlineto{\pgfqpoint{2.018575in}{0.499444in}}%
\pgfpathlineto{\pgfqpoint{2.018575in}{1.535535in}}%
\pgfpathlineto{\pgfqpoint{1.957189in}{1.535535in}}%
\pgfpathlineto{\pgfqpoint{1.957189in}{0.499444in}}%
\pgfpathclose%
\pgfusepath{stroke}%
\end{pgfscope}%
\begin{pgfscope}%
\pgfpathrectangle{\pgfqpoint{0.445556in}{0.499444in}}{\pgfqpoint{3.875000in}{1.155000in}}%
\pgfusepath{clip}%
\pgfsetbuttcap%
\pgfsetmiterjoin%
\pgfsetlinewidth{1.003750pt}%
\definecolor{currentstroke}{rgb}{0.000000,0.000000,0.000000}%
\pgfsetstrokecolor{currentstroke}%
\pgfsetdash{}{0pt}%
\pgfpathmoveto{\pgfqpoint{2.110655in}{0.499444in}}%
\pgfpathlineto{\pgfqpoint{2.172041in}{0.499444in}}%
\pgfpathlineto{\pgfqpoint{2.172041in}{1.450812in}}%
\pgfpathlineto{\pgfqpoint{2.110655in}{1.450812in}}%
\pgfpathlineto{\pgfqpoint{2.110655in}{0.499444in}}%
\pgfpathclose%
\pgfusepath{stroke}%
\end{pgfscope}%
\begin{pgfscope}%
\pgfpathrectangle{\pgfqpoint{0.445556in}{0.499444in}}{\pgfqpoint{3.875000in}{1.155000in}}%
\pgfusepath{clip}%
\pgfsetbuttcap%
\pgfsetmiterjoin%
\pgfsetlinewidth{1.003750pt}%
\definecolor{currentstroke}{rgb}{0.000000,0.000000,0.000000}%
\pgfsetstrokecolor{currentstroke}%
\pgfsetdash{}{0pt}%
\pgfpathmoveto{\pgfqpoint{2.264120in}{0.499444in}}%
\pgfpathlineto{\pgfqpoint{2.325506in}{0.499444in}}%
\pgfpathlineto{\pgfqpoint{2.325506in}{1.379398in}}%
\pgfpathlineto{\pgfqpoint{2.264120in}{1.379398in}}%
\pgfpathlineto{\pgfqpoint{2.264120in}{0.499444in}}%
\pgfpathclose%
\pgfusepath{stroke}%
\end{pgfscope}%
\begin{pgfscope}%
\pgfpathrectangle{\pgfqpoint{0.445556in}{0.499444in}}{\pgfqpoint{3.875000in}{1.155000in}}%
\pgfusepath{clip}%
\pgfsetbuttcap%
\pgfsetmiterjoin%
\pgfsetlinewidth{1.003750pt}%
\definecolor{currentstroke}{rgb}{0.000000,0.000000,0.000000}%
\pgfsetstrokecolor{currentstroke}%
\pgfsetdash{}{0pt}%
\pgfpathmoveto{\pgfqpoint{2.417585in}{0.499444in}}%
\pgfpathlineto{\pgfqpoint{2.478972in}{0.499444in}}%
\pgfpathlineto{\pgfqpoint{2.478972in}{1.248329in}}%
\pgfpathlineto{\pgfqpoint{2.417585in}{1.248329in}}%
\pgfpathlineto{\pgfqpoint{2.417585in}{0.499444in}}%
\pgfpathclose%
\pgfusepath{stroke}%
\end{pgfscope}%
\begin{pgfscope}%
\pgfpathrectangle{\pgfqpoint{0.445556in}{0.499444in}}{\pgfqpoint{3.875000in}{1.155000in}}%
\pgfusepath{clip}%
\pgfsetbuttcap%
\pgfsetmiterjoin%
\pgfsetlinewidth{1.003750pt}%
\definecolor{currentstroke}{rgb}{0.000000,0.000000,0.000000}%
\pgfsetstrokecolor{currentstroke}%
\pgfsetdash{}{0pt}%
\pgfpathmoveto{\pgfqpoint{2.571051in}{0.499444in}}%
\pgfpathlineto{\pgfqpoint{2.632437in}{0.499444in}}%
\pgfpathlineto{\pgfqpoint{2.632437in}{1.129640in}}%
\pgfpathlineto{\pgfqpoint{2.571051in}{1.129640in}}%
\pgfpathlineto{\pgfqpoint{2.571051in}{0.499444in}}%
\pgfpathclose%
\pgfusepath{stroke}%
\end{pgfscope}%
\begin{pgfscope}%
\pgfpathrectangle{\pgfqpoint{0.445556in}{0.499444in}}{\pgfqpoint{3.875000in}{1.155000in}}%
\pgfusepath{clip}%
\pgfsetbuttcap%
\pgfsetmiterjoin%
\pgfsetlinewidth{1.003750pt}%
\definecolor{currentstroke}{rgb}{0.000000,0.000000,0.000000}%
\pgfsetstrokecolor{currentstroke}%
\pgfsetdash{}{0pt}%
\pgfpathmoveto{\pgfqpoint{2.724516in}{0.499444in}}%
\pgfpathlineto{\pgfqpoint{2.785902in}{0.499444in}}%
\pgfpathlineto{\pgfqpoint{2.785902in}{1.011880in}}%
\pgfpathlineto{\pgfqpoint{2.724516in}{1.011880in}}%
\pgfpathlineto{\pgfqpoint{2.724516in}{0.499444in}}%
\pgfpathclose%
\pgfusepath{stroke}%
\end{pgfscope}%
\begin{pgfscope}%
\pgfpathrectangle{\pgfqpoint{0.445556in}{0.499444in}}{\pgfqpoint{3.875000in}{1.155000in}}%
\pgfusepath{clip}%
\pgfsetbuttcap%
\pgfsetmiterjoin%
\pgfsetlinewidth{1.003750pt}%
\definecolor{currentstroke}{rgb}{0.000000,0.000000,0.000000}%
\pgfsetstrokecolor{currentstroke}%
\pgfsetdash{}{0pt}%
\pgfpathmoveto{\pgfqpoint{2.877981in}{0.499444in}}%
\pgfpathlineto{\pgfqpoint{2.939368in}{0.499444in}}%
\pgfpathlineto{\pgfqpoint{2.939368in}{0.912921in}}%
\pgfpathlineto{\pgfqpoint{2.877981in}{0.912921in}}%
\pgfpathlineto{\pgfqpoint{2.877981in}{0.499444in}}%
\pgfpathclose%
\pgfusepath{stroke}%
\end{pgfscope}%
\begin{pgfscope}%
\pgfpathrectangle{\pgfqpoint{0.445556in}{0.499444in}}{\pgfqpoint{3.875000in}{1.155000in}}%
\pgfusepath{clip}%
\pgfsetbuttcap%
\pgfsetmiterjoin%
\pgfsetlinewidth{1.003750pt}%
\definecolor{currentstroke}{rgb}{0.000000,0.000000,0.000000}%
\pgfsetstrokecolor{currentstroke}%
\pgfsetdash{}{0pt}%
\pgfpathmoveto{\pgfqpoint{3.031447in}{0.499444in}}%
\pgfpathlineto{\pgfqpoint{3.092833in}{0.499444in}}%
\pgfpathlineto{\pgfqpoint{3.092833in}{0.805374in}}%
\pgfpathlineto{\pgfqpoint{3.031447in}{0.805374in}}%
\pgfpathlineto{\pgfqpoint{3.031447in}{0.499444in}}%
\pgfpathclose%
\pgfusepath{stroke}%
\end{pgfscope}%
\begin{pgfscope}%
\pgfpathrectangle{\pgfqpoint{0.445556in}{0.499444in}}{\pgfqpoint{3.875000in}{1.155000in}}%
\pgfusepath{clip}%
\pgfsetbuttcap%
\pgfsetmiterjoin%
\pgfsetlinewidth{1.003750pt}%
\definecolor{currentstroke}{rgb}{0.000000,0.000000,0.000000}%
\pgfsetstrokecolor{currentstroke}%
\pgfsetdash{}{0pt}%
\pgfpathmoveto{\pgfqpoint{3.184912in}{0.499444in}}%
\pgfpathlineto{\pgfqpoint{3.246298in}{0.499444in}}%
\pgfpathlineto{\pgfqpoint{3.246298in}{0.730710in}}%
\pgfpathlineto{\pgfqpoint{3.184912in}{0.730710in}}%
\pgfpathlineto{\pgfqpoint{3.184912in}{0.499444in}}%
\pgfpathclose%
\pgfusepath{stroke}%
\end{pgfscope}%
\begin{pgfscope}%
\pgfpathrectangle{\pgfqpoint{0.445556in}{0.499444in}}{\pgfqpoint{3.875000in}{1.155000in}}%
\pgfusepath{clip}%
\pgfsetbuttcap%
\pgfsetmiterjoin%
\pgfsetlinewidth{1.003750pt}%
\definecolor{currentstroke}{rgb}{0.000000,0.000000,0.000000}%
\pgfsetstrokecolor{currentstroke}%
\pgfsetdash{}{0pt}%
\pgfpathmoveto{\pgfqpoint{3.338377in}{0.499444in}}%
\pgfpathlineto{\pgfqpoint{3.399764in}{0.499444in}}%
\pgfpathlineto{\pgfqpoint{3.399764in}{0.672603in}}%
\pgfpathlineto{\pgfqpoint{3.338377in}{0.672603in}}%
\pgfpathlineto{\pgfqpoint{3.338377in}{0.499444in}}%
\pgfpathclose%
\pgfusepath{stroke}%
\end{pgfscope}%
\begin{pgfscope}%
\pgfpathrectangle{\pgfqpoint{0.445556in}{0.499444in}}{\pgfqpoint{3.875000in}{1.155000in}}%
\pgfusepath{clip}%
\pgfsetbuttcap%
\pgfsetmiterjoin%
\pgfsetlinewidth{1.003750pt}%
\definecolor{currentstroke}{rgb}{0.000000,0.000000,0.000000}%
\pgfsetstrokecolor{currentstroke}%
\pgfsetdash{}{0pt}%
\pgfpathmoveto{\pgfqpoint{3.491843in}{0.499444in}}%
\pgfpathlineto{\pgfqpoint{3.553229in}{0.499444in}}%
\pgfpathlineto{\pgfqpoint{3.553229in}{0.616663in}}%
\pgfpathlineto{\pgfqpoint{3.491843in}{0.616663in}}%
\pgfpathlineto{\pgfqpoint{3.491843in}{0.499444in}}%
\pgfpathclose%
\pgfusepath{stroke}%
\end{pgfscope}%
\begin{pgfscope}%
\pgfpathrectangle{\pgfqpoint{0.445556in}{0.499444in}}{\pgfqpoint{3.875000in}{1.155000in}}%
\pgfusepath{clip}%
\pgfsetbuttcap%
\pgfsetmiterjoin%
\pgfsetlinewidth{1.003750pt}%
\definecolor{currentstroke}{rgb}{0.000000,0.000000,0.000000}%
\pgfsetstrokecolor{currentstroke}%
\pgfsetdash{}{0pt}%
\pgfpathmoveto{\pgfqpoint{3.645308in}{0.499444in}}%
\pgfpathlineto{\pgfqpoint{3.706694in}{0.499444in}}%
\pgfpathlineto{\pgfqpoint{3.706694in}{0.574031in}}%
\pgfpathlineto{\pgfqpoint{3.645308in}{0.574031in}}%
\pgfpathlineto{\pgfqpoint{3.645308in}{0.499444in}}%
\pgfpathclose%
\pgfusepath{stroke}%
\end{pgfscope}%
\begin{pgfscope}%
\pgfpathrectangle{\pgfqpoint{0.445556in}{0.499444in}}{\pgfqpoint{3.875000in}{1.155000in}}%
\pgfusepath{clip}%
\pgfsetbuttcap%
\pgfsetmiterjoin%
\pgfsetlinewidth{1.003750pt}%
\definecolor{currentstroke}{rgb}{0.000000,0.000000,0.000000}%
\pgfsetstrokecolor{currentstroke}%
\pgfsetdash{}{0pt}%
\pgfpathmoveto{\pgfqpoint{3.798774in}{0.499444in}}%
\pgfpathlineto{\pgfqpoint{3.860160in}{0.499444in}}%
\pgfpathlineto{\pgfqpoint{3.860160in}{0.550587in}}%
\pgfpathlineto{\pgfqpoint{3.798774in}{0.550587in}}%
\pgfpathlineto{\pgfqpoint{3.798774in}{0.499444in}}%
\pgfpathclose%
\pgfusepath{stroke}%
\end{pgfscope}%
\begin{pgfscope}%
\pgfpathrectangle{\pgfqpoint{0.445556in}{0.499444in}}{\pgfqpoint{3.875000in}{1.155000in}}%
\pgfusepath{clip}%
\pgfsetbuttcap%
\pgfsetmiterjoin%
\pgfsetlinewidth{1.003750pt}%
\definecolor{currentstroke}{rgb}{0.000000,0.000000,0.000000}%
\pgfsetstrokecolor{currentstroke}%
\pgfsetdash{}{0pt}%
\pgfpathmoveto{\pgfqpoint{3.952239in}{0.499444in}}%
\pgfpathlineto{\pgfqpoint{4.013625in}{0.499444in}}%
\pgfpathlineto{\pgfqpoint{4.013625in}{0.526679in}}%
\pgfpathlineto{\pgfqpoint{3.952239in}{0.526679in}}%
\pgfpathlineto{\pgfqpoint{3.952239in}{0.499444in}}%
\pgfpathclose%
\pgfusepath{stroke}%
\end{pgfscope}%
\begin{pgfscope}%
\pgfpathrectangle{\pgfqpoint{0.445556in}{0.499444in}}{\pgfqpoint{3.875000in}{1.155000in}}%
\pgfusepath{clip}%
\pgfsetbuttcap%
\pgfsetmiterjoin%
\pgfsetlinewidth{1.003750pt}%
\definecolor{currentstroke}{rgb}{0.000000,0.000000,0.000000}%
\pgfsetstrokecolor{currentstroke}%
\pgfsetdash{}{0pt}%
\pgfpathmoveto{\pgfqpoint{4.105704in}{0.499444in}}%
\pgfpathlineto{\pgfqpoint{4.167090in}{0.499444in}}%
\pgfpathlineto{\pgfqpoint{4.167090in}{0.506640in}}%
\pgfpathlineto{\pgfqpoint{4.105704in}{0.506640in}}%
\pgfpathlineto{\pgfqpoint{4.105704in}{0.499444in}}%
\pgfpathclose%
\pgfusepath{stroke}%
\end{pgfscope}%
\begin{pgfscope}%
\pgfpathrectangle{\pgfqpoint{0.445556in}{0.499444in}}{\pgfqpoint{3.875000in}{1.155000in}}%
\pgfusepath{clip}%
\pgfsetbuttcap%
\pgfsetmiterjoin%
\definecolor{currentfill}{rgb}{0.000000,0.000000,0.000000}%
\pgfsetfillcolor{currentfill}%
\pgfsetlinewidth{0.000000pt}%
\definecolor{currentstroke}{rgb}{0.000000,0.000000,0.000000}%
\pgfsetstrokecolor{currentstroke}%
\pgfsetstrokeopacity{0.000000}%
\pgfsetdash{}{0pt}%
\pgfpathmoveto{\pgfqpoint{0.483922in}{0.499444in}}%
\pgfpathlineto{\pgfqpoint{0.545308in}{0.499444in}}%
\pgfpathlineto{\pgfqpoint{0.545308in}{0.500527in}}%
\pgfpathlineto{\pgfqpoint{0.483922in}{0.500527in}}%
\pgfpathlineto{\pgfqpoint{0.483922in}{0.499444in}}%
\pgfpathclose%
\pgfusepath{fill}%
\end{pgfscope}%
\begin{pgfscope}%
\pgfpathrectangle{\pgfqpoint{0.445556in}{0.499444in}}{\pgfqpoint{3.875000in}{1.155000in}}%
\pgfusepath{clip}%
\pgfsetbuttcap%
\pgfsetmiterjoin%
\definecolor{currentfill}{rgb}{0.000000,0.000000,0.000000}%
\pgfsetfillcolor{currentfill}%
\pgfsetlinewidth{0.000000pt}%
\definecolor{currentstroke}{rgb}{0.000000,0.000000,0.000000}%
\pgfsetstrokecolor{currentstroke}%
\pgfsetstrokeopacity{0.000000}%
\pgfsetdash{}{0pt}%
\pgfpathmoveto{\pgfqpoint{0.637387in}{0.499444in}}%
\pgfpathlineto{\pgfqpoint{0.698774in}{0.499444in}}%
\pgfpathlineto{\pgfqpoint{0.698774in}{0.502230in}}%
\pgfpathlineto{\pgfqpoint{0.637387in}{0.502230in}}%
\pgfpathlineto{\pgfqpoint{0.637387in}{0.499444in}}%
\pgfpathclose%
\pgfusepath{fill}%
\end{pgfscope}%
\begin{pgfscope}%
\pgfpathrectangle{\pgfqpoint{0.445556in}{0.499444in}}{\pgfqpoint{3.875000in}{1.155000in}}%
\pgfusepath{clip}%
\pgfsetbuttcap%
\pgfsetmiterjoin%
\definecolor{currentfill}{rgb}{0.000000,0.000000,0.000000}%
\pgfsetfillcolor{currentfill}%
\pgfsetlinewidth{0.000000pt}%
\definecolor{currentstroke}{rgb}{0.000000,0.000000,0.000000}%
\pgfsetstrokecolor{currentstroke}%
\pgfsetstrokeopacity{0.000000}%
\pgfsetdash{}{0pt}%
\pgfpathmoveto{\pgfqpoint{0.790853in}{0.499444in}}%
\pgfpathlineto{\pgfqpoint{0.852239in}{0.499444in}}%
\pgfpathlineto{\pgfqpoint{0.852239in}{0.506330in}}%
\pgfpathlineto{\pgfqpoint{0.790853in}{0.506330in}}%
\pgfpathlineto{\pgfqpoint{0.790853in}{0.499444in}}%
\pgfpathclose%
\pgfusepath{fill}%
\end{pgfscope}%
\begin{pgfscope}%
\pgfpathrectangle{\pgfqpoint{0.445556in}{0.499444in}}{\pgfqpoint{3.875000in}{1.155000in}}%
\pgfusepath{clip}%
\pgfsetbuttcap%
\pgfsetmiterjoin%
\definecolor{currentfill}{rgb}{0.000000,0.000000,0.000000}%
\pgfsetfillcolor{currentfill}%
\pgfsetlinewidth{0.000000pt}%
\definecolor{currentstroke}{rgb}{0.000000,0.000000,0.000000}%
\pgfsetstrokecolor{currentstroke}%
\pgfsetstrokeopacity{0.000000}%
\pgfsetdash{}{0pt}%
\pgfpathmoveto{\pgfqpoint{0.944318in}{0.499444in}}%
\pgfpathlineto{\pgfqpoint{1.005704in}{0.499444in}}%
\pgfpathlineto{\pgfqpoint{1.005704in}{0.514068in}}%
\pgfpathlineto{\pgfqpoint{0.944318in}{0.514068in}}%
\pgfpathlineto{\pgfqpoint{0.944318in}{0.499444in}}%
\pgfpathclose%
\pgfusepath{fill}%
\end{pgfscope}%
\begin{pgfscope}%
\pgfpathrectangle{\pgfqpoint{0.445556in}{0.499444in}}{\pgfqpoint{3.875000in}{1.155000in}}%
\pgfusepath{clip}%
\pgfsetbuttcap%
\pgfsetmiterjoin%
\definecolor{currentfill}{rgb}{0.000000,0.000000,0.000000}%
\pgfsetfillcolor{currentfill}%
\pgfsetlinewidth{0.000000pt}%
\definecolor{currentstroke}{rgb}{0.000000,0.000000,0.000000}%
\pgfsetstrokecolor{currentstroke}%
\pgfsetstrokeopacity{0.000000}%
\pgfsetdash{}{0pt}%
\pgfpathmoveto{\pgfqpoint{1.097783in}{0.499444in}}%
\pgfpathlineto{\pgfqpoint{1.159170in}{0.499444in}}%
\pgfpathlineto{\pgfqpoint{1.159170in}{0.524977in}}%
\pgfpathlineto{\pgfqpoint{1.097783in}{0.524977in}}%
\pgfpathlineto{\pgfqpoint{1.097783in}{0.499444in}}%
\pgfpathclose%
\pgfusepath{fill}%
\end{pgfscope}%
\begin{pgfscope}%
\pgfpathrectangle{\pgfqpoint{0.445556in}{0.499444in}}{\pgfqpoint{3.875000in}{1.155000in}}%
\pgfusepath{clip}%
\pgfsetbuttcap%
\pgfsetmiterjoin%
\definecolor{currentfill}{rgb}{0.000000,0.000000,0.000000}%
\pgfsetfillcolor{currentfill}%
\pgfsetlinewidth{0.000000pt}%
\definecolor{currentstroke}{rgb}{0.000000,0.000000,0.000000}%
\pgfsetstrokecolor{currentstroke}%
\pgfsetstrokeopacity{0.000000}%
\pgfsetdash{}{0pt}%
\pgfpathmoveto{\pgfqpoint{1.251249in}{0.499444in}}%
\pgfpathlineto{\pgfqpoint{1.312635in}{0.499444in}}%
\pgfpathlineto{\pgfqpoint{1.312635in}{0.535267in}}%
\pgfpathlineto{\pgfqpoint{1.251249in}{0.535267in}}%
\pgfpathlineto{\pgfqpoint{1.251249in}{0.499444in}}%
\pgfpathclose%
\pgfusepath{fill}%
\end{pgfscope}%
\begin{pgfscope}%
\pgfpathrectangle{\pgfqpoint{0.445556in}{0.499444in}}{\pgfqpoint{3.875000in}{1.155000in}}%
\pgfusepath{clip}%
\pgfsetbuttcap%
\pgfsetmiterjoin%
\definecolor{currentfill}{rgb}{0.000000,0.000000,0.000000}%
\pgfsetfillcolor{currentfill}%
\pgfsetlinewidth{0.000000pt}%
\definecolor{currentstroke}{rgb}{0.000000,0.000000,0.000000}%
\pgfsetstrokecolor{currentstroke}%
\pgfsetstrokeopacity{0.000000}%
\pgfsetdash{}{0pt}%
\pgfpathmoveto{\pgfqpoint{1.404714in}{0.499444in}}%
\pgfpathlineto{\pgfqpoint{1.466100in}{0.499444in}}%
\pgfpathlineto{\pgfqpoint{1.466100in}{0.549659in}}%
\pgfpathlineto{\pgfqpoint{1.404714in}{0.549659in}}%
\pgfpathlineto{\pgfqpoint{1.404714in}{0.499444in}}%
\pgfpathclose%
\pgfusepath{fill}%
\end{pgfscope}%
\begin{pgfscope}%
\pgfpathrectangle{\pgfqpoint{0.445556in}{0.499444in}}{\pgfqpoint{3.875000in}{1.155000in}}%
\pgfusepath{clip}%
\pgfsetbuttcap%
\pgfsetmiterjoin%
\definecolor{currentfill}{rgb}{0.000000,0.000000,0.000000}%
\pgfsetfillcolor{currentfill}%
\pgfsetlinewidth{0.000000pt}%
\definecolor{currentstroke}{rgb}{0.000000,0.000000,0.000000}%
\pgfsetstrokecolor{currentstroke}%
\pgfsetstrokeopacity{0.000000}%
\pgfsetdash{}{0pt}%
\pgfpathmoveto{\pgfqpoint{1.558179in}{0.499444in}}%
\pgfpathlineto{\pgfqpoint{1.619566in}{0.499444in}}%
\pgfpathlineto{\pgfqpoint{1.619566in}{0.568228in}}%
\pgfpathlineto{\pgfqpoint{1.558179in}{0.568228in}}%
\pgfpathlineto{\pgfqpoint{1.558179in}{0.499444in}}%
\pgfpathclose%
\pgfusepath{fill}%
\end{pgfscope}%
\begin{pgfscope}%
\pgfpathrectangle{\pgfqpoint{0.445556in}{0.499444in}}{\pgfqpoint{3.875000in}{1.155000in}}%
\pgfusepath{clip}%
\pgfsetbuttcap%
\pgfsetmiterjoin%
\definecolor{currentfill}{rgb}{0.000000,0.000000,0.000000}%
\pgfsetfillcolor{currentfill}%
\pgfsetlinewidth{0.000000pt}%
\definecolor{currentstroke}{rgb}{0.000000,0.000000,0.000000}%
\pgfsetstrokecolor{currentstroke}%
\pgfsetstrokeopacity{0.000000}%
\pgfsetdash{}{0pt}%
\pgfpathmoveto{\pgfqpoint{1.711645in}{0.499444in}}%
\pgfpathlineto{\pgfqpoint{1.773031in}{0.499444in}}%
\pgfpathlineto{\pgfqpoint{1.773031in}{0.590743in}}%
\pgfpathlineto{\pgfqpoint{1.711645in}{0.590743in}}%
\pgfpathlineto{\pgfqpoint{1.711645in}{0.499444in}}%
\pgfpathclose%
\pgfusepath{fill}%
\end{pgfscope}%
\begin{pgfscope}%
\pgfpathrectangle{\pgfqpoint{0.445556in}{0.499444in}}{\pgfqpoint{3.875000in}{1.155000in}}%
\pgfusepath{clip}%
\pgfsetbuttcap%
\pgfsetmiterjoin%
\definecolor{currentfill}{rgb}{0.000000,0.000000,0.000000}%
\pgfsetfillcolor{currentfill}%
\pgfsetlinewidth{0.000000pt}%
\definecolor{currentstroke}{rgb}{0.000000,0.000000,0.000000}%
\pgfsetstrokecolor{currentstroke}%
\pgfsetstrokeopacity{0.000000}%
\pgfsetdash{}{0pt}%
\pgfpathmoveto{\pgfqpoint{1.865110in}{0.499444in}}%
\pgfpathlineto{\pgfqpoint{1.926496in}{0.499444in}}%
\pgfpathlineto{\pgfqpoint{1.926496in}{0.605521in}}%
\pgfpathlineto{\pgfqpoint{1.865110in}{0.605521in}}%
\pgfpathlineto{\pgfqpoint{1.865110in}{0.499444in}}%
\pgfpathclose%
\pgfusepath{fill}%
\end{pgfscope}%
\begin{pgfscope}%
\pgfpathrectangle{\pgfqpoint{0.445556in}{0.499444in}}{\pgfqpoint{3.875000in}{1.155000in}}%
\pgfusepath{clip}%
\pgfsetbuttcap%
\pgfsetmiterjoin%
\definecolor{currentfill}{rgb}{0.000000,0.000000,0.000000}%
\pgfsetfillcolor{currentfill}%
\pgfsetlinewidth{0.000000pt}%
\definecolor{currentstroke}{rgb}{0.000000,0.000000,0.000000}%
\pgfsetstrokecolor{currentstroke}%
\pgfsetstrokeopacity{0.000000}%
\pgfsetdash{}{0pt}%
\pgfpathmoveto{\pgfqpoint{2.018575in}{0.499444in}}%
\pgfpathlineto{\pgfqpoint{2.079962in}{0.499444in}}%
\pgfpathlineto{\pgfqpoint{2.079962in}{0.624632in}}%
\pgfpathlineto{\pgfqpoint{2.018575in}{0.624632in}}%
\pgfpathlineto{\pgfqpoint{2.018575in}{0.499444in}}%
\pgfpathclose%
\pgfusepath{fill}%
\end{pgfscope}%
\begin{pgfscope}%
\pgfpathrectangle{\pgfqpoint{0.445556in}{0.499444in}}{\pgfqpoint{3.875000in}{1.155000in}}%
\pgfusepath{clip}%
\pgfsetbuttcap%
\pgfsetmiterjoin%
\definecolor{currentfill}{rgb}{0.000000,0.000000,0.000000}%
\pgfsetfillcolor{currentfill}%
\pgfsetlinewidth{0.000000pt}%
\definecolor{currentstroke}{rgb}{0.000000,0.000000,0.000000}%
\pgfsetstrokecolor{currentstroke}%
\pgfsetstrokeopacity{0.000000}%
\pgfsetdash{}{0pt}%
\pgfpathmoveto{\pgfqpoint{2.172041in}{0.499444in}}%
\pgfpathlineto{\pgfqpoint{2.233427in}{0.499444in}}%
\pgfpathlineto{\pgfqpoint{2.233427in}{0.643356in}}%
\pgfpathlineto{\pgfqpoint{2.172041in}{0.643356in}}%
\pgfpathlineto{\pgfqpoint{2.172041in}{0.499444in}}%
\pgfpathclose%
\pgfusepath{fill}%
\end{pgfscope}%
\begin{pgfscope}%
\pgfpathrectangle{\pgfqpoint{0.445556in}{0.499444in}}{\pgfqpoint{3.875000in}{1.155000in}}%
\pgfusepath{clip}%
\pgfsetbuttcap%
\pgfsetmiterjoin%
\definecolor{currentfill}{rgb}{0.000000,0.000000,0.000000}%
\pgfsetfillcolor{currentfill}%
\pgfsetlinewidth{0.000000pt}%
\definecolor{currentstroke}{rgb}{0.000000,0.000000,0.000000}%
\pgfsetstrokecolor{currentstroke}%
\pgfsetstrokeopacity{0.000000}%
\pgfsetdash{}{0pt}%
\pgfpathmoveto{\pgfqpoint{2.325506in}{0.499444in}}%
\pgfpathlineto{\pgfqpoint{2.386892in}{0.499444in}}%
\pgfpathlineto{\pgfqpoint{2.386892in}{0.662622in}}%
\pgfpathlineto{\pgfqpoint{2.325506in}{0.662622in}}%
\pgfpathlineto{\pgfqpoint{2.325506in}{0.499444in}}%
\pgfpathclose%
\pgfusepath{fill}%
\end{pgfscope}%
\begin{pgfscope}%
\pgfpathrectangle{\pgfqpoint{0.445556in}{0.499444in}}{\pgfqpoint{3.875000in}{1.155000in}}%
\pgfusepath{clip}%
\pgfsetbuttcap%
\pgfsetmiterjoin%
\definecolor{currentfill}{rgb}{0.000000,0.000000,0.000000}%
\pgfsetfillcolor{currentfill}%
\pgfsetlinewidth{0.000000pt}%
\definecolor{currentstroke}{rgb}{0.000000,0.000000,0.000000}%
\pgfsetstrokecolor{currentstroke}%
\pgfsetstrokeopacity{0.000000}%
\pgfsetdash{}{0pt}%
\pgfpathmoveto{\pgfqpoint{2.478972in}{0.499444in}}%
\pgfpathlineto{\pgfqpoint{2.540358in}{0.499444in}}%
\pgfpathlineto{\pgfqpoint{2.540358in}{0.672216in}}%
\pgfpathlineto{\pgfqpoint{2.478972in}{0.672216in}}%
\pgfpathlineto{\pgfqpoint{2.478972in}{0.499444in}}%
\pgfpathclose%
\pgfusepath{fill}%
\end{pgfscope}%
\begin{pgfscope}%
\pgfpathrectangle{\pgfqpoint{0.445556in}{0.499444in}}{\pgfqpoint{3.875000in}{1.155000in}}%
\pgfusepath{clip}%
\pgfsetbuttcap%
\pgfsetmiterjoin%
\definecolor{currentfill}{rgb}{0.000000,0.000000,0.000000}%
\pgfsetfillcolor{currentfill}%
\pgfsetlinewidth{0.000000pt}%
\definecolor{currentstroke}{rgb}{0.000000,0.000000,0.000000}%
\pgfsetstrokecolor{currentstroke}%
\pgfsetstrokeopacity{0.000000}%
\pgfsetdash{}{0pt}%
\pgfpathmoveto{\pgfqpoint{2.632437in}{0.499444in}}%
\pgfpathlineto{\pgfqpoint{2.693823in}{0.499444in}}%
\pgfpathlineto{\pgfqpoint{2.693823in}{0.684286in}}%
\pgfpathlineto{\pgfqpoint{2.632437in}{0.684286in}}%
\pgfpathlineto{\pgfqpoint{2.632437in}{0.499444in}}%
\pgfpathclose%
\pgfusepath{fill}%
\end{pgfscope}%
\begin{pgfscope}%
\pgfpathrectangle{\pgfqpoint{0.445556in}{0.499444in}}{\pgfqpoint{3.875000in}{1.155000in}}%
\pgfusepath{clip}%
\pgfsetbuttcap%
\pgfsetmiterjoin%
\definecolor{currentfill}{rgb}{0.000000,0.000000,0.000000}%
\pgfsetfillcolor{currentfill}%
\pgfsetlinewidth{0.000000pt}%
\definecolor{currentstroke}{rgb}{0.000000,0.000000,0.000000}%
\pgfsetstrokecolor{currentstroke}%
\pgfsetstrokeopacity{0.000000}%
\pgfsetdash{}{0pt}%
\pgfpathmoveto{\pgfqpoint{2.785902in}{0.499444in}}%
\pgfpathlineto{\pgfqpoint{2.847288in}{0.499444in}}%
\pgfpathlineto{\pgfqpoint{2.847288in}{0.686607in}}%
\pgfpathlineto{\pgfqpoint{2.785902in}{0.686607in}}%
\pgfpathlineto{\pgfqpoint{2.785902in}{0.499444in}}%
\pgfpathclose%
\pgfusepath{fill}%
\end{pgfscope}%
\begin{pgfscope}%
\pgfpathrectangle{\pgfqpoint{0.445556in}{0.499444in}}{\pgfqpoint{3.875000in}{1.155000in}}%
\pgfusepath{clip}%
\pgfsetbuttcap%
\pgfsetmiterjoin%
\definecolor{currentfill}{rgb}{0.000000,0.000000,0.000000}%
\pgfsetfillcolor{currentfill}%
\pgfsetlinewidth{0.000000pt}%
\definecolor{currentstroke}{rgb}{0.000000,0.000000,0.000000}%
\pgfsetstrokecolor{currentstroke}%
\pgfsetstrokeopacity{0.000000}%
\pgfsetdash{}{0pt}%
\pgfpathmoveto{\pgfqpoint{2.939368in}{0.499444in}}%
\pgfpathlineto{\pgfqpoint{3.000754in}{0.499444in}}%
\pgfpathlineto{\pgfqpoint{3.000754in}{0.682816in}}%
\pgfpathlineto{\pgfqpoint{2.939368in}{0.682816in}}%
\pgfpathlineto{\pgfqpoint{2.939368in}{0.499444in}}%
\pgfpathclose%
\pgfusepath{fill}%
\end{pgfscope}%
\begin{pgfscope}%
\pgfpathrectangle{\pgfqpoint{0.445556in}{0.499444in}}{\pgfqpoint{3.875000in}{1.155000in}}%
\pgfusepath{clip}%
\pgfsetbuttcap%
\pgfsetmiterjoin%
\definecolor{currentfill}{rgb}{0.000000,0.000000,0.000000}%
\pgfsetfillcolor{currentfill}%
\pgfsetlinewidth{0.000000pt}%
\definecolor{currentstroke}{rgb}{0.000000,0.000000,0.000000}%
\pgfsetstrokecolor{currentstroke}%
\pgfsetstrokeopacity{0.000000}%
\pgfsetdash{}{0pt}%
\pgfpathmoveto{\pgfqpoint{3.092833in}{0.499444in}}%
\pgfpathlineto{\pgfqpoint{3.154219in}{0.499444in}}%
\pgfpathlineto{\pgfqpoint{3.154219in}{0.681037in}}%
\pgfpathlineto{\pgfqpoint{3.092833in}{0.681037in}}%
\pgfpathlineto{\pgfqpoint{3.092833in}{0.499444in}}%
\pgfpathclose%
\pgfusepath{fill}%
\end{pgfscope}%
\begin{pgfscope}%
\pgfpathrectangle{\pgfqpoint{0.445556in}{0.499444in}}{\pgfqpoint{3.875000in}{1.155000in}}%
\pgfusepath{clip}%
\pgfsetbuttcap%
\pgfsetmiterjoin%
\definecolor{currentfill}{rgb}{0.000000,0.000000,0.000000}%
\pgfsetfillcolor{currentfill}%
\pgfsetlinewidth{0.000000pt}%
\definecolor{currentstroke}{rgb}{0.000000,0.000000,0.000000}%
\pgfsetstrokecolor{currentstroke}%
\pgfsetstrokeopacity{0.000000}%
\pgfsetdash{}{0pt}%
\pgfpathmoveto{\pgfqpoint{3.246298in}{0.499444in}}%
\pgfpathlineto{\pgfqpoint{3.307684in}{0.499444in}}%
\pgfpathlineto{\pgfqpoint{3.307684in}{0.677787in}}%
\pgfpathlineto{\pgfqpoint{3.246298in}{0.677787in}}%
\pgfpathlineto{\pgfqpoint{3.246298in}{0.499444in}}%
\pgfpathclose%
\pgfusepath{fill}%
\end{pgfscope}%
\begin{pgfscope}%
\pgfpathrectangle{\pgfqpoint{0.445556in}{0.499444in}}{\pgfqpoint{3.875000in}{1.155000in}}%
\pgfusepath{clip}%
\pgfsetbuttcap%
\pgfsetmiterjoin%
\definecolor{currentfill}{rgb}{0.000000,0.000000,0.000000}%
\pgfsetfillcolor{currentfill}%
\pgfsetlinewidth{0.000000pt}%
\definecolor{currentstroke}{rgb}{0.000000,0.000000,0.000000}%
\pgfsetstrokecolor{currentstroke}%
\pgfsetstrokeopacity{0.000000}%
\pgfsetdash{}{0pt}%
\pgfpathmoveto{\pgfqpoint{3.399764in}{0.499444in}}%
\pgfpathlineto{\pgfqpoint{3.461150in}{0.499444in}}%
\pgfpathlineto{\pgfqpoint{3.461150in}{0.659527in}}%
\pgfpathlineto{\pgfqpoint{3.399764in}{0.659527in}}%
\pgfpathlineto{\pgfqpoint{3.399764in}{0.499444in}}%
\pgfpathclose%
\pgfusepath{fill}%
\end{pgfscope}%
\begin{pgfscope}%
\pgfpathrectangle{\pgfqpoint{0.445556in}{0.499444in}}{\pgfqpoint{3.875000in}{1.155000in}}%
\pgfusepath{clip}%
\pgfsetbuttcap%
\pgfsetmiterjoin%
\definecolor{currentfill}{rgb}{0.000000,0.000000,0.000000}%
\pgfsetfillcolor{currentfill}%
\pgfsetlinewidth{0.000000pt}%
\definecolor{currentstroke}{rgb}{0.000000,0.000000,0.000000}%
\pgfsetstrokecolor{currentstroke}%
\pgfsetstrokeopacity{0.000000}%
\pgfsetdash{}{0pt}%
\pgfpathmoveto{\pgfqpoint{3.553229in}{0.499444in}}%
\pgfpathlineto{\pgfqpoint{3.614615in}{0.499444in}}%
\pgfpathlineto{\pgfqpoint{3.614615in}{0.657206in}}%
\pgfpathlineto{\pgfqpoint{3.553229in}{0.657206in}}%
\pgfpathlineto{\pgfqpoint{3.553229in}{0.499444in}}%
\pgfpathclose%
\pgfusepath{fill}%
\end{pgfscope}%
\begin{pgfscope}%
\pgfpathrectangle{\pgfqpoint{0.445556in}{0.499444in}}{\pgfqpoint{3.875000in}{1.155000in}}%
\pgfusepath{clip}%
\pgfsetbuttcap%
\pgfsetmiterjoin%
\definecolor{currentfill}{rgb}{0.000000,0.000000,0.000000}%
\pgfsetfillcolor{currentfill}%
\pgfsetlinewidth{0.000000pt}%
\definecolor{currentstroke}{rgb}{0.000000,0.000000,0.000000}%
\pgfsetstrokecolor{currentstroke}%
\pgfsetstrokeopacity{0.000000}%
\pgfsetdash{}{0pt}%
\pgfpathmoveto{\pgfqpoint{3.706694in}{0.499444in}}%
\pgfpathlineto{\pgfqpoint{3.768080in}{0.499444in}}%
\pgfpathlineto{\pgfqpoint{3.768080in}{0.639333in}}%
\pgfpathlineto{\pgfqpoint{3.706694in}{0.639333in}}%
\pgfpathlineto{\pgfqpoint{3.706694in}{0.499444in}}%
\pgfpathclose%
\pgfusepath{fill}%
\end{pgfscope}%
\begin{pgfscope}%
\pgfpathrectangle{\pgfqpoint{0.445556in}{0.499444in}}{\pgfqpoint{3.875000in}{1.155000in}}%
\pgfusepath{clip}%
\pgfsetbuttcap%
\pgfsetmiterjoin%
\definecolor{currentfill}{rgb}{0.000000,0.000000,0.000000}%
\pgfsetfillcolor{currentfill}%
\pgfsetlinewidth{0.000000pt}%
\definecolor{currentstroke}{rgb}{0.000000,0.000000,0.000000}%
\pgfsetstrokecolor{currentstroke}%
\pgfsetstrokeopacity{0.000000}%
\pgfsetdash{}{0pt}%
\pgfpathmoveto{\pgfqpoint{3.860160in}{0.499444in}}%
\pgfpathlineto{\pgfqpoint{3.921546in}{0.499444in}}%
\pgfpathlineto{\pgfqpoint{3.921546in}{0.623626in}}%
\pgfpathlineto{\pgfqpoint{3.860160in}{0.623626in}}%
\pgfpathlineto{\pgfqpoint{3.860160in}{0.499444in}}%
\pgfpathclose%
\pgfusepath{fill}%
\end{pgfscope}%
\begin{pgfscope}%
\pgfpathrectangle{\pgfqpoint{0.445556in}{0.499444in}}{\pgfqpoint{3.875000in}{1.155000in}}%
\pgfusepath{clip}%
\pgfsetbuttcap%
\pgfsetmiterjoin%
\definecolor{currentfill}{rgb}{0.000000,0.000000,0.000000}%
\pgfsetfillcolor{currentfill}%
\pgfsetlinewidth{0.000000pt}%
\definecolor{currentstroke}{rgb}{0.000000,0.000000,0.000000}%
\pgfsetstrokecolor{currentstroke}%
\pgfsetstrokeopacity{0.000000}%
\pgfsetdash{}{0pt}%
\pgfpathmoveto{\pgfqpoint{4.013625in}{0.499444in}}%
\pgfpathlineto{\pgfqpoint{4.075011in}{0.499444in}}%
\pgfpathlineto{\pgfqpoint{4.075011in}{0.580840in}}%
\pgfpathlineto{\pgfqpoint{4.013625in}{0.580840in}}%
\pgfpathlineto{\pgfqpoint{4.013625in}{0.499444in}}%
\pgfpathclose%
\pgfusepath{fill}%
\end{pgfscope}%
\begin{pgfscope}%
\pgfpathrectangle{\pgfqpoint{0.445556in}{0.499444in}}{\pgfqpoint{3.875000in}{1.155000in}}%
\pgfusepath{clip}%
\pgfsetbuttcap%
\pgfsetmiterjoin%
\definecolor{currentfill}{rgb}{0.000000,0.000000,0.000000}%
\pgfsetfillcolor{currentfill}%
\pgfsetlinewidth{0.000000pt}%
\definecolor{currentstroke}{rgb}{0.000000,0.000000,0.000000}%
\pgfsetstrokecolor{currentstroke}%
\pgfsetstrokeopacity{0.000000}%
\pgfsetdash{}{0pt}%
\pgfpathmoveto{\pgfqpoint{4.167090in}{0.499444in}}%
\pgfpathlineto{\pgfqpoint{4.228476in}{0.499444in}}%
\pgfpathlineto{\pgfqpoint{4.228476in}{0.529774in}}%
\pgfpathlineto{\pgfqpoint{4.167090in}{0.529774in}}%
\pgfpathlineto{\pgfqpoint{4.167090in}{0.499444in}}%
\pgfpathclose%
\pgfusepath{fill}%
\end{pgfscope}%
\begin{pgfscope}%
\pgfsetbuttcap%
\pgfsetroundjoin%
\definecolor{currentfill}{rgb}{0.000000,0.000000,0.000000}%
\pgfsetfillcolor{currentfill}%
\pgfsetlinewidth{0.803000pt}%
\definecolor{currentstroke}{rgb}{0.000000,0.000000,0.000000}%
\pgfsetstrokecolor{currentstroke}%
\pgfsetdash{}{0pt}%
\pgfsys@defobject{currentmarker}{\pgfqpoint{0.000000in}{-0.048611in}}{\pgfqpoint{0.000000in}{0.000000in}}{%
\pgfpathmoveto{\pgfqpoint{0.000000in}{0.000000in}}%
\pgfpathlineto{\pgfqpoint{0.000000in}{-0.048611in}}%
\pgfusepath{stroke,fill}%
}%
\begin{pgfscope}%
\pgfsys@transformshift{0.483922in}{0.499444in}%
\pgfsys@useobject{currentmarker}{}%
\end{pgfscope}%
\end{pgfscope}%
\begin{pgfscope}%
\definecolor{textcolor}{rgb}{0.000000,0.000000,0.000000}%
\pgfsetstrokecolor{textcolor}%
\pgfsetfillcolor{textcolor}%
\pgftext[x=0.483922in,y=0.402222in,,top]{\color{textcolor}\rmfamily\fontsize{10.000000}{12.000000}\selectfont 0.0}%
\end{pgfscope}%
\begin{pgfscope}%
\pgfsetbuttcap%
\pgfsetroundjoin%
\definecolor{currentfill}{rgb}{0.000000,0.000000,0.000000}%
\pgfsetfillcolor{currentfill}%
\pgfsetlinewidth{0.803000pt}%
\definecolor{currentstroke}{rgb}{0.000000,0.000000,0.000000}%
\pgfsetstrokecolor{currentstroke}%
\pgfsetdash{}{0pt}%
\pgfsys@defobject{currentmarker}{\pgfqpoint{0.000000in}{-0.048611in}}{\pgfqpoint{0.000000in}{0.000000in}}{%
\pgfpathmoveto{\pgfqpoint{0.000000in}{0.000000in}}%
\pgfpathlineto{\pgfqpoint{0.000000in}{-0.048611in}}%
\pgfusepath{stroke,fill}%
}%
\begin{pgfscope}%
\pgfsys@transformshift{0.867585in}{0.499444in}%
\pgfsys@useobject{currentmarker}{}%
\end{pgfscope}%
\end{pgfscope}%
\begin{pgfscope}%
\definecolor{textcolor}{rgb}{0.000000,0.000000,0.000000}%
\pgfsetstrokecolor{textcolor}%
\pgfsetfillcolor{textcolor}%
\pgftext[x=0.867585in,y=0.402222in,,top]{\color{textcolor}\rmfamily\fontsize{10.000000}{12.000000}\selectfont 0.1}%
\end{pgfscope}%
\begin{pgfscope}%
\pgfsetbuttcap%
\pgfsetroundjoin%
\definecolor{currentfill}{rgb}{0.000000,0.000000,0.000000}%
\pgfsetfillcolor{currentfill}%
\pgfsetlinewidth{0.803000pt}%
\definecolor{currentstroke}{rgb}{0.000000,0.000000,0.000000}%
\pgfsetstrokecolor{currentstroke}%
\pgfsetdash{}{0pt}%
\pgfsys@defobject{currentmarker}{\pgfqpoint{0.000000in}{-0.048611in}}{\pgfqpoint{0.000000in}{0.000000in}}{%
\pgfpathmoveto{\pgfqpoint{0.000000in}{0.000000in}}%
\pgfpathlineto{\pgfqpoint{0.000000in}{-0.048611in}}%
\pgfusepath{stroke,fill}%
}%
\begin{pgfscope}%
\pgfsys@transformshift{1.251249in}{0.499444in}%
\pgfsys@useobject{currentmarker}{}%
\end{pgfscope}%
\end{pgfscope}%
\begin{pgfscope}%
\definecolor{textcolor}{rgb}{0.000000,0.000000,0.000000}%
\pgfsetstrokecolor{textcolor}%
\pgfsetfillcolor{textcolor}%
\pgftext[x=1.251249in,y=0.402222in,,top]{\color{textcolor}\rmfamily\fontsize{10.000000}{12.000000}\selectfont 0.2}%
\end{pgfscope}%
\begin{pgfscope}%
\pgfsetbuttcap%
\pgfsetroundjoin%
\definecolor{currentfill}{rgb}{0.000000,0.000000,0.000000}%
\pgfsetfillcolor{currentfill}%
\pgfsetlinewidth{0.803000pt}%
\definecolor{currentstroke}{rgb}{0.000000,0.000000,0.000000}%
\pgfsetstrokecolor{currentstroke}%
\pgfsetdash{}{0pt}%
\pgfsys@defobject{currentmarker}{\pgfqpoint{0.000000in}{-0.048611in}}{\pgfqpoint{0.000000in}{0.000000in}}{%
\pgfpathmoveto{\pgfqpoint{0.000000in}{0.000000in}}%
\pgfpathlineto{\pgfqpoint{0.000000in}{-0.048611in}}%
\pgfusepath{stroke,fill}%
}%
\begin{pgfscope}%
\pgfsys@transformshift{1.634912in}{0.499444in}%
\pgfsys@useobject{currentmarker}{}%
\end{pgfscope}%
\end{pgfscope}%
\begin{pgfscope}%
\definecolor{textcolor}{rgb}{0.000000,0.000000,0.000000}%
\pgfsetstrokecolor{textcolor}%
\pgfsetfillcolor{textcolor}%
\pgftext[x=1.634912in,y=0.402222in,,top]{\color{textcolor}\rmfamily\fontsize{10.000000}{12.000000}\selectfont 0.3}%
\end{pgfscope}%
\begin{pgfscope}%
\pgfsetbuttcap%
\pgfsetroundjoin%
\definecolor{currentfill}{rgb}{0.000000,0.000000,0.000000}%
\pgfsetfillcolor{currentfill}%
\pgfsetlinewidth{0.803000pt}%
\definecolor{currentstroke}{rgb}{0.000000,0.000000,0.000000}%
\pgfsetstrokecolor{currentstroke}%
\pgfsetdash{}{0pt}%
\pgfsys@defobject{currentmarker}{\pgfqpoint{0.000000in}{-0.048611in}}{\pgfqpoint{0.000000in}{0.000000in}}{%
\pgfpathmoveto{\pgfqpoint{0.000000in}{0.000000in}}%
\pgfpathlineto{\pgfqpoint{0.000000in}{-0.048611in}}%
\pgfusepath{stroke,fill}%
}%
\begin{pgfscope}%
\pgfsys@transformshift{2.018575in}{0.499444in}%
\pgfsys@useobject{currentmarker}{}%
\end{pgfscope}%
\end{pgfscope}%
\begin{pgfscope}%
\definecolor{textcolor}{rgb}{0.000000,0.000000,0.000000}%
\pgfsetstrokecolor{textcolor}%
\pgfsetfillcolor{textcolor}%
\pgftext[x=2.018575in,y=0.402222in,,top]{\color{textcolor}\rmfamily\fontsize{10.000000}{12.000000}\selectfont 0.4}%
\end{pgfscope}%
\begin{pgfscope}%
\pgfsetbuttcap%
\pgfsetroundjoin%
\definecolor{currentfill}{rgb}{0.000000,0.000000,0.000000}%
\pgfsetfillcolor{currentfill}%
\pgfsetlinewidth{0.803000pt}%
\definecolor{currentstroke}{rgb}{0.000000,0.000000,0.000000}%
\pgfsetstrokecolor{currentstroke}%
\pgfsetdash{}{0pt}%
\pgfsys@defobject{currentmarker}{\pgfqpoint{0.000000in}{-0.048611in}}{\pgfqpoint{0.000000in}{0.000000in}}{%
\pgfpathmoveto{\pgfqpoint{0.000000in}{0.000000in}}%
\pgfpathlineto{\pgfqpoint{0.000000in}{-0.048611in}}%
\pgfusepath{stroke,fill}%
}%
\begin{pgfscope}%
\pgfsys@transformshift{2.402239in}{0.499444in}%
\pgfsys@useobject{currentmarker}{}%
\end{pgfscope}%
\end{pgfscope}%
\begin{pgfscope}%
\definecolor{textcolor}{rgb}{0.000000,0.000000,0.000000}%
\pgfsetstrokecolor{textcolor}%
\pgfsetfillcolor{textcolor}%
\pgftext[x=2.402239in,y=0.402222in,,top]{\color{textcolor}\rmfamily\fontsize{10.000000}{12.000000}\selectfont 0.5}%
\end{pgfscope}%
\begin{pgfscope}%
\pgfsetbuttcap%
\pgfsetroundjoin%
\definecolor{currentfill}{rgb}{0.000000,0.000000,0.000000}%
\pgfsetfillcolor{currentfill}%
\pgfsetlinewidth{0.803000pt}%
\definecolor{currentstroke}{rgb}{0.000000,0.000000,0.000000}%
\pgfsetstrokecolor{currentstroke}%
\pgfsetdash{}{0pt}%
\pgfsys@defobject{currentmarker}{\pgfqpoint{0.000000in}{-0.048611in}}{\pgfqpoint{0.000000in}{0.000000in}}{%
\pgfpathmoveto{\pgfqpoint{0.000000in}{0.000000in}}%
\pgfpathlineto{\pgfqpoint{0.000000in}{-0.048611in}}%
\pgfusepath{stroke,fill}%
}%
\begin{pgfscope}%
\pgfsys@transformshift{2.785902in}{0.499444in}%
\pgfsys@useobject{currentmarker}{}%
\end{pgfscope}%
\end{pgfscope}%
\begin{pgfscope}%
\definecolor{textcolor}{rgb}{0.000000,0.000000,0.000000}%
\pgfsetstrokecolor{textcolor}%
\pgfsetfillcolor{textcolor}%
\pgftext[x=2.785902in,y=0.402222in,,top]{\color{textcolor}\rmfamily\fontsize{10.000000}{12.000000}\selectfont 0.6}%
\end{pgfscope}%
\begin{pgfscope}%
\pgfsetbuttcap%
\pgfsetroundjoin%
\definecolor{currentfill}{rgb}{0.000000,0.000000,0.000000}%
\pgfsetfillcolor{currentfill}%
\pgfsetlinewidth{0.803000pt}%
\definecolor{currentstroke}{rgb}{0.000000,0.000000,0.000000}%
\pgfsetstrokecolor{currentstroke}%
\pgfsetdash{}{0pt}%
\pgfsys@defobject{currentmarker}{\pgfqpoint{0.000000in}{-0.048611in}}{\pgfqpoint{0.000000in}{0.000000in}}{%
\pgfpathmoveto{\pgfqpoint{0.000000in}{0.000000in}}%
\pgfpathlineto{\pgfqpoint{0.000000in}{-0.048611in}}%
\pgfusepath{stroke,fill}%
}%
\begin{pgfscope}%
\pgfsys@transformshift{3.169566in}{0.499444in}%
\pgfsys@useobject{currentmarker}{}%
\end{pgfscope}%
\end{pgfscope}%
\begin{pgfscope}%
\definecolor{textcolor}{rgb}{0.000000,0.000000,0.000000}%
\pgfsetstrokecolor{textcolor}%
\pgfsetfillcolor{textcolor}%
\pgftext[x=3.169566in,y=0.402222in,,top]{\color{textcolor}\rmfamily\fontsize{10.000000}{12.000000}\selectfont 0.7}%
\end{pgfscope}%
\begin{pgfscope}%
\pgfsetbuttcap%
\pgfsetroundjoin%
\definecolor{currentfill}{rgb}{0.000000,0.000000,0.000000}%
\pgfsetfillcolor{currentfill}%
\pgfsetlinewidth{0.803000pt}%
\definecolor{currentstroke}{rgb}{0.000000,0.000000,0.000000}%
\pgfsetstrokecolor{currentstroke}%
\pgfsetdash{}{0pt}%
\pgfsys@defobject{currentmarker}{\pgfqpoint{0.000000in}{-0.048611in}}{\pgfqpoint{0.000000in}{0.000000in}}{%
\pgfpathmoveto{\pgfqpoint{0.000000in}{0.000000in}}%
\pgfpathlineto{\pgfqpoint{0.000000in}{-0.048611in}}%
\pgfusepath{stroke,fill}%
}%
\begin{pgfscope}%
\pgfsys@transformshift{3.553229in}{0.499444in}%
\pgfsys@useobject{currentmarker}{}%
\end{pgfscope}%
\end{pgfscope}%
\begin{pgfscope}%
\definecolor{textcolor}{rgb}{0.000000,0.000000,0.000000}%
\pgfsetstrokecolor{textcolor}%
\pgfsetfillcolor{textcolor}%
\pgftext[x=3.553229in,y=0.402222in,,top]{\color{textcolor}\rmfamily\fontsize{10.000000}{12.000000}\selectfont 0.8}%
\end{pgfscope}%
\begin{pgfscope}%
\pgfsetbuttcap%
\pgfsetroundjoin%
\definecolor{currentfill}{rgb}{0.000000,0.000000,0.000000}%
\pgfsetfillcolor{currentfill}%
\pgfsetlinewidth{0.803000pt}%
\definecolor{currentstroke}{rgb}{0.000000,0.000000,0.000000}%
\pgfsetstrokecolor{currentstroke}%
\pgfsetdash{}{0pt}%
\pgfsys@defobject{currentmarker}{\pgfqpoint{0.000000in}{-0.048611in}}{\pgfqpoint{0.000000in}{0.000000in}}{%
\pgfpathmoveto{\pgfqpoint{0.000000in}{0.000000in}}%
\pgfpathlineto{\pgfqpoint{0.000000in}{-0.048611in}}%
\pgfusepath{stroke,fill}%
}%
\begin{pgfscope}%
\pgfsys@transformshift{3.936892in}{0.499444in}%
\pgfsys@useobject{currentmarker}{}%
\end{pgfscope}%
\end{pgfscope}%
\begin{pgfscope}%
\definecolor{textcolor}{rgb}{0.000000,0.000000,0.000000}%
\pgfsetstrokecolor{textcolor}%
\pgfsetfillcolor{textcolor}%
\pgftext[x=3.936892in,y=0.402222in,,top]{\color{textcolor}\rmfamily\fontsize{10.000000}{12.000000}\selectfont 0.9}%
\end{pgfscope}%
\begin{pgfscope}%
\pgfsetbuttcap%
\pgfsetroundjoin%
\definecolor{currentfill}{rgb}{0.000000,0.000000,0.000000}%
\pgfsetfillcolor{currentfill}%
\pgfsetlinewidth{0.803000pt}%
\definecolor{currentstroke}{rgb}{0.000000,0.000000,0.000000}%
\pgfsetstrokecolor{currentstroke}%
\pgfsetdash{}{0pt}%
\pgfsys@defobject{currentmarker}{\pgfqpoint{0.000000in}{-0.048611in}}{\pgfqpoint{0.000000in}{0.000000in}}{%
\pgfpathmoveto{\pgfqpoint{0.000000in}{0.000000in}}%
\pgfpathlineto{\pgfqpoint{0.000000in}{-0.048611in}}%
\pgfusepath{stroke,fill}%
}%
\begin{pgfscope}%
\pgfsys@transformshift{4.320556in}{0.499444in}%
\pgfsys@useobject{currentmarker}{}%
\end{pgfscope}%
\end{pgfscope}%
\begin{pgfscope}%
\definecolor{textcolor}{rgb}{0.000000,0.000000,0.000000}%
\pgfsetstrokecolor{textcolor}%
\pgfsetfillcolor{textcolor}%
\pgftext[x=4.320556in,y=0.402222in,,top]{\color{textcolor}\rmfamily\fontsize{10.000000}{12.000000}\selectfont 1.0}%
\end{pgfscope}%
\begin{pgfscope}%
\definecolor{textcolor}{rgb}{0.000000,0.000000,0.000000}%
\pgfsetstrokecolor{textcolor}%
\pgfsetfillcolor{textcolor}%
\pgftext[x=2.383056in,y=0.223333in,,top]{\color{textcolor}\rmfamily\fontsize{10.000000}{12.000000}\selectfont \(\displaystyle p\)}%
\end{pgfscope}%
\begin{pgfscope}%
\pgfsetbuttcap%
\pgfsetroundjoin%
\definecolor{currentfill}{rgb}{0.000000,0.000000,0.000000}%
\pgfsetfillcolor{currentfill}%
\pgfsetlinewidth{0.803000pt}%
\definecolor{currentstroke}{rgb}{0.000000,0.000000,0.000000}%
\pgfsetstrokecolor{currentstroke}%
\pgfsetdash{}{0pt}%
\pgfsys@defobject{currentmarker}{\pgfqpoint{-0.048611in}{0.000000in}}{\pgfqpoint{-0.000000in}{0.000000in}}{%
\pgfpathmoveto{\pgfqpoint{-0.000000in}{0.000000in}}%
\pgfpathlineto{\pgfqpoint{-0.048611in}{0.000000in}}%
\pgfusepath{stroke,fill}%
}%
\begin{pgfscope}%
\pgfsys@transformshift{0.445556in}{0.499444in}%
\pgfsys@useobject{currentmarker}{}%
\end{pgfscope}%
\end{pgfscope}%
\begin{pgfscope}%
\definecolor{textcolor}{rgb}{0.000000,0.000000,0.000000}%
\pgfsetstrokecolor{textcolor}%
\pgfsetfillcolor{textcolor}%
\pgftext[x=0.278889in, y=0.451250in, left, base]{\color{textcolor}\rmfamily\fontsize{10.000000}{12.000000}\selectfont \(\displaystyle {0}\)}%
\end{pgfscope}%
\begin{pgfscope}%
\pgfsetbuttcap%
\pgfsetroundjoin%
\definecolor{currentfill}{rgb}{0.000000,0.000000,0.000000}%
\pgfsetfillcolor{currentfill}%
\pgfsetlinewidth{0.803000pt}%
\definecolor{currentstroke}{rgb}{0.000000,0.000000,0.000000}%
\pgfsetstrokecolor{currentstroke}%
\pgfsetdash{}{0pt}%
\pgfsys@defobject{currentmarker}{\pgfqpoint{-0.048611in}{0.000000in}}{\pgfqpoint{-0.000000in}{0.000000in}}{%
\pgfpathmoveto{\pgfqpoint{-0.000000in}{0.000000in}}%
\pgfpathlineto{\pgfqpoint{-0.048611in}{0.000000in}}%
\pgfusepath{stroke,fill}%
}%
\begin{pgfscope}%
\pgfsys@transformshift{0.445556in}{0.830705in}%
\pgfsys@useobject{currentmarker}{}%
\end{pgfscope}%
\end{pgfscope}%
\begin{pgfscope}%
\definecolor{textcolor}{rgb}{0.000000,0.000000,0.000000}%
\pgfsetstrokecolor{textcolor}%
\pgfsetfillcolor{textcolor}%
\pgftext[x=0.278889in, y=0.782511in, left, base]{\color{textcolor}\rmfamily\fontsize{10.000000}{12.000000}\selectfont \(\displaystyle {2}\)}%
\end{pgfscope}%
\begin{pgfscope}%
\pgfsetbuttcap%
\pgfsetroundjoin%
\definecolor{currentfill}{rgb}{0.000000,0.000000,0.000000}%
\pgfsetfillcolor{currentfill}%
\pgfsetlinewidth{0.803000pt}%
\definecolor{currentstroke}{rgb}{0.000000,0.000000,0.000000}%
\pgfsetstrokecolor{currentstroke}%
\pgfsetdash{}{0pt}%
\pgfsys@defobject{currentmarker}{\pgfqpoint{-0.048611in}{0.000000in}}{\pgfqpoint{-0.000000in}{0.000000in}}{%
\pgfpathmoveto{\pgfqpoint{-0.000000in}{0.000000in}}%
\pgfpathlineto{\pgfqpoint{-0.048611in}{0.000000in}}%
\pgfusepath{stroke,fill}%
}%
\begin{pgfscope}%
\pgfsys@transformshift{0.445556in}{1.161966in}%
\pgfsys@useobject{currentmarker}{}%
\end{pgfscope}%
\end{pgfscope}%
\begin{pgfscope}%
\definecolor{textcolor}{rgb}{0.000000,0.000000,0.000000}%
\pgfsetstrokecolor{textcolor}%
\pgfsetfillcolor{textcolor}%
\pgftext[x=0.278889in, y=1.113772in, left, base]{\color{textcolor}\rmfamily\fontsize{10.000000}{12.000000}\selectfont \(\displaystyle {4}\)}%
\end{pgfscope}%
\begin{pgfscope}%
\pgfsetbuttcap%
\pgfsetroundjoin%
\definecolor{currentfill}{rgb}{0.000000,0.000000,0.000000}%
\pgfsetfillcolor{currentfill}%
\pgfsetlinewidth{0.803000pt}%
\definecolor{currentstroke}{rgb}{0.000000,0.000000,0.000000}%
\pgfsetstrokecolor{currentstroke}%
\pgfsetdash{}{0pt}%
\pgfsys@defobject{currentmarker}{\pgfqpoint{-0.048611in}{0.000000in}}{\pgfqpoint{-0.000000in}{0.000000in}}{%
\pgfpathmoveto{\pgfqpoint{-0.000000in}{0.000000in}}%
\pgfpathlineto{\pgfqpoint{-0.048611in}{0.000000in}}%
\pgfusepath{stroke,fill}%
}%
\begin{pgfscope}%
\pgfsys@transformshift{0.445556in}{1.493228in}%
\pgfsys@useobject{currentmarker}{}%
\end{pgfscope}%
\end{pgfscope}%
\begin{pgfscope}%
\definecolor{textcolor}{rgb}{0.000000,0.000000,0.000000}%
\pgfsetstrokecolor{textcolor}%
\pgfsetfillcolor{textcolor}%
\pgftext[x=0.278889in, y=1.445033in, left, base]{\color{textcolor}\rmfamily\fontsize{10.000000}{12.000000}\selectfont \(\displaystyle {6}\)}%
\end{pgfscope}%
\begin{pgfscope}%
\definecolor{textcolor}{rgb}{0.000000,0.000000,0.000000}%
\pgfsetstrokecolor{textcolor}%
\pgfsetfillcolor{textcolor}%
\pgftext[x=0.223333in,y=1.076944in,,bottom,rotate=90.000000]{\color{textcolor}\rmfamily\fontsize{10.000000}{12.000000}\selectfont Percent of Data Set}%
\end{pgfscope}%
\begin{pgfscope}%
\pgfsetrectcap%
\pgfsetmiterjoin%
\pgfsetlinewidth{0.803000pt}%
\definecolor{currentstroke}{rgb}{0.000000,0.000000,0.000000}%
\pgfsetstrokecolor{currentstroke}%
\pgfsetdash{}{0pt}%
\pgfpathmoveto{\pgfqpoint{0.445556in}{0.499444in}}%
\pgfpathlineto{\pgfqpoint{0.445556in}{1.654444in}}%
\pgfusepath{stroke}%
\end{pgfscope}%
\begin{pgfscope}%
\pgfsetrectcap%
\pgfsetmiterjoin%
\pgfsetlinewidth{0.803000pt}%
\definecolor{currentstroke}{rgb}{0.000000,0.000000,0.000000}%
\pgfsetstrokecolor{currentstroke}%
\pgfsetdash{}{0pt}%
\pgfpathmoveto{\pgfqpoint{4.320556in}{0.499444in}}%
\pgfpathlineto{\pgfqpoint{4.320556in}{1.654444in}}%
\pgfusepath{stroke}%
\end{pgfscope}%
\begin{pgfscope}%
\pgfsetrectcap%
\pgfsetmiterjoin%
\pgfsetlinewidth{0.803000pt}%
\definecolor{currentstroke}{rgb}{0.000000,0.000000,0.000000}%
\pgfsetstrokecolor{currentstroke}%
\pgfsetdash{}{0pt}%
\pgfpathmoveto{\pgfqpoint{0.445556in}{0.499444in}}%
\pgfpathlineto{\pgfqpoint{4.320556in}{0.499444in}}%
\pgfusepath{stroke}%
\end{pgfscope}%
\begin{pgfscope}%
\pgfsetrectcap%
\pgfsetmiterjoin%
\pgfsetlinewidth{0.803000pt}%
\definecolor{currentstroke}{rgb}{0.000000,0.000000,0.000000}%
\pgfsetstrokecolor{currentstroke}%
\pgfsetdash{}{0pt}%
\pgfpathmoveto{\pgfqpoint{0.445556in}{1.654444in}}%
\pgfpathlineto{\pgfqpoint{4.320556in}{1.654444in}}%
\pgfusepath{stroke}%
\end{pgfscope}%
\begin{pgfscope}%
\pgfsetbuttcap%
\pgfsetmiterjoin%
\definecolor{currentfill}{rgb}{1.000000,1.000000,1.000000}%
\pgfsetfillcolor{currentfill}%
\pgfsetfillopacity{0.800000}%
\pgfsetlinewidth{1.003750pt}%
\definecolor{currentstroke}{rgb}{0.800000,0.800000,0.800000}%
\pgfsetstrokecolor{currentstroke}%
\pgfsetstrokeopacity{0.800000}%
\pgfsetdash{}{0pt}%
\pgfpathmoveto{\pgfqpoint{3.543611in}{1.154445in}}%
\pgfpathlineto{\pgfqpoint{4.223333in}{1.154445in}}%
\pgfpathquadraticcurveto{\pgfqpoint{4.251111in}{1.154445in}}{\pgfqpoint{4.251111in}{1.182222in}}%
\pgfpathlineto{\pgfqpoint{4.251111in}{1.557222in}}%
\pgfpathquadraticcurveto{\pgfqpoint{4.251111in}{1.585000in}}{\pgfqpoint{4.223333in}{1.585000in}}%
\pgfpathlineto{\pgfqpoint{3.543611in}{1.585000in}}%
\pgfpathquadraticcurveto{\pgfqpoint{3.515833in}{1.585000in}}{\pgfqpoint{3.515833in}{1.557222in}}%
\pgfpathlineto{\pgfqpoint{3.515833in}{1.182222in}}%
\pgfpathquadraticcurveto{\pgfqpoint{3.515833in}{1.154445in}}{\pgfqpoint{3.543611in}{1.154445in}}%
\pgfpathlineto{\pgfqpoint{3.543611in}{1.154445in}}%
\pgfpathclose%
\pgfusepath{stroke,fill}%
\end{pgfscope}%
\begin{pgfscope}%
\pgfsetbuttcap%
\pgfsetmiterjoin%
\pgfsetlinewidth{1.003750pt}%
\definecolor{currentstroke}{rgb}{0.000000,0.000000,0.000000}%
\pgfsetstrokecolor{currentstroke}%
\pgfsetdash{}{0pt}%
\pgfpathmoveto{\pgfqpoint{3.571389in}{1.432222in}}%
\pgfpathlineto{\pgfqpoint{3.849167in}{1.432222in}}%
\pgfpathlineto{\pgfqpoint{3.849167in}{1.529444in}}%
\pgfpathlineto{\pgfqpoint{3.571389in}{1.529444in}}%
\pgfpathlineto{\pgfqpoint{3.571389in}{1.432222in}}%
\pgfpathclose%
\pgfusepath{stroke}%
\end{pgfscope}%
\begin{pgfscope}%
\definecolor{textcolor}{rgb}{0.000000,0.000000,0.000000}%
\pgfsetstrokecolor{textcolor}%
\pgfsetfillcolor{textcolor}%
\pgftext[x=3.960278in,y=1.432222in,left,base]{\color{textcolor}\rmfamily\fontsize{10.000000}{12.000000}\selectfont Neg}%
\end{pgfscope}%
\begin{pgfscope}%
\pgfsetbuttcap%
\pgfsetmiterjoin%
\definecolor{currentfill}{rgb}{0.000000,0.000000,0.000000}%
\pgfsetfillcolor{currentfill}%
\pgfsetlinewidth{0.000000pt}%
\definecolor{currentstroke}{rgb}{0.000000,0.000000,0.000000}%
\pgfsetstrokecolor{currentstroke}%
\pgfsetstrokeopacity{0.000000}%
\pgfsetdash{}{0pt}%
\pgfpathmoveto{\pgfqpoint{3.571389in}{1.236944in}}%
\pgfpathlineto{\pgfqpoint{3.849167in}{1.236944in}}%
\pgfpathlineto{\pgfqpoint{3.849167in}{1.334167in}}%
\pgfpathlineto{\pgfqpoint{3.571389in}{1.334167in}}%
\pgfpathlineto{\pgfqpoint{3.571389in}{1.236944in}}%
\pgfpathclose%
\pgfusepath{fill}%
\end{pgfscope}%
\begin{pgfscope}%
\definecolor{textcolor}{rgb}{0.000000,0.000000,0.000000}%
\pgfsetstrokecolor{textcolor}%
\pgfsetfillcolor{textcolor}%
\pgftext[x=3.960278in,y=1.236944in,left,base]{\color{textcolor}\rmfamily\fontsize{10.000000}{12.000000}\selectfont Pos}%
\end{pgfscope}%
\end{pgfpicture}%
\makeatother%
\endgroup%
	
&
	\vskip 0pt
	\hfil ROC Curve
	
	%% Creator: Matplotlib, PGF backend
%%
%% To include the figure in your LaTeX document, write
%%   \input{<filename>.pgf}
%%
%% Make sure the required packages are loaded in your preamble
%%   \usepackage{pgf}
%%
%% Also ensure that all the required font packages are loaded; for instance,
%% the lmodern package is sometimes necessary when using math font.
%%   \usepackage{lmodern}
%%
%% Figures using additional raster images can only be included by \input if
%% they are in the same directory as the main LaTeX file. For loading figures
%% from other directories you can use the `import` package
%%   \usepackage{import}
%%
%% and then include the figures with
%%   \import{<path to file>}{<filename>.pgf}
%%
%% Matplotlib used the following preamble
%%   
%%   \usepackage{fontspec}
%%   \makeatletter\@ifpackageloaded{underscore}{}{\usepackage[strings]{underscore}}\makeatother
%%
\begingroup%
\makeatletter%
\begin{pgfpicture}%
\pgfpathrectangle{\pgfpointorigin}{\pgfqpoint{2.221861in}{1.754444in}}%
\pgfusepath{use as bounding box, clip}%
\begin{pgfscope}%
\pgfsetbuttcap%
\pgfsetmiterjoin%
\definecolor{currentfill}{rgb}{1.000000,1.000000,1.000000}%
\pgfsetfillcolor{currentfill}%
\pgfsetlinewidth{0.000000pt}%
\definecolor{currentstroke}{rgb}{1.000000,1.000000,1.000000}%
\pgfsetstrokecolor{currentstroke}%
\pgfsetdash{}{0pt}%
\pgfpathmoveto{\pgfqpoint{0.000000in}{0.000000in}}%
\pgfpathlineto{\pgfqpoint{2.221861in}{0.000000in}}%
\pgfpathlineto{\pgfqpoint{2.221861in}{1.754444in}}%
\pgfpathlineto{\pgfqpoint{0.000000in}{1.754444in}}%
\pgfpathlineto{\pgfqpoint{0.000000in}{0.000000in}}%
\pgfpathclose%
\pgfusepath{fill}%
\end{pgfscope}%
\begin{pgfscope}%
\pgfsetbuttcap%
\pgfsetmiterjoin%
\definecolor{currentfill}{rgb}{1.000000,1.000000,1.000000}%
\pgfsetfillcolor{currentfill}%
\pgfsetlinewidth{0.000000pt}%
\definecolor{currentstroke}{rgb}{0.000000,0.000000,0.000000}%
\pgfsetstrokecolor{currentstroke}%
\pgfsetstrokeopacity{0.000000}%
\pgfsetdash{}{0pt}%
\pgfpathmoveto{\pgfqpoint{0.553581in}{0.499444in}}%
\pgfpathlineto{\pgfqpoint{2.103581in}{0.499444in}}%
\pgfpathlineto{\pgfqpoint{2.103581in}{1.654444in}}%
\pgfpathlineto{\pgfqpoint{0.553581in}{1.654444in}}%
\pgfpathlineto{\pgfqpoint{0.553581in}{0.499444in}}%
\pgfpathclose%
\pgfusepath{fill}%
\end{pgfscope}%
\begin{pgfscope}%
\pgfsetbuttcap%
\pgfsetroundjoin%
\definecolor{currentfill}{rgb}{0.000000,0.000000,0.000000}%
\pgfsetfillcolor{currentfill}%
\pgfsetlinewidth{0.803000pt}%
\definecolor{currentstroke}{rgb}{0.000000,0.000000,0.000000}%
\pgfsetstrokecolor{currentstroke}%
\pgfsetdash{}{0pt}%
\pgfsys@defobject{currentmarker}{\pgfqpoint{0.000000in}{-0.048611in}}{\pgfqpoint{0.000000in}{0.000000in}}{%
\pgfpathmoveto{\pgfqpoint{0.000000in}{0.000000in}}%
\pgfpathlineto{\pgfqpoint{0.000000in}{-0.048611in}}%
\pgfusepath{stroke,fill}%
}%
\begin{pgfscope}%
\pgfsys@transformshift{0.624035in}{0.499444in}%
\pgfsys@useobject{currentmarker}{}%
\end{pgfscope}%
\end{pgfscope}%
\begin{pgfscope}%
\definecolor{textcolor}{rgb}{0.000000,0.000000,0.000000}%
\pgfsetstrokecolor{textcolor}%
\pgfsetfillcolor{textcolor}%
\pgftext[x=0.624035in,y=0.402222in,,top]{\color{textcolor}\rmfamily\fontsize{10.000000}{12.000000}\selectfont \(\displaystyle {0.0}\)}%
\end{pgfscope}%
\begin{pgfscope}%
\pgfsetbuttcap%
\pgfsetroundjoin%
\definecolor{currentfill}{rgb}{0.000000,0.000000,0.000000}%
\pgfsetfillcolor{currentfill}%
\pgfsetlinewidth{0.803000pt}%
\definecolor{currentstroke}{rgb}{0.000000,0.000000,0.000000}%
\pgfsetstrokecolor{currentstroke}%
\pgfsetdash{}{0pt}%
\pgfsys@defobject{currentmarker}{\pgfqpoint{0.000000in}{-0.048611in}}{\pgfqpoint{0.000000in}{0.000000in}}{%
\pgfpathmoveto{\pgfqpoint{0.000000in}{0.000000in}}%
\pgfpathlineto{\pgfqpoint{0.000000in}{-0.048611in}}%
\pgfusepath{stroke,fill}%
}%
\begin{pgfscope}%
\pgfsys@transformshift{1.328581in}{0.499444in}%
\pgfsys@useobject{currentmarker}{}%
\end{pgfscope}%
\end{pgfscope}%
\begin{pgfscope}%
\definecolor{textcolor}{rgb}{0.000000,0.000000,0.000000}%
\pgfsetstrokecolor{textcolor}%
\pgfsetfillcolor{textcolor}%
\pgftext[x=1.328581in,y=0.402222in,,top]{\color{textcolor}\rmfamily\fontsize{10.000000}{12.000000}\selectfont \(\displaystyle {0.5}\)}%
\end{pgfscope}%
\begin{pgfscope}%
\pgfsetbuttcap%
\pgfsetroundjoin%
\definecolor{currentfill}{rgb}{0.000000,0.000000,0.000000}%
\pgfsetfillcolor{currentfill}%
\pgfsetlinewidth{0.803000pt}%
\definecolor{currentstroke}{rgb}{0.000000,0.000000,0.000000}%
\pgfsetstrokecolor{currentstroke}%
\pgfsetdash{}{0pt}%
\pgfsys@defobject{currentmarker}{\pgfqpoint{0.000000in}{-0.048611in}}{\pgfqpoint{0.000000in}{0.000000in}}{%
\pgfpathmoveto{\pgfqpoint{0.000000in}{0.000000in}}%
\pgfpathlineto{\pgfqpoint{0.000000in}{-0.048611in}}%
\pgfusepath{stroke,fill}%
}%
\begin{pgfscope}%
\pgfsys@transformshift{2.033126in}{0.499444in}%
\pgfsys@useobject{currentmarker}{}%
\end{pgfscope}%
\end{pgfscope}%
\begin{pgfscope}%
\definecolor{textcolor}{rgb}{0.000000,0.000000,0.000000}%
\pgfsetstrokecolor{textcolor}%
\pgfsetfillcolor{textcolor}%
\pgftext[x=2.033126in,y=0.402222in,,top]{\color{textcolor}\rmfamily\fontsize{10.000000}{12.000000}\selectfont \(\displaystyle {1.0}\)}%
\end{pgfscope}%
\begin{pgfscope}%
\definecolor{textcolor}{rgb}{0.000000,0.000000,0.000000}%
\pgfsetstrokecolor{textcolor}%
\pgfsetfillcolor{textcolor}%
\pgftext[x=1.328581in,y=0.223333in,,top]{\color{textcolor}\rmfamily\fontsize{10.000000}{12.000000}\selectfont False positive rate}%
\end{pgfscope}%
\begin{pgfscope}%
\pgfsetbuttcap%
\pgfsetroundjoin%
\definecolor{currentfill}{rgb}{0.000000,0.000000,0.000000}%
\pgfsetfillcolor{currentfill}%
\pgfsetlinewidth{0.803000pt}%
\definecolor{currentstroke}{rgb}{0.000000,0.000000,0.000000}%
\pgfsetstrokecolor{currentstroke}%
\pgfsetdash{}{0pt}%
\pgfsys@defobject{currentmarker}{\pgfqpoint{-0.048611in}{0.000000in}}{\pgfqpoint{-0.000000in}{0.000000in}}{%
\pgfpathmoveto{\pgfqpoint{-0.000000in}{0.000000in}}%
\pgfpathlineto{\pgfqpoint{-0.048611in}{0.000000in}}%
\pgfusepath{stroke,fill}%
}%
\begin{pgfscope}%
\pgfsys@transformshift{0.553581in}{0.551944in}%
\pgfsys@useobject{currentmarker}{}%
\end{pgfscope}%
\end{pgfscope}%
\begin{pgfscope}%
\definecolor{textcolor}{rgb}{0.000000,0.000000,0.000000}%
\pgfsetstrokecolor{textcolor}%
\pgfsetfillcolor{textcolor}%
\pgftext[x=0.278889in, y=0.503750in, left, base]{\color{textcolor}\rmfamily\fontsize{10.000000}{12.000000}\selectfont \(\displaystyle {0.0}\)}%
\end{pgfscope}%
\begin{pgfscope}%
\pgfsetbuttcap%
\pgfsetroundjoin%
\definecolor{currentfill}{rgb}{0.000000,0.000000,0.000000}%
\pgfsetfillcolor{currentfill}%
\pgfsetlinewidth{0.803000pt}%
\definecolor{currentstroke}{rgb}{0.000000,0.000000,0.000000}%
\pgfsetstrokecolor{currentstroke}%
\pgfsetdash{}{0pt}%
\pgfsys@defobject{currentmarker}{\pgfqpoint{-0.048611in}{0.000000in}}{\pgfqpoint{-0.000000in}{0.000000in}}{%
\pgfpathmoveto{\pgfqpoint{-0.000000in}{0.000000in}}%
\pgfpathlineto{\pgfqpoint{-0.048611in}{0.000000in}}%
\pgfusepath{stroke,fill}%
}%
\begin{pgfscope}%
\pgfsys@transformshift{0.553581in}{1.076944in}%
\pgfsys@useobject{currentmarker}{}%
\end{pgfscope}%
\end{pgfscope}%
\begin{pgfscope}%
\definecolor{textcolor}{rgb}{0.000000,0.000000,0.000000}%
\pgfsetstrokecolor{textcolor}%
\pgfsetfillcolor{textcolor}%
\pgftext[x=0.278889in, y=1.028750in, left, base]{\color{textcolor}\rmfamily\fontsize{10.000000}{12.000000}\selectfont \(\displaystyle {0.5}\)}%
\end{pgfscope}%
\begin{pgfscope}%
\pgfsetbuttcap%
\pgfsetroundjoin%
\definecolor{currentfill}{rgb}{0.000000,0.000000,0.000000}%
\pgfsetfillcolor{currentfill}%
\pgfsetlinewidth{0.803000pt}%
\definecolor{currentstroke}{rgb}{0.000000,0.000000,0.000000}%
\pgfsetstrokecolor{currentstroke}%
\pgfsetdash{}{0pt}%
\pgfsys@defobject{currentmarker}{\pgfqpoint{-0.048611in}{0.000000in}}{\pgfqpoint{-0.000000in}{0.000000in}}{%
\pgfpathmoveto{\pgfqpoint{-0.000000in}{0.000000in}}%
\pgfpathlineto{\pgfqpoint{-0.048611in}{0.000000in}}%
\pgfusepath{stroke,fill}%
}%
\begin{pgfscope}%
\pgfsys@transformshift{0.553581in}{1.601944in}%
\pgfsys@useobject{currentmarker}{}%
\end{pgfscope}%
\end{pgfscope}%
\begin{pgfscope}%
\definecolor{textcolor}{rgb}{0.000000,0.000000,0.000000}%
\pgfsetstrokecolor{textcolor}%
\pgfsetfillcolor{textcolor}%
\pgftext[x=0.278889in, y=1.553750in, left, base]{\color{textcolor}\rmfamily\fontsize{10.000000}{12.000000}\selectfont \(\displaystyle {1.0}\)}%
\end{pgfscope}%
\begin{pgfscope}%
\definecolor{textcolor}{rgb}{0.000000,0.000000,0.000000}%
\pgfsetstrokecolor{textcolor}%
\pgfsetfillcolor{textcolor}%
\pgftext[x=0.223333in,y=1.076944in,,bottom,rotate=90.000000]{\color{textcolor}\rmfamily\fontsize{10.000000}{12.000000}\selectfont True positive rate}%
\end{pgfscope}%
\begin{pgfscope}%
\pgfpathrectangle{\pgfqpoint{0.553581in}{0.499444in}}{\pgfqpoint{1.550000in}{1.155000in}}%
\pgfusepath{clip}%
\pgfsetbuttcap%
\pgfsetroundjoin%
\pgfsetlinewidth{1.505625pt}%
\definecolor{currentstroke}{rgb}{0.000000,0.000000,0.000000}%
\pgfsetstrokecolor{currentstroke}%
\pgfsetdash{{5.550000pt}{2.400000pt}}{0.000000pt}%
\pgfpathmoveto{\pgfqpoint{0.624035in}{0.551944in}}%
\pgfpathlineto{\pgfqpoint{2.033126in}{1.601944in}}%
\pgfusepath{stroke}%
\end{pgfscope}%
\begin{pgfscope}%
\pgfpathrectangle{\pgfqpoint{0.553581in}{0.499444in}}{\pgfqpoint{1.550000in}{1.155000in}}%
\pgfusepath{clip}%
\pgfsetrectcap%
\pgfsetroundjoin%
\pgfsetlinewidth{1.505625pt}%
\definecolor{currentstroke}{rgb}{0.000000,0.000000,0.000000}%
\pgfsetstrokecolor{currentstroke}%
\pgfsetdash{}{0pt}%
\pgfpathmoveto{\pgfqpoint{0.624035in}{0.551944in}}%
\pgfpathlineto{\pgfqpoint{0.625818in}{0.578454in}}%
\pgfpathlineto{\pgfqpoint{0.635863in}{0.672977in}}%
\pgfpathlineto{\pgfqpoint{0.642954in}{0.717523in}}%
\pgfpathlineto{\pgfqpoint{0.652061in}{0.766011in}}%
\pgfpathlineto{\pgfqpoint{0.660082in}{0.797456in}}%
\pgfpathlineto{\pgfqpoint{0.680307in}{0.866246in}}%
\pgfpathlineto{\pgfqpoint{0.692854in}{0.901510in}}%
\pgfpathlineto{\pgfqpoint{0.692924in}{0.901789in}}%
\pgfpathlineto{\pgfqpoint{0.715377in}{0.955057in}}%
\pgfpathlineto{\pgfqpoint{0.723835in}{0.974645in}}%
\pgfpathlineto{\pgfqpoint{0.765613in}{1.048215in}}%
\pgfpathlineto{\pgfqpoint{0.817389in}{1.123306in}}%
\pgfpathlineto{\pgfqpoint{0.847119in}{1.160680in}}%
\pgfpathlineto{\pgfqpoint{0.879485in}{1.196068in}}%
\pgfpathlineto{\pgfqpoint{0.880837in}{1.197248in}}%
\pgfpathlineto{\pgfqpoint{0.881064in}{1.197465in}}%
\pgfpathlineto{\pgfqpoint{0.936850in}{1.250454in}}%
\pgfpathlineto{\pgfqpoint{0.977643in}{1.283607in}}%
\pgfpathlineto{\pgfqpoint{0.999563in}{1.300153in}}%
\pgfpathlineto{\pgfqpoint{1.092633in}{1.363292in}}%
\pgfpathlineto{\pgfqpoint{1.116703in}{1.376827in}}%
\pgfpathlineto{\pgfqpoint{1.141766in}{1.389989in}}%
\pgfpathlineto{\pgfqpoint{1.167416in}{1.404609in}}%
\pgfpathlineto{\pgfqpoint{1.246452in}{1.440215in}}%
\pgfpathlineto{\pgfqpoint{1.273478in}{1.452507in}}%
\pgfpathlineto{\pgfqpoint{1.299753in}{1.462689in}}%
\pgfpathlineto{\pgfqpoint{1.438524in}{1.510618in}}%
\pgfpathlineto{\pgfqpoint{1.465792in}{1.519403in}}%
\pgfpathlineto{\pgfqpoint{1.521602in}{1.533651in}}%
\pgfpathlineto{\pgfqpoint{1.576935in}{1.547000in}}%
\pgfpathlineto{\pgfqpoint{1.707928in}{1.571678in}}%
\pgfpathlineto{\pgfqpoint{1.802459in}{1.584933in}}%
\pgfpathlineto{\pgfqpoint{1.865414in}{1.591762in}}%
\pgfpathlineto{\pgfqpoint{1.936570in}{1.597629in}}%
\pgfpathlineto{\pgfqpoint{1.989793in}{1.600392in}}%
\pgfpathlineto{\pgfqpoint{2.033126in}{1.601944in}}%
\pgfpathlineto{\pgfqpoint{2.033126in}{1.601944in}}%
\pgfusepath{stroke}%
\end{pgfscope}%
\begin{pgfscope}%
\pgfsetrectcap%
\pgfsetmiterjoin%
\pgfsetlinewidth{0.803000pt}%
\definecolor{currentstroke}{rgb}{0.000000,0.000000,0.000000}%
\pgfsetstrokecolor{currentstroke}%
\pgfsetdash{}{0pt}%
\pgfpathmoveto{\pgfqpoint{0.553581in}{0.499444in}}%
\pgfpathlineto{\pgfqpoint{0.553581in}{1.654444in}}%
\pgfusepath{stroke}%
\end{pgfscope}%
\begin{pgfscope}%
\pgfsetrectcap%
\pgfsetmiterjoin%
\pgfsetlinewidth{0.803000pt}%
\definecolor{currentstroke}{rgb}{0.000000,0.000000,0.000000}%
\pgfsetstrokecolor{currentstroke}%
\pgfsetdash{}{0pt}%
\pgfpathmoveto{\pgfqpoint{2.103581in}{0.499444in}}%
\pgfpathlineto{\pgfqpoint{2.103581in}{1.654444in}}%
\pgfusepath{stroke}%
\end{pgfscope}%
\begin{pgfscope}%
\pgfsetrectcap%
\pgfsetmiterjoin%
\pgfsetlinewidth{0.803000pt}%
\definecolor{currentstroke}{rgb}{0.000000,0.000000,0.000000}%
\pgfsetstrokecolor{currentstroke}%
\pgfsetdash{}{0pt}%
\pgfpathmoveto{\pgfqpoint{0.553581in}{0.499444in}}%
\pgfpathlineto{\pgfqpoint{2.103581in}{0.499444in}}%
\pgfusepath{stroke}%
\end{pgfscope}%
\begin{pgfscope}%
\pgfsetrectcap%
\pgfsetmiterjoin%
\pgfsetlinewidth{0.803000pt}%
\definecolor{currentstroke}{rgb}{0.000000,0.000000,0.000000}%
\pgfsetstrokecolor{currentstroke}%
\pgfsetdash{}{0pt}%
\pgfpathmoveto{\pgfqpoint{0.553581in}{1.654444in}}%
\pgfpathlineto{\pgfqpoint{2.103581in}{1.654444in}}%
\pgfusepath{stroke}%
\end{pgfscope}%
\begin{pgfscope}%
\pgfsetbuttcap%
\pgfsetmiterjoin%
\definecolor{currentfill}{rgb}{1.000000,1.000000,1.000000}%
\pgfsetfillcolor{currentfill}%
\pgfsetfillopacity{0.800000}%
\pgfsetlinewidth{1.003750pt}%
\definecolor{currentstroke}{rgb}{0.800000,0.800000,0.800000}%
\pgfsetstrokecolor{currentstroke}%
\pgfsetstrokeopacity{0.800000}%
\pgfsetdash{}{0pt}%
\pgfpathmoveto{\pgfqpoint{0.832747in}{0.568889in}}%
\pgfpathlineto{\pgfqpoint{2.006358in}{0.568889in}}%
\pgfpathquadraticcurveto{\pgfqpoint{2.034136in}{0.568889in}}{\pgfqpoint{2.034136in}{0.596666in}}%
\pgfpathlineto{\pgfqpoint{2.034136in}{0.776388in}}%
\pgfpathquadraticcurveto{\pgfqpoint{2.034136in}{0.804166in}}{\pgfqpoint{2.006358in}{0.804166in}}%
\pgfpathlineto{\pgfqpoint{0.832747in}{0.804166in}}%
\pgfpathquadraticcurveto{\pgfqpoint{0.804970in}{0.804166in}}{\pgfqpoint{0.804970in}{0.776388in}}%
\pgfpathlineto{\pgfqpoint{0.804970in}{0.596666in}}%
\pgfpathquadraticcurveto{\pgfqpoint{0.804970in}{0.568889in}}{\pgfqpoint{0.832747in}{0.568889in}}%
\pgfpathlineto{\pgfqpoint{0.832747in}{0.568889in}}%
\pgfpathclose%
\pgfusepath{stroke,fill}%
\end{pgfscope}%
\begin{pgfscope}%
\pgfsetrectcap%
\pgfsetroundjoin%
\pgfsetlinewidth{1.505625pt}%
\definecolor{currentstroke}{rgb}{0.000000,0.000000,0.000000}%
\pgfsetstrokecolor{currentstroke}%
\pgfsetdash{}{0pt}%
\pgfpathmoveto{\pgfqpoint{0.860525in}{0.700000in}}%
\pgfpathlineto{\pgfqpoint{0.999414in}{0.700000in}}%
\pgfpathlineto{\pgfqpoint{1.138303in}{0.700000in}}%
\pgfusepath{stroke}%
\end{pgfscope}%
\begin{pgfscope}%
\definecolor{textcolor}{rgb}{0.000000,0.000000,0.000000}%
\pgfsetstrokecolor{textcolor}%
\pgfsetfillcolor{textcolor}%
\pgftext[x=1.249414in,y=0.651388in,left,base]{\color{textcolor}\rmfamily\fontsize{10.000000}{12.000000}\selectfont AUC=0.799}%
\end{pgfscope}%
\end{pgfpicture}%
\makeatother%
\endgroup%

\end{tabular}

\

%
\verb|KBFC_Hard_Tomek_0_alpha_0_5_gamma_0_0_v1_Test|

\

This model is almost as effective at separating the two classes ($\text{ROC}=0.778$), but the distribution is skewed to the left.  Its results were nearly continuous, with the 214,070 samples returning 210,157 unique values of $p$, so we can fine tune the decision threshold.  

\noindent\begin{tabular}{@{\hspace{-6pt}}p{4.3in} @{\hspace{-6pt}}p{2.0in}}
	\vskip 0pt
	\hfil Raw Model Output
	
	%% Creator: Matplotlib, PGF backend
%%
%% To include the figure in your LaTeX document, write
%%   \input{<filename>.pgf}
%%
%% Make sure the required packages are loaded in your preamble
%%   \usepackage{pgf}
%%
%% Also ensure that all the required font packages are loaded; for instance,
%% the lmodern package is sometimes necessary when using math font.
%%   \usepackage{lmodern}
%%
%% Figures using additional raster images can only be included by \input if
%% they are in the same directory as the main LaTeX file. For loading figures
%% from other directories you can use the `import` package
%%   \usepackage{import}
%%
%% and then include the figures with
%%   \import{<path to file>}{<filename>.pgf}
%%
%% Matplotlib used the following preamble
%%   
%%   \usepackage{fontspec}
%%   \makeatletter\@ifpackageloaded{underscore}{}{\usepackage[strings]{underscore}}\makeatother
%%
\begingroup%
\makeatletter%
\begin{pgfpicture}%
\pgfpathrectangle{\pgfpointorigin}{\pgfqpoint{4.102500in}{1.754444in}}%
\pgfusepath{use as bounding box, clip}%
\begin{pgfscope}%
\pgfsetbuttcap%
\pgfsetmiterjoin%
\definecolor{currentfill}{rgb}{1.000000,1.000000,1.000000}%
\pgfsetfillcolor{currentfill}%
\pgfsetlinewidth{0.000000pt}%
\definecolor{currentstroke}{rgb}{1.000000,1.000000,1.000000}%
\pgfsetstrokecolor{currentstroke}%
\pgfsetdash{}{0pt}%
\pgfpathmoveto{\pgfqpoint{0.000000in}{0.000000in}}%
\pgfpathlineto{\pgfqpoint{4.102500in}{0.000000in}}%
\pgfpathlineto{\pgfqpoint{4.102500in}{1.754444in}}%
\pgfpathlineto{\pgfqpoint{0.000000in}{1.754444in}}%
\pgfpathlineto{\pgfqpoint{0.000000in}{0.000000in}}%
\pgfpathclose%
\pgfusepath{fill}%
\end{pgfscope}%
\begin{pgfscope}%
\pgfsetbuttcap%
\pgfsetmiterjoin%
\definecolor{currentfill}{rgb}{1.000000,1.000000,1.000000}%
\pgfsetfillcolor{currentfill}%
\pgfsetlinewidth{0.000000pt}%
\definecolor{currentstroke}{rgb}{0.000000,0.000000,0.000000}%
\pgfsetstrokecolor{currentstroke}%
\pgfsetstrokeopacity{0.000000}%
\pgfsetdash{}{0pt}%
\pgfpathmoveto{\pgfqpoint{0.515000in}{0.499444in}}%
\pgfpathlineto{\pgfqpoint{4.002500in}{0.499444in}}%
\pgfpathlineto{\pgfqpoint{4.002500in}{1.654444in}}%
\pgfpathlineto{\pgfqpoint{0.515000in}{1.654444in}}%
\pgfpathlineto{\pgfqpoint{0.515000in}{0.499444in}}%
\pgfpathclose%
\pgfusepath{fill}%
\end{pgfscope}%
\begin{pgfscope}%
\pgfpathrectangle{\pgfqpoint{0.515000in}{0.499444in}}{\pgfqpoint{3.487500in}{1.155000in}}%
\pgfusepath{clip}%
\pgfsetbuttcap%
\pgfsetmiterjoin%
\pgfsetlinewidth{1.003750pt}%
\definecolor{currentstroke}{rgb}{0.000000,0.000000,0.000000}%
\pgfsetstrokecolor{currentstroke}%
\pgfsetdash{}{0pt}%
\pgfpathmoveto{\pgfqpoint{0.610114in}{0.499444in}}%
\pgfpathlineto{\pgfqpoint{0.673523in}{0.499444in}}%
\pgfpathlineto{\pgfqpoint{0.673523in}{0.499444in}}%
\pgfpathlineto{\pgfqpoint{0.610114in}{0.499444in}}%
\pgfpathlineto{\pgfqpoint{0.610114in}{0.499444in}}%
\pgfpathclose%
\pgfusepath{stroke}%
\end{pgfscope}%
\begin{pgfscope}%
\pgfpathrectangle{\pgfqpoint{0.515000in}{0.499444in}}{\pgfqpoint{3.487500in}{1.155000in}}%
\pgfusepath{clip}%
\pgfsetbuttcap%
\pgfsetmiterjoin%
\pgfsetlinewidth{1.003750pt}%
\definecolor{currentstroke}{rgb}{0.000000,0.000000,0.000000}%
\pgfsetstrokecolor{currentstroke}%
\pgfsetdash{}{0pt}%
\pgfpathmoveto{\pgfqpoint{0.768637in}{0.499444in}}%
\pgfpathlineto{\pgfqpoint{0.832046in}{0.499444in}}%
\pgfpathlineto{\pgfqpoint{0.832046in}{1.599444in}}%
\pgfpathlineto{\pgfqpoint{0.768637in}{1.599444in}}%
\pgfpathlineto{\pgfqpoint{0.768637in}{0.499444in}}%
\pgfpathclose%
\pgfusepath{stroke}%
\end{pgfscope}%
\begin{pgfscope}%
\pgfpathrectangle{\pgfqpoint{0.515000in}{0.499444in}}{\pgfqpoint{3.487500in}{1.155000in}}%
\pgfusepath{clip}%
\pgfsetbuttcap%
\pgfsetmiterjoin%
\pgfsetlinewidth{1.003750pt}%
\definecolor{currentstroke}{rgb}{0.000000,0.000000,0.000000}%
\pgfsetstrokecolor{currentstroke}%
\pgfsetdash{}{0pt}%
\pgfpathmoveto{\pgfqpoint{0.927159in}{0.499444in}}%
\pgfpathlineto{\pgfqpoint{0.990568in}{0.499444in}}%
\pgfpathlineto{\pgfqpoint{0.990568in}{1.346951in}}%
\pgfpathlineto{\pgfqpoint{0.927159in}{1.346951in}}%
\pgfpathlineto{\pgfqpoint{0.927159in}{0.499444in}}%
\pgfpathclose%
\pgfusepath{stroke}%
\end{pgfscope}%
\begin{pgfscope}%
\pgfpathrectangle{\pgfqpoint{0.515000in}{0.499444in}}{\pgfqpoint{3.487500in}{1.155000in}}%
\pgfusepath{clip}%
\pgfsetbuttcap%
\pgfsetmiterjoin%
\pgfsetlinewidth{1.003750pt}%
\definecolor{currentstroke}{rgb}{0.000000,0.000000,0.000000}%
\pgfsetstrokecolor{currentstroke}%
\pgfsetdash{}{0pt}%
\pgfpathmoveto{\pgfqpoint{1.085682in}{0.499444in}}%
\pgfpathlineto{\pgfqpoint{1.149091in}{0.499444in}}%
\pgfpathlineto{\pgfqpoint{1.149091in}{1.032251in}}%
\pgfpathlineto{\pgfqpoint{1.085682in}{1.032251in}}%
\pgfpathlineto{\pgfqpoint{1.085682in}{0.499444in}}%
\pgfpathclose%
\pgfusepath{stroke}%
\end{pgfscope}%
\begin{pgfscope}%
\pgfpathrectangle{\pgfqpoint{0.515000in}{0.499444in}}{\pgfqpoint{3.487500in}{1.155000in}}%
\pgfusepath{clip}%
\pgfsetbuttcap%
\pgfsetmiterjoin%
\pgfsetlinewidth{1.003750pt}%
\definecolor{currentstroke}{rgb}{0.000000,0.000000,0.000000}%
\pgfsetstrokecolor{currentstroke}%
\pgfsetdash{}{0pt}%
\pgfpathmoveto{\pgfqpoint{1.244205in}{0.499444in}}%
\pgfpathlineto{\pgfqpoint{1.307614in}{0.499444in}}%
\pgfpathlineto{\pgfqpoint{1.307614in}{0.848431in}}%
\pgfpathlineto{\pgfqpoint{1.244205in}{0.848431in}}%
\pgfpathlineto{\pgfqpoint{1.244205in}{0.499444in}}%
\pgfpathclose%
\pgfusepath{stroke}%
\end{pgfscope}%
\begin{pgfscope}%
\pgfpathrectangle{\pgfqpoint{0.515000in}{0.499444in}}{\pgfqpoint{3.487500in}{1.155000in}}%
\pgfusepath{clip}%
\pgfsetbuttcap%
\pgfsetmiterjoin%
\pgfsetlinewidth{1.003750pt}%
\definecolor{currentstroke}{rgb}{0.000000,0.000000,0.000000}%
\pgfsetstrokecolor{currentstroke}%
\pgfsetdash{}{0pt}%
\pgfpathmoveto{\pgfqpoint{1.402728in}{0.499444in}}%
\pgfpathlineto{\pgfqpoint{1.466137in}{0.499444in}}%
\pgfpathlineto{\pgfqpoint{1.466137in}{0.740608in}}%
\pgfpathlineto{\pgfqpoint{1.402728in}{0.740608in}}%
\pgfpathlineto{\pgfqpoint{1.402728in}{0.499444in}}%
\pgfpathclose%
\pgfusepath{stroke}%
\end{pgfscope}%
\begin{pgfscope}%
\pgfpathrectangle{\pgfqpoint{0.515000in}{0.499444in}}{\pgfqpoint{3.487500in}{1.155000in}}%
\pgfusepath{clip}%
\pgfsetbuttcap%
\pgfsetmiterjoin%
\pgfsetlinewidth{1.003750pt}%
\definecolor{currentstroke}{rgb}{0.000000,0.000000,0.000000}%
\pgfsetstrokecolor{currentstroke}%
\pgfsetdash{}{0pt}%
\pgfpathmoveto{\pgfqpoint{1.561250in}{0.499444in}}%
\pgfpathlineto{\pgfqpoint{1.624659in}{0.499444in}}%
\pgfpathlineto{\pgfqpoint{1.624659in}{0.660907in}}%
\pgfpathlineto{\pgfqpoint{1.561250in}{0.660907in}}%
\pgfpathlineto{\pgfqpoint{1.561250in}{0.499444in}}%
\pgfpathclose%
\pgfusepath{stroke}%
\end{pgfscope}%
\begin{pgfscope}%
\pgfpathrectangle{\pgfqpoint{0.515000in}{0.499444in}}{\pgfqpoint{3.487500in}{1.155000in}}%
\pgfusepath{clip}%
\pgfsetbuttcap%
\pgfsetmiterjoin%
\pgfsetlinewidth{1.003750pt}%
\definecolor{currentstroke}{rgb}{0.000000,0.000000,0.000000}%
\pgfsetstrokecolor{currentstroke}%
\pgfsetdash{}{0pt}%
\pgfpathmoveto{\pgfqpoint{1.719773in}{0.499444in}}%
\pgfpathlineto{\pgfqpoint{1.783182in}{0.499444in}}%
\pgfpathlineto{\pgfqpoint{1.783182in}{0.611530in}}%
\pgfpathlineto{\pgfqpoint{1.719773in}{0.611530in}}%
\pgfpathlineto{\pgfqpoint{1.719773in}{0.499444in}}%
\pgfpathclose%
\pgfusepath{stroke}%
\end{pgfscope}%
\begin{pgfscope}%
\pgfpathrectangle{\pgfqpoint{0.515000in}{0.499444in}}{\pgfqpoint{3.487500in}{1.155000in}}%
\pgfusepath{clip}%
\pgfsetbuttcap%
\pgfsetmiterjoin%
\pgfsetlinewidth{1.003750pt}%
\definecolor{currentstroke}{rgb}{0.000000,0.000000,0.000000}%
\pgfsetstrokecolor{currentstroke}%
\pgfsetdash{}{0pt}%
\pgfpathmoveto{\pgfqpoint{1.878296in}{0.499444in}}%
\pgfpathlineto{\pgfqpoint{1.941705in}{0.499444in}}%
\pgfpathlineto{\pgfqpoint{1.941705in}{0.577063in}}%
\pgfpathlineto{\pgfqpoint{1.878296in}{0.577063in}}%
\pgfpathlineto{\pgfqpoint{1.878296in}{0.499444in}}%
\pgfpathclose%
\pgfusepath{stroke}%
\end{pgfscope}%
\begin{pgfscope}%
\pgfpathrectangle{\pgfqpoint{0.515000in}{0.499444in}}{\pgfqpoint{3.487500in}{1.155000in}}%
\pgfusepath{clip}%
\pgfsetbuttcap%
\pgfsetmiterjoin%
\pgfsetlinewidth{1.003750pt}%
\definecolor{currentstroke}{rgb}{0.000000,0.000000,0.000000}%
\pgfsetstrokecolor{currentstroke}%
\pgfsetdash{}{0pt}%
\pgfpathmoveto{\pgfqpoint{2.036818in}{0.499444in}}%
\pgfpathlineto{\pgfqpoint{2.100228in}{0.499444in}}%
\pgfpathlineto{\pgfqpoint{2.100228in}{0.554326in}}%
\pgfpathlineto{\pgfqpoint{2.036818in}{0.554326in}}%
\pgfpathlineto{\pgfqpoint{2.036818in}{0.499444in}}%
\pgfpathclose%
\pgfusepath{stroke}%
\end{pgfscope}%
\begin{pgfscope}%
\pgfpathrectangle{\pgfqpoint{0.515000in}{0.499444in}}{\pgfqpoint{3.487500in}{1.155000in}}%
\pgfusepath{clip}%
\pgfsetbuttcap%
\pgfsetmiterjoin%
\pgfsetlinewidth{1.003750pt}%
\definecolor{currentstroke}{rgb}{0.000000,0.000000,0.000000}%
\pgfsetstrokecolor{currentstroke}%
\pgfsetdash{}{0pt}%
\pgfpathmoveto{\pgfqpoint{2.195341in}{0.499444in}}%
\pgfpathlineto{\pgfqpoint{2.258750in}{0.499444in}}%
\pgfpathlineto{\pgfqpoint{2.258750in}{0.536953in}}%
\pgfpathlineto{\pgfqpoint{2.195341in}{0.536953in}}%
\pgfpathlineto{\pgfqpoint{2.195341in}{0.499444in}}%
\pgfpathclose%
\pgfusepath{stroke}%
\end{pgfscope}%
\begin{pgfscope}%
\pgfpathrectangle{\pgfqpoint{0.515000in}{0.499444in}}{\pgfqpoint{3.487500in}{1.155000in}}%
\pgfusepath{clip}%
\pgfsetbuttcap%
\pgfsetmiterjoin%
\pgfsetlinewidth{1.003750pt}%
\definecolor{currentstroke}{rgb}{0.000000,0.000000,0.000000}%
\pgfsetstrokecolor{currentstroke}%
\pgfsetdash{}{0pt}%
\pgfpathmoveto{\pgfqpoint{2.353864in}{0.499444in}}%
\pgfpathlineto{\pgfqpoint{2.417273in}{0.499444in}}%
\pgfpathlineto{\pgfqpoint{2.417273in}{0.528186in}}%
\pgfpathlineto{\pgfqpoint{2.353864in}{0.528186in}}%
\pgfpathlineto{\pgfqpoint{2.353864in}{0.499444in}}%
\pgfpathclose%
\pgfusepath{stroke}%
\end{pgfscope}%
\begin{pgfscope}%
\pgfpathrectangle{\pgfqpoint{0.515000in}{0.499444in}}{\pgfqpoint{3.487500in}{1.155000in}}%
\pgfusepath{clip}%
\pgfsetbuttcap%
\pgfsetmiterjoin%
\pgfsetlinewidth{1.003750pt}%
\definecolor{currentstroke}{rgb}{0.000000,0.000000,0.000000}%
\pgfsetstrokecolor{currentstroke}%
\pgfsetdash{}{0pt}%
\pgfpathmoveto{\pgfqpoint{2.512387in}{0.499444in}}%
\pgfpathlineto{\pgfqpoint{2.575796in}{0.499444in}}%
\pgfpathlineto{\pgfqpoint{2.575796in}{0.519540in}}%
\pgfpathlineto{\pgfqpoint{2.512387in}{0.519540in}}%
\pgfpathlineto{\pgfqpoint{2.512387in}{0.499444in}}%
\pgfpathclose%
\pgfusepath{stroke}%
\end{pgfscope}%
\begin{pgfscope}%
\pgfpathrectangle{\pgfqpoint{0.515000in}{0.499444in}}{\pgfqpoint{3.487500in}{1.155000in}}%
\pgfusepath{clip}%
\pgfsetbuttcap%
\pgfsetmiterjoin%
\pgfsetlinewidth{1.003750pt}%
\definecolor{currentstroke}{rgb}{0.000000,0.000000,0.000000}%
\pgfsetstrokecolor{currentstroke}%
\pgfsetdash{}{0pt}%
\pgfpathmoveto{\pgfqpoint{2.670909in}{0.499444in}}%
\pgfpathlineto{\pgfqpoint{2.734318in}{0.499444in}}%
\pgfpathlineto{\pgfqpoint{2.734318in}{0.513475in}}%
\pgfpathlineto{\pgfqpoint{2.670909in}{0.513475in}}%
\pgfpathlineto{\pgfqpoint{2.670909in}{0.499444in}}%
\pgfpathclose%
\pgfusepath{stroke}%
\end{pgfscope}%
\begin{pgfscope}%
\pgfpathrectangle{\pgfqpoint{0.515000in}{0.499444in}}{\pgfqpoint{3.487500in}{1.155000in}}%
\pgfusepath{clip}%
\pgfsetbuttcap%
\pgfsetmiterjoin%
\pgfsetlinewidth{1.003750pt}%
\definecolor{currentstroke}{rgb}{0.000000,0.000000,0.000000}%
\pgfsetstrokecolor{currentstroke}%
\pgfsetdash{}{0pt}%
\pgfpathmoveto{\pgfqpoint{2.829432in}{0.499444in}}%
\pgfpathlineto{\pgfqpoint{2.892841in}{0.499444in}}%
\pgfpathlineto{\pgfqpoint{2.892841in}{0.509932in}}%
\pgfpathlineto{\pgfqpoint{2.829432in}{0.509932in}}%
\pgfpathlineto{\pgfqpoint{2.829432in}{0.499444in}}%
\pgfpathclose%
\pgfusepath{stroke}%
\end{pgfscope}%
\begin{pgfscope}%
\pgfpathrectangle{\pgfqpoint{0.515000in}{0.499444in}}{\pgfqpoint{3.487500in}{1.155000in}}%
\pgfusepath{clip}%
\pgfsetbuttcap%
\pgfsetmiterjoin%
\pgfsetlinewidth{1.003750pt}%
\definecolor{currentstroke}{rgb}{0.000000,0.000000,0.000000}%
\pgfsetstrokecolor{currentstroke}%
\pgfsetdash{}{0pt}%
\pgfpathmoveto{\pgfqpoint{2.987955in}{0.499444in}}%
\pgfpathlineto{\pgfqpoint{3.051364in}{0.499444in}}%
\pgfpathlineto{\pgfqpoint{3.051364in}{0.507630in}}%
\pgfpathlineto{\pgfqpoint{2.987955in}{0.507630in}}%
\pgfpathlineto{\pgfqpoint{2.987955in}{0.499444in}}%
\pgfpathclose%
\pgfusepath{stroke}%
\end{pgfscope}%
\begin{pgfscope}%
\pgfpathrectangle{\pgfqpoint{0.515000in}{0.499444in}}{\pgfqpoint{3.487500in}{1.155000in}}%
\pgfusepath{clip}%
\pgfsetbuttcap%
\pgfsetmiterjoin%
\pgfsetlinewidth{1.003750pt}%
\definecolor{currentstroke}{rgb}{0.000000,0.000000,0.000000}%
\pgfsetstrokecolor{currentstroke}%
\pgfsetdash{}{0pt}%
\pgfpathmoveto{\pgfqpoint{3.146478in}{0.499444in}}%
\pgfpathlineto{\pgfqpoint{3.209887in}{0.499444in}}%
\pgfpathlineto{\pgfqpoint{3.209887in}{0.505709in}}%
\pgfpathlineto{\pgfqpoint{3.146478in}{0.505709in}}%
\pgfpathlineto{\pgfqpoint{3.146478in}{0.499444in}}%
\pgfpathclose%
\pgfusepath{stroke}%
\end{pgfscope}%
\begin{pgfscope}%
\pgfpathrectangle{\pgfqpoint{0.515000in}{0.499444in}}{\pgfqpoint{3.487500in}{1.155000in}}%
\pgfusepath{clip}%
\pgfsetbuttcap%
\pgfsetmiterjoin%
\pgfsetlinewidth{1.003750pt}%
\definecolor{currentstroke}{rgb}{0.000000,0.000000,0.000000}%
\pgfsetstrokecolor{currentstroke}%
\pgfsetdash{}{0pt}%
\pgfpathmoveto{\pgfqpoint{3.305000in}{0.499444in}}%
\pgfpathlineto{\pgfqpoint{3.368409in}{0.499444in}}%
\pgfpathlineto{\pgfqpoint{3.368409in}{0.503147in}}%
\pgfpathlineto{\pgfqpoint{3.305000in}{0.503147in}}%
\pgfpathlineto{\pgfqpoint{3.305000in}{0.499444in}}%
\pgfpathclose%
\pgfusepath{stroke}%
\end{pgfscope}%
\begin{pgfscope}%
\pgfpathrectangle{\pgfqpoint{0.515000in}{0.499444in}}{\pgfqpoint{3.487500in}{1.155000in}}%
\pgfusepath{clip}%
\pgfsetbuttcap%
\pgfsetmiterjoin%
\pgfsetlinewidth{1.003750pt}%
\definecolor{currentstroke}{rgb}{0.000000,0.000000,0.000000}%
\pgfsetstrokecolor{currentstroke}%
\pgfsetdash{}{0pt}%
\pgfpathmoveto{\pgfqpoint{3.463523in}{0.499444in}}%
\pgfpathlineto{\pgfqpoint{3.526932in}{0.499444in}}%
\pgfpathlineto{\pgfqpoint{3.526932in}{0.501266in}}%
\pgfpathlineto{\pgfqpoint{3.463523in}{0.501266in}}%
\pgfpathlineto{\pgfqpoint{3.463523in}{0.499444in}}%
\pgfpathclose%
\pgfusepath{stroke}%
\end{pgfscope}%
\begin{pgfscope}%
\pgfpathrectangle{\pgfqpoint{0.515000in}{0.499444in}}{\pgfqpoint{3.487500in}{1.155000in}}%
\pgfusepath{clip}%
\pgfsetbuttcap%
\pgfsetmiterjoin%
\pgfsetlinewidth{1.003750pt}%
\definecolor{currentstroke}{rgb}{0.000000,0.000000,0.000000}%
\pgfsetstrokecolor{currentstroke}%
\pgfsetdash{}{0pt}%
\pgfpathmoveto{\pgfqpoint{3.622046in}{0.499444in}}%
\pgfpathlineto{\pgfqpoint{3.685455in}{0.499444in}}%
\pgfpathlineto{\pgfqpoint{3.685455in}{0.499744in}}%
\pgfpathlineto{\pgfqpoint{3.622046in}{0.499744in}}%
\pgfpathlineto{\pgfqpoint{3.622046in}{0.499444in}}%
\pgfpathclose%
\pgfusepath{stroke}%
\end{pgfscope}%
\begin{pgfscope}%
\pgfpathrectangle{\pgfqpoint{0.515000in}{0.499444in}}{\pgfqpoint{3.487500in}{1.155000in}}%
\pgfusepath{clip}%
\pgfsetbuttcap%
\pgfsetmiterjoin%
\pgfsetlinewidth{1.003750pt}%
\definecolor{currentstroke}{rgb}{0.000000,0.000000,0.000000}%
\pgfsetstrokecolor{currentstroke}%
\pgfsetdash{}{0pt}%
\pgfpathmoveto{\pgfqpoint{3.780568in}{0.499444in}}%
\pgfpathlineto{\pgfqpoint{3.843978in}{0.499444in}}%
\pgfpathlineto{\pgfqpoint{3.843978in}{0.499444in}}%
\pgfpathlineto{\pgfqpoint{3.780568in}{0.499444in}}%
\pgfpathlineto{\pgfqpoint{3.780568in}{0.499444in}}%
\pgfpathclose%
\pgfusepath{stroke}%
\end{pgfscope}%
\begin{pgfscope}%
\pgfpathrectangle{\pgfqpoint{0.515000in}{0.499444in}}{\pgfqpoint{3.487500in}{1.155000in}}%
\pgfusepath{clip}%
\pgfsetbuttcap%
\pgfsetmiterjoin%
\definecolor{currentfill}{rgb}{0.000000,0.000000,0.000000}%
\pgfsetfillcolor{currentfill}%
\pgfsetlinewidth{0.000000pt}%
\definecolor{currentstroke}{rgb}{0.000000,0.000000,0.000000}%
\pgfsetstrokecolor{currentstroke}%
\pgfsetstrokeopacity{0.000000}%
\pgfsetdash{}{0pt}%
\pgfpathmoveto{\pgfqpoint{0.673523in}{0.499444in}}%
\pgfpathlineto{\pgfqpoint{0.736932in}{0.499444in}}%
\pgfpathlineto{\pgfqpoint{0.736932in}{0.499444in}}%
\pgfpathlineto{\pgfqpoint{0.673523in}{0.499444in}}%
\pgfpathlineto{\pgfqpoint{0.673523in}{0.499444in}}%
\pgfpathclose%
\pgfusepath{fill}%
\end{pgfscope}%
\begin{pgfscope}%
\pgfpathrectangle{\pgfqpoint{0.515000in}{0.499444in}}{\pgfqpoint{3.487500in}{1.155000in}}%
\pgfusepath{clip}%
\pgfsetbuttcap%
\pgfsetmiterjoin%
\definecolor{currentfill}{rgb}{0.000000,0.000000,0.000000}%
\pgfsetfillcolor{currentfill}%
\pgfsetlinewidth{0.000000pt}%
\definecolor{currentstroke}{rgb}{0.000000,0.000000,0.000000}%
\pgfsetstrokecolor{currentstroke}%
\pgfsetstrokeopacity{0.000000}%
\pgfsetdash{}{0pt}%
\pgfpathmoveto{\pgfqpoint{0.832046in}{0.499444in}}%
\pgfpathlineto{\pgfqpoint{0.895455in}{0.499444in}}%
\pgfpathlineto{\pgfqpoint{0.895455in}{0.535532in}}%
\pgfpathlineto{\pgfqpoint{0.832046in}{0.535532in}}%
\pgfpathlineto{\pgfqpoint{0.832046in}{0.499444in}}%
\pgfpathclose%
\pgfusepath{fill}%
\end{pgfscope}%
\begin{pgfscope}%
\pgfpathrectangle{\pgfqpoint{0.515000in}{0.499444in}}{\pgfqpoint{3.487500in}{1.155000in}}%
\pgfusepath{clip}%
\pgfsetbuttcap%
\pgfsetmiterjoin%
\definecolor{currentfill}{rgb}{0.000000,0.000000,0.000000}%
\pgfsetfillcolor{currentfill}%
\pgfsetlinewidth{0.000000pt}%
\definecolor{currentstroke}{rgb}{0.000000,0.000000,0.000000}%
\pgfsetstrokecolor{currentstroke}%
\pgfsetstrokeopacity{0.000000}%
\pgfsetdash{}{0pt}%
\pgfpathmoveto{\pgfqpoint{0.990568in}{0.499444in}}%
\pgfpathlineto{\pgfqpoint{1.053978in}{0.499444in}}%
\pgfpathlineto{\pgfqpoint{1.053978in}{0.575602in}}%
\pgfpathlineto{\pgfqpoint{0.990568in}{0.575602in}}%
\pgfpathlineto{\pgfqpoint{0.990568in}{0.499444in}}%
\pgfpathclose%
\pgfusepath{fill}%
\end{pgfscope}%
\begin{pgfscope}%
\pgfpathrectangle{\pgfqpoint{0.515000in}{0.499444in}}{\pgfqpoint{3.487500in}{1.155000in}}%
\pgfusepath{clip}%
\pgfsetbuttcap%
\pgfsetmiterjoin%
\definecolor{currentfill}{rgb}{0.000000,0.000000,0.000000}%
\pgfsetfillcolor{currentfill}%
\pgfsetlinewidth{0.000000pt}%
\definecolor{currentstroke}{rgb}{0.000000,0.000000,0.000000}%
\pgfsetstrokecolor{currentstroke}%
\pgfsetstrokeopacity{0.000000}%
\pgfsetdash{}{0pt}%
\pgfpathmoveto{\pgfqpoint{1.149091in}{0.499444in}}%
\pgfpathlineto{\pgfqpoint{1.212500in}{0.499444in}}%
\pgfpathlineto{\pgfqpoint{1.212500in}{0.576243in}}%
\pgfpathlineto{\pgfqpoint{1.149091in}{0.576243in}}%
\pgfpathlineto{\pgfqpoint{1.149091in}{0.499444in}}%
\pgfpathclose%
\pgfusepath{fill}%
\end{pgfscope}%
\begin{pgfscope}%
\pgfpathrectangle{\pgfqpoint{0.515000in}{0.499444in}}{\pgfqpoint{3.487500in}{1.155000in}}%
\pgfusepath{clip}%
\pgfsetbuttcap%
\pgfsetmiterjoin%
\definecolor{currentfill}{rgb}{0.000000,0.000000,0.000000}%
\pgfsetfillcolor{currentfill}%
\pgfsetlinewidth{0.000000pt}%
\definecolor{currentstroke}{rgb}{0.000000,0.000000,0.000000}%
\pgfsetstrokecolor{currentstroke}%
\pgfsetstrokeopacity{0.000000}%
\pgfsetdash{}{0pt}%
\pgfpathmoveto{\pgfqpoint{1.307614in}{0.499444in}}%
\pgfpathlineto{\pgfqpoint{1.371023in}{0.499444in}}%
\pgfpathlineto{\pgfqpoint{1.371023in}{0.571940in}}%
\pgfpathlineto{\pgfqpoint{1.307614in}{0.571940in}}%
\pgfpathlineto{\pgfqpoint{1.307614in}{0.499444in}}%
\pgfpathclose%
\pgfusepath{fill}%
\end{pgfscope}%
\begin{pgfscope}%
\pgfpathrectangle{\pgfqpoint{0.515000in}{0.499444in}}{\pgfqpoint{3.487500in}{1.155000in}}%
\pgfusepath{clip}%
\pgfsetbuttcap%
\pgfsetmiterjoin%
\definecolor{currentfill}{rgb}{0.000000,0.000000,0.000000}%
\pgfsetfillcolor{currentfill}%
\pgfsetlinewidth{0.000000pt}%
\definecolor{currentstroke}{rgb}{0.000000,0.000000,0.000000}%
\pgfsetstrokecolor{currentstroke}%
\pgfsetstrokeopacity{0.000000}%
\pgfsetdash{}{0pt}%
\pgfpathmoveto{\pgfqpoint{1.466137in}{0.499444in}}%
\pgfpathlineto{\pgfqpoint{1.529546in}{0.499444in}}%
\pgfpathlineto{\pgfqpoint{1.529546in}{0.564794in}}%
\pgfpathlineto{\pgfqpoint{1.466137in}{0.564794in}}%
\pgfpathlineto{\pgfqpoint{1.466137in}{0.499444in}}%
\pgfpathclose%
\pgfusepath{fill}%
\end{pgfscope}%
\begin{pgfscope}%
\pgfpathrectangle{\pgfqpoint{0.515000in}{0.499444in}}{\pgfqpoint{3.487500in}{1.155000in}}%
\pgfusepath{clip}%
\pgfsetbuttcap%
\pgfsetmiterjoin%
\definecolor{currentfill}{rgb}{0.000000,0.000000,0.000000}%
\pgfsetfillcolor{currentfill}%
\pgfsetlinewidth{0.000000pt}%
\definecolor{currentstroke}{rgb}{0.000000,0.000000,0.000000}%
\pgfsetstrokecolor{currentstroke}%
\pgfsetstrokeopacity{0.000000}%
\pgfsetdash{}{0pt}%
\pgfpathmoveto{\pgfqpoint{1.624659in}{0.499444in}}%
\pgfpathlineto{\pgfqpoint{1.688068in}{0.499444in}}%
\pgfpathlineto{\pgfqpoint{1.688068in}{0.557128in}}%
\pgfpathlineto{\pgfqpoint{1.624659in}{0.557128in}}%
\pgfpathlineto{\pgfqpoint{1.624659in}{0.499444in}}%
\pgfpathclose%
\pgfusepath{fill}%
\end{pgfscope}%
\begin{pgfscope}%
\pgfpathrectangle{\pgfqpoint{0.515000in}{0.499444in}}{\pgfqpoint{3.487500in}{1.155000in}}%
\pgfusepath{clip}%
\pgfsetbuttcap%
\pgfsetmiterjoin%
\definecolor{currentfill}{rgb}{0.000000,0.000000,0.000000}%
\pgfsetfillcolor{currentfill}%
\pgfsetlinewidth{0.000000pt}%
\definecolor{currentstroke}{rgb}{0.000000,0.000000,0.000000}%
\pgfsetstrokecolor{currentstroke}%
\pgfsetstrokeopacity{0.000000}%
\pgfsetdash{}{0pt}%
\pgfpathmoveto{\pgfqpoint{1.783182in}{0.499444in}}%
\pgfpathlineto{\pgfqpoint{1.846591in}{0.499444in}}%
\pgfpathlineto{\pgfqpoint{1.846591in}{0.548922in}}%
\pgfpathlineto{\pgfqpoint{1.783182in}{0.548922in}}%
\pgfpathlineto{\pgfqpoint{1.783182in}{0.499444in}}%
\pgfpathclose%
\pgfusepath{fill}%
\end{pgfscope}%
\begin{pgfscope}%
\pgfpathrectangle{\pgfqpoint{0.515000in}{0.499444in}}{\pgfqpoint{3.487500in}{1.155000in}}%
\pgfusepath{clip}%
\pgfsetbuttcap%
\pgfsetmiterjoin%
\definecolor{currentfill}{rgb}{0.000000,0.000000,0.000000}%
\pgfsetfillcolor{currentfill}%
\pgfsetlinewidth{0.000000pt}%
\definecolor{currentstroke}{rgb}{0.000000,0.000000,0.000000}%
\pgfsetstrokecolor{currentstroke}%
\pgfsetstrokeopacity{0.000000}%
\pgfsetdash{}{0pt}%
\pgfpathmoveto{\pgfqpoint{1.941705in}{0.499444in}}%
\pgfpathlineto{\pgfqpoint{2.005114in}{0.499444in}}%
\pgfpathlineto{\pgfqpoint{2.005114in}{0.542597in}}%
\pgfpathlineto{\pgfqpoint{1.941705in}{0.542597in}}%
\pgfpathlineto{\pgfqpoint{1.941705in}{0.499444in}}%
\pgfpathclose%
\pgfusepath{fill}%
\end{pgfscope}%
\begin{pgfscope}%
\pgfpathrectangle{\pgfqpoint{0.515000in}{0.499444in}}{\pgfqpoint{3.487500in}{1.155000in}}%
\pgfusepath{clip}%
\pgfsetbuttcap%
\pgfsetmiterjoin%
\definecolor{currentfill}{rgb}{0.000000,0.000000,0.000000}%
\pgfsetfillcolor{currentfill}%
\pgfsetlinewidth{0.000000pt}%
\definecolor{currentstroke}{rgb}{0.000000,0.000000,0.000000}%
\pgfsetstrokecolor{currentstroke}%
\pgfsetstrokeopacity{0.000000}%
\pgfsetdash{}{0pt}%
\pgfpathmoveto{\pgfqpoint{2.100228in}{0.499444in}}%
\pgfpathlineto{\pgfqpoint{2.163637in}{0.499444in}}%
\pgfpathlineto{\pgfqpoint{2.163637in}{0.534451in}}%
\pgfpathlineto{\pgfqpoint{2.100228in}{0.534451in}}%
\pgfpathlineto{\pgfqpoint{2.100228in}{0.499444in}}%
\pgfpathclose%
\pgfusepath{fill}%
\end{pgfscope}%
\begin{pgfscope}%
\pgfpathrectangle{\pgfqpoint{0.515000in}{0.499444in}}{\pgfqpoint{3.487500in}{1.155000in}}%
\pgfusepath{clip}%
\pgfsetbuttcap%
\pgfsetmiterjoin%
\definecolor{currentfill}{rgb}{0.000000,0.000000,0.000000}%
\pgfsetfillcolor{currentfill}%
\pgfsetlinewidth{0.000000pt}%
\definecolor{currentstroke}{rgb}{0.000000,0.000000,0.000000}%
\pgfsetstrokecolor{currentstroke}%
\pgfsetstrokeopacity{0.000000}%
\pgfsetdash{}{0pt}%
\pgfpathmoveto{\pgfqpoint{2.258750in}{0.499444in}}%
\pgfpathlineto{\pgfqpoint{2.322159in}{0.499444in}}%
\pgfpathlineto{\pgfqpoint{2.322159in}{0.528907in}}%
\pgfpathlineto{\pgfqpoint{2.258750in}{0.528907in}}%
\pgfpathlineto{\pgfqpoint{2.258750in}{0.499444in}}%
\pgfpathclose%
\pgfusepath{fill}%
\end{pgfscope}%
\begin{pgfscope}%
\pgfpathrectangle{\pgfqpoint{0.515000in}{0.499444in}}{\pgfqpoint{3.487500in}{1.155000in}}%
\pgfusepath{clip}%
\pgfsetbuttcap%
\pgfsetmiterjoin%
\definecolor{currentfill}{rgb}{0.000000,0.000000,0.000000}%
\pgfsetfillcolor{currentfill}%
\pgfsetlinewidth{0.000000pt}%
\definecolor{currentstroke}{rgb}{0.000000,0.000000,0.000000}%
\pgfsetstrokecolor{currentstroke}%
\pgfsetstrokeopacity{0.000000}%
\pgfsetdash{}{0pt}%
\pgfpathmoveto{\pgfqpoint{2.417273in}{0.499444in}}%
\pgfpathlineto{\pgfqpoint{2.480682in}{0.499444in}}%
\pgfpathlineto{\pgfqpoint{2.480682in}{0.525304in}}%
\pgfpathlineto{\pgfqpoint{2.417273in}{0.525304in}}%
\pgfpathlineto{\pgfqpoint{2.417273in}{0.499444in}}%
\pgfpathclose%
\pgfusepath{fill}%
\end{pgfscope}%
\begin{pgfscope}%
\pgfpathrectangle{\pgfqpoint{0.515000in}{0.499444in}}{\pgfqpoint{3.487500in}{1.155000in}}%
\pgfusepath{clip}%
\pgfsetbuttcap%
\pgfsetmiterjoin%
\definecolor{currentfill}{rgb}{0.000000,0.000000,0.000000}%
\pgfsetfillcolor{currentfill}%
\pgfsetlinewidth{0.000000pt}%
\definecolor{currentstroke}{rgb}{0.000000,0.000000,0.000000}%
\pgfsetstrokecolor{currentstroke}%
\pgfsetstrokeopacity{0.000000}%
\pgfsetdash{}{0pt}%
\pgfpathmoveto{\pgfqpoint{2.575796in}{0.499444in}}%
\pgfpathlineto{\pgfqpoint{2.639205in}{0.499444in}}%
\pgfpathlineto{\pgfqpoint{2.639205in}{0.521501in}}%
\pgfpathlineto{\pgfqpoint{2.575796in}{0.521501in}}%
\pgfpathlineto{\pgfqpoint{2.575796in}{0.499444in}}%
\pgfpathclose%
\pgfusepath{fill}%
\end{pgfscope}%
\begin{pgfscope}%
\pgfpathrectangle{\pgfqpoint{0.515000in}{0.499444in}}{\pgfqpoint{3.487500in}{1.155000in}}%
\pgfusepath{clip}%
\pgfsetbuttcap%
\pgfsetmiterjoin%
\definecolor{currentfill}{rgb}{0.000000,0.000000,0.000000}%
\pgfsetfillcolor{currentfill}%
\pgfsetlinewidth{0.000000pt}%
\definecolor{currentstroke}{rgb}{0.000000,0.000000,0.000000}%
\pgfsetstrokecolor{currentstroke}%
\pgfsetstrokeopacity{0.000000}%
\pgfsetdash{}{0pt}%
\pgfpathmoveto{\pgfqpoint{2.734318in}{0.499444in}}%
\pgfpathlineto{\pgfqpoint{2.797728in}{0.499444in}}%
\pgfpathlineto{\pgfqpoint{2.797728in}{0.518299in}}%
\pgfpathlineto{\pgfqpoint{2.734318in}{0.518299in}}%
\pgfpathlineto{\pgfqpoint{2.734318in}{0.499444in}}%
\pgfpathclose%
\pgfusepath{fill}%
\end{pgfscope}%
\begin{pgfscope}%
\pgfpathrectangle{\pgfqpoint{0.515000in}{0.499444in}}{\pgfqpoint{3.487500in}{1.155000in}}%
\pgfusepath{clip}%
\pgfsetbuttcap%
\pgfsetmiterjoin%
\definecolor{currentfill}{rgb}{0.000000,0.000000,0.000000}%
\pgfsetfillcolor{currentfill}%
\pgfsetlinewidth{0.000000pt}%
\definecolor{currentstroke}{rgb}{0.000000,0.000000,0.000000}%
\pgfsetstrokecolor{currentstroke}%
\pgfsetstrokeopacity{0.000000}%
\pgfsetdash{}{0pt}%
\pgfpathmoveto{\pgfqpoint{2.892841in}{0.499444in}}%
\pgfpathlineto{\pgfqpoint{2.956250in}{0.499444in}}%
\pgfpathlineto{\pgfqpoint{2.956250in}{0.517418in}}%
\pgfpathlineto{\pgfqpoint{2.892841in}{0.517418in}}%
\pgfpathlineto{\pgfqpoint{2.892841in}{0.499444in}}%
\pgfpathclose%
\pgfusepath{fill}%
\end{pgfscope}%
\begin{pgfscope}%
\pgfpathrectangle{\pgfqpoint{0.515000in}{0.499444in}}{\pgfqpoint{3.487500in}{1.155000in}}%
\pgfusepath{clip}%
\pgfsetbuttcap%
\pgfsetmiterjoin%
\definecolor{currentfill}{rgb}{0.000000,0.000000,0.000000}%
\pgfsetfillcolor{currentfill}%
\pgfsetlinewidth{0.000000pt}%
\definecolor{currentstroke}{rgb}{0.000000,0.000000,0.000000}%
\pgfsetstrokecolor{currentstroke}%
\pgfsetstrokeopacity{0.000000}%
\pgfsetdash{}{0pt}%
\pgfpathmoveto{\pgfqpoint{3.051364in}{0.499444in}}%
\pgfpathlineto{\pgfqpoint{3.114773in}{0.499444in}}%
\pgfpathlineto{\pgfqpoint{3.114773in}{0.515116in}}%
\pgfpathlineto{\pgfqpoint{3.051364in}{0.515116in}}%
\pgfpathlineto{\pgfqpoint{3.051364in}{0.499444in}}%
\pgfpathclose%
\pgfusepath{fill}%
\end{pgfscope}%
\begin{pgfscope}%
\pgfpathrectangle{\pgfqpoint{0.515000in}{0.499444in}}{\pgfqpoint{3.487500in}{1.155000in}}%
\pgfusepath{clip}%
\pgfsetbuttcap%
\pgfsetmiterjoin%
\definecolor{currentfill}{rgb}{0.000000,0.000000,0.000000}%
\pgfsetfillcolor{currentfill}%
\pgfsetlinewidth{0.000000pt}%
\definecolor{currentstroke}{rgb}{0.000000,0.000000,0.000000}%
\pgfsetstrokecolor{currentstroke}%
\pgfsetstrokeopacity{0.000000}%
\pgfsetdash{}{0pt}%
\pgfpathmoveto{\pgfqpoint{3.209887in}{0.499444in}}%
\pgfpathlineto{\pgfqpoint{3.273296in}{0.499444in}}%
\pgfpathlineto{\pgfqpoint{3.273296in}{0.513695in}}%
\pgfpathlineto{\pgfqpoint{3.209887in}{0.513695in}}%
\pgfpathlineto{\pgfqpoint{3.209887in}{0.499444in}}%
\pgfpathclose%
\pgfusepath{fill}%
\end{pgfscope}%
\begin{pgfscope}%
\pgfpathrectangle{\pgfqpoint{0.515000in}{0.499444in}}{\pgfqpoint{3.487500in}{1.155000in}}%
\pgfusepath{clip}%
\pgfsetbuttcap%
\pgfsetmiterjoin%
\definecolor{currentfill}{rgb}{0.000000,0.000000,0.000000}%
\pgfsetfillcolor{currentfill}%
\pgfsetlinewidth{0.000000pt}%
\definecolor{currentstroke}{rgb}{0.000000,0.000000,0.000000}%
\pgfsetstrokecolor{currentstroke}%
\pgfsetstrokeopacity{0.000000}%
\pgfsetdash{}{0pt}%
\pgfpathmoveto{\pgfqpoint{3.368409in}{0.499444in}}%
\pgfpathlineto{\pgfqpoint{3.431818in}{0.499444in}}%
\pgfpathlineto{\pgfqpoint{3.431818in}{0.511653in}}%
\pgfpathlineto{\pgfqpoint{3.368409in}{0.511653in}}%
\pgfpathlineto{\pgfqpoint{3.368409in}{0.499444in}}%
\pgfpathclose%
\pgfusepath{fill}%
\end{pgfscope}%
\begin{pgfscope}%
\pgfpathrectangle{\pgfqpoint{0.515000in}{0.499444in}}{\pgfqpoint{3.487500in}{1.155000in}}%
\pgfusepath{clip}%
\pgfsetbuttcap%
\pgfsetmiterjoin%
\definecolor{currentfill}{rgb}{0.000000,0.000000,0.000000}%
\pgfsetfillcolor{currentfill}%
\pgfsetlinewidth{0.000000pt}%
\definecolor{currentstroke}{rgb}{0.000000,0.000000,0.000000}%
\pgfsetstrokecolor{currentstroke}%
\pgfsetstrokeopacity{0.000000}%
\pgfsetdash{}{0pt}%
\pgfpathmoveto{\pgfqpoint{3.526932in}{0.499444in}}%
\pgfpathlineto{\pgfqpoint{3.590341in}{0.499444in}}%
\pgfpathlineto{\pgfqpoint{3.590341in}{0.505949in}}%
\pgfpathlineto{\pgfqpoint{3.526932in}{0.505949in}}%
\pgfpathlineto{\pgfqpoint{3.526932in}{0.499444in}}%
\pgfpathclose%
\pgfusepath{fill}%
\end{pgfscope}%
\begin{pgfscope}%
\pgfpathrectangle{\pgfqpoint{0.515000in}{0.499444in}}{\pgfqpoint{3.487500in}{1.155000in}}%
\pgfusepath{clip}%
\pgfsetbuttcap%
\pgfsetmiterjoin%
\definecolor{currentfill}{rgb}{0.000000,0.000000,0.000000}%
\pgfsetfillcolor{currentfill}%
\pgfsetlinewidth{0.000000pt}%
\definecolor{currentstroke}{rgb}{0.000000,0.000000,0.000000}%
\pgfsetstrokecolor{currentstroke}%
\pgfsetstrokeopacity{0.000000}%
\pgfsetdash{}{0pt}%
\pgfpathmoveto{\pgfqpoint{3.685455in}{0.499444in}}%
\pgfpathlineto{\pgfqpoint{3.748864in}{0.499444in}}%
\pgfpathlineto{\pgfqpoint{3.748864in}{0.501386in}}%
\pgfpathlineto{\pgfqpoint{3.685455in}{0.501386in}}%
\pgfpathlineto{\pgfqpoint{3.685455in}{0.499444in}}%
\pgfpathclose%
\pgfusepath{fill}%
\end{pgfscope}%
\begin{pgfscope}%
\pgfpathrectangle{\pgfqpoint{0.515000in}{0.499444in}}{\pgfqpoint{3.487500in}{1.155000in}}%
\pgfusepath{clip}%
\pgfsetbuttcap%
\pgfsetmiterjoin%
\definecolor{currentfill}{rgb}{0.000000,0.000000,0.000000}%
\pgfsetfillcolor{currentfill}%
\pgfsetlinewidth{0.000000pt}%
\definecolor{currentstroke}{rgb}{0.000000,0.000000,0.000000}%
\pgfsetstrokecolor{currentstroke}%
\pgfsetstrokeopacity{0.000000}%
\pgfsetdash{}{0pt}%
\pgfpathmoveto{\pgfqpoint{3.843978in}{0.499444in}}%
\pgfpathlineto{\pgfqpoint{3.907387in}{0.499444in}}%
\pgfpathlineto{\pgfqpoint{3.907387in}{0.499464in}}%
\pgfpathlineto{\pgfqpoint{3.843978in}{0.499464in}}%
\pgfpathlineto{\pgfqpoint{3.843978in}{0.499444in}}%
\pgfpathclose%
\pgfusepath{fill}%
\end{pgfscope}%
\begin{pgfscope}%
\pgfsetbuttcap%
\pgfsetroundjoin%
\definecolor{currentfill}{rgb}{0.000000,0.000000,0.000000}%
\pgfsetfillcolor{currentfill}%
\pgfsetlinewidth{0.803000pt}%
\definecolor{currentstroke}{rgb}{0.000000,0.000000,0.000000}%
\pgfsetstrokecolor{currentstroke}%
\pgfsetdash{}{0pt}%
\pgfsys@defobject{currentmarker}{\pgfqpoint{0.000000in}{-0.048611in}}{\pgfqpoint{0.000000in}{0.000000in}}{%
\pgfpathmoveto{\pgfqpoint{0.000000in}{0.000000in}}%
\pgfpathlineto{\pgfqpoint{0.000000in}{-0.048611in}}%
\pgfusepath{stroke,fill}%
}%
\begin{pgfscope}%
\pgfsys@transformshift{0.515000in}{0.499444in}%
\pgfsys@useobject{currentmarker}{}%
\end{pgfscope}%
\end{pgfscope}%
\begin{pgfscope}%
\pgfsetbuttcap%
\pgfsetroundjoin%
\definecolor{currentfill}{rgb}{0.000000,0.000000,0.000000}%
\pgfsetfillcolor{currentfill}%
\pgfsetlinewidth{0.803000pt}%
\definecolor{currentstroke}{rgb}{0.000000,0.000000,0.000000}%
\pgfsetstrokecolor{currentstroke}%
\pgfsetdash{}{0pt}%
\pgfsys@defobject{currentmarker}{\pgfqpoint{0.000000in}{-0.048611in}}{\pgfqpoint{0.000000in}{0.000000in}}{%
\pgfpathmoveto{\pgfqpoint{0.000000in}{0.000000in}}%
\pgfpathlineto{\pgfqpoint{0.000000in}{-0.048611in}}%
\pgfusepath{stroke,fill}%
}%
\begin{pgfscope}%
\pgfsys@transformshift{0.673523in}{0.499444in}%
\pgfsys@useobject{currentmarker}{}%
\end{pgfscope}%
\end{pgfscope}%
\begin{pgfscope}%
\definecolor{textcolor}{rgb}{0.000000,0.000000,0.000000}%
\pgfsetstrokecolor{textcolor}%
\pgfsetfillcolor{textcolor}%
\pgftext[x=0.673523in,y=0.402222in,,top]{\color{textcolor}\rmfamily\fontsize{10.000000}{12.000000}\selectfont 0.0}%
\end{pgfscope}%
\begin{pgfscope}%
\pgfsetbuttcap%
\pgfsetroundjoin%
\definecolor{currentfill}{rgb}{0.000000,0.000000,0.000000}%
\pgfsetfillcolor{currentfill}%
\pgfsetlinewidth{0.803000pt}%
\definecolor{currentstroke}{rgb}{0.000000,0.000000,0.000000}%
\pgfsetstrokecolor{currentstroke}%
\pgfsetdash{}{0pt}%
\pgfsys@defobject{currentmarker}{\pgfqpoint{0.000000in}{-0.048611in}}{\pgfqpoint{0.000000in}{0.000000in}}{%
\pgfpathmoveto{\pgfqpoint{0.000000in}{0.000000in}}%
\pgfpathlineto{\pgfqpoint{0.000000in}{-0.048611in}}%
\pgfusepath{stroke,fill}%
}%
\begin{pgfscope}%
\pgfsys@transformshift{0.832046in}{0.499444in}%
\pgfsys@useobject{currentmarker}{}%
\end{pgfscope}%
\end{pgfscope}%
\begin{pgfscope}%
\pgfsetbuttcap%
\pgfsetroundjoin%
\definecolor{currentfill}{rgb}{0.000000,0.000000,0.000000}%
\pgfsetfillcolor{currentfill}%
\pgfsetlinewidth{0.803000pt}%
\definecolor{currentstroke}{rgb}{0.000000,0.000000,0.000000}%
\pgfsetstrokecolor{currentstroke}%
\pgfsetdash{}{0pt}%
\pgfsys@defobject{currentmarker}{\pgfqpoint{0.000000in}{-0.048611in}}{\pgfqpoint{0.000000in}{0.000000in}}{%
\pgfpathmoveto{\pgfqpoint{0.000000in}{0.000000in}}%
\pgfpathlineto{\pgfqpoint{0.000000in}{-0.048611in}}%
\pgfusepath{stroke,fill}%
}%
\begin{pgfscope}%
\pgfsys@transformshift{0.990568in}{0.499444in}%
\pgfsys@useobject{currentmarker}{}%
\end{pgfscope}%
\end{pgfscope}%
\begin{pgfscope}%
\definecolor{textcolor}{rgb}{0.000000,0.000000,0.000000}%
\pgfsetstrokecolor{textcolor}%
\pgfsetfillcolor{textcolor}%
\pgftext[x=0.990568in,y=0.402222in,,top]{\color{textcolor}\rmfamily\fontsize{10.000000}{12.000000}\selectfont 0.1}%
\end{pgfscope}%
\begin{pgfscope}%
\pgfsetbuttcap%
\pgfsetroundjoin%
\definecolor{currentfill}{rgb}{0.000000,0.000000,0.000000}%
\pgfsetfillcolor{currentfill}%
\pgfsetlinewidth{0.803000pt}%
\definecolor{currentstroke}{rgb}{0.000000,0.000000,0.000000}%
\pgfsetstrokecolor{currentstroke}%
\pgfsetdash{}{0pt}%
\pgfsys@defobject{currentmarker}{\pgfqpoint{0.000000in}{-0.048611in}}{\pgfqpoint{0.000000in}{0.000000in}}{%
\pgfpathmoveto{\pgfqpoint{0.000000in}{0.000000in}}%
\pgfpathlineto{\pgfqpoint{0.000000in}{-0.048611in}}%
\pgfusepath{stroke,fill}%
}%
\begin{pgfscope}%
\pgfsys@transformshift{1.149091in}{0.499444in}%
\pgfsys@useobject{currentmarker}{}%
\end{pgfscope}%
\end{pgfscope}%
\begin{pgfscope}%
\pgfsetbuttcap%
\pgfsetroundjoin%
\definecolor{currentfill}{rgb}{0.000000,0.000000,0.000000}%
\pgfsetfillcolor{currentfill}%
\pgfsetlinewidth{0.803000pt}%
\definecolor{currentstroke}{rgb}{0.000000,0.000000,0.000000}%
\pgfsetstrokecolor{currentstroke}%
\pgfsetdash{}{0pt}%
\pgfsys@defobject{currentmarker}{\pgfqpoint{0.000000in}{-0.048611in}}{\pgfqpoint{0.000000in}{0.000000in}}{%
\pgfpathmoveto{\pgfqpoint{0.000000in}{0.000000in}}%
\pgfpathlineto{\pgfqpoint{0.000000in}{-0.048611in}}%
\pgfusepath{stroke,fill}%
}%
\begin{pgfscope}%
\pgfsys@transformshift{1.307614in}{0.499444in}%
\pgfsys@useobject{currentmarker}{}%
\end{pgfscope}%
\end{pgfscope}%
\begin{pgfscope}%
\definecolor{textcolor}{rgb}{0.000000,0.000000,0.000000}%
\pgfsetstrokecolor{textcolor}%
\pgfsetfillcolor{textcolor}%
\pgftext[x=1.307614in,y=0.402222in,,top]{\color{textcolor}\rmfamily\fontsize{10.000000}{12.000000}\selectfont 0.2}%
\end{pgfscope}%
\begin{pgfscope}%
\pgfsetbuttcap%
\pgfsetroundjoin%
\definecolor{currentfill}{rgb}{0.000000,0.000000,0.000000}%
\pgfsetfillcolor{currentfill}%
\pgfsetlinewidth{0.803000pt}%
\definecolor{currentstroke}{rgb}{0.000000,0.000000,0.000000}%
\pgfsetstrokecolor{currentstroke}%
\pgfsetdash{}{0pt}%
\pgfsys@defobject{currentmarker}{\pgfqpoint{0.000000in}{-0.048611in}}{\pgfqpoint{0.000000in}{0.000000in}}{%
\pgfpathmoveto{\pgfqpoint{0.000000in}{0.000000in}}%
\pgfpathlineto{\pgfqpoint{0.000000in}{-0.048611in}}%
\pgfusepath{stroke,fill}%
}%
\begin{pgfscope}%
\pgfsys@transformshift{1.466137in}{0.499444in}%
\pgfsys@useobject{currentmarker}{}%
\end{pgfscope}%
\end{pgfscope}%
\begin{pgfscope}%
\pgfsetbuttcap%
\pgfsetroundjoin%
\definecolor{currentfill}{rgb}{0.000000,0.000000,0.000000}%
\pgfsetfillcolor{currentfill}%
\pgfsetlinewidth{0.803000pt}%
\definecolor{currentstroke}{rgb}{0.000000,0.000000,0.000000}%
\pgfsetstrokecolor{currentstroke}%
\pgfsetdash{}{0pt}%
\pgfsys@defobject{currentmarker}{\pgfqpoint{0.000000in}{-0.048611in}}{\pgfqpoint{0.000000in}{0.000000in}}{%
\pgfpathmoveto{\pgfqpoint{0.000000in}{0.000000in}}%
\pgfpathlineto{\pgfqpoint{0.000000in}{-0.048611in}}%
\pgfusepath{stroke,fill}%
}%
\begin{pgfscope}%
\pgfsys@transformshift{1.624659in}{0.499444in}%
\pgfsys@useobject{currentmarker}{}%
\end{pgfscope}%
\end{pgfscope}%
\begin{pgfscope}%
\definecolor{textcolor}{rgb}{0.000000,0.000000,0.000000}%
\pgfsetstrokecolor{textcolor}%
\pgfsetfillcolor{textcolor}%
\pgftext[x=1.624659in,y=0.402222in,,top]{\color{textcolor}\rmfamily\fontsize{10.000000}{12.000000}\selectfont 0.3}%
\end{pgfscope}%
\begin{pgfscope}%
\pgfsetbuttcap%
\pgfsetroundjoin%
\definecolor{currentfill}{rgb}{0.000000,0.000000,0.000000}%
\pgfsetfillcolor{currentfill}%
\pgfsetlinewidth{0.803000pt}%
\definecolor{currentstroke}{rgb}{0.000000,0.000000,0.000000}%
\pgfsetstrokecolor{currentstroke}%
\pgfsetdash{}{0pt}%
\pgfsys@defobject{currentmarker}{\pgfqpoint{0.000000in}{-0.048611in}}{\pgfqpoint{0.000000in}{0.000000in}}{%
\pgfpathmoveto{\pgfqpoint{0.000000in}{0.000000in}}%
\pgfpathlineto{\pgfqpoint{0.000000in}{-0.048611in}}%
\pgfusepath{stroke,fill}%
}%
\begin{pgfscope}%
\pgfsys@transformshift{1.783182in}{0.499444in}%
\pgfsys@useobject{currentmarker}{}%
\end{pgfscope}%
\end{pgfscope}%
\begin{pgfscope}%
\pgfsetbuttcap%
\pgfsetroundjoin%
\definecolor{currentfill}{rgb}{0.000000,0.000000,0.000000}%
\pgfsetfillcolor{currentfill}%
\pgfsetlinewidth{0.803000pt}%
\definecolor{currentstroke}{rgb}{0.000000,0.000000,0.000000}%
\pgfsetstrokecolor{currentstroke}%
\pgfsetdash{}{0pt}%
\pgfsys@defobject{currentmarker}{\pgfqpoint{0.000000in}{-0.048611in}}{\pgfqpoint{0.000000in}{0.000000in}}{%
\pgfpathmoveto{\pgfqpoint{0.000000in}{0.000000in}}%
\pgfpathlineto{\pgfqpoint{0.000000in}{-0.048611in}}%
\pgfusepath{stroke,fill}%
}%
\begin{pgfscope}%
\pgfsys@transformshift{1.941705in}{0.499444in}%
\pgfsys@useobject{currentmarker}{}%
\end{pgfscope}%
\end{pgfscope}%
\begin{pgfscope}%
\definecolor{textcolor}{rgb}{0.000000,0.000000,0.000000}%
\pgfsetstrokecolor{textcolor}%
\pgfsetfillcolor{textcolor}%
\pgftext[x=1.941705in,y=0.402222in,,top]{\color{textcolor}\rmfamily\fontsize{10.000000}{12.000000}\selectfont 0.4}%
\end{pgfscope}%
\begin{pgfscope}%
\pgfsetbuttcap%
\pgfsetroundjoin%
\definecolor{currentfill}{rgb}{0.000000,0.000000,0.000000}%
\pgfsetfillcolor{currentfill}%
\pgfsetlinewidth{0.803000pt}%
\definecolor{currentstroke}{rgb}{0.000000,0.000000,0.000000}%
\pgfsetstrokecolor{currentstroke}%
\pgfsetdash{}{0pt}%
\pgfsys@defobject{currentmarker}{\pgfqpoint{0.000000in}{-0.048611in}}{\pgfqpoint{0.000000in}{0.000000in}}{%
\pgfpathmoveto{\pgfqpoint{0.000000in}{0.000000in}}%
\pgfpathlineto{\pgfqpoint{0.000000in}{-0.048611in}}%
\pgfusepath{stroke,fill}%
}%
\begin{pgfscope}%
\pgfsys@transformshift{2.100228in}{0.499444in}%
\pgfsys@useobject{currentmarker}{}%
\end{pgfscope}%
\end{pgfscope}%
\begin{pgfscope}%
\pgfsetbuttcap%
\pgfsetroundjoin%
\definecolor{currentfill}{rgb}{0.000000,0.000000,0.000000}%
\pgfsetfillcolor{currentfill}%
\pgfsetlinewidth{0.803000pt}%
\definecolor{currentstroke}{rgb}{0.000000,0.000000,0.000000}%
\pgfsetstrokecolor{currentstroke}%
\pgfsetdash{}{0pt}%
\pgfsys@defobject{currentmarker}{\pgfqpoint{0.000000in}{-0.048611in}}{\pgfqpoint{0.000000in}{0.000000in}}{%
\pgfpathmoveto{\pgfqpoint{0.000000in}{0.000000in}}%
\pgfpathlineto{\pgfqpoint{0.000000in}{-0.048611in}}%
\pgfusepath{stroke,fill}%
}%
\begin{pgfscope}%
\pgfsys@transformshift{2.258750in}{0.499444in}%
\pgfsys@useobject{currentmarker}{}%
\end{pgfscope}%
\end{pgfscope}%
\begin{pgfscope}%
\definecolor{textcolor}{rgb}{0.000000,0.000000,0.000000}%
\pgfsetstrokecolor{textcolor}%
\pgfsetfillcolor{textcolor}%
\pgftext[x=2.258750in,y=0.402222in,,top]{\color{textcolor}\rmfamily\fontsize{10.000000}{12.000000}\selectfont 0.5}%
\end{pgfscope}%
\begin{pgfscope}%
\pgfsetbuttcap%
\pgfsetroundjoin%
\definecolor{currentfill}{rgb}{0.000000,0.000000,0.000000}%
\pgfsetfillcolor{currentfill}%
\pgfsetlinewidth{0.803000pt}%
\definecolor{currentstroke}{rgb}{0.000000,0.000000,0.000000}%
\pgfsetstrokecolor{currentstroke}%
\pgfsetdash{}{0pt}%
\pgfsys@defobject{currentmarker}{\pgfqpoint{0.000000in}{-0.048611in}}{\pgfqpoint{0.000000in}{0.000000in}}{%
\pgfpathmoveto{\pgfqpoint{0.000000in}{0.000000in}}%
\pgfpathlineto{\pgfqpoint{0.000000in}{-0.048611in}}%
\pgfusepath{stroke,fill}%
}%
\begin{pgfscope}%
\pgfsys@transformshift{2.417273in}{0.499444in}%
\pgfsys@useobject{currentmarker}{}%
\end{pgfscope}%
\end{pgfscope}%
\begin{pgfscope}%
\pgfsetbuttcap%
\pgfsetroundjoin%
\definecolor{currentfill}{rgb}{0.000000,0.000000,0.000000}%
\pgfsetfillcolor{currentfill}%
\pgfsetlinewidth{0.803000pt}%
\definecolor{currentstroke}{rgb}{0.000000,0.000000,0.000000}%
\pgfsetstrokecolor{currentstroke}%
\pgfsetdash{}{0pt}%
\pgfsys@defobject{currentmarker}{\pgfqpoint{0.000000in}{-0.048611in}}{\pgfqpoint{0.000000in}{0.000000in}}{%
\pgfpathmoveto{\pgfqpoint{0.000000in}{0.000000in}}%
\pgfpathlineto{\pgfqpoint{0.000000in}{-0.048611in}}%
\pgfusepath{stroke,fill}%
}%
\begin{pgfscope}%
\pgfsys@transformshift{2.575796in}{0.499444in}%
\pgfsys@useobject{currentmarker}{}%
\end{pgfscope}%
\end{pgfscope}%
\begin{pgfscope}%
\definecolor{textcolor}{rgb}{0.000000,0.000000,0.000000}%
\pgfsetstrokecolor{textcolor}%
\pgfsetfillcolor{textcolor}%
\pgftext[x=2.575796in,y=0.402222in,,top]{\color{textcolor}\rmfamily\fontsize{10.000000}{12.000000}\selectfont 0.6}%
\end{pgfscope}%
\begin{pgfscope}%
\pgfsetbuttcap%
\pgfsetroundjoin%
\definecolor{currentfill}{rgb}{0.000000,0.000000,0.000000}%
\pgfsetfillcolor{currentfill}%
\pgfsetlinewidth{0.803000pt}%
\definecolor{currentstroke}{rgb}{0.000000,0.000000,0.000000}%
\pgfsetstrokecolor{currentstroke}%
\pgfsetdash{}{0pt}%
\pgfsys@defobject{currentmarker}{\pgfqpoint{0.000000in}{-0.048611in}}{\pgfqpoint{0.000000in}{0.000000in}}{%
\pgfpathmoveto{\pgfqpoint{0.000000in}{0.000000in}}%
\pgfpathlineto{\pgfqpoint{0.000000in}{-0.048611in}}%
\pgfusepath{stroke,fill}%
}%
\begin{pgfscope}%
\pgfsys@transformshift{2.734318in}{0.499444in}%
\pgfsys@useobject{currentmarker}{}%
\end{pgfscope}%
\end{pgfscope}%
\begin{pgfscope}%
\pgfsetbuttcap%
\pgfsetroundjoin%
\definecolor{currentfill}{rgb}{0.000000,0.000000,0.000000}%
\pgfsetfillcolor{currentfill}%
\pgfsetlinewidth{0.803000pt}%
\definecolor{currentstroke}{rgb}{0.000000,0.000000,0.000000}%
\pgfsetstrokecolor{currentstroke}%
\pgfsetdash{}{0pt}%
\pgfsys@defobject{currentmarker}{\pgfqpoint{0.000000in}{-0.048611in}}{\pgfqpoint{0.000000in}{0.000000in}}{%
\pgfpathmoveto{\pgfqpoint{0.000000in}{0.000000in}}%
\pgfpathlineto{\pgfqpoint{0.000000in}{-0.048611in}}%
\pgfusepath{stroke,fill}%
}%
\begin{pgfscope}%
\pgfsys@transformshift{2.892841in}{0.499444in}%
\pgfsys@useobject{currentmarker}{}%
\end{pgfscope}%
\end{pgfscope}%
\begin{pgfscope}%
\definecolor{textcolor}{rgb}{0.000000,0.000000,0.000000}%
\pgfsetstrokecolor{textcolor}%
\pgfsetfillcolor{textcolor}%
\pgftext[x=2.892841in,y=0.402222in,,top]{\color{textcolor}\rmfamily\fontsize{10.000000}{12.000000}\selectfont 0.7}%
\end{pgfscope}%
\begin{pgfscope}%
\pgfsetbuttcap%
\pgfsetroundjoin%
\definecolor{currentfill}{rgb}{0.000000,0.000000,0.000000}%
\pgfsetfillcolor{currentfill}%
\pgfsetlinewidth{0.803000pt}%
\definecolor{currentstroke}{rgb}{0.000000,0.000000,0.000000}%
\pgfsetstrokecolor{currentstroke}%
\pgfsetdash{}{0pt}%
\pgfsys@defobject{currentmarker}{\pgfqpoint{0.000000in}{-0.048611in}}{\pgfqpoint{0.000000in}{0.000000in}}{%
\pgfpathmoveto{\pgfqpoint{0.000000in}{0.000000in}}%
\pgfpathlineto{\pgfqpoint{0.000000in}{-0.048611in}}%
\pgfusepath{stroke,fill}%
}%
\begin{pgfscope}%
\pgfsys@transformshift{3.051364in}{0.499444in}%
\pgfsys@useobject{currentmarker}{}%
\end{pgfscope}%
\end{pgfscope}%
\begin{pgfscope}%
\pgfsetbuttcap%
\pgfsetroundjoin%
\definecolor{currentfill}{rgb}{0.000000,0.000000,0.000000}%
\pgfsetfillcolor{currentfill}%
\pgfsetlinewidth{0.803000pt}%
\definecolor{currentstroke}{rgb}{0.000000,0.000000,0.000000}%
\pgfsetstrokecolor{currentstroke}%
\pgfsetdash{}{0pt}%
\pgfsys@defobject{currentmarker}{\pgfqpoint{0.000000in}{-0.048611in}}{\pgfqpoint{0.000000in}{0.000000in}}{%
\pgfpathmoveto{\pgfqpoint{0.000000in}{0.000000in}}%
\pgfpathlineto{\pgfqpoint{0.000000in}{-0.048611in}}%
\pgfusepath{stroke,fill}%
}%
\begin{pgfscope}%
\pgfsys@transformshift{3.209887in}{0.499444in}%
\pgfsys@useobject{currentmarker}{}%
\end{pgfscope}%
\end{pgfscope}%
\begin{pgfscope}%
\definecolor{textcolor}{rgb}{0.000000,0.000000,0.000000}%
\pgfsetstrokecolor{textcolor}%
\pgfsetfillcolor{textcolor}%
\pgftext[x=3.209887in,y=0.402222in,,top]{\color{textcolor}\rmfamily\fontsize{10.000000}{12.000000}\selectfont 0.8}%
\end{pgfscope}%
\begin{pgfscope}%
\pgfsetbuttcap%
\pgfsetroundjoin%
\definecolor{currentfill}{rgb}{0.000000,0.000000,0.000000}%
\pgfsetfillcolor{currentfill}%
\pgfsetlinewidth{0.803000pt}%
\definecolor{currentstroke}{rgb}{0.000000,0.000000,0.000000}%
\pgfsetstrokecolor{currentstroke}%
\pgfsetdash{}{0pt}%
\pgfsys@defobject{currentmarker}{\pgfqpoint{0.000000in}{-0.048611in}}{\pgfqpoint{0.000000in}{0.000000in}}{%
\pgfpathmoveto{\pgfqpoint{0.000000in}{0.000000in}}%
\pgfpathlineto{\pgfqpoint{0.000000in}{-0.048611in}}%
\pgfusepath{stroke,fill}%
}%
\begin{pgfscope}%
\pgfsys@transformshift{3.368409in}{0.499444in}%
\pgfsys@useobject{currentmarker}{}%
\end{pgfscope}%
\end{pgfscope}%
\begin{pgfscope}%
\pgfsetbuttcap%
\pgfsetroundjoin%
\definecolor{currentfill}{rgb}{0.000000,0.000000,0.000000}%
\pgfsetfillcolor{currentfill}%
\pgfsetlinewidth{0.803000pt}%
\definecolor{currentstroke}{rgb}{0.000000,0.000000,0.000000}%
\pgfsetstrokecolor{currentstroke}%
\pgfsetdash{}{0pt}%
\pgfsys@defobject{currentmarker}{\pgfqpoint{0.000000in}{-0.048611in}}{\pgfqpoint{0.000000in}{0.000000in}}{%
\pgfpathmoveto{\pgfqpoint{0.000000in}{0.000000in}}%
\pgfpathlineto{\pgfqpoint{0.000000in}{-0.048611in}}%
\pgfusepath{stroke,fill}%
}%
\begin{pgfscope}%
\pgfsys@transformshift{3.526932in}{0.499444in}%
\pgfsys@useobject{currentmarker}{}%
\end{pgfscope}%
\end{pgfscope}%
\begin{pgfscope}%
\definecolor{textcolor}{rgb}{0.000000,0.000000,0.000000}%
\pgfsetstrokecolor{textcolor}%
\pgfsetfillcolor{textcolor}%
\pgftext[x=3.526932in,y=0.402222in,,top]{\color{textcolor}\rmfamily\fontsize{10.000000}{12.000000}\selectfont 0.9}%
\end{pgfscope}%
\begin{pgfscope}%
\pgfsetbuttcap%
\pgfsetroundjoin%
\definecolor{currentfill}{rgb}{0.000000,0.000000,0.000000}%
\pgfsetfillcolor{currentfill}%
\pgfsetlinewidth{0.803000pt}%
\definecolor{currentstroke}{rgb}{0.000000,0.000000,0.000000}%
\pgfsetstrokecolor{currentstroke}%
\pgfsetdash{}{0pt}%
\pgfsys@defobject{currentmarker}{\pgfqpoint{0.000000in}{-0.048611in}}{\pgfqpoint{0.000000in}{0.000000in}}{%
\pgfpathmoveto{\pgfqpoint{0.000000in}{0.000000in}}%
\pgfpathlineto{\pgfqpoint{0.000000in}{-0.048611in}}%
\pgfusepath{stroke,fill}%
}%
\begin{pgfscope}%
\pgfsys@transformshift{3.685455in}{0.499444in}%
\pgfsys@useobject{currentmarker}{}%
\end{pgfscope}%
\end{pgfscope}%
\begin{pgfscope}%
\pgfsetbuttcap%
\pgfsetroundjoin%
\definecolor{currentfill}{rgb}{0.000000,0.000000,0.000000}%
\pgfsetfillcolor{currentfill}%
\pgfsetlinewidth{0.803000pt}%
\definecolor{currentstroke}{rgb}{0.000000,0.000000,0.000000}%
\pgfsetstrokecolor{currentstroke}%
\pgfsetdash{}{0pt}%
\pgfsys@defobject{currentmarker}{\pgfqpoint{0.000000in}{-0.048611in}}{\pgfqpoint{0.000000in}{0.000000in}}{%
\pgfpathmoveto{\pgfqpoint{0.000000in}{0.000000in}}%
\pgfpathlineto{\pgfqpoint{0.000000in}{-0.048611in}}%
\pgfusepath{stroke,fill}%
}%
\begin{pgfscope}%
\pgfsys@transformshift{3.843978in}{0.499444in}%
\pgfsys@useobject{currentmarker}{}%
\end{pgfscope}%
\end{pgfscope}%
\begin{pgfscope}%
\definecolor{textcolor}{rgb}{0.000000,0.000000,0.000000}%
\pgfsetstrokecolor{textcolor}%
\pgfsetfillcolor{textcolor}%
\pgftext[x=3.843978in,y=0.402222in,,top]{\color{textcolor}\rmfamily\fontsize{10.000000}{12.000000}\selectfont 1.0}%
\end{pgfscope}%
\begin{pgfscope}%
\pgfsetbuttcap%
\pgfsetroundjoin%
\definecolor{currentfill}{rgb}{0.000000,0.000000,0.000000}%
\pgfsetfillcolor{currentfill}%
\pgfsetlinewidth{0.803000pt}%
\definecolor{currentstroke}{rgb}{0.000000,0.000000,0.000000}%
\pgfsetstrokecolor{currentstroke}%
\pgfsetdash{}{0pt}%
\pgfsys@defobject{currentmarker}{\pgfqpoint{0.000000in}{-0.048611in}}{\pgfqpoint{0.000000in}{0.000000in}}{%
\pgfpathmoveto{\pgfqpoint{0.000000in}{0.000000in}}%
\pgfpathlineto{\pgfqpoint{0.000000in}{-0.048611in}}%
\pgfusepath{stroke,fill}%
}%
\begin{pgfscope}%
\pgfsys@transformshift{4.002500in}{0.499444in}%
\pgfsys@useobject{currentmarker}{}%
\end{pgfscope}%
\end{pgfscope}%
\begin{pgfscope}%
\definecolor{textcolor}{rgb}{0.000000,0.000000,0.000000}%
\pgfsetstrokecolor{textcolor}%
\pgfsetfillcolor{textcolor}%
\pgftext[x=2.258750in,y=0.223333in,,top]{\color{textcolor}\rmfamily\fontsize{10.000000}{12.000000}\selectfont \(\displaystyle p\)}%
\end{pgfscope}%
\begin{pgfscope}%
\pgfsetbuttcap%
\pgfsetroundjoin%
\definecolor{currentfill}{rgb}{0.000000,0.000000,0.000000}%
\pgfsetfillcolor{currentfill}%
\pgfsetlinewidth{0.803000pt}%
\definecolor{currentstroke}{rgb}{0.000000,0.000000,0.000000}%
\pgfsetstrokecolor{currentstroke}%
\pgfsetdash{}{0pt}%
\pgfsys@defobject{currentmarker}{\pgfqpoint{-0.048611in}{0.000000in}}{\pgfqpoint{-0.000000in}{0.000000in}}{%
\pgfpathmoveto{\pgfqpoint{-0.000000in}{0.000000in}}%
\pgfpathlineto{\pgfqpoint{-0.048611in}{0.000000in}}%
\pgfusepath{stroke,fill}%
}%
\begin{pgfscope}%
\pgfsys@transformshift{0.515000in}{0.499444in}%
\pgfsys@useobject{currentmarker}{}%
\end{pgfscope}%
\end{pgfscope}%
\begin{pgfscope}%
\definecolor{textcolor}{rgb}{0.000000,0.000000,0.000000}%
\pgfsetstrokecolor{textcolor}%
\pgfsetfillcolor{textcolor}%
\pgftext[x=0.348333in, y=0.451250in, left, base]{\color{textcolor}\rmfamily\fontsize{10.000000}{12.000000}\selectfont \(\displaystyle {0}\)}%
\end{pgfscope}%
\begin{pgfscope}%
\pgfsetbuttcap%
\pgfsetroundjoin%
\definecolor{currentfill}{rgb}{0.000000,0.000000,0.000000}%
\pgfsetfillcolor{currentfill}%
\pgfsetlinewidth{0.803000pt}%
\definecolor{currentstroke}{rgb}{0.000000,0.000000,0.000000}%
\pgfsetstrokecolor{currentstroke}%
\pgfsetdash{}{0pt}%
\pgfsys@defobject{currentmarker}{\pgfqpoint{-0.048611in}{0.000000in}}{\pgfqpoint{-0.000000in}{0.000000in}}{%
\pgfpathmoveto{\pgfqpoint{-0.000000in}{0.000000in}}%
\pgfpathlineto{\pgfqpoint{-0.048611in}{0.000000in}}%
\pgfusepath{stroke,fill}%
}%
\begin{pgfscope}%
\pgfsys@transformshift{0.515000in}{0.927911in}%
\pgfsys@useobject{currentmarker}{}%
\end{pgfscope}%
\end{pgfscope}%
\begin{pgfscope}%
\definecolor{textcolor}{rgb}{0.000000,0.000000,0.000000}%
\pgfsetstrokecolor{textcolor}%
\pgfsetfillcolor{textcolor}%
\pgftext[x=0.278889in, y=0.879717in, left, base]{\color{textcolor}\rmfamily\fontsize{10.000000}{12.000000}\selectfont \(\displaystyle {10}\)}%
\end{pgfscope}%
\begin{pgfscope}%
\pgfsetbuttcap%
\pgfsetroundjoin%
\definecolor{currentfill}{rgb}{0.000000,0.000000,0.000000}%
\pgfsetfillcolor{currentfill}%
\pgfsetlinewidth{0.803000pt}%
\definecolor{currentstroke}{rgb}{0.000000,0.000000,0.000000}%
\pgfsetstrokecolor{currentstroke}%
\pgfsetdash{}{0pt}%
\pgfsys@defobject{currentmarker}{\pgfqpoint{-0.048611in}{0.000000in}}{\pgfqpoint{-0.000000in}{0.000000in}}{%
\pgfpathmoveto{\pgfqpoint{-0.000000in}{0.000000in}}%
\pgfpathlineto{\pgfqpoint{-0.048611in}{0.000000in}}%
\pgfusepath{stroke,fill}%
}%
\begin{pgfscope}%
\pgfsys@transformshift{0.515000in}{1.356379in}%
\pgfsys@useobject{currentmarker}{}%
\end{pgfscope}%
\end{pgfscope}%
\begin{pgfscope}%
\definecolor{textcolor}{rgb}{0.000000,0.000000,0.000000}%
\pgfsetstrokecolor{textcolor}%
\pgfsetfillcolor{textcolor}%
\pgftext[x=0.278889in, y=1.308184in, left, base]{\color{textcolor}\rmfamily\fontsize{10.000000}{12.000000}\selectfont \(\displaystyle {20}\)}%
\end{pgfscope}%
\begin{pgfscope}%
\definecolor{textcolor}{rgb}{0.000000,0.000000,0.000000}%
\pgfsetstrokecolor{textcolor}%
\pgfsetfillcolor{textcolor}%
\pgftext[x=0.223333in,y=1.076944in,,bottom,rotate=90.000000]{\color{textcolor}\rmfamily\fontsize{10.000000}{12.000000}\selectfont Percent of Data Set}%
\end{pgfscope}%
\begin{pgfscope}%
\pgfsetrectcap%
\pgfsetmiterjoin%
\pgfsetlinewidth{0.803000pt}%
\definecolor{currentstroke}{rgb}{0.000000,0.000000,0.000000}%
\pgfsetstrokecolor{currentstroke}%
\pgfsetdash{}{0pt}%
\pgfpathmoveto{\pgfqpoint{0.515000in}{0.499444in}}%
\pgfpathlineto{\pgfqpoint{0.515000in}{1.654444in}}%
\pgfusepath{stroke}%
\end{pgfscope}%
\begin{pgfscope}%
\pgfsetrectcap%
\pgfsetmiterjoin%
\pgfsetlinewidth{0.803000pt}%
\definecolor{currentstroke}{rgb}{0.000000,0.000000,0.000000}%
\pgfsetstrokecolor{currentstroke}%
\pgfsetdash{}{0pt}%
\pgfpathmoveto{\pgfqpoint{4.002500in}{0.499444in}}%
\pgfpathlineto{\pgfqpoint{4.002500in}{1.654444in}}%
\pgfusepath{stroke}%
\end{pgfscope}%
\begin{pgfscope}%
\pgfsetrectcap%
\pgfsetmiterjoin%
\pgfsetlinewidth{0.803000pt}%
\definecolor{currentstroke}{rgb}{0.000000,0.000000,0.000000}%
\pgfsetstrokecolor{currentstroke}%
\pgfsetdash{}{0pt}%
\pgfpathmoveto{\pgfqpoint{0.515000in}{0.499444in}}%
\pgfpathlineto{\pgfqpoint{4.002500in}{0.499444in}}%
\pgfusepath{stroke}%
\end{pgfscope}%
\begin{pgfscope}%
\pgfsetrectcap%
\pgfsetmiterjoin%
\pgfsetlinewidth{0.803000pt}%
\definecolor{currentstroke}{rgb}{0.000000,0.000000,0.000000}%
\pgfsetstrokecolor{currentstroke}%
\pgfsetdash{}{0pt}%
\pgfpathmoveto{\pgfqpoint{0.515000in}{1.654444in}}%
\pgfpathlineto{\pgfqpoint{4.002500in}{1.654444in}}%
\pgfusepath{stroke}%
\end{pgfscope}%
\begin{pgfscope}%
\pgfsetbuttcap%
\pgfsetmiterjoin%
\definecolor{currentfill}{rgb}{1.000000,1.000000,1.000000}%
\pgfsetfillcolor{currentfill}%
\pgfsetfillopacity{0.800000}%
\pgfsetlinewidth{1.003750pt}%
\definecolor{currentstroke}{rgb}{0.800000,0.800000,0.800000}%
\pgfsetstrokecolor{currentstroke}%
\pgfsetstrokeopacity{0.800000}%
\pgfsetdash{}{0pt}%
\pgfpathmoveto{\pgfqpoint{3.225556in}{1.154445in}}%
\pgfpathlineto{\pgfqpoint{3.905278in}{1.154445in}}%
\pgfpathquadraticcurveto{\pgfqpoint{3.933056in}{1.154445in}}{\pgfqpoint{3.933056in}{1.182222in}}%
\pgfpathlineto{\pgfqpoint{3.933056in}{1.557222in}}%
\pgfpathquadraticcurveto{\pgfqpoint{3.933056in}{1.585000in}}{\pgfqpoint{3.905278in}{1.585000in}}%
\pgfpathlineto{\pgfqpoint{3.225556in}{1.585000in}}%
\pgfpathquadraticcurveto{\pgfqpoint{3.197778in}{1.585000in}}{\pgfqpoint{3.197778in}{1.557222in}}%
\pgfpathlineto{\pgfqpoint{3.197778in}{1.182222in}}%
\pgfpathquadraticcurveto{\pgfqpoint{3.197778in}{1.154445in}}{\pgfqpoint{3.225556in}{1.154445in}}%
\pgfpathlineto{\pgfqpoint{3.225556in}{1.154445in}}%
\pgfpathclose%
\pgfusepath{stroke,fill}%
\end{pgfscope}%
\begin{pgfscope}%
\pgfsetbuttcap%
\pgfsetmiterjoin%
\pgfsetlinewidth{1.003750pt}%
\definecolor{currentstroke}{rgb}{0.000000,0.000000,0.000000}%
\pgfsetstrokecolor{currentstroke}%
\pgfsetdash{}{0pt}%
\pgfpathmoveto{\pgfqpoint{3.253334in}{1.432222in}}%
\pgfpathlineto{\pgfqpoint{3.531111in}{1.432222in}}%
\pgfpathlineto{\pgfqpoint{3.531111in}{1.529444in}}%
\pgfpathlineto{\pgfqpoint{3.253334in}{1.529444in}}%
\pgfpathlineto{\pgfqpoint{3.253334in}{1.432222in}}%
\pgfpathclose%
\pgfusepath{stroke}%
\end{pgfscope}%
\begin{pgfscope}%
\definecolor{textcolor}{rgb}{0.000000,0.000000,0.000000}%
\pgfsetstrokecolor{textcolor}%
\pgfsetfillcolor{textcolor}%
\pgftext[x=3.642223in,y=1.432222in,left,base]{\color{textcolor}\rmfamily\fontsize{10.000000}{12.000000}\selectfont Neg}%
\end{pgfscope}%
\begin{pgfscope}%
\pgfsetbuttcap%
\pgfsetmiterjoin%
\definecolor{currentfill}{rgb}{0.000000,0.000000,0.000000}%
\pgfsetfillcolor{currentfill}%
\pgfsetlinewidth{0.000000pt}%
\definecolor{currentstroke}{rgb}{0.000000,0.000000,0.000000}%
\pgfsetstrokecolor{currentstroke}%
\pgfsetstrokeopacity{0.000000}%
\pgfsetdash{}{0pt}%
\pgfpathmoveto{\pgfqpoint{3.253334in}{1.236944in}}%
\pgfpathlineto{\pgfqpoint{3.531111in}{1.236944in}}%
\pgfpathlineto{\pgfqpoint{3.531111in}{1.334167in}}%
\pgfpathlineto{\pgfqpoint{3.253334in}{1.334167in}}%
\pgfpathlineto{\pgfqpoint{3.253334in}{1.236944in}}%
\pgfpathclose%
\pgfusepath{fill}%
\end{pgfscope}%
\begin{pgfscope}%
\definecolor{textcolor}{rgb}{0.000000,0.000000,0.000000}%
\pgfsetstrokecolor{textcolor}%
\pgfsetfillcolor{textcolor}%
\pgftext[x=3.642223in,y=1.236944in,left,base]{\color{textcolor}\rmfamily\fontsize{10.000000}{12.000000}\selectfont Pos}%
\end{pgfscope}%
\end{pgfpicture}%
\makeatother%
\endgroup%
	
&
	\vskip 0pt
	\hfil ROC Curve
	
	%% Creator: Matplotlib, PGF backend
%%
%% To include the figure in your LaTeX document, write
%%   \input{<filename>.pgf}
%%
%% Make sure the required packages are loaded in your preamble
%%   \usepackage{pgf}
%%
%% Also ensure that all the required font packages are loaded; for instance,
%% the lmodern package is sometimes necessary when using math font.
%%   \usepackage{lmodern}
%%
%% Figures using additional raster images can only be included by \input if
%% they are in the same directory as the main LaTeX file. For loading figures
%% from other directories you can use the `import` package
%%   \usepackage{import}
%%
%% and then include the figures with
%%   \import{<path to file>}{<filename>.pgf}
%%
%% Matplotlib used the following preamble
%%   
%%   \usepackage{fontspec}
%%   \makeatletter\@ifpackageloaded{underscore}{}{\usepackage[strings]{underscore}}\makeatother
%%
\begingroup%
\makeatletter%
\begin{pgfpicture}%
\pgfpathrectangle{\pgfpointorigin}{\pgfqpoint{2.221861in}{1.754444in}}%
\pgfusepath{use as bounding box, clip}%
\begin{pgfscope}%
\pgfsetbuttcap%
\pgfsetmiterjoin%
\definecolor{currentfill}{rgb}{1.000000,1.000000,1.000000}%
\pgfsetfillcolor{currentfill}%
\pgfsetlinewidth{0.000000pt}%
\definecolor{currentstroke}{rgb}{1.000000,1.000000,1.000000}%
\pgfsetstrokecolor{currentstroke}%
\pgfsetdash{}{0pt}%
\pgfpathmoveto{\pgfqpoint{0.000000in}{0.000000in}}%
\pgfpathlineto{\pgfqpoint{2.221861in}{0.000000in}}%
\pgfpathlineto{\pgfqpoint{2.221861in}{1.754444in}}%
\pgfpathlineto{\pgfqpoint{0.000000in}{1.754444in}}%
\pgfpathlineto{\pgfqpoint{0.000000in}{0.000000in}}%
\pgfpathclose%
\pgfusepath{fill}%
\end{pgfscope}%
\begin{pgfscope}%
\pgfsetbuttcap%
\pgfsetmiterjoin%
\definecolor{currentfill}{rgb}{1.000000,1.000000,1.000000}%
\pgfsetfillcolor{currentfill}%
\pgfsetlinewidth{0.000000pt}%
\definecolor{currentstroke}{rgb}{0.000000,0.000000,0.000000}%
\pgfsetstrokecolor{currentstroke}%
\pgfsetstrokeopacity{0.000000}%
\pgfsetdash{}{0pt}%
\pgfpathmoveto{\pgfqpoint{0.553581in}{0.499444in}}%
\pgfpathlineto{\pgfqpoint{2.103581in}{0.499444in}}%
\pgfpathlineto{\pgfqpoint{2.103581in}{1.654444in}}%
\pgfpathlineto{\pgfqpoint{0.553581in}{1.654444in}}%
\pgfpathlineto{\pgfqpoint{0.553581in}{0.499444in}}%
\pgfpathclose%
\pgfusepath{fill}%
\end{pgfscope}%
\begin{pgfscope}%
\pgfsetbuttcap%
\pgfsetroundjoin%
\definecolor{currentfill}{rgb}{0.000000,0.000000,0.000000}%
\pgfsetfillcolor{currentfill}%
\pgfsetlinewidth{0.803000pt}%
\definecolor{currentstroke}{rgb}{0.000000,0.000000,0.000000}%
\pgfsetstrokecolor{currentstroke}%
\pgfsetdash{}{0pt}%
\pgfsys@defobject{currentmarker}{\pgfqpoint{0.000000in}{-0.048611in}}{\pgfqpoint{0.000000in}{0.000000in}}{%
\pgfpathmoveto{\pgfqpoint{0.000000in}{0.000000in}}%
\pgfpathlineto{\pgfqpoint{0.000000in}{-0.048611in}}%
\pgfusepath{stroke,fill}%
}%
\begin{pgfscope}%
\pgfsys@transformshift{0.624035in}{0.499444in}%
\pgfsys@useobject{currentmarker}{}%
\end{pgfscope}%
\end{pgfscope}%
\begin{pgfscope}%
\definecolor{textcolor}{rgb}{0.000000,0.000000,0.000000}%
\pgfsetstrokecolor{textcolor}%
\pgfsetfillcolor{textcolor}%
\pgftext[x=0.624035in,y=0.402222in,,top]{\color{textcolor}\rmfamily\fontsize{10.000000}{12.000000}\selectfont \(\displaystyle {0.0}\)}%
\end{pgfscope}%
\begin{pgfscope}%
\pgfsetbuttcap%
\pgfsetroundjoin%
\definecolor{currentfill}{rgb}{0.000000,0.000000,0.000000}%
\pgfsetfillcolor{currentfill}%
\pgfsetlinewidth{0.803000pt}%
\definecolor{currentstroke}{rgb}{0.000000,0.000000,0.000000}%
\pgfsetstrokecolor{currentstroke}%
\pgfsetdash{}{0pt}%
\pgfsys@defobject{currentmarker}{\pgfqpoint{0.000000in}{-0.048611in}}{\pgfqpoint{0.000000in}{0.000000in}}{%
\pgfpathmoveto{\pgfqpoint{0.000000in}{0.000000in}}%
\pgfpathlineto{\pgfqpoint{0.000000in}{-0.048611in}}%
\pgfusepath{stroke,fill}%
}%
\begin{pgfscope}%
\pgfsys@transformshift{1.328581in}{0.499444in}%
\pgfsys@useobject{currentmarker}{}%
\end{pgfscope}%
\end{pgfscope}%
\begin{pgfscope}%
\definecolor{textcolor}{rgb}{0.000000,0.000000,0.000000}%
\pgfsetstrokecolor{textcolor}%
\pgfsetfillcolor{textcolor}%
\pgftext[x=1.328581in,y=0.402222in,,top]{\color{textcolor}\rmfamily\fontsize{10.000000}{12.000000}\selectfont \(\displaystyle {0.5}\)}%
\end{pgfscope}%
\begin{pgfscope}%
\pgfsetbuttcap%
\pgfsetroundjoin%
\definecolor{currentfill}{rgb}{0.000000,0.000000,0.000000}%
\pgfsetfillcolor{currentfill}%
\pgfsetlinewidth{0.803000pt}%
\definecolor{currentstroke}{rgb}{0.000000,0.000000,0.000000}%
\pgfsetstrokecolor{currentstroke}%
\pgfsetdash{}{0pt}%
\pgfsys@defobject{currentmarker}{\pgfqpoint{0.000000in}{-0.048611in}}{\pgfqpoint{0.000000in}{0.000000in}}{%
\pgfpathmoveto{\pgfqpoint{0.000000in}{0.000000in}}%
\pgfpathlineto{\pgfqpoint{0.000000in}{-0.048611in}}%
\pgfusepath{stroke,fill}%
}%
\begin{pgfscope}%
\pgfsys@transformshift{2.033126in}{0.499444in}%
\pgfsys@useobject{currentmarker}{}%
\end{pgfscope}%
\end{pgfscope}%
\begin{pgfscope}%
\definecolor{textcolor}{rgb}{0.000000,0.000000,0.000000}%
\pgfsetstrokecolor{textcolor}%
\pgfsetfillcolor{textcolor}%
\pgftext[x=2.033126in,y=0.402222in,,top]{\color{textcolor}\rmfamily\fontsize{10.000000}{12.000000}\selectfont \(\displaystyle {1.0}\)}%
\end{pgfscope}%
\begin{pgfscope}%
\definecolor{textcolor}{rgb}{0.000000,0.000000,0.000000}%
\pgfsetstrokecolor{textcolor}%
\pgfsetfillcolor{textcolor}%
\pgftext[x=1.328581in,y=0.223333in,,top]{\color{textcolor}\rmfamily\fontsize{10.000000}{12.000000}\selectfont False positive rate}%
\end{pgfscope}%
\begin{pgfscope}%
\pgfsetbuttcap%
\pgfsetroundjoin%
\definecolor{currentfill}{rgb}{0.000000,0.000000,0.000000}%
\pgfsetfillcolor{currentfill}%
\pgfsetlinewidth{0.803000pt}%
\definecolor{currentstroke}{rgb}{0.000000,0.000000,0.000000}%
\pgfsetstrokecolor{currentstroke}%
\pgfsetdash{}{0pt}%
\pgfsys@defobject{currentmarker}{\pgfqpoint{-0.048611in}{0.000000in}}{\pgfqpoint{-0.000000in}{0.000000in}}{%
\pgfpathmoveto{\pgfqpoint{-0.000000in}{0.000000in}}%
\pgfpathlineto{\pgfqpoint{-0.048611in}{0.000000in}}%
\pgfusepath{stroke,fill}%
}%
\begin{pgfscope}%
\pgfsys@transformshift{0.553581in}{0.551944in}%
\pgfsys@useobject{currentmarker}{}%
\end{pgfscope}%
\end{pgfscope}%
\begin{pgfscope}%
\definecolor{textcolor}{rgb}{0.000000,0.000000,0.000000}%
\pgfsetstrokecolor{textcolor}%
\pgfsetfillcolor{textcolor}%
\pgftext[x=0.278889in, y=0.503750in, left, base]{\color{textcolor}\rmfamily\fontsize{10.000000}{12.000000}\selectfont \(\displaystyle {0.0}\)}%
\end{pgfscope}%
\begin{pgfscope}%
\pgfsetbuttcap%
\pgfsetroundjoin%
\definecolor{currentfill}{rgb}{0.000000,0.000000,0.000000}%
\pgfsetfillcolor{currentfill}%
\pgfsetlinewidth{0.803000pt}%
\definecolor{currentstroke}{rgb}{0.000000,0.000000,0.000000}%
\pgfsetstrokecolor{currentstroke}%
\pgfsetdash{}{0pt}%
\pgfsys@defobject{currentmarker}{\pgfqpoint{-0.048611in}{0.000000in}}{\pgfqpoint{-0.000000in}{0.000000in}}{%
\pgfpathmoveto{\pgfqpoint{-0.000000in}{0.000000in}}%
\pgfpathlineto{\pgfqpoint{-0.048611in}{0.000000in}}%
\pgfusepath{stroke,fill}%
}%
\begin{pgfscope}%
\pgfsys@transformshift{0.553581in}{1.076944in}%
\pgfsys@useobject{currentmarker}{}%
\end{pgfscope}%
\end{pgfscope}%
\begin{pgfscope}%
\definecolor{textcolor}{rgb}{0.000000,0.000000,0.000000}%
\pgfsetstrokecolor{textcolor}%
\pgfsetfillcolor{textcolor}%
\pgftext[x=0.278889in, y=1.028750in, left, base]{\color{textcolor}\rmfamily\fontsize{10.000000}{12.000000}\selectfont \(\displaystyle {0.5}\)}%
\end{pgfscope}%
\begin{pgfscope}%
\pgfsetbuttcap%
\pgfsetroundjoin%
\definecolor{currentfill}{rgb}{0.000000,0.000000,0.000000}%
\pgfsetfillcolor{currentfill}%
\pgfsetlinewidth{0.803000pt}%
\definecolor{currentstroke}{rgb}{0.000000,0.000000,0.000000}%
\pgfsetstrokecolor{currentstroke}%
\pgfsetdash{}{0pt}%
\pgfsys@defobject{currentmarker}{\pgfqpoint{-0.048611in}{0.000000in}}{\pgfqpoint{-0.000000in}{0.000000in}}{%
\pgfpathmoveto{\pgfqpoint{-0.000000in}{0.000000in}}%
\pgfpathlineto{\pgfqpoint{-0.048611in}{0.000000in}}%
\pgfusepath{stroke,fill}%
}%
\begin{pgfscope}%
\pgfsys@transformshift{0.553581in}{1.601944in}%
\pgfsys@useobject{currentmarker}{}%
\end{pgfscope}%
\end{pgfscope}%
\begin{pgfscope}%
\definecolor{textcolor}{rgb}{0.000000,0.000000,0.000000}%
\pgfsetstrokecolor{textcolor}%
\pgfsetfillcolor{textcolor}%
\pgftext[x=0.278889in, y=1.553750in, left, base]{\color{textcolor}\rmfamily\fontsize{10.000000}{12.000000}\selectfont \(\displaystyle {1.0}\)}%
\end{pgfscope}%
\begin{pgfscope}%
\definecolor{textcolor}{rgb}{0.000000,0.000000,0.000000}%
\pgfsetstrokecolor{textcolor}%
\pgfsetfillcolor{textcolor}%
\pgftext[x=0.223333in,y=1.076944in,,bottom,rotate=90.000000]{\color{textcolor}\rmfamily\fontsize{10.000000}{12.000000}\selectfont True positive rate}%
\end{pgfscope}%
\begin{pgfscope}%
\pgfpathrectangle{\pgfqpoint{0.553581in}{0.499444in}}{\pgfqpoint{1.550000in}{1.155000in}}%
\pgfusepath{clip}%
\pgfsetbuttcap%
\pgfsetroundjoin%
\pgfsetlinewidth{1.505625pt}%
\definecolor{currentstroke}{rgb}{0.000000,0.000000,0.000000}%
\pgfsetstrokecolor{currentstroke}%
\pgfsetdash{{5.550000pt}{2.400000pt}}{0.000000pt}%
\pgfpathmoveto{\pgfqpoint{0.624035in}{0.551944in}}%
\pgfpathlineto{\pgfqpoint{2.033126in}{1.601944in}}%
\pgfusepath{stroke}%
\end{pgfscope}%
\begin{pgfscope}%
\pgfpathrectangle{\pgfqpoint{0.553581in}{0.499444in}}{\pgfqpoint{1.550000in}{1.155000in}}%
\pgfusepath{clip}%
\pgfsetrectcap%
\pgfsetroundjoin%
\pgfsetlinewidth{1.505625pt}%
\definecolor{currentstroke}{rgb}{0.000000,0.000000,0.000000}%
\pgfsetstrokecolor{currentstroke}%
\pgfsetdash{}{0pt}%
\pgfpathmoveto{\pgfqpoint{0.624035in}{0.551944in}}%
\pgfpathlineto{\pgfqpoint{0.625145in}{0.568893in}}%
\pgfpathlineto{\pgfqpoint{0.625239in}{0.569949in}}%
\pgfpathlineto{\pgfqpoint{0.626341in}{0.584414in}}%
\pgfpathlineto{\pgfqpoint{0.626388in}{0.585345in}}%
\pgfpathlineto{\pgfqpoint{0.627498in}{0.595310in}}%
\pgfpathlineto{\pgfqpoint{0.627616in}{0.596396in}}%
\pgfpathlineto{\pgfqpoint{0.628726in}{0.605306in}}%
\pgfpathlineto{\pgfqpoint{0.628781in}{0.606237in}}%
\pgfpathlineto{\pgfqpoint{0.629891in}{0.616698in}}%
\pgfpathlineto{\pgfqpoint{0.629992in}{0.617785in}}%
\pgfpathlineto{\pgfqpoint{0.631102in}{0.624986in}}%
\pgfpathlineto{\pgfqpoint{0.631235in}{0.625887in}}%
\pgfpathlineto{\pgfqpoint{0.632330in}{0.633150in}}%
\pgfpathlineto{\pgfqpoint{0.632533in}{0.634175in}}%
\pgfpathlineto{\pgfqpoint{0.633643in}{0.643208in}}%
\pgfpathlineto{\pgfqpoint{0.633885in}{0.644263in}}%
\pgfpathlineto{\pgfqpoint{0.634996in}{0.651869in}}%
\pgfpathlineto{\pgfqpoint{0.635128in}{0.652924in}}%
\pgfpathlineto{\pgfqpoint{0.636239in}{0.659785in}}%
\pgfpathlineto{\pgfqpoint{0.636442in}{0.660840in}}%
\pgfpathlineto{\pgfqpoint{0.637552in}{0.666862in}}%
\pgfpathlineto{\pgfqpoint{0.637833in}{0.667949in}}%
\pgfpathlineto{\pgfqpoint{0.638943in}{0.674219in}}%
\pgfpathlineto{\pgfqpoint{0.639084in}{0.675306in}}%
\pgfpathlineto{\pgfqpoint{0.640194in}{0.680955in}}%
\pgfpathlineto{\pgfqpoint{0.640444in}{0.682042in}}%
\pgfpathlineto{\pgfqpoint{0.641547in}{0.687567in}}%
\pgfpathlineto{\pgfqpoint{0.641766in}{0.688654in}}%
\pgfpathlineto{\pgfqpoint{0.642868in}{0.693124in}}%
\pgfpathlineto{\pgfqpoint{0.643040in}{0.694210in}}%
\pgfpathlineto{\pgfqpoint{0.644111in}{0.699549in}}%
\pgfpathlineto{\pgfqpoint{0.644431in}{0.700574in}}%
\pgfpathlineto{\pgfqpoint{0.645542in}{0.705696in}}%
\pgfpathlineto{\pgfqpoint{0.645901in}{0.706782in}}%
\pgfpathlineto{\pgfqpoint{0.647003in}{0.711687in}}%
\pgfpathlineto{\pgfqpoint{0.647308in}{0.712773in}}%
\pgfpathlineto{\pgfqpoint{0.648418in}{0.717585in}}%
\pgfpathlineto{\pgfqpoint{0.648661in}{0.718609in}}%
\pgfpathlineto{\pgfqpoint{0.649771in}{0.723390in}}%
\pgfpathlineto{\pgfqpoint{0.650076in}{0.724476in}}%
\pgfpathlineto{\pgfqpoint{0.651186in}{0.727953in}}%
\pgfpathlineto{\pgfqpoint{0.651483in}{0.729040in}}%
\pgfpathlineto{\pgfqpoint{0.652593in}{0.733820in}}%
\pgfpathlineto{\pgfqpoint{0.652921in}{0.734813in}}%
\pgfpathlineto{\pgfqpoint{0.654016in}{0.740059in}}%
\pgfpathlineto{\pgfqpoint{0.654462in}{0.741115in}}%
\pgfpathlineto{\pgfqpoint{0.655556in}{0.745740in}}%
\pgfpathlineto{\pgfqpoint{0.655861in}{0.746827in}}%
\pgfpathlineto{\pgfqpoint{0.656971in}{0.750334in}}%
\pgfpathlineto{\pgfqpoint{0.657323in}{0.751390in}}%
\pgfpathlineto{\pgfqpoint{0.658425in}{0.754711in}}%
\pgfpathlineto{\pgfqpoint{0.658800in}{0.755736in}}%
\pgfpathlineto{\pgfqpoint{0.659895in}{0.759492in}}%
\pgfpathlineto{\pgfqpoint{0.660161in}{0.760516in}}%
\pgfpathlineto{\pgfqpoint{0.661271in}{0.765173in}}%
\pgfpathlineto{\pgfqpoint{0.661544in}{0.766228in}}%
\pgfpathlineto{\pgfqpoint{0.662647in}{0.769922in}}%
\pgfpathlineto{\pgfqpoint{0.663037in}{0.770853in}}%
\pgfpathlineto{\pgfqpoint{0.664148in}{0.774144in}}%
\pgfpathlineto{\pgfqpoint{0.664531in}{0.775199in}}%
\pgfpathlineto{\pgfqpoint{0.665641in}{0.778769in}}%
\pgfpathlineto{\pgfqpoint{0.666094in}{0.779855in}}%
\pgfpathlineto{\pgfqpoint{0.667204in}{0.783239in}}%
\pgfpathlineto{\pgfqpoint{0.667462in}{0.784294in}}%
\pgfpathlineto{\pgfqpoint{0.668565in}{0.787802in}}%
\pgfpathlineto{\pgfqpoint{0.669010in}{0.788889in}}%
\pgfpathlineto{\pgfqpoint{0.670120in}{0.792521in}}%
\pgfpathlineto{\pgfqpoint{0.670511in}{0.793607in}}%
\pgfpathlineto{\pgfqpoint{0.671582in}{0.797860in}}%
\pgfpathlineto{\pgfqpoint{0.671981in}{0.798915in}}%
\pgfpathlineto{\pgfqpoint{0.671981in}{0.798946in}}%
\pgfpathlineto{\pgfqpoint{0.673091in}{0.802113in}}%
\pgfpathlineto{\pgfqpoint{0.673568in}{0.803199in}}%
\pgfpathlineto{\pgfqpoint{0.674670in}{0.806241in}}%
\pgfpathlineto{\pgfqpoint{0.675163in}{0.807328in}}%
\pgfpathlineto{\pgfqpoint{0.676265in}{0.810370in}}%
\pgfpathlineto{\pgfqpoint{0.676562in}{0.811425in}}%
\pgfpathlineto{\pgfqpoint{0.677672in}{0.814467in}}%
\pgfpathlineto{\pgfqpoint{0.678079in}{0.815461in}}%
\pgfpathlineto{\pgfqpoint{0.679181in}{0.819093in}}%
\pgfpathlineto{\pgfqpoint{0.679689in}{0.820148in}}%
\pgfpathlineto{\pgfqpoint{0.680776in}{0.823066in}}%
\pgfpathlineto{\pgfqpoint{0.681104in}{0.824091in}}%
\pgfpathlineto{\pgfqpoint{0.682214in}{0.826760in}}%
\pgfpathlineto{\pgfqpoint{0.682668in}{0.827847in}}%
\pgfpathlineto{\pgfqpoint{0.683778in}{0.831013in}}%
\pgfpathlineto{\pgfqpoint{0.684239in}{0.832099in}}%
\pgfpathlineto{\pgfqpoint{0.685302in}{0.834831in}}%
\pgfpathlineto{\pgfqpoint{0.685888in}{0.835918in}}%
\pgfpathlineto{\pgfqpoint{0.686991in}{0.839332in}}%
\pgfpathlineto{\pgfqpoint{0.687475in}{0.840419in}}%
\pgfpathlineto{\pgfqpoint{0.688578in}{0.843523in}}%
\pgfpathlineto{\pgfqpoint{0.689039in}{0.844578in}}%
\pgfpathlineto{\pgfqpoint{0.690141in}{0.846844in}}%
\pgfpathlineto{\pgfqpoint{0.690790in}{0.847931in}}%
\pgfpathlineto{\pgfqpoint{0.691900in}{0.850414in}}%
\pgfpathlineto{\pgfqpoint{0.692393in}{0.851439in}}%
\pgfpathlineto{\pgfqpoint{0.693495in}{0.853922in}}%
\pgfpathlineto{\pgfqpoint{0.693792in}{0.854915in}}%
\pgfpathlineto{\pgfqpoint{0.694871in}{0.857926in}}%
\pgfpathlineto{\pgfqpoint{0.695301in}{0.859013in}}%
\pgfpathlineto{\pgfqpoint{0.696395in}{0.861372in}}%
\pgfpathlineto{\pgfqpoint{0.696880in}{0.862459in}}%
\pgfpathlineto{\pgfqpoint{0.697990in}{0.865345in}}%
\pgfpathlineto{\pgfqpoint{0.698475in}{0.866370in}}%
\pgfpathlineto{\pgfqpoint{0.699538in}{0.869381in}}%
\pgfpathlineto{\pgfqpoint{0.700023in}{0.870405in}}%
\pgfpathlineto{\pgfqpoint{0.701133in}{0.873013in}}%
\pgfpathlineto{\pgfqpoint{0.701672in}{0.874037in}}%
\pgfpathlineto{\pgfqpoint{0.702751in}{0.876521in}}%
\pgfpathlineto{\pgfqpoint{0.703181in}{0.877607in}}%
\pgfpathlineto{\pgfqpoint{0.704291in}{0.880494in}}%
\pgfpathlineto{\pgfqpoint{0.704838in}{0.881581in}}%
\pgfpathlineto{\pgfqpoint{0.705941in}{0.884561in}}%
\pgfpathlineto{\pgfqpoint{0.706543in}{0.885585in}}%
\pgfpathlineto{\pgfqpoint{0.707653in}{0.887975in}}%
\pgfpathlineto{\pgfqpoint{0.708075in}{0.889062in}}%
\pgfpathlineto{\pgfqpoint{0.709185in}{0.891762in}}%
\pgfpathlineto{\pgfqpoint{0.709842in}{0.892849in}}%
\pgfpathlineto{\pgfqpoint{0.710928in}{0.895860in}}%
\pgfpathlineto{\pgfqpoint{0.711476in}{0.896946in}}%
\pgfpathlineto{\pgfqpoint{0.712547in}{0.899554in}}%
\pgfpathlineto{\pgfqpoint{0.713297in}{0.900640in}}%
\pgfpathlineto{\pgfqpoint{0.714392in}{0.902844in}}%
\pgfpathlineto{\pgfqpoint{0.714853in}{0.903745in}}%
\pgfpathlineto{\pgfqpoint{0.714853in}{0.903900in}}%
\pgfpathlineto{\pgfqpoint{0.715955in}{0.906290in}}%
\pgfpathlineto{\pgfqpoint{0.716643in}{0.907377in}}%
\pgfpathlineto{\pgfqpoint{0.717753in}{0.909363in}}%
\pgfpathlineto{\pgfqpoint{0.718128in}{0.910450in}}%
\pgfpathlineto{\pgfqpoint{0.719238in}{0.912964in}}%
\pgfpathlineto{\pgfqpoint{0.719723in}{0.914051in}}%
\pgfpathlineto{\pgfqpoint{0.720825in}{0.916161in}}%
\pgfpathlineto{\pgfqpoint{0.721631in}{0.917248in}}%
\pgfpathlineto{\pgfqpoint{0.722725in}{0.919142in}}%
\pgfpathlineto{\pgfqpoint{0.723225in}{0.920228in}}%
\pgfpathlineto{\pgfqpoint{0.724320in}{0.923115in}}%
\pgfpathlineto{\pgfqpoint{0.724875in}{0.924201in}}%
\pgfpathlineto{\pgfqpoint{0.725954in}{0.926312in}}%
\pgfpathlineto{\pgfqpoint{0.726525in}{0.927399in}}%
\pgfpathlineto{\pgfqpoint{0.727627in}{0.929789in}}%
\pgfpathlineto{\pgfqpoint{0.728221in}{0.930875in}}%
\pgfpathlineto{\pgfqpoint{0.729323in}{0.933079in}}%
\pgfpathlineto{\pgfqpoint{0.729824in}{0.934166in}}%
\pgfpathlineto{\pgfqpoint{0.730887in}{0.936184in}}%
\pgfpathlineto{\pgfqpoint{0.731637in}{0.937270in}}%
\pgfpathlineto{\pgfqpoint{0.732732in}{0.940033in}}%
\pgfpathlineto{\pgfqpoint{0.733240in}{0.941119in}}%
\pgfpathlineto{\pgfqpoint{0.734334in}{0.943447in}}%
\pgfpathlineto{\pgfqpoint{0.734975in}{0.944534in}}%
\pgfpathlineto{\pgfqpoint{0.736078in}{0.947017in}}%
\pgfpathlineto{\pgfqpoint{0.736953in}{0.948104in}}%
\pgfpathlineto{\pgfqpoint{0.738040in}{0.949997in}}%
\pgfpathlineto{\pgfqpoint{0.738470in}{0.951084in}}%
\pgfpathlineto{\pgfqpoint{0.739549in}{0.953381in}}%
\pgfpathlineto{\pgfqpoint{0.739572in}{0.953381in}}%
\pgfpathlineto{\pgfqpoint{0.740049in}{0.954467in}}%
\pgfpathlineto{\pgfqpoint{0.741159in}{0.955957in}}%
\pgfpathlineto{\pgfqpoint{0.741706in}{0.957044in}}%
\pgfpathlineto{\pgfqpoint{0.742770in}{0.959031in}}%
\pgfpathlineto{\pgfqpoint{0.742801in}{0.959031in}}%
\pgfpathlineto{\pgfqpoint{0.743340in}{0.960117in}}%
\pgfpathlineto{\pgfqpoint{0.744450in}{0.961980in}}%
\pgfpathlineto{\pgfqpoint{0.745334in}{0.963066in}}%
\pgfpathlineto{\pgfqpoint{0.746413in}{0.965953in}}%
\pgfpathlineto{\pgfqpoint{0.746944in}{0.967040in}}%
\pgfpathlineto{\pgfqpoint{0.748046in}{0.969181in}}%
\pgfpathlineto{\pgfqpoint{0.748688in}{0.970268in}}%
\pgfpathlineto{\pgfqpoint{0.749790in}{0.971882in}}%
\pgfpathlineto{\pgfqpoint{0.750533in}{0.972969in}}%
\pgfpathlineto{\pgfqpoint{0.751627in}{0.974924in}}%
\pgfpathlineto{\pgfqpoint{0.752714in}{0.976011in}}%
\pgfpathlineto{\pgfqpoint{0.753808in}{0.977501in}}%
\pgfpathlineto{\pgfqpoint{0.754535in}{0.978587in}}%
\pgfpathlineto{\pgfqpoint{0.755630in}{0.980481in}}%
\pgfpathlineto{\pgfqpoint{0.756318in}{0.981567in}}%
\pgfpathlineto{\pgfqpoint{0.757412in}{0.983275in}}%
\pgfpathlineto{\pgfqpoint{0.757428in}{0.983275in}}%
\pgfpathlineto{\pgfqpoint{0.758092in}{0.984330in}}%
\pgfpathlineto{\pgfqpoint{0.759179in}{0.985975in}}%
\pgfpathlineto{\pgfqpoint{0.759945in}{0.987062in}}%
\pgfpathlineto{\pgfqpoint{0.761024in}{0.988738in}}%
\pgfpathlineto{\pgfqpoint{0.761884in}{0.989824in}}%
\pgfpathlineto{\pgfqpoint{0.762963in}{0.991749in}}%
\pgfpathlineto{\pgfqpoint{0.763705in}{0.992836in}}%
\pgfpathlineto{\pgfqpoint{0.764815in}{0.994543in}}%
\pgfpathlineto{\pgfqpoint{0.765433in}{0.995629in}}%
\pgfpathlineto{\pgfqpoint{0.766512in}{0.997647in}}%
\pgfpathlineto{\pgfqpoint{0.767270in}{0.998702in}}%
\pgfpathlineto{\pgfqpoint{0.768380in}{1.001062in}}%
\pgfpathlineto{\pgfqpoint{0.768998in}{1.002117in}}%
\pgfpathlineto{\pgfqpoint{0.770069in}{1.004197in}}%
\pgfpathlineto{\pgfqpoint{0.770600in}{1.005221in}}%
\pgfpathlineto{\pgfqpoint{0.771679in}{1.006804in}}%
\pgfpathlineto{\pgfqpoint{0.771695in}{1.006804in}}%
\pgfpathlineto{\pgfqpoint{0.772289in}{1.007891in}}%
\pgfpathlineto{\pgfqpoint{0.773383in}{1.009412in}}%
\pgfpathlineto{\pgfqpoint{0.774032in}{1.010498in}}%
\pgfpathlineto{\pgfqpoint{0.775142in}{1.011895in}}%
\pgfpathlineto{\pgfqpoint{0.775987in}{1.012951in}}%
\pgfpathlineto{\pgfqpoint{0.777097in}{1.014844in}}%
\pgfpathlineto{\pgfqpoint{0.777754in}{1.015931in}}%
\pgfpathlineto{\pgfqpoint{0.778840in}{1.017762in}}%
\pgfpathlineto{\pgfqpoint{0.779599in}{1.018849in}}%
\pgfpathlineto{\pgfqpoint{0.780709in}{1.021053in}}%
\pgfpathlineto{\pgfqpoint{0.781608in}{1.022139in}}%
\pgfpathlineto{\pgfqpoint{0.782640in}{1.023691in}}%
\pgfpathlineto{\pgfqpoint{0.783437in}{1.024778in}}%
\pgfpathlineto{\pgfqpoint{0.784508in}{1.026485in}}%
\pgfpathlineto{\pgfqpoint{0.785274in}{1.027572in}}%
\pgfpathlineto{\pgfqpoint{0.786376in}{1.029434in}}%
\pgfpathlineto{\pgfqpoint{0.786853in}{1.030490in}}%
\pgfpathlineto{\pgfqpoint{0.787948in}{1.032166in}}%
\pgfpathlineto{\pgfqpoint{0.788659in}{1.033252in}}%
\pgfpathlineto{\pgfqpoint{0.789769in}{1.035208in}}%
\pgfpathlineto{\pgfqpoint{0.790434in}{1.036294in}}%
\pgfpathlineto{\pgfqpoint{0.791520in}{1.038343in}}%
\pgfpathlineto{\pgfqpoint{0.791544in}{1.038343in}}%
\pgfpathlineto{\pgfqpoint{0.792068in}{1.039430in}}%
\pgfpathlineto{\pgfqpoint{0.793131in}{1.040951in}}%
\pgfpathlineto{\pgfqpoint{0.793960in}{1.042037in}}%
\pgfpathlineto{\pgfqpoint{0.795070in}{1.043527in}}%
\pgfpathlineto{\pgfqpoint{0.795914in}{1.044583in}}%
\pgfpathlineto{\pgfqpoint{0.797016in}{1.046166in}}%
\pgfpathlineto{\pgfqpoint{0.798236in}{1.047252in}}%
\pgfpathlineto{\pgfqpoint{0.799276in}{1.048711in}}%
\pgfpathlineto{\pgfqpoint{0.799322in}{1.048711in}}%
\pgfpathlineto{\pgfqpoint{0.800120in}{1.049798in}}%
\pgfpathlineto{\pgfqpoint{0.801167in}{1.051567in}}%
\pgfpathlineto{\pgfqpoint{0.801918in}{1.052654in}}%
\pgfpathlineto{\pgfqpoint{0.802997in}{1.054237in}}%
\pgfpathlineto{\pgfqpoint{0.803896in}{1.055323in}}%
\pgfpathlineto{\pgfqpoint{0.804935in}{1.056906in}}%
\pgfpathlineto{\pgfqpoint{0.805670in}{1.057962in}}%
\pgfpathlineto{\pgfqpoint{0.806749in}{1.059390in}}%
\pgfpathlineto{\pgfqpoint{0.808055in}{1.060445in}}%
\pgfpathlineto{\pgfqpoint{0.809165in}{1.061997in}}%
\pgfpathlineto{\pgfqpoint{0.810126in}{1.063084in}}%
\pgfpathlineto{\pgfqpoint{0.811236in}{1.065040in}}%
\pgfpathlineto{\pgfqpoint{0.812057in}{1.066064in}}%
\pgfpathlineto{\pgfqpoint{0.813152in}{1.067368in}}%
\pgfpathlineto{\pgfqpoint{0.814184in}{1.068454in}}%
\pgfpathlineto{\pgfqpoint{0.815270in}{1.070099in}}%
\pgfpathlineto{\pgfqpoint{0.816044in}{1.071155in}}%
\pgfpathlineto{\pgfqpoint{0.817154in}{1.072210in}}%
\pgfpathlineto{\pgfqpoint{0.817999in}{1.073266in}}%
\pgfpathlineto{\pgfqpoint{0.819101in}{1.074973in}}%
\pgfpathlineto{\pgfqpoint{0.819883in}{1.075997in}}%
\pgfpathlineto{\pgfqpoint{0.820962in}{1.078046in}}%
\pgfpathlineto{\pgfqpoint{0.821814in}{1.079102in}}%
\pgfpathlineto{\pgfqpoint{0.822924in}{1.080902in}}%
\pgfpathlineto{\pgfqpoint{0.823909in}{1.081989in}}%
\pgfpathlineto{\pgfqpoint{0.825011in}{1.083323in}}%
\pgfpathlineto{\pgfqpoint{0.825613in}{1.084410in}}%
\pgfpathlineto{\pgfqpoint{0.826692in}{1.086210in}}%
\pgfpathlineto{\pgfqpoint{0.827833in}{1.087297in}}%
\pgfpathlineto{\pgfqpoint{0.828936in}{1.088569in}}%
\pgfpathlineto{\pgfqpoint{0.830030in}{1.089656in}}%
\pgfpathlineto{\pgfqpoint{0.831132in}{1.091332in}}%
\pgfpathlineto{\pgfqpoint{0.832188in}{1.092419in}}%
\pgfpathlineto{\pgfqpoint{0.833235in}{1.094064in}}%
\pgfpathlineto{\pgfqpoint{0.833962in}{1.095119in}}%
\pgfpathlineto{\pgfqpoint{0.835065in}{1.096423in}}%
\pgfpathlineto{\pgfqpoint{0.836034in}{1.097479in}}%
\pgfpathlineto{\pgfqpoint{0.837136in}{1.099341in}}%
\pgfpathlineto{\pgfqpoint{0.838082in}{1.100428in}}%
\pgfpathlineto{\pgfqpoint{0.839192in}{1.102135in}}%
\pgfpathlineto{\pgfqpoint{0.839701in}{1.103221in}}%
\pgfpathlineto{\pgfqpoint{0.840748in}{1.104246in}}%
\pgfpathlineto{\pgfqpoint{0.841639in}{1.105332in}}%
\pgfpathlineto{\pgfqpoint{0.842726in}{1.106729in}}%
\pgfpathlineto{\pgfqpoint{0.843688in}{1.107753in}}%
\pgfpathlineto{\pgfqpoint{0.844782in}{1.108809in}}%
\pgfpathlineto{\pgfqpoint{0.845658in}{1.109895in}}%
\pgfpathlineto{\pgfqpoint{0.846768in}{1.111541in}}%
\pgfpathlineto{\pgfqpoint{0.847534in}{1.112627in}}%
\pgfpathlineto{\pgfqpoint{0.848636in}{1.113900in}}%
\pgfpathlineto{\pgfqpoint{0.849519in}{1.114986in}}%
\pgfpathlineto{\pgfqpoint{0.850630in}{1.116042in}}%
\pgfpathlineto{\pgfqpoint{0.851615in}{1.117128in}}%
\pgfpathlineto{\pgfqpoint{0.852748in}{1.118649in}}%
\pgfpathlineto{\pgfqpoint{0.853796in}{1.119705in}}%
\pgfpathlineto{\pgfqpoint{0.854875in}{1.120822in}}%
\pgfpathlineto{\pgfqpoint{0.855867in}{1.121909in}}%
\pgfpathlineto{\pgfqpoint{0.856962in}{1.123306in}}%
\pgfpathlineto{\pgfqpoint{0.857900in}{1.124392in}}%
\pgfpathlineto{\pgfqpoint{0.859002in}{1.126317in}}%
\pgfpathlineto{\pgfqpoint{0.860339in}{1.127372in}}%
\pgfpathlineto{\pgfqpoint{0.861441in}{1.128583in}}%
\pgfpathlineto{\pgfqpoint{0.862684in}{1.129669in}}%
\pgfpathlineto{\pgfqpoint{0.863771in}{1.130911in}}%
\pgfpathlineto{\pgfqpoint{0.864803in}{1.131966in}}%
\pgfpathlineto{\pgfqpoint{0.865874in}{1.133239in}}%
\pgfpathlineto{\pgfqpoint{0.867187in}{1.134326in}}%
\pgfpathlineto{\pgfqpoint{0.868282in}{1.135785in}}%
\pgfpathlineto{\pgfqpoint{0.869759in}{1.136871in}}%
\pgfpathlineto{\pgfqpoint{0.870869in}{1.138082in}}%
\pgfpathlineto{\pgfqpoint{0.871987in}{1.139137in}}%
\pgfpathlineto{\pgfqpoint{0.873098in}{1.140348in}}%
\pgfpathlineto{\pgfqpoint{0.874200in}{1.141434in}}%
\pgfpathlineto{\pgfqpoint{0.875310in}{1.142521in}}%
\pgfpathlineto{\pgfqpoint{0.876506in}{1.143607in}}%
\pgfpathlineto{\pgfqpoint{0.877616in}{1.144787in}}%
\pgfpathlineto{\pgfqpoint{0.878523in}{1.145842in}}%
\pgfpathlineto{\pgfqpoint{0.879633in}{1.147332in}}%
\pgfpathlineto{\pgfqpoint{0.880696in}{1.148419in}}%
\pgfpathlineto{\pgfqpoint{0.881720in}{1.149226in}}%
\pgfpathlineto{\pgfqpoint{0.883143in}{1.150312in}}%
\pgfpathlineto{\pgfqpoint{0.884253in}{1.151616in}}%
\pgfpathlineto{\pgfqpoint{0.885418in}{1.152702in}}%
\pgfpathlineto{\pgfqpoint{0.886481in}{1.153944in}}%
\pgfpathlineto{\pgfqpoint{0.886520in}{1.153944in}}%
\pgfpathlineto{\pgfqpoint{0.887818in}{1.155031in}}%
\pgfpathlineto{\pgfqpoint{0.888928in}{1.156521in}}%
\pgfpathlineto{\pgfqpoint{0.889929in}{1.157607in}}%
\pgfpathlineto{\pgfqpoint{0.891039in}{1.158600in}}%
\pgfpathlineto{\pgfqpoint{0.891891in}{1.159687in}}%
\pgfpathlineto{\pgfqpoint{0.892962in}{1.160680in}}%
\pgfpathlineto{\pgfqpoint{0.894002in}{1.161767in}}%
\pgfpathlineto{\pgfqpoint{0.895089in}{1.162698in}}%
\pgfpathlineto{\pgfqpoint{0.896034in}{1.163785in}}%
\pgfpathlineto{\pgfqpoint{0.897113in}{1.165119in}}%
\pgfpathlineto{\pgfqpoint{0.898067in}{1.166206in}}%
\pgfpathlineto{\pgfqpoint{0.899130in}{1.167572in}}%
\pgfpathlineto{\pgfqpoint{0.900279in}{1.168658in}}%
\pgfpathlineto{\pgfqpoint{0.901390in}{1.169838in}}%
\pgfpathlineto{\pgfqpoint{0.902672in}{1.170924in}}%
\pgfpathlineto{\pgfqpoint{0.903774in}{1.172042in}}%
\pgfpathlineto{\pgfqpoint{0.904853in}{1.173128in}}%
\pgfpathlineto{\pgfqpoint{0.905900in}{1.174122in}}%
\pgfpathlineto{\pgfqpoint{0.906721in}{1.175208in}}%
\pgfpathlineto{\pgfqpoint{0.907800in}{1.176388in}}%
\pgfpathlineto{\pgfqpoint{0.909090in}{1.177474in}}%
\pgfpathlineto{\pgfqpoint{0.910169in}{1.178064in}}%
\pgfpathlineto{\pgfqpoint{0.911310in}{1.179119in}}%
\pgfpathlineto{\pgfqpoint{0.912412in}{1.180082in}}%
\pgfpathlineto{\pgfqpoint{0.913499in}{1.181106in}}%
\pgfpathlineto{\pgfqpoint{0.914609in}{1.182503in}}%
\pgfpathlineto{\pgfqpoint{0.916079in}{1.183558in}}%
\pgfpathlineto{\pgfqpoint{0.917189in}{1.185017in}}%
\pgfpathlineto{\pgfqpoint{0.918080in}{1.186104in}}%
\pgfpathlineto{\pgfqpoint{0.919175in}{1.187625in}}%
\pgfpathlineto{\pgfqpoint{0.920183in}{1.188711in}}%
\pgfpathlineto{\pgfqpoint{0.921207in}{1.189643in}}%
\pgfpathlineto{\pgfqpoint{0.922450in}{1.190698in}}%
\pgfpathlineto{\pgfqpoint{0.923545in}{1.191722in}}%
\pgfpathlineto{\pgfqpoint{0.924889in}{1.192809in}}%
\pgfpathlineto{\pgfqpoint{0.925992in}{1.194299in}}%
\pgfpathlineto{\pgfqpoint{0.927149in}{1.195354in}}%
\pgfpathlineto{\pgfqpoint{0.928259in}{1.196534in}}%
\pgfpathlineto{\pgfqpoint{0.929486in}{1.197620in}}%
\pgfpathlineto{\pgfqpoint{0.930463in}{1.198428in}}%
\pgfpathlineto{\pgfqpoint{0.931878in}{1.199514in}}%
\pgfpathlineto{\pgfqpoint{0.932942in}{1.200414in}}%
\pgfpathlineto{\pgfqpoint{0.932981in}{1.200414in}}%
\pgfpathlineto{\pgfqpoint{0.934646in}{1.201439in}}%
\pgfpathlineto{\pgfqpoint{0.935646in}{1.202370in}}%
\pgfpathlineto{\pgfqpoint{0.937194in}{1.203456in}}%
\pgfpathlineto{\pgfqpoint{0.938289in}{1.204605in}}%
\pgfpathlineto{\pgfqpoint{0.939375in}{1.205691in}}%
\pgfpathlineto{\pgfqpoint{0.940478in}{1.206654in}}%
\pgfpathlineto{\pgfqpoint{0.941588in}{1.207709in}}%
\pgfpathlineto{\pgfqpoint{0.942682in}{1.208858in}}%
\pgfpathlineto{\pgfqpoint{0.944347in}{1.209944in}}%
\pgfpathlineto{\pgfqpoint{0.945426in}{1.210875in}}%
\pgfpathlineto{\pgfqpoint{0.946372in}{1.211962in}}%
\pgfpathlineto{\pgfqpoint{0.947467in}{1.213048in}}%
\pgfpathlineto{\pgfqpoint{0.948600in}{1.214135in}}%
\pgfpathlineto{\pgfqpoint{0.949695in}{1.215190in}}%
\pgfpathlineto{\pgfqpoint{0.950891in}{1.216277in}}%
\pgfpathlineto{\pgfqpoint{0.952001in}{1.217487in}}%
\pgfpathlineto{\pgfqpoint{0.953588in}{1.218574in}}%
\pgfpathlineto{\pgfqpoint{0.954659in}{1.219505in}}%
\pgfpathlineto{\pgfqpoint{0.954690in}{1.219505in}}%
\pgfpathlineto{\pgfqpoint{0.955417in}{1.220561in}}%
\pgfpathlineto{\pgfqpoint{0.956520in}{1.222020in}}%
\pgfpathlineto{\pgfqpoint{0.957778in}{1.223106in}}%
\pgfpathlineto{\pgfqpoint{0.958865in}{1.224037in}}%
\pgfpathlineto{\pgfqpoint{0.960014in}{1.225093in}}%
\pgfpathlineto{\pgfqpoint{0.961069in}{1.226086in}}%
\pgfpathlineto{\pgfqpoint{0.962461in}{1.227173in}}%
\pgfpathlineto{\pgfqpoint{0.963548in}{1.228073in}}%
\pgfpathlineto{\pgfqpoint{0.965111in}{1.229128in}}%
\pgfpathlineto{\pgfqpoint{0.966198in}{1.230153in}}%
\pgfpathlineto{\pgfqpoint{0.967347in}{1.231239in}}%
\pgfpathlineto{\pgfqpoint{0.968387in}{1.232294in}}%
\pgfpathlineto{\pgfqpoint{0.969880in}{1.233381in}}%
\pgfpathlineto{\pgfqpoint{0.970935in}{1.234467in}}%
\pgfpathlineto{\pgfqpoint{0.971952in}{1.235554in}}%
\pgfpathlineto{\pgfqpoint{0.973062in}{1.236392in}}%
\pgfpathlineto{\pgfqpoint{0.974438in}{1.237479in}}%
\pgfpathlineto{\pgfqpoint{0.975430in}{1.238255in}}%
\pgfpathlineto{\pgfqpoint{0.976908in}{1.239341in}}%
\pgfpathlineto{\pgfqpoint{0.978018in}{1.240396in}}%
\pgfpathlineto{\pgfqpoint{0.979644in}{1.241483in}}%
\pgfpathlineto{\pgfqpoint{0.980746in}{1.242694in}}%
\pgfpathlineto{\pgfqpoint{0.982193in}{1.243780in}}%
\pgfpathlineto{\pgfqpoint{0.983295in}{1.244960in}}%
\pgfpathlineto{\pgfqpoint{0.984444in}{1.246046in}}%
\pgfpathlineto{\pgfqpoint{0.985546in}{1.246946in}}%
\pgfpathlineto{\pgfqpoint{0.987352in}{1.248002in}}%
\pgfpathlineto{\pgfqpoint{0.988392in}{1.249026in}}%
\pgfpathlineto{\pgfqpoint{0.989971in}{1.250113in}}%
\pgfpathlineto{\pgfqpoint{0.991066in}{1.250982in}}%
\pgfpathlineto{\pgfqpoint{0.992184in}{1.252068in}}%
\pgfpathlineto{\pgfqpoint{0.993098in}{1.252751in}}%
\pgfpathlineto{\pgfqpoint{0.993176in}{1.252751in}}%
\pgfpathlineto{\pgfqpoint{0.994889in}{1.253838in}}%
\pgfpathlineto{\pgfqpoint{0.995967in}{1.254645in}}%
\pgfpathlineto{\pgfqpoint{0.997203in}{1.255731in}}%
\pgfpathlineto{\pgfqpoint{0.998258in}{1.256414in}}%
\pgfpathlineto{\pgfqpoint{0.999407in}{1.257501in}}%
\pgfpathlineto{\pgfqpoint{1.000455in}{1.258401in}}%
\pgfpathlineto{\pgfqpoint{1.000509in}{1.258401in}}%
\pgfpathlineto{\pgfqpoint{1.002143in}{1.259456in}}%
\pgfpathlineto{\pgfqpoint{1.003214in}{1.260481in}}%
\pgfpathlineto{\pgfqpoint{1.005020in}{1.261536in}}%
\pgfpathlineto{\pgfqpoint{1.006130in}{1.262498in}}%
\pgfpathlineto{\pgfqpoint{1.007451in}{1.263585in}}%
\pgfpathlineto{\pgfqpoint{1.008554in}{1.264392in}}%
\pgfpathlineto{\pgfqpoint{1.010188in}{1.265447in}}%
\pgfpathlineto{\pgfqpoint{1.011274in}{1.266317in}}%
\pgfpathlineto{\pgfqpoint{1.012861in}{1.267403in}}%
\pgfpathlineto{\pgfqpoint{1.013948in}{1.268490in}}%
\pgfpathlineto{\pgfqpoint{1.015379in}{1.269576in}}%
\pgfpathlineto{\pgfqpoint{1.016489in}{1.271035in}}%
\pgfpathlineto{\pgfqpoint{1.017505in}{1.272122in}}%
\pgfpathlineto{\pgfqpoint{1.018615in}{1.273208in}}%
\pgfpathlineto{\pgfqpoint{1.020116in}{1.274263in}}%
\pgfpathlineto{\pgfqpoint{1.021171in}{1.274915in}}%
\pgfpathlineto{\pgfqpoint{1.022719in}{1.276002in}}%
\pgfpathlineto{\pgfqpoint{1.023822in}{1.277306in}}%
\pgfpathlineto{\pgfqpoint{1.025096in}{1.278392in}}%
\pgfpathlineto{\pgfqpoint{1.026206in}{1.279572in}}%
\pgfpathlineto{\pgfqpoint{1.027472in}{1.280658in}}%
\pgfpathlineto{\pgfqpoint{1.028567in}{1.281745in}}%
\pgfpathlineto{\pgfqpoint{1.029974in}{1.282831in}}%
\pgfpathlineto{\pgfqpoint{1.031084in}{1.283980in}}%
\pgfpathlineto{\pgfqpoint{1.032812in}{1.285066in}}%
\pgfpathlineto{\pgfqpoint{1.033867in}{1.286401in}}%
\pgfpathlineto{\pgfqpoint{1.035384in}{1.287487in}}%
\pgfpathlineto{\pgfqpoint{1.036478in}{1.288388in}}%
\pgfpathlineto{\pgfqpoint{1.037901in}{1.289474in}}%
\pgfpathlineto{\pgfqpoint{1.038925in}{1.290095in}}%
\pgfpathlineto{\pgfqpoint{1.040137in}{1.291181in}}%
\pgfpathlineto{\pgfqpoint{1.041153in}{1.291833in}}%
\pgfpathlineto{\pgfqpoint{1.042686in}{1.292920in}}%
\pgfpathlineto{\pgfqpoint{1.043780in}{1.294161in}}%
\pgfpathlineto{\pgfqpoint{1.044953in}{1.295217in}}%
\pgfpathlineto{\pgfqpoint{1.045985in}{1.296086in}}%
\pgfpathlineto{\pgfqpoint{1.046008in}{1.296086in}}%
\pgfpathlineto{\pgfqpoint{1.047462in}{1.297142in}}%
\pgfpathlineto{\pgfqpoint{1.048400in}{1.297762in}}%
\pgfpathlineto{\pgfqpoint{1.050613in}{1.298849in}}%
\pgfpathlineto{\pgfqpoint{1.051684in}{1.299842in}}%
\pgfpathlineto{\pgfqpoint{1.053380in}{1.300929in}}%
\pgfpathlineto{\pgfqpoint{1.054412in}{1.301674in}}%
\pgfpathlineto{\pgfqpoint{1.055796in}{1.302760in}}%
\pgfpathlineto{\pgfqpoint{1.056804in}{1.303474in}}%
\pgfpathlineto{\pgfqpoint{1.058282in}{1.304561in}}%
\pgfpathlineto{\pgfqpoint{1.059314in}{1.305554in}}%
\pgfpathlineto{\pgfqpoint{1.061143in}{1.306640in}}%
\pgfpathlineto{\pgfqpoint{1.062245in}{1.307354in}}%
\pgfpathlineto{\pgfqpoint{1.064012in}{1.308441in}}%
\pgfpathlineto{\pgfqpoint{1.065107in}{1.309062in}}%
\pgfpathlineto{\pgfqpoint{1.066350in}{1.310117in}}%
\pgfpathlineto{\pgfqpoint{1.067421in}{1.311079in}}%
\pgfpathlineto{\pgfqpoint{1.068718in}{1.312166in}}%
\pgfpathlineto{\pgfqpoint{1.069797in}{1.313035in}}%
\pgfpathlineto{\pgfqpoint{1.071165in}{1.314122in}}%
\pgfpathlineto{\pgfqpoint{1.072197in}{1.314804in}}%
\pgfpathlineto{\pgfqpoint{1.073823in}{1.315891in}}%
\pgfpathlineto{\pgfqpoint{1.074910in}{1.316760in}}%
\pgfpathlineto{\pgfqpoint{1.076872in}{1.317816in}}%
\pgfpathlineto{\pgfqpoint{1.077982in}{1.318871in}}%
\pgfpathlineto{\pgfqpoint{1.079585in}{1.319957in}}%
\pgfpathlineto{\pgfqpoint{1.080625in}{1.320671in}}%
\pgfpathlineto{\pgfqpoint{1.080679in}{1.320671in}}%
\pgfpathlineto{\pgfqpoint{1.082133in}{1.321758in}}%
\pgfpathlineto{\pgfqpoint{1.083204in}{1.322627in}}%
\pgfpathlineto{\pgfqpoint{1.084823in}{1.323714in}}%
\pgfpathlineto{\pgfqpoint{1.085909in}{1.324490in}}%
\pgfpathlineto{\pgfqpoint{1.088427in}{1.325576in}}%
\pgfpathlineto{\pgfqpoint{1.089380in}{1.326197in}}%
\pgfpathlineto{\pgfqpoint{1.091468in}{1.327283in}}%
\pgfpathlineto{\pgfqpoint{1.092570in}{1.328277in}}%
\pgfpathlineto{\pgfqpoint{1.094454in}{1.329363in}}%
\pgfpathlineto{\pgfqpoint{1.095564in}{1.330015in}}%
\pgfpathlineto{\pgfqpoint{1.097135in}{1.331102in}}%
\pgfpathlineto{\pgfqpoint{1.098222in}{1.332033in}}%
\pgfpathlineto{\pgfqpoint{1.099762in}{1.333119in}}%
\pgfpathlineto{\pgfqpoint{1.100810in}{1.333926in}}%
\pgfpathlineto{\pgfqpoint{1.102663in}{1.334982in}}%
\pgfpathlineto{\pgfqpoint{1.103757in}{1.335851in}}%
\pgfpathlineto{\pgfqpoint{1.105352in}{1.336938in}}%
\pgfpathlineto{\pgfqpoint{1.106462in}{1.337496in}}%
\pgfpathlineto{\pgfqpoint{1.107642in}{1.338583in}}%
\pgfpathlineto{\pgfqpoint{1.108667in}{1.339576in}}%
\pgfpathlineto{\pgfqpoint{1.108690in}{1.339576in}}%
\pgfpathlineto{\pgfqpoint{1.110644in}{1.340663in}}%
\pgfpathlineto{\pgfqpoint{1.111747in}{1.341221in}}%
\pgfpathlineto{\pgfqpoint{1.113138in}{1.342308in}}%
\pgfpathlineto{\pgfqpoint{1.114233in}{1.342867in}}%
\pgfpathlineto{\pgfqpoint{1.115765in}{1.343953in}}%
\pgfpathlineto{\pgfqpoint{1.116805in}{1.344698in}}%
\pgfpathlineto{\pgfqpoint{1.118321in}{1.345753in}}%
\pgfpathlineto{\pgfqpoint{1.119338in}{1.346685in}}%
\pgfpathlineto{\pgfqpoint{1.122535in}{1.347771in}}%
\pgfpathlineto{\pgfqpoint{1.123622in}{1.348765in}}%
\pgfpathlineto{\pgfqpoint{1.125529in}{1.349820in}}%
\pgfpathlineto{\pgfqpoint{1.126553in}{1.350503in}}%
\pgfpathlineto{\pgfqpoint{1.128484in}{1.351589in}}%
\pgfpathlineto{\pgfqpoint{1.129469in}{1.352148in}}%
\pgfpathlineto{\pgfqpoint{1.132377in}{1.353235in}}%
\pgfpathlineto{\pgfqpoint{1.133441in}{1.354073in}}%
\pgfpathlineto{\pgfqpoint{1.135418in}{1.355159in}}%
\pgfpathlineto{\pgfqpoint{1.136466in}{1.355594in}}%
\pgfpathlineto{\pgfqpoint{1.138413in}{1.356680in}}%
\pgfpathlineto{\pgfqpoint{1.139515in}{1.357456in}}%
\pgfpathlineto{\pgfqpoint{1.141790in}{1.358543in}}%
\pgfpathlineto{\pgfqpoint{1.142869in}{1.359288in}}%
\pgfpathlineto{\pgfqpoint{1.144581in}{1.360374in}}%
\pgfpathlineto{\pgfqpoint{1.145660in}{1.361088in}}%
\pgfpathlineto{\pgfqpoint{1.147403in}{1.362144in}}%
\pgfpathlineto{\pgfqpoint{1.148458in}{1.363075in}}%
\pgfpathlineto{\pgfqpoint{1.148474in}{1.363075in}}%
\pgfpathlineto{\pgfqpoint{1.149647in}{1.364161in}}%
\pgfpathlineto{\pgfqpoint{1.150655in}{1.364813in}}%
\pgfpathlineto{\pgfqpoint{1.152508in}{1.365900in}}%
\pgfpathlineto{\pgfqpoint{1.153579in}{1.366614in}}%
\pgfpathlineto{\pgfqpoint{1.155103in}{1.367669in}}%
\pgfpathlineto{\pgfqpoint{1.156057in}{1.368259in}}%
\pgfpathlineto{\pgfqpoint{1.156143in}{1.368259in}}%
\pgfpathlineto{\pgfqpoint{1.157363in}{1.369345in}}%
\pgfpathlineto{\pgfqpoint{1.158426in}{1.369966in}}%
\pgfpathlineto{\pgfqpoint{1.158449in}{1.369966in}}%
\pgfpathlineto{\pgfqpoint{1.160239in}{1.371053in}}%
\pgfpathlineto{\pgfqpoint{1.161326in}{1.371798in}}%
\pgfpathlineto{\pgfqpoint{1.163023in}{1.372853in}}%
\pgfpathlineto{\pgfqpoint{1.164047in}{1.373722in}}%
\pgfpathlineto{\pgfqpoint{1.166048in}{1.374809in}}%
\pgfpathlineto{\pgfqpoint{1.167135in}{1.375461in}}%
\pgfpathlineto{\pgfqpoint{1.169480in}{1.376547in}}%
\pgfpathlineto{\pgfqpoint{1.170528in}{1.377230in}}%
\pgfpathlineto{\pgfqpoint{1.173272in}{1.378317in}}%
\pgfpathlineto{\pgfqpoint{1.174335in}{1.378969in}}%
\pgfpathlineto{\pgfqpoint{1.176375in}{1.380055in}}%
\pgfpathlineto{\pgfqpoint{1.177462in}{1.380893in}}%
\pgfpathlineto{\pgfqpoint{1.180221in}{1.381980in}}%
\pgfpathlineto{\pgfqpoint{1.181316in}{1.382476in}}%
\pgfpathlineto{\pgfqpoint{1.183497in}{1.383563in}}%
\pgfpathlineto{\pgfqpoint{1.184591in}{1.384091in}}%
\pgfpathlineto{\pgfqpoint{1.184599in}{1.384091in}}%
\pgfpathlineto{\pgfqpoint{1.186929in}{1.385177in}}%
\pgfpathlineto{\pgfqpoint{1.188039in}{1.385984in}}%
\pgfpathlineto{\pgfqpoint{1.190408in}{1.387071in}}%
\pgfpathlineto{\pgfqpoint{1.191494in}{1.387722in}}%
\pgfpathlineto{\pgfqpoint{1.194106in}{1.388809in}}%
\pgfpathlineto{\pgfqpoint{1.195177in}{1.389523in}}%
\pgfpathlineto{\pgfqpoint{1.197420in}{1.390609in}}%
\pgfpathlineto{\pgfqpoint{1.198436in}{1.391416in}}%
\pgfpathlineto{\pgfqpoint{1.200485in}{1.392472in}}%
\pgfpathlineto{\pgfqpoint{1.201587in}{1.393341in}}%
\pgfpathlineto{\pgfqpoint{1.203752in}{1.394428in}}%
\pgfpathlineto{\pgfqpoint{1.204777in}{1.394893in}}%
\pgfpathlineto{\pgfqpoint{1.206942in}{1.395980in}}%
\pgfpathlineto{\pgfqpoint{1.208044in}{1.396818in}}%
\pgfpathlineto{\pgfqpoint{1.210405in}{1.397904in}}%
\pgfpathlineto{\pgfqpoint{1.211422in}{1.398339in}}%
\pgfpathlineto{\pgfqpoint{1.213869in}{1.399394in}}%
\pgfpathlineto{\pgfqpoint{1.214971in}{1.400077in}}%
\pgfpathlineto{\pgfqpoint{1.217183in}{1.401164in}}%
\pgfpathlineto{\pgfqpoint{1.218285in}{1.401722in}}%
\pgfpathlineto{\pgfqpoint{1.220482in}{1.402809in}}%
\pgfpathlineto{\pgfqpoint{1.221444in}{1.403244in}}%
\pgfpathlineto{\pgfqpoint{1.223703in}{1.404330in}}%
\pgfpathlineto{\pgfqpoint{1.224329in}{1.404796in}}%
\pgfpathlineto{\pgfqpoint{1.224563in}{1.404796in}}%
\pgfpathlineto{\pgfqpoint{1.227800in}{1.405882in}}%
\pgfpathlineto{\pgfqpoint{1.228886in}{1.406534in}}%
\pgfpathlineto{\pgfqpoint{1.231357in}{1.407620in}}%
\pgfpathlineto{\pgfqpoint{1.232459in}{1.408241in}}%
\pgfpathlineto{\pgfqpoint{1.235117in}{1.409328in}}%
\pgfpathlineto{\pgfqpoint{1.236102in}{1.410011in}}%
\pgfpathlineto{\pgfqpoint{1.238385in}{1.411097in}}%
\pgfpathlineto{\pgfqpoint{1.239487in}{1.411532in}}%
\pgfpathlineto{\pgfqpoint{1.242028in}{1.412618in}}%
\pgfpathlineto{\pgfqpoint{1.243091in}{1.413301in}}%
\pgfpathlineto{\pgfqpoint{1.244826in}{1.414388in}}%
\pgfpathlineto{\pgfqpoint{1.245929in}{1.415102in}}%
\pgfpathlineto{\pgfqpoint{1.248032in}{1.416188in}}%
\pgfpathlineto{\pgfqpoint{1.249142in}{1.416871in}}%
\pgfpathlineto{\pgfqpoint{1.251088in}{1.417957in}}%
\pgfpathlineto{\pgfqpoint{1.252183in}{1.418454in}}%
\pgfpathlineto{\pgfqpoint{1.254810in}{1.419541in}}%
\pgfpathlineto{\pgfqpoint{1.255912in}{1.420068in}}%
\pgfpathlineto{\pgfqpoint{1.257593in}{1.421155in}}%
\pgfpathlineto{\pgfqpoint{1.258648in}{1.421714in}}%
\pgfpathlineto{\pgfqpoint{1.261064in}{1.422800in}}%
\pgfpathlineto{\pgfqpoint{1.262119in}{1.423204in}}%
\pgfpathlineto{\pgfqpoint{1.264988in}{1.424290in}}%
\pgfpathlineto{\pgfqpoint{1.266059in}{1.424756in}}%
\pgfpathlineto{\pgfqpoint{1.268529in}{1.425842in}}%
\pgfpathlineto{\pgfqpoint{1.269624in}{1.426401in}}%
\pgfpathlineto{\pgfqpoint{1.271649in}{1.427487in}}%
\pgfpathlineto{\pgfqpoint{1.272509in}{1.427891in}}%
\pgfpathlineto{\pgfqpoint{1.275112in}{1.428977in}}%
\pgfpathlineto{\pgfqpoint{1.276183in}{1.429722in}}%
\pgfpathlineto{\pgfqpoint{1.278497in}{1.430809in}}%
\pgfpathlineto{\pgfqpoint{1.279584in}{1.431368in}}%
\pgfpathlineto{\pgfqpoint{1.282070in}{1.432454in}}%
\pgfpathlineto{\pgfqpoint{1.283141in}{1.432951in}}%
\pgfpathlineto{\pgfqpoint{1.284603in}{1.434037in}}%
\pgfpathlineto{\pgfqpoint{1.285681in}{1.434813in}}%
\pgfpathlineto{\pgfqpoint{1.288550in}{1.435900in}}%
\pgfpathlineto{\pgfqpoint{1.289410in}{1.436272in}}%
\pgfpathlineto{\pgfqpoint{1.291466in}{1.437359in}}%
\pgfpathlineto{\pgfqpoint{1.292514in}{1.437855in}}%
\pgfpathlineto{\pgfqpoint{1.295203in}{1.438942in}}%
\pgfpathlineto{\pgfqpoint{1.296251in}{1.439314in}}%
\pgfpathlineto{\pgfqpoint{1.298987in}{1.440401in}}%
\pgfpathlineto{\pgfqpoint{1.299988in}{1.440960in}}%
\pgfpathlineto{\pgfqpoint{1.302216in}{1.442046in}}%
\pgfpathlineto{\pgfqpoint{1.303326in}{1.442543in}}%
\pgfpathlineto{\pgfqpoint{1.306164in}{1.443629in}}%
\pgfpathlineto{\pgfqpoint{1.307227in}{1.444064in}}%
\pgfpathlineto{\pgfqpoint{1.310878in}{1.445150in}}%
\pgfpathlineto{\pgfqpoint{1.311784in}{1.445492in}}%
\pgfpathlineto{\pgfqpoint{1.315084in}{1.446578in}}%
\pgfpathlineto{\pgfqpoint{1.316061in}{1.447137in}}%
\pgfpathlineto{\pgfqpoint{1.317820in}{1.448224in}}%
\pgfpathlineto{\pgfqpoint{1.318883in}{1.448938in}}%
\pgfpathlineto{\pgfqpoint{1.320751in}{1.450024in}}%
\pgfpathlineto{\pgfqpoint{1.321854in}{1.450334in}}%
\pgfpathlineto{\pgfqpoint{1.324332in}{1.451421in}}%
\pgfpathlineto{\pgfqpoint{1.325411in}{1.452011in}}%
\pgfpathlineto{\pgfqpoint{1.327850in}{1.453097in}}%
\pgfpathlineto{\pgfqpoint{1.328788in}{1.453563in}}%
\pgfpathlineto{\pgfqpoint{1.331813in}{1.454649in}}%
\pgfpathlineto{\pgfqpoint{1.332837in}{1.455053in}}%
\pgfpathlineto{\pgfqpoint{1.336043in}{1.456139in}}%
\pgfpathlineto{\pgfqpoint{1.337106in}{1.457040in}}%
\pgfpathlineto{\pgfqpoint{1.339240in}{1.458126in}}%
\pgfpathlineto{\pgfqpoint{1.340295in}{1.458716in}}%
\pgfpathlineto{\pgfqpoint{1.340311in}{1.458716in}}%
\pgfpathlineto{\pgfqpoint{1.342086in}{1.459802in}}%
\pgfpathlineto{\pgfqpoint{1.343188in}{1.460299in}}%
\pgfpathlineto{\pgfqpoint{1.345776in}{1.461385in}}%
\pgfpathlineto{\pgfqpoint{1.346800in}{1.461944in}}%
\pgfpathlineto{\pgfqpoint{1.350513in}{1.463031in}}%
\pgfpathlineto{\pgfqpoint{1.351576in}{1.463496in}}%
\pgfpathlineto{\pgfqpoint{1.354054in}{1.464552in}}%
\pgfpathlineto{\pgfqpoint{1.355094in}{1.465017in}}%
\pgfpathlineto{\pgfqpoint{1.358401in}{1.466104in}}%
\pgfpathlineto{\pgfqpoint{1.359308in}{1.466663in}}%
\pgfpathlineto{\pgfqpoint{1.359355in}{1.466663in}}%
\pgfpathlineto{\pgfqpoint{1.363068in}{1.467749in}}%
\pgfpathlineto{\pgfqpoint{1.363975in}{1.468153in}}%
\pgfpathlineto{\pgfqpoint{1.364085in}{1.468153in}}%
\pgfpathlineto{\pgfqpoint{1.366453in}{1.469239in}}%
\pgfpathlineto{\pgfqpoint{1.367438in}{1.469829in}}%
\pgfpathlineto{\pgfqpoint{1.370518in}{1.470915in}}%
\pgfpathlineto{\pgfqpoint{1.371402in}{1.471505in}}%
\pgfpathlineto{\pgfqpoint{1.371621in}{1.471505in}}%
\pgfpathlineto{\pgfqpoint{1.374107in}{1.472592in}}%
\pgfpathlineto{\pgfqpoint{1.375092in}{1.473026in}}%
\pgfpathlineto{\pgfqpoint{1.375107in}{1.473026in}}%
\pgfpathlineto{\pgfqpoint{1.377148in}{1.474113in}}%
\pgfpathlineto{\pgfqpoint{1.378172in}{1.474702in}}%
\pgfpathlineto{\pgfqpoint{1.380822in}{1.475789in}}%
\pgfpathlineto{\pgfqpoint{1.381877in}{1.476255in}}%
\pgfpathlineto{\pgfqpoint{1.384754in}{1.477341in}}%
\pgfpathlineto{\pgfqpoint{1.385841in}{1.477776in}}%
\pgfpathlineto{\pgfqpoint{1.385857in}{1.477776in}}%
\pgfpathlineto{\pgfqpoint{1.390047in}{1.478862in}}%
\pgfpathlineto{\pgfqpoint{1.391016in}{1.479142in}}%
\pgfpathlineto{\pgfqpoint{1.391141in}{1.479142in}}%
\pgfpathlineto{\pgfqpoint{1.394902in}{1.480228in}}%
\pgfpathlineto{\pgfqpoint{1.395965in}{1.480632in}}%
\pgfpathlineto{\pgfqpoint{1.399092in}{1.481718in}}%
\pgfpathlineto{\pgfqpoint{1.400179in}{1.482122in}}%
\pgfpathlineto{\pgfqpoint{1.400194in}{1.482122in}}%
\pgfpathlineto{\pgfqpoint{1.403251in}{1.483208in}}%
\pgfpathlineto{\pgfqpoint{1.404361in}{1.483767in}}%
\pgfpathlineto{\pgfqpoint{1.406988in}{1.484853in}}%
\pgfpathlineto{\pgfqpoint{1.408012in}{1.485474in}}%
\pgfpathlineto{\pgfqpoint{1.411741in}{1.486561in}}%
\pgfpathlineto{\pgfqpoint{1.412804in}{1.486933in}}%
\pgfpathlineto{\pgfqpoint{1.415579in}{1.488020in}}%
\pgfpathlineto{\pgfqpoint{1.416682in}{1.488578in}}%
\pgfpathlineto{\pgfqpoint{1.420544in}{1.489665in}}%
\pgfpathlineto{\pgfqpoint{1.421591in}{1.490130in}}%
\pgfpathlineto{\pgfqpoint{1.424468in}{1.491217in}}%
\pgfpathlineto{\pgfqpoint{1.425461in}{1.491714in}}%
\pgfpathlineto{\pgfqpoint{1.425531in}{1.491714in}}%
\pgfpathlineto{\pgfqpoint{1.428353in}{1.492800in}}%
\pgfpathlineto{\pgfqpoint{1.429370in}{1.493173in}}%
\pgfpathlineto{\pgfqpoint{1.429463in}{1.493173in}}%
\pgfpathlineto{\pgfqpoint{1.432442in}{1.494259in}}%
\pgfpathlineto{\pgfqpoint{1.433380in}{1.494600in}}%
\pgfpathlineto{\pgfqpoint{1.433466in}{1.494600in}}%
\pgfpathlineto{\pgfqpoint{1.437179in}{1.495687in}}%
\pgfpathlineto{\pgfqpoint{1.438282in}{1.496059in}}%
\pgfpathlineto{\pgfqpoint{1.441213in}{1.497146in}}%
\pgfpathlineto{\pgfqpoint{1.442300in}{1.497581in}}%
\pgfpathlineto{\pgfqpoint{1.445286in}{1.498667in}}%
\pgfpathlineto{\pgfqpoint{1.446310in}{1.499226in}}%
\pgfpathlineto{\pgfqpoint{1.449281in}{1.500312in}}%
\pgfpathlineto{\pgfqpoint{1.450368in}{1.500902in}}%
\pgfpathlineto{\pgfqpoint{1.453354in}{1.501989in}}%
\pgfpathlineto{\pgfqpoint{1.454433in}{1.502330in}}%
\pgfpathlineto{\pgfqpoint{1.458561in}{1.503385in}}%
\pgfpathlineto{\pgfqpoint{1.459632in}{1.503789in}}%
\pgfpathlineto{\pgfqpoint{1.462923in}{1.504875in}}%
\pgfpathlineto{\pgfqpoint{1.463822in}{1.505124in}}%
\pgfpathlineto{\pgfqpoint{1.463900in}{1.505124in}}%
\pgfpathlineto{\pgfqpoint{1.467301in}{1.506210in}}%
\pgfpathlineto{\pgfqpoint{1.468395in}{1.506645in}}%
\pgfpathlineto{\pgfqpoint{1.470874in}{1.507731in}}%
\pgfpathlineto{\pgfqpoint{1.471819in}{1.508042in}}%
\pgfpathlineto{\pgfqpoint{1.471945in}{1.508042in}}%
\pgfpathlineto{\pgfqpoint{1.474915in}{1.509097in}}%
\pgfpathlineto{\pgfqpoint{1.476018in}{1.509408in}}%
\pgfpathlineto{\pgfqpoint{1.476025in}{1.509408in}}%
\pgfpathlineto{\pgfqpoint{1.478425in}{1.510494in}}%
\pgfpathlineto{\pgfqpoint{1.479629in}{1.510898in}}%
\pgfpathlineto{\pgfqpoint{1.484038in}{1.511984in}}%
\pgfpathlineto{\pgfqpoint{1.485070in}{1.512481in}}%
\pgfpathlineto{\pgfqpoint{1.488815in}{1.513567in}}%
\pgfpathlineto{\pgfqpoint{1.489823in}{1.514033in}}%
\pgfpathlineto{\pgfqpoint{1.492786in}{1.515119in}}%
\pgfpathlineto{\pgfqpoint{1.493826in}{1.515399in}}%
\pgfpathlineto{\pgfqpoint{1.497696in}{1.516485in}}%
\pgfpathlineto{\pgfqpoint{1.498790in}{1.516796in}}%
\pgfpathlineto{\pgfqpoint{1.501761in}{1.517820in}}%
\pgfpathlineto{\pgfqpoint{1.501761in}{1.517851in}}%
\pgfpathlineto{\pgfqpoint{1.502801in}{1.518286in}}%
\pgfpathlineto{\pgfqpoint{1.507366in}{1.519372in}}%
\pgfpathlineto{\pgfqpoint{1.508429in}{1.519807in}}%
\pgfpathlineto{\pgfqpoint{1.511729in}{1.520893in}}%
\pgfpathlineto{\pgfqpoint{1.512768in}{1.521328in}}%
\pgfpathlineto{\pgfqpoint{1.516411in}{1.522383in}}%
\pgfpathlineto{\pgfqpoint{1.517467in}{1.522694in}}%
\pgfpathlineto{\pgfqpoint{1.521196in}{1.523749in}}%
\pgfpathlineto{\pgfqpoint{1.522228in}{1.524122in}}%
\pgfpathlineto{\pgfqpoint{1.527309in}{1.525208in}}%
\pgfpathlineto{\pgfqpoint{1.528396in}{1.525612in}}%
\pgfpathlineto{\pgfqpoint{1.532265in}{1.526698in}}%
\pgfpathlineto{\pgfqpoint{1.533250in}{1.526977in}}%
\pgfpathlineto{\pgfqpoint{1.536847in}{1.528064in}}%
\pgfpathlineto{\pgfqpoint{1.537957in}{1.528281in}}%
\pgfpathlineto{\pgfqpoint{1.541350in}{1.529337in}}%
\pgfpathlineto{\pgfqpoint{1.542374in}{1.529678in}}%
\pgfpathlineto{\pgfqpoint{1.546095in}{1.530765in}}%
\pgfpathlineto{\pgfqpoint{1.546994in}{1.531137in}}%
\pgfpathlineto{\pgfqpoint{1.551747in}{1.532224in}}%
\pgfpathlineto{\pgfqpoint{1.552787in}{1.532410in}}%
\pgfpathlineto{\pgfqpoint{1.556883in}{1.533496in}}%
\pgfpathlineto{\pgfqpoint{1.557962in}{1.533931in}}%
\pgfpathlineto{\pgfqpoint{1.557978in}{1.533931in}}%
\pgfpathlineto{\pgfqpoint{1.561902in}{1.535017in}}%
\pgfpathlineto{\pgfqpoint{1.562911in}{1.535452in}}%
\pgfpathlineto{\pgfqpoint{1.567234in}{1.536538in}}%
\pgfpathlineto{\pgfqpoint{1.568297in}{1.536911in}}%
\pgfpathlineto{\pgfqpoint{1.573019in}{1.537966in}}%
\pgfpathlineto{\pgfqpoint{1.574090in}{1.538308in}}%
\pgfpathlineto{\pgfqpoint{1.578655in}{1.539394in}}%
\pgfpathlineto{\pgfqpoint{1.579601in}{1.539643in}}%
\pgfpathlineto{\pgfqpoint{1.583651in}{1.540729in}}%
\pgfpathlineto{\pgfqpoint{1.584737in}{1.541008in}}%
\pgfpathlineto{\pgfqpoint{1.588779in}{1.542095in}}%
\pgfpathlineto{\pgfqpoint{1.589874in}{1.542312in}}%
\pgfpathlineto{\pgfqpoint{1.593048in}{1.543399in}}%
\pgfpathlineto{\pgfqpoint{1.594119in}{1.543492in}}%
\pgfpathlineto{\pgfqpoint{1.594134in}{1.543492in}}%
\pgfpathlineto{\pgfqpoint{1.597996in}{1.544547in}}%
\pgfpathlineto{\pgfqpoint{1.599012in}{1.544951in}}%
\pgfpathlineto{\pgfqpoint{1.603789in}{1.546037in}}%
\pgfpathlineto{\pgfqpoint{1.604852in}{1.546317in}}%
\pgfpathlineto{\pgfqpoint{1.610841in}{1.547403in}}%
\pgfpathlineto{\pgfqpoint{1.611388in}{1.547651in}}%
\pgfpathlineto{\pgfqpoint{1.611708in}{1.547651in}}%
\pgfpathlineto{\pgfqpoint{1.617306in}{1.548738in}}%
\pgfpathlineto{\pgfqpoint{1.618361in}{1.549110in}}%
\pgfpathlineto{\pgfqpoint{1.623091in}{1.550197in}}%
\pgfpathlineto{\pgfqpoint{1.623966in}{1.550569in}}%
\pgfpathlineto{\pgfqpoint{1.624193in}{1.550569in}}%
\pgfpathlineto{\pgfqpoint{1.630181in}{1.551656in}}%
\pgfpathlineto{\pgfqpoint{1.631135in}{1.551780in}}%
\pgfpathlineto{\pgfqpoint{1.631206in}{1.551780in}}%
\pgfpathlineto{\pgfqpoint{1.636084in}{1.552836in}}%
\pgfpathlineto{\pgfqpoint{1.637053in}{1.553208in}}%
\pgfpathlineto{\pgfqpoint{1.637170in}{1.553208in}}%
\pgfpathlineto{\pgfqpoint{1.641157in}{1.554294in}}%
\pgfpathlineto{\pgfqpoint{1.642252in}{1.554481in}}%
\pgfpathlineto{\pgfqpoint{1.646919in}{1.555567in}}%
\pgfpathlineto{\pgfqpoint{1.648013in}{1.555878in}}%
\pgfpathlineto{\pgfqpoint{1.652524in}{1.556964in}}%
\pgfpathlineto{\pgfqpoint{1.653603in}{1.557150in}}%
\pgfpathlineto{\pgfqpoint{1.660655in}{1.558237in}}%
\pgfpathlineto{\pgfqpoint{1.661710in}{1.558454in}}%
\pgfpathlineto{\pgfqpoint{1.661726in}{1.558454in}}%
\pgfpathlineto{\pgfqpoint{1.667956in}{1.559541in}}%
\pgfpathlineto{\pgfqpoint{1.668980in}{1.559727in}}%
\pgfpathlineto{\pgfqpoint{1.674851in}{1.560813in}}%
\pgfpathlineto{\pgfqpoint{1.675883in}{1.560938in}}%
\pgfpathlineto{\pgfqpoint{1.675946in}{1.560938in}}%
\pgfpathlineto{\pgfqpoint{1.681020in}{1.562024in}}%
\pgfpathlineto{\pgfqpoint{1.682020in}{1.562334in}}%
\pgfpathlineto{\pgfqpoint{1.686554in}{1.563421in}}%
\pgfpathlineto{\pgfqpoint{1.687625in}{1.563669in}}%
\pgfpathlineto{\pgfqpoint{1.694622in}{1.564756in}}%
\pgfpathlineto{\pgfqpoint{1.695513in}{1.565035in}}%
\pgfpathlineto{\pgfqpoint{1.700657in}{1.566122in}}%
\pgfpathlineto{\pgfqpoint{1.701760in}{1.566277in}}%
\pgfpathlineto{\pgfqpoint{1.706583in}{1.567363in}}%
\pgfpathlineto{\pgfqpoint{1.707428in}{1.567674in}}%
\pgfpathlineto{\pgfqpoint{1.713815in}{1.568760in}}%
\pgfpathlineto{\pgfqpoint{1.714917in}{1.569040in}}%
\pgfpathlineto{\pgfqpoint{1.720194in}{1.570126in}}%
\pgfpathlineto{\pgfqpoint{1.721241in}{1.570374in}}%
\pgfpathlineto{\pgfqpoint{1.728645in}{1.571461in}}%
\pgfpathlineto{\pgfqpoint{1.729614in}{1.571616in}}%
\pgfpathlineto{\pgfqpoint{1.735477in}{1.572702in}}%
\pgfpathlineto{\pgfqpoint{1.736501in}{1.573199in}}%
\pgfpathlineto{\pgfqpoint{1.742685in}{1.574286in}}%
\pgfpathlineto{\pgfqpoint{1.743709in}{1.574565in}}%
\pgfpathlineto{\pgfqpoint{1.750706in}{1.575651in}}%
\pgfpathlineto{\pgfqpoint{1.751761in}{1.575838in}}%
\pgfpathlineto{\pgfqpoint{1.756249in}{1.576924in}}%
\pgfpathlineto{\pgfqpoint{1.756819in}{1.577079in}}%
\pgfpathlineto{\pgfqpoint{1.763832in}{1.578166in}}%
\pgfpathlineto{\pgfqpoint{1.764848in}{1.578445in}}%
\pgfpathlineto{\pgfqpoint{1.764934in}{1.578445in}}%
\pgfpathlineto{\pgfqpoint{1.772150in}{1.579532in}}%
\pgfpathlineto{\pgfqpoint{1.773197in}{1.579780in}}%
\pgfpathlineto{\pgfqpoint{1.773260in}{1.579780in}}%
\pgfpathlineto{\pgfqpoint{1.780155in}{1.580867in}}%
\pgfpathlineto{\pgfqpoint{1.781031in}{1.581146in}}%
\pgfpathlineto{\pgfqpoint{1.790834in}{1.582232in}}%
\pgfpathlineto{\pgfqpoint{1.791897in}{1.582388in}}%
\pgfpathlineto{\pgfqpoint{1.800966in}{1.583474in}}%
\pgfpathlineto{\pgfqpoint{1.802013in}{1.583660in}}%
\pgfpathlineto{\pgfqpoint{1.812215in}{1.584747in}}%
\pgfpathlineto{\pgfqpoint{1.813192in}{1.585057in}}%
\pgfpathlineto{\pgfqpoint{1.822409in}{1.586144in}}%
\pgfpathlineto{\pgfqpoint{1.822409in}{1.586175in}}%
\pgfpathlineto{\pgfqpoint{1.823113in}{1.586175in}}%
\pgfpathlineto{\pgfqpoint{1.834066in}{1.587261in}}%
\pgfpathlineto{\pgfqpoint{1.834511in}{1.587447in}}%
\pgfpathlineto{\pgfqpoint{1.834965in}{1.587447in}}%
\pgfpathlineto{\pgfqpoint{1.842524in}{1.588534in}}%
\pgfpathlineto{\pgfqpoint{1.843580in}{1.588751in}}%
\pgfpathlineto{\pgfqpoint{1.851952in}{1.589838in}}%
\pgfpathlineto{\pgfqpoint{1.852937in}{1.589993in}}%
\pgfpathlineto{\pgfqpoint{1.853055in}{1.589993in}}%
\pgfpathlineto{\pgfqpoint{1.865453in}{1.591079in}}%
\pgfpathlineto{\pgfqpoint{1.866462in}{1.591266in}}%
\pgfpathlineto{\pgfqpoint{1.876656in}{1.592352in}}%
\pgfpathlineto{\pgfqpoint{1.877680in}{1.592507in}}%
\pgfpathlineto{\pgfqpoint{1.892565in}{1.593594in}}%
\pgfpathlineto{\pgfqpoint{1.893284in}{1.593687in}}%
\pgfpathlineto{\pgfqpoint{1.893628in}{1.593687in}}%
\pgfpathlineto{\pgfqpoint{1.909091in}{1.594773in}}%
\pgfpathlineto{\pgfqpoint{1.909865in}{1.594898in}}%
\pgfpathlineto{\pgfqpoint{1.910108in}{1.594898in}}%
\pgfpathlineto{\pgfqpoint{1.920310in}{1.595984in}}%
\pgfpathlineto{\pgfqpoint{1.920427in}{1.596046in}}%
\pgfpathlineto{\pgfqpoint{1.920497in}{1.596046in}}%
\pgfpathlineto{\pgfqpoint{1.935578in}{1.597133in}}%
\pgfpathlineto{\pgfqpoint{1.936484in}{1.597350in}}%
\pgfpathlineto{\pgfqpoint{1.951127in}{1.598436in}}%
\pgfpathlineto{\pgfqpoint{1.951987in}{1.598498in}}%
\pgfpathlineto{\pgfqpoint{1.968388in}{1.599585in}}%
\pgfpathlineto{\pgfqpoint{1.969451in}{1.599709in}}%
\pgfpathlineto{\pgfqpoint{1.991599in}{1.600796in}}%
\pgfpathlineto{\pgfqpoint{1.992334in}{1.600889in}}%
\pgfpathlineto{\pgfqpoint{1.992654in}{1.600889in}}%
\pgfpathlineto{\pgfqpoint{2.033126in}{1.601944in}}%
\pgfpathlineto{\pgfqpoint{2.033126in}{1.601944in}}%
\pgfusepath{stroke}%
\end{pgfscope}%
\begin{pgfscope}%
\pgfsetrectcap%
\pgfsetmiterjoin%
\pgfsetlinewidth{0.803000pt}%
\definecolor{currentstroke}{rgb}{0.000000,0.000000,0.000000}%
\pgfsetstrokecolor{currentstroke}%
\pgfsetdash{}{0pt}%
\pgfpathmoveto{\pgfqpoint{0.553581in}{0.499444in}}%
\pgfpathlineto{\pgfqpoint{0.553581in}{1.654444in}}%
\pgfusepath{stroke}%
\end{pgfscope}%
\begin{pgfscope}%
\pgfsetrectcap%
\pgfsetmiterjoin%
\pgfsetlinewidth{0.803000pt}%
\definecolor{currentstroke}{rgb}{0.000000,0.000000,0.000000}%
\pgfsetstrokecolor{currentstroke}%
\pgfsetdash{}{0pt}%
\pgfpathmoveto{\pgfqpoint{2.103581in}{0.499444in}}%
\pgfpathlineto{\pgfqpoint{2.103581in}{1.654444in}}%
\pgfusepath{stroke}%
\end{pgfscope}%
\begin{pgfscope}%
\pgfsetrectcap%
\pgfsetmiterjoin%
\pgfsetlinewidth{0.803000pt}%
\definecolor{currentstroke}{rgb}{0.000000,0.000000,0.000000}%
\pgfsetstrokecolor{currentstroke}%
\pgfsetdash{}{0pt}%
\pgfpathmoveto{\pgfqpoint{0.553581in}{0.499444in}}%
\pgfpathlineto{\pgfqpoint{2.103581in}{0.499444in}}%
\pgfusepath{stroke}%
\end{pgfscope}%
\begin{pgfscope}%
\pgfsetrectcap%
\pgfsetmiterjoin%
\pgfsetlinewidth{0.803000pt}%
\definecolor{currentstroke}{rgb}{0.000000,0.000000,0.000000}%
\pgfsetstrokecolor{currentstroke}%
\pgfsetdash{}{0pt}%
\pgfpathmoveto{\pgfqpoint{0.553581in}{1.654444in}}%
\pgfpathlineto{\pgfqpoint{2.103581in}{1.654444in}}%
\pgfusepath{stroke}%
\end{pgfscope}%
\begin{pgfscope}%
\pgfsetbuttcap%
\pgfsetmiterjoin%
\definecolor{currentfill}{rgb}{1.000000,1.000000,1.000000}%
\pgfsetfillcolor{currentfill}%
\pgfsetfillopacity{0.800000}%
\pgfsetlinewidth{1.003750pt}%
\definecolor{currentstroke}{rgb}{0.800000,0.800000,0.800000}%
\pgfsetstrokecolor{currentstroke}%
\pgfsetstrokeopacity{0.800000}%
\pgfsetdash{}{0pt}%
\pgfpathmoveto{\pgfqpoint{0.832747in}{0.568889in}}%
\pgfpathlineto{\pgfqpoint{2.006358in}{0.568889in}}%
\pgfpathquadraticcurveto{\pgfqpoint{2.034136in}{0.568889in}}{\pgfqpoint{2.034136in}{0.596666in}}%
\pgfpathlineto{\pgfqpoint{2.034136in}{0.776388in}}%
\pgfpathquadraticcurveto{\pgfqpoint{2.034136in}{0.804166in}}{\pgfqpoint{2.006358in}{0.804166in}}%
\pgfpathlineto{\pgfqpoint{0.832747in}{0.804166in}}%
\pgfpathquadraticcurveto{\pgfqpoint{0.804970in}{0.804166in}}{\pgfqpoint{0.804970in}{0.776388in}}%
\pgfpathlineto{\pgfqpoint{0.804970in}{0.596666in}}%
\pgfpathquadraticcurveto{\pgfqpoint{0.804970in}{0.568889in}}{\pgfqpoint{0.832747in}{0.568889in}}%
\pgfpathlineto{\pgfqpoint{0.832747in}{0.568889in}}%
\pgfpathclose%
\pgfusepath{stroke,fill}%
\end{pgfscope}%
\begin{pgfscope}%
\pgfsetrectcap%
\pgfsetroundjoin%
\pgfsetlinewidth{1.505625pt}%
\definecolor{currentstroke}{rgb}{0.000000,0.000000,0.000000}%
\pgfsetstrokecolor{currentstroke}%
\pgfsetdash{}{0pt}%
\pgfpathmoveto{\pgfqpoint{0.860525in}{0.700000in}}%
\pgfpathlineto{\pgfqpoint{0.999414in}{0.700000in}}%
\pgfpathlineto{\pgfqpoint{1.138303in}{0.700000in}}%
\pgfusepath{stroke}%
\end{pgfscope}%
\begin{pgfscope}%
\definecolor{textcolor}{rgb}{0.000000,0.000000,0.000000}%
\pgfsetstrokecolor{textcolor}%
\pgfsetfillcolor{textcolor}%
\pgftext[x=1.249414in,y=0.651388in,left,base]{\color{textcolor}\rmfamily\fontsize{10.000000}{12.000000}\selectfont AUC=0.778}%
\end{pgfscope}%
\end{pgfpicture}%
\makeatother%
\endgroup%

\end{tabular}

\


%
\verb|AdaBoost_Hard_Tomek_0_v1_Test|

\

In this model the values are clustered very tightly, but in that small range the 214,070 samples return 210,442 different values of $p$, so there is much diversity that we can't see in this representation.  

\noindent\begin{tabular}{@{\hspace{-6pt}}p{4.3in} @{\hspace{-6pt}}p{2.0in}}
	\vskip 0pt
	\hfil Raw Model Output
	
	%% Creator: Matplotlib, PGF backend
%%
%% To include the figure in your LaTeX document, write
%%   \input{<filename>.pgf}
%%
%% Make sure the required packages are loaded in your preamble
%%   \usepackage{pgf}
%%
%% Also ensure that all the required font packages are loaded; for instance,
%% the lmodern package is sometimes necessary when using math font.
%%   \usepackage{lmodern}
%%
%% Figures using additional raster images can only be included by \input if
%% they are in the same directory as the main LaTeX file. For loading figures
%% from other directories you can use the `import` package
%%   \usepackage{import}
%%
%% and then include the figures with
%%   \import{<path to file>}{<filename>.pgf}
%%
%% Matplotlib used the following preamble
%%   
%%   \usepackage{fontspec}
%%   \makeatletter\@ifpackageloaded{underscore}{}{\usepackage[strings]{underscore}}\makeatother
%%
\begingroup%
\makeatletter%
\begin{pgfpicture}%
\pgfpathrectangle{\pgfpointorigin}{\pgfqpoint{4.102500in}{1.754444in}}%
\pgfusepath{use as bounding box, clip}%
\begin{pgfscope}%
\pgfsetbuttcap%
\pgfsetmiterjoin%
\definecolor{currentfill}{rgb}{1.000000,1.000000,1.000000}%
\pgfsetfillcolor{currentfill}%
\pgfsetlinewidth{0.000000pt}%
\definecolor{currentstroke}{rgb}{1.000000,1.000000,1.000000}%
\pgfsetstrokecolor{currentstroke}%
\pgfsetdash{}{0pt}%
\pgfpathmoveto{\pgfqpoint{0.000000in}{0.000000in}}%
\pgfpathlineto{\pgfqpoint{4.102500in}{0.000000in}}%
\pgfpathlineto{\pgfqpoint{4.102500in}{1.754444in}}%
\pgfpathlineto{\pgfqpoint{0.000000in}{1.754444in}}%
\pgfpathlineto{\pgfqpoint{0.000000in}{0.000000in}}%
\pgfpathclose%
\pgfusepath{fill}%
\end{pgfscope}%
\begin{pgfscope}%
\pgfsetbuttcap%
\pgfsetmiterjoin%
\definecolor{currentfill}{rgb}{1.000000,1.000000,1.000000}%
\pgfsetfillcolor{currentfill}%
\pgfsetlinewidth{0.000000pt}%
\definecolor{currentstroke}{rgb}{0.000000,0.000000,0.000000}%
\pgfsetstrokecolor{currentstroke}%
\pgfsetstrokeopacity{0.000000}%
\pgfsetdash{}{0pt}%
\pgfpathmoveto{\pgfqpoint{0.515000in}{0.499444in}}%
\pgfpathlineto{\pgfqpoint{4.002500in}{0.499444in}}%
\pgfpathlineto{\pgfqpoint{4.002500in}{1.654444in}}%
\pgfpathlineto{\pgfqpoint{0.515000in}{1.654444in}}%
\pgfpathlineto{\pgfqpoint{0.515000in}{0.499444in}}%
\pgfpathclose%
\pgfusepath{fill}%
\end{pgfscope}%
\begin{pgfscope}%
\pgfpathrectangle{\pgfqpoint{0.515000in}{0.499444in}}{\pgfqpoint{3.487500in}{1.155000in}}%
\pgfusepath{clip}%
\pgfsetbuttcap%
\pgfsetmiterjoin%
\pgfsetlinewidth{1.003750pt}%
\definecolor{currentstroke}{rgb}{0.000000,0.000000,0.000000}%
\pgfsetstrokecolor{currentstroke}%
\pgfsetdash{}{0pt}%
\pgfpathmoveto{\pgfqpoint{0.610114in}{0.499444in}}%
\pgfpathlineto{\pgfqpoint{0.673523in}{0.499444in}}%
\pgfpathlineto{\pgfqpoint{0.673523in}{0.499444in}}%
\pgfpathlineto{\pgfqpoint{0.610114in}{0.499444in}}%
\pgfpathlineto{\pgfqpoint{0.610114in}{0.499444in}}%
\pgfpathclose%
\pgfusepath{stroke}%
\end{pgfscope}%
\begin{pgfscope}%
\pgfpathrectangle{\pgfqpoint{0.515000in}{0.499444in}}{\pgfqpoint{3.487500in}{1.155000in}}%
\pgfusepath{clip}%
\pgfsetbuttcap%
\pgfsetmiterjoin%
\pgfsetlinewidth{1.003750pt}%
\definecolor{currentstroke}{rgb}{0.000000,0.000000,0.000000}%
\pgfsetstrokecolor{currentstroke}%
\pgfsetdash{}{0pt}%
\pgfpathmoveto{\pgfqpoint{0.768637in}{0.499444in}}%
\pgfpathlineto{\pgfqpoint{0.832046in}{0.499444in}}%
\pgfpathlineto{\pgfqpoint{0.832046in}{0.499444in}}%
\pgfpathlineto{\pgfqpoint{0.768637in}{0.499444in}}%
\pgfpathlineto{\pgfqpoint{0.768637in}{0.499444in}}%
\pgfpathclose%
\pgfusepath{stroke}%
\end{pgfscope}%
\begin{pgfscope}%
\pgfpathrectangle{\pgfqpoint{0.515000in}{0.499444in}}{\pgfqpoint{3.487500in}{1.155000in}}%
\pgfusepath{clip}%
\pgfsetbuttcap%
\pgfsetmiterjoin%
\pgfsetlinewidth{1.003750pt}%
\definecolor{currentstroke}{rgb}{0.000000,0.000000,0.000000}%
\pgfsetstrokecolor{currentstroke}%
\pgfsetdash{}{0pt}%
\pgfpathmoveto{\pgfqpoint{0.927159in}{0.499444in}}%
\pgfpathlineto{\pgfqpoint{0.990568in}{0.499444in}}%
\pgfpathlineto{\pgfqpoint{0.990568in}{0.499444in}}%
\pgfpathlineto{\pgfqpoint{0.927159in}{0.499444in}}%
\pgfpathlineto{\pgfqpoint{0.927159in}{0.499444in}}%
\pgfpathclose%
\pgfusepath{stroke}%
\end{pgfscope}%
\begin{pgfscope}%
\pgfpathrectangle{\pgfqpoint{0.515000in}{0.499444in}}{\pgfqpoint{3.487500in}{1.155000in}}%
\pgfusepath{clip}%
\pgfsetbuttcap%
\pgfsetmiterjoin%
\pgfsetlinewidth{1.003750pt}%
\definecolor{currentstroke}{rgb}{0.000000,0.000000,0.000000}%
\pgfsetstrokecolor{currentstroke}%
\pgfsetdash{}{0pt}%
\pgfpathmoveto{\pgfqpoint{1.085682in}{0.499444in}}%
\pgfpathlineto{\pgfqpoint{1.149091in}{0.499444in}}%
\pgfpathlineto{\pgfqpoint{1.149091in}{0.499444in}}%
\pgfpathlineto{\pgfqpoint{1.085682in}{0.499444in}}%
\pgfpathlineto{\pgfqpoint{1.085682in}{0.499444in}}%
\pgfpathclose%
\pgfusepath{stroke}%
\end{pgfscope}%
\begin{pgfscope}%
\pgfpathrectangle{\pgfqpoint{0.515000in}{0.499444in}}{\pgfqpoint{3.487500in}{1.155000in}}%
\pgfusepath{clip}%
\pgfsetbuttcap%
\pgfsetmiterjoin%
\pgfsetlinewidth{1.003750pt}%
\definecolor{currentstroke}{rgb}{0.000000,0.000000,0.000000}%
\pgfsetstrokecolor{currentstroke}%
\pgfsetdash{}{0pt}%
\pgfpathmoveto{\pgfqpoint{1.244205in}{0.499444in}}%
\pgfpathlineto{\pgfqpoint{1.307614in}{0.499444in}}%
\pgfpathlineto{\pgfqpoint{1.307614in}{0.499444in}}%
\pgfpathlineto{\pgfqpoint{1.244205in}{0.499444in}}%
\pgfpathlineto{\pgfqpoint{1.244205in}{0.499444in}}%
\pgfpathclose%
\pgfusepath{stroke}%
\end{pgfscope}%
\begin{pgfscope}%
\pgfpathrectangle{\pgfqpoint{0.515000in}{0.499444in}}{\pgfqpoint{3.487500in}{1.155000in}}%
\pgfusepath{clip}%
\pgfsetbuttcap%
\pgfsetmiterjoin%
\pgfsetlinewidth{1.003750pt}%
\definecolor{currentstroke}{rgb}{0.000000,0.000000,0.000000}%
\pgfsetstrokecolor{currentstroke}%
\pgfsetdash{}{0pt}%
\pgfpathmoveto{\pgfqpoint{1.402728in}{0.499444in}}%
\pgfpathlineto{\pgfqpoint{1.466137in}{0.499444in}}%
\pgfpathlineto{\pgfqpoint{1.466137in}{0.499444in}}%
\pgfpathlineto{\pgfqpoint{1.402728in}{0.499444in}}%
\pgfpathlineto{\pgfqpoint{1.402728in}{0.499444in}}%
\pgfpathclose%
\pgfusepath{stroke}%
\end{pgfscope}%
\begin{pgfscope}%
\pgfpathrectangle{\pgfqpoint{0.515000in}{0.499444in}}{\pgfqpoint{3.487500in}{1.155000in}}%
\pgfusepath{clip}%
\pgfsetbuttcap%
\pgfsetmiterjoin%
\pgfsetlinewidth{1.003750pt}%
\definecolor{currentstroke}{rgb}{0.000000,0.000000,0.000000}%
\pgfsetstrokecolor{currentstroke}%
\pgfsetdash{}{0pt}%
\pgfpathmoveto{\pgfqpoint{1.561250in}{0.499444in}}%
\pgfpathlineto{\pgfqpoint{1.624659in}{0.499444in}}%
\pgfpathlineto{\pgfqpoint{1.624659in}{0.499444in}}%
\pgfpathlineto{\pgfqpoint{1.561250in}{0.499444in}}%
\pgfpathlineto{\pgfqpoint{1.561250in}{0.499444in}}%
\pgfpathclose%
\pgfusepath{stroke}%
\end{pgfscope}%
\begin{pgfscope}%
\pgfpathrectangle{\pgfqpoint{0.515000in}{0.499444in}}{\pgfqpoint{3.487500in}{1.155000in}}%
\pgfusepath{clip}%
\pgfsetbuttcap%
\pgfsetmiterjoin%
\pgfsetlinewidth{1.003750pt}%
\definecolor{currentstroke}{rgb}{0.000000,0.000000,0.000000}%
\pgfsetstrokecolor{currentstroke}%
\pgfsetdash{}{0pt}%
\pgfpathmoveto{\pgfqpoint{1.719773in}{0.499444in}}%
\pgfpathlineto{\pgfqpoint{1.783182in}{0.499444in}}%
\pgfpathlineto{\pgfqpoint{1.783182in}{0.499444in}}%
\pgfpathlineto{\pgfqpoint{1.719773in}{0.499444in}}%
\pgfpathlineto{\pgfqpoint{1.719773in}{0.499444in}}%
\pgfpathclose%
\pgfusepath{stroke}%
\end{pgfscope}%
\begin{pgfscope}%
\pgfpathrectangle{\pgfqpoint{0.515000in}{0.499444in}}{\pgfqpoint{3.487500in}{1.155000in}}%
\pgfusepath{clip}%
\pgfsetbuttcap%
\pgfsetmiterjoin%
\pgfsetlinewidth{1.003750pt}%
\definecolor{currentstroke}{rgb}{0.000000,0.000000,0.000000}%
\pgfsetstrokecolor{currentstroke}%
\pgfsetdash{}{0pt}%
\pgfpathmoveto{\pgfqpoint{1.878296in}{0.499444in}}%
\pgfpathlineto{\pgfqpoint{1.941705in}{0.499444in}}%
\pgfpathlineto{\pgfqpoint{1.941705in}{0.499444in}}%
\pgfpathlineto{\pgfqpoint{1.878296in}{0.499444in}}%
\pgfpathlineto{\pgfqpoint{1.878296in}{0.499444in}}%
\pgfpathclose%
\pgfusepath{stroke}%
\end{pgfscope}%
\begin{pgfscope}%
\pgfpathrectangle{\pgfqpoint{0.515000in}{0.499444in}}{\pgfqpoint{3.487500in}{1.155000in}}%
\pgfusepath{clip}%
\pgfsetbuttcap%
\pgfsetmiterjoin%
\pgfsetlinewidth{1.003750pt}%
\definecolor{currentstroke}{rgb}{0.000000,0.000000,0.000000}%
\pgfsetstrokecolor{currentstroke}%
\pgfsetdash{}{0pt}%
\pgfpathmoveto{\pgfqpoint{2.036818in}{0.499444in}}%
\pgfpathlineto{\pgfqpoint{2.100228in}{0.499444in}}%
\pgfpathlineto{\pgfqpoint{2.100228in}{0.499444in}}%
\pgfpathlineto{\pgfqpoint{2.036818in}{0.499444in}}%
\pgfpathlineto{\pgfqpoint{2.036818in}{0.499444in}}%
\pgfpathclose%
\pgfusepath{stroke}%
\end{pgfscope}%
\begin{pgfscope}%
\pgfpathrectangle{\pgfqpoint{0.515000in}{0.499444in}}{\pgfqpoint{3.487500in}{1.155000in}}%
\pgfusepath{clip}%
\pgfsetbuttcap%
\pgfsetmiterjoin%
\pgfsetlinewidth{1.003750pt}%
\definecolor{currentstroke}{rgb}{0.000000,0.000000,0.000000}%
\pgfsetstrokecolor{currentstroke}%
\pgfsetdash{}{0pt}%
\pgfpathmoveto{\pgfqpoint{2.195341in}{0.499444in}}%
\pgfpathlineto{\pgfqpoint{2.258750in}{0.499444in}}%
\pgfpathlineto{\pgfqpoint{2.258750in}{1.599444in}}%
\pgfpathlineto{\pgfqpoint{2.195341in}{1.599444in}}%
\pgfpathlineto{\pgfqpoint{2.195341in}{0.499444in}}%
\pgfpathclose%
\pgfusepath{stroke}%
\end{pgfscope}%
\begin{pgfscope}%
\pgfpathrectangle{\pgfqpoint{0.515000in}{0.499444in}}{\pgfqpoint{3.487500in}{1.155000in}}%
\pgfusepath{clip}%
\pgfsetbuttcap%
\pgfsetmiterjoin%
\pgfsetlinewidth{1.003750pt}%
\definecolor{currentstroke}{rgb}{0.000000,0.000000,0.000000}%
\pgfsetstrokecolor{currentstroke}%
\pgfsetdash{}{0pt}%
\pgfpathmoveto{\pgfqpoint{2.353864in}{0.499444in}}%
\pgfpathlineto{\pgfqpoint{2.417273in}{0.499444in}}%
\pgfpathlineto{\pgfqpoint{2.417273in}{0.519649in}}%
\pgfpathlineto{\pgfqpoint{2.353864in}{0.519649in}}%
\pgfpathlineto{\pgfqpoint{2.353864in}{0.499444in}}%
\pgfpathclose%
\pgfusepath{stroke}%
\end{pgfscope}%
\begin{pgfscope}%
\pgfpathrectangle{\pgfqpoint{0.515000in}{0.499444in}}{\pgfqpoint{3.487500in}{1.155000in}}%
\pgfusepath{clip}%
\pgfsetbuttcap%
\pgfsetmiterjoin%
\pgfsetlinewidth{1.003750pt}%
\definecolor{currentstroke}{rgb}{0.000000,0.000000,0.000000}%
\pgfsetstrokecolor{currentstroke}%
\pgfsetdash{}{0pt}%
\pgfpathmoveto{\pgfqpoint{2.512387in}{0.499444in}}%
\pgfpathlineto{\pgfqpoint{2.575796in}{0.499444in}}%
\pgfpathlineto{\pgfqpoint{2.575796in}{0.499444in}}%
\pgfpathlineto{\pgfqpoint{2.512387in}{0.499444in}}%
\pgfpathlineto{\pgfqpoint{2.512387in}{0.499444in}}%
\pgfpathclose%
\pgfusepath{stroke}%
\end{pgfscope}%
\begin{pgfscope}%
\pgfpathrectangle{\pgfqpoint{0.515000in}{0.499444in}}{\pgfqpoint{3.487500in}{1.155000in}}%
\pgfusepath{clip}%
\pgfsetbuttcap%
\pgfsetmiterjoin%
\pgfsetlinewidth{1.003750pt}%
\definecolor{currentstroke}{rgb}{0.000000,0.000000,0.000000}%
\pgfsetstrokecolor{currentstroke}%
\pgfsetdash{}{0pt}%
\pgfpathmoveto{\pgfqpoint{2.670909in}{0.499444in}}%
\pgfpathlineto{\pgfqpoint{2.734318in}{0.499444in}}%
\pgfpathlineto{\pgfqpoint{2.734318in}{0.499444in}}%
\pgfpathlineto{\pgfqpoint{2.670909in}{0.499444in}}%
\pgfpathlineto{\pgfqpoint{2.670909in}{0.499444in}}%
\pgfpathclose%
\pgfusepath{stroke}%
\end{pgfscope}%
\begin{pgfscope}%
\pgfpathrectangle{\pgfqpoint{0.515000in}{0.499444in}}{\pgfqpoint{3.487500in}{1.155000in}}%
\pgfusepath{clip}%
\pgfsetbuttcap%
\pgfsetmiterjoin%
\pgfsetlinewidth{1.003750pt}%
\definecolor{currentstroke}{rgb}{0.000000,0.000000,0.000000}%
\pgfsetstrokecolor{currentstroke}%
\pgfsetdash{}{0pt}%
\pgfpathmoveto{\pgfqpoint{2.829432in}{0.499444in}}%
\pgfpathlineto{\pgfqpoint{2.892841in}{0.499444in}}%
\pgfpathlineto{\pgfqpoint{2.892841in}{0.499444in}}%
\pgfpathlineto{\pgfqpoint{2.829432in}{0.499444in}}%
\pgfpathlineto{\pgfqpoint{2.829432in}{0.499444in}}%
\pgfpathclose%
\pgfusepath{stroke}%
\end{pgfscope}%
\begin{pgfscope}%
\pgfpathrectangle{\pgfqpoint{0.515000in}{0.499444in}}{\pgfqpoint{3.487500in}{1.155000in}}%
\pgfusepath{clip}%
\pgfsetbuttcap%
\pgfsetmiterjoin%
\pgfsetlinewidth{1.003750pt}%
\definecolor{currentstroke}{rgb}{0.000000,0.000000,0.000000}%
\pgfsetstrokecolor{currentstroke}%
\pgfsetdash{}{0pt}%
\pgfpathmoveto{\pgfqpoint{2.987955in}{0.499444in}}%
\pgfpathlineto{\pgfqpoint{3.051364in}{0.499444in}}%
\pgfpathlineto{\pgfqpoint{3.051364in}{0.499444in}}%
\pgfpathlineto{\pgfqpoint{2.987955in}{0.499444in}}%
\pgfpathlineto{\pgfqpoint{2.987955in}{0.499444in}}%
\pgfpathclose%
\pgfusepath{stroke}%
\end{pgfscope}%
\begin{pgfscope}%
\pgfpathrectangle{\pgfqpoint{0.515000in}{0.499444in}}{\pgfqpoint{3.487500in}{1.155000in}}%
\pgfusepath{clip}%
\pgfsetbuttcap%
\pgfsetmiterjoin%
\pgfsetlinewidth{1.003750pt}%
\definecolor{currentstroke}{rgb}{0.000000,0.000000,0.000000}%
\pgfsetstrokecolor{currentstroke}%
\pgfsetdash{}{0pt}%
\pgfpathmoveto{\pgfqpoint{3.146478in}{0.499444in}}%
\pgfpathlineto{\pgfqpoint{3.209887in}{0.499444in}}%
\pgfpathlineto{\pgfqpoint{3.209887in}{0.499444in}}%
\pgfpathlineto{\pgfqpoint{3.146478in}{0.499444in}}%
\pgfpathlineto{\pgfqpoint{3.146478in}{0.499444in}}%
\pgfpathclose%
\pgfusepath{stroke}%
\end{pgfscope}%
\begin{pgfscope}%
\pgfpathrectangle{\pgfqpoint{0.515000in}{0.499444in}}{\pgfqpoint{3.487500in}{1.155000in}}%
\pgfusepath{clip}%
\pgfsetbuttcap%
\pgfsetmiterjoin%
\pgfsetlinewidth{1.003750pt}%
\definecolor{currentstroke}{rgb}{0.000000,0.000000,0.000000}%
\pgfsetstrokecolor{currentstroke}%
\pgfsetdash{}{0pt}%
\pgfpathmoveto{\pgfqpoint{3.305000in}{0.499444in}}%
\pgfpathlineto{\pgfqpoint{3.368409in}{0.499444in}}%
\pgfpathlineto{\pgfqpoint{3.368409in}{0.499444in}}%
\pgfpathlineto{\pgfqpoint{3.305000in}{0.499444in}}%
\pgfpathlineto{\pgfqpoint{3.305000in}{0.499444in}}%
\pgfpathclose%
\pgfusepath{stroke}%
\end{pgfscope}%
\begin{pgfscope}%
\pgfpathrectangle{\pgfqpoint{0.515000in}{0.499444in}}{\pgfqpoint{3.487500in}{1.155000in}}%
\pgfusepath{clip}%
\pgfsetbuttcap%
\pgfsetmiterjoin%
\pgfsetlinewidth{1.003750pt}%
\definecolor{currentstroke}{rgb}{0.000000,0.000000,0.000000}%
\pgfsetstrokecolor{currentstroke}%
\pgfsetdash{}{0pt}%
\pgfpathmoveto{\pgfqpoint{3.463523in}{0.499444in}}%
\pgfpathlineto{\pgfqpoint{3.526932in}{0.499444in}}%
\pgfpathlineto{\pgfqpoint{3.526932in}{0.499444in}}%
\pgfpathlineto{\pgfqpoint{3.463523in}{0.499444in}}%
\pgfpathlineto{\pgfqpoint{3.463523in}{0.499444in}}%
\pgfpathclose%
\pgfusepath{stroke}%
\end{pgfscope}%
\begin{pgfscope}%
\pgfpathrectangle{\pgfqpoint{0.515000in}{0.499444in}}{\pgfqpoint{3.487500in}{1.155000in}}%
\pgfusepath{clip}%
\pgfsetbuttcap%
\pgfsetmiterjoin%
\pgfsetlinewidth{1.003750pt}%
\definecolor{currentstroke}{rgb}{0.000000,0.000000,0.000000}%
\pgfsetstrokecolor{currentstroke}%
\pgfsetdash{}{0pt}%
\pgfpathmoveto{\pgfqpoint{3.622046in}{0.499444in}}%
\pgfpathlineto{\pgfqpoint{3.685455in}{0.499444in}}%
\pgfpathlineto{\pgfqpoint{3.685455in}{0.499444in}}%
\pgfpathlineto{\pgfqpoint{3.622046in}{0.499444in}}%
\pgfpathlineto{\pgfqpoint{3.622046in}{0.499444in}}%
\pgfpathclose%
\pgfusepath{stroke}%
\end{pgfscope}%
\begin{pgfscope}%
\pgfpathrectangle{\pgfqpoint{0.515000in}{0.499444in}}{\pgfqpoint{3.487500in}{1.155000in}}%
\pgfusepath{clip}%
\pgfsetbuttcap%
\pgfsetmiterjoin%
\pgfsetlinewidth{1.003750pt}%
\definecolor{currentstroke}{rgb}{0.000000,0.000000,0.000000}%
\pgfsetstrokecolor{currentstroke}%
\pgfsetdash{}{0pt}%
\pgfpathmoveto{\pgfqpoint{3.780568in}{0.499444in}}%
\pgfpathlineto{\pgfqpoint{3.843978in}{0.499444in}}%
\pgfpathlineto{\pgfqpoint{3.843978in}{0.499444in}}%
\pgfpathlineto{\pgfqpoint{3.780568in}{0.499444in}}%
\pgfpathlineto{\pgfqpoint{3.780568in}{0.499444in}}%
\pgfpathclose%
\pgfusepath{stroke}%
\end{pgfscope}%
\begin{pgfscope}%
\pgfpathrectangle{\pgfqpoint{0.515000in}{0.499444in}}{\pgfqpoint{3.487500in}{1.155000in}}%
\pgfusepath{clip}%
\pgfsetbuttcap%
\pgfsetmiterjoin%
\definecolor{currentfill}{rgb}{0.000000,0.000000,0.000000}%
\pgfsetfillcolor{currentfill}%
\pgfsetlinewidth{0.000000pt}%
\definecolor{currentstroke}{rgb}{0.000000,0.000000,0.000000}%
\pgfsetstrokecolor{currentstroke}%
\pgfsetstrokeopacity{0.000000}%
\pgfsetdash{}{0pt}%
\pgfpathmoveto{\pgfqpoint{0.673523in}{0.499444in}}%
\pgfpathlineto{\pgfqpoint{0.736932in}{0.499444in}}%
\pgfpathlineto{\pgfqpoint{0.736932in}{0.499444in}}%
\pgfpathlineto{\pgfqpoint{0.673523in}{0.499444in}}%
\pgfpathlineto{\pgfqpoint{0.673523in}{0.499444in}}%
\pgfpathclose%
\pgfusepath{fill}%
\end{pgfscope}%
\begin{pgfscope}%
\pgfpathrectangle{\pgfqpoint{0.515000in}{0.499444in}}{\pgfqpoint{3.487500in}{1.155000in}}%
\pgfusepath{clip}%
\pgfsetbuttcap%
\pgfsetmiterjoin%
\definecolor{currentfill}{rgb}{0.000000,0.000000,0.000000}%
\pgfsetfillcolor{currentfill}%
\pgfsetlinewidth{0.000000pt}%
\definecolor{currentstroke}{rgb}{0.000000,0.000000,0.000000}%
\pgfsetstrokecolor{currentstroke}%
\pgfsetstrokeopacity{0.000000}%
\pgfsetdash{}{0pt}%
\pgfpathmoveto{\pgfqpoint{0.832046in}{0.499444in}}%
\pgfpathlineto{\pgfqpoint{0.895455in}{0.499444in}}%
\pgfpathlineto{\pgfqpoint{0.895455in}{0.499444in}}%
\pgfpathlineto{\pgfqpoint{0.832046in}{0.499444in}}%
\pgfpathlineto{\pgfqpoint{0.832046in}{0.499444in}}%
\pgfpathclose%
\pgfusepath{fill}%
\end{pgfscope}%
\begin{pgfscope}%
\pgfpathrectangle{\pgfqpoint{0.515000in}{0.499444in}}{\pgfqpoint{3.487500in}{1.155000in}}%
\pgfusepath{clip}%
\pgfsetbuttcap%
\pgfsetmiterjoin%
\definecolor{currentfill}{rgb}{0.000000,0.000000,0.000000}%
\pgfsetfillcolor{currentfill}%
\pgfsetlinewidth{0.000000pt}%
\definecolor{currentstroke}{rgb}{0.000000,0.000000,0.000000}%
\pgfsetstrokecolor{currentstroke}%
\pgfsetstrokeopacity{0.000000}%
\pgfsetdash{}{0pt}%
\pgfpathmoveto{\pgfqpoint{0.990568in}{0.499444in}}%
\pgfpathlineto{\pgfqpoint{1.053978in}{0.499444in}}%
\pgfpathlineto{\pgfqpoint{1.053978in}{0.499444in}}%
\pgfpathlineto{\pgfqpoint{0.990568in}{0.499444in}}%
\pgfpathlineto{\pgfqpoint{0.990568in}{0.499444in}}%
\pgfpathclose%
\pgfusepath{fill}%
\end{pgfscope}%
\begin{pgfscope}%
\pgfpathrectangle{\pgfqpoint{0.515000in}{0.499444in}}{\pgfqpoint{3.487500in}{1.155000in}}%
\pgfusepath{clip}%
\pgfsetbuttcap%
\pgfsetmiterjoin%
\definecolor{currentfill}{rgb}{0.000000,0.000000,0.000000}%
\pgfsetfillcolor{currentfill}%
\pgfsetlinewidth{0.000000pt}%
\definecolor{currentstroke}{rgb}{0.000000,0.000000,0.000000}%
\pgfsetstrokecolor{currentstroke}%
\pgfsetstrokeopacity{0.000000}%
\pgfsetdash{}{0pt}%
\pgfpathmoveto{\pgfqpoint{1.149091in}{0.499444in}}%
\pgfpathlineto{\pgfqpoint{1.212500in}{0.499444in}}%
\pgfpathlineto{\pgfqpoint{1.212500in}{0.499444in}}%
\pgfpathlineto{\pgfqpoint{1.149091in}{0.499444in}}%
\pgfpathlineto{\pgfqpoint{1.149091in}{0.499444in}}%
\pgfpathclose%
\pgfusepath{fill}%
\end{pgfscope}%
\begin{pgfscope}%
\pgfpathrectangle{\pgfqpoint{0.515000in}{0.499444in}}{\pgfqpoint{3.487500in}{1.155000in}}%
\pgfusepath{clip}%
\pgfsetbuttcap%
\pgfsetmiterjoin%
\definecolor{currentfill}{rgb}{0.000000,0.000000,0.000000}%
\pgfsetfillcolor{currentfill}%
\pgfsetlinewidth{0.000000pt}%
\definecolor{currentstroke}{rgb}{0.000000,0.000000,0.000000}%
\pgfsetstrokecolor{currentstroke}%
\pgfsetstrokeopacity{0.000000}%
\pgfsetdash{}{0pt}%
\pgfpathmoveto{\pgfqpoint{1.307614in}{0.499444in}}%
\pgfpathlineto{\pgfqpoint{1.371023in}{0.499444in}}%
\pgfpathlineto{\pgfqpoint{1.371023in}{0.499444in}}%
\pgfpathlineto{\pgfqpoint{1.307614in}{0.499444in}}%
\pgfpathlineto{\pgfqpoint{1.307614in}{0.499444in}}%
\pgfpathclose%
\pgfusepath{fill}%
\end{pgfscope}%
\begin{pgfscope}%
\pgfpathrectangle{\pgfqpoint{0.515000in}{0.499444in}}{\pgfqpoint{3.487500in}{1.155000in}}%
\pgfusepath{clip}%
\pgfsetbuttcap%
\pgfsetmiterjoin%
\definecolor{currentfill}{rgb}{0.000000,0.000000,0.000000}%
\pgfsetfillcolor{currentfill}%
\pgfsetlinewidth{0.000000pt}%
\definecolor{currentstroke}{rgb}{0.000000,0.000000,0.000000}%
\pgfsetstrokecolor{currentstroke}%
\pgfsetstrokeopacity{0.000000}%
\pgfsetdash{}{0pt}%
\pgfpathmoveto{\pgfqpoint{1.466137in}{0.499444in}}%
\pgfpathlineto{\pgfqpoint{1.529546in}{0.499444in}}%
\pgfpathlineto{\pgfqpoint{1.529546in}{0.499444in}}%
\pgfpathlineto{\pgfqpoint{1.466137in}{0.499444in}}%
\pgfpathlineto{\pgfqpoint{1.466137in}{0.499444in}}%
\pgfpathclose%
\pgfusepath{fill}%
\end{pgfscope}%
\begin{pgfscope}%
\pgfpathrectangle{\pgfqpoint{0.515000in}{0.499444in}}{\pgfqpoint{3.487500in}{1.155000in}}%
\pgfusepath{clip}%
\pgfsetbuttcap%
\pgfsetmiterjoin%
\definecolor{currentfill}{rgb}{0.000000,0.000000,0.000000}%
\pgfsetfillcolor{currentfill}%
\pgfsetlinewidth{0.000000pt}%
\definecolor{currentstroke}{rgb}{0.000000,0.000000,0.000000}%
\pgfsetstrokecolor{currentstroke}%
\pgfsetstrokeopacity{0.000000}%
\pgfsetdash{}{0pt}%
\pgfpathmoveto{\pgfqpoint{1.624659in}{0.499444in}}%
\pgfpathlineto{\pgfqpoint{1.688068in}{0.499444in}}%
\pgfpathlineto{\pgfqpoint{1.688068in}{0.499444in}}%
\pgfpathlineto{\pgfqpoint{1.624659in}{0.499444in}}%
\pgfpathlineto{\pgfqpoint{1.624659in}{0.499444in}}%
\pgfpathclose%
\pgfusepath{fill}%
\end{pgfscope}%
\begin{pgfscope}%
\pgfpathrectangle{\pgfqpoint{0.515000in}{0.499444in}}{\pgfqpoint{3.487500in}{1.155000in}}%
\pgfusepath{clip}%
\pgfsetbuttcap%
\pgfsetmiterjoin%
\definecolor{currentfill}{rgb}{0.000000,0.000000,0.000000}%
\pgfsetfillcolor{currentfill}%
\pgfsetlinewidth{0.000000pt}%
\definecolor{currentstroke}{rgb}{0.000000,0.000000,0.000000}%
\pgfsetstrokecolor{currentstroke}%
\pgfsetstrokeopacity{0.000000}%
\pgfsetdash{}{0pt}%
\pgfpathmoveto{\pgfqpoint{1.783182in}{0.499444in}}%
\pgfpathlineto{\pgfqpoint{1.846591in}{0.499444in}}%
\pgfpathlineto{\pgfqpoint{1.846591in}{0.499444in}}%
\pgfpathlineto{\pgfqpoint{1.783182in}{0.499444in}}%
\pgfpathlineto{\pgfqpoint{1.783182in}{0.499444in}}%
\pgfpathclose%
\pgfusepath{fill}%
\end{pgfscope}%
\begin{pgfscope}%
\pgfpathrectangle{\pgfqpoint{0.515000in}{0.499444in}}{\pgfqpoint{3.487500in}{1.155000in}}%
\pgfusepath{clip}%
\pgfsetbuttcap%
\pgfsetmiterjoin%
\definecolor{currentfill}{rgb}{0.000000,0.000000,0.000000}%
\pgfsetfillcolor{currentfill}%
\pgfsetlinewidth{0.000000pt}%
\definecolor{currentstroke}{rgb}{0.000000,0.000000,0.000000}%
\pgfsetstrokecolor{currentstroke}%
\pgfsetstrokeopacity{0.000000}%
\pgfsetdash{}{0pt}%
\pgfpathmoveto{\pgfqpoint{1.941705in}{0.499444in}}%
\pgfpathlineto{\pgfqpoint{2.005114in}{0.499444in}}%
\pgfpathlineto{\pgfqpoint{2.005114in}{0.499444in}}%
\pgfpathlineto{\pgfqpoint{1.941705in}{0.499444in}}%
\pgfpathlineto{\pgfqpoint{1.941705in}{0.499444in}}%
\pgfpathclose%
\pgfusepath{fill}%
\end{pgfscope}%
\begin{pgfscope}%
\pgfpathrectangle{\pgfqpoint{0.515000in}{0.499444in}}{\pgfqpoint{3.487500in}{1.155000in}}%
\pgfusepath{clip}%
\pgfsetbuttcap%
\pgfsetmiterjoin%
\definecolor{currentfill}{rgb}{0.000000,0.000000,0.000000}%
\pgfsetfillcolor{currentfill}%
\pgfsetlinewidth{0.000000pt}%
\definecolor{currentstroke}{rgb}{0.000000,0.000000,0.000000}%
\pgfsetstrokecolor{currentstroke}%
\pgfsetstrokeopacity{0.000000}%
\pgfsetdash{}{0pt}%
\pgfpathmoveto{\pgfqpoint{2.100228in}{0.499444in}}%
\pgfpathlineto{\pgfqpoint{2.163637in}{0.499444in}}%
\pgfpathlineto{\pgfqpoint{2.163637in}{0.499444in}}%
\pgfpathlineto{\pgfqpoint{2.100228in}{0.499444in}}%
\pgfpathlineto{\pgfqpoint{2.100228in}{0.499444in}}%
\pgfpathclose%
\pgfusepath{fill}%
\end{pgfscope}%
\begin{pgfscope}%
\pgfpathrectangle{\pgfqpoint{0.515000in}{0.499444in}}{\pgfqpoint{3.487500in}{1.155000in}}%
\pgfusepath{clip}%
\pgfsetbuttcap%
\pgfsetmiterjoin%
\definecolor{currentfill}{rgb}{0.000000,0.000000,0.000000}%
\pgfsetfillcolor{currentfill}%
\pgfsetlinewidth{0.000000pt}%
\definecolor{currentstroke}{rgb}{0.000000,0.000000,0.000000}%
\pgfsetstrokecolor{currentstroke}%
\pgfsetstrokeopacity{0.000000}%
\pgfsetdash{}{0pt}%
\pgfpathmoveto{\pgfqpoint{2.258750in}{0.499444in}}%
\pgfpathlineto{\pgfqpoint{2.322159in}{0.499444in}}%
\pgfpathlineto{\pgfqpoint{2.322159in}{0.685879in}}%
\pgfpathlineto{\pgfqpoint{2.258750in}{0.685879in}}%
\pgfpathlineto{\pgfqpoint{2.258750in}{0.499444in}}%
\pgfpathclose%
\pgfusepath{fill}%
\end{pgfscope}%
\begin{pgfscope}%
\pgfpathrectangle{\pgfqpoint{0.515000in}{0.499444in}}{\pgfqpoint{3.487500in}{1.155000in}}%
\pgfusepath{clip}%
\pgfsetbuttcap%
\pgfsetmiterjoin%
\definecolor{currentfill}{rgb}{0.000000,0.000000,0.000000}%
\pgfsetfillcolor{currentfill}%
\pgfsetlinewidth{0.000000pt}%
\definecolor{currentstroke}{rgb}{0.000000,0.000000,0.000000}%
\pgfsetstrokecolor{currentstroke}%
\pgfsetstrokeopacity{0.000000}%
\pgfsetdash{}{0pt}%
\pgfpathmoveto{\pgfqpoint{2.417273in}{0.499444in}}%
\pgfpathlineto{\pgfqpoint{2.480682in}{0.499444in}}%
\pgfpathlineto{\pgfqpoint{2.480682in}{0.523229in}}%
\pgfpathlineto{\pgfqpoint{2.417273in}{0.523229in}}%
\pgfpathlineto{\pgfqpoint{2.417273in}{0.499444in}}%
\pgfpathclose%
\pgfusepath{fill}%
\end{pgfscope}%
\begin{pgfscope}%
\pgfpathrectangle{\pgfqpoint{0.515000in}{0.499444in}}{\pgfqpoint{3.487500in}{1.155000in}}%
\pgfusepath{clip}%
\pgfsetbuttcap%
\pgfsetmiterjoin%
\definecolor{currentfill}{rgb}{0.000000,0.000000,0.000000}%
\pgfsetfillcolor{currentfill}%
\pgfsetlinewidth{0.000000pt}%
\definecolor{currentstroke}{rgb}{0.000000,0.000000,0.000000}%
\pgfsetstrokecolor{currentstroke}%
\pgfsetstrokeopacity{0.000000}%
\pgfsetdash{}{0pt}%
\pgfpathmoveto{\pgfqpoint{2.575796in}{0.499444in}}%
\pgfpathlineto{\pgfqpoint{2.639205in}{0.499444in}}%
\pgfpathlineto{\pgfqpoint{2.639205in}{0.499444in}}%
\pgfpathlineto{\pgfqpoint{2.575796in}{0.499444in}}%
\pgfpathlineto{\pgfqpoint{2.575796in}{0.499444in}}%
\pgfpathclose%
\pgfusepath{fill}%
\end{pgfscope}%
\begin{pgfscope}%
\pgfpathrectangle{\pgfqpoint{0.515000in}{0.499444in}}{\pgfqpoint{3.487500in}{1.155000in}}%
\pgfusepath{clip}%
\pgfsetbuttcap%
\pgfsetmiterjoin%
\definecolor{currentfill}{rgb}{0.000000,0.000000,0.000000}%
\pgfsetfillcolor{currentfill}%
\pgfsetlinewidth{0.000000pt}%
\definecolor{currentstroke}{rgb}{0.000000,0.000000,0.000000}%
\pgfsetstrokecolor{currentstroke}%
\pgfsetstrokeopacity{0.000000}%
\pgfsetdash{}{0pt}%
\pgfpathmoveto{\pgfqpoint{2.734318in}{0.499444in}}%
\pgfpathlineto{\pgfqpoint{2.797728in}{0.499444in}}%
\pgfpathlineto{\pgfqpoint{2.797728in}{0.499444in}}%
\pgfpathlineto{\pgfqpoint{2.734318in}{0.499444in}}%
\pgfpathlineto{\pgfqpoint{2.734318in}{0.499444in}}%
\pgfpathclose%
\pgfusepath{fill}%
\end{pgfscope}%
\begin{pgfscope}%
\pgfpathrectangle{\pgfqpoint{0.515000in}{0.499444in}}{\pgfqpoint{3.487500in}{1.155000in}}%
\pgfusepath{clip}%
\pgfsetbuttcap%
\pgfsetmiterjoin%
\definecolor{currentfill}{rgb}{0.000000,0.000000,0.000000}%
\pgfsetfillcolor{currentfill}%
\pgfsetlinewidth{0.000000pt}%
\definecolor{currentstroke}{rgb}{0.000000,0.000000,0.000000}%
\pgfsetstrokecolor{currentstroke}%
\pgfsetstrokeopacity{0.000000}%
\pgfsetdash{}{0pt}%
\pgfpathmoveto{\pgfqpoint{2.892841in}{0.499444in}}%
\pgfpathlineto{\pgfqpoint{2.956250in}{0.499444in}}%
\pgfpathlineto{\pgfqpoint{2.956250in}{0.499444in}}%
\pgfpathlineto{\pgfqpoint{2.892841in}{0.499444in}}%
\pgfpathlineto{\pgfqpoint{2.892841in}{0.499444in}}%
\pgfpathclose%
\pgfusepath{fill}%
\end{pgfscope}%
\begin{pgfscope}%
\pgfpathrectangle{\pgfqpoint{0.515000in}{0.499444in}}{\pgfqpoint{3.487500in}{1.155000in}}%
\pgfusepath{clip}%
\pgfsetbuttcap%
\pgfsetmiterjoin%
\definecolor{currentfill}{rgb}{0.000000,0.000000,0.000000}%
\pgfsetfillcolor{currentfill}%
\pgfsetlinewidth{0.000000pt}%
\definecolor{currentstroke}{rgb}{0.000000,0.000000,0.000000}%
\pgfsetstrokecolor{currentstroke}%
\pgfsetstrokeopacity{0.000000}%
\pgfsetdash{}{0pt}%
\pgfpathmoveto{\pgfqpoint{3.051364in}{0.499444in}}%
\pgfpathlineto{\pgfqpoint{3.114773in}{0.499444in}}%
\pgfpathlineto{\pgfqpoint{3.114773in}{0.499444in}}%
\pgfpathlineto{\pgfqpoint{3.051364in}{0.499444in}}%
\pgfpathlineto{\pgfqpoint{3.051364in}{0.499444in}}%
\pgfpathclose%
\pgfusepath{fill}%
\end{pgfscope}%
\begin{pgfscope}%
\pgfpathrectangle{\pgfqpoint{0.515000in}{0.499444in}}{\pgfqpoint{3.487500in}{1.155000in}}%
\pgfusepath{clip}%
\pgfsetbuttcap%
\pgfsetmiterjoin%
\definecolor{currentfill}{rgb}{0.000000,0.000000,0.000000}%
\pgfsetfillcolor{currentfill}%
\pgfsetlinewidth{0.000000pt}%
\definecolor{currentstroke}{rgb}{0.000000,0.000000,0.000000}%
\pgfsetstrokecolor{currentstroke}%
\pgfsetstrokeopacity{0.000000}%
\pgfsetdash{}{0pt}%
\pgfpathmoveto{\pgfqpoint{3.209887in}{0.499444in}}%
\pgfpathlineto{\pgfqpoint{3.273296in}{0.499444in}}%
\pgfpathlineto{\pgfqpoint{3.273296in}{0.499444in}}%
\pgfpathlineto{\pgfqpoint{3.209887in}{0.499444in}}%
\pgfpathlineto{\pgfqpoint{3.209887in}{0.499444in}}%
\pgfpathclose%
\pgfusepath{fill}%
\end{pgfscope}%
\begin{pgfscope}%
\pgfpathrectangle{\pgfqpoint{0.515000in}{0.499444in}}{\pgfqpoint{3.487500in}{1.155000in}}%
\pgfusepath{clip}%
\pgfsetbuttcap%
\pgfsetmiterjoin%
\definecolor{currentfill}{rgb}{0.000000,0.000000,0.000000}%
\pgfsetfillcolor{currentfill}%
\pgfsetlinewidth{0.000000pt}%
\definecolor{currentstroke}{rgb}{0.000000,0.000000,0.000000}%
\pgfsetstrokecolor{currentstroke}%
\pgfsetstrokeopacity{0.000000}%
\pgfsetdash{}{0pt}%
\pgfpathmoveto{\pgfqpoint{3.368409in}{0.499444in}}%
\pgfpathlineto{\pgfqpoint{3.431818in}{0.499444in}}%
\pgfpathlineto{\pgfqpoint{3.431818in}{0.499444in}}%
\pgfpathlineto{\pgfqpoint{3.368409in}{0.499444in}}%
\pgfpathlineto{\pgfqpoint{3.368409in}{0.499444in}}%
\pgfpathclose%
\pgfusepath{fill}%
\end{pgfscope}%
\begin{pgfscope}%
\pgfpathrectangle{\pgfqpoint{0.515000in}{0.499444in}}{\pgfqpoint{3.487500in}{1.155000in}}%
\pgfusepath{clip}%
\pgfsetbuttcap%
\pgfsetmiterjoin%
\definecolor{currentfill}{rgb}{0.000000,0.000000,0.000000}%
\pgfsetfillcolor{currentfill}%
\pgfsetlinewidth{0.000000pt}%
\definecolor{currentstroke}{rgb}{0.000000,0.000000,0.000000}%
\pgfsetstrokecolor{currentstroke}%
\pgfsetstrokeopacity{0.000000}%
\pgfsetdash{}{0pt}%
\pgfpathmoveto{\pgfqpoint{3.526932in}{0.499444in}}%
\pgfpathlineto{\pgfqpoint{3.590341in}{0.499444in}}%
\pgfpathlineto{\pgfqpoint{3.590341in}{0.499444in}}%
\pgfpathlineto{\pgfqpoint{3.526932in}{0.499444in}}%
\pgfpathlineto{\pgfqpoint{3.526932in}{0.499444in}}%
\pgfpathclose%
\pgfusepath{fill}%
\end{pgfscope}%
\begin{pgfscope}%
\pgfpathrectangle{\pgfqpoint{0.515000in}{0.499444in}}{\pgfqpoint{3.487500in}{1.155000in}}%
\pgfusepath{clip}%
\pgfsetbuttcap%
\pgfsetmiterjoin%
\definecolor{currentfill}{rgb}{0.000000,0.000000,0.000000}%
\pgfsetfillcolor{currentfill}%
\pgfsetlinewidth{0.000000pt}%
\definecolor{currentstroke}{rgb}{0.000000,0.000000,0.000000}%
\pgfsetstrokecolor{currentstroke}%
\pgfsetstrokeopacity{0.000000}%
\pgfsetdash{}{0pt}%
\pgfpathmoveto{\pgfqpoint{3.685455in}{0.499444in}}%
\pgfpathlineto{\pgfqpoint{3.748864in}{0.499444in}}%
\pgfpathlineto{\pgfqpoint{3.748864in}{0.499444in}}%
\pgfpathlineto{\pgfqpoint{3.685455in}{0.499444in}}%
\pgfpathlineto{\pgfqpoint{3.685455in}{0.499444in}}%
\pgfpathclose%
\pgfusepath{fill}%
\end{pgfscope}%
\begin{pgfscope}%
\pgfpathrectangle{\pgfqpoint{0.515000in}{0.499444in}}{\pgfqpoint{3.487500in}{1.155000in}}%
\pgfusepath{clip}%
\pgfsetbuttcap%
\pgfsetmiterjoin%
\definecolor{currentfill}{rgb}{0.000000,0.000000,0.000000}%
\pgfsetfillcolor{currentfill}%
\pgfsetlinewidth{0.000000pt}%
\definecolor{currentstroke}{rgb}{0.000000,0.000000,0.000000}%
\pgfsetstrokecolor{currentstroke}%
\pgfsetstrokeopacity{0.000000}%
\pgfsetdash{}{0pt}%
\pgfpathmoveto{\pgfqpoint{3.843978in}{0.499444in}}%
\pgfpathlineto{\pgfqpoint{3.907387in}{0.499444in}}%
\pgfpathlineto{\pgfqpoint{3.907387in}{0.499444in}}%
\pgfpathlineto{\pgfqpoint{3.843978in}{0.499444in}}%
\pgfpathlineto{\pgfqpoint{3.843978in}{0.499444in}}%
\pgfpathclose%
\pgfusepath{fill}%
\end{pgfscope}%
\begin{pgfscope}%
\pgfsetbuttcap%
\pgfsetroundjoin%
\definecolor{currentfill}{rgb}{0.000000,0.000000,0.000000}%
\pgfsetfillcolor{currentfill}%
\pgfsetlinewidth{0.803000pt}%
\definecolor{currentstroke}{rgb}{0.000000,0.000000,0.000000}%
\pgfsetstrokecolor{currentstroke}%
\pgfsetdash{}{0pt}%
\pgfsys@defobject{currentmarker}{\pgfqpoint{0.000000in}{-0.048611in}}{\pgfqpoint{0.000000in}{0.000000in}}{%
\pgfpathmoveto{\pgfqpoint{0.000000in}{0.000000in}}%
\pgfpathlineto{\pgfqpoint{0.000000in}{-0.048611in}}%
\pgfusepath{stroke,fill}%
}%
\begin{pgfscope}%
\pgfsys@transformshift{0.515000in}{0.499444in}%
\pgfsys@useobject{currentmarker}{}%
\end{pgfscope}%
\end{pgfscope}%
\begin{pgfscope}%
\pgfsetbuttcap%
\pgfsetroundjoin%
\definecolor{currentfill}{rgb}{0.000000,0.000000,0.000000}%
\pgfsetfillcolor{currentfill}%
\pgfsetlinewidth{0.803000pt}%
\definecolor{currentstroke}{rgb}{0.000000,0.000000,0.000000}%
\pgfsetstrokecolor{currentstroke}%
\pgfsetdash{}{0pt}%
\pgfsys@defobject{currentmarker}{\pgfqpoint{0.000000in}{-0.048611in}}{\pgfqpoint{0.000000in}{0.000000in}}{%
\pgfpathmoveto{\pgfqpoint{0.000000in}{0.000000in}}%
\pgfpathlineto{\pgfqpoint{0.000000in}{-0.048611in}}%
\pgfusepath{stroke,fill}%
}%
\begin{pgfscope}%
\pgfsys@transformshift{0.673523in}{0.499444in}%
\pgfsys@useobject{currentmarker}{}%
\end{pgfscope}%
\end{pgfscope}%
\begin{pgfscope}%
\definecolor{textcolor}{rgb}{0.000000,0.000000,0.000000}%
\pgfsetstrokecolor{textcolor}%
\pgfsetfillcolor{textcolor}%
\pgftext[x=0.673523in,y=0.402222in,,top]{\color{textcolor}\rmfamily\fontsize{10.000000}{12.000000}\selectfont 0.0}%
\end{pgfscope}%
\begin{pgfscope}%
\pgfsetbuttcap%
\pgfsetroundjoin%
\definecolor{currentfill}{rgb}{0.000000,0.000000,0.000000}%
\pgfsetfillcolor{currentfill}%
\pgfsetlinewidth{0.803000pt}%
\definecolor{currentstroke}{rgb}{0.000000,0.000000,0.000000}%
\pgfsetstrokecolor{currentstroke}%
\pgfsetdash{}{0pt}%
\pgfsys@defobject{currentmarker}{\pgfqpoint{0.000000in}{-0.048611in}}{\pgfqpoint{0.000000in}{0.000000in}}{%
\pgfpathmoveto{\pgfqpoint{0.000000in}{0.000000in}}%
\pgfpathlineto{\pgfqpoint{0.000000in}{-0.048611in}}%
\pgfusepath{stroke,fill}%
}%
\begin{pgfscope}%
\pgfsys@transformshift{0.832046in}{0.499444in}%
\pgfsys@useobject{currentmarker}{}%
\end{pgfscope}%
\end{pgfscope}%
\begin{pgfscope}%
\pgfsetbuttcap%
\pgfsetroundjoin%
\definecolor{currentfill}{rgb}{0.000000,0.000000,0.000000}%
\pgfsetfillcolor{currentfill}%
\pgfsetlinewidth{0.803000pt}%
\definecolor{currentstroke}{rgb}{0.000000,0.000000,0.000000}%
\pgfsetstrokecolor{currentstroke}%
\pgfsetdash{}{0pt}%
\pgfsys@defobject{currentmarker}{\pgfqpoint{0.000000in}{-0.048611in}}{\pgfqpoint{0.000000in}{0.000000in}}{%
\pgfpathmoveto{\pgfqpoint{0.000000in}{0.000000in}}%
\pgfpathlineto{\pgfqpoint{0.000000in}{-0.048611in}}%
\pgfusepath{stroke,fill}%
}%
\begin{pgfscope}%
\pgfsys@transformshift{0.990568in}{0.499444in}%
\pgfsys@useobject{currentmarker}{}%
\end{pgfscope}%
\end{pgfscope}%
\begin{pgfscope}%
\definecolor{textcolor}{rgb}{0.000000,0.000000,0.000000}%
\pgfsetstrokecolor{textcolor}%
\pgfsetfillcolor{textcolor}%
\pgftext[x=0.990568in,y=0.402222in,,top]{\color{textcolor}\rmfamily\fontsize{10.000000}{12.000000}\selectfont 0.1}%
\end{pgfscope}%
\begin{pgfscope}%
\pgfsetbuttcap%
\pgfsetroundjoin%
\definecolor{currentfill}{rgb}{0.000000,0.000000,0.000000}%
\pgfsetfillcolor{currentfill}%
\pgfsetlinewidth{0.803000pt}%
\definecolor{currentstroke}{rgb}{0.000000,0.000000,0.000000}%
\pgfsetstrokecolor{currentstroke}%
\pgfsetdash{}{0pt}%
\pgfsys@defobject{currentmarker}{\pgfqpoint{0.000000in}{-0.048611in}}{\pgfqpoint{0.000000in}{0.000000in}}{%
\pgfpathmoveto{\pgfqpoint{0.000000in}{0.000000in}}%
\pgfpathlineto{\pgfqpoint{0.000000in}{-0.048611in}}%
\pgfusepath{stroke,fill}%
}%
\begin{pgfscope}%
\pgfsys@transformshift{1.149091in}{0.499444in}%
\pgfsys@useobject{currentmarker}{}%
\end{pgfscope}%
\end{pgfscope}%
\begin{pgfscope}%
\pgfsetbuttcap%
\pgfsetroundjoin%
\definecolor{currentfill}{rgb}{0.000000,0.000000,0.000000}%
\pgfsetfillcolor{currentfill}%
\pgfsetlinewidth{0.803000pt}%
\definecolor{currentstroke}{rgb}{0.000000,0.000000,0.000000}%
\pgfsetstrokecolor{currentstroke}%
\pgfsetdash{}{0pt}%
\pgfsys@defobject{currentmarker}{\pgfqpoint{0.000000in}{-0.048611in}}{\pgfqpoint{0.000000in}{0.000000in}}{%
\pgfpathmoveto{\pgfqpoint{0.000000in}{0.000000in}}%
\pgfpathlineto{\pgfqpoint{0.000000in}{-0.048611in}}%
\pgfusepath{stroke,fill}%
}%
\begin{pgfscope}%
\pgfsys@transformshift{1.307614in}{0.499444in}%
\pgfsys@useobject{currentmarker}{}%
\end{pgfscope}%
\end{pgfscope}%
\begin{pgfscope}%
\definecolor{textcolor}{rgb}{0.000000,0.000000,0.000000}%
\pgfsetstrokecolor{textcolor}%
\pgfsetfillcolor{textcolor}%
\pgftext[x=1.307614in,y=0.402222in,,top]{\color{textcolor}\rmfamily\fontsize{10.000000}{12.000000}\selectfont 0.2}%
\end{pgfscope}%
\begin{pgfscope}%
\pgfsetbuttcap%
\pgfsetroundjoin%
\definecolor{currentfill}{rgb}{0.000000,0.000000,0.000000}%
\pgfsetfillcolor{currentfill}%
\pgfsetlinewidth{0.803000pt}%
\definecolor{currentstroke}{rgb}{0.000000,0.000000,0.000000}%
\pgfsetstrokecolor{currentstroke}%
\pgfsetdash{}{0pt}%
\pgfsys@defobject{currentmarker}{\pgfqpoint{0.000000in}{-0.048611in}}{\pgfqpoint{0.000000in}{0.000000in}}{%
\pgfpathmoveto{\pgfqpoint{0.000000in}{0.000000in}}%
\pgfpathlineto{\pgfqpoint{0.000000in}{-0.048611in}}%
\pgfusepath{stroke,fill}%
}%
\begin{pgfscope}%
\pgfsys@transformshift{1.466137in}{0.499444in}%
\pgfsys@useobject{currentmarker}{}%
\end{pgfscope}%
\end{pgfscope}%
\begin{pgfscope}%
\pgfsetbuttcap%
\pgfsetroundjoin%
\definecolor{currentfill}{rgb}{0.000000,0.000000,0.000000}%
\pgfsetfillcolor{currentfill}%
\pgfsetlinewidth{0.803000pt}%
\definecolor{currentstroke}{rgb}{0.000000,0.000000,0.000000}%
\pgfsetstrokecolor{currentstroke}%
\pgfsetdash{}{0pt}%
\pgfsys@defobject{currentmarker}{\pgfqpoint{0.000000in}{-0.048611in}}{\pgfqpoint{0.000000in}{0.000000in}}{%
\pgfpathmoveto{\pgfqpoint{0.000000in}{0.000000in}}%
\pgfpathlineto{\pgfqpoint{0.000000in}{-0.048611in}}%
\pgfusepath{stroke,fill}%
}%
\begin{pgfscope}%
\pgfsys@transformshift{1.624659in}{0.499444in}%
\pgfsys@useobject{currentmarker}{}%
\end{pgfscope}%
\end{pgfscope}%
\begin{pgfscope}%
\definecolor{textcolor}{rgb}{0.000000,0.000000,0.000000}%
\pgfsetstrokecolor{textcolor}%
\pgfsetfillcolor{textcolor}%
\pgftext[x=1.624659in,y=0.402222in,,top]{\color{textcolor}\rmfamily\fontsize{10.000000}{12.000000}\selectfont 0.3}%
\end{pgfscope}%
\begin{pgfscope}%
\pgfsetbuttcap%
\pgfsetroundjoin%
\definecolor{currentfill}{rgb}{0.000000,0.000000,0.000000}%
\pgfsetfillcolor{currentfill}%
\pgfsetlinewidth{0.803000pt}%
\definecolor{currentstroke}{rgb}{0.000000,0.000000,0.000000}%
\pgfsetstrokecolor{currentstroke}%
\pgfsetdash{}{0pt}%
\pgfsys@defobject{currentmarker}{\pgfqpoint{0.000000in}{-0.048611in}}{\pgfqpoint{0.000000in}{0.000000in}}{%
\pgfpathmoveto{\pgfqpoint{0.000000in}{0.000000in}}%
\pgfpathlineto{\pgfqpoint{0.000000in}{-0.048611in}}%
\pgfusepath{stroke,fill}%
}%
\begin{pgfscope}%
\pgfsys@transformshift{1.783182in}{0.499444in}%
\pgfsys@useobject{currentmarker}{}%
\end{pgfscope}%
\end{pgfscope}%
\begin{pgfscope}%
\pgfsetbuttcap%
\pgfsetroundjoin%
\definecolor{currentfill}{rgb}{0.000000,0.000000,0.000000}%
\pgfsetfillcolor{currentfill}%
\pgfsetlinewidth{0.803000pt}%
\definecolor{currentstroke}{rgb}{0.000000,0.000000,0.000000}%
\pgfsetstrokecolor{currentstroke}%
\pgfsetdash{}{0pt}%
\pgfsys@defobject{currentmarker}{\pgfqpoint{0.000000in}{-0.048611in}}{\pgfqpoint{0.000000in}{0.000000in}}{%
\pgfpathmoveto{\pgfqpoint{0.000000in}{0.000000in}}%
\pgfpathlineto{\pgfqpoint{0.000000in}{-0.048611in}}%
\pgfusepath{stroke,fill}%
}%
\begin{pgfscope}%
\pgfsys@transformshift{1.941705in}{0.499444in}%
\pgfsys@useobject{currentmarker}{}%
\end{pgfscope}%
\end{pgfscope}%
\begin{pgfscope}%
\definecolor{textcolor}{rgb}{0.000000,0.000000,0.000000}%
\pgfsetstrokecolor{textcolor}%
\pgfsetfillcolor{textcolor}%
\pgftext[x=1.941705in,y=0.402222in,,top]{\color{textcolor}\rmfamily\fontsize{10.000000}{12.000000}\selectfont 0.4}%
\end{pgfscope}%
\begin{pgfscope}%
\pgfsetbuttcap%
\pgfsetroundjoin%
\definecolor{currentfill}{rgb}{0.000000,0.000000,0.000000}%
\pgfsetfillcolor{currentfill}%
\pgfsetlinewidth{0.803000pt}%
\definecolor{currentstroke}{rgb}{0.000000,0.000000,0.000000}%
\pgfsetstrokecolor{currentstroke}%
\pgfsetdash{}{0pt}%
\pgfsys@defobject{currentmarker}{\pgfqpoint{0.000000in}{-0.048611in}}{\pgfqpoint{0.000000in}{0.000000in}}{%
\pgfpathmoveto{\pgfqpoint{0.000000in}{0.000000in}}%
\pgfpathlineto{\pgfqpoint{0.000000in}{-0.048611in}}%
\pgfusepath{stroke,fill}%
}%
\begin{pgfscope}%
\pgfsys@transformshift{2.100228in}{0.499444in}%
\pgfsys@useobject{currentmarker}{}%
\end{pgfscope}%
\end{pgfscope}%
\begin{pgfscope}%
\pgfsetbuttcap%
\pgfsetroundjoin%
\definecolor{currentfill}{rgb}{0.000000,0.000000,0.000000}%
\pgfsetfillcolor{currentfill}%
\pgfsetlinewidth{0.803000pt}%
\definecolor{currentstroke}{rgb}{0.000000,0.000000,0.000000}%
\pgfsetstrokecolor{currentstroke}%
\pgfsetdash{}{0pt}%
\pgfsys@defobject{currentmarker}{\pgfqpoint{0.000000in}{-0.048611in}}{\pgfqpoint{0.000000in}{0.000000in}}{%
\pgfpathmoveto{\pgfqpoint{0.000000in}{0.000000in}}%
\pgfpathlineto{\pgfqpoint{0.000000in}{-0.048611in}}%
\pgfusepath{stroke,fill}%
}%
\begin{pgfscope}%
\pgfsys@transformshift{2.258750in}{0.499444in}%
\pgfsys@useobject{currentmarker}{}%
\end{pgfscope}%
\end{pgfscope}%
\begin{pgfscope}%
\definecolor{textcolor}{rgb}{0.000000,0.000000,0.000000}%
\pgfsetstrokecolor{textcolor}%
\pgfsetfillcolor{textcolor}%
\pgftext[x=2.258750in,y=0.402222in,,top]{\color{textcolor}\rmfamily\fontsize{10.000000}{12.000000}\selectfont 0.5}%
\end{pgfscope}%
\begin{pgfscope}%
\pgfsetbuttcap%
\pgfsetroundjoin%
\definecolor{currentfill}{rgb}{0.000000,0.000000,0.000000}%
\pgfsetfillcolor{currentfill}%
\pgfsetlinewidth{0.803000pt}%
\definecolor{currentstroke}{rgb}{0.000000,0.000000,0.000000}%
\pgfsetstrokecolor{currentstroke}%
\pgfsetdash{}{0pt}%
\pgfsys@defobject{currentmarker}{\pgfqpoint{0.000000in}{-0.048611in}}{\pgfqpoint{0.000000in}{0.000000in}}{%
\pgfpathmoveto{\pgfqpoint{0.000000in}{0.000000in}}%
\pgfpathlineto{\pgfqpoint{0.000000in}{-0.048611in}}%
\pgfusepath{stroke,fill}%
}%
\begin{pgfscope}%
\pgfsys@transformshift{2.417273in}{0.499444in}%
\pgfsys@useobject{currentmarker}{}%
\end{pgfscope}%
\end{pgfscope}%
\begin{pgfscope}%
\pgfsetbuttcap%
\pgfsetroundjoin%
\definecolor{currentfill}{rgb}{0.000000,0.000000,0.000000}%
\pgfsetfillcolor{currentfill}%
\pgfsetlinewidth{0.803000pt}%
\definecolor{currentstroke}{rgb}{0.000000,0.000000,0.000000}%
\pgfsetstrokecolor{currentstroke}%
\pgfsetdash{}{0pt}%
\pgfsys@defobject{currentmarker}{\pgfqpoint{0.000000in}{-0.048611in}}{\pgfqpoint{0.000000in}{0.000000in}}{%
\pgfpathmoveto{\pgfqpoint{0.000000in}{0.000000in}}%
\pgfpathlineto{\pgfqpoint{0.000000in}{-0.048611in}}%
\pgfusepath{stroke,fill}%
}%
\begin{pgfscope}%
\pgfsys@transformshift{2.575796in}{0.499444in}%
\pgfsys@useobject{currentmarker}{}%
\end{pgfscope}%
\end{pgfscope}%
\begin{pgfscope}%
\definecolor{textcolor}{rgb}{0.000000,0.000000,0.000000}%
\pgfsetstrokecolor{textcolor}%
\pgfsetfillcolor{textcolor}%
\pgftext[x=2.575796in,y=0.402222in,,top]{\color{textcolor}\rmfamily\fontsize{10.000000}{12.000000}\selectfont 0.6}%
\end{pgfscope}%
\begin{pgfscope}%
\pgfsetbuttcap%
\pgfsetroundjoin%
\definecolor{currentfill}{rgb}{0.000000,0.000000,0.000000}%
\pgfsetfillcolor{currentfill}%
\pgfsetlinewidth{0.803000pt}%
\definecolor{currentstroke}{rgb}{0.000000,0.000000,0.000000}%
\pgfsetstrokecolor{currentstroke}%
\pgfsetdash{}{0pt}%
\pgfsys@defobject{currentmarker}{\pgfqpoint{0.000000in}{-0.048611in}}{\pgfqpoint{0.000000in}{0.000000in}}{%
\pgfpathmoveto{\pgfqpoint{0.000000in}{0.000000in}}%
\pgfpathlineto{\pgfqpoint{0.000000in}{-0.048611in}}%
\pgfusepath{stroke,fill}%
}%
\begin{pgfscope}%
\pgfsys@transformshift{2.734318in}{0.499444in}%
\pgfsys@useobject{currentmarker}{}%
\end{pgfscope}%
\end{pgfscope}%
\begin{pgfscope}%
\pgfsetbuttcap%
\pgfsetroundjoin%
\definecolor{currentfill}{rgb}{0.000000,0.000000,0.000000}%
\pgfsetfillcolor{currentfill}%
\pgfsetlinewidth{0.803000pt}%
\definecolor{currentstroke}{rgb}{0.000000,0.000000,0.000000}%
\pgfsetstrokecolor{currentstroke}%
\pgfsetdash{}{0pt}%
\pgfsys@defobject{currentmarker}{\pgfqpoint{0.000000in}{-0.048611in}}{\pgfqpoint{0.000000in}{0.000000in}}{%
\pgfpathmoveto{\pgfqpoint{0.000000in}{0.000000in}}%
\pgfpathlineto{\pgfqpoint{0.000000in}{-0.048611in}}%
\pgfusepath{stroke,fill}%
}%
\begin{pgfscope}%
\pgfsys@transformshift{2.892841in}{0.499444in}%
\pgfsys@useobject{currentmarker}{}%
\end{pgfscope}%
\end{pgfscope}%
\begin{pgfscope}%
\definecolor{textcolor}{rgb}{0.000000,0.000000,0.000000}%
\pgfsetstrokecolor{textcolor}%
\pgfsetfillcolor{textcolor}%
\pgftext[x=2.892841in,y=0.402222in,,top]{\color{textcolor}\rmfamily\fontsize{10.000000}{12.000000}\selectfont 0.7}%
\end{pgfscope}%
\begin{pgfscope}%
\pgfsetbuttcap%
\pgfsetroundjoin%
\definecolor{currentfill}{rgb}{0.000000,0.000000,0.000000}%
\pgfsetfillcolor{currentfill}%
\pgfsetlinewidth{0.803000pt}%
\definecolor{currentstroke}{rgb}{0.000000,0.000000,0.000000}%
\pgfsetstrokecolor{currentstroke}%
\pgfsetdash{}{0pt}%
\pgfsys@defobject{currentmarker}{\pgfqpoint{0.000000in}{-0.048611in}}{\pgfqpoint{0.000000in}{0.000000in}}{%
\pgfpathmoveto{\pgfqpoint{0.000000in}{0.000000in}}%
\pgfpathlineto{\pgfqpoint{0.000000in}{-0.048611in}}%
\pgfusepath{stroke,fill}%
}%
\begin{pgfscope}%
\pgfsys@transformshift{3.051364in}{0.499444in}%
\pgfsys@useobject{currentmarker}{}%
\end{pgfscope}%
\end{pgfscope}%
\begin{pgfscope}%
\pgfsetbuttcap%
\pgfsetroundjoin%
\definecolor{currentfill}{rgb}{0.000000,0.000000,0.000000}%
\pgfsetfillcolor{currentfill}%
\pgfsetlinewidth{0.803000pt}%
\definecolor{currentstroke}{rgb}{0.000000,0.000000,0.000000}%
\pgfsetstrokecolor{currentstroke}%
\pgfsetdash{}{0pt}%
\pgfsys@defobject{currentmarker}{\pgfqpoint{0.000000in}{-0.048611in}}{\pgfqpoint{0.000000in}{0.000000in}}{%
\pgfpathmoveto{\pgfqpoint{0.000000in}{0.000000in}}%
\pgfpathlineto{\pgfqpoint{0.000000in}{-0.048611in}}%
\pgfusepath{stroke,fill}%
}%
\begin{pgfscope}%
\pgfsys@transformshift{3.209887in}{0.499444in}%
\pgfsys@useobject{currentmarker}{}%
\end{pgfscope}%
\end{pgfscope}%
\begin{pgfscope}%
\definecolor{textcolor}{rgb}{0.000000,0.000000,0.000000}%
\pgfsetstrokecolor{textcolor}%
\pgfsetfillcolor{textcolor}%
\pgftext[x=3.209887in,y=0.402222in,,top]{\color{textcolor}\rmfamily\fontsize{10.000000}{12.000000}\selectfont 0.8}%
\end{pgfscope}%
\begin{pgfscope}%
\pgfsetbuttcap%
\pgfsetroundjoin%
\definecolor{currentfill}{rgb}{0.000000,0.000000,0.000000}%
\pgfsetfillcolor{currentfill}%
\pgfsetlinewidth{0.803000pt}%
\definecolor{currentstroke}{rgb}{0.000000,0.000000,0.000000}%
\pgfsetstrokecolor{currentstroke}%
\pgfsetdash{}{0pt}%
\pgfsys@defobject{currentmarker}{\pgfqpoint{0.000000in}{-0.048611in}}{\pgfqpoint{0.000000in}{0.000000in}}{%
\pgfpathmoveto{\pgfqpoint{0.000000in}{0.000000in}}%
\pgfpathlineto{\pgfqpoint{0.000000in}{-0.048611in}}%
\pgfusepath{stroke,fill}%
}%
\begin{pgfscope}%
\pgfsys@transformshift{3.368409in}{0.499444in}%
\pgfsys@useobject{currentmarker}{}%
\end{pgfscope}%
\end{pgfscope}%
\begin{pgfscope}%
\pgfsetbuttcap%
\pgfsetroundjoin%
\definecolor{currentfill}{rgb}{0.000000,0.000000,0.000000}%
\pgfsetfillcolor{currentfill}%
\pgfsetlinewidth{0.803000pt}%
\definecolor{currentstroke}{rgb}{0.000000,0.000000,0.000000}%
\pgfsetstrokecolor{currentstroke}%
\pgfsetdash{}{0pt}%
\pgfsys@defobject{currentmarker}{\pgfqpoint{0.000000in}{-0.048611in}}{\pgfqpoint{0.000000in}{0.000000in}}{%
\pgfpathmoveto{\pgfqpoint{0.000000in}{0.000000in}}%
\pgfpathlineto{\pgfqpoint{0.000000in}{-0.048611in}}%
\pgfusepath{stroke,fill}%
}%
\begin{pgfscope}%
\pgfsys@transformshift{3.526932in}{0.499444in}%
\pgfsys@useobject{currentmarker}{}%
\end{pgfscope}%
\end{pgfscope}%
\begin{pgfscope}%
\definecolor{textcolor}{rgb}{0.000000,0.000000,0.000000}%
\pgfsetstrokecolor{textcolor}%
\pgfsetfillcolor{textcolor}%
\pgftext[x=3.526932in,y=0.402222in,,top]{\color{textcolor}\rmfamily\fontsize{10.000000}{12.000000}\selectfont 0.9}%
\end{pgfscope}%
\begin{pgfscope}%
\pgfsetbuttcap%
\pgfsetroundjoin%
\definecolor{currentfill}{rgb}{0.000000,0.000000,0.000000}%
\pgfsetfillcolor{currentfill}%
\pgfsetlinewidth{0.803000pt}%
\definecolor{currentstroke}{rgb}{0.000000,0.000000,0.000000}%
\pgfsetstrokecolor{currentstroke}%
\pgfsetdash{}{0pt}%
\pgfsys@defobject{currentmarker}{\pgfqpoint{0.000000in}{-0.048611in}}{\pgfqpoint{0.000000in}{0.000000in}}{%
\pgfpathmoveto{\pgfqpoint{0.000000in}{0.000000in}}%
\pgfpathlineto{\pgfqpoint{0.000000in}{-0.048611in}}%
\pgfusepath{stroke,fill}%
}%
\begin{pgfscope}%
\pgfsys@transformshift{3.685455in}{0.499444in}%
\pgfsys@useobject{currentmarker}{}%
\end{pgfscope}%
\end{pgfscope}%
\begin{pgfscope}%
\pgfsetbuttcap%
\pgfsetroundjoin%
\definecolor{currentfill}{rgb}{0.000000,0.000000,0.000000}%
\pgfsetfillcolor{currentfill}%
\pgfsetlinewidth{0.803000pt}%
\definecolor{currentstroke}{rgb}{0.000000,0.000000,0.000000}%
\pgfsetstrokecolor{currentstroke}%
\pgfsetdash{}{0pt}%
\pgfsys@defobject{currentmarker}{\pgfqpoint{0.000000in}{-0.048611in}}{\pgfqpoint{0.000000in}{0.000000in}}{%
\pgfpathmoveto{\pgfqpoint{0.000000in}{0.000000in}}%
\pgfpathlineto{\pgfqpoint{0.000000in}{-0.048611in}}%
\pgfusepath{stroke,fill}%
}%
\begin{pgfscope}%
\pgfsys@transformshift{3.843978in}{0.499444in}%
\pgfsys@useobject{currentmarker}{}%
\end{pgfscope}%
\end{pgfscope}%
\begin{pgfscope}%
\definecolor{textcolor}{rgb}{0.000000,0.000000,0.000000}%
\pgfsetstrokecolor{textcolor}%
\pgfsetfillcolor{textcolor}%
\pgftext[x=3.843978in,y=0.402222in,,top]{\color{textcolor}\rmfamily\fontsize{10.000000}{12.000000}\selectfont 1.0}%
\end{pgfscope}%
\begin{pgfscope}%
\pgfsetbuttcap%
\pgfsetroundjoin%
\definecolor{currentfill}{rgb}{0.000000,0.000000,0.000000}%
\pgfsetfillcolor{currentfill}%
\pgfsetlinewidth{0.803000pt}%
\definecolor{currentstroke}{rgb}{0.000000,0.000000,0.000000}%
\pgfsetstrokecolor{currentstroke}%
\pgfsetdash{}{0pt}%
\pgfsys@defobject{currentmarker}{\pgfqpoint{0.000000in}{-0.048611in}}{\pgfqpoint{0.000000in}{0.000000in}}{%
\pgfpathmoveto{\pgfqpoint{0.000000in}{0.000000in}}%
\pgfpathlineto{\pgfqpoint{0.000000in}{-0.048611in}}%
\pgfusepath{stroke,fill}%
}%
\begin{pgfscope}%
\pgfsys@transformshift{4.002500in}{0.499444in}%
\pgfsys@useobject{currentmarker}{}%
\end{pgfscope}%
\end{pgfscope}%
\begin{pgfscope}%
\definecolor{textcolor}{rgb}{0.000000,0.000000,0.000000}%
\pgfsetstrokecolor{textcolor}%
\pgfsetfillcolor{textcolor}%
\pgftext[x=2.258750in,y=0.223333in,,top]{\color{textcolor}\rmfamily\fontsize{10.000000}{12.000000}\selectfont \(\displaystyle p\)}%
\end{pgfscope}%
\begin{pgfscope}%
\pgfsetbuttcap%
\pgfsetroundjoin%
\definecolor{currentfill}{rgb}{0.000000,0.000000,0.000000}%
\pgfsetfillcolor{currentfill}%
\pgfsetlinewidth{0.803000pt}%
\definecolor{currentstroke}{rgb}{0.000000,0.000000,0.000000}%
\pgfsetstrokecolor{currentstroke}%
\pgfsetdash{}{0pt}%
\pgfsys@defobject{currentmarker}{\pgfqpoint{-0.048611in}{0.000000in}}{\pgfqpoint{-0.000000in}{0.000000in}}{%
\pgfpathmoveto{\pgfqpoint{-0.000000in}{0.000000in}}%
\pgfpathlineto{\pgfqpoint{-0.048611in}{0.000000in}}%
\pgfusepath{stroke,fill}%
}%
\begin{pgfscope}%
\pgfsys@transformshift{0.515000in}{0.499444in}%
\pgfsys@useobject{currentmarker}{}%
\end{pgfscope}%
\end{pgfscope}%
\begin{pgfscope}%
\definecolor{textcolor}{rgb}{0.000000,0.000000,0.000000}%
\pgfsetstrokecolor{textcolor}%
\pgfsetfillcolor{textcolor}%
\pgftext[x=0.348333in, y=0.451250in, left, base]{\color{textcolor}\rmfamily\fontsize{10.000000}{12.000000}\selectfont \(\displaystyle {0}\)}%
\end{pgfscope}%
\begin{pgfscope}%
\pgfsetbuttcap%
\pgfsetroundjoin%
\definecolor{currentfill}{rgb}{0.000000,0.000000,0.000000}%
\pgfsetfillcolor{currentfill}%
\pgfsetlinewidth{0.803000pt}%
\definecolor{currentstroke}{rgb}{0.000000,0.000000,0.000000}%
\pgfsetstrokecolor{currentstroke}%
\pgfsetdash{}{0pt}%
\pgfsys@defobject{currentmarker}{\pgfqpoint{-0.048611in}{0.000000in}}{\pgfqpoint{-0.000000in}{0.000000in}}{%
\pgfpathmoveto{\pgfqpoint{-0.000000in}{0.000000in}}%
\pgfpathlineto{\pgfqpoint{-0.048611in}{0.000000in}}%
\pgfusepath{stroke,fill}%
}%
\begin{pgfscope}%
\pgfsys@transformshift{0.515000in}{0.832050in}%
\pgfsys@useobject{currentmarker}{}%
\end{pgfscope}%
\end{pgfscope}%
\begin{pgfscope}%
\definecolor{textcolor}{rgb}{0.000000,0.000000,0.000000}%
\pgfsetstrokecolor{textcolor}%
\pgfsetfillcolor{textcolor}%
\pgftext[x=0.278889in, y=0.783856in, left, base]{\color{textcolor}\rmfamily\fontsize{10.000000}{12.000000}\selectfont \(\displaystyle {25}\)}%
\end{pgfscope}%
\begin{pgfscope}%
\pgfsetbuttcap%
\pgfsetroundjoin%
\definecolor{currentfill}{rgb}{0.000000,0.000000,0.000000}%
\pgfsetfillcolor{currentfill}%
\pgfsetlinewidth{0.803000pt}%
\definecolor{currentstroke}{rgb}{0.000000,0.000000,0.000000}%
\pgfsetstrokecolor{currentstroke}%
\pgfsetdash{}{0pt}%
\pgfsys@defobject{currentmarker}{\pgfqpoint{-0.048611in}{0.000000in}}{\pgfqpoint{-0.000000in}{0.000000in}}{%
\pgfpathmoveto{\pgfqpoint{-0.000000in}{0.000000in}}%
\pgfpathlineto{\pgfqpoint{-0.048611in}{0.000000in}}%
\pgfusepath{stroke,fill}%
}%
\begin{pgfscope}%
\pgfsys@transformshift{0.515000in}{1.164656in}%
\pgfsys@useobject{currentmarker}{}%
\end{pgfscope}%
\end{pgfscope}%
\begin{pgfscope}%
\definecolor{textcolor}{rgb}{0.000000,0.000000,0.000000}%
\pgfsetstrokecolor{textcolor}%
\pgfsetfillcolor{textcolor}%
\pgftext[x=0.278889in, y=1.116462in, left, base]{\color{textcolor}\rmfamily\fontsize{10.000000}{12.000000}\selectfont \(\displaystyle {50}\)}%
\end{pgfscope}%
\begin{pgfscope}%
\pgfsetbuttcap%
\pgfsetroundjoin%
\definecolor{currentfill}{rgb}{0.000000,0.000000,0.000000}%
\pgfsetfillcolor{currentfill}%
\pgfsetlinewidth{0.803000pt}%
\definecolor{currentstroke}{rgb}{0.000000,0.000000,0.000000}%
\pgfsetstrokecolor{currentstroke}%
\pgfsetdash{}{0pt}%
\pgfsys@defobject{currentmarker}{\pgfqpoint{-0.048611in}{0.000000in}}{\pgfqpoint{-0.000000in}{0.000000in}}{%
\pgfpathmoveto{\pgfqpoint{-0.000000in}{0.000000in}}%
\pgfpathlineto{\pgfqpoint{-0.048611in}{0.000000in}}%
\pgfusepath{stroke,fill}%
}%
\begin{pgfscope}%
\pgfsys@transformshift{0.515000in}{1.497262in}%
\pgfsys@useobject{currentmarker}{}%
\end{pgfscope}%
\end{pgfscope}%
\begin{pgfscope}%
\definecolor{textcolor}{rgb}{0.000000,0.000000,0.000000}%
\pgfsetstrokecolor{textcolor}%
\pgfsetfillcolor{textcolor}%
\pgftext[x=0.278889in, y=1.449067in, left, base]{\color{textcolor}\rmfamily\fontsize{10.000000}{12.000000}\selectfont \(\displaystyle {75}\)}%
\end{pgfscope}%
\begin{pgfscope}%
\definecolor{textcolor}{rgb}{0.000000,0.000000,0.000000}%
\pgfsetstrokecolor{textcolor}%
\pgfsetfillcolor{textcolor}%
\pgftext[x=0.223333in,y=1.076944in,,bottom,rotate=90.000000]{\color{textcolor}\rmfamily\fontsize{10.000000}{12.000000}\selectfont Percent of Data Set}%
\end{pgfscope}%
\begin{pgfscope}%
\pgfsetrectcap%
\pgfsetmiterjoin%
\pgfsetlinewidth{0.803000pt}%
\definecolor{currentstroke}{rgb}{0.000000,0.000000,0.000000}%
\pgfsetstrokecolor{currentstroke}%
\pgfsetdash{}{0pt}%
\pgfpathmoveto{\pgfqpoint{0.515000in}{0.499444in}}%
\pgfpathlineto{\pgfqpoint{0.515000in}{1.654444in}}%
\pgfusepath{stroke}%
\end{pgfscope}%
\begin{pgfscope}%
\pgfsetrectcap%
\pgfsetmiterjoin%
\pgfsetlinewidth{0.803000pt}%
\definecolor{currentstroke}{rgb}{0.000000,0.000000,0.000000}%
\pgfsetstrokecolor{currentstroke}%
\pgfsetdash{}{0pt}%
\pgfpathmoveto{\pgfqpoint{4.002500in}{0.499444in}}%
\pgfpathlineto{\pgfqpoint{4.002500in}{1.654444in}}%
\pgfusepath{stroke}%
\end{pgfscope}%
\begin{pgfscope}%
\pgfsetrectcap%
\pgfsetmiterjoin%
\pgfsetlinewidth{0.803000pt}%
\definecolor{currentstroke}{rgb}{0.000000,0.000000,0.000000}%
\pgfsetstrokecolor{currentstroke}%
\pgfsetdash{}{0pt}%
\pgfpathmoveto{\pgfqpoint{0.515000in}{0.499444in}}%
\pgfpathlineto{\pgfqpoint{4.002500in}{0.499444in}}%
\pgfusepath{stroke}%
\end{pgfscope}%
\begin{pgfscope}%
\pgfsetrectcap%
\pgfsetmiterjoin%
\pgfsetlinewidth{0.803000pt}%
\definecolor{currentstroke}{rgb}{0.000000,0.000000,0.000000}%
\pgfsetstrokecolor{currentstroke}%
\pgfsetdash{}{0pt}%
\pgfpathmoveto{\pgfqpoint{0.515000in}{1.654444in}}%
\pgfpathlineto{\pgfqpoint{4.002500in}{1.654444in}}%
\pgfusepath{stroke}%
\end{pgfscope}%
\begin{pgfscope}%
\pgfsetbuttcap%
\pgfsetmiterjoin%
\definecolor{currentfill}{rgb}{1.000000,1.000000,1.000000}%
\pgfsetfillcolor{currentfill}%
\pgfsetfillopacity{0.800000}%
\pgfsetlinewidth{1.003750pt}%
\definecolor{currentstroke}{rgb}{0.800000,0.800000,0.800000}%
\pgfsetstrokecolor{currentstroke}%
\pgfsetstrokeopacity{0.800000}%
\pgfsetdash{}{0pt}%
\pgfpathmoveto{\pgfqpoint{3.225556in}{1.154445in}}%
\pgfpathlineto{\pgfqpoint{3.905278in}{1.154445in}}%
\pgfpathquadraticcurveto{\pgfqpoint{3.933056in}{1.154445in}}{\pgfqpoint{3.933056in}{1.182222in}}%
\pgfpathlineto{\pgfqpoint{3.933056in}{1.557222in}}%
\pgfpathquadraticcurveto{\pgfqpoint{3.933056in}{1.585000in}}{\pgfqpoint{3.905278in}{1.585000in}}%
\pgfpathlineto{\pgfqpoint{3.225556in}{1.585000in}}%
\pgfpathquadraticcurveto{\pgfqpoint{3.197778in}{1.585000in}}{\pgfqpoint{3.197778in}{1.557222in}}%
\pgfpathlineto{\pgfqpoint{3.197778in}{1.182222in}}%
\pgfpathquadraticcurveto{\pgfqpoint{3.197778in}{1.154445in}}{\pgfqpoint{3.225556in}{1.154445in}}%
\pgfpathlineto{\pgfqpoint{3.225556in}{1.154445in}}%
\pgfpathclose%
\pgfusepath{stroke,fill}%
\end{pgfscope}%
\begin{pgfscope}%
\pgfsetbuttcap%
\pgfsetmiterjoin%
\pgfsetlinewidth{1.003750pt}%
\definecolor{currentstroke}{rgb}{0.000000,0.000000,0.000000}%
\pgfsetstrokecolor{currentstroke}%
\pgfsetdash{}{0pt}%
\pgfpathmoveto{\pgfqpoint{3.253334in}{1.432222in}}%
\pgfpathlineto{\pgfqpoint{3.531111in}{1.432222in}}%
\pgfpathlineto{\pgfqpoint{3.531111in}{1.529444in}}%
\pgfpathlineto{\pgfqpoint{3.253334in}{1.529444in}}%
\pgfpathlineto{\pgfqpoint{3.253334in}{1.432222in}}%
\pgfpathclose%
\pgfusepath{stroke}%
\end{pgfscope}%
\begin{pgfscope}%
\definecolor{textcolor}{rgb}{0.000000,0.000000,0.000000}%
\pgfsetstrokecolor{textcolor}%
\pgfsetfillcolor{textcolor}%
\pgftext[x=3.642223in,y=1.432222in,left,base]{\color{textcolor}\rmfamily\fontsize{10.000000}{12.000000}\selectfont Neg}%
\end{pgfscope}%
\begin{pgfscope}%
\pgfsetbuttcap%
\pgfsetmiterjoin%
\definecolor{currentfill}{rgb}{0.000000,0.000000,0.000000}%
\pgfsetfillcolor{currentfill}%
\pgfsetlinewidth{0.000000pt}%
\definecolor{currentstroke}{rgb}{0.000000,0.000000,0.000000}%
\pgfsetstrokecolor{currentstroke}%
\pgfsetstrokeopacity{0.000000}%
\pgfsetdash{}{0pt}%
\pgfpathmoveto{\pgfqpoint{3.253334in}{1.236944in}}%
\pgfpathlineto{\pgfqpoint{3.531111in}{1.236944in}}%
\pgfpathlineto{\pgfqpoint{3.531111in}{1.334167in}}%
\pgfpathlineto{\pgfqpoint{3.253334in}{1.334167in}}%
\pgfpathlineto{\pgfqpoint{3.253334in}{1.236944in}}%
\pgfpathclose%
\pgfusepath{fill}%
\end{pgfscope}%
\begin{pgfscope}%
\definecolor{textcolor}{rgb}{0.000000,0.000000,0.000000}%
\pgfsetstrokecolor{textcolor}%
\pgfsetfillcolor{textcolor}%
\pgftext[x=3.642223in,y=1.236944in,left,base]{\color{textcolor}\rmfamily\fontsize{10.000000}{12.000000}\selectfont Pos}%
\end{pgfscope}%
\end{pgfpicture}%
\makeatother%
\endgroup%
	
&
	\vskip 0pt
	\hfil ROC Curve
	
	%% Creator: Matplotlib, PGF backend
%%
%% To include the figure in your LaTeX document, write
%%   \input{<filename>.pgf}
%%
%% Make sure the required packages are loaded in your preamble
%%   \usepackage{pgf}
%%
%% Also ensure that all the required font packages are loaded; for instance,
%% the lmodern package is sometimes necessary when using math font.
%%   \usepackage{lmodern}
%%
%% Figures using additional raster images can only be included by \input if
%% they are in the same directory as the main LaTeX file. For loading figures
%% from other directories you can use the `import` package
%%   \usepackage{import}
%%
%% and then include the figures with
%%   \import{<path to file>}{<filename>.pgf}
%%
%% Matplotlib used the following preamble
%%   
%%   \usepackage{fontspec}
%%   \makeatletter\@ifpackageloaded{underscore}{}{\usepackage[strings]{underscore}}\makeatother
%%
\begingroup%
\makeatletter%
\begin{pgfpicture}%
\pgfpathrectangle{\pgfpointorigin}{\pgfqpoint{2.221861in}{1.754444in}}%
\pgfusepath{use as bounding box, clip}%
\begin{pgfscope}%
\pgfsetbuttcap%
\pgfsetmiterjoin%
\definecolor{currentfill}{rgb}{1.000000,1.000000,1.000000}%
\pgfsetfillcolor{currentfill}%
\pgfsetlinewidth{0.000000pt}%
\definecolor{currentstroke}{rgb}{1.000000,1.000000,1.000000}%
\pgfsetstrokecolor{currentstroke}%
\pgfsetdash{}{0pt}%
\pgfpathmoveto{\pgfqpoint{0.000000in}{0.000000in}}%
\pgfpathlineto{\pgfqpoint{2.221861in}{0.000000in}}%
\pgfpathlineto{\pgfqpoint{2.221861in}{1.754444in}}%
\pgfpathlineto{\pgfqpoint{0.000000in}{1.754444in}}%
\pgfpathlineto{\pgfqpoint{0.000000in}{0.000000in}}%
\pgfpathclose%
\pgfusepath{fill}%
\end{pgfscope}%
\begin{pgfscope}%
\pgfsetbuttcap%
\pgfsetmiterjoin%
\definecolor{currentfill}{rgb}{1.000000,1.000000,1.000000}%
\pgfsetfillcolor{currentfill}%
\pgfsetlinewidth{0.000000pt}%
\definecolor{currentstroke}{rgb}{0.000000,0.000000,0.000000}%
\pgfsetstrokecolor{currentstroke}%
\pgfsetstrokeopacity{0.000000}%
\pgfsetdash{}{0pt}%
\pgfpathmoveto{\pgfqpoint{0.553581in}{0.499444in}}%
\pgfpathlineto{\pgfqpoint{2.103581in}{0.499444in}}%
\pgfpathlineto{\pgfqpoint{2.103581in}{1.654444in}}%
\pgfpathlineto{\pgfqpoint{0.553581in}{1.654444in}}%
\pgfpathlineto{\pgfqpoint{0.553581in}{0.499444in}}%
\pgfpathclose%
\pgfusepath{fill}%
\end{pgfscope}%
\begin{pgfscope}%
\pgfsetbuttcap%
\pgfsetroundjoin%
\definecolor{currentfill}{rgb}{0.000000,0.000000,0.000000}%
\pgfsetfillcolor{currentfill}%
\pgfsetlinewidth{0.803000pt}%
\definecolor{currentstroke}{rgb}{0.000000,0.000000,0.000000}%
\pgfsetstrokecolor{currentstroke}%
\pgfsetdash{}{0pt}%
\pgfsys@defobject{currentmarker}{\pgfqpoint{0.000000in}{-0.048611in}}{\pgfqpoint{0.000000in}{0.000000in}}{%
\pgfpathmoveto{\pgfqpoint{0.000000in}{0.000000in}}%
\pgfpathlineto{\pgfqpoint{0.000000in}{-0.048611in}}%
\pgfusepath{stroke,fill}%
}%
\begin{pgfscope}%
\pgfsys@transformshift{0.624035in}{0.499444in}%
\pgfsys@useobject{currentmarker}{}%
\end{pgfscope}%
\end{pgfscope}%
\begin{pgfscope}%
\definecolor{textcolor}{rgb}{0.000000,0.000000,0.000000}%
\pgfsetstrokecolor{textcolor}%
\pgfsetfillcolor{textcolor}%
\pgftext[x=0.624035in,y=0.402222in,,top]{\color{textcolor}\rmfamily\fontsize{10.000000}{12.000000}\selectfont \(\displaystyle {0.0}\)}%
\end{pgfscope}%
\begin{pgfscope}%
\pgfsetbuttcap%
\pgfsetroundjoin%
\definecolor{currentfill}{rgb}{0.000000,0.000000,0.000000}%
\pgfsetfillcolor{currentfill}%
\pgfsetlinewidth{0.803000pt}%
\definecolor{currentstroke}{rgb}{0.000000,0.000000,0.000000}%
\pgfsetstrokecolor{currentstroke}%
\pgfsetdash{}{0pt}%
\pgfsys@defobject{currentmarker}{\pgfqpoint{0.000000in}{-0.048611in}}{\pgfqpoint{0.000000in}{0.000000in}}{%
\pgfpathmoveto{\pgfqpoint{0.000000in}{0.000000in}}%
\pgfpathlineto{\pgfqpoint{0.000000in}{-0.048611in}}%
\pgfusepath{stroke,fill}%
}%
\begin{pgfscope}%
\pgfsys@transformshift{1.328581in}{0.499444in}%
\pgfsys@useobject{currentmarker}{}%
\end{pgfscope}%
\end{pgfscope}%
\begin{pgfscope}%
\definecolor{textcolor}{rgb}{0.000000,0.000000,0.000000}%
\pgfsetstrokecolor{textcolor}%
\pgfsetfillcolor{textcolor}%
\pgftext[x=1.328581in,y=0.402222in,,top]{\color{textcolor}\rmfamily\fontsize{10.000000}{12.000000}\selectfont \(\displaystyle {0.5}\)}%
\end{pgfscope}%
\begin{pgfscope}%
\pgfsetbuttcap%
\pgfsetroundjoin%
\definecolor{currentfill}{rgb}{0.000000,0.000000,0.000000}%
\pgfsetfillcolor{currentfill}%
\pgfsetlinewidth{0.803000pt}%
\definecolor{currentstroke}{rgb}{0.000000,0.000000,0.000000}%
\pgfsetstrokecolor{currentstroke}%
\pgfsetdash{}{0pt}%
\pgfsys@defobject{currentmarker}{\pgfqpoint{0.000000in}{-0.048611in}}{\pgfqpoint{0.000000in}{0.000000in}}{%
\pgfpathmoveto{\pgfqpoint{0.000000in}{0.000000in}}%
\pgfpathlineto{\pgfqpoint{0.000000in}{-0.048611in}}%
\pgfusepath{stroke,fill}%
}%
\begin{pgfscope}%
\pgfsys@transformshift{2.033126in}{0.499444in}%
\pgfsys@useobject{currentmarker}{}%
\end{pgfscope}%
\end{pgfscope}%
\begin{pgfscope}%
\definecolor{textcolor}{rgb}{0.000000,0.000000,0.000000}%
\pgfsetstrokecolor{textcolor}%
\pgfsetfillcolor{textcolor}%
\pgftext[x=2.033126in,y=0.402222in,,top]{\color{textcolor}\rmfamily\fontsize{10.000000}{12.000000}\selectfont \(\displaystyle {1.0}\)}%
\end{pgfscope}%
\begin{pgfscope}%
\definecolor{textcolor}{rgb}{0.000000,0.000000,0.000000}%
\pgfsetstrokecolor{textcolor}%
\pgfsetfillcolor{textcolor}%
\pgftext[x=1.328581in,y=0.223333in,,top]{\color{textcolor}\rmfamily\fontsize{10.000000}{12.000000}\selectfont False positive rate}%
\end{pgfscope}%
\begin{pgfscope}%
\pgfsetbuttcap%
\pgfsetroundjoin%
\definecolor{currentfill}{rgb}{0.000000,0.000000,0.000000}%
\pgfsetfillcolor{currentfill}%
\pgfsetlinewidth{0.803000pt}%
\definecolor{currentstroke}{rgb}{0.000000,0.000000,0.000000}%
\pgfsetstrokecolor{currentstroke}%
\pgfsetdash{}{0pt}%
\pgfsys@defobject{currentmarker}{\pgfqpoint{-0.048611in}{0.000000in}}{\pgfqpoint{-0.000000in}{0.000000in}}{%
\pgfpathmoveto{\pgfqpoint{-0.000000in}{0.000000in}}%
\pgfpathlineto{\pgfqpoint{-0.048611in}{0.000000in}}%
\pgfusepath{stroke,fill}%
}%
\begin{pgfscope}%
\pgfsys@transformshift{0.553581in}{0.551944in}%
\pgfsys@useobject{currentmarker}{}%
\end{pgfscope}%
\end{pgfscope}%
\begin{pgfscope}%
\definecolor{textcolor}{rgb}{0.000000,0.000000,0.000000}%
\pgfsetstrokecolor{textcolor}%
\pgfsetfillcolor{textcolor}%
\pgftext[x=0.278889in, y=0.503750in, left, base]{\color{textcolor}\rmfamily\fontsize{10.000000}{12.000000}\selectfont \(\displaystyle {0.0}\)}%
\end{pgfscope}%
\begin{pgfscope}%
\pgfsetbuttcap%
\pgfsetroundjoin%
\definecolor{currentfill}{rgb}{0.000000,0.000000,0.000000}%
\pgfsetfillcolor{currentfill}%
\pgfsetlinewidth{0.803000pt}%
\definecolor{currentstroke}{rgb}{0.000000,0.000000,0.000000}%
\pgfsetstrokecolor{currentstroke}%
\pgfsetdash{}{0pt}%
\pgfsys@defobject{currentmarker}{\pgfqpoint{-0.048611in}{0.000000in}}{\pgfqpoint{-0.000000in}{0.000000in}}{%
\pgfpathmoveto{\pgfqpoint{-0.000000in}{0.000000in}}%
\pgfpathlineto{\pgfqpoint{-0.048611in}{0.000000in}}%
\pgfusepath{stroke,fill}%
}%
\begin{pgfscope}%
\pgfsys@transformshift{0.553581in}{1.076944in}%
\pgfsys@useobject{currentmarker}{}%
\end{pgfscope}%
\end{pgfscope}%
\begin{pgfscope}%
\definecolor{textcolor}{rgb}{0.000000,0.000000,0.000000}%
\pgfsetstrokecolor{textcolor}%
\pgfsetfillcolor{textcolor}%
\pgftext[x=0.278889in, y=1.028750in, left, base]{\color{textcolor}\rmfamily\fontsize{10.000000}{12.000000}\selectfont \(\displaystyle {0.5}\)}%
\end{pgfscope}%
\begin{pgfscope}%
\pgfsetbuttcap%
\pgfsetroundjoin%
\definecolor{currentfill}{rgb}{0.000000,0.000000,0.000000}%
\pgfsetfillcolor{currentfill}%
\pgfsetlinewidth{0.803000pt}%
\definecolor{currentstroke}{rgb}{0.000000,0.000000,0.000000}%
\pgfsetstrokecolor{currentstroke}%
\pgfsetdash{}{0pt}%
\pgfsys@defobject{currentmarker}{\pgfqpoint{-0.048611in}{0.000000in}}{\pgfqpoint{-0.000000in}{0.000000in}}{%
\pgfpathmoveto{\pgfqpoint{-0.000000in}{0.000000in}}%
\pgfpathlineto{\pgfqpoint{-0.048611in}{0.000000in}}%
\pgfusepath{stroke,fill}%
}%
\begin{pgfscope}%
\pgfsys@transformshift{0.553581in}{1.601944in}%
\pgfsys@useobject{currentmarker}{}%
\end{pgfscope}%
\end{pgfscope}%
\begin{pgfscope}%
\definecolor{textcolor}{rgb}{0.000000,0.000000,0.000000}%
\pgfsetstrokecolor{textcolor}%
\pgfsetfillcolor{textcolor}%
\pgftext[x=0.278889in, y=1.553750in, left, base]{\color{textcolor}\rmfamily\fontsize{10.000000}{12.000000}\selectfont \(\displaystyle {1.0}\)}%
\end{pgfscope}%
\begin{pgfscope}%
\definecolor{textcolor}{rgb}{0.000000,0.000000,0.000000}%
\pgfsetstrokecolor{textcolor}%
\pgfsetfillcolor{textcolor}%
\pgftext[x=0.223333in,y=1.076944in,,bottom,rotate=90.000000]{\color{textcolor}\rmfamily\fontsize{10.000000}{12.000000}\selectfont True positive rate}%
\end{pgfscope}%
\begin{pgfscope}%
\pgfpathrectangle{\pgfqpoint{0.553581in}{0.499444in}}{\pgfqpoint{1.550000in}{1.155000in}}%
\pgfusepath{clip}%
\pgfsetbuttcap%
\pgfsetroundjoin%
\pgfsetlinewidth{1.505625pt}%
\definecolor{currentstroke}{rgb}{0.000000,0.000000,0.000000}%
\pgfsetstrokecolor{currentstroke}%
\pgfsetdash{{5.550000pt}{2.400000pt}}{0.000000pt}%
\pgfpathmoveto{\pgfqpoint{0.624035in}{0.551944in}}%
\pgfpathlineto{\pgfqpoint{2.033126in}{1.601944in}}%
\pgfusepath{stroke}%
\end{pgfscope}%
\begin{pgfscope}%
\pgfpathrectangle{\pgfqpoint{0.553581in}{0.499444in}}{\pgfqpoint{1.550000in}{1.155000in}}%
\pgfusepath{clip}%
\pgfsetrectcap%
\pgfsetroundjoin%
\pgfsetlinewidth{1.505625pt}%
\definecolor{currentstroke}{rgb}{0.000000,0.000000,0.000000}%
\pgfsetstrokecolor{currentstroke}%
\pgfsetdash{}{0pt}%
\pgfpathmoveto{\pgfqpoint{0.624035in}{0.551944in}}%
\pgfpathlineto{\pgfqpoint{0.625130in}{0.562374in}}%
\pgfpathlineto{\pgfqpoint{0.625403in}{0.563430in}}%
\pgfpathlineto{\pgfqpoint{0.626513in}{0.572028in}}%
\pgfpathlineto{\pgfqpoint{0.626701in}{0.573084in}}%
\pgfpathlineto{\pgfqpoint{0.627796in}{0.578547in}}%
\pgfpathlineto{\pgfqpoint{0.627983in}{0.579603in}}%
\pgfpathlineto{\pgfqpoint{0.629093in}{0.584725in}}%
\pgfpathlineto{\pgfqpoint{0.629312in}{0.585780in}}%
\pgfpathlineto{\pgfqpoint{0.630407in}{0.592609in}}%
\pgfpathlineto{\pgfqpoint{0.630672in}{0.593634in}}%
\pgfpathlineto{\pgfqpoint{0.631775in}{0.598973in}}%
\pgfpathlineto{\pgfqpoint{0.631986in}{0.600059in}}%
\pgfpathlineto{\pgfqpoint{0.633088in}{0.605554in}}%
\pgfpathlineto{\pgfqpoint{0.633323in}{0.606640in}}%
\pgfpathlineto{\pgfqpoint{0.634433in}{0.611421in}}%
\pgfpathlineto{\pgfqpoint{0.634691in}{0.612507in}}%
\pgfpathlineto{\pgfqpoint{0.635793in}{0.617040in}}%
\pgfpathlineto{\pgfqpoint{0.636121in}{0.618126in}}%
\pgfpathlineto{\pgfqpoint{0.637231in}{0.622503in}}%
\pgfpathlineto{\pgfqpoint{0.637521in}{0.623527in}}%
\pgfpathlineto{\pgfqpoint{0.638623in}{0.628711in}}%
\pgfpathlineto{\pgfqpoint{0.638983in}{0.629767in}}%
\pgfpathlineto{\pgfqpoint{0.640077in}{0.633771in}}%
\pgfpathlineto{\pgfqpoint{0.640257in}{0.634858in}}%
\pgfpathlineto{\pgfqpoint{0.641351in}{0.639017in}}%
\pgfpathlineto{\pgfqpoint{0.641625in}{0.640042in}}%
\pgfpathlineto{\pgfqpoint{0.642735in}{0.644326in}}%
\pgfpathlineto{\pgfqpoint{0.642884in}{0.645226in}}%
\pgfpathlineto{\pgfqpoint{0.643986in}{0.649851in}}%
\pgfpathlineto{\pgfqpoint{0.644252in}{0.650844in}}%
\pgfpathlineto{\pgfqpoint{0.645330in}{0.655345in}}%
\pgfpathlineto{\pgfqpoint{0.645659in}{0.656339in}}%
\pgfpathlineto{\pgfqpoint{0.646761in}{0.660188in}}%
\pgfpathlineto{\pgfqpoint{0.647066in}{0.661275in}}%
\pgfpathlineto{\pgfqpoint{0.648160in}{0.665651in}}%
\pgfpathlineto{\pgfqpoint{0.648481in}{0.666645in}}%
\pgfpathlineto{\pgfqpoint{0.649591in}{0.671363in}}%
\pgfpathlineto{\pgfqpoint{0.649982in}{0.672419in}}%
\pgfpathlineto{\pgfqpoint{0.651053in}{0.677696in}}%
\pgfpathlineto{\pgfqpoint{0.651381in}{0.678751in}}%
\pgfpathlineto{\pgfqpoint{0.652484in}{0.682787in}}%
\pgfpathlineto{\pgfqpoint{0.652828in}{0.683842in}}%
\pgfpathlineto{\pgfqpoint{0.653930in}{0.688343in}}%
\pgfpathlineto{\pgfqpoint{0.654204in}{0.689399in}}%
\pgfpathlineto{\pgfqpoint{0.655314in}{0.693651in}}%
\pgfpathlineto{\pgfqpoint{0.655619in}{0.694738in}}%
\pgfpathlineto{\pgfqpoint{0.656721in}{0.698091in}}%
\pgfpathlineto{\pgfqpoint{0.657080in}{0.699177in}}%
\pgfpathlineto{\pgfqpoint{0.658191in}{0.703275in}}%
\pgfpathlineto{\pgfqpoint{0.658644in}{0.704361in}}%
\pgfpathlineto{\pgfqpoint{0.659746in}{0.707093in}}%
\pgfpathlineto{\pgfqpoint{0.660176in}{0.708117in}}%
\pgfpathlineto{\pgfqpoint{0.660176in}{0.708179in}}%
\pgfpathlineto{\pgfqpoint{0.661278in}{0.711377in}}%
\pgfpathlineto{\pgfqpoint{0.661599in}{0.712370in}}%
\pgfpathlineto{\pgfqpoint{0.662709in}{0.716343in}}%
\pgfpathlineto{\pgfqpoint{0.662998in}{0.717399in}}%
\pgfpathlineto{\pgfqpoint{0.664077in}{0.720751in}}%
\pgfpathlineto{\pgfqpoint{0.664546in}{0.721838in}}%
\pgfpathlineto{\pgfqpoint{0.665656in}{0.724694in}}%
\pgfpathlineto{\pgfqpoint{0.665977in}{0.725780in}}%
\pgfpathlineto{\pgfqpoint{0.667079in}{0.729226in}}%
\pgfpathlineto{\pgfqpoint{0.667556in}{0.730312in}}%
\pgfpathlineto{\pgfqpoint{0.668651in}{0.733385in}}%
\pgfpathlineto{\pgfqpoint{0.669088in}{0.734410in}}%
\pgfpathlineto{\pgfqpoint{0.669088in}{0.734441in}}%
\pgfpathlineto{\pgfqpoint{0.670198in}{0.737887in}}%
\pgfpathlineto{\pgfqpoint{0.670652in}{0.738973in}}%
\pgfpathlineto{\pgfqpoint{0.671754in}{0.741767in}}%
\pgfpathlineto{\pgfqpoint{0.672270in}{0.742822in}}%
\pgfpathlineto{\pgfqpoint{0.673372in}{0.746578in}}%
\pgfpathlineto{\pgfqpoint{0.673865in}{0.747603in}}%
\pgfpathlineto{\pgfqpoint{0.674975in}{0.751452in}}%
\pgfpathlineto{\pgfqpoint{0.675350in}{0.752538in}}%
\pgfpathlineto{\pgfqpoint{0.676453in}{0.756108in}}%
\pgfpathlineto{\pgfqpoint{0.676734in}{0.757040in}}%
\pgfpathlineto{\pgfqpoint{0.677844in}{0.760237in}}%
\pgfpathlineto{\pgfqpoint{0.678344in}{0.761261in}}%
\pgfpathlineto{\pgfqpoint{0.679415in}{0.764241in}}%
\pgfpathlineto{\pgfqpoint{0.679822in}{0.765328in}}%
\pgfpathlineto{\pgfqpoint{0.680916in}{0.768091in}}%
\pgfpathlineto{\pgfqpoint{0.681417in}{0.769177in}}%
\pgfpathlineto{\pgfqpoint{0.682488in}{0.772250in}}%
\pgfpathlineto{\pgfqpoint{0.683004in}{0.773337in}}%
\pgfpathlineto{\pgfqpoint{0.684114in}{0.776410in}}%
\pgfpathlineto{\pgfqpoint{0.684552in}{0.777496in}}%
\pgfpathlineto{\pgfqpoint{0.685662in}{0.779855in}}%
\pgfpathlineto{\pgfqpoint{0.686185in}{0.780942in}}%
\pgfpathlineto{\pgfqpoint{0.687288in}{0.784015in}}%
\pgfpathlineto{\pgfqpoint{0.687757in}{0.785071in}}%
\pgfpathlineto{\pgfqpoint{0.688867in}{0.787368in}}%
\pgfpathlineto{\pgfqpoint{0.689125in}{0.788423in}}%
\pgfpathlineto{\pgfqpoint{0.690212in}{0.791093in}}%
\pgfpathlineto{\pgfqpoint{0.690728in}{0.792179in}}%
\pgfpathlineto{\pgfqpoint{0.691830in}{0.794942in}}%
\pgfpathlineto{\pgfqpoint{0.692377in}{0.796028in}}%
\pgfpathlineto{\pgfqpoint{0.693448in}{0.799102in}}%
\pgfpathlineto{\pgfqpoint{0.694034in}{0.800188in}}%
\pgfpathlineto{\pgfqpoint{0.695137in}{0.803292in}}%
\pgfpathlineto{\pgfqpoint{0.695668in}{0.804379in}}%
\pgfpathlineto{\pgfqpoint{0.696731in}{0.806459in}}%
\pgfpathlineto{\pgfqpoint{0.697271in}{0.807545in}}%
\pgfpathlineto{\pgfqpoint{0.698365in}{0.810618in}}%
\pgfpathlineto{\pgfqpoint{0.698881in}{0.811705in}}%
\pgfpathlineto{\pgfqpoint{0.699976in}{0.814343in}}%
\pgfpathlineto{\pgfqpoint{0.700531in}{0.815430in}}%
\pgfpathlineto{\pgfqpoint{0.701618in}{0.817323in}}%
\pgfpathlineto{\pgfqpoint{0.702157in}{0.818317in}}%
\pgfpathlineto{\pgfqpoint{0.703251in}{0.820490in}}%
\pgfpathlineto{\pgfqpoint{0.703853in}{0.821545in}}%
\pgfpathlineto{\pgfqpoint{0.704963in}{0.824463in}}%
\pgfpathlineto{\pgfqpoint{0.705362in}{0.825549in}}%
\pgfpathlineto{\pgfqpoint{0.706464in}{0.828033in}}%
\pgfpathlineto{\pgfqpoint{0.707254in}{0.829057in}}%
\pgfpathlineto{\pgfqpoint{0.707254in}{0.829119in}}%
\pgfpathlineto{\pgfqpoint{0.708349in}{0.831758in}}%
\pgfpathlineto{\pgfqpoint{0.708943in}{0.832844in}}%
\pgfpathlineto{\pgfqpoint{0.710045in}{0.835483in}}%
\pgfpathlineto{\pgfqpoint{0.710569in}{0.836569in}}%
\pgfpathlineto{\pgfqpoint{0.711679in}{0.839115in}}%
\pgfpathlineto{\pgfqpoint{0.712218in}{0.840201in}}%
\pgfpathlineto{\pgfqpoint{0.713328in}{0.843306in}}%
\pgfpathlineto{\pgfqpoint{0.713930in}{0.844361in}}%
\pgfpathlineto{\pgfqpoint{0.715025in}{0.847403in}}%
\pgfpathlineto{\pgfqpoint{0.715431in}{0.848490in}}%
\pgfpathlineto{\pgfqpoint{0.716471in}{0.850880in}}%
\pgfpathlineto{\pgfqpoint{0.717034in}{0.851904in}}%
\pgfpathlineto{\pgfqpoint{0.718136in}{0.854108in}}%
\pgfpathlineto{\pgfqpoint{0.718769in}{0.855195in}}%
\pgfpathlineto{\pgfqpoint{0.719880in}{0.857740in}}%
\pgfpathlineto{\pgfqpoint{0.720333in}{0.858827in}}%
\pgfpathlineto{\pgfqpoint{0.721435in}{0.861217in}}%
\pgfpathlineto{\pgfqpoint{0.721990in}{0.862241in}}%
\pgfpathlineto{\pgfqpoint{0.723038in}{0.864135in}}%
\pgfpathlineto{\pgfqpoint{0.723749in}{0.865159in}}%
\pgfpathlineto{\pgfqpoint{0.724859in}{0.867984in}}%
\pgfpathlineto{\pgfqpoint{0.725375in}{0.869071in}}%
\pgfpathlineto{\pgfqpoint{0.726478in}{0.871181in}}%
\pgfpathlineto{\pgfqpoint{0.727080in}{0.872268in}}%
\pgfpathlineto{\pgfqpoint{0.728151in}{0.874844in}}%
\pgfpathlineto{\pgfqpoint{0.728885in}{0.875931in}}%
\pgfpathlineto{\pgfqpoint{0.729910in}{0.877980in}}%
\pgfpathlineto{\pgfqpoint{0.729956in}{0.877980in}}%
\pgfpathlineto{\pgfqpoint{0.730347in}{0.879066in}}%
\pgfpathlineto{\pgfqpoint{0.731364in}{0.881363in}}%
\pgfpathlineto{\pgfqpoint{0.731950in}{0.882450in}}%
\pgfpathlineto{\pgfqpoint{0.733060in}{0.884126in}}%
\pgfpathlineto{\pgfqpoint{0.733412in}{0.885150in}}%
\pgfpathlineto{\pgfqpoint{0.734499in}{0.887292in}}%
\pgfpathlineto{\pgfqpoint{0.735210in}{0.888379in}}%
\pgfpathlineto{\pgfqpoint{0.736312in}{0.890490in}}%
\pgfpathlineto{\pgfqpoint{0.736844in}{0.891514in}}%
\pgfpathlineto{\pgfqpoint{0.737954in}{0.893408in}}%
\pgfpathlineto{\pgfqpoint{0.738556in}{0.894494in}}%
\pgfpathlineto{\pgfqpoint{0.739643in}{0.896543in}}%
\pgfpathlineto{\pgfqpoint{0.740479in}{0.897629in}}%
\pgfpathlineto{\pgfqpoint{0.741589in}{0.900051in}}%
\pgfpathlineto{\pgfqpoint{0.742121in}{0.901137in}}%
\pgfpathlineto{\pgfqpoint{0.743215in}{0.903279in}}%
\pgfpathlineto{\pgfqpoint{0.743809in}{0.904272in}}%
\pgfpathlineto{\pgfqpoint{0.744919in}{0.906632in}}%
\pgfpathlineto{\pgfqpoint{0.745553in}{0.907718in}}%
\pgfpathlineto{\pgfqpoint{0.746647in}{0.909581in}}%
\pgfpathlineto{\pgfqpoint{0.747398in}{0.910667in}}%
\pgfpathlineto{\pgfqpoint{0.748476in}{0.912933in}}%
\pgfpathlineto{\pgfqpoint{0.749180in}{0.913989in}}%
\pgfpathlineto{\pgfqpoint{0.750282in}{0.916348in}}%
\pgfpathlineto{\pgfqpoint{0.750900in}{0.917434in}}%
\pgfpathlineto{\pgfqpoint{0.752010in}{0.919607in}}%
\pgfpathlineto{\pgfqpoint{0.752534in}{0.920694in}}%
\pgfpathlineto{\pgfqpoint{0.753628in}{0.922960in}}%
\pgfpathlineto{\pgfqpoint{0.754574in}{0.924046in}}%
\pgfpathlineto{\pgfqpoint{0.755645in}{0.926436in}}%
\pgfpathlineto{\pgfqpoint{0.756208in}{0.927523in}}%
\pgfpathlineto{\pgfqpoint{0.757279in}{0.929230in}}%
\pgfpathlineto{\pgfqpoint{0.757303in}{0.929230in}}%
\pgfpathlineto{\pgfqpoint{0.757779in}{0.930255in}}%
\pgfpathlineto{\pgfqpoint{0.758882in}{0.931651in}}%
\pgfpathlineto{\pgfqpoint{0.759476in}{0.932645in}}%
\pgfpathlineto{\pgfqpoint{0.760578in}{0.934197in}}%
\pgfpathlineto{\pgfqpoint{0.761329in}{0.935283in}}%
\pgfpathlineto{\pgfqpoint{0.762439in}{0.937146in}}%
\pgfpathlineto{\pgfqpoint{0.763009in}{0.938232in}}%
\pgfpathlineto{\pgfqpoint{0.764096in}{0.939878in}}%
\pgfpathlineto{\pgfqpoint{0.764854in}{0.940964in}}%
\pgfpathlineto{\pgfqpoint{0.765933in}{0.943044in}}%
\pgfpathlineto{\pgfqpoint{0.766512in}{0.944099in}}%
\pgfpathlineto{\pgfqpoint{0.767606in}{0.945589in}}%
\pgfpathlineto{\pgfqpoint{0.768396in}{0.946645in}}%
\pgfpathlineto{\pgfqpoint{0.769490in}{0.948663in}}%
\pgfpathlineto{\pgfqpoint{0.770092in}{0.949749in}}%
\pgfpathlineto{\pgfqpoint{0.771202in}{0.951891in}}%
\pgfpathlineto{\pgfqpoint{0.772000in}{0.952977in}}%
\pgfpathlineto{\pgfqpoint{0.773086in}{0.954654in}}%
\pgfpathlineto{\pgfqpoint{0.773939in}{0.955740in}}%
\pgfpathlineto{\pgfqpoint{0.775049in}{0.957634in}}%
\pgfpathlineto{\pgfqpoint{0.775783in}{0.958720in}}%
\pgfpathlineto{\pgfqpoint{0.776894in}{0.960148in}}%
\pgfpathlineto{\pgfqpoint{0.777636in}{0.961235in}}%
\pgfpathlineto{\pgfqpoint{0.778731in}{0.963377in}}%
\pgfpathlineto{\pgfqpoint{0.779450in}{0.964432in}}%
\pgfpathlineto{\pgfqpoint{0.780552in}{0.966201in}}%
\pgfpathlineto{\pgfqpoint{0.781217in}{0.967288in}}%
\pgfpathlineto{\pgfqpoint{0.782327in}{0.969740in}}%
\pgfpathlineto{\pgfqpoint{0.783202in}{0.970765in}}%
\pgfpathlineto{\pgfqpoint{0.784266in}{0.972813in}}%
\pgfpathlineto{\pgfqpoint{0.784930in}{0.973838in}}%
\pgfpathlineto{\pgfqpoint{0.786040in}{0.975173in}}%
\pgfpathlineto{\pgfqpoint{0.786759in}{0.976259in}}%
\pgfpathlineto{\pgfqpoint{0.787870in}{0.978153in}}%
\pgfpathlineto{\pgfqpoint{0.788612in}{0.979239in}}%
\pgfpathlineto{\pgfqpoint{0.789629in}{0.981040in}}%
\pgfpathlineto{\pgfqpoint{0.790301in}{0.982126in}}%
\pgfpathlineto{\pgfqpoint{0.791411in}{0.983989in}}%
\pgfpathlineto{\pgfqpoint{0.791896in}{0.985075in}}%
\pgfpathlineto{\pgfqpoint{0.792998in}{0.986472in}}%
\pgfpathlineto{\pgfqpoint{0.793670in}{0.987558in}}%
\pgfpathlineto{\pgfqpoint{0.794694in}{0.989297in}}%
\pgfpathlineto{\pgfqpoint{0.794741in}{0.989297in}}%
\pgfpathlineto{\pgfqpoint{0.795726in}{0.990383in}}%
\pgfpathlineto{\pgfqpoint{0.796829in}{0.992525in}}%
\pgfpathlineto{\pgfqpoint{0.797767in}{0.993581in}}%
\pgfpathlineto{\pgfqpoint{0.798861in}{0.995102in}}%
\pgfpathlineto{\pgfqpoint{0.799580in}{0.996188in}}%
\pgfpathlineto{\pgfqpoint{0.800636in}{0.997554in}}%
\pgfpathlineto{\pgfqpoint{0.801371in}{0.998640in}}%
\pgfpathlineto{\pgfqpoint{0.802481in}{1.000379in}}%
\pgfpathlineto{\pgfqpoint{0.803216in}{1.001465in}}%
\pgfpathlineto{\pgfqpoint{0.804326in}{1.002552in}}%
\pgfpathlineto{\pgfqpoint{0.805084in}{1.003607in}}%
\pgfpathlineto{\pgfqpoint{0.806186in}{1.005035in}}%
\pgfpathlineto{\pgfqpoint{0.807015in}{1.006122in}}%
\pgfpathlineto{\pgfqpoint{0.808109in}{1.007767in}}%
\pgfpathlineto{\pgfqpoint{0.808938in}{1.008853in}}%
\pgfpathlineto{\pgfqpoint{0.810040in}{1.010561in}}%
\pgfpathlineto{\pgfqpoint{0.811072in}{1.011647in}}%
\pgfpathlineto{\pgfqpoint{0.812151in}{1.013541in}}%
\pgfpathlineto{\pgfqpoint{0.813089in}{1.014627in}}%
\pgfpathlineto{\pgfqpoint{0.814176in}{1.015869in}}%
\pgfpathlineto{\pgfqpoint{0.815145in}{1.016955in}}%
\pgfpathlineto{\pgfqpoint{0.816248in}{1.018569in}}%
\pgfpathlineto{\pgfqpoint{0.816936in}{1.019625in}}%
\pgfpathlineto{\pgfqpoint{0.818030in}{1.021177in}}%
\pgfpathlineto{\pgfqpoint{0.818781in}{1.022263in}}%
\pgfpathlineto{\pgfqpoint{0.819891in}{1.023722in}}%
\pgfpathlineto{\pgfqpoint{0.820868in}{1.024809in}}%
\pgfpathlineto{\pgfqpoint{0.821962in}{1.026020in}}%
\pgfpathlineto{\pgfqpoint{0.822736in}{1.027075in}}%
\pgfpathlineto{\pgfqpoint{0.823823in}{1.028441in}}%
\pgfpathlineto{\pgfqpoint{0.824902in}{1.029496in}}%
\pgfpathlineto{\pgfqpoint{0.825957in}{1.031110in}}%
\pgfpathlineto{\pgfqpoint{0.826692in}{1.032197in}}%
\pgfpathlineto{\pgfqpoint{0.827763in}{1.033252in}}%
\pgfpathlineto{\pgfqpoint{0.828599in}{1.034308in}}%
\pgfpathlineto{\pgfqpoint{0.829710in}{1.035643in}}%
\pgfpathlineto{\pgfqpoint{0.830601in}{1.036698in}}%
\pgfpathlineto{\pgfqpoint{0.831695in}{1.038561in}}%
\pgfpathlineto{\pgfqpoint{0.832837in}{1.039647in}}%
\pgfpathlineto{\pgfqpoint{0.833947in}{1.040982in}}%
\pgfpathlineto{\pgfqpoint{0.834721in}{1.042037in}}%
\pgfpathlineto{\pgfqpoint{0.835792in}{1.043589in}}%
\pgfpathlineto{\pgfqpoint{0.836605in}{1.044645in}}%
\pgfpathlineto{\pgfqpoint{0.837676in}{1.045918in}}%
\pgfpathlineto{\pgfqpoint{0.838786in}{1.047004in}}%
\pgfpathlineto{\pgfqpoint{0.839880in}{1.048370in}}%
\pgfpathlineto{\pgfqpoint{0.840623in}{1.049456in}}%
\pgfpathlineto{\pgfqpoint{0.841733in}{1.050915in}}%
\pgfpathlineto{\pgfqpoint{0.842742in}{1.052002in}}%
\pgfpathlineto{\pgfqpoint{0.843836in}{1.053212in}}%
\pgfpathlineto{\pgfqpoint{0.844837in}{1.054268in}}%
\pgfpathlineto{\pgfqpoint{0.845900in}{1.055199in}}%
\pgfpathlineto{\pgfqpoint{0.847104in}{1.056286in}}%
\pgfpathlineto{\pgfqpoint{0.848206in}{1.057372in}}%
\pgfpathlineto{\pgfqpoint{0.849504in}{1.058459in}}%
\pgfpathlineto{\pgfqpoint{0.850614in}{1.060228in}}%
\pgfpathlineto{\pgfqpoint{0.851787in}{1.061190in}}%
\pgfpathlineto{\pgfqpoint{0.851787in}{1.061283in}}%
\pgfpathlineto{\pgfqpoint{0.852881in}{1.062680in}}%
\pgfpathlineto{\pgfqpoint{0.854007in}{1.063767in}}%
\pgfpathlineto{\pgfqpoint{0.855101in}{1.065195in}}%
\pgfpathlineto{\pgfqpoint{0.855805in}{1.066281in}}%
\pgfpathlineto{\pgfqpoint{0.856907in}{1.067523in}}%
\pgfpathlineto{\pgfqpoint{0.856915in}{1.067523in}}%
\pgfpathlineto{\pgfqpoint{0.857642in}{1.068609in}}%
\pgfpathlineto{\pgfqpoint{0.858713in}{1.069851in}}%
\pgfpathlineto{\pgfqpoint{0.859745in}{1.070938in}}%
\pgfpathlineto{\pgfqpoint{0.860855in}{1.072179in}}%
\pgfpathlineto{\pgfqpoint{0.861410in}{1.073204in}}%
\pgfpathlineto{\pgfqpoint{0.862505in}{1.074600in}}%
\pgfpathlineto{\pgfqpoint{0.863662in}{1.075687in}}%
\pgfpathlineto{\pgfqpoint{0.864756in}{1.077115in}}%
\pgfpathlineto{\pgfqpoint{0.865522in}{1.078201in}}%
\pgfpathlineto{\pgfqpoint{0.866624in}{1.079536in}}%
\pgfpathlineto{\pgfqpoint{0.867907in}{1.080592in}}%
\pgfpathlineto{\pgfqpoint{0.869017in}{1.082268in}}%
\pgfpathlineto{\pgfqpoint{0.870002in}{1.083354in}}%
\pgfpathlineto{\pgfqpoint{0.871096in}{1.084689in}}%
\pgfpathlineto{\pgfqpoint{0.872425in}{1.085776in}}%
\pgfpathlineto{\pgfqpoint{0.873535in}{1.086955in}}%
\pgfpathlineto{\pgfqpoint{0.874098in}{1.088042in}}%
\pgfpathlineto{\pgfqpoint{0.875193in}{1.089345in}}%
\pgfpathlineto{\pgfqpoint{0.876193in}{1.090432in}}%
\pgfpathlineto{\pgfqpoint{0.877288in}{1.091953in}}%
\pgfpathlineto{\pgfqpoint{0.878085in}{1.093040in}}%
\pgfpathlineto{\pgfqpoint{0.879172in}{1.094654in}}%
\pgfpathlineto{\pgfqpoint{0.879946in}{1.095740in}}%
\pgfpathlineto{\pgfqpoint{0.880993in}{1.097075in}}%
\pgfpathlineto{\pgfqpoint{0.882307in}{1.098130in}}%
\pgfpathlineto{\pgfqpoint{0.883417in}{1.099403in}}%
\pgfpathlineto{\pgfqpoint{0.884300in}{1.100490in}}%
\pgfpathlineto{\pgfqpoint{0.885356in}{1.101731in}}%
\pgfpathlineto{\pgfqpoint{0.886630in}{1.102818in}}%
\pgfpathlineto{\pgfqpoint{0.887740in}{1.103904in}}%
\pgfpathlineto{\pgfqpoint{0.888748in}{1.104991in}}%
\pgfpathlineto{\pgfqpoint{0.889812in}{1.106326in}}%
\pgfpathlineto{\pgfqpoint{0.889843in}{1.106326in}}%
\pgfpathlineto{\pgfqpoint{0.891086in}{1.107412in}}%
\pgfpathlineto{\pgfqpoint{0.892165in}{1.108871in}}%
\pgfpathlineto{\pgfqpoint{0.893158in}{1.109957in}}%
\pgfpathlineto{\pgfqpoint{0.894260in}{1.111447in}}%
\pgfpathlineto{\pgfqpoint{0.895253in}{1.112503in}}%
\pgfpathlineto{\pgfqpoint{0.896324in}{1.113651in}}%
\pgfpathlineto{\pgfqpoint{0.897207in}{1.114707in}}%
\pgfpathlineto{\pgfqpoint{0.898302in}{1.116321in}}%
\pgfpathlineto{\pgfqpoint{0.899560in}{1.117377in}}%
\pgfpathlineto{\pgfqpoint{0.900670in}{1.118370in}}%
\pgfpathlineto{\pgfqpoint{0.901702in}{1.119456in}}%
\pgfpathlineto{\pgfqpoint{0.902797in}{1.120481in}}%
\pgfpathlineto{\pgfqpoint{0.903774in}{1.121567in}}%
\pgfpathlineto{\pgfqpoint{0.904829in}{1.122561in}}%
\pgfpathlineto{\pgfqpoint{0.905744in}{1.123616in}}%
\pgfpathlineto{\pgfqpoint{0.906776in}{1.124889in}}%
\pgfpathlineto{\pgfqpoint{0.908089in}{1.125975in}}%
\pgfpathlineto{\pgfqpoint{0.909199in}{1.127341in}}%
\pgfpathlineto{\pgfqpoint{0.910349in}{1.128428in}}%
\pgfpathlineto{\pgfqpoint{0.911451in}{1.129700in}}%
\pgfpathlineto{\pgfqpoint{0.912467in}{1.130787in}}%
\pgfpathlineto{\pgfqpoint{0.913507in}{1.131687in}}%
\pgfpathlineto{\pgfqpoint{0.913538in}{1.131687in}}%
\pgfpathlineto{\pgfqpoint{0.914468in}{1.132773in}}%
\pgfpathlineto{\pgfqpoint{0.915579in}{1.134170in}}%
\pgfpathlineto{\pgfqpoint{0.916861in}{1.135257in}}%
\pgfpathlineto{\pgfqpoint{0.917971in}{1.136623in}}%
\pgfpathlineto{\pgfqpoint{0.918995in}{1.137709in}}%
\pgfpathlineto{\pgfqpoint{0.920058in}{1.139168in}}%
\pgfpathlineto{\pgfqpoint{0.921622in}{1.140255in}}%
\pgfpathlineto{\pgfqpoint{0.922716in}{1.141434in}}%
\pgfpathlineto{\pgfqpoint{0.924108in}{1.142521in}}%
\pgfpathlineto{\pgfqpoint{0.925179in}{1.143855in}}%
\pgfpathlineto{\pgfqpoint{0.926218in}{1.144911in}}%
\pgfpathlineto{\pgfqpoint{0.927305in}{1.146122in}}%
\pgfpathlineto{\pgfqpoint{0.928548in}{1.147208in}}%
\pgfpathlineto{\pgfqpoint{0.929650in}{1.148294in}}%
\pgfpathlineto{\pgfqpoint{0.930925in}{1.149319in}}%
\pgfpathlineto{\pgfqpoint{0.932003in}{1.150467in}}%
\pgfpathlineto{\pgfqpoint{0.933880in}{1.151554in}}%
\pgfpathlineto{\pgfqpoint{0.934959in}{1.152796in}}%
\pgfpathlineto{\pgfqpoint{0.935842in}{1.153882in}}%
\pgfpathlineto{\pgfqpoint{0.936803in}{1.155217in}}%
\pgfpathlineto{\pgfqpoint{0.938273in}{1.156303in}}%
\pgfpathlineto{\pgfqpoint{0.939368in}{1.157390in}}%
\pgfpathlineto{\pgfqpoint{0.940572in}{1.158414in}}%
\pgfpathlineto{\pgfqpoint{0.941682in}{1.159780in}}%
\pgfpathlineto{\pgfqpoint{0.942893in}{1.160867in}}%
\pgfpathlineto{\pgfqpoint{0.943980in}{1.162046in}}%
\pgfpathlineto{\pgfqpoint{0.945387in}{1.163133in}}%
\pgfpathlineto{\pgfqpoint{0.946497in}{1.164281in}}%
\pgfpathlineto{\pgfqpoint{0.948170in}{1.165368in}}%
\pgfpathlineto{\pgfqpoint{0.949249in}{1.165957in}}%
\pgfpathlineto{\pgfqpoint{0.949280in}{1.165957in}}%
\pgfpathlineto{\pgfqpoint{0.950015in}{1.167044in}}%
\pgfpathlineto{\pgfqpoint{0.951008in}{1.167882in}}%
\pgfpathlineto{\pgfqpoint{0.952368in}{1.168969in}}%
\pgfpathlineto{\pgfqpoint{0.953416in}{1.170024in}}%
\pgfpathlineto{\pgfqpoint{0.954651in}{1.171110in}}%
\pgfpathlineto{\pgfqpoint{0.955753in}{1.172725in}}%
\pgfpathlineto{\pgfqpoint{0.957247in}{1.173811in}}%
\pgfpathlineto{\pgfqpoint{0.958310in}{1.175146in}}%
\pgfpathlineto{\pgfqpoint{0.959811in}{1.176201in}}%
\pgfpathlineto{\pgfqpoint{0.960897in}{1.177567in}}%
\pgfpathlineto{\pgfqpoint{0.961976in}{1.178654in}}%
\pgfpathlineto{\pgfqpoint{0.963063in}{1.179740in}}%
\pgfpathlineto{\pgfqpoint{0.964408in}{1.180827in}}%
\pgfpathlineto{\pgfqpoint{0.965486in}{1.181913in}}%
\pgfpathlineto{\pgfqpoint{0.966385in}{1.183000in}}%
\pgfpathlineto{\pgfqpoint{0.967464in}{1.184241in}}%
\pgfpathlineto{\pgfqpoint{0.968879in}{1.185297in}}%
\pgfpathlineto{\pgfqpoint{0.969989in}{1.186476in}}%
\pgfpathlineto{\pgfqpoint{0.971600in}{1.187563in}}%
\pgfpathlineto{\pgfqpoint{0.972632in}{1.188463in}}%
\pgfpathlineto{\pgfqpoint{0.972655in}{1.188463in}}%
\pgfpathlineto{\pgfqpoint{0.973812in}{1.189549in}}%
\pgfpathlineto{\pgfqpoint{0.974914in}{1.190853in}}%
\pgfpathlineto{\pgfqpoint{0.975884in}{1.191909in}}%
\pgfpathlineto{\pgfqpoint{0.976971in}{1.192778in}}%
\pgfpathlineto{\pgfqpoint{0.978386in}{1.193833in}}%
\pgfpathlineto{\pgfqpoint{0.979433in}{1.194827in}}%
\pgfpathlineto{\pgfqpoint{0.980457in}{1.195913in}}%
\pgfpathlineto{\pgfqpoint{0.981442in}{1.196813in}}%
\pgfpathlineto{\pgfqpoint{0.982888in}{1.197900in}}%
\pgfpathlineto{\pgfqpoint{0.983952in}{1.198893in}}%
\pgfpathlineto{\pgfqpoint{0.985750in}{1.199949in}}%
\pgfpathlineto{\pgfqpoint{0.986836in}{1.201190in}}%
\pgfpathlineto{\pgfqpoint{0.988017in}{1.202277in}}%
\pgfpathlineto{\pgfqpoint{0.989127in}{1.203612in}}%
\pgfpathlineto{\pgfqpoint{0.990268in}{1.204636in}}%
\pgfpathlineto{\pgfqpoint{0.991371in}{1.205785in}}%
\pgfpathlineto{\pgfqpoint{0.992614in}{1.206871in}}%
\pgfpathlineto{\pgfqpoint{0.993661in}{1.207802in}}%
\pgfpathlineto{\pgfqpoint{0.994482in}{1.208889in}}%
\pgfpathlineto{\pgfqpoint{0.995530in}{1.209727in}}%
\pgfpathlineto{\pgfqpoint{0.996851in}{1.210813in}}%
\pgfpathlineto{\pgfqpoint{0.997961in}{1.211838in}}%
\pgfpathlineto{\pgfqpoint{0.999079in}{1.212924in}}%
\pgfpathlineto{\pgfqpoint{1.000189in}{1.214445in}}%
\pgfpathlineto{\pgfqpoint{1.001463in}{1.215501in}}%
\pgfpathlineto{\pgfqpoint{1.002573in}{1.216494in}}%
\pgfpathlineto{\pgfqpoint{1.004082in}{1.217581in}}%
\pgfpathlineto{\pgfqpoint{1.005153in}{1.218481in}}%
\pgfpathlineto{\pgfqpoint{1.006474in}{1.219567in}}%
\pgfpathlineto{\pgfqpoint{1.007545in}{1.220902in}}%
\pgfpathlineto{\pgfqpoint{1.008890in}{1.221989in}}%
\pgfpathlineto{\pgfqpoint{1.009953in}{1.223044in}}%
\pgfpathlineto{\pgfqpoint{1.009992in}{1.223044in}}%
\pgfpathlineto{\pgfqpoint{1.011095in}{1.224130in}}%
\pgfpathlineto{\pgfqpoint{1.012205in}{1.224782in}}%
\pgfpathlineto{\pgfqpoint{1.013463in}{1.225869in}}%
\pgfpathlineto{\pgfqpoint{1.014472in}{1.226893in}}%
\pgfpathlineto{\pgfqpoint{1.016137in}{1.227980in}}%
\pgfpathlineto{\pgfqpoint{1.017177in}{1.228911in}}%
\pgfpathlineto{\pgfqpoint{1.018678in}{1.229997in}}%
\pgfpathlineto{\pgfqpoint{1.019756in}{1.230742in}}%
\pgfpathlineto{\pgfqpoint{1.020953in}{1.231829in}}%
\pgfpathlineto{\pgfqpoint{1.022039in}{1.232977in}}%
\pgfpathlineto{\pgfqpoint{1.023173in}{1.234064in}}%
\pgfpathlineto{\pgfqpoint{1.024283in}{1.235150in}}%
\pgfpathlineto{\pgfqpoint{1.025518in}{1.236237in}}%
\pgfpathlineto{\pgfqpoint{1.026613in}{1.237075in}}%
\pgfpathlineto{\pgfqpoint{1.028043in}{1.238161in}}%
\pgfpathlineto{\pgfqpoint{1.029106in}{1.239403in}}%
\pgfpathlineto{\pgfqpoint{1.030490in}{1.240490in}}%
\pgfpathlineto{\pgfqpoint{1.031428in}{1.241204in}}%
\pgfpathlineto{\pgfqpoint{1.031452in}{1.241204in}}%
\pgfpathlineto{\pgfqpoint{1.032718in}{1.242290in}}%
\pgfpathlineto{\pgfqpoint{1.033789in}{1.243159in}}%
\pgfpathlineto{\pgfqpoint{1.035384in}{1.244246in}}%
\pgfpathlineto{\pgfqpoint{1.036494in}{1.245177in}}%
\pgfpathlineto{\pgfqpoint{1.037682in}{1.246263in}}%
\pgfpathlineto{\pgfqpoint{1.038769in}{1.247474in}}%
\pgfpathlineto{\pgfqpoint{1.039973in}{1.248561in}}%
\pgfpathlineto{\pgfqpoint{1.041044in}{1.249244in}}%
\pgfpathlineto{\pgfqpoint{1.042482in}{1.250330in}}%
\pgfpathlineto{\pgfqpoint{1.043592in}{1.251354in}}%
\pgfpathlineto{\pgfqpoint{1.045133in}{1.252348in}}%
\pgfpathlineto{\pgfqpoint{1.045133in}{1.252410in}}%
\pgfpathlineto{\pgfqpoint{1.046227in}{1.253279in}}%
\pgfpathlineto{\pgfqpoint{1.048009in}{1.254365in}}%
\pgfpathlineto{\pgfqpoint{1.049104in}{1.255328in}}%
\pgfpathlineto{\pgfqpoint{1.050128in}{1.256414in}}%
\pgfpathlineto{\pgfqpoint{1.051222in}{1.257470in}}%
\pgfpathlineto{\pgfqpoint{1.052903in}{1.258556in}}%
\pgfpathlineto{\pgfqpoint{1.053990in}{1.259643in}}%
\pgfpathlineto{\pgfqpoint{1.055295in}{1.260729in}}%
\pgfpathlineto{\pgfqpoint{1.056398in}{1.261660in}}%
\pgfpathlineto{\pgfqpoint{1.057946in}{1.262747in}}%
\pgfpathlineto{\pgfqpoint{1.058970in}{1.263740in}}%
\pgfpathlineto{\pgfqpoint{1.060869in}{1.264827in}}%
\pgfpathlineto{\pgfqpoint{1.061901in}{1.265416in}}%
\pgfpathlineto{\pgfqpoint{1.063395in}{1.266503in}}%
\pgfpathlineto{\pgfqpoint{1.064473in}{1.267403in}}%
\pgfpathlineto{\pgfqpoint{1.065818in}{1.268459in}}%
\pgfpathlineto{\pgfqpoint{1.066811in}{1.269110in}}%
\pgfpathlineto{\pgfqpoint{1.068077in}{1.270197in}}%
\pgfpathlineto{\pgfqpoint{1.069117in}{1.271035in}}%
\pgfpathlineto{\pgfqpoint{1.069187in}{1.271035in}}%
\pgfpathlineto{\pgfqpoint{1.070595in}{1.272122in}}%
\pgfpathlineto{\pgfqpoint{1.071681in}{1.273084in}}%
\pgfpathlineto{\pgfqpoint{1.073456in}{1.274170in}}%
\pgfpathlineto{\pgfqpoint{1.074527in}{1.274915in}}%
\pgfpathlineto{\pgfqpoint{1.075887in}{1.275971in}}%
\pgfpathlineto{\pgfqpoint{1.076794in}{1.276467in}}%
\pgfpathlineto{\pgfqpoint{1.078623in}{1.277554in}}%
\pgfpathlineto{\pgfqpoint{1.079710in}{1.278423in}}%
\pgfpathlineto{\pgfqpoint{1.081625in}{1.279510in}}%
\pgfpathlineto{\pgfqpoint{1.082665in}{1.280534in}}%
\pgfpathlineto{\pgfqpoint{1.084690in}{1.281620in}}%
\pgfpathlineto{\pgfqpoint{1.085745in}{1.282272in}}%
\pgfpathlineto{\pgfqpoint{1.087285in}{1.283359in}}%
\pgfpathlineto{\pgfqpoint{1.088270in}{1.284011in}}%
\pgfpathlineto{\pgfqpoint{1.089849in}{1.285066in}}%
\pgfpathlineto{\pgfqpoint{1.090897in}{1.286059in}}%
\pgfpathlineto{\pgfqpoint{1.092101in}{1.287146in}}%
\pgfpathlineto{\pgfqpoint{1.093195in}{1.288077in}}%
\pgfpathlineto{\pgfqpoint{1.094814in}{1.289164in}}%
\pgfpathlineto{\pgfqpoint{1.095916in}{1.290188in}}%
\pgfpathlineto{\pgfqpoint{1.097362in}{1.291275in}}%
\pgfpathlineto{\pgfqpoint{1.098371in}{1.291895in}}%
\pgfpathlineto{\pgfqpoint{1.099809in}{1.292951in}}%
\pgfpathlineto{\pgfqpoint{1.100919in}{1.293727in}}%
\pgfpathlineto{\pgfqpoint{1.102154in}{1.294813in}}%
\pgfpathlineto{\pgfqpoint{1.103155in}{1.295527in}}%
\pgfpathlineto{\pgfqpoint{1.104977in}{1.296614in}}%
\pgfpathlineto{\pgfqpoint{1.106055in}{1.297390in}}%
\pgfpathlineto{\pgfqpoint{1.107291in}{1.298476in}}%
\pgfpathlineto{\pgfqpoint{1.108401in}{1.299252in}}%
\pgfpathlineto{\pgfqpoint{1.110050in}{1.300339in}}%
\pgfpathlineto{\pgfqpoint{1.111090in}{1.301146in}}%
\pgfpathlineto{\pgfqpoint{1.112661in}{1.302201in}}%
\pgfpathlineto{\pgfqpoint{1.113771in}{1.303288in}}%
\pgfpathlineto{\pgfqpoint{1.115038in}{1.304374in}}%
\pgfpathlineto{\pgfqpoint{1.116148in}{1.304995in}}%
\pgfpathlineto{\pgfqpoint{1.117876in}{1.306082in}}%
\pgfpathlineto{\pgfqpoint{1.118947in}{1.306547in}}%
\pgfpathlineto{\pgfqpoint{1.120682in}{1.307634in}}%
\pgfpathlineto{\pgfqpoint{1.121769in}{1.308379in}}%
\pgfpathlineto{\pgfqpoint{1.123536in}{1.309465in}}%
\pgfpathlineto{\pgfqpoint{1.124591in}{1.310210in}}%
\pgfpathlineto{\pgfqpoint{1.125998in}{1.311297in}}%
\pgfpathlineto{\pgfqpoint{1.127108in}{1.312352in}}%
\pgfpathlineto{\pgfqpoint{1.128547in}{1.313439in}}%
\pgfpathlineto{\pgfqpoint{1.129602in}{1.314028in}}%
\pgfpathlineto{\pgfqpoint{1.131197in}{1.315084in}}%
\pgfpathlineto{\pgfqpoint{1.131197in}{1.315115in}}%
\pgfpathlineto{\pgfqpoint{1.132291in}{1.316046in}}%
\pgfpathlineto{\pgfqpoint{1.134074in}{1.317133in}}%
\pgfpathlineto{\pgfqpoint{1.135161in}{1.317567in}}%
\pgfpathlineto{\pgfqpoint{1.137005in}{1.318654in}}%
\pgfpathlineto{\pgfqpoint{1.137990in}{1.319461in}}%
\pgfpathlineto{\pgfqpoint{1.140258in}{1.320547in}}%
\pgfpathlineto{\pgfqpoint{1.141250in}{1.321106in}}%
\pgfpathlineto{\pgfqpoint{1.143127in}{1.322193in}}%
\pgfpathlineto{\pgfqpoint{1.144221in}{1.322938in}}%
\pgfpathlineto{\pgfqpoint{1.146207in}{1.324024in}}%
\pgfpathlineto{\pgfqpoint{1.147254in}{1.324986in}}%
\pgfpathlineto{\pgfqpoint{1.149076in}{1.326073in}}%
\pgfpathlineto{\pgfqpoint{1.150178in}{1.326694in}}%
\pgfpathlineto{\pgfqpoint{1.151593in}{1.327780in}}%
\pgfpathlineto{\pgfqpoint{1.152594in}{1.328711in}}%
\pgfpathlineto{\pgfqpoint{1.154650in}{1.329798in}}%
\pgfpathlineto{\pgfqpoint{1.155760in}{1.330481in}}%
\pgfpathlineto{\pgfqpoint{1.157542in}{1.331567in}}%
\pgfpathlineto{\pgfqpoint{1.158621in}{1.332281in}}%
\pgfpathlineto{\pgfqpoint{1.160607in}{1.333368in}}%
\pgfpathlineto{\pgfqpoint{1.161670in}{1.334051in}}%
\pgfpathlineto{\pgfqpoint{1.163140in}{1.335137in}}%
\pgfpathlineto{\pgfqpoint{1.164234in}{1.335882in}}%
\pgfpathlineto{\pgfqpoint{1.165524in}{1.336969in}}%
\pgfpathlineto{\pgfqpoint{1.166634in}{1.337962in}}%
\pgfpathlineto{\pgfqpoint{1.168417in}{1.339017in}}%
\pgfpathlineto{\pgfqpoint{1.169511in}{1.339824in}}%
\pgfpathlineto{\pgfqpoint{1.172193in}{1.340911in}}%
\pgfpathlineto{\pgfqpoint{1.173154in}{1.341718in}}%
\pgfpathlineto{\pgfqpoint{1.175070in}{1.342773in}}%
\pgfpathlineto{\pgfqpoint{1.176062in}{1.343705in}}%
\pgfpathlineto{\pgfqpoint{1.177665in}{1.344791in}}%
\pgfpathlineto{\pgfqpoint{1.178736in}{1.345474in}}%
\pgfpathlineto{\pgfqpoint{1.180354in}{1.346530in}}%
\pgfpathlineto{\pgfqpoint{1.181308in}{1.347492in}}%
\pgfpathlineto{\pgfqpoint{1.182801in}{1.348578in}}%
\pgfpathlineto{\pgfqpoint{1.183708in}{1.349044in}}%
\pgfpathlineto{\pgfqpoint{1.185365in}{1.350130in}}%
\pgfpathlineto{\pgfqpoint{1.186421in}{1.351000in}}%
\pgfpathlineto{\pgfqpoint{1.187851in}{1.352086in}}%
\pgfpathlineto{\pgfqpoint{1.188860in}{1.352893in}}%
\pgfpathlineto{\pgfqpoint{1.188899in}{1.352893in}}%
\pgfpathlineto{\pgfqpoint{1.190416in}{1.353980in}}%
\pgfpathlineto{\pgfqpoint{1.191315in}{1.354507in}}%
\pgfpathlineto{\pgfqpoint{1.191494in}{1.354507in}}%
\pgfpathlineto{\pgfqpoint{1.192917in}{1.355594in}}%
\pgfpathlineto{\pgfqpoint{1.194020in}{1.356339in}}%
\pgfpathlineto{\pgfqpoint{1.195646in}{1.357425in}}%
\pgfpathlineto{\pgfqpoint{1.196724in}{1.358046in}}%
\pgfpathlineto{\pgfqpoint{1.198710in}{1.359133in}}%
\pgfpathlineto{\pgfqpoint{1.199765in}{1.359847in}}%
\pgfpathlineto{\pgfqpoint{1.201822in}{1.360933in}}%
\pgfpathlineto{\pgfqpoint{1.202861in}{1.361647in}}%
\pgfpathlineto{\pgfqpoint{1.204222in}{1.362734in}}%
\pgfpathlineto{\pgfqpoint{1.205277in}{1.363354in}}%
\pgfpathlineto{\pgfqpoint{1.206934in}{1.364441in}}%
\pgfpathlineto{\pgfqpoint{1.208005in}{1.365000in}}%
\pgfpathlineto{\pgfqpoint{1.209960in}{1.366086in}}%
\pgfpathlineto{\pgfqpoint{1.211039in}{1.366831in}}%
\pgfpathlineto{\pgfqpoint{1.213423in}{1.367918in}}%
\pgfpathlineto{\pgfqpoint{1.214510in}{1.368663in}}%
\pgfpathlineto{\pgfqpoint{1.215885in}{1.369749in}}%
\pgfpathlineto{\pgfqpoint{1.216988in}{1.370494in}}%
\pgfpathlineto{\pgfqpoint{1.218856in}{1.371581in}}%
\pgfpathlineto{\pgfqpoint{1.219833in}{1.372263in}}%
\pgfpathlineto{\pgfqpoint{1.219919in}{1.372263in}}%
\pgfpathlineto{\pgfqpoint{1.221577in}{1.373319in}}%
\pgfpathlineto{\pgfqpoint{1.222640in}{1.373816in}}%
\pgfpathlineto{\pgfqpoint{1.224633in}{1.374902in}}%
\pgfpathlineto{\pgfqpoint{1.225728in}{1.375740in}}%
\pgfpathlineto{\pgfqpoint{1.227159in}{1.376827in}}%
\pgfpathlineto{\pgfqpoint{1.228245in}{1.377727in}}%
\pgfpathlineto{\pgfqpoint{1.230372in}{1.378813in}}%
\pgfpathlineto{\pgfqpoint{1.231482in}{1.379496in}}%
\pgfpathlineto{\pgfqpoint{1.232865in}{1.380583in}}%
\pgfpathlineto{\pgfqpoint{1.233952in}{1.381079in}}%
\pgfpathlineto{\pgfqpoint{1.236759in}{1.382104in}}%
\pgfpathlineto{\pgfqpoint{1.237845in}{1.382756in}}%
\pgfpathlineto{\pgfqpoint{1.239745in}{1.383842in}}%
\pgfpathlineto{\pgfqpoint{1.240808in}{1.384432in}}%
\pgfpathlineto{\pgfqpoint{1.240839in}{1.384432in}}%
\pgfpathlineto{\pgfqpoint{1.242411in}{1.385518in}}%
\pgfpathlineto{\pgfqpoint{1.243388in}{1.385953in}}%
\pgfpathlineto{\pgfqpoint{1.245741in}{1.387008in}}%
\pgfpathlineto{\pgfqpoint{1.246851in}{1.387816in}}%
\pgfpathlineto{\pgfqpoint{1.248923in}{1.388902in}}%
\pgfpathlineto{\pgfqpoint{1.249931in}{1.389368in}}%
\pgfpathlineto{\pgfqpoint{1.251909in}{1.390454in}}%
\pgfpathlineto{\pgfqpoint{1.253011in}{1.390951in}}%
\pgfpathlineto{\pgfqpoint{1.255239in}{1.392037in}}%
\pgfpathlineto{\pgfqpoint{1.256303in}{1.392596in}}%
\pgfpathlineto{\pgfqpoint{1.258515in}{1.393683in}}%
\pgfpathlineto{\pgfqpoint{1.259492in}{1.394490in}}%
\pgfpathlineto{\pgfqpoint{1.261705in}{1.395576in}}%
\pgfpathlineto{\pgfqpoint{1.262713in}{1.396135in}}%
\pgfpathlineto{\pgfqpoint{1.264191in}{1.397221in}}%
\pgfpathlineto{\pgfqpoint{1.265230in}{1.397780in}}%
\pgfpathlineto{\pgfqpoint{1.267685in}{1.398867in}}%
\pgfpathlineto{\pgfqpoint{1.268764in}{1.399518in}}%
\pgfpathlineto{\pgfqpoint{1.270328in}{1.400605in}}%
\pgfpathlineto{\pgfqpoint{1.271422in}{1.401319in}}%
\pgfpathlineto{\pgfqpoint{1.274119in}{1.402405in}}%
\pgfpathlineto{\pgfqpoint{1.275190in}{1.403150in}}%
\pgfpathlineto{\pgfqpoint{1.278106in}{1.404237in}}%
\pgfpathlineto{\pgfqpoint{1.279216in}{1.404889in}}%
\pgfpathlineto{\pgfqpoint{1.281397in}{1.405975in}}%
\pgfpathlineto{\pgfqpoint{1.282507in}{1.406534in}}%
\pgfpathlineto{\pgfqpoint{1.284947in}{1.407620in}}%
\pgfpathlineto{\pgfqpoint{1.285994in}{1.408396in}}%
\pgfpathlineto{\pgfqpoint{1.286041in}{1.408396in}}%
\pgfpathlineto{\pgfqpoint{1.288629in}{1.409483in}}%
\pgfpathlineto{\pgfqpoint{1.289614in}{1.410011in}}%
\pgfpathlineto{\pgfqpoint{1.289645in}{1.410011in}}%
\pgfpathlineto{\pgfqpoint{1.292076in}{1.411097in}}%
\pgfpathlineto{\pgfqpoint{1.293147in}{1.411718in}}%
\pgfpathlineto{\pgfqpoint{1.293186in}{1.411718in}}%
\pgfpathlineto{\pgfqpoint{1.295195in}{1.412804in}}%
\pgfpathlineto{\pgfqpoint{1.296462in}{1.413612in}}%
\pgfpathlineto{\pgfqpoint{1.298526in}{1.414667in}}%
\pgfpathlineto{\pgfqpoint{1.299636in}{1.415443in}}%
\pgfpathlineto{\pgfqpoint{1.300871in}{1.416530in}}%
\pgfpathlineto{\pgfqpoint{1.301958in}{1.416840in}}%
\pgfpathlineto{\pgfqpoint{1.303896in}{1.417864in}}%
\pgfpathlineto{\pgfqpoint{1.304968in}{1.418485in}}%
\pgfpathlineto{\pgfqpoint{1.306859in}{1.419572in}}%
\pgfpathlineto{\pgfqpoint{1.307907in}{1.420099in}}%
\pgfpathlineto{\pgfqpoint{1.310096in}{1.421186in}}%
\pgfpathlineto{\pgfqpoint{1.311143in}{1.421838in}}%
\pgfpathlineto{\pgfqpoint{1.314286in}{1.422924in}}%
\pgfpathlineto{\pgfqpoint{1.315381in}{1.423545in}}%
\pgfpathlineto{\pgfqpoint{1.318000in}{1.424600in}}%
\pgfpathlineto{\pgfqpoint{1.319071in}{1.425097in}}%
\pgfpathlineto{\pgfqpoint{1.320650in}{1.426153in}}%
\pgfpathlineto{\pgfqpoint{1.321643in}{1.426898in}}%
\pgfpathlineto{\pgfqpoint{1.324418in}{1.427984in}}%
\pgfpathlineto{\pgfqpoint{1.325528in}{1.428450in}}%
\pgfpathlineto{\pgfqpoint{1.327482in}{1.429536in}}%
\pgfpathlineto{\pgfqpoint{1.328452in}{1.430250in}}%
\pgfpathlineto{\pgfqpoint{1.331039in}{1.431337in}}%
\pgfpathlineto{\pgfqpoint{1.332134in}{1.431926in}}%
\pgfpathlineto{\pgfqpoint{1.334033in}{1.433013in}}%
\pgfpathlineto{\pgfqpoint{1.335026in}{1.433603in}}%
\pgfpathlineto{\pgfqpoint{1.337317in}{1.434689in}}%
\pgfpathlineto{\pgfqpoint{1.338052in}{1.435279in}}%
\pgfpathlineto{\pgfqpoint{1.338364in}{1.435279in}}%
\pgfpathlineto{\pgfqpoint{1.340507in}{1.436365in}}%
\pgfpathlineto{\pgfqpoint{1.341593in}{1.436831in}}%
\pgfpathlineto{\pgfqpoint{1.344126in}{1.437918in}}%
\pgfpathlineto{\pgfqpoint{1.345228in}{1.438445in}}%
\pgfpathlineto{\pgfqpoint{1.347793in}{1.439532in}}%
\pgfpathlineto{\pgfqpoint{1.348809in}{1.439935in}}%
\pgfpathlineto{\pgfqpoint{1.351967in}{1.441022in}}%
\pgfpathlineto{\pgfqpoint{1.352929in}{1.441674in}}%
\pgfpathlineto{\pgfqpoint{1.355837in}{1.442760in}}%
\pgfpathlineto{\pgfqpoint{1.356924in}{1.443257in}}%
\pgfpathlineto{\pgfqpoint{1.359339in}{1.444343in}}%
\pgfpathlineto{\pgfqpoint{1.360223in}{1.444685in}}%
\pgfpathlineto{\pgfqpoint{1.362849in}{1.445771in}}%
\pgfpathlineto{\pgfqpoint{1.363959in}{1.446392in}}%
\pgfpathlineto{\pgfqpoint{1.366289in}{1.447447in}}%
\pgfpathlineto{\pgfqpoint{1.367337in}{1.447913in}}%
\pgfpathlineto{\pgfqpoint{1.370370in}{1.449000in}}%
\pgfpathlineto{\pgfqpoint{1.371425in}{1.449620in}}%
\pgfpathlineto{\pgfqpoint{1.374654in}{1.450707in}}%
\pgfpathlineto{\pgfqpoint{1.375764in}{1.451266in}}%
\pgfpathlineto{\pgfqpoint{1.378563in}{1.452352in}}%
\pgfpathlineto{\pgfqpoint{1.379610in}{1.452725in}}%
\pgfpathlineto{\pgfqpoint{1.381956in}{1.453811in}}%
\pgfpathlineto{\pgfqpoint{1.382691in}{1.454184in}}%
\pgfpathlineto{\pgfqpoint{1.382948in}{1.454184in}}%
\pgfpathlineto{\pgfqpoint{1.385028in}{1.455270in}}%
\pgfpathlineto{\pgfqpoint{1.386083in}{1.455798in}}%
\pgfpathlineto{\pgfqpoint{1.386099in}{1.455798in}}%
\pgfpathlineto{\pgfqpoint{1.388296in}{1.456884in}}%
\pgfpathlineto{\pgfqpoint{1.389382in}{1.457381in}}%
\pgfpathlineto{\pgfqpoint{1.392635in}{1.458467in}}%
\pgfpathlineto{\pgfqpoint{1.393745in}{1.458995in}}%
\pgfpathlineto{\pgfqpoint{1.395941in}{1.460082in}}%
\pgfpathlineto{\pgfqpoint{1.396919in}{1.460640in}}%
\pgfpathlineto{\pgfqpoint{1.399655in}{1.461696in}}%
\pgfpathlineto{\pgfqpoint{1.400718in}{1.462224in}}%
\pgfpathlineto{\pgfqpoint{1.400765in}{1.462224in}}%
\pgfpathlineto{\pgfqpoint{1.403571in}{1.463310in}}%
\pgfpathlineto{\pgfqpoint{1.404541in}{1.463838in}}%
\pgfpathlineto{\pgfqpoint{1.406996in}{1.464924in}}%
\pgfpathlineto{\pgfqpoint{1.408090in}{1.465483in}}%
\pgfpathlineto{\pgfqpoint{1.411467in}{1.466569in}}%
\pgfpathlineto{\pgfqpoint{1.412382in}{1.466942in}}%
\pgfpathlineto{\pgfqpoint{1.416650in}{1.468028in}}%
\pgfpathlineto{\pgfqpoint{1.417542in}{1.468308in}}%
\pgfpathlineto{\pgfqpoint{1.417674in}{1.468308in}}%
\pgfpathlineto{\pgfqpoint{1.420286in}{1.469394in}}%
\pgfpathlineto{\pgfqpoint{1.421357in}{1.469829in}}%
\pgfpathlineto{\pgfqpoint{1.425187in}{1.470915in}}%
\pgfpathlineto{\pgfqpoint{1.426227in}{1.471536in}}%
\pgfpathlineto{\pgfqpoint{1.429205in}{1.472623in}}%
\pgfpathlineto{\pgfqpoint{1.430214in}{1.473212in}}%
\pgfpathlineto{\pgfqpoint{1.432778in}{1.474299in}}%
\pgfpathlineto{\pgfqpoint{1.433794in}{1.475013in}}%
\pgfpathlineto{\pgfqpoint{1.433865in}{1.475013in}}%
\pgfpathlineto{\pgfqpoint{1.436820in}{1.476099in}}%
\pgfpathlineto{\pgfqpoint{1.437922in}{1.476658in}}%
\pgfpathlineto{\pgfqpoint{1.440791in}{1.477745in}}%
\pgfpathlineto{\pgfqpoint{1.441886in}{1.478303in}}%
\pgfpathlineto{\pgfqpoint{1.444552in}{1.479390in}}%
\pgfpathlineto{\pgfqpoint{1.445466in}{1.479949in}}%
\pgfpathlineto{\pgfqpoint{1.448890in}{1.481035in}}%
\pgfpathlineto{\pgfqpoint{1.449914in}{1.481532in}}%
\pgfpathlineto{\pgfqpoint{1.449993in}{1.481532in}}%
\pgfpathlineto{\pgfqpoint{1.453315in}{1.482587in}}%
\pgfpathlineto{\pgfqpoint{1.454355in}{1.483053in}}%
\pgfpathlineto{\pgfqpoint{1.457412in}{1.484139in}}%
\pgfpathlineto{\pgfqpoint{1.458631in}{1.484791in}}%
\pgfpathlineto{\pgfqpoint{1.461750in}{1.485878in}}%
\pgfpathlineto{\pgfqpoint{1.462782in}{1.486281in}}%
\pgfpathlineto{\pgfqpoint{1.466144in}{1.487368in}}%
\pgfpathlineto{\pgfqpoint{1.467113in}{1.488082in}}%
\pgfpathlineto{\pgfqpoint{1.467238in}{1.488082in}}%
\pgfpathlineto{\pgfqpoint{1.469834in}{1.489168in}}%
\pgfpathlineto{\pgfqpoint{1.470811in}{1.489820in}}%
\pgfpathlineto{\pgfqpoint{1.470842in}{1.489820in}}%
\pgfpathlineto{\pgfqpoint{1.474438in}{1.490906in}}%
\pgfpathlineto{\pgfqpoint{1.475525in}{1.491403in}}%
\pgfpathlineto{\pgfqpoint{1.478543in}{1.492490in}}%
\pgfpathlineto{\pgfqpoint{1.479489in}{1.493173in}}%
\pgfpathlineto{\pgfqpoint{1.479528in}{1.493173in}}%
\pgfpathlineto{\pgfqpoint{1.482920in}{1.494259in}}%
\pgfpathlineto{\pgfqpoint{1.484023in}{1.494663in}}%
\pgfpathlineto{\pgfqpoint{1.486579in}{1.495749in}}%
\pgfpathlineto{\pgfqpoint{1.487439in}{1.496028in}}%
\pgfpathlineto{\pgfqpoint{1.490808in}{1.497115in}}%
\pgfpathlineto{\pgfqpoint{1.491770in}{1.497332in}}%
\pgfpathlineto{\pgfqpoint{1.491911in}{1.497332in}}%
\pgfpathlineto{\pgfqpoint{1.496101in}{1.498419in}}%
\pgfpathlineto{\pgfqpoint{1.497188in}{1.498977in}}%
\pgfpathlineto{\pgfqpoint{1.499893in}{1.500064in}}%
\pgfpathlineto{\pgfqpoint{1.500940in}{1.500530in}}%
\pgfpathlineto{\pgfqpoint{1.503864in}{1.501616in}}%
\pgfpathlineto{\pgfqpoint{1.504919in}{1.502237in}}%
\pgfpathlineto{\pgfqpoint{1.507679in}{1.503292in}}%
\pgfpathlineto{\pgfqpoint{1.508758in}{1.503820in}}%
\pgfpathlineto{\pgfqpoint{1.513042in}{1.504906in}}%
\pgfpathlineto{\pgfqpoint{1.514097in}{1.505217in}}%
\pgfpathlineto{\pgfqpoint{1.514144in}{1.505217in}}%
\pgfpathlineto{\pgfqpoint{1.516802in}{1.506303in}}%
\pgfpathlineto{\pgfqpoint{1.517865in}{1.506924in}}%
\pgfpathlineto{\pgfqpoint{1.517912in}{1.506924in}}%
\pgfpathlineto{\pgfqpoint{1.520985in}{1.508011in}}%
\pgfpathlineto{\pgfqpoint{1.521931in}{1.508383in}}%
\pgfpathlineto{\pgfqpoint{1.525183in}{1.509470in}}%
\pgfpathlineto{\pgfqpoint{1.526285in}{1.509935in}}%
\pgfpathlineto{\pgfqpoint{1.530233in}{1.511022in}}%
\pgfpathlineto{\pgfqpoint{1.530819in}{1.511332in}}%
\pgfpathlineto{\pgfqpoint{1.531304in}{1.511332in}}%
\pgfpathlineto{\pgfqpoint{1.536104in}{1.512419in}}%
\pgfpathlineto{\pgfqpoint{1.537105in}{1.512636in}}%
\pgfpathlineto{\pgfqpoint{1.540458in}{1.513722in}}%
\pgfpathlineto{\pgfqpoint{1.541529in}{1.514312in}}%
\pgfpathlineto{\pgfqpoint{1.545086in}{1.515368in}}%
\pgfpathlineto{\pgfqpoint{1.546079in}{1.515895in}}%
\pgfpathlineto{\pgfqpoint{1.551724in}{1.516982in}}%
\pgfpathlineto{\pgfqpoint{1.552709in}{1.517354in}}%
\pgfpathlineto{\pgfqpoint{1.552771in}{1.517354in}}%
\pgfpathlineto{\pgfqpoint{1.556219in}{1.518441in}}%
\pgfpathlineto{\pgfqpoint{1.557212in}{1.518751in}}%
\pgfpathlineto{\pgfqpoint{1.557329in}{1.518751in}}%
\pgfpathlineto{\pgfqpoint{1.560753in}{1.519838in}}%
\pgfpathlineto{\pgfqpoint{1.561808in}{1.520241in}}%
\pgfpathlineto{\pgfqpoint{1.565694in}{1.521328in}}%
\pgfpathlineto{\pgfqpoint{1.566632in}{1.521700in}}%
\pgfpathlineto{\pgfqpoint{1.570572in}{1.522787in}}%
\pgfpathlineto{\pgfqpoint{1.571510in}{1.523252in}}%
\pgfpathlineto{\pgfqpoint{1.574778in}{1.524339in}}%
\pgfpathlineto{\pgfqpoint{1.575849in}{1.524773in}}%
\pgfpathlineto{\pgfqpoint{1.578436in}{1.525860in}}%
\pgfpathlineto{\pgfqpoint{1.579515in}{1.526232in}}%
\pgfpathlineto{\pgfqpoint{1.579539in}{1.526232in}}%
\pgfpathlineto{\pgfqpoint{1.585035in}{1.527319in}}%
\pgfpathlineto{\pgfqpoint{1.586051in}{1.527816in}}%
\pgfpathlineto{\pgfqpoint{1.589334in}{1.528902in}}%
\pgfpathlineto{\pgfqpoint{1.590366in}{1.529275in}}%
\pgfpathlineto{\pgfqpoint{1.590397in}{1.529275in}}%
\pgfpathlineto{\pgfqpoint{1.595393in}{1.530361in}}%
\pgfpathlineto{\pgfqpoint{1.596503in}{1.530734in}}%
\pgfpathlineto{\pgfqpoint{1.601303in}{1.531820in}}%
\pgfpathlineto{\pgfqpoint{1.602358in}{1.532193in}}%
\pgfpathlineto{\pgfqpoint{1.605611in}{1.533279in}}%
\pgfpathlineto{\pgfqpoint{1.606322in}{1.533496in}}%
\pgfpathlineto{\pgfqpoint{1.610903in}{1.534583in}}%
\pgfpathlineto{\pgfqpoint{1.611896in}{1.534893in}}%
\pgfpathlineto{\pgfqpoint{1.615523in}{1.535980in}}%
\pgfpathlineto{\pgfqpoint{1.616407in}{1.536166in}}%
\pgfpathlineto{\pgfqpoint{1.619862in}{1.537252in}}%
\pgfpathlineto{\pgfqpoint{1.620957in}{1.537718in}}%
\pgfpathlineto{\pgfqpoint{1.624897in}{1.538804in}}%
\pgfpathlineto{\pgfqpoint{1.625905in}{1.539084in}}%
\pgfpathlineto{\pgfqpoint{1.629791in}{1.540170in}}%
\pgfpathlineto{\pgfqpoint{1.630830in}{1.540667in}}%
\pgfpathlineto{\pgfqpoint{1.634809in}{1.541753in}}%
\pgfpathlineto{\pgfqpoint{1.635904in}{1.541971in}}%
\pgfpathlineto{\pgfqpoint{1.639438in}{1.543026in}}%
\pgfpathlineto{\pgfqpoint{1.640430in}{1.543554in}}%
\pgfpathlineto{\pgfqpoint{1.644574in}{1.544640in}}%
\pgfpathlineto{\pgfqpoint{1.645684in}{1.544920in}}%
\pgfpathlineto{\pgfqpoint{1.650695in}{1.546006in}}%
\pgfpathlineto{\pgfqpoint{1.651719in}{1.546348in}}%
\pgfpathlineto{\pgfqpoint{1.651766in}{1.546348in}}%
\pgfpathlineto{\pgfqpoint{1.656605in}{1.547434in}}%
\pgfpathlineto{\pgfqpoint{1.657707in}{1.547838in}}%
\pgfpathlineto{\pgfqpoint{1.661624in}{1.548924in}}%
\pgfpathlineto{\pgfqpoint{1.662460in}{1.549235in}}%
\pgfpathlineto{\pgfqpoint{1.666682in}{1.550321in}}%
\pgfpathlineto{\pgfqpoint{1.667683in}{1.550507in}}%
\pgfpathlineto{\pgfqpoint{1.671435in}{1.551594in}}%
\pgfpathlineto{\pgfqpoint{1.672389in}{1.551749in}}%
\pgfpathlineto{\pgfqpoint{1.677588in}{1.552836in}}%
\pgfpathlineto{\pgfqpoint{1.678690in}{1.553115in}}%
\pgfpathlineto{\pgfqpoint{1.683951in}{1.554201in}}%
\pgfpathlineto{\pgfqpoint{1.684913in}{1.554450in}}%
\pgfpathlineto{\pgfqpoint{1.689009in}{1.555536in}}%
\pgfpathlineto{\pgfqpoint{1.690018in}{1.555785in}}%
\pgfpathlineto{\pgfqpoint{1.694372in}{1.556871in}}%
\pgfpathlineto{\pgfqpoint{1.695459in}{1.557088in}}%
\pgfpathlineto{\pgfqpoint{1.700556in}{1.558175in}}%
\pgfpathlineto{\pgfqpoint{1.701611in}{1.558485in}}%
\pgfpathlineto{\pgfqpoint{1.707834in}{1.559572in}}%
\pgfpathlineto{\pgfqpoint{1.708866in}{1.559820in}}%
\pgfpathlineto{\pgfqpoint{1.708897in}{1.559820in}}%
\pgfpathlineto{\pgfqpoint{1.713869in}{1.560906in}}%
\pgfpathlineto{\pgfqpoint{1.714893in}{1.561186in}}%
\pgfpathlineto{\pgfqpoint{1.719256in}{1.562272in}}%
\pgfpathlineto{\pgfqpoint{1.720327in}{1.562552in}}%
\pgfpathlineto{\pgfqpoint{1.724681in}{1.563638in}}%
\pgfpathlineto{\pgfqpoint{1.725580in}{1.563731in}}%
\pgfpathlineto{\pgfqpoint{1.732092in}{1.564818in}}%
\pgfpathlineto{\pgfqpoint{1.733030in}{1.565190in}}%
\pgfpathlineto{\pgfqpoint{1.739073in}{1.566277in}}%
\pgfpathlineto{\pgfqpoint{1.739793in}{1.566556in}}%
\pgfpathlineto{\pgfqpoint{1.744702in}{1.567643in}}%
\pgfpathlineto{\pgfqpoint{1.745711in}{1.567829in}}%
\pgfpathlineto{\pgfqpoint{1.750964in}{1.568915in}}%
\pgfpathlineto{\pgfqpoint{1.752066in}{1.569195in}}%
\pgfpathlineto{\pgfqpoint{1.758117in}{1.570281in}}%
\pgfpathlineto{\pgfqpoint{1.759157in}{1.570685in}}%
\pgfpathlineto{\pgfqpoint{1.764723in}{1.571771in}}%
\pgfpathlineto{\pgfqpoint{1.765732in}{1.571957in}}%
\pgfpathlineto{\pgfqpoint{1.765825in}{1.571957in}}%
\pgfpathlineto{\pgfqpoint{1.772736in}{1.573044in}}%
\pgfpathlineto{\pgfqpoint{1.773846in}{1.573199in}}%
\pgfpathlineto{\pgfqpoint{1.780992in}{1.574286in}}%
\pgfpathlineto{\pgfqpoint{1.782102in}{1.574534in}}%
\pgfpathlineto{\pgfqpoint{1.789208in}{1.575620in}}%
\pgfpathlineto{\pgfqpoint{1.789990in}{1.575807in}}%
\pgfpathlineto{\pgfqpoint{1.790318in}{1.575807in}}%
\pgfpathlineto{\pgfqpoint{1.796009in}{1.576862in}}%
\pgfpathlineto{\pgfqpoint{1.796744in}{1.577079in}}%
\pgfpathlineto{\pgfqpoint{1.804476in}{1.578166in}}%
\pgfpathlineto{\pgfqpoint{1.805547in}{1.578290in}}%
\pgfpathlineto{\pgfqpoint{1.811731in}{1.579377in}}%
\pgfpathlineto{\pgfqpoint{1.812168in}{1.579501in}}%
\pgfpathlineto{\pgfqpoint{1.812770in}{1.579501in}}%
\pgfpathlineto{\pgfqpoint{1.819533in}{1.580587in}}%
\pgfpathlineto{\pgfqpoint{1.820181in}{1.580804in}}%
\pgfpathlineto{\pgfqpoint{1.820385in}{1.580804in}}%
\pgfpathlineto{\pgfqpoint{1.830610in}{1.581891in}}%
\pgfpathlineto{\pgfqpoint{1.831666in}{1.582046in}}%
\pgfpathlineto{\pgfqpoint{1.838959in}{1.583133in}}%
\pgfpathlineto{\pgfqpoint{1.839835in}{1.583381in}}%
\pgfpathlineto{\pgfqpoint{1.847348in}{1.584467in}}%
\pgfpathlineto{\pgfqpoint{1.848247in}{1.584623in}}%
\pgfpathlineto{\pgfqpoint{1.855025in}{1.585709in}}%
\pgfpathlineto{\pgfqpoint{1.855861in}{1.585957in}}%
\pgfpathlineto{\pgfqpoint{1.862311in}{1.587013in}}%
\pgfpathlineto{\pgfqpoint{1.863366in}{1.587354in}}%
\pgfpathlineto{\pgfqpoint{1.863405in}{1.587354in}}%
\pgfpathlineto{\pgfqpoint{1.873685in}{1.588441in}}%
\pgfpathlineto{\pgfqpoint{1.874577in}{1.588658in}}%
\pgfpathlineto{\pgfqpoint{1.880682in}{1.589714in}}%
\pgfpathlineto{\pgfqpoint{1.881151in}{1.589869in}}%
\pgfpathlineto{\pgfqpoint{1.881730in}{1.589869in}}%
\pgfpathlineto{\pgfqpoint{1.893081in}{1.590955in}}%
\pgfpathlineto{\pgfqpoint{1.893464in}{1.591048in}}%
\pgfpathlineto{\pgfqpoint{1.893597in}{1.591048in}}%
\pgfpathlineto{\pgfqpoint{1.903205in}{1.592135in}}%
\pgfpathlineto{\pgfqpoint{1.904245in}{1.592321in}}%
\pgfpathlineto{\pgfqpoint{1.912414in}{1.593408in}}%
\pgfpathlineto{\pgfqpoint{1.913368in}{1.593501in}}%
\pgfpathlineto{\pgfqpoint{1.922264in}{1.594587in}}%
\pgfpathlineto{\pgfqpoint{1.922827in}{1.594742in}}%
\pgfpathlineto{\pgfqpoint{1.923108in}{1.594742in}}%
\pgfpathlineto{\pgfqpoint{1.935820in}{1.595829in}}%
\pgfpathlineto{\pgfqpoint{1.936641in}{1.595953in}}%
\pgfpathlineto{\pgfqpoint{1.936836in}{1.595953in}}%
\pgfpathlineto{\pgfqpoint{1.950845in}{1.597040in}}%
\pgfpathlineto{\pgfqpoint{1.951346in}{1.597226in}}%
\pgfpathlineto{\pgfqpoint{1.951666in}{1.597226in}}%
\pgfpathlineto{\pgfqpoint{1.968521in}{1.598312in}}%
\pgfpathlineto{\pgfqpoint{1.968990in}{1.598374in}}%
\pgfpathlineto{\pgfqpoint{1.969272in}{1.598374in}}%
\pgfpathlineto{\pgfqpoint{1.982585in}{1.599461in}}%
\pgfpathlineto{\pgfqpoint{1.982585in}{1.599492in}}%
\pgfpathlineto{\pgfqpoint{1.983203in}{1.599492in}}%
\pgfpathlineto{\pgfqpoint{2.001293in}{1.600578in}}%
\pgfpathlineto{\pgfqpoint{2.002168in}{1.600671in}}%
\pgfpathlineto{\pgfqpoint{2.002278in}{1.600671in}}%
\pgfpathlineto{\pgfqpoint{2.028592in}{1.601758in}}%
\pgfpathlineto{\pgfqpoint{2.028592in}{1.601789in}}%
\pgfpathlineto{\pgfqpoint{2.029514in}{1.601789in}}%
\pgfpathlineto{\pgfqpoint{2.033126in}{1.601944in}}%
\pgfpathlineto{\pgfqpoint{2.033126in}{1.601944in}}%
\pgfusepath{stroke}%
\end{pgfscope}%
\begin{pgfscope}%
\pgfsetrectcap%
\pgfsetmiterjoin%
\pgfsetlinewidth{0.803000pt}%
\definecolor{currentstroke}{rgb}{0.000000,0.000000,0.000000}%
\pgfsetstrokecolor{currentstroke}%
\pgfsetdash{}{0pt}%
\pgfpathmoveto{\pgfqpoint{0.553581in}{0.499444in}}%
\pgfpathlineto{\pgfqpoint{0.553581in}{1.654444in}}%
\pgfusepath{stroke}%
\end{pgfscope}%
\begin{pgfscope}%
\pgfsetrectcap%
\pgfsetmiterjoin%
\pgfsetlinewidth{0.803000pt}%
\definecolor{currentstroke}{rgb}{0.000000,0.000000,0.000000}%
\pgfsetstrokecolor{currentstroke}%
\pgfsetdash{}{0pt}%
\pgfpathmoveto{\pgfqpoint{2.103581in}{0.499444in}}%
\pgfpathlineto{\pgfqpoint{2.103581in}{1.654444in}}%
\pgfusepath{stroke}%
\end{pgfscope}%
\begin{pgfscope}%
\pgfsetrectcap%
\pgfsetmiterjoin%
\pgfsetlinewidth{0.803000pt}%
\definecolor{currentstroke}{rgb}{0.000000,0.000000,0.000000}%
\pgfsetstrokecolor{currentstroke}%
\pgfsetdash{}{0pt}%
\pgfpathmoveto{\pgfqpoint{0.553581in}{0.499444in}}%
\pgfpathlineto{\pgfqpoint{2.103581in}{0.499444in}}%
\pgfusepath{stroke}%
\end{pgfscope}%
\begin{pgfscope}%
\pgfsetrectcap%
\pgfsetmiterjoin%
\pgfsetlinewidth{0.803000pt}%
\definecolor{currentstroke}{rgb}{0.000000,0.000000,0.000000}%
\pgfsetstrokecolor{currentstroke}%
\pgfsetdash{}{0pt}%
\pgfpathmoveto{\pgfqpoint{0.553581in}{1.654444in}}%
\pgfpathlineto{\pgfqpoint{2.103581in}{1.654444in}}%
\pgfusepath{stroke}%
\end{pgfscope}%
\begin{pgfscope}%
\pgfsetbuttcap%
\pgfsetmiterjoin%
\definecolor{currentfill}{rgb}{1.000000,1.000000,1.000000}%
\pgfsetfillcolor{currentfill}%
\pgfsetfillopacity{0.800000}%
\pgfsetlinewidth{1.003750pt}%
\definecolor{currentstroke}{rgb}{0.800000,0.800000,0.800000}%
\pgfsetstrokecolor{currentstroke}%
\pgfsetstrokeopacity{0.800000}%
\pgfsetdash{}{0pt}%
\pgfpathmoveto{\pgfqpoint{0.832747in}{0.568889in}}%
\pgfpathlineto{\pgfqpoint{2.006358in}{0.568889in}}%
\pgfpathquadraticcurveto{\pgfqpoint{2.034136in}{0.568889in}}{\pgfqpoint{2.034136in}{0.596666in}}%
\pgfpathlineto{\pgfqpoint{2.034136in}{0.776388in}}%
\pgfpathquadraticcurveto{\pgfqpoint{2.034136in}{0.804166in}}{\pgfqpoint{2.006358in}{0.804166in}}%
\pgfpathlineto{\pgfqpoint{0.832747in}{0.804166in}}%
\pgfpathquadraticcurveto{\pgfqpoint{0.804970in}{0.804166in}}{\pgfqpoint{0.804970in}{0.776388in}}%
\pgfpathlineto{\pgfqpoint{0.804970in}{0.596666in}}%
\pgfpathquadraticcurveto{\pgfqpoint{0.804970in}{0.568889in}}{\pgfqpoint{0.832747in}{0.568889in}}%
\pgfpathlineto{\pgfqpoint{0.832747in}{0.568889in}}%
\pgfpathclose%
\pgfusepath{stroke,fill}%
\end{pgfscope}%
\begin{pgfscope}%
\pgfsetrectcap%
\pgfsetroundjoin%
\pgfsetlinewidth{1.505625pt}%
\definecolor{currentstroke}{rgb}{0.000000,0.000000,0.000000}%
\pgfsetstrokecolor{currentstroke}%
\pgfsetdash{}{0pt}%
\pgfpathmoveto{\pgfqpoint{0.860525in}{0.700000in}}%
\pgfpathlineto{\pgfqpoint{0.999414in}{0.700000in}}%
\pgfpathlineto{\pgfqpoint{1.138303in}{0.700000in}}%
\pgfusepath{stroke}%
\end{pgfscope}%
\begin{pgfscope}%
\definecolor{textcolor}{rgb}{0.000000,0.000000,0.000000}%
\pgfsetstrokecolor{textcolor}%
\pgfsetfillcolor{textcolor}%
\pgftext[x=1.249414in,y=0.651388in,left,base]{\color{textcolor}\rmfamily\fontsize{10.000000}{12.000000}\selectfont AUC=0.752}%
\end{pgfscope}%
\end{pgfpicture}%
\makeatother%
\endgroup%

\end{tabular}


\

To make a useful visualization of the results where we can see the interplay between the negative and positive classes, we can transform the data.  A transformation that preserves rank will have no effect on the ROC curve.  [Cite]  For the graph below, we mapped the smallest value in the set to 0 and the largest to 1.  

%
\noindent\begin{tabular}{@{\hspace{-6pt}}p{4.3in} @{\hspace{-6pt}}p{2.0in}}
	\vskip 0pt
	\hfil Raw Model Output
	
	%% Creator: Matplotlib, PGF backend
%%
%% To include the figure in your LaTeX document, write
%%   \input{<filename>.pgf}
%%
%% Make sure the required packages are loaded in your preamble
%%   \usepackage{pgf}
%%
%% Also ensure that all the required font packages are loaded; for instance,
%% the lmodern package is sometimes necessary when using math font.
%%   \usepackage{lmodern}
%%
%% Figures using additional raster images can only be included by \input if
%% they are in the same directory as the main LaTeX file. For loading figures
%% from other directories you can use the `import` package
%%   \usepackage{import}
%%
%% and then include the figures with
%%   \import{<path to file>}{<filename>.pgf}
%%
%% Matplotlib used the following preamble
%%   
%%   \usepackage{fontspec}
%%   \makeatletter\@ifpackageloaded{underscore}{}{\usepackage[strings]{underscore}}\makeatother
%%
\begingroup%
\makeatletter%
\begin{pgfpicture}%
\pgfpathrectangle{\pgfpointorigin}{\pgfqpoint{4.102500in}{1.754444in}}%
\pgfusepath{use as bounding box, clip}%
\begin{pgfscope}%
\pgfsetbuttcap%
\pgfsetmiterjoin%
\definecolor{currentfill}{rgb}{1.000000,1.000000,1.000000}%
\pgfsetfillcolor{currentfill}%
\pgfsetlinewidth{0.000000pt}%
\definecolor{currentstroke}{rgb}{1.000000,1.000000,1.000000}%
\pgfsetstrokecolor{currentstroke}%
\pgfsetdash{}{0pt}%
\pgfpathmoveto{\pgfqpoint{0.000000in}{0.000000in}}%
\pgfpathlineto{\pgfqpoint{4.102500in}{0.000000in}}%
\pgfpathlineto{\pgfqpoint{4.102500in}{1.754444in}}%
\pgfpathlineto{\pgfqpoint{0.000000in}{1.754444in}}%
\pgfpathlineto{\pgfqpoint{0.000000in}{0.000000in}}%
\pgfpathclose%
\pgfusepath{fill}%
\end{pgfscope}%
\begin{pgfscope}%
\pgfsetbuttcap%
\pgfsetmiterjoin%
\definecolor{currentfill}{rgb}{1.000000,1.000000,1.000000}%
\pgfsetfillcolor{currentfill}%
\pgfsetlinewidth{0.000000pt}%
\definecolor{currentstroke}{rgb}{0.000000,0.000000,0.000000}%
\pgfsetstrokecolor{currentstroke}%
\pgfsetstrokeopacity{0.000000}%
\pgfsetdash{}{0pt}%
\pgfpathmoveto{\pgfqpoint{0.515000in}{0.499444in}}%
\pgfpathlineto{\pgfqpoint{4.002500in}{0.499444in}}%
\pgfpathlineto{\pgfqpoint{4.002500in}{1.654444in}}%
\pgfpathlineto{\pgfqpoint{0.515000in}{1.654444in}}%
\pgfpathlineto{\pgfqpoint{0.515000in}{0.499444in}}%
\pgfpathclose%
\pgfusepath{fill}%
\end{pgfscope}%
\begin{pgfscope}%
\pgfpathrectangle{\pgfqpoint{0.515000in}{0.499444in}}{\pgfqpoint{3.487500in}{1.155000in}}%
\pgfusepath{clip}%
\pgfsetbuttcap%
\pgfsetmiterjoin%
\pgfsetlinewidth{1.003750pt}%
\definecolor{currentstroke}{rgb}{0.000000,0.000000,0.000000}%
\pgfsetstrokecolor{currentstroke}%
\pgfsetdash{}{0pt}%
\pgfpathmoveto{\pgfqpoint{0.610114in}{0.499444in}}%
\pgfpathlineto{\pgfqpoint{0.673523in}{0.499444in}}%
\pgfpathlineto{\pgfqpoint{0.673523in}{0.499481in}}%
\pgfpathlineto{\pgfqpoint{0.610114in}{0.499481in}}%
\pgfpathlineto{\pgfqpoint{0.610114in}{0.499444in}}%
\pgfpathclose%
\pgfusepath{stroke}%
\end{pgfscope}%
\begin{pgfscope}%
\pgfpathrectangle{\pgfqpoint{0.515000in}{0.499444in}}{\pgfqpoint{3.487500in}{1.155000in}}%
\pgfusepath{clip}%
\pgfsetbuttcap%
\pgfsetmiterjoin%
\pgfsetlinewidth{1.003750pt}%
\definecolor{currentstroke}{rgb}{0.000000,0.000000,0.000000}%
\pgfsetstrokecolor{currentstroke}%
\pgfsetdash{}{0pt}%
\pgfpathmoveto{\pgfqpoint{0.768637in}{0.499444in}}%
\pgfpathlineto{\pgfqpoint{0.832046in}{0.499444in}}%
\pgfpathlineto{\pgfqpoint{0.832046in}{0.500114in}}%
\pgfpathlineto{\pgfqpoint{0.768637in}{0.500114in}}%
\pgfpathlineto{\pgfqpoint{0.768637in}{0.499444in}}%
\pgfpathclose%
\pgfusepath{stroke}%
\end{pgfscope}%
\begin{pgfscope}%
\pgfpathrectangle{\pgfqpoint{0.515000in}{0.499444in}}{\pgfqpoint{3.487500in}{1.155000in}}%
\pgfusepath{clip}%
\pgfsetbuttcap%
\pgfsetmiterjoin%
\pgfsetlinewidth{1.003750pt}%
\definecolor{currentstroke}{rgb}{0.000000,0.000000,0.000000}%
\pgfsetstrokecolor{currentstroke}%
\pgfsetdash{}{0pt}%
\pgfpathmoveto{\pgfqpoint{0.927159in}{0.499444in}}%
\pgfpathlineto{\pgfqpoint{0.990568in}{0.499444in}}%
\pgfpathlineto{\pgfqpoint{0.990568in}{0.510007in}}%
\pgfpathlineto{\pgfqpoint{0.927159in}{0.510007in}}%
\pgfpathlineto{\pgfqpoint{0.927159in}{0.499444in}}%
\pgfpathclose%
\pgfusepath{stroke}%
\end{pgfscope}%
\begin{pgfscope}%
\pgfpathrectangle{\pgfqpoint{0.515000in}{0.499444in}}{\pgfqpoint{3.487500in}{1.155000in}}%
\pgfusepath{clip}%
\pgfsetbuttcap%
\pgfsetmiterjoin%
\pgfsetlinewidth{1.003750pt}%
\definecolor{currentstroke}{rgb}{0.000000,0.000000,0.000000}%
\pgfsetstrokecolor{currentstroke}%
\pgfsetdash{}{0pt}%
\pgfpathmoveto{\pgfqpoint{1.085682in}{0.499444in}}%
\pgfpathlineto{\pgfqpoint{1.149091in}{0.499444in}}%
\pgfpathlineto{\pgfqpoint{1.149091in}{0.553338in}}%
\pgfpathlineto{\pgfqpoint{1.085682in}{0.553338in}}%
\pgfpathlineto{\pgfqpoint{1.085682in}{0.499444in}}%
\pgfpathclose%
\pgfusepath{stroke}%
\end{pgfscope}%
\begin{pgfscope}%
\pgfpathrectangle{\pgfqpoint{0.515000in}{0.499444in}}{\pgfqpoint{3.487500in}{1.155000in}}%
\pgfusepath{clip}%
\pgfsetbuttcap%
\pgfsetmiterjoin%
\pgfsetlinewidth{1.003750pt}%
\definecolor{currentstroke}{rgb}{0.000000,0.000000,0.000000}%
\pgfsetstrokecolor{currentstroke}%
\pgfsetdash{}{0pt}%
\pgfpathmoveto{\pgfqpoint{1.244205in}{0.499444in}}%
\pgfpathlineto{\pgfqpoint{1.307614in}{0.499444in}}%
\pgfpathlineto{\pgfqpoint{1.307614in}{0.659563in}}%
\pgfpathlineto{\pgfqpoint{1.244205in}{0.659563in}}%
\pgfpathlineto{\pgfqpoint{1.244205in}{0.499444in}}%
\pgfpathclose%
\pgfusepath{stroke}%
\end{pgfscope}%
\begin{pgfscope}%
\pgfpathrectangle{\pgfqpoint{0.515000in}{0.499444in}}{\pgfqpoint{3.487500in}{1.155000in}}%
\pgfusepath{clip}%
\pgfsetbuttcap%
\pgfsetmiterjoin%
\pgfsetlinewidth{1.003750pt}%
\definecolor{currentstroke}{rgb}{0.000000,0.000000,0.000000}%
\pgfsetstrokecolor{currentstroke}%
\pgfsetdash{}{0pt}%
\pgfpathmoveto{\pgfqpoint{1.402728in}{0.499444in}}%
\pgfpathlineto{\pgfqpoint{1.466137in}{0.499444in}}%
\pgfpathlineto{\pgfqpoint{1.466137in}{0.859329in}}%
\pgfpathlineto{\pgfqpoint{1.402728in}{0.859329in}}%
\pgfpathlineto{\pgfqpoint{1.402728in}{0.499444in}}%
\pgfpathclose%
\pgfusepath{stroke}%
\end{pgfscope}%
\begin{pgfscope}%
\pgfpathrectangle{\pgfqpoint{0.515000in}{0.499444in}}{\pgfqpoint{3.487500in}{1.155000in}}%
\pgfusepath{clip}%
\pgfsetbuttcap%
\pgfsetmiterjoin%
\pgfsetlinewidth{1.003750pt}%
\definecolor{currentstroke}{rgb}{0.000000,0.000000,0.000000}%
\pgfsetstrokecolor{currentstroke}%
\pgfsetdash{}{0pt}%
\pgfpathmoveto{\pgfqpoint{1.561250in}{0.499444in}}%
\pgfpathlineto{\pgfqpoint{1.624659in}{0.499444in}}%
\pgfpathlineto{\pgfqpoint{1.624659in}{1.124891in}}%
\pgfpathlineto{\pgfqpoint{1.561250in}{1.124891in}}%
\pgfpathlineto{\pgfqpoint{1.561250in}{0.499444in}}%
\pgfpathclose%
\pgfusepath{stroke}%
\end{pgfscope}%
\begin{pgfscope}%
\pgfpathrectangle{\pgfqpoint{0.515000in}{0.499444in}}{\pgfqpoint{3.487500in}{1.155000in}}%
\pgfusepath{clip}%
\pgfsetbuttcap%
\pgfsetmiterjoin%
\pgfsetlinewidth{1.003750pt}%
\definecolor{currentstroke}{rgb}{0.000000,0.000000,0.000000}%
\pgfsetstrokecolor{currentstroke}%
\pgfsetdash{}{0pt}%
\pgfpathmoveto{\pgfqpoint{1.719773in}{0.499444in}}%
\pgfpathlineto{\pgfqpoint{1.783182in}{0.499444in}}%
\pgfpathlineto{\pgfqpoint{1.783182in}{1.402207in}}%
\pgfpathlineto{\pgfqpoint{1.719773in}{1.402207in}}%
\pgfpathlineto{\pgfqpoint{1.719773in}{0.499444in}}%
\pgfpathclose%
\pgfusepath{stroke}%
\end{pgfscope}%
\begin{pgfscope}%
\pgfpathrectangle{\pgfqpoint{0.515000in}{0.499444in}}{\pgfqpoint{3.487500in}{1.155000in}}%
\pgfusepath{clip}%
\pgfsetbuttcap%
\pgfsetmiterjoin%
\pgfsetlinewidth{1.003750pt}%
\definecolor{currentstroke}{rgb}{0.000000,0.000000,0.000000}%
\pgfsetstrokecolor{currentstroke}%
\pgfsetdash{}{0pt}%
\pgfpathmoveto{\pgfqpoint{1.878296in}{0.499444in}}%
\pgfpathlineto{\pgfqpoint{1.941705in}{0.499444in}}%
\pgfpathlineto{\pgfqpoint{1.941705in}{1.599444in}}%
\pgfpathlineto{\pgfqpoint{1.878296in}{1.599444in}}%
\pgfpathlineto{\pgfqpoint{1.878296in}{0.499444in}}%
\pgfpathclose%
\pgfusepath{stroke}%
\end{pgfscope}%
\begin{pgfscope}%
\pgfpathrectangle{\pgfqpoint{0.515000in}{0.499444in}}{\pgfqpoint{3.487500in}{1.155000in}}%
\pgfusepath{clip}%
\pgfsetbuttcap%
\pgfsetmiterjoin%
\pgfsetlinewidth{1.003750pt}%
\definecolor{currentstroke}{rgb}{0.000000,0.000000,0.000000}%
\pgfsetstrokecolor{currentstroke}%
\pgfsetdash{}{0pt}%
\pgfpathmoveto{\pgfqpoint{2.036818in}{0.499444in}}%
\pgfpathlineto{\pgfqpoint{2.100228in}{0.499444in}}%
\pgfpathlineto{\pgfqpoint{2.100228in}{1.592303in}}%
\pgfpathlineto{\pgfqpoint{2.036818in}{1.592303in}}%
\pgfpathlineto{\pgfqpoint{2.036818in}{0.499444in}}%
\pgfpathclose%
\pgfusepath{stroke}%
\end{pgfscope}%
\begin{pgfscope}%
\pgfpathrectangle{\pgfqpoint{0.515000in}{0.499444in}}{\pgfqpoint{3.487500in}{1.155000in}}%
\pgfusepath{clip}%
\pgfsetbuttcap%
\pgfsetmiterjoin%
\pgfsetlinewidth{1.003750pt}%
\definecolor{currentstroke}{rgb}{0.000000,0.000000,0.000000}%
\pgfsetstrokecolor{currentstroke}%
\pgfsetdash{}{0pt}%
\pgfpathmoveto{\pgfqpoint{2.195341in}{0.499444in}}%
\pgfpathlineto{\pgfqpoint{2.258750in}{0.499444in}}%
\pgfpathlineto{\pgfqpoint{2.258750in}{1.416266in}}%
\pgfpathlineto{\pgfqpoint{2.195341in}{1.416266in}}%
\pgfpathlineto{\pgfqpoint{2.195341in}{0.499444in}}%
\pgfpathclose%
\pgfusepath{stroke}%
\end{pgfscope}%
\begin{pgfscope}%
\pgfpathrectangle{\pgfqpoint{0.515000in}{0.499444in}}{\pgfqpoint{3.487500in}{1.155000in}}%
\pgfusepath{clip}%
\pgfsetbuttcap%
\pgfsetmiterjoin%
\pgfsetlinewidth{1.003750pt}%
\definecolor{currentstroke}{rgb}{0.000000,0.000000,0.000000}%
\pgfsetstrokecolor{currentstroke}%
\pgfsetdash{}{0pt}%
\pgfpathmoveto{\pgfqpoint{2.353864in}{0.499444in}}%
\pgfpathlineto{\pgfqpoint{2.417273in}{0.499444in}}%
\pgfpathlineto{\pgfqpoint{2.417273in}{1.151931in}}%
\pgfpathlineto{\pgfqpoint{2.353864in}{1.151931in}}%
\pgfpathlineto{\pgfqpoint{2.353864in}{0.499444in}}%
\pgfpathclose%
\pgfusepath{stroke}%
\end{pgfscope}%
\begin{pgfscope}%
\pgfpathrectangle{\pgfqpoint{0.515000in}{0.499444in}}{\pgfqpoint{3.487500in}{1.155000in}}%
\pgfusepath{clip}%
\pgfsetbuttcap%
\pgfsetmiterjoin%
\pgfsetlinewidth{1.003750pt}%
\definecolor{currentstroke}{rgb}{0.000000,0.000000,0.000000}%
\pgfsetstrokecolor{currentstroke}%
\pgfsetdash{}{0pt}%
\pgfpathmoveto{\pgfqpoint{2.512387in}{0.499444in}}%
\pgfpathlineto{\pgfqpoint{2.575796in}{0.499444in}}%
\pgfpathlineto{\pgfqpoint{2.575796in}{0.899424in}}%
\pgfpathlineto{\pgfqpoint{2.512387in}{0.899424in}}%
\pgfpathlineto{\pgfqpoint{2.512387in}{0.499444in}}%
\pgfpathclose%
\pgfusepath{stroke}%
\end{pgfscope}%
\begin{pgfscope}%
\pgfpathrectangle{\pgfqpoint{0.515000in}{0.499444in}}{\pgfqpoint{3.487500in}{1.155000in}}%
\pgfusepath{clip}%
\pgfsetbuttcap%
\pgfsetmiterjoin%
\pgfsetlinewidth{1.003750pt}%
\definecolor{currentstroke}{rgb}{0.000000,0.000000,0.000000}%
\pgfsetstrokecolor{currentstroke}%
\pgfsetdash{}{0pt}%
\pgfpathmoveto{\pgfqpoint{2.670909in}{0.499444in}}%
\pgfpathlineto{\pgfqpoint{2.734318in}{0.499444in}}%
\pgfpathlineto{\pgfqpoint{2.734318in}{0.722159in}}%
\pgfpathlineto{\pgfqpoint{2.670909in}{0.722159in}}%
\pgfpathlineto{\pgfqpoint{2.670909in}{0.499444in}}%
\pgfpathclose%
\pgfusepath{stroke}%
\end{pgfscope}%
\begin{pgfscope}%
\pgfpathrectangle{\pgfqpoint{0.515000in}{0.499444in}}{\pgfqpoint{3.487500in}{1.155000in}}%
\pgfusepath{clip}%
\pgfsetbuttcap%
\pgfsetmiterjoin%
\pgfsetlinewidth{1.003750pt}%
\definecolor{currentstroke}{rgb}{0.000000,0.000000,0.000000}%
\pgfsetstrokecolor{currentstroke}%
\pgfsetdash{}{0pt}%
\pgfpathmoveto{\pgfqpoint{2.829432in}{0.499444in}}%
\pgfpathlineto{\pgfqpoint{2.892841in}{0.499444in}}%
\pgfpathlineto{\pgfqpoint{2.892841in}{0.613554in}}%
\pgfpathlineto{\pgfqpoint{2.829432in}{0.613554in}}%
\pgfpathlineto{\pgfqpoint{2.829432in}{0.499444in}}%
\pgfpathclose%
\pgfusepath{stroke}%
\end{pgfscope}%
\begin{pgfscope}%
\pgfpathrectangle{\pgfqpoint{0.515000in}{0.499444in}}{\pgfqpoint{3.487500in}{1.155000in}}%
\pgfusepath{clip}%
\pgfsetbuttcap%
\pgfsetmiterjoin%
\pgfsetlinewidth{1.003750pt}%
\definecolor{currentstroke}{rgb}{0.000000,0.000000,0.000000}%
\pgfsetstrokecolor{currentstroke}%
\pgfsetdash{}{0pt}%
\pgfpathmoveto{\pgfqpoint{2.987955in}{0.499444in}}%
\pgfpathlineto{\pgfqpoint{3.051364in}{0.499444in}}%
\pgfpathlineto{\pgfqpoint{3.051364in}{0.553449in}}%
\pgfpathlineto{\pgfqpoint{2.987955in}{0.553449in}}%
\pgfpathlineto{\pgfqpoint{2.987955in}{0.499444in}}%
\pgfpathclose%
\pgfusepath{stroke}%
\end{pgfscope}%
\begin{pgfscope}%
\pgfpathrectangle{\pgfqpoint{0.515000in}{0.499444in}}{\pgfqpoint{3.487500in}{1.155000in}}%
\pgfusepath{clip}%
\pgfsetbuttcap%
\pgfsetmiterjoin%
\pgfsetlinewidth{1.003750pt}%
\definecolor{currentstroke}{rgb}{0.000000,0.000000,0.000000}%
\pgfsetstrokecolor{currentstroke}%
\pgfsetdash{}{0pt}%
\pgfpathmoveto{\pgfqpoint{3.146478in}{0.499444in}}%
\pgfpathlineto{\pgfqpoint{3.209887in}{0.499444in}}%
\pgfpathlineto{\pgfqpoint{3.209887in}{0.522802in}}%
\pgfpathlineto{\pgfqpoint{3.146478in}{0.522802in}}%
\pgfpathlineto{\pgfqpoint{3.146478in}{0.499444in}}%
\pgfpathclose%
\pgfusepath{stroke}%
\end{pgfscope}%
\begin{pgfscope}%
\pgfpathrectangle{\pgfqpoint{0.515000in}{0.499444in}}{\pgfqpoint{3.487500in}{1.155000in}}%
\pgfusepath{clip}%
\pgfsetbuttcap%
\pgfsetmiterjoin%
\pgfsetlinewidth{1.003750pt}%
\definecolor{currentstroke}{rgb}{0.000000,0.000000,0.000000}%
\pgfsetstrokecolor{currentstroke}%
\pgfsetdash{}{0pt}%
\pgfpathmoveto{\pgfqpoint{3.305000in}{0.499444in}}%
\pgfpathlineto{\pgfqpoint{3.368409in}{0.499444in}}%
\pgfpathlineto{\pgfqpoint{3.368409in}{0.509077in}}%
\pgfpathlineto{\pgfqpoint{3.305000in}{0.509077in}}%
\pgfpathlineto{\pgfqpoint{3.305000in}{0.499444in}}%
\pgfpathclose%
\pgfusepath{stroke}%
\end{pgfscope}%
\begin{pgfscope}%
\pgfpathrectangle{\pgfqpoint{0.515000in}{0.499444in}}{\pgfqpoint{3.487500in}{1.155000in}}%
\pgfusepath{clip}%
\pgfsetbuttcap%
\pgfsetmiterjoin%
\pgfsetlinewidth{1.003750pt}%
\definecolor{currentstroke}{rgb}{0.000000,0.000000,0.000000}%
\pgfsetstrokecolor{currentstroke}%
\pgfsetdash{}{0pt}%
\pgfpathmoveto{\pgfqpoint{3.463523in}{0.499444in}}%
\pgfpathlineto{\pgfqpoint{3.526932in}{0.499444in}}%
\pgfpathlineto{\pgfqpoint{3.526932in}{0.502940in}}%
\pgfpathlineto{\pgfqpoint{3.463523in}{0.502940in}}%
\pgfpathlineto{\pgfqpoint{3.463523in}{0.499444in}}%
\pgfpathclose%
\pgfusepath{stroke}%
\end{pgfscope}%
\begin{pgfscope}%
\pgfpathrectangle{\pgfqpoint{0.515000in}{0.499444in}}{\pgfqpoint{3.487500in}{1.155000in}}%
\pgfusepath{clip}%
\pgfsetbuttcap%
\pgfsetmiterjoin%
\pgfsetlinewidth{1.003750pt}%
\definecolor{currentstroke}{rgb}{0.000000,0.000000,0.000000}%
\pgfsetstrokecolor{currentstroke}%
\pgfsetdash{}{0pt}%
\pgfpathmoveto{\pgfqpoint{3.622046in}{0.499444in}}%
\pgfpathlineto{\pgfqpoint{3.685455in}{0.499444in}}%
\pgfpathlineto{\pgfqpoint{3.685455in}{0.500486in}}%
\pgfpathlineto{\pgfqpoint{3.622046in}{0.500486in}}%
\pgfpathlineto{\pgfqpoint{3.622046in}{0.499444in}}%
\pgfpathclose%
\pgfusepath{stroke}%
\end{pgfscope}%
\begin{pgfscope}%
\pgfpathrectangle{\pgfqpoint{0.515000in}{0.499444in}}{\pgfqpoint{3.487500in}{1.155000in}}%
\pgfusepath{clip}%
\pgfsetbuttcap%
\pgfsetmiterjoin%
\pgfsetlinewidth{1.003750pt}%
\definecolor{currentstroke}{rgb}{0.000000,0.000000,0.000000}%
\pgfsetstrokecolor{currentstroke}%
\pgfsetdash{}{0pt}%
\pgfpathmoveto{\pgfqpoint{3.780568in}{0.499444in}}%
\pgfpathlineto{\pgfqpoint{3.843978in}{0.499444in}}%
\pgfpathlineto{\pgfqpoint{3.843978in}{0.499519in}}%
\pgfpathlineto{\pgfqpoint{3.780568in}{0.499519in}}%
\pgfpathlineto{\pgfqpoint{3.780568in}{0.499444in}}%
\pgfpathclose%
\pgfusepath{stroke}%
\end{pgfscope}%
\begin{pgfscope}%
\pgfpathrectangle{\pgfqpoint{0.515000in}{0.499444in}}{\pgfqpoint{3.487500in}{1.155000in}}%
\pgfusepath{clip}%
\pgfsetbuttcap%
\pgfsetmiterjoin%
\definecolor{currentfill}{rgb}{0.000000,0.000000,0.000000}%
\pgfsetfillcolor{currentfill}%
\pgfsetlinewidth{0.000000pt}%
\definecolor{currentstroke}{rgb}{0.000000,0.000000,0.000000}%
\pgfsetstrokecolor{currentstroke}%
\pgfsetstrokeopacity{0.000000}%
\pgfsetdash{}{0pt}%
\pgfpathmoveto{\pgfqpoint{0.673523in}{0.499444in}}%
\pgfpathlineto{\pgfqpoint{0.736932in}{0.499444in}}%
\pgfpathlineto{\pgfqpoint{0.736932in}{0.499444in}}%
\pgfpathlineto{\pgfqpoint{0.673523in}{0.499444in}}%
\pgfpathlineto{\pgfqpoint{0.673523in}{0.499444in}}%
\pgfpathclose%
\pgfusepath{fill}%
\end{pgfscope}%
\begin{pgfscope}%
\pgfpathrectangle{\pgfqpoint{0.515000in}{0.499444in}}{\pgfqpoint{3.487500in}{1.155000in}}%
\pgfusepath{clip}%
\pgfsetbuttcap%
\pgfsetmiterjoin%
\definecolor{currentfill}{rgb}{0.000000,0.000000,0.000000}%
\pgfsetfillcolor{currentfill}%
\pgfsetlinewidth{0.000000pt}%
\definecolor{currentstroke}{rgb}{0.000000,0.000000,0.000000}%
\pgfsetstrokecolor{currentstroke}%
\pgfsetstrokeopacity{0.000000}%
\pgfsetdash{}{0pt}%
\pgfpathmoveto{\pgfqpoint{0.832046in}{0.499444in}}%
\pgfpathlineto{\pgfqpoint{0.895455in}{0.499444in}}%
\pgfpathlineto{\pgfqpoint{0.895455in}{0.499481in}}%
\pgfpathlineto{\pgfqpoint{0.832046in}{0.499481in}}%
\pgfpathlineto{\pgfqpoint{0.832046in}{0.499444in}}%
\pgfpathclose%
\pgfusepath{fill}%
\end{pgfscope}%
\begin{pgfscope}%
\pgfpathrectangle{\pgfqpoint{0.515000in}{0.499444in}}{\pgfqpoint{3.487500in}{1.155000in}}%
\pgfusepath{clip}%
\pgfsetbuttcap%
\pgfsetmiterjoin%
\definecolor{currentfill}{rgb}{0.000000,0.000000,0.000000}%
\pgfsetfillcolor{currentfill}%
\pgfsetlinewidth{0.000000pt}%
\definecolor{currentstroke}{rgb}{0.000000,0.000000,0.000000}%
\pgfsetstrokecolor{currentstroke}%
\pgfsetstrokeopacity{0.000000}%
\pgfsetdash{}{0pt}%
\pgfpathmoveto{\pgfqpoint{0.990568in}{0.499444in}}%
\pgfpathlineto{\pgfqpoint{1.053978in}{0.499444in}}%
\pgfpathlineto{\pgfqpoint{1.053978in}{0.499556in}}%
\pgfpathlineto{\pgfqpoint{0.990568in}{0.499556in}}%
\pgfpathlineto{\pgfqpoint{0.990568in}{0.499444in}}%
\pgfpathclose%
\pgfusepath{fill}%
\end{pgfscope}%
\begin{pgfscope}%
\pgfpathrectangle{\pgfqpoint{0.515000in}{0.499444in}}{\pgfqpoint{3.487500in}{1.155000in}}%
\pgfusepath{clip}%
\pgfsetbuttcap%
\pgfsetmiterjoin%
\definecolor{currentfill}{rgb}{0.000000,0.000000,0.000000}%
\pgfsetfillcolor{currentfill}%
\pgfsetlinewidth{0.000000pt}%
\definecolor{currentstroke}{rgb}{0.000000,0.000000,0.000000}%
\pgfsetstrokecolor{currentstroke}%
\pgfsetstrokeopacity{0.000000}%
\pgfsetdash{}{0pt}%
\pgfpathmoveto{\pgfqpoint{1.149091in}{0.499444in}}%
\pgfpathlineto{\pgfqpoint{1.212500in}{0.499444in}}%
\pgfpathlineto{\pgfqpoint{1.212500in}{0.499779in}}%
\pgfpathlineto{\pgfqpoint{1.149091in}{0.499779in}}%
\pgfpathlineto{\pgfqpoint{1.149091in}{0.499444in}}%
\pgfpathclose%
\pgfusepath{fill}%
\end{pgfscope}%
\begin{pgfscope}%
\pgfpathrectangle{\pgfqpoint{0.515000in}{0.499444in}}{\pgfqpoint{3.487500in}{1.155000in}}%
\pgfusepath{clip}%
\pgfsetbuttcap%
\pgfsetmiterjoin%
\definecolor{currentfill}{rgb}{0.000000,0.000000,0.000000}%
\pgfsetfillcolor{currentfill}%
\pgfsetlinewidth{0.000000pt}%
\definecolor{currentstroke}{rgb}{0.000000,0.000000,0.000000}%
\pgfsetstrokecolor{currentstroke}%
\pgfsetstrokeopacity{0.000000}%
\pgfsetdash{}{0pt}%
\pgfpathmoveto{\pgfqpoint{1.307614in}{0.499444in}}%
\pgfpathlineto{\pgfqpoint{1.371023in}{0.499444in}}%
\pgfpathlineto{\pgfqpoint{1.371023in}{0.501676in}}%
\pgfpathlineto{\pgfqpoint{1.307614in}{0.501676in}}%
\pgfpathlineto{\pgfqpoint{1.307614in}{0.499444in}}%
\pgfpathclose%
\pgfusepath{fill}%
\end{pgfscope}%
\begin{pgfscope}%
\pgfpathrectangle{\pgfqpoint{0.515000in}{0.499444in}}{\pgfqpoint{3.487500in}{1.155000in}}%
\pgfusepath{clip}%
\pgfsetbuttcap%
\pgfsetmiterjoin%
\definecolor{currentfill}{rgb}{0.000000,0.000000,0.000000}%
\pgfsetfillcolor{currentfill}%
\pgfsetlinewidth{0.000000pt}%
\definecolor{currentstroke}{rgb}{0.000000,0.000000,0.000000}%
\pgfsetstrokecolor{currentstroke}%
\pgfsetstrokeopacity{0.000000}%
\pgfsetdash{}{0pt}%
\pgfpathmoveto{\pgfqpoint{1.466137in}{0.499444in}}%
\pgfpathlineto{\pgfqpoint{1.529546in}{0.499444in}}%
\pgfpathlineto{\pgfqpoint{1.529546in}{0.507218in}}%
\pgfpathlineto{\pgfqpoint{1.466137in}{0.507218in}}%
\pgfpathlineto{\pgfqpoint{1.466137in}{0.499444in}}%
\pgfpathclose%
\pgfusepath{fill}%
\end{pgfscope}%
\begin{pgfscope}%
\pgfpathrectangle{\pgfqpoint{0.515000in}{0.499444in}}{\pgfqpoint{3.487500in}{1.155000in}}%
\pgfusepath{clip}%
\pgfsetbuttcap%
\pgfsetmiterjoin%
\definecolor{currentfill}{rgb}{0.000000,0.000000,0.000000}%
\pgfsetfillcolor{currentfill}%
\pgfsetlinewidth{0.000000pt}%
\definecolor{currentstroke}{rgb}{0.000000,0.000000,0.000000}%
\pgfsetstrokecolor{currentstroke}%
\pgfsetstrokeopacity{0.000000}%
\pgfsetdash{}{0pt}%
\pgfpathmoveto{\pgfqpoint{1.624659in}{0.499444in}}%
\pgfpathlineto{\pgfqpoint{1.688068in}{0.499444in}}%
\pgfpathlineto{\pgfqpoint{1.688068in}{0.522579in}}%
\pgfpathlineto{\pgfqpoint{1.624659in}{0.522579in}}%
\pgfpathlineto{\pgfqpoint{1.624659in}{0.499444in}}%
\pgfpathclose%
\pgfusepath{fill}%
\end{pgfscope}%
\begin{pgfscope}%
\pgfpathrectangle{\pgfqpoint{0.515000in}{0.499444in}}{\pgfqpoint{3.487500in}{1.155000in}}%
\pgfusepath{clip}%
\pgfsetbuttcap%
\pgfsetmiterjoin%
\definecolor{currentfill}{rgb}{0.000000,0.000000,0.000000}%
\pgfsetfillcolor{currentfill}%
\pgfsetlinewidth{0.000000pt}%
\definecolor{currentstroke}{rgb}{0.000000,0.000000,0.000000}%
\pgfsetstrokecolor{currentstroke}%
\pgfsetstrokeopacity{0.000000}%
\pgfsetdash{}{0pt}%
\pgfpathmoveto{\pgfqpoint{1.783182in}{0.499444in}}%
\pgfpathlineto{\pgfqpoint{1.846591in}{0.499444in}}%
\pgfpathlineto{\pgfqpoint{1.846591in}{0.553375in}}%
\pgfpathlineto{\pgfqpoint{1.783182in}{0.553375in}}%
\pgfpathlineto{\pgfqpoint{1.783182in}{0.499444in}}%
\pgfpathclose%
\pgfusepath{fill}%
\end{pgfscope}%
\begin{pgfscope}%
\pgfpathrectangle{\pgfqpoint{0.515000in}{0.499444in}}{\pgfqpoint{3.487500in}{1.155000in}}%
\pgfusepath{clip}%
\pgfsetbuttcap%
\pgfsetmiterjoin%
\definecolor{currentfill}{rgb}{0.000000,0.000000,0.000000}%
\pgfsetfillcolor{currentfill}%
\pgfsetlinewidth{0.000000pt}%
\definecolor{currentstroke}{rgb}{0.000000,0.000000,0.000000}%
\pgfsetstrokecolor{currentstroke}%
\pgfsetstrokeopacity{0.000000}%
\pgfsetdash{}{0pt}%
\pgfpathmoveto{\pgfqpoint{1.941705in}{0.499444in}}%
\pgfpathlineto{\pgfqpoint{2.005114in}{0.499444in}}%
\pgfpathlineto{\pgfqpoint{2.005114in}{0.601615in}}%
\pgfpathlineto{\pgfqpoint{1.941705in}{0.601615in}}%
\pgfpathlineto{\pgfqpoint{1.941705in}{0.499444in}}%
\pgfpathclose%
\pgfusepath{fill}%
\end{pgfscope}%
\begin{pgfscope}%
\pgfpathrectangle{\pgfqpoint{0.515000in}{0.499444in}}{\pgfqpoint{3.487500in}{1.155000in}}%
\pgfusepath{clip}%
\pgfsetbuttcap%
\pgfsetmiterjoin%
\definecolor{currentfill}{rgb}{0.000000,0.000000,0.000000}%
\pgfsetfillcolor{currentfill}%
\pgfsetlinewidth{0.000000pt}%
\definecolor{currentstroke}{rgb}{0.000000,0.000000,0.000000}%
\pgfsetstrokecolor{currentstroke}%
\pgfsetstrokeopacity{0.000000}%
\pgfsetdash{}{0pt}%
\pgfpathmoveto{\pgfqpoint{2.100228in}{0.499444in}}%
\pgfpathlineto{\pgfqpoint{2.163637in}{0.499444in}}%
\pgfpathlineto{\pgfqpoint{2.163637in}{0.655806in}}%
\pgfpathlineto{\pgfqpoint{2.100228in}{0.655806in}}%
\pgfpathlineto{\pgfqpoint{2.100228in}{0.499444in}}%
\pgfpathclose%
\pgfusepath{fill}%
\end{pgfscope}%
\begin{pgfscope}%
\pgfpathrectangle{\pgfqpoint{0.515000in}{0.499444in}}{\pgfqpoint{3.487500in}{1.155000in}}%
\pgfusepath{clip}%
\pgfsetbuttcap%
\pgfsetmiterjoin%
\definecolor{currentfill}{rgb}{0.000000,0.000000,0.000000}%
\pgfsetfillcolor{currentfill}%
\pgfsetlinewidth{0.000000pt}%
\definecolor{currentstroke}{rgb}{0.000000,0.000000,0.000000}%
\pgfsetstrokecolor{currentstroke}%
\pgfsetstrokeopacity{0.000000}%
\pgfsetdash{}{0pt}%
\pgfpathmoveto{\pgfqpoint{2.258750in}{0.499444in}}%
\pgfpathlineto{\pgfqpoint{2.322159in}{0.499444in}}%
\pgfpathlineto{\pgfqpoint{2.322159in}{0.690880in}}%
\pgfpathlineto{\pgfqpoint{2.258750in}{0.690880in}}%
\pgfpathlineto{\pgfqpoint{2.258750in}{0.499444in}}%
\pgfpathclose%
\pgfusepath{fill}%
\end{pgfscope}%
\begin{pgfscope}%
\pgfpathrectangle{\pgfqpoint{0.515000in}{0.499444in}}{\pgfqpoint{3.487500in}{1.155000in}}%
\pgfusepath{clip}%
\pgfsetbuttcap%
\pgfsetmiterjoin%
\definecolor{currentfill}{rgb}{0.000000,0.000000,0.000000}%
\pgfsetfillcolor{currentfill}%
\pgfsetlinewidth{0.000000pt}%
\definecolor{currentstroke}{rgb}{0.000000,0.000000,0.000000}%
\pgfsetstrokecolor{currentstroke}%
\pgfsetstrokeopacity{0.000000}%
\pgfsetdash{}{0pt}%
\pgfpathmoveto{\pgfqpoint{2.417273in}{0.499444in}}%
\pgfpathlineto{\pgfqpoint{2.480682in}{0.499444in}}%
\pgfpathlineto{\pgfqpoint{2.480682in}{0.690173in}}%
\pgfpathlineto{\pgfqpoint{2.417273in}{0.690173in}}%
\pgfpathlineto{\pgfqpoint{2.417273in}{0.499444in}}%
\pgfpathclose%
\pgfusepath{fill}%
\end{pgfscope}%
\begin{pgfscope}%
\pgfpathrectangle{\pgfqpoint{0.515000in}{0.499444in}}{\pgfqpoint{3.487500in}{1.155000in}}%
\pgfusepath{clip}%
\pgfsetbuttcap%
\pgfsetmiterjoin%
\definecolor{currentfill}{rgb}{0.000000,0.000000,0.000000}%
\pgfsetfillcolor{currentfill}%
\pgfsetlinewidth{0.000000pt}%
\definecolor{currentstroke}{rgb}{0.000000,0.000000,0.000000}%
\pgfsetstrokecolor{currentstroke}%
\pgfsetstrokeopacity{0.000000}%
\pgfsetdash{}{0pt}%
\pgfpathmoveto{\pgfqpoint{2.575796in}{0.499444in}}%
\pgfpathlineto{\pgfqpoint{2.639205in}{0.499444in}}%
\pgfpathlineto{\pgfqpoint{2.639205in}{0.678196in}}%
\pgfpathlineto{\pgfqpoint{2.575796in}{0.678196in}}%
\pgfpathlineto{\pgfqpoint{2.575796in}{0.499444in}}%
\pgfpathclose%
\pgfusepath{fill}%
\end{pgfscope}%
\begin{pgfscope}%
\pgfpathrectangle{\pgfqpoint{0.515000in}{0.499444in}}{\pgfqpoint{3.487500in}{1.155000in}}%
\pgfusepath{clip}%
\pgfsetbuttcap%
\pgfsetmiterjoin%
\definecolor{currentfill}{rgb}{0.000000,0.000000,0.000000}%
\pgfsetfillcolor{currentfill}%
\pgfsetlinewidth{0.000000pt}%
\definecolor{currentstroke}{rgb}{0.000000,0.000000,0.000000}%
\pgfsetstrokecolor{currentstroke}%
\pgfsetstrokeopacity{0.000000}%
\pgfsetdash{}{0pt}%
\pgfpathmoveto{\pgfqpoint{2.734318in}{0.499444in}}%
\pgfpathlineto{\pgfqpoint{2.797728in}{0.499444in}}%
\pgfpathlineto{\pgfqpoint{2.797728in}{0.637693in}}%
\pgfpathlineto{\pgfqpoint{2.734318in}{0.637693in}}%
\pgfpathlineto{\pgfqpoint{2.734318in}{0.499444in}}%
\pgfpathclose%
\pgfusepath{fill}%
\end{pgfscope}%
\begin{pgfscope}%
\pgfpathrectangle{\pgfqpoint{0.515000in}{0.499444in}}{\pgfqpoint{3.487500in}{1.155000in}}%
\pgfusepath{clip}%
\pgfsetbuttcap%
\pgfsetmiterjoin%
\definecolor{currentfill}{rgb}{0.000000,0.000000,0.000000}%
\pgfsetfillcolor{currentfill}%
\pgfsetlinewidth{0.000000pt}%
\definecolor{currentstroke}{rgb}{0.000000,0.000000,0.000000}%
\pgfsetstrokecolor{currentstroke}%
\pgfsetstrokeopacity{0.000000}%
\pgfsetdash{}{0pt}%
\pgfpathmoveto{\pgfqpoint{2.892841in}{0.499444in}}%
\pgfpathlineto{\pgfqpoint{2.956250in}{0.499444in}}%
\pgfpathlineto{\pgfqpoint{2.956250in}{0.598267in}}%
\pgfpathlineto{\pgfqpoint{2.892841in}{0.598267in}}%
\pgfpathlineto{\pgfqpoint{2.892841in}{0.499444in}}%
\pgfpathclose%
\pgfusepath{fill}%
\end{pgfscope}%
\begin{pgfscope}%
\pgfpathrectangle{\pgfqpoint{0.515000in}{0.499444in}}{\pgfqpoint{3.487500in}{1.155000in}}%
\pgfusepath{clip}%
\pgfsetbuttcap%
\pgfsetmiterjoin%
\definecolor{currentfill}{rgb}{0.000000,0.000000,0.000000}%
\pgfsetfillcolor{currentfill}%
\pgfsetlinewidth{0.000000pt}%
\definecolor{currentstroke}{rgb}{0.000000,0.000000,0.000000}%
\pgfsetstrokecolor{currentstroke}%
\pgfsetstrokeopacity{0.000000}%
\pgfsetdash{}{0pt}%
\pgfpathmoveto{\pgfqpoint{3.051364in}{0.499444in}}%
\pgfpathlineto{\pgfqpoint{3.114773in}{0.499444in}}%
\pgfpathlineto{\pgfqpoint{3.114773in}{0.556202in}}%
\pgfpathlineto{\pgfqpoint{3.051364in}{0.556202in}}%
\pgfpathlineto{\pgfqpoint{3.051364in}{0.499444in}}%
\pgfpathclose%
\pgfusepath{fill}%
\end{pgfscope}%
\begin{pgfscope}%
\pgfpathrectangle{\pgfqpoint{0.515000in}{0.499444in}}{\pgfqpoint{3.487500in}{1.155000in}}%
\pgfusepath{clip}%
\pgfsetbuttcap%
\pgfsetmiterjoin%
\definecolor{currentfill}{rgb}{0.000000,0.000000,0.000000}%
\pgfsetfillcolor{currentfill}%
\pgfsetlinewidth{0.000000pt}%
\definecolor{currentstroke}{rgb}{0.000000,0.000000,0.000000}%
\pgfsetstrokecolor{currentstroke}%
\pgfsetstrokeopacity{0.000000}%
\pgfsetdash{}{0pt}%
\pgfpathmoveto{\pgfqpoint{3.209887in}{0.499444in}}%
\pgfpathlineto{\pgfqpoint{3.273296in}{0.499444in}}%
\pgfpathlineto{\pgfqpoint{3.273296in}{0.528716in}}%
\pgfpathlineto{\pgfqpoint{3.209887in}{0.528716in}}%
\pgfpathlineto{\pgfqpoint{3.209887in}{0.499444in}}%
\pgfpathclose%
\pgfusepath{fill}%
\end{pgfscope}%
\begin{pgfscope}%
\pgfpathrectangle{\pgfqpoint{0.515000in}{0.499444in}}{\pgfqpoint{3.487500in}{1.155000in}}%
\pgfusepath{clip}%
\pgfsetbuttcap%
\pgfsetmiterjoin%
\definecolor{currentfill}{rgb}{0.000000,0.000000,0.000000}%
\pgfsetfillcolor{currentfill}%
\pgfsetlinewidth{0.000000pt}%
\definecolor{currentstroke}{rgb}{0.000000,0.000000,0.000000}%
\pgfsetstrokecolor{currentstroke}%
\pgfsetstrokeopacity{0.000000}%
\pgfsetdash{}{0pt}%
\pgfpathmoveto{\pgfqpoint{3.368409in}{0.499444in}}%
\pgfpathlineto{\pgfqpoint{3.431818in}{0.499444in}}%
\pgfpathlineto{\pgfqpoint{3.431818in}{0.516367in}}%
\pgfpathlineto{\pgfqpoint{3.368409in}{0.516367in}}%
\pgfpathlineto{\pgfqpoint{3.368409in}{0.499444in}}%
\pgfpathclose%
\pgfusepath{fill}%
\end{pgfscope}%
\begin{pgfscope}%
\pgfpathrectangle{\pgfqpoint{0.515000in}{0.499444in}}{\pgfqpoint{3.487500in}{1.155000in}}%
\pgfusepath{clip}%
\pgfsetbuttcap%
\pgfsetmiterjoin%
\definecolor{currentfill}{rgb}{0.000000,0.000000,0.000000}%
\pgfsetfillcolor{currentfill}%
\pgfsetlinewidth{0.000000pt}%
\definecolor{currentstroke}{rgb}{0.000000,0.000000,0.000000}%
\pgfsetstrokecolor{currentstroke}%
\pgfsetstrokeopacity{0.000000}%
\pgfsetdash{}{0pt}%
\pgfpathmoveto{\pgfqpoint{3.526932in}{0.499444in}}%
\pgfpathlineto{\pgfqpoint{3.590341in}{0.499444in}}%
\pgfpathlineto{\pgfqpoint{3.590341in}{0.507143in}}%
\pgfpathlineto{\pgfqpoint{3.526932in}{0.507143in}}%
\pgfpathlineto{\pgfqpoint{3.526932in}{0.499444in}}%
\pgfpathclose%
\pgfusepath{fill}%
\end{pgfscope}%
\begin{pgfscope}%
\pgfpathrectangle{\pgfqpoint{0.515000in}{0.499444in}}{\pgfqpoint{3.487500in}{1.155000in}}%
\pgfusepath{clip}%
\pgfsetbuttcap%
\pgfsetmiterjoin%
\definecolor{currentfill}{rgb}{0.000000,0.000000,0.000000}%
\pgfsetfillcolor{currentfill}%
\pgfsetlinewidth{0.000000pt}%
\definecolor{currentstroke}{rgb}{0.000000,0.000000,0.000000}%
\pgfsetstrokecolor{currentstroke}%
\pgfsetstrokeopacity{0.000000}%
\pgfsetdash{}{0pt}%
\pgfpathmoveto{\pgfqpoint{3.685455in}{0.499444in}}%
\pgfpathlineto{\pgfqpoint{3.748864in}{0.499444in}}%
\pgfpathlineto{\pgfqpoint{3.748864in}{0.502271in}}%
\pgfpathlineto{\pgfqpoint{3.685455in}{0.502271in}}%
\pgfpathlineto{\pgfqpoint{3.685455in}{0.499444in}}%
\pgfpathclose%
\pgfusepath{fill}%
\end{pgfscope}%
\begin{pgfscope}%
\pgfpathrectangle{\pgfqpoint{0.515000in}{0.499444in}}{\pgfqpoint{3.487500in}{1.155000in}}%
\pgfusepath{clip}%
\pgfsetbuttcap%
\pgfsetmiterjoin%
\definecolor{currentfill}{rgb}{0.000000,0.000000,0.000000}%
\pgfsetfillcolor{currentfill}%
\pgfsetlinewidth{0.000000pt}%
\definecolor{currentstroke}{rgb}{0.000000,0.000000,0.000000}%
\pgfsetstrokecolor{currentstroke}%
\pgfsetstrokeopacity{0.000000}%
\pgfsetdash{}{0pt}%
\pgfpathmoveto{\pgfqpoint{3.843978in}{0.499444in}}%
\pgfpathlineto{\pgfqpoint{3.907387in}{0.499444in}}%
\pgfpathlineto{\pgfqpoint{3.907387in}{0.499965in}}%
\pgfpathlineto{\pgfqpoint{3.843978in}{0.499965in}}%
\pgfpathlineto{\pgfqpoint{3.843978in}{0.499444in}}%
\pgfpathclose%
\pgfusepath{fill}%
\end{pgfscope}%
\begin{pgfscope}%
\pgfsetbuttcap%
\pgfsetroundjoin%
\definecolor{currentfill}{rgb}{0.000000,0.000000,0.000000}%
\pgfsetfillcolor{currentfill}%
\pgfsetlinewidth{0.803000pt}%
\definecolor{currentstroke}{rgb}{0.000000,0.000000,0.000000}%
\pgfsetstrokecolor{currentstroke}%
\pgfsetdash{}{0pt}%
\pgfsys@defobject{currentmarker}{\pgfqpoint{0.000000in}{-0.048611in}}{\pgfqpoint{0.000000in}{0.000000in}}{%
\pgfpathmoveto{\pgfqpoint{0.000000in}{0.000000in}}%
\pgfpathlineto{\pgfqpoint{0.000000in}{-0.048611in}}%
\pgfusepath{stroke,fill}%
}%
\begin{pgfscope}%
\pgfsys@transformshift{0.515000in}{0.499444in}%
\pgfsys@useobject{currentmarker}{}%
\end{pgfscope}%
\end{pgfscope}%
\begin{pgfscope}%
\pgfsetbuttcap%
\pgfsetroundjoin%
\definecolor{currentfill}{rgb}{0.000000,0.000000,0.000000}%
\pgfsetfillcolor{currentfill}%
\pgfsetlinewidth{0.803000pt}%
\definecolor{currentstroke}{rgb}{0.000000,0.000000,0.000000}%
\pgfsetstrokecolor{currentstroke}%
\pgfsetdash{}{0pt}%
\pgfsys@defobject{currentmarker}{\pgfqpoint{0.000000in}{-0.048611in}}{\pgfqpoint{0.000000in}{0.000000in}}{%
\pgfpathmoveto{\pgfqpoint{0.000000in}{0.000000in}}%
\pgfpathlineto{\pgfqpoint{0.000000in}{-0.048611in}}%
\pgfusepath{stroke,fill}%
}%
\begin{pgfscope}%
\pgfsys@transformshift{0.673523in}{0.499444in}%
\pgfsys@useobject{currentmarker}{}%
\end{pgfscope}%
\end{pgfscope}%
\begin{pgfscope}%
\definecolor{textcolor}{rgb}{0.000000,0.000000,0.000000}%
\pgfsetstrokecolor{textcolor}%
\pgfsetfillcolor{textcolor}%
\pgftext[x=0.673523in,y=0.402222in,,top]{\color{textcolor}\rmfamily\fontsize{10.000000}{12.000000}\selectfont 0.0}%
\end{pgfscope}%
\begin{pgfscope}%
\pgfsetbuttcap%
\pgfsetroundjoin%
\definecolor{currentfill}{rgb}{0.000000,0.000000,0.000000}%
\pgfsetfillcolor{currentfill}%
\pgfsetlinewidth{0.803000pt}%
\definecolor{currentstroke}{rgb}{0.000000,0.000000,0.000000}%
\pgfsetstrokecolor{currentstroke}%
\pgfsetdash{}{0pt}%
\pgfsys@defobject{currentmarker}{\pgfqpoint{0.000000in}{-0.048611in}}{\pgfqpoint{0.000000in}{0.000000in}}{%
\pgfpathmoveto{\pgfqpoint{0.000000in}{0.000000in}}%
\pgfpathlineto{\pgfqpoint{0.000000in}{-0.048611in}}%
\pgfusepath{stroke,fill}%
}%
\begin{pgfscope}%
\pgfsys@transformshift{0.832046in}{0.499444in}%
\pgfsys@useobject{currentmarker}{}%
\end{pgfscope}%
\end{pgfscope}%
\begin{pgfscope}%
\pgfsetbuttcap%
\pgfsetroundjoin%
\definecolor{currentfill}{rgb}{0.000000,0.000000,0.000000}%
\pgfsetfillcolor{currentfill}%
\pgfsetlinewidth{0.803000pt}%
\definecolor{currentstroke}{rgb}{0.000000,0.000000,0.000000}%
\pgfsetstrokecolor{currentstroke}%
\pgfsetdash{}{0pt}%
\pgfsys@defobject{currentmarker}{\pgfqpoint{0.000000in}{-0.048611in}}{\pgfqpoint{0.000000in}{0.000000in}}{%
\pgfpathmoveto{\pgfqpoint{0.000000in}{0.000000in}}%
\pgfpathlineto{\pgfqpoint{0.000000in}{-0.048611in}}%
\pgfusepath{stroke,fill}%
}%
\begin{pgfscope}%
\pgfsys@transformshift{0.990568in}{0.499444in}%
\pgfsys@useobject{currentmarker}{}%
\end{pgfscope}%
\end{pgfscope}%
\begin{pgfscope}%
\definecolor{textcolor}{rgb}{0.000000,0.000000,0.000000}%
\pgfsetstrokecolor{textcolor}%
\pgfsetfillcolor{textcolor}%
\pgftext[x=0.990568in,y=0.402222in,,top]{\color{textcolor}\rmfamily\fontsize{10.000000}{12.000000}\selectfont 0.1}%
\end{pgfscope}%
\begin{pgfscope}%
\pgfsetbuttcap%
\pgfsetroundjoin%
\definecolor{currentfill}{rgb}{0.000000,0.000000,0.000000}%
\pgfsetfillcolor{currentfill}%
\pgfsetlinewidth{0.803000pt}%
\definecolor{currentstroke}{rgb}{0.000000,0.000000,0.000000}%
\pgfsetstrokecolor{currentstroke}%
\pgfsetdash{}{0pt}%
\pgfsys@defobject{currentmarker}{\pgfqpoint{0.000000in}{-0.048611in}}{\pgfqpoint{0.000000in}{0.000000in}}{%
\pgfpathmoveto{\pgfqpoint{0.000000in}{0.000000in}}%
\pgfpathlineto{\pgfqpoint{0.000000in}{-0.048611in}}%
\pgfusepath{stroke,fill}%
}%
\begin{pgfscope}%
\pgfsys@transformshift{1.149091in}{0.499444in}%
\pgfsys@useobject{currentmarker}{}%
\end{pgfscope}%
\end{pgfscope}%
\begin{pgfscope}%
\pgfsetbuttcap%
\pgfsetroundjoin%
\definecolor{currentfill}{rgb}{0.000000,0.000000,0.000000}%
\pgfsetfillcolor{currentfill}%
\pgfsetlinewidth{0.803000pt}%
\definecolor{currentstroke}{rgb}{0.000000,0.000000,0.000000}%
\pgfsetstrokecolor{currentstroke}%
\pgfsetdash{}{0pt}%
\pgfsys@defobject{currentmarker}{\pgfqpoint{0.000000in}{-0.048611in}}{\pgfqpoint{0.000000in}{0.000000in}}{%
\pgfpathmoveto{\pgfqpoint{0.000000in}{0.000000in}}%
\pgfpathlineto{\pgfqpoint{0.000000in}{-0.048611in}}%
\pgfusepath{stroke,fill}%
}%
\begin{pgfscope}%
\pgfsys@transformshift{1.307614in}{0.499444in}%
\pgfsys@useobject{currentmarker}{}%
\end{pgfscope}%
\end{pgfscope}%
\begin{pgfscope}%
\definecolor{textcolor}{rgb}{0.000000,0.000000,0.000000}%
\pgfsetstrokecolor{textcolor}%
\pgfsetfillcolor{textcolor}%
\pgftext[x=1.307614in,y=0.402222in,,top]{\color{textcolor}\rmfamily\fontsize{10.000000}{12.000000}\selectfont 0.2}%
\end{pgfscope}%
\begin{pgfscope}%
\pgfsetbuttcap%
\pgfsetroundjoin%
\definecolor{currentfill}{rgb}{0.000000,0.000000,0.000000}%
\pgfsetfillcolor{currentfill}%
\pgfsetlinewidth{0.803000pt}%
\definecolor{currentstroke}{rgb}{0.000000,0.000000,0.000000}%
\pgfsetstrokecolor{currentstroke}%
\pgfsetdash{}{0pt}%
\pgfsys@defobject{currentmarker}{\pgfqpoint{0.000000in}{-0.048611in}}{\pgfqpoint{0.000000in}{0.000000in}}{%
\pgfpathmoveto{\pgfqpoint{0.000000in}{0.000000in}}%
\pgfpathlineto{\pgfqpoint{0.000000in}{-0.048611in}}%
\pgfusepath{stroke,fill}%
}%
\begin{pgfscope}%
\pgfsys@transformshift{1.466137in}{0.499444in}%
\pgfsys@useobject{currentmarker}{}%
\end{pgfscope}%
\end{pgfscope}%
\begin{pgfscope}%
\pgfsetbuttcap%
\pgfsetroundjoin%
\definecolor{currentfill}{rgb}{0.000000,0.000000,0.000000}%
\pgfsetfillcolor{currentfill}%
\pgfsetlinewidth{0.803000pt}%
\definecolor{currentstroke}{rgb}{0.000000,0.000000,0.000000}%
\pgfsetstrokecolor{currentstroke}%
\pgfsetdash{}{0pt}%
\pgfsys@defobject{currentmarker}{\pgfqpoint{0.000000in}{-0.048611in}}{\pgfqpoint{0.000000in}{0.000000in}}{%
\pgfpathmoveto{\pgfqpoint{0.000000in}{0.000000in}}%
\pgfpathlineto{\pgfqpoint{0.000000in}{-0.048611in}}%
\pgfusepath{stroke,fill}%
}%
\begin{pgfscope}%
\pgfsys@transformshift{1.624659in}{0.499444in}%
\pgfsys@useobject{currentmarker}{}%
\end{pgfscope}%
\end{pgfscope}%
\begin{pgfscope}%
\definecolor{textcolor}{rgb}{0.000000,0.000000,0.000000}%
\pgfsetstrokecolor{textcolor}%
\pgfsetfillcolor{textcolor}%
\pgftext[x=1.624659in,y=0.402222in,,top]{\color{textcolor}\rmfamily\fontsize{10.000000}{12.000000}\selectfont 0.3}%
\end{pgfscope}%
\begin{pgfscope}%
\pgfsetbuttcap%
\pgfsetroundjoin%
\definecolor{currentfill}{rgb}{0.000000,0.000000,0.000000}%
\pgfsetfillcolor{currentfill}%
\pgfsetlinewidth{0.803000pt}%
\definecolor{currentstroke}{rgb}{0.000000,0.000000,0.000000}%
\pgfsetstrokecolor{currentstroke}%
\pgfsetdash{}{0pt}%
\pgfsys@defobject{currentmarker}{\pgfqpoint{0.000000in}{-0.048611in}}{\pgfqpoint{0.000000in}{0.000000in}}{%
\pgfpathmoveto{\pgfqpoint{0.000000in}{0.000000in}}%
\pgfpathlineto{\pgfqpoint{0.000000in}{-0.048611in}}%
\pgfusepath{stroke,fill}%
}%
\begin{pgfscope}%
\pgfsys@transformshift{1.783182in}{0.499444in}%
\pgfsys@useobject{currentmarker}{}%
\end{pgfscope}%
\end{pgfscope}%
\begin{pgfscope}%
\pgfsetbuttcap%
\pgfsetroundjoin%
\definecolor{currentfill}{rgb}{0.000000,0.000000,0.000000}%
\pgfsetfillcolor{currentfill}%
\pgfsetlinewidth{0.803000pt}%
\definecolor{currentstroke}{rgb}{0.000000,0.000000,0.000000}%
\pgfsetstrokecolor{currentstroke}%
\pgfsetdash{}{0pt}%
\pgfsys@defobject{currentmarker}{\pgfqpoint{0.000000in}{-0.048611in}}{\pgfqpoint{0.000000in}{0.000000in}}{%
\pgfpathmoveto{\pgfqpoint{0.000000in}{0.000000in}}%
\pgfpathlineto{\pgfqpoint{0.000000in}{-0.048611in}}%
\pgfusepath{stroke,fill}%
}%
\begin{pgfscope}%
\pgfsys@transformshift{1.941705in}{0.499444in}%
\pgfsys@useobject{currentmarker}{}%
\end{pgfscope}%
\end{pgfscope}%
\begin{pgfscope}%
\definecolor{textcolor}{rgb}{0.000000,0.000000,0.000000}%
\pgfsetstrokecolor{textcolor}%
\pgfsetfillcolor{textcolor}%
\pgftext[x=1.941705in,y=0.402222in,,top]{\color{textcolor}\rmfamily\fontsize{10.000000}{12.000000}\selectfont 0.4}%
\end{pgfscope}%
\begin{pgfscope}%
\pgfsetbuttcap%
\pgfsetroundjoin%
\definecolor{currentfill}{rgb}{0.000000,0.000000,0.000000}%
\pgfsetfillcolor{currentfill}%
\pgfsetlinewidth{0.803000pt}%
\definecolor{currentstroke}{rgb}{0.000000,0.000000,0.000000}%
\pgfsetstrokecolor{currentstroke}%
\pgfsetdash{}{0pt}%
\pgfsys@defobject{currentmarker}{\pgfqpoint{0.000000in}{-0.048611in}}{\pgfqpoint{0.000000in}{0.000000in}}{%
\pgfpathmoveto{\pgfqpoint{0.000000in}{0.000000in}}%
\pgfpathlineto{\pgfqpoint{0.000000in}{-0.048611in}}%
\pgfusepath{stroke,fill}%
}%
\begin{pgfscope}%
\pgfsys@transformshift{2.100228in}{0.499444in}%
\pgfsys@useobject{currentmarker}{}%
\end{pgfscope}%
\end{pgfscope}%
\begin{pgfscope}%
\pgfsetbuttcap%
\pgfsetroundjoin%
\definecolor{currentfill}{rgb}{0.000000,0.000000,0.000000}%
\pgfsetfillcolor{currentfill}%
\pgfsetlinewidth{0.803000pt}%
\definecolor{currentstroke}{rgb}{0.000000,0.000000,0.000000}%
\pgfsetstrokecolor{currentstroke}%
\pgfsetdash{}{0pt}%
\pgfsys@defobject{currentmarker}{\pgfqpoint{0.000000in}{-0.048611in}}{\pgfqpoint{0.000000in}{0.000000in}}{%
\pgfpathmoveto{\pgfqpoint{0.000000in}{0.000000in}}%
\pgfpathlineto{\pgfqpoint{0.000000in}{-0.048611in}}%
\pgfusepath{stroke,fill}%
}%
\begin{pgfscope}%
\pgfsys@transformshift{2.258750in}{0.499444in}%
\pgfsys@useobject{currentmarker}{}%
\end{pgfscope}%
\end{pgfscope}%
\begin{pgfscope}%
\definecolor{textcolor}{rgb}{0.000000,0.000000,0.000000}%
\pgfsetstrokecolor{textcolor}%
\pgfsetfillcolor{textcolor}%
\pgftext[x=2.258750in,y=0.402222in,,top]{\color{textcolor}\rmfamily\fontsize{10.000000}{12.000000}\selectfont 0.5}%
\end{pgfscope}%
\begin{pgfscope}%
\pgfsetbuttcap%
\pgfsetroundjoin%
\definecolor{currentfill}{rgb}{0.000000,0.000000,0.000000}%
\pgfsetfillcolor{currentfill}%
\pgfsetlinewidth{0.803000pt}%
\definecolor{currentstroke}{rgb}{0.000000,0.000000,0.000000}%
\pgfsetstrokecolor{currentstroke}%
\pgfsetdash{}{0pt}%
\pgfsys@defobject{currentmarker}{\pgfqpoint{0.000000in}{-0.048611in}}{\pgfqpoint{0.000000in}{0.000000in}}{%
\pgfpathmoveto{\pgfqpoint{0.000000in}{0.000000in}}%
\pgfpathlineto{\pgfqpoint{0.000000in}{-0.048611in}}%
\pgfusepath{stroke,fill}%
}%
\begin{pgfscope}%
\pgfsys@transformshift{2.417273in}{0.499444in}%
\pgfsys@useobject{currentmarker}{}%
\end{pgfscope}%
\end{pgfscope}%
\begin{pgfscope}%
\pgfsetbuttcap%
\pgfsetroundjoin%
\definecolor{currentfill}{rgb}{0.000000,0.000000,0.000000}%
\pgfsetfillcolor{currentfill}%
\pgfsetlinewidth{0.803000pt}%
\definecolor{currentstroke}{rgb}{0.000000,0.000000,0.000000}%
\pgfsetstrokecolor{currentstroke}%
\pgfsetdash{}{0pt}%
\pgfsys@defobject{currentmarker}{\pgfqpoint{0.000000in}{-0.048611in}}{\pgfqpoint{0.000000in}{0.000000in}}{%
\pgfpathmoveto{\pgfqpoint{0.000000in}{0.000000in}}%
\pgfpathlineto{\pgfqpoint{0.000000in}{-0.048611in}}%
\pgfusepath{stroke,fill}%
}%
\begin{pgfscope}%
\pgfsys@transformshift{2.575796in}{0.499444in}%
\pgfsys@useobject{currentmarker}{}%
\end{pgfscope}%
\end{pgfscope}%
\begin{pgfscope}%
\definecolor{textcolor}{rgb}{0.000000,0.000000,0.000000}%
\pgfsetstrokecolor{textcolor}%
\pgfsetfillcolor{textcolor}%
\pgftext[x=2.575796in,y=0.402222in,,top]{\color{textcolor}\rmfamily\fontsize{10.000000}{12.000000}\selectfont 0.6}%
\end{pgfscope}%
\begin{pgfscope}%
\pgfsetbuttcap%
\pgfsetroundjoin%
\definecolor{currentfill}{rgb}{0.000000,0.000000,0.000000}%
\pgfsetfillcolor{currentfill}%
\pgfsetlinewidth{0.803000pt}%
\definecolor{currentstroke}{rgb}{0.000000,0.000000,0.000000}%
\pgfsetstrokecolor{currentstroke}%
\pgfsetdash{}{0pt}%
\pgfsys@defobject{currentmarker}{\pgfqpoint{0.000000in}{-0.048611in}}{\pgfqpoint{0.000000in}{0.000000in}}{%
\pgfpathmoveto{\pgfqpoint{0.000000in}{0.000000in}}%
\pgfpathlineto{\pgfqpoint{0.000000in}{-0.048611in}}%
\pgfusepath{stroke,fill}%
}%
\begin{pgfscope}%
\pgfsys@transformshift{2.734318in}{0.499444in}%
\pgfsys@useobject{currentmarker}{}%
\end{pgfscope}%
\end{pgfscope}%
\begin{pgfscope}%
\pgfsetbuttcap%
\pgfsetroundjoin%
\definecolor{currentfill}{rgb}{0.000000,0.000000,0.000000}%
\pgfsetfillcolor{currentfill}%
\pgfsetlinewidth{0.803000pt}%
\definecolor{currentstroke}{rgb}{0.000000,0.000000,0.000000}%
\pgfsetstrokecolor{currentstroke}%
\pgfsetdash{}{0pt}%
\pgfsys@defobject{currentmarker}{\pgfqpoint{0.000000in}{-0.048611in}}{\pgfqpoint{0.000000in}{0.000000in}}{%
\pgfpathmoveto{\pgfqpoint{0.000000in}{0.000000in}}%
\pgfpathlineto{\pgfqpoint{0.000000in}{-0.048611in}}%
\pgfusepath{stroke,fill}%
}%
\begin{pgfscope}%
\pgfsys@transformshift{2.892841in}{0.499444in}%
\pgfsys@useobject{currentmarker}{}%
\end{pgfscope}%
\end{pgfscope}%
\begin{pgfscope}%
\definecolor{textcolor}{rgb}{0.000000,0.000000,0.000000}%
\pgfsetstrokecolor{textcolor}%
\pgfsetfillcolor{textcolor}%
\pgftext[x=2.892841in,y=0.402222in,,top]{\color{textcolor}\rmfamily\fontsize{10.000000}{12.000000}\selectfont 0.7}%
\end{pgfscope}%
\begin{pgfscope}%
\pgfsetbuttcap%
\pgfsetroundjoin%
\definecolor{currentfill}{rgb}{0.000000,0.000000,0.000000}%
\pgfsetfillcolor{currentfill}%
\pgfsetlinewidth{0.803000pt}%
\definecolor{currentstroke}{rgb}{0.000000,0.000000,0.000000}%
\pgfsetstrokecolor{currentstroke}%
\pgfsetdash{}{0pt}%
\pgfsys@defobject{currentmarker}{\pgfqpoint{0.000000in}{-0.048611in}}{\pgfqpoint{0.000000in}{0.000000in}}{%
\pgfpathmoveto{\pgfqpoint{0.000000in}{0.000000in}}%
\pgfpathlineto{\pgfqpoint{0.000000in}{-0.048611in}}%
\pgfusepath{stroke,fill}%
}%
\begin{pgfscope}%
\pgfsys@transformshift{3.051364in}{0.499444in}%
\pgfsys@useobject{currentmarker}{}%
\end{pgfscope}%
\end{pgfscope}%
\begin{pgfscope}%
\pgfsetbuttcap%
\pgfsetroundjoin%
\definecolor{currentfill}{rgb}{0.000000,0.000000,0.000000}%
\pgfsetfillcolor{currentfill}%
\pgfsetlinewidth{0.803000pt}%
\definecolor{currentstroke}{rgb}{0.000000,0.000000,0.000000}%
\pgfsetstrokecolor{currentstroke}%
\pgfsetdash{}{0pt}%
\pgfsys@defobject{currentmarker}{\pgfqpoint{0.000000in}{-0.048611in}}{\pgfqpoint{0.000000in}{0.000000in}}{%
\pgfpathmoveto{\pgfqpoint{0.000000in}{0.000000in}}%
\pgfpathlineto{\pgfqpoint{0.000000in}{-0.048611in}}%
\pgfusepath{stroke,fill}%
}%
\begin{pgfscope}%
\pgfsys@transformshift{3.209887in}{0.499444in}%
\pgfsys@useobject{currentmarker}{}%
\end{pgfscope}%
\end{pgfscope}%
\begin{pgfscope}%
\definecolor{textcolor}{rgb}{0.000000,0.000000,0.000000}%
\pgfsetstrokecolor{textcolor}%
\pgfsetfillcolor{textcolor}%
\pgftext[x=3.209887in,y=0.402222in,,top]{\color{textcolor}\rmfamily\fontsize{10.000000}{12.000000}\selectfont 0.8}%
\end{pgfscope}%
\begin{pgfscope}%
\pgfsetbuttcap%
\pgfsetroundjoin%
\definecolor{currentfill}{rgb}{0.000000,0.000000,0.000000}%
\pgfsetfillcolor{currentfill}%
\pgfsetlinewidth{0.803000pt}%
\definecolor{currentstroke}{rgb}{0.000000,0.000000,0.000000}%
\pgfsetstrokecolor{currentstroke}%
\pgfsetdash{}{0pt}%
\pgfsys@defobject{currentmarker}{\pgfqpoint{0.000000in}{-0.048611in}}{\pgfqpoint{0.000000in}{0.000000in}}{%
\pgfpathmoveto{\pgfqpoint{0.000000in}{0.000000in}}%
\pgfpathlineto{\pgfqpoint{0.000000in}{-0.048611in}}%
\pgfusepath{stroke,fill}%
}%
\begin{pgfscope}%
\pgfsys@transformshift{3.368409in}{0.499444in}%
\pgfsys@useobject{currentmarker}{}%
\end{pgfscope}%
\end{pgfscope}%
\begin{pgfscope}%
\pgfsetbuttcap%
\pgfsetroundjoin%
\definecolor{currentfill}{rgb}{0.000000,0.000000,0.000000}%
\pgfsetfillcolor{currentfill}%
\pgfsetlinewidth{0.803000pt}%
\definecolor{currentstroke}{rgb}{0.000000,0.000000,0.000000}%
\pgfsetstrokecolor{currentstroke}%
\pgfsetdash{}{0pt}%
\pgfsys@defobject{currentmarker}{\pgfqpoint{0.000000in}{-0.048611in}}{\pgfqpoint{0.000000in}{0.000000in}}{%
\pgfpathmoveto{\pgfqpoint{0.000000in}{0.000000in}}%
\pgfpathlineto{\pgfqpoint{0.000000in}{-0.048611in}}%
\pgfusepath{stroke,fill}%
}%
\begin{pgfscope}%
\pgfsys@transformshift{3.526932in}{0.499444in}%
\pgfsys@useobject{currentmarker}{}%
\end{pgfscope}%
\end{pgfscope}%
\begin{pgfscope}%
\definecolor{textcolor}{rgb}{0.000000,0.000000,0.000000}%
\pgfsetstrokecolor{textcolor}%
\pgfsetfillcolor{textcolor}%
\pgftext[x=3.526932in,y=0.402222in,,top]{\color{textcolor}\rmfamily\fontsize{10.000000}{12.000000}\selectfont 0.9}%
\end{pgfscope}%
\begin{pgfscope}%
\pgfsetbuttcap%
\pgfsetroundjoin%
\definecolor{currentfill}{rgb}{0.000000,0.000000,0.000000}%
\pgfsetfillcolor{currentfill}%
\pgfsetlinewidth{0.803000pt}%
\definecolor{currentstroke}{rgb}{0.000000,0.000000,0.000000}%
\pgfsetstrokecolor{currentstroke}%
\pgfsetdash{}{0pt}%
\pgfsys@defobject{currentmarker}{\pgfqpoint{0.000000in}{-0.048611in}}{\pgfqpoint{0.000000in}{0.000000in}}{%
\pgfpathmoveto{\pgfqpoint{0.000000in}{0.000000in}}%
\pgfpathlineto{\pgfqpoint{0.000000in}{-0.048611in}}%
\pgfusepath{stroke,fill}%
}%
\begin{pgfscope}%
\pgfsys@transformshift{3.685455in}{0.499444in}%
\pgfsys@useobject{currentmarker}{}%
\end{pgfscope}%
\end{pgfscope}%
\begin{pgfscope}%
\pgfsetbuttcap%
\pgfsetroundjoin%
\definecolor{currentfill}{rgb}{0.000000,0.000000,0.000000}%
\pgfsetfillcolor{currentfill}%
\pgfsetlinewidth{0.803000pt}%
\definecolor{currentstroke}{rgb}{0.000000,0.000000,0.000000}%
\pgfsetstrokecolor{currentstroke}%
\pgfsetdash{}{0pt}%
\pgfsys@defobject{currentmarker}{\pgfqpoint{0.000000in}{-0.048611in}}{\pgfqpoint{0.000000in}{0.000000in}}{%
\pgfpathmoveto{\pgfqpoint{0.000000in}{0.000000in}}%
\pgfpathlineto{\pgfqpoint{0.000000in}{-0.048611in}}%
\pgfusepath{stroke,fill}%
}%
\begin{pgfscope}%
\pgfsys@transformshift{3.843978in}{0.499444in}%
\pgfsys@useobject{currentmarker}{}%
\end{pgfscope}%
\end{pgfscope}%
\begin{pgfscope}%
\definecolor{textcolor}{rgb}{0.000000,0.000000,0.000000}%
\pgfsetstrokecolor{textcolor}%
\pgfsetfillcolor{textcolor}%
\pgftext[x=3.843978in,y=0.402222in,,top]{\color{textcolor}\rmfamily\fontsize{10.000000}{12.000000}\selectfont 1.0}%
\end{pgfscope}%
\begin{pgfscope}%
\pgfsetbuttcap%
\pgfsetroundjoin%
\definecolor{currentfill}{rgb}{0.000000,0.000000,0.000000}%
\pgfsetfillcolor{currentfill}%
\pgfsetlinewidth{0.803000pt}%
\definecolor{currentstroke}{rgb}{0.000000,0.000000,0.000000}%
\pgfsetstrokecolor{currentstroke}%
\pgfsetdash{}{0pt}%
\pgfsys@defobject{currentmarker}{\pgfqpoint{0.000000in}{-0.048611in}}{\pgfqpoint{0.000000in}{0.000000in}}{%
\pgfpathmoveto{\pgfqpoint{0.000000in}{0.000000in}}%
\pgfpathlineto{\pgfqpoint{0.000000in}{-0.048611in}}%
\pgfusepath{stroke,fill}%
}%
\begin{pgfscope}%
\pgfsys@transformshift{4.002500in}{0.499444in}%
\pgfsys@useobject{currentmarker}{}%
\end{pgfscope}%
\end{pgfscope}%
\begin{pgfscope}%
\definecolor{textcolor}{rgb}{0.000000,0.000000,0.000000}%
\pgfsetstrokecolor{textcolor}%
\pgfsetfillcolor{textcolor}%
\pgftext[x=2.258750in,y=0.223333in,,top]{\color{textcolor}\rmfamily\fontsize{10.000000}{12.000000}\selectfont \(\displaystyle p\)}%
\end{pgfscope}%
\begin{pgfscope}%
\pgfsetbuttcap%
\pgfsetroundjoin%
\definecolor{currentfill}{rgb}{0.000000,0.000000,0.000000}%
\pgfsetfillcolor{currentfill}%
\pgfsetlinewidth{0.803000pt}%
\definecolor{currentstroke}{rgb}{0.000000,0.000000,0.000000}%
\pgfsetstrokecolor{currentstroke}%
\pgfsetdash{}{0pt}%
\pgfsys@defobject{currentmarker}{\pgfqpoint{-0.048611in}{0.000000in}}{\pgfqpoint{-0.000000in}{0.000000in}}{%
\pgfpathmoveto{\pgfqpoint{-0.000000in}{0.000000in}}%
\pgfpathlineto{\pgfqpoint{-0.048611in}{0.000000in}}%
\pgfusepath{stroke,fill}%
}%
\begin{pgfscope}%
\pgfsys@transformshift{0.515000in}{0.499444in}%
\pgfsys@useobject{currentmarker}{}%
\end{pgfscope}%
\end{pgfscope}%
\begin{pgfscope}%
\definecolor{textcolor}{rgb}{0.000000,0.000000,0.000000}%
\pgfsetstrokecolor{textcolor}%
\pgfsetfillcolor{textcolor}%
\pgftext[x=0.348333in, y=0.451250in, left, base]{\color{textcolor}\rmfamily\fontsize{10.000000}{12.000000}\selectfont \(\displaystyle {0}\)}%
\end{pgfscope}%
\begin{pgfscope}%
\pgfsetbuttcap%
\pgfsetroundjoin%
\definecolor{currentfill}{rgb}{0.000000,0.000000,0.000000}%
\pgfsetfillcolor{currentfill}%
\pgfsetlinewidth{0.803000pt}%
\definecolor{currentstroke}{rgb}{0.000000,0.000000,0.000000}%
\pgfsetstrokecolor{currentstroke}%
\pgfsetdash{}{0pt}%
\pgfsys@defobject{currentmarker}{\pgfqpoint{-0.048611in}{0.000000in}}{\pgfqpoint{-0.000000in}{0.000000in}}{%
\pgfpathmoveto{\pgfqpoint{-0.000000in}{0.000000in}}%
\pgfpathlineto{\pgfqpoint{-0.048611in}{0.000000in}}%
\pgfusepath{stroke,fill}%
}%
\begin{pgfscope}%
\pgfsys@transformshift{0.515000in}{0.897546in}%
\pgfsys@useobject{currentmarker}{}%
\end{pgfscope}%
\end{pgfscope}%
\begin{pgfscope}%
\definecolor{textcolor}{rgb}{0.000000,0.000000,0.000000}%
\pgfsetstrokecolor{textcolor}%
\pgfsetfillcolor{textcolor}%
\pgftext[x=0.348333in, y=0.849351in, left, base]{\color{textcolor}\rmfamily\fontsize{10.000000}{12.000000}\selectfont \(\displaystyle {5}\)}%
\end{pgfscope}%
\begin{pgfscope}%
\pgfsetbuttcap%
\pgfsetroundjoin%
\definecolor{currentfill}{rgb}{0.000000,0.000000,0.000000}%
\pgfsetfillcolor{currentfill}%
\pgfsetlinewidth{0.803000pt}%
\definecolor{currentstroke}{rgb}{0.000000,0.000000,0.000000}%
\pgfsetstrokecolor{currentstroke}%
\pgfsetdash{}{0pt}%
\pgfsys@defobject{currentmarker}{\pgfqpoint{-0.048611in}{0.000000in}}{\pgfqpoint{-0.000000in}{0.000000in}}{%
\pgfpathmoveto{\pgfqpoint{-0.000000in}{0.000000in}}%
\pgfpathlineto{\pgfqpoint{-0.048611in}{0.000000in}}%
\pgfusepath{stroke,fill}%
}%
\begin{pgfscope}%
\pgfsys@transformshift{0.515000in}{1.295647in}%
\pgfsys@useobject{currentmarker}{}%
\end{pgfscope}%
\end{pgfscope}%
\begin{pgfscope}%
\definecolor{textcolor}{rgb}{0.000000,0.000000,0.000000}%
\pgfsetstrokecolor{textcolor}%
\pgfsetfillcolor{textcolor}%
\pgftext[x=0.278889in, y=1.247453in, left, base]{\color{textcolor}\rmfamily\fontsize{10.000000}{12.000000}\selectfont \(\displaystyle {10}\)}%
\end{pgfscope}%
\begin{pgfscope}%
\definecolor{textcolor}{rgb}{0.000000,0.000000,0.000000}%
\pgfsetstrokecolor{textcolor}%
\pgfsetfillcolor{textcolor}%
\pgftext[x=0.223333in,y=1.076944in,,bottom,rotate=90.000000]{\color{textcolor}\rmfamily\fontsize{10.000000}{12.000000}\selectfont Percent of Data Set}%
\end{pgfscope}%
\begin{pgfscope}%
\pgfsetrectcap%
\pgfsetmiterjoin%
\pgfsetlinewidth{0.803000pt}%
\definecolor{currentstroke}{rgb}{0.000000,0.000000,0.000000}%
\pgfsetstrokecolor{currentstroke}%
\pgfsetdash{}{0pt}%
\pgfpathmoveto{\pgfqpoint{0.515000in}{0.499444in}}%
\pgfpathlineto{\pgfqpoint{0.515000in}{1.654444in}}%
\pgfusepath{stroke}%
\end{pgfscope}%
\begin{pgfscope}%
\pgfsetrectcap%
\pgfsetmiterjoin%
\pgfsetlinewidth{0.803000pt}%
\definecolor{currentstroke}{rgb}{0.000000,0.000000,0.000000}%
\pgfsetstrokecolor{currentstroke}%
\pgfsetdash{}{0pt}%
\pgfpathmoveto{\pgfqpoint{4.002500in}{0.499444in}}%
\pgfpathlineto{\pgfqpoint{4.002500in}{1.654444in}}%
\pgfusepath{stroke}%
\end{pgfscope}%
\begin{pgfscope}%
\pgfsetrectcap%
\pgfsetmiterjoin%
\pgfsetlinewidth{0.803000pt}%
\definecolor{currentstroke}{rgb}{0.000000,0.000000,0.000000}%
\pgfsetstrokecolor{currentstroke}%
\pgfsetdash{}{0pt}%
\pgfpathmoveto{\pgfqpoint{0.515000in}{0.499444in}}%
\pgfpathlineto{\pgfqpoint{4.002500in}{0.499444in}}%
\pgfusepath{stroke}%
\end{pgfscope}%
\begin{pgfscope}%
\pgfsetrectcap%
\pgfsetmiterjoin%
\pgfsetlinewidth{0.803000pt}%
\definecolor{currentstroke}{rgb}{0.000000,0.000000,0.000000}%
\pgfsetstrokecolor{currentstroke}%
\pgfsetdash{}{0pt}%
\pgfpathmoveto{\pgfqpoint{0.515000in}{1.654444in}}%
\pgfpathlineto{\pgfqpoint{4.002500in}{1.654444in}}%
\pgfusepath{stroke}%
\end{pgfscope}%
\begin{pgfscope}%
\pgfsetbuttcap%
\pgfsetmiterjoin%
\definecolor{currentfill}{rgb}{1.000000,1.000000,1.000000}%
\pgfsetfillcolor{currentfill}%
\pgfsetfillopacity{0.800000}%
\pgfsetlinewidth{1.003750pt}%
\definecolor{currentstroke}{rgb}{0.800000,0.800000,0.800000}%
\pgfsetstrokecolor{currentstroke}%
\pgfsetstrokeopacity{0.800000}%
\pgfsetdash{}{0pt}%
\pgfpathmoveto{\pgfqpoint{3.225556in}{1.154445in}}%
\pgfpathlineto{\pgfqpoint{3.905278in}{1.154445in}}%
\pgfpathquadraticcurveto{\pgfqpoint{3.933056in}{1.154445in}}{\pgfqpoint{3.933056in}{1.182222in}}%
\pgfpathlineto{\pgfqpoint{3.933056in}{1.557222in}}%
\pgfpathquadraticcurveto{\pgfqpoint{3.933056in}{1.585000in}}{\pgfqpoint{3.905278in}{1.585000in}}%
\pgfpathlineto{\pgfqpoint{3.225556in}{1.585000in}}%
\pgfpathquadraticcurveto{\pgfqpoint{3.197778in}{1.585000in}}{\pgfqpoint{3.197778in}{1.557222in}}%
\pgfpathlineto{\pgfqpoint{3.197778in}{1.182222in}}%
\pgfpathquadraticcurveto{\pgfqpoint{3.197778in}{1.154445in}}{\pgfqpoint{3.225556in}{1.154445in}}%
\pgfpathlineto{\pgfqpoint{3.225556in}{1.154445in}}%
\pgfpathclose%
\pgfusepath{stroke,fill}%
\end{pgfscope}%
\begin{pgfscope}%
\pgfsetbuttcap%
\pgfsetmiterjoin%
\pgfsetlinewidth{1.003750pt}%
\definecolor{currentstroke}{rgb}{0.000000,0.000000,0.000000}%
\pgfsetstrokecolor{currentstroke}%
\pgfsetdash{}{0pt}%
\pgfpathmoveto{\pgfqpoint{3.253334in}{1.432222in}}%
\pgfpathlineto{\pgfqpoint{3.531111in}{1.432222in}}%
\pgfpathlineto{\pgfqpoint{3.531111in}{1.529444in}}%
\pgfpathlineto{\pgfqpoint{3.253334in}{1.529444in}}%
\pgfpathlineto{\pgfqpoint{3.253334in}{1.432222in}}%
\pgfpathclose%
\pgfusepath{stroke}%
\end{pgfscope}%
\begin{pgfscope}%
\definecolor{textcolor}{rgb}{0.000000,0.000000,0.000000}%
\pgfsetstrokecolor{textcolor}%
\pgfsetfillcolor{textcolor}%
\pgftext[x=3.642223in,y=1.432222in,left,base]{\color{textcolor}\rmfamily\fontsize{10.000000}{12.000000}\selectfont Neg}%
\end{pgfscope}%
\begin{pgfscope}%
\pgfsetbuttcap%
\pgfsetmiterjoin%
\definecolor{currentfill}{rgb}{0.000000,0.000000,0.000000}%
\pgfsetfillcolor{currentfill}%
\pgfsetlinewidth{0.000000pt}%
\definecolor{currentstroke}{rgb}{0.000000,0.000000,0.000000}%
\pgfsetstrokecolor{currentstroke}%
\pgfsetstrokeopacity{0.000000}%
\pgfsetdash{}{0pt}%
\pgfpathmoveto{\pgfqpoint{3.253334in}{1.236944in}}%
\pgfpathlineto{\pgfqpoint{3.531111in}{1.236944in}}%
\pgfpathlineto{\pgfqpoint{3.531111in}{1.334167in}}%
\pgfpathlineto{\pgfqpoint{3.253334in}{1.334167in}}%
\pgfpathlineto{\pgfqpoint{3.253334in}{1.236944in}}%
\pgfpathclose%
\pgfusepath{fill}%
\end{pgfscope}%
\begin{pgfscope}%
\definecolor{textcolor}{rgb}{0.000000,0.000000,0.000000}%
\pgfsetstrokecolor{textcolor}%
\pgfsetfillcolor{textcolor}%
\pgftext[x=3.642223in,y=1.236944in,left,base]{\color{textcolor}\rmfamily\fontsize{10.000000}{12.000000}\selectfont Pos}%
\end{pgfscope}%
\end{pgfpicture}%
\makeatother%
\endgroup%
	
&
	\vskip 0pt
	\hfil ROC Curve
	
	%% Creator: Matplotlib, PGF backend
%%
%% To include the figure in your LaTeX document, write
%%   \input{<filename>.pgf}
%%
%% Make sure the required packages are loaded in your preamble
%%   \usepackage{pgf}
%%
%% Also ensure that all the required font packages are loaded; for instance,
%% the lmodern package is sometimes necessary when using math font.
%%   \usepackage{lmodern}
%%
%% Figures using additional raster images can only be included by \input if
%% they are in the same directory as the main LaTeX file. For loading figures
%% from other directories you can use the `import` package
%%   \usepackage{import}
%%
%% and then include the figures with
%%   \import{<path to file>}{<filename>.pgf}
%%
%% Matplotlib used the following preamble
%%   
%%   \usepackage{fontspec}
%%   \makeatletter\@ifpackageloaded{underscore}{}{\usepackage[strings]{underscore}}\makeatother
%%
\begingroup%
\makeatletter%
\begin{pgfpicture}%
\pgfpathrectangle{\pgfpointorigin}{\pgfqpoint{2.221861in}{1.754444in}}%
\pgfusepath{use as bounding box, clip}%
\begin{pgfscope}%
\pgfsetbuttcap%
\pgfsetmiterjoin%
\definecolor{currentfill}{rgb}{1.000000,1.000000,1.000000}%
\pgfsetfillcolor{currentfill}%
\pgfsetlinewidth{0.000000pt}%
\definecolor{currentstroke}{rgb}{1.000000,1.000000,1.000000}%
\pgfsetstrokecolor{currentstroke}%
\pgfsetdash{}{0pt}%
\pgfpathmoveto{\pgfqpoint{0.000000in}{0.000000in}}%
\pgfpathlineto{\pgfqpoint{2.221861in}{0.000000in}}%
\pgfpathlineto{\pgfqpoint{2.221861in}{1.754444in}}%
\pgfpathlineto{\pgfqpoint{0.000000in}{1.754444in}}%
\pgfpathlineto{\pgfqpoint{0.000000in}{0.000000in}}%
\pgfpathclose%
\pgfusepath{fill}%
\end{pgfscope}%
\begin{pgfscope}%
\pgfsetbuttcap%
\pgfsetmiterjoin%
\definecolor{currentfill}{rgb}{1.000000,1.000000,1.000000}%
\pgfsetfillcolor{currentfill}%
\pgfsetlinewidth{0.000000pt}%
\definecolor{currentstroke}{rgb}{0.000000,0.000000,0.000000}%
\pgfsetstrokecolor{currentstroke}%
\pgfsetstrokeopacity{0.000000}%
\pgfsetdash{}{0pt}%
\pgfpathmoveto{\pgfqpoint{0.553581in}{0.499444in}}%
\pgfpathlineto{\pgfqpoint{2.103581in}{0.499444in}}%
\pgfpathlineto{\pgfqpoint{2.103581in}{1.654444in}}%
\pgfpathlineto{\pgfqpoint{0.553581in}{1.654444in}}%
\pgfpathlineto{\pgfqpoint{0.553581in}{0.499444in}}%
\pgfpathclose%
\pgfusepath{fill}%
\end{pgfscope}%
\begin{pgfscope}%
\pgfsetbuttcap%
\pgfsetroundjoin%
\definecolor{currentfill}{rgb}{0.000000,0.000000,0.000000}%
\pgfsetfillcolor{currentfill}%
\pgfsetlinewidth{0.803000pt}%
\definecolor{currentstroke}{rgb}{0.000000,0.000000,0.000000}%
\pgfsetstrokecolor{currentstroke}%
\pgfsetdash{}{0pt}%
\pgfsys@defobject{currentmarker}{\pgfqpoint{0.000000in}{-0.048611in}}{\pgfqpoint{0.000000in}{0.000000in}}{%
\pgfpathmoveto{\pgfqpoint{0.000000in}{0.000000in}}%
\pgfpathlineto{\pgfqpoint{0.000000in}{-0.048611in}}%
\pgfusepath{stroke,fill}%
}%
\begin{pgfscope}%
\pgfsys@transformshift{0.624035in}{0.499444in}%
\pgfsys@useobject{currentmarker}{}%
\end{pgfscope}%
\end{pgfscope}%
\begin{pgfscope}%
\definecolor{textcolor}{rgb}{0.000000,0.000000,0.000000}%
\pgfsetstrokecolor{textcolor}%
\pgfsetfillcolor{textcolor}%
\pgftext[x=0.624035in,y=0.402222in,,top]{\color{textcolor}\rmfamily\fontsize{10.000000}{12.000000}\selectfont \(\displaystyle {0.0}\)}%
\end{pgfscope}%
\begin{pgfscope}%
\pgfsetbuttcap%
\pgfsetroundjoin%
\definecolor{currentfill}{rgb}{0.000000,0.000000,0.000000}%
\pgfsetfillcolor{currentfill}%
\pgfsetlinewidth{0.803000pt}%
\definecolor{currentstroke}{rgb}{0.000000,0.000000,0.000000}%
\pgfsetstrokecolor{currentstroke}%
\pgfsetdash{}{0pt}%
\pgfsys@defobject{currentmarker}{\pgfqpoint{0.000000in}{-0.048611in}}{\pgfqpoint{0.000000in}{0.000000in}}{%
\pgfpathmoveto{\pgfqpoint{0.000000in}{0.000000in}}%
\pgfpathlineto{\pgfqpoint{0.000000in}{-0.048611in}}%
\pgfusepath{stroke,fill}%
}%
\begin{pgfscope}%
\pgfsys@transformshift{1.328581in}{0.499444in}%
\pgfsys@useobject{currentmarker}{}%
\end{pgfscope}%
\end{pgfscope}%
\begin{pgfscope}%
\definecolor{textcolor}{rgb}{0.000000,0.000000,0.000000}%
\pgfsetstrokecolor{textcolor}%
\pgfsetfillcolor{textcolor}%
\pgftext[x=1.328581in,y=0.402222in,,top]{\color{textcolor}\rmfamily\fontsize{10.000000}{12.000000}\selectfont \(\displaystyle {0.5}\)}%
\end{pgfscope}%
\begin{pgfscope}%
\pgfsetbuttcap%
\pgfsetroundjoin%
\definecolor{currentfill}{rgb}{0.000000,0.000000,0.000000}%
\pgfsetfillcolor{currentfill}%
\pgfsetlinewidth{0.803000pt}%
\definecolor{currentstroke}{rgb}{0.000000,0.000000,0.000000}%
\pgfsetstrokecolor{currentstroke}%
\pgfsetdash{}{0pt}%
\pgfsys@defobject{currentmarker}{\pgfqpoint{0.000000in}{-0.048611in}}{\pgfqpoint{0.000000in}{0.000000in}}{%
\pgfpathmoveto{\pgfqpoint{0.000000in}{0.000000in}}%
\pgfpathlineto{\pgfqpoint{0.000000in}{-0.048611in}}%
\pgfusepath{stroke,fill}%
}%
\begin{pgfscope}%
\pgfsys@transformshift{2.033126in}{0.499444in}%
\pgfsys@useobject{currentmarker}{}%
\end{pgfscope}%
\end{pgfscope}%
\begin{pgfscope}%
\definecolor{textcolor}{rgb}{0.000000,0.000000,0.000000}%
\pgfsetstrokecolor{textcolor}%
\pgfsetfillcolor{textcolor}%
\pgftext[x=2.033126in,y=0.402222in,,top]{\color{textcolor}\rmfamily\fontsize{10.000000}{12.000000}\selectfont \(\displaystyle {1.0}\)}%
\end{pgfscope}%
\begin{pgfscope}%
\definecolor{textcolor}{rgb}{0.000000,0.000000,0.000000}%
\pgfsetstrokecolor{textcolor}%
\pgfsetfillcolor{textcolor}%
\pgftext[x=1.328581in,y=0.223333in,,top]{\color{textcolor}\rmfamily\fontsize{10.000000}{12.000000}\selectfont False positive rate}%
\end{pgfscope}%
\begin{pgfscope}%
\pgfsetbuttcap%
\pgfsetroundjoin%
\definecolor{currentfill}{rgb}{0.000000,0.000000,0.000000}%
\pgfsetfillcolor{currentfill}%
\pgfsetlinewidth{0.803000pt}%
\definecolor{currentstroke}{rgb}{0.000000,0.000000,0.000000}%
\pgfsetstrokecolor{currentstroke}%
\pgfsetdash{}{0pt}%
\pgfsys@defobject{currentmarker}{\pgfqpoint{-0.048611in}{0.000000in}}{\pgfqpoint{-0.000000in}{0.000000in}}{%
\pgfpathmoveto{\pgfqpoint{-0.000000in}{0.000000in}}%
\pgfpathlineto{\pgfqpoint{-0.048611in}{0.000000in}}%
\pgfusepath{stroke,fill}%
}%
\begin{pgfscope}%
\pgfsys@transformshift{0.553581in}{0.551944in}%
\pgfsys@useobject{currentmarker}{}%
\end{pgfscope}%
\end{pgfscope}%
\begin{pgfscope}%
\definecolor{textcolor}{rgb}{0.000000,0.000000,0.000000}%
\pgfsetstrokecolor{textcolor}%
\pgfsetfillcolor{textcolor}%
\pgftext[x=0.278889in, y=0.503750in, left, base]{\color{textcolor}\rmfamily\fontsize{10.000000}{12.000000}\selectfont \(\displaystyle {0.0}\)}%
\end{pgfscope}%
\begin{pgfscope}%
\pgfsetbuttcap%
\pgfsetroundjoin%
\definecolor{currentfill}{rgb}{0.000000,0.000000,0.000000}%
\pgfsetfillcolor{currentfill}%
\pgfsetlinewidth{0.803000pt}%
\definecolor{currentstroke}{rgb}{0.000000,0.000000,0.000000}%
\pgfsetstrokecolor{currentstroke}%
\pgfsetdash{}{0pt}%
\pgfsys@defobject{currentmarker}{\pgfqpoint{-0.048611in}{0.000000in}}{\pgfqpoint{-0.000000in}{0.000000in}}{%
\pgfpathmoveto{\pgfqpoint{-0.000000in}{0.000000in}}%
\pgfpathlineto{\pgfqpoint{-0.048611in}{0.000000in}}%
\pgfusepath{stroke,fill}%
}%
\begin{pgfscope}%
\pgfsys@transformshift{0.553581in}{1.076944in}%
\pgfsys@useobject{currentmarker}{}%
\end{pgfscope}%
\end{pgfscope}%
\begin{pgfscope}%
\definecolor{textcolor}{rgb}{0.000000,0.000000,0.000000}%
\pgfsetstrokecolor{textcolor}%
\pgfsetfillcolor{textcolor}%
\pgftext[x=0.278889in, y=1.028750in, left, base]{\color{textcolor}\rmfamily\fontsize{10.000000}{12.000000}\selectfont \(\displaystyle {0.5}\)}%
\end{pgfscope}%
\begin{pgfscope}%
\pgfsetbuttcap%
\pgfsetroundjoin%
\definecolor{currentfill}{rgb}{0.000000,0.000000,0.000000}%
\pgfsetfillcolor{currentfill}%
\pgfsetlinewidth{0.803000pt}%
\definecolor{currentstroke}{rgb}{0.000000,0.000000,0.000000}%
\pgfsetstrokecolor{currentstroke}%
\pgfsetdash{}{0pt}%
\pgfsys@defobject{currentmarker}{\pgfqpoint{-0.048611in}{0.000000in}}{\pgfqpoint{-0.000000in}{0.000000in}}{%
\pgfpathmoveto{\pgfqpoint{-0.000000in}{0.000000in}}%
\pgfpathlineto{\pgfqpoint{-0.048611in}{0.000000in}}%
\pgfusepath{stroke,fill}%
}%
\begin{pgfscope}%
\pgfsys@transformshift{0.553581in}{1.601944in}%
\pgfsys@useobject{currentmarker}{}%
\end{pgfscope}%
\end{pgfscope}%
\begin{pgfscope}%
\definecolor{textcolor}{rgb}{0.000000,0.000000,0.000000}%
\pgfsetstrokecolor{textcolor}%
\pgfsetfillcolor{textcolor}%
\pgftext[x=0.278889in, y=1.553750in, left, base]{\color{textcolor}\rmfamily\fontsize{10.000000}{12.000000}\selectfont \(\displaystyle {1.0}\)}%
\end{pgfscope}%
\begin{pgfscope}%
\definecolor{textcolor}{rgb}{0.000000,0.000000,0.000000}%
\pgfsetstrokecolor{textcolor}%
\pgfsetfillcolor{textcolor}%
\pgftext[x=0.223333in,y=1.076944in,,bottom,rotate=90.000000]{\color{textcolor}\rmfamily\fontsize{10.000000}{12.000000}\selectfont True positive rate}%
\end{pgfscope}%
\begin{pgfscope}%
\pgfpathrectangle{\pgfqpoint{0.553581in}{0.499444in}}{\pgfqpoint{1.550000in}{1.155000in}}%
\pgfusepath{clip}%
\pgfsetbuttcap%
\pgfsetroundjoin%
\pgfsetlinewidth{1.505625pt}%
\definecolor{currentstroke}{rgb}{0.000000,0.000000,0.000000}%
\pgfsetstrokecolor{currentstroke}%
\pgfsetdash{{5.550000pt}{2.400000pt}}{0.000000pt}%
\pgfpathmoveto{\pgfqpoint{0.624035in}{0.551944in}}%
\pgfpathlineto{\pgfqpoint{2.033126in}{1.601944in}}%
\pgfusepath{stroke}%
\end{pgfscope}%
\begin{pgfscope}%
\pgfpathrectangle{\pgfqpoint{0.553581in}{0.499444in}}{\pgfqpoint{1.550000in}{1.155000in}}%
\pgfusepath{clip}%
\pgfsetrectcap%
\pgfsetroundjoin%
\pgfsetlinewidth{1.505625pt}%
\definecolor{currentstroke}{rgb}{0.000000,0.000000,0.000000}%
\pgfsetstrokecolor{currentstroke}%
\pgfsetdash{}{0pt}%
\pgfpathmoveto{\pgfqpoint{0.624035in}{0.551944in}}%
\pgfpathlineto{\pgfqpoint{0.625130in}{0.562374in}}%
\pgfpathlineto{\pgfqpoint{0.625403in}{0.563430in}}%
\pgfpathlineto{\pgfqpoint{0.626513in}{0.572028in}}%
\pgfpathlineto{\pgfqpoint{0.626701in}{0.573084in}}%
\pgfpathlineto{\pgfqpoint{0.627796in}{0.578547in}}%
\pgfpathlineto{\pgfqpoint{0.627983in}{0.579603in}}%
\pgfpathlineto{\pgfqpoint{0.629093in}{0.584725in}}%
\pgfpathlineto{\pgfqpoint{0.629312in}{0.585780in}}%
\pgfpathlineto{\pgfqpoint{0.630407in}{0.592609in}}%
\pgfpathlineto{\pgfqpoint{0.630672in}{0.593634in}}%
\pgfpathlineto{\pgfqpoint{0.631775in}{0.598973in}}%
\pgfpathlineto{\pgfqpoint{0.631986in}{0.600059in}}%
\pgfpathlineto{\pgfqpoint{0.633088in}{0.605554in}}%
\pgfpathlineto{\pgfqpoint{0.633323in}{0.606640in}}%
\pgfpathlineto{\pgfqpoint{0.634433in}{0.611421in}}%
\pgfpathlineto{\pgfqpoint{0.634691in}{0.612507in}}%
\pgfpathlineto{\pgfqpoint{0.635793in}{0.617040in}}%
\pgfpathlineto{\pgfqpoint{0.636121in}{0.618126in}}%
\pgfpathlineto{\pgfqpoint{0.637231in}{0.622503in}}%
\pgfpathlineto{\pgfqpoint{0.637521in}{0.623527in}}%
\pgfpathlineto{\pgfqpoint{0.638623in}{0.628711in}}%
\pgfpathlineto{\pgfqpoint{0.638983in}{0.629767in}}%
\pgfpathlineto{\pgfqpoint{0.640077in}{0.633771in}}%
\pgfpathlineto{\pgfqpoint{0.640257in}{0.634858in}}%
\pgfpathlineto{\pgfqpoint{0.641351in}{0.639017in}}%
\pgfpathlineto{\pgfqpoint{0.641625in}{0.640042in}}%
\pgfpathlineto{\pgfqpoint{0.642735in}{0.644326in}}%
\pgfpathlineto{\pgfqpoint{0.642884in}{0.645226in}}%
\pgfpathlineto{\pgfqpoint{0.643986in}{0.649851in}}%
\pgfpathlineto{\pgfqpoint{0.644252in}{0.650844in}}%
\pgfpathlineto{\pgfqpoint{0.645330in}{0.655345in}}%
\pgfpathlineto{\pgfqpoint{0.645659in}{0.656339in}}%
\pgfpathlineto{\pgfqpoint{0.646761in}{0.660188in}}%
\pgfpathlineto{\pgfqpoint{0.647066in}{0.661275in}}%
\pgfpathlineto{\pgfqpoint{0.648160in}{0.665651in}}%
\pgfpathlineto{\pgfqpoint{0.648481in}{0.666645in}}%
\pgfpathlineto{\pgfqpoint{0.649591in}{0.671363in}}%
\pgfpathlineto{\pgfqpoint{0.649982in}{0.672419in}}%
\pgfpathlineto{\pgfqpoint{0.651053in}{0.677696in}}%
\pgfpathlineto{\pgfqpoint{0.651381in}{0.678751in}}%
\pgfpathlineto{\pgfqpoint{0.652484in}{0.682787in}}%
\pgfpathlineto{\pgfqpoint{0.652828in}{0.683842in}}%
\pgfpathlineto{\pgfqpoint{0.653930in}{0.688343in}}%
\pgfpathlineto{\pgfqpoint{0.654204in}{0.689399in}}%
\pgfpathlineto{\pgfqpoint{0.655314in}{0.693651in}}%
\pgfpathlineto{\pgfqpoint{0.655619in}{0.694738in}}%
\pgfpathlineto{\pgfqpoint{0.656721in}{0.698091in}}%
\pgfpathlineto{\pgfqpoint{0.657080in}{0.699177in}}%
\pgfpathlineto{\pgfqpoint{0.658191in}{0.703275in}}%
\pgfpathlineto{\pgfqpoint{0.658644in}{0.704361in}}%
\pgfpathlineto{\pgfqpoint{0.659746in}{0.707093in}}%
\pgfpathlineto{\pgfqpoint{0.660176in}{0.708117in}}%
\pgfpathlineto{\pgfqpoint{0.660176in}{0.708179in}}%
\pgfpathlineto{\pgfqpoint{0.661278in}{0.711377in}}%
\pgfpathlineto{\pgfqpoint{0.661599in}{0.712370in}}%
\pgfpathlineto{\pgfqpoint{0.662709in}{0.716343in}}%
\pgfpathlineto{\pgfqpoint{0.662998in}{0.717399in}}%
\pgfpathlineto{\pgfqpoint{0.664077in}{0.720751in}}%
\pgfpathlineto{\pgfqpoint{0.664546in}{0.721838in}}%
\pgfpathlineto{\pgfqpoint{0.665656in}{0.724694in}}%
\pgfpathlineto{\pgfqpoint{0.665977in}{0.725780in}}%
\pgfpathlineto{\pgfqpoint{0.667079in}{0.729226in}}%
\pgfpathlineto{\pgfqpoint{0.667556in}{0.730312in}}%
\pgfpathlineto{\pgfqpoint{0.668651in}{0.733385in}}%
\pgfpathlineto{\pgfqpoint{0.669088in}{0.734410in}}%
\pgfpathlineto{\pgfqpoint{0.669088in}{0.734441in}}%
\pgfpathlineto{\pgfqpoint{0.670198in}{0.737887in}}%
\pgfpathlineto{\pgfqpoint{0.670652in}{0.738973in}}%
\pgfpathlineto{\pgfqpoint{0.671754in}{0.741767in}}%
\pgfpathlineto{\pgfqpoint{0.672270in}{0.742822in}}%
\pgfpathlineto{\pgfqpoint{0.673372in}{0.746578in}}%
\pgfpathlineto{\pgfqpoint{0.673865in}{0.747603in}}%
\pgfpathlineto{\pgfqpoint{0.674975in}{0.751452in}}%
\pgfpathlineto{\pgfqpoint{0.675350in}{0.752538in}}%
\pgfpathlineto{\pgfqpoint{0.676453in}{0.756108in}}%
\pgfpathlineto{\pgfqpoint{0.676734in}{0.757040in}}%
\pgfpathlineto{\pgfqpoint{0.677844in}{0.760237in}}%
\pgfpathlineto{\pgfqpoint{0.678344in}{0.761261in}}%
\pgfpathlineto{\pgfqpoint{0.679415in}{0.764241in}}%
\pgfpathlineto{\pgfqpoint{0.679822in}{0.765328in}}%
\pgfpathlineto{\pgfqpoint{0.680916in}{0.768091in}}%
\pgfpathlineto{\pgfqpoint{0.681417in}{0.769177in}}%
\pgfpathlineto{\pgfqpoint{0.682488in}{0.772250in}}%
\pgfpathlineto{\pgfqpoint{0.683004in}{0.773337in}}%
\pgfpathlineto{\pgfqpoint{0.684114in}{0.776410in}}%
\pgfpathlineto{\pgfqpoint{0.684552in}{0.777496in}}%
\pgfpathlineto{\pgfqpoint{0.685662in}{0.779855in}}%
\pgfpathlineto{\pgfqpoint{0.686185in}{0.780942in}}%
\pgfpathlineto{\pgfqpoint{0.687288in}{0.784015in}}%
\pgfpathlineto{\pgfqpoint{0.687757in}{0.785071in}}%
\pgfpathlineto{\pgfqpoint{0.688867in}{0.787368in}}%
\pgfpathlineto{\pgfqpoint{0.689125in}{0.788423in}}%
\pgfpathlineto{\pgfqpoint{0.690212in}{0.791093in}}%
\pgfpathlineto{\pgfqpoint{0.690728in}{0.792179in}}%
\pgfpathlineto{\pgfqpoint{0.691830in}{0.794942in}}%
\pgfpathlineto{\pgfqpoint{0.692377in}{0.796028in}}%
\pgfpathlineto{\pgfqpoint{0.693448in}{0.799102in}}%
\pgfpathlineto{\pgfqpoint{0.694034in}{0.800188in}}%
\pgfpathlineto{\pgfqpoint{0.695137in}{0.803292in}}%
\pgfpathlineto{\pgfqpoint{0.695668in}{0.804379in}}%
\pgfpathlineto{\pgfqpoint{0.696731in}{0.806459in}}%
\pgfpathlineto{\pgfqpoint{0.697271in}{0.807545in}}%
\pgfpathlineto{\pgfqpoint{0.698365in}{0.810618in}}%
\pgfpathlineto{\pgfqpoint{0.698881in}{0.811705in}}%
\pgfpathlineto{\pgfqpoint{0.699976in}{0.814343in}}%
\pgfpathlineto{\pgfqpoint{0.700531in}{0.815430in}}%
\pgfpathlineto{\pgfqpoint{0.701618in}{0.817323in}}%
\pgfpathlineto{\pgfqpoint{0.702157in}{0.818317in}}%
\pgfpathlineto{\pgfqpoint{0.703251in}{0.820490in}}%
\pgfpathlineto{\pgfqpoint{0.703853in}{0.821545in}}%
\pgfpathlineto{\pgfqpoint{0.704963in}{0.824463in}}%
\pgfpathlineto{\pgfqpoint{0.705362in}{0.825549in}}%
\pgfpathlineto{\pgfqpoint{0.706464in}{0.828033in}}%
\pgfpathlineto{\pgfqpoint{0.707254in}{0.829057in}}%
\pgfpathlineto{\pgfqpoint{0.707254in}{0.829119in}}%
\pgfpathlineto{\pgfqpoint{0.708349in}{0.831758in}}%
\pgfpathlineto{\pgfqpoint{0.708943in}{0.832844in}}%
\pgfpathlineto{\pgfqpoint{0.710045in}{0.835483in}}%
\pgfpathlineto{\pgfqpoint{0.710569in}{0.836569in}}%
\pgfpathlineto{\pgfqpoint{0.711679in}{0.839115in}}%
\pgfpathlineto{\pgfqpoint{0.712218in}{0.840201in}}%
\pgfpathlineto{\pgfqpoint{0.713328in}{0.843306in}}%
\pgfpathlineto{\pgfqpoint{0.713930in}{0.844361in}}%
\pgfpathlineto{\pgfqpoint{0.715025in}{0.847403in}}%
\pgfpathlineto{\pgfqpoint{0.715431in}{0.848490in}}%
\pgfpathlineto{\pgfqpoint{0.716471in}{0.850880in}}%
\pgfpathlineto{\pgfqpoint{0.717034in}{0.851904in}}%
\pgfpathlineto{\pgfqpoint{0.718136in}{0.854108in}}%
\pgfpathlineto{\pgfqpoint{0.718769in}{0.855195in}}%
\pgfpathlineto{\pgfqpoint{0.719880in}{0.857740in}}%
\pgfpathlineto{\pgfqpoint{0.720333in}{0.858827in}}%
\pgfpathlineto{\pgfqpoint{0.721435in}{0.861217in}}%
\pgfpathlineto{\pgfqpoint{0.721990in}{0.862241in}}%
\pgfpathlineto{\pgfqpoint{0.723038in}{0.864135in}}%
\pgfpathlineto{\pgfqpoint{0.723749in}{0.865159in}}%
\pgfpathlineto{\pgfqpoint{0.724859in}{0.867984in}}%
\pgfpathlineto{\pgfqpoint{0.725375in}{0.869071in}}%
\pgfpathlineto{\pgfqpoint{0.726478in}{0.871181in}}%
\pgfpathlineto{\pgfqpoint{0.727080in}{0.872268in}}%
\pgfpathlineto{\pgfqpoint{0.728151in}{0.874844in}}%
\pgfpathlineto{\pgfqpoint{0.728885in}{0.875931in}}%
\pgfpathlineto{\pgfqpoint{0.729910in}{0.877980in}}%
\pgfpathlineto{\pgfqpoint{0.729956in}{0.877980in}}%
\pgfpathlineto{\pgfqpoint{0.730347in}{0.879066in}}%
\pgfpathlineto{\pgfqpoint{0.731364in}{0.881363in}}%
\pgfpathlineto{\pgfqpoint{0.731950in}{0.882450in}}%
\pgfpathlineto{\pgfqpoint{0.733060in}{0.884126in}}%
\pgfpathlineto{\pgfqpoint{0.733412in}{0.885150in}}%
\pgfpathlineto{\pgfqpoint{0.734499in}{0.887292in}}%
\pgfpathlineto{\pgfqpoint{0.735210in}{0.888379in}}%
\pgfpathlineto{\pgfqpoint{0.736312in}{0.890490in}}%
\pgfpathlineto{\pgfqpoint{0.736844in}{0.891514in}}%
\pgfpathlineto{\pgfqpoint{0.737954in}{0.893408in}}%
\pgfpathlineto{\pgfqpoint{0.738556in}{0.894494in}}%
\pgfpathlineto{\pgfqpoint{0.739643in}{0.896543in}}%
\pgfpathlineto{\pgfqpoint{0.740479in}{0.897629in}}%
\pgfpathlineto{\pgfqpoint{0.741589in}{0.900051in}}%
\pgfpathlineto{\pgfqpoint{0.742121in}{0.901137in}}%
\pgfpathlineto{\pgfqpoint{0.743215in}{0.903279in}}%
\pgfpathlineto{\pgfqpoint{0.743809in}{0.904272in}}%
\pgfpathlineto{\pgfqpoint{0.744919in}{0.906632in}}%
\pgfpathlineto{\pgfqpoint{0.745553in}{0.907718in}}%
\pgfpathlineto{\pgfqpoint{0.746647in}{0.909581in}}%
\pgfpathlineto{\pgfqpoint{0.747398in}{0.910667in}}%
\pgfpathlineto{\pgfqpoint{0.748476in}{0.912933in}}%
\pgfpathlineto{\pgfqpoint{0.749180in}{0.913989in}}%
\pgfpathlineto{\pgfqpoint{0.750282in}{0.916348in}}%
\pgfpathlineto{\pgfqpoint{0.750900in}{0.917434in}}%
\pgfpathlineto{\pgfqpoint{0.752010in}{0.919607in}}%
\pgfpathlineto{\pgfqpoint{0.752534in}{0.920694in}}%
\pgfpathlineto{\pgfqpoint{0.753628in}{0.922960in}}%
\pgfpathlineto{\pgfqpoint{0.754574in}{0.924046in}}%
\pgfpathlineto{\pgfqpoint{0.755645in}{0.926436in}}%
\pgfpathlineto{\pgfqpoint{0.756208in}{0.927523in}}%
\pgfpathlineto{\pgfqpoint{0.757279in}{0.929230in}}%
\pgfpathlineto{\pgfqpoint{0.757303in}{0.929230in}}%
\pgfpathlineto{\pgfqpoint{0.757779in}{0.930255in}}%
\pgfpathlineto{\pgfqpoint{0.758882in}{0.931651in}}%
\pgfpathlineto{\pgfqpoint{0.759476in}{0.932645in}}%
\pgfpathlineto{\pgfqpoint{0.760578in}{0.934197in}}%
\pgfpathlineto{\pgfqpoint{0.761329in}{0.935283in}}%
\pgfpathlineto{\pgfqpoint{0.762439in}{0.937146in}}%
\pgfpathlineto{\pgfqpoint{0.763009in}{0.938232in}}%
\pgfpathlineto{\pgfqpoint{0.764096in}{0.939878in}}%
\pgfpathlineto{\pgfqpoint{0.764854in}{0.940964in}}%
\pgfpathlineto{\pgfqpoint{0.765933in}{0.943044in}}%
\pgfpathlineto{\pgfqpoint{0.766512in}{0.944099in}}%
\pgfpathlineto{\pgfqpoint{0.767606in}{0.945589in}}%
\pgfpathlineto{\pgfqpoint{0.768396in}{0.946645in}}%
\pgfpathlineto{\pgfqpoint{0.769490in}{0.948663in}}%
\pgfpathlineto{\pgfqpoint{0.770092in}{0.949749in}}%
\pgfpathlineto{\pgfqpoint{0.771202in}{0.951891in}}%
\pgfpathlineto{\pgfqpoint{0.772000in}{0.952977in}}%
\pgfpathlineto{\pgfqpoint{0.773086in}{0.954654in}}%
\pgfpathlineto{\pgfqpoint{0.773939in}{0.955740in}}%
\pgfpathlineto{\pgfqpoint{0.775049in}{0.957634in}}%
\pgfpathlineto{\pgfqpoint{0.775783in}{0.958720in}}%
\pgfpathlineto{\pgfqpoint{0.776894in}{0.960148in}}%
\pgfpathlineto{\pgfqpoint{0.777636in}{0.961235in}}%
\pgfpathlineto{\pgfqpoint{0.778731in}{0.963377in}}%
\pgfpathlineto{\pgfqpoint{0.779450in}{0.964432in}}%
\pgfpathlineto{\pgfqpoint{0.780552in}{0.966201in}}%
\pgfpathlineto{\pgfqpoint{0.781217in}{0.967288in}}%
\pgfpathlineto{\pgfqpoint{0.782327in}{0.969740in}}%
\pgfpathlineto{\pgfqpoint{0.783202in}{0.970765in}}%
\pgfpathlineto{\pgfqpoint{0.784266in}{0.972813in}}%
\pgfpathlineto{\pgfqpoint{0.784930in}{0.973838in}}%
\pgfpathlineto{\pgfqpoint{0.786040in}{0.975173in}}%
\pgfpathlineto{\pgfqpoint{0.786759in}{0.976259in}}%
\pgfpathlineto{\pgfqpoint{0.787870in}{0.978153in}}%
\pgfpathlineto{\pgfqpoint{0.788612in}{0.979239in}}%
\pgfpathlineto{\pgfqpoint{0.789629in}{0.981040in}}%
\pgfpathlineto{\pgfqpoint{0.790301in}{0.982126in}}%
\pgfpathlineto{\pgfqpoint{0.791411in}{0.983989in}}%
\pgfpathlineto{\pgfqpoint{0.791896in}{0.985075in}}%
\pgfpathlineto{\pgfqpoint{0.792998in}{0.986472in}}%
\pgfpathlineto{\pgfqpoint{0.793670in}{0.987558in}}%
\pgfpathlineto{\pgfqpoint{0.794694in}{0.989297in}}%
\pgfpathlineto{\pgfqpoint{0.794741in}{0.989297in}}%
\pgfpathlineto{\pgfqpoint{0.795726in}{0.990383in}}%
\pgfpathlineto{\pgfqpoint{0.796829in}{0.992525in}}%
\pgfpathlineto{\pgfqpoint{0.797767in}{0.993581in}}%
\pgfpathlineto{\pgfqpoint{0.798861in}{0.995102in}}%
\pgfpathlineto{\pgfqpoint{0.799580in}{0.996188in}}%
\pgfpathlineto{\pgfqpoint{0.800636in}{0.997554in}}%
\pgfpathlineto{\pgfqpoint{0.801371in}{0.998640in}}%
\pgfpathlineto{\pgfqpoint{0.802481in}{1.000379in}}%
\pgfpathlineto{\pgfqpoint{0.803216in}{1.001465in}}%
\pgfpathlineto{\pgfqpoint{0.804326in}{1.002552in}}%
\pgfpathlineto{\pgfqpoint{0.805084in}{1.003607in}}%
\pgfpathlineto{\pgfqpoint{0.806186in}{1.005035in}}%
\pgfpathlineto{\pgfqpoint{0.807015in}{1.006122in}}%
\pgfpathlineto{\pgfqpoint{0.808109in}{1.007767in}}%
\pgfpathlineto{\pgfqpoint{0.808938in}{1.008853in}}%
\pgfpathlineto{\pgfqpoint{0.810040in}{1.010561in}}%
\pgfpathlineto{\pgfqpoint{0.811072in}{1.011647in}}%
\pgfpathlineto{\pgfqpoint{0.812151in}{1.013541in}}%
\pgfpathlineto{\pgfqpoint{0.813089in}{1.014627in}}%
\pgfpathlineto{\pgfqpoint{0.814176in}{1.015869in}}%
\pgfpathlineto{\pgfqpoint{0.815145in}{1.016955in}}%
\pgfpathlineto{\pgfqpoint{0.816248in}{1.018569in}}%
\pgfpathlineto{\pgfqpoint{0.816936in}{1.019625in}}%
\pgfpathlineto{\pgfqpoint{0.818030in}{1.021177in}}%
\pgfpathlineto{\pgfqpoint{0.818781in}{1.022263in}}%
\pgfpathlineto{\pgfqpoint{0.819891in}{1.023722in}}%
\pgfpathlineto{\pgfqpoint{0.820868in}{1.024809in}}%
\pgfpathlineto{\pgfqpoint{0.821962in}{1.026020in}}%
\pgfpathlineto{\pgfqpoint{0.822736in}{1.027075in}}%
\pgfpathlineto{\pgfqpoint{0.823823in}{1.028441in}}%
\pgfpathlineto{\pgfqpoint{0.824902in}{1.029496in}}%
\pgfpathlineto{\pgfqpoint{0.825957in}{1.031110in}}%
\pgfpathlineto{\pgfqpoint{0.826692in}{1.032197in}}%
\pgfpathlineto{\pgfqpoint{0.827763in}{1.033252in}}%
\pgfpathlineto{\pgfqpoint{0.828599in}{1.034308in}}%
\pgfpathlineto{\pgfqpoint{0.829710in}{1.035643in}}%
\pgfpathlineto{\pgfqpoint{0.830601in}{1.036698in}}%
\pgfpathlineto{\pgfqpoint{0.831695in}{1.038561in}}%
\pgfpathlineto{\pgfqpoint{0.832837in}{1.039647in}}%
\pgfpathlineto{\pgfqpoint{0.833947in}{1.040982in}}%
\pgfpathlineto{\pgfqpoint{0.834721in}{1.042037in}}%
\pgfpathlineto{\pgfqpoint{0.835792in}{1.043589in}}%
\pgfpathlineto{\pgfqpoint{0.836605in}{1.044645in}}%
\pgfpathlineto{\pgfqpoint{0.837676in}{1.045918in}}%
\pgfpathlineto{\pgfqpoint{0.838786in}{1.047004in}}%
\pgfpathlineto{\pgfqpoint{0.839880in}{1.048370in}}%
\pgfpathlineto{\pgfqpoint{0.840623in}{1.049456in}}%
\pgfpathlineto{\pgfqpoint{0.841733in}{1.050915in}}%
\pgfpathlineto{\pgfqpoint{0.842742in}{1.052002in}}%
\pgfpathlineto{\pgfqpoint{0.843836in}{1.053212in}}%
\pgfpathlineto{\pgfqpoint{0.844837in}{1.054268in}}%
\pgfpathlineto{\pgfqpoint{0.845900in}{1.055199in}}%
\pgfpathlineto{\pgfqpoint{0.847104in}{1.056286in}}%
\pgfpathlineto{\pgfqpoint{0.848206in}{1.057372in}}%
\pgfpathlineto{\pgfqpoint{0.849504in}{1.058459in}}%
\pgfpathlineto{\pgfqpoint{0.850614in}{1.060228in}}%
\pgfpathlineto{\pgfqpoint{0.851787in}{1.061190in}}%
\pgfpathlineto{\pgfqpoint{0.851787in}{1.061283in}}%
\pgfpathlineto{\pgfqpoint{0.852881in}{1.062680in}}%
\pgfpathlineto{\pgfqpoint{0.854007in}{1.063767in}}%
\pgfpathlineto{\pgfqpoint{0.855101in}{1.065195in}}%
\pgfpathlineto{\pgfqpoint{0.855805in}{1.066281in}}%
\pgfpathlineto{\pgfqpoint{0.856907in}{1.067523in}}%
\pgfpathlineto{\pgfqpoint{0.856915in}{1.067523in}}%
\pgfpathlineto{\pgfqpoint{0.857642in}{1.068609in}}%
\pgfpathlineto{\pgfqpoint{0.858713in}{1.069851in}}%
\pgfpathlineto{\pgfqpoint{0.859745in}{1.070938in}}%
\pgfpathlineto{\pgfqpoint{0.860855in}{1.072179in}}%
\pgfpathlineto{\pgfqpoint{0.861410in}{1.073204in}}%
\pgfpathlineto{\pgfqpoint{0.862505in}{1.074600in}}%
\pgfpathlineto{\pgfqpoint{0.863662in}{1.075687in}}%
\pgfpathlineto{\pgfqpoint{0.864756in}{1.077115in}}%
\pgfpathlineto{\pgfqpoint{0.865522in}{1.078201in}}%
\pgfpathlineto{\pgfqpoint{0.866624in}{1.079536in}}%
\pgfpathlineto{\pgfqpoint{0.867907in}{1.080592in}}%
\pgfpathlineto{\pgfqpoint{0.869017in}{1.082268in}}%
\pgfpathlineto{\pgfqpoint{0.870002in}{1.083354in}}%
\pgfpathlineto{\pgfqpoint{0.871096in}{1.084689in}}%
\pgfpathlineto{\pgfqpoint{0.872425in}{1.085776in}}%
\pgfpathlineto{\pgfqpoint{0.873535in}{1.086955in}}%
\pgfpathlineto{\pgfqpoint{0.874098in}{1.088042in}}%
\pgfpathlineto{\pgfqpoint{0.875193in}{1.089345in}}%
\pgfpathlineto{\pgfqpoint{0.876193in}{1.090432in}}%
\pgfpathlineto{\pgfqpoint{0.877288in}{1.091953in}}%
\pgfpathlineto{\pgfqpoint{0.878085in}{1.093040in}}%
\pgfpathlineto{\pgfqpoint{0.879172in}{1.094654in}}%
\pgfpathlineto{\pgfqpoint{0.879946in}{1.095740in}}%
\pgfpathlineto{\pgfqpoint{0.880993in}{1.097075in}}%
\pgfpathlineto{\pgfqpoint{0.882307in}{1.098130in}}%
\pgfpathlineto{\pgfqpoint{0.883417in}{1.099403in}}%
\pgfpathlineto{\pgfqpoint{0.884300in}{1.100490in}}%
\pgfpathlineto{\pgfqpoint{0.885356in}{1.101731in}}%
\pgfpathlineto{\pgfqpoint{0.886630in}{1.102818in}}%
\pgfpathlineto{\pgfqpoint{0.887740in}{1.103904in}}%
\pgfpathlineto{\pgfqpoint{0.888748in}{1.104991in}}%
\pgfpathlineto{\pgfqpoint{0.889812in}{1.106326in}}%
\pgfpathlineto{\pgfqpoint{0.889843in}{1.106326in}}%
\pgfpathlineto{\pgfqpoint{0.891086in}{1.107412in}}%
\pgfpathlineto{\pgfqpoint{0.892165in}{1.108871in}}%
\pgfpathlineto{\pgfqpoint{0.893158in}{1.109957in}}%
\pgfpathlineto{\pgfqpoint{0.894260in}{1.111447in}}%
\pgfpathlineto{\pgfqpoint{0.895253in}{1.112503in}}%
\pgfpathlineto{\pgfqpoint{0.896324in}{1.113651in}}%
\pgfpathlineto{\pgfqpoint{0.897207in}{1.114707in}}%
\pgfpathlineto{\pgfqpoint{0.898302in}{1.116321in}}%
\pgfpathlineto{\pgfqpoint{0.899560in}{1.117377in}}%
\pgfpathlineto{\pgfqpoint{0.900670in}{1.118370in}}%
\pgfpathlineto{\pgfqpoint{0.901702in}{1.119456in}}%
\pgfpathlineto{\pgfqpoint{0.902797in}{1.120481in}}%
\pgfpathlineto{\pgfqpoint{0.903774in}{1.121567in}}%
\pgfpathlineto{\pgfqpoint{0.904829in}{1.122561in}}%
\pgfpathlineto{\pgfqpoint{0.905744in}{1.123616in}}%
\pgfpathlineto{\pgfqpoint{0.906776in}{1.124889in}}%
\pgfpathlineto{\pgfqpoint{0.908089in}{1.125975in}}%
\pgfpathlineto{\pgfqpoint{0.909199in}{1.127341in}}%
\pgfpathlineto{\pgfqpoint{0.910349in}{1.128428in}}%
\pgfpathlineto{\pgfqpoint{0.911451in}{1.129700in}}%
\pgfpathlineto{\pgfqpoint{0.912467in}{1.130787in}}%
\pgfpathlineto{\pgfqpoint{0.913507in}{1.131687in}}%
\pgfpathlineto{\pgfqpoint{0.913538in}{1.131687in}}%
\pgfpathlineto{\pgfqpoint{0.914468in}{1.132773in}}%
\pgfpathlineto{\pgfqpoint{0.915579in}{1.134170in}}%
\pgfpathlineto{\pgfqpoint{0.916861in}{1.135257in}}%
\pgfpathlineto{\pgfqpoint{0.917971in}{1.136623in}}%
\pgfpathlineto{\pgfqpoint{0.918995in}{1.137709in}}%
\pgfpathlineto{\pgfqpoint{0.920058in}{1.139168in}}%
\pgfpathlineto{\pgfqpoint{0.921622in}{1.140255in}}%
\pgfpathlineto{\pgfqpoint{0.922716in}{1.141434in}}%
\pgfpathlineto{\pgfqpoint{0.924108in}{1.142521in}}%
\pgfpathlineto{\pgfqpoint{0.925179in}{1.143855in}}%
\pgfpathlineto{\pgfqpoint{0.926218in}{1.144911in}}%
\pgfpathlineto{\pgfqpoint{0.927305in}{1.146122in}}%
\pgfpathlineto{\pgfqpoint{0.928548in}{1.147208in}}%
\pgfpathlineto{\pgfqpoint{0.929650in}{1.148294in}}%
\pgfpathlineto{\pgfqpoint{0.930925in}{1.149319in}}%
\pgfpathlineto{\pgfqpoint{0.932003in}{1.150467in}}%
\pgfpathlineto{\pgfqpoint{0.933880in}{1.151554in}}%
\pgfpathlineto{\pgfqpoint{0.934959in}{1.152796in}}%
\pgfpathlineto{\pgfqpoint{0.935842in}{1.153882in}}%
\pgfpathlineto{\pgfqpoint{0.936803in}{1.155217in}}%
\pgfpathlineto{\pgfqpoint{0.938273in}{1.156303in}}%
\pgfpathlineto{\pgfqpoint{0.939368in}{1.157390in}}%
\pgfpathlineto{\pgfqpoint{0.940572in}{1.158414in}}%
\pgfpathlineto{\pgfqpoint{0.941682in}{1.159780in}}%
\pgfpathlineto{\pgfqpoint{0.942893in}{1.160867in}}%
\pgfpathlineto{\pgfqpoint{0.943980in}{1.162046in}}%
\pgfpathlineto{\pgfqpoint{0.945387in}{1.163133in}}%
\pgfpathlineto{\pgfqpoint{0.946497in}{1.164281in}}%
\pgfpathlineto{\pgfqpoint{0.948170in}{1.165368in}}%
\pgfpathlineto{\pgfqpoint{0.949249in}{1.165957in}}%
\pgfpathlineto{\pgfqpoint{0.949280in}{1.165957in}}%
\pgfpathlineto{\pgfqpoint{0.950015in}{1.167044in}}%
\pgfpathlineto{\pgfqpoint{0.951008in}{1.167882in}}%
\pgfpathlineto{\pgfqpoint{0.952368in}{1.168969in}}%
\pgfpathlineto{\pgfqpoint{0.953416in}{1.170024in}}%
\pgfpathlineto{\pgfqpoint{0.954651in}{1.171110in}}%
\pgfpathlineto{\pgfqpoint{0.955753in}{1.172725in}}%
\pgfpathlineto{\pgfqpoint{0.957247in}{1.173811in}}%
\pgfpathlineto{\pgfqpoint{0.958310in}{1.175146in}}%
\pgfpathlineto{\pgfqpoint{0.959811in}{1.176201in}}%
\pgfpathlineto{\pgfqpoint{0.960897in}{1.177567in}}%
\pgfpathlineto{\pgfqpoint{0.961976in}{1.178654in}}%
\pgfpathlineto{\pgfqpoint{0.963063in}{1.179740in}}%
\pgfpathlineto{\pgfqpoint{0.964408in}{1.180827in}}%
\pgfpathlineto{\pgfqpoint{0.965486in}{1.181913in}}%
\pgfpathlineto{\pgfqpoint{0.966385in}{1.183000in}}%
\pgfpathlineto{\pgfqpoint{0.967464in}{1.184241in}}%
\pgfpathlineto{\pgfqpoint{0.968879in}{1.185297in}}%
\pgfpathlineto{\pgfqpoint{0.969989in}{1.186476in}}%
\pgfpathlineto{\pgfqpoint{0.971600in}{1.187563in}}%
\pgfpathlineto{\pgfqpoint{0.972632in}{1.188463in}}%
\pgfpathlineto{\pgfqpoint{0.972655in}{1.188463in}}%
\pgfpathlineto{\pgfqpoint{0.973812in}{1.189549in}}%
\pgfpathlineto{\pgfqpoint{0.974914in}{1.190853in}}%
\pgfpathlineto{\pgfqpoint{0.975884in}{1.191909in}}%
\pgfpathlineto{\pgfqpoint{0.976971in}{1.192778in}}%
\pgfpathlineto{\pgfqpoint{0.978386in}{1.193833in}}%
\pgfpathlineto{\pgfqpoint{0.979433in}{1.194827in}}%
\pgfpathlineto{\pgfqpoint{0.980457in}{1.195913in}}%
\pgfpathlineto{\pgfqpoint{0.981442in}{1.196813in}}%
\pgfpathlineto{\pgfqpoint{0.982888in}{1.197900in}}%
\pgfpathlineto{\pgfqpoint{0.983952in}{1.198893in}}%
\pgfpathlineto{\pgfqpoint{0.985750in}{1.199949in}}%
\pgfpathlineto{\pgfqpoint{0.986836in}{1.201190in}}%
\pgfpathlineto{\pgfqpoint{0.988017in}{1.202277in}}%
\pgfpathlineto{\pgfqpoint{0.989127in}{1.203612in}}%
\pgfpathlineto{\pgfqpoint{0.990268in}{1.204636in}}%
\pgfpathlineto{\pgfqpoint{0.991371in}{1.205785in}}%
\pgfpathlineto{\pgfqpoint{0.992614in}{1.206871in}}%
\pgfpathlineto{\pgfqpoint{0.993661in}{1.207802in}}%
\pgfpathlineto{\pgfqpoint{0.994482in}{1.208889in}}%
\pgfpathlineto{\pgfqpoint{0.995530in}{1.209727in}}%
\pgfpathlineto{\pgfqpoint{0.996851in}{1.210813in}}%
\pgfpathlineto{\pgfqpoint{0.997961in}{1.211838in}}%
\pgfpathlineto{\pgfqpoint{0.999079in}{1.212924in}}%
\pgfpathlineto{\pgfqpoint{1.000189in}{1.214445in}}%
\pgfpathlineto{\pgfqpoint{1.001463in}{1.215501in}}%
\pgfpathlineto{\pgfqpoint{1.002573in}{1.216494in}}%
\pgfpathlineto{\pgfqpoint{1.004082in}{1.217581in}}%
\pgfpathlineto{\pgfqpoint{1.005153in}{1.218481in}}%
\pgfpathlineto{\pgfqpoint{1.006474in}{1.219567in}}%
\pgfpathlineto{\pgfqpoint{1.007545in}{1.220902in}}%
\pgfpathlineto{\pgfqpoint{1.008890in}{1.221989in}}%
\pgfpathlineto{\pgfqpoint{1.009953in}{1.223044in}}%
\pgfpathlineto{\pgfqpoint{1.009992in}{1.223044in}}%
\pgfpathlineto{\pgfqpoint{1.011095in}{1.224130in}}%
\pgfpathlineto{\pgfqpoint{1.012205in}{1.224782in}}%
\pgfpathlineto{\pgfqpoint{1.013463in}{1.225869in}}%
\pgfpathlineto{\pgfqpoint{1.014472in}{1.226893in}}%
\pgfpathlineto{\pgfqpoint{1.016137in}{1.227980in}}%
\pgfpathlineto{\pgfqpoint{1.017177in}{1.228911in}}%
\pgfpathlineto{\pgfqpoint{1.018678in}{1.229997in}}%
\pgfpathlineto{\pgfqpoint{1.019756in}{1.230742in}}%
\pgfpathlineto{\pgfqpoint{1.020953in}{1.231829in}}%
\pgfpathlineto{\pgfqpoint{1.022039in}{1.232977in}}%
\pgfpathlineto{\pgfqpoint{1.023173in}{1.234064in}}%
\pgfpathlineto{\pgfqpoint{1.024283in}{1.235150in}}%
\pgfpathlineto{\pgfqpoint{1.025518in}{1.236237in}}%
\pgfpathlineto{\pgfqpoint{1.026613in}{1.237075in}}%
\pgfpathlineto{\pgfqpoint{1.028043in}{1.238161in}}%
\pgfpathlineto{\pgfqpoint{1.029106in}{1.239403in}}%
\pgfpathlineto{\pgfqpoint{1.030490in}{1.240490in}}%
\pgfpathlineto{\pgfqpoint{1.031428in}{1.241204in}}%
\pgfpathlineto{\pgfqpoint{1.031452in}{1.241204in}}%
\pgfpathlineto{\pgfqpoint{1.032718in}{1.242290in}}%
\pgfpathlineto{\pgfqpoint{1.033789in}{1.243159in}}%
\pgfpathlineto{\pgfqpoint{1.035384in}{1.244246in}}%
\pgfpathlineto{\pgfqpoint{1.036494in}{1.245177in}}%
\pgfpathlineto{\pgfqpoint{1.037682in}{1.246263in}}%
\pgfpathlineto{\pgfqpoint{1.038769in}{1.247474in}}%
\pgfpathlineto{\pgfqpoint{1.039973in}{1.248561in}}%
\pgfpathlineto{\pgfqpoint{1.041044in}{1.249244in}}%
\pgfpathlineto{\pgfqpoint{1.042482in}{1.250330in}}%
\pgfpathlineto{\pgfqpoint{1.043592in}{1.251354in}}%
\pgfpathlineto{\pgfqpoint{1.045133in}{1.252348in}}%
\pgfpathlineto{\pgfqpoint{1.045133in}{1.252410in}}%
\pgfpathlineto{\pgfqpoint{1.046227in}{1.253279in}}%
\pgfpathlineto{\pgfqpoint{1.048009in}{1.254365in}}%
\pgfpathlineto{\pgfqpoint{1.049104in}{1.255328in}}%
\pgfpathlineto{\pgfqpoint{1.050128in}{1.256414in}}%
\pgfpathlineto{\pgfqpoint{1.051222in}{1.257470in}}%
\pgfpathlineto{\pgfqpoint{1.052903in}{1.258556in}}%
\pgfpathlineto{\pgfqpoint{1.053990in}{1.259643in}}%
\pgfpathlineto{\pgfqpoint{1.055295in}{1.260729in}}%
\pgfpathlineto{\pgfqpoint{1.056398in}{1.261660in}}%
\pgfpathlineto{\pgfqpoint{1.057946in}{1.262747in}}%
\pgfpathlineto{\pgfqpoint{1.058970in}{1.263740in}}%
\pgfpathlineto{\pgfqpoint{1.060869in}{1.264827in}}%
\pgfpathlineto{\pgfqpoint{1.061901in}{1.265416in}}%
\pgfpathlineto{\pgfqpoint{1.063395in}{1.266503in}}%
\pgfpathlineto{\pgfqpoint{1.064473in}{1.267403in}}%
\pgfpathlineto{\pgfqpoint{1.065818in}{1.268459in}}%
\pgfpathlineto{\pgfqpoint{1.066811in}{1.269110in}}%
\pgfpathlineto{\pgfqpoint{1.068077in}{1.270197in}}%
\pgfpathlineto{\pgfqpoint{1.069117in}{1.271035in}}%
\pgfpathlineto{\pgfqpoint{1.069187in}{1.271035in}}%
\pgfpathlineto{\pgfqpoint{1.070595in}{1.272122in}}%
\pgfpathlineto{\pgfqpoint{1.071681in}{1.273084in}}%
\pgfpathlineto{\pgfqpoint{1.073456in}{1.274170in}}%
\pgfpathlineto{\pgfqpoint{1.074527in}{1.274915in}}%
\pgfpathlineto{\pgfqpoint{1.075887in}{1.275971in}}%
\pgfpathlineto{\pgfqpoint{1.076794in}{1.276467in}}%
\pgfpathlineto{\pgfqpoint{1.078623in}{1.277554in}}%
\pgfpathlineto{\pgfqpoint{1.079710in}{1.278423in}}%
\pgfpathlineto{\pgfqpoint{1.081625in}{1.279510in}}%
\pgfpathlineto{\pgfqpoint{1.082665in}{1.280534in}}%
\pgfpathlineto{\pgfqpoint{1.084690in}{1.281620in}}%
\pgfpathlineto{\pgfqpoint{1.085745in}{1.282272in}}%
\pgfpathlineto{\pgfqpoint{1.087285in}{1.283359in}}%
\pgfpathlineto{\pgfqpoint{1.088270in}{1.284011in}}%
\pgfpathlineto{\pgfqpoint{1.089849in}{1.285066in}}%
\pgfpathlineto{\pgfqpoint{1.090897in}{1.286059in}}%
\pgfpathlineto{\pgfqpoint{1.092101in}{1.287146in}}%
\pgfpathlineto{\pgfqpoint{1.093195in}{1.288077in}}%
\pgfpathlineto{\pgfqpoint{1.094814in}{1.289164in}}%
\pgfpathlineto{\pgfqpoint{1.095916in}{1.290188in}}%
\pgfpathlineto{\pgfqpoint{1.097362in}{1.291275in}}%
\pgfpathlineto{\pgfqpoint{1.098371in}{1.291895in}}%
\pgfpathlineto{\pgfqpoint{1.099809in}{1.292951in}}%
\pgfpathlineto{\pgfqpoint{1.100919in}{1.293727in}}%
\pgfpathlineto{\pgfqpoint{1.102154in}{1.294813in}}%
\pgfpathlineto{\pgfqpoint{1.103155in}{1.295527in}}%
\pgfpathlineto{\pgfqpoint{1.104977in}{1.296614in}}%
\pgfpathlineto{\pgfqpoint{1.106055in}{1.297390in}}%
\pgfpathlineto{\pgfqpoint{1.107291in}{1.298476in}}%
\pgfpathlineto{\pgfqpoint{1.108401in}{1.299252in}}%
\pgfpathlineto{\pgfqpoint{1.110050in}{1.300339in}}%
\pgfpathlineto{\pgfqpoint{1.111090in}{1.301146in}}%
\pgfpathlineto{\pgfqpoint{1.112661in}{1.302201in}}%
\pgfpathlineto{\pgfqpoint{1.113771in}{1.303288in}}%
\pgfpathlineto{\pgfqpoint{1.115038in}{1.304374in}}%
\pgfpathlineto{\pgfqpoint{1.116148in}{1.304995in}}%
\pgfpathlineto{\pgfqpoint{1.117876in}{1.306082in}}%
\pgfpathlineto{\pgfqpoint{1.118947in}{1.306547in}}%
\pgfpathlineto{\pgfqpoint{1.120682in}{1.307634in}}%
\pgfpathlineto{\pgfqpoint{1.121769in}{1.308379in}}%
\pgfpathlineto{\pgfqpoint{1.123536in}{1.309465in}}%
\pgfpathlineto{\pgfqpoint{1.124591in}{1.310210in}}%
\pgfpathlineto{\pgfqpoint{1.125998in}{1.311297in}}%
\pgfpathlineto{\pgfqpoint{1.127108in}{1.312352in}}%
\pgfpathlineto{\pgfqpoint{1.128547in}{1.313439in}}%
\pgfpathlineto{\pgfqpoint{1.129602in}{1.314028in}}%
\pgfpathlineto{\pgfqpoint{1.131197in}{1.315084in}}%
\pgfpathlineto{\pgfqpoint{1.131197in}{1.315115in}}%
\pgfpathlineto{\pgfqpoint{1.132291in}{1.316046in}}%
\pgfpathlineto{\pgfqpoint{1.134074in}{1.317133in}}%
\pgfpathlineto{\pgfqpoint{1.135161in}{1.317567in}}%
\pgfpathlineto{\pgfqpoint{1.137005in}{1.318654in}}%
\pgfpathlineto{\pgfqpoint{1.137990in}{1.319461in}}%
\pgfpathlineto{\pgfqpoint{1.140258in}{1.320547in}}%
\pgfpathlineto{\pgfqpoint{1.141250in}{1.321106in}}%
\pgfpathlineto{\pgfqpoint{1.143127in}{1.322193in}}%
\pgfpathlineto{\pgfqpoint{1.144221in}{1.322938in}}%
\pgfpathlineto{\pgfqpoint{1.146207in}{1.324024in}}%
\pgfpathlineto{\pgfqpoint{1.147254in}{1.324986in}}%
\pgfpathlineto{\pgfqpoint{1.149076in}{1.326073in}}%
\pgfpathlineto{\pgfqpoint{1.150178in}{1.326694in}}%
\pgfpathlineto{\pgfqpoint{1.151593in}{1.327780in}}%
\pgfpathlineto{\pgfqpoint{1.152594in}{1.328711in}}%
\pgfpathlineto{\pgfqpoint{1.154650in}{1.329798in}}%
\pgfpathlineto{\pgfqpoint{1.155760in}{1.330481in}}%
\pgfpathlineto{\pgfqpoint{1.157542in}{1.331567in}}%
\pgfpathlineto{\pgfqpoint{1.158621in}{1.332281in}}%
\pgfpathlineto{\pgfqpoint{1.160607in}{1.333368in}}%
\pgfpathlineto{\pgfqpoint{1.161670in}{1.334051in}}%
\pgfpathlineto{\pgfqpoint{1.163140in}{1.335137in}}%
\pgfpathlineto{\pgfqpoint{1.164234in}{1.335882in}}%
\pgfpathlineto{\pgfqpoint{1.165524in}{1.336969in}}%
\pgfpathlineto{\pgfqpoint{1.166634in}{1.337962in}}%
\pgfpathlineto{\pgfqpoint{1.168417in}{1.339017in}}%
\pgfpathlineto{\pgfqpoint{1.169511in}{1.339824in}}%
\pgfpathlineto{\pgfqpoint{1.172193in}{1.340911in}}%
\pgfpathlineto{\pgfqpoint{1.173154in}{1.341718in}}%
\pgfpathlineto{\pgfqpoint{1.175070in}{1.342773in}}%
\pgfpathlineto{\pgfqpoint{1.176062in}{1.343705in}}%
\pgfpathlineto{\pgfqpoint{1.177665in}{1.344791in}}%
\pgfpathlineto{\pgfqpoint{1.178736in}{1.345474in}}%
\pgfpathlineto{\pgfqpoint{1.180354in}{1.346530in}}%
\pgfpathlineto{\pgfqpoint{1.181308in}{1.347492in}}%
\pgfpathlineto{\pgfqpoint{1.182801in}{1.348578in}}%
\pgfpathlineto{\pgfqpoint{1.183708in}{1.349044in}}%
\pgfpathlineto{\pgfqpoint{1.185365in}{1.350130in}}%
\pgfpathlineto{\pgfqpoint{1.186421in}{1.351000in}}%
\pgfpathlineto{\pgfqpoint{1.187851in}{1.352086in}}%
\pgfpathlineto{\pgfqpoint{1.188860in}{1.352893in}}%
\pgfpathlineto{\pgfqpoint{1.188899in}{1.352893in}}%
\pgfpathlineto{\pgfqpoint{1.190416in}{1.353980in}}%
\pgfpathlineto{\pgfqpoint{1.191315in}{1.354507in}}%
\pgfpathlineto{\pgfqpoint{1.191494in}{1.354507in}}%
\pgfpathlineto{\pgfqpoint{1.192917in}{1.355594in}}%
\pgfpathlineto{\pgfqpoint{1.194020in}{1.356339in}}%
\pgfpathlineto{\pgfqpoint{1.195646in}{1.357425in}}%
\pgfpathlineto{\pgfqpoint{1.196724in}{1.358046in}}%
\pgfpathlineto{\pgfqpoint{1.198710in}{1.359133in}}%
\pgfpathlineto{\pgfqpoint{1.199765in}{1.359847in}}%
\pgfpathlineto{\pgfqpoint{1.201822in}{1.360933in}}%
\pgfpathlineto{\pgfqpoint{1.202861in}{1.361647in}}%
\pgfpathlineto{\pgfqpoint{1.204222in}{1.362734in}}%
\pgfpathlineto{\pgfqpoint{1.205277in}{1.363354in}}%
\pgfpathlineto{\pgfqpoint{1.206934in}{1.364441in}}%
\pgfpathlineto{\pgfqpoint{1.208005in}{1.365000in}}%
\pgfpathlineto{\pgfqpoint{1.209960in}{1.366086in}}%
\pgfpathlineto{\pgfqpoint{1.211039in}{1.366831in}}%
\pgfpathlineto{\pgfqpoint{1.213423in}{1.367918in}}%
\pgfpathlineto{\pgfqpoint{1.214510in}{1.368663in}}%
\pgfpathlineto{\pgfqpoint{1.215885in}{1.369749in}}%
\pgfpathlineto{\pgfqpoint{1.216988in}{1.370494in}}%
\pgfpathlineto{\pgfqpoint{1.218856in}{1.371581in}}%
\pgfpathlineto{\pgfqpoint{1.219833in}{1.372263in}}%
\pgfpathlineto{\pgfqpoint{1.219919in}{1.372263in}}%
\pgfpathlineto{\pgfqpoint{1.221577in}{1.373319in}}%
\pgfpathlineto{\pgfqpoint{1.222640in}{1.373816in}}%
\pgfpathlineto{\pgfqpoint{1.224633in}{1.374902in}}%
\pgfpathlineto{\pgfqpoint{1.225728in}{1.375740in}}%
\pgfpathlineto{\pgfqpoint{1.227159in}{1.376827in}}%
\pgfpathlineto{\pgfqpoint{1.228245in}{1.377727in}}%
\pgfpathlineto{\pgfqpoint{1.230372in}{1.378813in}}%
\pgfpathlineto{\pgfqpoint{1.231482in}{1.379496in}}%
\pgfpathlineto{\pgfqpoint{1.232865in}{1.380583in}}%
\pgfpathlineto{\pgfqpoint{1.233952in}{1.381079in}}%
\pgfpathlineto{\pgfqpoint{1.236759in}{1.382104in}}%
\pgfpathlineto{\pgfqpoint{1.237845in}{1.382756in}}%
\pgfpathlineto{\pgfqpoint{1.239745in}{1.383842in}}%
\pgfpathlineto{\pgfqpoint{1.240808in}{1.384432in}}%
\pgfpathlineto{\pgfqpoint{1.240839in}{1.384432in}}%
\pgfpathlineto{\pgfqpoint{1.242411in}{1.385518in}}%
\pgfpathlineto{\pgfqpoint{1.243388in}{1.385953in}}%
\pgfpathlineto{\pgfqpoint{1.245741in}{1.387008in}}%
\pgfpathlineto{\pgfqpoint{1.246851in}{1.387816in}}%
\pgfpathlineto{\pgfqpoint{1.248923in}{1.388902in}}%
\pgfpathlineto{\pgfqpoint{1.249931in}{1.389368in}}%
\pgfpathlineto{\pgfqpoint{1.251909in}{1.390454in}}%
\pgfpathlineto{\pgfqpoint{1.253011in}{1.390951in}}%
\pgfpathlineto{\pgfqpoint{1.255239in}{1.392037in}}%
\pgfpathlineto{\pgfqpoint{1.256303in}{1.392596in}}%
\pgfpathlineto{\pgfqpoint{1.258515in}{1.393683in}}%
\pgfpathlineto{\pgfqpoint{1.259492in}{1.394490in}}%
\pgfpathlineto{\pgfqpoint{1.261705in}{1.395576in}}%
\pgfpathlineto{\pgfqpoint{1.262713in}{1.396135in}}%
\pgfpathlineto{\pgfqpoint{1.264191in}{1.397221in}}%
\pgfpathlineto{\pgfqpoint{1.265230in}{1.397780in}}%
\pgfpathlineto{\pgfqpoint{1.267685in}{1.398867in}}%
\pgfpathlineto{\pgfqpoint{1.268764in}{1.399518in}}%
\pgfpathlineto{\pgfqpoint{1.270328in}{1.400605in}}%
\pgfpathlineto{\pgfqpoint{1.271422in}{1.401319in}}%
\pgfpathlineto{\pgfqpoint{1.274119in}{1.402405in}}%
\pgfpathlineto{\pgfqpoint{1.275190in}{1.403150in}}%
\pgfpathlineto{\pgfqpoint{1.278106in}{1.404237in}}%
\pgfpathlineto{\pgfqpoint{1.279216in}{1.404889in}}%
\pgfpathlineto{\pgfqpoint{1.281397in}{1.405975in}}%
\pgfpathlineto{\pgfqpoint{1.282507in}{1.406534in}}%
\pgfpathlineto{\pgfqpoint{1.284947in}{1.407620in}}%
\pgfpathlineto{\pgfqpoint{1.285994in}{1.408396in}}%
\pgfpathlineto{\pgfqpoint{1.286041in}{1.408396in}}%
\pgfpathlineto{\pgfqpoint{1.288629in}{1.409483in}}%
\pgfpathlineto{\pgfqpoint{1.289614in}{1.410011in}}%
\pgfpathlineto{\pgfqpoint{1.289645in}{1.410011in}}%
\pgfpathlineto{\pgfqpoint{1.292076in}{1.411097in}}%
\pgfpathlineto{\pgfqpoint{1.293147in}{1.411718in}}%
\pgfpathlineto{\pgfqpoint{1.293186in}{1.411718in}}%
\pgfpathlineto{\pgfqpoint{1.295195in}{1.412804in}}%
\pgfpathlineto{\pgfqpoint{1.296462in}{1.413612in}}%
\pgfpathlineto{\pgfqpoint{1.298526in}{1.414667in}}%
\pgfpathlineto{\pgfqpoint{1.299636in}{1.415443in}}%
\pgfpathlineto{\pgfqpoint{1.300871in}{1.416530in}}%
\pgfpathlineto{\pgfqpoint{1.301958in}{1.416840in}}%
\pgfpathlineto{\pgfqpoint{1.303896in}{1.417864in}}%
\pgfpathlineto{\pgfqpoint{1.304968in}{1.418485in}}%
\pgfpathlineto{\pgfqpoint{1.306859in}{1.419572in}}%
\pgfpathlineto{\pgfqpoint{1.307907in}{1.420099in}}%
\pgfpathlineto{\pgfqpoint{1.310096in}{1.421186in}}%
\pgfpathlineto{\pgfqpoint{1.311143in}{1.421838in}}%
\pgfpathlineto{\pgfqpoint{1.314286in}{1.422924in}}%
\pgfpathlineto{\pgfqpoint{1.315381in}{1.423545in}}%
\pgfpathlineto{\pgfqpoint{1.318000in}{1.424600in}}%
\pgfpathlineto{\pgfqpoint{1.319071in}{1.425097in}}%
\pgfpathlineto{\pgfqpoint{1.320650in}{1.426153in}}%
\pgfpathlineto{\pgfqpoint{1.321643in}{1.426898in}}%
\pgfpathlineto{\pgfqpoint{1.324418in}{1.427984in}}%
\pgfpathlineto{\pgfqpoint{1.325528in}{1.428450in}}%
\pgfpathlineto{\pgfqpoint{1.327482in}{1.429536in}}%
\pgfpathlineto{\pgfqpoint{1.328452in}{1.430250in}}%
\pgfpathlineto{\pgfqpoint{1.331039in}{1.431337in}}%
\pgfpathlineto{\pgfqpoint{1.332134in}{1.431926in}}%
\pgfpathlineto{\pgfqpoint{1.334033in}{1.433013in}}%
\pgfpathlineto{\pgfqpoint{1.335026in}{1.433603in}}%
\pgfpathlineto{\pgfqpoint{1.337317in}{1.434689in}}%
\pgfpathlineto{\pgfqpoint{1.338052in}{1.435279in}}%
\pgfpathlineto{\pgfqpoint{1.338364in}{1.435279in}}%
\pgfpathlineto{\pgfqpoint{1.340507in}{1.436365in}}%
\pgfpathlineto{\pgfqpoint{1.341593in}{1.436831in}}%
\pgfpathlineto{\pgfqpoint{1.344126in}{1.437918in}}%
\pgfpathlineto{\pgfqpoint{1.345228in}{1.438445in}}%
\pgfpathlineto{\pgfqpoint{1.347793in}{1.439532in}}%
\pgfpathlineto{\pgfqpoint{1.348809in}{1.439935in}}%
\pgfpathlineto{\pgfqpoint{1.351967in}{1.441022in}}%
\pgfpathlineto{\pgfqpoint{1.352929in}{1.441674in}}%
\pgfpathlineto{\pgfqpoint{1.355837in}{1.442760in}}%
\pgfpathlineto{\pgfqpoint{1.356924in}{1.443257in}}%
\pgfpathlineto{\pgfqpoint{1.359339in}{1.444343in}}%
\pgfpathlineto{\pgfqpoint{1.360223in}{1.444685in}}%
\pgfpathlineto{\pgfqpoint{1.362849in}{1.445771in}}%
\pgfpathlineto{\pgfqpoint{1.363959in}{1.446392in}}%
\pgfpathlineto{\pgfqpoint{1.366289in}{1.447447in}}%
\pgfpathlineto{\pgfqpoint{1.367337in}{1.447913in}}%
\pgfpathlineto{\pgfqpoint{1.370370in}{1.449000in}}%
\pgfpathlineto{\pgfqpoint{1.371425in}{1.449620in}}%
\pgfpathlineto{\pgfqpoint{1.374654in}{1.450707in}}%
\pgfpathlineto{\pgfqpoint{1.375764in}{1.451266in}}%
\pgfpathlineto{\pgfqpoint{1.378563in}{1.452352in}}%
\pgfpathlineto{\pgfqpoint{1.379610in}{1.452725in}}%
\pgfpathlineto{\pgfqpoint{1.381956in}{1.453811in}}%
\pgfpathlineto{\pgfqpoint{1.382691in}{1.454184in}}%
\pgfpathlineto{\pgfqpoint{1.382948in}{1.454184in}}%
\pgfpathlineto{\pgfqpoint{1.385028in}{1.455270in}}%
\pgfpathlineto{\pgfqpoint{1.386083in}{1.455798in}}%
\pgfpathlineto{\pgfqpoint{1.386099in}{1.455798in}}%
\pgfpathlineto{\pgfqpoint{1.388296in}{1.456884in}}%
\pgfpathlineto{\pgfqpoint{1.389382in}{1.457381in}}%
\pgfpathlineto{\pgfqpoint{1.392635in}{1.458467in}}%
\pgfpathlineto{\pgfqpoint{1.393745in}{1.458995in}}%
\pgfpathlineto{\pgfqpoint{1.395941in}{1.460082in}}%
\pgfpathlineto{\pgfqpoint{1.396919in}{1.460640in}}%
\pgfpathlineto{\pgfqpoint{1.399655in}{1.461696in}}%
\pgfpathlineto{\pgfqpoint{1.400718in}{1.462224in}}%
\pgfpathlineto{\pgfqpoint{1.400765in}{1.462224in}}%
\pgfpathlineto{\pgfqpoint{1.403571in}{1.463310in}}%
\pgfpathlineto{\pgfqpoint{1.404541in}{1.463838in}}%
\pgfpathlineto{\pgfqpoint{1.406996in}{1.464924in}}%
\pgfpathlineto{\pgfqpoint{1.408090in}{1.465483in}}%
\pgfpathlineto{\pgfqpoint{1.411467in}{1.466569in}}%
\pgfpathlineto{\pgfqpoint{1.412382in}{1.466942in}}%
\pgfpathlineto{\pgfqpoint{1.416650in}{1.468028in}}%
\pgfpathlineto{\pgfqpoint{1.417542in}{1.468308in}}%
\pgfpathlineto{\pgfqpoint{1.417674in}{1.468308in}}%
\pgfpathlineto{\pgfqpoint{1.420286in}{1.469394in}}%
\pgfpathlineto{\pgfqpoint{1.421357in}{1.469829in}}%
\pgfpathlineto{\pgfqpoint{1.425187in}{1.470915in}}%
\pgfpathlineto{\pgfqpoint{1.426227in}{1.471536in}}%
\pgfpathlineto{\pgfqpoint{1.429205in}{1.472623in}}%
\pgfpathlineto{\pgfqpoint{1.430214in}{1.473212in}}%
\pgfpathlineto{\pgfqpoint{1.432778in}{1.474299in}}%
\pgfpathlineto{\pgfqpoint{1.433794in}{1.475013in}}%
\pgfpathlineto{\pgfqpoint{1.433865in}{1.475013in}}%
\pgfpathlineto{\pgfqpoint{1.436820in}{1.476099in}}%
\pgfpathlineto{\pgfqpoint{1.437922in}{1.476658in}}%
\pgfpathlineto{\pgfqpoint{1.440791in}{1.477745in}}%
\pgfpathlineto{\pgfqpoint{1.441886in}{1.478303in}}%
\pgfpathlineto{\pgfqpoint{1.444552in}{1.479390in}}%
\pgfpathlineto{\pgfqpoint{1.445466in}{1.479949in}}%
\pgfpathlineto{\pgfqpoint{1.448890in}{1.481035in}}%
\pgfpathlineto{\pgfqpoint{1.449914in}{1.481532in}}%
\pgfpathlineto{\pgfqpoint{1.449993in}{1.481532in}}%
\pgfpathlineto{\pgfqpoint{1.453315in}{1.482587in}}%
\pgfpathlineto{\pgfqpoint{1.454355in}{1.483053in}}%
\pgfpathlineto{\pgfqpoint{1.457412in}{1.484139in}}%
\pgfpathlineto{\pgfqpoint{1.458631in}{1.484791in}}%
\pgfpathlineto{\pgfqpoint{1.461750in}{1.485878in}}%
\pgfpathlineto{\pgfqpoint{1.462782in}{1.486281in}}%
\pgfpathlineto{\pgfqpoint{1.466144in}{1.487368in}}%
\pgfpathlineto{\pgfqpoint{1.467113in}{1.488082in}}%
\pgfpathlineto{\pgfqpoint{1.467238in}{1.488082in}}%
\pgfpathlineto{\pgfqpoint{1.469834in}{1.489168in}}%
\pgfpathlineto{\pgfqpoint{1.470811in}{1.489820in}}%
\pgfpathlineto{\pgfqpoint{1.470842in}{1.489820in}}%
\pgfpathlineto{\pgfqpoint{1.474438in}{1.490906in}}%
\pgfpathlineto{\pgfqpoint{1.475525in}{1.491403in}}%
\pgfpathlineto{\pgfqpoint{1.478543in}{1.492490in}}%
\pgfpathlineto{\pgfqpoint{1.479489in}{1.493173in}}%
\pgfpathlineto{\pgfqpoint{1.479528in}{1.493173in}}%
\pgfpathlineto{\pgfqpoint{1.482920in}{1.494259in}}%
\pgfpathlineto{\pgfqpoint{1.484023in}{1.494663in}}%
\pgfpathlineto{\pgfqpoint{1.486579in}{1.495749in}}%
\pgfpathlineto{\pgfqpoint{1.487439in}{1.496028in}}%
\pgfpathlineto{\pgfqpoint{1.490808in}{1.497115in}}%
\pgfpathlineto{\pgfqpoint{1.491770in}{1.497332in}}%
\pgfpathlineto{\pgfqpoint{1.491911in}{1.497332in}}%
\pgfpathlineto{\pgfqpoint{1.496101in}{1.498419in}}%
\pgfpathlineto{\pgfqpoint{1.497188in}{1.498977in}}%
\pgfpathlineto{\pgfqpoint{1.499893in}{1.500064in}}%
\pgfpathlineto{\pgfqpoint{1.500940in}{1.500530in}}%
\pgfpathlineto{\pgfqpoint{1.503864in}{1.501616in}}%
\pgfpathlineto{\pgfqpoint{1.504919in}{1.502237in}}%
\pgfpathlineto{\pgfqpoint{1.507679in}{1.503292in}}%
\pgfpathlineto{\pgfqpoint{1.508758in}{1.503820in}}%
\pgfpathlineto{\pgfqpoint{1.513042in}{1.504906in}}%
\pgfpathlineto{\pgfqpoint{1.514097in}{1.505217in}}%
\pgfpathlineto{\pgfqpoint{1.514144in}{1.505217in}}%
\pgfpathlineto{\pgfqpoint{1.516802in}{1.506303in}}%
\pgfpathlineto{\pgfqpoint{1.517865in}{1.506924in}}%
\pgfpathlineto{\pgfqpoint{1.517912in}{1.506924in}}%
\pgfpathlineto{\pgfqpoint{1.520985in}{1.508011in}}%
\pgfpathlineto{\pgfqpoint{1.521931in}{1.508383in}}%
\pgfpathlineto{\pgfqpoint{1.525183in}{1.509470in}}%
\pgfpathlineto{\pgfqpoint{1.526285in}{1.509935in}}%
\pgfpathlineto{\pgfqpoint{1.530233in}{1.511022in}}%
\pgfpathlineto{\pgfqpoint{1.530819in}{1.511332in}}%
\pgfpathlineto{\pgfqpoint{1.531304in}{1.511332in}}%
\pgfpathlineto{\pgfqpoint{1.536104in}{1.512419in}}%
\pgfpathlineto{\pgfqpoint{1.537105in}{1.512636in}}%
\pgfpathlineto{\pgfqpoint{1.540458in}{1.513722in}}%
\pgfpathlineto{\pgfqpoint{1.541529in}{1.514312in}}%
\pgfpathlineto{\pgfqpoint{1.545086in}{1.515368in}}%
\pgfpathlineto{\pgfqpoint{1.546079in}{1.515895in}}%
\pgfpathlineto{\pgfqpoint{1.551724in}{1.516982in}}%
\pgfpathlineto{\pgfqpoint{1.552709in}{1.517354in}}%
\pgfpathlineto{\pgfqpoint{1.552771in}{1.517354in}}%
\pgfpathlineto{\pgfqpoint{1.556219in}{1.518441in}}%
\pgfpathlineto{\pgfqpoint{1.557212in}{1.518751in}}%
\pgfpathlineto{\pgfqpoint{1.557329in}{1.518751in}}%
\pgfpathlineto{\pgfqpoint{1.560753in}{1.519838in}}%
\pgfpathlineto{\pgfqpoint{1.561808in}{1.520241in}}%
\pgfpathlineto{\pgfqpoint{1.565694in}{1.521328in}}%
\pgfpathlineto{\pgfqpoint{1.566632in}{1.521700in}}%
\pgfpathlineto{\pgfqpoint{1.570572in}{1.522787in}}%
\pgfpathlineto{\pgfqpoint{1.571510in}{1.523252in}}%
\pgfpathlineto{\pgfqpoint{1.574778in}{1.524339in}}%
\pgfpathlineto{\pgfqpoint{1.575849in}{1.524773in}}%
\pgfpathlineto{\pgfqpoint{1.578436in}{1.525860in}}%
\pgfpathlineto{\pgfqpoint{1.579515in}{1.526232in}}%
\pgfpathlineto{\pgfqpoint{1.579539in}{1.526232in}}%
\pgfpathlineto{\pgfqpoint{1.585035in}{1.527319in}}%
\pgfpathlineto{\pgfqpoint{1.586051in}{1.527816in}}%
\pgfpathlineto{\pgfqpoint{1.589334in}{1.528902in}}%
\pgfpathlineto{\pgfqpoint{1.590366in}{1.529275in}}%
\pgfpathlineto{\pgfqpoint{1.590397in}{1.529275in}}%
\pgfpathlineto{\pgfqpoint{1.595393in}{1.530361in}}%
\pgfpathlineto{\pgfqpoint{1.596503in}{1.530734in}}%
\pgfpathlineto{\pgfqpoint{1.601303in}{1.531820in}}%
\pgfpathlineto{\pgfqpoint{1.602358in}{1.532193in}}%
\pgfpathlineto{\pgfqpoint{1.605611in}{1.533279in}}%
\pgfpathlineto{\pgfqpoint{1.606322in}{1.533496in}}%
\pgfpathlineto{\pgfqpoint{1.610903in}{1.534583in}}%
\pgfpathlineto{\pgfqpoint{1.611896in}{1.534893in}}%
\pgfpathlineto{\pgfqpoint{1.615523in}{1.535980in}}%
\pgfpathlineto{\pgfqpoint{1.616407in}{1.536166in}}%
\pgfpathlineto{\pgfqpoint{1.619862in}{1.537252in}}%
\pgfpathlineto{\pgfqpoint{1.620957in}{1.537718in}}%
\pgfpathlineto{\pgfqpoint{1.624897in}{1.538804in}}%
\pgfpathlineto{\pgfqpoint{1.625905in}{1.539084in}}%
\pgfpathlineto{\pgfqpoint{1.629791in}{1.540170in}}%
\pgfpathlineto{\pgfqpoint{1.630830in}{1.540667in}}%
\pgfpathlineto{\pgfqpoint{1.634809in}{1.541753in}}%
\pgfpathlineto{\pgfqpoint{1.635904in}{1.541971in}}%
\pgfpathlineto{\pgfqpoint{1.639438in}{1.543026in}}%
\pgfpathlineto{\pgfqpoint{1.640430in}{1.543554in}}%
\pgfpathlineto{\pgfqpoint{1.644574in}{1.544640in}}%
\pgfpathlineto{\pgfqpoint{1.645684in}{1.544920in}}%
\pgfpathlineto{\pgfqpoint{1.650695in}{1.546006in}}%
\pgfpathlineto{\pgfqpoint{1.651719in}{1.546348in}}%
\pgfpathlineto{\pgfqpoint{1.651766in}{1.546348in}}%
\pgfpathlineto{\pgfqpoint{1.656605in}{1.547434in}}%
\pgfpathlineto{\pgfqpoint{1.657707in}{1.547838in}}%
\pgfpathlineto{\pgfqpoint{1.661624in}{1.548924in}}%
\pgfpathlineto{\pgfqpoint{1.662460in}{1.549235in}}%
\pgfpathlineto{\pgfqpoint{1.666682in}{1.550321in}}%
\pgfpathlineto{\pgfqpoint{1.667683in}{1.550507in}}%
\pgfpathlineto{\pgfqpoint{1.671435in}{1.551594in}}%
\pgfpathlineto{\pgfqpoint{1.672389in}{1.551749in}}%
\pgfpathlineto{\pgfqpoint{1.677588in}{1.552836in}}%
\pgfpathlineto{\pgfqpoint{1.678690in}{1.553115in}}%
\pgfpathlineto{\pgfqpoint{1.683951in}{1.554201in}}%
\pgfpathlineto{\pgfqpoint{1.684913in}{1.554450in}}%
\pgfpathlineto{\pgfqpoint{1.689009in}{1.555536in}}%
\pgfpathlineto{\pgfqpoint{1.690018in}{1.555785in}}%
\pgfpathlineto{\pgfqpoint{1.694372in}{1.556871in}}%
\pgfpathlineto{\pgfqpoint{1.695459in}{1.557088in}}%
\pgfpathlineto{\pgfqpoint{1.700556in}{1.558175in}}%
\pgfpathlineto{\pgfqpoint{1.701611in}{1.558485in}}%
\pgfpathlineto{\pgfqpoint{1.707834in}{1.559572in}}%
\pgfpathlineto{\pgfqpoint{1.708866in}{1.559820in}}%
\pgfpathlineto{\pgfqpoint{1.708897in}{1.559820in}}%
\pgfpathlineto{\pgfqpoint{1.713869in}{1.560906in}}%
\pgfpathlineto{\pgfqpoint{1.714893in}{1.561186in}}%
\pgfpathlineto{\pgfqpoint{1.719256in}{1.562272in}}%
\pgfpathlineto{\pgfqpoint{1.720327in}{1.562552in}}%
\pgfpathlineto{\pgfqpoint{1.724681in}{1.563638in}}%
\pgfpathlineto{\pgfqpoint{1.725580in}{1.563731in}}%
\pgfpathlineto{\pgfqpoint{1.732092in}{1.564818in}}%
\pgfpathlineto{\pgfqpoint{1.733030in}{1.565190in}}%
\pgfpathlineto{\pgfqpoint{1.739073in}{1.566277in}}%
\pgfpathlineto{\pgfqpoint{1.739793in}{1.566556in}}%
\pgfpathlineto{\pgfqpoint{1.744702in}{1.567643in}}%
\pgfpathlineto{\pgfqpoint{1.745711in}{1.567829in}}%
\pgfpathlineto{\pgfqpoint{1.750964in}{1.568915in}}%
\pgfpathlineto{\pgfqpoint{1.752066in}{1.569195in}}%
\pgfpathlineto{\pgfqpoint{1.758117in}{1.570281in}}%
\pgfpathlineto{\pgfqpoint{1.759157in}{1.570685in}}%
\pgfpathlineto{\pgfqpoint{1.764723in}{1.571771in}}%
\pgfpathlineto{\pgfqpoint{1.765732in}{1.571957in}}%
\pgfpathlineto{\pgfqpoint{1.765825in}{1.571957in}}%
\pgfpathlineto{\pgfqpoint{1.772736in}{1.573044in}}%
\pgfpathlineto{\pgfqpoint{1.773846in}{1.573199in}}%
\pgfpathlineto{\pgfqpoint{1.780992in}{1.574286in}}%
\pgfpathlineto{\pgfqpoint{1.782102in}{1.574534in}}%
\pgfpathlineto{\pgfqpoint{1.789208in}{1.575620in}}%
\pgfpathlineto{\pgfqpoint{1.789990in}{1.575807in}}%
\pgfpathlineto{\pgfqpoint{1.790318in}{1.575807in}}%
\pgfpathlineto{\pgfqpoint{1.796009in}{1.576862in}}%
\pgfpathlineto{\pgfqpoint{1.796744in}{1.577079in}}%
\pgfpathlineto{\pgfqpoint{1.804476in}{1.578166in}}%
\pgfpathlineto{\pgfqpoint{1.805547in}{1.578290in}}%
\pgfpathlineto{\pgfqpoint{1.811731in}{1.579377in}}%
\pgfpathlineto{\pgfqpoint{1.812168in}{1.579501in}}%
\pgfpathlineto{\pgfqpoint{1.812770in}{1.579501in}}%
\pgfpathlineto{\pgfqpoint{1.819533in}{1.580587in}}%
\pgfpathlineto{\pgfqpoint{1.820181in}{1.580804in}}%
\pgfpathlineto{\pgfqpoint{1.820385in}{1.580804in}}%
\pgfpathlineto{\pgfqpoint{1.830610in}{1.581891in}}%
\pgfpathlineto{\pgfqpoint{1.831666in}{1.582046in}}%
\pgfpathlineto{\pgfqpoint{1.838959in}{1.583133in}}%
\pgfpathlineto{\pgfqpoint{1.839835in}{1.583381in}}%
\pgfpathlineto{\pgfqpoint{1.847348in}{1.584467in}}%
\pgfpathlineto{\pgfqpoint{1.848247in}{1.584623in}}%
\pgfpathlineto{\pgfqpoint{1.855025in}{1.585709in}}%
\pgfpathlineto{\pgfqpoint{1.855861in}{1.585957in}}%
\pgfpathlineto{\pgfqpoint{1.862311in}{1.587013in}}%
\pgfpathlineto{\pgfqpoint{1.863366in}{1.587354in}}%
\pgfpathlineto{\pgfqpoint{1.863405in}{1.587354in}}%
\pgfpathlineto{\pgfqpoint{1.873685in}{1.588441in}}%
\pgfpathlineto{\pgfqpoint{1.874577in}{1.588658in}}%
\pgfpathlineto{\pgfqpoint{1.880682in}{1.589714in}}%
\pgfpathlineto{\pgfqpoint{1.881151in}{1.589869in}}%
\pgfpathlineto{\pgfqpoint{1.881730in}{1.589869in}}%
\pgfpathlineto{\pgfqpoint{1.893081in}{1.590955in}}%
\pgfpathlineto{\pgfqpoint{1.893464in}{1.591048in}}%
\pgfpathlineto{\pgfqpoint{1.893597in}{1.591048in}}%
\pgfpathlineto{\pgfqpoint{1.903205in}{1.592135in}}%
\pgfpathlineto{\pgfqpoint{1.904245in}{1.592321in}}%
\pgfpathlineto{\pgfqpoint{1.912414in}{1.593408in}}%
\pgfpathlineto{\pgfqpoint{1.913368in}{1.593501in}}%
\pgfpathlineto{\pgfqpoint{1.922264in}{1.594587in}}%
\pgfpathlineto{\pgfqpoint{1.922827in}{1.594742in}}%
\pgfpathlineto{\pgfqpoint{1.923108in}{1.594742in}}%
\pgfpathlineto{\pgfqpoint{1.935820in}{1.595829in}}%
\pgfpathlineto{\pgfqpoint{1.936641in}{1.595953in}}%
\pgfpathlineto{\pgfqpoint{1.936836in}{1.595953in}}%
\pgfpathlineto{\pgfqpoint{1.950845in}{1.597040in}}%
\pgfpathlineto{\pgfqpoint{1.951346in}{1.597226in}}%
\pgfpathlineto{\pgfqpoint{1.951666in}{1.597226in}}%
\pgfpathlineto{\pgfqpoint{1.968521in}{1.598312in}}%
\pgfpathlineto{\pgfqpoint{1.968990in}{1.598374in}}%
\pgfpathlineto{\pgfqpoint{1.969272in}{1.598374in}}%
\pgfpathlineto{\pgfqpoint{1.982585in}{1.599461in}}%
\pgfpathlineto{\pgfqpoint{1.982585in}{1.599492in}}%
\pgfpathlineto{\pgfqpoint{1.983203in}{1.599492in}}%
\pgfpathlineto{\pgfqpoint{2.001293in}{1.600578in}}%
\pgfpathlineto{\pgfqpoint{2.002168in}{1.600671in}}%
\pgfpathlineto{\pgfqpoint{2.002278in}{1.600671in}}%
\pgfpathlineto{\pgfqpoint{2.028592in}{1.601758in}}%
\pgfpathlineto{\pgfqpoint{2.028592in}{1.601789in}}%
\pgfpathlineto{\pgfqpoint{2.029514in}{1.601789in}}%
\pgfpathlineto{\pgfqpoint{2.033126in}{1.601944in}}%
\pgfpathlineto{\pgfqpoint{2.033126in}{1.601944in}}%
\pgfusepath{stroke}%
\end{pgfscope}%
\begin{pgfscope}%
\pgfsetrectcap%
\pgfsetmiterjoin%
\pgfsetlinewidth{0.803000pt}%
\definecolor{currentstroke}{rgb}{0.000000,0.000000,0.000000}%
\pgfsetstrokecolor{currentstroke}%
\pgfsetdash{}{0pt}%
\pgfpathmoveto{\pgfqpoint{0.553581in}{0.499444in}}%
\pgfpathlineto{\pgfqpoint{0.553581in}{1.654444in}}%
\pgfusepath{stroke}%
\end{pgfscope}%
\begin{pgfscope}%
\pgfsetrectcap%
\pgfsetmiterjoin%
\pgfsetlinewidth{0.803000pt}%
\definecolor{currentstroke}{rgb}{0.000000,0.000000,0.000000}%
\pgfsetstrokecolor{currentstroke}%
\pgfsetdash{}{0pt}%
\pgfpathmoveto{\pgfqpoint{2.103581in}{0.499444in}}%
\pgfpathlineto{\pgfqpoint{2.103581in}{1.654444in}}%
\pgfusepath{stroke}%
\end{pgfscope}%
\begin{pgfscope}%
\pgfsetrectcap%
\pgfsetmiterjoin%
\pgfsetlinewidth{0.803000pt}%
\definecolor{currentstroke}{rgb}{0.000000,0.000000,0.000000}%
\pgfsetstrokecolor{currentstroke}%
\pgfsetdash{}{0pt}%
\pgfpathmoveto{\pgfqpoint{0.553581in}{0.499444in}}%
\pgfpathlineto{\pgfqpoint{2.103581in}{0.499444in}}%
\pgfusepath{stroke}%
\end{pgfscope}%
\begin{pgfscope}%
\pgfsetrectcap%
\pgfsetmiterjoin%
\pgfsetlinewidth{0.803000pt}%
\definecolor{currentstroke}{rgb}{0.000000,0.000000,0.000000}%
\pgfsetstrokecolor{currentstroke}%
\pgfsetdash{}{0pt}%
\pgfpathmoveto{\pgfqpoint{0.553581in}{1.654444in}}%
\pgfpathlineto{\pgfqpoint{2.103581in}{1.654444in}}%
\pgfusepath{stroke}%
\end{pgfscope}%
\begin{pgfscope}%
\pgfsetbuttcap%
\pgfsetmiterjoin%
\definecolor{currentfill}{rgb}{1.000000,1.000000,1.000000}%
\pgfsetfillcolor{currentfill}%
\pgfsetfillopacity{0.800000}%
\pgfsetlinewidth{1.003750pt}%
\definecolor{currentstroke}{rgb}{0.800000,0.800000,0.800000}%
\pgfsetstrokecolor{currentstroke}%
\pgfsetstrokeopacity{0.800000}%
\pgfsetdash{}{0pt}%
\pgfpathmoveto{\pgfqpoint{0.832747in}{0.568889in}}%
\pgfpathlineto{\pgfqpoint{2.006358in}{0.568889in}}%
\pgfpathquadraticcurveto{\pgfqpoint{2.034136in}{0.568889in}}{\pgfqpoint{2.034136in}{0.596666in}}%
\pgfpathlineto{\pgfqpoint{2.034136in}{0.776388in}}%
\pgfpathquadraticcurveto{\pgfqpoint{2.034136in}{0.804166in}}{\pgfqpoint{2.006358in}{0.804166in}}%
\pgfpathlineto{\pgfqpoint{0.832747in}{0.804166in}}%
\pgfpathquadraticcurveto{\pgfqpoint{0.804970in}{0.804166in}}{\pgfqpoint{0.804970in}{0.776388in}}%
\pgfpathlineto{\pgfqpoint{0.804970in}{0.596666in}}%
\pgfpathquadraticcurveto{\pgfqpoint{0.804970in}{0.568889in}}{\pgfqpoint{0.832747in}{0.568889in}}%
\pgfpathlineto{\pgfqpoint{0.832747in}{0.568889in}}%
\pgfpathclose%
\pgfusepath{stroke,fill}%
\end{pgfscope}%
\begin{pgfscope}%
\pgfsetrectcap%
\pgfsetroundjoin%
\pgfsetlinewidth{1.505625pt}%
\definecolor{currentstroke}{rgb}{0.000000,0.000000,0.000000}%
\pgfsetstrokecolor{currentstroke}%
\pgfsetdash{}{0pt}%
\pgfpathmoveto{\pgfqpoint{0.860525in}{0.700000in}}%
\pgfpathlineto{\pgfqpoint{0.999414in}{0.700000in}}%
\pgfpathlineto{\pgfqpoint{1.138303in}{0.700000in}}%
\pgfusepath{stroke}%
\end{pgfscope}%
\begin{pgfscope}%
\definecolor{textcolor}{rgb}{0.000000,0.000000,0.000000}%
\pgfsetstrokecolor{textcolor}%
\pgfsetfillcolor{textcolor}%
\pgftext[x=1.249414in,y=0.651388in,left,base]{\color{textcolor}\rmfamily\fontsize{10.000000}{12.000000}\selectfont AUC=0.752}%
\end{pgfscope}%
\end{pgfpicture}%
\makeatother%
\endgroup%

\end{tabular}

The distribution has long tails, so we can make a more useful visualization by truncating the ends.  For this graph we mapped the 0.01 quantile to 0 and the 0.99 quantile to 1 leaving the center 98\% of the distribution and truncated the ends.  Our goal in clipping the tails is to make all of the models' results have approximately the same granularity when we choose the decision thresholds that give us the (politically) desired results.  

%
\noindent\begin{tabular}{@{\hspace{-6pt}}p{4.3in} @{\hspace{-6pt}}p{2.0in}}
	\vskip 0pt
	\hfil Raw Model Output
	
	%% Creator: Matplotlib, PGF backend
%%
%% To include the figure in your LaTeX document, write
%%   \input{<filename>.pgf}
%%
%% Make sure the required packages are loaded in your preamble
%%   \usepackage{pgf}
%%
%% Also ensure that all the required font packages are loaded; for instance,
%% the lmodern package is sometimes necessary when using math font.
%%   \usepackage{lmodern}
%%
%% Figures using additional raster images can only be included by \input if
%% they are in the same directory as the main LaTeX file. For loading figures
%% from other directories you can use the `import` package
%%   \usepackage{import}
%%
%% and then include the figures with
%%   \import{<path to file>}{<filename>.pgf}
%%
%% Matplotlib used the following preamble
%%   
%%   \usepackage{fontspec}
%%   \makeatletter\@ifpackageloaded{underscore}{}{\usepackage[strings]{underscore}}\makeatother
%%
\begingroup%
\makeatletter%
\begin{pgfpicture}%
\pgfpathrectangle{\pgfpointorigin}{\pgfqpoint{4.102500in}{1.767624in}}%
\pgfusepath{use as bounding box, clip}%
\begin{pgfscope}%
\pgfsetbuttcap%
\pgfsetmiterjoin%
\definecolor{currentfill}{rgb}{1.000000,1.000000,1.000000}%
\pgfsetfillcolor{currentfill}%
\pgfsetlinewidth{0.000000pt}%
\definecolor{currentstroke}{rgb}{1.000000,1.000000,1.000000}%
\pgfsetstrokecolor{currentstroke}%
\pgfsetdash{}{0pt}%
\pgfpathmoveto{\pgfqpoint{0.000000in}{0.000000in}}%
\pgfpathlineto{\pgfqpoint{4.102500in}{0.000000in}}%
\pgfpathlineto{\pgfqpoint{4.102500in}{1.767624in}}%
\pgfpathlineto{\pgfqpoint{0.000000in}{1.767624in}}%
\pgfpathlineto{\pgfqpoint{0.000000in}{0.000000in}}%
\pgfpathclose%
\pgfusepath{fill}%
\end{pgfscope}%
\begin{pgfscope}%
\pgfsetbuttcap%
\pgfsetmiterjoin%
\definecolor{currentfill}{rgb}{1.000000,1.000000,1.000000}%
\pgfsetfillcolor{currentfill}%
\pgfsetlinewidth{0.000000pt}%
\definecolor{currentstroke}{rgb}{0.000000,0.000000,0.000000}%
\pgfsetstrokecolor{currentstroke}%
\pgfsetstrokeopacity{0.000000}%
\pgfsetdash{}{0pt}%
\pgfpathmoveto{\pgfqpoint{0.515000in}{0.499444in}}%
\pgfpathlineto{\pgfqpoint{4.002500in}{0.499444in}}%
\pgfpathlineto{\pgfqpoint{4.002500in}{1.654444in}}%
\pgfpathlineto{\pgfqpoint{0.515000in}{1.654444in}}%
\pgfpathlineto{\pgfqpoint{0.515000in}{0.499444in}}%
\pgfpathclose%
\pgfusepath{fill}%
\end{pgfscope}%
\begin{pgfscope}%
\pgfpathrectangle{\pgfqpoint{0.515000in}{0.499444in}}{\pgfqpoint{3.487500in}{1.155000in}}%
\pgfusepath{clip}%
\pgfsetbuttcap%
\pgfsetmiterjoin%
\pgfsetlinewidth{1.003750pt}%
\definecolor{currentstroke}{rgb}{0.000000,0.000000,0.000000}%
\pgfsetstrokecolor{currentstroke}%
\pgfsetdash{}{0pt}%
\pgfpathmoveto{\pgfqpoint{0.610114in}{0.499444in}}%
\pgfpathlineto{\pgfqpoint{0.673523in}{0.499444in}}%
\pgfpathlineto{\pgfqpoint{0.673523in}{0.593775in}}%
\pgfpathlineto{\pgfqpoint{0.610114in}{0.593775in}}%
\pgfpathlineto{\pgfqpoint{0.610114in}{0.499444in}}%
\pgfpathclose%
\pgfusepath{stroke}%
\end{pgfscope}%
\begin{pgfscope}%
\pgfpathrectangle{\pgfqpoint{0.515000in}{0.499444in}}{\pgfqpoint{3.487500in}{1.155000in}}%
\pgfusepath{clip}%
\pgfsetbuttcap%
\pgfsetmiterjoin%
\pgfsetlinewidth{1.003750pt}%
\definecolor{currentstroke}{rgb}{0.000000,0.000000,0.000000}%
\pgfsetstrokecolor{currentstroke}%
\pgfsetdash{}{0pt}%
\pgfpathmoveto{\pgfqpoint{0.768637in}{0.499444in}}%
\pgfpathlineto{\pgfqpoint{0.832046in}{0.499444in}}%
\pgfpathlineto{\pgfqpoint{0.832046in}{0.634950in}}%
\pgfpathlineto{\pgfqpoint{0.768637in}{0.634950in}}%
\pgfpathlineto{\pgfqpoint{0.768637in}{0.499444in}}%
\pgfpathclose%
\pgfusepath{stroke}%
\end{pgfscope}%
\begin{pgfscope}%
\pgfpathrectangle{\pgfqpoint{0.515000in}{0.499444in}}{\pgfqpoint{3.487500in}{1.155000in}}%
\pgfusepath{clip}%
\pgfsetbuttcap%
\pgfsetmiterjoin%
\pgfsetlinewidth{1.003750pt}%
\definecolor{currentstroke}{rgb}{0.000000,0.000000,0.000000}%
\pgfsetstrokecolor{currentstroke}%
\pgfsetdash{}{0pt}%
\pgfpathmoveto{\pgfqpoint{0.927159in}{0.499444in}}%
\pgfpathlineto{\pgfqpoint{0.990568in}{0.499444in}}%
\pgfpathlineto{\pgfqpoint{0.990568in}{0.748429in}}%
\pgfpathlineto{\pgfqpoint{0.927159in}{0.748429in}}%
\pgfpathlineto{\pgfqpoint{0.927159in}{0.499444in}}%
\pgfpathclose%
\pgfusepath{stroke}%
\end{pgfscope}%
\begin{pgfscope}%
\pgfpathrectangle{\pgfqpoint{0.515000in}{0.499444in}}{\pgfqpoint{3.487500in}{1.155000in}}%
\pgfusepath{clip}%
\pgfsetbuttcap%
\pgfsetmiterjoin%
\pgfsetlinewidth{1.003750pt}%
\definecolor{currentstroke}{rgb}{0.000000,0.000000,0.000000}%
\pgfsetstrokecolor{currentstroke}%
\pgfsetdash{}{0pt}%
\pgfpathmoveto{\pgfqpoint{1.085682in}{0.499444in}}%
\pgfpathlineto{\pgfqpoint{1.149091in}{0.499444in}}%
\pgfpathlineto{\pgfqpoint{1.149091in}{0.912762in}}%
\pgfpathlineto{\pgfqpoint{1.085682in}{0.912762in}}%
\pgfpathlineto{\pgfqpoint{1.085682in}{0.499444in}}%
\pgfpathclose%
\pgfusepath{stroke}%
\end{pgfscope}%
\begin{pgfscope}%
\pgfpathrectangle{\pgfqpoint{0.515000in}{0.499444in}}{\pgfqpoint{3.487500in}{1.155000in}}%
\pgfusepath{clip}%
\pgfsetbuttcap%
\pgfsetmiterjoin%
\pgfsetlinewidth{1.003750pt}%
\definecolor{currentstroke}{rgb}{0.000000,0.000000,0.000000}%
\pgfsetstrokecolor{currentstroke}%
\pgfsetdash{}{0pt}%
\pgfpathmoveto{\pgfqpoint{1.244205in}{0.499444in}}%
\pgfpathlineto{\pgfqpoint{1.307614in}{0.499444in}}%
\pgfpathlineto{\pgfqpoint{1.307614in}{1.095668in}}%
\pgfpathlineto{\pgfqpoint{1.244205in}{1.095668in}}%
\pgfpathlineto{\pgfqpoint{1.244205in}{0.499444in}}%
\pgfpathclose%
\pgfusepath{stroke}%
\end{pgfscope}%
\begin{pgfscope}%
\pgfpathrectangle{\pgfqpoint{0.515000in}{0.499444in}}{\pgfqpoint{3.487500in}{1.155000in}}%
\pgfusepath{clip}%
\pgfsetbuttcap%
\pgfsetmiterjoin%
\pgfsetlinewidth{1.003750pt}%
\definecolor{currentstroke}{rgb}{0.000000,0.000000,0.000000}%
\pgfsetstrokecolor{currentstroke}%
\pgfsetdash{}{0pt}%
\pgfpathmoveto{\pgfqpoint{1.402728in}{0.499444in}}%
\pgfpathlineto{\pgfqpoint{1.466137in}{0.499444in}}%
\pgfpathlineto{\pgfqpoint{1.466137in}{1.290869in}}%
\pgfpathlineto{\pgfqpoint{1.402728in}{1.290869in}}%
\pgfpathlineto{\pgfqpoint{1.402728in}{0.499444in}}%
\pgfpathclose%
\pgfusepath{stroke}%
\end{pgfscope}%
\begin{pgfscope}%
\pgfpathrectangle{\pgfqpoint{0.515000in}{0.499444in}}{\pgfqpoint{3.487500in}{1.155000in}}%
\pgfusepath{clip}%
\pgfsetbuttcap%
\pgfsetmiterjoin%
\pgfsetlinewidth{1.003750pt}%
\definecolor{currentstroke}{rgb}{0.000000,0.000000,0.000000}%
\pgfsetstrokecolor{currentstroke}%
\pgfsetdash{}{0pt}%
\pgfpathmoveto{\pgfqpoint{1.561250in}{0.499444in}}%
\pgfpathlineto{\pgfqpoint{1.624659in}{0.499444in}}%
\pgfpathlineto{\pgfqpoint{1.624659in}{1.459492in}}%
\pgfpathlineto{\pgfqpoint{1.561250in}{1.459492in}}%
\pgfpathlineto{\pgfqpoint{1.561250in}{0.499444in}}%
\pgfpathclose%
\pgfusepath{stroke}%
\end{pgfscope}%
\begin{pgfscope}%
\pgfpathrectangle{\pgfqpoint{0.515000in}{0.499444in}}{\pgfqpoint{3.487500in}{1.155000in}}%
\pgfusepath{clip}%
\pgfsetbuttcap%
\pgfsetmiterjoin%
\pgfsetlinewidth{1.003750pt}%
\definecolor{currentstroke}{rgb}{0.000000,0.000000,0.000000}%
\pgfsetstrokecolor{currentstroke}%
\pgfsetdash{}{0pt}%
\pgfpathmoveto{\pgfqpoint{1.719773in}{0.499444in}}%
\pgfpathlineto{\pgfqpoint{1.783182in}{0.499444in}}%
\pgfpathlineto{\pgfqpoint{1.783182in}{1.585789in}}%
\pgfpathlineto{\pgfqpoint{1.719773in}{1.585789in}}%
\pgfpathlineto{\pgfqpoint{1.719773in}{0.499444in}}%
\pgfpathclose%
\pgfusepath{stroke}%
\end{pgfscope}%
\begin{pgfscope}%
\pgfpathrectangle{\pgfqpoint{0.515000in}{0.499444in}}{\pgfqpoint{3.487500in}{1.155000in}}%
\pgfusepath{clip}%
\pgfsetbuttcap%
\pgfsetmiterjoin%
\pgfsetlinewidth{1.003750pt}%
\definecolor{currentstroke}{rgb}{0.000000,0.000000,0.000000}%
\pgfsetstrokecolor{currentstroke}%
\pgfsetdash{}{0pt}%
\pgfpathmoveto{\pgfqpoint{1.878296in}{0.499444in}}%
\pgfpathlineto{\pgfqpoint{1.941705in}{0.499444in}}%
\pgfpathlineto{\pgfqpoint{1.941705in}{1.599444in}}%
\pgfpathlineto{\pgfqpoint{1.878296in}{1.599444in}}%
\pgfpathlineto{\pgfqpoint{1.878296in}{0.499444in}}%
\pgfpathclose%
\pgfusepath{stroke}%
\end{pgfscope}%
\begin{pgfscope}%
\pgfpathrectangle{\pgfqpoint{0.515000in}{0.499444in}}{\pgfqpoint{3.487500in}{1.155000in}}%
\pgfusepath{clip}%
\pgfsetbuttcap%
\pgfsetmiterjoin%
\pgfsetlinewidth{1.003750pt}%
\definecolor{currentstroke}{rgb}{0.000000,0.000000,0.000000}%
\pgfsetstrokecolor{currentstroke}%
\pgfsetdash{}{0pt}%
\pgfpathmoveto{\pgfqpoint{2.036818in}{0.499444in}}%
\pgfpathlineto{\pgfqpoint{2.100228in}{0.499444in}}%
\pgfpathlineto{\pgfqpoint{2.100228in}{1.510712in}}%
\pgfpathlineto{\pgfqpoint{2.036818in}{1.510712in}}%
\pgfpathlineto{\pgfqpoint{2.036818in}{0.499444in}}%
\pgfpathclose%
\pgfusepath{stroke}%
\end{pgfscope}%
\begin{pgfscope}%
\pgfpathrectangle{\pgfqpoint{0.515000in}{0.499444in}}{\pgfqpoint{3.487500in}{1.155000in}}%
\pgfusepath{clip}%
\pgfsetbuttcap%
\pgfsetmiterjoin%
\pgfsetlinewidth{1.003750pt}%
\definecolor{currentstroke}{rgb}{0.000000,0.000000,0.000000}%
\pgfsetstrokecolor{currentstroke}%
\pgfsetdash{}{0pt}%
\pgfpathmoveto{\pgfqpoint{2.195341in}{0.499444in}}%
\pgfpathlineto{\pgfqpoint{2.258750in}{0.499444in}}%
\pgfpathlineto{\pgfqpoint{2.258750in}{1.363225in}}%
\pgfpathlineto{\pgfqpoint{2.195341in}{1.363225in}}%
\pgfpathlineto{\pgfqpoint{2.195341in}{0.499444in}}%
\pgfpathclose%
\pgfusepath{stroke}%
\end{pgfscope}%
\begin{pgfscope}%
\pgfpathrectangle{\pgfqpoint{0.515000in}{0.499444in}}{\pgfqpoint{3.487500in}{1.155000in}}%
\pgfusepath{clip}%
\pgfsetbuttcap%
\pgfsetmiterjoin%
\pgfsetlinewidth{1.003750pt}%
\definecolor{currentstroke}{rgb}{0.000000,0.000000,0.000000}%
\pgfsetstrokecolor{currentstroke}%
\pgfsetdash{}{0pt}%
\pgfpathmoveto{\pgfqpoint{2.353864in}{0.499444in}}%
\pgfpathlineto{\pgfqpoint{2.417273in}{0.499444in}}%
\pgfpathlineto{\pgfqpoint{2.417273in}{1.189632in}}%
\pgfpathlineto{\pgfqpoint{2.353864in}{1.189632in}}%
\pgfpathlineto{\pgfqpoint{2.353864in}{0.499444in}}%
\pgfpathclose%
\pgfusepath{stroke}%
\end{pgfscope}%
\begin{pgfscope}%
\pgfpathrectangle{\pgfqpoint{0.515000in}{0.499444in}}{\pgfqpoint{3.487500in}{1.155000in}}%
\pgfusepath{clip}%
\pgfsetbuttcap%
\pgfsetmiterjoin%
\pgfsetlinewidth{1.003750pt}%
\definecolor{currentstroke}{rgb}{0.000000,0.000000,0.000000}%
\pgfsetstrokecolor{currentstroke}%
\pgfsetdash{}{0pt}%
\pgfpathmoveto{\pgfqpoint{2.512387in}{0.499444in}}%
\pgfpathlineto{\pgfqpoint{2.575796in}{0.499444in}}%
\pgfpathlineto{\pgfqpoint{2.575796in}{1.004215in}}%
\pgfpathlineto{\pgfqpoint{2.512387in}{1.004215in}}%
\pgfpathlineto{\pgfqpoint{2.512387in}{0.499444in}}%
\pgfpathclose%
\pgfusepath{stroke}%
\end{pgfscope}%
\begin{pgfscope}%
\pgfpathrectangle{\pgfqpoint{0.515000in}{0.499444in}}{\pgfqpoint{3.487500in}{1.155000in}}%
\pgfusepath{clip}%
\pgfsetbuttcap%
\pgfsetmiterjoin%
\pgfsetlinewidth{1.003750pt}%
\definecolor{currentstroke}{rgb}{0.000000,0.000000,0.000000}%
\pgfsetstrokecolor{currentstroke}%
\pgfsetdash{}{0pt}%
\pgfpathmoveto{\pgfqpoint{2.670909in}{0.499444in}}%
\pgfpathlineto{\pgfqpoint{2.734318in}{0.499444in}}%
\pgfpathlineto{\pgfqpoint{2.734318in}{0.846892in}}%
\pgfpathlineto{\pgfqpoint{2.670909in}{0.846892in}}%
\pgfpathlineto{\pgfqpoint{2.670909in}{0.499444in}}%
\pgfpathclose%
\pgfusepath{stroke}%
\end{pgfscope}%
\begin{pgfscope}%
\pgfpathrectangle{\pgfqpoint{0.515000in}{0.499444in}}{\pgfqpoint{3.487500in}{1.155000in}}%
\pgfusepath{clip}%
\pgfsetbuttcap%
\pgfsetmiterjoin%
\pgfsetlinewidth{1.003750pt}%
\definecolor{currentstroke}{rgb}{0.000000,0.000000,0.000000}%
\pgfsetstrokecolor{currentstroke}%
\pgfsetdash{}{0pt}%
\pgfpathmoveto{\pgfqpoint{2.829432in}{0.499444in}}%
\pgfpathlineto{\pgfqpoint{2.892841in}{0.499444in}}%
\pgfpathlineto{\pgfqpoint{2.892841in}{0.734564in}}%
\pgfpathlineto{\pgfqpoint{2.829432in}{0.734564in}}%
\pgfpathlineto{\pgfqpoint{2.829432in}{0.499444in}}%
\pgfpathclose%
\pgfusepath{stroke}%
\end{pgfscope}%
\begin{pgfscope}%
\pgfpathrectangle{\pgfqpoint{0.515000in}{0.499444in}}{\pgfqpoint{3.487500in}{1.155000in}}%
\pgfusepath{clip}%
\pgfsetbuttcap%
\pgfsetmiterjoin%
\pgfsetlinewidth{1.003750pt}%
\definecolor{currentstroke}{rgb}{0.000000,0.000000,0.000000}%
\pgfsetstrokecolor{currentstroke}%
\pgfsetdash{}{0pt}%
\pgfpathmoveto{\pgfqpoint{2.987955in}{0.499444in}}%
\pgfpathlineto{\pgfqpoint{3.051364in}{0.499444in}}%
\pgfpathlineto{\pgfqpoint{3.051364in}{0.648448in}}%
\pgfpathlineto{\pgfqpoint{2.987955in}{0.648448in}}%
\pgfpathlineto{\pgfqpoint{2.987955in}{0.499444in}}%
\pgfpathclose%
\pgfusepath{stroke}%
\end{pgfscope}%
\begin{pgfscope}%
\pgfpathrectangle{\pgfqpoint{0.515000in}{0.499444in}}{\pgfqpoint{3.487500in}{1.155000in}}%
\pgfusepath{clip}%
\pgfsetbuttcap%
\pgfsetmiterjoin%
\pgfsetlinewidth{1.003750pt}%
\definecolor{currentstroke}{rgb}{0.000000,0.000000,0.000000}%
\pgfsetstrokecolor{currentstroke}%
\pgfsetdash{}{0pt}%
\pgfpathmoveto{\pgfqpoint{3.146478in}{0.499444in}}%
\pgfpathlineto{\pgfqpoint{3.209887in}{0.499444in}}%
\pgfpathlineto{\pgfqpoint{3.209887in}{0.586084in}}%
\pgfpathlineto{\pgfqpoint{3.146478in}{0.586084in}}%
\pgfpathlineto{\pgfqpoint{3.146478in}{0.499444in}}%
\pgfpathclose%
\pgfusepath{stroke}%
\end{pgfscope}%
\begin{pgfscope}%
\pgfpathrectangle{\pgfqpoint{0.515000in}{0.499444in}}{\pgfqpoint{3.487500in}{1.155000in}}%
\pgfusepath{clip}%
\pgfsetbuttcap%
\pgfsetmiterjoin%
\pgfsetlinewidth{1.003750pt}%
\definecolor{currentstroke}{rgb}{0.000000,0.000000,0.000000}%
\pgfsetstrokecolor{currentstroke}%
\pgfsetdash{}{0pt}%
\pgfpathmoveto{\pgfqpoint{3.305000in}{0.499444in}}%
\pgfpathlineto{\pgfqpoint{3.368409in}{0.499444in}}%
\pgfpathlineto{\pgfqpoint{3.368409in}{0.553071in}}%
\pgfpathlineto{\pgfqpoint{3.305000in}{0.553071in}}%
\pgfpathlineto{\pgfqpoint{3.305000in}{0.499444in}}%
\pgfpathclose%
\pgfusepath{stroke}%
\end{pgfscope}%
\begin{pgfscope}%
\pgfpathrectangle{\pgfqpoint{0.515000in}{0.499444in}}{\pgfqpoint{3.487500in}{1.155000in}}%
\pgfusepath{clip}%
\pgfsetbuttcap%
\pgfsetmiterjoin%
\pgfsetlinewidth{1.003750pt}%
\definecolor{currentstroke}{rgb}{0.000000,0.000000,0.000000}%
\pgfsetstrokecolor{currentstroke}%
\pgfsetdash{}{0pt}%
\pgfpathmoveto{\pgfqpoint{3.463523in}{0.499444in}}%
\pgfpathlineto{\pgfqpoint{3.526932in}{0.499444in}}%
\pgfpathlineto{\pgfqpoint{3.526932in}{0.528062in}}%
\pgfpathlineto{\pgfqpoint{3.463523in}{0.528062in}}%
\pgfpathlineto{\pgfqpoint{3.463523in}{0.499444in}}%
\pgfpathclose%
\pgfusepath{stroke}%
\end{pgfscope}%
\begin{pgfscope}%
\pgfpathrectangle{\pgfqpoint{0.515000in}{0.499444in}}{\pgfqpoint{3.487500in}{1.155000in}}%
\pgfusepath{clip}%
\pgfsetbuttcap%
\pgfsetmiterjoin%
\pgfsetlinewidth{1.003750pt}%
\definecolor{currentstroke}{rgb}{0.000000,0.000000,0.000000}%
\pgfsetstrokecolor{currentstroke}%
\pgfsetdash{}{0pt}%
\pgfpathmoveto{\pgfqpoint{3.622046in}{0.499444in}}%
\pgfpathlineto{\pgfqpoint{3.685455in}{0.499444in}}%
\pgfpathlineto{\pgfqpoint{3.685455in}{0.516291in}}%
\pgfpathlineto{\pgfqpoint{3.622046in}{0.516291in}}%
\pgfpathlineto{\pgfqpoint{3.622046in}{0.499444in}}%
\pgfpathclose%
\pgfusepath{stroke}%
\end{pgfscope}%
\begin{pgfscope}%
\pgfpathrectangle{\pgfqpoint{0.515000in}{0.499444in}}{\pgfqpoint{3.487500in}{1.155000in}}%
\pgfusepath{clip}%
\pgfsetbuttcap%
\pgfsetmiterjoin%
\pgfsetlinewidth{1.003750pt}%
\definecolor{currentstroke}{rgb}{0.000000,0.000000,0.000000}%
\pgfsetstrokecolor{currentstroke}%
\pgfsetdash{}{0pt}%
\pgfpathmoveto{\pgfqpoint{3.780568in}{0.499444in}}%
\pgfpathlineto{\pgfqpoint{3.843978in}{0.499444in}}%
\pgfpathlineto{\pgfqpoint{3.843978in}{0.516134in}}%
\pgfpathlineto{\pgfqpoint{3.780568in}{0.516134in}}%
\pgfpathlineto{\pgfqpoint{3.780568in}{0.499444in}}%
\pgfpathclose%
\pgfusepath{stroke}%
\end{pgfscope}%
\begin{pgfscope}%
\pgfpathrectangle{\pgfqpoint{0.515000in}{0.499444in}}{\pgfqpoint{3.487500in}{1.155000in}}%
\pgfusepath{clip}%
\pgfsetbuttcap%
\pgfsetmiterjoin%
\definecolor{currentfill}{rgb}{0.000000,0.000000,0.000000}%
\pgfsetfillcolor{currentfill}%
\pgfsetlinewidth{0.000000pt}%
\definecolor{currentstroke}{rgb}{0.000000,0.000000,0.000000}%
\pgfsetstrokecolor{currentstroke}%
\pgfsetstrokeopacity{0.000000}%
\pgfsetdash{}{0pt}%
\pgfpathmoveto{\pgfqpoint{0.673523in}{0.499444in}}%
\pgfpathlineto{\pgfqpoint{0.736932in}{0.499444in}}%
\pgfpathlineto{\pgfqpoint{0.736932in}{0.500229in}}%
\pgfpathlineto{\pgfqpoint{0.673523in}{0.500229in}}%
\pgfpathlineto{\pgfqpoint{0.673523in}{0.499444in}}%
\pgfpathclose%
\pgfusepath{fill}%
\end{pgfscope}%
\begin{pgfscope}%
\pgfpathrectangle{\pgfqpoint{0.515000in}{0.499444in}}{\pgfqpoint{3.487500in}{1.155000in}}%
\pgfusepath{clip}%
\pgfsetbuttcap%
\pgfsetmiterjoin%
\definecolor{currentfill}{rgb}{0.000000,0.000000,0.000000}%
\pgfsetfillcolor{currentfill}%
\pgfsetlinewidth{0.000000pt}%
\definecolor{currentstroke}{rgb}{0.000000,0.000000,0.000000}%
\pgfsetstrokecolor{currentstroke}%
\pgfsetstrokeopacity{0.000000}%
\pgfsetdash{}{0pt}%
\pgfpathmoveto{\pgfqpoint{0.832046in}{0.499444in}}%
\pgfpathlineto{\pgfqpoint{0.895455in}{0.499444in}}%
\pgfpathlineto{\pgfqpoint{0.895455in}{0.501118in}}%
\pgfpathlineto{\pgfqpoint{0.832046in}{0.501118in}}%
\pgfpathlineto{\pgfqpoint{0.832046in}{0.499444in}}%
\pgfpathclose%
\pgfusepath{fill}%
\end{pgfscope}%
\begin{pgfscope}%
\pgfpathrectangle{\pgfqpoint{0.515000in}{0.499444in}}{\pgfqpoint{3.487500in}{1.155000in}}%
\pgfusepath{clip}%
\pgfsetbuttcap%
\pgfsetmiterjoin%
\definecolor{currentfill}{rgb}{0.000000,0.000000,0.000000}%
\pgfsetfillcolor{currentfill}%
\pgfsetlinewidth{0.000000pt}%
\definecolor{currentstroke}{rgb}{0.000000,0.000000,0.000000}%
\pgfsetstrokecolor{currentstroke}%
\pgfsetstrokeopacity{0.000000}%
\pgfsetdash{}{0pt}%
\pgfpathmoveto{\pgfqpoint{0.990568in}{0.499444in}}%
\pgfpathlineto{\pgfqpoint{1.053978in}{0.499444in}}%
\pgfpathlineto{\pgfqpoint{1.053978in}{0.503996in}}%
\pgfpathlineto{\pgfqpoint{0.990568in}{0.503996in}}%
\pgfpathlineto{\pgfqpoint{0.990568in}{0.499444in}}%
\pgfpathclose%
\pgfusepath{fill}%
\end{pgfscope}%
\begin{pgfscope}%
\pgfpathrectangle{\pgfqpoint{0.515000in}{0.499444in}}{\pgfqpoint{3.487500in}{1.155000in}}%
\pgfusepath{clip}%
\pgfsetbuttcap%
\pgfsetmiterjoin%
\definecolor{currentfill}{rgb}{0.000000,0.000000,0.000000}%
\pgfsetfillcolor{currentfill}%
\pgfsetlinewidth{0.000000pt}%
\definecolor{currentstroke}{rgb}{0.000000,0.000000,0.000000}%
\pgfsetstrokecolor{currentstroke}%
\pgfsetstrokeopacity{0.000000}%
\pgfsetdash{}{0pt}%
\pgfpathmoveto{\pgfqpoint{1.149091in}{0.499444in}}%
\pgfpathlineto{\pgfqpoint{1.212500in}{0.499444in}}%
\pgfpathlineto{\pgfqpoint{1.212500in}{0.509960in}}%
\pgfpathlineto{\pgfqpoint{1.149091in}{0.509960in}}%
\pgfpathlineto{\pgfqpoint{1.149091in}{0.499444in}}%
\pgfpathclose%
\pgfusepath{fill}%
\end{pgfscope}%
\begin{pgfscope}%
\pgfpathrectangle{\pgfqpoint{0.515000in}{0.499444in}}{\pgfqpoint{3.487500in}{1.155000in}}%
\pgfusepath{clip}%
\pgfsetbuttcap%
\pgfsetmiterjoin%
\definecolor{currentfill}{rgb}{0.000000,0.000000,0.000000}%
\pgfsetfillcolor{currentfill}%
\pgfsetlinewidth{0.000000pt}%
\definecolor{currentstroke}{rgb}{0.000000,0.000000,0.000000}%
\pgfsetstrokecolor{currentstroke}%
\pgfsetstrokeopacity{0.000000}%
\pgfsetdash{}{0pt}%
\pgfpathmoveto{\pgfqpoint{1.307614in}{0.499444in}}%
\pgfpathlineto{\pgfqpoint{1.371023in}{0.499444in}}%
\pgfpathlineto{\pgfqpoint{1.371023in}{0.520372in}}%
\pgfpathlineto{\pgfqpoint{1.307614in}{0.520372in}}%
\pgfpathlineto{\pgfqpoint{1.307614in}{0.499444in}}%
\pgfpathclose%
\pgfusepath{fill}%
\end{pgfscope}%
\begin{pgfscope}%
\pgfpathrectangle{\pgfqpoint{0.515000in}{0.499444in}}{\pgfqpoint{3.487500in}{1.155000in}}%
\pgfusepath{clip}%
\pgfsetbuttcap%
\pgfsetmiterjoin%
\definecolor{currentfill}{rgb}{0.000000,0.000000,0.000000}%
\pgfsetfillcolor{currentfill}%
\pgfsetlinewidth{0.000000pt}%
\definecolor{currentstroke}{rgb}{0.000000,0.000000,0.000000}%
\pgfsetstrokecolor{currentstroke}%
\pgfsetstrokeopacity{0.000000}%
\pgfsetdash{}{0pt}%
\pgfpathmoveto{\pgfqpoint{1.466137in}{0.499444in}}%
\pgfpathlineto{\pgfqpoint{1.529546in}{0.499444in}}%
\pgfpathlineto{\pgfqpoint{1.529546in}{0.537951in}}%
\pgfpathlineto{\pgfqpoint{1.466137in}{0.537951in}}%
\pgfpathlineto{\pgfqpoint{1.466137in}{0.499444in}}%
\pgfpathclose%
\pgfusepath{fill}%
\end{pgfscope}%
\begin{pgfscope}%
\pgfpathrectangle{\pgfqpoint{0.515000in}{0.499444in}}{\pgfqpoint{3.487500in}{1.155000in}}%
\pgfusepath{clip}%
\pgfsetbuttcap%
\pgfsetmiterjoin%
\definecolor{currentfill}{rgb}{0.000000,0.000000,0.000000}%
\pgfsetfillcolor{currentfill}%
\pgfsetlinewidth{0.000000pt}%
\definecolor{currentstroke}{rgb}{0.000000,0.000000,0.000000}%
\pgfsetstrokecolor{currentstroke}%
\pgfsetstrokeopacity{0.000000}%
\pgfsetdash{}{0pt}%
\pgfpathmoveto{\pgfqpoint{1.624659in}{0.499444in}}%
\pgfpathlineto{\pgfqpoint{1.688068in}{0.499444in}}%
\pgfpathlineto{\pgfqpoint{1.688068in}{0.566307in}}%
\pgfpathlineto{\pgfqpoint{1.624659in}{0.566307in}}%
\pgfpathlineto{\pgfqpoint{1.624659in}{0.499444in}}%
\pgfpathclose%
\pgfusepath{fill}%
\end{pgfscope}%
\begin{pgfscope}%
\pgfpathrectangle{\pgfqpoint{0.515000in}{0.499444in}}{\pgfqpoint{3.487500in}{1.155000in}}%
\pgfusepath{clip}%
\pgfsetbuttcap%
\pgfsetmiterjoin%
\definecolor{currentfill}{rgb}{0.000000,0.000000,0.000000}%
\pgfsetfillcolor{currentfill}%
\pgfsetlinewidth{0.000000pt}%
\definecolor{currentstroke}{rgb}{0.000000,0.000000,0.000000}%
\pgfsetstrokecolor{currentstroke}%
\pgfsetstrokeopacity{0.000000}%
\pgfsetdash{}{0pt}%
\pgfpathmoveto{\pgfqpoint{1.783182in}{0.499444in}}%
\pgfpathlineto{\pgfqpoint{1.846591in}{0.499444in}}%
\pgfpathlineto{\pgfqpoint{1.846591in}{0.601779in}}%
\pgfpathlineto{\pgfqpoint{1.783182in}{0.601779in}}%
\pgfpathlineto{\pgfqpoint{1.783182in}{0.499444in}}%
\pgfpathclose%
\pgfusepath{fill}%
\end{pgfscope}%
\begin{pgfscope}%
\pgfpathrectangle{\pgfqpoint{0.515000in}{0.499444in}}{\pgfqpoint{3.487500in}{1.155000in}}%
\pgfusepath{clip}%
\pgfsetbuttcap%
\pgfsetmiterjoin%
\definecolor{currentfill}{rgb}{0.000000,0.000000,0.000000}%
\pgfsetfillcolor{currentfill}%
\pgfsetlinewidth{0.000000pt}%
\definecolor{currentstroke}{rgb}{0.000000,0.000000,0.000000}%
\pgfsetstrokecolor{currentstroke}%
\pgfsetstrokeopacity{0.000000}%
\pgfsetdash{}{0pt}%
\pgfpathmoveto{\pgfqpoint{1.941705in}{0.499444in}}%
\pgfpathlineto{\pgfqpoint{2.005114in}{0.499444in}}%
\pgfpathlineto{\pgfqpoint{2.005114in}{0.638089in}}%
\pgfpathlineto{\pgfqpoint{1.941705in}{0.638089in}}%
\pgfpathlineto{\pgfqpoint{1.941705in}{0.499444in}}%
\pgfpathclose%
\pgfusepath{fill}%
\end{pgfscope}%
\begin{pgfscope}%
\pgfpathrectangle{\pgfqpoint{0.515000in}{0.499444in}}{\pgfqpoint{3.487500in}{1.155000in}}%
\pgfusepath{clip}%
\pgfsetbuttcap%
\pgfsetmiterjoin%
\definecolor{currentfill}{rgb}{0.000000,0.000000,0.000000}%
\pgfsetfillcolor{currentfill}%
\pgfsetlinewidth{0.000000pt}%
\definecolor{currentstroke}{rgb}{0.000000,0.000000,0.000000}%
\pgfsetstrokecolor{currentstroke}%
\pgfsetstrokeopacity{0.000000}%
\pgfsetdash{}{0pt}%
\pgfpathmoveto{\pgfqpoint{2.100228in}{0.499444in}}%
\pgfpathlineto{\pgfqpoint{2.163637in}{0.499444in}}%
\pgfpathlineto{\pgfqpoint{2.163637in}{0.669480in}}%
\pgfpathlineto{\pgfqpoint{2.100228in}{0.669480in}}%
\pgfpathlineto{\pgfqpoint{2.100228in}{0.499444in}}%
\pgfpathclose%
\pgfusepath{fill}%
\end{pgfscope}%
\begin{pgfscope}%
\pgfpathrectangle{\pgfqpoint{0.515000in}{0.499444in}}{\pgfqpoint{3.487500in}{1.155000in}}%
\pgfusepath{clip}%
\pgfsetbuttcap%
\pgfsetmiterjoin%
\definecolor{currentfill}{rgb}{0.000000,0.000000,0.000000}%
\pgfsetfillcolor{currentfill}%
\pgfsetlinewidth{0.000000pt}%
\definecolor{currentstroke}{rgb}{0.000000,0.000000,0.000000}%
\pgfsetstrokecolor{currentstroke}%
\pgfsetstrokeopacity{0.000000}%
\pgfsetdash{}{0pt}%
\pgfpathmoveto{\pgfqpoint{2.258750in}{0.499444in}}%
\pgfpathlineto{\pgfqpoint{2.322159in}{0.499444in}}%
\pgfpathlineto{\pgfqpoint{2.322159in}{0.688838in}}%
\pgfpathlineto{\pgfqpoint{2.258750in}{0.688838in}}%
\pgfpathlineto{\pgfqpoint{2.258750in}{0.499444in}}%
\pgfpathclose%
\pgfusepath{fill}%
\end{pgfscope}%
\begin{pgfscope}%
\pgfpathrectangle{\pgfqpoint{0.515000in}{0.499444in}}{\pgfqpoint{3.487500in}{1.155000in}}%
\pgfusepath{clip}%
\pgfsetbuttcap%
\pgfsetmiterjoin%
\definecolor{currentfill}{rgb}{0.000000,0.000000,0.000000}%
\pgfsetfillcolor{currentfill}%
\pgfsetlinewidth{0.000000pt}%
\definecolor{currentstroke}{rgb}{0.000000,0.000000,0.000000}%
\pgfsetstrokecolor{currentstroke}%
\pgfsetstrokeopacity{0.000000}%
\pgfsetdash{}{0pt}%
\pgfpathmoveto{\pgfqpoint{2.417273in}{0.499444in}}%
\pgfpathlineto{\pgfqpoint{2.480682in}{0.499444in}}%
\pgfpathlineto{\pgfqpoint{2.480682in}{0.689884in}}%
\pgfpathlineto{\pgfqpoint{2.417273in}{0.689884in}}%
\pgfpathlineto{\pgfqpoint{2.417273in}{0.499444in}}%
\pgfpathclose%
\pgfusepath{fill}%
\end{pgfscope}%
\begin{pgfscope}%
\pgfpathrectangle{\pgfqpoint{0.515000in}{0.499444in}}{\pgfqpoint{3.487500in}{1.155000in}}%
\pgfusepath{clip}%
\pgfsetbuttcap%
\pgfsetmiterjoin%
\definecolor{currentfill}{rgb}{0.000000,0.000000,0.000000}%
\pgfsetfillcolor{currentfill}%
\pgfsetlinewidth{0.000000pt}%
\definecolor{currentstroke}{rgb}{0.000000,0.000000,0.000000}%
\pgfsetstrokecolor{currentstroke}%
\pgfsetstrokeopacity{0.000000}%
\pgfsetdash{}{0pt}%
\pgfpathmoveto{\pgfqpoint{2.575796in}{0.499444in}}%
\pgfpathlineto{\pgfqpoint{2.639205in}{0.499444in}}%
\pgfpathlineto{\pgfqpoint{2.639205in}{0.682141in}}%
\pgfpathlineto{\pgfqpoint{2.575796in}{0.682141in}}%
\pgfpathlineto{\pgfqpoint{2.575796in}{0.499444in}}%
\pgfpathclose%
\pgfusepath{fill}%
\end{pgfscope}%
\begin{pgfscope}%
\pgfpathrectangle{\pgfqpoint{0.515000in}{0.499444in}}{\pgfqpoint{3.487500in}{1.155000in}}%
\pgfusepath{clip}%
\pgfsetbuttcap%
\pgfsetmiterjoin%
\definecolor{currentfill}{rgb}{0.000000,0.000000,0.000000}%
\pgfsetfillcolor{currentfill}%
\pgfsetlinewidth{0.000000pt}%
\definecolor{currentstroke}{rgb}{0.000000,0.000000,0.000000}%
\pgfsetstrokecolor{currentstroke}%
\pgfsetstrokeopacity{0.000000}%
\pgfsetdash{}{0pt}%
\pgfpathmoveto{\pgfqpoint{2.734318in}{0.499444in}}%
\pgfpathlineto{\pgfqpoint{2.797728in}{0.499444in}}%
\pgfpathlineto{\pgfqpoint{2.797728in}{0.667806in}}%
\pgfpathlineto{\pgfqpoint{2.734318in}{0.667806in}}%
\pgfpathlineto{\pgfqpoint{2.734318in}{0.499444in}}%
\pgfpathclose%
\pgfusepath{fill}%
\end{pgfscope}%
\begin{pgfscope}%
\pgfpathrectangle{\pgfqpoint{0.515000in}{0.499444in}}{\pgfqpoint{3.487500in}{1.155000in}}%
\pgfusepath{clip}%
\pgfsetbuttcap%
\pgfsetmiterjoin%
\definecolor{currentfill}{rgb}{0.000000,0.000000,0.000000}%
\pgfsetfillcolor{currentfill}%
\pgfsetlinewidth{0.000000pt}%
\definecolor{currentstroke}{rgb}{0.000000,0.000000,0.000000}%
\pgfsetstrokecolor{currentstroke}%
\pgfsetstrokeopacity{0.000000}%
\pgfsetdash{}{0pt}%
\pgfpathmoveto{\pgfqpoint{2.892841in}{0.499444in}}%
\pgfpathlineto{\pgfqpoint{2.956250in}{0.499444in}}%
\pgfpathlineto{\pgfqpoint{2.956250in}{0.638821in}}%
\pgfpathlineto{\pgfqpoint{2.892841in}{0.638821in}}%
\pgfpathlineto{\pgfqpoint{2.892841in}{0.499444in}}%
\pgfpathclose%
\pgfusepath{fill}%
\end{pgfscope}%
\begin{pgfscope}%
\pgfpathrectangle{\pgfqpoint{0.515000in}{0.499444in}}{\pgfqpoint{3.487500in}{1.155000in}}%
\pgfusepath{clip}%
\pgfsetbuttcap%
\pgfsetmiterjoin%
\definecolor{currentfill}{rgb}{0.000000,0.000000,0.000000}%
\pgfsetfillcolor{currentfill}%
\pgfsetlinewidth{0.000000pt}%
\definecolor{currentstroke}{rgb}{0.000000,0.000000,0.000000}%
\pgfsetstrokecolor{currentstroke}%
\pgfsetstrokeopacity{0.000000}%
\pgfsetdash{}{0pt}%
\pgfpathmoveto{\pgfqpoint{3.051364in}{0.499444in}}%
\pgfpathlineto{\pgfqpoint{3.114773in}{0.499444in}}%
\pgfpathlineto{\pgfqpoint{3.114773in}{0.611511in}}%
\pgfpathlineto{\pgfqpoint{3.051364in}{0.611511in}}%
\pgfpathlineto{\pgfqpoint{3.051364in}{0.499444in}}%
\pgfpathclose%
\pgfusepath{fill}%
\end{pgfscope}%
\begin{pgfscope}%
\pgfpathrectangle{\pgfqpoint{0.515000in}{0.499444in}}{\pgfqpoint{3.487500in}{1.155000in}}%
\pgfusepath{clip}%
\pgfsetbuttcap%
\pgfsetmiterjoin%
\definecolor{currentfill}{rgb}{0.000000,0.000000,0.000000}%
\pgfsetfillcolor{currentfill}%
\pgfsetlinewidth{0.000000pt}%
\definecolor{currentstroke}{rgb}{0.000000,0.000000,0.000000}%
\pgfsetstrokecolor{currentstroke}%
\pgfsetstrokeopacity{0.000000}%
\pgfsetdash{}{0pt}%
\pgfpathmoveto{\pgfqpoint{3.209887in}{0.499444in}}%
\pgfpathlineto{\pgfqpoint{3.273296in}{0.499444in}}%
\pgfpathlineto{\pgfqpoint{3.273296in}{0.585299in}}%
\pgfpathlineto{\pgfqpoint{3.209887in}{0.585299in}}%
\pgfpathlineto{\pgfqpoint{3.209887in}{0.499444in}}%
\pgfpathclose%
\pgfusepath{fill}%
\end{pgfscope}%
\begin{pgfscope}%
\pgfpathrectangle{\pgfqpoint{0.515000in}{0.499444in}}{\pgfqpoint{3.487500in}{1.155000in}}%
\pgfusepath{clip}%
\pgfsetbuttcap%
\pgfsetmiterjoin%
\definecolor{currentfill}{rgb}{0.000000,0.000000,0.000000}%
\pgfsetfillcolor{currentfill}%
\pgfsetlinewidth{0.000000pt}%
\definecolor{currentstroke}{rgb}{0.000000,0.000000,0.000000}%
\pgfsetstrokecolor{currentstroke}%
\pgfsetstrokeopacity{0.000000}%
\pgfsetdash{}{0pt}%
\pgfpathmoveto{\pgfqpoint{3.368409in}{0.499444in}}%
\pgfpathlineto{\pgfqpoint{3.431818in}{0.499444in}}%
\pgfpathlineto{\pgfqpoint{3.431818in}{0.553856in}}%
\pgfpathlineto{\pgfqpoint{3.368409in}{0.553856in}}%
\pgfpathlineto{\pgfqpoint{3.368409in}{0.499444in}}%
\pgfpathclose%
\pgfusepath{fill}%
\end{pgfscope}%
\begin{pgfscope}%
\pgfpathrectangle{\pgfqpoint{0.515000in}{0.499444in}}{\pgfqpoint{3.487500in}{1.155000in}}%
\pgfusepath{clip}%
\pgfsetbuttcap%
\pgfsetmiterjoin%
\definecolor{currentfill}{rgb}{0.000000,0.000000,0.000000}%
\pgfsetfillcolor{currentfill}%
\pgfsetlinewidth{0.000000pt}%
\definecolor{currentstroke}{rgb}{0.000000,0.000000,0.000000}%
\pgfsetstrokecolor{currentstroke}%
\pgfsetstrokeopacity{0.000000}%
\pgfsetdash{}{0pt}%
\pgfpathmoveto{\pgfqpoint{3.526932in}{0.499444in}}%
\pgfpathlineto{\pgfqpoint{3.590341in}{0.499444in}}%
\pgfpathlineto{\pgfqpoint{3.590341in}{0.536800in}}%
\pgfpathlineto{\pgfqpoint{3.526932in}{0.536800in}}%
\pgfpathlineto{\pgfqpoint{3.526932in}{0.499444in}}%
\pgfpathclose%
\pgfusepath{fill}%
\end{pgfscope}%
\begin{pgfscope}%
\pgfpathrectangle{\pgfqpoint{0.515000in}{0.499444in}}{\pgfqpoint{3.487500in}{1.155000in}}%
\pgfusepath{clip}%
\pgfsetbuttcap%
\pgfsetmiterjoin%
\definecolor{currentfill}{rgb}{0.000000,0.000000,0.000000}%
\pgfsetfillcolor{currentfill}%
\pgfsetlinewidth{0.000000pt}%
\definecolor{currentstroke}{rgb}{0.000000,0.000000,0.000000}%
\pgfsetstrokecolor{currentstroke}%
\pgfsetstrokeopacity{0.000000}%
\pgfsetdash{}{0pt}%
\pgfpathmoveto{\pgfqpoint{3.685455in}{0.499444in}}%
\pgfpathlineto{\pgfqpoint{3.748864in}{0.499444in}}%
\pgfpathlineto{\pgfqpoint{3.748864in}{0.520424in}}%
\pgfpathlineto{\pgfqpoint{3.685455in}{0.520424in}}%
\pgfpathlineto{\pgfqpoint{3.685455in}{0.499444in}}%
\pgfpathclose%
\pgfusepath{fill}%
\end{pgfscope}%
\begin{pgfscope}%
\pgfpathrectangle{\pgfqpoint{0.515000in}{0.499444in}}{\pgfqpoint{3.487500in}{1.155000in}}%
\pgfusepath{clip}%
\pgfsetbuttcap%
\pgfsetmiterjoin%
\definecolor{currentfill}{rgb}{0.000000,0.000000,0.000000}%
\pgfsetfillcolor{currentfill}%
\pgfsetlinewidth{0.000000pt}%
\definecolor{currentstroke}{rgb}{0.000000,0.000000,0.000000}%
\pgfsetstrokecolor{currentstroke}%
\pgfsetstrokeopacity{0.000000}%
\pgfsetdash{}{0pt}%
\pgfpathmoveto{\pgfqpoint{3.843978in}{0.499444in}}%
\pgfpathlineto{\pgfqpoint{3.907387in}{0.499444in}}%
\pgfpathlineto{\pgfqpoint{3.907387in}{0.533347in}}%
\pgfpathlineto{\pgfqpoint{3.843978in}{0.533347in}}%
\pgfpathlineto{\pgfqpoint{3.843978in}{0.499444in}}%
\pgfpathclose%
\pgfusepath{fill}%
\end{pgfscope}%
\begin{pgfscope}%
\pgfsetbuttcap%
\pgfsetroundjoin%
\definecolor{currentfill}{rgb}{0.000000,0.000000,0.000000}%
\pgfsetfillcolor{currentfill}%
\pgfsetlinewidth{0.803000pt}%
\definecolor{currentstroke}{rgb}{0.000000,0.000000,0.000000}%
\pgfsetstrokecolor{currentstroke}%
\pgfsetdash{}{0pt}%
\pgfsys@defobject{currentmarker}{\pgfqpoint{0.000000in}{-0.048611in}}{\pgfqpoint{0.000000in}{0.000000in}}{%
\pgfpathmoveto{\pgfqpoint{0.000000in}{0.000000in}}%
\pgfpathlineto{\pgfqpoint{0.000000in}{-0.048611in}}%
\pgfusepath{stroke,fill}%
}%
\begin{pgfscope}%
\pgfsys@transformshift{0.515000in}{0.499444in}%
\pgfsys@useobject{currentmarker}{}%
\end{pgfscope}%
\end{pgfscope}%
\begin{pgfscope}%
\pgfsetbuttcap%
\pgfsetroundjoin%
\definecolor{currentfill}{rgb}{0.000000,0.000000,0.000000}%
\pgfsetfillcolor{currentfill}%
\pgfsetlinewidth{0.803000pt}%
\definecolor{currentstroke}{rgb}{0.000000,0.000000,0.000000}%
\pgfsetstrokecolor{currentstroke}%
\pgfsetdash{}{0pt}%
\pgfsys@defobject{currentmarker}{\pgfqpoint{0.000000in}{-0.048611in}}{\pgfqpoint{0.000000in}{0.000000in}}{%
\pgfpathmoveto{\pgfqpoint{0.000000in}{0.000000in}}%
\pgfpathlineto{\pgfqpoint{0.000000in}{-0.048611in}}%
\pgfusepath{stroke,fill}%
}%
\begin{pgfscope}%
\pgfsys@transformshift{0.673523in}{0.499444in}%
\pgfsys@useobject{currentmarker}{}%
\end{pgfscope}%
\end{pgfscope}%
\begin{pgfscope}%
\definecolor{textcolor}{rgb}{0.000000,0.000000,0.000000}%
\pgfsetstrokecolor{textcolor}%
\pgfsetfillcolor{textcolor}%
\pgftext[x=0.673523in,y=0.402222in,,top]{\color{textcolor}\rmfamily\fontsize{10.000000}{12.000000}\selectfont 0.0}%
\end{pgfscope}%
\begin{pgfscope}%
\pgfsetbuttcap%
\pgfsetroundjoin%
\definecolor{currentfill}{rgb}{0.000000,0.000000,0.000000}%
\pgfsetfillcolor{currentfill}%
\pgfsetlinewidth{0.803000pt}%
\definecolor{currentstroke}{rgb}{0.000000,0.000000,0.000000}%
\pgfsetstrokecolor{currentstroke}%
\pgfsetdash{}{0pt}%
\pgfsys@defobject{currentmarker}{\pgfqpoint{0.000000in}{-0.048611in}}{\pgfqpoint{0.000000in}{0.000000in}}{%
\pgfpathmoveto{\pgfqpoint{0.000000in}{0.000000in}}%
\pgfpathlineto{\pgfqpoint{0.000000in}{-0.048611in}}%
\pgfusepath{stroke,fill}%
}%
\begin{pgfscope}%
\pgfsys@transformshift{0.832046in}{0.499444in}%
\pgfsys@useobject{currentmarker}{}%
\end{pgfscope}%
\end{pgfscope}%
\begin{pgfscope}%
\pgfsetbuttcap%
\pgfsetroundjoin%
\definecolor{currentfill}{rgb}{0.000000,0.000000,0.000000}%
\pgfsetfillcolor{currentfill}%
\pgfsetlinewidth{0.803000pt}%
\definecolor{currentstroke}{rgb}{0.000000,0.000000,0.000000}%
\pgfsetstrokecolor{currentstroke}%
\pgfsetdash{}{0pt}%
\pgfsys@defobject{currentmarker}{\pgfqpoint{0.000000in}{-0.048611in}}{\pgfqpoint{0.000000in}{0.000000in}}{%
\pgfpathmoveto{\pgfqpoint{0.000000in}{0.000000in}}%
\pgfpathlineto{\pgfqpoint{0.000000in}{-0.048611in}}%
\pgfusepath{stroke,fill}%
}%
\begin{pgfscope}%
\pgfsys@transformshift{0.990568in}{0.499444in}%
\pgfsys@useobject{currentmarker}{}%
\end{pgfscope}%
\end{pgfscope}%
\begin{pgfscope}%
\definecolor{textcolor}{rgb}{0.000000,0.000000,0.000000}%
\pgfsetstrokecolor{textcolor}%
\pgfsetfillcolor{textcolor}%
\pgftext[x=0.990568in,y=0.402222in,,top]{\color{textcolor}\rmfamily\fontsize{10.000000}{12.000000}\selectfont 0.1}%
\end{pgfscope}%
\begin{pgfscope}%
\pgfsetbuttcap%
\pgfsetroundjoin%
\definecolor{currentfill}{rgb}{0.000000,0.000000,0.000000}%
\pgfsetfillcolor{currentfill}%
\pgfsetlinewidth{0.803000pt}%
\definecolor{currentstroke}{rgb}{0.000000,0.000000,0.000000}%
\pgfsetstrokecolor{currentstroke}%
\pgfsetdash{}{0pt}%
\pgfsys@defobject{currentmarker}{\pgfqpoint{0.000000in}{-0.048611in}}{\pgfqpoint{0.000000in}{0.000000in}}{%
\pgfpathmoveto{\pgfqpoint{0.000000in}{0.000000in}}%
\pgfpathlineto{\pgfqpoint{0.000000in}{-0.048611in}}%
\pgfusepath{stroke,fill}%
}%
\begin{pgfscope}%
\pgfsys@transformshift{1.149091in}{0.499444in}%
\pgfsys@useobject{currentmarker}{}%
\end{pgfscope}%
\end{pgfscope}%
\begin{pgfscope}%
\pgfsetbuttcap%
\pgfsetroundjoin%
\definecolor{currentfill}{rgb}{0.000000,0.000000,0.000000}%
\pgfsetfillcolor{currentfill}%
\pgfsetlinewidth{0.803000pt}%
\definecolor{currentstroke}{rgb}{0.000000,0.000000,0.000000}%
\pgfsetstrokecolor{currentstroke}%
\pgfsetdash{}{0pt}%
\pgfsys@defobject{currentmarker}{\pgfqpoint{0.000000in}{-0.048611in}}{\pgfqpoint{0.000000in}{0.000000in}}{%
\pgfpathmoveto{\pgfqpoint{0.000000in}{0.000000in}}%
\pgfpathlineto{\pgfqpoint{0.000000in}{-0.048611in}}%
\pgfusepath{stroke,fill}%
}%
\begin{pgfscope}%
\pgfsys@transformshift{1.307614in}{0.499444in}%
\pgfsys@useobject{currentmarker}{}%
\end{pgfscope}%
\end{pgfscope}%
\begin{pgfscope}%
\definecolor{textcolor}{rgb}{0.000000,0.000000,0.000000}%
\pgfsetstrokecolor{textcolor}%
\pgfsetfillcolor{textcolor}%
\pgftext[x=1.307614in,y=0.402222in,,top]{\color{textcolor}\rmfamily\fontsize{10.000000}{12.000000}\selectfont 0.2}%
\end{pgfscope}%
\begin{pgfscope}%
\pgfsetbuttcap%
\pgfsetroundjoin%
\definecolor{currentfill}{rgb}{0.000000,0.000000,0.000000}%
\pgfsetfillcolor{currentfill}%
\pgfsetlinewidth{0.803000pt}%
\definecolor{currentstroke}{rgb}{0.000000,0.000000,0.000000}%
\pgfsetstrokecolor{currentstroke}%
\pgfsetdash{}{0pt}%
\pgfsys@defobject{currentmarker}{\pgfqpoint{0.000000in}{-0.048611in}}{\pgfqpoint{0.000000in}{0.000000in}}{%
\pgfpathmoveto{\pgfqpoint{0.000000in}{0.000000in}}%
\pgfpathlineto{\pgfqpoint{0.000000in}{-0.048611in}}%
\pgfusepath{stroke,fill}%
}%
\begin{pgfscope}%
\pgfsys@transformshift{1.466137in}{0.499444in}%
\pgfsys@useobject{currentmarker}{}%
\end{pgfscope}%
\end{pgfscope}%
\begin{pgfscope}%
\pgfsetbuttcap%
\pgfsetroundjoin%
\definecolor{currentfill}{rgb}{0.000000,0.000000,0.000000}%
\pgfsetfillcolor{currentfill}%
\pgfsetlinewidth{0.803000pt}%
\definecolor{currentstroke}{rgb}{0.000000,0.000000,0.000000}%
\pgfsetstrokecolor{currentstroke}%
\pgfsetdash{}{0pt}%
\pgfsys@defobject{currentmarker}{\pgfqpoint{0.000000in}{-0.048611in}}{\pgfqpoint{0.000000in}{0.000000in}}{%
\pgfpathmoveto{\pgfqpoint{0.000000in}{0.000000in}}%
\pgfpathlineto{\pgfqpoint{0.000000in}{-0.048611in}}%
\pgfusepath{stroke,fill}%
}%
\begin{pgfscope}%
\pgfsys@transformshift{1.624659in}{0.499444in}%
\pgfsys@useobject{currentmarker}{}%
\end{pgfscope}%
\end{pgfscope}%
\begin{pgfscope}%
\definecolor{textcolor}{rgb}{0.000000,0.000000,0.000000}%
\pgfsetstrokecolor{textcolor}%
\pgfsetfillcolor{textcolor}%
\pgftext[x=1.624659in,y=0.402222in,,top]{\color{textcolor}\rmfamily\fontsize{10.000000}{12.000000}\selectfont 0.3}%
\end{pgfscope}%
\begin{pgfscope}%
\pgfsetbuttcap%
\pgfsetroundjoin%
\definecolor{currentfill}{rgb}{0.000000,0.000000,0.000000}%
\pgfsetfillcolor{currentfill}%
\pgfsetlinewidth{0.803000pt}%
\definecolor{currentstroke}{rgb}{0.000000,0.000000,0.000000}%
\pgfsetstrokecolor{currentstroke}%
\pgfsetdash{}{0pt}%
\pgfsys@defobject{currentmarker}{\pgfqpoint{0.000000in}{-0.048611in}}{\pgfqpoint{0.000000in}{0.000000in}}{%
\pgfpathmoveto{\pgfqpoint{0.000000in}{0.000000in}}%
\pgfpathlineto{\pgfqpoint{0.000000in}{-0.048611in}}%
\pgfusepath{stroke,fill}%
}%
\begin{pgfscope}%
\pgfsys@transformshift{1.783182in}{0.499444in}%
\pgfsys@useobject{currentmarker}{}%
\end{pgfscope}%
\end{pgfscope}%
\begin{pgfscope}%
\pgfsetbuttcap%
\pgfsetroundjoin%
\definecolor{currentfill}{rgb}{0.000000,0.000000,0.000000}%
\pgfsetfillcolor{currentfill}%
\pgfsetlinewidth{0.803000pt}%
\definecolor{currentstroke}{rgb}{0.000000,0.000000,0.000000}%
\pgfsetstrokecolor{currentstroke}%
\pgfsetdash{}{0pt}%
\pgfsys@defobject{currentmarker}{\pgfqpoint{0.000000in}{-0.048611in}}{\pgfqpoint{0.000000in}{0.000000in}}{%
\pgfpathmoveto{\pgfqpoint{0.000000in}{0.000000in}}%
\pgfpathlineto{\pgfqpoint{0.000000in}{-0.048611in}}%
\pgfusepath{stroke,fill}%
}%
\begin{pgfscope}%
\pgfsys@transformshift{1.941705in}{0.499444in}%
\pgfsys@useobject{currentmarker}{}%
\end{pgfscope}%
\end{pgfscope}%
\begin{pgfscope}%
\definecolor{textcolor}{rgb}{0.000000,0.000000,0.000000}%
\pgfsetstrokecolor{textcolor}%
\pgfsetfillcolor{textcolor}%
\pgftext[x=1.941705in,y=0.402222in,,top]{\color{textcolor}\rmfamily\fontsize{10.000000}{12.000000}\selectfont 0.4}%
\end{pgfscope}%
\begin{pgfscope}%
\pgfsetbuttcap%
\pgfsetroundjoin%
\definecolor{currentfill}{rgb}{0.000000,0.000000,0.000000}%
\pgfsetfillcolor{currentfill}%
\pgfsetlinewidth{0.803000pt}%
\definecolor{currentstroke}{rgb}{0.000000,0.000000,0.000000}%
\pgfsetstrokecolor{currentstroke}%
\pgfsetdash{}{0pt}%
\pgfsys@defobject{currentmarker}{\pgfqpoint{0.000000in}{-0.048611in}}{\pgfqpoint{0.000000in}{0.000000in}}{%
\pgfpathmoveto{\pgfqpoint{0.000000in}{0.000000in}}%
\pgfpathlineto{\pgfqpoint{0.000000in}{-0.048611in}}%
\pgfusepath{stroke,fill}%
}%
\begin{pgfscope}%
\pgfsys@transformshift{2.100228in}{0.499444in}%
\pgfsys@useobject{currentmarker}{}%
\end{pgfscope}%
\end{pgfscope}%
\begin{pgfscope}%
\pgfsetbuttcap%
\pgfsetroundjoin%
\definecolor{currentfill}{rgb}{0.000000,0.000000,0.000000}%
\pgfsetfillcolor{currentfill}%
\pgfsetlinewidth{0.803000pt}%
\definecolor{currentstroke}{rgb}{0.000000,0.000000,0.000000}%
\pgfsetstrokecolor{currentstroke}%
\pgfsetdash{}{0pt}%
\pgfsys@defobject{currentmarker}{\pgfqpoint{0.000000in}{-0.048611in}}{\pgfqpoint{0.000000in}{0.000000in}}{%
\pgfpathmoveto{\pgfqpoint{0.000000in}{0.000000in}}%
\pgfpathlineto{\pgfqpoint{0.000000in}{-0.048611in}}%
\pgfusepath{stroke,fill}%
}%
\begin{pgfscope}%
\pgfsys@transformshift{2.258750in}{0.499444in}%
\pgfsys@useobject{currentmarker}{}%
\end{pgfscope}%
\end{pgfscope}%
\begin{pgfscope}%
\definecolor{textcolor}{rgb}{0.000000,0.000000,0.000000}%
\pgfsetstrokecolor{textcolor}%
\pgfsetfillcolor{textcolor}%
\pgftext[x=2.258750in,y=0.402222in,,top]{\color{textcolor}\rmfamily\fontsize{10.000000}{12.000000}\selectfont 0.5}%
\end{pgfscope}%
\begin{pgfscope}%
\pgfsetbuttcap%
\pgfsetroundjoin%
\definecolor{currentfill}{rgb}{0.000000,0.000000,0.000000}%
\pgfsetfillcolor{currentfill}%
\pgfsetlinewidth{0.803000pt}%
\definecolor{currentstroke}{rgb}{0.000000,0.000000,0.000000}%
\pgfsetstrokecolor{currentstroke}%
\pgfsetdash{}{0pt}%
\pgfsys@defobject{currentmarker}{\pgfqpoint{0.000000in}{-0.048611in}}{\pgfqpoint{0.000000in}{0.000000in}}{%
\pgfpathmoveto{\pgfqpoint{0.000000in}{0.000000in}}%
\pgfpathlineto{\pgfqpoint{0.000000in}{-0.048611in}}%
\pgfusepath{stroke,fill}%
}%
\begin{pgfscope}%
\pgfsys@transformshift{2.417273in}{0.499444in}%
\pgfsys@useobject{currentmarker}{}%
\end{pgfscope}%
\end{pgfscope}%
\begin{pgfscope}%
\pgfsetbuttcap%
\pgfsetroundjoin%
\definecolor{currentfill}{rgb}{0.000000,0.000000,0.000000}%
\pgfsetfillcolor{currentfill}%
\pgfsetlinewidth{0.803000pt}%
\definecolor{currentstroke}{rgb}{0.000000,0.000000,0.000000}%
\pgfsetstrokecolor{currentstroke}%
\pgfsetdash{}{0pt}%
\pgfsys@defobject{currentmarker}{\pgfqpoint{0.000000in}{-0.048611in}}{\pgfqpoint{0.000000in}{0.000000in}}{%
\pgfpathmoveto{\pgfqpoint{0.000000in}{0.000000in}}%
\pgfpathlineto{\pgfqpoint{0.000000in}{-0.048611in}}%
\pgfusepath{stroke,fill}%
}%
\begin{pgfscope}%
\pgfsys@transformshift{2.575796in}{0.499444in}%
\pgfsys@useobject{currentmarker}{}%
\end{pgfscope}%
\end{pgfscope}%
\begin{pgfscope}%
\definecolor{textcolor}{rgb}{0.000000,0.000000,0.000000}%
\pgfsetstrokecolor{textcolor}%
\pgfsetfillcolor{textcolor}%
\pgftext[x=2.575796in,y=0.402222in,,top]{\color{textcolor}\rmfamily\fontsize{10.000000}{12.000000}\selectfont 0.6}%
\end{pgfscope}%
\begin{pgfscope}%
\pgfsetbuttcap%
\pgfsetroundjoin%
\definecolor{currentfill}{rgb}{0.000000,0.000000,0.000000}%
\pgfsetfillcolor{currentfill}%
\pgfsetlinewidth{0.803000pt}%
\definecolor{currentstroke}{rgb}{0.000000,0.000000,0.000000}%
\pgfsetstrokecolor{currentstroke}%
\pgfsetdash{}{0pt}%
\pgfsys@defobject{currentmarker}{\pgfqpoint{0.000000in}{-0.048611in}}{\pgfqpoint{0.000000in}{0.000000in}}{%
\pgfpathmoveto{\pgfqpoint{0.000000in}{0.000000in}}%
\pgfpathlineto{\pgfqpoint{0.000000in}{-0.048611in}}%
\pgfusepath{stroke,fill}%
}%
\begin{pgfscope}%
\pgfsys@transformshift{2.734318in}{0.499444in}%
\pgfsys@useobject{currentmarker}{}%
\end{pgfscope}%
\end{pgfscope}%
\begin{pgfscope}%
\pgfsetbuttcap%
\pgfsetroundjoin%
\definecolor{currentfill}{rgb}{0.000000,0.000000,0.000000}%
\pgfsetfillcolor{currentfill}%
\pgfsetlinewidth{0.803000pt}%
\definecolor{currentstroke}{rgb}{0.000000,0.000000,0.000000}%
\pgfsetstrokecolor{currentstroke}%
\pgfsetdash{}{0pt}%
\pgfsys@defobject{currentmarker}{\pgfqpoint{0.000000in}{-0.048611in}}{\pgfqpoint{0.000000in}{0.000000in}}{%
\pgfpathmoveto{\pgfqpoint{0.000000in}{0.000000in}}%
\pgfpathlineto{\pgfqpoint{0.000000in}{-0.048611in}}%
\pgfusepath{stroke,fill}%
}%
\begin{pgfscope}%
\pgfsys@transformshift{2.892841in}{0.499444in}%
\pgfsys@useobject{currentmarker}{}%
\end{pgfscope}%
\end{pgfscope}%
\begin{pgfscope}%
\definecolor{textcolor}{rgb}{0.000000,0.000000,0.000000}%
\pgfsetstrokecolor{textcolor}%
\pgfsetfillcolor{textcolor}%
\pgftext[x=2.892841in,y=0.402222in,,top]{\color{textcolor}\rmfamily\fontsize{10.000000}{12.000000}\selectfont 0.7}%
\end{pgfscope}%
\begin{pgfscope}%
\pgfsetbuttcap%
\pgfsetroundjoin%
\definecolor{currentfill}{rgb}{0.000000,0.000000,0.000000}%
\pgfsetfillcolor{currentfill}%
\pgfsetlinewidth{0.803000pt}%
\definecolor{currentstroke}{rgb}{0.000000,0.000000,0.000000}%
\pgfsetstrokecolor{currentstroke}%
\pgfsetdash{}{0pt}%
\pgfsys@defobject{currentmarker}{\pgfqpoint{0.000000in}{-0.048611in}}{\pgfqpoint{0.000000in}{0.000000in}}{%
\pgfpathmoveto{\pgfqpoint{0.000000in}{0.000000in}}%
\pgfpathlineto{\pgfqpoint{0.000000in}{-0.048611in}}%
\pgfusepath{stroke,fill}%
}%
\begin{pgfscope}%
\pgfsys@transformshift{3.051364in}{0.499444in}%
\pgfsys@useobject{currentmarker}{}%
\end{pgfscope}%
\end{pgfscope}%
\begin{pgfscope}%
\pgfsetbuttcap%
\pgfsetroundjoin%
\definecolor{currentfill}{rgb}{0.000000,0.000000,0.000000}%
\pgfsetfillcolor{currentfill}%
\pgfsetlinewidth{0.803000pt}%
\definecolor{currentstroke}{rgb}{0.000000,0.000000,0.000000}%
\pgfsetstrokecolor{currentstroke}%
\pgfsetdash{}{0pt}%
\pgfsys@defobject{currentmarker}{\pgfqpoint{0.000000in}{-0.048611in}}{\pgfqpoint{0.000000in}{0.000000in}}{%
\pgfpathmoveto{\pgfqpoint{0.000000in}{0.000000in}}%
\pgfpathlineto{\pgfqpoint{0.000000in}{-0.048611in}}%
\pgfusepath{stroke,fill}%
}%
\begin{pgfscope}%
\pgfsys@transformshift{3.209887in}{0.499444in}%
\pgfsys@useobject{currentmarker}{}%
\end{pgfscope}%
\end{pgfscope}%
\begin{pgfscope}%
\definecolor{textcolor}{rgb}{0.000000,0.000000,0.000000}%
\pgfsetstrokecolor{textcolor}%
\pgfsetfillcolor{textcolor}%
\pgftext[x=3.209887in,y=0.402222in,,top]{\color{textcolor}\rmfamily\fontsize{10.000000}{12.000000}\selectfont 0.8}%
\end{pgfscope}%
\begin{pgfscope}%
\pgfsetbuttcap%
\pgfsetroundjoin%
\definecolor{currentfill}{rgb}{0.000000,0.000000,0.000000}%
\pgfsetfillcolor{currentfill}%
\pgfsetlinewidth{0.803000pt}%
\definecolor{currentstroke}{rgb}{0.000000,0.000000,0.000000}%
\pgfsetstrokecolor{currentstroke}%
\pgfsetdash{}{0pt}%
\pgfsys@defobject{currentmarker}{\pgfqpoint{0.000000in}{-0.048611in}}{\pgfqpoint{0.000000in}{0.000000in}}{%
\pgfpathmoveto{\pgfqpoint{0.000000in}{0.000000in}}%
\pgfpathlineto{\pgfqpoint{0.000000in}{-0.048611in}}%
\pgfusepath{stroke,fill}%
}%
\begin{pgfscope}%
\pgfsys@transformshift{3.368409in}{0.499444in}%
\pgfsys@useobject{currentmarker}{}%
\end{pgfscope}%
\end{pgfscope}%
\begin{pgfscope}%
\pgfsetbuttcap%
\pgfsetroundjoin%
\definecolor{currentfill}{rgb}{0.000000,0.000000,0.000000}%
\pgfsetfillcolor{currentfill}%
\pgfsetlinewidth{0.803000pt}%
\definecolor{currentstroke}{rgb}{0.000000,0.000000,0.000000}%
\pgfsetstrokecolor{currentstroke}%
\pgfsetdash{}{0pt}%
\pgfsys@defobject{currentmarker}{\pgfqpoint{0.000000in}{-0.048611in}}{\pgfqpoint{0.000000in}{0.000000in}}{%
\pgfpathmoveto{\pgfqpoint{0.000000in}{0.000000in}}%
\pgfpathlineto{\pgfqpoint{0.000000in}{-0.048611in}}%
\pgfusepath{stroke,fill}%
}%
\begin{pgfscope}%
\pgfsys@transformshift{3.526932in}{0.499444in}%
\pgfsys@useobject{currentmarker}{}%
\end{pgfscope}%
\end{pgfscope}%
\begin{pgfscope}%
\definecolor{textcolor}{rgb}{0.000000,0.000000,0.000000}%
\pgfsetstrokecolor{textcolor}%
\pgfsetfillcolor{textcolor}%
\pgftext[x=3.526932in,y=0.402222in,,top]{\color{textcolor}\rmfamily\fontsize{10.000000}{12.000000}\selectfont 0.9}%
\end{pgfscope}%
\begin{pgfscope}%
\pgfsetbuttcap%
\pgfsetroundjoin%
\definecolor{currentfill}{rgb}{0.000000,0.000000,0.000000}%
\pgfsetfillcolor{currentfill}%
\pgfsetlinewidth{0.803000pt}%
\definecolor{currentstroke}{rgb}{0.000000,0.000000,0.000000}%
\pgfsetstrokecolor{currentstroke}%
\pgfsetdash{}{0pt}%
\pgfsys@defobject{currentmarker}{\pgfqpoint{0.000000in}{-0.048611in}}{\pgfqpoint{0.000000in}{0.000000in}}{%
\pgfpathmoveto{\pgfqpoint{0.000000in}{0.000000in}}%
\pgfpathlineto{\pgfqpoint{0.000000in}{-0.048611in}}%
\pgfusepath{stroke,fill}%
}%
\begin{pgfscope}%
\pgfsys@transformshift{3.685455in}{0.499444in}%
\pgfsys@useobject{currentmarker}{}%
\end{pgfscope}%
\end{pgfscope}%
\begin{pgfscope}%
\pgfsetbuttcap%
\pgfsetroundjoin%
\definecolor{currentfill}{rgb}{0.000000,0.000000,0.000000}%
\pgfsetfillcolor{currentfill}%
\pgfsetlinewidth{0.803000pt}%
\definecolor{currentstroke}{rgb}{0.000000,0.000000,0.000000}%
\pgfsetstrokecolor{currentstroke}%
\pgfsetdash{}{0pt}%
\pgfsys@defobject{currentmarker}{\pgfqpoint{0.000000in}{-0.048611in}}{\pgfqpoint{0.000000in}{0.000000in}}{%
\pgfpathmoveto{\pgfqpoint{0.000000in}{0.000000in}}%
\pgfpathlineto{\pgfqpoint{0.000000in}{-0.048611in}}%
\pgfusepath{stroke,fill}%
}%
\begin{pgfscope}%
\pgfsys@transformshift{3.843978in}{0.499444in}%
\pgfsys@useobject{currentmarker}{}%
\end{pgfscope}%
\end{pgfscope}%
\begin{pgfscope}%
\definecolor{textcolor}{rgb}{0.000000,0.000000,0.000000}%
\pgfsetstrokecolor{textcolor}%
\pgfsetfillcolor{textcolor}%
\pgftext[x=3.843978in,y=0.402222in,,top]{\color{textcolor}\rmfamily\fontsize{10.000000}{12.000000}\selectfont 1.0}%
\end{pgfscope}%
\begin{pgfscope}%
\pgfsetbuttcap%
\pgfsetroundjoin%
\definecolor{currentfill}{rgb}{0.000000,0.000000,0.000000}%
\pgfsetfillcolor{currentfill}%
\pgfsetlinewidth{0.803000pt}%
\definecolor{currentstroke}{rgb}{0.000000,0.000000,0.000000}%
\pgfsetstrokecolor{currentstroke}%
\pgfsetdash{}{0pt}%
\pgfsys@defobject{currentmarker}{\pgfqpoint{0.000000in}{-0.048611in}}{\pgfqpoint{0.000000in}{0.000000in}}{%
\pgfpathmoveto{\pgfqpoint{0.000000in}{0.000000in}}%
\pgfpathlineto{\pgfqpoint{0.000000in}{-0.048611in}}%
\pgfusepath{stroke,fill}%
}%
\begin{pgfscope}%
\pgfsys@transformshift{4.002500in}{0.499444in}%
\pgfsys@useobject{currentmarker}{}%
\end{pgfscope}%
\end{pgfscope}%
\begin{pgfscope}%
\definecolor{textcolor}{rgb}{0.000000,0.000000,0.000000}%
\pgfsetstrokecolor{textcolor}%
\pgfsetfillcolor{textcolor}%
\pgftext[x=2.258750in,y=0.223333in,,top]{\color{textcolor}\rmfamily\fontsize{10.000000}{12.000000}\selectfont \(\displaystyle p\)}%
\end{pgfscope}%
\begin{pgfscope}%
\pgfsetbuttcap%
\pgfsetroundjoin%
\definecolor{currentfill}{rgb}{0.000000,0.000000,0.000000}%
\pgfsetfillcolor{currentfill}%
\pgfsetlinewidth{0.803000pt}%
\definecolor{currentstroke}{rgb}{0.000000,0.000000,0.000000}%
\pgfsetstrokecolor{currentstroke}%
\pgfsetdash{}{0pt}%
\pgfsys@defobject{currentmarker}{\pgfqpoint{-0.048611in}{0.000000in}}{\pgfqpoint{-0.000000in}{0.000000in}}{%
\pgfpathmoveto{\pgfqpoint{-0.000000in}{0.000000in}}%
\pgfpathlineto{\pgfqpoint{-0.048611in}{0.000000in}}%
\pgfusepath{stroke,fill}%
}%
\begin{pgfscope}%
\pgfsys@transformshift{0.515000in}{0.499444in}%
\pgfsys@useobject{currentmarker}{}%
\end{pgfscope}%
\end{pgfscope}%
\begin{pgfscope}%
\definecolor{textcolor}{rgb}{0.000000,0.000000,0.000000}%
\pgfsetstrokecolor{textcolor}%
\pgfsetfillcolor{textcolor}%
\pgftext[x=0.348333in, y=0.451250in, left, base]{\color{textcolor}\rmfamily\fontsize{10.000000}{12.000000}\selectfont \(\displaystyle {0}\)}%
\end{pgfscope}%
\begin{pgfscope}%
\pgfsetbuttcap%
\pgfsetroundjoin%
\definecolor{currentfill}{rgb}{0.000000,0.000000,0.000000}%
\pgfsetfillcolor{currentfill}%
\pgfsetlinewidth{0.803000pt}%
\definecolor{currentstroke}{rgb}{0.000000,0.000000,0.000000}%
\pgfsetstrokecolor{currentstroke}%
\pgfsetdash{}{0pt}%
\pgfsys@defobject{currentmarker}{\pgfqpoint{-0.048611in}{0.000000in}}{\pgfqpoint{-0.000000in}{0.000000in}}{%
\pgfpathmoveto{\pgfqpoint{-0.000000in}{0.000000in}}%
\pgfpathlineto{\pgfqpoint{-0.048611in}{0.000000in}}%
\pgfusepath{stroke,fill}%
}%
\begin{pgfscope}%
\pgfsys@transformshift{0.515000in}{1.059437in}%
\pgfsys@useobject{currentmarker}{}%
\end{pgfscope}%
\end{pgfscope}%
\begin{pgfscope}%
\definecolor{textcolor}{rgb}{0.000000,0.000000,0.000000}%
\pgfsetstrokecolor{textcolor}%
\pgfsetfillcolor{textcolor}%
\pgftext[x=0.348333in, y=1.011243in, left, base]{\color{textcolor}\rmfamily\fontsize{10.000000}{12.000000}\selectfont \(\displaystyle {5}\)}%
\end{pgfscope}%
\begin{pgfscope}%
\pgfsetbuttcap%
\pgfsetroundjoin%
\definecolor{currentfill}{rgb}{0.000000,0.000000,0.000000}%
\pgfsetfillcolor{currentfill}%
\pgfsetlinewidth{0.803000pt}%
\definecolor{currentstroke}{rgb}{0.000000,0.000000,0.000000}%
\pgfsetstrokecolor{currentstroke}%
\pgfsetdash{}{0pt}%
\pgfsys@defobject{currentmarker}{\pgfqpoint{-0.048611in}{0.000000in}}{\pgfqpoint{-0.000000in}{0.000000in}}{%
\pgfpathmoveto{\pgfqpoint{-0.000000in}{0.000000in}}%
\pgfpathlineto{\pgfqpoint{-0.048611in}{0.000000in}}%
\pgfusepath{stroke,fill}%
}%
\begin{pgfscope}%
\pgfsys@transformshift{0.515000in}{1.619430in}%
\pgfsys@useobject{currentmarker}{}%
\end{pgfscope}%
\end{pgfscope}%
\begin{pgfscope}%
\definecolor{textcolor}{rgb}{0.000000,0.000000,0.000000}%
\pgfsetstrokecolor{textcolor}%
\pgfsetfillcolor{textcolor}%
\pgftext[x=0.278889in, y=1.571235in, left, base]{\color{textcolor}\rmfamily\fontsize{10.000000}{12.000000}\selectfont \(\displaystyle {10}\)}%
\end{pgfscope}%
\begin{pgfscope}%
\definecolor{textcolor}{rgb}{0.000000,0.000000,0.000000}%
\pgfsetstrokecolor{textcolor}%
\pgfsetfillcolor{textcolor}%
\pgftext[x=0.223333in,y=1.076944in,,bottom,rotate=90.000000]{\color{textcolor}\rmfamily\fontsize{10.000000}{12.000000}\selectfont Percent of Data Set}%
\end{pgfscope}%
\begin{pgfscope}%
\pgfsetrectcap%
\pgfsetmiterjoin%
\pgfsetlinewidth{0.803000pt}%
\definecolor{currentstroke}{rgb}{0.000000,0.000000,0.000000}%
\pgfsetstrokecolor{currentstroke}%
\pgfsetdash{}{0pt}%
\pgfpathmoveto{\pgfqpoint{0.515000in}{0.499444in}}%
\pgfpathlineto{\pgfqpoint{0.515000in}{1.654444in}}%
\pgfusepath{stroke}%
\end{pgfscope}%
\begin{pgfscope}%
\pgfsetrectcap%
\pgfsetmiterjoin%
\pgfsetlinewidth{0.803000pt}%
\definecolor{currentstroke}{rgb}{0.000000,0.000000,0.000000}%
\pgfsetstrokecolor{currentstroke}%
\pgfsetdash{}{0pt}%
\pgfpathmoveto{\pgfqpoint{4.002500in}{0.499444in}}%
\pgfpathlineto{\pgfqpoint{4.002500in}{1.654444in}}%
\pgfusepath{stroke}%
\end{pgfscope}%
\begin{pgfscope}%
\pgfsetrectcap%
\pgfsetmiterjoin%
\pgfsetlinewidth{0.803000pt}%
\definecolor{currentstroke}{rgb}{0.000000,0.000000,0.000000}%
\pgfsetstrokecolor{currentstroke}%
\pgfsetdash{}{0pt}%
\pgfpathmoveto{\pgfqpoint{0.515000in}{0.499444in}}%
\pgfpathlineto{\pgfqpoint{4.002500in}{0.499444in}}%
\pgfusepath{stroke}%
\end{pgfscope}%
\begin{pgfscope}%
\pgfsetrectcap%
\pgfsetmiterjoin%
\pgfsetlinewidth{0.803000pt}%
\definecolor{currentstroke}{rgb}{0.000000,0.000000,0.000000}%
\pgfsetstrokecolor{currentstroke}%
\pgfsetdash{}{0pt}%
\pgfpathmoveto{\pgfqpoint{0.515000in}{1.654444in}}%
\pgfpathlineto{\pgfqpoint{4.002500in}{1.654444in}}%
\pgfusepath{stroke}%
\end{pgfscope}%
\begin{pgfscope}%
\pgfsetbuttcap%
\pgfsetmiterjoin%
\definecolor{currentfill}{rgb}{1.000000,1.000000,1.000000}%
\pgfsetfillcolor{currentfill}%
\pgfsetfillopacity{0.800000}%
\pgfsetlinewidth{1.003750pt}%
\definecolor{currentstroke}{rgb}{0.800000,0.800000,0.800000}%
\pgfsetstrokecolor{currentstroke}%
\pgfsetstrokeopacity{0.800000}%
\pgfsetdash{}{0pt}%
\pgfpathmoveto{\pgfqpoint{3.225556in}{1.154445in}}%
\pgfpathlineto{\pgfqpoint{3.905278in}{1.154445in}}%
\pgfpathquadraticcurveto{\pgfqpoint{3.933056in}{1.154445in}}{\pgfqpoint{3.933056in}{1.182222in}}%
\pgfpathlineto{\pgfqpoint{3.933056in}{1.557222in}}%
\pgfpathquadraticcurveto{\pgfqpoint{3.933056in}{1.585000in}}{\pgfqpoint{3.905278in}{1.585000in}}%
\pgfpathlineto{\pgfqpoint{3.225556in}{1.585000in}}%
\pgfpathquadraticcurveto{\pgfqpoint{3.197778in}{1.585000in}}{\pgfqpoint{3.197778in}{1.557222in}}%
\pgfpathlineto{\pgfqpoint{3.197778in}{1.182222in}}%
\pgfpathquadraticcurveto{\pgfqpoint{3.197778in}{1.154445in}}{\pgfqpoint{3.225556in}{1.154445in}}%
\pgfpathlineto{\pgfqpoint{3.225556in}{1.154445in}}%
\pgfpathclose%
\pgfusepath{stroke,fill}%
\end{pgfscope}%
\begin{pgfscope}%
\pgfsetbuttcap%
\pgfsetmiterjoin%
\pgfsetlinewidth{1.003750pt}%
\definecolor{currentstroke}{rgb}{0.000000,0.000000,0.000000}%
\pgfsetstrokecolor{currentstroke}%
\pgfsetdash{}{0pt}%
\pgfpathmoveto{\pgfqpoint{3.253334in}{1.432222in}}%
\pgfpathlineto{\pgfqpoint{3.531111in}{1.432222in}}%
\pgfpathlineto{\pgfqpoint{3.531111in}{1.529444in}}%
\pgfpathlineto{\pgfqpoint{3.253334in}{1.529444in}}%
\pgfpathlineto{\pgfqpoint{3.253334in}{1.432222in}}%
\pgfpathclose%
\pgfusepath{stroke}%
\end{pgfscope}%
\begin{pgfscope}%
\definecolor{textcolor}{rgb}{0.000000,0.000000,0.000000}%
\pgfsetstrokecolor{textcolor}%
\pgfsetfillcolor{textcolor}%
\pgftext[x=3.642223in,y=1.432222in,left,base]{\color{textcolor}\rmfamily\fontsize{10.000000}{12.000000}\selectfont Neg}%
\end{pgfscope}%
\begin{pgfscope}%
\pgfsetbuttcap%
\pgfsetmiterjoin%
\definecolor{currentfill}{rgb}{0.000000,0.000000,0.000000}%
\pgfsetfillcolor{currentfill}%
\pgfsetlinewidth{0.000000pt}%
\definecolor{currentstroke}{rgb}{0.000000,0.000000,0.000000}%
\pgfsetstrokecolor{currentstroke}%
\pgfsetstrokeopacity{0.000000}%
\pgfsetdash{}{0pt}%
\pgfpathmoveto{\pgfqpoint{3.253334in}{1.236944in}}%
\pgfpathlineto{\pgfqpoint{3.531111in}{1.236944in}}%
\pgfpathlineto{\pgfqpoint{3.531111in}{1.334167in}}%
\pgfpathlineto{\pgfqpoint{3.253334in}{1.334167in}}%
\pgfpathlineto{\pgfqpoint{3.253334in}{1.236944in}}%
\pgfpathclose%
\pgfusepath{fill}%
\end{pgfscope}%
\begin{pgfscope}%
\definecolor{textcolor}{rgb}{0.000000,0.000000,0.000000}%
\pgfsetstrokecolor{textcolor}%
\pgfsetfillcolor{textcolor}%
\pgftext[x=3.642223in,y=1.236944in,left,base]{\color{textcolor}\rmfamily\fontsize{10.000000}{12.000000}\selectfont Pos}%
\end{pgfscope}%
\end{pgfpicture}%
\makeatother%
\endgroup%
	
&
	\vskip 0pt
	\hfil ROC Curve
	
	%% Creator: Matplotlib, PGF backend
%%
%% To include the figure in your LaTeX document, write
%%   \input{<filename>.pgf}
%%
%% Make sure the required packages are loaded in your preamble
%%   \usepackage{pgf}
%%
%% Also ensure that all the required font packages are loaded; for instance,
%% the lmodern package is sometimes necessary when using math font.
%%   \usepackage{lmodern}
%%
%% Figures using additional raster images can only be included by \input if
%% they are in the same directory as the main LaTeX file. For loading figures
%% from other directories you can use the `import` package
%%   \usepackage{import}
%%
%% and then include the figures with
%%   \import{<path to file>}{<filename>.pgf}
%%
%% Matplotlib used the following preamble
%%   
%%   \usepackage{fontspec}
%%   \makeatletter\@ifpackageloaded{underscore}{}{\usepackage[strings]{underscore}}\makeatother
%%
\begingroup%
\makeatletter%
\begin{pgfpicture}%
\pgfpathrectangle{\pgfpointorigin}{\pgfqpoint{2.221861in}{1.754444in}}%
\pgfusepath{use as bounding box, clip}%
\begin{pgfscope}%
\pgfsetbuttcap%
\pgfsetmiterjoin%
\definecolor{currentfill}{rgb}{1.000000,1.000000,1.000000}%
\pgfsetfillcolor{currentfill}%
\pgfsetlinewidth{0.000000pt}%
\definecolor{currentstroke}{rgb}{1.000000,1.000000,1.000000}%
\pgfsetstrokecolor{currentstroke}%
\pgfsetdash{}{0pt}%
\pgfpathmoveto{\pgfqpoint{0.000000in}{0.000000in}}%
\pgfpathlineto{\pgfqpoint{2.221861in}{0.000000in}}%
\pgfpathlineto{\pgfqpoint{2.221861in}{1.754444in}}%
\pgfpathlineto{\pgfqpoint{0.000000in}{1.754444in}}%
\pgfpathlineto{\pgfqpoint{0.000000in}{0.000000in}}%
\pgfpathclose%
\pgfusepath{fill}%
\end{pgfscope}%
\begin{pgfscope}%
\pgfsetbuttcap%
\pgfsetmiterjoin%
\definecolor{currentfill}{rgb}{1.000000,1.000000,1.000000}%
\pgfsetfillcolor{currentfill}%
\pgfsetlinewidth{0.000000pt}%
\definecolor{currentstroke}{rgb}{0.000000,0.000000,0.000000}%
\pgfsetstrokecolor{currentstroke}%
\pgfsetstrokeopacity{0.000000}%
\pgfsetdash{}{0pt}%
\pgfpathmoveto{\pgfqpoint{0.553581in}{0.499444in}}%
\pgfpathlineto{\pgfqpoint{2.103581in}{0.499444in}}%
\pgfpathlineto{\pgfqpoint{2.103581in}{1.654444in}}%
\pgfpathlineto{\pgfqpoint{0.553581in}{1.654444in}}%
\pgfpathlineto{\pgfqpoint{0.553581in}{0.499444in}}%
\pgfpathclose%
\pgfusepath{fill}%
\end{pgfscope}%
\begin{pgfscope}%
\pgfsetbuttcap%
\pgfsetroundjoin%
\definecolor{currentfill}{rgb}{0.000000,0.000000,0.000000}%
\pgfsetfillcolor{currentfill}%
\pgfsetlinewidth{0.803000pt}%
\definecolor{currentstroke}{rgb}{0.000000,0.000000,0.000000}%
\pgfsetstrokecolor{currentstroke}%
\pgfsetdash{}{0pt}%
\pgfsys@defobject{currentmarker}{\pgfqpoint{0.000000in}{-0.048611in}}{\pgfqpoint{0.000000in}{0.000000in}}{%
\pgfpathmoveto{\pgfqpoint{0.000000in}{0.000000in}}%
\pgfpathlineto{\pgfqpoint{0.000000in}{-0.048611in}}%
\pgfusepath{stroke,fill}%
}%
\begin{pgfscope}%
\pgfsys@transformshift{0.624035in}{0.499444in}%
\pgfsys@useobject{currentmarker}{}%
\end{pgfscope}%
\end{pgfscope}%
\begin{pgfscope}%
\definecolor{textcolor}{rgb}{0.000000,0.000000,0.000000}%
\pgfsetstrokecolor{textcolor}%
\pgfsetfillcolor{textcolor}%
\pgftext[x=0.624035in,y=0.402222in,,top]{\color{textcolor}\rmfamily\fontsize{10.000000}{12.000000}\selectfont \(\displaystyle {0.0}\)}%
\end{pgfscope}%
\begin{pgfscope}%
\pgfsetbuttcap%
\pgfsetroundjoin%
\definecolor{currentfill}{rgb}{0.000000,0.000000,0.000000}%
\pgfsetfillcolor{currentfill}%
\pgfsetlinewidth{0.803000pt}%
\definecolor{currentstroke}{rgb}{0.000000,0.000000,0.000000}%
\pgfsetstrokecolor{currentstroke}%
\pgfsetdash{}{0pt}%
\pgfsys@defobject{currentmarker}{\pgfqpoint{0.000000in}{-0.048611in}}{\pgfqpoint{0.000000in}{0.000000in}}{%
\pgfpathmoveto{\pgfqpoint{0.000000in}{0.000000in}}%
\pgfpathlineto{\pgfqpoint{0.000000in}{-0.048611in}}%
\pgfusepath{stroke,fill}%
}%
\begin{pgfscope}%
\pgfsys@transformshift{1.328581in}{0.499444in}%
\pgfsys@useobject{currentmarker}{}%
\end{pgfscope}%
\end{pgfscope}%
\begin{pgfscope}%
\definecolor{textcolor}{rgb}{0.000000,0.000000,0.000000}%
\pgfsetstrokecolor{textcolor}%
\pgfsetfillcolor{textcolor}%
\pgftext[x=1.328581in,y=0.402222in,,top]{\color{textcolor}\rmfamily\fontsize{10.000000}{12.000000}\selectfont \(\displaystyle {0.5}\)}%
\end{pgfscope}%
\begin{pgfscope}%
\pgfsetbuttcap%
\pgfsetroundjoin%
\definecolor{currentfill}{rgb}{0.000000,0.000000,0.000000}%
\pgfsetfillcolor{currentfill}%
\pgfsetlinewidth{0.803000pt}%
\definecolor{currentstroke}{rgb}{0.000000,0.000000,0.000000}%
\pgfsetstrokecolor{currentstroke}%
\pgfsetdash{}{0pt}%
\pgfsys@defobject{currentmarker}{\pgfqpoint{0.000000in}{-0.048611in}}{\pgfqpoint{0.000000in}{0.000000in}}{%
\pgfpathmoveto{\pgfqpoint{0.000000in}{0.000000in}}%
\pgfpathlineto{\pgfqpoint{0.000000in}{-0.048611in}}%
\pgfusepath{stroke,fill}%
}%
\begin{pgfscope}%
\pgfsys@transformshift{2.033126in}{0.499444in}%
\pgfsys@useobject{currentmarker}{}%
\end{pgfscope}%
\end{pgfscope}%
\begin{pgfscope}%
\definecolor{textcolor}{rgb}{0.000000,0.000000,0.000000}%
\pgfsetstrokecolor{textcolor}%
\pgfsetfillcolor{textcolor}%
\pgftext[x=2.033126in,y=0.402222in,,top]{\color{textcolor}\rmfamily\fontsize{10.000000}{12.000000}\selectfont \(\displaystyle {1.0}\)}%
\end{pgfscope}%
\begin{pgfscope}%
\definecolor{textcolor}{rgb}{0.000000,0.000000,0.000000}%
\pgfsetstrokecolor{textcolor}%
\pgfsetfillcolor{textcolor}%
\pgftext[x=1.328581in,y=0.223333in,,top]{\color{textcolor}\rmfamily\fontsize{10.000000}{12.000000}\selectfont False positive rate}%
\end{pgfscope}%
\begin{pgfscope}%
\pgfsetbuttcap%
\pgfsetroundjoin%
\definecolor{currentfill}{rgb}{0.000000,0.000000,0.000000}%
\pgfsetfillcolor{currentfill}%
\pgfsetlinewidth{0.803000pt}%
\definecolor{currentstroke}{rgb}{0.000000,0.000000,0.000000}%
\pgfsetstrokecolor{currentstroke}%
\pgfsetdash{}{0pt}%
\pgfsys@defobject{currentmarker}{\pgfqpoint{-0.048611in}{0.000000in}}{\pgfqpoint{-0.000000in}{0.000000in}}{%
\pgfpathmoveto{\pgfqpoint{-0.000000in}{0.000000in}}%
\pgfpathlineto{\pgfqpoint{-0.048611in}{0.000000in}}%
\pgfusepath{stroke,fill}%
}%
\begin{pgfscope}%
\pgfsys@transformshift{0.553581in}{0.551944in}%
\pgfsys@useobject{currentmarker}{}%
\end{pgfscope}%
\end{pgfscope}%
\begin{pgfscope}%
\definecolor{textcolor}{rgb}{0.000000,0.000000,0.000000}%
\pgfsetstrokecolor{textcolor}%
\pgfsetfillcolor{textcolor}%
\pgftext[x=0.278889in, y=0.503750in, left, base]{\color{textcolor}\rmfamily\fontsize{10.000000}{12.000000}\selectfont \(\displaystyle {0.0}\)}%
\end{pgfscope}%
\begin{pgfscope}%
\pgfsetbuttcap%
\pgfsetroundjoin%
\definecolor{currentfill}{rgb}{0.000000,0.000000,0.000000}%
\pgfsetfillcolor{currentfill}%
\pgfsetlinewidth{0.803000pt}%
\definecolor{currentstroke}{rgb}{0.000000,0.000000,0.000000}%
\pgfsetstrokecolor{currentstroke}%
\pgfsetdash{}{0pt}%
\pgfsys@defobject{currentmarker}{\pgfqpoint{-0.048611in}{0.000000in}}{\pgfqpoint{-0.000000in}{0.000000in}}{%
\pgfpathmoveto{\pgfqpoint{-0.000000in}{0.000000in}}%
\pgfpathlineto{\pgfqpoint{-0.048611in}{0.000000in}}%
\pgfusepath{stroke,fill}%
}%
\begin{pgfscope}%
\pgfsys@transformshift{0.553581in}{1.076944in}%
\pgfsys@useobject{currentmarker}{}%
\end{pgfscope}%
\end{pgfscope}%
\begin{pgfscope}%
\definecolor{textcolor}{rgb}{0.000000,0.000000,0.000000}%
\pgfsetstrokecolor{textcolor}%
\pgfsetfillcolor{textcolor}%
\pgftext[x=0.278889in, y=1.028750in, left, base]{\color{textcolor}\rmfamily\fontsize{10.000000}{12.000000}\selectfont \(\displaystyle {0.5}\)}%
\end{pgfscope}%
\begin{pgfscope}%
\pgfsetbuttcap%
\pgfsetroundjoin%
\definecolor{currentfill}{rgb}{0.000000,0.000000,0.000000}%
\pgfsetfillcolor{currentfill}%
\pgfsetlinewidth{0.803000pt}%
\definecolor{currentstroke}{rgb}{0.000000,0.000000,0.000000}%
\pgfsetstrokecolor{currentstroke}%
\pgfsetdash{}{0pt}%
\pgfsys@defobject{currentmarker}{\pgfqpoint{-0.048611in}{0.000000in}}{\pgfqpoint{-0.000000in}{0.000000in}}{%
\pgfpathmoveto{\pgfqpoint{-0.000000in}{0.000000in}}%
\pgfpathlineto{\pgfqpoint{-0.048611in}{0.000000in}}%
\pgfusepath{stroke,fill}%
}%
\begin{pgfscope}%
\pgfsys@transformshift{0.553581in}{1.601944in}%
\pgfsys@useobject{currentmarker}{}%
\end{pgfscope}%
\end{pgfscope}%
\begin{pgfscope}%
\definecolor{textcolor}{rgb}{0.000000,0.000000,0.000000}%
\pgfsetstrokecolor{textcolor}%
\pgfsetfillcolor{textcolor}%
\pgftext[x=0.278889in, y=1.553750in, left, base]{\color{textcolor}\rmfamily\fontsize{10.000000}{12.000000}\selectfont \(\displaystyle {1.0}\)}%
\end{pgfscope}%
\begin{pgfscope}%
\definecolor{textcolor}{rgb}{0.000000,0.000000,0.000000}%
\pgfsetstrokecolor{textcolor}%
\pgfsetfillcolor{textcolor}%
\pgftext[x=0.223333in,y=1.076944in,,bottom,rotate=90.000000]{\color{textcolor}\rmfamily\fontsize{10.000000}{12.000000}\selectfont True positive rate}%
\end{pgfscope}%
\begin{pgfscope}%
\pgfpathrectangle{\pgfqpoint{0.553581in}{0.499444in}}{\pgfqpoint{1.550000in}{1.155000in}}%
\pgfusepath{clip}%
\pgfsetbuttcap%
\pgfsetroundjoin%
\pgfsetlinewidth{1.505625pt}%
\definecolor{currentstroke}{rgb}{0.000000,0.000000,0.000000}%
\pgfsetstrokecolor{currentstroke}%
\pgfsetdash{{5.550000pt}{2.400000pt}}{0.000000pt}%
\pgfpathmoveto{\pgfqpoint{0.624035in}{0.551944in}}%
\pgfpathlineto{\pgfqpoint{2.033126in}{1.601944in}}%
\pgfusepath{stroke}%
\end{pgfscope}%
\begin{pgfscope}%
\pgfpathrectangle{\pgfqpoint{0.553581in}{0.499444in}}{\pgfqpoint{1.550000in}{1.155000in}}%
\pgfusepath{clip}%
\pgfsetrectcap%
\pgfsetroundjoin%
\pgfsetlinewidth{1.505625pt}%
\definecolor{currentstroke}{rgb}{0.000000,0.000000,0.000000}%
\pgfsetstrokecolor{currentstroke}%
\pgfsetdash{}{0pt}%
\pgfpathmoveto{\pgfqpoint{0.624035in}{0.551944in}}%
\pgfpathlineto{\pgfqpoint{0.628155in}{0.580037in}}%
\pgfpathlineto{\pgfqpoint{0.628413in}{0.581124in}}%
\pgfpathlineto{\pgfqpoint{0.629515in}{0.587270in}}%
\pgfpathlineto{\pgfqpoint{0.629664in}{0.588201in}}%
\pgfpathlineto{\pgfqpoint{0.630774in}{0.594068in}}%
\pgfpathlineto{\pgfqpoint{0.631001in}{0.595031in}}%
\pgfpathlineto{\pgfqpoint{0.632111in}{0.600587in}}%
\pgfpathlineto{\pgfqpoint{0.632377in}{0.601674in}}%
\pgfpathlineto{\pgfqpoint{0.633471in}{0.607572in}}%
\pgfpathlineto{\pgfqpoint{0.633753in}{0.608627in}}%
\pgfpathlineto{\pgfqpoint{0.633753in}{0.608658in}}%
\pgfpathlineto{\pgfqpoint{0.634863in}{0.613345in}}%
\pgfpathlineto{\pgfqpoint{0.635222in}{0.614432in}}%
\pgfpathlineto{\pgfqpoint{0.636325in}{0.618902in}}%
\pgfpathlineto{\pgfqpoint{0.636504in}{0.619895in}}%
\pgfpathlineto{\pgfqpoint{0.637614in}{0.624024in}}%
\pgfpathlineto{\pgfqpoint{0.637943in}{0.625110in}}%
\pgfpathlineto{\pgfqpoint{0.639045in}{0.630077in}}%
\pgfpathlineto{\pgfqpoint{0.639334in}{0.631164in}}%
\pgfpathlineto{\pgfqpoint{0.640437in}{0.635416in}}%
\pgfpathlineto{\pgfqpoint{0.640749in}{0.636472in}}%
\pgfpathlineto{\pgfqpoint{0.641859in}{0.641314in}}%
\pgfpathlineto{\pgfqpoint{0.642227in}{0.642339in}}%
\pgfpathlineto{\pgfqpoint{0.643321in}{0.647119in}}%
\pgfpathlineto{\pgfqpoint{0.643689in}{0.648206in}}%
\pgfpathlineto{\pgfqpoint{0.644799in}{0.653048in}}%
\pgfpathlineto{\pgfqpoint{0.645080in}{0.654042in}}%
\pgfpathlineto{\pgfqpoint{0.646175in}{0.658357in}}%
\pgfpathlineto{\pgfqpoint{0.646581in}{0.659443in}}%
\pgfpathlineto{\pgfqpoint{0.647691in}{0.664161in}}%
\pgfpathlineto{\pgfqpoint{0.648043in}{0.665217in}}%
\pgfpathlineto{\pgfqpoint{0.649130in}{0.669501in}}%
\pgfpathlineto{\pgfqpoint{0.649146in}{0.669501in}}%
\pgfpathlineto{\pgfqpoint{0.649411in}{0.670494in}}%
\pgfpathlineto{\pgfqpoint{0.650521in}{0.675399in}}%
\pgfpathlineto{\pgfqpoint{0.650779in}{0.676423in}}%
\pgfpathlineto{\pgfqpoint{0.651882in}{0.680552in}}%
\pgfpathlineto{\pgfqpoint{0.652124in}{0.681638in}}%
\pgfpathlineto{\pgfqpoint{0.653226in}{0.685518in}}%
\pgfpathlineto{\pgfqpoint{0.653453in}{0.686481in}}%
\pgfpathlineto{\pgfqpoint{0.653453in}{0.686605in}}%
\pgfpathlineto{\pgfqpoint{0.654547in}{0.690982in}}%
\pgfpathlineto{\pgfqpoint{0.654821in}{0.692068in}}%
\pgfpathlineto{\pgfqpoint{0.655931in}{0.695793in}}%
\pgfpathlineto{\pgfqpoint{0.656439in}{0.696849in}}%
\pgfpathlineto{\pgfqpoint{0.657549in}{0.701195in}}%
\pgfpathlineto{\pgfqpoint{0.657854in}{0.702281in}}%
\pgfpathlineto{\pgfqpoint{0.658957in}{0.705168in}}%
\pgfpathlineto{\pgfqpoint{0.659418in}{0.706255in}}%
\pgfpathlineto{\pgfqpoint{0.660497in}{0.709017in}}%
\pgfpathlineto{\pgfqpoint{0.660911in}{0.710073in}}%
\pgfpathlineto{\pgfqpoint{0.662021in}{0.713984in}}%
\pgfpathlineto{\pgfqpoint{0.662342in}{0.715071in}}%
\pgfpathlineto{\pgfqpoint{0.663436in}{0.718920in}}%
\pgfpathlineto{\pgfqpoint{0.663905in}{0.720006in}}%
\pgfpathlineto{\pgfqpoint{0.665008in}{0.723142in}}%
\pgfpathlineto{\pgfqpoint{0.665437in}{0.724197in}}%
\pgfpathlineto{\pgfqpoint{0.666540in}{0.727705in}}%
\pgfpathlineto{\pgfqpoint{0.667001in}{0.728729in}}%
\pgfpathlineto{\pgfqpoint{0.668111in}{0.732330in}}%
\pgfpathlineto{\pgfqpoint{0.668674in}{0.733385in}}%
\pgfpathlineto{\pgfqpoint{0.669753in}{0.736365in}}%
\pgfpathlineto{\pgfqpoint{0.670144in}{0.737452in}}%
\pgfpathlineto{\pgfqpoint{0.671254in}{0.740525in}}%
\pgfpathlineto{\pgfqpoint{0.675835in}{0.754277in}}%
\pgfpathlineto{\pgfqpoint{0.676187in}{0.755363in}}%
\pgfpathlineto{\pgfqpoint{0.677281in}{0.758778in}}%
\pgfpathlineto{\pgfqpoint{0.677758in}{0.759864in}}%
\pgfpathlineto{\pgfqpoint{0.678813in}{0.763093in}}%
\pgfpathlineto{\pgfqpoint{0.679400in}{0.764179in}}%
\pgfpathlineto{\pgfqpoint{0.680510in}{0.767408in}}%
\pgfpathlineto{\pgfqpoint{0.681112in}{0.768432in}}%
\pgfpathlineto{\pgfqpoint{0.682198in}{0.771629in}}%
\pgfpathlineto{\pgfqpoint{0.682699in}{0.772716in}}%
\pgfpathlineto{\pgfqpoint{0.683754in}{0.775789in}}%
\pgfpathlineto{\pgfqpoint{0.684294in}{0.776875in}}%
\pgfpathlineto{\pgfqpoint{0.685388in}{0.779607in}}%
\pgfpathlineto{\pgfqpoint{0.686107in}{0.780694in}}%
\pgfpathlineto{\pgfqpoint{0.687217in}{0.783922in}}%
\pgfpathlineto{\pgfqpoint{0.687710in}{0.784977in}}%
\pgfpathlineto{\pgfqpoint{0.688804in}{0.787306in}}%
\pgfpathlineto{\pgfqpoint{0.689117in}{0.788392in}}%
\pgfpathlineto{\pgfqpoint{0.690212in}{0.791093in}}%
\pgfpathlineto{\pgfqpoint{0.690728in}{0.792179in}}%
\pgfpathlineto{\pgfqpoint{0.691830in}{0.794942in}}%
\pgfpathlineto{\pgfqpoint{0.692377in}{0.796028in}}%
\pgfpathlineto{\pgfqpoint{0.693448in}{0.799102in}}%
\pgfpathlineto{\pgfqpoint{0.694034in}{0.800188in}}%
\pgfpathlineto{\pgfqpoint{0.695137in}{0.803292in}}%
\pgfpathlineto{\pgfqpoint{0.695668in}{0.804379in}}%
\pgfpathlineto{\pgfqpoint{0.696731in}{0.806459in}}%
\pgfpathlineto{\pgfqpoint{0.697271in}{0.807545in}}%
\pgfpathlineto{\pgfqpoint{0.698365in}{0.810618in}}%
\pgfpathlineto{\pgfqpoint{0.698881in}{0.811705in}}%
\pgfpathlineto{\pgfqpoint{0.699976in}{0.814343in}}%
\pgfpathlineto{\pgfqpoint{0.700531in}{0.815430in}}%
\pgfpathlineto{\pgfqpoint{0.701618in}{0.817323in}}%
\pgfpathlineto{\pgfqpoint{0.702157in}{0.818317in}}%
\pgfpathlineto{\pgfqpoint{0.703251in}{0.820490in}}%
\pgfpathlineto{\pgfqpoint{0.703853in}{0.821545in}}%
\pgfpathlineto{\pgfqpoint{0.704963in}{0.824463in}}%
\pgfpathlineto{\pgfqpoint{0.705362in}{0.825549in}}%
\pgfpathlineto{\pgfqpoint{0.706464in}{0.828033in}}%
\pgfpathlineto{\pgfqpoint{0.707254in}{0.829057in}}%
\pgfpathlineto{\pgfqpoint{0.707254in}{0.829119in}}%
\pgfpathlineto{\pgfqpoint{0.708349in}{0.831758in}}%
\pgfpathlineto{\pgfqpoint{0.708943in}{0.832844in}}%
\pgfpathlineto{\pgfqpoint{0.710045in}{0.835483in}}%
\pgfpathlineto{\pgfqpoint{0.710569in}{0.836569in}}%
\pgfpathlineto{\pgfqpoint{0.711679in}{0.839115in}}%
\pgfpathlineto{\pgfqpoint{0.712218in}{0.840201in}}%
\pgfpathlineto{\pgfqpoint{0.713328in}{0.843306in}}%
\pgfpathlineto{\pgfqpoint{0.713930in}{0.844361in}}%
\pgfpathlineto{\pgfqpoint{0.715025in}{0.847403in}}%
\pgfpathlineto{\pgfqpoint{0.715431in}{0.848490in}}%
\pgfpathlineto{\pgfqpoint{0.716471in}{0.850880in}}%
\pgfpathlineto{\pgfqpoint{0.717034in}{0.851904in}}%
\pgfpathlineto{\pgfqpoint{0.718136in}{0.854108in}}%
\pgfpathlineto{\pgfqpoint{0.718769in}{0.855195in}}%
\pgfpathlineto{\pgfqpoint{0.719880in}{0.857740in}}%
\pgfpathlineto{\pgfqpoint{0.720333in}{0.858827in}}%
\pgfpathlineto{\pgfqpoint{0.721435in}{0.861217in}}%
\pgfpathlineto{\pgfqpoint{0.721990in}{0.862241in}}%
\pgfpathlineto{\pgfqpoint{0.723038in}{0.864135in}}%
\pgfpathlineto{\pgfqpoint{0.723749in}{0.865159in}}%
\pgfpathlineto{\pgfqpoint{0.724859in}{0.867984in}}%
\pgfpathlineto{\pgfqpoint{0.725375in}{0.869071in}}%
\pgfpathlineto{\pgfqpoint{0.726478in}{0.871181in}}%
\pgfpathlineto{\pgfqpoint{0.727080in}{0.872268in}}%
\pgfpathlineto{\pgfqpoint{0.728151in}{0.874844in}}%
\pgfpathlineto{\pgfqpoint{0.728885in}{0.875931in}}%
\pgfpathlineto{\pgfqpoint{0.729910in}{0.877980in}}%
\pgfpathlineto{\pgfqpoint{0.729956in}{0.877980in}}%
\pgfpathlineto{\pgfqpoint{0.730347in}{0.879066in}}%
\pgfpathlineto{\pgfqpoint{0.731364in}{0.881363in}}%
\pgfpathlineto{\pgfqpoint{0.731950in}{0.882450in}}%
\pgfpathlineto{\pgfqpoint{0.733060in}{0.884126in}}%
\pgfpathlineto{\pgfqpoint{0.733412in}{0.885150in}}%
\pgfpathlineto{\pgfqpoint{0.734499in}{0.887292in}}%
\pgfpathlineto{\pgfqpoint{0.735210in}{0.888379in}}%
\pgfpathlineto{\pgfqpoint{0.736312in}{0.890490in}}%
\pgfpathlineto{\pgfqpoint{0.736844in}{0.891514in}}%
\pgfpathlineto{\pgfqpoint{0.737954in}{0.893408in}}%
\pgfpathlineto{\pgfqpoint{0.738556in}{0.894494in}}%
\pgfpathlineto{\pgfqpoint{0.739643in}{0.896543in}}%
\pgfpathlineto{\pgfqpoint{0.740479in}{0.897629in}}%
\pgfpathlineto{\pgfqpoint{0.741589in}{0.900051in}}%
\pgfpathlineto{\pgfqpoint{0.742121in}{0.901137in}}%
\pgfpathlineto{\pgfqpoint{0.743215in}{0.903279in}}%
\pgfpathlineto{\pgfqpoint{0.743809in}{0.904272in}}%
\pgfpathlineto{\pgfqpoint{0.744919in}{0.906632in}}%
\pgfpathlineto{\pgfqpoint{0.745553in}{0.907718in}}%
\pgfpathlineto{\pgfqpoint{0.746647in}{0.909581in}}%
\pgfpathlineto{\pgfqpoint{0.747398in}{0.910667in}}%
\pgfpathlineto{\pgfqpoint{0.748476in}{0.912933in}}%
\pgfpathlineto{\pgfqpoint{0.749180in}{0.913989in}}%
\pgfpathlineto{\pgfqpoint{0.750282in}{0.916348in}}%
\pgfpathlineto{\pgfqpoint{0.750900in}{0.917434in}}%
\pgfpathlineto{\pgfqpoint{0.752010in}{0.919607in}}%
\pgfpathlineto{\pgfqpoint{0.752534in}{0.920694in}}%
\pgfpathlineto{\pgfqpoint{0.753628in}{0.922960in}}%
\pgfpathlineto{\pgfqpoint{0.754574in}{0.924046in}}%
\pgfpathlineto{\pgfqpoint{0.755645in}{0.926436in}}%
\pgfpathlineto{\pgfqpoint{0.756208in}{0.927523in}}%
\pgfpathlineto{\pgfqpoint{0.757279in}{0.929230in}}%
\pgfpathlineto{\pgfqpoint{0.757303in}{0.929230in}}%
\pgfpathlineto{\pgfqpoint{0.757779in}{0.930255in}}%
\pgfpathlineto{\pgfqpoint{0.758882in}{0.931651in}}%
\pgfpathlineto{\pgfqpoint{0.759476in}{0.932645in}}%
\pgfpathlineto{\pgfqpoint{0.760578in}{0.934197in}}%
\pgfpathlineto{\pgfqpoint{0.761329in}{0.935283in}}%
\pgfpathlineto{\pgfqpoint{0.762439in}{0.937146in}}%
\pgfpathlineto{\pgfqpoint{0.763009in}{0.938232in}}%
\pgfpathlineto{\pgfqpoint{0.764096in}{0.939878in}}%
\pgfpathlineto{\pgfqpoint{0.764854in}{0.940964in}}%
\pgfpathlineto{\pgfqpoint{0.765933in}{0.943044in}}%
\pgfpathlineto{\pgfqpoint{0.766512in}{0.944099in}}%
\pgfpathlineto{\pgfqpoint{0.767606in}{0.945589in}}%
\pgfpathlineto{\pgfqpoint{0.768396in}{0.946645in}}%
\pgfpathlineto{\pgfqpoint{0.769490in}{0.948663in}}%
\pgfpathlineto{\pgfqpoint{0.770092in}{0.949749in}}%
\pgfpathlineto{\pgfqpoint{0.771202in}{0.951891in}}%
\pgfpathlineto{\pgfqpoint{0.772000in}{0.952977in}}%
\pgfpathlineto{\pgfqpoint{0.773086in}{0.954654in}}%
\pgfpathlineto{\pgfqpoint{0.773939in}{0.955740in}}%
\pgfpathlineto{\pgfqpoint{0.775049in}{0.957634in}}%
\pgfpathlineto{\pgfqpoint{0.775783in}{0.958720in}}%
\pgfpathlineto{\pgfqpoint{0.776894in}{0.960148in}}%
\pgfpathlineto{\pgfqpoint{0.777636in}{0.961235in}}%
\pgfpathlineto{\pgfqpoint{0.778731in}{0.963377in}}%
\pgfpathlineto{\pgfqpoint{0.779450in}{0.964432in}}%
\pgfpathlineto{\pgfqpoint{0.780552in}{0.966201in}}%
\pgfpathlineto{\pgfqpoint{0.781217in}{0.967288in}}%
\pgfpathlineto{\pgfqpoint{0.782327in}{0.969740in}}%
\pgfpathlineto{\pgfqpoint{0.783202in}{0.970765in}}%
\pgfpathlineto{\pgfqpoint{0.784266in}{0.972813in}}%
\pgfpathlineto{\pgfqpoint{0.784930in}{0.973838in}}%
\pgfpathlineto{\pgfqpoint{0.786040in}{0.975173in}}%
\pgfpathlineto{\pgfqpoint{0.786759in}{0.976259in}}%
\pgfpathlineto{\pgfqpoint{0.787870in}{0.978153in}}%
\pgfpathlineto{\pgfqpoint{0.788612in}{0.979239in}}%
\pgfpathlineto{\pgfqpoint{0.789629in}{0.981040in}}%
\pgfpathlineto{\pgfqpoint{0.790301in}{0.982126in}}%
\pgfpathlineto{\pgfqpoint{0.791411in}{0.983989in}}%
\pgfpathlineto{\pgfqpoint{0.791896in}{0.985075in}}%
\pgfpathlineto{\pgfqpoint{0.792998in}{0.986472in}}%
\pgfpathlineto{\pgfqpoint{0.793670in}{0.987558in}}%
\pgfpathlineto{\pgfqpoint{0.794694in}{0.989297in}}%
\pgfpathlineto{\pgfqpoint{0.794741in}{0.989297in}}%
\pgfpathlineto{\pgfqpoint{0.795726in}{0.990383in}}%
\pgfpathlineto{\pgfqpoint{0.796829in}{0.992525in}}%
\pgfpathlineto{\pgfqpoint{0.797767in}{0.993581in}}%
\pgfpathlineto{\pgfqpoint{0.798861in}{0.995102in}}%
\pgfpathlineto{\pgfqpoint{0.799580in}{0.996188in}}%
\pgfpathlineto{\pgfqpoint{0.800636in}{0.997554in}}%
\pgfpathlineto{\pgfqpoint{0.801371in}{0.998640in}}%
\pgfpathlineto{\pgfqpoint{0.802481in}{1.000379in}}%
\pgfpathlineto{\pgfqpoint{0.803216in}{1.001465in}}%
\pgfpathlineto{\pgfqpoint{0.804326in}{1.002552in}}%
\pgfpathlineto{\pgfqpoint{0.805084in}{1.003607in}}%
\pgfpathlineto{\pgfqpoint{0.806186in}{1.005035in}}%
\pgfpathlineto{\pgfqpoint{0.807015in}{1.006122in}}%
\pgfpathlineto{\pgfqpoint{0.808109in}{1.007767in}}%
\pgfpathlineto{\pgfqpoint{0.808938in}{1.008853in}}%
\pgfpathlineto{\pgfqpoint{0.810040in}{1.010561in}}%
\pgfpathlineto{\pgfqpoint{0.811072in}{1.011647in}}%
\pgfpathlineto{\pgfqpoint{0.812151in}{1.013541in}}%
\pgfpathlineto{\pgfqpoint{0.813089in}{1.014627in}}%
\pgfpathlineto{\pgfqpoint{0.814176in}{1.015869in}}%
\pgfpathlineto{\pgfqpoint{0.815145in}{1.016955in}}%
\pgfpathlineto{\pgfqpoint{0.816248in}{1.018569in}}%
\pgfpathlineto{\pgfqpoint{0.816936in}{1.019625in}}%
\pgfpathlineto{\pgfqpoint{0.818030in}{1.021177in}}%
\pgfpathlineto{\pgfqpoint{0.818781in}{1.022263in}}%
\pgfpathlineto{\pgfqpoint{0.819891in}{1.023722in}}%
\pgfpathlineto{\pgfqpoint{0.820868in}{1.024809in}}%
\pgfpathlineto{\pgfqpoint{0.821962in}{1.026020in}}%
\pgfpathlineto{\pgfqpoint{0.822736in}{1.027075in}}%
\pgfpathlineto{\pgfqpoint{0.823823in}{1.028441in}}%
\pgfpathlineto{\pgfqpoint{0.824902in}{1.029496in}}%
\pgfpathlineto{\pgfqpoint{0.825957in}{1.031110in}}%
\pgfpathlineto{\pgfqpoint{0.826692in}{1.032197in}}%
\pgfpathlineto{\pgfqpoint{0.827763in}{1.033252in}}%
\pgfpathlineto{\pgfqpoint{0.828599in}{1.034308in}}%
\pgfpathlineto{\pgfqpoint{0.829710in}{1.035643in}}%
\pgfpathlineto{\pgfqpoint{0.830601in}{1.036698in}}%
\pgfpathlineto{\pgfqpoint{0.831695in}{1.038561in}}%
\pgfpathlineto{\pgfqpoint{0.832837in}{1.039647in}}%
\pgfpathlineto{\pgfqpoint{0.833947in}{1.040982in}}%
\pgfpathlineto{\pgfqpoint{0.834721in}{1.042037in}}%
\pgfpathlineto{\pgfqpoint{0.835792in}{1.043589in}}%
\pgfpathlineto{\pgfqpoint{0.836605in}{1.044645in}}%
\pgfpathlineto{\pgfqpoint{0.837676in}{1.045918in}}%
\pgfpathlineto{\pgfqpoint{0.838786in}{1.047004in}}%
\pgfpathlineto{\pgfqpoint{0.839880in}{1.048370in}}%
\pgfpathlineto{\pgfqpoint{0.840623in}{1.049456in}}%
\pgfpathlineto{\pgfqpoint{0.841733in}{1.050915in}}%
\pgfpathlineto{\pgfqpoint{0.842742in}{1.052002in}}%
\pgfpathlineto{\pgfqpoint{0.843836in}{1.053212in}}%
\pgfpathlineto{\pgfqpoint{0.844837in}{1.054268in}}%
\pgfpathlineto{\pgfqpoint{0.845900in}{1.055199in}}%
\pgfpathlineto{\pgfqpoint{0.847104in}{1.056286in}}%
\pgfpathlineto{\pgfqpoint{0.848206in}{1.057372in}}%
\pgfpathlineto{\pgfqpoint{0.849504in}{1.058459in}}%
\pgfpathlineto{\pgfqpoint{0.850614in}{1.060228in}}%
\pgfpathlineto{\pgfqpoint{0.851787in}{1.061190in}}%
\pgfpathlineto{\pgfqpoint{0.851787in}{1.061283in}}%
\pgfpathlineto{\pgfqpoint{0.852881in}{1.062680in}}%
\pgfpathlineto{\pgfqpoint{0.854007in}{1.063767in}}%
\pgfpathlineto{\pgfqpoint{0.855101in}{1.065195in}}%
\pgfpathlineto{\pgfqpoint{0.855805in}{1.066281in}}%
\pgfpathlineto{\pgfqpoint{0.856907in}{1.067523in}}%
\pgfpathlineto{\pgfqpoint{0.856915in}{1.067523in}}%
\pgfpathlineto{\pgfqpoint{0.857642in}{1.068609in}}%
\pgfpathlineto{\pgfqpoint{0.858713in}{1.069851in}}%
\pgfpathlineto{\pgfqpoint{0.859745in}{1.070938in}}%
\pgfpathlineto{\pgfqpoint{0.860855in}{1.072179in}}%
\pgfpathlineto{\pgfqpoint{0.861410in}{1.073204in}}%
\pgfpathlineto{\pgfqpoint{0.862505in}{1.074600in}}%
\pgfpathlineto{\pgfqpoint{0.863662in}{1.075687in}}%
\pgfpathlineto{\pgfqpoint{0.864756in}{1.077115in}}%
\pgfpathlineto{\pgfqpoint{0.865522in}{1.078201in}}%
\pgfpathlineto{\pgfqpoint{0.866624in}{1.079536in}}%
\pgfpathlineto{\pgfqpoint{0.867907in}{1.080592in}}%
\pgfpathlineto{\pgfqpoint{0.869017in}{1.082268in}}%
\pgfpathlineto{\pgfqpoint{0.870002in}{1.083354in}}%
\pgfpathlineto{\pgfqpoint{0.871096in}{1.084689in}}%
\pgfpathlineto{\pgfqpoint{0.872425in}{1.085776in}}%
\pgfpathlineto{\pgfqpoint{0.873535in}{1.086955in}}%
\pgfpathlineto{\pgfqpoint{0.874098in}{1.088042in}}%
\pgfpathlineto{\pgfqpoint{0.875193in}{1.089345in}}%
\pgfpathlineto{\pgfqpoint{0.876193in}{1.090432in}}%
\pgfpathlineto{\pgfqpoint{0.877288in}{1.091953in}}%
\pgfpathlineto{\pgfqpoint{0.878085in}{1.093040in}}%
\pgfpathlineto{\pgfqpoint{0.879172in}{1.094654in}}%
\pgfpathlineto{\pgfqpoint{0.879946in}{1.095740in}}%
\pgfpathlineto{\pgfqpoint{0.880993in}{1.097075in}}%
\pgfpathlineto{\pgfqpoint{0.882307in}{1.098130in}}%
\pgfpathlineto{\pgfqpoint{0.883417in}{1.099403in}}%
\pgfpathlineto{\pgfqpoint{0.884300in}{1.100490in}}%
\pgfpathlineto{\pgfqpoint{0.885356in}{1.101731in}}%
\pgfpathlineto{\pgfqpoint{0.886630in}{1.102818in}}%
\pgfpathlineto{\pgfqpoint{0.887740in}{1.103904in}}%
\pgfpathlineto{\pgfqpoint{0.888748in}{1.104991in}}%
\pgfpathlineto{\pgfqpoint{0.889812in}{1.106326in}}%
\pgfpathlineto{\pgfqpoint{0.889843in}{1.106326in}}%
\pgfpathlineto{\pgfqpoint{0.891086in}{1.107412in}}%
\pgfpathlineto{\pgfqpoint{0.892165in}{1.108871in}}%
\pgfpathlineto{\pgfqpoint{0.893158in}{1.109957in}}%
\pgfpathlineto{\pgfqpoint{0.894260in}{1.111447in}}%
\pgfpathlineto{\pgfqpoint{0.895253in}{1.112503in}}%
\pgfpathlineto{\pgfqpoint{0.896324in}{1.113651in}}%
\pgfpathlineto{\pgfqpoint{0.897207in}{1.114707in}}%
\pgfpathlineto{\pgfqpoint{0.898302in}{1.116321in}}%
\pgfpathlineto{\pgfqpoint{0.899560in}{1.117377in}}%
\pgfpathlineto{\pgfqpoint{0.900670in}{1.118370in}}%
\pgfpathlineto{\pgfqpoint{0.901702in}{1.119456in}}%
\pgfpathlineto{\pgfqpoint{0.902797in}{1.120481in}}%
\pgfpathlineto{\pgfqpoint{0.903774in}{1.121567in}}%
\pgfpathlineto{\pgfqpoint{0.904829in}{1.122561in}}%
\pgfpathlineto{\pgfqpoint{0.905744in}{1.123616in}}%
\pgfpathlineto{\pgfqpoint{0.906776in}{1.124889in}}%
\pgfpathlineto{\pgfqpoint{0.908089in}{1.125975in}}%
\pgfpathlineto{\pgfqpoint{0.909199in}{1.127341in}}%
\pgfpathlineto{\pgfqpoint{0.910349in}{1.128428in}}%
\pgfpathlineto{\pgfqpoint{0.911451in}{1.129700in}}%
\pgfpathlineto{\pgfqpoint{0.912467in}{1.130787in}}%
\pgfpathlineto{\pgfqpoint{0.913507in}{1.131687in}}%
\pgfpathlineto{\pgfqpoint{0.913538in}{1.131687in}}%
\pgfpathlineto{\pgfqpoint{0.914468in}{1.132773in}}%
\pgfpathlineto{\pgfqpoint{0.915579in}{1.134170in}}%
\pgfpathlineto{\pgfqpoint{0.916861in}{1.135257in}}%
\pgfpathlineto{\pgfqpoint{0.917971in}{1.136623in}}%
\pgfpathlineto{\pgfqpoint{0.918995in}{1.137709in}}%
\pgfpathlineto{\pgfqpoint{0.920058in}{1.139168in}}%
\pgfpathlineto{\pgfqpoint{0.921622in}{1.140255in}}%
\pgfpathlineto{\pgfqpoint{0.922716in}{1.141434in}}%
\pgfpathlineto{\pgfqpoint{0.924108in}{1.142521in}}%
\pgfpathlineto{\pgfqpoint{0.925179in}{1.143855in}}%
\pgfpathlineto{\pgfqpoint{0.926218in}{1.144911in}}%
\pgfpathlineto{\pgfqpoint{0.927305in}{1.146122in}}%
\pgfpathlineto{\pgfqpoint{0.928548in}{1.147208in}}%
\pgfpathlineto{\pgfqpoint{0.929650in}{1.148294in}}%
\pgfpathlineto{\pgfqpoint{0.930925in}{1.149319in}}%
\pgfpathlineto{\pgfqpoint{0.932003in}{1.150467in}}%
\pgfpathlineto{\pgfqpoint{0.933880in}{1.151554in}}%
\pgfpathlineto{\pgfqpoint{0.934959in}{1.152796in}}%
\pgfpathlineto{\pgfqpoint{0.935842in}{1.153882in}}%
\pgfpathlineto{\pgfqpoint{0.936803in}{1.155217in}}%
\pgfpathlineto{\pgfqpoint{0.938273in}{1.156303in}}%
\pgfpathlineto{\pgfqpoint{0.939368in}{1.157390in}}%
\pgfpathlineto{\pgfqpoint{0.940572in}{1.158414in}}%
\pgfpathlineto{\pgfqpoint{0.941682in}{1.159780in}}%
\pgfpathlineto{\pgfqpoint{0.942893in}{1.160867in}}%
\pgfpathlineto{\pgfqpoint{0.943980in}{1.162046in}}%
\pgfpathlineto{\pgfqpoint{0.945387in}{1.163133in}}%
\pgfpathlineto{\pgfqpoint{0.946497in}{1.164281in}}%
\pgfpathlineto{\pgfqpoint{0.948170in}{1.165368in}}%
\pgfpathlineto{\pgfqpoint{0.949249in}{1.165957in}}%
\pgfpathlineto{\pgfqpoint{0.949280in}{1.165957in}}%
\pgfpathlineto{\pgfqpoint{0.950015in}{1.167044in}}%
\pgfpathlineto{\pgfqpoint{0.951008in}{1.167882in}}%
\pgfpathlineto{\pgfqpoint{0.952368in}{1.168969in}}%
\pgfpathlineto{\pgfqpoint{0.953416in}{1.170024in}}%
\pgfpathlineto{\pgfqpoint{0.954651in}{1.171110in}}%
\pgfpathlineto{\pgfqpoint{0.955753in}{1.172725in}}%
\pgfpathlineto{\pgfqpoint{0.957247in}{1.173811in}}%
\pgfpathlineto{\pgfqpoint{0.958310in}{1.175146in}}%
\pgfpathlineto{\pgfqpoint{0.959811in}{1.176201in}}%
\pgfpathlineto{\pgfqpoint{0.960897in}{1.177567in}}%
\pgfpathlineto{\pgfqpoint{0.961976in}{1.178654in}}%
\pgfpathlineto{\pgfqpoint{0.963063in}{1.179740in}}%
\pgfpathlineto{\pgfqpoint{0.964408in}{1.180827in}}%
\pgfpathlineto{\pgfqpoint{0.965486in}{1.181913in}}%
\pgfpathlineto{\pgfqpoint{0.966385in}{1.183000in}}%
\pgfpathlineto{\pgfqpoint{0.967464in}{1.184241in}}%
\pgfpathlineto{\pgfqpoint{0.968879in}{1.185297in}}%
\pgfpathlineto{\pgfqpoint{0.969989in}{1.186476in}}%
\pgfpathlineto{\pgfqpoint{0.971600in}{1.187563in}}%
\pgfpathlineto{\pgfqpoint{0.972632in}{1.188463in}}%
\pgfpathlineto{\pgfqpoint{0.972655in}{1.188463in}}%
\pgfpathlineto{\pgfqpoint{0.973812in}{1.189549in}}%
\pgfpathlineto{\pgfqpoint{0.974914in}{1.190853in}}%
\pgfpathlineto{\pgfqpoint{0.975884in}{1.191909in}}%
\pgfpathlineto{\pgfqpoint{0.976971in}{1.192778in}}%
\pgfpathlineto{\pgfqpoint{0.978386in}{1.193833in}}%
\pgfpathlineto{\pgfqpoint{0.979433in}{1.194827in}}%
\pgfpathlineto{\pgfqpoint{0.980457in}{1.195913in}}%
\pgfpathlineto{\pgfqpoint{0.981442in}{1.196813in}}%
\pgfpathlineto{\pgfqpoint{0.982888in}{1.197900in}}%
\pgfpathlineto{\pgfqpoint{0.983952in}{1.198893in}}%
\pgfpathlineto{\pgfqpoint{0.985750in}{1.199949in}}%
\pgfpathlineto{\pgfqpoint{0.986836in}{1.201190in}}%
\pgfpathlineto{\pgfqpoint{0.988017in}{1.202277in}}%
\pgfpathlineto{\pgfqpoint{0.989127in}{1.203612in}}%
\pgfpathlineto{\pgfqpoint{0.990268in}{1.204636in}}%
\pgfpathlineto{\pgfqpoint{0.991371in}{1.205785in}}%
\pgfpathlineto{\pgfqpoint{0.992614in}{1.206871in}}%
\pgfpathlineto{\pgfqpoint{0.993661in}{1.207802in}}%
\pgfpathlineto{\pgfqpoint{0.994482in}{1.208889in}}%
\pgfpathlineto{\pgfqpoint{0.995530in}{1.209727in}}%
\pgfpathlineto{\pgfqpoint{0.996851in}{1.210813in}}%
\pgfpathlineto{\pgfqpoint{0.997961in}{1.211838in}}%
\pgfpathlineto{\pgfqpoint{0.999079in}{1.212924in}}%
\pgfpathlineto{\pgfqpoint{1.000189in}{1.214445in}}%
\pgfpathlineto{\pgfqpoint{1.001463in}{1.215501in}}%
\pgfpathlineto{\pgfqpoint{1.002573in}{1.216494in}}%
\pgfpathlineto{\pgfqpoint{1.004082in}{1.217581in}}%
\pgfpathlineto{\pgfqpoint{1.005153in}{1.218481in}}%
\pgfpathlineto{\pgfqpoint{1.006474in}{1.219567in}}%
\pgfpathlineto{\pgfqpoint{1.007545in}{1.220902in}}%
\pgfpathlineto{\pgfqpoint{1.008890in}{1.221989in}}%
\pgfpathlineto{\pgfqpoint{1.009953in}{1.223044in}}%
\pgfpathlineto{\pgfqpoint{1.009992in}{1.223044in}}%
\pgfpathlineto{\pgfqpoint{1.011095in}{1.224130in}}%
\pgfpathlineto{\pgfqpoint{1.012205in}{1.224782in}}%
\pgfpathlineto{\pgfqpoint{1.013463in}{1.225869in}}%
\pgfpathlineto{\pgfqpoint{1.014472in}{1.226893in}}%
\pgfpathlineto{\pgfqpoint{1.016137in}{1.227980in}}%
\pgfpathlineto{\pgfqpoint{1.017177in}{1.228911in}}%
\pgfpathlineto{\pgfqpoint{1.018678in}{1.229997in}}%
\pgfpathlineto{\pgfqpoint{1.019756in}{1.230742in}}%
\pgfpathlineto{\pgfqpoint{1.020953in}{1.231829in}}%
\pgfpathlineto{\pgfqpoint{1.022039in}{1.232977in}}%
\pgfpathlineto{\pgfqpoint{1.023173in}{1.234064in}}%
\pgfpathlineto{\pgfqpoint{1.024283in}{1.235150in}}%
\pgfpathlineto{\pgfqpoint{1.025518in}{1.236237in}}%
\pgfpathlineto{\pgfqpoint{1.026613in}{1.237075in}}%
\pgfpathlineto{\pgfqpoint{1.028043in}{1.238161in}}%
\pgfpathlineto{\pgfqpoint{1.029106in}{1.239403in}}%
\pgfpathlineto{\pgfqpoint{1.030490in}{1.240490in}}%
\pgfpathlineto{\pgfqpoint{1.031428in}{1.241204in}}%
\pgfpathlineto{\pgfqpoint{1.031452in}{1.241204in}}%
\pgfpathlineto{\pgfqpoint{1.032718in}{1.242290in}}%
\pgfpathlineto{\pgfqpoint{1.033789in}{1.243159in}}%
\pgfpathlineto{\pgfqpoint{1.035384in}{1.244246in}}%
\pgfpathlineto{\pgfqpoint{1.036494in}{1.245177in}}%
\pgfpathlineto{\pgfqpoint{1.037682in}{1.246263in}}%
\pgfpathlineto{\pgfqpoint{1.038769in}{1.247474in}}%
\pgfpathlineto{\pgfqpoint{1.039973in}{1.248561in}}%
\pgfpathlineto{\pgfqpoint{1.041044in}{1.249244in}}%
\pgfpathlineto{\pgfqpoint{1.042482in}{1.250330in}}%
\pgfpathlineto{\pgfqpoint{1.043592in}{1.251354in}}%
\pgfpathlineto{\pgfqpoint{1.045133in}{1.252348in}}%
\pgfpathlineto{\pgfqpoint{1.045133in}{1.252410in}}%
\pgfpathlineto{\pgfqpoint{1.046227in}{1.253279in}}%
\pgfpathlineto{\pgfqpoint{1.048009in}{1.254365in}}%
\pgfpathlineto{\pgfqpoint{1.049104in}{1.255328in}}%
\pgfpathlineto{\pgfqpoint{1.050128in}{1.256414in}}%
\pgfpathlineto{\pgfqpoint{1.051222in}{1.257470in}}%
\pgfpathlineto{\pgfqpoint{1.052903in}{1.258556in}}%
\pgfpathlineto{\pgfqpoint{1.053990in}{1.259643in}}%
\pgfpathlineto{\pgfqpoint{1.055295in}{1.260729in}}%
\pgfpathlineto{\pgfqpoint{1.056398in}{1.261660in}}%
\pgfpathlineto{\pgfqpoint{1.057946in}{1.262747in}}%
\pgfpathlineto{\pgfqpoint{1.058970in}{1.263740in}}%
\pgfpathlineto{\pgfqpoint{1.060869in}{1.264827in}}%
\pgfpathlineto{\pgfqpoint{1.061901in}{1.265416in}}%
\pgfpathlineto{\pgfqpoint{1.063395in}{1.266503in}}%
\pgfpathlineto{\pgfqpoint{1.064473in}{1.267403in}}%
\pgfpathlineto{\pgfqpoint{1.065818in}{1.268459in}}%
\pgfpathlineto{\pgfqpoint{1.066811in}{1.269110in}}%
\pgfpathlineto{\pgfqpoint{1.068077in}{1.270197in}}%
\pgfpathlineto{\pgfqpoint{1.069117in}{1.271035in}}%
\pgfpathlineto{\pgfqpoint{1.069187in}{1.271035in}}%
\pgfpathlineto{\pgfqpoint{1.070595in}{1.272122in}}%
\pgfpathlineto{\pgfqpoint{1.071681in}{1.273084in}}%
\pgfpathlineto{\pgfqpoint{1.073456in}{1.274170in}}%
\pgfpathlineto{\pgfqpoint{1.074527in}{1.274915in}}%
\pgfpathlineto{\pgfqpoint{1.075887in}{1.275971in}}%
\pgfpathlineto{\pgfqpoint{1.076794in}{1.276467in}}%
\pgfpathlineto{\pgfqpoint{1.078623in}{1.277554in}}%
\pgfpathlineto{\pgfqpoint{1.079710in}{1.278423in}}%
\pgfpathlineto{\pgfqpoint{1.081625in}{1.279510in}}%
\pgfpathlineto{\pgfqpoint{1.082665in}{1.280534in}}%
\pgfpathlineto{\pgfqpoint{1.084690in}{1.281620in}}%
\pgfpathlineto{\pgfqpoint{1.085745in}{1.282272in}}%
\pgfpathlineto{\pgfqpoint{1.087285in}{1.283359in}}%
\pgfpathlineto{\pgfqpoint{1.088270in}{1.284011in}}%
\pgfpathlineto{\pgfqpoint{1.089849in}{1.285066in}}%
\pgfpathlineto{\pgfqpoint{1.090897in}{1.286059in}}%
\pgfpathlineto{\pgfqpoint{1.092101in}{1.287146in}}%
\pgfpathlineto{\pgfqpoint{1.093195in}{1.288077in}}%
\pgfpathlineto{\pgfqpoint{1.094814in}{1.289164in}}%
\pgfpathlineto{\pgfqpoint{1.095916in}{1.290188in}}%
\pgfpathlineto{\pgfqpoint{1.097362in}{1.291275in}}%
\pgfpathlineto{\pgfqpoint{1.098371in}{1.291895in}}%
\pgfpathlineto{\pgfqpoint{1.099809in}{1.292951in}}%
\pgfpathlineto{\pgfqpoint{1.100919in}{1.293727in}}%
\pgfpathlineto{\pgfqpoint{1.102154in}{1.294813in}}%
\pgfpathlineto{\pgfqpoint{1.103155in}{1.295527in}}%
\pgfpathlineto{\pgfqpoint{1.104977in}{1.296614in}}%
\pgfpathlineto{\pgfqpoint{1.106055in}{1.297390in}}%
\pgfpathlineto{\pgfqpoint{1.107291in}{1.298476in}}%
\pgfpathlineto{\pgfqpoint{1.108401in}{1.299252in}}%
\pgfpathlineto{\pgfqpoint{1.110050in}{1.300339in}}%
\pgfpathlineto{\pgfqpoint{1.111090in}{1.301146in}}%
\pgfpathlineto{\pgfqpoint{1.112661in}{1.302201in}}%
\pgfpathlineto{\pgfqpoint{1.113771in}{1.303288in}}%
\pgfpathlineto{\pgfqpoint{1.115038in}{1.304374in}}%
\pgfpathlineto{\pgfqpoint{1.116148in}{1.304995in}}%
\pgfpathlineto{\pgfqpoint{1.117876in}{1.306082in}}%
\pgfpathlineto{\pgfqpoint{1.118947in}{1.306547in}}%
\pgfpathlineto{\pgfqpoint{1.120682in}{1.307634in}}%
\pgfpathlineto{\pgfqpoint{1.121769in}{1.308379in}}%
\pgfpathlineto{\pgfqpoint{1.123536in}{1.309465in}}%
\pgfpathlineto{\pgfqpoint{1.124591in}{1.310210in}}%
\pgfpathlineto{\pgfqpoint{1.125998in}{1.311297in}}%
\pgfpathlineto{\pgfqpoint{1.127108in}{1.312352in}}%
\pgfpathlineto{\pgfqpoint{1.128547in}{1.313439in}}%
\pgfpathlineto{\pgfqpoint{1.129602in}{1.314028in}}%
\pgfpathlineto{\pgfqpoint{1.131197in}{1.315084in}}%
\pgfpathlineto{\pgfqpoint{1.131197in}{1.315115in}}%
\pgfpathlineto{\pgfqpoint{1.132291in}{1.316046in}}%
\pgfpathlineto{\pgfqpoint{1.134074in}{1.317133in}}%
\pgfpathlineto{\pgfqpoint{1.135161in}{1.317567in}}%
\pgfpathlineto{\pgfqpoint{1.137005in}{1.318654in}}%
\pgfpathlineto{\pgfqpoint{1.137990in}{1.319461in}}%
\pgfpathlineto{\pgfqpoint{1.140258in}{1.320547in}}%
\pgfpathlineto{\pgfqpoint{1.141250in}{1.321106in}}%
\pgfpathlineto{\pgfqpoint{1.143127in}{1.322193in}}%
\pgfpathlineto{\pgfqpoint{1.144221in}{1.322938in}}%
\pgfpathlineto{\pgfqpoint{1.146207in}{1.324024in}}%
\pgfpathlineto{\pgfqpoint{1.147254in}{1.324986in}}%
\pgfpathlineto{\pgfqpoint{1.149076in}{1.326073in}}%
\pgfpathlineto{\pgfqpoint{1.150178in}{1.326694in}}%
\pgfpathlineto{\pgfqpoint{1.151593in}{1.327780in}}%
\pgfpathlineto{\pgfqpoint{1.152594in}{1.328711in}}%
\pgfpathlineto{\pgfqpoint{1.154650in}{1.329798in}}%
\pgfpathlineto{\pgfqpoint{1.155760in}{1.330481in}}%
\pgfpathlineto{\pgfqpoint{1.157542in}{1.331567in}}%
\pgfpathlineto{\pgfqpoint{1.158621in}{1.332281in}}%
\pgfpathlineto{\pgfqpoint{1.160607in}{1.333368in}}%
\pgfpathlineto{\pgfqpoint{1.161670in}{1.334051in}}%
\pgfpathlineto{\pgfqpoint{1.163140in}{1.335137in}}%
\pgfpathlineto{\pgfqpoint{1.164234in}{1.335882in}}%
\pgfpathlineto{\pgfqpoint{1.165524in}{1.336969in}}%
\pgfpathlineto{\pgfqpoint{1.166634in}{1.337962in}}%
\pgfpathlineto{\pgfqpoint{1.168417in}{1.339017in}}%
\pgfpathlineto{\pgfqpoint{1.169511in}{1.339824in}}%
\pgfpathlineto{\pgfqpoint{1.172193in}{1.340911in}}%
\pgfpathlineto{\pgfqpoint{1.173154in}{1.341718in}}%
\pgfpathlineto{\pgfqpoint{1.175070in}{1.342773in}}%
\pgfpathlineto{\pgfqpoint{1.176062in}{1.343705in}}%
\pgfpathlineto{\pgfqpoint{1.177665in}{1.344791in}}%
\pgfpathlineto{\pgfqpoint{1.178736in}{1.345474in}}%
\pgfpathlineto{\pgfqpoint{1.180354in}{1.346530in}}%
\pgfpathlineto{\pgfqpoint{1.181308in}{1.347492in}}%
\pgfpathlineto{\pgfqpoint{1.182801in}{1.348578in}}%
\pgfpathlineto{\pgfqpoint{1.183708in}{1.349044in}}%
\pgfpathlineto{\pgfqpoint{1.185365in}{1.350130in}}%
\pgfpathlineto{\pgfqpoint{1.186421in}{1.351000in}}%
\pgfpathlineto{\pgfqpoint{1.187851in}{1.352086in}}%
\pgfpathlineto{\pgfqpoint{1.188860in}{1.352893in}}%
\pgfpathlineto{\pgfqpoint{1.188899in}{1.352893in}}%
\pgfpathlineto{\pgfqpoint{1.190416in}{1.353980in}}%
\pgfpathlineto{\pgfqpoint{1.191315in}{1.354507in}}%
\pgfpathlineto{\pgfqpoint{1.191494in}{1.354507in}}%
\pgfpathlineto{\pgfqpoint{1.192917in}{1.355594in}}%
\pgfpathlineto{\pgfqpoint{1.194020in}{1.356339in}}%
\pgfpathlineto{\pgfqpoint{1.195646in}{1.357425in}}%
\pgfpathlineto{\pgfqpoint{1.196724in}{1.358046in}}%
\pgfpathlineto{\pgfqpoint{1.198710in}{1.359133in}}%
\pgfpathlineto{\pgfqpoint{1.199765in}{1.359847in}}%
\pgfpathlineto{\pgfqpoint{1.201822in}{1.360933in}}%
\pgfpathlineto{\pgfqpoint{1.202861in}{1.361647in}}%
\pgfpathlineto{\pgfqpoint{1.204222in}{1.362734in}}%
\pgfpathlineto{\pgfqpoint{1.205277in}{1.363354in}}%
\pgfpathlineto{\pgfqpoint{1.206934in}{1.364441in}}%
\pgfpathlineto{\pgfqpoint{1.208005in}{1.365000in}}%
\pgfpathlineto{\pgfqpoint{1.209960in}{1.366086in}}%
\pgfpathlineto{\pgfqpoint{1.211039in}{1.366831in}}%
\pgfpathlineto{\pgfqpoint{1.213423in}{1.367918in}}%
\pgfpathlineto{\pgfqpoint{1.214510in}{1.368663in}}%
\pgfpathlineto{\pgfqpoint{1.215885in}{1.369749in}}%
\pgfpathlineto{\pgfqpoint{1.216988in}{1.370494in}}%
\pgfpathlineto{\pgfqpoint{1.218856in}{1.371581in}}%
\pgfpathlineto{\pgfqpoint{1.219833in}{1.372263in}}%
\pgfpathlineto{\pgfqpoint{1.219919in}{1.372263in}}%
\pgfpathlineto{\pgfqpoint{1.221577in}{1.373319in}}%
\pgfpathlineto{\pgfqpoint{1.222640in}{1.373816in}}%
\pgfpathlineto{\pgfqpoint{1.224633in}{1.374902in}}%
\pgfpathlineto{\pgfqpoint{1.225728in}{1.375740in}}%
\pgfpathlineto{\pgfqpoint{1.227159in}{1.376827in}}%
\pgfpathlineto{\pgfqpoint{1.228245in}{1.377727in}}%
\pgfpathlineto{\pgfqpoint{1.230372in}{1.378813in}}%
\pgfpathlineto{\pgfqpoint{1.231482in}{1.379496in}}%
\pgfpathlineto{\pgfqpoint{1.232865in}{1.380583in}}%
\pgfpathlineto{\pgfqpoint{1.233952in}{1.381079in}}%
\pgfpathlineto{\pgfqpoint{1.236759in}{1.382104in}}%
\pgfpathlineto{\pgfqpoint{1.237845in}{1.382756in}}%
\pgfpathlineto{\pgfqpoint{1.239745in}{1.383842in}}%
\pgfpathlineto{\pgfqpoint{1.240808in}{1.384432in}}%
\pgfpathlineto{\pgfqpoint{1.240839in}{1.384432in}}%
\pgfpathlineto{\pgfqpoint{1.242411in}{1.385518in}}%
\pgfpathlineto{\pgfqpoint{1.243388in}{1.385953in}}%
\pgfpathlineto{\pgfqpoint{1.245741in}{1.387008in}}%
\pgfpathlineto{\pgfqpoint{1.246851in}{1.387816in}}%
\pgfpathlineto{\pgfqpoint{1.248923in}{1.388902in}}%
\pgfpathlineto{\pgfqpoint{1.249931in}{1.389368in}}%
\pgfpathlineto{\pgfqpoint{1.251909in}{1.390454in}}%
\pgfpathlineto{\pgfqpoint{1.253011in}{1.390951in}}%
\pgfpathlineto{\pgfqpoint{1.255239in}{1.392037in}}%
\pgfpathlineto{\pgfqpoint{1.256303in}{1.392596in}}%
\pgfpathlineto{\pgfqpoint{1.258515in}{1.393683in}}%
\pgfpathlineto{\pgfqpoint{1.259492in}{1.394490in}}%
\pgfpathlineto{\pgfqpoint{1.261705in}{1.395576in}}%
\pgfpathlineto{\pgfqpoint{1.262713in}{1.396135in}}%
\pgfpathlineto{\pgfqpoint{1.264191in}{1.397221in}}%
\pgfpathlineto{\pgfqpoint{1.265230in}{1.397780in}}%
\pgfpathlineto{\pgfqpoint{1.267685in}{1.398867in}}%
\pgfpathlineto{\pgfqpoint{1.268764in}{1.399518in}}%
\pgfpathlineto{\pgfqpoint{1.270328in}{1.400605in}}%
\pgfpathlineto{\pgfqpoint{1.271422in}{1.401319in}}%
\pgfpathlineto{\pgfqpoint{1.274119in}{1.402405in}}%
\pgfpathlineto{\pgfqpoint{1.275190in}{1.403150in}}%
\pgfpathlineto{\pgfqpoint{1.278106in}{1.404237in}}%
\pgfpathlineto{\pgfqpoint{1.279216in}{1.404889in}}%
\pgfpathlineto{\pgfqpoint{1.281397in}{1.405975in}}%
\pgfpathlineto{\pgfqpoint{1.282507in}{1.406534in}}%
\pgfpathlineto{\pgfqpoint{1.284947in}{1.407620in}}%
\pgfpathlineto{\pgfqpoint{1.285994in}{1.408396in}}%
\pgfpathlineto{\pgfqpoint{1.286041in}{1.408396in}}%
\pgfpathlineto{\pgfqpoint{1.288629in}{1.409483in}}%
\pgfpathlineto{\pgfqpoint{1.289614in}{1.410011in}}%
\pgfpathlineto{\pgfqpoint{1.289645in}{1.410011in}}%
\pgfpathlineto{\pgfqpoint{1.292076in}{1.411097in}}%
\pgfpathlineto{\pgfqpoint{1.293147in}{1.411718in}}%
\pgfpathlineto{\pgfqpoint{1.293186in}{1.411718in}}%
\pgfpathlineto{\pgfqpoint{1.295195in}{1.412804in}}%
\pgfpathlineto{\pgfqpoint{1.296462in}{1.413612in}}%
\pgfpathlineto{\pgfqpoint{1.298526in}{1.414667in}}%
\pgfpathlineto{\pgfqpoint{1.299636in}{1.415443in}}%
\pgfpathlineto{\pgfqpoint{1.300871in}{1.416530in}}%
\pgfpathlineto{\pgfqpoint{1.301958in}{1.416840in}}%
\pgfpathlineto{\pgfqpoint{1.303896in}{1.417864in}}%
\pgfpathlineto{\pgfqpoint{1.304968in}{1.418485in}}%
\pgfpathlineto{\pgfqpoint{1.306859in}{1.419572in}}%
\pgfpathlineto{\pgfqpoint{1.307907in}{1.420099in}}%
\pgfpathlineto{\pgfqpoint{1.310096in}{1.421186in}}%
\pgfpathlineto{\pgfqpoint{1.311143in}{1.421838in}}%
\pgfpathlineto{\pgfqpoint{1.314286in}{1.422924in}}%
\pgfpathlineto{\pgfqpoint{1.315381in}{1.423545in}}%
\pgfpathlineto{\pgfqpoint{1.318000in}{1.424600in}}%
\pgfpathlineto{\pgfqpoint{1.319071in}{1.425097in}}%
\pgfpathlineto{\pgfqpoint{1.320650in}{1.426153in}}%
\pgfpathlineto{\pgfqpoint{1.321643in}{1.426898in}}%
\pgfpathlineto{\pgfqpoint{1.324418in}{1.427984in}}%
\pgfpathlineto{\pgfqpoint{1.325528in}{1.428450in}}%
\pgfpathlineto{\pgfqpoint{1.327482in}{1.429536in}}%
\pgfpathlineto{\pgfqpoint{1.328452in}{1.430250in}}%
\pgfpathlineto{\pgfqpoint{1.331039in}{1.431337in}}%
\pgfpathlineto{\pgfqpoint{1.332134in}{1.431926in}}%
\pgfpathlineto{\pgfqpoint{1.334033in}{1.433013in}}%
\pgfpathlineto{\pgfqpoint{1.335026in}{1.433603in}}%
\pgfpathlineto{\pgfqpoint{1.337317in}{1.434689in}}%
\pgfpathlineto{\pgfqpoint{1.338052in}{1.435279in}}%
\pgfpathlineto{\pgfqpoint{1.338364in}{1.435279in}}%
\pgfpathlineto{\pgfqpoint{1.340507in}{1.436365in}}%
\pgfpathlineto{\pgfqpoint{1.341593in}{1.436831in}}%
\pgfpathlineto{\pgfqpoint{1.344126in}{1.437918in}}%
\pgfpathlineto{\pgfqpoint{1.345228in}{1.438445in}}%
\pgfpathlineto{\pgfqpoint{1.347793in}{1.439532in}}%
\pgfpathlineto{\pgfqpoint{1.348809in}{1.439935in}}%
\pgfpathlineto{\pgfqpoint{1.351967in}{1.441022in}}%
\pgfpathlineto{\pgfqpoint{1.352929in}{1.441674in}}%
\pgfpathlineto{\pgfqpoint{1.355837in}{1.442760in}}%
\pgfpathlineto{\pgfqpoint{1.356924in}{1.443257in}}%
\pgfpathlineto{\pgfqpoint{1.359339in}{1.444343in}}%
\pgfpathlineto{\pgfqpoint{1.360223in}{1.444685in}}%
\pgfpathlineto{\pgfqpoint{1.362849in}{1.445771in}}%
\pgfpathlineto{\pgfqpoint{1.363959in}{1.446392in}}%
\pgfpathlineto{\pgfqpoint{1.366289in}{1.447447in}}%
\pgfpathlineto{\pgfqpoint{1.367337in}{1.447913in}}%
\pgfpathlineto{\pgfqpoint{1.370370in}{1.449000in}}%
\pgfpathlineto{\pgfqpoint{1.371425in}{1.449620in}}%
\pgfpathlineto{\pgfqpoint{1.374654in}{1.450707in}}%
\pgfpathlineto{\pgfqpoint{1.375764in}{1.451266in}}%
\pgfpathlineto{\pgfqpoint{1.378563in}{1.452352in}}%
\pgfpathlineto{\pgfqpoint{1.379610in}{1.452725in}}%
\pgfpathlineto{\pgfqpoint{1.381956in}{1.453811in}}%
\pgfpathlineto{\pgfqpoint{1.382691in}{1.454184in}}%
\pgfpathlineto{\pgfqpoint{1.382948in}{1.454184in}}%
\pgfpathlineto{\pgfqpoint{1.385028in}{1.455270in}}%
\pgfpathlineto{\pgfqpoint{1.386083in}{1.455798in}}%
\pgfpathlineto{\pgfqpoint{1.386099in}{1.455798in}}%
\pgfpathlineto{\pgfqpoint{1.388296in}{1.456884in}}%
\pgfpathlineto{\pgfqpoint{1.389382in}{1.457381in}}%
\pgfpathlineto{\pgfqpoint{1.392635in}{1.458467in}}%
\pgfpathlineto{\pgfqpoint{1.393745in}{1.458995in}}%
\pgfpathlineto{\pgfqpoint{1.395941in}{1.460082in}}%
\pgfpathlineto{\pgfqpoint{1.396919in}{1.460640in}}%
\pgfpathlineto{\pgfqpoint{1.399655in}{1.461696in}}%
\pgfpathlineto{\pgfqpoint{1.400718in}{1.462224in}}%
\pgfpathlineto{\pgfqpoint{1.400765in}{1.462224in}}%
\pgfpathlineto{\pgfqpoint{1.403571in}{1.463310in}}%
\pgfpathlineto{\pgfqpoint{1.404541in}{1.463838in}}%
\pgfpathlineto{\pgfqpoint{1.406996in}{1.464924in}}%
\pgfpathlineto{\pgfqpoint{1.408090in}{1.465483in}}%
\pgfpathlineto{\pgfqpoint{1.411467in}{1.466569in}}%
\pgfpathlineto{\pgfqpoint{1.412382in}{1.466942in}}%
\pgfpathlineto{\pgfqpoint{1.416650in}{1.468028in}}%
\pgfpathlineto{\pgfqpoint{1.417542in}{1.468308in}}%
\pgfpathlineto{\pgfqpoint{1.417674in}{1.468308in}}%
\pgfpathlineto{\pgfqpoint{1.420286in}{1.469394in}}%
\pgfpathlineto{\pgfqpoint{1.421357in}{1.469829in}}%
\pgfpathlineto{\pgfqpoint{1.425187in}{1.470915in}}%
\pgfpathlineto{\pgfqpoint{1.426227in}{1.471536in}}%
\pgfpathlineto{\pgfqpoint{1.429205in}{1.472623in}}%
\pgfpathlineto{\pgfqpoint{1.430214in}{1.473212in}}%
\pgfpathlineto{\pgfqpoint{1.432778in}{1.474299in}}%
\pgfpathlineto{\pgfqpoint{1.433794in}{1.475013in}}%
\pgfpathlineto{\pgfqpoint{1.433865in}{1.475013in}}%
\pgfpathlineto{\pgfqpoint{1.436820in}{1.476099in}}%
\pgfpathlineto{\pgfqpoint{1.437922in}{1.476658in}}%
\pgfpathlineto{\pgfqpoint{1.440791in}{1.477745in}}%
\pgfpathlineto{\pgfqpoint{1.441886in}{1.478303in}}%
\pgfpathlineto{\pgfqpoint{1.444552in}{1.479390in}}%
\pgfpathlineto{\pgfqpoint{1.445466in}{1.479949in}}%
\pgfpathlineto{\pgfqpoint{1.448890in}{1.481035in}}%
\pgfpathlineto{\pgfqpoint{1.449914in}{1.481532in}}%
\pgfpathlineto{\pgfqpoint{1.449993in}{1.481532in}}%
\pgfpathlineto{\pgfqpoint{1.453315in}{1.482587in}}%
\pgfpathlineto{\pgfqpoint{1.454355in}{1.483053in}}%
\pgfpathlineto{\pgfqpoint{1.457412in}{1.484139in}}%
\pgfpathlineto{\pgfqpoint{1.458631in}{1.484791in}}%
\pgfpathlineto{\pgfqpoint{1.461750in}{1.485878in}}%
\pgfpathlineto{\pgfqpoint{1.462782in}{1.486281in}}%
\pgfpathlineto{\pgfqpoint{1.466144in}{1.487368in}}%
\pgfpathlineto{\pgfqpoint{1.467113in}{1.488082in}}%
\pgfpathlineto{\pgfqpoint{1.467238in}{1.488082in}}%
\pgfpathlineto{\pgfqpoint{1.469834in}{1.489168in}}%
\pgfpathlineto{\pgfqpoint{1.470811in}{1.489820in}}%
\pgfpathlineto{\pgfqpoint{1.470842in}{1.489820in}}%
\pgfpathlineto{\pgfqpoint{1.474438in}{1.490906in}}%
\pgfpathlineto{\pgfqpoint{1.475525in}{1.491403in}}%
\pgfpathlineto{\pgfqpoint{1.478543in}{1.492490in}}%
\pgfpathlineto{\pgfqpoint{1.479489in}{1.493173in}}%
\pgfpathlineto{\pgfqpoint{1.479528in}{1.493173in}}%
\pgfpathlineto{\pgfqpoint{1.482920in}{1.494259in}}%
\pgfpathlineto{\pgfqpoint{1.484023in}{1.494663in}}%
\pgfpathlineto{\pgfqpoint{1.486579in}{1.495749in}}%
\pgfpathlineto{\pgfqpoint{1.487439in}{1.496028in}}%
\pgfpathlineto{\pgfqpoint{1.490808in}{1.497115in}}%
\pgfpathlineto{\pgfqpoint{1.491770in}{1.497332in}}%
\pgfpathlineto{\pgfqpoint{1.491911in}{1.497332in}}%
\pgfpathlineto{\pgfqpoint{1.496101in}{1.498419in}}%
\pgfpathlineto{\pgfqpoint{1.497188in}{1.498977in}}%
\pgfpathlineto{\pgfqpoint{1.499893in}{1.500064in}}%
\pgfpathlineto{\pgfqpoint{1.500940in}{1.500530in}}%
\pgfpathlineto{\pgfqpoint{1.503864in}{1.501616in}}%
\pgfpathlineto{\pgfqpoint{1.504919in}{1.502237in}}%
\pgfpathlineto{\pgfqpoint{1.507679in}{1.503292in}}%
\pgfpathlineto{\pgfqpoint{1.508758in}{1.503820in}}%
\pgfpathlineto{\pgfqpoint{1.513042in}{1.504906in}}%
\pgfpathlineto{\pgfqpoint{1.514097in}{1.505217in}}%
\pgfpathlineto{\pgfqpoint{1.514144in}{1.505217in}}%
\pgfpathlineto{\pgfqpoint{1.516802in}{1.506303in}}%
\pgfpathlineto{\pgfqpoint{1.517865in}{1.506924in}}%
\pgfpathlineto{\pgfqpoint{1.517912in}{1.506924in}}%
\pgfpathlineto{\pgfqpoint{1.520985in}{1.508011in}}%
\pgfpathlineto{\pgfqpoint{1.521931in}{1.508383in}}%
\pgfpathlineto{\pgfqpoint{1.525183in}{1.509470in}}%
\pgfpathlineto{\pgfqpoint{1.526285in}{1.509935in}}%
\pgfpathlineto{\pgfqpoint{1.530233in}{1.511022in}}%
\pgfpathlineto{\pgfqpoint{1.530819in}{1.511332in}}%
\pgfpathlineto{\pgfqpoint{1.531304in}{1.511332in}}%
\pgfpathlineto{\pgfqpoint{1.536104in}{1.512419in}}%
\pgfpathlineto{\pgfqpoint{1.537105in}{1.512636in}}%
\pgfpathlineto{\pgfqpoint{1.540458in}{1.513722in}}%
\pgfpathlineto{\pgfqpoint{1.541529in}{1.514312in}}%
\pgfpathlineto{\pgfqpoint{1.545086in}{1.515368in}}%
\pgfpathlineto{\pgfqpoint{1.546079in}{1.515895in}}%
\pgfpathlineto{\pgfqpoint{1.551724in}{1.516982in}}%
\pgfpathlineto{\pgfqpoint{1.552709in}{1.517354in}}%
\pgfpathlineto{\pgfqpoint{1.552771in}{1.517354in}}%
\pgfpathlineto{\pgfqpoint{1.556219in}{1.518441in}}%
\pgfpathlineto{\pgfqpoint{1.557212in}{1.518751in}}%
\pgfpathlineto{\pgfqpoint{1.557329in}{1.518751in}}%
\pgfpathlineto{\pgfqpoint{1.560753in}{1.519838in}}%
\pgfpathlineto{\pgfqpoint{1.561808in}{1.520241in}}%
\pgfpathlineto{\pgfqpoint{1.565694in}{1.521328in}}%
\pgfpathlineto{\pgfqpoint{1.566632in}{1.521700in}}%
\pgfpathlineto{\pgfqpoint{1.570572in}{1.522787in}}%
\pgfpathlineto{\pgfqpoint{1.571510in}{1.523252in}}%
\pgfpathlineto{\pgfqpoint{1.574778in}{1.524339in}}%
\pgfpathlineto{\pgfqpoint{1.575849in}{1.524773in}}%
\pgfpathlineto{\pgfqpoint{1.578436in}{1.525860in}}%
\pgfpathlineto{\pgfqpoint{1.579515in}{1.526232in}}%
\pgfpathlineto{\pgfqpoint{1.579539in}{1.526232in}}%
\pgfpathlineto{\pgfqpoint{1.585035in}{1.527319in}}%
\pgfpathlineto{\pgfqpoint{1.586051in}{1.527816in}}%
\pgfpathlineto{\pgfqpoint{1.589334in}{1.528902in}}%
\pgfpathlineto{\pgfqpoint{1.590366in}{1.529275in}}%
\pgfpathlineto{\pgfqpoint{1.590397in}{1.529275in}}%
\pgfpathlineto{\pgfqpoint{1.595393in}{1.530361in}}%
\pgfpathlineto{\pgfqpoint{1.596503in}{1.530734in}}%
\pgfpathlineto{\pgfqpoint{1.601303in}{1.531820in}}%
\pgfpathlineto{\pgfqpoint{1.602358in}{1.532193in}}%
\pgfpathlineto{\pgfqpoint{1.605611in}{1.533279in}}%
\pgfpathlineto{\pgfqpoint{1.606322in}{1.533496in}}%
\pgfpathlineto{\pgfqpoint{1.610903in}{1.534583in}}%
\pgfpathlineto{\pgfqpoint{1.611896in}{1.534893in}}%
\pgfpathlineto{\pgfqpoint{1.615523in}{1.535980in}}%
\pgfpathlineto{\pgfqpoint{1.616407in}{1.536166in}}%
\pgfpathlineto{\pgfqpoint{1.619862in}{1.537252in}}%
\pgfpathlineto{\pgfqpoint{1.620957in}{1.537718in}}%
\pgfpathlineto{\pgfqpoint{1.624897in}{1.538804in}}%
\pgfpathlineto{\pgfqpoint{1.625905in}{1.539084in}}%
\pgfpathlineto{\pgfqpoint{1.629791in}{1.540170in}}%
\pgfpathlineto{\pgfqpoint{1.630830in}{1.540667in}}%
\pgfpathlineto{\pgfqpoint{1.634809in}{1.541753in}}%
\pgfpathlineto{\pgfqpoint{1.635904in}{1.541971in}}%
\pgfpathlineto{\pgfqpoint{1.639438in}{1.543026in}}%
\pgfpathlineto{\pgfqpoint{1.640430in}{1.543554in}}%
\pgfpathlineto{\pgfqpoint{1.644574in}{1.544640in}}%
\pgfpathlineto{\pgfqpoint{1.645684in}{1.544920in}}%
\pgfpathlineto{\pgfqpoint{1.650695in}{1.546006in}}%
\pgfpathlineto{\pgfqpoint{1.651719in}{1.546348in}}%
\pgfpathlineto{\pgfqpoint{1.651766in}{1.546348in}}%
\pgfpathlineto{\pgfqpoint{1.656605in}{1.547434in}}%
\pgfpathlineto{\pgfqpoint{1.657707in}{1.547838in}}%
\pgfpathlineto{\pgfqpoint{1.661624in}{1.548924in}}%
\pgfpathlineto{\pgfqpoint{1.662460in}{1.549235in}}%
\pgfpathlineto{\pgfqpoint{1.666682in}{1.550321in}}%
\pgfpathlineto{\pgfqpoint{1.667683in}{1.550507in}}%
\pgfpathlineto{\pgfqpoint{1.671435in}{1.551594in}}%
\pgfpathlineto{\pgfqpoint{1.672389in}{1.551749in}}%
\pgfpathlineto{\pgfqpoint{1.677588in}{1.552836in}}%
\pgfpathlineto{\pgfqpoint{1.678690in}{1.553115in}}%
\pgfpathlineto{\pgfqpoint{1.683951in}{1.554201in}}%
\pgfpathlineto{\pgfqpoint{1.684913in}{1.554450in}}%
\pgfpathlineto{\pgfqpoint{1.689009in}{1.555536in}}%
\pgfpathlineto{\pgfqpoint{1.690018in}{1.555785in}}%
\pgfpathlineto{\pgfqpoint{1.694372in}{1.556871in}}%
\pgfpathlineto{\pgfqpoint{1.695459in}{1.557088in}}%
\pgfpathlineto{\pgfqpoint{1.700556in}{1.558175in}}%
\pgfpathlineto{\pgfqpoint{1.701611in}{1.558485in}}%
\pgfpathlineto{\pgfqpoint{1.707834in}{1.559572in}}%
\pgfpathlineto{\pgfqpoint{1.708866in}{1.559820in}}%
\pgfpathlineto{\pgfqpoint{1.708897in}{1.559820in}}%
\pgfpathlineto{\pgfqpoint{1.713869in}{1.560906in}}%
\pgfpathlineto{\pgfqpoint{1.714893in}{1.561186in}}%
\pgfpathlineto{\pgfqpoint{1.719256in}{1.562272in}}%
\pgfpathlineto{\pgfqpoint{1.720327in}{1.562552in}}%
\pgfpathlineto{\pgfqpoint{1.724681in}{1.563638in}}%
\pgfpathlineto{\pgfqpoint{1.725580in}{1.563731in}}%
\pgfpathlineto{\pgfqpoint{1.732092in}{1.564818in}}%
\pgfpathlineto{\pgfqpoint{1.733030in}{1.565190in}}%
\pgfpathlineto{\pgfqpoint{1.739073in}{1.566277in}}%
\pgfpathlineto{\pgfqpoint{1.739793in}{1.566556in}}%
\pgfpathlineto{\pgfqpoint{1.744702in}{1.567643in}}%
\pgfpathlineto{\pgfqpoint{1.745711in}{1.567829in}}%
\pgfpathlineto{\pgfqpoint{1.750964in}{1.568915in}}%
\pgfpathlineto{\pgfqpoint{1.752066in}{1.569195in}}%
\pgfpathlineto{\pgfqpoint{1.758117in}{1.570281in}}%
\pgfpathlineto{\pgfqpoint{1.759157in}{1.570685in}}%
\pgfpathlineto{\pgfqpoint{1.764723in}{1.571771in}}%
\pgfpathlineto{\pgfqpoint{1.765732in}{1.571957in}}%
\pgfpathlineto{\pgfqpoint{1.765825in}{1.571957in}}%
\pgfpathlineto{\pgfqpoint{1.772736in}{1.573044in}}%
\pgfpathlineto{\pgfqpoint{1.773846in}{1.573199in}}%
\pgfpathlineto{\pgfqpoint{1.780992in}{1.574286in}}%
\pgfpathlineto{\pgfqpoint{1.782102in}{1.574534in}}%
\pgfpathlineto{\pgfqpoint{1.789208in}{1.575620in}}%
\pgfpathlineto{\pgfqpoint{1.789990in}{1.575807in}}%
\pgfpathlineto{\pgfqpoint{1.790318in}{1.575807in}}%
\pgfpathlineto{\pgfqpoint{1.796009in}{1.576862in}}%
\pgfpathlineto{\pgfqpoint{1.796744in}{1.577079in}}%
\pgfpathlineto{\pgfqpoint{1.804476in}{1.578166in}}%
\pgfpathlineto{\pgfqpoint{1.805547in}{1.578290in}}%
\pgfpathlineto{\pgfqpoint{1.811731in}{1.579377in}}%
\pgfpathlineto{\pgfqpoint{1.812168in}{1.579501in}}%
\pgfpathlineto{\pgfqpoint{1.812770in}{1.579501in}}%
\pgfpathlineto{\pgfqpoint{1.819533in}{1.580587in}}%
\pgfpathlineto{\pgfqpoint{1.820181in}{1.580804in}}%
\pgfpathlineto{\pgfqpoint{1.820385in}{1.580804in}}%
\pgfpathlineto{\pgfqpoint{1.830610in}{1.581891in}}%
\pgfpathlineto{\pgfqpoint{1.831666in}{1.582046in}}%
\pgfpathlineto{\pgfqpoint{1.838959in}{1.583133in}}%
\pgfpathlineto{\pgfqpoint{1.839835in}{1.583381in}}%
\pgfpathlineto{\pgfqpoint{1.847348in}{1.584467in}}%
\pgfpathlineto{\pgfqpoint{1.848247in}{1.584623in}}%
\pgfpathlineto{\pgfqpoint{1.855025in}{1.585709in}}%
\pgfpathlineto{\pgfqpoint{1.855861in}{1.585957in}}%
\pgfpathlineto{\pgfqpoint{1.862311in}{1.587013in}}%
\pgfpathlineto{\pgfqpoint{1.863366in}{1.587354in}}%
\pgfpathlineto{\pgfqpoint{1.863405in}{1.587354in}}%
\pgfpathlineto{\pgfqpoint{1.873685in}{1.588441in}}%
\pgfpathlineto{\pgfqpoint{1.874577in}{1.588658in}}%
\pgfpathlineto{\pgfqpoint{1.880682in}{1.589714in}}%
\pgfpathlineto{\pgfqpoint{1.881151in}{1.589869in}}%
\pgfpathlineto{\pgfqpoint{1.881730in}{1.589869in}}%
\pgfpathlineto{\pgfqpoint{1.893081in}{1.590955in}}%
\pgfpathlineto{\pgfqpoint{1.893464in}{1.591048in}}%
\pgfpathlineto{\pgfqpoint{1.893597in}{1.591048in}}%
\pgfpathlineto{\pgfqpoint{1.903205in}{1.592135in}}%
\pgfpathlineto{\pgfqpoint{1.904245in}{1.592321in}}%
\pgfpathlineto{\pgfqpoint{1.912414in}{1.593408in}}%
\pgfpathlineto{\pgfqpoint{1.913368in}{1.593501in}}%
\pgfpathlineto{\pgfqpoint{1.922264in}{1.594587in}}%
\pgfpathlineto{\pgfqpoint{1.922827in}{1.594742in}}%
\pgfpathlineto{\pgfqpoint{1.923108in}{1.594742in}}%
\pgfpathlineto{\pgfqpoint{1.935820in}{1.595829in}}%
\pgfpathlineto{\pgfqpoint{1.936641in}{1.595953in}}%
\pgfpathlineto{\pgfqpoint{1.936836in}{1.595953in}}%
\pgfpathlineto{\pgfqpoint{1.950845in}{1.597040in}}%
\pgfpathlineto{\pgfqpoint{1.951346in}{1.597226in}}%
\pgfpathlineto{\pgfqpoint{1.951666in}{1.597226in}}%
\pgfpathlineto{\pgfqpoint{1.968521in}{1.598312in}}%
\pgfpathlineto{\pgfqpoint{1.968990in}{1.598374in}}%
\pgfpathlineto{\pgfqpoint{1.969272in}{1.598374in}}%
\pgfpathlineto{\pgfqpoint{1.982585in}{1.599461in}}%
\pgfpathlineto{\pgfqpoint{1.982585in}{1.599492in}}%
\pgfpathlineto{\pgfqpoint{1.983203in}{1.599492in}}%
\pgfpathlineto{\pgfqpoint{2.001293in}{1.600578in}}%
\pgfpathlineto{\pgfqpoint{2.002168in}{1.600671in}}%
\pgfpathlineto{\pgfqpoint{2.002278in}{1.600671in}}%
\pgfpathlineto{\pgfqpoint{2.019031in}{1.601479in}}%
\pgfpathlineto{\pgfqpoint{2.033126in}{1.601944in}}%
\pgfpathlineto{\pgfqpoint{2.033126in}{1.601944in}}%
\pgfusepath{stroke}%
\end{pgfscope}%
\begin{pgfscope}%
\pgfsetrectcap%
\pgfsetmiterjoin%
\pgfsetlinewidth{0.803000pt}%
\definecolor{currentstroke}{rgb}{0.000000,0.000000,0.000000}%
\pgfsetstrokecolor{currentstroke}%
\pgfsetdash{}{0pt}%
\pgfpathmoveto{\pgfqpoint{0.553581in}{0.499444in}}%
\pgfpathlineto{\pgfqpoint{0.553581in}{1.654444in}}%
\pgfusepath{stroke}%
\end{pgfscope}%
\begin{pgfscope}%
\pgfsetrectcap%
\pgfsetmiterjoin%
\pgfsetlinewidth{0.803000pt}%
\definecolor{currentstroke}{rgb}{0.000000,0.000000,0.000000}%
\pgfsetstrokecolor{currentstroke}%
\pgfsetdash{}{0pt}%
\pgfpathmoveto{\pgfqpoint{2.103581in}{0.499444in}}%
\pgfpathlineto{\pgfqpoint{2.103581in}{1.654444in}}%
\pgfusepath{stroke}%
\end{pgfscope}%
\begin{pgfscope}%
\pgfsetrectcap%
\pgfsetmiterjoin%
\pgfsetlinewidth{0.803000pt}%
\definecolor{currentstroke}{rgb}{0.000000,0.000000,0.000000}%
\pgfsetstrokecolor{currentstroke}%
\pgfsetdash{}{0pt}%
\pgfpathmoveto{\pgfqpoint{0.553581in}{0.499444in}}%
\pgfpathlineto{\pgfqpoint{2.103581in}{0.499444in}}%
\pgfusepath{stroke}%
\end{pgfscope}%
\begin{pgfscope}%
\pgfsetrectcap%
\pgfsetmiterjoin%
\pgfsetlinewidth{0.803000pt}%
\definecolor{currentstroke}{rgb}{0.000000,0.000000,0.000000}%
\pgfsetstrokecolor{currentstroke}%
\pgfsetdash{}{0pt}%
\pgfpathmoveto{\pgfqpoint{0.553581in}{1.654444in}}%
\pgfpathlineto{\pgfqpoint{2.103581in}{1.654444in}}%
\pgfusepath{stroke}%
\end{pgfscope}%
\begin{pgfscope}%
\pgfsetbuttcap%
\pgfsetmiterjoin%
\definecolor{currentfill}{rgb}{1.000000,1.000000,1.000000}%
\pgfsetfillcolor{currentfill}%
\pgfsetfillopacity{0.800000}%
\pgfsetlinewidth{1.003750pt}%
\definecolor{currentstroke}{rgb}{0.800000,0.800000,0.800000}%
\pgfsetstrokecolor{currentstroke}%
\pgfsetstrokeopacity{0.800000}%
\pgfsetdash{}{0pt}%
\pgfpathmoveto{\pgfqpoint{0.832747in}{0.568889in}}%
\pgfpathlineto{\pgfqpoint{2.006358in}{0.568889in}}%
\pgfpathquadraticcurveto{\pgfqpoint{2.034136in}{0.568889in}}{\pgfqpoint{2.034136in}{0.596666in}}%
\pgfpathlineto{\pgfqpoint{2.034136in}{0.776388in}}%
\pgfpathquadraticcurveto{\pgfqpoint{2.034136in}{0.804166in}}{\pgfqpoint{2.006358in}{0.804166in}}%
\pgfpathlineto{\pgfqpoint{0.832747in}{0.804166in}}%
\pgfpathquadraticcurveto{\pgfqpoint{0.804970in}{0.804166in}}{\pgfqpoint{0.804970in}{0.776388in}}%
\pgfpathlineto{\pgfqpoint{0.804970in}{0.596666in}}%
\pgfpathquadraticcurveto{\pgfqpoint{0.804970in}{0.568889in}}{\pgfqpoint{0.832747in}{0.568889in}}%
\pgfpathlineto{\pgfqpoint{0.832747in}{0.568889in}}%
\pgfpathclose%
\pgfusepath{stroke,fill}%
\end{pgfscope}%
\begin{pgfscope}%
\pgfsetrectcap%
\pgfsetroundjoin%
\pgfsetlinewidth{1.505625pt}%
\definecolor{currentstroke}{rgb}{0.000000,0.000000,0.000000}%
\pgfsetstrokecolor{currentstroke}%
\pgfsetdash{}{0pt}%
\pgfpathmoveto{\pgfqpoint{0.860525in}{0.700000in}}%
\pgfpathlineto{\pgfqpoint{0.999414in}{0.700000in}}%
\pgfpathlineto{\pgfqpoint{1.138303in}{0.700000in}}%
\pgfusepath{stroke}%
\end{pgfscope}%
\begin{pgfscope}%
\definecolor{textcolor}{rgb}{0.000000,0.000000,0.000000}%
\pgfsetstrokecolor{textcolor}%
\pgfsetfillcolor{textcolor}%
\pgftext[x=1.249414in,y=0.651388in,left,base]{\color{textcolor}\rmfamily\fontsize{10.000000}{12.000000}\selectfont AUC=0.752}%
\end{pgfscope}%
\end{pgfpicture}%
\makeatother%
\endgroup%

\end{tabular}

\

%
\verb|Bagging_Hard_Tomek_0_v1_Test|

\

This model returned 217 different values, but most of them were rare.  Taking out the 5\% of the data set with the least frequent values, 95\% of the samples had only 10 values of $p$.  It may be a useful model, but we will not be able to fine tune the decision threshold.  

\noindent\begin{tabular}{@{\hspace{-6pt}}p{4.3in} @{\hspace{-6pt}}p{2.0in}}
	\vskip 0pt
	\hfil Raw Model Output
	
	%% Creator: Matplotlib, PGF backend
%%
%% To include the figure in your LaTeX document, write
%%   \input{<filename>.pgf}
%%
%% Make sure the required packages are loaded in your preamble
%%   \usepackage{pgf}
%%
%% Also ensure that all the required font packages are loaded; for instance,
%% the lmodern package is sometimes necessary when using math font.
%%   \usepackage{lmodern}
%%
%% Figures using additional raster images can only be included by \input if
%% they are in the same directory as the main LaTeX file. For loading figures
%% from other directories you can use the `import` package
%%   \usepackage{import}
%%
%% and then include the figures with
%%   \import{<path to file>}{<filename>.pgf}
%%
%% Matplotlib used the following preamble
%%   
%%   \usepackage{fontspec}
%%   \makeatletter\@ifpackageloaded{underscore}{}{\usepackage[strings]{underscore}}\makeatother
%%
\begingroup%
\makeatletter%
\begin{pgfpicture}%
\pgfpathrectangle{\pgfpointorigin}{\pgfqpoint{4.191250in}{1.754444in}}%
\pgfusepath{use as bounding box, clip}%
\begin{pgfscope}%
\pgfsetbuttcap%
\pgfsetmiterjoin%
\definecolor{currentfill}{rgb}{1.000000,1.000000,1.000000}%
\pgfsetfillcolor{currentfill}%
\pgfsetlinewidth{0.000000pt}%
\definecolor{currentstroke}{rgb}{1.000000,1.000000,1.000000}%
\pgfsetstrokecolor{currentstroke}%
\pgfsetdash{}{0pt}%
\pgfpathmoveto{\pgfqpoint{0.000000in}{0.000000in}}%
\pgfpathlineto{\pgfqpoint{4.191250in}{0.000000in}}%
\pgfpathlineto{\pgfqpoint{4.191250in}{1.754444in}}%
\pgfpathlineto{\pgfqpoint{0.000000in}{1.754444in}}%
\pgfpathlineto{\pgfqpoint{0.000000in}{0.000000in}}%
\pgfpathclose%
\pgfusepath{fill}%
\end{pgfscope}%
\begin{pgfscope}%
\pgfsetbuttcap%
\pgfsetmiterjoin%
\definecolor{currentfill}{rgb}{1.000000,1.000000,1.000000}%
\pgfsetfillcolor{currentfill}%
\pgfsetlinewidth{0.000000pt}%
\definecolor{currentstroke}{rgb}{0.000000,0.000000,0.000000}%
\pgfsetstrokecolor{currentstroke}%
\pgfsetstrokeopacity{0.000000}%
\pgfsetdash{}{0pt}%
\pgfpathmoveto{\pgfqpoint{0.515000in}{0.499444in}}%
\pgfpathlineto{\pgfqpoint{4.002500in}{0.499444in}}%
\pgfpathlineto{\pgfqpoint{4.002500in}{1.654444in}}%
\pgfpathlineto{\pgfqpoint{0.515000in}{1.654444in}}%
\pgfpathlineto{\pgfqpoint{0.515000in}{0.499444in}}%
\pgfpathclose%
\pgfusepath{fill}%
\end{pgfscope}%
\begin{pgfscope}%
\pgfpathrectangle{\pgfqpoint{0.515000in}{0.499444in}}{\pgfqpoint{3.487500in}{1.155000in}}%
\pgfusepath{clip}%
\pgfsetbuttcap%
\pgfsetmiterjoin%
\pgfsetlinewidth{1.003750pt}%
\definecolor{currentstroke}{rgb}{0.000000,0.000000,0.000000}%
\pgfsetstrokecolor{currentstroke}%
\pgfsetdash{}{0pt}%
\pgfpathmoveto{\pgfqpoint{0.505000in}{0.499444in}}%
\pgfpathlineto{\pgfqpoint{0.549530in}{0.499444in}}%
\pgfpathlineto{\pgfqpoint{0.549530in}{0.499779in}}%
\pgfpathlineto{\pgfqpoint{0.505000in}{0.499779in}}%
\pgfusepath{stroke}%
\end{pgfscope}%
\begin{pgfscope}%
\pgfpathrectangle{\pgfqpoint{0.515000in}{0.499444in}}{\pgfqpoint{3.487500in}{1.155000in}}%
\pgfusepath{clip}%
\pgfsetbuttcap%
\pgfsetmiterjoin%
\pgfsetlinewidth{1.003750pt}%
\definecolor{currentstroke}{rgb}{0.000000,0.000000,0.000000}%
\pgfsetstrokecolor{currentstroke}%
\pgfsetdash{}{0pt}%
\pgfpathmoveto{\pgfqpoint{0.632401in}{0.499444in}}%
\pgfpathlineto{\pgfqpoint{0.687649in}{0.499444in}}%
\pgfpathlineto{\pgfqpoint{0.687649in}{0.502938in}}%
\pgfpathlineto{\pgfqpoint{0.632401in}{0.502938in}}%
\pgfpathlineto{\pgfqpoint{0.632401in}{0.499444in}}%
\pgfpathclose%
\pgfusepath{stroke}%
\end{pgfscope}%
\begin{pgfscope}%
\pgfpathrectangle{\pgfqpoint{0.515000in}{0.499444in}}{\pgfqpoint{3.487500in}{1.155000in}}%
\pgfusepath{clip}%
\pgfsetbuttcap%
\pgfsetmiterjoin%
\pgfsetlinewidth{1.003750pt}%
\definecolor{currentstroke}{rgb}{0.000000,0.000000,0.000000}%
\pgfsetstrokecolor{currentstroke}%
\pgfsetdash{}{0pt}%
\pgfpathmoveto{\pgfqpoint{0.770520in}{0.499444in}}%
\pgfpathlineto{\pgfqpoint{0.825768in}{0.499444in}}%
\pgfpathlineto{\pgfqpoint{0.825768in}{1.573461in}}%
\pgfpathlineto{\pgfqpoint{0.770520in}{1.573461in}}%
\pgfpathlineto{\pgfqpoint{0.770520in}{0.499444in}}%
\pgfpathclose%
\pgfusepath{stroke}%
\end{pgfscope}%
\begin{pgfscope}%
\pgfpathrectangle{\pgfqpoint{0.515000in}{0.499444in}}{\pgfqpoint{3.487500in}{1.155000in}}%
\pgfusepath{clip}%
\pgfsetbuttcap%
\pgfsetmiterjoin%
\pgfsetlinewidth{1.003750pt}%
\definecolor{currentstroke}{rgb}{0.000000,0.000000,0.000000}%
\pgfsetstrokecolor{currentstroke}%
\pgfsetdash{}{0pt}%
\pgfpathmoveto{\pgfqpoint{0.908639in}{0.499444in}}%
\pgfpathlineto{\pgfqpoint{0.963886in}{0.499444in}}%
\pgfpathlineto{\pgfqpoint{0.963886in}{0.503905in}}%
\pgfpathlineto{\pgfqpoint{0.908639in}{0.503905in}}%
\pgfpathlineto{\pgfqpoint{0.908639in}{0.499444in}}%
\pgfpathclose%
\pgfusepath{stroke}%
\end{pgfscope}%
\begin{pgfscope}%
\pgfpathrectangle{\pgfqpoint{0.515000in}{0.499444in}}{\pgfqpoint{3.487500in}{1.155000in}}%
\pgfusepath{clip}%
\pgfsetbuttcap%
\pgfsetmiterjoin%
\pgfsetlinewidth{1.003750pt}%
\definecolor{currentstroke}{rgb}{0.000000,0.000000,0.000000}%
\pgfsetstrokecolor{currentstroke}%
\pgfsetdash{}{0pt}%
\pgfpathmoveto{\pgfqpoint{1.046758in}{0.499444in}}%
\pgfpathlineto{\pgfqpoint{1.102005in}{0.499444in}}%
\pgfpathlineto{\pgfqpoint{1.102005in}{1.599444in}}%
\pgfpathlineto{\pgfqpoint{1.046758in}{1.599444in}}%
\pgfpathlineto{\pgfqpoint{1.046758in}{0.499444in}}%
\pgfpathclose%
\pgfusepath{stroke}%
\end{pgfscope}%
\begin{pgfscope}%
\pgfpathrectangle{\pgfqpoint{0.515000in}{0.499444in}}{\pgfqpoint{3.487500in}{1.155000in}}%
\pgfusepath{clip}%
\pgfsetbuttcap%
\pgfsetmiterjoin%
\pgfsetlinewidth{1.003750pt}%
\definecolor{currentstroke}{rgb}{0.000000,0.000000,0.000000}%
\pgfsetstrokecolor{currentstroke}%
\pgfsetdash{}{0pt}%
\pgfpathmoveto{\pgfqpoint{1.184877in}{0.499444in}}%
\pgfpathlineto{\pgfqpoint{1.240124in}{0.499444in}}%
\pgfpathlineto{\pgfqpoint{1.240124in}{0.500299in}}%
\pgfpathlineto{\pgfqpoint{1.184877in}{0.500299in}}%
\pgfpathlineto{\pgfqpoint{1.184877in}{0.499444in}}%
\pgfpathclose%
\pgfusepath{stroke}%
\end{pgfscope}%
\begin{pgfscope}%
\pgfpathrectangle{\pgfqpoint{0.515000in}{0.499444in}}{\pgfqpoint{3.487500in}{1.155000in}}%
\pgfusepath{clip}%
\pgfsetbuttcap%
\pgfsetmiterjoin%
\pgfsetlinewidth{1.003750pt}%
\definecolor{currentstroke}{rgb}{0.000000,0.000000,0.000000}%
\pgfsetstrokecolor{currentstroke}%
\pgfsetdash{}{0pt}%
\pgfpathmoveto{\pgfqpoint{1.322995in}{0.499444in}}%
\pgfpathlineto{\pgfqpoint{1.378243in}{0.499444in}}%
\pgfpathlineto{\pgfqpoint{1.378243in}{0.507734in}}%
\pgfpathlineto{\pgfqpoint{1.322995in}{0.507734in}}%
\pgfpathlineto{\pgfqpoint{1.322995in}{0.499444in}}%
\pgfpathclose%
\pgfusepath{stroke}%
\end{pgfscope}%
\begin{pgfscope}%
\pgfpathrectangle{\pgfqpoint{0.515000in}{0.499444in}}{\pgfqpoint{3.487500in}{1.155000in}}%
\pgfusepath{clip}%
\pgfsetbuttcap%
\pgfsetmiterjoin%
\pgfsetlinewidth{1.003750pt}%
\definecolor{currentstroke}{rgb}{0.000000,0.000000,0.000000}%
\pgfsetstrokecolor{currentstroke}%
\pgfsetdash{}{0pt}%
\pgfpathmoveto{\pgfqpoint{1.461114in}{0.499444in}}%
\pgfpathlineto{\pgfqpoint{1.516362in}{0.499444in}}%
\pgfpathlineto{\pgfqpoint{1.516362in}{1.517554in}}%
\pgfpathlineto{\pgfqpoint{1.461114in}{1.517554in}}%
\pgfpathlineto{\pgfqpoint{1.461114in}{0.499444in}}%
\pgfpathclose%
\pgfusepath{stroke}%
\end{pgfscope}%
\begin{pgfscope}%
\pgfpathrectangle{\pgfqpoint{0.515000in}{0.499444in}}{\pgfqpoint{3.487500in}{1.155000in}}%
\pgfusepath{clip}%
\pgfsetbuttcap%
\pgfsetmiterjoin%
\pgfsetlinewidth{1.003750pt}%
\definecolor{currentstroke}{rgb}{0.000000,0.000000,0.000000}%
\pgfsetstrokecolor{currentstroke}%
\pgfsetdash{}{0pt}%
\pgfpathmoveto{\pgfqpoint{1.599233in}{0.499444in}}%
\pgfpathlineto{\pgfqpoint{1.654480in}{0.499444in}}%
\pgfpathlineto{\pgfqpoint{1.654480in}{0.506024in}}%
\pgfpathlineto{\pgfqpoint{1.599233in}{0.506024in}}%
\pgfpathlineto{\pgfqpoint{1.599233in}{0.499444in}}%
\pgfpathclose%
\pgfusepath{stroke}%
\end{pgfscope}%
\begin{pgfscope}%
\pgfpathrectangle{\pgfqpoint{0.515000in}{0.499444in}}{\pgfqpoint{3.487500in}{1.155000in}}%
\pgfusepath{clip}%
\pgfsetbuttcap%
\pgfsetmiterjoin%
\pgfsetlinewidth{1.003750pt}%
\definecolor{currentstroke}{rgb}{0.000000,0.000000,0.000000}%
\pgfsetstrokecolor{currentstroke}%
\pgfsetdash{}{0pt}%
\pgfpathmoveto{\pgfqpoint{1.737352in}{0.499444in}}%
\pgfpathlineto{\pgfqpoint{1.792599in}{0.499444in}}%
\pgfpathlineto{\pgfqpoint{1.792599in}{1.355371in}}%
\pgfpathlineto{\pgfqpoint{1.737352in}{1.355371in}}%
\pgfpathlineto{\pgfqpoint{1.737352in}{0.499444in}}%
\pgfpathclose%
\pgfusepath{stroke}%
\end{pgfscope}%
\begin{pgfscope}%
\pgfpathrectangle{\pgfqpoint{0.515000in}{0.499444in}}{\pgfqpoint{3.487500in}{1.155000in}}%
\pgfusepath{clip}%
\pgfsetbuttcap%
\pgfsetmiterjoin%
\pgfsetlinewidth{1.003750pt}%
\definecolor{currentstroke}{rgb}{0.000000,0.000000,0.000000}%
\pgfsetstrokecolor{currentstroke}%
\pgfsetdash{}{0pt}%
\pgfpathmoveto{\pgfqpoint{1.875471in}{0.499444in}}%
\pgfpathlineto{\pgfqpoint{1.930718in}{0.499444in}}%
\pgfpathlineto{\pgfqpoint{1.930718in}{0.500262in}}%
\pgfpathlineto{\pgfqpoint{1.875471in}{0.500262in}}%
\pgfpathlineto{\pgfqpoint{1.875471in}{0.499444in}}%
\pgfpathclose%
\pgfusepath{stroke}%
\end{pgfscope}%
\begin{pgfscope}%
\pgfpathrectangle{\pgfqpoint{0.515000in}{0.499444in}}{\pgfqpoint{3.487500in}{1.155000in}}%
\pgfusepath{clip}%
\pgfsetbuttcap%
\pgfsetmiterjoin%
\pgfsetlinewidth{1.003750pt}%
\definecolor{currentstroke}{rgb}{0.000000,0.000000,0.000000}%
\pgfsetstrokecolor{currentstroke}%
\pgfsetdash{}{0pt}%
\pgfpathmoveto{\pgfqpoint{2.013589in}{0.499444in}}%
\pgfpathlineto{\pgfqpoint{2.068837in}{0.499444in}}%
\pgfpathlineto{\pgfqpoint{2.068837in}{0.508365in}}%
\pgfpathlineto{\pgfqpoint{2.013589in}{0.508365in}}%
\pgfpathlineto{\pgfqpoint{2.013589in}{0.499444in}}%
\pgfpathclose%
\pgfusepath{stroke}%
\end{pgfscope}%
\begin{pgfscope}%
\pgfpathrectangle{\pgfqpoint{0.515000in}{0.499444in}}{\pgfqpoint{3.487500in}{1.155000in}}%
\pgfusepath{clip}%
\pgfsetbuttcap%
\pgfsetmiterjoin%
\pgfsetlinewidth{1.003750pt}%
\definecolor{currentstroke}{rgb}{0.000000,0.000000,0.000000}%
\pgfsetstrokecolor{currentstroke}%
\pgfsetdash{}{0pt}%
\pgfpathmoveto{\pgfqpoint{2.151708in}{0.499444in}}%
\pgfpathlineto{\pgfqpoint{2.206956in}{0.499444in}}%
\pgfpathlineto{\pgfqpoint{2.206956in}{1.172596in}}%
\pgfpathlineto{\pgfqpoint{2.151708in}{1.172596in}}%
\pgfpathlineto{\pgfqpoint{2.151708in}{0.499444in}}%
\pgfpathclose%
\pgfusepath{stroke}%
\end{pgfscope}%
\begin{pgfscope}%
\pgfpathrectangle{\pgfqpoint{0.515000in}{0.499444in}}{\pgfqpoint{3.487500in}{1.155000in}}%
\pgfusepath{clip}%
\pgfsetbuttcap%
\pgfsetmiterjoin%
\pgfsetlinewidth{1.003750pt}%
\definecolor{currentstroke}{rgb}{0.000000,0.000000,0.000000}%
\pgfsetstrokecolor{currentstroke}%
\pgfsetdash{}{0pt}%
\pgfpathmoveto{\pgfqpoint{2.289827in}{0.499444in}}%
\pgfpathlineto{\pgfqpoint{2.345075in}{0.499444in}}%
\pgfpathlineto{\pgfqpoint{2.345075in}{0.504165in}}%
\pgfpathlineto{\pgfqpoint{2.289827in}{0.504165in}}%
\pgfpathlineto{\pgfqpoint{2.289827in}{0.499444in}}%
\pgfpathclose%
\pgfusepath{stroke}%
\end{pgfscope}%
\begin{pgfscope}%
\pgfpathrectangle{\pgfqpoint{0.515000in}{0.499444in}}{\pgfqpoint{3.487500in}{1.155000in}}%
\pgfusepath{clip}%
\pgfsetbuttcap%
\pgfsetmiterjoin%
\pgfsetlinewidth{1.003750pt}%
\definecolor{currentstroke}{rgb}{0.000000,0.000000,0.000000}%
\pgfsetstrokecolor{currentstroke}%
\pgfsetdash{}{0pt}%
\pgfpathmoveto{\pgfqpoint{2.427946in}{0.499444in}}%
\pgfpathlineto{\pgfqpoint{2.483193in}{0.499444in}}%
\pgfpathlineto{\pgfqpoint{2.483193in}{0.973055in}}%
\pgfpathlineto{\pgfqpoint{2.427946in}{0.973055in}}%
\pgfpathlineto{\pgfqpoint{2.427946in}{0.499444in}}%
\pgfpathclose%
\pgfusepath{stroke}%
\end{pgfscope}%
\begin{pgfscope}%
\pgfpathrectangle{\pgfqpoint{0.515000in}{0.499444in}}{\pgfqpoint{3.487500in}{1.155000in}}%
\pgfusepath{clip}%
\pgfsetbuttcap%
\pgfsetmiterjoin%
\pgfsetlinewidth{1.003750pt}%
\definecolor{currentstroke}{rgb}{0.000000,0.000000,0.000000}%
\pgfsetstrokecolor{currentstroke}%
\pgfsetdash{}{0pt}%
\pgfpathmoveto{\pgfqpoint{2.566065in}{0.499444in}}%
\pgfpathlineto{\pgfqpoint{2.621312in}{0.499444in}}%
\pgfpathlineto{\pgfqpoint{2.621312in}{0.500559in}}%
\pgfpathlineto{\pgfqpoint{2.566065in}{0.500559in}}%
\pgfpathlineto{\pgfqpoint{2.566065in}{0.499444in}}%
\pgfpathclose%
\pgfusepath{stroke}%
\end{pgfscope}%
\begin{pgfscope}%
\pgfpathrectangle{\pgfqpoint{0.515000in}{0.499444in}}{\pgfqpoint{3.487500in}{1.155000in}}%
\pgfusepath{clip}%
\pgfsetbuttcap%
\pgfsetmiterjoin%
\pgfsetlinewidth{1.003750pt}%
\definecolor{currentstroke}{rgb}{0.000000,0.000000,0.000000}%
\pgfsetstrokecolor{currentstroke}%
\pgfsetdash{}{0pt}%
\pgfpathmoveto{\pgfqpoint{2.704183in}{0.499444in}}%
\pgfpathlineto{\pgfqpoint{2.759431in}{0.499444in}}%
\pgfpathlineto{\pgfqpoint{2.759431in}{0.504202in}}%
\pgfpathlineto{\pgfqpoint{2.704183in}{0.504202in}}%
\pgfpathlineto{\pgfqpoint{2.704183in}{0.499444in}}%
\pgfpathclose%
\pgfusepath{stroke}%
\end{pgfscope}%
\begin{pgfscope}%
\pgfpathrectangle{\pgfqpoint{0.515000in}{0.499444in}}{\pgfqpoint{3.487500in}{1.155000in}}%
\pgfusepath{clip}%
\pgfsetbuttcap%
\pgfsetmiterjoin%
\pgfsetlinewidth{1.003750pt}%
\definecolor{currentstroke}{rgb}{0.000000,0.000000,0.000000}%
\pgfsetstrokecolor{currentstroke}%
\pgfsetdash{}{0pt}%
\pgfpathmoveto{\pgfqpoint{2.842302in}{0.499444in}}%
\pgfpathlineto{\pgfqpoint{2.897550in}{0.499444in}}%
\pgfpathlineto{\pgfqpoint{2.897550in}{0.805557in}}%
\pgfpathlineto{\pgfqpoint{2.842302in}{0.805557in}}%
\pgfpathlineto{\pgfqpoint{2.842302in}{0.499444in}}%
\pgfpathclose%
\pgfusepath{stroke}%
\end{pgfscope}%
\begin{pgfscope}%
\pgfpathrectangle{\pgfqpoint{0.515000in}{0.499444in}}{\pgfqpoint{3.487500in}{1.155000in}}%
\pgfusepath{clip}%
\pgfsetbuttcap%
\pgfsetmiterjoin%
\pgfsetlinewidth{1.003750pt}%
\definecolor{currentstroke}{rgb}{0.000000,0.000000,0.000000}%
\pgfsetstrokecolor{currentstroke}%
\pgfsetdash{}{0pt}%
\pgfpathmoveto{\pgfqpoint{2.980421in}{0.499444in}}%
\pgfpathlineto{\pgfqpoint{3.035669in}{0.499444in}}%
\pgfpathlineto{\pgfqpoint{3.035669in}{0.501637in}}%
\pgfpathlineto{\pgfqpoint{2.980421in}{0.501637in}}%
\pgfpathlineto{\pgfqpoint{2.980421in}{0.499444in}}%
\pgfpathclose%
\pgfusepath{stroke}%
\end{pgfscope}%
\begin{pgfscope}%
\pgfpathrectangle{\pgfqpoint{0.515000in}{0.499444in}}{\pgfqpoint{3.487500in}{1.155000in}}%
\pgfusepath{clip}%
\pgfsetbuttcap%
\pgfsetmiterjoin%
\pgfsetlinewidth{1.003750pt}%
\definecolor{currentstroke}{rgb}{0.000000,0.000000,0.000000}%
\pgfsetstrokecolor{currentstroke}%
\pgfsetdash{}{0pt}%
\pgfpathmoveto{\pgfqpoint{3.118540in}{0.499444in}}%
\pgfpathlineto{\pgfqpoint{3.173787in}{0.499444in}}%
\pgfpathlineto{\pgfqpoint{3.173787in}{0.690546in}}%
\pgfpathlineto{\pgfqpoint{3.118540in}{0.690546in}}%
\pgfpathlineto{\pgfqpoint{3.118540in}{0.499444in}}%
\pgfpathclose%
\pgfusepath{stroke}%
\end{pgfscope}%
\begin{pgfscope}%
\pgfpathrectangle{\pgfqpoint{0.515000in}{0.499444in}}{\pgfqpoint{3.487500in}{1.155000in}}%
\pgfusepath{clip}%
\pgfsetbuttcap%
\pgfsetmiterjoin%
\pgfsetlinewidth{1.003750pt}%
\definecolor{currentstroke}{rgb}{0.000000,0.000000,0.000000}%
\pgfsetstrokecolor{currentstroke}%
\pgfsetdash{}{0pt}%
\pgfpathmoveto{\pgfqpoint{3.256659in}{0.499444in}}%
\pgfpathlineto{\pgfqpoint{3.311906in}{0.499444in}}%
\pgfpathlineto{\pgfqpoint{3.311906in}{0.499742in}}%
\pgfpathlineto{\pgfqpoint{3.256659in}{0.499742in}}%
\pgfpathlineto{\pgfqpoint{3.256659in}{0.499444in}}%
\pgfpathclose%
\pgfusepath{stroke}%
\end{pgfscope}%
\begin{pgfscope}%
\pgfpathrectangle{\pgfqpoint{0.515000in}{0.499444in}}{\pgfqpoint{3.487500in}{1.155000in}}%
\pgfusepath{clip}%
\pgfsetbuttcap%
\pgfsetmiterjoin%
\pgfsetlinewidth{1.003750pt}%
\definecolor{currentstroke}{rgb}{0.000000,0.000000,0.000000}%
\pgfsetstrokecolor{currentstroke}%
\pgfsetdash{}{0pt}%
\pgfpathmoveto{\pgfqpoint{3.394778in}{0.499444in}}%
\pgfpathlineto{\pgfqpoint{3.450025in}{0.499444in}}%
\pgfpathlineto{\pgfqpoint{3.450025in}{0.501080in}}%
\pgfpathlineto{\pgfqpoint{3.394778in}{0.501080in}}%
\pgfpathlineto{\pgfqpoint{3.394778in}{0.499444in}}%
\pgfpathclose%
\pgfusepath{stroke}%
\end{pgfscope}%
\begin{pgfscope}%
\pgfpathrectangle{\pgfqpoint{0.515000in}{0.499444in}}{\pgfqpoint{3.487500in}{1.155000in}}%
\pgfusepath{clip}%
\pgfsetbuttcap%
\pgfsetmiterjoin%
\pgfsetlinewidth{1.003750pt}%
\definecolor{currentstroke}{rgb}{0.000000,0.000000,0.000000}%
\pgfsetstrokecolor{currentstroke}%
\pgfsetdash{}{0pt}%
\pgfpathmoveto{\pgfqpoint{3.532896in}{0.499444in}}%
\pgfpathlineto{\pgfqpoint{3.588144in}{0.499444in}}%
\pgfpathlineto{\pgfqpoint{3.588144in}{0.598508in}}%
\pgfpathlineto{\pgfqpoint{3.532896in}{0.598508in}}%
\pgfpathlineto{\pgfqpoint{3.532896in}{0.499444in}}%
\pgfpathclose%
\pgfusepath{stroke}%
\end{pgfscope}%
\begin{pgfscope}%
\pgfpathrectangle{\pgfqpoint{0.515000in}{0.499444in}}{\pgfqpoint{3.487500in}{1.155000in}}%
\pgfusepath{clip}%
\pgfsetbuttcap%
\pgfsetmiterjoin%
\pgfsetlinewidth{1.003750pt}%
\definecolor{currentstroke}{rgb}{0.000000,0.000000,0.000000}%
\pgfsetstrokecolor{currentstroke}%
\pgfsetdash{}{0pt}%
\pgfpathmoveto{\pgfqpoint{3.671015in}{0.499444in}}%
\pgfpathlineto{\pgfqpoint{3.726263in}{0.499444in}}%
\pgfpathlineto{\pgfqpoint{3.726263in}{0.500002in}}%
\pgfpathlineto{\pgfqpoint{3.671015in}{0.500002in}}%
\pgfpathlineto{\pgfqpoint{3.671015in}{0.499444in}}%
\pgfpathclose%
\pgfusepath{stroke}%
\end{pgfscope}%
\begin{pgfscope}%
\pgfpathrectangle{\pgfqpoint{0.515000in}{0.499444in}}{\pgfqpoint{3.487500in}{1.155000in}}%
\pgfusepath{clip}%
\pgfsetbuttcap%
\pgfsetmiterjoin%
\pgfsetlinewidth{1.003750pt}%
\definecolor{currentstroke}{rgb}{0.000000,0.000000,0.000000}%
\pgfsetstrokecolor{currentstroke}%
\pgfsetdash{}{0pt}%
\pgfpathmoveto{\pgfqpoint{3.809134in}{0.499444in}}%
\pgfpathlineto{\pgfqpoint{3.864381in}{0.499444in}}%
\pgfpathlineto{\pgfqpoint{3.864381in}{0.538103in}}%
\pgfpathlineto{\pgfqpoint{3.809134in}{0.538103in}}%
\pgfpathlineto{\pgfqpoint{3.809134in}{0.499444in}}%
\pgfpathclose%
\pgfusepath{stroke}%
\end{pgfscope}%
\begin{pgfscope}%
\pgfpathrectangle{\pgfqpoint{0.515000in}{0.499444in}}{\pgfqpoint{3.487500in}{1.155000in}}%
\pgfusepath{clip}%
\pgfsetbuttcap%
\pgfsetmiterjoin%
\definecolor{currentfill}{rgb}{0.000000,0.000000,0.000000}%
\pgfsetfillcolor{currentfill}%
\pgfsetlinewidth{0.000000pt}%
\definecolor{currentstroke}{rgb}{0.000000,0.000000,0.000000}%
\pgfsetstrokecolor{currentstroke}%
\pgfsetstrokeopacity{0.000000}%
\pgfsetdash{}{0pt}%
\pgfpathmoveto{\pgfqpoint{0.549530in}{0.499444in}}%
\pgfpathlineto{\pgfqpoint{0.604778in}{0.499444in}}%
\pgfpathlineto{\pgfqpoint{0.604778in}{0.499444in}}%
\pgfpathlineto{\pgfqpoint{0.549530in}{0.499444in}}%
\pgfpathlineto{\pgfqpoint{0.549530in}{0.499444in}}%
\pgfpathclose%
\pgfusepath{fill}%
\end{pgfscope}%
\begin{pgfscope}%
\pgfpathrectangle{\pgfqpoint{0.515000in}{0.499444in}}{\pgfqpoint{3.487500in}{1.155000in}}%
\pgfusepath{clip}%
\pgfsetbuttcap%
\pgfsetmiterjoin%
\definecolor{currentfill}{rgb}{0.000000,0.000000,0.000000}%
\pgfsetfillcolor{currentfill}%
\pgfsetlinewidth{0.000000pt}%
\definecolor{currentstroke}{rgb}{0.000000,0.000000,0.000000}%
\pgfsetstrokecolor{currentstroke}%
\pgfsetstrokeopacity{0.000000}%
\pgfsetdash{}{0pt}%
\pgfpathmoveto{\pgfqpoint{0.687649in}{0.499444in}}%
\pgfpathlineto{\pgfqpoint{0.742896in}{0.499444in}}%
\pgfpathlineto{\pgfqpoint{0.742896in}{0.499630in}}%
\pgfpathlineto{\pgfqpoint{0.687649in}{0.499630in}}%
\pgfpathlineto{\pgfqpoint{0.687649in}{0.499444in}}%
\pgfpathclose%
\pgfusepath{fill}%
\end{pgfscope}%
\begin{pgfscope}%
\pgfpathrectangle{\pgfqpoint{0.515000in}{0.499444in}}{\pgfqpoint{3.487500in}{1.155000in}}%
\pgfusepath{clip}%
\pgfsetbuttcap%
\pgfsetmiterjoin%
\definecolor{currentfill}{rgb}{0.000000,0.000000,0.000000}%
\pgfsetfillcolor{currentfill}%
\pgfsetlinewidth{0.000000pt}%
\definecolor{currentstroke}{rgb}{0.000000,0.000000,0.000000}%
\pgfsetstrokecolor{currentstroke}%
\pgfsetstrokeopacity{0.000000}%
\pgfsetdash{}{0pt}%
\pgfpathmoveto{\pgfqpoint{0.825768in}{0.499444in}}%
\pgfpathlineto{\pgfqpoint{0.881015in}{0.499444in}}%
\pgfpathlineto{\pgfqpoint{0.881015in}{0.557173in}}%
\pgfpathlineto{\pgfqpoint{0.825768in}{0.557173in}}%
\pgfpathlineto{\pgfqpoint{0.825768in}{0.499444in}}%
\pgfpathclose%
\pgfusepath{fill}%
\end{pgfscope}%
\begin{pgfscope}%
\pgfpathrectangle{\pgfqpoint{0.515000in}{0.499444in}}{\pgfqpoint{3.487500in}{1.155000in}}%
\pgfusepath{clip}%
\pgfsetbuttcap%
\pgfsetmiterjoin%
\definecolor{currentfill}{rgb}{0.000000,0.000000,0.000000}%
\pgfsetfillcolor{currentfill}%
\pgfsetlinewidth{0.000000pt}%
\definecolor{currentstroke}{rgb}{0.000000,0.000000,0.000000}%
\pgfsetstrokecolor{currentstroke}%
\pgfsetstrokeopacity{0.000000}%
\pgfsetdash{}{0pt}%
\pgfpathmoveto{\pgfqpoint{0.963886in}{0.499444in}}%
\pgfpathlineto{\pgfqpoint{1.019134in}{0.499444in}}%
\pgfpathlineto{\pgfqpoint{1.019134in}{0.499853in}}%
\pgfpathlineto{\pgfqpoint{0.963886in}{0.499853in}}%
\pgfpathlineto{\pgfqpoint{0.963886in}{0.499444in}}%
\pgfpathclose%
\pgfusepath{fill}%
\end{pgfscope}%
\begin{pgfscope}%
\pgfpathrectangle{\pgfqpoint{0.515000in}{0.499444in}}{\pgfqpoint{3.487500in}{1.155000in}}%
\pgfusepath{clip}%
\pgfsetbuttcap%
\pgfsetmiterjoin%
\definecolor{currentfill}{rgb}{0.000000,0.000000,0.000000}%
\pgfsetfillcolor{currentfill}%
\pgfsetlinewidth{0.000000pt}%
\definecolor{currentstroke}{rgb}{0.000000,0.000000,0.000000}%
\pgfsetstrokecolor{currentstroke}%
\pgfsetstrokeopacity{0.000000}%
\pgfsetdash{}{0pt}%
\pgfpathmoveto{\pgfqpoint{1.102005in}{0.499444in}}%
\pgfpathlineto{\pgfqpoint{1.157253in}{0.499444in}}%
\pgfpathlineto{\pgfqpoint{1.157253in}{0.589141in}}%
\pgfpathlineto{\pgfqpoint{1.102005in}{0.589141in}}%
\pgfpathlineto{\pgfqpoint{1.102005in}{0.499444in}}%
\pgfpathclose%
\pgfusepath{fill}%
\end{pgfscope}%
\begin{pgfscope}%
\pgfpathrectangle{\pgfqpoint{0.515000in}{0.499444in}}{\pgfqpoint{3.487500in}{1.155000in}}%
\pgfusepath{clip}%
\pgfsetbuttcap%
\pgfsetmiterjoin%
\definecolor{currentfill}{rgb}{0.000000,0.000000,0.000000}%
\pgfsetfillcolor{currentfill}%
\pgfsetlinewidth{0.000000pt}%
\definecolor{currentstroke}{rgb}{0.000000,0.000000,0.000000}%
\pgfsetstrokecolor{currentstroke}%
\pgfsetstrokeopacity{0.000000}%
\pgfsetdash{}{0pt}%
\pgfpathmoveto{\pgfqpoint{1.240124in}{0.499444in}}%
\pgfpathlineto{\pgfqpoint{1.295372in}{0.499444in}}%
\pgfpathlineto{\pgfqpoint{1.295372in}{0.499481in}}%
\pgfpathlineto{\pgfqpoint{1.240124in}{0.499481in}}%
\pgfpathlineto{\pgfqpoint{1.240124in}{0.499444in}}%
\pgfpathclose%
\pgfusepath{fill}%
\end{pgfscope}%
\begin{pgfscope}%
\pgfpathrectangle{\pgfqpoint{0.515000in}{0.499444in}}{\pgfqpoint{3.487500in}{1.155000in}}%
\pgfusepath{clip}%
\pgfsetbuttcap%
\pgfsetmiterjoin%
\definecolor{currentfill}{rgb}{0.000000,0.000000,0.000000}%
\pgfsetfillcolor{currentfill}%
\pgfsetlinewidth{0.000000pt}%
\definecolor{currentstroke}{rgb}{0.000000,0.000000,0.000000}%
\pgfsetstrokecolor{currentstroke}%
\pgfsetstrokeopacity{0.000000}%
\pgfsetdash{}{0pt}%
\pgfpathmoveto{\pgfqpoint{1.378243in}{0.499444in}}%
\pgfpathlineto{\pgfqpoint{1.433490in}{0.499444in}}%
\pgfpathlineto{\pgfqpoint{1.433490in}{0.500336in}}%
\pgfpathlineto{\pgfqpoint{1.378243in}{0.500336in}}%
\pgfpathlineto{\pgfqpoint{1.378243in}{0.499444in}}%
\pgfpathclose%
\pgfusepath{fill}%
\end{pgfscope}%
\begin{pgfscope}%
\pgfpathrectangle{\pgfqpoint{0.515000in}{0.499444in}}{\pgfqpoint{3.487500in}{1.155000in}}%
\pgfusepath{clip}%
\pgfsetbuttcap%
\pgfsetmiterjoin%
\definecolor{currentfill}{rgb}{0.000000,0.000000,0.000000}%
\pgfsetfillcolor{currentfill}%
\pgfsetlinewidth{0.000000pt}%
\definecolor{currentstroke}{rgb}{0.000000,0.000000,0.000000}%
\pgfsetstrokecolor{currentstroke}%
\pgfsetstrokeopacity{0.000000}%
\pgfsetdash{}{0pt}%
\pgfpathmoveto{\pgfqpoint{1.516362in}{0.499444in}}%
\pgfpathlineto{\pgfqpoint{1.571609in}{0.499444in}}%
\pgfpathlineto{\pgfqpoint{1.571609in}{0.616165in}}%
\pgfpathlineto{\pgfqpoint{1.516362in}{0.616165in}}%
\pgfpathlineto{\pgfqpoint{1.516362in}{0.499444in}}%
\pgfpathclose%
\pgfusepath{fill}%
\end{pgfscope}%
\begin{pgfscope}%
\pgfpathrectangle{\pgfqpoint{0.515000in}{0.499444in}}{\pgfqpoint{3.487500in}{1.155000in}}%
\pgfusepath{clip}%
\pgfsetbuttcap%
\pgfsetmiterjoin%
\definecolor{currentfill}{rgb}{0.000000,0.000000,0.000000}%
\pgfsetfillcolor{currentfill}%
\pgfsetlinewidth{0.000000pt}%
\definecolor{currentstroke}{rgb}{0.000000,0.000000,0.000000}%
\pgfsetstrokecolor{currentstroke}%
\pgfsetstrokeopacity{0.000000}%
\pgfsetdash{}{0pt}%
\pgfpathmoveto{\pgfqpoint{1.654480in}{0.499444in}}%
\pgfpathlineto{\pgfqpoint{1.709728in}{0.499444in}}%
\pgfpathlineto{\pgfqpoint{1.709728in}{0.500448in}}%
\pgfpathlineto{\pgfqpoint{1.654480in}{0.500448in}}%
\pgfpathlineto{\pgfqpoint{1.654480in}{0.499444in}}%
\pgfpathclose%
\pgfusepath{fill}%
\end{pgfscope}%
\begin{pgfscope}%
\pgfpathrectangle{\pgfqpoint{0.515000in}{0.499444in}}{\pgfqpoint{3.487500in}{1.155000in}}%
\pgfusepath{clip}%
\pgfsetbuttcap%
\pgfsetmiterjoin%
\definecolor{currentfill}{rgb}{0.000000,0.000000,0.000000}%
\pgfsetfillcolor{currentfill}%
\pgfsetlinewidth{0.000000pt}%
\definecolor{currentstroke}{rgb}{0.000000,0.000000,0.000000}%
\pgfsetstrokecolor{currentstroke}%
\pgfsetstrokeopacity{0.000000}%
\pgfsetdash{}{0pt}%
\pgfpathmoveto{\pgfqpoint{1.792599in}{0.499444in}}%
\pgfpathlineto{\pgfqpoint{1.847847in}{0.499444in}}%
\pgfpathlineto{\pgfqpoint{1.847847in}{0.640550in}}%
\pgfpathlineto{\pgfqpoint{1.792599in}{0.640550in}}%
\pgfpathlineto{\pgfqpoint{1.792599in}{0.499444in}}%
\pgfpathclose%
\pgfusepath{fill}%
\end{pgfscope}%
\begin{pgfscope}%
\pgfpathrectangle{\pgfqpoint{0.515000in}{0.499444in}}{\pgfqpoint{3.487500in}{1.155000in}}%
\pgfusepath{clip}%
\pgfsetbuttcap%
\pgfsetmiterjoin%
\definecolor{currentfill}{rgb}{0.000000,0.000000,0.000000}%
\pgfsetfillcolor{currentfill}%
\pgfsetlinewidth{0.000000pt}%
\definecolor{currentstroke}{rgb}{0.000000,0.000000,0.000000}%
\pgfsetstrokecolor{currentstroke}%
\pgfsetstrokeopacity{0.000000}%
\pgfsetdash{}{0pt}%
\pgfpathmoveto{\pgfqpoint{1.930718in}{0.499444in}}%
\pgfpathlineto{\pgfqpoint{1.985966in}{0.499444in}}%
\pgfpathlineto{\pgfqpoint{1.985966in}{0.499630in}}%
\pgfpathlineto{\pgfqpoint{1.930718in}{0.499630in}}%
\pgfpathlineto{\pgfqpoint{1.930718in}{0.499444in}}%
\pgfpathclose%
\pgfusepath{fill}%
\end{pgfscope}%
\begin{pgfscope}%
\pgfpathrectangle{\pgfqpoint{0.515000in}{0.499444in}}{\pgfqpoint{3.487500in}{1.155000in}}%
\pgfusepath{clip}%
\pgfsetbuttcap%
\pgfsetmiterjoin%
\definecolor{currentfill}{rgb}{0.000000,0.000000,0.000000}%
\pgfsetfillcolor{currentfill}%
\pgfsetlinewidth{0.000000pt}%
\definecolor{currentstroke}{rgb}{0.000000,0.000000,0.000000}%
\pgfsetstrokecolor{currentstroke}%
\pgfsetstrokeopacity{0.000000}%
\pgfsetdash{}{0pt}%
\pgfpathmoveto{\pgfqpoint{2.068837in}{0.499444in}}%
\pgfpathlineto{\pgfqpoint{2.124084in}{0.499444in}}%
\pgfpathlineto{\pgfqpoint{2.124084in}{0.501340in}}%
\pgfpathlineto{\pgfqpoint{2.068837in}{0.501340in}}%
\pgfpathlineto{\pgfqpoint{2.068837in}{0.499444in}}%
\pgfpathclose%
\pgfusepath{fill}%
\end{pgfscope}%
\begin{pgfscope}%
\pgfpathrectangle{\pgfqpoint{0.515000in}{0.499444in}}{\pgfqpoint{3.487500in}{1.155000in}}%
\pgfusepath{clip}%
\pgfsetbuttcap%
\pgfsetmiterjoin%
\definecolor{currentfill}{rgb}{0.000000,0.000000,0.000000}%
\pgfsetfillcolor{currentfill}%
\pgfsetlinewidth{0.000000pt}%
\definecolor{currentstroke}{rgb}{0.000000,0.000000,0.000000}%
\pgfsetstrokecolor{currentstroke}%
\pgfsetstrokeopacity{0.000000}%
\pgfsetdash{}{0pt}%
\pgfpathmoveto{\pgfqpoint{2.206956in}{0.499444in}}%
\pgfpathlineto{\pgfqpoint{2.262203in}{0.499444in}}%
\pgfpathlineto{\pgfqpoint{2.262203in}{0.652928in}}%
\pgfpathlineto{\pgfqpoint{2.206956in}{0.652928in}}%
\pgfpathlineto{\pgfqpoint{2.206956in}{0.499444in}}%
\pgfpathclose%
\pgfusepath{fill}%
\end{pgfscope}%
\begin{pgfscope}%
\pgfpathrectangle{\pgfqpoint{0.515000in}{0.499444in}}{\pgfqpoint{3.487500in}{1.155000in}}%
\pgfusepath{clip}%
\pgfsetbuttcap%
\pgfsetmiterjoin%
\definecolor{currentfill}{rgb}{0.000000,0.000000,0.000000}%
\pgfsetfillcolor{currentfill}%
\pgfsetlinewidth{0.000000pt}%
\definecolor{currentstroke}{rgb}{0.000000,0.000000,0.000000}%
\pgfsetstrokecolor{currentstroke}%
\pgfsetstrokeopacity{0.000000}%
\pgfsetdash{}{0pt}%
\pgfpathmoveto{\pgfqpoint{2.345075in}{0.499444in}}%
\pgfpathlineto{\pgfqpoint{2.400322in}{0.499444in}}%
\pgfpathlineto{\pgfqpoint{2.400322in}{0.500782in}}%
\pgfpathlineto{\pgfqpoint{2.345075in}{0.500782in}}%
\pgfpathlineto{\pgfqpoint{2.345075in}{0.499444in}}%
\pgfpathclose%
\pgfusepath{fill}%
\end{pgfscope}%
\begin{pgfscope}%
\pgfpathrectangle{\pgfqpoint{0.515000in}{0.499444in}}{\pgfqpoint{3.487500in}{1.155000in}}%
\pgfusepath{clip}%
\pgfsetbuttcap%
\pgfsetmiterjoin%
\definecolor{currentfill}{rgb}{0.000000,0.000000,0.000000}%
\pgfsetfillcolor{currentfill}%
\pgfsetlinewidth{0.000000pt}%
\definecolor{currentstroke}{rgb}{0.000000,0.000000,0.000000}%
\pgfsetstrokecolor{currentstroke}%
\pgfsetstrokeopacity{0.000000}%
\pgfsetdash{}{0pt}%
\pgfpathmoveto{\pgfqpoint{2.483193in}{0.499444in}}%
\pgfpathlineto{\pgfqpoint{2.538441in}{0.499444in}}%
\pgfpathlineto{\pgfqpoint{2.538441in}{0.656943in}}%
\pgfpathlineto{\pgfqpoint{2.483193in}{0.656943in}}%
\pgfpathlineto{\pgfqpoint{2.483193in}{0.499444in}}%
\pgfpathclose%
\pgfusepath{fill}%
\end{pgfscope}%
\begin{pgfscope}%
\pgfpathrectangle{\pgfqpoint{0.515000in}{0.499444in}}{\pgfqpoint{3.487500in}{1.155000in}}%
\pgfusepath{clip}%
\pgfsetbuttcap%
\pgfsetmiterjoin%
\definecolor{currentfill}{rgb}{0.000000,0.000000,0.000000}%
\pgfsetfillcolor{currentfill}%
\pgfsetlinewidth{0.000000pt}%
\definecolor{currentstroke}{rgb}{0.000000,0.000000,0.000000}%
\pgfsetstrokecolor{currentstroke}%
\pgfsetstrokeopacity{0.000000}%
\pgfsetdash{}{0pt}%
\pgfpathmoveto{\pgfqpoint{2.621312in}{0.499444in}}%
\pgfpathlineto{\pgfqpoint{2.676560in}{0.499444in}}%
\pgfpathlineto{\pgfqpoint{2.676560in}{0.499890in}}%
\pgfpathlineto{\pgfqpoint{2.621312in}{0.499890in}}%
\pgfpathlineto{\pgfqpoint{2.621312in}{0.499444in}}%
\pgfpathclose%
\pgfusepath{fill}%
\end{pgfscope}%
\begin{pgfscope}%
\pgfpathrectangle{\pgfqpoint{0.515000in}{0.499444in}}{\pgfqpoint{3.487500in}{1.155000in}}%
\pgfusepath{clip}%
\pgfsetbuttcap%
\pgfsetmiterjoin%
\definecolor{currentfill}{rgb}{0.000000,0.000000,0.000000}%
\pgfsetfillcolor{currentfill}%
\pgfsetlinewidth{0.000000pt}%
\definecolor{currentstroke}{rgb}{0.000000,0.000000,0.000000}%
\pgfsetstrokecolor{currentstroke}%
\pgfsetstrokeopacity{0.000000}%
\pgfsetdash{}{0pt}%
\pgfpathmoveto{\pgfqpoint{2.759431in}{0.499444in}}%
\pgfpathlineto{\pgfqpoint{2.814678in}{0.499444in}}%
\pgfpathlineto{\pgfqpoint{2.814678in}{0.502046in}}%
\pgfpathlineto{\pgfqpoint{2.759431in}{0.502046in}}%
\pgfpathlineto{\pgfqpoint{2.759431in}{0.499444in}}%
\pgfpathclose%
\pgfusepath{fill}%
\end{pgfscope}%
\begin{pgfscope}%
\pgfpathrectangle{\pgfqpoint{0.515000in}{0.499444in}}{\pgfqpoint{3.487500in}{1.155000in}}%
\pgfusepath{clip}%
\pgfsetbuttcap%
\pgfsetmiterjoin%
\definecolor{currentfill}{rgb}{0.000000,0.000000,0.000000}%
\pgfsetfillcolor{currentfill}%
\pgfsetlinewidth{0.000000pt}%
\definecolor{currentstroke}{rgb}{0.000000,0.000000,0.000000}%
\pgfsetstrokecolor{currentstroke}%
\pgfsetstrokeopacity{0.000000}%
\pgfsetdash{}{0pt}%
\pgfpathmoveto{\pgfqpoint{2.897550in}{0.499444in}}%
\pgfpathlineto{\pgfqpoint{2.952797in}{0.499444in}}%
\pgfpathlineto{\pgfqpoint{2.952797in}{0.658578in}}%
\pgfpathlineto{\pgfqpoint{2.897550in}{0.658578in}}%
\pgfpathlineto{\pgfqpoint{2.897550in}{0.499444in}}%
\pgfpathclose%
\pgfusepath{fill}%
\end{pgfscope}%
\begin{pgfscope}%
\pgfpathrectangle{\pgfqpoint{0.515000in}{0.499444in}}{\pgfqpoint{3.487500in}{1.155000in}}%
\pgfusepath{clip}%
\pgfsetbuttcap%
\pgfsetmiterjoin%
\definecolor{currentfill}{rgb}{0.000000,0.000000,0.000000}%
\pgfsetfillcolor{currentfill}%
\pgfsetlinewidth{0.000000pt}%
\definecolor{currentstroke}{rgb}{0.000000,0.000000,0.000000}%
\pgfsetstrokecolor{currentstroke}%
\pgfsetstrokeopacity{0.000000}%
\pgfsetdash{}{0pt}%
\pgfpathmoveto{\pgfqpoint{3.035669in}{0.499444in}}%
\pgfpathlineto{\pgfqpoint{3.090916in}{0.499444in}}%
\pgfpathlineto{\pgfqpoint{3.090916in}{0.500782in}}%
\pgfpathlineto{\pgfqpoint{3.035669in}{0.500782in}}%
\pgfpathlineto{\pgfqpoint{3.035669in}{0.499444in}}%
\pgfpathclose%
\pgfusepath{fill}%
\end{pgfscope}%
\begin{pgfscope}%
\pgfpathrectangle{\pgfqpoint{0.515000in}{0.499444in}}{\pgfqpoint{3.487500in}{1.155000in}}%
\pgfusepath{clip}%
\pgfsetbuttcap%
\pgfsetmiterjoin%
\definecolor{currentfill}{rgb}{0.000000,0.000000,0.000000}%
\pgfsetfillcolor{currentfill}%
\pgfsetlinewidth{0.000000pt}%
\definecolor{currentstroke}{rgb}{0.000000,0.000000,0.000000}%
\pgfsetstrokecolor{currentstroke}%
\pgfsetstrokeopacity{0.000000}%
\pgfsetdash{}{0pt}%
\pgfpathmoveto{\pgfqpoint{3.173787in}{0.499444in}}%
\pgfpathlineto{\pgfqpoint{3.229035in}{0.499444in}}%
\pgfpathlineto{\pgfqpoint{3.229035in}{0.644156in}}%
\pgfpathlineto{\pgfqpoint{3.173787in}{0.644156in}}%
\pgfpathlineto{\pgfqpoint{3.173787in}{0.499444in}}%
\pgfpathclose%
\pgfusepath{fill}%
\end{pgfscope}%
\begin{pgfscope}%
\pgfpathrectangle{\pgfqpoint{0.515000in}{0.499444in}}{\pgfqpoint{3.487500in}{1.155000in}}%
\pgfusepath{clip}%
\pgfsetbuttcap%
\pgfsetmiterjoin%
\definecolor{currentfill}{rgb}{0.000000,0.000000,0.000000}%
\pgfsetfillcolor{currentfill}%
\pgfsetlinewidth{0.000000pt}%
\definecolor{currentstroke}{rgb}{0.000000,0.000000,0.000000}%
\pgfsetstrokecolor{currentstroke}%
\pgfsetstrokeopacity{0.000000}%
\pgfsetdash{}{0pt}%
\pgfpathmoveto{\pgfqpoint{3.311906in}{0.499444in}}%
\pgfpathlineto{\pgfqpoint{3.367154in}{0.499444in}}%
\pgfpathlineto{\pgfqpoint{3.367154in}{0.500002in}}%
\pgfpathlineto{\pgfqpoint{3.311906in}{0.500002in}}%
\pgfpathlineto{\pgfqpoint{3.311906in}{0.499444in}}%
\pgfpathclose%
\pgfusepath{fill}%
\end{pgfscope}%
\begin{pgfscope}%
\pgfpathrectangle{\pgfqpoint{0.515000in}{0.499444in}}{\pgfqpoint{3.487500in}{1.155000in}}%
\pgfusepath{clip}%
\pgfsetbuttcap%
\pgfsetmiterjoin%
\definecolor{currentfill}{rgb}{0.000000,0.000000,0.000000}%
\pgfsetfillcolor{currentfill}%
\pgfsetlinewidth{0.000000pt}%
\definecolor{currentstroke}{rgb}{0.000000,0.000000,0.000000}%
\pgfsetstrokecolor{currentstroke}%
\pgfsetstrokeopacity{0.000000}%
\pgfsetdash{}{0pt}%
\pgfpathmoveto{\pgfqpoint{3.450025in}{0.499444in}}%
\pgfpathlineto{\pgfqpoint{3.505273in}{0.499444in}}%
\pgfpathlineto{\pgfqpoint{3.505273in}{0.500968in}}%
\pgfpathlineto{\pgfqpoint{3.450025in}{0.500968in}}%
\pgfpathlineto{\pgfqpoint{3.450025in}{0.499444in}}%
\pgfpathclose%
\pgfusepath{fill}%
\end{pgfscope}%
\begin{pgfscope}%
\pgfpathrectangle{\pgfqpoint{0.515000in}{0.499444in}}{\pgfqpoint{3.487500in}{1.155000in}}%
\pgfusepath{clip}%
\pgfsetbuttcap%
\pgfsetmiterjoin%
\definecolor{currentfill}{rgb}{0.000000,0.000000,0.000000}%
\pgfsetfillcolor{currentfill}%
\pgfsetlinewidth{0.000000pt}%
\definecolor{currentstroke}{rgb}{0.000000,0.000000,0.000000}%
\pgfsetstrokecolor{currentstroke}%
\pgfsetstrokeopacity{0.000000}%
\pgfsetdash{}{0pt}%
\pgfpathmoveto{\pgfqpoint{3.588144in}{0.499444in}}%
\pgfpathlineto{\pgfqpoint{3.643391in}{0.499444in}}%
\pgfpathlineto{\pgfqpoint{3.643391in}{0.621704in}}%
\pgfpathlineto{\pgfqpoint{3.588144in}{0.621704in}}%
\pgfpathlineto{\pgfqpoint{3.588144in}{0.499444in}}%
\pgfpathclose%
\pgfusepath{fill}%
\end{pgfscope}%
\begin{pgfscope}%
\pgfpathrectangle{\pgfqpoint{0.515000in}{0.499444in}}{\pgfqpoint{3.487500in}{1.155000in}}%
\pgfusepath{clip}%
\pgfsetbuttcap%
\pgfsetmiterjoin%
\definecolor{currentfill}{rgb}{0.000000,0.000000,0.000000}%
\pgfsetfillcolor{currentfill}%
\pgfsetlinewidth{0.000000pt}%
\definecolor{currentstroke}{rgb}{0.000000,0.000000,0.000000}%
\pgfsetstrokecolor{currentstroke}%
\pgfsetstrokeopacity{0.000000}%
\pgfsetdash{}{0pt}%
\pgfpathmoveto{\pgfqpoint{3.726263in}{0.499444in}}%
\pgfpathlineto{\pgfqpoint{3.781510in}{0.499444in}}%
\pgfpathlineto{\pgfqpoint{3.781510in}{0.499965in}}%
\pgfpathlineto{\pgfqpoint{3.726263in}{0.499965in}}%
\pgfpathlineto{\pgfqpoint{3.726263in}{0.499444in}}%
\pgfpathclose%
\pgfusepath{fill}%
\end{pgfscope}%
\begin{pgfscope}%
\pgfpathrectangle{\pgfqpoint{0.515000in}{0.499444in}}{\pgfqpoint{3.487500in}{1.155000in}}%
\pgfusepath{clip}%
\pgfsetbuttcap%
\pgfsetmiterjoin%
\definecolor{currentfill}{rgb}{0.000000,0.000000,0.000000}%
\pgfsetfillcolor{currentfill}%
\pgfsetlinewidth{0.000000pt}%
\definecolor{currentstroke}{rgb}{0.000000,0.000000,0.000000}%
\pgfsetstrokecolor{currentstroke}%
\pgfsetstrokeopacity{0.000000}%
\pgfsetdash{}{0pt}%
\pgfpathmoveto{\pgfqpoint{3.864381in}{0.499444in}}%
\pgfpathlineto{\pgfqpoint{3.919629in}{0.499444in}}%
\pgfpathlineto{\pgfqpoint{3.919629in}{0.576316in}}%
\pgfpathlineto{\pgfqpoint{3.864381in}{0.576316in}}%
\pgfpathlineto{\pgfqpoint{3.864381in}{0.499444in}}%
\pgfpathclose%
\pgfusepath{fill}%
\end{pgfscope}%
\begin{pgfscope}%
\pgfsetbuttcap%
\pgfsetroundjoin%
\definecolor{currentfill}{rgb}{0.000000,0.000000,0.000000}%
\pgfsetfillcolor{currentfill}%
\pgfsetlinewidth{0.803000pt}%
\definecolor{currentstroke}{rgb}{0.000000,0.000000,0.000000}%
\pgfsetstrokecolor{currentstroke}%
\pgfsetdash{}{0pt}%
\pgfsys@defobject{currentmarker}{\pgfqpoint{0.000000in}{-0.048611in}}{\pgfqpoint{0.000000in}{0.000000in}}{%
\pgfpathmoveto{\pgfqpoint{0.000000in}{0.000000in}}%
\pgfpathlineto{\pgfqpoint{0.000000in}{-0.048611in}}%
\pgfusepath{stroke,fill}%
}%
\begin{pgfscope}%
\pgfsys@transformshift{0.549530in}{0.499444in}%
\pgfsys@useobject{currentmarker}{}%
\end{pgfscope}%
\end{pgfscope}%
\begin{pgfscope}%
\definecolor{textcolor}{rgb}{0.000000,0.000000,0.000000}%
\pgfsetstrokecolor{textcolor}%
\pgfsetfillcolor{textcolor}%
\pgftext[x=0.549530in,y=0.402222in,,top]{\color{textcolor}\rmfamily\fontsize{10.000000}{12.000000}\selectfont 0.0}%
\end{pgfscope}%
\begin{pgfscope}%
\pgfsetbuttcap%
\pgfsetroundjoin%
\definecolor{currentfill}{rgb}{0.000000,0.000000,0.000000}%
\pgfsetfillcolor{currentfill}%
\pgfsetlinewidth{0.803000pt}%
\definecolor{currentstroke}{rgb}{0.000000,0.000000,0.000000}%
\pgfsetstrokecolor{currentstroke}%
\pgfsetdash{}{0pt}%
\pgfsys@defobject{currentmarker}{\pgfqpoint{0.000000in}{-0.048611in}}{\pgfqpoint{0.000000in}{0.000000in}}{%
\pgfpathmoveto{\pgfqpoint{0.000000in}{0.000000in}}%
\pgfpathlineto{\pgfqpoint{0.000000in}{-0.048611in}}%
\pgfusepath{stroke,fill}%
}%
\begin{pgfscope}%
\pgfsys@transformshift{0.894827in}{0.499444in}%
\pgfsys@useobject{currentmarker}{}%
\end{pgfscope}%
\end{pgfscope}%
\begin{pgfscope}%
\definecolor{textcolor}{rgb}{0.000000,0.000000,0.000000}%
\pgfsetstrokecolor{textcolor}%
\pgfsetfillcolor{textcolor}%
\pgftext[x=0.894827in,y=0.402222in,,top]{\color{textcolor}\rmfamily\fontsize{10.000000}{12.000000}\selectfont 0.1}%
\end{pgfscope}%
\begin{pgfscope}%
\pgfsetbuttcap%
\pgfsetroundjoin%
\definecolor{currentfill}{rgb}{0.000000,0.000000,0.000000}%
\pgfsetfillcolor{currentfill}%
\pgfsetlinewidth{0.803000pt}%
\definecolor{currentstroke}{rgb}{0.000000,0.000000,0.000000}%
\pgfsetstrokecolor{currentstroke}%
\pgfsetdash{}{0pt}%
\pgfsys@defobject{currentmarker}{\pgfqpoint{0.000000in}{-0.048611in}}{\pgfqpoint{0.000000in}{0.000000in}}{%
\pgfpathmoveto{\pgfqpoint{0.000000in}{0.000000in}}%
\pgfpathlineto{\pgfqpoint{0.000000in}{-0.048611in}}%
\pgfusepath{stroke,fill}%
}%
\begin{pgfscope}%
\pgfsys@transformshift{1.240124in}{0.499444in}%
\pgfsys@useobject{currentmarker}{}%
\end{pgfscope}%
\end{pgfscope}%
\begin{pgfscope}%
\definecolor{textcolor}{rgb}{0.000000,0.000000,0.000000}%
\pgfsetstrokecolor{textcolor}%
\pgfsetfillcolor{textcolor}%
\pgftext[x=1.240124in,y=0.402222in,,top]{\color{textcolor}\rmfamily\fontsize{10.000000}{12.000000}\selectfont 0.2}%
\end{pgfscope}%
\begin{pgfscope}%
\pgfsetbuttcap%
\pgfsetroundjoin%
\definecolor{currentfill}{rgb}{0.000000,0.000000,0.000000}%
\pgfsetfillcolor{currentfill}%
\pgfsetlinewidth{0.803000pt}%
\definecolor{currentstroke}{rgb}{0.000000,0.000000,0.000000}%
\pgfsetstrokecolor{currentstroke}%
\pgfsetdash{}{0pt}%
\pgfsys@defobject{currentmarker}{\pgfqpoint{0.000000in}{-0.048611in}}{\pgfqpoint{0.000000in}{0.000000in}}{%
\pgfpathmoveto{\pgfqpoint{0.000000in}{0.000000in}}%
\pgfpathlineto{\pgfqpoint{0.000000in}{-0.048611in}}%
\pgfusepath{stroke,fill}%
}%
\begin{pgfscope}%
\pgfsys@transformshift{1.585421in}{0.499444in}%
\pgfsys@useobject{currentmarker}{}%
\end{pgfscope}%
\end{pgfscope}%
\begin{pgfscope}%
\definecolor{textcolor}{rgb}{0.000000,0.000000,0.000000}%
\pgfsetstrokecolor{textcolor}%
\pgfsetfillcolor{textcolor}%
\pgftext[x=1.585421in,y=0.402222in,,top]{\color{textcolor}\rmfamily\fontsize{10.000000}{12.000000}\selectfont 0.3}%
\end{pgfscope}%
\begin{pgfscope}%
\pgfsetbuttcap%
\pgfsetroundjoin%
\definecolor{currentfill}{rgb}{0.000000,0.000000,0.000000}%
\pgfsetfillcolor{currentfill}%
\pgfsetlinewidth{0.803000pt}%
\definecolor{currentstroke}{rgb}{0.000000,0.000000,0.000000}%
\pgfsetstrokecolor{currentstroke}%
\pgfsetdash{}{0pt}%
\pgfsys@defobject{currentmarker}{\pgfqpoint{0.000000in}{-0.048611in}}{\pgfqpoint{0.000000in}{0.000000in}}{%
\pgfpathmoveto{\pgfqpoint{0.000000in}{0.000000in}}%
\pgfpathlineto{\pgfqpoint{0.000000in}{-0.048611in}}%
\pgfusepath{stroke,fill}%
}%
\begin{pgfscope}%
\pgfsys@transformshift{1.930718in}{0.499444in}%
\pgfsys@useobject{currentmarker}{}%
\end{pgfscope}%
\end{pgfscope}%
\begin{pgfscope}%
\definecolor{textcolor}{rgb}{0.000000,0.000000,0.000000}%
\pgfsetstrokecolor{textcolor}%
\pgfsetfillcolor{textcolor}%
\pgftext[x=1.930718in,y=0.402222in,,top]{\color{textcolor}\rmfamily\fontsize{10.000000}{12.000000}\selectfont 0.4}%
\end{pgfscope}%
\begin{pgfscope}%
\pgfsetbuttcap%
\pgfsetroundjoin%
\definecolor{currentfill}{rgb}{0.000000,0.000000,0.000000}%
\pgfsetfillcolor{currentfill}%
\pgfsetlinewidth{0.803000pt}%
\definecolor{currentstroke}{rgb}{0.000000,0.000000,0.000000}%
\pgfsetstrokecolor{currentstroke}%
\pgfsetdash{}{0pt}%
\pgfsys@defobject{currentmarker}{\pgfqpoint{0.000000in}{-0.048611in}}{\pgfqpoint{0.000000in}{0.000000in}}{%
\pgfpathmoveto{\pgfqpoint{0.000000in}{0.000000in}}%
\pgfpathlineto{\pgfqpoint{0.000000in}{-0.048611in}}%
\pgfusepath{stroke,fill}%
}%
\begin{pgfscope}%
\pgfsys@transformshift{2.276015in}{0.499444in}%
\pgfsys@useobject{currentmarker}{}%
\end{pgfscope}%
\end{pgfscope}%
\begin{pgfscope}%
\definecolor{textcolor}{rgb}{0.000000,0.000000,0.000000}%
\pgfsetstrokecolor{textcolor}%
\pgfsetfillcolor{textcolor}%
\pgftext[x=2.276015in,y=0.402222in,,top]{\color{textcolor}\rmfamily\fontsize{10.000000}{12.000000}\selectfont 0.5}%
\end{pgfscope}%
\begin{pgfscope}%
\pgfsetbuttcap%
\pgfsetroundjoin%
\definecolor{currentfill}{rgb}{0.000000,0.000000,0.000000}%
\pgfsetfillcolor{currentfill}%
\pgfsetlinewidth{0.803000pt}%
\definecolor{currentstroke}{rgb}{0.000000,0.000000,0.000000}%
\pgfsetstrokecolor{currentstroke}%
\pgfsetdash{}{0pt}%
\pgfsys@defobject{currentmarker}{\pgfqpoint{0.000000in}{-0.048611in}}{\pgfqpoint{0.000000in}{0.000000in}}{%
\pgfpathmoveto{\pgfqpoint{0.000000in}{0.000000in}}%
\pgfpathlineto{\pgfqpoint{0.000000in}{-0.048611in}}%
\pgfusepath{stroke,fill}%
}%
\begin{pgfscope}%
\pgfsys@transformshift{2.621312in}{0.499444in}%
\pgfsys@useobject{currentmarker}{}%
\end{pgfscope}%
\end{pgfscope}%
\begin{pgfscope}%
\definecolor{textcolor}{rgb}{0.000000,0.000000,0.000000}%
\pgfsetstrokecolor{textcolor}%
\pgfsetfillcolor{textcolor}%
\pgftext[x=2.621312in,y=0.402222in,,top]{\color{textcolor}\rmfamily\fontsize{10.000000}{12.000000}\selectfont 0.6}%
\end{pgfscope}%
\begin{pgfscope}%
\pgfsetbuttcap%
\pgfsetroundjoin%
\definecolor{currentfill}{rgb}{0.000000,0.000000,0.000000}%
\pgfsetfillcolor{currentfill}%
\pgfsetlinewidth{0.803000pt}%
\definecolor{currentstroke}{rgb}{0.000000,0.000000,0.000000}%
\pgfsetstrokecolor{currentstroke}%
\pgfsetdash{}{0pt}%
\pgfsys@defobject{currentmarker}{\pgfqpoint{0.000000in}{-0.048611in}}{\pgfqpoint{0.000000in}{0.000000in}}{%
\pgfpathmoveto{\pgfqpoint{0.000000in}{0.000000in}}%
\pgfpathlineto{\pgfqpoint{0.000000in}{-0.048611in}}%
\pgfusepath{stroke,fill}%
}%
\begin{pgfscope}%
\pgfsys@transformshift{2.966609in}{0.499444in}%
\pgfsys@useobject{currentmarker}{}%
\end{pgfscope}%
\end{pgfscope}%
\begin{pgfscope}%
\definecolor{textcolor}{rgb}{0.000000,0.000000,0.000000}%
\pgfsetstrokecolor{textcolor}%
\pgfsetfillcolor{textcolor}%
\pgftext[x=2.966609in,y=0.402222in,,top]{\color{textcolor}\rmfamily\fontsize{10.000000}{12.000000}\selectfont 0.7}%
\end{pgfscope}%
\begin{pgfscope}%
\pgfsetbuttcap%
\pgfsetroundjoin%
\definecolor{currentfill}{rgb}{0.000000,0.000000,0.000000}%
\pgfsetfillcolor{currentfill}%
\pgfsetlinewidth{0.803000pt}%
\definecolor{currentstroke}{rgb}{0.000000,0.000000,0.000000}%
\pgfsetstrokecolor{currentstroke}%
\pgfsetdash{}{0pt}%
\pgfsys@defobject{currentmarker}{\pgfqpoint{0.000000in}{-0.048611in}}{\pgfqpoint{0.000000in}{0.000000in}}{%
\pgfpathmoveto{\pgfqpoint{0.000000in}{0.000000in}}%
\pgfpathlineto{\pgfqpoint{0.000000in}{-0.048611in}}%
\pgfusepath{stroke,fill}%
}%
\begin{pgfscope}%
\pgfsys@transformshift{3.311906in}{0.499444in}%
\pgfsys@useobject{currentmarker}{}%
\end{pgfscope}%
\end{pgfscope}%
\begin{pgfscope}%
\definecolor{textcolor}{rgb}{0.000000,0.000000,0.000000}%
\pgfsetstrokecolor{textcolor}%
\pgfsetfillcolor{textcolor}%
\pgftext[x=3.311906in,y=0.402222in,,top]{\color{textcolor}\rmfamily\fontsize{10.000000}{12.000000}\selectfont 0.8}%
\end{pgfscope}%
\begin{pgfscope}%
\pgfsetbuttcap%
\pgfsetroundjoin%
\definecolor{currentfill}{rgb}{0.000000,0.000000,0.000000}%
\pgfsetfillcolor{currentfill}%
\pgfsetlinewidth{0.803000pt}%
\definecolor{currentstroke}{rgb}{0.000000,0.000000,0.000000}%
\pgfsetstrokecolor{currentstroke}%
\pgfsetdash{}{0pt}%
\pgfsys@defobject{currentmarker}{\pgfqpoint{0.000000in}{-0.048611in}}{\pgfqpoint{0.000000in}{0.000000in}}{%
\pgfpathmoveto{\pgfqpoint{0.000000in}{0.000000in}}%
\pgfpathlineto{\pgfqpoint{0.000000in}{-0.048611in}}%
\pgfusepath{stroke,fill}%
}%
\begin{pgfscope}%
\pgfsys@transformshift{3.657203in}{0.499444in}%
\pgfsys@useobject{currentmarker}{}%
\end{pgfscope}%
\end{pgfscope}%
\begin{pgfscope}%
\definecolor{textcolor}{rgb}{0.000000,0.000000,0.000000}%
\pgfsetstrokecolor{textcolor}%
\pgfsetfillcolor{textcolor}%
\pgftext[x=3.657203in,y=0.402222in,,top]{\color{textcolor}\rmfamily\fontsize{10.000000}{12.000000}\selectfont 0.9}%
\end{pgfscope}%
\begin{pgfscope}%
\pgfsetbuttcap%
\pgfsetroundjoin%
\definecolor{currentfill}{rgb}{0.000000,0.000000,0.000000}%
\pgfsetfillcolor{currentfill}%
\pgfsetlinewidth{0.803000pt}%
\definecolor{currentstroke}{rgb}{0.000000,0.000000,0.000000}%
\pgfsetstrokecolor{currentstroke}%
\pgfsetdash{}{0pt}%
\pgfsys@defobject{currentmarker}{\pgfqpoint{0.000000in}{-0.048611in}}{\pgfqpoint{0.000000in}{0.000000in}}{%
\pgfpathmoveto{\pgfqpoint{0.000000in}{0.000000in}}%
\pgfpathlineto{\pgfqpoint{0.000000in}{-0.048611in}}%
\pgfusepath{stroke,fill}%
}%
\begin{pgfscope}%
\pgfsys@transformshift{4.002500in}{0.499444in}%
\pgfsys@useobject{currentmarker}{}%
\end{pgfscope}%
\end{pgfscope}%
\begin{pgfscope}%
\definecolor{textcolor}{rgb}{0.000000,0.000000,0.000000}%
\pgfsetstrokecolor{textcolor}%
\pgfsetfillcolor{textcolor}%
\pgftext[x=4.002500in,y=0.402222in,,top]{\color{textcolor}\rmfamily\fontsize{10.000000}{12.000000}\selectfont 1.0}%
\end{pgfscope}%
\begin{pgfscope}%
\definecolor{textcolor}{rgb}{0.000000,0.000000,0.000000}%
\pgfsetstrokecolor{textcolor}%
\pgfsetfillcolor{textcolor}%
\pgftext[x=2.258750in,y=0.223333in,,top]{\color{textcolor}\rmfamily\fontsize{10.000000}{12.000000}\selectfont \(\displaystyle p\)}%
\end{pgfscope}%
\begin{pgfscope}%
\pgfsetbuttcap%
\pgfsetroundjoin%
\definecolor{currentfill}{rgb}{0.000000,0.000000,0.000000}%
\pgfsetfillcolor{currentfill}%
\pgfsetlinewidth{0.803000pt}%
\definecolor{currentstroke}{rgb}{0.000000,0.000000,0.000000}%
\pgfsetstrokecolor{currentstroke}%
\pgfsetdash{}{0pt}%
\pgfsys@defobject{currentmarker}{\pgfqpoint{-0.048611in}{0.000000in}}{\pgfqpoint{-0.000000in}{0.000000in}}{%
\pgfpathmoveto{\pgfqpoint{-0.000000in}{0.000000in}}%
\pgfpathlineto{\pgfqpoint{-0.048611in}{0.000000in}}%
\pgfusepath{stroke,fill}%
}%
\begin{pgfscope}%
\pgfsys@transformshift{0.515000in}{0.499444in}%
\pgfsys@useobject{currentmarker}{}%
\end{pgfscope}%
\end{pgfscope}%
\begin{pgfscope}%
\definecolor{textcolor}{rgb}{0.000000,0.000000,0.000000}%
\pgfsetstrokecolor{textcolor}%
\pgfsetfillcolor{textcolor}%
\pgftext[x=0.348333in, y=0.451250in, left, base]{\color{textcolor}\rmfamily\fontsize{10.000000}{12.000000}\selectfont \(\displaystyle {0}\)}%
\end{pgfscope}%
\begin{pgfscope}%
\pgfsetbuttcap%
\pgfsetroundjoin%
\definecolor{currentfill}{rgb}{0.000000,0.000000,0.000000}%
\pgfsetfillcolor{currentfill}%
\pgfsetlinewidth{0.803000pt}%
\definecolor{currentstroke}{rgb}{0.000000,0.000000,0.000000}%
\pgfsetstrokecolor{currentstroke}%
\pgfsetdash{}{0pt}%
\pgfsys@defobject{currentmarker}{\pgfqpoint{-0.048611in}{0.000000in}}{\pgfqpoint{-0.000000in}{0.000000in}}{%
\pgfpathmoveto{\pgfqpoint{-0.000000in}{0.000000in}}%
\pgfpathlineto{\pgfqpoint{-0.048611in}{0.000000in}}%
\pgfusepath{stroke,fill}%
}%
\begin{pgfscope}%
\pgfsys@transformshift{0.515000in}{0.897317in}%
\pgfsys@useobject{currentmarker}{}%
\end{pgfscope}%
\end{pgfscope}%
\begin{pgfscope}%
\definecolor{textcolor}{rgb}{0.000000,0.000000,0.000000}%
\pgfsetstrokecolor{textcolor}%
\pgfsetfillcolor{textcolor}%
\pgftext[x=0.348333in, y=0.849122in, left, base]{\color{textcolor}\rmfamily\fontsize{10.000000}{12.000000}\selectfont \(\displaystyle {5}\)}%
\end{pgfscope}%
\begin{pgfscope}%
\pgfsetbuttcap%
\pgfsetroundjoin%
\definecolor{currentfill}{rgb}{0.000000,0.000000,0.000000}%
\pgfsetfillcolor{currentfill}%
\pgfsetlinewidth{0.803000pt}%
\definecolor{currentstroke}{rgb}{0.000000,0.000000,0.000000}%
\pgfsetstrokecolor{currentstroke}%
\pgfsetdash{}{0pt}%
\pgfsys@defobject{currentmarker}{\pgfqpoint{-0.048611in}{0.000000in}}{\pgfqpoint{-0.000000in}{0.000000in}}{%
\pgfpathmoveto{\pgfqpoint{-0.000000in}{0.000000in}}%
\pgfpathlineto{\pgfqpoint{-0.048611in}{0.000000in}}%
\pgfusepath{stroke,fill}%
}%
\begin{pgfscope}%
\pgfsys@transformshift{0.515000in}{1.295190in}%
\pgfsys@useobject{currentmarker}{}%
\end{pgfscope}%
\end{pgfscope}%
\begin{pgfscope}%
\definecolor{textcolor}{rgb}{0.000000,0.000000,0.000000}%
\pgfsetstrokecolor{textcolor}%
\pgfsetfillcolor{textcolor}%
\pgftext[x=0.278889in, y=1.246995in, left, base]{\color{textcolor}\rmfamily\fontsize{10.000000}{12.000000}\selectfont \(\displaystyle {10}\)}%
\end{pgfscope}%
\begin{pgfscope}%
\definecolor{textcolor}{rgb}{0.000000,0.000000,0.000000}%
\pgfsetstrokecolor{textcolor}%
\pgfsetfillcolor{textcolor}%
\pgftext[x=0.223333in,y=1.076944in,,bottom,rotate=90.000000]{\color{textcolor}\rmfamily\fontsize{10.000000}{12.000000}\selectfont Percent of Data Set}%
\end{pgfscope}%
\begin{pgfscope}%
\pgfsetrectcap%
\pgfsetmiterjoin%
\pgfsetlinewidth{0.803000pt}%
\definecolor{currentstroke}{rgb}{0.000000,0.000000,0.000000}%
\pgfsetstrokecolor{currentstroke}%
\pgfsetdash{}{0pt}%
\pgfpathmoveto{\pgfqpoint{0.515000in}{0.499444in}}%
\pgfpathlineto{\pgfqpoint{0.515000in}{1.654444in}}%
\pgfusepath{stroke}%
\end{pgfscope}%
\begin{pgfscope}%
\pgfsetrectcap%
\pgfsetmiterjoin%
\pgfsetlinewidth{0.803000pt}%
\definecolor{currentstroke}{rgb}{0.000000,0.000000,0.000000}%
\pgfsetstrokecolor{currentstroke}%
\pgfsetdash{}{0pt}%
\pgfpathmoveto{\pgfqpoint{4.002500in}{0.499444in}}%
\pgfpathlineto{\pgfqpoint{4.002500in}{1.654444in}}%
\pgfusepath{stroke}%
\end{pgfscope}%
\begin{pgfscope}%
\pgfsetrectcap%
\pgfsetmiterjoin%
\pgfsetlinewidth{0.803000pt}%
\definecolor{currentstroke}{rgb}{0.000000,0.000000,0.000000}%
\pgfsetstrokecolor{currentstroke}%
\pgfsetdash{}{0pt}%
\pgfpathmoveto{\pgfqpoint{0.515000in}{0.499444in}}%
\pgfpathlineto{\pgfqpoint{4.002500in}{0.499444in}}%
\pgfusepath{stroke}%
\end{pgfscope}%
\begin{pgfscope}%
\pgfsetrectcap%
\pgfsetmiterjoin%
\pgfsetlinewidth{0.803000pt}%
\definecolor{currentstroke}{rgb}{0.000000,0.000000,0.000000}%
\pgfsetstrokecolor{currentstroke}%
\pgfsetdash{}{0pt}%
\pgfpathmoveto{\pgfqpoint{0.515000in}{1.654444in}}%
\pgfpathlineto{\pgfqpoint{4.002500in}{1.654444in}}%
\pgfusepath{stroke}%
\end{pgfscope}%
\begin{pgfscope}%
\pgfsetbuttcap%
\pgfsetmiterjoin%
\definecolor{currentfill}{rgb}{1.000000,1.000000,1.000000}%
\pgfsetfillcolor{currentfill}%
\pgfsetfillopacity{0.800000}%
\pgfsetlinewidth{1.003750pt}%
\definecolor{currentstroke}{rgb}{0.800000,0.800000,0.800000}%
\pgfsetstrokecolor{currentstroke}%
\pgfsetstrokeopacity{0.800000}%
\pgfsetdash{}{0pt}%
\pgfpathmoveto{\pgfqpoint{3.225556in}{1.154445in}}%
\pgfpathlineto{\pgfqpoint{3.905278in}{1.154445in}}%
\pgfpathquadraticcurveto{\pgfqpoint{3.933056in}{1.154445in}}{\pgfqpoint{3.933056in}{1.182222in}}%
\pgfpathlineto{\pgfqpoint{3.933056in}{1.557222in}}%
\pgfpathquadraticcurveto{\pgfqpoint{3.933056in}{1.585000in}}{\pgfqpoint{3.905278in}{1.585000in}}%
\pgfpathlineto{\pgfqpoint{3.225556in}{1.585000in}}%
\pgfpathquadraticcurveto{\pgfqpoint{3.197778in}{1.585000in}}{\pgfqpoint{3.197778in}{1.557222in}}%
\pgfpathlineto{\pgfqpoint{3.197778in}{1.182222in}}%
\pgfpathquadraticcurveto{\pgfqpoint{3.197778in}{1.154445in}}{\pgfqpoint{3.225556in}{1.154445in}}%
\pgfpathlineto{\pgfqpoint{3.225556in}{1.154445in}}%
\pgfpathclose%
\pgfusepath{stroke,fill}%
\end{pgfscope}%
\begin{pgfscope}%
\pgfsetbuttcap%
\pgfsetmiterjoin%
\pgfsetlinewidth{1.003750pt}%
\definecolor{currentstroke}{rgb}{0.000000,0.000000,0.000000}%
\pgfsetstrokecolor{currentstroke}%
\pgfsetdash{}{0pt}%
\pgfpathmoveto{\pgfqpoint{3.253334in}{1.432222in}}%
\pgfpathlineto{\pgfqpoint{3.531111in}{1.432222in}}%
\pgfpathlineto{\pgfqpoint{3.531111in}{1.529444in}}%
\pgfpathlineto{\pgfqpoint{3.253334in}{1.529444in}}%
\pgfpathlineto{\pgfqpoint{3.253334in}{1.432222in}}%
\pgfpathclose%
\pgfusepath{stroke}%
\end{pgfscope}%
\begin{pgfscope}%
\definecolor{textcolor}{rgb}{0.000000,0.000000,0.000000}%
\pgfsetstrokecolor{textcolor}%
\pgfsetfillcolor{textcolor}%
\pgftext[x=3.642223in,y=1.432222in,left,base]{\color{textcolor}\rmfamily\fontsize{10.000000}{12.000000}\selectfont Neg}%
\end{pgfscope}%
\begin{pgfscope}%
\pgfsetbuttcap%
\pgfsetmiterjoin%
\definecolor{currentfill}{rgb}{0.000000,0.000000,0.000000}%
\pgfsetfillcolor{currentfill}%
\pgfsetlinewidth{0.000000pt}%
\definecolor{currentstroke}{rgb}{0.000000,0.000000,0.000000}%
\pgfsetstrokecolor{currentstroke}%
\pgfsetstrokeopacity{0.000000}%
\pgfsetdash{}{0pt}%
\pgfpathmoveto{\pgfqpoint{3.253334in}{1.236944in}}%
\pgfpathlineto{\pgfqpoint{3.531111in}{1.236944in}}%
\pgfpathlineto{\pgfqpoint{3.531111in}{1.334167in}}%
\pgfpathlineto{\pgfqpoint{3.253334in}{1.334167in}}%
\pgfpathlineto{\pgfqpoint{3.253334in}{1.236944in}}%
\pgfpathclose%
\pgfusepath{fill}%
\end{pgfscope}%
\begin{pgfscope}%
\definecolor{textcolor}{rgb}{0.000000,0.000000,0.000000}%
\pgfsetstrokecolor{textcolor}%
\pgfsetfillcolor{textcolor}%
\pgftext[x=3.642223in,y=1.236944in,left,base]{\color{textcolor}\rmfamily\fontsize{10.000000}{12.000000}\selectfont Pos}%
\end{pgfscope}%
\end{pgfpicture}%
\makeatother%
\endgroup%
	
&
	\vskip 0pt
	\hfil ROC Curve
	
	%% Creator: Matplotlib, PGF backend
%%
%% To include the figure in your LaTeX document, write
%%   \input{<filename>.pgf}
%%
%% Make sure the required packages are loaded in your preamble
%%   \usepackage{pgf}
%%
%% Also ensure that all the required font packages are loaded; for instance,
%% the lmodern package is sometimes necessary when using math font.
%%   \usepackage{lmodern}
%%
%% Figures using additional raster images can only be included by \input if
%% they are in the same directory as the main LaTeX file. For loading figures
%% from other directories you can use the `import` package
%%   \usepackage{import}
%%
%% and then include the figures with
%%   \import{<path to file>}{<filename>.pgf}
%%
%% Matplotlib used the following preamble
%%   
%%   \usepackage{fontspec}
%%   \makeatletter\@ifpackageloaded{underscore}{}{\usepackage[strings]{underscore}}\makeatother
%%
\begingroup%
\makeatletter%
\begin{pgfpicture}%
\pgfpathrectangle{\pgfpointorigin}{\pgfqpoint{2.221861in}{1.754444in}}%
\pgfusepath{use as bounding box, clip}%
\begin{pgfscope}%
\pgfsetbuttcap%
\pgfsetmiterjoin%
\definecolor{currentfill}{rgb}{1.000000,1.000000,1.000000}%
\pgfsetfillcolor{currentfill}%
\pgfsetlinewidth{0.000000pt}%
\definecolor{currentstroke}{rgb}{1.000000,1.000000,1.000000}%
\pgfsetstrokecolor{currentstroke}%
\pgfsetdash{}{0pt}%
\pgfpathmoveto{\pgfqpoint{0.000000in}{0.000000in}}%
\pgfpathlineto{\pgfqpoint{2.221861in}{0.000000in}}%
\pgfpathlineto{\pgfqpoint{2.221861in}{1.754444in}}%
\pgfpathlineto{\pgfqpoint{0.000000in}{1.754444in}}%
\pgfpathlineto{\pgfqpoint{0.000000in}{0.000000in}}%
\pgfpathclose%
\pgfusepath{fill}%
\end{pgfscope}%
\begin{pgfscope}%
\pgfsetbuttcap%
\pgfsetmiterjoin%
\definecolor{currentfill}{rgb}{1.000000,1.000000,1.000000}%
\pgfsetfillcolor{currentfill}%
\pgfsetlinewidth{0.000000pt}%
\definecolor{currentstroke}{rgb}{0.000000,0.000000,0.000000}%
\pgfsetstrokecolor{currentstroke}%
\pgfsetstrokeopacity{0.000000}%
\pgfsetdash{}{0pt}%
\pgfpathmoveto{\pgfqpoint{0.553581in}{0.499444in}}%
\pgfpathlineto{\pgfqpoint{2.103581in}{0.499444in}}%
\pgfpathlineto{\pgfqpoint{2.103581in}{1.654444in}}%
\pgfpathlineto{\pgfqpoint{0.553581in}{1.654444in}}%
\pgfpathlineto{\pgfqpoint{0.553581in}{0.499444in}}%
\pgfpathclose%
\pgfusepath{fill}%
\end{pgfscope}%
\begin{pgfscope}%
\pgfsetbuttcap%
\pgfsetroundjoin%
\definecolor{currentfill}{rgb}{0.000000,0.000000,0.000000}%
\pgfsetfillcolor{currentfill}%
\pgfsetlinewidth{0.803000pt}%
\definecolor{currentstroke}{rgb}{0.000000,0.000000,0.000000}%
\pgfsetstrokecolor{currentstroke}%
\pgfsetdash{}{0pt}%
\pgfsys@defobject{currentmarker}{\pgfqpoint{0.000000in}{-0.048611in}}{\pgfqpoint{0.000000in}{0.000000in}}{%
\pgfpathmoveto{\pgfqpoint{0.000000in}{0.000000in}}%
\pgfpathlineto{\pgfqpoint{0.000000in}{-0.048611in}}%
\pgfusepath{stroke,fill}%
}%
\begin{pgfscope}%
\pgfsys@transformshift{0.624035in}{0.499444in}%
\pgfsys@useobject{currentmarker}{}%
\end{pgfscope}%
\end{pgfscope}%
\begin{pgfscope}%
\definecolor{textcolor}{rgb}{0.000000,0.000000,0.000000}%
\pgfsetstrokecolor{textcolor}%
\pgfsetfillcolor{textcolor}%
\pgftext[x=0.624035in,y=0.402222in,,top]{\color{textcolor}\rmfamily\fontsize{10.000000}{12.000000}\selectfont \(\displaystyle {0.0}\)}%
\end{pgfscope}%
\begin{pgfscope}%
\pgfsetbuttcap%
\pgfsetroundjoin%
\definecolor{currentfill}{rgb}{0.000000,0.000000,0.000000}%
\pgfsetfillcolor{currentfill}%
\pgfsetlinewidth{0.803000pt}%
\definecolor{currentstroke}{rgb}{0.000000,0.000000,0.000000}%
\pgfsetstrokecolor{currentstroke}%
\pgfsetdash{}{0pt}%
\pgfsys@defobject{currentmarker}{\pgfqpoint{0.000000in}{-0.048611in}}{\pgfqpoint{0.000000in}{0.000000in}}{%
\pgfpathmoveto{\pgfqpoint{0.000000in}{0.000000in}}%
\pgfpathlineto{\pgfqpoint{0.000000in}{-0.048611in}}%
\pgfusepath{stroke,fill}%
}%
\begin{pgfscope}%
\pgfsys@transformshift{1.328581in}{0.499444in}%
\pgfsys@useobject{currentmarker}{}%
\end{pgfscope}%
\end{pgfscope}%
\begin{pgfscope}%
\definecolor{textcolor}{rgb}{0.000000,0.000000,0.000000}%
\pgfsetstrokecolor{textcolor}%
\pgfsetfillcolor{textcolor}%
\pgftext[x=1.328581in,y=0.402222in,,top]{\color{textcolor}\rmfamily\fontsize{10.000000}{12.000000}\selectfont \(\displaystyle {0.5}\)}%
\end{pgfscope}%
\begin{pgfscope}%
\pgfsetbuttcap%
\pgfsetroundjoin%
\definecolor{currentfill}{rgb}{0.000000,0.000000,0.000000}%
\pgfsetfillcolor{currentfill}%
\pgfsetlinewidth{0.803000pt}%
\definecolor{currentstroke}{rgb}{0.000000,0.000000,0.000000}%
\pgfsetstrokecolor{currentstroke}%
\pgfsetdash{}{0pt}%
\pgfsys@defobject{currentmarker}{\pgfqpoint{0.000000in}{-0.048611in}}{\pgfqpoint{0.000000in}{0.000000in}}{%
\pgfpathmoveto{\pgfqpoint{0.000000in}{0.000000in}}%
\pgfpathlineto{\pgfqpoint{0.000000in}{-0.048611in}}%
\pgfusepath{stroke,fill}%
}%
\begin{pgfscope}%
\pgfsys@transformshift{2.033126in}{0.499444in}%
\pgfsys@useobject{currentmarker}{}%
\end{pgfscope}%
\end{pgfscope}%
\begin{pgfscope}%
\definecolor{textcolor}{rgb}{0.000000,0.000000,0.000000}%
\pgfsetstrokecolor{textcolor}%
\pgfsetfillcolor{textcolor}%
\pgftext[x=2.033126in,y=0.402222in,,top]{\color{textcolor}\rmfamily\fontsize{10.000000}{12.000000}\selectfont \(\displaystyle {1.0}\)}%
\end{pgfscope}%
\begin{pgfscope}%
\definecolor{textcolor}{rgb}{0.000000,0.000000,0.000000}%
\pgfsetstrokecolor{textcolor}%
\pgfsetfillcolor{textcolor}%
\pgftext[x=1.328581in,y=0.223333in,,top]{\color{textcolor}\rmfamily\fontsize{10.000000}{12.000000}\selectfont False positive rate}%
\end{pgfscope}%
\begin{pgfscope}%
\pgfsetbuttcap%
\pgfsetroundjoin%
\definecolor{currentfill}{rgb}{0.000000,0.000000,0.000000}%
\pgfsetfillcolor{currentfill}%
\pgfsetlinewidth{0.803000pt}%
\definecolor{currentstroke}{rgb}{0.000000,0.000000,0.000000}%
\pgfsetstrokecolor{currentstroke}%
\pgfsetdash{}{0pt}%
\pgfsys@defobject{currentmarker}{\pgfqpoint{-0.048611in}{0.000000in}}{\pgfqpoint{-0.000000in}{0.000000in}}{%
\pgfpathmoveto{\pgfqpoint{-0.000000in}{0.000000in}}%
\pgfpathlineto{\pgfqpoint{-0.048611in}{0.000000in}}%
\pgfusepath{stroke,fill}%
}%
\begin{pgfscope}%
\pgfsys@transformshift{0.553581in}{0.551944in}%
\pgfsys@useobject{currentmarker}{}%
\end{pgfscope}%
\end{pgfscope}%
\begin{pgfscope}%
\definecolor{textcolor}{rgb}{0.000000,0.000000,0.000000}%
\pgfsetstrokecolor{textcolor}%
\pgfsetfillcolor{textcolor}%
\pgftext[x=0.278889in, y=0.503750in, left, base]{\color{textcolor}\rmfamily\fontsize{10.000000}{12.000000}\selectfont \(\displaystyle {0.0}\)}%
\end{pgfscope}%
\begin{pgfscope}%
\pgfsetbuttcap%
\pgfsetroundjoin%
\definecolor{currentfill}{rgb}{0.000000,0.000000,0.000000}%
\pgfsetfillcolor{currentfill}%
\pgfsetlinewidth{0.803000pt}%
\definecolor{currentstroke}{rgb}{0.000000,0.000000,0.000000}%
\pgfsetstrokecolor{currentstroke}%
\pgfsetdash{}{0pt}%
\pgfsys@defobject{currentmarker}{\pgfqpoint{-0.048611in}{0.000000in}}{\pgfqpoint{-0.000000in}{0.000000in}}{%
\pgfpathmoveto{\pgfqpoint{-0.000000in}{0.000000in}}%
\pgfpathlineto{\pgfqpoint{-0.048611in}{0.000000in}}%
\pgfusepath{stroke,fill}%
}%
\begin{pgfscope}%
\pgfsys@transformshift{0.553581in}{1.076944in}%
\pgfsys@useobject{currentmarker}{}%
\end{pgfscope}%
\end{pgfscope}%
\begin{pgfscope}%
\definecolor{textcolor}{rgb}{0.000000,0.000000,0.000000}%
\pgfsetstrokecolor{textcolor}%
\pgfsetfillcolor{textcolor}%
\pgftext[x=0.278889in, y=1.028750in, left, base]{\color{textcolor}\rmfamily\fontsize{10.000000}{12.000000}\selectfont \(\displaystyle {0.5}\)}%
\end{pgfscope}%
\begin{pgfscope}%
\pgfsetbuttcap%
\pgfsetroundjoin%
\definecolor{currentfill}{rgb}{0.000000,0.000000,0.000000}%
\pgfsetfillcolor{currentfill}%
\pgfsetlinewidth{0.803000pt}%
\definecolor{currentstroke}{rgb}{0.000000,0.000000,0.000000}%
\pgfsetstrokecolor{currentstroke}%
\pgfsetdash{}{0pt}%
\pgfsys@defobject{currentmarker}{\pgfqpoint{-0.048611in}{0.000000in}}{\pgfqpoint{-0.000000in}{0.000000in}}{%
\pgfpathmoveto{\pgfqpoint{-0.000000in}{0.000000in}}%
\pgfpathlineto{\pgfqpoint{-0.048611in}{0.000000in}}%
\pgfusepath{stroke,fill}%
}%
\begin{pgfscope}%
\pgfsys@transformshift{0.553581in}{1.601944in}%
\pgfsys@useobject{currentmarker}{}%
\end{pgfscope}%
\end{pgfscope}%
\begin{pgfscope}%
\definecolor{textcolor}{rgb}{0.000000,0.000000,0.000000}%
\pgfsetstrokecolor{textcolor}%
\pgfsetfillcolor{textcolor}%
\pgftext[x=0.278889in, y=1.553750in, left, base]{\color{textcolor}\rmfamily\fontsize{10.000000}{12.000000}\selectfont \(\displaystyle {1.0}\)}%
\end{pgfscope}%
\begin{pgfscope}%
\definecolor{textcolor}{rgb}{0.000000,0.000000,0.000000}%
\pgfsetstrokecolor{textcolor}%
\pgfsetfillcolor{textcolor}%
\pgftext[x=0.223333in,y=1.076944in,,bottom,rotate=90.000000]{\color{textcolor}\rmfamily\fontsize{10.000000}{12.000000}\selectfont True positive rate}%
\end{pgfscope}%
\begin{pgfscope}%
\pgfpathrectangle{\pgfqpoint{0.553581in}{0.499444in}}{\pgfqpoint{1.550000in}{1.155000in}}%
\pgfusepath{clip}%
\pgfsetbuttcap%
\pgfsetroundjoin%
\pgfsetlinewidth{1.505625pt}%
\definecolor{currentstroke}{rgb}{0.000000,0.000000,0.000000}%
\pgfsetstrokecolor{currentstroke}%
\pgfsetdash{{5.550000pt}{2.400000pt}}{0.000000pt}%
\pgfpathmoveto{\pgfqpoint{0.624035in}{0.551944in}}%
\pgfpathlineto{\pgfqpoint{2.033126in}{1.601944in}}%
\pgfusepath{stroke}%
\end{pgfscope}%
\begin{pgfscope}%
\pgfpathrectangle{\pgfqpoint{0.553581in}{0.499444in}}{\pgfqpoint{1.550000in}{1.155000in}}%
\pgfusepath{clip}%
\pgfsetrectcap%
\pgfsetroundjoin%
\pgfsetlinewidth{1.505625pt}%
\definecolor{currentstroke}{rgb}{0.000000,0.000000,0.000000}%
\pgfsetstrokecolor{currentstroke}%
\pgfsetdash{}{0pt}%
\pgfpathmoveto{\pgfqpoint{0.624035in}{0.551944in}}%
\pgfpathlineto{\pgfqpoint{0.632298in}{0.616605in}}%
\pgfpathlineto{\pgfqpoint{0.653523in}{0.720410in}}%
\pgfpathlineto{\pgfqpoint{0.694261in}{0.842436in}}%
\pgfpathlineto{\pgfqpoint{0.759789in}{0.977811in}}%
\pgfpathlineto{\pgfqpoint{0.860417in}{1.110578in}}%
\pgfpathlineto{\pgfqpoint{1.004004in}{1.240365in}}%
\pgfpathlineto{\pgfqpoint{1.185428in}{1.359040in}}%
\pgfpathlineto{\pgfqpoint{1.401437in}{1.457288in}}%
\pgfpathlineto{\pgfqpoint{1.633715in}{1.532534in}}%
\pgfpathlineto{\pgfqpoint{1.860395in}{1.580898in}}%
\pgfpathlineto{\pgfqpoint{2.033126in}{1.601944in}}%
\pgfpathlineto{\pgfqpoint{2.033126in}{1.601944in}}%
\pgfusepath{stroke}%
\end{pgfscope}%
\begin{pgfscope}%
\pgfsetrectcap%
\pgfsetmiterjoin%
\pgfsetlinewidth{0.803000pt}%
\definecolor{currentstroke}{rgb}{0.000000,0.000000,0.000000}%
\pgfsetstrokecolor{currentstroke}%
\pgfsetdash{}{0pt}%
\pgfpathmoveto{\pgfqpoint{0.553581in}{0.499444in}}%
\pgfpathlineto{\pgfqpoint{0.553581in}{1.654444in}}%
\pgfusepath{stroke}%
\end{pgfscope}%
\begin{pgfscope}%
\pgfsetrectcap%
\pgfsetmiterjoin%
\pgfsetlinewidth{0.803000pt}%
\definecolor{currentstroke}{rgb}{0.000000,0.000000,0.000000}%
\pgfsetstrokecolor{currentstroke}%
\pgfsetdash{}{0pt}%
\pgfpathmoveto{\pgfqpoint{2.103581in}{0.499444in}}%
\pgfpathlineto{\pgfqpoint{2.103581in}{1.654444in}}%
\pgfusepath{stroke}%
\end{pgfscope}%
\begin{pgfscope}%
\pgfsetrectcap%
\pgfsetmiterjoin%
\pgfsetlinewidth{0.803000pt}%
\definecolor{currentstroke}{rgb}{0.000000,0.000000,0.000000}%
\pgfsetstrokecolor{currentstroke}%
\pgfsetdash{}{0pt}%
\pgfpathmoveto{\pgfqpoint{0.553581in}{0.499444in}}%
\pgfpathlineto{\pgfqpoint{2.103581in}{0.499444in}}%
\pgfusepath{stroke}%
\end{pgfscope}%
\begin{pgfscope}%
\pgfsetrectcap%
\pgfsetmiterjoin%
\pgfsetlinewidth{0.803000pt}%
\definecolor{currentstroke}{rgb}{0.000000,0.000000,0.000000}%
\pgfsetstrokecolor{currentstroke}%
\pgfsetdash{}{0pt}%
\pgfpathmoveto{\pgfqpoint{0.553581in}{1.654444in}}%
\pgfpathlineto{\pgfqpoint{2.103581in}{1.654444in}}%
\pgfusepath{stroke}%
\end{pgfscope}%
\begin{pgfscope}%
\pgfsetbuttcap%
\pgfsetmiterjoin%
\definecolor{currentfill}{rgb}{1.000000,1.000000,1.000000}%
\pgfsetfillcolor{currentfill}%
\pgfsetfillopacity{0.800000}%
\pgfsetlinewidth{1.003750pt}%
\definecolor{currentstroke}{rgb}{0.800000,0.800000,0.800000}%
\pgfsetstrokecolor{currentstroke}%
\pgfsetstrokeopacity{0.800000}%
\pgfsetdash{}{0pt}%
\pgfpathmoveto{\pgfqpoint{0.832747in}{0.568889in}}%
\pgfpathlineto{\pgfqpoint{2.006358in}{0.568889in}}%
\pgfpathquadraticcurveto{\pgfqpoint{2.034136in}{0.568889in}}{\pgfqpoint{2.034136in}{0.596666in}}%
\pgfpathlineto{\pgfqpoint{2.034136in}{0.776388in}}%
\pgfpathquadraticcurveto{\pgfqpoint{2.034136in}{0.804166in}}{\pgfqpoint{2.006358in}{0.804166in}}%
\pgfpathlineto{\pgfqpoint{0.832747in}{0.804166in}}%
\pgfpathquadraticcurveto{\pgfqpoint{0.804970in}{0.804166in}}{\pgfqpoint{0.804970in}{0.776388in}}%
\pgfpathlineto{\pgfqpoint{0.804970in}{0.596666in}}%
\pgfpathquadraticcurveto{\pgfqpoint{0.804970in}{0.568889in}}{\pgfqpoint{0.832747in}{0.568889in}}%
\pgfpathlineto{\pgfqpoint{0.832747in}{0.568889in}}%
\pgfpathclose%
\pgfusepath{stroke,fill}%
\end{pgfscope}%
\begin{pgfscope}%
\pgfsetrectcap%
\pgfsetroundjoin%
\pgfsetlinewidth{1.505625pt}%
\definecolor{currentstroke}{rgb}{0.000000,0.000000,0.000000}%
\pgfsetstrokecolor{currentstroke}%
\pgfsetdash{}{0pt}%
\pgfpathmoveto{\pgfqpoint{0.860525in}{0.700000in}}%
\pgfpathlineto{\pgfqpoint{0.999414in}{0.700000in}}%
\pgfpathlineto{\pgfqpoint{1.138303in}{0.700000in}}%
\pgfusepath{stroke}%
\end{pgfscope}%
\begin{pgfscope}%
\definecolor{textcolor}{rgb}{0.000000,0.000000,0.000000}%
\pgfsetstrokecolor{textcolor}%
\pgfsetfillcolor{textcolor}%
\pgftext[x=1.249414in,y=0.651388in,left,base]{\color{textcolor}\rmfamily\fontsize{10.000000}{12.000000}\selectfont AUC=0.758}%
\end{pgfscope}%
\end{pgfpicture}%
\makeatother%
\endgroup%

\end{tabular}

\

\

Other stuff

\


%%%
\begin{comment}
If we set the discrimination threshold about $0.7$, the model would classify almost all of the samples, both positive and negative class, correctly, with about the same number of false positives (sending an ambulance when one is not needed, negative class samples with $p > 0.7$) and false negatives (not sending an ambulance when one is needed, positive class samples with $p < 0.7$).  If we (as a society) were willing to tolerate more false positives, we could set the discrimination threshold lower, and if budgets were tighter we could increase the $p$ threshold.  

The table below gives the number of true negatives (TN), false positives (FP), false negatives (FN), and true positives (TP) for the 499,496 samples in the test set, along with the precision and recall values, for different discrimination thresholds $p$.  The precision is the proportion of ambulances we sent that were needed, and the recall is the proportion of ambulances needed that we sent.  

$$\text{Precision} = \frac{TP}{FP+TP}, \qquad \text{Recall} = \frac{TP}{FN + TP}$$

\begin{center}
\begin{tabular}{rrrrrrrrrrrrrr}
\toprule
$p$ &   TN &       FP &      FN &      TP &  Precision &   Recall \\
\midrule
0.50 &  346,776 &   73,794 &       1 &  78,925 &  0.52 &  1.00       \\
0.60 &  390,335 &   30,235 &      89 &  78,837 &  0.72 &  1.00  \\
0.70 & 411,040 &    9,530 &   2,838 &  76,088 &  0.89 &  0.96 \\
0.80 & 418,739 &    1,831 &  19,174 &  59,752 &  0.97 &  0.76  \\
0.90 & 420,496 &       74 &  53,736 &  25,190 &  1.00 &  0.32 & \\
\bottomrule
\end{tabular}
\end{center}

\end{comment}
%%%



