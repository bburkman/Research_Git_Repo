% Analysis of Results

Our ML algorithms assign to each sample (feature vector, crash person) a probability $p \in [0,1]$ that the person needs an ambulance.  The histogram below left shows the percentage of the dataset in each range of $p$, showing the percentages for the negative class (``Does not need an ambulance'') and the positive class (``Needs an ambulance'').  On the right, the Receiver Operating Characteristic (ROC) curve, and particularly the area under the curve (AUC), is a metric for how well the model separates the two classes, with $AUC=1.0$ being perfect and $AUC=0.5$ (the dashed line) being just random assignment with no insight.  

We would love to have results like in the graphs below, where the machine learning (ML) algorithm nearly perfectly separates the two classes.  There is some overlap between $p=0.6$ and $p=0.8$ with some samples the algorithm misclassifies, but the model clearly separates most samples.  Having an AUC of 0.996 would be amazing.  

[Put in \verb|BRFC_Hard_alpha_0_5_Train_Pred_Wide.pgf| once we have it.)

\begin{comment}

\noindent\begin{tabular}{@{\hspace{-6pt}}p{4.3in} @{\hspace{-6pt}}p{2.0in}}
	\vskip 0pt
	\hfil Raw Model Output
	
	%% Creator: Matplotlib, PGF backend
%%
%% To include the figure in your LaTeX document, write
%%   \input{<filename>.pgf}
%%
%% Make sure the required packages are loaded in your preamble
%%   \usepackage{pgf}
%%
%% Also ensure that all the required font packages are loaded; for instance,
%% the lmodern package is sometimes necessary when using math font.
%%   \usepackage{lmodern}
%%
%% Figures using additional raster images can only be included by \input if
%% they are in the same directory as the main LaTeX file. For loading figures
%% from other directories you can use the `import` package
%%   \usepackage{import}
%%
%% and then include the figures with
%%   \import{<path to file>}{<filename>.pgf}
%%
%% Matplotlib used the following preamble
%%   
%%   \usepackage{fontspec}
%%   \makeatletter\@ifpackageloaded{underscore}{}{\usepackage[strings]{underscore}}\makeatother
%%
\begingroup%
\makeatletter%
\begin{pgfpicture}%
\pgfpathrectangle{\pgfpointorigin}{\pgfqpoint{4.509306in}{1.754444in}}%
\pgfusepath{use as bounding box, clip}%
\begin{pgfscope}%
\pgfsetbuttcap%
\pgfsetmiterjoin%
\definecolor{currentfill}{rgb}{1.000000,1.000000,1.000000}%
\pgfsetfillcolor{currentfill}%
\pgfsetlinewidth{0.000000pt}%
\definecolor{currentstroke}{rgb}{1.000000,1.000000,1.000000}%
\pgfsetstrokecolor{currentstroke}%
\pgfsetdash{}{0pt}%
\pgfpathmoveto{\pgfqpoint{0.000000in}{0.000000in}}%
\pgfpathlineto{\pgfqpoint{4.509306in}{0.000000in}}%
\pgfpathlineto{\pgfqpoint{4.509306in}{1.754444in}}%
\pgfpathlineto{\pgfqpoint{0.000000in}{1.754444in}}%
\pgfpathlineto{\pgfqpoint{0.000000in}{0.000000in}}%
\pgfpathclose%
\pgfusepath{fill}%
\end{pgfscope}%
\begin{pgfscope}%
\pgfsetbuttcap%
\pgfsetmiterjoin%
\definecolor{currentfill}{rgb}{1.000000,1.000000,1.000000}%
\pgfsetfillcolor{currentfill}%
\pgfsetlinewidth{0.000000pt}%
\definecolor{currentstroke}{rgb}{0.000000,0.000000,0.000000}%
\pgfsetstrokecolor{currentstroke}%
\pgfsetstrokeopacity{0.000000}%
\pgfsetdash{}{0pt}%
\pgfpathmoveto{\pgfqpoint{0.445556in}{0.499444in}}%
\pgfpathlineto{\pgfqpoint{4.320556in}{0.499444in}}%
\pgfpathlineto{\pgfqpoint{4.320556in}{1.654444in}}%
\pgfpathlineto{\pgfqpoint{0.445556in}{1.654444in}}%
\pgfpathlineto{\pgfqpoint{0.445556in}{0.499444in}}%
\pgfpathclose%
\pgfusepath{fill}%
\end{pgfscope}%
\begin{pgfscope}%
\pgfpathrectangle{\pgfqpoint{0.445556in}{0.499444in}}{\pgfqpoint{3.875000in}{1.155000in}}%
\pgfusepath{clip}%
\pgfsetbuttcap%
\pgfsetmiterjoin%
\pgfsetlinewidth{1.003750pt}%
\definecolor{currentstroke}{rgb}{0.000000,0.000000,0.000000}%
\pgfsetstrokecolor{currentstroke}%
\pgfsetdash{}{0pt}%
\pgfpathmoveto{\pgfqpoint{0.435556in}{0.499444in}}%
\pgfpathlineto{\pgfqpoint{0.483922in}{0.499444in}}%
\pgfpathlineto{\pgfqpoint{0.483922in}{0.632510in}}%
\pgfpathlineto{\pgfqpoint{0.435556in}{0.632510in}}%
\pgfusepath{stroke}%
\end{pgfscope}%
\begin{pgfscope}%
\pgfpathrectangle{\pgfqpoint{0.445556in}{0.499444in}}{\pgfqpoint{3.875000in}{1.155000in}}%
\pgfusepath{clip}%
\pgfsetbuttcap%
\pgfsetmiterjoin%
\pgfsetlinewidth{1.003750pt}%
\definecolor{currentstroke}{rgb}{0.000000,0.000000,0.000000}%
\pgfsetstrokecolor{currentstroke}%
\pgfsetdash{}{0pt}%
\pgfpathmoveto{\pgfqpoint{0.576001in}{0.499444in}}%
\pgfpathlineto{\pgfqpoint{0.637387in}{0.499444in}}%
\pgfpathlineto{\pgfqpoint{0.637387in}{0.855505in}}%
\pgfpathlineto{\pgfqpoint{0.576001in}{0.855505in}}%
\pgfpathlineto{\pgfqpoint{0.576001in}{0.499444in}}%
\pgfpathclose%
\pgfusepath{stroke}%
\end{pgfscope}%
\begin{pgfscope}%
\pgfpathrectangle{\pgfqpoint{0.445556in}{0.499444in}}{\pgfqpoint{3.875000in}{1.155000in}}%
\pgfusepath{clip}%
\pgfsetbuttcap%
\pgfsetmiterjoin%
\pgfsetlinewidth{1.003750pt}%
\definecolor{currentstroke}{rgb}{0.000000,0.000000,0.000000}%
\pgfsetstrokecolor{currentstroke}%
\pgfsetdash{}{0pt}%
\pgfpathmoveto{\pgfqpoint{0.729467in}{0.499444in}}%
\pgfpathlineto{\pgfqpoint{0.790853in}{0.499444in}}%
\pgfpathlineto{\pgfqpoint{0.790853in}{1.087595in}}%
\pgfpathlineto{\pgfqpoint{0.729467in}{1.087595in}}%
\pgfpathlineto{\pgfqpoint{0.729467in}{0.499444in}}%
\pgfpathclose%
\pgfusepath{stroke}%
\end{pgfscope}%
\begin{pgfscope}%
\pgfpathrectangle{\pgfqpoint{0.445556in}{0.499444in}}{\pgfqpoint{3.875000in}{1.155000in}}%
\pgfusepath{clip}%
\pgfsetbuttcap%
\pgfsetmiterjoin%
\pgfsetlinewidth{1.003750pt}%
\definecolor{currentstroke}{rgb}{0.000000,0.000000,0.000000}%
\pgfsetstrokecolor{currentstroke}%
\pgfsetdash{}{0pt}%
\pgfpathmoveto{\pgfqpoint{0.882932in}{0.499444in}}%
\pgfpathlineto{\pgfqpoint{0.944318in}{0.499444in}}%
\pgfpathlineto{\pgfqpoint{0.944318in}{1.269324in}}%
\pgfpathlineto{\pgfqpoint{0.882932in}{1.269324in}}%
\pgfpathlineto{\pgfqpoint{0.882932in}{0.499444in}}%
\pgfpathclose%
\pgfusepath{stroke}%
\end{pgfscope}%
\begin{pgfscope}%
\pgfpathrectangle{\pgfqpoint{0.445556in}{0.499444in}}{\pgfqpoint{3.875000in}{1.155000in}}%
\pgfusepath{clip}%
\pgfsetbuttcap%
\pgfsetmiterjoin%
\pgfsetlinewidth{1.003750pt}%
\definecolor{currentstroke}{rgb}{0.000000,0.000000,0.000000}%
\pgfsetstrokecolor{currentstroke}%
\pgfsetdash{}{0pt}%
\pgfpathmoveto{\pgfqpoint{1.036397in}{0.499444in}}%
\pgfpathlineto{\pgfqpoint{1.097783in}{0.499444in}}%
\pgfpathlineto{\pgfqpoint{1.097783in}{1.428301in}}%
\pgfpathlineto{\pgfqpoint{1.036397in}{1.428301in}}%
\pgfpathlineto{\pgfqpoint{1.036397in}{0.499444in}}%
\pgfpathclose%
\pgfusepath{stroke}%
\end{pgfscope}%
\begin{pgfscope}%
\pgfpathrectangle{\pgfqpoint{0.445556in}{0.499444in}}{\pgfqpoint{3.875000in}{1.155000in}}%
\pgfusepath{clip}%
\pgfsetbuttcap%
\pgfsetmiterjoin%
\pgfsetlinewidth{1.003750pt}%
\definecolor{currentstroke}{rgb}{0.000000,0.000000,0.000000}%
\pgfsetstrokecolor{currentstroke}%
\pgfsetdash{}{0pt}%
\pgfpathmoveto{\pgfqpoint{1.189863in}{0.499444in}}%
\pgfpathlineto{\pgfqpoint{1.251249in}{0.499444in}}%
\pgfpathlineto{\pgfqpoint{1.251249in}{1.521593in}}%
\pgfpathlineto{\pgfqpoint{1.189863in}{1.521593in}}%
\pgfpathlineto{\pgfqpoint{1.189863in}{0.499444in}}%
\pgfpathclose%
\pgfusepath{stroke}%
\end{pgfscope}%
\begin{pgfscope}%
\pgfpathrectangle{\pgfqpoint{0.445556in}{0.499444in}}{\pgfqpoint{3.875000in}{1.155000in}}%
\pgfusepath{clip}%
\pgfsetbuttcap%
\pgfsetmiterjoin%
\pgfsetlinewidth{1.003750pt}%
\definecolor{currentstroke}{rgb}{0.000000,0.000000,0.000000}%
\pgfsetstrokecolor{currentstroke}%
\pgfsetdash{}{0pt}%
\pgfpathmoveto{\pgfqpoint{1.343328in}{0.499444in}}%
\pgfpathlineto{\pgfqpoint{1.404714in}{0.499444in}}%
\pgfpathlineto{\pgfqpoint{1.404714in}{1.590875in}}%
\pgfpathlineto{\pgfqpoint{1.343328in}{1.590875in}}%
\pgfpathlineto{\pgfqpoint{1.343328in}{0.499444in}}%
\pgfpathclose%
\pgfusepath{stroke}%
\end{pgfscope}%
\begin{pgfscope}%
\pgfpathrectangle{\pgfqpoint{0.445556in}{0.499444in}}{\pgfqpoint{3.875000in}{1.155000in}}%
\pgfusepath{clip}%
\pgfsetbuttcap%
\pgfsetmiterjoin%
\pgfsetlinewidth{1.003750pt}%
\definecolor{currentstroke}{rgb}{0.000000,0.000000,0.000000}%
\pgfsetstrokecolor{currentstroke}%
\pgfsetdash{}{0pt}%
\pgfpathmoveto{\pgfqpoint{1.496793in}{0.499444in}}%
\pgfpathlineto{\pgfqpoint{1.558179in}{0.499444in}}%
\pgfpathlineto{\pgfqpoint{1.558179in}{1.599444in}}%
\pgfpathlineto{\pgfqpoint{1.496793in}{1.599444in}}%
\pgfpathlineto{\pgfqpoint{1.496793in}{0.499444in}}%
\pgfpathclose%
\pgfusepath{stroke}%
\end{pgfscope}%
\begin{pgfscope}%
\pgfpathrectangle{\pgfqpoint{0.445556in}{0.499444in}}{\pgfqpoint{3.875000in}{1.155000in}}%
\pgfusepath{clip}%
\pgfsetbuttcap%
\pgfsetmiterjoin%
\pgfsetlinewidth{1.003750pt}%
\definecolor{currentstroke}{rgb}{0.000000,0.000000,0.000000}%
\pgfsetstrokecolor{currentstroke}%
\pgfsetdash{}{0pt}%
\pgfpathmoveto{\pgfqpoint{1.650259in}{0.499444in}}%
\pgfpathlineto{\pgfqpoint{1.711645in}{0.499444in}}%
\pgfpathlineto{\pgfqpoint{1.711645in}{1.566953in}}%
\pgfpathlineto{\pgfqpoint{1.650259in}{1.566953in}}%
\pgfpathlineto{\pgfqpoint{1.650259in}{0.499444in}}%
\pgfpathclose%
\pgfusepath{stroke}%
\end{pgfscope}%
\begin{pgfscope}%
\pgfpathrectangle{\pgfqpoint{0.445556in}{0.499444in}}{\pgfqpoint{3.875000in}{1.155000in}}%
\pgfusepath{clip}%
\pgfsetbuttcap%
\pgfsetmiterjoin%
\pgfsetlinewidth{1.003750pt}%
\definecolor{currentstroke}{rgb}{0.000000,0.000000,0.000000}%
\pgfsetstrokecolor{currentstroke}%
\pgfsetdash{}{0pt}%
\pgfpathmoveto{\pgfqpoint{1.803724in}{0.499444in}}%
\pgfpathlineto{\pgfqpoint{1.865110in}{0.499444in}}%
\pgfpathlineto{\pgfqpoint{1.865110in}{1.506942in}}%
\pgfpathlineto{\pgfqpoint{1.803724in}{1.506942in}}%
\pgfpathlineto{\pgfqpoint{1.803724in}{0.499444in}}%
\pgfpathclose%
\pgfusepath{stroke}%
\end{pgfscope}%
\begin{pgfscope}%
\pgfpathrectangle{\pgfqpoint{0.445556in}{0.499444in}}{\pgfqpoint{3.875000in}{1.155000in}}%
\pgfusepath{clip}%
\pgfsetbuttcap%
\pgfsetmiterjoin%
\pgfsetlinewidth{1.003750pt}%
\definecolor{currentstroke}{rgb}{0.000000,0.000000,0.000000}%
\pgfsetstrokecolor{currentstroke}%
\pgfsetdash{}{0pt}%
\pgfpathmoveto{\pgfqpoint{1.957189in}{0.499444in}}%
\pgfpathlineto{\pgfqpoint{2.018575in}{0.499444in}}%
\pgfpathlineto{\pgfqpoint{2.018575in}{1.416545in}}%
\pgfpathlineto{\pgfqpoint{1.957189in}{1.416545in}}%
\pgfpathlineto{\pgfqpoint{1.957189in}{0.499444in}}%
\pgfpathclose%
\pgfusepath{stroke}%
\end{pgfscope}%
\begin{pgfscope}%
\pgfpathrectangle{\pgfqpoint{0.445556in}{0.499444in}}{\pgfqpoint{3.875000in}{1.155000in}}%
\pgfusepath{clip}%
\pgfsetbuttcap%
\pgfsetmiterjoin%
\pgfsetlinewidth{1.003750pt}%
\definecolor{currentstroke}{rgb}{0.000000,0.000000,0.000000}%
\pgfsetstrokecolor{currentstroke}%
\pgfsetdash{}{0pt}%
\pgfpathmoveto{\pgfqpoint{2.110655in}{0.499444in}}%
\pgfpathlineto{\pgfqpoint{2.172041in}{0.499444in}}%
\pgfpathlineto{\pgfqpoint{2.172041in}{1.298482in}}%
\pgfpathlineto{\pgfqpoint{2.110655in}{1.298482in}}%
\pgfpathlineto{\pgfqpoint{2.110655in}{0.499444in}}%
\pgfpathclose%
\pgfusepath{stroke}%
\end{pgfscope}%
\begin{pgfscope}%
\pgfpathrectangle{\pgfqpoint{0.445556in}{0.499444in}}{\pgfqpoint{3.875000in}{1.155000in}}%
\pgfusepath{clip}%
\pgfsetbuttcap%
\pgfsetmiterjoin%
\pgfsetlinewidth{1.003750pt}%
\definecolor{currentstroke}{rgb}{0.000000,0.000000,0.000000}%
\pgfsetstrokecolor{currentstroke}%
\pgfsetdash{}{0pt}%
\pgfpathmoveto{\pgfqpoint{2.264120in}{0.499444in}}%
\pgfpathlineto{\pgfqpoint{2.325506in}{0.499444in}}%
\pgfpathlineto{\pgfqpoint{2.325506in}{1.165884in}}%
\pgfpathlineto{\pgfqpoint{2.264120in}{1.165884in}}%
\pgfpathlineto{\pgfqpoint{2.264120in}{0.499444in}}%
\pgfpathclose%
\pgfusepath{stroke}%
\end{pgfscope}%
\begin{pgfscope}%
\pgfpathrectangle{\pgfqpoint{0.445556in}{0.499444in}}{\pgfqpoint{3.875000in}{1.155000in}}%
\pgfusepath{clip}%
\pgfsetbuttcap%
\pgfsetmiterjoin%
\pgfsetlinewidth{1.003750pt}%
\definecolor{currentstroke}{rgb}{0.000000,0.000000,0.000000}%
\pgfsetstrokecolor{currentstroke}%
\pgfsetdash{}{0pt}%
\pgfpathmoveto{\pgfqpoint{2.417585in}{0.499444in}}%
\pgfpathlineto{\pgfqpoint{2.478972in}{0.499444in}}%
\pgfpathlineto{\pgfqpoint{2.478972in}{1.041650in}}%
\pgfpathlineto{\pgfqpoint{2.417585in}{1.041650in}}%
\pgfpathlineto{\pgfqpoint{2.417585in}{0.499444in}}%
\pgfpathclose%
\pgfusepath{stroke}%
\end{pgfscope}%
\begin{pgfscope}%
\pgfpathrectangle{\pgfqpoint{0.445556in}{0.499444in}}{\pgfqpoint{3.875000in}{1.155000in}}%
\pgfusepath{clip}%
\pgfsetbuttcap%
\pgfsetmiterjoin%
\pgfsetlinewidth{1.003750pt}%
\definecolor{currentstroke}{rgb}{0.000000,0.000000,0.000000}%
\pgfsetstrokecolor{currentstroke}%
\pgfsetdash{}{0pt}%
\pgfpathmoveto{\pgfqpoint{2.571051in}{0.499444in}}%
\pgfpathlineto{\pgfqpoint{2.632437in}{0.499444in}}%
\pgfpathlineto{\pgfqpoint{2.632437in}{0.916218in}}%
\pgfpathlineto{\pgfqpoint{2.571051in}{0.916218in}}%
\pgfpathlineto{\pgfqpoint{2.571051in}{0.499444in}}%
\pgfpathclose%
\pgfusepath{stroke}%
\end{pgfscope}%
\begin{pgfscope}%
\pgfpathrectangle{\pgfqpoint{0.445556in}{0.499444in}}{\pgfqpoint{3.875000in}{1.155000in}}%
\pgfusepath{clip}%
\pgfsetbuttcap%
\pgfsetmiterjoin%
\pgfsetlinewidth{1.003750pt}%
\definecolor{currentstroke}{rgb}{0.000000,0.000000,0.000000}%
\pgfsetstrokecolor{currentstroke}%
\pgfsetdash{}{0pt}%
\pgfpathmoveto{\pgfqpoint{2.724516in}{0.499444in}}%
\pgfpathlineto{\pgfqpoint{2.785902in}{0.499444in}}%
\pgfpathlineto{\pgfqpoint{2.785902in}{0.806958in}}%
\pgfpathlineto{\pgfqpoint{2.724516in}{0.806958in}}%
\pgfpathlineto{\pgfqpoint{2.724516in}{0.499444in}}%
\pgfpathclose%
\pgfusepath{stroke}%
\end{pgfscope}%
\begin{pgfscope}%
\pgfpathrectangle{\pgfqpoint{0.445556in}{0.499444in}}{\pgfqpoint{3.875000in}{1.155000in}}%
\pgfusepath{clip}%
\pgfsetbuttcap%
\pgfsetmiterjoin%
\pgfsetlinewidth{1.003750pt}%
\definecolor{currentstroke}{rgb}{0.000000,0.000000,0.000000}%
\pgfsetstrokecolor{currentstroke}%
\pgfsetdash{}{0pt}%
\pgfpathmoveto{\pgfqpoint{2.877981in}{0.499444in}}%
\pgfpathlineto{\pgfqpoint{2.939368in}{0.499444in}}%
\pgfpathlineto{\pgfqpoint{2.939368in}{0.716063in}}%
\pgfpathlineto{\pgfqpoint{2.877981in}{0.716063in}}%
\pgfpathlineto{\pgfqpoint{2.877981in}{0.499444in}}%
\pgfpathclose%
\pgfusepath{stroke}%
\end{pgfscope}%
\begin{pgfscope}%
\pgfpathrectangle{\pgfqpoint{0.445556in}{0.499444in}}{\pgfqpoint{3.875000in}{1.155000in}}%
\pgfusepath{clip}%
\pgfsetbuttcap%
\pgfsetmiterjoin%
\pgfsetlinewidth{1.003750pt}%
\definecolor{currentstroke}{rgb}{0.000000,0.000000,0.000000}%
\pgfsetstrokecolor{currentstroke}%
\pgfsetdash{}{0pt}%
\pgfpathmoveto{\pgfqpoint{3.031447in}{0.499444in}}%
\pgfpathlineto{\pgfqpoint{3.092833in}{0.499444in}}%
\pgfpathlineto{\pgfqpoint{3.092833in}{0.645904in}}%
\pgfpathlineto{\pgfqpoint{3.031447in}{0.645904in}}%
\pgfpathlineto{\pgfqpoint{3.031447in}{0.499444in}}%
\pgfpathclose%
\pgfusepath{stroke}%
\end{pgfscope}%
\begin{pgfscope}%
\pgfpathrectangle{\pgfqpoint{0.445556in}{0.499444in}}{\pgfqpoint{3.875000in}{1.155000in}}%
\pgfusepath{clip}%
\pgfsetbuttcap%
\pgfsetmiterjoin%
\pgfsetlinewidth{1.003750pt}%
\definecolor{currentstroke}{rgb}{0.000000,0.000000,0.000000}%
\pgfsetstrokecolor{currentstroke}%
\pgfsetdash{}{0pt}%
\pgfpathmoveto{\pgfqpoint{3.184912in}{0.499444in}}%
\pgfpathlineto{\pgfqpoint{3.246298in}{0.499444in}}%
\pgfpathlineto{\pgfqpoint{3.246298in}{0.598936in}}%
\pgfpathlineto{\pgfqpoint{3.184912in}{0.598936in}}%
\pgfpathlineto{\pgfqpoint{3.184912in}{0.499444in}}%
\pgfpathclose%
\pgfusepath{stroke}%
\end{pgfscope}%
\begin{pgfscope}%
\pgfpathrectangle{\pgfqpoint{0.445556in}{0.499444in}}{\pgfqpoint{3.875000in}{1.155000in}}%
\pgfusepath{clip}%
\pgfsetbuttcap%
\pgfsetmiterjoin%
\pgfsetlinewidth{1.003750pt}%
\definecolor{currentstroke}{rgb}{0.000000,0.000000,0.000000}%
\pgfsetstrokecolor{currentstroke}%
\pgfsetdash{}{0pt}%
\pgfpathmoveto{\pgfqpoint{3.338377in}{0.499444in}}%
\pgfpathlineto{\pgfqpoint{3.399764in}{0.499444in}}%
\pgfpathlineto{\pgfqpoint{3.399764in}{0.560040in}}%
\pgfpathlineto{\pgfqpoint{3.338377in}{0.560040in}}%
\pgfpathlineto{\pgfqpoint{3.338377in}{0.499444in}}%
\pgfpathclose%
\pgfusepath{stroke}%
\end{pgfscope}%
\begin{pgfscope}%
\pgfpathrectangle{\pgfqpoint{0.445556in}{0.499444in}}{\pgfqpoint{3.875000in}{1.155000in}}%
\pgfusepath{clip}%
\pgfsetbuttcap%
\pgfsetmiterjoin%
\pgfsetlinewidth{1.003750pt}%
\definecolor{currentstroke}{rgb}{0.000000,0.000000,0.000000}%
\pgfsetstrokecolor{currentstroke}%
\pgfsetdash{}{0pt}%
\pgfpathmoveto{\pgfqpoint{3.491843in}{0.499444in}}%
\pgfpathlineto{\pgfqpoint{3.553229in}{0.499444in}}%
\pgfpathlineto{\pgfqpoint{3.553229in}{0.535591in}}%
\pgfpathlineto{\pgfqpoint{3.491843in}{0.535591in}}%
\pgfpathlineto{\pgfqpoint{3.491843in}{0.499444in}}%
\pgfpathclose%
\pgfusepath{stroke}%
\end{pgfscope}%
\begin{pgfscope}%
\pgfpathrectangle{\pgfqpoint{0.445556in}{0.499444in}}{\pgfqpoint{3.875000in}{1.155000in}}%
\pgfusepath{clip}%
\pgfsetbuttcap%
\pgfsetmiterjoin%
\pgfsetlinewidth{1.003750pt}%
\definecolor{currentstroke}{rgb}{0.000000,0.000000,0.000000}%
\pgfsetstrokecolor{currentstroke}%
\pgfsetdash{}{0pt}%
\pgfpathmoveto{\pgfqpoint{3.645308in}{0.499444in}}%
\pgfpathlineto{\pgfqpoint{3.706694in}{0.499444in}}%
\pgfpathlineto{\pgfqpoint{3.706694in}{0.512137in}}%
\pgfpathlineto{\pgfqpoint{3.645308in}{0.512137in}}%
\pgfpathlineto{\pgfqpoint{3.645308in}{0.499444in}}%
\pgfpathclose%
\pgfusepath{stroke}%
\end{pgfscope}%
\begin{pgfscope}%
\pgfpathrectangle{\pgfqpoint{0.445556in}{0.499444in}}{\pgfqpoint{3.875000in}{1.155000in}}%
\pgfusepath{clip}%
\pgfsetbuttcap%
\pgfsetmiterjoin%
\pgfsetlinewidth{1.003750pt}%
\definecolor{currentstroke}{rgb}{0.000000,0.000000,0.000000}%
\pgfsetstrokecolor{currentstroke}%
\pgfsetdash{}{0pt}%
\pgfpathmoveto{\pgfqpoint{3.798774in}{0.499444in}}%
\pgfpathlineto{\pgfqpoint{3.860160in}{0.499444in}}%
\pgfpathlineto{\pgfqpoint{3.860160in}{0.503305in}}%
\pgfpathlineto{\pgfqpoint{3.798774in}{0.503305in}}%
\pgfpathlineto{\pgfqpoint{3.798774in}{0.499444in}}%
\pgfpathclose%
\pgfusepath{stroke}%
\end{pgfscope}%
\begin{pgfscope}%
\pgfpathrectangle{\pgfqpoint{0.445556in}{0.499444in}}{\pgfqpoint{3.875000in}{1.155000in}}%
\pgfusepath{clip}%
\pgfsetbuttcap%
\pgfsetmiterjoin%
\pgfsetlinewidth{1.003750pt}%
\definecolor{currentstroke}{rgb}{0.000000,0.000000,0.000000}%
\pgfsetstrokecolor{currentstroke}%
\pgfsetdash{}{0pt}%
\pgfpathmoveto{\pgfqpoint{3.952239in}{0.499444in}}%
\pgfpathlineto{\pgfqpoint{4.013625in}{0.499444in}}%
\pgfpathlineto{\pgfqpoint{4.013625in}{0.500234in}}%
\pgfpathlineto{\pgfqpoint{3.952239in}{0.500234in}}%
\pgfpathlineto{\pgfqpoint{3.952239in}{0.499444in}}%
\pgfpathclose%
\pgfusepath{stroke}%
\end{pgfscope}%
\begin{pgfscope}%
\pgfpathrectangle{\pgfqpoint{0.445556in}{0.499444in}}{\pgfqpoint{3.875000in}{1.155000in}}%
\pgfusepath{clip}%
\pgfsetbuttcap%
\pgfsetmiterjoin%
\pgfsetlinewidth{1.003750pt}%
\definecolor{currentstroke}{rgb}{0.000000,0.000000,0.000000}%
\pgfsetstrokecolor{currentstroke}%
\pgfsetdash{}{0pt}%
\pgfpathmoveto{\pgfqpoint{4.105704in}{0.499444in}}%
\pgfpathlineto{\pgfqpoint{4.167090in}{0.499444in}}%
\pgfpathlineto{\pgfqpoint{4.167090in}{0.499503in}}%
\pgfpathlineto{\pgfqpoint{4.105704in}{0.499503in}}%
\pgfpathlineto{\pgfqpoint{4.105704in}{0.499444in}}%
\pgfpathclose%
\pgfusepath{stroke}%
\end{pgfscope}%
\begin{pgfscope}%
\pgfpathrectangle{\pgfqpoint{0.445556in}{0.499444in}}{\pgfqpoint{3.875000in}{1.155000in}}%
\pgfusepath{clip}%
\pgfsetbuttcap%
\pgfsetmiterjoin%
\definecolor{currentfill}{rgb}{0.000000,0.000000,0.000000}%
\pgfsetfillcolor{currentfill}%
\pgfsetlinewidth{0.000000pt}%
\definecolor{currentstroke}{rgb}{0.000000,0.000000,0.000000}%
\pgfsetstrokecolor{currentstroke}%
\pgfsetstrokeopacity{0.000000}%
\pgfsetdash{}{0pt}%
\pgfpathmoveto{\pgfqpoint{0.483922in}{0.499444in}}%
\pgfpathlineto{\pgfqpoint{0.545308in}{0.499444in}}%
\pgfpathlineto{\pgfqpoint{0.545308in}{0.499444in}}%
\pgfpathlineto{\pgfqpoint{0.483922in}{0.499444in}}%
\pgfpathlineto{\pgfqpoint{0.483922in}{0.499444in}}%
\pgfpathclose%
\pgfusepath{fill}%
\end{pgfscope}%
\begin{pgfscope}%
\pgfpathrectangle{\pgfqpoint{0.445556in}{0.499444in}}{\pgfqpoint{3.875000in}{1.155000in}}%
\pgfusepath{clip}%
\pgfsetbuttcap%
\pgfsetmiterjoin%
\definecolor{currentfill}{rgb}{0.000000,0.000000,0.000000}%
\pgfsetfillcolor{currentfill}%
\pgfsetlinewidth{0.000000pt}%
\definecolor{currentstroke}{rgb}{0.000000,0.000000,0.000000}%
\pgfsetstrokecolor{currentstroke}%
\pgfsetstrokeopacity{0.000000}%
\pgfsetdash{}{0pt}%
\pgfpathmoveto{\pgfqpoint{0.637387in}{0.499444in}}%
\pgfpathlineto{\pgfqpoint{0.698774in}{0.499444in}}%
\pgfpathlineto{\pgfqpoint{0.698774in}{0.499444in}}%
\pgfpathlineto{\pgfqpoint{0.637387in}{0.499444in}}%
\pgfpathlineto{\pgfqpoint{0.637387in}{0.499444in}}%
\pgfpathclose%
\pgfusepath{fill}%
\end{pgfscope}%
\begin{pgfscope}%
\pgfpathrectangle{\pgfqpoint{0.445556in}{0.499444in}}{\pgfqpoint{3.875000in}{1.155000in}}%
\pgfusepath{clip}%
\pgfsetbuttcap%
\pgfsetmiterjoin%
\definecolor{currentfill}{rgb}{0.000000,0.000000,0.000000}%
\pgfsetfillcolor{currentfill}%
\pgfsetlinewidth{0.000000pt}%
\definecolor{currentstroke}{rgb}{0.000000,0.000000,0.000000}%
\pgfsetstrokecolor{currentstroke}%
\pgfsetstrokeopacity{0.000000}%
\pgfsetdash{}{0pt}%
\pgfpathmoveto{\pgfqpoint{0.790853in}{0.499444in}}%
\pgfpathlineto{\pgfqpoint{0.852239in}{0.499444in}}%
\pgfpathlineto{\pgfqpoint{0.852239in}{0.499444in}}%
\pgfpathlineto{\pgfqpoint{0.790853in}{0.499444in}}%
\pgfpathlineto{\pgfqpoint{0.790853in}{0.499444in}}%
\pgfpathclose%
\pgfusepath{fill}%
\end{pgfscope}%
\begin{pgfscope}%
\pgfpathrectangle{\pgfqpoint{0.445556in}{0.499444in}}{\pgfqpoint{3.875000in}{1.155000in}}%
\pgfusepath{clip}%
\pgfsetbuttcap%
\pgfsetmiterjoin%
\definecolor{currentfill}{rgb}{0.000000,0.000000,0.000000}%
\pgfsetfillcolor{currentfill}%
\pgfsetlinewidth{0.000000pt}%
\definecolor{currentstroke}{rgb}{0.000000,0.000000,0.000000}%
\pgfsetstrokecolor{currentstroke}%
\pgfsetstrokeopacity{0.000000}%
\pgfsetdash{}{0pt}%
\pgfpathmoveto{\pgfqpoint{0.944318in}{0.499444in}}%
\pgfpathlineto{\pgfqpoint{1.005704in}{0.499444in}}%
\pgfpathlineto{\pgfqpoint{1.005704in}{0.499444in}}%
\pgfpathlineto{\pgfqpoint{0.944318in}{0.499444in}}%
\pgfpathlineto{\pgfqpoint{0.944318in}{0.499444in}}%
\pgfpathclose%
\pgfusepath{fill}%
\end{pgfscope}%
\begin{pgfscope}%
\pgfpathrectangle{\pgfqpoint{0.445556in}{0.499444in}}{\pgfqpoint{3.875000in}{1.155000in}}%
\pgfusepath{clip}%
\pgfsetbuttcap%
\pgfsetmiterjoin%
\definecolor{currentfill}{rgb}{0.000000,0.000000,0.000000}%
\pgfsetfillcolor{currentfill}%
\pgfsetlinewidth{0.000000pt}%
\definecolor{currentstroke}{rgb}{0.000000,0.000000,0.000000}%
\pgfsetstrokecolor{currentstroke}%
\pgfsetstrokeopacity{0.000000}%
\pgfsetdash{}{0pt}%
\pgfpathmoveto{\pgfqpoint{1.097783in}{0.499444in}}%
\pgfpathlineto{\pgfqpoint{1.159170in}{0.499444in}}%
\pgfpathlineto{\pgfqpoint{1.159170in}{0.499444in}}%
\pgfpathlineto{\pgfqpoint{1.097783in}{0.499444in}}%
\pgfpathlineto{\pgfqpoint{1.097783in}{0.499444in}}%
\pgfpathclose%
\pgfusepath{fill}%
\end{pgfscope}%
\begin{pgfscope}%
\pgfpathrectangle{\pgfqpoint{0.445556in}{0.499444in}}{\pgfqpoint{3.875000in}{1.155000in}}%
\pgfusepath{clip}%
\pgfsetbuttcap%
\pgfsetmiterjoin%
\definecolor{currentfill}{rgb}{0.000000,0.000000,0.000000}%
\pgfsetfillcolor{currentfill}%
\pgfsetlinewidth{0.000000pt}%
\definecolor{currentstroke}{rgb}{0.000000,0.000000,0.000000}%
\pgfsetstrokecolor{currentstroke}%
\pgfsetstrokeopacity{0.000000}%
\pgfsetdash{}{0pt}%
\pgfpathmoveto{\pgfqpoint{1.251249in}{0.499444in}}%
\pgfpathlineto{\pgfqpoint{1.312635in}{0.499444in}}%
\pgfpathlineto{\pgfqpoint{1.312635in}{0.499444in}}%
\pgfpathlineto{\pgfqpoint{1.251249in}{0.499444in}}%
\pgfpathlineto{\pgfqpoint{1.251249in}{0.499444in}}%
\pgfpathclose%
\pgfusepath{fill}%
\end{pgfscope}%
\begin{pgfscope}%
\pgfpathrectangle{\pgfqpoint{0.445556in}{0.499444in}}{\pgfqpoint{3.875000in}{1.155000in}}%
\pgfusepath{clip}%
\pgfsetbuttcap%
\pgfsetmiterjoin%
\definecolor{currentfill}{rgb}{0.000000,0.000000,0.000000}%
\pgfsetfillcolor{currentfill}%
\pgfsetlinewidth{0.000000pt}%
\definecolor{currentstroke}{rgb}{0.000000,0.000000,0.000000}%
\pgfsetstrokecolor{currentstroke}%
\pgfsetstrokeopacity{0.000000}%
\pgfsetdash{}{0pt}%
\pgfpathmoveto{\pgfqpoint{1.404714in}{0.499444in}}%
\pgfpathlineto{\pgfqpoint{1.466100in}{0.499444in}}%
\pgfpathlineto{\pgfqpoint{1.466100in}{0.499444in}}%
\pgfpathlineto{\pgfqpoint{1.404714in}{0.499444in}}%
\pgfpathlineto{\pgfqpoint{1.404714in}{0.499444in}}%
\pgfpathclose%
\pgfusepath{fill}%
\end{pgfscope}%
\begin{pgfscope}%
\pgfpathrectangle{\pgfqpoint{0.445556in}{0.499444in}}{\pgfqpoint{3.875000in}{1.155000in}}%
\pgfusepath{clip}%
\pgfsetbuttcap%
\pgfsetmiterjoin%
\definecolor{currentfill}{rgb}{0.000000,0.000000,0.000000}%
\pgfsetfillcolor{currentfill}%
\pgfsetlinewidth{0.000000pt}%
\definecolor{currentstroke}{rgb}{0.000000,0.000000,0.000000}%
\pgfsetstrokecolor{currentstroke}%
\pgfsetstrokeopacity{0.000000}%
\pgfsetdash{}{0pt}%
\pgfpathmoveto{\pgfqpoint{1.558179in}{0.499444in}}%
\pgfpathlineto{\pgfqpoint{1.619566in}{0.499444in}}%
\pgfpathlineto{\pgfqpoint{1.619566in}{0.499444in}}%
\pgfpathlineto{\pgfqpoint{1.558179in}{0.499444in}}%
\pgfpathlineto{\pgfqpoint{1.558179in}{0.499444in}}%
\pgfpathclose%
\pgfusepath{fill}%
\end{pgfscope}%
\begin{pgfscope}%
\pgfpathrectangle{\pgfqpoint{0.445556in}{0.499444in}}{\pgfqpoint{3.875000in}{1.155000in}}%
\pgfusepath{clip}%
\pgfsetbuttcap%
\pgfsetmiterjoin%
\definecolor{currentfill}{rgb}{0.000000,0.000000,0.000000}%
\pgfsetfillcolor{currentfill}%
\pgfsetlinewidth{0.000000pt}%
\definecolor{currentstroke}{rgb}{0.000000,0.000000,0.000000}%
\pgfsetstrokecolor{currentstroke}%
\pgfsetstrokeopacity{0.000000}%
\pgfsetdash{}{0pt}%
\pgfpathmoveto{\pgfqpoint{1.711645in}{0.499444in}}%
\pgfpathlineto{\pgfqpoint{1.773031in}{0.499444in}}%
\pgfpathlineto{\pgfqpoint{1.773031in}{0.499444in}}%
\pgfpathlineto{\pgfqpoint{1.711645in}{0.499444in}}%
\pgfpathlineto{\pgfqpoint{1.711645in}{0.499444in}}%
\pgfpathclose%
\pgfusepath{fill}%
\end{pgfscope}%
\begin{pgfscope}%
\pgfpathrectangle{\pgfqpoint{0.445556in}{0.499444in}}{\pgfqpoint{3.875000in}{1.155000in}}%
\pgfusepath{clip}%
\pgfsetbuttcap%
\pgfsetmiterjoin%
\definecolor{currentfill}{rgb}{0.000000,0.000000,0.000000}%
\pgfsetfillcolor{currentfill}%
\pgfsetlinewidth{0.000000pt}%
\definecolor{currentstroke}{rgb}{0.000000,0.000000,0.000000}%
\pgfsetstrokecolor{currentstroke}%
\pgfsetstrokeopacity{0.000000}%
\pgfsetdash{}{0pt}%
\pgfpathmoveto{\pgfqpoint{1.865110in}{0.499444in}}%
\pgfpathlineto{\pgfqpoint{1.926496in}{0.499444in}}%
\pgfpathlineto{\pgfqpoint{1.926496in}{0.499444in}}%
\pgfpathlineto{\pgfqpoint{1.865110in}{0.499444in}}%
\pgfpathlineto{\pgfqpoint{1.865110in}{0.499444in}}%
\pgfpathclose%
\pgfusepath{fill}%
\end{pgfscope}%
\begin{pgfscope}%
\pgfpathrectangle{\pgfqpoint{0.445556in}{0.499444in}}{\pgfqpoint{3.875000in}{1.155000in}}%
\pgfusepath{clip}%
\pgfsetbuttcap%
\pgfsetmiterjoin%
\definecolor{currentfill}{rgb}{0.000000,0.000000,0.000000}%
\pgfsetfillcolor{currentfill}%
\pgfsetlinewidth{0.000000pt}%
\definecolor{currentstroke}{rgb}{0.000000,0.000000,0.000000}%
\pgfsetstrokecolor{currentstroke}%
\pgfsetstrokeopacity{0.000000}%
\pgfsetdash{}{0pt}%
\pgfpathmoveto{\pgfqpoint{2.018575in}{0.499444in}}%
\pgfpathlineto{\pgfqpoint{2.079962in}{0.499444in}}%
\pgfpathlineto{\pgfqpoint{2.079962in}{0.499444in}}%
\pgfpathlineto{\pgfqpoint{2.018575in}{0.499444in}}%
\pgfpathlineto{\pgfqpoint{2.018575in}{0.499444in}}%
\pgfpathclose%
\pgfusepath{fill}%
\end{pgfscope}%
\begin{pgfscope}%
\pgfpathrectangle{\pgfqpoint{0.445556in}{0.499444in}}{\pgfqpoint{3.875000in}{1.155000in}}%
\pgfusepath{clip}%
\pgfsetbuttcap%
\pgfsetmiterjoin%
\definecolor{currentfill}{rgb}{0.000000,0.000000,0.000000}%
\pgfsetfillcolor{currentfill}%
\pgfsetlinewidth{0.000000pt}%
\definecolor{currentstroke}{rgb}{0.000000,0.000000,0.000000}%
\pgfsetstrokecolor{currentstroke}%
\pgfsetstrokeopacity{0.000000}%
\pgfsetdash{}{0pt}%
\pgfpathmoveto{\pgfqpoint{2.172041in}{0.499444in}}%
\pgfpathlineto{\pgfqpoint{2.233427in}{0.499444in}}%
\pgfpathlineto{\pgfqpoint{2.233427in}{0.499444in}}%
\pgfpathlineto{\pgfqpoint{2.172041in}{0.499444in}}%
\pgfpathlineto{\pgfqpoint{2.172041in}{0.499444in}}%
\pgfpathclose%
\pgfusepath{fill}%
\end{pgfscope}%
\begin{pgfscope}%
\pgfpathrectangle{\pgfqpoint{0.445556in}{0.499444in}}{\pgfqpoint{3.875000in}{1.155000in}}%
\pgfusepath{clip}%
\pgfsetbuttcap%
\pgfsetmiterjoin%
\definecolor{currentfill}{rgb}{0.000000,0.000000,0.000000}%
\pgfsetfillcolor{currentfill}%
\pgfsetlinewidth{0.000000pt}%
\definecolor{currentstroke}{rgb}{0.000000,0.000000,0.000000}%
\pgfsetstrokecolor{currentstroke}%
\pgfsetstrokeopacity{0.000000}%
\pgfsetdash{}{0pt}%
\pgfpathmoveto{\pgfqpoint{2.325506in}{0.499444in}}%
\pgfpathlineto{\pgfqpoint{2.386892in}{0.499444in}}%
\pgfpathlineto{\pgfqpoint{2.386892in}{0.499561in}}%
\pgfpathlineto{\pgfqpoint{2.325506in}{0.499561in}}%
\pgfpathlineto{\pgfqpoint{2.325506in}{0.499444in}}%
\pgfpathclose%
\pgfusepath{fill}%
\end{pgfscope}%
\begin{pgfscope}%
\pgfpathrectangle{\pgfqpoint{0.445556in}{0.499444in}}{\pgfqpoint{3.875000in}{1.155000in}}%
\pgfusepath{clip}%
\pgfsetbuttcap%
\pgfsetmiterjoin%
\definecolor{currentfill}{rgb}{0.000000,0.000000,0.000000}%
\pgfsetfillcolor{currentfill}%
\pgfsetlinewidth{0.000000pt}%
\definecolor{currentstroke}{rgb}{0.000000,0.000000,0.000000}%
\pgfsetstrokecolor{currentstroke}%
\pgfsetstrokeopacity{0.000000}%
\pgfsetdash{}{0pt}%
\pgfpathmoveto{\pgfqpoint{2.478972in}{0.499444in}}%
\pgfpathlineto{\pgfqpoint{2.540358in}{0.499444in}}%
\pgfpathlineto{\pgfqpoint{2.540358in}{0.499854in}}%
\pgfpathlineto{\pgfqpoint{2.478972in}{0.499854in}}%
\pgfpathlineto{\pgfqpoint{2.478972in}{0.499444in}}%
\pgfpathclose%
\pgfusepath{fill}%
\end{pgfscope}%
\begin{pgfscope}%
\pgfpathrectangle{\pgfqpoint{0.445556in}{0.499444in}}{\pgfqpoint{3.875000in}{1.155000in}}%
\pgfusepath{clip}%
\pgfsetbuttcap%
\pgfsetmiterjoin%
\definecolor{currentfill}{rgb}{0.000000,0.000000,0.000000}%
\pgfsetfillcolor{currentfill}%
\pgfsetlinewidth{0.000000pt}%
\definecolor{currentstroke}{rgb}{0.000000,0.000000,0.000000}%
\pgfsetstrokecolor{currentstroke}%
\pgfsetstrokeopacity{0.000000}%
\pgfsetdash{}{0pt}%
\pgfpathmoveto{\pgfqpoint{2.632437in}{0.499444in}}%
\pgfpathlineto{\pgfqpoint{2.693823in}{0.499444in}}%
\pgfpathlineto{\pgfqpoint{2.693823in}{0.501521in}}%
\pgfpathlineto{\pgfqpoint{2.632437in}{0.501521in}}%
\pgfpathlineto{\pgfqpoint{2.632437in}{0.499444in}}%
\pgfpathclose%
\pgfusepath{fill}%
\end{pgfscope}%
\begin{pgfscope}%
\pgfpathrectangle{\pgfqpoint{0.445556in}{0.499444in}}{\pgfqpoint{3.875000in}{1.155000in}}%
\pgfusepath{clip}%
\pgfsetbuttcap%
\pgfsetmiterjoin%
\definecolor{currentfill}{rgb}{0.000000,0.000000,0.000000}%
\pgfsetfillcolor{currentfill}%
\pgfsetlinewidth{0.000000pt}%
\definecolor{currentstroke}{rgb}{0.000000,0.000000,0.000000}%
\pgfsetstrokecolor{currentstroke}%
\pgfsetstrokeopacity{0.000000}%
\pgfsetdash{}{0pt}%
\pgfpathmoveto{\pgfqpoint{2.785902in}{0.499444in}}%
\pgfpathlineto{\pgfqpoint{2.847288in}{0.499444in}}%
\pgfpathlineto{\pgfqpoint{2.847288in}{0.511171in}}%
\pgfpathlineto{\pgfqpoint{2.785902in}{0.511171in}}%
\pgfpathlineto{\pgfqpoint{2.785902in}{0.499444in}}%
\pgfpathclose%
\pgfusepath{fill}%
\end{pgfscope}%
\begin{pgfscope}%
\pgfpathrectangle{\pgfqpoint{0.445556in}{0.499444in}}{\pgfqpoint{3.875000in}{1.155000in}}%
\pgfusepath{clip}%
\pgfsetbuttcap%
\pgfsetmiterjoin%
\definecolor{currentfill}{rgb}{0.000000,0.000000,0.000000}%
\pgfsetfillcolor{currentfill}%
\pgfsetlinewidth{0.000000pt}%
\definecolor{currentstroke}{rgb}{0.000000,0.000000,0.000000}%
\pgfsetstrokecolor{currentstroke}%
\pgfsetstrokeopacity{0.000000}%
\pgfsetdash{}{0pt}%
\pgfpathmoveto{\pgfqpoint{2.939368in}{0.499444in}}%
\pgfpathlineto{\pgfqpoint{3.000754in}{0.499444in}}%
\pgfpathlineto{\pgfqpoint{3.000754in}{0.534889in}}%
\pgfpathlineto{\pgfqpoint{2.939368in}{0.534889in}}%
\pgfpathlineto{\pgfqpoint{2.939368in}{0.499444in}}%
\pgfpathclose%
\pgfusepath{fill}%
\end{pgfscope}%
\begin{pgfscope}%
\pgfpathrectangle{\pgfqpoint{0.445556in}{0.499444in}}{\pgfqpoint{3.875000in}{1.155000in}}%
\pgfusepath{clip}%
\pgfsetbuttcap%
\pgfsetmiterjoin%
\definecolor{currentfill}{rgb}{0.000000,0.000000,0.000000}%
\pgfsetfillcolor{currentfill}%
\pgfsetlinewidth{0.000000pt}%
\definecolor{currentstroke}{rgb}{0.000000,0.000000,0.000000}%
\pgfsetstrokecolor{currentstroke}%
\pgfsetstrokeopacity{0.000000}%
\pgfsetdash{}{0pt}%
\pgfpathmoveto{\pgfqpoint{3.092833in}{0.499444in}}%
\pgfpathlineto{\pgfqpoint{3.154219in}{0.499444in}}%
\pgfpathlineto{\pgfqpoint{3.154219in}{0.582998in}}%
\pgfpathlineto{\pgfqpoint{3.092833in}{0.582998in}}%
\pgfpathlineto{\pgfqpoint{3.092833in}{0.499444in}}%
\pgfpathclose%
\pgfusepath{fill}%
\end{pgfscope}%
\begin{pgfscope}%
\pgfpathrectangle{\pgfqpoint{0.445556in}{0.499444in}}{\pgfqpoint{3.875000in}{1.155000in}}%
\pgfusepath{clip}%
\pgfsetbuttcap%
\pgfsetmiterjoin%
\definecolor{currentfill}{rgb}{0.000000,0.000000,0.000000}%
\pgfsetfillcolor{currentfill}%
\pgfsetlinewidth{0.000000pt}%
\definecolor{currentstroke}{rgb}{0.000000,0.000000,0.000000}%
\pgfsetstrokecolor{currentstroke}%
\pgfsetstrokeopacity{0.000000}%
\pgfsetdash{}{0pt}%
\pgfpathmoveto{\pgfqpoint{3.246298in}{0.499444in}}%
\pgfpathlineto{\pgfqpoint{3.307684in}{0.499444in}}%
\pgfpathlineto{\pgfqpoint{3.307684in}{0.662340in}}%
\pgfpathlineto{\pgfqpoint{3.246298in}{0.662340in}}%
\pgfpathlineto{\pgfqpoint{3.246298in}{0.499444in}}%
\pgfpathclose%
\pgfusepath{fill}%
\end{pgfscope}%
\begin{pgfscope}%
\pgfpathrectangle{\pgfqpoint{0.445556in}{0.499444in}}{\pgfqpoint{3.875000in}{1.155000in}}%
\pgfusepath{clip}%
\pgfsetbuttcap%
\pgfsetmiterjoin%
\definecolor{currentfill}{rgb}{0.000000,0.000000,0.000000}%
\pgfsetfillcolor{currentfill}%
\pgfsetlinewidth{0.000000pt}%
\definecolor{currentstroke}{rgb}{0.000000,0.000000,0.000000}%
\pgfsetstrokecolor{currentstroke}%
\pgfsetstrokeopacity{0.000000}%
\pgfsetdash{}{0pt}%
\pgfpathmoveto{\pgfqpoint{3.399764in}{0.499444in}}%
\pgfpathlineto{\pgfqpoint{3.461150in}{0.499444in}}%
\pgfpathlineto{\pgfqpoint{3.461150in}{0.763967in}}%
\pgfpathlineto{\pgfqpoint{3.399764in}{0.763967in}}%
\pgfpathlineto{\pgfqpoint{3.399764in}{0.499444in}}%
\pgfpathclose%
\pgfusepath{fill}%
\end{pgfscope}%
\begin{pgfscope}%
\pgfpathrectangle{\pgfqpoint{0.445556in}{0.499444in}}{\pgfqpoint{3.875000in}{1.155000in}}%
\pgfusepath{clip}%
\pgfsetbuttcap%
\pgfsetmiterjoin%
\definecolor{currentfill}{rgb}{0.000000,0.000000,0.000000}%
\pgfsetfillcolor{currentfill}%
\pgfsetlinewidth{0.000000pt}%
\definecolor{currentstroke}{rgb}{0.000000,0.000000,0.000000}%
\pgfsetstrokecolor{currentstroke}%
\pgfsetstrokeopacity{0.000000}%
\pgfsetdash{}{0pt}%
\pgfpathmoveto{\pgfqpoint{3.553229in}{0.499444in}}%
\pgfpathlineto{\pgfqpoint{3.614615in}{0.499444in}}%
\pgfpathlineto{\pgfqpoint{3.614615in}{0.871794in}}%
\pgfpathlineto{\pgfqpoint{3.553229in}{0.871794in}}%
\pgfpathlineto{\pgfqpoint{3.553229in}{0.499444in}}%
\pgfpathclose%
\pgfusepath{fill}%
\end{pgfscope}%
\begin{pgfscope}%
\pgfpathrectangle{\pgfqpoint{0.445556in}{0.499444in}}{\pgfqpoint{3.875000in}{1.155000in}}%
\pgfusepath{clip}%
\pgfsetbuttcap%
\pgfsetmiterjoin%
\definecolor{currentfill}{rgb}{0.000000,0.000000,0.000000}%
\pgfsetfillcolor{currentfill}%
\pgfsetlinewidth{0.000000pt}%
\definecolor{currentstroke}{rgb}{0.000000,0.000000,0.000000}%
\pgfsetstrokecolor{currentstroke}%
\pgfsetstrokeopacity{0.000000}%
\pgfsetdash{}{0pt}%
\pgfpathmoveto{\pgfqpoint{3.706694in}{0.499444in}}%
\pgfpathlineto{\pgfqpoint{3.768080in}{0.499444in}}%
\pgfpathlineto{\pgfqpoint{3.768080in}{0.925810in}}%
\pgfpathlineto{\pgfqpoint{3.706694in}{0.925810in}}%
\pgfpathlineto{\pgfqpoint{3.706694in}{0.499444in}}%
\pgfpathclose%
\pgfusepath{fill}%
\end{pgfscope}%
\begin{pgfscope}%
\pgfpathrectangle{\pgfqpoint{0.445556in}{0.499444in}}{\pgfqpoint{3.875000in}{1.155000in}}%
\pgfusepath{clip}%
\pgfsetbuttcap%
\pgfsetmiterjoin%
\definecolor{currentfill}{rgb}{0.000000,0.000000,0.000000}%
\pgfsetfillcolor{currentfill}%
\pgfsetlinewidth{0.000000pt}%
\definecolor{currentstroke}{rgb}{0.000000,0.000000,0.000000}%
\pgfsetstrokecolor{currentstroke}%
\pgfsetstrokeopacity{0.000000}%
\pgfsetdash{}{0pt}%
\pgfpathmoveto{\pgfqpoint{3.860160in}{0.499444in}}%
\pgfpathlineto{\pgfqpoint{3.921546in}{0.499444in}}%
\pgfpathlineto{\pgfqpoint{3.921546in}{0.913586in}}%
\pgfpathlineto{\pgfqpoint{3.860160in}{0.913586in}}%
\pgfpathlineto{\pgfqpoint{3.860160in}{0.499444in}}%
\pgfpathclose%
\pgfusepath{fill}%
\end{pgfscope}%
\begin{pgfscope}%
\pgfpathrectangle{\pgfqpoint{0.445556in}{0.499444in}}{\pgfqpoint{3.875000in}{1.155000in}}%
\pgfusepath{clip}%
\pgfsetbuttcap%
\pgfsetmiterjoin%
\definecolor{currentfill}{rgb}{0.000000,0.000000,0.000000}%
\pgfsetfillcolor{currentfill}%
\pgfsetlinewidth{0.000000pt}%
\definecolor{currentstroke}{rgb}{0.000000,0.000000,0.000000}%
\pgfsetstrokecolor{currentstroke}%
\pgfsetstrokeopacity{0.000000}%
\pgfsetdash{}{0pt}%
\pgfpathmoveto{\pgfqpoint{4.013625in}{0.499444in}}%
\pgfpathlineto{\pgfqpoint{4.075011in}{0.499444in}}%
\pgfpathlineto{\pgfqpoint{4.075011in}{0.851907in}}%
\pgfpathlineto{\pgfqpoint{4.013625in}{0.851907in}}%
\pgfpathlineto{\pgfqpoint{4.013625in}{0.499444in}}%
\pgfpathclose%
\pgfusepath{fill}%
\end{pgfscope}%
\begin{pgfscope}%
\pgfpathrectangle{\pgfqpoint{0.445556in}{0.499444in}}{\pgfqpoint{3.875000in}{1.155000in}}%
\pgfusepath{clip}%
\pgfsetbuttcap%
\pgfsetmiterjoin%
\definecolor{currentfill}{rgb}{0.000000,0.000000,0.000000}%
\pgfsetfillcolor{currentfill}%
\pgfsetlinewidth{0.000000pt}%
\definecolor{currentstroke}{rgb}{0.000000,0.000000,0.000000}%
\pgfsetstrokecolor{currentstroke}%
\pgfsetstrokeopacity{0.000000}%
\pgfsetdash{}{0pt}%
\pgfpathmoveto{\pgfqpoint{4.167090in}{0.499444in}}%
\pgfpathlineto{\pgfqpoint{4.228476in}{0.499444in}}%
\pgfpathlineto{\pgfqpoint{4.228476in}{0.681583in}}%
\pgfpathlineto{\pgfqpoint{4.167090in}{0.681583in}}%
\pgfpathlineto{\pgfqpoint{4.167090in}{0.499444in}}%
\pgfpathclose%
\pgfusepath{fill}%
\end{pgfscope}%
\begin{pgfscope}%
\pgfsetbuttcap%
\pgfsetroundjoin%
\definecolor{currentfill}{rgb}{0.000000,0.000000,0.000000}%
\pgfsetfillcolor{currentfill}%
\pgfsetlinewidth{0.803000pt}%
\definecolor{currentstroke}{rgb}{0.000000,0.000000,0.000000}%
\pgfsetstrokecolor{currentstroke}%
\pgfsetdash{}{0pt}%
\pgfsys@defobject{currentmarker}{\pgfqpoint{0.000000in}{-0.048611in}}{\pgfqpoint{0.000000in}{0.000000in}}{%
\pgfpathmoveto{\pgfqpoint{0.000000in}{0.000000in}}%
\pgfpathlineto{\pgfqpoint{0.000000in}{-0.048611in}}%
\pgfusepath{stroke,fill}%
}%
\begin{pgfscope}%
\pgfsys@transformshift{0.483922in}{0.499444in}%
\pgfsys@useobject{currentmarker}{}%
\end{pgfscope}%
\end{pgfscope}%
\begin{pgfscope}%
\definecolor{textcolor}{rgb}{0.000000,0.000000,0.000000}%
\pgfsetstrokecolor{textcolor}%
\pgfsetfillcolor{textcolor}%
\pgftext[x=0.483922in,y=0.402222in,,top]{\color{textcolor}\rmfamily\fontsize{10.000000}{12.000000}\selectfont 0.0}%
\end{pgfscope}%
\begin{pgfscope}%
\pgfsetbuttcap%
\pgfsetroundjoin%
\definecolor{currentfill}{rgb}{0.000000,0.000000,0.000000}%
\pgfsetfillcolor{currentfill}%
\pgfsetlinewidth{0.803000pt}%
\definecolor{currentstroke}{rgb}{0.000000,0.000000,0.000000}%
\pgfsetstrokecolor{currentstroke}%
\pgfsetdash{}{0pt}%
\pgfsys@defobject{currentmarker}{\pgfqpoint{0.000000in}{-0.048611in}}{\pgfqpoint{0.000000in}{0.000000in}}{%
\pgfpathmoveto{\pgfqpoint{0.000000in}{0.000000in}}%
\pgfpathlineto{\pgfqpoint{0.000000in}{-0.048611in}}%
\pgfusepath{stroke,fill}%
}%
\begin{pgfscope}%
\pgfsys@transformshift{0.867585in}{0.499444in}%
\pgfsys@useobject{currentmarker}{}%
\end{pgfscope}%
\end{pgfscope}%
\begin{pgfscope}%
\definecolor{textcolor}{rgb}{0.000000,0.000000,0.000000}%
\pgfsetstrokecolor{textcolor}%
\pgfsetfillcolor{textcolor}%
\pgftext[x=0.867585in,y=0.402222in,,top]{\color{textcolor}\rmfamily\fontsize{10.000000}{12.000000}\selectfont 0.1}%
\end{pgfscope}%
\begin{pgfscope}%
\pgfsetbuttcap%
\pgfsetroundjoin%
\definecolor{currentfill}{rgb}{0.000000,0.000000,0.000000}%
\pgfsetfillcolor{currentfill}%
\pgfsetlinewidth{0.803000pt}%
\definecolor{currentstroke}{rgb}{0.000000,0.000000,0.000000}%
\pgfsetstrokecolor{currentstroke}%
\pgfsetdash{}{0pt}%
\pgfsys@defobject{currentmarker}{\pgfqpoint{0.000000in}{-0.048611in}}{\pgfqpoint{0.000000in}{0.000000in}}{%
\pgfpathmoveto{\pgfqpoint{0.000000in}{0.000000in}}%
\pgfpathlineto{\pgfqpoint{0.000000in}{-0.048611in}}%
\pgfusepath{stroke,fill}%
}%
\begin{pgfscope}%
\pgfsys@transformshift{1.251249in}{0.499444in}%
\pgfsys@useobject{currentmarker}{}%
\end{pgfscope}%
\end{pgfscope}%
\begin{pgfscope}%
\definecolor{textcolor}{rgb}{0.000000,0.000000,0.000000}%
\pgfsetstrokecolor{textcolor}%
\pgfsetfillcolor{textcolor}%
\pgftext[x=1.251249in,y=0.402222in,,top]{\color{textcolor}\rmfamily\fontsize{10.000000}{12.000000}\selectfont 0.2}%
\end{pgfscope}%
\begin{pgfscope}%
\pgfsetbuttcap%
\pgfsetroundjoin%
\definecolor{currentfill}{rgb}{0.000000,0.000000,0.000000}%
\pgfsetfillcolor{currentfill}%
\pgfsetlinewidth{0.803000pt}%
\definecolor{currentstroke}{rgb}{0.000000,0.000000,0.000000}%
\pgfsetstrokecolor{currentstroke}%
\pgfsetdash{}{0pt}%
\pgfsys@defobject{currentmarker}{\pgfqpoint{0.000000in}{-0.048611in}}{\pgfqpoint{0.000000in}{0.000000in}}{%
\pgfpathmoveto{\pgfqpoint{0.000000in}{0.000000in}}%
\pgfpathlineto{\pgfqpoint{0.000000in}{-0.048611in}}%
\pgfusepath{stroke,fill}%
}%
\begin{pgfscope}%
\pgfsys@transformshift{1.634912in}{0.499444in}%
\pgfsys@useobject{currentmarker}{}%
\end{pgfscope}%
\end{pgfscope}%
\begin{pgfscope}%
\definecolor{textcolor}{rgb}{0.000000,0.000000,0.000000}%
\pgfsetstrokecolor{textcolor}%
\pgfsetfillcolor{textcolor}%
\pgftext[x=1.634912in,y=0.402222in,,top]{\color{textcolor}\rmfamily\fontsize{10.000000}{12.000000}\selectfont 0.3}%
\end{pgfscope}%
\begin{pgfscope}%
\pgfsetbuttcap%
\pgfsetroundjoin%
\definecolor{currentfill}{rgb}{0.000000,0.000000,0.000000}%
\pgfsetfillcolor{currentfill}%
\pgfsetlinewidth{0.803000pt}%
\definecolor{currentstroke}{rgb}{0.000000,0.000000,0.000000}%
\pgfsetstrokecolor{currentstroke}%
\pgfsetdash{}{0pt}%
\pgfsys@defobject{currentmarker}{\pgfqpoint{0.000000in}{-0.048611in}}{\pgfqpoint{0.000000in}{0.000000in}}{%
\pgfpathmoveto{\pgfqpoint{0.000000in}{0.000000in}}%
\pgfpathlineto{\pgfqpoint{0.000000in}{-0.048611in}}%
\pgfusepath{stroke,fill}%
}%
\begin{pgfscope}%
\pgfsys@transformshift{2.018575in}{0.499444in}%
\pgfsys@useobject{currentmarker}{}%
\end{pgfscope}%
\end{pgfscope}%
\begin{pgfscope}%
\definecolor{textcolor}{rgb}{0.000000,0.000000,0.000000}%
\pgfsetstrokecolor{textcolor}%
\pgfsetfillcolor{textcolor}%
\pgftext[x=2.018575in,y=0.402222in,,top]{\color{textcolor}\rmfamily\fontsize{10.000000}{12.000000}\selectfont 0.4}%
\end{pgfscope}%
\begin{pgfscope}%
\pgfsetbuttcap%
\pgfsetroundjoin%
\definecolor{currentfill}{rgb}{0.000000,0.000000,0.000000}%
\pgfsetfillcolor{currentfill}%
\pgfsetlinewidth{0.803000pt}%
\definecolor{currentstroke}{rgb}{0.000000,0.000000,0.000000}%
\pgfsetstrokecolor{currentstroke}%
\pgfsetdash{}{0pt}%
\pgfsys@defobject{currentmarker}{\pgfqpoint{0.000000in}{-0.048611in}}{\pgfqpoint{0.000000in}{0.000000in}}{%
\pgfpathmoveto{\pgfqpoint{0.000000in}{0.000000in}}%
\pgfpathlineto{\pgfqpoint{0.000000in}{-0.048611in}}%
\pgfusepath{stroke,fill}%
}%
\begin{pgfscope}%
\pgfsys@transformshift{2.402239in}{0.499444in}%
\pgfsys@useobject{currentmarker}{}%
\end{pgfscope}%
\end{pgfscope}%
\begin{pgfscope}%
\definecolor{textcolor}{rgb}{0.000000,0.000000,0.000000}%
\pgfsetstrokecolor{textcolor}%
\pgfsetfillcolor{textcolor}%
\pgftext[x=2.402239in,y=0.402222in,,top]{\color{textcolor}\rmfamily\fontsize{10.000000}{12.000000}\selectfont 0.5}%
\end{pgfscope}%
\begin{pgfscope}%
\pgfsetbuttcap%
\pgfsetroundjoin%
\definecolor{currentfill}{rgb}{0.000000,0.000000,0.000000}%
\pgfsetfillcolor{currentfill}%
\pgfsetlinewidth{0.803000pt}%
\definecolor{currentstroke}{rgb}{0.000000,0.000000,0.000000}%
\pgfsetstrokecolor{currentstroke}%
\pgfsetdash{}{0pt}%
\pgfsys@defobject{currentmarker}{\pgfqpoint{0.000000in}{-0.048611in}}{\pgfqpoint{0.000000in}{0.000000in}}{%
\pgfpathmoveto{\pgfqpoint{0.000000in}{0.000000in}}%
\pgfpathlineto{\pgfqpoint{0.000000in}{-0.048611in}}%
\pgfusepath{stroke,fill}%
}%
\begin{pgfscope}%
\pgfsys@transformshift{2.785902in}{0.499444in}%
\pgfsys@useobject{currentmarker}{}%
\end{pgfscope}%
\end{pgfscope}%
\begin{pgfscope}%
\definecolor{textcolor}{rgb}{0.000000,0.000000,0.000000}%
\pgfsetstrokecolor{textcolor}%
\pgfsetfillcolor{textcolor}%
\pgftext[x=2.785902in,y=0.402222in,,top]{\color{textcolor}\rmfamily\fontsize{10.000000}{12.000000}\selectfont 0.6}%
\end{pgfscope}%
\begin{pgfscope}%
\pgfsetbuttcap%
\pgfsetroundjoin%
\definecolor{currentfill}{rgb}{0.000000,0.000000,0.000000}%
\pgfsetfillcolor{currentfill}%
\pgfsetlinewidth{0.803000pt}%
\definecolor{currentstroke}{rgb}{0.000000,0.000000,0.000000}%
\pgfsetstrokecolor{currentstroke}%
\pgfsetdash{}{0pt}%
\pgfsys@defobject{currentmarker}{\pgfqpoint{0.000000in}{-0.048611in}}{\pgfqpoint{0.000000in}{0.000000in}}{%
\pgfpathmoveto{\pgfqpoint{0.000000in}{0.000000in}}%
\pgfpathlineto{\pgfqpoint{0.000000in}{-0.048611in}}%
\pgfusepath{stroke,fill}%
}%
\begin{pgfscope}%
\pgfsys@transformshift{3.169566in}{0.499444in}%
\pgfsys@useobject{currentmarker}{}%
\end{pgfscope}%
\end{pgfscope}%
\begin{pgfscope}%
\definecolor{textcolor}{rgb}{0.000000,0.000000,0.000000}%
\pgfsetstrokecolor{textcolor}%
\pgfsetfillcolor{textcolor}%
\pgftext[x=3.169566in,y=0.402222in,,top]{\color{textcolor}\rmfamily\fontsize{10.000000}{12.000000}\selectfont 0.7}%
\end{pgfscope}%
\begin{pgfscope}%
\pgfsetbuttcap%
\pgfsetroundjoin%
\definecolor{currentfill}{rgb}{0.000000,0.000000,0.000000}%
\pgfsetfillcolor{currentfill}%
\pgfsetlinewidth{0.803000pt}%
\definecolor{currentstroke}{rgb}{0.000000,0.000000,0.000000}%
\pgfsetstrokecolor{currentstroke}%
\pgfsetdash{}{0pt}%
\pgfsys@defobject{currentmarker}{\pgfqpoint{0.000000in}{-0.048611in}}{\pgfqpoint{0.000000in}{0.000000in}}{%
\pgfpathmoveto{\pgfqpoint{0.000000in}{0.000000in}}%
\pgfpathlineto{\pgfqpoint{0.000000in}{-0.048611in}}%
\pgfusepath{stroke,fill}%
}%
\begin{pgfscope}%
\pgfsys@transformshift{3.553229in}{0.499444in}%
\pgfsys@useobject{currentmarker}{}%
\end{pgfscope}%
\end{pgfscope}%
\begin{pgfscope}%
\definecolor{textcolor}{rgb}{0.000000,0.000000,0.000000}%
\pgfsetstrokecolor{textcolor}%
\pgfsetfillcolor{textcolor}%
\pgftext[x=3.553229in,y=0.402222in,,top]{\color{textcolor}\rmfamily\fontsize{10.000000}{12.000000}\selectfont 0.8}%
\end{pgfscope}%
\begin{pgfscope}%
\pgfsetbuttcap%
\pgfsetroundjoin%
\definecolor{currentfill}{rgb}{0.000000,0.000000,0.000000}%
\pgfsetfillcolor{currentfill}%
\pgfsetlinewidth{0.803000pt}%
\definecolor{currentstroke}{rgb}{0.000000,0.000000,0.000000}%
\pgfsetstrokecolor{currentstroke}%
\pgfsetdash{}{0pt}%
\pgfsys@defobject{currentmarker}{\pgfqpoint{0.000000in}{-0.048611in}}{\pgfqpoint{0.000000in}{0.000000in}}{%
\pgfpathmoveto{\pgfqpoint{0.000000in}{0.000000in}}%
\pgfpathlineto{\pgfqpoint{0.000000in}{-0.048611in}}%
\pgfusepath{stroke,fill}%
}%
\begin{pgfscope}%
\pgfsys@transformshift{3.936892in}{0.499444in}%
\pgfsys@useobject{currentmarker}{}%
\end{pgfscope}%
\end{pgfscope}%
\begin{pgfscope}%
\definecolor{textcolor}{rgb}{0.000000,0.000000,0.000000}%
\pgfsetstrokecolor{textcolor}%
\pgfsetfillcolor{textcolor}%
\pgftext[x=3.936892in,y=0.402222in,,top]{\color{textcolor}\rmfamily\fontsize{10.000000}{12.000000}\selectfont 0.9}%
\end{pgfscope}%
\begin{pgfscope}%
\pgfsetbuttcap%
\pgfsetroundjoin%
\definecolor{currentfill}{rgb}{0.000000,0.000000,0.000000}%
\pgfsetfillcolor{currentfill}%
\pgfsetlinewidth{0.803000pt}%
\definecolor{currentstroke}{rgb}{0.000000,0.000000,0.000000}%
\pgfsetstrokecolor{currentstroke}%
\pgfsetdash{}{0pt}%
\pgfsys@defobject{currentmarker}{\pgfqpoint{0.000000in}{-0.048611in}}{\pgfqpoint{0.000000in}{0.000000in}}{%
\pgfpathmoveto{\pgfqpoint{0.000000in}{0.000000in}}%
\pgfpathlineto{\pgfqpoint{0.000000in}{-0.048611in}}%
\pgfusepath{stroke,fill}%
}%
\begin{pgfscope}%
\pgfsys@transformshift{4.320556in}{0.499444in}%
\pgfsys@useobject{currentmarker}{}%
\end{pgfscope}%
\end{pgfscope}%
\begin{pgfscope}%
\definecolor{textcolor}{rgb}{0.000000,0.000000,0.000000}%
\pgfsetstrokecolor{textcolor}%
\pgfsetfillcolor{textcolor}%
\pgftext[x=4.320556in,y=0.402222in,,top]{\color{textcolor}\rmfamily\fontsize{10.000000}{12.000000}\selectfont 1.0}%
\end{pgfscope}%
\begin{pgfscope}%
\definecolor{textcolor}{rgb}{0.000000,0.000000,0.000000}%
\pgfsetstrokecolor{textcolor}%
\pgfsetfillcolor{textcolor}%
\pgftext[x=2.383056in,y=0.223333in,,top]{\color{textcolor}\rmfamily\fontsize{10.000000}{12.000000}\selectfont \(\displaystyle p\)}%
\end{pgfscope}%
\begin{pgfscope}%
\pgfsetbuttcap%
\pgfsetroundjoin%
\definecolor{currentfill}{rgb}{0.000000,0.000000,0.000000}%
\pgfsetfillcolor{currentfill}%
\pgfsetlinewidth{0.803000pt}%
\definecolor{currentstroke}{rgb}{0.000000,0.000000,0.000000}%
\pgfsetstrokecolor{currentstroke}%
\pgfsetdash{}{0pt}%
\pgfsys@defobject{currentmarker}{\pgfqpoint{-0.048611in}{0.000000in}}{\pgfqpoint{-0.000000in}{0.000000in}}{%
\pgfpathmoveto{\pgfqpoint{-0.000000in}{0.000000in}}%
\pgfpathlineto{\pgfqpoint{-0.048611in}{0.000000in}}%
\pgfusepath{stroke,fill}%
}%
\begin{pgfscope}%
\pgfsys@transformshift{0.445556in}{0.499444in}%
\pgfsys@useobject{currentmarker}{}%
\end{pgfscope}%
\end{pgfscope}%
\begin{pgfscope}%
\definecolor{textcolor}{rgb}{0.000000,0.000000,0.000000}%
\pgfsetstrokecolor{textcolor}%
\pgfsetfillcolor{textcolor}%
\pgftext[x=0.278889in, y=0.451250in, left, base]{\color{textcolor}\rmfamily\fontsize{10.000000}{12.000000}\selectfont \(\displaystyle {0}\)}%
\end{pgfscope}%
\begin{pgfscope}%
\pgfsetbuttcap%
\pgfsetroundjoin%
\definecolor{currentfill}{rgb}{0.000000,0.000000,0.000000}%
\pgfsetfillcolor{currentfill}%
\pgfsetlinewidth{0.803000pt}%
\definecolor{currentstroke}{rgb}{0.000000,0.000000,0.000000}%
\pgfsetstrokecolor{currentstroke}%
\pgfsetdash{}{0pt}%
\pgfsys@defobject{currentmarker}{\pgfqpoint{-0.048611in}{0.000000in}}{\pgfqpoint{-0.000000in}{0.000000in}}{%
\pgfpathmoveto{\pgfqpoint{-0.000000in}{0.000000in}}%
\pgfpathlineto{\pgfqpoint{-0.048611in}{0.000000in}}%
\pgfusepath{stroke,fill}%
}%
\begin{pgfscope}%
\pgfsys@transformshift{0.445556in}{0.791601in}%
\pgfsys@useobject{currentmarker}{}%
\end{pgfscope}%
\end{pgfscope}%
\begin{pgfscope}%
\definecolor{textcolor}{rgb}{0.000000,0.000000,0.000000}%
\pgfsetstrokecolor{textcolor}%
\pgfsetfillcolor{textcolor}%
\pgftext[x=0.278889in, y=0.743407in, left, base]{\color{textcolor}\rmfamily\fontsize{10.000000}{12.000000}\selectfont \(\displaystyle {2}\)}%
\end{pgfscope}%
\begin{pgfscope}%
\pgfsetbuttcap%
\pgfsetroundjoin%
\definecolor{currentfill}{rgb}{0.000000,0.000000,0.000000}%
\pgfsetfillcolor{currentfill}%
\pgfsetlinewidth{0.803000pt}%
\definecolor{currentstroke}{rgb}{0.000000,0.000000,0.000000}%
\pgfsetstrokecolor{currentstroke}%
\pgfsetdash{}{0pt}%
\pgfsys@defobject{currentmarker}{\pgfqpoint{-0.048611in}{0.000000in}}{\pgfqpoint{-0.000000in}{0.000000in}}{%
\pgfpathmoveto{\pgfqpoint{-0.000000in}{0.000000in}}%
\pgfpathlineto{\pgfqpoint{-0.048611in}{0.000000in}}%
\pgfusepath{stroke,fill}%
}%
\begin{pgfscope}%
\pgfsys@transformshift{0.445556in}{1.083759in}%
\pgfsys@useobject{currentmarker}{}%
\end{pgfscope}%
\end{pgfscope}%
\begin{pgfscope}%
\definecolor{textcolor}{rgb}{0.000000,0.000000,0.000000}%
\pgfsetstrokecolor{textcolor}%
\pgfsetfillcolor{textcolor}%
\pgftext[x=0.278889in, y=1.035564in, left, base]{\color{textcolor}\rmfamily\fontsize{10.000000}{12.000000}\selectfont \(\displaystyle {4}\)}%
\end{pgfscope}%
\begin{pgfscope}%
\pgfsetbuttcap%
\pgfsetroundjoin%
\definecolor{currentfill}{rgb}{0.000000,0.000000,0.000000}%
\pgfsetfillcolor{currentfill}%
\pgfsetlinewidth{0.803000pt}%
\definecolor{currentstroke}{rgb}{0.000000,0.000000,0.000000}%
\pgfsetstrokecolor{currentstroke}%
\pgfsetdash{}{0pt}%
\pgfsys@defobject{currentmarker}{\pgfqpoint{-0.048611in}{0.000000in}}{\pgfqpoint{-0.000000in}{0.000000in}}{%
\pgfpathmoveto{\pgfqpoint{-0.000000in}{0.000000in}}%
\pgfpathlineto{\pgfqpoint{-0.048611in}{0.000000in}}%
\pgfusepath{stroke,fill}%
}%
\begin{pgfscope}%
\pgfsys@transformshift{0.445556in}{1.375916in}%
\pgfsys@useobject{currentmarker}{}%
\end{pgfscope}%
\end{pgfscope}%
\begin{pgfscope}%
\definecolor{textcolor}{rgb}{0.000000,0.000000,0.000000}%
\pgfsetstrokecolor{textcolor}%
\pgfsetfillcolor{textcolor}%
\pgftext[x=0.278889in, y=1.327722in, left, base]{\color{textcolor}\rmfamily\fontsize{10.000000}{12.000000}\selectfont \(\displaystyle {6}\)}%
\end{pgfscope}%
\begin{pgfscope}%
\definecolor{textcolor}{rgb}{0.000000,0.000000,0.000000}%
\pgfsetstrokecolor{textcolor}%
\pgfsetfillcolor{textcolor}%
\pgftext[x=0.223333in,y=1.076944in,,bottom,rotate=90.000000]{\color{textcolor}\rmfamily\fontsize{10.000000}{12.000000}\selectfont Percent of Data Set}%
\end{pgfscope}%
\begin{pgfscope}%
\pgfsetrectcap%
\pgfsetmiterjoin%
\pgfsetlinewidth{0.803000pt}%
\definecolor{currentstroke}{rgb}{0.000000,0.000000,0.000000}%
\pgfsetstrokecolor{currentstroke}%
\pgfsetdash{}{0pt}%
\pgfpathmoveto{\pgfqpoint{0.445556in}{0.499444in}}%
\pgfpathlineto{\pgfqpoint{0.445556in}{1.654444in}}%
\pgfusepath{stroke}%
\end{pgfscope}%
\begin{pgfscope}%
\pgfsetrectcap%
\pgfsetmiterjoin%
\pgfsetlinewidth{0.803000pt}%
\definecolor{currentstroke}{rgb}{0.000000,0.000000,0.000000}%
\pgfsetstrokecolor{currentstroke}%
\pgfsetdash{}{0pt}%
\pgfpathmoveto{\pgfqpoint{4.320556in}{0.499444in}}%
\pgfpathlineto{\pgfqpoint{4.320556in}{1.654444in}}%
\pgfusepath{stroke}%
\end{pgfscope}%
\begin{pgfscope}%
\pgfsetrectcap%
\pgfsetmiterjoin%
\pgfsetlinewidth{0.803000pt}%
\definecolor{currentstroke}{rgb}{0.000000,0.000000,0.000000}%
\pgfsetstrokecolor{currentstroke}%
\pgfsetdash{}{0pt}%
\pgfpathmoveto{\pgfqpoint{0.445556in}{0.499444in}}%
\pgfpathlineto{\pgfqpoint{4.320556in}{0.499444in}}%
\pgfusepath{stroke}%
\end{pgfscope}%
\begin{pgfscope}%
\pgfsetrectcap%
\pgfsetmiterjoin%
\pgfsetlinewidth{0.803000pt}%
\definecolor{currentstroke}{rgb}{0.000000,0.000000,0.000000}%
\pgfsetstrokecolor{currentstroke}%
\pgfsetdash{}{0pt}%
\pgfpathmoveto{\pgfqpoint{0.445556in}{1.654444in}}%
\pgfpathlineto{\pgfqpoint{4.320556in}{1.654444in}}%
\pgfusepath{stroke}%
\end{pgfscope}%
\begin{pgfscope}%
\pgfsetbuttcap%
\pgfsetmiterjoin%
\definecolor{currentfill}{rgb}{1.000000,1.000000,1.000000}%
\pgfsetfillcolor{currentfill}%
\pgfsetfillopacity{0.800000}%
\pgfsetlinewidth{1.003750pt}%
\definecolor{currentstroke}{rgb}{0.800000,0.800000,0.800000}%
\pgfsetstrokecolor{currentstroke}%
\pgfsetstrokeopacity{0.800000}%
\pgfsetdash{}{0pt}%
\pgfpathmoveto{\pgfqpoint{3.543611in}{1.154445in}}%
\pgfpathlineto{\pgfqpoint{4.223333in}{1.154445in}}%
\pgfpathquadraticcurveto{\pgfqpoint{4.251111in}{1.154445in}}{\pgfqpoint{4.251111in}{1.182222in}}%
\pgfpathlineto{\pgfqpoint{4.251111in}{1.557222in}}%
\pgfpathquadraticcurveto{\pgfqpoint{4.251111in}{1.585000in}}{\pgfqpoint{4.223333in}{1.585000in}}%
\pgfpathlineto{\pgfqpoint{3.543611in}{1.585000in}}%
\pgfpathquadraticcurveto{\pgfqpoint{3.515833in}{1.585000in}}{\pgfqpoint{3.515833in}{1.557222in}}%
\pgfpathlineto{\pgfqpoint{3.515833in}{1.182222in}}%
\pgfpathquadraticcurveto{\pgfqpoint{3.515833in}{1.154445in}}{\pgfqpoint{3.543611in}{1.154445in}}%
\pgfpathlineto{\pgfqpoint{3.543611in}{1.154445in}}%
\pgfpathclose%
\pgfusepath{stroke,fill}%
\end{pgfscope}%
\begin{pgfscope}%
\pgfsetbuttcap%
\pgfsetmiterjoin%
\pgfsetlinewidth{1.003750pt}%
\definecolor{currentstroke}{rgb}{0.000000,0.000000,0.000000}%
\pgfsetstrokecolor{currentstroke}%
\pgfsetdash{}{0pt}%
\pgfpathmoveto{\pgfqpoint{3.571389in}{1.432222in}}%
\pgfpathlineto{\pgfqpoint{3.849167in}{1.432222in}}%
\pgfpathlineto{\pgfqpoint{3.849167in}{1.529444in}}%
\pgfpathlineto{\pgfqpoint{3.571389in}{1.529444in}}%
\pgfpathlineto{\pgfqpoint{3.571389in}{1.432222in}}%
\pgfpathclose%
\pgfusepath{stroke}%
\end{pgfscope}%
\begin{pgfscope}%
\definecolor{textcolor}{rgb}{0.000000,0.000000,0.000000}%
\pgfsetstrokecolor{textcolor}%
\pgfsetfillcolor{textcolor}%
\pgftext[x=3.960278in,y=1.432222in,left,base]{\color{textcolor}\rmfamily\fontsize{10.000000}{12.000000}\selectfont Neg}%
\end{pgfscope}%
\begin{pgfscope}%
\pgfsetbuttcap%
\pgfsetmiterjoin%
\definecolor{currentfill}{rgb}{0.000000,0.000000,0.000000}%
\pgfsetfillcolor{currentfill}%
\pgfsetlinewidth{0.000000pt}%
\definecolor{currentstroke}{rgb}{0.000000,0.000000,0.000000}%
\pgfsetstrokecolor{currentstroke}%
\pgfsetstrokeopacity{0.000000}%
\pgfsetdash{}{0pt}%
\pgfpathmoveto{\pgfqpoint{3.571389in}{1.236944in}}%
\pgfpathlineto{\pgfqpoint{3.849167in}{1.236944in}}%
\pgfpathlineto{\pgfqpoint{3.849167in}{1.334167in}}%
\pgfpathlineto{\pgfqpoint{3.571389in}{1.334167in}}%
\pgfpathlineto{\pgfqpoint{3.571389in}{1.236944in}}%
\pgfpathclose%
\pgfusepath{fill}%
\end{pgfscope}%
\begin{pgfscope}%
\definecolor{textcolor}{rgb}{0.000000,0.000000,0.000000}%
\pgfsetstrokecolor{textcolor}%
\pgfsetfillcolor{textcolor}%
\pgftext[x=3.960278in,y=1.236944in,left,base]{\color{textcolor}\rmfamily\fontsize{10.000000}{12.000000}\selectfont Pos}%
\end{pgfscope}%
\end{pgfpicture}%
\makeatother%
\endgroup%
	
&
	\vskip 0pt
	\hfil ROC Curve
	
	%% Creator: Matplotlib, PGF backend
%%
%% To include the figure in your LaTeX document, write
%%   \input{<filename>.pgf}
%%
%% Make sure the required packages are loaded in your preamble
%%   \usepackage{pgf}
%%
%% Also ensure that all the required font packages are loaded; for instance,
%% the lmodern package is sometimes necessary when using math font.
%%   \usepackage{lmodern}
%%
%% Figures using additional raster images can only be included by \input if
%% they are in the same directory as the main LaTeX file. For loading figures
%% from other directories you can use the `import` package
%%   \usepackage{import}
%%
%% and then include the figures with
%%   \import{<path to file>}{<filename>.pgf}
%%
%% Matplotlib used the following preamble
%%   
%%   \usepackage{fontspec}
%%   \makeatletter\@ifpackageloaded{underscore}{}{\usepackage[strings]{underscore}}\makeatother
%%
\begingroup%
\makeatletter%
\begin{pgfpicture}%
\pgfpathrectangle{\pgfpointorigin}{\pgfqpoint{2.221861in}{1.754444in}}%
\pgfusepath{use as bounding box, clip}%
\begin{pgfscope}%
\pgfsetbuttcap%
\pgfsetmiterjoin%
\definecolor{currentfill}{rgb}{1.000000,1.000000,1.000000}%
\pgfsetfillcolor{currentfill}%
\pgfsetlinewidth{0.000000pt}%
\definecolor{currentstroke}{rgb}{1.000000,1.000000,1.000000}%
\pgfsetstrokecolor{currentstroke}%
\pgfsetdash{}{0pt}%
\pgfpathmoveto{\pgfqpoint{0.000000in}{0.000000in}}%
\pgfpathlineto{\pgfqpoint{2.221861in}{0.000000in}}%
\pgfpathlineto{\pgfqpoint{2.221861in}{1.754444in}}%
\pgfpathlineto{\pgfqpoint{0.000000in}{1.754444in}}%
\pgfpathlineto{\pgfqpoint{0.000000in}{0.000000in}}%
\pgfpathclose%
\pgfusepath{fill}%
\end{pgfscope}%
\begin{pgfscope}%
\pgfsetbuttcap%
\pgfsetmiterjoin%
\definecolor{currentfill}{rgb}{1.000000,1.000000,1.000000}%
\pgfsetfillcolor{currentfill}%
\pgfsetlinewidth{0.000000pt}%
\definecolor{currentstroke}{rgb}{0.000000,0.000000,0.000000}%
\pgfsetstrokecolor{currentstroke}%
\pgfsetstrokeopacity{0.000000}%
\pgfsetdash{}{0pt}%
\pgfpathmoveto{\pgfqpoint{0.553581in}{0.499444in}}%
\pgfpathlineto{\pgfqpoint{2.103581in}{0.499444in}}%
\pgfpathlineto{\pgfqpoint{2.103581in}{1.654444in}}%
\pgfpathlineto{\pgfqpoint{0.553581in}{1.654444in}}%
\pgfpathlineto{\pgfqpoint{0.553581in}{0.499444in}}%
\pgfpathclose%
\pgfusepath{fill}%
\end{pgfscope}%
\begin{pgfscope}%
\pgfsetbuttcap%
\pgfsetroundjoin%
\definecolor{currentfill}{rgb}{0.000000,0.000000,0.000000}%
\pgfsetfillcolor{currentfill}%
\pgfsetlinewidth{0.803000pt}%
\definecolor{currentstroke}{rgb}{0.000000,0.000000,0.000000}%
\pgfsetstrokecolor{currentstroke}%
\pgfsetdash{}{0pt}%
\pgfsys@defobject{currentmarker}{\pgfqpoint{0.000000in}{-0.048611in}}{\pgfqpoint{0.000000in}{0.000000in}}{%
\pgfpathmoveto{\pgfqpoint{0.000000in}{0.000000in}}%
\pgfpathlineto{\pgfqpoint{0.000000in}{-0.048611in}}%
\pgfusepath{stroke,fill}%
}%
\begin{pgfscope}%
\pgfsys@transformshift{0.624035in}{0.499444in}%
\pgfsys@useobject{currentmarker}{}%
\end{pgfscope}%
\end{pgfscope}%
\begin{pgfscope}%
\definecolor{textcolor}{rgb}{0.000000,0.000000,0.000000}%
\pgfsetstrokecolor{textcolor}%
\pgfsetfillcolor{textcolor}%
\pgftext[x=0.624035in,y=0.402222in,,top]{\color{textcolor}\rmfamily\fontsize{10.000000}{12.000000}\selectfont \(\displaystyle {0.0}\)}%
\end{pgfscope}%
\begin{pgfscope}%
\pgfsetbuttcap%
\pgfsetroundjoin%
\definecolor{currentfill}{rgb}{0.000000,0.000000,0.000000}%
\pgfsetfillcolor{currentfill}%
\pgfsetlinewidth{0.803000pt}%
\definecolor{currentstroke}{rgb}{0.000000,0.000000,0.000000}%
\pgfsetstrokecolor{currentstroke}%
\pgfsetdash{}{0pt}%
\pgfsys@defobject{currentmarker}{\pgfqpoint{0.000000in}{-0.048611in}}{\pgfqpoint{0.000000in}{0.000000in}}{%
\pgfpathmoveto{\pgfqpoint{0.000000in}{0.000000in}}%
\pgfpathlineto{\pgfqpoint{0.000000in}{-0.048611in}}%
\pgfusepath{stroke,fill}%
}%
\begin{pgfscope}%
\pgfsys@transformshift{1.328581in}{0.499444in}%
\pgfsys@useobject{currentmarker}{}%
\end{pgfscope}%
\end{pgfscope}%
\begin{pgfscope}%
\definecolor{textcolor}{rgb}{0.000000,0.000000,0.000000}%
\pgfsetstrokecolor{textcolor}%
\pgfsetfillcolor{textcolor}%
\pgftext[x=1.328581in,y=0.402222in,,top]{\color{textcolor}\rmfamily\fontsize{10.000000}{12.000000}\selectfont \(\displaystyle {0.5}\)}%
\end{pgfscope}%
\begin{pgfscope}%
\pgfsetbuttcap%
\pgfsetroundjoin%
\definecolor{currentfill}{rgb}{0.000000,0.000000,0.000000}%
\pgfsetfillcolor{currentfill}%
\pgfsetlinewidth{0.803000pt}%
\definecolor{currentstroke}{rgb}{0.000000,0.000000,0.000000}%
\pgfsetstrokecolor{currentstroke}%
\pgfsetdash{}{0pt}%
\pgfsys@defobject{currentmarker}{\pgfqpoint{0.000000in}{-0.048611in}}{\pgfqpoint{0.000000in}{0.000000in}}{%
\pgfpathmoveto{\pgfqpoint{0.000000in}{0.000000in}}%
\pgfpathlineto{\pgfqpoint{0.000000in}{-0.048611in}}%
\pgfusepath{stroke,fill}%
}%
\begin{pgfscope}%
\pgfsys@transformshift{2.033126in}{0.499444in}%
\pgfsys@useobject{currentmarker}{}%
\end{pgfscope}%
\end{pgfscope}%
\begin{pgfscope}%
\definecolor{textcolor}{rgb}{0.000000,0.000000,0.000000}%
\pgfsetstrokecolor{textcolor}%
\pgfsetfillcolor{textcolor}%
\pgftext[x=2.033126in,y=0.402222in,,top]{\color{textcolor}\rmfamily\fontsize{10.000000}{12.000000}\selectfont \(\displaystyle {1.0}\)}%
\end{pgfscope}%
\begin{pgfscope}%
\definecolor{textcolor}{rgb}{0.000000,0.000000,0.000000}%
\pgfsetstrokecolor{textcolor}%
\pgfsetfillcolor{textcolor}%
\pgftext[x=1.328581in,y=0.223333in,,top]{\color{textcolor}\rmfamily\fontsize{10.000000}{12.000000}\selectfont False positive rate}%
\end{pgfscope}%
\begin{pgfscope}%
\pgfsetbuttcap%
\pgfsetroundjoin%
\definecolor{currentfill}{rgb}{0.000000,0.000000,0.000000}%
\pgfsetfillcolor{currentfill}%
\pgfsetlinewidth{0.803000pt}%
\definecolor{currentstroke}{rgb}{0.000000,0.000000,0.000000}%
\pgfsetstrokecolor{currentstroke}%
\pgfsetdash{}{0pt}%
\pgfsys@defobject{currentmarker}{\pgfqpoint{-0.048611in}{0.000000in}}{\pgfqpoint{-0.000000in}{0.000000in}}{%
\pgfpathmoveto{\pgfqpoint{-0.000000in}{0.000000in}}%
\pgfpathlineto{\pgfqpoint{-0.048611in}{0.000000in}}%
\pgfusepath{stroke,fill}%
}%
\begin{pgfscope}%
\pgfsys@transformshift{0.553581in}{0.551944in}%
\pgfsys@useobject{currentmarker}{}%
\end{pgfscope}%
\end{pgfscope}%
\begin{pgfscope}%
\definecolor{textcolor}{rgb}{0.000000,0.000000,0.000000}%
\pgfsetstrokecolor{textcolor}%
\pgfsetfillcolor{textcolor}%
\pgftext[x=0.278889in, y=0.503750in, left, base]{\color{textcolor}\rmfamily\fontsize{10.000000}{12.000000}\selectfont \(\displaystyle {0.0}\)}%
\end{pgfscope}%
\begin{pgfscope}%
\pgfsetbuttcap%
\pgfsetroundjoin%
\definecolor{currentfill}{rgb}{0.000000,0.000000,0.000000}%
\pgfsetfillcolor{currentfill}%
\pgfsetlinewidth{0.803000pt}%
\definecolor{currentstroke}{rgb}{0.000000,0.000000,0.000000}%
\pgfsetstrokecolor{currentstroke}%
\pgfsetdash{}{0pt}%
\pgfsys@defobject{currentmarker}{\pgfqpoint{-0.048611in}{0.000000in}}{\pgfqpoint{-0.000000in}{0.000000in}}{%
\pgfpathmoveto{\pgfqpoint{-0.000000in}{0.000000in}}%
\pgfpathlineto{\pgfqpoint{-0.048611in}{0.000000in}}%
\pgfusepath{stroke,fill}%
}%
\begin{pgfscope}%
\pgfsys@transformshift{0.553581in}{1.076944in}%
\pgfsys@useobject{currentmarker}{}%
\end{pgfscope}%
\end{pgfscope}%
\begin{pgfscope}%
\definecolor{textcolor}{rgb}{0.000000,0.000000,0.000000}%
\pgfsetstrokecolor{textcolor}%
\pgfsetfillcolor{textcolor}%
\pgftext[x=0.278889in, y=1.028750in, left, base]{\color{textcolor}\rmfamily\fontsize{10.000000}{12.000000}\selectfont \(\displaystyle {0.5}\)}%
\end{pgfscope}%
\begin{pgfscope}%
\pgfsetbuttcap%
\pgfsetroundjoin%
\definecolor{currentfill}{rgb}{0.000000,0.000000,0.000000}%
\pgfsetfillcolor{currentfill}%
\pgfsetlinewidth{0.803000pt}%
\definecolor{currentstroke}{rgb}{0.000000,0.000000,0.000000}%
\pgfsetstrokecolor{currentstroke}%
\pgfsetdash{}{0pt}%
\pgfsys@defobject{currentmarker}{\pgfqpoint{-0.048611in}{0.000000in}}{\pgfqpoint{-0.000000in}{0.000000in}}{%
\pgfpathmoveto{\pgfqpoint{-0.000000in}{0.000000in}}%
\pgfpathlineto{\pgfqpoint{-0.048611in}{0.000000in}}%
\pgfusepath{stroke,fill}%
}%
\begin{pgfscope}%
\pgfsys@transformshift{0.553581in}{1.601944in}%
\pgfsys@useobject{currentmarker}{}%
\end{pgfscope}%
\end{pgfscope}%
\begin{pgfscope}%
\definecolor{textcolor}{rgb}{0.000000,0.000000,0.000000}%
\pgfsetstrokecolor{textcolor}%
\pgfsetfillcolor{textcolor}%
\pgftext[x=0.278889in, y=1.553750in, left, base]{\color{textcolor}\rmfamily\fontsize{10.000000}{12.000000}\selectfont \(\displaystyle {1.0}\)}%
\end{pgfscope}%
\begin{pgfscope}%
\definecolor{textcolor}{rgb}{0.000000,0.000000,0.000000}%
\pgfsetstrokecolor{textcolor}%
\pgfsetfillcolor{textcolor}%
\pgftext[x=0.223333in,y=1.076944in,,bottom,rotate=90.000000]{\color{textcolor}\rmfamily\fontsize{10.000000}{12.000000}\selectfont True positive rate}%
\end{pgfscope}%
\begin{pgfscope}%
\pgfpathrectangle{\pgfqpoint{0.553581in}{0.499444in}}{\pgfqpoint{1.550000in}{1.155000in}}%
\pgfusepath{clip}%
\pgfsetbuttcap%
\pgfsetroundjoin%
\pgfsetlinewidth{1.505625pt}%
\definecolor{currentstroke}{rgb}{0.000000,0.000000,0.000000}%
\pgfsetstrokecolor{currentstroke}%
\pgfsetdash{{5.550000pt}{2.400000pt}}{0.000000pt}%
\pgfpathmoveto{\pgfqpoint{0.624035in}{0.551944in}}%
\pgfpathlineto{\pgfqpoint{2.033126in}{1.601944in}}%
\pgfusepath{stroke}%
\end{pgfscope}%
\begin{pgfscope}%
\pgfpathrectangle{\pgfqpoint{0.553581in}{0.499444in}}{\pgfqpoint{1.550000in}{1.155000in}}%
\pgfusepath{clip}%
\pgfsetrectcap%
\pgfsetroundjoin%
\pgfsetlinewidth{1.505625pt}%
\definecolor{currentstroke}{rgb}{0.000000,0.000000,0.000000}%
\pgfsetstrokecolor{currentstroke}%
\pgfsetdash{}{0pt}%
\pgfpathmoveto{\pgfqpoint{0.624035in}{0.551944in}}%
\pgfpathlineto{\pgfqpoint{0.625087in}{1.081068in}}%
\pgfpathlineto{\pgfqpoint{0.626766in}{1.222951in}}%
\pgfpathlineto{\pgfqpoint{0.630063in}{1.346143in}}%
\pgfpathlineto{\pgfqpoint{0.630170in}{1.346861in}}%
\pgfpathlineto{\pgfqpoint{0.633316in}{1.414483in}}%
\pgfpathlineto{\pgfqpoint{0.637112in}{1.467192in}}%
\pgfpathlineto{\pgfqpoint{0.642134in}{1.509564in}}%
\pgfpathlineto{\pgfqpoint{0.648493in}{1.541240in}}%
\pgfpathlineto{\pgfqpoint{0.648510in}{1.541293in}}%
\pgfpathlineto{\pgfqpoint{0.655596in}{1.563550in}}%
\pgfpathlineto{\pgfqpoint{0.660310in}{1.572583in}}%
\pgfpathlineto{\pgfqpoint{0.670656in}{1.585341in}}%
\pgfpathlineto{\pgfqpoint{0.682708in}{1.593071in}}%
\pgfpathlineto{\pgfqpoint{0.683120in}{1.593230in}}%
\pgfpathlineto{\pgfqpoint{0.697859in}{1.597421in}}%
\pgfpathlineto{\pgfqpoint{0.715458in}{1.600108in}}%
\pgfpathlineto{\pgfqpoint{0.747388in}{1.601439in}}%
\pgfpathlineto{\pgfqpoint{0.871277in}{1.601931in}}%
\pgfpathlineto{\pgfqpoint{2.033126in}{1.601944in}}%
\pgfpathlineto{\pgfqpoint{2.033126in}{1.601944in}}%
\pgfusepath{stroke}%
\end{pgfscope}%
\begin{pgfscope}%
\pgfsetrectcap%
\pgfsetmiterjoin%
\pgfsetlinewidth{0.803000pt}%
\definecolor{currentstroke}{rgb}{0.000000,0.000000,0.000000}%
\pgfsetstrokecolor{currentstroke}%
\pgfsetdash{}{0pt}%
\pgfpathmoveto{\pgfqpoint{0.553581in}{0.499444in}}%
\pgfpathlineto{\pgfqpoint{0.553581in}{1.654444in}}%
\pgfusepath{stroke}%
\end{pgfscope}%
\begin{pgfscope}%
\pgfsetrectcap%
\pgfsetmiterjoin%
\pgfsetlinewidth{0.803000pt}%
\definecolor{currentstroke}{rgb}{0.000000,0.000000,0.000000}%
\pgfsetstrokecolor{currentstroke}%
\pgfsetdash{}{0pt}%
\pgfpathmoveto{\pgfqpoint{2.103581in}{0.499444in}}%
\pgfpathlineto{\pgfqpoint{2.103581in}{1.654444in}}%
\pgfusepath{stroke}%
\end{pgfscope}%
\begin{pgfscope}%
\pgfsetrectcap%
\pgfsetmiterjoin%
\pgfsetlinewidth{0.803000pt}%
\definecolor{currentstroke}{rgb}{0.000000,0.000000,0.000000}%
\pgfsetstrokecolor{currentstroke}%
\pgfsetdash{}{0pt}%
\pgfpathmoveto{\pgfqpoint{0.553581in}{0.499444in}}%
\pgfpathlineto{\pgfqpoint{2.103581in}{0.499444in}}%
\pgfusepath{stroke}%
\end{pgfscope}%
\begin{pgfscope}%
\pgfsetrectcap%
\pgfsetmiterjoin%
\pgfsetlinewidth{0.803000pt}%
\definecolor{currentstroke}{rgb}{0.000000,0.000000,0.000000}%
\pgfsetstrokecolor{currentstroke}%
\pgfsetdash{}{0pt}%
\pgfpathmoveto{\pgfqpoint{0.553581in}{1.654444in}}%
\pgfpathlineto{\pgfqpoint{2.103581in}{1.654444in}}%
\pgfusepath{stroke}%
\end{pgfscope}%
\begin{pgfscope}%
\pgfsetbuttcap%
\pgfsetmiterjoin%
\definecolor{currentfill}{rgb}{1.000000,1.000000,1.000000}%
\pgfsetfillcolor{currentfill}%
\pgfsetfillopacity{0.800000}%
\pgfsetlinewidth{1.003750pt}%
\definecolor{currentstroke}{rgb}{0.800000,0.800000,0.800000}%
\pgfsetstrokecolor{currentstroke}%
\pgfsetstrokeopacity{0.800000}%
\pgfsetdash{}{0pt}%
\pgfpathmoveto{\pgfqpoint{0.832747in}{1.349722in}}%
\pgfpathlineto{\pgfqpoint{2.006358in}{1.349722in}}%
\pgfpathquadraticcurveto{\pgfqpoint{2.034136in}{1.349722in}}{\pgfqpoint{2.034136in}{1.377500in}}%
\pgfpathlineto{\pgfqpoint{2.034136in}{1.557222in}}%
\pgfpathquadraticcurveto{\pgfqpoint{2.034136in}{1.585000in}}{\pgfqpoint{2.006358in}{1.585000in}}%
\pgfpathlineto{\pgfqpoint{0.832747in}{1.585000in}}%
\pgfpathquadraticcurveto{\pgfqpoint{0.804970in}{1.585000in}}{\pgfqpoint{0.804970in}{1.557222in}}%
\pgfpathlineto{\pgfqpoint{0.804970in}{1.377500in}}%
\pgfpathquadraticcurveto{\pgfqpoint{0.804970in}{1.349722in}}{\pgfqpoint{0.832747in}{1.349722in}}%
\pgfpathlineto{\pgfqpoint{0.832747in}{1.349722in}}%
\pgfpathclose%
\pgfusepath{stroke,fill}%
\end{pgfscope}%
\begin{pgfscope}%
\pgfsetrectcap%
\pgfsetroundjoin%
\pgfsetlinewidth{1.505625pt}%
\definecolor{currentstroke}{rgb}{0.000000,0.000000,0.000000}%
\pgfsetstrokecolor{currentstroke}%
\pgfsetdash{}{0pt}%
\pgfpathmoveto{\pgfqpoint{0.860525in}{1.480833in}}%
\pgfpathlineto{\pgfqpoint{0.999414in}{1.480833in}}%
\pgfpathlineto{\pgfqpoint{1.138303in}{1.480833in}}%
\pgfusepath{stroke}%
\end{pgfscope}%
\begin{pgfscope}%
\definecolor{textcolor}{rgb}{0.000000,0.000000,0.000000}%
\pgfsetstrokecolor{textcolor}%
\pgfsetfillcolor{textcolor}%
\pgftext[x=1.249414in,y=1.432222in,left,base]{\color{textcolor}\rmfamily\fontsize{10.000000}{12.000000}\selectfont AUC=0.996}%
\end{pgfscope}%
\end{pgfpicture}%
\makeatother%
\endgroup%

	
\end{tabular}
\end{comment}

Unfortunately, our test results do not look quite that nice.  They do not separate the two classes as well.  Some distributions are clustered to one side or in the middle.  Some models give the results in $p \in [0,1]$ rounded to two decimal places so that we cannot hope for a level of detail beyond that, and one algorithm, Bagging, gives $p$ rounded to only one decimal place.  

Let us look at some examples.  In all of them, AUC is in the range $[0.7,0.8]$, so the various models separate the positive and negative classes about equally well overall, with none being dramatically better or worse.  We will later show how we investigated which models do a better job in the ranges of interest.  

\

%
\verb|BRFC_5_Fold_alpha_0_5_Hard_Test|

\

This model does not separate the negative and positive classes as well as the ideal, giving a much lower AUC (area under the ROC curve).  These results are actually from the same model as the ideal above, but the ideal are the results on the training set and below on the test set, showing overfitting.  

In these results, the 100 most frequent values comprised 93\% of the results, meaning that, while there is some noise making the distribution look continuous, it is mostly discrete to two decimal places, so we cannot hope for fine detail in tuning the decision threshold.  

\noindent\begin{tabular}{@{\hspace{-6pt}}p{4.3in} @{\hspace{-6pt}}p{2.0in}}
	\vskip 0pt
	\hfil Raw Model Output
	
	%% Creator: Matplotlib, PGF backend
%%
%% To include the figure in your LaTeX document, write
%%   \input{<filename>.pgf}
%%
%% Make sure the required packages are loaded in your preamble
%%   \usepackage{pgf}
%%
%% Also ensure that all the required font packages are loaded; for instance,
%% the lmodern package is sometimes necessary when using math font.
%%   \usepackage{lmodern}
%%
%% Figures using additional raster images can only be included by \input if
%% they are in the same directory as the main LaTeX file. For loading figures
%% from other directories you can use the `import` package
%%   \usepackage{import}
%%
%% and then include the figures with
%%   \import{<path to file>}{<filename>.pgf}
%%
%% Matplotlib used the following preamble
%%   
%%   \usepackage{fontspec}
%%   \makeatletter\@ifpackageloaded{underscore}{}{\usepackage[strings]{underscore}}\makeatother
%%
\begingroup%
\makeatletter%
\begin{pgfpicture}%
\pgfpathrectangle{\pgfpointorigin}{\pgfqpoint{4.141081in}{1.754444in}}%
\pgfusepath{use as bounding box, clip}%
\begin{pgfscope}%
\pgfsetbuttcap%
\pgfsetmiterjoin%
\definecolor{currentfill}{rgb}{1.000000,1.000000,1.000000}%
\pgfsetfillcolor{currentfill}%
\pgfsetlinewidth{0.000000pt}%
\definecolor{currentstroke}{rgb}{1.000000,1.000000,1.000000}%
\pgfsetstrokecolor{currentstroke}%
\pgfsetdash{}{0pt}%
\pgfpathmoveto{\pgfqpoint{0.000000in}{0.000000in}}%
\pgfpathlineto{\pgfqpoint{4.141081in}{0.000000in}}%
\pgfpathlineto{\pgfqpoint{4.141081in}{1.754444in}}%
\pgfpathlineto{\pgfqpoint{0.000000in}{1.754444in}}%
\pgfpathlineto{\pgfqpoint{0.000000in}{0.000000in}}%
\pgfpathclose%
\pgfusepath{fill}%
\end{pgfscope}%
\begin{pgfscope}%
\pgfsetbuttcap%
\pgfsetmiterjoin%
\definecolor{currentfill}{rgb}{1.000000,1.000000,1.000000}%
\pgfsetfillcolor{currentfill}%
\pgfsetlinewidth{0.000000pt}%
\definecolor{currentstroke}{rgb}{0.000000,0.000000,0.000000}%
\pgfsetstrokecolor{currentstroke}%
\pgfsetstrokeopacity{0.000000}%
\pgfsetdash{}{0pt}%
\pgfpathmoveto{\pgfqpoint{0.553581in}{0.499444in}}%
\pgfpathlineto{\pgfqpoint{4.041081in}{0.499444in}}%
\pgfpathlineto{\pgfqpoint{4.041081in}{1.654444in}}%
\pgfpathlineto{\pgfqpoint{0.553581in}{1.654444in}}%
\pgfpathlineto{\pgfqpoint{0.553581in}{0.499444in}}%
\pgfpathclose%
\pgfusepath{fill}%
\end{pgfscope}%
\begin{pgfscope}%
\pgfpathrectangle{\pgfqpoint{0.553581in}{0.499444in}}{\pgfqpoint{3.487500in}{1.155000in}}%
\pgfusepath{clip}%
\pgfsetbuttcap%
\pgfsetmiterjoin%
\pgfsetlinewidth{1.003750pt}%
\definecolor{currentstroke}{rgb}{0.000000,0.000000,0.000000}%
\pgfsetstrokecolor{currentstroke}%
\pgfsetdash{}{0pt}%
\pgfpathmoveto{\pgfqpoint{0.648694in}{0.499444in}}%
\pgfpathlineto{\pgfqpoint{0.712103in}{0.499444in}}%
\pgfpathlineto{\pgfqpoint{0.712103in}{0.510776in}}%
\pgfpathlineto{\pgfqpoint{0.648694in}{0.510776in}}%
\pgfpathlineto{\pgfqpoint{0.648694in}{0.499444in}}%
\pgfpathclose%
\pgfusepath{stroke}%
\end{pgfscope}%
\begin{pgfscope}%
\pgfpathrectangle{\pgfqpoint{0.553581in}{0.499444in}}{\pgfqpoint{3.487500in}{1.155000in}}%
\pgfusepath{clip}%
\pgfsetbuttcap%
\pgfsetmiterjoin%
\pgfsetlinewidth{1.003750pt}%
\definecolor{currentstroke}{rgb}{0.000000,0.000000,0.000000}%
\pgfsetstrokecolor{currentstroke}%
\pgfsetdash{}{0pt}%
\pgfpathmoveto{\pgfqpoint{0.807217in}{0.499444in}}%
\pgfpathlineto{\pgfqpoint{0.870626in}{0.499444in}}%
\pgfpathlineto{\pgfqpoint{0.870626in}{0.654936in}}%
\pgfpathlineto{\pgfqpoint{0.807217in}{0.654936in}}%
\pgfpathlineto{\pgfqpoint{0.807217in}{0.499444in}}%
\pgfpathclose%
\pgfusepath{stroke}%
\end{pgfscope}%
\begin{pgfscope}%
\pgfpathrectangle{\pgfqpoint{0.553581in}{0.499444in}}{\pgfqpoint{3.487500in}{1.155000in}}%
\pgfusepath{clip}%
\pgfsetbuttcap%
\pgfsetmiterjoin%
\pgfsetlinewidth{1.003750pt}%
\definecolor{currentstroke}{rgb}{0.000000,0.000000,0.000000}%
\pgfsetstrokecolor{currentstroke}%
\pgfsetdash{}{0pt}%
\pgfpathmoveto{\pgfqpoint{0.965740in}{0.499444in}}%
\pgfpathlineto{\pgfqpoint{1.029149in}{0.499444in}}%
\pgfpathlineto{\pgfqpoint{1.029149in}{0.902702in}}%
\pgfpathlineto{\pgfqpoint{0.965740in}{0.902702in}}%
\pgfpathlineto{\pgfqpoint{0.965740in}{0.499444in}}%
\pgfpathclose%
\pgfusepath{stroke}%
\end{pgfscope}%
\begin{pgfscope}%
\pgfpathrectangle{\pgfqpoint{0.553581in}{0.499444in}}{\pgfqpoint{3.487500in}{1.155000in}}%
\pgfusepath{clip}%
\pgfsetbuttcap%
\pgfsetmiterjoin%
\pgfsetlinewidth{1.003750pt}%
\definecolor{currentstroke}{rgb}{0.000000,0.000000,0.000000}%
\pgfsetstrokecolor{currentstroke}%
\pgfsetdash{}{0pt}%
\pgfpathmoveto{\pgfqpoint{1.124263in}{0.499444in}}%
\pgfpathlineto{\pgfqpoint{1.187672in}{0.499444in}}%
\pgfpathlineto{\pgfqpoint{1.187672in}{1.156824in}}%
\pgfpathlineto{\pgfqpoint{1.124263in}{1.156824in}}%
\pgfpathlineto{\pgfqpoint{1.124263in}{0.499444in}}%
\pgfpathclose%
\pgfusepath{stroke}%
\end{pgfscope}%
\begin{pgfscope}%
\pgfpathrectangle{\pgfqpoint{0.553581in}{0.499444in}}{\pgfqpoint{3.487500in}{1.155000in}}%
\pgfusepath{clip}%
\pgfsetbuttcap%
\pgfsetmiterjoin%
\pgfsetlinewidth{1.003750pt}%
\definecolor{currentstroke}{rgb}{0.000000,0.000000,0.000000}%
\pgfsetstrokecolor{currentstroke}%
\pgfsetdash{}{0pt}%
\pgfpathmoveto{\pgfqpoint{1.282785in}{0.499444in}}%
\pgfpathlineto{\pgfqpoint{1.346194in}{0.499444in}}%
\pgfpathlineto{\pgfqpoint{1.346194in}{1.338930in}}%
\pgfpathlineto{\pgfqpoint{1.282785in}{1.338930in}}%
\pgfpathlineto{\pgfqpoint{1.282785in}{0.499444in}}%
\pgfpathclose%
\pgfusepath{stroke}%
\end{pgfscope}%
\begin{pgfscope}%
\pgfpathrectangle{\pgfqpoint{0.553581in}{0.499444in}}{\pgfqpoint{3.487500in}{1.155000in}}%
\pgfusepath{clip}%
\pgfsetbuttcap%
\pgfsetmiterjoin%
\pgfsetlinewidth{1.003750pt}%
\definecolor{currentstroke}{rgb}{0.000000,0.000000,0.000000}%
\pgfsetstrokecolor{currentstroke}%
\pgfsetdash{}{0pt}%
\pgfpathmoveto{\pgfqpoint{1.441308in}{0.499444in}}%
\pgfpathlineto{\pgfqpoint{1.504717in}{0.499444in}}%
\pgfpathlineto{\pgfqpoint{1.504717in}{1.477909in}}%
\pgfpathlineto{\pgfqpoint{1.441308in}{1.477909in}}%
\pgfpathlineto{\pgfqpoint{1.441308in}{0.499444in}}%
\pgfpathclose%
\pgfusepath{stroke}%
\end{pgfscope}%
\begin{pgfscope}%
\pgfpathrectangle{\pgfqpoint{0.553581in}{0.499444in}}{\pgfqpoint{3.487500in}{1.155000in}}%
\pgfusepath{clip}%
\pgfsetbuttcap%
\pgfsetmiterjoin%
\pgfsetlinewidth{1.003750pt}%
\definecolor{currentstroke}{rgb}{0.000000,0.000000,0.000000}%
\pgfsetstrokecolor{currentstroke}%
\pgfsetdash{}{0pt}%
\pgfpathmoveto{\pgfqpoint{1.599831in}{0.499444in}}%
\pgfpathlineto{\pgfqpoint{1.663240in}{0.499444in}}%
\pgfpathlineto{\pgfqpoint{1.663240in}{1.561778in}}%
\pgfpathlineto{\pgfqpoint{1.599831in}{1.561778in}}%
\pgfpathlineto{\pgfqpoint{1.599831in}{0.499444in}}%
\pgfpathclose%
\pgfusepath{stroke}%
\end{pgfscope}%
\begin{pgfscope}%
\pgfpathrectangle{\pgfqpoint{0.553581in}{0.499444in}}{\pgfqpoint{3.487500in}{1.155000in}}%
\pgfusepath{clip}%
\pgfsetbuttcap%
\pgfsetmiterjoin%
\pgfsetlinewidth{1.003750pt}%
\definecolor{currentstroke}{rgb}{0.000000,0.000000,0.000000}%
\pgfsetstrokecolor{currentstroke}%
\pgfsetdash{}{0pt}%
\pgfpathmoveto{\pgfqpoint{1.758353in}{0.499444in}}%
\pgfpathlineto{\pgfqpoint{1.821763in}{0.499444in}}%
\pgfpathlineto{\pgfqpoint{1.821763in}{1.599444in}}%
\pgfpathlineto{\pgfqpoint{1.758353in}{1.599444in}}%
\pgfpathlineto{\pgfqpoint{1.758353in}{0.499444in}}%
\pgfpathclose%
\pgfusepath{stroke}%
\end{pgfscope}%
\begin{pgfscope}%
\pgfpathrectangle{\pgfqpoint{0.553581in}{0.499444in}}{\pgfqpoint{3.487500in}{1.155000in}}%
\pgfusepath{clip}%
\pgfsetbuttcap%
\pgfsetmiterjoin%
\pgfsetlinewidth{1.003750pt}%
\definecolor{currentstroke}{rgb}{0.000000,0.000000,0.000000}%
\pgfsetstrokecolor{currentstroke}%
\pgfsetdash{}{0pt}%
\pgfpathmoveto{\pgfqpoint{1.916876in}{0.499444in}}%
\pgfpathlineto{\pgfqpoint{1.980285in}{0.499444in}}%
\pgfpathlineto{\pgfqpoint{1.980285in}{1.580154in}}%
\pgfpathlineto{\pgfqpoint{1.916876in}{1.580154in}}%
\pgfpathlineto{\pgfqpoint{1.916876in}{0.499444in}}%
\pgfpathclose%
\pgfusepath{stroke}%
\end{pgfscope}%
\begin{pgfscope}%
\pgfpathrectangle{\pgfqpoint{0.553581in}{0.499444in}}{\pgfqpoint{3.487500in}{1.155000in}}%
\pgfusepath{clip}%
\pgfsetbuttcap%
\pgfsetmiterjoin%
\pgfsetlinewidth{1.003750pt}%
\definecolor{currentstroke}{rgb}{0.000000,0.000000,0.000000}%
\pgfsetstrokecolor{currentstroke}%
\pgfsetdash{}{0pt}%
\pgfpathmoveto{\pgfqpoint{2.075399in}{0.499444in}}%
\pgfpathlineto{\pgfqpoint{2.138808in}{0.499444in}}%
\pgfpathlineto{\pgfqpoint{2.138808in}{1.516172in}}%
\pgfpathlineto{\pgfqpoint{2.075399in}{1.516172in}}%
\pgfpathlineto{\pgfqpoint{2.075399in}{0.499444in}}%
\pgfpathclose%
\pgfusepath{stroke}%
\end{pgfscope}%
\begin{pgfscope}%
\pgfpathrectangle{\pgfqpoint{0.553581in}{0.499444in}}{\pgfqpoint{3.487500in}{1.155000in}}%
\pgfusepath{clip}%
\pgfsetbuttcap%
\pgfsetmiterjoin%
\pgfsetlinewidth{1.003750pt}%
\definecolor{currentstroke}{rgb}{0.000000,0.000000,0.000000}%
\pgfsetstrokecolor{currentstroke}%
\pgfsetdash{}{0pt}%
\pgfpathmoveto{\pgfqpoint{2.233922in}{0.499444in}}%
\pgfpathlineto{\pgfqpoint{2.297331in}{0.499444in}}%
\pgfpathlineto{\pgfqpoint{2.297331in}{1.414374in}}%
\pgfpathlineto{\pgfqpoint{2.233922in}{1.414374in}}%
\pgfpathlineto{\pgfqpoint{2.233922in}{0.499444in}}%
\pgfpathclose%
\pgfusepath{stroke}%
\end{pgfscope}%
\begin{pgfscope}%
\pgfpathrectangle{\pgfqpoint{0.553581in}{0.499444in}}{\pgfqpoint{3.487500in}{1.155000in}}%
\pgfusepath{clip}%
\pgfsetbuttcap%
\pgfsetmiterjoin%
\pgfsetlinewidth{1.003750pt}%
\definecolor{currentstroke}{rgb}{0.000000,0.000000,0.000000}%
\pgfsetstrokecolor{currentstroke}%
\pgfsetdash{}{0pt}%
\pgfpathmoveto{\pgfqpoint{2.392444in}{0.499444in}}%
\pgfpathlineto{\pgfqpoint{2.455853in}{0.499444in}}%
\pgfpathlineto{\pgfqpoint{2.455853in}{1.289075in}}%
\pgfpathlineto{\pgfqpoint{2.392444in}{1.289075in}}%
\pgfpathlineto{\pgfqpoint{2.392444in}{0.499444in}}%
\pgfpathclose%
\pgfusepath{stroke}%
\end{pgfscope}%
\begin{pgfscope}%
\pgfpathrectangle{\pgfqpoint{0.553581in}{0.499444in}}{\pgfqpoint{3.487500in}{1.155000in}}%
\pgfusepath{clip}%
\pgfsetbuttcap%
\pgfsetmiterjoin%
\pgfsetlinewidth{1.003750pt}%
\definecolor{currentstroke}{rgb}{0.000000,0.000000,0.000000}%
\pgfsetstrokecolor{currentstroke}%
\pgfsetdash{}{0pt}%
\pgfpathmoveto{\pgfqpoint{2.550967in}{0.499444in}}%
\pgfpathlineto{\pgfqpoint{2.614376in}{0.499444in}}%
\pgfpathlineto{\pgfqpoint{2.614376in}{1.145846in}}%
\pgfpathlineto{\pgfqpoint{2.550967in}{1.145846in}}%
\pgfpathlineto{\pgfqpoint{2.550967in}{0.499444in}}%
\pgfpathclose%
\pgfusepath{stroke}%
\end{pgfscope}%
\begin{pgfscope}%
\pgfpathrectangle{\pgfqpoint{0.553581in}{0.499444in}}{\pgfqpoint{3.487500in}{1.155000in}}%
\pgfusepath{clip}%
\pgfsetbuttcap%
\pgfsetmiterjoin%
\pgfsetlinewidth{1.003750pt}%
\definecolor{currentstroke}{rgb}{0.000000,0.000000,0.000000}%
\pgfsetstrokecolor{currentstroke}%
\pgfsetdash{}{0pt}%
\pgfpathmoveto{\pgfqpoint{2.709490in}{0.499444in}}%
\pgfpathlineto{\pgfqpoint{2.772899in}{0.499444in}}%
\pgfpathlineto{\pgfqpoint{2.772899in}{1.007743in}}%
\pgfpathlineto{\pgfqpoint{2.709490in}{1.007743in}}%
\pgfpathlineto{\pgfqpoint{2.709490in}{0.499444in}}%
\pgfpathclose%
\pgfusepath{stroke}%
\end{pgfscope}%
\begin{pgfscope}%
\pgfpathrectangle{\pgfqpoint{0.553581in}{0.499444in}}{\pgfqpoint{3.487500in}{1.155000in}}%
\pgfusepath{clip}%
\pgfsetbuttcap%
\pgfsetmiterjoin%
\pgfsetlinewidth{1.003750pt}%
\definecolor{currentstroke}{rgb}{0.000000,0.000000,0.000000}%
\pgfsetstrokecolor{currentstroke}%
\pgfsetdash{}{0pt}%
\pgfpathmoveto{\pgfqpoint{2.868013in}{0.499444in}}%
\pgfpathlineto{\pgfqpoint{2.931422in}{0.499444in}}%
\pgfpathlineto{\pgfqpoint{2.931422in}{0.874224in}}%
\pgfpathlineto{\pgfqpoint{2.868013in}{0.874224in}}%
\pgfpathlineto{\pgfqpoint{2.868013in}{0.499444in}}%
\pgfpathclose%
\pgfusepath{stroke}%
\end{pgfscope}%
\begin{pgfscope}%
\pgfpathrectangle{\pgfqpoint{0.553581in}{0.499444in}}{\pgfqpoint{3.487500in}{1.155000in}}%
\pgfusepath{clip}%
\pgfsetbuttcap%
\pgfsetmiterjoin%
\pgfsetlinewidth{1.003750pt}%
\definecolor{currentstroke}{rgb}{0.000000,0.000000,0.000000}%
\pgfsetstrokecolor{currentstroke}%
\pgfsetdash{}{0pt}%
\pgfpathmoveto{\pgfqpoint{3.026535in}{0.499444in}}%
\pgfpathlineto{\pgfqpoint{3.089944in}{0.499444in}}%
\pgfpathlineto{\pgfqpoint{3.089944in}{0.757684in}}%
\pgfpathlineto{\pgfqpoint{3.026535in}{0.757684in}}%
\pgfpathlineto{\pgfqpoint{3.026535in}{0.499444in}}%
\pgfpathclose%
\pgfusepath{stroke}%
\end{pgfscope}%
\begin{pgfscope}%
\pgfpathrectangle{\pgfqpoint{0.553581in}{0.499444in}}{\pgfqpoint{3.487500in}{1.155000in}}%
\pgfusepath{clip}%
\pgfsetbuttcap%
\pgfsetmiterjoin%
\pgfsetlinewidth{1.003750pt}%
\definecolor{currentstroke}{rgb}{0.000000,0.000000,0.000000}%
\pgfsetstrokecolor{currentstroke}%
\pgfsetdash{}{0pt}%
\pgfpathmoveto{\pgfqpoint{3.185058in}{0.499444in}}%
\pgfpathlineto{\pgfqpoint{3.248467in}{0.499444in}}%
\pgfpathlineto{\pgfqpoint{3.248467in}{0.671095in}}%
\pgfpathlineto{\pgfqpoint{3.185058in}{0.671095in}}%
\pgfpathlineto{\pgfqpoint{3.185058in}{0.499444in}}%
\pgfpathclose%
\pgfusepath{stroke}%
\end{pgfscope}%
\begin{pgfscope}%
\pgfpathrectangle{\pgfqpoint{0.553581in}{0.499444in}}{\pgfqpoint{3.487500in}{1.155000in}}%
\pgfusepath{clip}%
\pgfsetbuttcap%
\pgfsetmiterjoin%
\pgfsetlinewidth{1.003750pt}%
\definecolor{currentstroke}{rgb}{0.000000,0.000000,0.000000}%
\pgfsetstrokecolor{currentstroke}%
\pgfsetdash{}{0pt}%
\pgfpathmoveto{\pgfqpoint{3.343581in}{0.499444in}}%
\pgfpathlineto{\pgfqpoint{3.406990in}{0.499444in}}%
\pgfpathlineto{\pgfqpoint{3.406990in}{0.613990in}}%
\pgfpathlineto{\pgfqpoint{3.343581in}{0.613990in}}%
\pgfpathlineto{\pgfqpoint{3.343581in}{0.499444in}}%
\pgfpathclose%
\pgfusepath{stroke}%
\end{pgfscope}%
\begin{pgfscope}%
\pgfpathrectangle{\pgfqpoint{0.553581in}{0.499444in}}{\pgfqpoint{3.487500in}{1.155000in}}%
\pgfusepath{clip}%
\pgfsetbuttcap%
\pgfsetmiterjoin%
\pgfsetlinewidth{1.003750pt}%
\definecolor{currentstroke}{rgb}{0.000000,0.000000,0.000000}%
\pgfsetstrokecolor{currentstroke}%
\pgfsetdash{}{0pt}%
\pgfpathmoveto{\pgfqpoint{3.502103in}{0.499444in}}%
\pgfpathlineto{\pgfqpoint{3.565513in}{0.499444in}}%
\pgfpathlineto{\pgfqpoint{3.565513in}{0.567862in}}%
\pgfpathlineto{\pgfqpoint{3.502103in}{0.567862in}}%
\pgfpathlineto{\pgfqpoint{3.502103in}{0.499444in}}%
\pgfpathclose%
\pgfusepath{stroke}%
\end{pgfscope}%
\begin{pgfscope}%
\pgfpathrectangle{\pgfqpoint{0.553581in}{0.499444in}}{\pgfqpoint{3.487500in}{1.155000in}}%
\pgfusepath{clip}%
\pgfsetbuttcap%
\pgfsetmiterjoin%
\pgfsetlinewidth{1.003750pt}%
\definecolor{currentstroke}{rgb}{0.000000,0.000000,0.000000}%
\pgfsetstrokecolor{currentstroke}%
\pgfsetdash{}{0pt}%
\pgfpathmoveto{\pgfqpoint{3.660626in}{0.499444in}}%
\pgfpathlineto{\pgfqpoint{3.724035in}{0.499444in}}%
\pgfpathlineto{\pgfqpoint{3.724035in}{0.535619in}}%
\pgfpathlineto{\pgfqpoint{3.660626in}{0.535619in}}%
\pgfpathlineto{\pgfqpoint{3.660626in}{0.499444in}}%
\pgfpathclose%
\pgfusepath{stroke}%
\end{pgfscope}%
\begin{pgfscope}%
\pgfpathrectangle{\pgfqpoint{0.553581in}{0.499444in}}{\pgfqpoint{3.487500in}{1.155000in}}%
\pgfusepath{clip}%
\pgfsetbuttcap%
\pgfsetmiterjoin%
\pgfsetlinewidth{1.003750pt}%
\definecolor{currentstroke}{rgb}{0.000000,0.000000,0.000000}%
\pgfsetstrokecolor{currentstroke}%
\pgfsetdash{}{0pt}%
\pgfpathmoveto{\pgfqpoint{3.819149in}{0.499444in}}%
\pgfpathlineto{\pgfqpoint{3.882558in}{0.499444in}}%
\pgfpathlineto{\pgfqpoint{3.882558in}{0.508837in}}%
\pgfpathlineto{\pgfqpoint{3.819149in}{0.508837in}}%
\pgfpathlineto{\pgfqpoint{3.819149in}{0.499444in}}%
\pgfpathclose%
\pgfusepath{stroke}%
\end{pgfscope}%
\begin{pgfscope}%
\pgfpathrectangle{\pgfqpoint{0.553581in}{0.499444in}}{\pgfqpoint{3.487500in}{1.155000in}}%
\pgfusepath{clip}%
\pgfsetbuttcap%
\pgfsetmiterjoin%
\definecolor{currentfill}{rgb}{0.000000,0.000000,0.000000}%
\pgfsetfillcolor{currentfill}%
\pgfsetlinewidth{0.000000pt}%
\definecolor{currentstroke}{rgb}{0.000000,0.000000,0.000000}%
\pgfsetstrokecolor{currentstroke}%
\pgfsetstrokeopacity{0.000000}%
\pgfsetdash{}{0pt}%
\pgfpathmoveto{\pgfqpoint{0.712103in}{0.499444in}}%
\pgfpathlineto{\pgfqpoint{0.775513in}{0.499444in}}%
\pgfpathlineto{\pgfqpoint{0.775513in}{0.499444in}}%
\pgfpathlineto{\pgfqpoint{0.712103in}{0.499444in}}%
\pgfpathlineto{\pgfqpoint{0.712103in}{0.499444in}}%
\pgfpathclose%
\pgfusepath{fill}%
\end{pgfscope}%
\begin{pgfscope}%
\pgfpathrectangle{\pgfqpoint{0.553581in}{0.499444in}}{\pgfqpoint{3.487500in}{1.155000in}}%
\pgfusepath{clip}%
\pgfsetbuttcap%
\pgfsetmiterjoin%
\definecolor{currentfill}{rgb}{0.000000,0.000000,0.000000}%
\pgfsetfillcolor{currentfill}%
\pgfsetlinewidth{0.000000pt}%
\definecolor{currentstroke}{rgb}{0.000000,0.000000,0.000000}%
\pgfsetstrokecolor{currentstroke}%
\pgfsetstrokeopacity{0.000000}%
\pgfsetdash{}{0pt}%
\pgfpathmoveto{\pgfqpoint{0.870626in}{0.499444in}}%
\pgfpathlineto{\pgfqpoint{0.934035in}{0.499444in}}%
\pgfpathlineto{\pgfqpoint{0.934035in}{0.500823in}}%
\pgfpathlineto{\pgfqpoint{0.870626in}{0.500823in}}%
\pgfpathlineto{\pgfqpoint{0.870626in}{0.499444in}}%
\pgfpathclose%
\pgfusepath{fill}%
\end{pgfscope}%
\begin{pgfscope}%
\pgfpathrectangle{\pgfqpoint{0.553581in}{0.499444in}}{\pgfqpoint{3.487500in}{1.155000in}}%
\pgfusepath{clip}%
\pgfsetbuttcap%
\pgfsetmiterjoin%
\definecolor{currentfill}{rgb}{0.000000,0.000000,0.000000}%
\pgfsetfillcolor{currentfill}%
\pgfsetlinewidth{0.000000pt}%
\definecolor{currentstroke}{rgb}{0.000000,0.000000,0.000000}%
\pgfsetstrokecolor{currentstroke}%
\pgfsetstrokeopacity{0.000000}%
\pgfsetdash{}{0pt}%
\pgfpathmoveto{\pgfqpoint{1.029149in}{0.499444in}}%
\pgfpathlineto{\pgfqpoint{1.092558in}{0.499444in}}%
\pgfpathlineto{\pgfqpoint{1.092558in}{0.505035in}}%
\pgfpathlineto{\pgfqpoint{1.029149in}{0.505035in}}%
\pgfpathlineto{\pgfqpoint{1.029149in}{0.499444in}}%
\pgfpathclose%
\pgfusepath{fill}%
\end{pgfscope}%
\begin{pgfscope}%
\pgfpathrectangle{\pgfqpoint{0.553581in}{0.499444in}}{\pgfqpoint{3.487500in}{1.155000in}}%
\pgfusepath{clip}%
\pgfsetbuttcap%
\pgfsetmiterjoin%
\definecolor{currentfill}{rgb}{0.000000,0.000000,0.000000}%
\pgfsetfillcolor{currentfill}%
\pgfsetlinewidth{0.000000pt}%
\definecolor{currentstroke}{rgb}{0.000000,0.000000,0.000000}%
\pgfsetstrokecolor{currentstroke}%
\pgfsetstrokeopacity{0.000000}%
\pgfsetdash{}{0pt}%
\pgfpathmoveto{\pgfqpoint{1.187672in}{0.499444in}}%
\pgfpathlineto{\pgfqpoint{1.251081in}{0.499444in}}%
\pgfpathlineto{\pgfqpoint{1.251081in}{0.512136in}}%
\pgfpathlineto{\pgfqpoint{1.187672in}{0.512136in}}%
\pgfpathlineto{\pgfqpoint{1.187672in}{0.499444in}}%
\pgfpathclose%
\pgfusepath{fill}%
\end{pgfscope}%
\begin{pgfscope}%
\pgfpathrectangle{\pgfqpoint{0.553581in}{0.499444in}}{\pgfqpoint{3.487500in}{1.155000in}}%
\pgfusepath{clip}%
\pgfsetbuttcap%
\pgfsetmiterjoin%
\definecolor{currentfill}{rgb}{0.000000,0.000000,0.000000}%
\pgfsetfillcolor{currentfill}%
\pgfsetlinewidth{0.000000pt}%
\definecolor{currentstroke}{rgb}{0.000000,0.000000,0.000000}%
\pgfsetstrokecolor{currentstroke}%
\pgfsetstrokeopacity{0.000000}%
\pgfsetdash{}{0pt}%
\pgfpathmoveto{\pgfqpoint{1.346194in}{0.499444in}}%
\pgfpathlineto{\pgfqpoint{1.409603in}{0.499444in}}%
\pgfpathlineto{\pgfqpoint{1.409603in}{0.522871in}}%
\pgfpathlineto{\pgfqpoint{1.346194in}{0.522871in}}%
\pgfpathlineto{\pgfqpoint{1.346194in}{0.499444in}}%
\pgfpathclose%
\pgfusepath{fill}%
\end{pgfscope}%
\begin{pgfscope}%
\pgfpathrectangle{\pgfqpoint{0.553581in}{0.499444in}}{\pgfqpoint{3.487500in}{1.155000in}}%
\pgfusepath{clip}%
\pgfsetbuttcap%
\pgfsetmiterjoin%
\definecolor{currentfill}{rgb}{0.000000,0.000000,0.000000}%
\pgfsetfillcolor{currentfill}%
\pgfsetlinewidth{0.000000pt}%
\definecolor{currentstroke}{rgb}{0.000000,0.000000,0.000000}%
\pgfsetstrokecolor{currentstroke}%
\pgfsetstrokeopacity{0.000000}%
\pgfsetdash{}{0pt}%
\pgfpathmoveto{\pgfqpoint{1.504717in}{0.499444in}}%
\pgfpathlineto{\pgfqpoint{1.568126in}{0.499444in}}%
\pgfpathlineto{\pgfqpoint{1.568126in}{0.537036in}}%
\pgfpathlineto{\pgfqpoint{1.504717in}{0.537036in}}%
\pgfpathlineto{\pgfqpoint{1.504717in}{0.499444in}}%
\pgfpathclose%
\pgfusepath{fill}%
\end{pgfscope}%
\begin{pgfscope}%
\pgfpathrectangle{\pgfqpoint{0.553581in}{0.499444in}}{\pgfqpoint{3.487500in}{1.155000in}}%
\pgfusepath{clip}%
\pgfsetbuttcap%
\pgfsetmiterjoin%
\definecolor{currentfill}{rgb}{0.000000,0.000000,0.000000}%
\pgfsetfillcolor{currentfill}%
\pgfsetlinewidth{0.000000pt}%
\definecolor{currentstroke}{rgb}{0.000000,0.000000,0.000000}%
\pgfsetstrokecolor{currentstroke}%
\pgfsetstrokeopacity{0.000000}%
\pgfsetdash{}{0pt}%
\pgfpathmoveto{\pgfqpoint{1.663240in}{0.499444in}}%
\pgfpathlineto{\pgfqpoint{1.726649in}{0.499444in}}%
\pgfpathlineto{\pgfqpoint{1.726649in}{0.555487in}}%
\pgfpathlineto{\pgfqpoint{1.663240in}{0.555487in}}%
\pgfpathlineto{\pgfqpoint{1.663240in}{0.499444in}}%
\pgfpathclose%
\pgfusepath{fill}%
\end{pgfscope}%
\begin{pgfscope}%
\pgfpathrectangle{\pgfqpoint{0.553581in}{0.499444in}}{\pgfqpoint{3.487500in}{1.155000in}}%
\pgfusepath{clip}%
\pgfsetbuttcap%
\pgfsetmiterjoin%
\definecolor{currentfill}{rgb}{0.000000,0.000000,0.000000}%
\pgfsetfillcolor{currentfill}%
\pgfsetlinewidth{0.000000pt}%
\definecolor{currentstroke}{rgb}{0.000000,0.000000,0.000000}%
\pgfsetstrokecolor{currentstroke}%
\pgfsetstrokeopacity{0.000000}%
\pgfsetdash{}{0pt}%
\pgfpathmoveto{\pgfqpoint{1.821763in}{0.499444in}}%
\pgfpathlineto{\pgfqpoint{1.885172in}{0.499444in}}%
\pgfpathlineto{\pgfqpoint{1.885172in}{0.576827in}}%
\pgfpathlineto{\pgfqpoint{1.821763in}{0.576827in}}%
\pgfpathlineto{\pgfqpoint{1.821763in}{0.499444in}}%
\pgfpathclose%
\pgfusepath{fill}%
\end{pgfscope}%
\begin{pgfscope}%
\pgfpathrectangle{\pgfqpoint{0.553581in}{0.499444in}}{\pgfqpoint{3.487500in}{1.155000in}}%
\pgfusepath{clip}%
\pgfsetbuttcap%
\pgfsetmiterjoin%
\definecolor{currentfill}{rgb}{0.000000,0.000000,0.000000}%
\pgfsetfillcolor{currentfill}%
\pgfsetlinewidth{0.000000pt}%
\definecolor{currentstroke}{rgb}{0.000000,0.000000,0.000000}%
\pgfsetstrokecolor{currentstroke}%
\pgfsetstrokeopacity{0.000000}%
\pgfsetdash{}{0pt}%
\pgfpathmoveto{\pgfqpoint{1.980285in}{0.499444in}}%
\pgfpathlineto{\pgfqpoint{2.043694in}{0.499444in}}%
\pgfpathlineto{\pgfqpoint{2.043694in}{0.600161in}}%
\pgfpathlineto{\pgfqpoint{1.980285in}{0.600161in}}%
\pgfpathlineto{\pgfqpoint{1.980285in}{0.499444in}}%
\pgfpathclose%
\pgfusepath{fill}%
\end{pgfscope}%
\begin{pgfscope}%
\pgfpathrectangle{\pgfqpoint{0.553581in}{0.499444in}}{\pgfqpoint{3.487500in}{1.155000in}}%
\pgfusepath{clip}%
\pgfsetbuttcap%
\pgfsetmiterjoin%
\definecolor{currentfill}{rgb}{0.000000,0.000000,0.000000}%
\pgfsetfillcolor{currentfill}%
\pgfsetlinewidth{0.000000pt}%
\definecolor{currentstroke}{rgb}{0.000000,0.000000,0.000000}%
\pgfsetstrokecolor{currentstroke}%
\pgfsetstrokeopacity{0.000000}%
\pgfsetdash{}{0pt}%
\pgfpathmoveto{\pgfqpoint{2.138808in}{0.499444in}}%
\pgfpathlineto{\pgfqpoint{2.202217in}{0.499444in}}%
\pgfpathlineto{\pgfqpoint{2.202217in}{0.625042in}}%
\pgfpathlineto{\pgfqpoint{2.138808in}{0.625042in}}%
\pgfpathlineto{\pgfqpoint{2.138808in}{0.499444in}}%
\pgfpathclose%
\pgfusepath{fill}%
\end{pgfscope}%
\begin{pgfscope}%
\pgfpathrectangle{\pgfqpoint{0.553581in}{0.499444in}}{\pgfqpoint{3.487500in}{1.155000in}}%
\pgfusepath{clip}%
\pgfsetbuttcap%
\pgfsetmiterjoin%
\definecolor{currentfill}{rgb}{0.000000,0.000000,0.000000}%
\pgfsetfillcolor{currentfill}%
\pgfsetlinewidth{0.000000pt}%
\definecolor{currentstroke}{rgb}{0.000000,0.000000,0.000000}%
\pgfsetstrokecolor{currentstroke}%
\pgfsetstrokeopacity{0.000000}%
\pgfsetdash{}{0pt}%
\pgfpathmoveto{\pgfqpoint{2.297331in}{0.499444in}}%
\pgfpathlineto{\pgfqpoint{2.360740in}{0.499444in}}%
\pgfpathlineto{\pgfqpoint{2.360740in}{0.648805in}}%
\pgfpathlineto{\pgfqpoint{2.297331in}{0.648805in}}%
\pgfpathlineto{\pgfqpoint{2.297331in}{0.499444in}}%
\pgfpathclose%
\pgfusepath{fill}%
\end{pgfscope}%
\begin{pgfscope}%
\pgfpathrectangle{\pgfqpoint{0.553581in}{0.499444in}}{\pgfqpoint{3.487500in}{1.155000in}}%
\pgfusepath{clip}%
\pgfsetbuttcap%
\pgfsetmiterjoin%
\definecolor{currentfill}{rgb}{0.000000,0.000000,0.000000}%
\pgfsetfillcolor{currentfill}%
\pgfsetlinewidth{0.000000pt}%
\definecolor{currentstroke}{rgb}{0.000000,0.000000,0.000000}%
\pgfsetstrokecolor{currentstroke}%
\pgfsetstrokeopacity{0.000000}%
\pgfsetdash{}{0pt}%
\pgfpathmoveto{\pgfqpoint{2.455853in}{0.499444in}}%
\pgfpathlineto{\pgfqpoint{2.519263in}{0.499444in}}%
\pgfpathlineto{\pgfqpoint{2.519263in}{0.669660in}}%
\pgfpathlineto{\pgfqpoint{2.455853in}{0.669660in}}%
\pgfpathlineto{\pgfqpoint{2.455853in}{0.499444in}}%
\pgfpathclose%
\pgfusepath{fill}%
\end{pgfscope}%
\begin{pgfscope}%
\pgfpathrectangle{\pgfqpoint{0.553581in}{0.499444in}}{\pgfqpoint{3.487500in}{1.155000in}}%
\pgfusepath{clip}%
\pgfsetbuttcap%
\pgfsetmiterjoin%
\definecolor{currentfill}{rgb}{0.000000,0.000000,0.000000}%
\pgfsetfillcolor{currentfill}%
\pgfsetlinewidth{0.000000pt}%
\definecolor{currentstroke}{rgb}{0.000000,0.000000,0.000000}%
\pgfsetstrokecolor{currentstroke}%
\pgfsetstrokeopacity{0.000000}%
\pgfsetdash{}{0pt}%
\pgfpathmoveto{\pgfqpoint{2.614376in}{0.499444in}}%
\pgfpathlineto{\pgfqpoint{2.677785in}{0.499444in}}%
\pgfpathlineto{\pgfqpoint{2.677785in}{0.686732in}}%
\pgfpathlineto{\pgfqpoint{2.614376in}{0.686732in}}%
\pgfpathlineto{\pgfqpoint{2.614376in}{0.499444in}}%
\pgfpathclose%
\pgfusepath{fill}%
\end{pgfscope}%
\begin{pgfscope}%
\pgfpathrectangle{\pgfqpoint{0.553581in}{0.499444in}}{\pgfqpoint{3.487500in}{1.155000in}}%
\pgfusepath{clip}%
\pgfsetbuttcap%
\pgfsetmiterjoin%
\definecolor{currentfill}{rgb}{0.000000,0.000000,0.000000}%
\pgfsetfillcolor{currentfill}%
\pgfsetlinewidth{0.000000pt}%
\definecolor{currentstroke}{rgb}{0.000000,0.000000,0.000000}%
\pgfsetstrokecolor{currentstroke}%
\pgfsetstrokeopacity{0.000000}%
\pgfsetdash{}{0pt}%
\pgfpathmoveto{\pgfqpoint{2.772899in}{0.499444in}}%
\pgfpathlineto{\pgfqpoint{2.836308in}{0.499444in}}%
\pgfpathlineto{\pgfqpoint{2.836308in}{0.688372in}}%
\pgfpathlineto{\pgfqpoint{2.772899in}{0.688372in}}%
\pgfpathlineto{\pgfqpoint{2.772899in}{0.499444in}}%
\pgfpathclose%
\pgfusepath{fill}%
\end{pgfscope}%
\begin{pgfscope}%
\pgfpathrectangle{\pgfqpoint{0.553581in}{0.499444in}}{\pgfqpoint{3.487500in}{1.155000in}}%
\pgfusepath{clip}%
\pgfsetbuttcap%
\pgfsetmiterjoin%
\definecolor{currentfill}{rgb}{0.000000,0.000000,0.000000}%
\pgfsetfillcolor{currentfill}%
\pgfsetlinewidth{0.000000pt}%
\definecolor{currentstroke}{rgb}{0.000000,0.000000,0.000000}%
\pgfsetstrokecolor{currentstroke}%
\pgfsetstrokeopacity{0.000000}%
\pgfsetdash{}{0pt}%
\pgfpathmoveto{\pgfqpoint{2.931422in}{0.499444in}}%
\pgfpathlineto{\pgfqpoint{2.994831in}{0.499444in}}%
\pgfpathlineto{\pgfqpoint{2.994831in}{0.685613in}}%
\pgfpathlineto{\pgfqpoint{2.931422in}{0.685613in}}%
\pgfpathlineto{\pgfqpoint{2.931422in}{0.499444in}}%
\pgfpathclose%
\pgfusepath{fill}%
\end{pgfscope}%
\begin{pgfscope}%
\pgfpathrectangle{\pgfqpoint{0.553581in}{0.499444in}}{\pgfqpoint{3.487500in}{1.155000in}}%
\pgfusepath{clip}%
\pgfsetbuttcap%
\pgfsetmiterjoin%
\definecolor{currentfill}{rgb}{0.000000,0.000000,0.000000}%
\pgfsetfillcolor{currentfill}%
\pgfsetlinewidth{0.000000pt}%
\definecolor{currentstroke}{rgb}{0.000000,0.000000,0.000000}%
\pgfsetstrokecolor{currentstroke}%
\pgfsetstrokeopacity{0.000000}%
\pgfsetdash{}{0pt}%
\pgfpathmoveto{\pgfqpoint{3.089944in}{0.499444in}}%
\pgfpathlineto{\pgfqpoint{3.153353in}{0.499444in}}%
\pgfpathlineto{\pgfqpoint{3.153353in}{0.675400in}}%
\pgfpathlineto{\pgfqpoint{3.089944in}{0.675400in}}%
\pgfpathlineto{\pgfqpoint{3.089944in}{0.499444in}}%
\pgfpathclose%
\pgfusepath{fill}%
\end{pgfscope}%
\begin{pgfscope}%
\pgfpathrectangle{\pgfqpoint{0.553581in}{0.499444in}}{\pgfqpoint{3.487500in}{1.155000in}}%
\pgfusepath{clip}%
\pgfsetbuttcap%
\pgfsetmiterjoin%
\definecolor{currentfill}{rgb}{0.000000,0.000000,0.000000}%
\pgfsetfillcolor{currentfill}%
\pgfsetlinewidth{0.000000pt}%
\definecolor{currentstroke}{rgb}{0.000000,0.000000,0.000000}%
\pgfsetstrokecolor{currentstroke}%
\pgfsetstrokeopacity{0.000000}%
\pgfsetdash{}{0pt}%
\pgfpathmoveto{\pgfqpoint{3.248467in}{0.499444in}}%
\pgfpathlineto{\pgfqpoint{3.311876in}{0.499444in}}%
\pgfpathlineto{\pgfqpoint{3.311876in}{0.663677in}}%
\pgfpathlineto{\pgfqpoint{3.248467in}{0.663677in}}%
\pgfpathlineto{\pgfqpoint{3.248467in}{0.499444in}}%
\pgfpathclose%
\pgfusepath{fill}%
\end{pgfscope}%
\begin{pgfscope}%
\pgfpathrectangle{\pgfqpoint{0.553581in}{0.499444in}}{\pgfqpoint{3.487500in}{1.155000in}}%
\pgfusepath{clip}%
\pgfsetbuttcap%
\pgfsetmiterjoin%
\definecolor{currentfill}{rgb}{0.000000,0.000000,0.000000}%
\pgfsetfillcolor{currentfill}%
\pgfsetlinewidth{0.000000pt}%
\definecolor{currentstroke}{rgb}{0.000000,0.000000,0.000000}%
\pgfsetstrokecolor{currentstroke}%
\pgfsetstrokeopacity{0.000000}%
\pgfsetdash{}{0pt}%
\pgfpathmoveto{\pgfqpoint{3.406990in}{0.499444in}}%
\pgfpathlineto{\pgfqpoint{3.470399in}{0.499444in}}%
\pgfpathlineto{\pgfqpoint{3.470399in}{0.655290in}}%
\pgfpathlineto{\pgfqpoint{3.406990in}{0.655290in}}%
\pgfpathlineto{\pgfqpoint{3.406990in}{0.499444in}}%
\pgfpathclose%
\pgfusepath{fill}%
\end{pgfscope}%
\begin{pgfscope}%
\pgfpathrectangle{\pgfqpoint{0.553581in}{0.499444in}}{\pgfqpoint{3.487500in}{1.155000in}}%
\pgfusepath{clip}%
\pgfsetbuttcap%
\pgfsetmiterjoin%
\definecolor{currentfill}{rgb}{0.000000,0.000000,0.000000}%
\pgfsetfillcolor{currentfill}%
\pgfsetlinewidth{0.000000pt}%
\definecolor{currentstroke}{rgb}{0.000000,0.000000,0.000000}%
\pgfsetstrokecolor{currentstroke}%
\pgfsetstrokeopacity{0.000000}%
\pgfsetdash{}{0pt}%
\pgfpathmoveto{\pgfqpoint{3.565513in}{0.499444in}}%
\pgfpathlineto{\pgfqpoint{3.628922in}{0.499444in}}%
\pgfpathlineto{\pgfqpoint{3.628922in}{0.639747in}}%
\pgfpathlineto{\pgfqpoint{3.565513in}{0.639747in}}%
\pgfpathlineto{\pgfqpoint{3.565513in}{0.499444in}}%
\pgfpathclose%
\pgfusepath{fill}%
\end{pgfscope}%
\begin{pgfscope}%
\pgfpathrectangle{\pgfqpoint{0.553581in}{0.499444in}}{\pgfqpoint{3.487500in}{1.155000in}}%
\pgfusepath{clip}%
\pgfsetbuttcap%
\pgfsetmiterjoin%
\definecolor{currentfill}{rgb}{0.000000,0.000000,0.000000}%
\pgfsetfillcolor{currentfill}%
\pgfsetlinewidth{0.000000pt}%
\definecolor{currentstroke}{rgb}{0.000000,0.000000,0.000000}%
\pgfsetstrokecolor{currentstroke}%
\pgfsetstrokeopacity{0.000000}%
\pgfsetdash{}{0pt}%
\pgfpathmoveto{\pgfqpoint{3.724035in}{0.499444in}}%
\pgfpathlineto{\pgfqpoint{3.787444in}{0.499444in}}%
\pgfpathlineto{\pgfqpoint{3.787444in}{0.603832in}}%
\pgfpathlineto{\pgfqpoint{3.724035in}{0.603832in}}%
\pgfpathlineto{\pgfqpoint{3.724035in}{0.499444in}}%
\pgfpathclose%
\pgfusepath{fill}%
\end{pgfscope}%
\begin{pgfscope}%
\pgfpathrectangle{\pgfqpoint{0.553581in}{0.499444in}}{\pgfqpoint{3.487500in}{1.155000in}}%
\pgfusepath{clip}%
\pgfsetbuttcap%
\pgfsetmiterjoin%
\definecolor{currentfill}{rgb}{0.000000,0.000000,0.000000}%
\pgfsetfillcolor{currentfill}%
\pgfsetlinewidth{0.000000pt}%
\definecolor{currentstroke}{rgb}{0.000000,0.000000,0.000000}%
\pgfsetstrokecolor{currentstroke}%
\pgfsetstrokeopacity{0.000000}%
\pgfsetdash{}{0pt}%
\pgfpathmoveto{\pgfqpoint{3.882558in}{0.499444in}}%
\pgfpathlineto{\pgfqpoint{3.945967in}{0.499444in}}%
\pgfpathlineto{\pgfqpoint{3.945967in}{0.537725in}}%
\pgfpathlineto{\pgfqpoint{3.882558in}{0.537725in}}%
\pgfpathlineto{\pgfqpoint{3.882558in}{0.499444in}}%
\pgfpathclose%
\pgfusepath{fill}%
\end{pgfscope}%
\begin{pgfscope}%
\pgfsetbuttcap%
\pgfsetroundjoin%
\definecolor{currentfill}{rgb}{0.000000,0.000000,0.000000}%
\pgfsetfillcolor{currentfill}%
\pgfsetlinewidth{0.803000pt}%
\definecolor{currentstroke}{rgb}{0.000000,0.000000,0.000000}%
\pgfsetstrokecolor{currentstroke}%
\pgfsetdash{}{0pt}%
\pgfsys@defobject{currentmarker}{\pgfqpoint{0.000000in}{-0.048611in}}{\pgfqpoint{0.000000in}{0.000000in}}{%
\pgfpathmoveto{\pgfqpoint{0.000000in}{0.000000in}}%
\pgfpathlineto{\pgfqpoint{0.000000in}{-0.048611in}}%
\pgfusepath{stroke,fill}%
}%
\begin{pgfscope}%
\pgfsys@transformshift{0.553581in}{0.499444in}%
\pgfsys@useobject{currentmarker}{}%
\end{pgfscope}%
\end{pgfscope}%
\begin{pgfscope}%
\pgfsetbuttcap%
\pgfsetroundjoin%
\definecolor{currentfill}{rgb}{0.000000,0.000000,0.000000}%
\pgfsetfillcolor{currentfill}%
\pgfsetlinewidth{0.803000pt}%
\definecolor{currentstroke}{rgb}{0.000000,0.000000,0.000000}%
\pgfsetstrokecolor{currentstroke}%
\pgfsetdash{}{0pt}%
\pgfsys@defobject{currentmarker}{\pgfqpoint{0.000000in}{-0.048611in}}{\pgfqpoint{0.000000in}{0.000000in}}{%
\pgfpathmoveto{\pgfqpoint{0.000000in}{0.000000in}}%
\pgfpathlineto{\pgfqpoint{0.000000in}{-0.048611in}}%
\pgfusepath{stroke,fill}%
}%
\begin{pgfscope}%
\pgfsys@transformshift{0.712103in}{0.499444in}%
\pgfsys@useobject{currentmarker}{}%
\end{pgfscope}%
\end{pgfscope}%
\begin{pgfscope}%
\definecolor{textcolor}{rgb}{0.000000,0.000000,0.000000}%
\pgfsetstrokecolor{textcolor}%
\pgfsetfillcolor{textcolor}%
\pgftext[x=0.712103in,y=0.402222in,,top]{\color{textcolor}\rmfamily\fontsize{10.000000}{12.000000}\selectfont 0.0}%
\end{pgfscope}%
\begin{pgfscope}%
\pgfsetbuttcap%
\pgfsetroundjoin%
\definecolor{currentfill}{rgb}{0.000000,0.000000,0.000000}%
\pgfsetfillcolor{currentfill}%
\pgfsetlinewidth{0.803000pt}%
\definecolor{currentstroke}{rgb}{0.000000,0.000000,0.000000}%
\pgfsetstrokecolor{currentstroke}%
\pgfsetdash{}{0pt}%
\pgfsys@defobject{currentmarker}{\pgfqpoint{0.000000in}{-0.048611in}}{\pgfqpoint{0.000000in}{0.000000in}}{%
\pgfpathmoveto{\pgfqpoint{0.000000in}{0.000000in}}%
\pgfpathlineto{\pgfqpoint{0.000000in}{-0.048611in}}%
\pgfusepath{stroke,fill}%
}%
\begin{pgfscope}%
\pgfsys@transformshift{0.870626in}{0.499444in}%
\pgfsys@useobject{currentmarker}{}%
\end{pgfscope}%
\end{pgfscope}%
\begin{pgfscope}%
\pgfsetbuttcap%
\pgfsetroundjoin%
\definecolor{currentfill}{rgb}{0.000000,0.000000,0.000000}%
\pgfsetfillcolor{currentfill}%
\pgfsetlinewidth{0.803000pt}%
\definecolor{currentstroke}{rgb}{0.000000,0.000000,0.000000}%
\pgfsetstrokecolor{currentstroke}%
\pgfsetdash{}{0pt}%
\pgfsys@defobject{currentmarker}{\pgfqpoint{0.000000in}{-0.048611in}}{\pgfqpoint{0.000000in}{0.000000in}}{%
\pgfpathmoveto{\pgfqpoint{0.000000in}{0.000000in}}%
\pgfpathlineto{\pgfqpoint{0.000000in}{-0.048611in}}%
\pgfusepath{stroke,fill}%
}%
\begin{pgfscope}%
\pgfsys@transformshift{1.029149in}{0.499444in}%
\pgfsys@useobject{currentmarker}{}%
\end{pgfscope}%
\end{pgfscope}%
\begin{pgfscope}%
\definecolor{textcolor}{rgb}{0.000000,0.000000,0.000000}%
\pgfsetstrokecolor{textcolor}%
\pgfsetfillcolor{textcolor}%
\pgftext[x=1.029149in,y=0.402222in,,top]{\color{textcolor}\rmfamily\fontsize{10.000000}{12.000000}\selectfont 0.1}%
\end{pgfscope}%
\begin{pgfscope}%
\pgfsetbuttcap%
\pgfsetroundjoin%
\definecolor{currentfill}{rgb}{0.000000,0.000000,0.000000}%
\pgfsetfillcolor{currentfill}%
\pgfsetlinewidth{0.803000pt}%
\definecolor{currentstroke}{rgb}{0.000000,0.000000,0.000000}%
\pgfsetstrokecolor{currentstroke}%
\pgfsetdash{}{0pt}%
\pgfsys@defobject{currentmarker}{\pgfqpoint{0.000000in}{-0.048611in}}{\pgfqpoint{0.000000in}{0.000000in}}{%
\pgfpathmoveto{\pgfqpoint{0.000000in}{0.000000in}}%
\pgfpathlineto{\pgfqpoint{0.000000in}{-0.048611in}}%
\pgfusepath{stroke,fill}%
}%
\begin{pgfscope}%
\pgfsys@transformshift{1.187672in}{0.499444in}%
\pgfsys@useobject{currentmarker}{}%
\end{pgfscope}%
\end{pgfscope}%
\begin{pgfscope}%
\pgfsetbuttcap%
\pgfsetroundjoin%
\definecolor{currentfill}{rgb}{0.000000,0.000000,0.000000}%
\pgfsetfillcolor{currentfill}%
\pgfsetlinewidth{0.803000pt}%
\definecolor{currentstroke}{rgb}{0.000000,0.000000,0.000000}%
\pgfsetstrokecolor{currentstroke}%
\pgfsetdash{}{0pt}%
\pgfsys@defobject{currentmarker}{\pgfqpoint{0.000000in}{-0.048611in}}{\pgfqpoint{0.000000in}{0.000000in}}{%
\pgfpathmoveto{\pgfqpoint{0.000000in}{0.000000in}}%
\pgfpathlineto{\pgfqpoint{0.000000in}{-0.048611in}}%
\pgfusepath{stroke,fill}%
}%
\begin{pgfscope}%
\pgfsys@transformshift{1.346194in}{0.499444in}%
\pgfsys@useobject{currentmarker}{}%
\end{pgfscope}%
\end{pgfscope}%
\begin{pgfscope}%
\definecolor{textcolor}{rgb}{0.000000,0.000000,0.000000}%
\pgfsetstrokecolor{textcolor}%
\pgfsetfillcolor{textcolor}%
\pgftext[x=1.346194in,y=0.402222in,,top]{\color{textcolor}\rmfamily\fontsize{10.000000}{12.000000}\selectfont 0.2}%
\end{pgfscope}%
\begin{pgfscope}%
\pgfsetbuttcap%
\pgfsetroundjoin%
\definecolor{currentfill}{rgb}{0.000000,0.000000,0.000000}%
\pgfsetfillcolor{currentfill}%
\pgfsetlinewidth{0.803000pt}%
\definecolor{currentstroke}{rgb}{0.000000,0.000000,0.000000}%
\pgfsetstrokecolor{currentstroke}%
\pgfsetdash{}{0pt}%
\pgfsys@defobject{currentmarker}{\pgfqpoint{0.000000in}{-0.048611in}}{\pgfqpoint{0.000000in}{0.000000in}}{%
\pgfpathmoveto{\pgfqpoint{0.000000in}{0.000000in}}%
\pgfpathlineto{\pgfqpoint{0.000000in}{-0.048611in}}%
\pgfusepath{stroke,fill}%
}%
\begin{pgfscope}%
\pgfsys@transformshift{1.504717in}{0.499444in}%
\pgfsys@useobject{currentmarker}{}%
\end{pgfscope}%
\end{pgfscope}%
\begin{pgfscope}%
\pgfsetbuttcap%
\pgfsetroundjoin%
\definecolor{currentfill}{rgb}{0.000000,0.000000,0.000000}%
\pgfsetfillcolor{currentfill}%
\pgfsetlinewidth{0.803000pt}%
\definecolor{currentstroke}{rgb}{0.000000,0.000000,0.000000}%
\pgfsetstrokecolor{currentstroke}%
\pgfsetdash{}{0pt}%
\pgfsys@defobject{currentmarker}{\pgfqpoint{0.000000in}{-0.048611in}}{\pgfqpoint{0.000000in}{0.000000in}}{%
\pgfpathmoveto{\pgfqpoint{0.000000in}{0.000000in}}%
\pgfpathlineto{\pgfqpoint{0.000000in}{-0.048611in}}%
\pgfusepath{stroke,fill}%
}%
\begin{pgfscope}%
\pgfsys@transformshift{1.663240in}{0.499444in}%
\pgfsys@useobject{currentmarker}{}%
\end{pgfscope}%
\end{pgfscope}%
\begin{pgfscope}%
\definecolor{textcolor}{rgb}{0.000000,0.000000,0.000000}%
\pgfsetstrokecolor{textcolor}%
\pgfsetfillcolor{textcolor}%
\pgftext[x=1.663240in,y=0.402222in,,top]{\color{textcolor}\rmfamily\fontsize{10.000000}{12.000000}\selectfont 0.3}%
\end{pgfscope}%
\begin{pgfscope}%
\pgfsetbuttcap%
\pgfsetroundjoin%
\definecolor{currentfill}{rgb}{0.000000,0.000000,0.000000}%
\pgfsetfillcolor{currentfill}%
\pgfsetlinewidth{0.803000pt}%
\definecolor{currentstroke}{rgb}{0.000000,0.000000,0.000000}%
\pgfsetstrokecolor{currentstroke}%
\pgfsetdash{}{0pt}%
\pgfsys@defobject{currentmarker}{\pgfqpoint{0.000000in}{-0.048611in}}{\pgfqpoint{0.000000in}{0.000000in}}{%
\pgfpathmoveto{\pgfqpoint{0.000000in}{0.000000in}}%
\pgfpathlineto{\pgfqpoint{0.000000in}{-0.048611in}}%
\pgfusepath{stroke,fill}%
}%
\begin{pgfscope}%
\pgfsys@transformshift{1.821763in}{0.499444in}%
\pgfsys@useobject{currentmarker}{}%
\end{pgfscope}%
\end{pgfscope}%
\begin{pgfscope}%
\pgfsetbuttcap%
\pgfsetroundjoin%
\definecolor{currentfill}{rgb}{0.000000,0.000000,0.000000}%
\pgfsetfillcolor{currentfill}%
\pgfsetlinewidth{0.803000pt}%
\definecolor{currentstroke}{rgb}{0.000000,0.000000,0.000000}%
\pgfsetstrokecolor{currentstroke}%
\pgfsetdash{}{0pt}%
\pgfsys@defobject{currentmarker}{\pgfqpoint{0.000000in}{-0.048611in}}{\pgfqpoint{0.000000in}{0.000000in}}{%
\pgfpathmoveto{\pgfqpoint{0.000000in}{0.000000in}}%
\pgfpathlineto{\pgfqpoint{0.000000in}{-0.048611in}}%
\pgfusepath{stroke,fill}%
}%
\begin{pgfscope}%
\pgfsys@transformshift{1.980285in}{0.499444in}%
\pgfsys@useobject{currentmarker}{}%
\end{pgfscope}%
\end{pgfscope}%
\begin{pgfscope}%
\definecolor{textcolor}{rgb}{0.000000,0.000000,0.000000}%
\pgfsetstrokecolor{textcolor}%
\pgfsetfillcolor{textcolor}%
\pgftext[x=1.980285in,y=0.402222in,,top]{\color{textcolor}\rmfamily\fontsize{10.000000}{12.000000}\selectfont 0.4}%
\end{pgfscope}%
\begin{pgfscope}%
\pgfsetbuttcap%
\pgfsetroundjoin%
\definecolor{currentfill}{rgb}{0.000000,0.000000,0.000000}%
\pgfsetfillcolor{currentfill}%
\pgfsetlinewidth{0.803000pt}%
\definecolor{currentstroke}{rgb}{0.000000,0.000000,0.000000}%
\pgfsetstrokecolor{currentstroke}%
\pgfsetdash{}{0pt}%
\pgfsys@defobject{currentmarker}{\pgfqpoint{0.000000in}{-0.048611in}}{\pgfqpoint{0.000000in}{0.000000in}}{%
\pgfpathmoveto{\pgfqpoint{0.000000in}{0.000000in}}%
\pgfpathlineto{\pgfqpoint{0.000000in}{-0.048611in}}%
\pgfusepath{stroke,fill}%
}%
\begin{pgfscope}%
\pgfsys@transformshift{2.138808in}{0.499444in}%
\pgfsys@useobject{currentmarker}{}%
\end{pgfscope}%
\end{pgfscope}%
\begin{pgfscope}%
\pgfsetbuttcap%
\pgfsetroundjoin%
\definecolor{currentfill}{rgb}{0.000000,0.000000,0.000000}%
\pgfsetfillcolor{currentfill}%
\pgfsetlinewidth{0.803000pt}%
\definecolor{currentstroke}{rgb}{0.000000,0.000000,0.000000}%
\pgfsetstrokecolor{currentstroke}%
\pgfsetdash{}{0pt}%
\pgfsys@defobject{currentmarker}{\pgfqpoint{0.000000in}{-0.048611in}}{\pgfqpoint{0.000000in}{0.000000in}}{%
\pgfpathmoveto{\pgfqpoint{0.000000in}{0.000000in}}%
\pgfpathlineto{\pgfqpoint{0.000000in}{-0.048611in}}%
\pgfusepath{stroke,fill}%
}%
\begin{pgfscope}%
\pgfsys@transformshift{2.297331in}{0.499444in}%
\pgfsys@useobject{currentmarker}{}%
\end{pgfscope}%
\end{pgfscope}%
\begin{pgfscope}%
\definecolor{textcolor}{rgb}{0.000000,0.000000,0.000000}%
\pgfsetstrokecolor{textcolor}%
\pgfsetfillcolor{textcolor}%
\pgftext[x=2.297331in,y=0.402222in,,top]{\color{textcolor}\rmfamily\fontsize{10.000000}{12.000000}\selectfont 0.5}%
\end{pgfscope}%
\begin{pgfscope}%
\pgfsetbuttcap%
\pgfsetroundjoin%
\definecolor{currentfill}{rgb}{0.000000,0.000000,0.000000}%
\pgfsetfillcolor{currentfill}%
\pgfsetlinewidth{0.803000pt}%
\definecolor{currentstroke}{rgb}{0.000000,0.000000,0.000000}%
\pgfsetstrokecolor{currentstroke}%
\pgfsetdash{}{0pt}%
\pgfsys@defobject{currentmarker}{\pgfqpoint{0.000000in}{-0.048611in}}{\pgfqpoint{0.000000in}{0.000000in}}{%
\pgfpathmoveto{\pgfqpoint{0.000000in}{0.000000in}}%
\pgfpathlineto{\pgfqpoint{0.000000in}{-0.048611in}}%
\pgfusepath{stroke,fill}%
}%
\begin{pgfscope}%
\pgfsys@transformshift{2.455853in}{0.499444in}%
\pgfsys@useobject{currentmarker}{}%
\end{pgfscope}%
\end{pgfscope}%
\begin{pgfscope}%
\pgfsetbuttcap%
\pgfsetroundjoin%
\definecolor{currentfill}{rgb}{0.000000,0.000000,0.000000}%
\pgfsetfillcolor{currentfill}%
\pgfsetlinewidth{0.803000pt}%
\definecolor{currentstroke}{rgb}{0.000000,0.000000,0.000000}%
\pgfsetstrokecolor{currentstroke}%
\pgfsetdash{}{0pt}%
\pgfsys@defobject{currentmarker}{\pgfqpoint{0.000000in}{-0.048611in}}{\pgfqpoint{0.000000in}{0.000000in}}{%
\pgfpathmoveto{\pgfqpoint{0.000000in}{0.000000in}}%
\pgfpathlineto{\pgfqpoint{0.000000in}{-0.048611in}}%
\pgfusepath{stroke,fill}%
}%
\begin{pgfscope}%
\pgfsys@transformshift{2.614376in}{0.499444in}%
\pgfsys@useobject{currentmarker}{}%
\end{pgfscope}%
\end{pgfscope}%
\begin{pgfscope}%
\definecolor{textcolor}{rgb}{0.000000,0.000000,0.000000}%
\pgfsetstrokecolor{textcolor}%
\pgfsetfillcolor{textcolor}%
\pgftext[x=2.614376in,y=0.402222in,,top]{\color{textcolor}\rmfamily\fontsize{10.000000}{12.000000}\selectfont 0.6}%
\end{pgfscope}%
\begin{pgfscope}%
\pgfsetbuttcap%
\pgfsetroundjoin%
\definecolor{currentfill}{rgb}{0.000000,0.000000,0.000000}%
\pgfsetfillcolor{currentfill}%
\pgfsetlinewidth{0.803000pt}%
\definecolor{currentstroke}{rgb}{0.000000,0.000000,0.000000}%
\pgfsetstrokecolor{currentstroke}%
\pgfsetdash{}{0pt}%
\pgfsys@defobject{currentmarker}{\pgfqpoint{0.000000in}{-0.048611in}}{\pgfqpoint{0.000000in}{0.000000in}}{%
\pgfpathmoveto{\pgfqpoint{0.000000in}{0.000000in}}%
\pgfpathlineto{\pgfqpoint{0.000000in}{-0.048611in}}%
\pgfusepath{stroke,fill}%
}%
\begin{pgfscope}%
\pgfsys@transformshift{2.772899in}{0.499444in}%
\pgfsys@useobject{currentmarker}{}%
\end{pgfscope}%
\end{pgfscope}%
\begin{pgfscope}%
\pgfsetbuttcap%
\pgfsetroundjoin%
\definecolor{currentfill}{rgb}{0.000000,0.000000,0.000000}%
\pgfsetfillcolor{currentfill}%
\pgfsetlinewidth{0.803000pt}%
\definecolor{currentstroke}{rgb}{0.000000,0.000000,0.000000}%
\pgfsetstrokecolor{currentstroke}%
\pgfsetdash{}{0pt}%
\pgfsys@defobject{currentmarker}{\pgfqpoint{0.000000in}{-0.048611in}}{\pgfqpoint{0.000000in}{0.000000in}}{%
\pgfpathmoveto{\pgfqpoint{0.000000in}{0.000000in}}%
\pgfpathlineto{\pgfqpoint{0.000000in}{-0.048611in}}%
\pgfusepath{stroke,fill}%
}%
\begin{pgfscope}%
\pgfsys@transformshift{2.931422in}{0.499444in}%
\pgfsys@useobject{currentmarker}{}%
\end{pgfscope}%
\end{pgfscope}%
\begin{pgfscope}%
\definecolor{textcolor}{rgb}{0.000000,0.000000,0.000000}%
\pgfsetstrokecolor{textcolor}%
\pgfsetfillcolor{textcolor}%
\pgftext[x=2.931422in,y=0.402222in,,top]{\color{textcolor}\rmfamily\fontsize{10.000000}{12.000000}\selectfont 0.7}%
\end{pgfscope}%
\begin{pgfscope}%
\pgfsetbuttcap%
\pgfsetroundjoin%
\definecolor{currentfill}{rgb}{0.000000,0.000000,0.000000}%
\pgfsetfillcolor{currentfill}%
\pgfsetlinewidth{0.803000pt}%
\definecolor{currentstroke}{rgb}{0.000000,0.000000,0.000000}%
\pgfsetstrokecolor{currentstroke}%
\pgfsetdash{}{0pt}%
\pgfsys@defobject{currentmarker}{\pgfqpoint{0.000000in}{-0.048611in}}{\pgfqpoint{0.000000in}{0.000000in}}{%
\pgfpathmoveto{\pgfqpoint{0.000000in}{0.000000in}}%
\pgfpathlineto{\pgfqpoint{0.000000in}{-0.048611in}}%
\pgfusepath{stroke,fill}%
}%
\begin{pgfscope}%
\pgfsys@transformshift{3.089944in}{0.499444in}%
\pgfsys@useobject{currentmarker}{}%
\end{pgfscope}%
\end{pgfscope}%
\begin{pgfscope}%
\pgfsetbuttcap%
\pgfsetroundjoin%
\definecolor{currentfill}{rgb}{0.000000,0.000000,0.000000}%
\pgfsetfillcolor{currentfill}%
\pgfsetlinewidth{0.803000pt}%
\definecolor{currentstroke}{rgb}{0.000000,0.000000,0.000000}%
\pgfsetstrokecolor{currentstroke}%
\pgfsetdash{}{0pt}%
\pgfsys@defobject{currentmarker}{\pgfqpoint{0.000000in}{-0.048611in}}{\pgfqpoint{0.000000in}{0.000000in}}{%
\pgfpathmoveto{\pgfqpoint{0.000000in}{0.000000in}}%
\pgfpathlineto{\pgfqpoint{0.000000in}{-0.048611in}}%
\pgfusepath{stroke,fill}%
}%
\begin{pgfscope}%
\pgfsys@transformshift{3.248467in}{0.499444in}%
\pgfsys@useobject{currentmarker}{}%
\end{pgfscope}%
\end{pgfscope}%
\begin{pgfscope}%
\definecolor{textcolor}{rgb}{0.000000,0.000000,0.000000}%
\pgfsetstrokecolor{textcolor}%
\pgfsetfillcolor{textcolor}%
\pgftext[x=3.248467in,y=0.402222in,,top]{\color{textcolor}\rmfamily\fontsize{10.000000}{12.000000}\selectfont 0.8}%
\end{pgfscope}%
\begin{pgfscope}%
\pgfsetbuttcap%
\pgfsetroundjoin%
\definecolor{currentfill}{rgb}{0.000000,0.000000,0.000000}%
\pgfsetfillcolor{currentfill}%
\pgfsetlinewidth{0.803000pt}%
\definecolor{currentstroke}{rgb}{0.000000,0.000000,0.000000}%
\pgfsetstrokecolor{currentstroke}%
\pgfsetdash{}{0pt}%
\pgfsys@defobject{currentmarker}{\pgfqpoint{0.000000in}{-0.048611in}}{\pgfqpoint{0.000000in}{0.000000in}}{%
\pgfpathmoveto{\pgfqpoint{0.000000in}{0.000000in}}%
\pgfpathlineto{\pgfqpoint{0.000000in}{-0.048611in}}%
\pgfusepath{stroke,fill}%
}%
\begin{pgfscope}%
\pgfsys@transformshift{3.406990in}{0.499444in}%
\pgfsys@useobject{currentmarker}{}%
\end{pgfscope}%
\end{pgfscope}%
\begin{pgfscope}%
\pgfsetbuttcap%
\pgfsetroundjoin%
\definecolor{currentfill}{rgb}{0.000000,0.000000,0.000000}%
\pgfsetfillcolor{currentfill}%
\pgfsetlinewidth{0.803000pt}%
\definecolor{currentstroke}{rgb}{0.000000,0.000000,0.000000}%
\pgfsetstrokecolor{currentstroke}%
\pgfsetdash{}{0pt}%
\pgfsys@defobject{currentmarker}{\pgfqpoint{0.000000in}{-0.048611in}}{\pgfqpoint{0.000000in}{0.000000in}}{%
\pgfpathmoveto{\pgfqpoint{0.000000in}{0.000000in}}%
\pgfpathlineto{\pgfqpoint{0.000000in}{-0.048611in}}%
\pgfusepath{stroke,fill}%
}%
\begin{pgfscope}%
\pgfsys@transformshift{3.565513in}{0.499444in}%
\pgfsys@useobject{currentmarker}{}%
\end{pgfscope}%
\end{pgfscope}%
\begin{pgfscope}%
\definecolor{textcolor}{rgb}{0.000000,0.000000,0.000000}%
\pgfsetstrokecolor{textcolor}%
\pgfsetfillcolor{textcolor}%
\pgftext[x=3.565513in,y=0.402222in,,top]{\color{textcolor}\rmfamily\fontsize{10.000000}{12.000000}\selectfont 0.9}%
\end{pgfscope}%
\begin{pgfscope}%
\pgfsetbuttcap%
\pgfsetroundjoin%
\definecolor{currentfill}{rgb}{0.000000,0.000000,0.000000}%
\pgfsetfillcolor{currentfill}%
\pgfsetlinewidth{0.803000pt}%
\definecolor{currentstroke}{rgb}{0.000000,0.000000,0.000000}%
\pgfsetstrokecolor{currentstroke}%
\pgfsetdash{}{0pt}%
\pgfsys@defobject{currentmarker}{\pgfqpoint{0.000000in}{-0.048611in}}{\pgfqpoint{0.000000in}{0.000000in}}{%
\pgfpathmoveto{\pgfqpoint{0.000000in}{0.000000in}}%
\pgfpathlineto{\pgfqpoint{0.000000in}{-0.048611in}}%
\pgfusepath{stroke,fill}%
}%
\begin{pgfscope}%
\pgfsys@transformshift{3.724035in}{0.499444in}%
\pgfsys@useobject{currentmarker}{}%
\end{pgfscope}%
\end{pgfscope}%
\begin{pgfscope}%
\pgfsetbuttcap%
\pgfsetroundjoin%
\definecolor{currentfill}{rgb}{0.000000,0.000000,0.000000}%
\pgfsetfillcolor{currentfill}%
\pgfsetlinewidth{0.803000pt}%
\definecolor{currentstroke}{rgb}{0.000000,0.000000,0.000000}%
\pgfsetstrokecolor{currentstroke}%
\pgfsetdash{}{0pt}%
\pgfsys@defobject{currentmarker}{\pgfqpoint{0.000000in}{-0.048611in}}{\pgfqpoint{0.000000in}{0.000000in}}{%
\pgfpathmoveto{\pgfqpoint{0.000000in}{0.000000in}}%
\pgfpathlineto{\pgfqpoint{0.000000in}{-0.048611in}}%
\pgfusepath{stroke,fill}%
}%
\begin{pgfscope}%
\pgfsys@transformshift{3.882558in}{0.499444in}%
\pgfsys@useobject{currentmarker}{}%
\end{pgfscope}%
\end{pgfscope}%
\begin{pgfscope}%
\definecolor{textcolor}{rgb}{0.000000,0.000000,0.000000}%
\pgfsetstrokecolor{textcolor}%
\pgfsetfillcolor{textcolor}%
\pgftext[x=3.882558in,y=0.402222in,,top]{\color{textcolor}\rmfamily\fontsize{10.000000}{12.000000}\selectfont 1.0}%
\end{pgfscope}%
\begin{pgfscope}%
\pgfsetbuttcap%
\pgfsetroundjoin%
\definecolor{currentfill}{rgb}{0.000000,0.000000,0.000000}%
\pgfsetfillcolor{currentfill}%
\pgfsetlinewidth{0.803000pt}%
\definecolor{currentstroke}{rgb}{0.000000,0.000000,0.000000}%
\pgfsetstrokecolor{currentstroke}%
\pgfsetdash{}{0pt}%
\pgfsys@defobject{currentmarker}{\pgfqpoint{0.000000in}{-0.048611in}}{\pgfqpoint{0.000000in}{0.000000in}}{%
\pgfpathmoveto{\pgfqpoint{0.000000in}{0.000000in}}%
\pgfpathlineto{\pgfqpoint{0.000000in}{-0.048611in}}%
\pgfusepath{stroke,fill}%
}%
\begin{pgfscope}%
\pgfsys@transformshift{4.041081in}{0.499444in}%
\pgfsys@useobject{currentmarker}{}%
\end{pgfscope}%
\end{pgfscope}%
\begin{pgfscope}%
\definecolor{textcolor}{rgb}{0.000000,0.000000,0.000000}%
\pgfsetstrokecolor{textcolor}%
\pgfsetfillcolor{textcolor}%
\pgftext[x=2.297331in,y=0.223333in,,top]{\color{textcolor}\rmfamily\fontsize{10.000000}{12.000000}\selectfont \(\displaystyle p\)}%
\end{pgfscope}%
\begin{pgfscope}%
\pgfsetbuttcap%
\pgfsetroundjoin%
\definecolor{currentfill}{rgb}{0.000000,0.000000,0.000000}%
\pgfsetfillcolor{currentfill}%
\pgfsetlinewidth{0.803000pt}%
\definecolor{currentstroke}{rgb}{0.000000,0.000000,0.000000}%
\pgfsetstrokecolor{currentstroke}%
\pgfsetdash{}{0pt}%
\pgfsys@defobject{currentmarker}{\pgfqpoint{-0.048611in}{0.000000in}}{\pgfqpoint{-0.000000in}{0.000000in}}{%
\pgfpathmoveto{\pgfqpoint{-0.000000in}{0.000000in}}%
\pgfpathlineto{\pgfqpoint{-0.048611in}{0.000000in}}%
\pgfusepath{stroke,fill}%
}%
\begin{pgfscope}%
\pgfsys@transformshift{0.553581in}{0.499444in}%
\pgfsys@useobject{currentmarker}{}%
\end{pgfscope}%
\end{pgfscope}%
\begin{pgfscope}%
\definecolor{textcolor}{rgb}{0.000000,0.000000,0.000000}%
\pgfsetstrokecolor{textcolor}%
\pgfsetfillcolor{textcolor}%
\pgftext[x=0.278889in, y=0.451250in, left, base]{\color{textcolor}\rmfamily\fontsize{10.000000}{12.000000}\selectfont \(\displaystyle {0.0}\)}%
\end{pgfscope}%
\begin{pgfscope}%
\pgfsetbuttcap%
\pgfsetroundjoin%
\definecolor{currentfill}{rgb}{0.000000,0.000000,0.000000}%
\pgfsetfillcolor{currentfill}%
\pgfsetlinewidth{0.803000pt}%
\definecolor{currentstroke}{rgb}{0.000000,0.000000,0.000000}%
\pgfsetstrokecolor{currentstroke}%
\pgfsetdash{}{0pt}%
\pgfsys@defobject{currentmarker}{\pgfqpoint{-0.048611in}{0.000000in}}{\pgfqpoint{-0.000000in}{0.000000in}}{%
\pgfpathmoveto{\pgfqpoint{-0.000000in}{0.000000in}}%
\pgfpathlineto{\pgfqpoint{-0.048611in}{0.000000in}}%
\pgfusepath{stroke,fill}%
}%
\begin{pgfscope}%
\pgfsys@transformshift{0.553581in}{0.831920in}%
\pgfsys@useobject{currentmarker}{}%
\end{pgfscope}%
\end{pgfscope}%
\begin{pgfscope}%
\definecolor{textcolor}{rgb}{0.000000,0.000000,0.000000}%
\pgfsetstrokecolor{textcolor}%
\pgfsetfillcolor{textcolor}%
\pgftext[x=0.278889in, y=0.783726in, left, base]{\color{textcolor}\rmfamily\fontsize{10.000000}{12.000000}\selectfont \(\displaystyle {2.5}\)}%
\end{pgfscope}%
\begin{pgfscope}%
\pgfsetbuttcap%
\pgfsetroundjoin%
\definecolor{currentfill}{rgb}{0.000000,0.000000,0.000000}%
\pgfsetfillcolor{currentfill}%
\pgfsetlinewidth{0.803000pt}%
\definecolor{currentstroke}{rgb}{0.000000,0.000000,0.000000}%
\pgfsetstrokecolor{currentstroke}%
\pgfsetdash{}{0pt}%
\pgfsys@defobject{currentmarker}{\pgfqpoint{-0.048611in}{0.000000in}}{\pgfqpoint{-0.000000in}{0.000000in}}{%
\pgfpathmoveto{\pgfqpoint{-0.000000in}{0.000000in}}%
\pgfpathlineto{\pgfqpoint{-0.048611in}{0.000000in}}%
\pgfusepath{stroke,fill}%
}%
\begin{pgfscope}%
\pgfsys@transformshift{0.553581in}{1.164396in}%
\pgfsys@useobject{currentmarker}{}%
\end{pgfscope}%
\end{pgfscope}%
\begin{pgfscope}%
\definecolor{textcolor}{rgb}{0.000000,0.000000,0.000000}%
\pgfsetstrokecolor{textcolor}%
\pgfsetfillcolor{textcolor}%
\pgftext[x=0.278889in, y=1.116202in, left, base]{\color{textcolor}\rmfamily\fontsize{10.000000}{12.000000}\selectfont \(\displaystyle {5.0}\)}%
\end{pgfscope}%
\begin{pgfscope}%
\pgfsetbuttcap%
\pgfsetroundjoin%
\definecolor{currentfill}{rgb}{0.000000,0.000000,0.000000}%
\pgfsetfillcolor{currentfill}%
\pgfsetlinewidth{0.803000pt}%
\definecolor{currentstroke}{rgb}{0.000000,0.000000,0.000000}%
\pgfsetstrokecolor{currentstroke}%
\pgfsetdash{}{0pt}%
\pgfsys@defobject{currentmarker}{\pgfqpoint{-0.048611in}{0.000000in}}{\pgfqpoint{-0.000000in}{0.000000in}}{%
\pgfpathmoveto{\pgfqpoint{-0.000000in}{0.000000in}}%
\pgfpathlineto{\pgfqpoint{-0.048611in}{0.000000in}}%
\pgfusepath{stroke,fill}%
}%
\begin{pgfscope}%
\pgfsys@transformshift{0.553581in}{1.496872in}%
\pgfsys@useobject{currentmarker}{}%
\end{pgfscope}%
\end{pgfscope}%
\begin{pgfscope}%
\definecolor{textcolor}{rgb}{0.000000,0.000000,0.000000}%
\pgfsetstrokecolor{textcolor}%
\pgfsetfillcolor{textcolor}%
\pgftext[x=0.278889in, y=1.448678in, left, base]{\color{textcolor}\rmfamily\fontsize{10.000000}{12.000000}\selectfont \(\displaystyle {7.5}\)}%
\end{pgfscope}%
\begin{pgfscope}%
\definecolor{textcolor}{rgb}{0.000000,0.000000,0.000000}%
\pgfsetstrokecolor{textcolor}%
\pgfsetfillcolor{textcolor}%
\pgftext[x=0.223333in,y=1.076944in,,bottom,rotate=90.000000]{\color{textcolor}\rmfamily\fontsize{10.000000}{12.000000}\selectfont Percent of Data Set}%
\end{pgfscope}%
\begin{pgfscope}%
\pgfsetrectcap%
\pgfsetmiterjoin%
\pgfsetlinewidth{0.803000pt}%
\definecolor{currentstroke}{rgb}{0.000000,0.000000,0.000000}%
\pgfsetstrokecolor{currentstroke}%
\pgfsetdash{}{0pt}%
\pgfpathmoveto{\pgfqpoint{0.553581in}{0.499444in}}%
\pgfpathlineto{\pgfqpoint{0.553581in}{1.654444in}}%
\pgfusepath{stroke}%
\end{pgfscope}%
\begin{pgfscope}%
\pgfsetrectcap%
\pgfsetmiterjoin%
\pgfsetlinewidth{0.803000pt}%
\definecolor{currentstroke}{rgb}{0.000000,0.000000,0.000000}%
\pgfsetstrokecolor{currentstroke}%
\pgfsetdash{}{0pt}%
\pgfpathmoveto{\pgfqpoint{4.041081in}{0.499444in}}%
\pgfpathlineto{\pgfqpoint{4.041081in}{1.654444in}}%
\pgfusepath{stroke}%
\end{pgfscope}%
\begin{pgfscope}%
\pgfsetrectcap%
\pgfsetmiterjoin%
\pgfsetlinewidth{0.803000pt}%
\definecolor{currentstroke}{rgb}{0.000000,0.000000,0.000000}%
\pgfsetstrokecolor{currentstroke}%
\pgfsetdash{}{0pt}%
\pgfpathmoveto{\pgfqpoint{0.553581in}{0.499444in}}%
\pgfpathlineto{\pgfqpoint{4.041081in}{0.499444in}}%
\pgfusepath{stroke}%
\end{pgfscope}%
\begin{pgfscope}%
\pgfsetrectcap%
\pgfsetmiterjoin%
\pgfsetlinewidth{0.803000pt}%
\definecolor{currentstroke}{rgb}{0.000000,0.000000,0.000000}%
\pgfsetstrokecolor{currentstroke}%
\pgfsetdash{}{0pt}%
\pgfpathmoveto{\pgfqpoint{0.553581in}{1.654444in}}%
\pgfpathlineto{\pgfqpoint{4.041081in}{1.654444in}}%
\pgfusepath{stroke}%
\end{pgfscope}%
\begin{pgfscope}%
\pgfsetbuttcap%
\pgfsetmiterjoin%
\definecolor{currentfill}{rgb}{1.000000,1.000000,1.000000}%
\pgfsetfillcolor{currentfill}%
\pgfsetfillopacity{0.800000}%
\pgfsetlinewidth{1.003750pt}%
\definecolor{currentstroke}{rgb}{0.800000,0.800000,0.800000}%
\pgfsetstrokecolor{currentstroke}%
\pgfsetstrokeopacity{0.800000}%
\pgfsetdash{}{0pt}%
\pgfpathmoveto{\pgfqpoint{3.264136in}{1.154445in}}%
\pgfpathlineto{\pgfqpoint{3.943858in}{1.154445in}}%
\pgfpathquadraticcurveto{\pgfqpoint{3.971636in}{1.154445in}}{\pgfqpoint{3.971636in}{1.182222in}}%
\pgfpathlineto{\pgfqpoint{3.971636in}{1.557222in}}%
\pgfpathquadraticcurveto{\pgfqpoint{3.971636in}{1.585000in}}{\pgfqpoint{3.943858in}{1.585000in}}%
\pgfpathlineto{\pgfqpoint{3.264136in}{1.585000in}}%
\pgfpathquadraticcurveto{\pgfqpoint{3.236358in}{1.585000in}}{\pgfqpoint{3.236358in}{1.557222in}}%
\pgfpathlineto{\pgfqpoint{3.236358in}{1.182222in}}%
\pgfpathquadraticcurveto{\pgfqpoint{3.236358in}{1.154445in}}{\pgfqpoint{3.264136in}{1.154445in}}%
\pgfpathlineto{\pgfqpoint{3.264136in}{1.154445in}}%
\pgfpathclose%
\pgfusepath{stroke,fill}%
\end{pgfscope}%
\begin{pgfscope}%
\pgfsetbuttcap%
\pgfsetmiterjoin%
\pgfsetlinewidth{1.003750pt}%
\definecolor{currentstroke}{rgb}{0.000000,0.000000,0.000000}%
\pgfsetstrokecolor{currentstroke}%
\pgfsetdash{}{0pt}%
\pgfpathmoveto{\pgfqpoint{3.291914in}{1.432222in}}%
\pgfpathlineto{\pgfqpoint{3.569692in}{1.432222in}}%
\pgfpathlineto{\pgfqpoint{3.569692in}{1.529444in}}%
\pgfpathlineto{\pgfqpoint{3.291914in}{1.529444in}}%
\pgfpathlineto{\pgfqpoint{3.291914in}{1.432222in}}%
\pgfpathclose%
\pgfusepath{stroke}%
\end{pgfscope}%
\begin{pgfscope}%
\definecolor{textcolor}{rgb}{0.000000,0.000000,0.000000}%
\pgfsetstrokecolor{textcolor}%
\pgfsetfillcolor{textcolor}%
\pgftext[x=3.680803in,y=1.432222in,left,base]{\color{textcolor}\rmfamily\fontsize{10.000000}{12.000000}\selectfont Neg}%
\end{pgfscope}%
\begin{pgfscope}%
\pgfsetbuttcap%
\pgfsetmiterjoin%
\definecolor{currentfill}{rgb}{0.000000,0.000000,0.000000}%
\pgfsetfillcolor{currentfill}%
\pgfsetlinewidth{0.000000pt}%
\definecolor{currentstroke}{rgb}{0.000000,0.000000,0.000000}%
\pgfsetstrokecolor{currentstroke}%
\pgfsetstrokeopacity{0.000000}%
\pgfsetdash{}{0pt}%
\pgfpathmoveto{\pgfqpoint{3.291914in}{1.236944in}}%
\pgfpathlineto{\pgfqpoint{3.569692in}{1.236944in}}%
\pgfpathlineto{\pgfqpoint{3.569692in}{1.334167in}}%
\pgfpathlineto{\pgfqpoint{3.291914in}{1.334167in}}%
\pgfpathlineto{\pgfqpoint{3.291914in}{1.236944in}}%
\pgfpathclose%
\pgfusepath{fill}%
\end{pgfscope}%
\begin{pgfscope}%
\definecolor{textcolor}{rgb}{0.000000,0.000000,0.000000}%
\pgfsetstrokecolor{textcolor}%
\pgfsetfillcolor{textcolor}%
\pgftext[x=3.680803in,y=1.236944in,left,base]{\color{textcolor}\rmfamily\fontsize{10.000000}{12.000000}\selectfont Pos}%
\end{pgfscope}%
\end{pgfpicture}%
\makeatother%
\endgroup%
	
&
	\vskip 0pt
	\hfil ROC Curve
	
	%% Creator: Matplotlib, PGF backend
%%
%% To include the figure in your LaTeX document, write
%%   \input{<filename>.pgf}
%%
%% Make sure the required packages are loaded in your preamble
%%   \usepackage{pgf}
%%
%% Also ensure that all the required font packages are loaded; for instance,
%% the lmodern package is sometimes necessary when using math font.
%%   \usepackage{lmodern}
%%
%% Figures using additional raster images can only be included by \input if
%% they are in the same directory as the main LaTeX file. For loading figures
%% from other directories you can use the `import` package
%%   \usepackage{import}
%%
%% and then include the figures with
%%   \import{<path to file>}{<filename>.pgf}
%%
%% Matplotlib used the following preamble
%%   
%%   \usepackage{fontspec}
%%   \makeatletter\@ifpackageloaded{underscore}{}{\usepackage[strings]{underscore}}\makeatother
%%
\begingroup%
\makeatletter%
\begin{pgfpicture}%
\pgfpathrectangle{\pgfpointorigin}{\pgfqpoint{2.221861in}{1.754444in}}%
\pgfusepath{use as bounding box, clip}%
\begin{pgfscope}%
\pgfsetbuttcap%
\pgfsetmiterjoin%
\definecolor{currentfill}{rgb}{1.000000,1.000000,1.000000}%
\pgfsetfillcolor{currentfill}%
\pgfsetlinewidth{0.000000pt}%
\definecolor{currentstroke}{rgb}{1.000000,1.000000,1.000000}%
\pgfsetstrokecolor{currentstroke}%
\pgfsetdash{}{0pt}%
\pgfpathmoveto{\pgfqpoint{0.000000in}{0.000000in}}%
\pgfpathlineto{\pgfqpoint{2.221861in}{0.000000in}}%
\pgfpathlineto{\pgfqpoint{2.221861in}{1.754444in}}%
\pgfpathlineto{\pgfqpoint{0.000000in}{1.754444in}}%
\pgfpathlineto{\pgfqpoint{0.000000in}{0.000000in}}%
\pgfpathclose%
\pgfusepath{fill}%
\end{pgfscope}%
\begin{pgfscope}%
\pgfsetbuttcap%
\pgfsetmiterjoin%
\definecolor{currentfill}{rgb}{1.000000,1.000000,1.000000}%
\pgfsetfillcolor{currentfill}%
\pgfsetlinewidth{0.000000pt}%
\definecolor{currentstroke}{rgb}{0.000000,0.000000,0.000000}%
\pgfsetstrokecolor{currentstroke}%
\pgfsetstrokeopacity{0.000000}%
\pgfsetdash{}{0pt}%
\pgfpathmoveto{\pgfqpoint{0.553581in}{0.499444in}}%
\pgfpathlineto{\pgfqpoint{2.103581in}{0.499444in}}%
\pgfpathlineto{\pgfqpoint{2.103581in}{1.654444in}}%
\pgfpathlineto{\pgfqpoint{0.553581in}{1.654444in}}%
\pgfpathlineto{\pgfqpoint{0.553581in}{0.499444in}}%
\pgfpathclose%
\pgfusepath{fill}%
\end{pgfscope}%
\begin{pgfscope}%
\pgfsetbuttcap%
\pgfsetroundjoin%
\definecolor{currentfill}{rgb}{0.000000,0.000000,0.000000}%
\pgfsetfillcolor{currentfill}%
\pgfsetlinewidth{0.803000pt}%
\definecolor{currentstroke}{rgb}{0.000000,0.000000,0.000000}%
\pgfsetstrokecolor{currentstroke}%
\pgfsetdash{}{0pt}%
\pgfsys@defobject{currentmarker}{\pgfqpoint{0.000000in}{-0.048611in}}{\pgfqpoint{0.000000in}{0.000000in}}{%
\pgfpathmoveto{\pgfqpoint{0.000000in}{0.000000in}}%
\pgfpathlineto{\pgfqpoint{0.000000in}{-0.048611in}}%
\pgfusepath{stroke,fill}%
}%
\begin{pgfscope}%
\pgfsys@transformshift{0.624035in}{0.499444in}%
\pgfsys@useobject{currentmarker}{}%
\end{pgfscope}%
\end{pgfscope}%
\begin{pgfscope}%
\definecolor{textcolor}{rgb}{0.000000,0.000000,0.000000}%
\pgfsetstrokecolor{textcolor}%
\pgfsetfillcolor{textcolor}%
\pgftext[x=0.624035in,y=0.402222in,,top]{\color{textcolor}\rmfamily\fontsize{10.000000}{12.000000}\selectfont \(\displaystyle {0.0}\)}%
\end{pgfscope}%
\begin{pgfscope}%
\pgfsetbuttcap%
\pgfsetroundjoin%
\definecolor{currentfill}{rgb}{0.000000,0.000000,0.000000}%
\pgfsetfillcolor{currentfill}%
\pgfsetlinewidth{0.803000pt}%
\definecolor{currentstroke}{rgb}{0.000000,0.000000,0.000000}%
\pgfsetstrokecolor{currentstroke}%
\pgfsetdash{}{0pt}%
\pgfsys@defobject{currentmarker}{\pgfqpoint{0.000000in}{-0.048611in}}{\pgfqpoint{0.000000in}{0.000000in}}{%
\pgfpathmoveto{\pgfqpoint{0.000000in}{0.000000in}}%
\pgfpathlineto{\pgfqpoint{0.000000in}{-0.048611in}}%
\pgfusepath{stroke,fill}%
}%
\begin{pgfscope}%
\pgfsys@transformshift{1.328581in}{0.499444in}%
\pgfsys@useobject{currentmarker}{}%
\end{pgfscope}%
\end{pgfscope}%
\begin{pgfscope}%
\definecolor{textcolor}{rgb}{0.000000,0.000000,0.000000}%
\pgfsetstrokecolor{textcolor}%
\pgfsetfillcolor{textcolor}%
\pgftext[x=1.328581in,y=0.402222in,,top]{\color{textcolor}\rmfamily\fontsize{10.000000}{12.000000}\selectfont \(\displaystyle {0.5}\)}%
\end{pgfscope}%
\begin{pgfscope}%
\pgfsetbuttcap%
\pgfsetroundjoin%
\definecolor{currentfill}{rgb}{0.000000,0.000000,0.000000}%
\pgfsetfillcolor{currentfill}%
\pgfsetlinewidth{0.803000pt}%
\definecolor{currentstroke}{rgb}{0.000000,0.000000,0.000000}%
\pgfsetstrokecolor{currentstroke}%
\pgfsetdash{}{0pt}%
\pgfsys@defobject{currentmarker}{\pgfqpoint{0.000000in}{-0.048611in}}{\pgfqpoint{0.000000in}{0.000000in}}{%
\pgfpathmoveto{\pgfqpoint{0.000000in}{0.000000in}}%
\pgfpathlineto{\pgfqpoint{0.000000in}{-0.048611in}}%
\pgfusepath{stroke,fill}%
}%
\begin{pgfscope}%
\pgfsys@transformshift{2.033126in}{0.499444in}%
\pgfsys@useobject{currentmarker}{}%
\end{pgfscope}%
\end{pgfscope}%
\begin{pgfscope}%
\definecolor{textcolor}{rgb}{0.000000,0.000000,0.000000}%
\pgfsetstrokecolor{textcolor}%
\pgfsetfillcolor{textcolor}%
\pgftext[x=2.033126in,y=0.402222in,,top]{\color{textcolor}\rmfamily\fontsize{10.000000}{12.000000}\selectfont \(\displaystyle {1.0}\)}%
\end{pgfscope}%
\begin{pgfscope}%
\definecolor{textcolor}{rgb}{0.000000,0.000000,0.000000}%
\pgfsetstrokecolor{textcolor}%
\pgfsetfillcolor{textcolor}%
\pgftext[x=1.328581in,y=0.223333in,,top]{\color{textcolor}\rmfamily\fontsize{10.000000}{12.000000}\selectfont False positive rate}%
\end{pgfscope}%
\begin{pgfscope}%
\pgfsetbuttcap%
\pgfsetroundjoin%
\definecolor{currentfill}{rgb}{0.000000,0.000000,0.000000}%
\pgfsetfillcolor{currentfill}%
\pgfsetlinewidth{0.803000pt}%
\definecolor{currentstroke}{rgb}{0.000000,0.000000,0.000000}%
\pgfsetstrokecolor{currentstroke}%
\pgfsetdash{}{0pt}%
\pgfsys@defobject{currentmarker}{\pgfqpoint{-0.048611in}{0.000000in}}{\pgfqpoint{-0.000000in}{0.000000in}}{%
\pgfpathmoveto{\pgfqpoint{-0.000000in}{0.000000in}}%
\pgfpathlineto{\pgfqpoint{-0.048611in}{0.000000in}}%
\pgfusepath{stroke,fill}%
}%
\begin{pgfscope}%
\pgfsys@transformshift{0.553581in}{0.551944in}%
\pgfsys@useobject{currentmarker}{}%
\end{pgfscope}%
\end{pgfscope}%
\begin{pgfscope}%
\definecolor{textcolor}{rgb}{0.000000,0.000000,0.000000}%
\pgfsetstrokecolor{textcolor}%
\pgfsetfillcolor{textcolor}%
\pgftext[x=0.278889in, y=0.503750in, left, base]{\color{textcolor}\rmfamily\fontsize{10.000000}{12.000000}\selectfont \(\displaystyle {0.0}\)}%
\end{pgfscope}%
\begin{pgfscope}%
\pgfsetbuttcap%
\pgfsetroundjoin%
\definecolor{currentfill}{rgb}{0.000000,0.000000,0.000000}%
\pgfsetfillcolor{currentfill}%
\pgfsetlinewidth{0.803000pt}%
\definecolor{currentstroke}{rgb}{0.000000,0.000000,0.000000}%
\pgfsetstrokecolor{currentstroke}%
\pgfsetdash{}{0pt}%
\pgfsys@defobject{currentmarker}{\pgfqpoint{-0.048611in}{0.000000in}}{\pgfqpoint{-0.000000in}{0.000000in}}{%
\pgfpathmoveto{\pgfqpoint{-0.000000in}{0.000000in}}%
\pgfpathlineto{\pgfqpoint{-0.048611in}{0.000000in}}%
\pgfusepath{stroke,fill}%
}%
\begin{pgfscope}%
\pgfsys@transformshift{0.553581in}{1.076944in}%
\pgfsys@useobject{currentmarker}{}%
\end{pgfscope}%
\end{pgfscope}%
\begin{pgfscope}%
\definecolor{textcolor}{rgb}{0.000000,0.000000,0.000000}%
\pgfsetstrokecolor{textcolor}%
\pgfsetfillcolor{textcolor}%
\pgftext[x=0.278889in, y=1.028750in, left, base]{\color{textcolor}\rmfamily\fontsize{10.000000}{12.000000}\selectfont \(\displaystyle {0.5}\)}%
\end{pgfscope}%
\begin{pgfscope}%
\pgfsetbuttcap%
\pgfsetroundjoin%
\definecolor{currentfill}{rgb}{0.000000,0.000000,0.000000}%
\pgfsetfillcolor{currentfill}%
\pgfsetlinewidth{0.803000pt}%
\definecolor{currentstroke}{rgb}{0.000000,0.000000,0.000000}%
\pgfsetstrokecolor{currentstroke}%
\pgfsetdash{}{0pt}%
\pgfsys@defobject{currentmarker}{\pgfqpoint{-0.048611in}{0.000000in}}{\pgfqpoint{-0.000000in}{0.000000in}}{%
\pgfpathmoveto{\pgfqpoint{-0.000000in}{0.000000in}}%
\pgfpathlineto{\pgfqpoint{-0.048611in}{0.000000in}}%
\pgfusepath{stroke,fill}%
}%
\begin{pgfscope}%
\pgfsys@transformshift{0.553581in}{1.601944in}%
\pgfsys@useobject{currentmarker}{}%
\end{pgfscope}%
\end{pgfscope}%
\begin{pgfscope}%
\definecolor{textcolor}{rgb}{0.000000,0.000000,0.000000}%
\pgfsetstrokecolor{textcolor}%
\pgfsetfillcolor{textcolor}%
\pgftext[x=0.278889in, y=1.553750in, left, base]{\color{textcolor}\rmfamily\fontsize{10.000000}{12.000000}\selectfont \(\displaystyle {1.0}\)}%
\end{pgfscope}%
\begin{pgfscope}%
\definecolor{textcolor}{rgb}{0.000000,0.000000,0.000000}%
\pgfsetstrokecolor{textcolor}%
\pgfsetfillcolor{textcolor}%
\pgftext[x=0.223333in,y=1.076944in,,bottom,rotate=90.000000]{\color{textcolor}\rmfamily\fontsize{10.000000}{12.000000}\selectfont True positive rate}%
\end{pgfscope}%
\begin{pgfscope}%
\pgfpathrectangle{\pgfqpoint{0.553581in}{0.499444in}}{\pgfqpoint{1.550000in}{1.155000in}}%
\pgfusepath{clip}%
\pgfsetbuttcap%
\pgfsetroundjoin%
\pgfsetlinewidth{1.505625pt}%
\definecolor{currentstroke}{rgb}{0.000000,0.000000,0.000000}%
\pgfsetstrokecolor{currentstroke}%
\pgfsetdash{{5.550000pt}{2.400000pt}}{0.000000pt}%
\pgfpathmoveto{\pgfqpoint{0.624035in}{0.551944in}}%
\pgfpathlineto{\pgfqpoint{2.033126in}{1.601944in}}%
\pgfusepath{stroke}%
\end{pgfscope}%
\begin{pgfscope}%
\pgfpathrectangle{\pgfqpoint{0.553581in}{0.499444in}}{\pgfqpoint{1.550000in}{1.155000in}}%
\pgfusepath{clip}%
\pgfsetrectcap%
\pgfsetroundjoin%
\pgfsetlinewidth{1.505625pt}%
\definecolor{currentstroke}{rgb}{0.000000,0.000000,0.000000}%
\pgfsetstrokecolor{currentstroke}%
\pgfsetdash{}{0pt}%
\pgfpathmoveto{\pgfqpoint{0.624035in}{0.551944in}}%
\pgfpathlineto{\pgfqpoint{0.625846in}{0.579044in}}%
\pgfpathlineto{\pgfqpoint{0.631120in}{0.636381in}}%
\pgfpathlineto{\pgfqpoint{0.638217in}{0.692312in}}%
\pgfpathlineto{\pgfqpoint{0.640586in}{0.707306in}}%
\pgfpathlineto{\pgfqpoint{0.643382in}{0.723873in}}%
\pgfpathlineto{\pgfqpoint{0.652793in}{0.771208in}}%
\pgfpathlineto{\pgfqpoint{0.664250in}{0.817594in}}%
\pgfpathlineto{\pgfqpoint{0.664532in}{0.818665in}}%
\pgfpathlineto{\pgfqpoint{0.664532in}{0.818693in}}%
\pgfpathlineto{\pgfqpoint{0.664618in}{0.818907in}}%
\pgfpathlineto{\pgfqpoint{0.674394in}{0.853271in}}%
\pgfpathlineto{\pgfqpoint{0.692246in}{0.905095in}}%
\pgfpathlineto{\pgfqpoint{0.706890in}{0.941190in}}%
\pgfpathlineto{\pgfqpoint{0.723825in}{0.977612in}}%
\pgfpathlineto{\pgfqpoint{0.754052in}{1.034214in}}%
\pgfpathlineto{\pgfqpoint{0.777420in}{1.072684in}}%
\pgfpathlineto{\pgfqpoint{0.803802in}{1.110288in}}%
\pgfpathlineto{\pgfqpoint{0.848438in}{1.167336in}}%
\pgfpathlineto{\pgfqpoint{0.881530in}{1.203572in}}%
\pgfpathlineto{\pgfqpoint{0.916283in}{1.238401in}}%
\pgfpathlineto{\pgfqpoint{0.937147in}{1.256756in}}%
\pgfpathlineto{\pgfqpoint{0.977427in}{1.291147in}}%
\pgfpathlineto{\pgfqpoint{1.020682in}{1.322856in}}%
\pgfpathlineto{\pgfqpoint{1.042870in}{1.337998in}}%
\pgfpathlineto{\pgfqpoint{1.089190in}{1.367994in}}%
\pgfpathlineto{\pgfqpoint{1.138296in}{1.395522in}}%
\pgfpathlineto{\pgfqpoint{1.189259in}{1.421336in}}%
\pgfpathlineto{\pgfqpoint{1.241799in}{1.444636in}}%
\pgfpathlineto{\pgfqpoint{1.295668in}{1.465599in}}%
\pgfpathlineto{\pgfqpoint{1.350250in}{1.485630in}}%
\pgfpathlineto{\pgfqpoint{1.405484in}{1.503911in}}%
\pgfpathlineto{\pgfqpoint{1.460721in}{1.519388in}}%
\pgfpathlineto{\pgfqpoint{1.461316in}{1.519491in}}%
\pgfpathlineto{\pgfqpoint{1.543674in}{1.540072in}}%
\pgfpathlineto{\pgfqpoint{1.624087in}{1.556918in}}%
\pgfpathlineto{\pgfqpoint{1.699571in}{1.570077in}}%
\pgfpathlineto{\pgfqpoint{1.701093in}{1.570198in}}%
\pgfpathlineto{\pgfqpoint{1.749465in}{1.577564in}}%
\pgfpathlineto{\pgfqpoint{1.858917in}{1.590294in}}%
\pgfpathlineto{\pgfqpoint{1.946936in}{1.597567in}}%
\pgfpathlineto{\pgfqpoint{2.033126in}{1.601944in}}%
\pgfpathlineto{\pgfqpoint{2.033126in}{1.601944in}}%
\pgfusepath{stroke}%
\end{pgfscope}%
\begin{pgfscope}%
\pgfsetrectcap%
\pgfsetmiterjoin%
\pgfsetlinewidth{0.803000pt}%
\definecolor{currentstroke}{rgb}{0.000000,0.000000,0.000000}%
\pgfsetstrokecolor{currentstroke}%
\pgfsetdash{}{0pt}%
\pgfpathmoveto{\pgfqpoint{0.553581in}{0.499444in}}%
\pgfpathlineto{\pgfqpoint{0.553581in}{1.654444in}}%
\pgfusepath{stroke}%
\end{pgfscope}%
\begin{pgfscope}%
\pgfsetrectcap%
\pgfsetmiterjoin%
\pgfsetlinewidth{0.803000pt}%
\definecolor{currentstroke}{rgb}{0.000000,0.000000,0.000000}%
\pgfsetstrokecolor{currentstroke}%
\pgfsetdash{}{0pt}%
\pgfpathmoveto{\pgfqpoint{2.103581in}{0.499444in}}%
\pgfpathlineto{\pgfqpoint{2.103581in}{1.654444in}}%
\pgfusepath{stroke}%
\end{pgfscope}%
\begin{pgfscope}%
\pgfsetrectcap%
\pgfsetmiterjoin%
\pgfsetlinewidth{0.803000pt}%
\definecolor{currentstroke}{rgb}{0.000000,0.000000,0.000000}%
\pgfsetstrokecolor{currentstroke}%
\pgfsetdash{}{0pt}%
\pgfpathmoveto{\pgfqpoint{0.553581in}{0.499444in}}%
\pgfpathlineto{\pgfqpoint{2.103581in}{0.499444in}}%
\pgfusepath{stroke}%
\end{pgfscope}%
\begin{pgfscope}%
\pgfsetrectcap%
\pgfsetmiterjoin%
\pgfsetlinewidth{0.803000pt}%
\definecolor{currentstroke}{rgb}{0.000000,0.000000,0.000000}%
\pgfsetstrokecolor{currentstroke}%
\pgfsetdash{}{0pt}%
\pgfpathmoveto{\pgfqpoint{0.553581in}{1.654444in}}%
\pgfpathlineto{\pgfqpoint{2.103581in}{1.654444in}}%
\pgfusepath{stroke}%
\end{pgfscope}%
\begin{pgfscope}%
\pgfsetbuttcap%
\pgfsetmiterjoin%
\definecolor{currentfill}{rgb}{1.000000,1.000000,1.000000}%
\pgfsetfillcolor{currentfill}%
\pgfsetfillopacity{0.800000}%
\pgfsetlinewidth{1.003750pt}%
\definecolor{currentstroke}{rgb}{0.800000,0.800000,0.800000}%
\pgfsetstrokecolor{currentstroke}%
\pgfsetstrokeopacity{0.800000}%
\pgfsetdash{}{0pt}%
\pgfpathmoveto{\pgfqpoint{0.832747in}{0.568889in}}%
\pgfpathlineto{\pgfqpoint{2.006358in}{0.568889in}}%
\pgfpathquadraticcurveto{\pgfqpoint{2.034136in}{0.568889in}}{\pgfqpoint{2.034136in}{0.596666in}}%
\pgfpathlineto{\pgfqpoint{2.034136in}{0.776388in}}%
\pgfpathquadraticcurveto{\pgfqpoint{2.034136in}{0.804166in}}{\pgfqpoint{2.006358in}{0.804166in}}%
\pgfpathlineto{\pgfqpoint{0.832747in}{0.804166in}}%
\pgfpathquadraticcurveto{\pgfqpoint{0.804970in}{0.804166in}}{\pgfqpoint{0.804970in}{0.776388in}}%
\pgfpathlineto{\pgfqpoint{0.804970in}{0.596666in}}%
\pgfpathquadraticcurveto{\pgfqpoint{0.804970in}{0.568889in}}{\pgfqpoint{0.832747in}{0.568889in}}%
\pgfpathlineto{\pgfqpoint{0.832747in}{0.568889in}}%
\pgfpathclose%
\pgfusepath{stroke,fill}%
\end{pgfscope}%
\begin{pgfscope}%
\pgfsetrectcap%
\pgfsetroundjoin%
\pgfsetlinewidth{1.505625pt}%
\definecolor{currentstroke}{rgb}{0.000000,0.000000,0.000000}%
\pgfsetstrokecolor{currentstroke}%
\pgfsetdash{}{0pt}%
\pgfpathmoveto{\pgfqpoint{0.860525in}{0.700000in}}%
\pgfpathlineto{\pgfqpoint{0.999414in}{0.700000in}}%
\pgfpathlineto{\pgfqpoint{1.138303in}{0.700000in}}%
\pgfusepath{stroke}%
\end{pgfscope}%
\begin{pgfscope}%
\definecolor{textcolor}{rgb}{0.000000,0.000000,0.000000}%
\pgfsetstrokecolor{textcolor}%
\pgfsetfillcolor{textcolor}%
\pgftext[x=1.249414in,y=0.651388in,left,base]{\color{textcolor}\rmfamily\fontsize{10.000000}{12.000000}\selectfont AUC=0.802}%
\end{pgfscope}%
\end{pgfpicture}%
\makeatother%
\endgroup%

\end{tabular}

\

\

%
\verb|AdaBoost_5_Fold_Hard_Test|

\

In this model the values are clustered very tightly, but in that small range the 214,070 samples return 210,442 different values of $p$, so there is much diversity that we can't see in this representation.  


\

\verb|AdaBoost_5_Fold_Hard_Test|



\noindent\begin{tabular}{@{\hspace{-6pt}}p{4.3in} @{\hspace{-6pt}}p{2.0in}}
	\vskip 0pt
	\hfil Raw Model Output
	
	%% Creator: Matplotlib, PGF backend
%%
%% To include the figure in your LaTeX document, write
%%   \input{<filename>.pgf}
%%
%% Make sure the required packages are loaded in your preamble
%%   \usepackage{pgf}
%%
%% Also ensure that all the required font packages are loaded; for instance,
%% the lmodern package is sometimes necessary when using math font.
%%   \usepackage{lmodern}
%%
%% Figures using additional raster images can only be included by \input if
%% they are in the same directory as the main LaTeX file. For loading figures
%% from other directories you can use the `import` package
%%   \usepackage{import}
%%
%% and then include the figures with
%%   \import{<path to file>}{<filename>.pgf}
%%
%% Matplotlib used the following preamble
%%   
%%   \usepackage{fontspec}
%%   \makeatletter\@ifpackageloaded{underscore}{}{\usepackage[strings]{underscore}}\makeatother
%%
\begingroup%
\makeatletter%
\begin{pgfpicture}%
\pgfpathrectangle{\pgfpointorigin}{\pgfqpoint{4.102500in}{1.754444in}}%
\pgfusepath{use as bounding box, clip}%
\begin{pgfscope}%
\pgfsetbuttcap%
\pgfsetmiterjoin%
\definecolor{currentfill}{rgb}{1.000000,1.000000,1.000000}%
\pgfsetfillcolor{currentfill}%
\pgfsetlinewidth{0.000000pt}%
\definecolor{currentstroke}{rgb}{1.000000,1.000000,1.000000}%
\pgfsetstrokecolor{currentstroke}%
\pgfsetdash{}{0pt}%
\pgfpathmoveto{\pgfqpoint{0.000000in}{0.000000in}}%
\pgfpathlineto{\pgfqpoint{4.102500in}{0.000000in}}%
\pgfpathlineto{\pgfqpoint{4.102500in}{1.754444in}}%
\pgfpathlineto{\pgfqpoint{0.000000in}{1.754444in}}%
\pgfpathlineto{\pgfqpoint{0.000000in}{0.000000in}}%
\pgfpathclose%
\pgfusepath{fill}%
\end{pgfscope}%
\begin{pgfscope}%
\pgfsetbuttcap%
\pgfsetmiterjoin%
\definecolor{currentfill}{rgb}{1.000000,1.000000,1.000000}%
\pgfsetfillcolor{currentfill}%
\pgfsetlinewidth{0.000000pt}%
\definecolor{currentstroke}{rgb}{0.000000,0.000000,0.000000}%
\pgfsetstrokecolor{currentstroke}%
\pgfsetstrokeopacity{0.000000}%
\pgfsetdash{}{0pt}%
\pgfpathmoveto{\pgfqpoint{0.515000in}{0.499444in}}%
\pgfpathlineto{\pgfqpoint{4.002500in}{0.499444in}}%
\pgfpathlineto{\pgfqpoint{4.002500in}{1.654444in}}%
\pgfpathlineto{\pgfqpoint{0.515000in}{1.654444in}}%
\pgfpathlineto{\pgfqpoint{0.515000in}{0.499444in}}%
\pgfpathclose%
\pgfusepath{fill}%
\end{pgfscope}%
\begin{pgfscope}%
\pgfpathrectangle{\pgfqpoint{0.515000in}{0.499444in}}{\pgfqpoint{3.487500in}{1.155000in}}%
\pgfusepath{clip}%
\pgfsetbuttcap%
\pgfsetmiterjoin%
\pgfsetlinewidth{1.003750pt}%
\definecolor{currentstroke}{rgb}{0.000000,0.000000,0.000000}%
\pgfsetstrokecolor{currentstroke}%
\pgfsetdash{}{0pt}%
\pgfpathmoveto{\pgfqpoint{0.610114in}{0.499444in}}%
\pgfpathlineto{\pgfqpoint{0.673523in}{0.499444in}}%
\pgfpathlineto{\pgfqpoint{0.673523in}{0.499444in}}%
\pgfpathlineto{\pgfqpoint{0.610114in}{0.499444in}}%
\pgfpathlineto{\pgfqpoint{0.610114in}{0.499444in}}%
\pgfpathclose%
\pgfusepath{stroke}%
\end{pgfscope}%
\begin{pgfscope}%
\pgfpathrectangle{\pgfqpoint{0.515000in}{0.499444in}}{\pgfqpoint{3.487500in}{1.155000in}}%
\pgfusepath{clip}%
\pgfsetbuttcap%
\pgfsetmiterjoin%
\pgfsetlinewidth{1.003750pt}%
\definecolor{currentstroke}{rgb}{0.000000,0.000000,0.000000}%
\pgfsetstrokecolor{currentstroke}%
\pgfsetdash{}{0pt}%
\pgfpathmoveto{\pgfqpoint{0.768637in}{0.499444in}}%
\pgfpathlineto{\pgfqpoint{0.832046in}{0.499444in}}%
\pgfpathlineto{\pgfqpoint{0.832046in}{0.499444in}}%
\pgfpathlineto{\pgfqpoint{0.768637in}{0.499444in}}%
\pgfpathlineto{\pgfqpoint{0.768637in}{0.499444in}}%
\pgfpathclose%
\pgfusepath{stroke}%
\end{pgfscope}%
\begin{pgfscope}%
\pgfpathrectangle{\pgfqpoint{0.515000in}{0.499444in}}{\pgfqpoint{3.487500in}{1.155000in}}%
\pgfusepath{clip}%
\pgfsetbuttcap%
\pgfsetmiterjoin%
\pgfsetlinewidth{1.003750pt}%
\definecolor{currentstroke}{rgb}{0.000000,0.000000,0.000000}%
\pgfsetstrokecolor{currentstroke}%
\pgfsetdash{}{0pt}%
\pgfpathmoveto{\pgfqpoint{0.927159in}{0.499444in}}%
\pgfpathlineto{\pgfqpoint{0.990568in}{0.499444in}}%
\pgfpathlineto{\pgfqpoint{0.990568in}{0.499444in}}%
\pgfpathlineto{\pgfqpoint{0.927159in}{0.499444in}}%
\pgfpathlineto{\pgfqpoint{0.927159in}{0.499444in}}%
\pgfpathclose%
\pgfusepath{stroke}%
\end{pgfscope}%
\begin{pgfscope}%
\pgfpathrectangle{\pgfqpoint{0.515000in}{0.499444in}}{\pgfqpoint{3.487500in}{1.155000in}}%
\pgfusepath{clip}%
\pgfsetbuttcap%
\pgfsetmiterjoin%
\pgfsetlinewidth{1.003750pt}%
\definecolor{currentstroke}{rgb}{0.000000,0.000000,0.000000}%
\pgfsetstrokecolor{currentstroke}%
\pgfsetdash{}{0pt}%
\pgfpathmoveto{\pgfqpoint{1.085682in}{0.499444in}}%
\pgfpathlineto{\pgfqpoint{1.149091in}{0.499444in}}%
\pgfpathlineto{\pgfqpoint{1.149091in}{0.499444in}}%
\pgfpathlineto{\pgfqpoint{1.085682in}{0.499444in}}%
\pgfpathlineto{\pgfqpoint{1.085682in}{0.499444in}}%
\pgfpathclose%
\pgfusepath{stroke}%
\end{pgfscope}%
\begin{pgfscope}%
\pgfpathrectangle{\pgfqpoint{0.515000in}{0.499444in}}{\pgfqpoint{3.487500in}{1.155000in}}%
\pgfusepath{clip}%
\pgfsetbuttcap%
\pgfsetmiterjoin%
\pgfsetlinewidth{1.003750pt}%
\definecolor{currentstroke}{rgb}{0.000000,0.000000,0.000000}%
\pgfsetstrokecolor{currentstroke}%
\pgfsetdash{}{0pt}%
\pgfpathmoveto{\pgfqpoint{1.244205in}{0.499444in}}%
\pgfpathlineto{\pgfqpoint{1.307614in}{0.499444in}}%
\pgfpathlineto{\pgfqpoint{1.307614in}{0.499444in}}%
\pgfpathlineto{\pgfqpoint{1.244205in}{0.499444in}}%
\pgfpathlineto{\pgfqpoint{1.244205in}{0.499444in}}%
\pgfpathclose%
\pgfusepath{stroke}%
\end{pgfscope}%
\begin{pgfscope}%
\pgfpathrectangle{\pgfqpoint{0.515000in}{0.499444in}}{\pgfqpoint{3.487500in}{1.155000in}}%
\pgfusepath{clip}%
\pgfsetbuttcap%
\pgfsetmiterjoin%
\pgfsetlinewidth{1.003750pt}%
\definecolor{currentstroke}{rgb}{0.000000,0.000000,0.000000}%
\pgfsetstrokecolor{currentstroke}%
\pgfsetdash{}{0pt}%
\pgfpathmoveto{\pgfqpoint{1.402728in}{0.499444in}}%
\pgfpathlineto{\pgfqpoint{1.466137in}{0.499444in}}%
\pgfpathlineto{\pgfqpoint{1.466137in}{0.499444in}}%
\pgfpathlineto{\pgfqpoint{1.402728in}{0.499444in}}%
\pgfpathlineto{\pgfqpoint{1.402728in}{0.499444in}}%
\pgfpathclose%
\pgfusepath{stroke}%
\end{pgfscope}%
\begin{pgfscope}%
\pgfpathrectangle{\pgfqpoint{0.515000in}{0.499444in}}{\pgfqpoint{3.487500in}{1.155000in}}%
\pgfusepath{clip}%
\pgfsetbuttcap%
\pgfsetmiterjoin%
\pgfsetlinewidth{1.003750pt}%
\definecolor{currentstroke}{rgb}{0.000000,0.000000,0.000000}%
\pgfsetstrokecolor{currentstroke}%
\pgfsetdash{}{0pt}%
\pgfpathmoveto{\pgfqpoint{1.561250in}{0.499444in}}%
\pgfpathlineto{\pgfqpoint{1.624659in}{0.499444in}}%
\pgfpathlineto{\pgfqpoint{1.624659in}{0.499444in}}%
\pgfpathlineto{\pgfqpoint{1.561250in}{0.499444in}}%
\pgfpathlineto{\pgfqpoint{1.561250in}{0.499444in}}%
\pgfpathclose%
\pgfusepath{stroke}%
\end{pgfscope}%
\begin{pgfscope}%
\pgfpathrectangle{\pgfqpoint{0.515000in}{0.499444in}}{\pgfqpoint{3.487500in}{1.155000in}}%
\pgfusepath{clip}%
\pgfsetbuttcap%
\pgfsetmiterjoin%
\pgfsetlinewidth{1.003750pt}%
\definecolor{currentstroke}{rgb}{0.000000,0.000000,0.000000}%
\pgfsetstrokecolor{currentstroke}%
\pgfsetdash{}{0pt}%
\pgfpathmoveto{\pgfqpoint{1.719773in}{0.499444in}}%
\pgfpathlineto{\pgfqpoint{1.783182in}{0.499444in}}%
\pgfpathlineto{\pgfqpoint{1.783182in}{0.499444in}}%
\pgfpathlineto{\pgfqpoint{1.719773in}{0.499444in}}%
\pgfpathlineto{\pgfqpoint{1.719773in}{0.499444in}}%
\pgfpathclose%
\pgfusepath{stroke}%
\end{pgfscope}%
\begin{pgfscope}%
\pgfpathrectangle{\pgfqpoint{0.515000in}{0.499444in}}{\pgfqpoint{3.487500in}{1.155000in}}%
\pgfusepath{clip}%
\pgfsetbuttcap%
\pgfsetmiterjoin%
\pgfsetlinewidth{1.003750pt}%
\definecolor{currentstroke}{rgb}{0.000000,0.000000,0.000000}%
\pgfsetstrokecolor{currentstroke}%
\pgfsetdash{}{0pt}%
\pgfpathmoveto{\pgfqpoint{1.878296in}{0.499444in}}%
\pgfpathlineto{\pgfqpoint{1.941705in}{0.499444in}}%
\pgfpathlineto{\pgfqpoint{1.941705in}{0.499444in}}%
\pgfpathlineto{\pgfqpoint{1.878296in}{0.499444in}}%
\pgfpathlineto{\pgfqpoint{1.878296in}{0.499444in}}%
\pgfpathclose%
\pgfusepath{stroke}%
\end{pgfscope}%
\begin{pgfscope}%
\pgfpathrectangle{\pgfqpoint{0.515000in}{0.499444in}}{\pgfqpoint{3.487500in}{1.155000in}}%
\pgfusepath{clip}%
\pgfsetbuttcap%
\pgfsetmiterjoin%
\pgfsetlinewidth{1.003750pt}%
\definecolor{currentstroke}{rgb}{0.000000,0.000000,0.000000}%
\pgfsetstrokecolor{currentstroke}%
\pgfsetdash{}{0pt}%
\pgfpathmoveto{\pgfqpoint{2.036818in}{0.499444in}}%
\pgfpathlineto{\pgfqpoint{2.100228in}{0.499444in}}%
\pgfpathlineto{\pgfqpoint{2.100228in}{0.499444in}}%
\pgfpathlineto{\pgfqpoint{2.036818in}{0.499444in}}%
\pgfpathlineto{\pgfqpoint{2.036818in}{0.499444in}}%
\pgfpathclose%
\pgfusepath{stroke}%
\end{pgfscope}%
\begin{pgfscope}%
\pgfpathrectangle{\pgfqpoint{0.515000in}{0.499444in}}{\pgfqpoint{3.487500in}{1.155000in}}%
\pgfusepath{clip}%
\pgfsetbuttcap%
\pgfsetmiterjoin%
\pgfsetlinewidth{1.003750pt}%
\definecolor{currentstroke}{rgb}{0.000000,0.000000,0.000000}%
\pgfsetstrokecolor{currentstroke}%
\pgfsetdash{}{0pt}%
\pgfpathmoveto{\pgfqpoint{2.195341in}{0.499444in}}%
\pgfpathlineto{\pgfqpoint{2.258750in}{0.499444in}}%
\pgfpathlineto{\pgfqpoint{2.258750in}{1.599444in}}%
\pgfpathlineto{\pgfqpoint{2.195341in}{1.599444in}}%
\pgfpathlineto{\pgfqpoint{2.195341in}{0.499444in}}%
\pgfpathclose%
\pgfusepath{stroke}%
\end{pgfscope}%
\begin{pgfscope}%
\pgfpathrectangle{\pgfqpoint{0.515000in}{0.499444in}}{\pgfqpoint{3.487500in}{1.155000in}}%
\pgfusepath{clip}%
\pgfsetbuttcap%
\pgfsetmiterjoin%
\pgfsetlinewidth{1.003750pt}%
\definecolor{currentstroke}{rgb}{0.000000,0.000000,0.000000}%
\pgfsetstrokecolor{currentstroke}%
\pgfsetdash{}{0pt}%
\pgfpathmoveto{\pgfqpoint{2.353864in}{0.499444in}}%
\pgfpathlineto{\pgfqpoint{2.417273in}{0.499444in}}%
\pgfpathlineto{\pgfqpoint{2.417273in}{0.518544in}}%
\pgfpathlineto{\pgfqpoint{2.353864in}{0.518544in}}%
\pgfpathlineto{\pgfqpoint{2.353864in}{0.499444in}}%
\pgfpathclose%
\pgfusepath{stroke}%
\end{pgfscope}%
\begin{pgfscope}%
\pgfpathrectangle{\pgfqpoint{0.515000in}{0.499444in}}{\pgfqpoint{3.487500in}{1.155000in}}%
\pgfusepath{clip}%
\pgfsetbuttcap%
\pgfsetmiterjoin%
\pgfsetlinewidth{1.003750pt}%
\definecolor{currentstroke}{rgb}{0.000000,0.000000,0.000000}%
\pgfsetstrokecolor{currentstroke}%
\pgfsetdash{}{0pt}%
\pgfpathmoveto{\pgfqpoint{2.512387in}{0.499444in}}%
\pgfpathlineto{\pgfqpoint{2.575796in}{0.499444in}}%
\pgfpathlineto{\pgfqpoint{2.575796in}{0.499444in}}%
\pgfpathlineto{\pgfqpoint{2.512387in}{0.499444in}}%
\pgfpathlineto{\pgfqpoint{2.512387in}{0.499444in}}%
\pgfpathclose%
\pgfusepath{stroke}%
\end{pgfscope}%
\begin{pgfscope}%
\pgfpathrectangle{\pgfqpoint{0.515000in}{0.499444in}}{\pgfqpoint{3.487500in}{1.155000in}}%
\pgfusepath{clip}%
\pgfsetbuttcap%
\pgfsetmiterjoin%
\pgfsetlinewidth{1.003750pt}%
\definecolor{currentstroke}{rgb}{0.000000,0.000000,0.000000}%
\pgfsetstrokecolor{currentstroke}%
\pgfsetdash{}{0pt}%
\pgfpathmoveto{\pgfqpoint{2.670909in}{0.499444in}}%
\pgfpathlineto{\pgfqpoint{2.734318in}{0.499444in}}%
\pgfpathlineto{\pgfqpoint{2.734318in}{0.499444in}}%
\pgfpathlineto{\pgfqpoint{2.670909in}{0.499444in}}%
\pgfpathlineto{\pgfqpoint{2.670909in}{0.499444in}}%
\pgfpathclose%
\pgfusepath{stroke}%
\end{pgfscope}%
\begin{pgfscope}%
\pgfpathrectangle{\pgfqpoint{0.515000in}{0.499444in}}{\pgfqpoint{3.487500in}{1.155000in}}%
\pgfusepath{clip}%
\pgfsetbuttcap%
\pgfsetmiterjoin%
\pgfsetlinewidth{1.003750pt}%
\definecolor{currentstroke}{rgb}{0.000000,0.000000,0.000000}%
\pgfsetstrokecolor{currentstroke}%
\pgfsetdash{}{0pt}%
\pgfpathmoveto{\pgfqpoint{2.829432in}{0.499444in}}%
\pgfpathlineto{\pgfqpoint{2.892841in}{0.499444in}}%
\pgfpathlineto{\pgfqpoint{2.892841in}{0.499444in}}%
\pgfpathlineto{\pgfqpoint{2.829432in}{0.499444in}}%
\pgfpathlineto{\pgfqpoint{2.829432in}{0.499444in}}%
\pgfpathclose%
\pgfusepath{stroke}%
\end{pgfscope}%
\begin{pgfscope}%
\pgfpathrectangle{\pgfqpoint{0.515000in}{0.499444in}}{\pgfqpoint{3.487500in}{1.155000in}}%
\pgfusepath{clip}%
\pgfsetbuttcap%
\pgfsetmiterjoin%
\pgfsetlinewidth{1.003750pt}%
\definecolor{currentstroke}{rgb}{0.000000,0.000000,0.000000}%
\pgfsetstrokecolor{currentstroke}%
\pgfsetdash{}{0pt}%
\pgfpathmoveto{\pgfqpoint{2.987955in}{0.499444in}}%
\pgfpathlineto{\pgfqpoint{3.051364in}{0.499444in}}%
\pgfpathlineto{\pgfqpoint{3.051364in}{0.499444in}}%
\pgfpathlineto{\pgfqpoint{2.987955in}{0.499444in}}%
\pgfpathlineto{\pgfqpoint{2.987955in}{0.499444in}}%
\pgfpathclose%
\pgfusepath{stroke}%
\end{pgfscope}%
\begin{pgfscope}%
\pgfpathrectangle{\pgfqpoint{0.515000in}{0.499444in}}{\pgfqpoint{3.487500in}{1.155000in}}%
\pgfusepath{clip}%
\pgfsetbuttcap%
\pgfsetmiterjoin%
\pgfsetlinewidth{1.003750pt}%
\definecolor{currentstroke}{rgb}{0.000000,0.000000,0.000000}%
\pgfsetstrokecolor{currentstroke}%
\pgfsetdash{}{0pt}%
\pgfpathmoveto{\pgfqpoint{3.146478in}{0.499444in}}%
\pgfpathlineto{\pgfqpoint{3.209887in}{0.499444in}}%
\pgfpathlineto{\pgfqpoint{3.209887in}{0.499444in}}%
\pgfpathlineto{\pgfqpoint{3.146478in}{0.499444in}}%
\pgfpathlineto{\pgfqpoint{3.146478in}{0.499444in}}%
\pgfpathclose%
\pgfusepath{stroke}%
\end{pgfscope}%
\begin{pgfscope}%
\pgfpathrectangle{\pgfqpoint{0.515000in}{0.499444in}}{\pgfqpoint{3.487500in}{1.155000in}}%
\pgfusepath{clip}%
\pgfsetbuttcap%
\pgfsetmiterjoin%
\pgfsetlinewidth{1.003750pt}%
\definecolor{currentstroke}{rgb}{0.000000,0.000000,0.000000}%
\pgfsetstrokecolor{currentstroke}%
\pgfsetdash{}{0pt}%
\pgfpathmoveto{\pgfqpoint{3.305000in}{0.499444in}}%
\pgfpathlineto{\pgfqpoint{3.368409in}{0.499444in}}%
\pgfpathlineto{\pgfqpoint{3.368409in}{0.499444in}}%
\pgfpathlineto{\pgfqpoint{3.305000in}{0.499444in}}%
\pgfpathlineto{\pgfqpoint{3.305000in}{0.499444in}}%
\pgfpathclose%
\pgfusepath{stroke}%
\end{pgfscope}%
\begin{pgfscope}%
\pgfpathrectangle{\pgfqpoint{0.515000in}{0.499444in}}{\pgfqpoint{3.487500in}{1.155000in}}%
\pgfusepath{clip}%
\pgfsetbuttcap%
\pgfsetmiterjoin%
\pgfsetlinewidth{1.003750pt}%
\definecolor{currentstroke}{rgb}{0.000000,0.000000,0.000000}%
\pgfsetstrokecolor{currentstroke}%
\pgfsetdash{}{0pt}%
\pgfpathmoveto{\pgfqpoint{3.463523in}{0.499444in}}%
\pgfpathlineto{\pgfqpoint{3.526932in}{0.499444in}}%
\pgfpathlineto{\pgfqpoint{3.526932in}{0.499444in}}%
\pgfpathlineto{\pgfqpoint{3.463523in}{0.499444in}}%
\pgfpathlineto{\pgfqpoint{3.463523in}{0.499444in}}%
\pgfpathclose%
\pgfusepath{stroke}%
\end{pgfscope}%
\begin{pgfscope}%
\pgfpathrectangle{\pgfqpoint{0.515000in}{0.499444in}}{\pgfqpoint{3.487500in}{1.155000in}}%
\pgfusepath{clip}%
\pgfsetbuttcap%
\pgfsetmiterjoin%
\pgfsetlinewidth{1.003750pt}%
\definecolor{currentstroke}{rgb}{0.000000,0.000000,0.000000}%
\pgfsetstrokecolor{currentstroke}%
\pgfsetdash{}{0pt}%
\pgfpathmoveto{\pgfqpoint{3.622046in}{0.499444in}}%
\pgfpathlineto{\pgfqpoint{3.685455in}{0.499444in}}%
\pgfpathlineto{\pgfqpoint{3.685455in}{0.499444in}}%
\pgfpathlineto{\pgfqpoint{3.622046in}{0.499444in}}%
\pgfpathlineto{\pgfqpoint{3.622046in}{0.499444in}}%
\pgfpathclose%
\pgfusepath{stroke}%
\end{pgfscope}%
\begin{pgfscope}%
\pgfpathrectangle{\pgfqpoint{0.515000in}{0.499444in}}{\pgfqpoint{3.487500in}{1.155000in}}%
\pgfusepath{clip}%
\pgfsetbuttcap%
\pgfsetmiterjoin%
\pgfsetlinewidth{1.003750pt}%
\definecolor{currentstroke}{rgb}{0.000000,0.000000,0.000000}%
\pgfsetstrokecolor{currentstroke}%
\pgfsetdash{}{0pt}%
\pgfpathmoveto{\pgfqpoint{3.780568in}{0.499444in}}%
\pgfpathlineto{\pgfqpoint{3.843978in}{0.499444in}}%
\pgfpathlineto{\pgfqpoint{3.843978in}{0.499444in}}%
\pgfpathlineto{\pgfqpoint{3.780568in}{0.499444in}}%
\pgfpathlineto{\pgfqpoint{3.780568in}{0.499444in}}%
\pgfpathclose%
\pgfusepath{stroke}%
\end{pgfscope}%
\begin{pgfscope}%
\pgfpathrectangle{\pgfqpoint{0.515000in}{0.499444in}}{\pgfqpoint{3.487500in}{1.155000in}}%
\pgfusepath{clip}%
\pgfsetbuttcap%
\pgfsetmiterjoin%
\definecolor{currentfill}{rgb}{0.000000,0.000000,0.000000}%
\pgfsetfillcolor{currentfill}%
\pgfsetlinewidth{0.000000pt}%
\definecolor{currentstroke}{rgb}{0.000000,0.000000,0.000000}%
\pgfsetstrokecolor{currentstroke}%
\pgfsetstrokeopacity{0.000000}%
\pgfsetdash{}{0pt}%
\pgfpathmoveto{\pgfqpoint{0.673523in}{0.499444in}}%
\pgfpathlineto{\pgfqpoint{0.736932in}{0.499444in}}%
\pgfpathlineto{\pgfqpoint{0.736932in}{0.499444in}}%
\pgfpathlineto{\pgfqpoint{0.673523in}{0.499444in}}%
\pgfpathlineto{\pgfqpoint{0.673523in}{0.499444in}}%
\pgfpathclose%
\pgfusepath{fill}%
\end{pgfscope}%
\begin{pgfscope}%
\pgfpathrectangle{\pgfqpoint{0.515000in}{0.499444in}}{\pgfqpoint{3.487500in}{1.155000in}}%
\pgfusepath{clip}%
\pgfsetbuttcap%
\pgfsetmiterjoin%
\definecolor{currentfill}{rgb}{0.000000,0.000000,0.000000}%
\pgfsetfillcolor{currentfill}%
\pgfsetlinewidth{0.000000pt}%
\definecolor{currentstroke}{rgb}{0.000000,0.000000,0.000000}%
\pgfsetstrokecolor{currentstroke}%
\pgfsetstrokeopacity{0.000000}%
\pgfsetdash{}{0pt}%
\pgfpathmoveto{\pgfqpoint{0.832046in}{0.499444in}}%
\pgfpathlineto{\pgfqpoint{0.895455in}{0.499444in}}%
\pgfpathlineto{\pgfqpoint{0.895455in}{0.499444in}}%
\pgfpathlineto{\pgfqpoint{0.832046in}{0.499444in}}%
\pgfpathlineto{\pgfqpoint{0.832046in}{0.499444in}}%
\pgfpathclose%
\pgfusepath{fill}%
\end{pgfscope}%
\begin{pgfscope}%
\pgfpathrectangle{\pgfqpoint{0.515000in}{0.499444in}}{\pgfqpoint{3.487500in}{1.155000in}}%
\pgfusepath{clip}%
\pgfsetbuttcap%
\pgfsetmiterjoin%
\definecolor{currentfill}{rgb}{0.000000,0.000000,0.000000}%
\pgfsetfillcolor{currentfill}%
\pgfsetlinewidth{0.000000pt}%
\definecolor{currentstroke}{rgb}{0.000000,0.000000,0.000000}%
\pgfsetstrokecolor{currentstroke}%
\pgfsetstrokeopacity{0.000000}%
\pgfsetdash{}{0pt}%
\pgfpathmoveto{\pgfqpoint{0.990568in}{0.499444in}}%
\pgfpathlineto{\pgfqpoint{1.053978in}{0.499444in}}%
\pgfpathlineto{\pgfqpoint{1.053978in}{0.499444in}}%
\pgfpathlineto{\pgfqpoint{0.990568in}{0.499444in}}%
\pgfpathlineto{\pgfqpoint{0.990568in}{0.499444in}}%
\pgfpathclose%
\pgfusepath{fill}%
\end{pgfscope}%
\begin{pgfscope}%
\pgfpathrectangle{\pgfqpoint{0.515000in}{0.499444in}}{\pgfqpoint{3.487500in}{1.155000in}}%
\pgfusepath{clip}%
\pgfsetbuttcap%
\pgfsetmiterjoin%
\definecolor{currentfill}{rgb}{0.000000,0.000000,0.000000}%
\pgfsetfillcolor{currentfill}%
\pgfsetlinewidth{0.000000pt}%
\definecolor{currentstroke}{rgb}{0.000000,0.000000,0.000000}%
\pgfsetstrokecolor{currentstroke}%
\pgfsetstrokeopacity{0.000000}%
\pgfsetdash{}{0pt}%
\pgfpathmoveto{\pgfqpoint{1.149091in}{0.499444in}}%
\pgfpathlineto{\pgfqpoint{1.212500in}{0.499444in}}%
\pgfpathlineto{\pgfqpoint{1.212500in}{0.499444in}}%
\pgfpathlineto{\pgfqpoint{1.149091in}{0.499444in}}%
\pgfpathlineto{\pgfqpoint{1.149091in}{0.499444in}}%
\pgfpathclose%
\pgfusepath{fill}%
\end{pgfscope}%
\begin{pgfscope}%
\pgfpathrectangle{\pgfqpoint{0.515000in}{0.499444in}}{\pgfqpoint{3.487500in}{1.155000in}}%
\pgfusepath{clip}%
\pgfsetbuttcap%
\pgfsetmiterjoin%
\definecolor{currentfill}{rgb}{0.000000,0.000000,0.000000}%
\pgfsetfillcolor{currentfill}%
\pgfsetlinewidth{0.000000pt}%
\definecolor{currentstroke}{rgb}{0.000000,0.000000,0.000000}%
\pgfsetstrokecolor{currentstroke}%
\pgfsetstrokeopacity{0.000000}%
\pgfsetdash{}{0pt}%
\pgfpathmoveto{\pgfqpoint{1.307614in}{0.499444in}}%
\pgfpathlineto{\pgfqpoint{1.371023in}{0.499444in}}%
\pgfpathlineto{\pgfqpoint{1.371023in}{0.499444in}}%
\pgfpathlineto{\pgfqpoint{1.307614in}{0.499444in}}%
\pgfpathlineto{\pgfqpoint{1.307614in}{0.499444in}}%
\pgfpathclose%
\pgfusepath{fill}%
\end{pgfscope}%
\begin{pgfscope}%
\pgfpathrectangle{\pgfqpoint{0.515000in}{0.499444in}}{\pgfqpoint{3.487500in}{1.155000in}}%
\pgfusepath{clip}%
\pgfsetbuttcap%
\pgfsetmiterjoin%
\definecolor{currentfill}{rgb}{0.000000,0.000000,0.000000}%
\pgfsetfillcolor{currentfill}%
\pgfsetlinewidth{0.000000pt}%
\definecolor{currentstroke}{rgb}{0.000000,0.000000,0.000000}%
\pgfsetstrokecolor{currentstroke}%
\pgfsetstrokeopacity{0.000000}%
\pgfsetdash{}{0pt}%
\pgfpathmoveto{\pgfqpoint{1.466137in}{0.499444in}}%
\pgfpathlineto{\pgfqpoint{1.529546in}{0.499444in}}%
\pgfpathlineto{\pgfqpoint{1.529546in}{0.499444in}}%
\pgfpathlineto{\pgfqpoint{1.466137in}{0.499444in}}%
\pgfpathlineto{\pgfqpoint{1.466137in}{0.499444in}}%
\pgfpathclose%
\pgfusepath{fill}%
\end{pgfscope}%
\begin{pgfscope}%
\pgfpathrectangle{\pgfqpoint{0.515000in}{0.499444in}}{\pgfqpoint{3.487500in}{1.155000in}}%
\pgfusepath{clip}%
\pgfsetbuttcap%
\pgfsetmiterjoin%
\definecolor{currentfill}{rgb}{0.000000,0.000000,0.000000}%
\pgfsetfillcolor{currentfill}%
\pgfsetlinewidth{0.000000pt}%
\definecolor{currentstroke}{rgb}{0.000000,0.000000,0.000000}%
\pgfsetstrokecolor{currentstroke}%
\pgfsetstrokeopacity{0.000000}%
\pgfsetdash{}{0pt}%
\pgfpathmoveto{\pgfqpoint{1.624659in}{0.499444in}}%
\pgfpathlineto{\pgfqpoint{1.688068in}{0.499444in}}%
\pgfpathlineto{\pgfqpoint{1.688068in}{0.499444in}}%
\pgfpathlineto{\pgfqpoint{1.624659in}{0.499444in}}%
\pgfpathlineto{\pgfqpoint{1.624659in}{0.499444in}}%
\pgfpathclose%
\pgfusepath{fill}%
\end{pgfscope}%
\begin{pgfscope}%
\pgfpathrectangle{\pgfqpoint{0.515000in}{0.499444in}}{\pgfqpoint{3.487500in}{1.155000in}}%
\pgfusepath{clip}%
\pgfsetbuttcap%
\pgfsetmiterjoin%
\definecolor{currentfill}{rgb}{0.000000,0.000000,0.000000}%
\pgfsetfillcolor{currentfill}%
\pgfsetlinewidth{0.000000pt}%
\definecolor{currentstroke}{rgb}{0.000000,0.000000,0.000000}%
\pgfsetstrokecolor{currentstroke}%
\pgfsetstrokeopacity{0.000000}%
\pgfsetdash{}{0pt}%
\pgfpathmoveto{\pgfqpoint{1.783182in}{0.499444in}}%
\pgfpathlineto{\pgfqpoint{1.846591in}{0.499444in}}%
\pgfpathlineto{\pgfqpoint{1.846591in}{0.499444in}}%
\pgfpathlineto{\pgfqpoint{1.783182in}{0.499444in}}%
\pgfpathlineto{\pgfqpoint{1.783182in}{0.499444in}}%
\pgfpathclose%
\pgfusepath{fill}%
\end{pgfscope}%
\begin{pgfscope}%
\pgfpathrectangle{\pgfqpoint{0.515000in}{0.499444in}}{\pgfqpoint{3.487500in}{1.155000in}}%
\pgfusepath{clip}%
\pgfsetbuttcap%
\pgfsetmiterjoin%
\definecolor{currentfill}{rgb}{0.000000,0.000000,0.000000}%
\pgfsetfillcolor{currentfill}%
\pgfsetlinewidth{0.000000pt}%
\definecolor{currentstroke}{rgb}{0.000000,0.000000,0.000000}%
\pgfsetstrokecolor{currentstroke}%
\pgfsetstrokeopacity{0.000000}%
\pgfsetdash{}{0pt}%
\pgfpathmoveto{\pgfqpoint{1.941705in}{0.499444in}}%
\pgfpathlineto{\pgfqpoint{2.005114in}{0.499444in}}%
\pgfpathlineto{\pgfqpoint{2.005114in}{0.499444in}}%
\pgfpathlineto{\pgfqpoint{1.941705in}{0.499444in}}%
\pgfpathlineto{\pgfqpoint{1.941705in}{0.499444in}}%
\pgfpathclose%
\pgfusepath{fill}%
\end{pgfscope}%
\begin{pgfscope}%
\pgfpathrectangle{\pgfqpoint{0.515000in}{0.499444in}}{\pgfqpoint{3.487500in}{1.155000in}}%
\pgfusepath{clip}%
\pgfsetbuttcap%
\pgfsetmiterjoin%
\definecolor{currentfill}{rgb}{0.000000,0.000000,0.000000}%
\pgfsetfillcolor{currentfill}%
\pgfsetlinewidth{0.000000pt}%
\definecolor{currentstroke}{rgb}{0.000000,0.000000,0.000000}%
\pgfsetstrokecolor{currentstroke}%
\pgfsetstrokeopacity{0.000000}%
\pgfsetdash{}{0pt}%
\pgfpathmoveto{\pgfqpoint{2.100228in}{0.499444in}}%
\pgfpathlineto{\pgfqpoint{2.163637in}{0.499444in}}%
\pgfpathlineto{\pgfqpoint{2.163637in}{0.499444in}}%
\pgfpathlineto{\pgfqpoint{2.100228in}{0.499444in}}%
\pgfpathlineto{\pgfqpoint{2.100228in}{0.499444in}}%
\pgfpathclose%
\pgfusepath{fill}%
\end{pgfscope}%
\begin{pgfscope}%
\pgfpathrectangle{\pgfqpoint{0.515000in}{0.499444in}}{\pgfqpoint{3.487500in}{1.155000in}}%
\pgfusepath{clip}%
\pgfsetbuttcap%
\pgfsetmiterjoin%
\definecolor{currentfill}{rgb}{0.000000,0.000000,0.000000}%
\pgfsetfillcolor{currentfill}%
\pgfsetlinewidth{0.000000pt}%
\definecolor{currentstroke}{rgb}{0.000000,0.000000,0.000000}%
\pgfsetstrokecolor{currentstroke}%
\pgfsetstrokeopacity{0.000000}%
\pgfsetdash{}{0pt}%
\pgfpathmoveto{\pgfqpoint{2.258750in}{0.499444in}}%
\pgfpathlineto{\pgfqpoint{2.322159in}{0.499444in}}%
\pgfpathlineto{\pgfqpoint{2.322159in}{0.686855in}}%
\pgfpathlineto{\pgfqpoint{2.258750in}{0.686855in}}%
\pgfpathlineto{\pgfqpoint{2.258750in}{0.499444in}}%
\pgfpathclose%
\pgfusepath{fill}%
\end{pgfscope}%
\begin{pgfscope}%
\pgfpathrectangle{\pgfqpoint{0.515000in}{0.499444in}}{\pgfqpoint{3.487500in}{1.155000in}}%
\pgfusepath{clip}%
\pgfsetbuttcap%
\pgfsetmiterjoin%
\definecolor{currentfill}{rgb}{0.000000,0.000000,0.000000}%
\pgfsetfillcolor{currentfill}%
\pgfsetlinewidth{0.000000pt}%
\definecolor{currentstroke}{rgb}{0.000000,0.000000,0.000000}%
\pgfsetstrokecolor{currentstroke}%
\pgfsetstrokeopacity{0.000000}%
\pgfsetdash{}{0pt}%
\pgfpathmoveto{\pgfqpoint{2.417273in}{0.499444in}}%
\pgfpathlineto{\pgfqpoint{2.480682in}{0.499444in}}%
\pgfpathlineto{\pgfqpoint{2.480682in}{0.522047in}}%
\pgfpathlineto{\pgfqpoint{2.417273in}{0.522047in}}%
\pgfpathlineto{\pgfqpoint{2.417273in}{0.499444in}}%
\pgfpathclose%
\pgfusepath{fill}%
\end{pgfscope}%
\begin{pgfscope}%
\pgfpathrectangle{\pgfqpoint{0.515000in}{0.499444in}}{\pgfqpoint{3.487500in}{1.155000in}}%
\pgfusepath{clip}%
\pgfsetbuttcap%
\pgfsetmiterjoin%
\definecolor{currentfill}{rgb}{0.000000,0.000000,0.000000}%
\pgfsetfillcolor{currentfill}%
\pgfsetlinewidth{0.000000pt}%
\definecolor{currentstroke}{rgb}{0.000000,0.000000,0.000000}%
\pgfsetstrokecolor{currentstroke}%
\pgfsetstrokeopacity{0.000000}%
\pgfsetdash{}{0pt}%
\pgfpathmoveto{\pgfqpoint{2.575796in}{0.499444in}}%
\pgfpathlineto{\pgfqpoint{2.639205in}{0.499444in}}%
\pgfpathlineto{\pgfqpoint{2.639205in}{0.499444in}}%
\pgfpathlineto{\pgfqpoint{2.575796in}{0.499444in}}%
\pgfpathlineto{\pgfqpoint{2.575796in}{0.499444in}}%
\pgfpathclose%
\pgfusepath{fill}%
\end{pgfscope}%
\begin{pgfscope}%
\pgfpathrectangle{\pgfqpoint{0.515000in}{0.499444in}}{\pgfqpoint{3.487500in}{1.155000in}}%
\pgfusepath{clip}%
\pgfsetbuttcap%
\pgfsetmiterjoin%
\definecolor{currentfill}{rgb}{0.000000,0.000000,0.000000}%
\pgfsetfillcolor{currentfill}%
\pgfsetlinewidth{0.000000pt}%
\definecolor{currentstroke}{rgb}{0.000000,0.000000,0.000000}%
\pgfsetstrokecolor{currentstroke}%
\pgfsetstrokeopacity{0.000000}%
\pgfsetdash{}{0pt}%
\pgfpathmoveto{\pgfqpoint{2.734318in}{0.499444in}}%
\pgfpathlineto{\pgfqpoint{2.797728in}{0.499444in}}%
\pgfpathlineto{\pgfqpoint{2.797728in}{0.499444in}}%
\pgfpathlineto{\pgfqpoint{2.734318in}{0.499444in}}%
\pgfpathlineto{\pgfqpoint{2.734318in}{0.499444in}}%
\pgfpathclose%
\pgfusepath{fill}%
\end{pgfscope}%
\begin{pgfscope}%
\pgfpathrectangle{\pgfqpoint{0.515000in}{0.499444in}}{\pgfqpoint{3.487500in}{1.155000in}}%
\pgfusepath{clip}%
\pgfsetbuttcap%
\pgfsetmiterjoin%
\definecolor{currentfill}{rgb}{0.000000,0.000000,0.000000}%
\pgfsetfillcolor{currentfill}%
\pgfsetlinewidth{0.000000pt}%
\definecolor{currentstroke}{rgb}{0.000000,0.000000,0.000000}%
\pgfsetstrokecolor{currentstroke}%
\pgfsetstrokeopacity{0.000000}%
\pgfsetdash{}{0pt}%
\pgfpathmoveto{\pgfqpoint{2.892841in}{0.499444in}}%
\pgfpathlineto{\pgfqpoint{2.956250in}{0.499444in}}%
\pgfpathlineto{\pgfqpoint{2.956250in}{0.499444in}}%
\pgfpathlineto{\pgfqpoint{2.892841in}{0.499444in}}%
\pgfpathlineto{\pgfqpoint{2.892841in}{0.499444in}}%
\pgfpathclose%
\pgfusepath{fill}%
\end{pgfscope}%
\begin{pgfscope}%
\pgfpathrectangle{\pgfqpoint{0.515000in}{0.499444in}}{\pgfqpoint{3.487500in}{1.155000in}}%
\pgfusepath{clip}%
\pgfsetbuttcap%
\pgfsetmiterjoin%
\definecolor{currentfill}{rgb}{0.000000,0.000000,0.000000}%
\pgfsetfillcolor{currentfill}%
\pgfsetlinewidth{0.000000pt}%
\definecolor{currentstroke}{rgb}{0.000000,0.000000,0.000000}%
\pgfsetstrokecolor{currentstroke}%
\pgfsetstrokeopacity{0.000000}%
\pgfsetdash{}{0pt}%
\pgfpathmoveto{\pgfqpoint{3.051364in}{0.499444in}}%
\pgfpathlineto{\pgfqpoint{3.114773in}{0.499444in}}%
\pgfpathlineto{\pgfqpoint{3.114773in}{0.499444in}}%
\pgfpathlineto{\pgfqpoint{3.051364in}{0.499444in}}%
\pgfpathlineto{\pgfqpoint{3.051364in}{0.499444in}}%
\pgfpathclose%
\pgfusepath{fill}%
\end{pgfscope}%
\begin{pgfscope}%
\pgfpathrectangle{\pgfqpoint{0.515000in}{0.499444in}}{\pgfqpoint{3.487500in}{1.155000in}}%
\pgfusepath{clip}%
\pgfsetbuttcap%
\pgfsetmiterjoin%
\definecolor{currentfill}{rgb}{0.000000,0.000000,0.000000}%
\pgfsetfillcolor{currentfill}%
\pgfsetlinewidth{0.000000pt}%
\definecolor{currentstroke}{rgb}{0.000000,0.000000,0.000000}%
\pgfsetstrokecolor{currentstroke}%
\pgfsetstrokeopacity{0.000000}%
\pgfsetdash{}{0pt}%
\pgfpathmoveto{\pgfqpoint{3.209887in}{0.499444in}}%
\pgfpathlineto{\pgfqpoint{3.273296in}{0.499444in}}%
\pgfpathlineto{\pgfqpoint{3.273296in}{0.499444in}}%
\pgfpathlineto{\pgfqpoint{3.209887in}{0.499444in}}%
\pgfpathlineto{\pgfqpoint{3.209887in}{0.499444in}}%
\pgfpathclose%
\pgfusepath{fill}%
\end{pgfscope}%
\begin{pgfscope}%
\pgfpathrectangle{\pgfqpoint{0.515000in}{0.499444in}}{\pgfqpoint{3.487500in}{1.155000in}}%
\pgfusepath{clip}%
\pgfsetbuttcap%
\pgfsetmiterjoin%
\definecolor{currentfill}{rgb}{0.000000,0.000000,0.000000}%
\pgfsetfillcolor{currentfill}%
\pgfsetlinewidth{0.000000pt}%
\definecolor{currentstroke}{rgb}{0.000000,0.000000,0.000000}%
\pgfsetstrokecolor{currentstroke}%
\pgfsetstrokeopacity{0.000000}%
\pgfsetdash{}{0pt}%
\pgfpathmoveto{\pgfqpoint{3.368409in}{0.499444in}}%
\pgfpathlineto{\pgfqpoint{3.431818in}{0.499444in}}%
\pgfpathlineto{\pgfqpoint{3.431818in}{0.499444in}}%
\pgfpathlineto{\pgfqpoint{3.368409in}{0.499444in}}%
\pgfpathlineto{\pgfqpoint{3.368409in}{0.499444in}}%
\pgfpathclose%
\pgfusepath{fill}%
\end{pgfscope}%
\begin{pgfscope}%
\pgfpathrectangle{\pgfqpoint{0.515000in}{0.499444in}}{\pgfqpoint{3.487500in}{1.155000in}}%
\pgfusepath{clip}%
\pgfsetbuttcap%
\pgfsetmiterjoin%
\definecolor{currentfill}{rgb}{0.000000,0.000000,0.000000}%
\pgfsetfillcolor{currentfill}%
\pgfsetlinewidth{0.000000pt}%
\definecolor{currentstroke}{rgb}{0.000000,0.000000,0.000000}%
\pgfsetstrokecolor{currentstroke}%
\pgfsetstrokeopacity{0.000000}%
\pgfsetdash{}{0pt}%
\pgfpathmoveto{\pgfqpoint{3.526932in}{0.499444in}}%
\pgfpathlineto{\pgfqpoint{3.590341in}{0.499444in}}%
\pgfpathlineto{\pgfqpoint{3.590341in}{0.499444in}}%
\pgfpathlineto{\pgfqpoint{3.526932in}{0.499444in}}%
\pgfpathlineto{\pgfqpoint{3.526932in}{0.499444in}}%
\pgfpathclose%
\pgfusepath{fill}%
\end{pgfscope}%
\begin{pgfscope}%
\pgfpathrectangle{\pgfqpoint{0.515000in}{0.499444in}}{\pgfqpoint{3.487500in}{1.155000in}}%
\pgfusepath{clip}%
\pgfsetbuttcap%
\pgfsetmiterjoin%
\definecolor{currentfill}{rgb}{0.000000,0.000000,0.000000}%
\pgfsetfillcolor{currentfill}%
\pgfsetlinewidth{0.000000pt}%
\definecolor{currentstroke}{rgb}{0.000000,0.000000,0.000000}%
\pgfsetstrokecolor{currentstroke}%
\pgfsetstrokeopacity{0.000000}%
\pgfsetdash{}{0pt}%
\pgfpathmoveto{\pgfqpoint{3.685455in}{0.499444in}}%
\pgfpathlineto{\pgfqpoint{3.748864in}{0.499444in}}%
\pgfpathlineto{\pgfqpoint{3.748864in}{0.499444in}}%
\pgfpathlineto{\pgfqpoint{3.685455in}{0.499444in}}%
\pgfpathlineto{\pgfqpoint{3.685455in}{0.499444in}}%
\pgfpathclose%
\pgfusepath{fill}%
\end{pgfscope}%
\begin{pgfscope}%
\pgfpathrectangle{\pgfqpoint{0.515000in}{0.499444in}}{\pgfqpoint{3.487500in}{1.155000in}}%
\pgfusepath{clip}%
\pgfsetbuttcap%
\pgfsetmiterjoin%
\definecolor{currentfill}{rgb}{0.000000,0.000000,0.000000}%
\pgfsetfillcolor{currentfill}%
\pgfsetlinewidth{0.000000pt}%
\definecolor{currentstroke}{rgb}{0.000000,0.000000,0.000000}%
\pgfsetstrokecolor{currentstroke}%
\pgfsetstrokeopacity{0.000000}%
\pgfsetdash{}{0pt}%
\pgfpathmoveto{\pgfqpoint{3.843978in}{0.499444in}}%
\pgfpathlineto{\pgfqpoint{3.907387in}{0.499444in}}%
\pgfpathlineto{\pgfqpoint{3.907387in}{0.499444in}}%
\pgfpathlineto{\pgfqpoint{3.843978in}{0.499444in}}%
\pgfpathlineto{\pgfqpoint{3.843978in}{0.499444in}}%
\pgfpathclose%
\pgfusepath{fill}%
\end{pgfscope}%
\begin{pgfscope}%
\pgfsetbuttcap%
\pgfsetroundjoin%
\definecolor{currentfill}{rgb}{0.000000,0.000000,0.000000}%
\pgfsetfillcolor{currentfill}%
\pgfsetlinewidth{0.803000pt}%
\definecolor{currentstroke}{rgb}{0.000000,0.000000,0.000000}%
\pgfsetstrokecolor{currentstroke}%
\pgfsetdash{}{0pt}%
\pgfsys@defobject{currentmarker}{\pgfqpoint{0.000000in}{-0.048611in}}{\pgfqpoint{0.000000in}{0.000000in}}{%
\pgfpathmoveto{\pgfqpoint{0.000000in}{0.000000in}}%
\pgfpathlineto{\pgfqpoint{0.000000in}{-0.048611in}}%
\pgfusepath{stroke,fill}%
}%
\begin{pgfscope}%
\pgfsys@transformshift{0.515000in}{0.499444in}%
\pgfsys@useobject{currentmarker}{}%
\end{pgfscope}%
\end{pgfscope}%
\begin{pgfscope}%
\pgfsetbuttcap%
\pgfsetroundjoin%
\definecolor{currentfill}{rgb}{0.000000,0.000000,0.000000}%
\pgfsetfillcolor{currentfill}%
\pgfsetlinewidth{0.803000pt}%
\definecolor{currentstroke}{rgb}{0.000000,0.000000,0.000000}%
\pgfsetstrokecolor{currentstroke}%
\pgfsetdash{}{0pt}%
\pgfsys@defobject{currentmarker}{\pgfqpoint{0.000000in}{-0.048611in}}{\pgfqpoint{0.000000in}{0.000000in}}{%
\pgfpathmoveto{\pgfqpoint{0.000000in}{0.000000in}}%
\pgfpathlineto{\pgfqpoint{0.000000in}{-0.048611in}}%
\pgfusepath{stroke,fill}%
}%
\begin{pgfscope}%
\pgfsys@transformshift{0.673523in}{0.499444in}%
\pgfsys@useobject{currentmarker}{}%
\end{pgfscope}%
\end{pgfscope}%
\begin{pgfscope}%
\definecolor{textcolor}{rgb}{0.000000,0.000000,0.000000}%
\pgfsetstrokecolor{textcolor}%
\pgfsetfillcolor{textcolor}%
\pgftext[x=0.673523in,y=0.402222in,,top]{\color{textcolor}\rmfamily\fontsize{10.000000}{12.000000}\selectfont 0.0}%
\end{pgfscope}%
\begin{pgfscope}%
\pgfsetbuttcap%
\pgfsetroundjoin%
\definecolor{currentfill}{rgb}{0.000000,0.000000,0.000000}%
\pgfsetfillcolor{currentfill}%
\pgfsetlinewidth{0.803000pt}%
\definecolor{currentstroke}{rgb}{0.000000,0.000000,0.000000}%
\pgfsetstrokecolor{currentstroke}%
\pgfsetdash{}{0pt}%
\pgfsys@defobject{currentmarker}{\pgfqpoint{0.000000in}{-0.048611in}}{\pgfqpoint{0.000000in}{0.000000in}}{%
\pgfpathmoveto{\pgfqpoint{0.000000in}{0.000000in}}%
\pgfpathlineto{\pgfqpoint{0.000000in}{-0.048611in}}%
\pgfusepath{stroke,fill}%
}%
\begin{pgfscope}%
\pgfsys@transformshift{0.832046in}{0.499444in}%
\pgfsys@useobject{currentmarker}{}%
\end{pgfscope}%
\end{pgfscope}%
\begin{pgfscope}%
\pgfsetbuttcap%
\pgfsetroundjoin%
\definecolor{currentfill}{rgb}{0.000000,0.000000,0.000000}%
\pgfsetfillcolor{currentfill}%
\pgfsetlinewidth{0.803000pt}%
\definecolor{currentstroke}{rgb}{0.000000,0.000000,0.000000}%
\pgfsetstrokecolor{currentstroke}%
\pgfsetdash{}{0pt}%
\pgfsys@defobject{currentmarker}{\pgfqpoint{0.000000in}{-0.048611in}}{\pgfqpoint{0.000000in}{0.000000in}}{%
\pgfpathmoveto{\pgfqpoint{0.000000in}{0.000000in}}%
\pgfpathlineto{\pgfqpoint{0.000000in}{-0.048611in}}%
\pgfusepath{stroke,fill}%
}%
\begin{pgfscope}%
\pgfsys@transformshift{0.990568in}{0.499444in}%
\pgfsys@useobject{currentmarker}{}%
\end{pgfscope}%
\end{pgfscope}%
\begin{pgfscope}%
\definecolor{textcolor}{rgb}{0.000000,0.000000,0.000000}%
\pgfsetstrokecolor{textcolor}%
\pgfsetfillcolor{textcolor}%
\pgftext[x=0.990568in,y=0.402222in,,top]{\color{textcolor}\rmfamily\fontsize{10.000000}{12.000000}\selectfont 0.1}%
\end{pgfscope}%
\begin{pgfscope}%
\pgfsetbuttcap%
\pgfsetroundjoin%
\definecolor{currentfill}{rgb}{0.000000,0.000000,0.000000}%
\pgfsetfillcolor{currentfill}%
\pgfsetlinewidth{0.803000pt}%
\definecolor{currentstroke}{rgb}{0.000000,0.000000,0.000000}%
\pgfsetstrokecolor{currentstroke}%
\pgfsetdash{}{0pt}%
\pgfsys@defobject{currentmarker}{\pgfqpoint{0.000000in}{-0.048611in}}{\pgfqpoint{0.000000in}{0.000000in}}{%
\pgfpathmoveto{\pgfqpoint{0.000000in}{0.000000in}}%
\pgfpathlineto{\pgfqpoint{0.000000in}{-0.048611in}}%
\pgfusepath{stroke,fill}%
}%
\begin{pgfscope}%
\pgfsys@transformshift{1.149091in}{0.499444in}%
\pgfsys@useobject{currentmarker}{}%
\end{pgfscope}%
\end{pgfscope}%
\begin{pgfscope}%
\pgfsetbuttcap%
\pgfsetroundjoin%
\definecolor{currentfill}{rgb}{0.000000,0.000000,0.000000}%
\pgfsetfillcolor{currentfill}%
\pgfsetlinewidth{0.803000pt}%
\definecolor{currentstroke}{rgb}{0.000000,0.000000,0.000000}%
\pgfsetstrokecolor{currentstroke}%
\pgfsetdash{}{0pt}%
\pgfsys@defobject{currentmarker}{\pgfqpoint{0.000000in}{-0.048611in}}{\pgfqpoint{0.000000in}{0.000000in}}{%
\pgfpathmoveto{\pgfqpoint{0.000000in}{0.000000in}}%
\pgfpathlineto{\pgfqpoint{0.000000in}{-0.048611in}}%
\pgfusepath{stroke,fill}%
}%
\begin{pgfscope}%
\pgfsys@transformshift{1.307614in}{0.499444in}%
\pgfsys@useobject{currentmarker}{}%
\end{pgfscope}%
\end{pgfscope}%
\begin{pgfscope}%
\definecolor{textcolor}{rgb}{0.000000,0.000000,0.000000}%
\pgfsetstrokecolor{textcolor}%
\pgfsetfillcolor{textcolor}%
\pgftext[x=1.307614in,y=0.402222in,,top]{\color{textcolor}\rmfamily\fontsize{10.000000}{12.000000}\selectfont 0.2}%
\end{pgfscope}%
\begin{pgfscope}%
\pgfsetbuttcap%
\pgfsetroundjoin%
\definecolor{currentfill}{rgb}{0.000000,0.000000,0.000000}%
\pgfsetfillcolor{currentfill}%
\pgfsetlinewidth{0.803000pt}%
\definecolor{currentstroke}{rgb}{0.000000,0.000000,0.000000}%
\pgfsetstrokecolor{currentstroke}%
\pgfsetdash{}{0pt}%
\pgfsys@defobject{currentmarker}{\pgfqpoint{0.000000in}{-0.048611in}}{\pgfqpoint{0.000000in}{0.000000in}}{%
\pgfpathmoveto{\pgfqpoint{0.000000in}{0.000000in}}%
\pgfpathlineto{\pgfqpoint{0.000000in}{-0.048611in}}%
\pgfusepath{stroke,fill}%
}%
\begin{pgfscope}%
\pgfsys@transformshift{1.466137in}{0.499444in}%
\pgfsys@useobject{currentmarker}{}%
\end{pgfscope}%
\end{pgfscope}%
\begin{pgfscope}%
\pgfsetbuttcap%
\pgfsetroundjoin%
\definecolor{currentfill}{rgb}{0.000000,0.000000,0.000000}%
\pgfsetfillcolor{currentfill}%
\pgfsetlinewidth{0.803000pt}%
\definecolor{currentstroke}{rgb}{0.000000,0.000000,0.000000}%
\pgfsetstrokecolor{currentstroke}%
\pgfsetdash{}{0pt}%
\pgfsys@defobject{currentmarker}{\pgfqpoint{0.000000in}{-0.048611in}}{\pgfqpoint{0.000000in}{0.000000in}}{%
\pgfpathmoveto{\pgfqpoint{0.000000in}{0.000000in}}%
\pgfpathlineto{\pgfqpoint{0.000000in}{-0.048611in}}%
\pgfusepath{stroke,fill}%
}%
\begin{pgfscope}%
\pgfsys@transformshift{1.624659in}{0.499444in}%
\pgfsys@useobject{currentmarker}{}%
\end{pgfscope}%
\end{pgfscope}%
\begin{pgfscope}%
\definecolor{textcolor}{rgb}{0.000000,0.000000,0.000000}%
\pgfsetstrokecolor{textcolor}%
\pgfsetfillcolor{textcolor}%
\pgftext[x=1.624659in,y=0.402222in,,top]{\color{textcolor}\rmfamily\fontsize{10.000000}{12.000000}\selectfont 0.3}%
\end{pgfscope}%
\begin{pgfscope}%
\pgfsetbuttcap%
\pgfsetroundjoin%
\definecolor{currentfill}{rgb}{0.000000,0.000000,0.000000}%
\pgfsetfillcolor{currentfill}%
\pgfsetlinewidth{0.803000pt}%
\definecolor{currentstroke}{rgb}{0.000000,0.000000,0.000000}%
\pgfsetstrokecolor{currentstroke}%
\pgfsetdash{}{0pt}%
\pgfsys@defobject{currentmarker}{\pgfqpoint{0.000000in}{-0.048611in}}{\pgfqpoint{0.000000in}{0.000000in}}{%
\pgfpathmoveto{\pgfqpoint{0.000000in}{0.000000in}}%
\pgfpathlineto{\pgfqpoint{0.000000in}{-0.048611in}}%
\pgfusepath{stroke,fill}%
}%
\begin{pgfscope}%
\pgfsys@transformshift{1.783182in}{0.499444in}%
\pgfsys@useobject{currentmarker}{}%
\end{pgfscope}%
\end{pgfscope}%
\begin{pgfscope}%
\pgfsetbuttcap%
\pgfsetroundjoin%
\definecolor{currentfill}{rgb}{0.000000,0.000000,0.000000}%
\pgfsetfillcolor{currentfill}%
\pgfsetlinewidth{0.803000pt}%
\definecolor{currentstroke}{rgb}{0.000000,0.000000,0.000000}%
\pgfsetstrokecolor{currentstroke}%
\pgfsetdash{}{0pt}%
\pgfsys@defobject{currentmarker}{\pgfqpoint{0.000000in}{-0.048611in}}{\pgfqpoint{0.000000in}{0.000000in}}{%
\pgfpathmoveto{\pgfqpoint{0.000000in}{0.000000in}}%
\pgfpathlineto{\pgfqpoint{0.000000in}{-0.048611in}}%
\pgfusepath{stroke,fill}%
}%
\begin{pgfscope}%
\pgfsys@transformshift{1.941705in}{0.499444in}%
\pgfsys@useobject{currentmarker}{}%
\end{pgfscope}%
\end{pgfscope}%
\begin{pgfscope}%
\definecolor{textcolor}{rgb}{0.000000,0.000000,0.000000}%
\pgfsetstrokecolor{textcolor}%
\pgfsetfillcolor{textcolor}%
\pgftext[x=1.941705in,y=0.402222in,,top]{\color{textcolor}\rmfamily\fontsize{10.000000}{12.000000}\selectfont 0.4}%
\end{pgfscope}%
\begin{pgfscope}%
\pgfsetbuttcap%
\pgfsetroundjoin%
\definecolor{currentfill}{rgb}{0.000000,0.000000,0.000000}%
\pgfsetfillcolor{currentfill}%
\pgfsetlinewidth{0.803000pt}%
\definecolor{currentstroke}{rgb}{0.000000,0.000000,0.000000}%
\pgfsetstrokecolor{currentstroke}%
\pgfsetdash{}{0pt}%
\pgfsys@defobject{currentmarker}{\pgfqpoint{0.000000in}{-0.048611in}}{\pgfqpoint{0.000000in}{0.000000in}}{%
\pgfpathmoveto{\pgfqpoint{0.000000in}{0.000000in}}%
\pgfpathlineto{\pgfqpoint{0.000000in}{-0.048611in}}%
\pgfusepath{stroke,fill}%
}%
\begin{pgfscope}%
\pgfsys@transformshift{2.100228in}{0.499444in}%
\pgfsys@useobject{currentmarker}{}%
\end{pgfscope}%
\end{pgfscope}%
\begin{pgfscope}%
\pgfsetbuttcap%
\pgfsetroundjoin%
\definecolor{currentfill}{rgb}{0.000000,0.000000,0.000000}%
\pgfsetfillcolor{currentfill}%
\pgfsetlinewidth{0.803000pt}%
\definecolor{currentstroke}{rgb}{0.000000,0.000000,0.000000}%
\pgfsetstrokecolor{currentstroke}%
\pgfsetdash{}{0pt}%
\pgfsys@defobject{currentmarker}{\pgfqpoint{0.000000in}{-0.048611in}}{\pgfqpoint{0.000000in}{0.000000in}}{%
\pgfpathmoveto{\pgfqpoint{0.000000in}{0.000000in}}%
\pgfpathlineto{\pgfqpoint{0.000000in}{-0.048611in}}%
\pgfusepath{stroke,fill}%
}%
\begin{pgfscope}%
\pgfsys@transformshift{2.258750in}{0.499444in}%
\pgfsys@useobject{currentmarker}{}%
\end{pgfscope}%
\end{pgfscope}%
\begin{pgfscope}%
\definecolor{textcolor}{rgb}{0.000000,0.000000,0.000000}%
\pgfsetstrokecolor{textcolor}%
\pgfsetfillcolor{textcolor}%
\pgftext[x=2.258750in,y=0.402222in,,top]{\color{textcolor}\rmfamily\fontsize{10.000000}{12.000000}\selectfont 0.5}%
\end{pgfscope}%
\begin{pgfscope}%
\pgfsetbuttcap%
\pgfsetroundjoin%
\definecolor{currentfill}{rgb}{0.000000,0.000000,0.000000}%
\pgfsetfillcolor{currentfill}%
\pgfsetlinewidth{0.803000pt}%
\definecolor{currentstroke}{rgb}{0.000000,0.000000,0.000000}%
\pgfsetstrokecolor{currentstroke}%
\pgfsetdash{}{0pt}%
\pgfsys@defobject{currentmarker}{\pgfqpoint{0.000000in}{-0.048611in}}{\pgfqpoint{0.000000in}{0.000000in}}{%
\pgfpathmoveto{\pgfqpoint{0.000000in}{0.000000in}}%
\pgfpathlineto{\pgfqpoint{0.000000in}{-0.048611in}}%
\pgfusepath{stroke,fill}%
}%
\begin{pgfscope}%
\pgfsys@transformshift{2.417273in}{0.499444in}%
\pgfsys@useobject{currentmarker}{}%
\end{pgfscope}%
\end{pgfscope}%
\begin{pgfscope}%
\pgfsetbuttcap%
\pgfsetroundjoin%
\definecolor{currentfill}{rgb}{0.000000,0.000000,0.000000}%
\pgfsetfillcolor{currentfill}%
\pgfsetlinewidth{0.803000pt}%
\definecolor{currentstroke}{rgb}{0.000000,0.000000,0.000000}%
\pgfsetstrokecolor{currentstroke}%
\pgfsetdash{}{0pt}%
\pgfsys@defobject{currentmarker}{\pgfqpoint{0.000000in}{-0.048611in}}{\pgfqpoint{0.000000in}{0.000000in}}{%
\pgfpathmoveto{\pgfqpoint{0.000000in}{0.000000in}}%
\pgfpathlineto{\pgfqpoint{0.000000in}{-0.048611in}}%
\pgfusepath{stroke,fill}%
}%
\begin{pgfscope}%
\pgfsys@transformshift{2.575796in}{0.499444in}%
\pgfsys@useobject{currentmarker}{}%
\end{pgfscope}%
\end{pgfscope}%
\begin{pgfscope}%
\definecolor{textcolor}{rgb}{0.000000,0.000000,0.000000}%
\pgfsetstrokecolor{textcolor}%
\pgfsetfillcolor{textcolor}%
\pgftext[x=2.575796in,y=0.402222in,,top]{\color{textcolor}\rmfamily\fontsize{10.000000}{12.000000}\selectfont 0.6}%
\end{pgfscope}%
\begin{pgfscope}%
\pgfsetbuttcap%
\pgfsetroundjoin%
\definecolor{currentfill}{rgb}{0.000000,0.000000,0.000000}%
\pgfsetfillcolor{currentfill}%
\pgfsetlinewidth{0.803000pt}%
\definecolor{currentstroke}{rgb}{0.000000,0.000000,0.000000}%
\pgfsetstrokecolor{currentstroke}%
\pgfsetdash{}{0pt}%
\pgfsys@defobject{currentmarker}{\pgfqpoint{0.000000in}{-0.048611in}}{\pgfqpoint{0.000000in}{0.000000in}}{%
\pgfpathmoveto{\pgfqpoint{0.000000in}{0.000000in}}%
\pgfpathlineto{\pgfqpoint{0.000000in}{-0.048611in}}%
\pgfusepath{stroke,fill}%
}%
\begin{pgfscope}%
\pgfsys@transformshift{2.734318in}{0.499444in}%
\pgfsys@useobject{currentmarker}{}%
\end{pgfscope}%
\end{pgfscope}%
\begin{pgfscope}%
\pgfsetbuttcap%
\pgfsetroundjoin%
\definecolor{currentfill}{rgb}{0.000000,0.000000,0.000000}%
\pgfsetfillcolor{currentfill}%
\pgfsetlinewidth{0.803000pt}%
\definecolor{currentstroke}{rgb}{0.000000,0.000000,0.000000}%
\pgfsetstrokecolor{currentstroke}%
\pgfsetdash{}{0pt}%
\pgfsys@defobject{currentmarker}{\pgfqpoint{0.000000in}{-0.048611in}}{\pgfqpoint{0.000000in}{0.000000in}}{%
\pgfpathmoveto{\pgfqpoint{0.000000in}{0.000000in}}%
\pgfpathlineto{\pgfqpoint{0.000000in}{-0.048611in}}%
\pgfusepath{stroke,fill}%
}%
\begin{pgfscope}%
\pgfsys@transformshift{2.892841in}{0.499444in}%
\pgfsys@useobject{currentmarker}{}%
\end{pgfscope}%
\end{pgfscope}%
\begin{pgfscope}%
\definecolor{textcolor}{rgb}{0.000000,0.000000,0.000000}%
\pgfsetstrokecolor{textcolor}%
\pgfsetfillcolor{textcolor}%
\pgftext[x=2.892841in,y=0.402222in,,top]{\color{textcolor}\rmfamily\fontsize{10.000000}{12.000000}\selectfont 0.7}%
\end{pgfscope}%
\begin{pgfscope}%
\pgfsetbuttcap%
\pgfsetroundjoin%
\definecolor{currentfill}{rgb}{0.000000,0.000000,0.000000}%
\pgfsetfillcolor{currentfill}%
\pgfsetlinewidth{0.803000pt}%
\definecolor{currentstroke}{rgb}{0.000000,0.000000,0.000000}%
\pgfsetstrokecolor{currentstroke}%
\pgfsetdash{}{0pt}%
\pgfsys@defobject{currentmarker}{\pgfqpoint{0.000000in}{-0.048611in}}{\pgfqpoint{0.000000in}{0.000000in}}{%
\pgfpathmoveto{\pgfqpoint{0.000000in}{0.000000in}}%
\pgfpathlineto{\pgfqpoint{0.000000in}{-0.048611in}}%
\pgfusepath{stroke,fill}%
}%
\begin{pgfscope}%
\pgfsys@transformshift{3.051364in}{0.499444in}%
\pgfsys@useobject{currentmarker}{}%
\end{pgfscope}%
\end{pgfscope}%
\begin{pgfscope}%
\pgfsetbuttcap%
\pgfsetroundjoin%
\definecolor{currentfill}{rgb}{0.000000,0.000000,0.000000}%
\pgfsetfillcolor{currentfill}%
\pgfsetlinewidth{0.803000pt}%
\definecolor{currentstroke}{rgb}{0.000000,0.000000,0.000000}%
\pgfsetstrokecolor{currentstroke}%
\pgfsetdash{}{0pt}%
\pgfsys@defobject{currentmarker}{\pgfqpoint{0.000000in}{-0.048611in}}{\pgfqpoint{0.000000in}{0.000000in}}{%
\pgfpathmoveto{\pgfqpoint{0.000000in}{0.000000in}}%
\pgfpathlineto{\pgfqpoint{0.000000in}{-0.048611in}}%
\pgfusepath{stroke,fill}%
}%
\begin{pgfscope}%
\pgfsys@transformshift{3.209887in}{0.499444in}%
\pgfsys@useobject{currentmarker}{}%
\end{pgfscope}%
\end{pgfscope}%
\begin{pgfscope}%
\definecolor{textcolor}{rgb}{0.000000,0.000000,0.000000}%
\pgfsetstrokecolor{textcolor}%
\pgfsetfillcolor{textcolor}%
\pgftext[x=3.209887in,y=0.402222in,,top]{\color{textcolor}\rmfamily\fontsize{10.000000}{12.000000}\selectfont 0.8}%
\end{pgfscope}%
\begin{pgfscope}%
\pgfsetbuttcap%
\pgfsetroundjoin%
\definecolor{currentfill}{rgb}{0.000000,0.000000,0.000000}%
\pgfsetfillcolor{currentfill}%
\pgfsetlinewidth{0.803000pt}%
\definecolor{currentstroke}{rgb}{0.000000,0.000000,0.000000}%
\pgfsetstrokecolor{currentstroke}%
\pgfsetdash{}{0pt}%
\pgfsys@defobject{currentmarker}{\pgfqpoint{0.000000in}{-0.048611in}}{\pgfqpoint{0.000000in}{0.000000in}}{%
\pgfpathmoveto{\pgfqpoint{0.000000in}{0.000000in}}%
\pgfpathlineto{\pgfqpoint{0.000000in}{-0.048611in}}%
\pgfusepath{stroke,fill}%
}%
\begin{pgfscope}%
\pgfsys@transformshift{3.368409in}{0.499444in}%
\pgfsys@useobject{currentmarker}{}%
\end{pgfscope}%
\end{pgfscope}%
\begin{pgfscope}%
\pgfsetbuttcap%
\pgfsetroundjoin%
\definecolor{currentfill}{rgb}{0.000000,0.000000,0.000000}%
\pgfsetfillcolor{currentfill}%
\pgfsetlinewidth{0.803000pt}%
\definecolor{currentstroke}{rgb}{0.000000,0.000000,0.000000}%
\pgfsetstrokecolor{currentstroke}%
\pgfsetdash{}{0pt}%
\pgfsys@defobject{currentmarker}{\pgfqpoint{0.000000in}{-0.048611in}}{\pgfqpoint{0.000000in}{0.000000in}}{%
\pgfpathmoveto{\pgfqpoint{0.000000in}{0.000000in}}%
\pgfpathlineto{\pgfqpoint{0.000000in}{-0.048611in}}%
\pgfusepath{stroke,fill}%
}%
\begin{pgfscope}%
\pgfsys@transformshift{3.526932in}{0.499444in}%
\pgfsys@useobject{currentmarker}{}%
\end{pgfscope}%
\end{pgfscope}%
\begin{pgfscope}%
\definecolor{textcolor}{rgb}{0.000000,0.000000,0.000000}%
\pgfsetstrokecolor{textcolor}%
\pgfsetfillcolor{textcolor}%
\pgftext[x=3.526932in,y=0.402222in,,top]{\color{textcolor}\rmfamily\fontsize{10.000000}{12.000000}\selectfont 0.9}%
\end{pgfscope}%
\begin{pgfscope}%
\pgfsetbuttcap%
\pgfsetroundjoin%
\definecolor{currentfill}{rgb}{0.000000,0.000000,0.000000}%
\pgfsetfillcolor{currentfill}%
\pgfsetlinewidth{0.803000pt}%
\definecolor{currentstroke}{rgb}{0.000000,0.000000,0.000000}%
\pgfsetstrokecolor{currentstroke}%
\pgfsetdash{}{0pt}%
\pgfsys@defobject{currentmarker}{\pgfqpoint{0.000000in}{-0.048611in}}{\pgfqpoint{0.000000in}{0.000000in}}{%
\pgfpathmoveto{\pgfqpoint{0.000000in}{0.000000in}}%
\pgfpathlineto{\pgfqpoint{0.000000in}{-0.048611in}}%
\pgfusepath{stroke,fill}%
}%
\begin{pgfscope}%
\pgfsys@transformshift{3.685455in}{0.499444in}%
\pgfsys@useobject{currentmarker}{}%
\end{pgfscope}%
\end{pgfscope}%
\begin{pgfscope}%
\pgfsetbuttcap%
\pgfsetroundjoin%
\definecolor{currentfill}{rgb}{0.000000,0.000000,0.000000}%
\pgfsetfillcolor{currentfill}%
\pgfsetlinewidth{0.803000pt}%
\definecolor{currentstroke}{rgb}{0.000000,0.000000,0.000000}%
\pgfsetstrokecolor{currentstroke}%
\pgfsetdash{}{0pt}%
\pgfsys@defobject{currentmarker}{\pgfqpoint{0.000000in}{-0.048611in}}{\pgfqpoint{0.000000in}{0.000000in}}{%
\pgfpathmoveto{\pgfqpoint{0.000000in}{0.000000in}}%
\pgfpathlineto{\pgfqpoint{0.000000in}{-0.048611in}}%
\pgfusepath{stroke,fill}%
}%
\begin{pgfscope}%
\pgfsys@transformshift{3.843978in}{0.499444in}%
\pgfsys@useobject{currentmarker}{}%
\end{pgfscope}%
\end{pgfscope}%
\begin{pgfscope}%
\definecolor{textcolor}{rgb}{0.000000,0.000000,0.000000}%
\pgfsetstrokecolor{textcolor}%
\pgfsetfillcolor{textcolor}%
\pgftext[x=3.843978in,y=0.402222in,,top]{\color{textcolor}\rmfamily\fontsize{10.000000}{12.000000}\selectfont 1.0}%
\end{pgfscope}%
\begin{pgfscope}%
\pgfsetbuttcap%
\pgfsetroundjoin%
\definecolor{currentfill}{rgb}{0.000000,0.000000,0.000000}%
\pgfsetfillcolor{currentfill}%
\pgfsetlinewidth{0.803000pt}%
\definecolor{currentstroke}{rgb}{0.000000,0.000000,0.000000}%
\pgfsetstrokecolor{currentstroke}%
\pgfsetdash{}{0pt}%
\pgfsys@defobject{currentmarker}{\pgfqpoint{0.000000in}{-0.048611in}}{\pgfqpoint{0.000000in}{0.000000in}}{%
\pgfpathmoveto{\pgfqpoint{0.000000in}{0.000000in}}%
\pgfpathlineto{\pgfqpoint{0.000000in}{-0.048611in}}%
\pgfusepath{stroke,fill}%
}%
\begin{pgfscope}%
\pgfsys@transformshift{4.002500in}{0.499444in}%
\pgfsys@useobject{currentmarker}{}%
\end{pgfscope}%
\end{pgfscope}%
\begin{pgfscope}%
\definecolor{textcolor}{rgb}{0.000000,0.000000,0.000000}%
\pgfsetstrokecolor{textcolor}%
\pgfsetfillcolor{textcolor}%
\pgftext[x=2.258750in,y=0.223333in,,top]{\color{textcolor}\rmfamily\fontsize{10.000000}{12.000000}\selectfont \(\displaystyle p\)}%
\end{pgfscope}%
\begin{pgfscope}%
\pgfsetbuttcap%
\pgfsetroundjoin%
\definecolor{currentfill}{rgb}{0.000000,0.000000,0.000000}%
\pgfsetfillcolor{currentfill}%
\pgfsetlinewidth{0.803000pt}%
\definecolor{currentstroke}{rgb}{0.000000,0.000000,0.000000}%
\pgfsetstrokecolor{currentstroke}%
\pgfsetdash{}{0pt}%
\pgfsys@defobject{currentmarker}{\pgfqpoint{-0.048611in}{0.000000in}}{\pgfqpoint{-0.000000in}{0.000000in}}{%
\pgfpathmoveto{\pgfqpoint{-0.000000in}{0.000000in}}%
\pgfpathlineto{\pgfqpoint{-0.048611in}{0.000000in}}%
\pgfusepath{stroke,fill}%
}%
\begin{pgfscope}%
\pgfsys@transformshift{0.515000in}{0.499444in}%
\pgfsys@useobject{currentmarker}{}%
\end{pgfscope}%
\end{pgfscope}%
\begin{pgfscope}%
\definecolor{textcolor}{rgb}{0.000000,0.000000,0.000000}%
\pgfsetstrokecolor{textcolor}%
\pgfsetfillcolor{textcolor}%
\pgftext[x=0.348333in, y=0.451250in, left, base]{\color{textcolor}\rmfamily\fontsize{10.000000}{12.000000}\selectfont \(\displaystyle {0}\)}%
\end{pgfscope}%
\begin{pgfscope}%
\pgfsetbuttcap%
\pgfsetroundjoin%
\definecolor{currentfill}{rgb}{0.000000,0.000000,0.000000}%
\pgfsetfillcolor{currentfill}%
\pgfsetlinewidth{0.803000pt}%
\definecolor{currentstroke}{rgb}{0.000000,0.000000,0.000000}%
\pgfsetstrokecolor{currentstroke}%
\pgfsetdash{}{0pt}%
\pgfsys@defobject{currentmarker}{\pgfqpoint{-0.048611in}{0.000000in}}{\pgfqpoint{-0.000000in}{0.000000in}}{%
\pgfpathmoveto{\pgfqpoint{-0.000000in}{0.000000in}}%
\pgfpathlineto{\pgfqpoint{-0.048611in}{0.000000in}}%
\pgfusepath{stroke,fill}%
}%
\begin{pgfscope}%
\pgfsys@transformshift{0.515000in}{0.831723in}%
\pgfsys@useobject{currentmarker}{}%
\end{pgfscope}%
\end{pgfscope}%
\begin{pgfscope}%
\definecolor{textcolor}{rgb}{0.000000,0.000000,0.000000}%
\pgfsetstrokecolor{textcolor}%
\pgfsetfillcolor{textcolor}%
\pgftext[x=0.278889in, y=0.783528in, left, base]{\color{textcolor}\rmfamily\fontsize{10.000000}{12.000000}\selectfont \(\displaystyle {25}\)}%
\end{pgfscope}%
\begin{pgfscope}%
\pgfsetbuttcap%
\pgfsetroundjoin%
\definecolor{currentfill}{rgb}{0.000000,0.000000,0.000000}%
\pgfsetfillcolor{currentfill}%
\pgfsetlinewidth{0.803000pt}%
\definecolor{currentstroke}{rgb}{0.000000,0.000000,0.000000}%
\pgfsetstrokecolor{currentstroke}%
\pgfsetdash{}{0pt}%
\pgfsys@defobject{currentmarker}{\pgfqpoint{-0.048611in}{0.000000in}}{\pgfqpoint{-0.000000in}{0.000000in}}{%
\pgfpathmoveto{\pgfqpoint{-0.000000in}{0.000000in}}%
\pgfpathlineto{\pgfqpoint{-0.048611in}{0.000000in}}%
\pgfusepath{stroke,fill}%
}%
\begin{pgfscope}%
\pgfsys@transformshift{0.515000in}{1.164001in}%
\pgfsys@useobject{currentmarker}{}%
\end{pgfscope}%
\end{pgfscope}%
\begin{pgfscope}%
\definecolor{textcolor}{rgb}{0.000000,0.000000,0.000000}%
\pgfsetstrokecolor{textcolor}%
\pgfsetfillcolor{textcolor}%
\pgftext[x=0.278889in, y=1.115806in, left, base]{\color{textcolor}\rmfamily\fontsize{10.000000}{12.000000}\selectfont \(\displaystyle {50}\)}%
\end{pgfscope}%
\begin{pgfscope}%
\pgfsetbuttcap%
\pgfsetroundjoin%
\definecolor{currentfill}{rgb}{0.000000,0.000000,0.000000}%
\pgfsetfillcolor{currentfill}%
\pgfsetlinewidth{0.803000pt}%
\definecolor{currentstroke}{rgb}{0.000000,0.000000,0.000000}%
\pgfsetstrokecolor{currentstroke}%
\pgfsetdash{}{0pt}%
\pgfsys@defobject{currentmarker}{\pgfqpoint{-0.048611in}{0.000000in}}{\pgfqpoint{-0.000000in}{0.000000in}}{%
\pgfpathmoveto{\pgfqpoint{-0.000000in}{0.000000in}}%
\pgfpathlineto{\pgfqpoint{-0.048611in}{0.000000in}}%
\pgfusepath{stroke,fill}%
}%
\begin{pgfscope}%
\pgfsys@transformshift{0.515000in}{1.496279in}%
\pgfsys@useobject{currentmarker}{}%
\end{pgfscope}%
\end{pgfscope}%
\begin{pgfscope}%
\definecolor{textcolor}{rgb}{0.000000,0.000000,0.000000}%
\pgfsetstrokecolor{textcolor}%
\pgfsetfillcolor{textcolor}%
\pgftext[x=0.278889in, y=1.448085in, left, base]{\color{textcolor}\rmfamily\fontsize{10.000000}{12.000000}\selectfont \(\displaystyle {75}\)}%
\end{pgfscope}%
\begin{pgfscope}%
\definecolor{textcolor}{rgb}{0.000000,0.000000,0.000000}%
\pgfsetstrokecolor{textcolor}%
\pgfsetfillcolor{textcolor}%
\pgftext[x=0.223333in,y=1.076944in,,bottom,rotate=90.000000]{\color{textcolor}\rmfamily\fontsize{10.000000}{12.000000}\selectfont Percent of Data Set}%
\end{pgfscope}%
\begin{pgfscope}%
\pgfsetrectcap%
\pgfsetmiterjoin%
\pgfsetlinewidth{0.803000pt}%
\definecolor{currentstroke}{rgb}{0.000000,0.000000,0.000000}%
\pgfsetstrokecolor{currentstroke}%
\pgfsetdash{}{0pt}%
\pgfpathmoveto{\pgfqpoint{0.515000in}{0.499444in}}%
\pgfpathlineto{\pgfqpoint{0.515000in}{1.654444in}}%
\pgfusepath{stroke}%
\end{pgfscope}%
\begin{pgfscope}%
\pgfsetrectcap%
\pgfsetmiterjoin%
\pgfsetlinewidth{0.803000pt}%
\definecolor{currentstroke}{rgb}{0.000000,0.000000,0.000000}%
\pgfsetstrokecolor{currentstroke}%
\pgfsetdash{}{0pt}%
\pgfpathmoveto{\pgfqpoint{4.002500in}{0.499444in}}%
\pgfpathlineto{\pgfqpoint{4.002500in}{1.654444in}}%
\pgfusepath{stroke}%
\end{pgfscope}%
\begin{pgfscope}%
\pgfsetrectcap%
\pgfsetmiterjoin%
\pgfsetlinewidth{0.803000pt}%
\definecolor{currentstroke}{rgb}{0.000000,0.000000,0.000000}%
\pgfsetstrokecolor{currentstroke}%
\pgfsetdash{}{0pt}%
\pgfpathmoveto{\pgfqpoint{0.515000in}{0.499444in}}%
\pgfpathlineto{\pgfqpoint{4.002500in}{0.499444in}}%
\pgfusepath{stroke}%
\end{pgfscope}%
\begin{pgfscope}%
\pgfsetrectcap%
\pgfsetmiterjoin%
\pgfsetlinewidth{0.803000pt}%
\definecolor{currentstroke}{rgb}{0.000000,0.000000,0.000000}%
\pgfsetstrokecolor{currentstroke}%
\pgfsetdash{}{0pt}%
\pgfpathmoveto{\pgfqpoint{0.515000in}{1.654444in}}%
\pgfpathlineto{\pgfqpoint{4.002500in}{1.654444in}}%
\pgfusepath{stroke}%
\end{pgfscope}%
\begin{pgfscope}%
\pgfsetbuttcap%
\pgfsetmiterjoin%
\definecolor{currentfill}{rgb}{1.000000,1.000000,1.000000}%
\pgfsetfillcolor{currentfill}%
\pgfsetfillopacity{0.800000}%
\pgfsetlinewidth{1.003750pt}%
\definecolor{currentstroke}{rgb}{0.800000,0.800000,0.800000}%
\pgfsetstrokecolor{currentstroke}%
\pgfsetstrokeopacity{0.800000}%
\pgfsetdash{}{0pt}%
\pgfpathmoveto{\pgfqpoint{3.225556in}{1.154445in}}%
\pgfpathlineto{\pgfqpoint{3.905278in}{1.154445in}}%
\pgfpathquadraticcurveto{\pgfqpoint{3.933056in}{1.154445in}}{\pgfqpoint{3.933056in}{1.182222in}}%
\pgfpathlineto{\pgfqpoint{3.933056in}{1.557222in}}%
\pgfpathquadraticcurveto{\pgfqpoint{3.933056in}{1.585000in}}{\pgfqpoint{3.905278in}{1.585000in}}%
\pgfpathlineto{\pgfqpoint{3.225556in}{1.585000in}}%
\pgfpathquadraticcurveto{\pgfqpoint{3.197778in}{1.585000in}}{\pgfqpoint{3.197778in}{1.557222in}}%
\pgfpathlineto{\pgfqpoint{3.197778in}{1.182222in}}%
\pgfpathquadraticcurveto{\pgfqpoint{3.197778in}{1.154445in}}{\pgfqpoint{3.225556in}{1.154445in}}%
\pgfpathlineto{\pgfqpoint{3.225556in}{1.154445in}}%
\pgfpathclose%
\pgfusepath{stroke,fill}%
\end{pgfscope}%
\begin{pgfscope}%
\pgfsetbuttcap%
\pgfsetmiterjoin%
\pgfsetlinewidth{1.003750pt}%
\definecolor{currentstroke}{rgb}{0.000000,0.000000,0.000000}%
\pgfsetstrokecolor{currentstroke}%
\pgfsetdash{}{0pt}%
\pgfpathmoveto{\pgfqpoint{3.253334in}{1.432222in}}%
\pgfpathlineto{\pgfqpoint{3.531111in}{1.432222in}}%
\pgfpathlineto{\pgfqpoint{3.531111in}{1.529444in}}%
\pgfpathlineto{\pgfqpoint{3.253334in}{1.529444in}}%
\pgfpathlineto{\pgfqpoint{3.253334in}{1.432222in}}%
\pgfpathclose%
\pgfusepath{stroke}%
\end{pgfscope}%
\begin{pgfscope}%
\definecolor{textcolor}{rgb}{0.000000,0.000000,0.000000}%
\pgfsetstrokecolor{textcolor}%
\pgfsetfillcolor{textcolor}%
\pgftext[x=3.642223in,y=1.432222in,left,base]{\color{textcolor}\rmfamily\fontsize{10.000000}{12.000000}\selectfont Neg}%
\end{pgfscope}%
\begin{pgfscope}%
\pgfsetbuttcap%
\pgfsetmiterjoin%
\definecolor{currentfill}{rgb}{0.000000,0.000000,0.000000}%
\pgfsetfillcolor{currentfill}%
\pgfsetlinewidth{0.000000pt}%
\definecolor{currentstroke}{rgb}{0.000000,0.000000,0.000000}%
\pgfsetstrokecolor{currentstroke}%
\pgfsetstrokeopacity{0.000000}%
\pgfsetdash{}{0pt}%
\pgfpathmoveto{\pgfqpoint{3.253334in}{1.236944in}}%
\pgfpathlineto{\pgfqpoint{3.531111in}{1.236944in}}%
\pgfpathlineto{\pgfqpoint{3.531111in}{1.334167in}}%
\pgfpathlineto{\pgfqpoint{3.253334in}{1.334167in}}%
\pgfpathlineto{\pgfqpoint{3.253334in}{1.236944in}}%
\pgfpathclose%
\pgfusepath{fill}%
\end{pgfscope}%
\begin{pgfscope}%
\definecolor{textcolor}{rgb}{0.000000,0.000000,0.000000}%
\pgfsetstrokecolor{textcolor}%
\pgfsetfillcolor{textcolor}%
\pgftext[x=3.642223in,y=1.236944in,left,base]{\color{textcolor}\rmfamily\fontsize{10.000000}{12.000000}\selectfont Pos}%
\end{pgfscope}%
\end{pgfpicture}%
\makeatother%
\endgroup%
	
&
	\vskip 0pt
	\hfil ROC Curve
	
	%% Creator: Matplotlib, PGF backend
%%
%% To include the figure in your LaTeX document, write
%%   \input{<filename>.pgf}
%%
%% Make sure the required packages are loaded in your preamble
%%   \usepackage{pgf}
%%
%% Also ensure that all the required font packages are loaded; for instance,
%% the lmodern package is sometimes necessary when using math font.
%%   \usepackage{lmodern}
%%
%% Figures using additional raster images can only be included by \input if
%% they are in the same directory as the main LaTeX file. For loading figures
%% from other directories you can use the `import` package
%%   \usepackage{import}
%%
%% and then include the figures with
%%   \import{<path to file>}{<filename>.pgf}
%%
%% Matplotlib used the following preamble
%%   
%%   \usepackage{fontspec}
%%   \makeatletter\@ifpackageloaded{underscore}{}{\usepackage[strings]{underscore}}\makeatother
%%
\begingroup%
\makeatletter%
\begin{pgfpicture}%
\pgfpathrectangle{\pgfpointorigin}{\pgfqpoint{2.221861in}{1.754444in}}%
\pgfusepath{use as bounding box, clip}%
\begin{pgfscope}%
\pgfsetbuttcap%
\pgfsetmiterjoin%
\definecolor{currentfill}{rgb}{1.000000,1.000000,1.000000}%
\pgfsetfillcolor{currentfill}%
\pgfsetlinewidth{0.000000pt}%
\definecolor{currentstroke}{rgb}{1.000000,1.000000,1.000000}%
\pgfsetstrokecolor{currentstroke}%
\pgfsetdash{}{0pt}%
\pgfpathmoveto{\pgfqpoint{0.000000in}{0.000000in}}%
\pgfpathlineto{\pgfqpoint{2.221861in}{0.000000in}}%
\pgfpathlineto{\pgfqpoint{2.221861in}{1.754444in}}%
\pgfpathlineto{\pgfqpoint{0.000000in}{1.754444in}}%
\pgfpathlineto{\pgfqpoint{0.000000in}{0.000000in}}%
\pgfpathclose%
\pgfusepath{fill}%
\end{pgfscope}%
\begin{pgfscope}%
\pgfsetbuttcap%
\pgfsetmiterjoin%
\definecolor{currentfill}{rgb}{1.000000,1.000000,1.000000}%
\pgfsetfillcolor{currentfill}%
\pgfsetlinewidth{0.000000pt}%
\definecolor{currentstroke}{rgb}{0.000000,0.000000,0.000000}%
\pgfsetstrokecolor{currentstroke}%
\pgfsetstrokeopacity{0.000000}%
\pgfsetdash{}{0pt}%
\pgfpathmoveto{\pgfqpoint{0.553581in}{0.499444in}}%
\pgfpathlineto{\pgfqpoint{2.103581in}{0.499444in}}%
\pgfpathlineto{\pgfqpoint{2.103581in}{1.654444in}}%
\pgfpathlineto{\pgfqpoint{0.553581in}{1.654444in}}%
\pgfpathlineto{\pgfqpoint{0.553581in}{0.499444in}}%
\pgfpathclose%
\pgfusepath{fill}%
\end{pgfscope}%
\begin{pgfscope}%
\pgfsetbuttcap%
\pgfsetroundjoin%
\definecolor{currentfill}{rgb}{0.000000,0.000000,0.000000}%
\pgfsetfillcolor{currentfill}%
\pgfsetlinewidth{0.803000pt}%
\definecolor{currentstroke}{rgb}{0.000000,0.000000,0.000000}%
\pgfsetstrokecolor{currentstroke}%
\pgfsetdash{}{0pt}%
\pgfsys@defobject{currentmarker}{\pgfqpoint{0.000000in}{-0.048611in}}{\pgfqpoint{0.000000in}{0.000000in}}{%
\pgfpathmoveto{\pgfqpoint{0.000000in}{0.000000in}}%
\pgfpathlineto{\pgfqpoint{0.000000in}{-0.048611in}}%
\pgfusepath{stroke,fill}%
}%
\begin{pgfscope}%
\pgfsys@transformshift{0.624035in}{0.499444in}%
\pgfsys@useobject{currentmarker}{}%
\end{pgfscope}%
\end{pgfscope}%
\begin{pgfscope}%
\definecolor{textcolor}{rgb}{0.000000,0.000000,0.000000}%
\pgfsetstrokecolor{textcolor}%
\pgfsetfillcolor{textcolor}%
\pgftext[x=0.624035in,y=0.402222in,,top]{\color{textcolor}\rmfamily\fontsize{10.000000}{12.000000}\selectfont \(\displaystyle {0.0}\)}%
\end{pgfscope}%
\begin{pgfscope}%
\pgfsetbuttcap%
\pgfsetroundjoin%
\definecolor{currentfill}{rgb}{0.000000,0.000000,0.000000}%
\pgfsetfillcolor{currentfill}%
\pgfsetlinewidth{0.803000pt}%
\definecolor{currentstroke}{rgb}{0.000000,0.000000,0.000000}%
\pgfsetstrokecolor{currentstroke}%
\pgfsetdash{}{0pt}%
\pgfsys@defobject{currentmarker}{\pgfqpoint{0.000000in}{-0.048611in}}{\pgfqpoint{0.000000in}{0.000000in}}{%
\pgfpathmoveto{\pgfqpoint{0.000000in}{0.000000in}}%
\pgfpathlineto{\pgfqpoint{0.000000in}{-0.048611in}}%
\pgfusepath{stroke,fill}%
}%
\begin{pgfscope}%
\pgfsys@transformshift{1.328581in}{0.499444in}%
\pgfsys@useobject{currentmarker}{}%
\end{pgfscope}%
\end{pgfscope}%
\begin{pgfscope}%
\definecolor{textcolor}{rgb}{0.000000,0.000000,0.000000}%
\pgfsetstrokecolor{textcolor}%
\pgfsetfillcolor{textcolor}%
\pgftext[x=1.328581in,y=0.402222in,,top]{\color{textcolor}\rmfamily\fontsize{10.000000}{12.000000}\selectfont \(\displaystyle {0.5}\)}%
\end{pgfscope}%
\begin{pgfscope}%
\pgfsetbuttcap%
\pgfsetroundjoin%
\definecolor{currentfill}{rgb}{0.000000,0.000000,0.000000}%
\pgfsetfillcolor{currentfill}%
\pgfsetlinewidth{0.803000pt}%
\definecolor{currentstroke}{rgb}{0.000000,0.000000,0.000000}%
\pgfsetstrokecolor{currentstroke}%
\pgfsetdash{}{0pt}%
\pgfsys@defobject{currentmarker}{\pgfqpoint{0.000000in}{-0.048611in}}{\pgfqpoint{0.000000in}{0.000000in}}{%
\pgfpathmoveto{\pgfqpoint{0.000000in}{0.000000in}}%
\pgfpathlineto{\pgfqpoint{0.000000in}{-0.048611in}}%
\pgfusepath{stroke,fill}%
}%
\begin{pgfscope}%
\pgfsys@transformshift{2.033126in}{0.499444in}%
\pgfsys@useobject{currentmarker}{}%
\end{pgfscope}%
\end{pgfscope}%
\begin{pgfscope}%
\definecolor{textcolor}{rgb}{0.000000,0.000000,0.000000}%
\pgfsetstrokecolor{textcolor}%
\pgfsetfillcolor{textcolor}%
\pgftext[x=2.033126in,y=0.402222in,,top]{\color{textcolor}\rmfamily\fontsize{10.000000}{12.000000}\selectfont \(\displaystyle {1.0}\)}%
\end{pgfscope}%
\begin{pgfscope}%
\definecolor{textcolor}{rgb}{0.000000,0.000000,0.000000}%
\pgfsetstrokecolor{textcolor}%
\pgfsetfillcolor{textcolor}%
\pgftext[x=1.328581in,y=0.223333in,,top]{\color{textcolor}\rmfamily\fontsize{10.000000}{12.000000}\selectfont False positive rate}%
\end{pgfscope}%
\begin{pgfscope}%
\pgfsetbuttcap%
\pgfsetroundjoin%
\definecolor{currentfill}{rgb}{0.000000,0.000000,0.000000}%
\pgfsetfillcolor{currentfill}%
\pgfsetlinewidth{0.803000pt}%
\definecolor{currentstroke}{rgb}{0.000000,0.000000,0.000000}%
\pgfsetstrokecolor{currentstroke}%
\pgfsetdash{}{0pt}%
\pgfsys@defobject{currentmarker}{\pgfqpoint{-0.048611in}{0.000000in}}{\pgfqpoint{-0.000000in}{0.000000in}}{%
\pgfpathmoveto{\pgfqpoint{-0.000000in}{0.000000in}}%
\pgfpathlineto{\pgfqpoint{-0.048611in}{0.000000in}}%
\pgfusepath{stroke,fill}%
}%
\begin{pgfscope}%
\pgfsys@transformshift{0.553581in}{0.551944in}%
\pgfsys@useobject{currentmarker}{}%
\end{pgfscope}%
\end{pgfscope}%
\begin{pgfscope}%
\definecolor{textcolor}{rgb}{0.000000,0.000000,0.000000}%
\pgfsetstrokecolor{textcolor}%
\pgfsetfillcolor{textcolor}%
\pgftext[x=0.278889in, y=0.503750in, left, base]{\color{textcolor}\rmfamily\fontsize{10.000000}{12.000000}\selectfont \(\displaystyle {0.0}\)}%
\end{pgfscope}%
\begin{pgfscope}%
\pgfsetbuttcap%
\pgfsetroundjoin%
\definecolor{currentfill}{rgb}{0.000000,0.000000,0.000000}%
\pgfsetfillcolor{currentfill}%
\pgfsetlinewidth{0.803000pt}%
\definecolor{currentstroke}{rgb}{0.000000,0.000000,0.000000}%
\pgfsetstrokecolor{currentstroke}%
\pgfsetdash{}{0pt}%
\pgfsys@defobject{currentmarker}{\pgfqpoint{-0.048611in}{0.000000in}}{\pgfqpoint{-0.000000in}{0.000000in}}{%
\pgfpathmoveto{\pgfqpoint{-0.000000in}{0.000000in}}%
\pgfpathlineto{\pgfqpoint{-0.048611in}{0.000000in}}%
\pgfusepath{stroke,fill}%
}%
\begin{pgfscope}%
\pgfsys@transformshift{0.553581in}{1.076944in}%
\pgfsys@useobject{currentmarker}{}%
\end{pgfscope}%
\end{pgfscope}%
\begin{pgfscope}%
\definecolor{textcolor}{rgb}{0.000000,0.000000,0.000000}%
\pgfsetstrokecolor{textcolor}%
\pgfsetfillcolor{textcolor}%
\pgftext[x=0.278889in, y=1.028750in, left, base]{\color{textcolor}\rmfamily\fontsize{10.000000}{12.000000}\selectfont \(\displaystyle {0.5}\)}%
\end{pgfscope}%
\begin{pgfscope}%
\pgfsetbuttcap%
\pgfsetroundjoin%
\definecolor{currentfill}{rgb}{0.000000,0.000000,0.000000}%
\pgfsetfillcolor{currentfill}%
\pgfsetlinewidth{0.803000pt}%
\definecolor{currentstroke}{rgb}{0.000000,0.000000,0.000000}%
\pgfsetstrokecolor{currentstroke}%
\pgfsetdash{}{0pt}%
\pgfsys@defobject{currentmarker}{\pgfqpoint{-0.048611in}{0.000000in}}{\pgfqpoint{-0.000000in}{0.000000in}}{%
\pgfpathmoveto{\pgfqpoint{-0.000000in}{0.000000in}}%
\pgfpathlineto{\pgfqpoint{-0.048611in}{0.000000in}}%
\pgfusepath{stroke,fill}%
}%
\begin{pgfscope}%
\pgfsys@transformshift{0.553581in}{1.601944in}%
\pgfsys@useobject{currentmarker}{}%
\end{pgfscope}%
\end{pgfscope}%
\begin{pgfscope}%
\definecolor{textcolor}{rgb}{0.000000,0.000000,0.000000}%
\pgfsetstrokecolor{textcolor}%
\pgfsetfillcolor{textcolor}%
\pgftext[x=0.278889in, y=1.553750in, left, base]{\color{textcolor}\rmfamily\fontsize{10.000000}{12.000000}\selectfont \(\displaystyle {1.0}\)}%
\end{pgfscope}%
\begin{pgfscope}%
\definecolor{textcolor}{rgb}{0.000000,0.000000,0.000000}%
\pgfsetstrokecolor{textcolor}%
\pgfsetfillcolor{textcolor}%
\pgftext[x=0.223333in,y=1.076944in,,bottom,rotate=90.000000]{\color{textcolor}\rmfamily\fontsize{10.000000}{12.000000}\selectfont True positive rate}%
\end{pgfscope}%
\begin{pgfscope}%
\pgfpathrectangle{\pgfqpoint{0.553581in}{0.499444in}}{\pgfqpoint{1.550000in}{1.155000in}}%
\pgfusepath{clip}%
\pgfsetbuttcap%
\pgfsetroundjoin%
\pgfsetlinewidth{1.505625pt}%
\definecolor{currentstroke}{rgb}{0.000000,0.000000,0.000000}%
\pgfsetstrokecolor{currentstroke}%
\pgfsetdash{{5.550000pt}{2.400000pt}}{0.000000pt}%
\pgfpathmoveto{\pgfqpoint{0.624035in}{0.551944in}}%
\pgfpathlineto{\pgfqpoint{2.033126in}{1.601944in}}%
\pgfusepath{stroke}%
\end{pgfscope}%
\begin{pgfscope}%
\pgfpathrectangle{\pgfqpoint{0.553581in}{0.499444in}}{\pgfqpoint{1.550000in}{1.155000in}}%
\pgfusepath{clip}%
\pgfsetrectcap%
\pgfsetroundjoin%
\pgfsetlinewidth{1.505625pt}%
\definecolor{currentstroke}{rgb}{0.000000,0.000000,0.000000}%
\pgfsetstrokecolor{currentstroke}%
\pgfsetdash{}{0pt}%
\pgfpathmoveto{\pgfqpoint{0.624035in}{0.551944in}}%
\pgfpathlineto{\pgfqpoint{0.624793in}{0.559478in}}%
\pgfpathlineto{\pgfqpoint{0.625893in}{0.567803in}}%
\pgfpathlineto{\pgfqpoint{0.626087in}{0.568893in}}%
\pgfpathlineto{\pgfqpoint{0.627197in}{0.575952in}}%
\pgfpathlineto{\pgfqpoint{0.627422in}{0.577051in}}%
\pgfpathlineto{\pgfqpoint{0.628529in}{0.583635in}}%
\pgfpathlineto{\pgfqpoint{0.628745in}{0.584734in}}%
\pgfpathlineto{\pgfqpoint{0.629849in}{0.590340in}}%
\pgfpathlineto{\pgfqpoint{0.630110in}{0.591439in}}%
\pgfpathlineto{\pgfqpoint{0.631219in}{0.597324in}}%
\pgfpathlineto{\pgfqpoint{0.631407in}{0.598423in}}%
\pgfpathlineto{\pgfqpoint{0.632516in}{0.603331in}}%
\pgfpathlineto{\pgfqpoint{0.632746in}{0.604439in}}%
\pgfpathlineto{\pgfqpoint{0.633843in}{0.609365in}}%
\pgfpathlineto{\pgfqpoint{0.634143in}{0.610464in}}%
\pgfpathlineto{\pgfqpoint{0.635243in}{0.614878in}}%
\pgfpathlineto{\pgfqpoint{0.635497in}{0.615987in}}%
\pgfpathlineto{\pgfqpoint{0.636606in}{0.620438in}}%
\pgfpathlineto{\pgfqpoint{0.636902in}{0.621546in}}%
\pgfpathlineto{\pgfqpoint{0.638011in}{0.626463in}}%
\pgfpathlineto{\pgfqpoint{0.638259in}{0.627562in}}%
\pgfpathlineto{\pgfqpoint{0.639364in}{0.632312in}}%
\pgfpathlineto{\pgfqpoint{0.639624in}{0.633410in}}%
\pgfpathlineto{\pgfqpoint{0.640729in}{0.637974in}}%
\pgfpathlineto{\pgfqpoint{0.641083in}{0.639063in}}%
\pgfpathlineto{\pgfqpoint{0.642190in}{0.643505in}}%
\pgfpathlineto{\pgfqpoint{0.642439in}{0.644604in}}%
\pgfpathlineto{\pgfqpoint{0.643543in}{0.648832in}}%
\pgfpathlineto{\pgfqpoint{0.643886in}{0.649940in}}%
\pgfpathlineto{\pgfqpoint{0.644986in}{0.653619in}}%
\pgfpathlineto{\pgfqpoint{0.645267in}{0.654680in}}%
\pgfpathlineto{\pgfqpoint{0.646372in}{0.659132in}}%
\pgfpathlineto{\pgfqpoint{0.646667in}{0.660231in}}%
\pgfpathlineto{\pgfqpoint{0.647772in}{0.664170in}}%
\pgfpathlineto{\pgfqpoint{0.648140in}{0.665269in}}%
\pgfpathlineto{\pgfqpoint{0.649245in}{0.668919in}}%
\pgfpathlineto{\pgfqpoint{0.649571in}{0.670018in}}%
\pgfpathlineto{\pgfqpoint{0.650678in}{0.673929in}}%
\pgfpathlineto{\pgfqpoint{0.650943in}{0.675038in}}%
\pgfpathlineto{\pgfqpoint{0.652052in}{0.679172in}}%
\pgfpathlineto{\pgfqpoint{0.652390in}{0.680281in}}%
\pgfpathlineto{\pgfqpoint{0.653483in}{0.684182in}}%
\pgfpathlineto{\pgfqpoint{0.653832in}{0.685281in}}%
\pgfpathlineto{\pgfqpoint{0.654942in}{0.689332in}}%
\pgfpathlineto{\pgfqpoint{0.655246in}{0.690440in}}%
\pgfpathlineto{\pgfqpoint{0.656353in}{0.694166in}}%
\pgfpathlineto{\pgfqpoint{0.656694in}{0.695274in}}%
\pgfpathlineto{\pgfqpoint{0.657798in}{0.698310in}}%
\pgfpathlineto{\pgfqpoint{0.657803in}{0.698310in}}%
\pgfpathlineto{\pgfqpoint{0.658190in}{0.699418in}}%
\pgfpathlineto{\pgfqpoint{0.659299in}{0.703404in}}%
\pgfpathlineto{\pgfqpoint{0.659684in}{0.704512in}}%
\pgfpathlineto{\pgfqpoint{0.660793in}{0.707892in}}%
\pgfpathlineto{\pgfqpoint{0.661098in}{0.709000in}}%
\pgfpathlineto{\pgfqpoint{0.662205in}{0.712856in}}%
\pgfpathlineto{\pgfqpoint{0.662536in}{0.713955in}}%
\pgfpathlineto{\pgfqpoint{0.663645in}{0.717587in}}%
\pgfpathlineto{\pgfqpoint{0.664084in}{0.718685in}}%
\pgfpathlineto{\pgfqpoint{0.665193in}{0.721908in}}%
\pgfpathlineto{\pgfqpoint{0.665474in}{0.723016in}}%
\pgfpathlineto{\pgfqpoint{0.666584in}{0.726061in}}%
\pgfpathlineto{\pgfqpoint{0.666896in}{0.727160in}}%
\pgfpathlineto{\pgfqpoint{0.668003in}{0.730354in}}%
\pgfpathlineto{\pgfqpoint{0.668350in}{0.731462in}}%
\pgfpathlineto{\pgfqpoint{0.669452in}{0.734750in}}%
\pgfpathlineto{\pgfqpoint{0.669893in}{0.735858in}}%
\pgfpathlineto{\pgfqpoint{0.670986in}{0.738931in}}%
\pgfpathlineto{\pgfqpoint{0.671382in}{0.740030in}}%
\pgfpathlineto{\pgfqpoint{0.672489in}{0.743233in}}%
\pgfpathlineto{\pgfqpoint{0.672817in}{0.744342in}}%
\pgfpathlineto{\pgfqpoint{0.673927in}{0.746968in}}%
\pgfpathlineto{\pgfqpoint{0.674311in}{0.748076in}}%
\pgfpathlineto{\pgfqpoint{0.675414in}{0.751065in}}%
\pgfpathlineto{\pgfqpoint{0.675862in}{0.752173in}}%
\pgfpathlineto{\pgfqpoint{0.676964in}{0.755284in}}%
\pgfpathlineto{\pgfqpoint{0.677316in}{0.756392in}}%
\pgfpathlineto{\pgfqpoint{0.678413in}{0.759307in}}%
\pgfpathlineto{\pgfqpoint{0.678868in}{0.760415in}}%
\pgfpathlineto{\pgfqpoint{0.679978in}{0.763078in}}%
\pgfpathlineto{\pgfqpoint{0.680329in}{0.764187in}}%
\pgfpathlineto{\pgfqpoint{0.681436in}{0.767316in}}%
\pgfpathlineto{\pgfqpoint{0.681852in}{0.768396in}}%
\pgfpathlineto{\pgfqpoint{0.682956in}{0.771395in}}%
\pgfpathlineto{\pgfqpoint{0.683313in}{0.772493in}}%
\pgfpathlineto{\pgfqpoint{0.684422in}{0.775138in}}%
\pgfpathlineto{\pgfqpoint{0.684863in}{0.776246in}}%
\pgfpathlineto{\pgfqpoint{0.685970in}{0.778668in}}%
\pgfpathlineto{\pgfqpoint{0.686416in}{0.779776in}}%
\pgfpathlineto{\pgfqpoint{0.687513in}{0.782411in}}%
\pgfpathlineto{\pgfqpoint{0.687945in}{0.783519in}}%
\pgfpathlineto{\pgfqpoint{0.689049in}{0.786509in}}%
\pgfpathlineto{\pgfqpoint{0.689568in}{0.787617in}}%
\pgfpathlineto{\pgfqpoint{0.690675in}{0.790206in}}%
\pgfpathlineto{\pgfqpoint{0.691167in}{0.791314in}}%
\pgfpathlineto{\pgfqpoint{0.692276in}{0.793968in}}%
\pgfpathlineto{\pgfqpoint{0.692715in}{0.795076in}}%
\pgfpathlineto{\pgfqpoint{0.693824in}{0.797600in}}%
\pgfpathlineto{\pgfqpoint{0.694298in}{0.798708in}}%
\pgfpathlineto{\pgfqpoint{0.695407in}{0.801437in}}%
\pgfpathlineto{\pgfqpoint{0.695853in}{0.802545in}}%
\pgfpathlineto{\pgfqpoint{0.696958in}{0.805208in}}%
\pgfpathlineto{\pgfqpoint{0.696962in}{0.805208in}}%
\pgfpathlineto{\pgfqpoint{0.697436in}{0.806317in}}%
\pgfpathlineto{\pgfqpoint{0.698545in}{0.808933in}}%
\pgfpathlineto{\pgfqpoint{0.699047in}{0.810032in}}%
\pgfpathlineto{\pgfqpoint{0.700150in}{0.812500in}}%
\pgfpathlineto{\pgfqpoint{0.700607in}{0.813608in}}%
\pgfpathlineto{\pgfqpoint{0.701716in}{0.815955in}}%
\pgfpathlineto{\pgfqpoint{0.702138in}{0.817054in}}%
\pgfpathlineto{\pgfqpoint{0.703248in}{0.819354in}}%
\pgfpathlineto{\pgfqpoint{0.703843in}{0.820462in}}%
\pgfpathlineto{\pgfqpoint{0.704953in}{0.822949in}}%
\pgfpathlineto{\pgfqpoint{0.705450in}{0.824057in}}%
\pgfpathlineto{\pgfqpoint{0.706559in}{0.826739in}}%
\pgfpathlineto{\pgfqpoint{0.707141in}{0.827847in}}%
\pgfpathlineto{\pgfqpoint{0.708250in}{0.830380in}}%
\pgfpathlineto{\pgfqpoint{0.708783in}{0.831488in}}%
\pgfpathlineto{\pgfqpoint{0.709892in}{0.833994in}}%
\pgfpathlineto{\pgfqpoint{0.710354in}{0.835083in}}%
\pgfpathlineto{\pgfqpoint{0.711461in}{0.837691in}}%
\pgfpathlineto{\pgfqpoint{0.712000in}{0.838799in}}%
\pgfpathlineto{\pgfqpoint{0.713100in}{0.841229in}}%
\pgfpathlineto{\pgfqpoint{0.713583in}{0.842338in}}%
\pgfpathlineto{\pgfqpoint{0.714693in}{0.845020in}}%
\pgfpathlineto{\pgfqpoint{0.715253in}{0.846128in}}%
\pgfpathlineto{\pgfqpoint{0.716334in}{0.848111in}}%
\pgfpathlineto{\pgfqpoint{0.716865in}{0.849210in}}%
\pgfpathlineto{\pgfqpoint{0.717964in}{0.851594in}}%
\pgfpathlineto{\pgfqpoint{0.718459in}{0.852702in}}%
\pgfpathlineto{\pgfqpoint{0.719569in}{0.854947in}}%
\pgfpathlineto{\pgfqpoint{0.720103in}{0.856055in}}%
\pgfpathlineto{\pgfqpoint{0.721206in}{0.858374in}}%
\pgfpathlineto{\pgfqpoint{0.721773in}{0.859482in}}%
\pgfpathlineto{\pgfqpoint{0.722883in}{0.861875in}}%
\pgfpathlineto{\pgfqpoint{0.723436in}{0.862984in}}%
\pgfpathlineto{\pgfqpoint{0.724545in}{0.865004in}}%
\pgfpathlineto{\pgfqpoint{0.725085in}{0.866113in}}%
\pgfpathlineto{\pgfqpoint{0.726185in}{0.868059in}}%
\pgfpathlineto{\pgfqpoint{0.726773in}{0.869167in}}%
\pgfpathlineto{\pgfqpoint{0.727878in}{0.871365in}}%
\pgfpathlineto{\pgfqpoint{0.728474in}{0.872473in}}%
\pgfpathlineto{\pgfqpoint{0.729583in}{0.874699in}}%
\pgfpathlineto{\pgfqpoint{0.730214in}{0.875807in}}%
\pgfpathlineto{\pgfqpoint{0.731321in}{0.878098in}}%
\pgfpathlineto{\pgfqpoint{0.731919in}{0.879197in}}%
\pgfpathlineto{\pgfqpoint{0.733028in}{0.881413in}}%
\pgfpathlineto{\pgfqpoint{0.733516in}{0.882521in}}%
\pgfpathlineto{\pgfqpoint{0.734625in}{0.884468in}}%
\pgfpathlineto{\pgfqpoint{0.735235in}{0.885566in}}%
\pgfpathlineto{\pgfqpoint{0.736316in}{0.887503in}}%
\pgfpathlineto{\pgfqpoint{0.736886in}{0.888602in}}%
\pgfpathlineto{\pgfqpoint{0.737989in}{0.890446in}}%
\pgfpathlineto{\pgfqpoint{0.737996in}{0.890446in}}%
\pgfpathlineto{\pgfqpoint{0.738631in}{0.891527in}}%
\pgfpathlineto{\pgfqpoint{0.739738in}{0.893603in}}%
\pgfpathlineto{\pgfqpoint{0.740367in}{0.894711in}}%
\pgfpathlineto{\pgfqpoint{0.741467in}{0.896984in}}%
\pgfpathlineto{\pgfqpoint{0.742098in}{0.898092in}}%
\pgfpathlineto{\pgfqpoint{0.743202in}{0.900252in}}%
\pgfpathlineto{\pgfqpoint{0.743913in}{0.901351in}}%
\pgfpathlineto{\pgfqpoint{0.745022in}{0.903353in}}%
\pgfpathlineto{\pgfqpoint{0.745627in}{0.904462in}}%
\pgfpathlineto{\pgfqpoint{0.746706in}{0.906436in}}%
\pgfpathlineto{\pgfqpoint{0.746732in}{0.906436in}}%
\pgfpathlineto{\pgfqpoint{0.747410in}{0.907544in}}%
\pgfpathlineto{\pgfqpoint{0.748514in}{0.909546in}}%
\pgfpathlineto{\pgfqpoint{0.749098in}{0.910645in}}%
\pgfpathlineto{\pgfqpoint{0.750208in}{0.912610in}}%
\pgfpathlineto{\pgfqpoint{0.750829in}{0.913700in}}%
\pgfpathlineto{\pgfqpoint{0.751936in}{0.915627in}}%
\pgfpathlineto{\pgfqpoint{0.752628in}{0.916726in}}%
\pgfpathlineto{\pgfqpoint{0.753737in}{0.918281in}}%
\pgfpathlineto{\pgfqpoint{0.754382in}{0.919390in}}%
\pgfpathlineto{\pgfqpoint{0.755492in}{0.921429in}}%
\pgfpathlineto{\pgfqpoint{0.756176in}{0.922519in}}%
\pgfpathlineto{\pgfqpoint{0.757281in}{0.924391in}}%
\pgfpathlineto{\pgfqpoint{0.758001in}{0.925489in}}%
\pgfpathlineto{\pgfqpoint{0.759106in}{0.927743in}}%
\pgfpathlineto{\pgfqpoint{0.759868in}{0.928851in}}%
\pgfpathlineto{\pgfqpoint{0.760977in}{0.930947in}}%
\pgfpathlineto{\pgfqpoint{0.761657in}{0.932045in}}%
\pgfpathlineto{\pgfqpoint{0.762755in}{0.934113in}}%
\pgfpathlineto{\pgfqpoint{0.763294in}{0.935221in}}%
\pgfpathlineto{\pgfqpoint{0.764399in}{0.937056in}}%
\pgfpathlineto{\pgfqpoint{0.765039in}{0.938164in}}%
\pgfpathlineto{\pgfqpoint{0.766139in}{0.939803in}}%
\pgfpathlineto{\pgfqpoint{0.766149in}{0.939803in}}%
\pgfpathlineto{\pgfqpoint{0.766606in}{0.940911in}}%
\pgfpathlineto{\pgfqpoint{0.767715in}{0.942327in}}%
\pgfpathlineto{\pgfqpoint{0.768438in}{0.943435in}}%
\pgfpathlineto{\pgfqpoint{0.769545in}{0.945176in}}%
\pgfpathlineto{\pgfqpoint{0.770201in}{0.946266in}}%
\pgfpathlineto{\pgfqpoint{0.771308in}{0.948016in}}%
\pgfpathlineto{\pgfqpoint{0.771974in}{0.949115in}}%
\pgfpathlineto{\pgfqpoint{0.773084in}{0.950689in}}%
\pgfpathlineto{\pgfqpoint{0.773715in}{0.951788in}}%
\pgfpathlineto{\pgfqpoint{0.774824in}{0.953585in}}%
\pgfpathlineto{\pgfqpoint{0.775593in}{0.954675in}}%
\pgfpathlineto{\pgfqpoint{0.776695in}{0.956454in}}%
\pgfpathlineto{\pgfqpoint{0.777364in}{0.957553in}}%
\pgfpathlineto{\pgfqpoint{0.778473in}{0.959368in}}%
\pgfpathlineto{\pgfqpoint{0.779149in}{0.960477in}}%
\pgfpathlineto{\pgfqpoint{0.780256in}{0.962097in}}%
\pgfpathlineto{\pgfqpoint{0.780858in}{0.963205in}}%
\pgfpathlineto{\pgfqpoint{0.781951in}{0.964984in}}%
\pgfpathlineto{\pgfqpoint{0.782610in}{0.966092in}}%
\pgfpathlineto{\pgfqpoint{0.783720in}{0.967834in}}%
\pgfpathlineto{\pgfqpoint{0.784477in}{0.968942in}}%
\pgfpathlineto{\pgfqpoint{0.785584in}{0.970730in}}%
\pgfpathlineto{\pgfqpoint{0.786220in}{0.971838in}}%
\pgfpathlineto{\pgfqpoint{0.787329in}{0.973701in}}%
\pgfpathlineto{\pgfqpoint{0.788056in}{0.974809in}}%
\pgfpathlineto{\pgfqpoint{0.789165in}{0.976345in}}%
\pgfpathlineto{\pgfqpoint{0.789881in}{0.977453in}}%
\pgfpathlineto{\pgfqpoint{0.790985in}{0.978990in}}%
\pgfpathlineto{\pgfqpoint{0.791717in}{0.980089in}}%
\pgfpathlineto{\pgfqpoint{0.792826in}{0.981849in}}%
\pgfpathlineto{\pgfqpoint{0.793560in}{0.982957in}}%
\pgfpathlineto{\pgfqpoint{0.794665in}{0.984782in}}%
\pgfpathlineto{\pgfqpoint{0.795357in}{0.985881in}}%
\pgfpathlineto{\pgfqpoint{0.796464in}{0.987483in}}%
\pgfpathlineto{\pgfqpoint{0.797207in}{0.988591in}}%
\pgfpathlineto{\pgfqpoint{0.798312in}{0.990240in}}%
\pgfpathlineto{\pgfqpoint{0.799107in}{0.991348in}}%
\pgfpathlineto{\pgfqpoint{0.800216in}{0.993024in}}%
\pgfpathlineto{\pgfqpoint{0.800990in}{0.994132in}}%
\pgfpathlineto{\pgfqpoint{0.802093in}{0.995622in}}%
\pgfpathlineto{\pgfqpoint{0.802799in}{0.996703in}}%
\pgfpathlineto{\pgfqpoint{0.803899in}{0.998397in}}%
\pgfpathlineto{\pgfqpoint{0.804715in}{0.999506in}}%
\pgfpathlineto{\pgfqpoint{0.805817in}{1.001107in}}%
\pgfpathlineto{\pgfqpoint{0.806765in}{1.002216in}}%
\pgfpathlineto{\pgfqpoint{0.807874in}{1.003780in}}%
\pgfpathlineto{\pgfqpoint{0.808622in}{1.004879in}}%
\pgfpathlineto{\pgfqpoint{0.809731in}{1.006350in}}%
\pgfpathlineto{\pgfqpoint{0.810512in}{1.007459in}}%
\pgfpathlineto{\pgfqpoint{0.811622in}{1.008976in}}%
\pgfpathlineto{\pgfqpoint{0.812522in}{1.010085in}}%
\pgfpathlineto{\pgfqpoint{0.813632in}{1.011351in}}%
\pgfpathlineto{\pgfqpoint{0.814485in}{1.012459in}}%
\pgfpathlineto{\pgfqpoint{0.815581in}{1.014061in}}%
\pgfpathlineto{\pgfqpoint{0.816406in}{1.015169in}}%
\pgfpathlineto{\pgfqpoint{0.817499in}{1.016920in}}%
\pgfpathlineto{\pgfqpoint{0.818371in}{1.018028in}}%
\pgfpathlineto{\pgfqpoint{0.819478in}{1.019304in}}%
\pgfpathlineto{\pgfqpoint{0.820325in}{1.020412in}}%
\pgfpathlineto{\pgfqpoint{0.821434in}{1.022182in}}%
\pgfpathlineto{\pgfqpoint{0.822405in}{1.023290in}}%
\pgfpathlineto{\pgfqpoint{0.823515in}{1.024743in}}%
\pgfpathlineto{\pgfqpoint{0.824289in}{1.025851in}}%
\pgfpathlineto{\pgfqpoint{0.825396in}{1.027294in}}%
\pgfpathlineto{\pgfqpoint{0.826212in}{1.028402in}}%
\pgfpathlineto{\pgfqpoint{0.827319in}{1.029809in}}%
\pgfpathlineto{\pgfqpoint{0.828184in}{1.030917in}}%
\pgfpathlineto{\pgfqpoint{0.829282in}{1.032528in}}%
\pgfpathlineto{\pgfqpoint{0.830018in}{1.033627in}}%
\pgfpathlineto{\pgfqpoint{0.831125in}{1.035294in}}%
\pgfpathlineto{\pgfqpoint{0.831946in}{1.036393in}}%
\pgfpathlineto{\pgfqpoint{0.833046in}{1.037873in}}%
\pgfpathlineto{\pgfqpoint{0.834000in}{1.038982in}}%
\pgfpathlineto{\pgfqpoint{0.835105in}{1.040388in}}%
\pgfpathlineto{\pgfqpoint{0.835879in}{1.041468in}}%
\pgfpathlineto{\pgfqpoint{0.836986in}{1.042753in}}%
\pgfpathlineto{\pgfqpoint{0.837870in}{1.043852in}}%
\pgfpathlineto{\pgfqpoint{0.838970in}{1.045240in}}%
\pgfpathlineto{\pgfqpoint{0.839807in}{1.046348in}}%
\pgfpathlineto{\pgfqpoint{0.840917in}{1.047717in}}%
\pgfpathlineto{\pgfqpoint{0.841740in}{1.048825in}}%
\pgfpathlineto{\pgfqpoint{0.842845in}{1.049905in}}%
\pgfpathlineto{\pgfqpoint{0.843785in}{1.051004in}}%
\pgfpathlineto{\pgfqpoint{0.844883in}{1.052503in}}%
\pgfpathlineto{\pgfqpoint{0.845811in}{1.053612in}}%
\pgfpathlineto{\pgfqpoint{0.846904in}{1.055334in}}%
\pgfpathlineto{\pgfqpoint{0.847847in}{1.056433in}}%
\pgfpathlineto{\pgfqpoint{0.848952in}{1.057923in}}%
\pgfpathlineto{\pgfqpoint{0.849876in}{1.059031in}}%
\pgfpathlineto{\pgfqpoint{0.850980in}{1.060289in}}%
\pgfpathlineto{\pgfqpoint{0.851776in}{1.061397in}}%
\pgfpathlineto{\pgfqpoint{0.852873in}{1.062905in}}%
\pgfpathlineto{\pgfqpoint{0.853839in}{1.064004in}}%
\pgfpathlineto{\pgfqpoint{0.854949in}{1.065606in}}%
\pgfpathlineto{\pgfqpoint{0.855920in}{1.066714in}}%
\pgfpathlineto{\pgfqpoint{0.857015in}{1.068139in}}%
\pgfpathlineto{\pgfqpoint{0.857883in}{1.069238in}}%
\pgfpathlineto{\pgfqpoint{0.858980in}{1.070830in}}%
\pgfpathlineto{\pgfqpoint{0.859871in}{1.071939in}}%
\pgfpathlineto{\pgfqpoint{0.860981in}{1.073475in}}%
\pgfpathlineto{\pgfqpoint{0.862095in}{1.074574in}}%
\pgfpathlineto{\pgfqpoint{0.863195in}{1.075897in}}%
\pgfpathlineto{\pgfqpoint{0.864156in}{1.077005in}}%
\pgfpathlineto{\pgfqpoint{0.865266in}{1.078364in}}%
\pgfpathlineto{\pgfqpoint{0.866023in}{1.079473in}}%
\pgfpathlineto{\pgfqpoint{0.867133in}{1.080814in}}%
\pgfpathlineto{\pgfqpoint{0.868186in}{1.081922in}}%
\pgfpathlineto{\pgfqpoint{0.869288in}{1.083123in}}%
\pgfpathlineto{\pgfqpoint{0.870242in}{1.084222in}}%
\pgfpathlineto{\pgfqpoint{0.871349in}{1.085805in}}%
\pgfpathlineto{\pgfqpoint{0.872290in}{1.086895in}}%
\pgfpathlineto{\pgfqpoint{0.873397in}{1.088347in}}%
\pgfpathlineto{\pgfqpoint{0.874377in}{1.089456in}}%
\pgfpathlineto{\pgfqpoint{0.875435in}{1.090862in}}%
\pgfpathlineto{\pgfqpoint{0.876418in}{1.091970in}}%
\pgfpathlineto{\pgfqpoint{0.877499in}{1.093078in}}%
\pgfpathlineto{\pgfqpoint{0.878578in}{1.094186in}}%
\pgfpathlineto{\pgfqpoint{0.879687in}{1.095341in}}%
\pgfpathlineto{\pgfqpoint{0.880559in}{1.096449in}}%
\pgfpathlineto{\pgfqpoint{0.881659in}{1.097623in}}%
\pgfpathlineto{\pgfqpoint{0.881669in}{1.097623in}}%
\pgfpathlineto{\pgfqpoint{0.882680in}{1.098731in}}%
\pgfpathlineto{\pgfqpoint{0.883768in}{1.100016in}}%
\pgfpathlineto{\pgfqpoint{0.883784in}{1.100016in}}%
\pgfpathlineto{\pgfqpoint{0.884659in}{1.101115in}}%
\pgfpathlineto{\pgfqpoint{0.885745in}{1.102512in}}%
\pgfpathlineto{\pgfqpoint{0.886889in}{1.103620in}}%
\pgfpathlineto{\pgfqpoint{0.887987in}{1.104840in}}%
\pgfpathlineto{\pgfqpoint{0.889052in}{1.105948in}}%
\pgfpathlineto{\pgfqpoint{0.890147in}{1.106945in}}%
\pgfpathlineto{\pgfqpoint{0.891144in}{1.108043in}}%
\pgfpathlineto{\pgfqpoint{0.892239in}{1.109366in}}%
\pgfpathlineto{\pgfqpoint{0.893177in}{1.110474in}}%
\pgfpathlineto{\pgfqpoint{0.894270in}{1.111694in}}%
\pgfpathlineto{\pgfqpoint{0.895241in}{1.112802in}}%
\pgfpathlineto{\pgfqpoint{0.896350in}{1.114013in}}%
\pgfpathlineto{\pgfqpoint{0.897302in}{1.115121in}}%
\pgfpathlineto{\pgfqpoint{0.898409in}{1.116415in}}%
\pgfpathlineto{\pgfqpoint{0.899409in}{1.117514in}}%
\pgfpathlineto{\pgfqpoint{0.900513in}{1.118809in}}%
\pgfpathlineto{\pgfqpoint{0.901522in}{1.119908in}}%
\pgfpathlineto{\pgfqpoint{0.902612in}{1.121323in}}%
\pgfpathlineto{\pgfqpoint{0.903787in}{1.122431in}}%
\pgfpathlineto{\pgfqpoint{0.904875in}{1.123782in}}%
\pgfpathlineto{\pgfqpoint{0.906090in}{1.124890in}}%
\pgfpathlineto{\pgfqpoint{0.907179in}{1.126249in}}%
\pgfpathlineto{\pgfqpoint{0.908023in}{1.127348in}}%
\pgfpathlineto{\pgfqpoint{0.909120in}{1.128568in}}%
\pgfpathlineto{\pgfqpoint{0.909130in}{1.128568in}}%
\pgfpathlineto{\pgfqpoint{0.910298in}{1.129667in}}%
\pgfpathlineto{\pgfqpoint{0.911402in}{1.131055in}}%
\pgfpathlineto{\pgfqpoint{0.912502in}{1.132163in}}%
\pgfpathlineto{\pgfqpoint{0.913602in}{1.133355in}}%
\pgfpathlineto{\pgfqpoint{0.913612in}{1.133355in}}%
\pgfpathlineto{\pgfqpoint{0.914700in}{1.134463in}}%
\pgfpathlineto{\pgfqpoint{0.915807in}{1.135664in}}%
\pgfpathlineto{\pgfqpoint{0.916862in}{1.136773in}}%
\pgfpathlineto{\pgfqpoint{0.917967in}{1.137704in}}%
\pgfpathlineto{\pgfqpoint{0.918900in}{1.138812in}}%
\pgfpathlineto{\pgfqpoint{0.920010in}{1.140051in}}%
\pgfpathlineto{\pgfqpoint{0.921035in}{1.141159in}}%
\pgfpathlineto{\pgfqpoint{0.922144in}{1.142183in}}%
\pgfpathlineto{\pgfqpoint{0.923335in}{1.143291in}}%
\pgfpathlineto{\pgfqpoint{0.924445in}{1.144558in}}%
\pgfpathlineto{\pgfqpoint{0.925636in}{1.145666in}}%
\pgfpathlineto{\pgfqpoint{0.926743in}{1.146663in}}%
\pgfpathlineto{\pgfqpoint{0.927770in}{1.147771in}}%
\pgfpathlineto{\pgfqpoint{0.928873in}{1.148860in}}%
\pgfpathlineto{\pgfqpoint{0.929832in}{1.149969in}}%
\pgfpathlineto{\pgfqpoint{0.930918in}{1.151067in}}%
\pgfpathlineto{\pgfqpoint{0.930932in}{1.151067in}}%
\pgfpathlineto{\pgfqpoint{0.932069in}{1.152176in}}%
\pgfpathlineto{\pgfqpoint{0.933171in}{1.153368in}}%
\pgfpathlineto{\pgfqpoint{0.933179in}{1.153368in}}%
\pgfpathlineto{\pgfqpoint{0.934215in}{1.154457in}}%
\pgfpathlineto{\pgfqpoint{0.935313in}{1.155714in}}%
\pgfpathlineto{\pgfqpoint{0.936438in}{1.156823in}}%
\pgfpathlineto{\pgfqpoint{0.937545in}{1.157968in}}%
\pgfpathlineto{\pgfqpoint{0.938643in}{1.159058in}}%
\pgfpathlineto{\pgfqpoint{0.938643in}{1.159067in}}%
\pgfpathlineto{\pgfqpoint{0.939750in}{1.160333in}}%
\pgfpathlineto{\pgfqpoint{0.940754in}{1.161442in}}%
\pgfpathlineto{\pgfqpoint{0.941842in}{1.162503in}}%
\pgfpathlineto{\pgfqpoint{0.942914in}{1.163602in}}%
\pgfpathlineto{\pgfqpoint{0.944014in}{1.164701in}}%
\pgfpathlineto{\pgfqpoint{0.945449in}{1.165809in}}%
\pgfpathlineto{\pgfqpoint{0.946558in}{1.166703in}}%
\pgfpathlineto{\pgfqpoint{0.947783in}{1.167811in}}%
\pgfpathlineto{\pgfqpoint{0.948890in}{1.168854in}}%
\pgfpathlineto{\pgfqpoint{0.950004in}{1.169963in}}%
\pgfpathlineto{\pgfqpoint{0.951104in}{1.170903in}}%
\pgfpathlineto{\pgfqpoint{0.952368in}{1.171974in}}%
\pgfpathlineto{\pgfqpoint{0.953475in}{1.173036in}}%
\pgfpathlineto{\pgfqpoint{0.954633in}{1.174144in}}%
\pgfpathlineto{\pgfqpoint{0.955712in}{1.175224in}}%
\pgfpathlineto{\pgfqpoint{0.956887in}{1.176323in}}%
\pgfpathlineto{\pgfqpoint{0.957996in}{1.177655in}}%
\pgfpathlineto{\pgfqpoint{0.958986in}{1.178763in}}%
\pgfpathlineto{\pgfqpoint{0.960067in}{1.179862in}}%
\pgfpathlineto{\pgfqpoint{0.961310in}{1.180970in}}%
\pgfpathlineto{\pgfqpoint{0.962415in}{1.182106in}}%
\pgfpathlineto{\pgfqpoint{0.963503in}{1.183214in}}%
\pgfpathlineto{\pgfqpoint{0.964613in}{1.184434in}}%
\pgfpathlineto{\pgfqpoint{0.965818in}{1.185543in}}%
\pgfpathlineto{\pgfqpoint{0.966918in}{1.186632in}}%
\pgfpathlineto{\pgfqpoint{0.967955in}{1.187740in}}%
\pgfpathlineto{\pgfqpoint{0.969052in}{1.188737in}}%
\pgfpathlineto{\pgfqpoint{0.970276in}{1.189845in}}%
\pgfpathlineto{\pgfqpoint{0.971372in}{1.190767in}}%
\pgfpathlineto{\pgfqpoint{0.972720in}{1.191875in}}%
\pgfpathlineto{\pgfqpoint{0.973827in}{1.192955in}}%
\pgfpathlineto{\pgfqpoint{0.975105in}{1.194064in}}%
\pgfpathlineto{\pgfqpoint{0.976212in}{1.195293in}}%
\pgfpathlineto{\pgfqpoint{0.977720in}{1.196401in}}%
\pgfpathlineto{\pgfqpoint{0.979203in}{1.197779in}}%
\pgfpathlineto{\pgfqpoint{0.980589in}{1.198887in}}%
\pgfpathlineto{\pgfqpoint{0.981698in}{1.199996in}}%
\pgfpathlineto{\pgfqpoint{0.982857in}{1.201095in}}%
\pgfpathlineto{\pgfqpoint{0.983964in}{1.202287in}}%
\pgfpathlineto{\pgfqpoint{0.985319in}{1.203385in}}%
\pgfpathlineto{\pgfqpoint{0.986410in}{1.204242in}}%
\pgfpathlineto{\pgfqpoint{0.987730in}{1.205350in}}%
\pgfpathlineto{\pgfqpoint{0.988837in}{1.206300in}}%
\pgfpathlineto{\pgfqpoint{0.990017in}{1.207408in}}%
\pgfpathlineto{\pgfqpoint{0.991089in}{1.208144in}}%
\pgfpathlineto{\pgfqpoint{0.992522in}{1.209252in}}%
\pgfpathlineto{\pgfqpoint{0.993631in}{1.210277in}}%
\pgfpathlineto{\pgfqpoint{0.995038in}{1.211385in}}%
\pgfpathlineto{\pgfqpoint{0.996147in}{1.212512in}}%
\pgfpathlineto{\pgfqpoint{0.997440in}{1.213611in}}%
\pgfpathlineto{\pgfqpoint{0.998542in}{1.214607in}}%
\pgfpathlineto{\pgfqpoint{0.999853in}{1.215715in}}%
\pgfpathlineto{\pgfqpoint{1.000962in}{1.216646in}}%
\pgfpathlineto{\pgfqpoint{1.002161in}{1.217755in}}%
\pgfpathlineto{\pgfqpoint{1.003270in}{1.218472in}}%
\pgfpathlineto{\pgfqpoint{1.004337in}{1.219580in}}%
\pgfpathlineto{\pgfqpoint{1.005432in}{1.220688in}}%
\pgfpathlineto{\pgfqpoint{1.006406in}{1.221796in}}%
\pgfpathlineto{\pgfqpoint{1.007513in}{1.222858in}}%
\pgfpathlineto{\pgfqpoint{1.008721in}{1.223966in}}%
\pgfpathlineto{\pgfqpoint{1.009825in}{1.225000in}}%
\pgfpathlineto{\pgfqpoint{1.011289in}{1.226108in}}%
\pgfpathlineto{\pgfqpoint{1.012384in}{1.227170in}}%
\pgfpathlineto{\pgfqpoint{1.013920in}{1.228259in}}%
\pgfpathlineto{\pgfqpoint{1.015027in}{1.229470in}}%
\pgfpathlineto{\pgfqpoint{1.016392in}{1.230569in}}%
\pgfpathlineto{\pgfqpoint{1.017494in}{1.231584in}}%
\pgfpathlineto{\pgfqpoint{1.018883in}{1.232692in}}%
\pgfpathlineto{\pgfqpoint{1.019966in}{1.233558in}}%
\pgfpathlineto{\pgfqpoint{1.021383in}{1.234666in}}%
\pgfpathlineto{\pgfqpoint{1.022487in}{1.235728in}}%
\pgfpathlineto{\pgfqpoint{1.023784in}{1.236836in}}%
\pgfpathlineto{\pgfqpoint{1.024873in}{1.237488in}}%
\pgfpathlineto{\pgfqpoint{1.026188in}{1.238596in}}%
\pgfpathlineto{\pgfqpoint{1.027286in}{1.239378in}}%
\pgfpathlineto{\pgfqpoint{1.028834in}{1.240487in}}%
\pgfpathlineto{\pgfqpoint{1.029936in}{1.241446in}}%
\pgfpathlineto{\pgfqpoint{1.031313in}{1.242554in}}%
\pgfpathlineto{\pgfqpoint{1.032378in}{1.243383in}}%
\pgfpathlineto{\pgfqpoint{1.033904in}{1.244491in}}%
\pgfpathlineto{\pgfqpoint{1.035011in}{1.245487in}}%
\pgfpathlineto{\pgfqpoint{1.036322in}{1.246596in}}%
\pgfpathlineto{\pgfqpoint{1.037420in}{1.247518in}}%
\pgfpathlineto{\pgfqpoint{1.038954in}{1.248626in}}%
\pgfpathlineto{\pgfqpoint{1.040063in}{1.249687in}}%
\pgfpathlineto{\pgfqpoint{1.041280in}{1.250786in}}%
\pgfpathlineto{\pgfqpoint{1.042369in}{1.251745in}}%
\pgfpathlineto{\pgfqpoint{1.043682in}{1.252854in}}%
\pgfpathlineto{\pgfqpoint{1.044787in}{1.253766in}}%
\pgfpathlineto{\pgfqpoint{1.046128in}{1.254875in}}%
\pgfpathlineto{\pgfqpoint{1.047212in}{1.255703in}}%
\pgfpathlineto{\pgfqpoint{1.048642in}{1.256812in}}%
\pgfpathlineto{\pgfqpoint{1.049747in}{1.257706in}}%
\pgfpathlineto{\pgfqpoint{1.051006in}{1.258814in}}%
\pgfpathlineto{\pgfqpoint{1.052102in}{1.259708in}}%
\pgfpathlineto{\pgfqpoint{1.053657in}{1.260816in}}%
\pgfpathlineto{\pgfqpoint{1.054747in}{1.261915in}}%
\pgfpathlineto{\pgfqpoint{1.056464in}{1.263023in}}%
\pgfpathlineto{\pgfqpoint{1.057568in}{1.263936in}}%
\pgfpathlineto{\pgfqpoint{1.059039in}{1.265044in}}%
\pgfpathlineto{\pgfqpoint{1.060148in}{1.265863in}}%
\pgfpathlineto{\pgfqpoint{1.061731in}{1.266972in}}%
\pgfpathlineto{\pgfqpoint{1.062838in}{1.267856in}}%
\pgfpathlineto{\pgfqpoint{1.064351in}{1.268964in}}%
\pgfpathlineto{\pgfqpoint{1.065458in}{1.269803in}}%
\pgfpathlineto{\pgfqpoint{1.066771in}{1.270892in}}%
\pgfpathlineto{\pgfqpoint{1.067867in}{1.271758in}}%
\pgfpathlineto{\pgfqpoint{1.067881in}{1.271758in}}%
\pgfpathlineto{\pgfqpoint{1.069408in}{1.272866in}}%
\pgfpathlineto{\pgfqpoint{1.070505in}{1.273695in}}%
\pgfpathlineto{\pgfqpoint{1.071884in}{1.274785in}}%
\pgfpathlineto{\pgfqpoint{1.072991in}{1.275725in}}%
\pgfpathlineto{\pgfqpoint{1.074276in}{1.276834in}}%
\pgfpathlineto{\pgfqpoint{1.075379in}{1.277700in}}%
\pgfpathlineto{\pgfqpoint{1.076842in}{1.278808in}}%
\pgfpathlineto{\pgfqpoint{1.077951in}{1.279609in}}%
\pgfpathlineto{\pgfqpoint{1.079598in}{1.280717in}}%
\pgfpathlineto{\pgfqpoint{1.080707in}{1.281508in}}%
\pgfpathlineto{\pgfqpoint{1.081863in}{1.282617in}}%
\pgfpathlineto{\pgfqpoint{1.082956in}{1.283539in}}%
\pgfpathlineto{\pgfqpoint{1.084448in}{1.284647in}}%
\pgfpathlineto{\pgfqpoint{1.085522in}{1.285280in}}%
\pgfpathlineto{\pgfqpoint{1.086929in}{1.286388in}}%
\pgfpathlineto{\pgfqpoint{1.088039in}{1.287273in}}%
\pgfpathlineto{\pgfqpoint{1.089615in}{1.288381in}}%
\pgfpathlineto{\pgfqpoint{1.090710in}{1.289154in}}%
\pgfpathlineto{\pgfqpoint{1.092162in}{1.290262in}}%
\pgfpathlineto{\pgfqpoint{1.093240in}{1.291091in}}%
\pgfpathlineto{\pgfqpoint{1.093271in}{1.291091in}}%
\pgfpathlineto{\pgfqpoint{1.094906in}{1.292199in}}%
\pgfpathlineto{\pgfqpoint{1.096003in}{1.292972in}}%
\pgfpathlineto{\pgfqpoint{1.097697in}{1.294080in}}%
\pgfpathlineto{\pgfqpoint{1.098796in}{1.294751in}}%
\pgfpathlineto{\pgfqpoint{1.100328in}{1.295859in}}%
\pgfpathlineto{\pgfqpoint{1.101426in}{1.296595in}}%
\pgfpathlineto{\pgfqpoint{1.102908in}{1.297703in}}%
\pgfpathlineto{\pgfqpoint{1.104012in}{1.298476in}}%
\pgfpathlineto{\pgfqpoint{1.105596in}{1.299584in}}%
\pgfpathlineto{\pgfqpoint{1.106686in}{1.300301in}}%
\pgfpathlineto{\pgfqpoint{1.108147in}{1.301409in}}%
\pgfpathlineto{\pgfqpoint{1.109254in}{1.302136in}}%
\pgfpathlineto{\pgfqpoint{1.110999in}{1.303244in}}%
\pgfpathlineto{\pgfqpoint{1.112106in}{1.304110in}}%
\pgfpathlineto{\pgfqpoint{1.113574in}{1.305218in}}%
\pgfpathlineto{\pgfqpoint{1.114679in}{1.306094in}}%
\pgfpathlineto{\pgfqpoint{1.116255in}{1.307202in}}%
\pgfpathlineto{\pgfqpoint{1.117357in}{1.307919in}}%
\pgfpathlineto{\pgfqpoint{1.119184in}{1.309027in}}%
\pgfpathlineto{\pgfqpoint{1.120270in}{1.309912in}}%
\pgfpathlineto{\pgfqpoint{1.121938in}{1.311020in}}%
\pgfpathlineto{\pgfqpoint{1.123026in}{1.311746in}}%
\pgfpathlineto{\pgfqpoint{1.124822in}{1.312854in}}%
\pgfpathlineto{\pgfqpoint{1.125925in}{1.313655in}}%
\pgfpathlineto{\pgfqpoint{1.127533in}{1.314764in}}%
\pgfpathlineto{\pgfqpoint{1.128608in}{1.315453in}}%
\pgfpathlineto{\pgfqpoint{1.130195in}{1.316561in}}%
\pgfpathlineto{\pgfqpoint{1.131300in}{1.317259in}}%
\pgfpathlineto{\pgfqpoint{1.133207in}{1.318368in}}%
\pgfpathlineto{\pgfqpoint{1.134316in}{1.319038in}}%
\pgfpathlineto{\pgfqpoint{1.136131in}{1.320137in}}%
\pgfpathlineto{\pgfqpoint{1.137236in}{1.320938in}}%
\pgfpathlineto{\pgfqpoint{1.138983in}{1.322046in}}%
\pgfpathlineto{\pgfqpoint{1.140081in}{1.322893in}}%
\pgfpathlineto{\pgfqpoint{1.140088in}{1.322893in}}%
\pgfpathlineto{\pgfqpoint{1.141854in}{1.324002in}}%
\pgfpathlineto{\pgfqpoint{1.142930in}{1.324784in}}%
\pgfpathlineto{\pgfqpoint{1.144403in}{1.325892in}}%
\pgfpathlineto{\pgfqpoint{1.145503in}{1.326674in}}%
\pgfpathlineto{\pgfqpoint{1.145508in}{1.326674in}}%
\pgfpathlineto{\pgfqpoint{1.147304in}{1.327773in}}%
\pgfpathlineto{\pgfqpoint{1.148407in}{1.328481in}}%
\pgfpathlineto{\pgfqpoint{1.150140in}{1.329589in}}%
\pgfpathlineto{\pgfqpoint{1.151240in}{1.330483in}}%
\pgfpathlineto{\pgfqpoint{1.152935in}{1.331591in}}%
\pgfpathlineto{\pgfqpoint{1.153979in}{1.332411in}}%
\pgfpathlineto{\pgfqpoint{1.153993in}{1.332411in}}%
\pgfpathlineto{\pgfqpoint{1.155654in}{1.333519in}}%
\pgfpathlineto{\pgfqpoint{1.156754in}{1.334301in}}%
\pgfpathlineto{\pgfqpoint{1.158688in}{1.335410in}}%
\pgfpathlineto{\pgfqpoint{1.159774in}{1.336201in}}%
\pgfpathlineto{\pgfqpoint{1.161477in}{1.337300in}}%
\pgfpathlineto{\pgfqpoint{1.162572in}{1.337905in}}%
\pgfpathlineto{\pgfqpoint{1.164448in}{1.339013in}}%
\pgfpathlineto{\pgfqpoint{1.165558in}{1.339637in}}%
\pgfpathlineto{\pgfqpoint{1.167321in}{1.340746in}}%
\pgfpathlineto{\pgfqpoint{1.168424in}{1.341491in}}%
\pgfpathlineto{\pgfqpoint{1.170401in}{1.342599in}}%
\pgfpathlineto{\pgfqpoint{1.171501in}{1.343316in}}%
\pgfpathlineto{\pgfqpoint{1.173255in}{1.344424in}}%
\pgfpathlineto{\pgfqpoint{1.174355in}{1.345281in}}%
\pgfpathlineto{\pgfqpoint{1.176421in}{1.346389in}}%
\pgfpathlineto{\pgfqpoint{1.177528in}{1.347162in}}%
\pgfpathlineto{\pgfqpoint{1.178980in}{1.348270in}}%
\pgfpathlineto{\pgfqpoint{1.180080in}{1.348969in}}%
\pgfpathlineto{\pgfqpoint{1.181773in}{1.350077in}}%
\pgfpathlineto{\pgfqpoint{1.182840in}{1.350747in}}%
\pgfpathlineto{\pgfqpoint{1.184656in}{1.351855in}}%
\pgfpathlineto{\pgfqpoint{1.185751in}{1.352489in}}%
\pgfpathlineto{\pgfqpoint{1.187604in}{1.353597in}}%
\pgfpathlineto{\pgfqpoint{1.188659in}{1.354267in}}%
\pgfpathlineto{\pgfqpoint{1.190545in}{1.355376in}}%
\pgfpathlineto{\pgfqpoint{1.191647in}{1.355972in}}%
\pgfpathlineto{\pgfqpoint{1.193586in}{1.357080in}}%
\pgfpathlineto{\pgfqpoint{1.194658in}{1.357769in}}%
\pgfpathlineto{\pgfqpoint{1.196689in}{1.358877in}}%
\pgfpathlineto{\pgfqpoint{1.197759in}{1.359482in}}%
\pgfpathlineto{\pgfqpoint{1.199656in}{1.360591in}}%
\pgfpathlineto{\pgfqpoint{1.200763in}{1.361429in}}%
\pgfpathlineto{\pgfqpoint{1.203087in}{1.362537in}}%
\pgfpathlineto{\pgfqpoint{1.204190in}{1.363245in}}%
\pgfpathlineto{\pgfqpoint{1.204197in}{1.363245in}}%
\pgfpathlineto{\pgfqpoint{1.205822in}{1.364353in}}%
\pgfpathlineto{\pgfqpoint{1.206917in}{1.365023in}}%
\pgfpathlineto{\pgfqpoint{1.209009in}{1.366132in}}%
\pgfpathlineto{\pgfqpoint{1.210107in}{1.366895in}}%
\pgfpathlineto{\pgfqpoint{1.211842in}{1.367985in}}%
\pgfpathlineto{\pgfqpoint{1.212933in}{1.368618in}}%
\pgfpathlineto{\pgfqpoint{1.214856in}{1.369717in}}%
\pgfpathlineto{\pgfqpoint{1.215932in}{1.370341in}}%
\pgfpathlineto{\pgfqpoint{1.218034in}{1.371449in}}%
\pgfpathlineto{\pgfqpoint{1.219131in}{1.372101in}}%
\pgfpathlineto{\pgfqpoint{1.220778in}{1.373209in}}%
\pgfpathlineto{\pgfqpoint{1.221875in}{1.373852in}}%
\pgfpathlineto{\pgfqpoint{1.223970in}{1.374951in}}%
\pgfpathlineto{\pgfqpoint{1.225067in}{1.375621in}}%
\pgfpathlineto{\pgfqpoint{1.226707in}{1.376729in}}%
\pgfpathlineto{\pgfqpoint{1.227790in}{1.377307in}}%
\pgfpathlineto{\pgfqpoint{1.229697in}{1.378415in}}%
\pgfpathlineto{\pgfqpoint{1.230785in}{1.379039in}}%
\pgfpathlineto{\pgfqpoint{1.232805in}{1.380147in}}%
\pgfpathlineto{\pgfqpoint{1.233886in}{1.380948in}}%
\pgfpathlineto{\pgfqpoint{1.235778in}{1.382056in}}%
\pgfpathlineto{\pgfqpoint{1.236888in}{1.382643in}}%
\pgfpathlineto{\pgfqpoint{1.238891in}{1.383751in}}%
\pgfpathlineto{\pgfqpoint{1.239991in}{1.384412in}}%
\pgfpathlineto{\pgfqpoint{1.242104in}{1.385511in}}%
\pgfpathlineto{\pgfqpoint{1.243211in}{1.386144in}}%
\pgfpathlineto{\pgfqpoint{1.245361in}{1.387253in}}%
\pgfpathlineto{\pgfqpoint{1.246419in}{1.387802in}}%
\pgfpathlineto{\pgfqpoint{1.249041in}{1.388910in}}%
\pgfpathlineto{\pgfqpoint{1.250143in}{1.389627in}}%
\pgfpathlineto{\pgfqpoint{1.252414in}{1.390735in}}%
\pgfpathlineto{\pgfqpoint{1.253521in}{1.391425in}}%
\pgfpathlineto{\pgfqpoint{1.255481in}{1.392533in}}%
\pgfpathlineto{\pgfqpoint{1.256576in}{1.393129in}}%
\pgfpathlineto{\pgfqpoint{1.258790in}{1.394237in}}%
\pgfpathlineto{\pgfqpoint{1.259900in}{1.394945in}}%
\pgfpathlineto{\pgfqpoint{1.261910in}{1.396053in}}%
\pgfpathlineto{\pgfqpoint{1.263019in}{1.396695in}}%
\pgfpathlineto{\pgfqpoint{1.265184in}{1.397794in}}%
\pgfpathlineto{\pgfqpoint{1.266293in}{1.398362in}}%
\pgfpathlineto{\pgfqpoint{1.268732in}{1.399471in}}%
\pgfpathlineto{\pgfqpoint{1.269842in}{1.400216in}}%
\pgfpathlineto{\pgfqpoint{1.271936in}{1.401324in}}%
\pgfpathlineto{\pgfqpoint{1.273038in}{1.401957in}}%
\pgfpathlineto{\pgfqpoint{1.275527in}{1.403065in}}%
\pgfpathlineto{\pgfqpoint{1.276636in}{1.403736in}}%
\pgfpathlineto{\pgfqpoint{1.278897in}{1.404844in}}%
\pgfpathlineto{\pgfqpoint{1.280004in}{1.405580in}}%
\pgfpathlineto{\pgfqpoint{1.282218in}{1.406688in}}%
\pgfpathlineto{\pgfqpoint{1.283299in}{1.407247in}}%
\pgfpathlineto{\pgfqpoint{1.285757in}{1.408355in}}%
\pgfpathlineto{\pgfqpoint{1.286859in}{1.408867in}}%
\pgfpathlineto{\pgfqpoint{1.288951in}{1.409966in}}%
\pgfpathlineto{\pgfqpoint{1.290056in}{1.410636in}}%
\pgfpathlineto{\pgfqpoint{1.292422in}{1.411745in}}%
\pgfpathlineto{\pgfqpoint{1.293508in}{1.412313in}}%
\pgfpathlineto{\pgfqpoint{1.295954in}{1.413421in}}%
\pgfpathlineto{\pgfqpoint{1.297054in}{1.413970in}}%
\pgfpathlineto{\pgfqpoint{1.299343in}{1.415078in}}%
\pgfpathlineto{\pgfqpoint{1.300434in}{1.415591in}}%
\pgfpathlineto{\pgfqpoint{1.302957in}{1.416699in}}%
\pgfpathlineto{\pgfqpoint{1.304036in}{1.417360in}}%
\pgfpathlineto{\pgfqpoint{1.306123in}{1.418468in}}%
\pgfpathlineto{\pgfqpoint{1.307214in}{1.419018in}}%
\pgfpathlineto{\pgfqpoint{1.309524in}{1.420117in}}%
\pgfpathlineto{\pgfqpoint{1.310633in}{1.420666in}}%
\pgfpathlineto{\pgfqpoint{1.312571in}{1.421774in}}%
\pgfpathlineto{\pgfqpoint{1.313670in}{1.422389in}}%
\pgfpathlineto{\pgfqpoint{1.316018in}{1.423497in}}%
\pgfpathlineto{\pgfqpoint{1.317127in}{1.424186in}}%
\pgfpathlineto{\pgfqpoint{1.319287in}{1.425294in}}%
\pgfpathlineto{\pgfqpoint{1.320390in}{1.425853in}}%
\pgfpathlineto{\pgfqpoint{1.322151in}{1.426961in}}%
\pgfpathlineto{\pgfqpoint{1.323258in}{1.427408in}}%
\pgfpathlineto{\pgfqpoint{1.325207in}{1.428507in}}%
\pgfpathlineto{\pgfqpoint{1.326309in}{1.429029in}}%
\pgfpathlineto{\pgfqpoint{1.328901in}{1.430137in}}%
\pgfpathlineto{\pgfqpoint{1.330003in}{1.430770in}}%
\pgfpathlineto{\pgfqpoint{1.332386in}{1.431878in}}%
\pgfpathlineto{\pgfqpoint{1.333495in}{1.432567in}}%
\pgfpathlineto{\pgfqpoint{1.335658in}{1.433676in}}%
\pgfpathlineto{\pgfqpoint{1.336767in}{1.434123in}}%
\pgfpathlineto{\pgfqpoint{1.339047in}{1.435231in}}%
\pgfpathlineto{\pgfqpoint{1.340114in}{1.435901in}}%
\pgfpathlineto{\pgfqpoint{1.342907in}{1.437009in}}%
\pgfpathlineto{\pgfqpoint{1.343979in}{1.437587in}}%
\pgfpathlineto{\pgfqpoint{1.346106in}{1.438695in}}%
\pgfpathlineto{\pgfqpoint{1.347215in}{1.439217in}}%
\pgfpathlineto{\pgfqpoint{1.349305in}{1.440325in}}%
\pgfpathlineto{\pgfqpoint{1.350384in}{1.440865in}}%
\pgfpathlineto{\pgfqpoint{1.352380in}{1.441973in}}%
\pgfpathlineto{\pgfqpoint{1.353484in}{1.442448in}}%
\pgfpathlineto{\pgfqpoint{1.355635in}{1.443556in}}%
\pgfpathlineto{\pgfqpoint{1.356740in}{1.444050in}}%
\pgfpathlineto{\pgfqpoint{1.359495in}{1.445158in}}%
\pgfpathlineto{\pgfqpoint{1.360600in}{1.445596in}}%
\pgfpathlineto{\pgfqpoint{1.363191in}{1.446704in}}%
\pgfpathlineto{\pgfqpoint{1.364268in}{1.447300in}}%
\pgfpathlineto{\pgfqpoint{1.366620in}{1.448408in}}%
\pgfpathlineto{\pgfqpoint{1.367713in}{1.449041in}}%
\pgfpathlineto{\pgfqpoint{1.370729in}{1.450150in}}%
\pgfpathlineto{\pgfqpoint{1.371839in}{1.450578in}}%
\pgfpathlineto{\pgfqpoint{1.374343in}{1.451686in}}%
\pgfpathlineto{\pgfqpoint{1.375446in}{1.452245in}}%
\pgfpathlineto{\pgfqpoint{1.378072in}{1.453344in}}%
\pgfpathlineto{\pgfqpoint{1.379123in}{1.453875in}}%
\pgfpathlineto{\pgfqpoint{1.381635in}{1.454983in}}%
\pgfpathlineto{\pgfqpoint{1.382728in}{1.455541in}}%
\pgfpathlineto{\pgfqpoint{1.385725in}{1.456650in}}%
\pgfpathlineto{\pgfqpoint{1.386780in}{1.457022in}}%
\pgfpathlineto{\pgfqpoint{1.389323in}{1.458121in}}%
\pgfpathlineto{\pgfqpoint{1.390376in}{1.458447in}}%
\pgfpathlineto{\pgfqpoint{1.390411in}{1.458447in}}%
\pgfpathlineto{\pgfqpoint{1.392721in}{1.459555in}}%
\pgfpathlineto{\pgfqpoint{1.393830in}{1.460002in}}%
\pgfpathlineto{\pgfqpoint{1.396565in}{1.461110in}}%
\pgfpathlineto{\pgfqpoint{1.397649in}{1.461511in}}%
\pgfpathlineto{\pgfqpoint{1.400242in}{1.462619in}}%
\pgfpathlineto{\pgfqpoint{1.401328in}{1.463075in}}%
\pgfpathlineto{\pgfqpoint{1.401335in}{1.463075in}}%
\pgfpathlineto{\pgfqpoint{1.403906in}{1.464184in}}%
\pgfpathlineto{\pgfqpoint{1.404973in}{1.464742in}}%
\pgfpathlineto{\pgfqpoint{1.408153in}{1.465850in}}%
\pgfpathlineto{\pgfqpoint{1.409213in}{1.466223in}}%
\pgfpathlineto{\pgfqpoint{1.411866in}{1.467331in}}%
\pgfpathlineto{\pgfqpoint{1.412935in}{1.467806in}}%
\pgfpathlineto{\pgfqpoint{1.415982in}{1.468914in}}%
\pgfpathlineto{\pgfqpoint{1.417075in}{1.469277in}}%
\pgfpathlineto{\pgfqpoint{1.419561in}{1.470386in}}%
\pgfpathlineto{\pgfqpoint{1.420661in}{1.470786in}}%
\pgfpathlineto{\pgfqpoint{1.423496in}{1.471894in}}%
\pgfpathlineto{\pgfqpoint{1.424591in}{1.472360in}}%
\pgfpathlineto{\pgfqpoint{1.427256in}{1.473468in}}%
\pgfpathlineto{\pgfqpoint{1.428339in}{1.473906in}}%
\pgfpathlineto{\pgfqpoint{1.431693in}{1.475005in}}%
\pgfpathlineto{\pgfqpoint{1.432779in}{1.475480in}}%
\pgfpathlineto{\pgfqpoint{1.432800in}{1.475480in}}%
\pgfpathlineto{\pgfqpoint{1.435520in}{1.476569in}}%
\pgfpathlineto{\pgfqpoint{1.436588in}{1.476877in}}%
\pgfpathlineto{\pgfqpoint{1.436627in}{1.476877in}}%
\pgfpathlineto{\pgfqpoint{1.439716in}{1.477985in}}%
\pgfpathlineto{\pgfqpoint{1.440802in}{1.478432in}}%
\pgfpathlineto{\pgfqpoint{1.443551in}{1.479540in}}%
\pgfpathlineto{\pgfqpoint{1.444634in}{1.479903in}}%
\pgfpathlineto{\pgfqpoint{1.447730in}{1.481011in}}%
\pgfpathlineto{\pgfqpoint{1.448764in}{1.481365in}}%
\pgfpathlineto{\pgfqpoint{1.451513in}{1.482473in}}%
\pgfpathlineto{\pgfqpoint{1.452618in}{1.482874in}}%
\pgfpathlineto{\pgfqpoint{1.455298in}{1.483982in}}%
\pgfpathlineto{\pgfqpoint{1.456398in}{1.484364in}}%
\pgfpathlineto{\pgfqpoint{1.458392in}{1.485463in}}%
\pgfpathlineto{\pgfqpoint{1.459482in}{1.486012in}}%
\pgfpathlineto{\pgfqpoint{1.462278in}{1.487120in}}%
\pgfpathlineto{\pgfqpoint{1.463354in}{1.487484in}}%
\pgfpathlineto{\pgfqpoint{1.466959in}{1.488592in}}%
\pgfpathlineto{\pgfqpoint{1.468054in}{1.489039in}}%
\pgfpathlineto{\pgfqpoint{1.470866in}{1.490147in}}%
\pgfpathlineto{\pgfqpoint{1.471945in}{1.490557in}}%
\pgfpathlineto{\pgfqpoint{1.475247in}{1.491665in}}%
\pgfpathlineto{\pgfqpoint{1.476354in}{1.492149in}}%
\pgfpathlineto{\pgfqpoint{1.479495in}{1.493257in}}%
\pgfpathlineto{\pgfqpoint{1.480571in}{1.493658in}}%
\pgfpathlineto{\pgfqpoint{1.483878in}{1.494766in}}%
\pgfpathlineto{\pgfqpoint{1.484959in}{1.495232in}}%
\pgfpathlineto{\pgfqpoint{1.488627in}{1.496340in}}%
\pgfpathlineto{\pgfqpoint{1.489727in}{1.496833in}}%
\pgfpathlineto{\pgfqpoint{1.493271in}{1.497942in}}%
\pgfpathlineto{\pgfqpoint{1.494345in}{1.498361in}}%
\pgfpathlineto{\pgfqpoint{1.497603in}{1.499469in}}%
\pgfpathlineto{\pgfqpoint{1.498654in}{1.499832in}}%
\pgfpathlineto{\pgfqpoint{1.502465in}{1.500940in}}%
\pgfpathlineto{\pgfqpoint{1.503553in}{1.501331in}}%
\pgfpathlineto{\pgfqpoint{1.506623in}{1.502440in}}%
\pgfpathlineto{\pgfqpoint{1.507695in}{1.502812in}}%
\pgfpathlineto{\pgfqpoint{1.510779in}{1.503920in}}%
\pgfpathlineto{\pgfqpoint{1.511883in}{1.504311in}}%
\pgfpathlineto{\pgfqpoint{1.515828in}{1.505420in}}%
\pgfpathlineto{\pgfqpoint{1.516930in}{1.505801in}}%
\pgfpathlineto{\pgfqpoint{1.520019in}{1.506910in}}%
\pgfpathlineto{\pgfqpoint{1.521129in}{1.507263in}}%
\pgfpathlineto{\pgfqpoint{1.524281in}{1.508372in}}%
\pgfpathlineto{\pgfqpoint{1.525322in}{1.508809in}}%
\pgfpathlineto{\pgfqpoint{1.528582in}{1.509917in}}%
\pgfpathlineto{\pgfqpoint{1.529658in}{1.510374in}}%
\pgfpathlineto{\pgfqpoint{1.533561in}{1.511482in}}%
\pgfpathlineto{\pgfqpoint{1.534659in}{1.511780in}}%
\pgfpathlineto{\pgfqpoint{1.538847in}{1.512888in}}%
\pgfpathlineto{\pgfqpoint{1.539945in}{1.513354in}}%
\pgfpathlineto{\pgfqpoint{1.543383in}{1.514462in}}%
\pgfpathlineto{\pgfqpoint{1.544467in}{1.514937in}}%
\pgfpathlineto{\pgfqpoint{1.547933in}{1.516045in}}%
\pgfpathlineto{\pgfqpoint{1.549002in}{1.516399in}}%
\pgfpathlineto{\pgfqpoint{1.552931in}{1.517507in}}%
\pgfpathlineto{\pgfqpoint{1.554024in}{1.517842in}}%
\pgfpathlineto{\pgfqpoint{1.558370in}{1.518951in}}%
\pgfpathlineto{\pgfqpoint{1.559434in}{1.519407in}}%
\pgfpathlineto{\pgfqpoint{1.559479in}{1.519407in}}%
\pgfpathlineto{\pgfqpoint{1.563072in}{1.520506in}}%
\pgfpathlineto{\pgfqpoint{1.564172in}{1.520822in}}%
\pgfpathlineto{\pgfqpoint{1.568147in}{1.521931in}}%
\pgfpathlineto{\pgfqpoint{1.569238in}{1.522303in}}%
\pgfpathlineto{\pgfqpoint{1.572709in}{1.523411in}}%
\pgfpathlineto{\pgfqpoint{1.573816in}{1.523924in}}%
\pgfpathlineto{\pgfqpoint{1.578448in}{1.525022in}}%
\pgfpathlineto{\pgfqpoint{1.579543in}{1.525311in}}%
\pgfpathlineto{\pgfqpoint{1.583575in}{1.526419in}}%
\pgfpathlineto{\pgfqpoint{1.584658in}{1.526792in}}%
\pgfpathlineto{\pgfqpoint{1.588389in}{1.527900in}}%
\pgfpathlineto{\pgfqpoint{1.589492in}{1.528161in}}%
\pgfpathlineto{\pgfqpoint{1.593514in}{1.529269in}}%
\pgfpathlineto{\pgfqpoint{1.594621in}{1.529641in}}%
\pgfpathlineto{\pgfqpoint{1.598481in}{1.530750in}}%
\pgfpathlineto{\pgfqpoint{1.599548in}{1.531085in}}%
\pgfpathlineto{\pgfqpoint{1.604169in}{1.532193in}}%
\pgfpathlineto{\pgfqpoint{1.605224in}{1.532435in}}%
\pgfpathlineto{\pgfqpoint{1.605259in}{1.532435in}}%
\pgfpathlineto{\pgfqpoint{1.609980in}{1.533543in}}%
\pgfpathlineto{\pgfqpoint{1.611068in}{1.533935in}}%
\pgfpathlineto{\pgfqpoint{1.615328in}{1.535043in}}%
\pgfpathlineto{\pgfqpoint{1.616413in}{1.535387in}}%
\pgfpathlineto{\pgfqpoint{1.621081in}{1.536496in}}%
\pgfpathlineto{\pgfqpoint{1.622173in}{1.536784in}}%
\pgfpathlineto{\pgfqpoint{1.622190in}{1.536784in}}%
\pgfpathlineto{\pgfqpoint{1.626388in}{1.537892in}}%
\pgfpathlineto{\pgfqpoint{1.627464in}{1.538172in}}%
\pgfpathlineto{\pgfqpoint{1.627488in}{1.538172in}}%
\pgfpathlineto{\pgfqpoint{1.631663in}{1.539280in}}%
\pgfpathlineto{\pgfqpoint{1.632758in}{1.539597in}}%
\pgfpathlineto{\pgfqpoint{1.637064in}{1.540705in}}%
\pgfpathlineto{\pgfqpoint{1.638168in}{1.540975in}}%
\pgfpathlineto{\pgfqpoint{1.642327in}{1.542083in}}%
\pgfpathlineto{\pgfqpoint{1.643368in}{1.542400in}}%
\pgfpathlineto{\pgfqpoint{1.643403in}{1.542400in}}%
\pgfpathlineto{\pgfqpoint{1.647634in}{1.543508in}}%
\pgfpathlineto{\pgfqpoint{1.648725in}{1.543871in}}%
\pgfpathlineto{\pgfqpoint{1.648741in}{1.543871in}}%
\pgfpathlineto{\pgfqpoint{1.652768in}{1.544979in}}%
\pgfpathlineto{\pgfqpoint{1.653823in}{1.545231in}}%
\pgfpathlineto{\pgfqpoint{1.653847in}{1.545231in}}%
\pgfpathlineto{\pgfqpoint{1.658171in}{1.546339in}}%
\pgfpathlineto{\pgfqpoint{1.659180in}{1.546656in}}%
\pgfpathlineto{\pgfqpoint{1.663767in}{1.547764in}}%
\pgfpathlineto{\pgfqpoint{1.664870in}{1.548099in}}%
\pgfpathlineto{\pgfqpoint{1.669190in}{1.549207in}}%
\pgfpathlineto{\pgfqpoint{1.670212in}{1.549440in}}%
\pgfpathlineto{\pgfqpoint{1.670243in}{1.549440in}}%
\pgfpathlineto{\pgfqpoint{1.675738in}{1.550548in}}%
\pgfpathlineto{\pgfqpoint{1.676821in}{1.550828in}}%
\pgfpathlineto{\pgfqpoint{1.681601in}{1.551936in}}%
\pgfpathlineto{\pgfqpoint{1.682544in}{1.552159in}}%
\pgfpathlineto{\pgfqpoint{1.687049in}{1.553267in}}%
\pgfpathlineto{\pgfqpoint{1.688140in}{1.553565in}}%
\pgfpathlineto{\pgfqpoint{1.692793in}{1.554674in}}%
\pgfpathlineto{\pgfqpoint{1.693897in}{1.554897in}}%
\pgfpathlineto{\pgfqpoint{1.698403in}{1.556005in}}%
\pgfpathlineto{\pgfqpoint{1.699503in}{1.556285in}}%
\pgfpathlineto{\pgfqpoint{1.704275in}{1.557393in}}%
\pgfpathlineto{\pgfqpoint{1.705342in}{1.557533in}}%
\pgfpathlineto{\pgfqpoint{1.711018in}{1.558641in}}%
\pgfpathlineto{\pgfqpoint{1.712064in}{1.558948in}}%
\pgfpathlineto{\pgfqpoint{1.712118in}{1.558948in}}%
\pgfpathlineto{\pgfqpoint{1.717686in}{1.560056in}}%
\pgfpathlineto{\pgfqpoint{1.718786in}{1.560392in}}%
\pgfpathlineto{\pgfqpoint{1.724466in}{1.561500in}}%
\pgfpathlineto{\pgfqpoint{1.725571in}{1.561835in}}%
\pgfpathlineto{\pgfqpoint{1.731169in}{1.562943in}}%
\pgfpathlineto{\pgfqpoint{1.732079in}{1.563204in}}%
\pgfpathlineto{\pgfqpoint{1.732116in}{1.563204in}}%
\pgfpathlineto{\pgfqpoint{1.737951in}{1.564312in}}%
\pgfpathlineto{\pgfqpoint{1.739009in}{1.564489in}}%
\pgfpathlineto{\pgfqpoint{1.739016in}{1.564489in}}%
\pgfpathlineto{\pgfqpoint{1.744312in}{1.565597in}}%
\pgfpathlineto{\pgfqpoint{1.745414in}{1.565830in}}%
\pgfpathlineto{\pgfqpoint{1.750663in}{1.566938in}}%
\pgfpathlineto{\pgfqpoint{1.751683in}{1.567218in}}%
\pgfpathlineto{\pgfqpoint{1.751718in}{1.567218in}}%
\pgfpathlineto{\pgfqpoint{1.758496in}{1.568326in}}%
\pgfpathlineto{\pgfqpoint{1.759538in}{1.568624in}}%
\pgfpathlineto{\pgfqpoint{1.765668in}{1.569732in}}%
\pgfpathlineto{\pgfqpoint{1.766726in}{1.569974in}}%
\pgfpathlineto{\pgfqpoint{1.772953in}{1.571082in}}%
\pgfpathlineto{\pgfqpoint{1.774060in}{1.571315in}}%
\pgfpathlineto{\pgfqpoint{1.779531in}{1.572423in}}%
\pgfpathlineto{\pgfqpoint{1.780495in}{1.572638in}}%
\pgfpathlineto{\pgfqpoint{1.780638in}{1.572638in}}%
\pgfpathlineto{\pgfqpoint{1.787144in}{1.573746in}}%
\pgfpathlineto{\pgfqpoint{1.788007in}{1.573857in}}%
\pgfpathlineto{\pgfqpoint{1.788035in}{1.573857in}}%
\pgfpathlineto{\pgfqpoint{1.794342in}{1.574966in}}%
\pgfpathlineto{\pgfqpoint{1.795449in}{1.575096in}}%
\pgfpathlineto{\pgfqpoint{1.801071in}{1.576204in}}%
\pgfpathlineto{\pgfqpoint{1.802163in}{1.576428in}}%
\pgfpathlineto{\pgfqpoint{1.808852in}{1.577536in}}%
\pgfpathlineto{\pgfqpoint{1.809849in}{1.577759in}}%
\pgfpathlineto{\pgfqpoint{1.809922in}{1.577759in}}%
\pgfpathlineto{\pgfqpoint{1.816887in}{1.578868in}}%
\pgfpathlineto{\pgfqpoint{1.817872in}{1.579082in}}%
\pgfpathlineto{\pgfqpoint{1.825445in}{1.580190in}}%
\pgfpathlineto{\pgfqpoint{1.826484in}{1.580441in}}%
\pgfpathlineto{\pgfqpoint{1.833192in}{1.581550in}}%
\pgfpathlineto{\pgfqpoint{1.834278in}{1.581773in}}%
\pgfpathlineto{\pgfqpoint{1.841630in}{1.582881in}}%
\pgfpathlineto{\pgfqpoint{1.842613in}{1.583012in}}%
\pgfpathlineto{\pgfqpoint{1.849963in}{1.584120in}}%
\pgfpathlineto{\pgfqpoint{1.851039in}{1.584288in}}%
\pgfpathlineto{\pgfqpoint{1.858268in}{1.585396in}}%
\pgfpathlineto{\pgfqpoint{1.859335in}{1.585563in}}%
\pgfpathlineto{\pgfqpoint{1.866934in}{1.586672in}}%
\pgfpathlineto{\pgfqpoint{1.867958in}{1.586774in}}%
\pgfpathlineto{\pgfqpoint{1.867987in}{1.586774in}}%
\pgfpathlineto{\pgfqpoint{1.876423in}{1.587882in}}%
\pgfpathlineto{\pgfqpoint{1.877523in}{1.588050in}}%
\pgfpathlineto{\pgfqpoint{1.884816in}{1.589158in}}%
\pgfpathlineto{\pgfqpoint{1.885919in}{1.589363in}}%
\pgfpathlineto{\pgfqpoint{1.895839in}{1.590471in}}%
\pgfpathlineto{\pgfqpoint{1.896592in}{1.590555in}}%
\pgfpathlineto{\pgfqpoint{1.896780in}{1.590555in}}%
\pgfpathlineto{\pgfqpoint{1.906890in}{1.591663in}}%
\pgfpathlineto{\pgfqpoint{1.907901in}{1.591803in}}%
\pgfpathlineto{\pgfqpoint{1.917719in}{1.592911in}}%
\pgfpathlineto{\pgfqpoint{1.918765in}{1.593060in}}%
\pgfpathlineto{\pgfqpoint{1.929058in}{1.594168in}}%
\pgfpathlineto{\pgfqpoint{1.930130in}{1.594280in}}%
\pgfpathlineto{\pgfqpoint{1.930158in}{1.594280in}}%
\pgfpathlineto{\pgfqpoint{1.940895in}{1.595388in}}%
\pgfpathlineto{\pgfqpoint{1.941917in}{1.595509in}}%
\pgfpathlineto{\pgfqpoint{1.941974in}{1.595509in}}%
\pgfpathlineto{\pgfqpoint{1.953541in}{1.596617in}}%
\pgfpathlineto{\pgfqpoint{1.954561in}{1.596720in}}%
\pgfpathlineto{\pgfqpoint{1.954627in}{1.596720in}}%
\pgfpathlineto{\pgfqpoint{1.968248in}{1.597828in}}%
\pgfpathlineto{\pgfqpoint{1.969343in}{1.597930in}}%
\pgfpathlineto{\pgfqpoint{1.983413in}{1.599039in}}%
\pgfpathlineto{\pgfqpoint{1.984506in}{1.599150in}}%
\pgfpathlineto{\pgfqpoint{1.999921in}{1.600259in}}%
\pgfpathlineto{\pgfqpoint{2.001005in}{1.600361in}}%
\pgfpathlineto{\pgfqpoint{2.020056in}{1.601469in}}%
\pgfpathlineto{\pgfqpoint{2.021067in}{1.601525in}}%
\pgfpathlineto{\pgfqpoint{2.021109in}{1.601525in}}%
\pgfpathlineto{\pgfqpoint{2.033126in}{1.601944in}}%
\pgfpathlineto{\pgfqpoint{2.033126in}{1.601944in}}%
\pgfusepath{stroke}%
\end{pgfscope}%
\begin{pgfscope}%
\pgfsetrectcap%
\pgfsetmiterjoin%
\pgfsetlinewidth{0.803000pt}%
\definecolor{currentstroke}{rgb}{0.000000,0.000000,0.000000}%
\pgfsetstrokecolor{currentstroke}%
\pgfsetdash{}{0pt}%
\pgfpathmoveto{\pgfqpoint{0.553581in}{0.499444in}}%
\pgfpathlineto{\pgfqpoint{0.553581in}{1.654444in}}%
\pgfusepath{stroke}%
\end{pgfscope}%
\begin{pgfscope}%
\pgfsetrectcap%
\pgfsetmiterjoin%
\pgfsetlinewidth{0.803000pt}%
\definecolor{currentstroke}{rgb}{0.000000,0.000000,0.000000}%
\pgfsetstrokecolor{currentstroke}%
\pgfsetdash{}{0pt}%
\pgfpathmoveto{\pgfqpoint{2.103581in}{0.499444in}}%
\pgfpathlineto{\pgfqpoint{2.103581in}{1.654444in}}%
\pgfusepath{stroke}%
\end{pgfscope}%
\begin{pgfscope}%
\pgfsetrectcap%
\pgfsetmiterjoin%
\pgfsetlinewidth{0.803000pt}%
\definecolor{currentstroke}{rgb}{0.000000,0.000000,0.000000}%
\pgfsetstrokecolor{currentstroke}%
\pgfsetdash{}{0pt}%
\pgfpathmoveto{\pgfqpoint{0.553581in}{0.499444in}}%
\pgfpathlineto{\pgfqpoint{2.103581in}{0.499444in}}%
\pgfusepath{stroke}%
\end{pgfscope}%
\begin{pgfscope}%
\pgfsetrectcap%
\pgfsetmiterjoin%
\pgfsetlinewidth{0.803000pt}%
\definecolor{currentstroke}{rgb}{0.000000,0.000000,0.000000}%
\pgfsetstrokecolor{currentstroke}%
\pgfsetdash{}{0pt}%
\pgfpathmoveto{\pgfqpoint{0.553581in}{1.654444in}}%
\pgfpathlineto{\pgfqpoint{2.103581in}{1.654444in}}%
\pgfusepath{stroke}%
\end{pgfscope}%
\begin{pgfscope}%
\pgfsetbuttcap%
\pgfsetmiterjoin%
\definecolor{currentfill}{rgb}{1.000000,1.000000,1.000000}%
\pgfsetfillcolor{currentfill}%
\pgfsetfillopacity{0.800000}%
\pgfsetlinewidth{1.003750pt}%
\definecolor{currentstroke}{rgb}{0.800000,0.800000,0.800000}%
\pgfsetstrokecolor{currentstroke}%
\pgfsetstrokeopacity{0.800000}%
\pgfsetdash{}{0pt}%
\pgfpathmoveto{\pgfqpoint{0.832747in}{0.568889in}}%
\pgfpathlineto{\pgfqpoint{2.006358in}{0.568889in}}%
\pgfpathquadraticcurveto{\pgfqpoint{2.034136in}{0.568889in}}{\pgfqpoint{2.034136in}{0.596666in}}%
\pgfpathlineto{\pgfqpoint{2.034136in}{0.776388in}}%
\pgfpathquadraticcurveto{\pgfqpoint{2.034136in}{0.804166in}}{\pgfqpoint{2.006358in}{0.804166in}}%
\pgfpathlineto{\pgfqpoint{0.832747in}{0.804166in}}%
\pgfpathquadraticcurveto{\pgfqpoint{0.804970in}{0.804166in}}{\pgfqpoint{0.804970in}{0.776388in}}%
\pgfpathlineto{\pgfqpoint{0.804970in}{0.596666in}}%
\pgfpathquadraticcurveto{\pgfqpoint{0.804970in}{0.568889in}}{\pgfqpoint{0.832747in}{0.568889in}}%
\pgfpathlineto{\pgfqpoint{0.832747in}{0.568889in}}%
\pgfpathclose%
\pgfusepath{stroke,fill}%
\end{pgfscope}%
\begin{pgfscope}%
\pgfsetrectcap%
\pgfsetroundjoin%
\pgfsetlinewidth{1.505625pt}%
\definecolor{currentstroke}{rgb}{0.000000,0.000000,0.000000}%
\pgfsetstrokecolor{currentstroke}%
\pgfsetdash{}{0pt}%
\pgfpathmoveto{\pgfqpoint{0.860525in}{0.700000in}}%
\pgfpathlineto{\pgfqpoint{0.999414in}{0.700000in}}%
\pgfpathlineto{\pgfqpoint{1.138303in}{0.700000in}}%
\pgfusepath{stroke}%
\end{pgfscope}%
\begin{pgfscope}%
\definecolor{textcolor}{rgb}{0.000000,0.000000,0.000000}%
\pgfsetstrokecolor{textcolor}%
\pgfsetfillcolor{textcolor}%
\pgftext[x=1.249414in,y=0.651388in,left,base]{\color{textcolor}\rmfamily\fontsize{10.000000}{12.000000}\selectfont AUC=0.752}%
\end{pgfscope}%
\end{pgfpicture}%
\makeatother%
\endgroup%

\end{tabular}


\

In this work we used two methods to give the results of different models similar distributions.  This case illustrates directly transforming the \verb|y_proba| values.  

To make a useful visualization of the results where we can see the interplay between the negative and positive classes, we can transform the data.  A transformation that preserves rank will have no effect on the ROC curve.  [Cite]  For the graph below, we mapped the smallest value in the set to 0 and the largest to 1.  

\

\verb|AdaBoost_5_Fold_Hard_Test_Transformed_100|

%
\noindent\begin{tabular}{@{\hspace{-6pt}}p{4.3in} @{\hspace{-6pt}}p{2.0in}}
	\vskip 0pt
	\hfil Raw Model Output
	
	%% Creator: Matplotlib, PGF backend
%%
%% To include the figure in your LaTeX document, write
%%   \input{<filename>.pgf}
%%
%% Make sure the required packages are loaded in your preamble
%%   \usepackage{pgf}
%%
%% Also ensure that all the required font packages are loaded; for instance,
%% the lmodern package is sometimes necessary when using math font.
%%   \usepackage{lmodern}
%%
%% Figures using additional raster images can only be included by \input if
%% they are in the same directory as the main LaTeX file. For loading figures
%% from other directories you can use the `import` package
%%   \usepackage{import}
%%
%% and then include the figures with
%%   \import{<path to file>}{<filename>.pgf}
%%
%% Matplotlib used the following preamble
%%   
%%   \usepackage{fontspec}
%%   \makeatletter\@ifpackageloaded{underscore}{}{\usepackage[strings]{underscore}}\makeatother
%%
\begingroup%
\makeatletter%
\begin{pgfpicture}%
\pgfpathrectangle{\pgfpointorigin}{\pgfqpoint{4.102500in}{1.754444in}}%
\pgfusepath{use as bounding box, clip}%
\begin{pgfscope}%
\pgfsetbuttcap%
\pgfsetmiterjoin%
\definecolor{currentfill}{rgb}{1.000000,1.000000,1.000000}%
\pgfsetfillcolor{currentfill}%
\pgfsetlinewidth{0.000000pt}%
\definecolor{currentstroke}{rgb}{1.000000,1.000000,1.000000}%
\pgfsetstrokecolor{currentstroke}%
\pgfsetdash{}{0pt}%
\pgfpathmoveto{\pgfqpoint{0.000000in}{0.000000in}}%
\pgfpathlineto{\pgfqpoint{4.102500in}{0.000000in}}%
\pgfpathlineto{\pgfqpoint{4.102500in}{1.754444in}}%
\pgfpathlineto{\pgfqpoint{0.000000in}{1.754444in}}%
\pgfpathlineto{\pgfqpoint{0.000000in}{0.000000in}}%
\pgfpathclose%
\pgfusepath{fill}%
\end{pgfscope}%
\begin{pgfscope}%
\pgfsetbuttcap%
\pgfsetmiterjoin%
\definecolor{currentfill}{rgb}{1.000000,1.000000,1.000000}%
\pgfsetfillcolor{currentfill}%
\pgfsetlinewidth{0.000000pt}%
\definecolor{currentstroke}{rgb}{0.000000,0.000000,0.000000}%
\pgfsetstrokecolor{currentstroke}%
\pgfsetstrokeopacity{0.000000}%
\pgfsetdash{}{0pt}%
\pgfpathmoveto{\pgfqpoint{0.515000in}{0.499444in}}%
\pgfpathlineto{\pgfqpoint{4.002500in}{0.499444in}}%
\pgfpathlineto{\pgfqpoint{4.002500in}{1.654444in}}%
\pgfpathlineto{\pgfqpoint{0.515000in}{1.654444in}}%
\pgfpathlineto{\pgfqpoint{0.515000in}{0.499444in}}%
\pgfpathclose%
\pgfusepath{fill}%
\end{pgfscope}%
\begin{pgfscope}%
\pgfpathrectangle{\pgfqpoint{0.515000in}{0.499444in}}{\pgfqpoint{3.487500in}{1.155000in}}%
\pgfusepath{clip}%
\pgfsetbuttcap%
\pgfsetmiterjoin%
\pgfsetlinewidth{1.003750pt}%
\definecolor{currentstroke}{rgb}{0.000000,0.000000,0.000000}%
\pgfsetstrokecolor{currentstroke}%
\pgfsetdash{}{0pt}%
\pgfpathmoveto{\pgfqpoint{0.610114in}{0.499444in}}%
\pgfpathlineto{\pgfqpoint{0.673523in}{0.499444in}}%
\pgfpathlineto{\pgfqpoint{0.673523in}{0.499455in}}%
\pgfpathlineto{\pgfqpoint{0.610114in}{0.499455in}}%
\pgfpathlineto{\pgfqpoint{0.610114in}{0.499444in}}%
\pgfpathclose%
\pgfusepath{stroke}%
\end{pgfscope}%
\begin{pgfscope}%
\pgfpathrectangle{\pgfqpoint{0.515000in}{0.499444in}}{\pgfqpoint{3.487500in}{1.155000in}}%
\pgfusepath{clip}%
\pgfsetbuttcap%
\pgfsetmiterjoin%
\pgfsetlinewidth{1.003750pt}%
\definecolor{currentstroke}{rgb}{0.000000,0.000000,0.000000}%
\pgfsetstrokecolor{currentstroke}%
\pgfsetdash{}{0pt}%
\pgfpathmoveto{\pgfqpoint{0.768637in}{0.499444in}}%
\pgfpathlineto{\pgfqpoint{0.832046in}{0.499444in}}%
\pgfpathlineto{\pgfqpoint{0.832046in}{0.500264in}}%
\pgfpathlineto{\pgfqpoint{0.768637in}{0.500264in}}%
\pgfpathlineto{\pgfqpoint{0.768637in}{0.499444in}}%
\pgfpathclose%
\pgfusepath{stroke}%
\end{pgfscope}%
\begin{pgfscope}%
\pgfpathrectangle{\pgfqpoint{0.515000in}{0.499444in}}{\pgfqpoint{3.487500in}{1.155000in}}%
\pgfusepath{clip}%
\pgfsetbuttcap%
\pgfsetmiterjoin%
\pgfsetlinewidth{1.003750pt}%
\definecolor{currentstroke}{rgb}{0.000000,0.000000,0.000000}%
\pgfsetstrokecolor{currentstroke}%
\pgfsetdash{}{0pt}%
\pgfpathmoveto{\pgfqpoint{0.927159in}{0.499444in}}%
\pgfpathlineto{\pgfqpoint{0.990568in}{0.499444in}}%
\pgfpathlineto{\pgfqpoint{0.990568in}{0.508497in}}%
\pgfpathlineto{\pgfqpoint{0.927159in}{0.508497in}}%
\pgfpathlineto{\pgfqpoint{0.927159in}{0.499444in}}%
\pgfpathclose%
\pgfusepath{stroke}%
\end{pgfscope}%
\begin{pgfscope}%
\pgfpathrectangle{\pgfqpoint{0.515000in}{0.499444in}}{\pgfqpoint{3.487500in}{1.155000in}}%
\pgfusepath{clip}%
\pgfsetbuttcap%
\pgfsetmiterjoin%
\pgfsetlinewidth{1.003750pt}%
\definecolor{currentstroke}{rgb}{0.000000,0.000000,0.000000}%
\pgfsetstrokecolor{currentstroke}%
\pgfsetdash{}{0pt}%
\pgfpathmoveto{\pgfqpoint{1.085682in}{0.499444in}}%
\pgfpathlineto{\pgfqpoint{1.149091in}{0.499444in}}%
\pgfpathlineto{\pgfqpoint{1.149091in}{0.547836in}}%
\pgfpathlineto{\pgfqpoint{1.085682in}{0.547836in}}%
\pgfpathlineto{\pgfqpoint{1.085682in}{0.499444in}}%
\pgfpathclose%
\pgfusepath{stroke}%
\end{pgfscope}%
\begin{pgfscope}%
\pgfpathrectangle{\pgfqpoint{0.515000in}{0.499444in}}{\pgfqpoint{3.487500in}{1.155000in}}%
\pgfusepath{clip}%
\pgfsetbuttcap%
\pgfsetmiterjoin%
\pgfsetlinewidth{1.003750pt}%
\definecolor{currentstroke}{rgb}{0.000000,0.000000,0.000000}%
\pgfsetstrokecolor{currentstroke}%
\pgfsetdash{}{0pt}%
\pgfpathmoveto{\pgfqpoint{1.244205in}{0.499444in}}%
\pgfpathlineto{\pgfqpoint{1.307614in}{0.499444in}}%
\pgfpathlineto{\pgfqpoint{1.307614in}{0.649999in}}%
\pgfpathlineto{\pgfqpoint{1.244205in}{0.649999in}}%
\pgfpathlineto{\pgfqpoint{1.244205in}{0.499444in}}%
\pgfpathclose%
\pgfusepath{stroke}%
\end{pgfscope}%
\begin{pgfscope}%
\pgfpathrectangle{\pgfqpoint{0.515000in}{0.499444in}}{\pgfqpoint{3.487500in}{1.155000in}}%
\pgfusepath{clip}%
\pgfsetbuttcap%
\pgfsetmiterjoin%
\pgfsetlinewidth{1.003750pt}%
\definecolor{currentstroke}{rgb}{0.000000,0.000000,0.000000}%
\pgfsetstrokecolor{currentstroke}%
\pgfsetdash{}{0pt}%
\pgfpathmoveto{\pgfqpoint{1.402728in}{0.499444in}}%
\pgfpathlineto{\pgfqpoint{1.466137in}{0.499444in}}%
\pgfpathlineto{\pgfqpoint{1.466137in}{0.844364in}}%
\pgfpathlineto{\pgfqpoint{1.402728in}{0.844364in}}%
\pgfpathlineto{\pgfqpoint{1.402728in}{0.499444in}}%
\pgfpathclose%
\pgfusepath{stroke}%
\end{pgfscope}%
\begin{pgfscope}%
\pgfpathrectangle{\pgfqpoint{0.515000in}{0.499444in}}{\pgfqpoint{3.487500in}{1.155000in}}%
\pgfusepath{clip}%
\pgfsetbuttcap%
\pgfsetmiterjoin%
\pgfsetlinewidth{1.003750pt}%
\definecolor{currentstroke}{rgb}{0.000000,0.000000,0.000000}%
\pgfsetstrokecolor{currentstroke}%
\pgfsetdash{}{0pt}%
\pgfpathmoveto{\pgfqpoint{1.561250in}{0.499444in}}%
\pgfpathlineto{\pgfqpoint{1.624659in}{0.499444in}}%
\pgfpathlineto{\pgfqpoint{1.624659in}{1.116390in}}%
\pgfpathlineto{\pgfqpoint{1.561250in}{1.116390in}}%
\pgfpathlineto{\pgfqpoint{1.561250in}{0.499444in}}%
\pgfpathclose%
\pgfusepath{stroke}%
\end{pgfscope}%
\begin{pgfscope}%
\pgfpathrectangle{\pgfqpoint{0.515000in}{0.499444in}}{\pgfqpoint{3.487500in}{1.155000in}}%
\pgfusepath{clip}%
\pgfsetbuttcap%
\pgfsetmiterjoin%
\pgfsetlinewidth{1.003750pt}%
\definecolor{currentstroke}{rgb}{0.000000,0.000000,0.000000}%
\pgfsetstrokecolor{currentstroke}%
\pgfsetdash{}{0pt}%
\pgfpathmoveto{\pgfqpoint{1.719773in}{0.499444in}}%
\pgfpathlineto{\pgfqpoint{1.783182in}{0.499444in}}%
\pgfpathlineto{\pgfqpoint{1.783182in}{1.393599in}}%
\pgfpathlineto{\pgfqpoint{1.719773in}{1.393599in}}%
\pgfpathlineto{\pgfqpoint{1.719773in}{0.499444in}}%
\pgfpathclose%
\pgfusepath{stroke}%
\end{pgfscope}%
\begin{pgfscope}%
\pgfpathrectangle{\pgfqpoint{0.515000in}{0.499444in}}{\pgfqpoint{3.487500in}{1.155000in}}%
\pgfusepath{clip}%
\pgfsetbuttcap%
\pgfsetmiterjoin%
\pgfsetlinewidth{1.003750pt}%
\definecolor{currentstroke}{rgb}{0.000000,0.000000,0.000000}%
\pgfsetstrokecolor{currentstroke}%
\pgfsetdash{}{0pt}%
\pgfpathmoveto{\pgfqpoint{1.878296in}{0.499444in}}%
\pgfpathlineto{\pgfqpoint{1.941705in}{0.499444in}}%
\pgfpathlineto{\pgfqpoint{1.941705in}{1.599444in}}%
\pgfpathlineto{\pgfqpoint{1.878296in}{1.599444in}}%
\pgfpathlineto{\pgfqpoint{1.878296in}{0.499444in}}%
\pgfpathclose%
\pgfusepath{stroke}%
\end{pgfscope}%
\begin{pgfscope}%
\pgfpathrectangle{\pgfqpoint{0.515000in}{0.499444in}}{\pgfqpoint{3.487500in}{1.155000in}}%
\pgfusepath{clip}%
\pgfsetbuttcap%
\pgfsetmiterjoin%
\pgfsetlinewidth{1.003750pt}%
\definecolor{currentstroke}{rgb}{0.000000,0.000000,0.000000}%
\pgfsetstrokecolor{currentstroke}%
\pgfsetdash{}{0pt}%
\pgfpathmoveto{\pgfqpoint{2.036818in}{0.499444in}}%
\pgfpathlineto{\pgfqpoint{2.100228in}{0.499444in}}%
\pgfpathlineto{\pgfqpoint{2.100228in}{1.590293in}}%
\pgfpathlineto{\pgfqpoint{2.036818in}{1.590293in}}%
\pgfpathlineto{\pgfqpoint{2.036818in}{0.499444in}}%
\pgfpathclose%
\pgfusepath{stroke}%
\end{pgfscope}%
\begin{pgfscope}%
\pgfpathrectangle{\pgfqpoint{0.515000in}{0.499444in}}{\pgfqpoint{3.487500in}{1.155000in}}%
\pgfusepath{clip}%
\pgfsetbuttcap%
\pgfsetmiterjoin%
\pgfsetlinewidth{1.003750pt}%
\definecolor{currentstroke}{rgb}{0.000000,0.000000,0.000000}%
\pgfsetstrokecolor{currentstroke}%
\pgfsetdash{}{0pt}%
\pgfpathmoveto{\pgfqpoint{2.195341in}{0.499444in}}%
\pgfpathlineto{\pgfqpoint{2.258750in}{0.499444in}}%
\pgfpathlineto{\pgfqpoint{2.258750in}{1.407998in}}%
\pgfpathlineto{\pgfqpoint{2.195341in}{1.407998in}}%
\pgfpathlineto{\pgfqpoint{2.195341in}{0.499444in}}%
\pgfpathclose%
\pgfusepath{stroke}%
\end{pgfscope}%
\begin{pgfscope}%
\pgfpathrectangle{\pgfqpoint{0.515000in}{0.499444in}}{\pgfqpoint{3.487500in}{1.155000in}}%
\pgfusepath{clip}%
\pgfsetbuttcap%
\pgfsetmiterjoin%
\pgfsetlinewidth{1.003750pt}%
\definecolor{currentstroke}{rgb}{0.000000,0.000000,0.000000}%
\pgfsetstrokecolor{currentstroke}%
\pgfsetdash{}{0pt}%
\pgfpathmoveto{\pgfqpoint{2.353864in}{0.499444in}}%
\pgfpathlineto{\pgfqpoint{2.417273in}{0.499444in}}%
\pgfpathlineto{\pgfqpoint{2.417273in}{1.136650in}}%
\pgfpathlineto{\pgfqpoint{2.353864in}{1.136650in}}%
\pgfpathlineto{\pgfqpoint{2.353864in}{0.499444in}}%
\pgfpathclose%
\pgfusepath{stroke}%
\end{pgfscope}%
\begin{pgfscope}%
\pgfpathrectangle{\pgfqpoint{0.515000in}{0.499444in}}{\pgfqpoint{3.487500in}{1.155000in}}%
\pgfusepath{clip}%
\pgfsetbuttcap%
\pgfsetmiterjoin%
\pgfsetlinewidth{1.003750pt}%
\definecolor{currentstroke}{rgb}{0.000000,0.000000,0.000000}%
\pgfsetstrokecolor{currentstroke}%
\pgfsetdash{}{0pt}%
\pgfpathmoveto{\pgfqpoint{2.512387in}{0.499444in}}%
\pgfpathlineto{\pgfqpoint{2.575796in}{0.499444in}}%
\pgfpathlineto{\pgfqpoint{2.575796in}{0.881297in}}%
\pgfpathlineto{\pgfqpoint{2.512387in}{0.881297in}}%
\pgfpathlineto{\pgfqpoint{2.512387in}{0.499444in}}%
\pgfpathclose%
\pgfusepath{stroke}%
\end{pgfscope}%
\begin{pgfscope}%
\pgfpathrectangle{\pgfqpoint{0.515000in}{0.499444in}}{\pgfqpoint{3.487500in}{1.155000in}}%
\pgfusepath{clip}%
\pgfsetbuttcap%
\pgfsetmiterjoin%
\pgfsetlinewidth{1.003750pt}%
\definecolor{currentstroke}{rgb}{0.000000,0.000000,0.000000}%
\pgfsetstrokecolor{currentstroke}%
\pgfsetdash{}{0pt}%
\pgfpathmoveto{\pgfqpoint{2.670909in}{0.499444in}}%
\pgfpathlineto{\pgfqpoint{2.734318in}{0.499444in}}%
\pgfpathlineto{\pgfqpoint{2.734318in}{0.707280in}}%
\pgfpathlineto{\pgfqpoint{2.670909in}{0.707280in}}%
\pgfpathlineto{\pgfqpoint{2.670909in}{0.499444in}}%
\pgfpathclose%
\pgfusepath{stroke}%
\end{pgfscope}%
\begin{pgfscope}%
\pgfpathrectangle{\pgfqpoint{0.515000in}{0.499444in}}{\pgfqpoint{3.487500in}{1.155000in}}%
\pgfusepath{clip}%
\pgfsetbuttcap%
\pgfsetmiterjoin%
\pgfsetlinewidth{1.003750pt}%
\definecolor{currentstroke}{rgb}{0.000000,0.000000,0.000000}%
\pgfsetstrokecolor{currentstroke}%
\pgfsetdash{}{0pt}%
\pgfpathmoveto{\pgfqpoint{2.829432in}{0.499444in}}%
\pgfpathlineto{\pgfqpoint{2.892841in}{0.499444in}}%
\pgfpathlineto{\pgfqpoint{2.892841in}{0.601323in}}%
\pgfpathlineto{\pgfqpoint{2.829432in}{0.601323in}}%
\pgfpathlineto{\pgfqpoint{2.829432in}{0.499444in}}%
\pgfpathclose%
\pgfusepath{stroke}%
\end{pgfscope}%
\begin{pgfscope}%
\pgfpathrectangle{\pgfqpoint{0.515000in}{0.499444in}}{\pgfqpoint{3.487500in}{1.155000in}}%
\pgfusepath{clip}%
\pgfsetbuttcap%
\pgfsetmiterjoin%
\pgfsetlinewidth{1.003750pt}%
\definecolor{currentstroke}{rgb}{0.000000,0.000000,0.000000}%
\pgfsetstrokecolor{currentstroke}%
\pgfsetdash{}{0pt}%
\pgfpathmoveto{\pgfqpoint{2.987955in}{0.499444in}}%
\pgfpathlineto{\pgfqpoint{3.051364in}{0.499444in}}%
\pgfpathlineto{\pgfqpoint{3.051364in}{0.545529in}}%
\pgfpathlineto{\pgfqpoint{2.987955in}{0.545529in}}%
\pgfpathlineto{\pgfqpoint{2.987955in}{0.499444in}}%
\pgfpathclose%
\pgfusepath{stroke}%
\end{pgfscope}%
\begin{pgfscope}%
\pgfpathrectangle{\pgfqpoint{0.515000in}{0.499444in}}{\pgfqpoint{3.487500in}{1.155000in}}%
\pgfusepath{clip}%
\pgfsetbuttcap%
\pgfsetmiterjoin%
\pgfsetlinewidth{1.003750pt}%
\definecolor{currentstroke}{rgb}{0.000000,0.000000,0.000000}%
\pgfsetstrokecolor{currentstroke}%
\pgfsetdash{}{0pt}%
\pgfpathmoveto{\pgfqpoint{3.146478in}{0.499444in}}%
\pgfpathlineto{\pgfqpoint{3.209887in}{0.499444in}}%
\pgfpathlineto{\pgfqpoint{3.209887in}{0.518633in}}%
\pgfpathlineto{\pgfqpoint{3.146478in}{0.518633in}}%
\pgfpathlineto{\pgfqpoint{3.146478in}{0.499444in}}%
\pgfpathclose%
\pgfusepath{stroke}%
\end{pgfscope}%
\begin{pgfscope}%
\pgfpathrectangle{\pgfqpoint{0.515000in}{0.499444in}}{\pgfqpoint{3.487500in}{1.155000in}}%
\pgfusepath{clip}%
\pgfsetbuttcap%
\pgfsetmiterjoin%
\pgfsetlinewidth{1.003750pt}%
\definecolor{currentstroke}{rgb}{0.000000,0.000000,0.000000}%
\pgfsetstrokecolor{currentstroke}%
\pgfsetdash{}{0pt}%
\pgfpathmoveto{\pgfqpoint{3.305000in}{0.499444in}}%
\pgfpathlineto{\pgfqpoint{3.368409in}{0.499444in}}%
\pgfpathlineto{\pgfqpoint{3.368409in}{0.506923in}}%
\pgfpathlineto{\pgfqpoint{3.305000in}{0.506923in}}%
\pgfpathlineto{\pgfqpoint{3.305000in}{0.499444in}}%
\pgfpathclose%
\pgfusepath{stroke}%
\end{pgfscope}%
\begin{pgfscope}%
\pgfpathrectangle{\pgfqpoint{0.515000in}{0.499444in}}{\pgfqpoint{3.487500in}{1.155000in}}%
\pgfusepath{clip}%
\pgfsetbuttcap%
\pgfsetmiterjoin%
\pgfsetlinewidth{1.003750pt}%
\definecolor{currentstroke}{rgb}{0.000000,0.000000,0.000000}%
\pgfsetstrokecolor{currentstroke}%
\pgfsetdash{}{0pt}%
\pgfpathmoveto{\pgfqpoint{3.463523in}{0.499444in}}%
\pgfpathlineto{\pgfqpoint{3.526932in}{0.499444in}}%
\pgfpathlineto{\pgfqpoint{3.526932in}{0.502145in}}%
\pgfpathlineto{\pgfqpoint{3.463523in}{0.502145in}}%
\pgfpathlineto{\pgfqpoint{3.463523in}{0.499444in}}%
\pgfpathclose%
\pgfusepath{stroke}%
\end{pgfscope}%
\begin{pgfscope}%
\pgfpathrectangle{\pgfqpoint{0.515000in}{0.499444in}}{\pgfqpoint{3.487500in}{1.155000in}}%
\pgfusepath{clip}%
\pgfsetbuttcap%
\pgfsetmiterjoin%
\pgfsetlinewidth{1.003750pt}%
\definecolor{currentstroke}{rgb}{0.000000,0.000000,0.000000}%
\pgfsetstrokecolor{currentstroke}%
\pgfsetdash{}{0pt}%
\pgfpathmoveto{\pgfqpoint{3.622046in}{0.499444in}}%
\pgfpathlineto{\pgfqpoint{3.685455in}{0.499444in}}%
\pgfpathlineto{\pgfqpoint{3.685455in}{0.499947in}}%
\pgfpathlineto{\pgfqpoint{3.622046in}{0.499947in}}%
\pgfpathlineto{\pgfqpoint{3.622046in}{0.499444in}}%
\pgfpathclose%
\pgfusepath{stroke}%
\end{pgfscope}%
\begin{pgfscope}%
\pgfpathrectangle{\pgfqpoint{0.515000in}{0.499444in}}{\pgfqpoint{3.487500in}{1.155000in}}%
\pgfusepath{clip}%
\pgfsetbuttcap%
\pgfsetmiterjoin%
\pgfsetlinewidth{1.003750pt}%
\definecolor{currentstroke}{rgb}{0.000000,0.000000,0.000000}%
\pgfsetstrokecolor{currentstroke}%
\pgfsetdash{}{0pt}%
\pgfpathmoveto{\pgfqpoint{3.780568in}{0.499444in}}%
\pgfpathlineto{\pgfqpoint{3.843978in}{0.499444in}}%
\pgfpathlineto{\pgfqpoint{3.843978in}{0.499488in}}%
\pgfpathlineto{\pgfqpoint{3.780568in}{0.499488in}}%
\pgfpathlineto{\pgfqpoint{3.780568in}{0.499444in}}%
\pgfpathclose%
\pgfusepath{stroke}%
\end{pgfscope}%
\begin{pgfscope}%
\pgfpathrectangle{\pgfqpoint{0.515000in}{0.499444in}}{\pgfqpoint{3.487500in}{1.155000in}}%
\pgfusepath{clip}%
\pgfsetbuttcap%
\pgfsetmiterjoin%
\definecolor{currentfill}{rgb}{0.000000,0.000000,0.000000}%
\pgfsetfillcolor{currentfill}%
\pgfsetlinewidth{0.000000pt}%
\definecolor{currentstroke}{rgb}{0.000000,0.000000,0.000000}%
\pgfsetstrokecolor{currentstroke}%
\pgfsetstrokeopacity{0.000000}%
\pgfsetdash{}{0pt}%
\pgfpathmoveto{\pgfqpoint{0.673523in}{0.499444in}}%
\pgfpathlineto{\pgfqpoint{0.736932in}{0.499444in}}%
\pgfpathlineto{\pgfqpoint{0.736932in}{0.499444in}}%
\pgfpathlineto{\pgfqpoint{0.673523in}{0.499444in}}%
\pgfpathlineto{\pgfqpoint{0.673523in}{0.499444in}}%
\pgfpathclose%
\pgfusepath{fill}%
\end{pgfscope}%
\begin{pgfscope}%
\pgfpathrectangle{\pgfqpoint{0.515000in}{0.499444in}}{\pgfqpoint{3.487500in}{1.155000in}}%
\pgfusepath{clip}%
\pgfsetbuttcap%
\pgfsetmiterjoin%
\definecolor{currentfill}{rgb}{0.000000,0.000000,0.000000}%
\pgfsetfillcolor{currentfill}%
\pgfsetlinewidth{0.000000pt}%
\definecolor{currentstroke}{rgb}{0.000000,0.000000,0.000000}%
\pgfsetstrokecolor{currentstroke}%
\pgfsetstrokeopacity{0.000000}%
\pgfsetdash{}{0pt}%
\pgfpathmoveto{\pgfqpoint{0.832046in}{0.499444in}}%
\pgfpathlineto{\pgfqpoint{0.895455in}{0.499444in}}%
\pgfpathlineto{\pgfqpoint{0.895455in}{0.499477in}}%
\pgfpathlineto{\pgfqpoint{0.832046in}{0.499477in}}%
\pgfpathlineto{\pgfqpoint{0.832046in}{0.499444in}}%
\pgfpathclose%
\pgfusepath{fill}%
\end{pgfscope}%
\begin{pgfscope}%
\pgfpathrectangle{\pgfqpoint{0.515000in}{0.499444in}}{\pgfqpoint{3.487500in}{1.155000in}}%
\pgfusepath{clip}%
\pgfsetbuttcap%
\pgfsetmiterjoin%
\definecolor{currentfill}{rgb}{0.000000,0.000000,0.000000}%
\pgfsetfillcolor{currentfill}%
\pgfsetlinewidth{0.000000pt}%
\definecolor{currentstroke}{rgb}{0.000000,0.000000,0.000000}%
\pgfsetstrokecolor{currentstroke}%
\pgfsetstrokeopacity{0.000000}%
\pgfsetdash{}{0pt}%
\pgfpathmoveto{\pgfqpoint{0.990568in}{0.499444in}}%
\pgfpathlineto{\pgfqpoint{1.053978in}{0.499444in}}%
\pgfpathlineto{\pgfqpoint{1.053978in}{0.499521in}}%
\pgfpathlineto{\pgfqpoint{0.990568in}{0.499521in}}%
\pgfpathlineto{\pgfqpoint{0.990568in}{0.499444in}}%
\pgfpathclose%
\pgfusepath{fill}%
\end{pgfscope}%
\begin{pgfscope}%
\pgfpathrectangle{\pgfqpoint{0.515000in}{0.499444in}}{\pgfqpoint{3.487500in}{1.155000in}}%
\pgfusepath{clip}%
\pgfsetbuttcap%
\pgfsetmiterjoin%
\definecolor{currentfill}{rgb}{0.000000,0.000000,0.000000}%
\pgfsetfillcolor{currentfill}%
\pgfsetlinewidth{0.000000pt}%
\definecolor{currentstroke}{rgb}{0.000000,0.000000,0.000000}%
\pgfsetstrokecolor{currentstroke}%
\pgfsetstrokeopacity{0.000000}%
\pgfsetdash{}{0pt}%
\pgfpathmoveto{\pgfqpoint{1.149091in}{0.499444in}}%
\pgfpathlineto{\pgfqpoint{1.212500in}{0.499444in}}%
\pgfpathlineto{\pgfqpoint{1.212500in}{0.499871in}}%
\pgfpathlineto{\pgfqpoint{1.149091in}{0.499871in}}%
\pgfpathlineto{\pgfqpoint{1.149091in}{0.499444in}}%
\pgfpathclose%
\pgfusepath{fill}%
\end{pgfscope}%
\begin{pgfscope}%
\pgfpathrectangle{\pgfqpoint{0.515000in}{0.499444in}}{\pgfqpoint{3.487500in}{1.155000in}}%
\pgfusepath{clip}%
\pgfsetbuttcap%
\pgfsetmiterjoin%
\definecolor{currentfill}{rgb}{0.000000,0.000000,0.000000}%
\pgfsetfillcolor{currentfill}%
\pgfsetlinewidth{0.000000pt}%
\definecolor{currentstroke}{rgb}{0.000000,0.000000,0.000000}%
\pgfsetstrokecolor{currentstroke}%
\pgfsetstrokeopacity{0.000000}%
\pgfsetdash{}{0pt}%
\pgfpathmoveto{\pgfqpoint{1.307614in}{0.499444in}}%
\pgfpathlineto{\pgfqpoint{1.371023in}{0.499444in}}%
\pgfpathlineto{\pgfqpoint{1.371023in}{0.501850in}}%
\pgfpathlineto{\pgfqpoint{1.307614in}{0.501850in}}%
\pgfpathlineto{\pgfqpoint{1.307614in}{0.499444in}}%
\pgfpathclose%
\pgfusepath{fill}%
\end{pgfscope}%
\begin{pgfscope}%
\pgfpathrectangle{\pgfqpoint{0.515000in}{0.499444in}}{\pgfqpoint{3.487500in}{1.155000in}}%
\pgfusepath{clip}%
\pgfsetbuttcap%
\pgfsetmiterjoin%
\definecolor{currentfill}{rgb}{0.000000,0.000000,0.000000}%
\pgfsetfillcolor{currentfill}%
\pgfsetlinewidth{0.000000pt}%
\definecolor{currentstroke}{rgb}{0.000000,0.000000,0.000000}%
\pgfsetstrokecolor{currentstroke}%
\pgfsetstrokeopacity{0.000000}%
\pgfsetdash{}{0pt}%
\pgfpathmoveto{\pgfqpoint{1.466137in}{0.499444in}}%
\pgfpathlineto{\pgfqpoint{1.529546in}{0.499444in}}%
\pgfpathlineto{\pgfqpoint{1.529546in}{0.507513in}}%
\pgfpathlineto{\pgfqpoint{1.466137in}{0.507513in}}%
\pgfpathlineto{\pgfqpoint{1.466137in}{0.499444in}}%
\pgfpathclose%
\pgfusepath{fill}%
\end{pgfscope}%
\begin{pgfscope}%
\pgfpathrectangle{\pgfqpoint{0.515000in}{0.499444in}}{\pgfqpoint{3.487500in}{1.155000in}}%
\pgfusepath{clip}%
\pgfsetbuttcap%
\pgfsetmiterjoin%
\definecolor{currentfill}{rgb}{0.000000,0.000000,0.000000}%
\pgfsetfillcolor{currentfill}%
\pgfsetlinewidth{0.000000pt}%
\definecolor{currentstroke}{rgb}{0.000000,0.000000,0.000000}%
\pgfsetstrokecolor{currentstroke}%
\pgfsetstrokeopacity{0.000000}%
\pgfsetdash{}{0pt}%
\pgfpathmoveto{\pgfqpoint{1.624659in}{0.499444in}}%
\pgfpathlineto{\pgfqpoint{1.688068in}{0.499444in}}%
\pgfpathlineto{\pgfqpoint{1.688068in}{0.522525in}}%
\pgfpathlineto{\pgfqpoint{1.624659in}{0.522525in}}%
\pgfpathlineto{\pgfqpoint{1.624659in}{0.499444in}}%
\pgfpathclose%
\pgfusepath{fill}%
\end{pgfscope}%
\begin{pgfscope}%
\pgfpathrectangle{\pgfqpoint{0.515000in}{0.499444in}}{\pgfqpoint{3.487500in}{1.155000in}}%
\pgfusepath{clip}%
\pgfsetbuttcap%
\pgfsetmiterjoin%
\definecolor{currentfill}{rgb}{0.000000,0.000000,0.000000}%
\pgfsetfillcolor{currentfill}%
\pgfsetlinewidth{0.000000pt}%
\definecolor{currentstroke}{rgb}{0.000000,0.000000,0.000000}%
\pgfsetstrokecolor{currentstroke}%
\pgfsetstrokeopacity{0.000000}%
\pgfsetdash{}{0pt}%
\pgfpathmoveto{\pgfqpoint{1.783182in}{0.499444in}}%
\pgfpathlineto{\pgfqpoint{1.846591in}{0.499444in}}%
\pgfpathlineto{\pgfqpoint{1.846591in}{0.551794in}}%
\pgfpathlineto{\pgfqpoint{1.783182in}{0.551794in}}%
\pgfpathlineto{\pgfqpoint{1.783182in}{0.499444in}}%
\pgfpathclose%
\pgfusepath{fill}%
\end{pgfscope}%
\begin{pgfscope}%
\pgfpathrectangle{\pgfqpoint{0.515000in}{0.499444in}}{\pgfqpoint{3.487500in}{1.155000in}}%
\pgfusepath{clip}%
\pgfsetbuttcap%
\pgfsetmiterjoin%
\definecolor{currentfill}{rgb}{0.000000,0.000000,0.000000}%
\pgfsetfillcolor{currentfill}%
\pgfsetlinewidth{0.000000pt}%
\definecolor{currentstroke}{rgb}{0.000000,0.000000,0.000000}%
\pgfsetstrokecolor{currentstroke}%
\pgfsetstrokeopacity{0.000000}%
\pgfsetdash{}{0pt}%
\pgfpathmoveto{\pgfqpoint{1.941705in}{0.499444in}}%
\pgfpathlineto{\pgfqpoint{2.005114in}{0.499444in}}%
\pgfpathlineto{\pgfqpoint{2.005114in}{0.599770in}}%
\pgfpathlineto{\pgfqpoint{1.941705in}{0.599770in}}%
\pgfpathlineto{\pgfqpoint{1.941705in}{0.499444in}}%
\pgfpathclose%
\pgfusepath{fill}%
\end{pgfscope}%
\begin{pgfscope}%
\pgfpathrectangle{\pgfqpoint{0.515000in}{0.499444in}}{\pgfqpoint{3.487500in}{1.155000in}}%
\pgfusepath{clip}%
\pgfsetbuttcap%
\pgfsetmiterjoin%
\definecolor{currentfill}{rgb}{0.000000,0.000000,0.000000}%
\pgfsetfillcolor{currentfill}%
\pgfsetlinewidth{0.000000pt}%
\definecolor{currentstroke}{rgb}{0.000000,0.000000,0.000000}%
\pgfsetstrokecolor{currentstroke}%
\pgfsetstrokeopacity{0.000000}%
\pgfsetdash{}{0pt}%
\pgfpathmoveto{\pgfqpoint{2.100228in}{0.499444in}}%
\pgfpathlineto{\pgfqpoint{2.163637in}{0.499444in}}%
\pgfpathlineto{\pgfqpoint{2.163637in}{0.655608in}}%
\pgfpathlineto{\pgfqpoint{2.100228in}{0.655608in}}%
\pgfpathlineto{\pgfqpoint{2.100228in}{0.499444in}}%
\pgfpathclose%
\pgfusepath{fill}%
\end{pgfscope}%
\begin{pgfscope}%
\pgfpathrectangle{\pgfqpoint{0.515000in}{0.499444in}}{\pgfqpoint{3.487500in}{1.155000in}}%
\pgfusepath{clip}%
\pgfsetbuttcap%
\pgfsetmiterjoin%
\definecolor{currentfill}{rgb}{0.000000,0.000000,0.000000}%
\pgfsetfillcolor{currentfill}%
\pgfsetlinewidth{0.000000pt}%
\definecolor{currentstroke}{rgb}{0.000000,0.000000,0.000000}%
\pgfsetstrokecolor{currentstroke}%
\pgfsetstrokeopacity{0.000000}%
\pgfsetdash{}{0pt}%
\pgfpathmoveto{\pgfqpoint{2.258750in}{0.499444in}}%
\pgfpathlineto{\pgfqpoint{2.322159in}{0.499444in}}%
\pgfpathlineto{\pgfqpoint{2.322159in}{0.692530in}}%
\pgfpathlineto{\pgfqpoint{2.258750in}{0.692530in}}%
\pgfpathlineto{\pgfqpoint{2.258750in}{0.499444in}}%
\pgfpathclose%
\pgfusepath{fill}%
\end{pgfscope}%
\begin{pgfscope}%
\pgfpathrectangle{\pgfqpoint{0.515000in}{0.499444in}}{\pgfqpoint{3.487500in}{1.155000in}}%
\pgfusepath{clip}%
\pgfsetbuttcap%
\pgfsetmiterjoin%
\definecolor{currentfill}{rgb}{0.000000,0.000000,0.000000}%
\pgfsetfillcolor{currentfill}%
\pgfsetlinewidth{0.000000pt}%
\definecolor{currentstroke}{rgb}{0.000000,0.000000,0.000000}%
\pgfsetstrokecolor{currentstroke}%
\pgfsetstrokeopacity{0.000000}%
\pgfsetdash{}{0pt}%
\pgfpathmoveto{\pgfqpoint{2.417273in}{0.499444in}}%
\pgfpathlineto{\pgfqpoint{2.480682in}{0.499444in}}%
\pgfpathlineto{\pgfqpoint{2.480682in}{0.698686in}}%
\pgfpathlineto{\pgfqpoint{2.417273in}{0.698686in}}%
\pgfpathlineto{\pgfqpoint{2.417273in}{0.499444in}}%
\pgfpathclose%
\pgfusepath{fill}%
\end{pgfscope}%
\begin{pgfscope}%
\pgfpathrectangle{\pgfqpoint{0.515000in}{0.499444in}}{\pgfqpoint{3.487500in}{1.155000in}}%
\pgfusepath{clip}%
\pgfsetbuttcap%
\pgfsetmiterjoin%
\definecolor{currentfill}{rgb}{0.000000,0.000000,0.000000}%
\pgfsetfillcolor{currentfill}%
\pgfsetlinewidth{0.000000pt}%
\definecolor{currentstroke}{rgb}{0.000000,0.000000,0.000000}%
\pgfsetstrokecolor{currentstroke}%
\pgfsetstrokeopacity{0.000000}%
\pgfsetdash{}{0pt}%
\pgfpathmoveto{\pgfqpoint{2.575796in}{0.499444in}}%
\pgfpathlineto{\pgfqpoint{2.639205in}{0.499444in}}%
\pgfpathlineto{\pgfqpoint{2.639205in}{0.673834in}}%
\pgfpathlineto{\pgfqpoint{2.575796in}{0.673834in}}%
\pgfpathlineto{\pgfqpoint{2.575796in}{0.499444in}}%
\pgfpathclose%
\pgfusepath{fill}%
\end{pgfscope}%
\begin{pgfscope}%
\pgfpathrectangle{\pgfqpoint{0.515000in}{0.499444in}}{\pgfqpoint{3.487500in}{1.155000in}}%
\pgfusepath{clip}%
\pgfsetbuttcap%
\pgfsetmiterjoin%
\definecolor{currentfill}{rgb}{0.000000,0.000000,0.000000}%
\pgfsetfillcolor{currentfill}%
\pgfsetlinewidth{0.000000pt}%
\definecolor{currentstroke}{rgb}{0.000000,0.000000,0.000000}%
\pgfsetstrokecolor{currentstroke}%
\pgfsetstrokeopacity{0.000000}%
\pgfsetdash{}{0pt}%
\pgfpathmoveto{\pgfqpoint{2.734318in}{0.499444in}}%
\pgfpathlineto{\pgfqpoint{2.797728in}{0.499444in}}%
\pgfpathlineto{\pgfqpoint{2.797728in}{0.633697in}}%
\pgfpathlineto{\pgfqpoint{2.734318in}{0.633697in}}%
\pgfpathlineto{\pgfqpoint{2.734318in}{0.499444in}}%
\pgfpathclose%
\pgfusepath{fill}%
\end{pgfscope}%
\begin{pgfscope}%
\pgfpathrectangle{\pgfqpoint{0.515000in}{0.499444in}}{\pgfqpoint{3.487500in}{1.155000in}}%
\pgfusepath{clip}%
\pgfsetbuttcap%
\pgfsetmiterjoin%
\definecolor{currentfill}{rgb}{0.000000,0.000000,0.000000}%
\pgfsetfillcolor{currentfill}%
\pgfsetlinewidth{0.000000pt}%
\definecolor{currentstroke}{rgb}{0.000000,0.000000,0.000000}%
\pgfsetstrokecolor{currentstroke}%
\pgfsetstrokeopacity{0.000000}%
\pgfsetdash{}{0pt}%
\pgfpathmoveto{\pgfqpoint{2.892841in}{0.499444in}}%
\pgfpathlineto{\pgfqpoint{2.956250in}{0.499444in}}%
\pgfpathlineto{\pgfqpoint{2.956250in}{0.589033in}}%
\pgfpathlineto{\pgfqpoint{2.892841in}{0.589033in}}%
\pgfpathlineto{\pgfqpoint{2.892841in}{0.499444in}}%
\pgfpathclose%
\pgfusepath{fill}%
\end{pgfscope}%
\begin{pgfscope}%
\pgfpathrectangle{\pgfqpoint{0.515000in}{0.499444in}}{\pgfqpoint{3.487500in}{1.155000in}}%
\pgfusepath{clip}%
\pgfsetbuttcap%
\pgfsetmiterjoin%
\definecolor{currentfill}{rgb}{0.000000,0.000000,0.000000}%
\pgfsetfillcolor{currentfill}%
\pgfsetlinewidth{0.000000pt}%
\definecolor{currentstroke}{rgb}{0.000000,0.000000,0.000000}%
\pgfsetstrokecolor{currentstroke}%
\pgfsetstrokeopacity{0.000000}%
\pgfsetdash{}{0pt}%
\pgfpathmoveto{\pgfqpoint{3.051364in}{0.499444in}}%
\pgfpathlineto{\pgfqpoint{3.114773in}{0.499444in}}%
\pgfpathlineto{\pgfqpoint{3.114773in}{0.549935in}}%
\pgfpathlineto{\pgfqpoint{3.051364in}{0.549935in}}%
\pgfpathlineto{\pgfqpoint{3.051364in}{0.499444in}}%
\pgfpathclose%
\pgfusepath{fill}%
\end{pgfscope}%
\begin{pgfscope}%
\pgfpathrectangle{\pgfqpoint{0.515000in}{0.499444in}}{\pgfqpoint{3.487500in}{1.155000in}}%
\pgfusepath{clip}%
\pgfsetbuttcap%
\pgfsetmiterjoin%
\definecolor{currentfill}{rgb}{0.000000,0.000000,0.000000}%
\pgfsetfillcolor{currentfill}%
\pgfsetlinewidth{0.000000pt}%
\definecolor{currentstroke}{rgb}{0.000000,0.000000,0.000000}%
\pgfsetstrokecolor{currentstroke}%
\pgfsetstrokeopacity{0.000000}%
\pgfsetdash{}{0pt}%
\pgfpathmoveto{\pgfqpoint{3.209887in}{0.499444in}}%
\pgfpathlineto{\pgfqpoint{3.273296in}{0.499444in}}%
\pgfpathlineto{\pgfqpoint{3.273296in}{0.525838in}}%
\pgfpathlineto{\pgfqpoint{3.209887in}{0.525838in}}%
\pgfpathlineto{\pgfqpoint{3.209887in}{0.499444in}}%
\pgfpathclose%
\pgfusepath{fill}%
\end{pgfscope}%
\begin{pgfscope}%
\pgfpathrectangle{\pgfqpoint{0.515000in}{0.499444in}}{\pgfqpoint{3.487500in}{1.155000in}}%
\pgfusepath{clip}%
\pgfsetbuttcap%
\pgfsetmiterjoin%
\definecolor{currentfill}{rgb}{0.000000,0.000000,0.000000}%
\pgfsetfillcolor{currentfill}%
\pgfsetlinewidth{0.000000pt}%
\definecolor{currentstroke}{rgb}{0.000000,0.000000,0.000000}%
\pgfsetstrokecolor{currentstroke}%
\pgfsetstrokeopacity{0.000000}%
\pgfsetdash{}{0pt}%
\pgfpathmoveto{\pgfqpoint{3.368409in}{0.499444in}}%
\pgfpathlineto{\pgfqpoint{3.431818in}{0.499444in}}%
\pgfpathlineto{\pgfqpoint{3.431818in}{0.513691in}}%
\pgfpathlineto{\pgfqpoint{3.368409in}{0.513691in}}%
\pgfpathlineto{\pgfqpoint{3.368409in}{0.499444in}}%
\pgfpathclose%
\pgfusepath{fill}%
\end{pgfscope}%
\begin{pgfscope}%
\pgfpathrectangle{\pgfqpoint{0.515000in}{0.499444in}}{\pgfqpoint{3.487500in}{1.155000in}}%
\pgfusepath{clip}%
\pgfsetbuttcap%
\pgfsetmiterjoin%
\definecolor{currentfill}{rgb}{0.000000,0.000000,0.000000}%
\pgfsetfillcolor{currentfill}%
\pgfsetlinewidth{0.000000pt}%
\definecolor{currentstroke}{rgb}{0.000000,0.000000,0.000000}%
\pgfsetstrokecolor{currentstroke}%
\pgfsetstrokeopacity{0.000000}%
\pgfsetdash{}{0pt}%
\pgfpathmoveto{\pgfqpoint{3.526932in}{0.499444in}}%
\pgfpathlineto{\pgfqpoint{3.590341in}{0.499444in}}%
\pgfpathlineto{\pgfqpoint{3.590341in}{0.505720in}}%
\pgfpathlineto{\pgfqpoint{3.526932in}{0.505720in}}%
\pgfpathlineto{\pgfqpoint{3.526932in}{0.499444in}}%
\pgfpathclose%
\pgfusepath{fill}%
\end{pgfscope}%
\begin{pgfscope}%
\pgfpathrectangle{\pgfqpoint{0.515000in}{0.499444in}}{\pgfqpoint{3.487500in}{1.155000in}}%
\pgfusepath{clip}%
\pgfsetbuttcap%
\pgfsetmiterjoin%
\definecolor{currentfill}{rgb}{0.000000,0.000000,0.000000}%
\pgfsetfillcolor{currentfill}%
\pgfsetlinewidth{0.000000pt}%
\definecolor{currentstroke}{rgb}{0.000000,0.000000,0.000000}%
\pgfsetstrokecolor{currentstroke}%
\pgfsetstrokeopacity{0.000000}%
\pgfsetdash{}{0pt}%
\pgfpathmoveto{\pgfqpoint{3.685455in}{0.499444in}}%
\pgfpathlineto{\pgfqpoint{3.748864in}{0.499444in}}%
\pgfpathlineto{\pgfqpoint{3.748864in}{0.501030in}}%
\pgfpathlineto{\pgfqpoint{3.685455in}{0.501030in}}%
\pgfpathlineto{\pgfqpoint{3.685455in}{0.499444in}}%
\pgfpathclose%
\pgfusepath{fill}%
\end{pgfscope}%
\begin{pgfscope}%
\pgfpathrectangle{\pgfqpoint{0.515000in}{0.499444in}}{\pgfqpoint{3.487500in}{1.155000in}}%
\pgfusepath{clip}%
\pgfsetbuttcap%
\pgfsetmiterjoin%
\definecolor{currentfill}{rgb}{0.000000,0.000000,0.000000}%
\pgfsetfillcolor{currentfill}%
\pgfsetlinewidth{0.000000pt}%
\definecolor{currentstroke}{rgb}{0.000000,0.000000,0.000000}%
\pgfsetstrokecolor{currentstroke}%
\pgfsetstrokeopacity{0.000000}%
\pgfsetdash{}{0pt}%
\pgfpathmoveto{\pgfqpoint{3.843978in}{0.499444in}}%
\pgfpathlineto{\pgfqpoint{3.907387in}{0.499444in}}%
\pgfpathlineto{\pgfqpoint{3.907387in}{0.499728in}}%
\pgfpathlineto{\pgfqpoint{3.843978in}{0.499728in}}%
\pgfpathlineto{\pgfqpoint{3.843978in}{0.499444in}}%
\pgfpathclose%
\pgfusepath{fill}%
\end{pgfscope}%
\begin{pgfscope}%
\pgfsetbuttcap%
\pgfsetroundjoin%
\definecolor{currentfill}{rgb}{0.000000,0.000000,0.000000}%
\pgfsetfillcolor{currentfill}%
\pgfsetlinewidth{0.803000pt}%
\definecolor{currentstroke}{rgb}{0.000000,0.000000,0.000000}%
\pgfsetstrokecolor{currentstroke}%
\pgfsetdash{}{0pt}%
\pgfsys@defobject{currentmarker}{\pgfqpoint{0.000000in}{-0.048611in}}{\pgfqpoint{0.000000in}{0.000000in}}{%
\pgfpathmoveto{\pgfqpoint{0.000000in}{0.000000in}}%
\pgfpathlineto{\pgfqpoint{0.000000in}{-0.048611in}}%
\pgfusepath{stroke,fill}%
}%
\begin{pgfscope}%
\pgfsys@transformshift{0.515000in}{0.499444in}%
\pgfsys@useobject{currentmarker}{}%
\end{pgfscope}%
\end{pgfscope}%
\begin{pgfscope}%
\pgfsetbuttcap%
\pgfsetroundjoin%
\definecolor{currentfill}{rgb}{0.000000,0.000000,0.000000}%
\pgfsetfillcolor{currentfill}%
\pgfsetlinewidth{0.803000pt}%
\definecolor{currentstroke}{rgb}{0.000000,0.000000,0.000000}%
\pgfsetstrokecolor{currentstroke}%
\pgfsetdash{}{0pt}%
\pgfsys@defobject{currentmarker}{\pgfqpoint{0.000000in}{-0.048611in}}{\pgfqpoint{0.000000in}{0.000000in}}{%
\pgfpathmoveto{\pgfqpoint{0.000000in}{0.000000in}}%
\pgfpathlineto{\pgfqpoint{0.000000in}{-0.048611in}}%
\pgfusepath{stroke,fill}%
}%
\begin{pgfscope}%
\pgfsys@transformshift{0.673523in}{0.499444in}%
\pgfsys@useobject{currentmarker}{}%
\end{pgfscope}%
\end{pgfscope}%
\begin{pgfscope}%
\definecolor{textcolor}{rgb}{0.000000,0.000000,0.000000}%
\pgfsetstrokecolor{textcolor}%
\pgfsetfillcolor{textcolor}%
\pgftext[x=0.673523in,y=0.402222in,,top]{\color{textcolor}\rmfamily\fontsize{10.000000}{12.000000}\selectfont 0.0}%
\end{pgfscope}%
\begin{pgfscope}%
\pgfsetbuttcap%
\pgfsetroundjoin%
\definecolor{currentfill}{rgb}{0.000000,0.000000,0.000000}%
\pgfsetfillcolor{currentfill}%
\pgfsetlinewidth{0.803000pt}%
\definecolor{currentstroke}{rgb}{0.000000,0.000000,0.000000}%
\pgfsetstrokecolor{currentstroke}%
\pgfsetdash{}{0pt}%
\pgfsys@defobject{currentmarker}{\pgfqpoint{0.000000in}{-0.048611in}}{\pgfqpoint{0.000000in}{0.000000in}}{%
\pgfpathmoveto{\pgfqpoint{0.000000in}{0.000000in}}%
\pgfpathlineto{\pgfqpoint{0.000000in}{-0.048611in}}%
\pgfusepath{stroke,fill}%
}%
\begin{pgfscope}%
\pgfsys@transformshift{0.832046in}{0.499444in}%
\pgfsys@useobject{currentmarker}{}%
\end{pgfscope}%
\end{pgfscope}%
\begin{pgfscope}%
\pgfsetbuttcap%
\pgfsetroundjoin%
\definecolor{currentfill}{rgb}{0.000000,0.000000,0.000000}%
\pgfsetfillcolor{currentfill}%
\pgfsetlinewidth{0.803000pt}%
\definecolor{currentstroke}{rgb}{0.000000,0.000000,0.000000}%
\pgfsetstrokecolor{currentstroke}%
\pgfsetdash{}{0pt}%
\pgfsys@defobject{currentmarker}{\pgfqpoint{0.000000in}{-0.048611in}}{\pgfqpoint{0.000000in}{0.000000in}}{%
\pgfpathmoveto{\pgfqpoint{0.000000in}{0.000000in}}%
\pgfpathlineto{\pgfqpoint{0.000000in}{-0.048611in}}%
\pgfusepath{stroke,fill}%
}%
\begin{pgfscope}%
\pgfsys@transformshift{0.990568in}{0.499444in}%
\pgfsys@useobject{currentmarker}{}%
\end{pgfscope}%
\end{pgfscope}%
\begin{pgfscope}%
\definecolor{textcolor}{rgb}{0.000000,0.000000,0.000000}%
\pgfsetstrokecolor{textcolor}%
\pgfsetfillcolor{textcolor}%
\pgftext[x=0.990568in,y=0.402222in,,top]{\color{textcolor}\rmfamily\fontsize{10.000000}{12.000000}\selectfont 0.1}%
\end{pgfscope}%
\begin{pgfscope}%
\pgfsetbuttcap%
\pgfsetroundjoin%
\definecolor{currentfill}{rgb}{0.000000,0.000000,0.000000}%
\pgfsetfillcolor{currentfill}%
\pgfsetlinewidth{0.803000pt}%
\definecolor{currentstroke}{rgb}{0.000000,0.000000,0.000000}%
\pgfsetstrokecolor{currentstroke}%
\pgfsetdash{}{0pt}%
\pgfsys@defobject{currentmarker}{\pgfqpoint{0.000000in}{-0.048611in}}{\pgfqpoint{0.000000in}{0.000000in}}{%
\pgfpathmoveto{\pgfqpoint{0.000000in}{0.000000in}}%
\pgfpathlineto{\pgfqpoint{0.000000in}{-0.048611in}}%
\pgfusepath{stroke,fill}%
}%
\begin{pgfscope}%
\pgfsys@transformshift{1.149091in}{0.499444in}%
\pgfsys@useobject{currentmarker}{}%
\end{pgfscope}%
\end{pgfscope}%
\begin{pgfscope}%
\pgfsetbuttcap%
\pgfsetroundjoin%
\definecolor{currentfill}{rgb}{0.000000,0.000000,0.000000}%
\pgfsetfillcolor{currentfill}%
\pgfsetlinewidth{0.803000pt}%
\definecolor{currentstroke}{rgb}{0.000000,0.000000,0.000000}%
\pgfsetstrokecolor{currentstroke}%
\pgfsetdash{}{0pt}%
\pgfsys@defobject{currentmarker}{\pgfqpoint{0.000000in}{-0.048611in}}{\pgfqpoint{0.000000in}{0.000000in}}{%
\pgfpathmoveto{\pgfqpoint{0.000000in}{0.000000in}}%
\pgfpathlineto{\pgfqpoint{0.000000in}{-0.048611in}}%
\pgfusepath{stroke,fill}%
}%
\begin{pgfscope}%
\pgfsys@transformshift{1.307614in}{0.499444in}%
\pgfsys@useobject{currentmarker}{}%
\end{pgfscope}%
\end{pgfscope}%
\begin{pgfscope}%
\definecolor{textcolor}{rgb}{0.000000,0.000000,0.000000}%
\pgfsetstrokecolor{textcolor}%
\pgfsetfillcolor{textcolor}%
\pgftext[x=1.307614in,y=0.402222in,,top]{\color{textcolor}\rmfamily\fontsize{10.000000}{12.000000}\selectfont 0.2}%
\end{pgfscope}%
\begin{pgfscope}%
\pgfsetbuttcap%
\pgfsetroundjoin%
\definecolor{currentfill}{rgb}{0.000000,0.000000,0.000000}%
\pgfsetfillcolor{currentfill}%
\pgfsetlinewidth{0.803000pt}%
\definecolor{currentstroke}{rgb}{0.000000,0.000000,0.000000}%
\pgfsetstrokecolor{currentstroke}%
\pgfsetdash{}{0pt}%
\pgfsys@defobject{currentmarker}{\pgfqpoint{0.000000in}{-0.048611in}}{\pgfqpoint{0.000000in}{0.000000in}}{%
\pgfpathmoveto{\pgfqpoint{0.000000in}{0.000000in}}%
\pgfpathlineto{\pgfqpoint{0.000000in}{-0.048611in}}%
\pgfusepath{stroke,fill}%
}%
\begin{pgfscope}%
\pgfsys@transformshift{1.466137in}{0.499444in}%
\pgfsys@useobject{currentmarker}{}%
\end{pgfscope}%
\end{pgfscope}%
\begin{pgfscope}%
\pgfsetbuttcap%
\pgfsetroundjoin%
\definecolor{currentfill}{rgb}{0.000000,0.000000,0.000000}%
\pgfsetfillcolor{currentfill}%
\pgfsetlinewidth{0.803000pt}%
\definecolor{currentstroke}{rgb}{0.000000,0.000000,0.000000}%
\pgfsetstrokecolor{currentstroke}%
\pgfsetdash{}{0pt}%
\pgfsys@defobject{currentmarker}{\pgfqpoint{0.000000in}{-0.048611in}}{\pgfqpoint{0.000000in}{0.000000in}}{%
\pgfpathmoveto{\pgfqpoint{0.000000in}{0.000000in}}%
\pgfpathlineto{\pgfqpoint{0.000000in}{-0.048611in}}%
\pgfusepath{stroke,fill}%
}%
\begin{pgfscope}%
\pgfsys@transformshift{1.624659in}{0.499444in}%
\pgfsys@useobject{currentmarker}{}%
\end{pgfscope}%
\end{pgfscope}%
\begin{pgfscope}%
\definecolor{textcolor}{rgb}{0.000000,0.000000,0.000000}%
\pgfsetstrokecolor{textcolor}%
\pgfsetfillcolor{textcolor}%
\pgftext[x=1.624659in,y=0.402222in,,top]{\color{textcolor}\rmfamily\fontsize{10.000000}{12.000000}\selectfont 0.3}%
\end{pgfscope}%
\begin{pgfscope}%
\pgfsetbuttcap%
\pgfsetroundjoin%
\definecolor{currentfill}{rgb}{0.000000,0.000000,0.000000}%
\pgfsetfillcolor{currentfill}%
\pgfsetlinewidth{0.803000pt}%
\definecolor{currentstroke}{rgb}{0.000000,0.000000,0.000000}%
\pgfsetstrokecolor{currentstroke}%
\pgfsetdash{}{0pt}%
\pgfsys@defobject{currentmarker}{\pgfqpoint{0.000000in}{-0.048611in}}{\pgfqpoint{0.000000in}{0.000000in}}{%
\pgfpathmoveto{\pgfqpoint{0.000000in}{0.000000in}}%
\pgfpathlineto{\pgfqpoint{0.000000in}{-0.048611in}}%
\pgfusepath{stroke,fill}%
}%
\begin{pgfscope}%
\pgfsys@transformshift{1.783182in}{0.499444in}%
\pgfsys@useobject{currentmarker}{}%
\end{pgfscope}%
\end{pgfscope}%
\begin{pgfscope}%
\pgfsetbuttcap%
\pgfsetroundjoin%
\definecolor{currentfill}{rgb}{0.000000,0.000000,0.000000}%
\pgfsetfillcolor{currentfill}%
\pgfsetlinewidth{0.803000pt}%
\definecolor{currentstroke}{rgb}{0.000000,0.000000,0.000000}%
\pgfsetstrokecolor{currentstroke}%
\pgfsetdash{}{0pt}%
\pgfsys@defobject{currentmarker}{\pgfqpoint{0.000000in}{-0.048611in}}{\pgfqpoint{0.000000in}{0.000000in}}{%
\pgfpathmoveto{\pgfqpoint{0.000000in}{0.000000in}}%
\pgfpathlineto{\pgfqpoint{0.000000in}{-0.048611in}}%
\pgfusepath{stroke,fill}%
}%
\begin{pgfscope}%
\pgfsys@transformshift{1.941705in}{0.499444in}%
\pgfsys@useobject{currentmarker}{}%
\end{pgfscope}%
\end{pgfscope}%
\begin{pgfscope}%
\definecolor{textcolor}{rgb}{0.000000,0.000000,0.000000}%
\pgfsetstrokecolor{textcolor}%
\pgfsetfillcolor{textcolor}%
\pgftext[x=1.941705in,y=0.402222in,,top]{\color{textcolor}\rmfamily\fontsize{10.000000}{12.000000}\selectfont 0.4}%
\end{pgfscope}%
\begin{pgfscope}%
\pgfsetbuttcap%
\pgfsetroundjoin%
\definecolor{currentfill}{rgb}{0.000000,0.000000,0.000000}%
\pgfsetfillcolor{currentfill}%
\pgfsetlinewidth{0.803000pt}%
\definecolor{currentstroke}{rgb}{0.000000,0.000000,0.000000}%
\pgfsetstrokecolor{currentstroke}%
\pgfsetdash{}{0pt}%
\pgfsys@defobject{currentmarker}{\pgfqpoint{0.000000in}{-0.048611in}}{\pgfqpoint{0.000000in}{0.000000in}}{%
\pgfpathmoveto{\pgfqpoint{0.000000in}{0.000000in}}%
\pgfpathlineto{\pgfqpoint{0.000000in}{-0.048611in}}%
\pgfusepath{stroke,fill}%
}%
\begin{pgfscope}%
\pgfsys@transformshift{2.100228in}{0.499444in}%
\pgfsys@useobject{currentmarker}{}%
\end{pgfscope}%
\end{pgfscope}%
\begin{pgfscope}%
\pgfsetbuttcap%
\pgfsetroundjoin%
\definecolor{currentfill}{rgb}{0.000000,0.000000,0.000000}%
\pgfsetfillcolor{currentfill}%
\pgfsetlinewidth{0.803000pt}%
\definecolor{currentstroke}{rgb}{0.000000,0.000000,0.000000}%
\pgfsetstrokecolor{currentstroke}%
\pgfsetdash{}{0pt}%
\pgfsys@defobject{currentmarker}{\pgfqpoint{0.000000in}{-0.048611in}}{\pgfqpoint{0.000000in}{0.000000in}}{%
\pgfpathmoveto{\pgfqpoint{0.000000in}{0.000000in}}%
\pgfpathlineto{\pgfqpoint{0.000000in}{-0.048611in}}%
\pgfusepath{stroke,fill}%
}%
\begin{pgfscope}%
\pgfsys@transformshift{2.258750in}{0.499444in}%
\pgfsys@useobject{currentmarker}{}%
\end{pgfscope}%
\end{pgfscope}%
\begin{pgfscope}%
\definecolor{textcolor}{rgb}{0.000000,0.000000,0.000000}%
\pgfsetstrokecolor{textcolor}%
\pgfsetfillcolor{textcolor}%
\pgftext[x=2.258750in,y=0.402222in,,top]{\color{textcolor}\rmfamily\fontsize{10.000000}{12.000000}\selectfont 0.5}%
\end{pgfscope}%
\begin{pgfscope}%
\pgfsetbuttcap%
\pgfsetroundjoin%
\definecolor{currentfill}{rgb}{0.000000,0.000000,0.000000}%
\pgfsetfillcolor{currentfill}%
\pgfsetlinewidth{0.803000pt}%
\definecolor{currentstroke}{rgb}{0.000000,0.000000,0.000000}%
\pgfsetstrokecolor{currentstroke}%
\pgfsetdash{}{0pt}%
\pgfsys@defobject{currentmarker}{\pgfqpoint{0.000000in}{-0.048611in}}{\pgfqpoint{0.000000in}{0.000000in}}{%
\pgfpathmoveto{\pgfqpoint{0.000000in}{0.000000in}}%
\pgfpathlineto{\pgfqpoint{0.000000in}{-0.048611in}}%
\pgfusepath{stroke,fill}%
}%
\begin{pgfscope}%
\pgfsys@transformshift{2.417273in}{0.499444in}%
\pgfsys@useobject{currentmarker}{}%
\end{pgfscope}%
\end{pgfscope}%
\begin{pgfscope}%
\pgfsetbuttcap%
\pgfsetroundjoin%
\definecolor{currentfill}{rgb}{0.000000,0.000000,0.000000}%
\pgfsetfillcolor{currentfill}%
\pgfsetlinewidth{0.803000pt}%
\definecolor{currentstroke}{rgb}{0.000000,0.000000,0.000000}%
\pgfsetstrokecolor{currentstroke}%
\pgfsetdash{}{0pt}%
\pgfsys@defobject{currentmarker}{\pgfqpoint{0.000000in}{-0.048611in}}{\pgfqpoint{0.000000in}{0.000000in}}{%
\pgfpathmoveto{\pgfqpoint{0.000000in}{0.000000in}}%
\pgfpathlineto{\pgfqpoint{0.000000in}{-0.048611in}}%
\pgfusepath{stroke,fill}%
}%
\begin{pgfscope}%
\pgfsys@transformshift{2.575796in}{0.499444in}%
\pgfsys@useobject{currentmarker}{}%
\end{pgfscope}%
\end{pgfscope}%
\begin{pgfscope}%
\definecolor{textcolor}{rgb}{0.000000,0.000000,0.000000}%
\pgfsetstrokecolor{textcolor}%
\pgfsetfillcolor{textcolor}%
\pgftext[x=2.575796in,y=0.402222in,,top]{\color{textcolor}\rmfamily\fontsize{10.000000}{12.000000}\selectfont 0.6}%
\end{pgfscope}%
\begin{pgfscope}%
\pgfsetbuttcap%
\pgfsetroundjoin%
\definecolor{currentfill}{rgb}{0.000000,0.000000,0.000000}%
\pgfsetfillcolor{currentfill}%
\pgfsetlinewidth{0.803000pt}%
\definecolor{currentstroke}{rgb}{0.000000,0.000000,0.000000}%
\pgfsetstrokecolor{currentstroke}%
\pgfsetdash{}{0pt}%
\pgfsys@defobject{currentmarker}{\pgfqpoint{0.000000in}{-0.048611in}}{\pgfqpoint{0.000000in}{0.000000in}}{%
\pgfpathmoveto{\pgfqpoint{0.000000in}{0.000000in}}%
\pgfpathlineto{\pgfqpoint{0.000000in}{-0.048611in}}%
\pgfusepath{stroke,fill}%
}%
\begin{pgfscope}%
\pgfsys@transformshift{2.734318in}{0.499444in}%
\pgfsys@useobject{currentmarker}{}%
\end{pgfscope}%
\end{pgfscope}%
\begin{pgfscope}%
\pgfsetbuttcap%
\pgfsetroundjoin%
\definecolor{currentfill}{rgb}{0.000000,0.000000,0.000000}%
\pgfsetfillcolor{currentfill}%
\pgfsetlinewidth{0.803000pt}%
\definecolor{currentstroke}{rgb}{0.000000,0.000000,0.000000}%
\pgfsetstrokecolor{currentstroke}%
\pgfsetdash{}{0pt}%
\pgfsys@defobject{currentmarker}{\pgfqpoint{0.000000in}{-0.048611in}}{\pgfqpoint{0.000000in}{0.000000in}}{%
\pgfpathmoveto{\pgfqpoint{0.000000in}{0.000000in}}%
\pgfpathlineto{\pgfqpoint{0.000000in}{-0.048611in}}%
\pgfusepath{stroke,fill}%
}%
\begin{pgfscope}%
\pgfsys@transformshift{2.892841in}{0.499444in}%
\pgfsys@useobject{currentmarker}{}%
\end{pgfscope}%
\end{pgfscope}%
\begin{pgfscope}%
\definecolor{textcolor}{rgb}{0.000000,0.000000,0.000000}%
\pgfsetstrokecolor{textcolor}%
\pgfsetfillcolor{textcolor}%
\pgftext[x=2.892841in,y=0.402222in,,top]{\color{textcolor}\rmfamily\fontsize{10.000000}{12.000000}\selectfont 0.7}%
\end{pgfscope}%
\begin{pgfscope}%
\pgfsetbuttcap%
\pgfsetroundjoin%
\definecolor{currentfill}{rgb}{0.000000,0.000000,0.000000}%
\pgfsetfillcolor{currentfill}%
\pgfsetlinewidth{0.803000pt}%
\definecolor{currentstroke}{rgb}{0.000000,0.000000,0.000000}%
\pgfsetstrokecolor{currentstroke}%
\pgfsetdash{}{0pt}%
\pgfsys@defobject{currentmarker}{\pgfqpoint{0.000000in}{-0.048611in}}{\pgfqpoint{0.000000in}{0.000000in}}{%
\pgfpathmoveto{\pgfqpoint{0.000000in}{0.000000in}}%
\pgfpathlineto{\pgfqpoint{0.000000in}{-0.048611in}}%
\pgfusepath{stroke,fill}%
}%
\begin{pgfscope}%
\pgfsys@transformshift{3.051364in}{0.499444in}%
\pgfsys@useobject{currentmarker}{}%
\end{pgfscope}%
\end{pgfscope}%
\begin{pgfscope}%
\pgfsetbuttcap%
\pgfsetroundjoin%
\definecolor{currentfill}{rgb}{0.000000,0.000000,0.000000}%
\pgfsetfillcolor{currentfill}%
\pgfsetlinewidth{0.803000pt}%
\definecolor{currentstroke}{rgb}{0.000000,0.000000,0.000000}%
\pgfsetstrokecolor{currentstroke}%
\pgfsetdash{}{0pt}%
\pgfsys@defobject{currentmarker}{\pgfqpoint{0.000000in}{-0.048611in}}{\pgfqpoint{0.000000in}{0.000000in}}{%
\pgfpathmoveto{\pgfqpoint{0.000000in}{0.000000in}}%
\pgfpathlineto{\pgfqpoint{0.000000in}{-0.048611in}}%
\pgfusepath{stroke,fill}%
}%
\begin{pgfscope}%
\pgfsys@transformshift{3.209887in}{0.499444in}%
\pgfsys@useobject{currentmarker}{}%
\end{pgfscope}%
\end{pgfscope}%
\begin{pgfscope}%
\definecolor{textcolor}{rgb}{0.000000,0.000000,0.000000}%
\pgfsetstrokecolor{textcolor}%
\pgfsetfillcolor{textcolor}%
\pgftext[x=3.209887in,y=0.402222in,,top]{\color{textcolor}\rmfamily\fontsize{10.000000}{12.000000}\selectfont 0.8}%
\end{pgfscope}%
\begin{pgfscope}%
\pgfsetbuttcap%
\pgfsetroundjoin%
\definecolor{currentfill}{rgb}{0.000000,0.000000,0.000000}%
\pgfsetfillcolor{currentfill}%
\pgfsetlinewidth{0.803000pt}%
\definecolor{currentstroke}{rgb}{0.000000,0.000000,0.000000}%
\pgfsetstrokecolor{currentstroke}%
\pgfsetdash{}{0pt}%
\pgfsys@defobject{currentmarker}{\pgfqpoint{0.000000in}{-0.048611in}}{\pgfqpoint{0.000000in}{0.000000in}}{%
\pgfpathmoveto{\pgfqpoint{0.000000in}{0.000000in}}%
\pgfpathlineto{\pgfqpoint{0.000000in}{-0.048611in}}%
\pgfusepath{stroke,fill}%
}%
\begin{pgfscope}%
\pgfsys@transformshift{3.368409in}{0.499444in}%
\pgfsys@useobject{currentmarker}{}%
\end{pgfscope}%
\end{pgfscope}%
\begin{pgfscope}%
\pgfsetbuttcap%
\pgfsetroundjoin%
\definecolor{currentfill}{rgb}{0.000000,0.000000,0.000000}%
\pgfsetfillcolor{currentfill}%
\pgfsetlinewidth{0.803000pt}%
\definecolor{currentstroke}{rgb}{0.000000,0.000000,0.000000}%
\pgfsetstrokecolor{currentstroke}%
\pgfsetdash{}{0pt}%
\pgfsys@defobject{currentmarker}{\pgfqpoint{0.000000in}{-0.048611in}}{\pgfqpoint{0.000000in}{0.000000in}}{%
\pgfpathmoveto{\pgfqpoint{0.000000in}{0.000000in}}%
\pgfpathlineto{\pgfqpoint{0.000000in}{-0.048611in}}%
\pgfusepath{stroke,fill}%
}%
\begin{pgfscope}%
\pgfsys@transformshift{3.526932in}{0.499444in}%
\pgfsys@useobject{currentmarker}{}%
\end{pgfscope}%
\end{pgfscope}%
\begin{pgfscope}%
\definecolor{textcolor}{rgb}{0.000000,0.000000,0.000000}%
\pgfsetstrokecolor{textcolor}%
\pgfsetfillcolor{textcolor}%
\pgftext[x=3.526932in,y=0.402222in,,top]{\color{textcolor}\rmfamily\fontsize{10.000000}{12.000000}\selectfont 0.9}%
\end{pgfscope}%
\begin{pgfscope}%
\pgfsetbuttcap%
\pgfsetroundjoin%
\definecolor{currentfill}{rgb}{0.000000,0.000000,0.000000}%
\pgfsetfillcolor{currentfill}%
\pgfsetlinewidth{0.803000pt}%
\definecolor{currentstroke}{rgb}{0.000000,0.000000,0.000000}%
\pgfsetstrokecolor{currentstroke}%
\pgfsetdash{}{0pt}%
\pgfsys@defobject{currentmarker}{\pgfqpoint{0.000000in}{-0.048611in}}{\pgfqpoint{0.000000in}{0.000000in}}{%
\pgfpathmoveto{\pgfqpoint{0.000000in}{0.000000in}}%
\pgfpathlineto{\pgfqpoint{0.000000in}{-0.048611in}}%
\pgfusepath{stroke,fill}%
}%
\begin{pgfscope}%
\pgfsys@transformshift{3.685455in}{0.499444in}%
\pgfsys@useobject{currentmarker}{}%
\end{pgfscope}%
\end{pgfscope}%
\begin{pgfscope}%
\pgfsetbuttcap%
\pgfsetroundjoin%
\definecolor{currentfill}{rgb}{0.000000,0.000000,0.000000}%
\pgfsetfillcolor{currentfill}%
\pgfsetlinewidth{0.803000pt}%
\definecolor{currentstroke}{rgb}{0.000000,0.000000,0.000000}%
\pgfsetstrokecolor{currentstroke}%
\pgfsetdash{}{0pt}%
\pgfsys@defobject{currentmarker}{\pgfqpoint{0.000000in}{-0.048611in}}{\pgfqpoint{0.000000in}{0.000000in}}{%
\pgfpathmoveto{\pgfqpoint{0.000000in}{0.000000in}}%
\pgfpathlineto{\pgfqpoint{0.000000in}{-0.048611in}}%
\pgfusepath{stroke,fill}%
}%
\begin{pgfscope}%
\pgfsys@transformshift{3.843978in}{0.499444in}%
\pgfsys@useobject{currentmarker}{}%
\end{pgfscope}%
\end{pgfscope}%
\begin{pgfscope}%
\definecolor{textcolor}{rgb}{0.000000,0.000000,0.000000}%
\pgfsetstrokecolor{textcolor}%
\pgfsetfillcolor{textcolor}%
\pgftext[x=3.843978in,y=0.402222in,,top]{\color{textcolor}\rmfamily\fontsize{10.000000}{12.000000}\selectfont 1.0}%
\end{pgfscope}%
\begin{pgfscope}%
\pgfsetbuttcap%
\pgfsetroundjoin%
\definecolor{currentfill}{rgb}{0.000000,0.000000,0.000000}%
\pgfsetfillcolor{currentfill}%
\pgfsetlinewidth{0.803000pt}%
\definecolor{currentstroke}{rgb}{0.000000,0.000000,0.000000}%
\pgfsetstrokecolor{currentstroke}%
\pgfsetdash{}{0pt}%
\pgfsys@defobject{currentmarker}{\pgfqpoint{0.000000in}{-0.048611in}}{\pgfqpoint{0.000000in}{0.000000in}}{%
\pgfpathmoveto{\pgfqpoint{0.000000in}{0.000000in}}%
\pgfpathlineto{\pgfqpoint{0.000000in}{-0.048611in}}%
\pgfusepath{stroke,fill}%
}%
\begin{pgfscope}%
\pgfsys@transformshift{4.002500in}{0.499444in}%
\pgfsys@useobject{currentmarker}{}%
\end{pgfscope}%
\end{pgfscope}%
\begin{pgfscope}%
\definecolor{textcolor}{rgb}{0.000000,0.000000,0.000000}%
\pgfsetstrokecolor{textcolor}%
\pgfsetfillcolor{textcolor}%
\pgftext[x=2.258750in,y=0.223333in,,top]{\color{textcolor}\rmfamily\fontsize{10.000000}{12.000000}\selectfont \(\displaystyle p\)}%
\end{pgfscope}%
\begin{pgfscope}%
\pgfsetbuttcap%
\pgfsetroundjoin%
\definecolor{currentfill}{rgb}{0.000000,0.000000,0.000000}%
\pgfsetfillcolor{currentfill}%
\pgfsetlinewidth{0.803000pt}%
\definecolor{currentstroke}{rgb}{0.000000,0.000000,0.000000}%
\pgfsetstrokecolor{currentstroke}%
\pgfsetdash{}{0pt}%
\pgfsys@defobject{currentmarker}{\pgfqpoint{-0.048611in}{0.000000in}}{\pgfqpoint{-0.000000in}{0.000000in}}{%
\pgfpathmoveto{\pgfqpoint{-0.000000in}{0.000000in}}%
\pgfpathlineto{\pgfqpoint{-0.048611in}{0.000000in}}%
\pgfusepath{stroke,fill}%
}%
\begin{pgfscope}%
\pgfsys@transformshift{0.515000in}{0.499444in}%
\pgfsys@useobject{currentmarker}{}%
\end{pgfscope}%
\end{pgfscope}%
\begin{pgfscope}%
\definecolor{textcolor}{rgb}{0.000000,0.000000,0.000000}%
\pgfsetstrokecolor{textcolor}%
\pgfsetfillcolor{textcolor}%
\pgftext[x=0.348333in, y=0.451250in, left, base]{\color{textcolor}\rmfamily\fontsize{10.000000}{12.000000}\selectfont \(\displaystyle {0}\)}%
\end{pgfscope}%
\begin{pgfscope}%
\pgfsetbuttcap%
\pgfsetroundjoin%
\definecolor{currentfill}{rgb}{0.000000,0.000000,0.000000}%
\pgfsetfillcolor{currentfill}%
\pgfsetlinewidth{0.803000pt}%
\definecolor{currentstroke}{rgb}{0.000000,0.000000,0.000000}%
\pgfsetstrokecolor{currentstroke}%
\pgfsetdash{}{0pt}%
\pgfsys@defobject{currentmarker}{\pgfqpoint{-0.048611in}{0.000000in}}{\pgfqpoint{-0.000000in}{0.000000in}}{%
\pgfpathmoveto{\pgfqpoint{-0.000000in}{0.000000in}}%
\pgfpathlineto{\pgfqpoint{-0.048611in}{0.000000in}}%
\pgfusepath{stroke,fill}%
}%
\begin{pgfscope}%
\pgfsys@transformshift{0.515000in}{0.889534in}%
\pgfsys@useobject{currentmarker}{}%
\end{pgfscope}%
\end{pgfscope}%
\begin{pgfscope}%
\definecolor{textcolor}{rgb}{0.000000,0.000000,0.000000}%
\pgfsetstrokecolor{textcolor}%
\pgfsetfillcolor{textcolor}%
\pgftext[x=0.348333in, y=0.841339in, left, base]{\color{textcolor}\rmfamily\fontsize{10.000000}{12.000000}\selectfont \(\displaystyle {5}\)}%
\end{pgfscope}%
\begin{pgfscope}%
\pgfsetbuttcap%
\pgfsetroundjoin%
\definecolor{currentfill}{rgb}{0.000000,0.000000,0.000000}%
\pgfsetfillcolor{currentfill}%
\pgfsetlinewidth{0.803000pt}%
\definecolor{currentstroke}{rgb}{0.000000,0.000000,0.000000}%
\pgfsetstrokecolor{currentstroke}%
\pgfsetdash{}{0pt}%
\pgfsys@defobject{currentmarker}{\pgfqpoint{-0.048611in}{0.000000in}}{\pgfqpoint{-0.000000in}{0.000000in}}{%
\pgfpathmoveto{\pgfqpoint{-0.000000in}{0.000000in}}%
\pgfpathlineto{\pgfqpoint{-0.048611in}{0.000000in}}%
\pgfusepath{stroke,fill}%
}%
\begin{pgfscope}%
\pgfsys@transformshift{0.515000in}{1.279623in}%
\pgfsys@useobject{currentmarker}{}%
\end{pgfscope}%
\end{pgfscope}%
\begin{pgfscope}%
\definecolor{textcolor}{rgb}{0.000000,0.000000,0.000000}%
\pgfsetstrokecolor{textcolor}%
\pgfsetfillcolor{textcolor}%
\pgftext[x=0.278889in, y=1.231429in, left, base]{\color{textcolor}\rmfamily\fontsize{10.000000}{12.000000}\selectfont \(\displaystyle {10}\)}%
\end{pgfscope}%
\begin{pgfscope}%
\definecolor{textcolor}{rgb}{0.000000,0.000000,0.000000}%
\pgfsetstrokecolor{textcolor}%
\pgfsetfillcolor{textcolor}%
\pgftext[x=0.223333in,y=1.076944in,,bottom,rotate=90.000000]{\color{textcolor}\rmfamily\fontsize{10.000000}{12.000000}\selectfont Percent of Data Set}%
\end{pgfscope}%
\begin{pgfscope}%
\pgfsetrectcap%
\pgfsetmiterjoin%
\pgfsetlinewidth{0.803000pt}%
\definecolor{currentstroke}{rgb}{0.000000,0.000000,0.000000}%
\pgfsetstrokecolor{currentstroke}%
\pgfsetdash{}{0pt}%
\pgfpathmoveto{\pgfqpoint{0.515000in}{0.499444in}}%
\pgfpathlineto{\pgfqpoint{0.515000in}{1.654444in}}%
\pgfusepath{stroke}%
\end{pgfscope}%
\begin{pgfscope}%
\pgfsetrectcap%
\pgfsetmiterjoin%
\pgfsetlinewidth{0.803000pt}%
\definecolor{currentstroke}{rgb}{0.000000,0.000000,0.000000}%
\pgfsetstrokecolor{currentstroke}%
\pgfsetdash{}{0pt}%
\pgfpathmoveto{\pgfqpoint{4.002500in}{0.499444in}}%
\pgfpathlineto{\pgfqpoint{4.002500in}{1.654444in}}%
\pgfusepath{stroke}%
\end{pgfscope}%
\begin{pgfscope}%
\pgfsetrectcap%
\pgfsetmiterjoin%
\pgfsetlinewidth{0.803000pt}%
\definecolor{currentstroke}{rgb}{0.000000,0.000000,0.000000}%
\pgfsetstrokecolor{currentstroke}%
\pgfsetdash{}{0pt}%
\pgfpathmoveto{\pgfqpoint{0.515000in}{0.499444in}}%
\pgfpathlineto{\pgfqpoint{4.002500in}{0.499444in}}%
\pgfusepath{stroke}%
\end{pgfscope}%
\begin{pgfscope}%
\pgfsetrectcap%
\pgfsetmiterjoin%
\pgfsetlinewidth{0.803000pt}%
\definecolor{currentstroke}{rgb}{0.000000,0.000000,0.000000}%
\pgfsetstrokecolor{currentstroke}%
\pgfsetdash{}{0pt}%
\pgfpathmoveto{\pgfqpoint{0.515000in}{1.654444in}}%
\pgfpathlineto{\pgfqpoint{4.002500in}{1.654444in}}%
\pgfusepath{stroke}%
\end{pgfscope}%
\begin{pgfscope}%
\pgfsetbuttcap%
\pgfsetmiterjoin%
\definecolor{currentfill}{rgb}{1.000000,1.000000,1.000000}%
\pgfsetfillcolor{currentfill}%
\pgfsetfillopacity{0.800000}%
\pgfsetlinewidth{1.003750pt}%
\definecolor{currentstroke}{rgb}{0.800000,0.800000,0.800000}%
\pgfsetstrokecolor{currentstroke}%
\pgfsetstrokeopacity{0.800000}%
\pgfsetdash{}{0pt}%
\pgfpathmoveto{\pgfqpoint{3.225556in}{1.154445in}}%
\pgfpathlineto{\pgfqpoint{3.905278in}{1.154445in}}%
\pgfpathquadraticcurveto{\pgfqpoint{3.933056in}{1.154445in}}{\pgfqpoint{3.933056in}{1.182222in}}%
\pgfpathlineto{\pgfqpoint{3.933056in}{1.557222in}}%
\pgfpathquadraticcurveto{\pgfqpoint{3.933056in}{1.585000in}}{\pgfqpoint{3.905278in}{1.585000in}}%
\pgfpathlineto{\pgfqpoint{3.225556in}{1.585000in}}%
\pgfpathquadraticcurveto{\pgfqpoint{3.197778in}{1.585000in}}{\pgfqpoint{3.197778in}{1.557222in}}%
\pgfpathlineto{\pgfqpoint{3.197778in}{1.182222in}}%
\pgfpathquadraticcurveto{\pgfqpoint{3.197778in}{1.154445in}}{\pgfqpoint{3.225556in}{1.154445in}}%
\pgfpathlineto{\pgfqpoint{3.225556in}{1.154445in}}%
\pgfpathclose%
\pgfusepath{stroke,fill}%
\end{pgfscope}%
\begin{pgfscope}%
\pgfsetbuttcap%
\pgfsetmiterjoin%
\pgfsetlinewidth{1.003750pt}%
\definecolor{currentstroke}{rgb}{0.000000,0.000000,0.000000}%
\pgfsetstrokecolor{currentstroke}%
\pgfsetdash{}{0pt}%
\pgfpathmoveto{\pgfqpoint{3.253334in}{1.432222in}}%
\pgfpathlineto{\pgfqpoint{3.531111in}{1.432222in}}%
\pgfpathlineto{\pgfqpoint{3.531111in}{1.529444in}}%
\pgfpathlineto{\pgfqpoint{3.253334in}{1.529444in}}%
\pgfpathlineto{\pgfqpoint{3.253334in}{1.432222in}}%
\pgfpathclose%
\pgfusepath{stroke}%
\end{pgfscope}%
\begin{pgfscope}%
\definecolor{textcolor}{rgb}{0.000000,0.000000,0.000000}%
\pgfsetstrokecolor{textcolor}%
\pgfsetfillcolor{textcolor}%
\pgftext[x=3.642223in,y=1.432222in,left,base]{\color{textcolor}\rmfamily\fontsize{10.000000}{12.000000}\selectfont Neg}%
\end{pgfscope}%
\begin{pgfscope}%
\pgfsetbuttcap%
\pgfsetmiterjoin%
\definecolor{currentfill}{rgb}{0.000000,0.000000,0.000000}%
\pgfsetfillcolor{currentfill}%
\pgfsetlinewidth{0.000000pt}%
\definecolor{currentstroke}{rgb}{0.000000,0.000000,0.000000}%
\pgfsetstrokecolor{currentstroke}%
\pgfsetstrokeopacity{0.000000}%
\pgfsetdash{}{0pt}%
\pgfpathmoveto{\pgfqpoint{3.253334in}{1.236944in}}%
\pgfpathlineto{\pgfqpoint{3.531111in}{1.236944in}}%
\pgfpathlineto{\pgfqpoint{3.531111in}{1.334167in}}%
\pgfpathlineto{\pgfqpoint{3.253334in}{1.334167in}}%
\pgfpathlineto{\pgfqpoint{3.253334in}{1.236944in}}%
\pgfpathclose%
\pgfusepath{fill}%
\end{pgfscope}%
\begin{pgfscope}%
\definecolor{textcolor}{rgb}{0.000000,0.000000,0.000000}%
\pgfsetstrokecolor{textcolor}%
\pgfsetfillcolor{textcolor}%
\pgftext[x=3.642223in,y=1.236944in,left,base]{\color{textcolor}\rmfamily\fontsize{10.000000}{12.000000}\selectfont Pos}%
\end{pgfscope}%
\end{pgfpicture}%
\makeatother%
\endgroup%
	
&
	\vskip 0pt
	\hfil ROC Curve
	
	%% Creator: Matplotlib, PGF backend
%%
%% To include the figure in your LaTeX document, write
%%   \input{<filename>.pgf}
%%
%% Make sure the required packages are loaded in your preamble
%%   \usepackage{pgf}
%%
%% Also ensure that all the required font packages are loaded; for instance,
%% the lmodern package is sometimes necessary when using math font.
%%   \usepackage{lmodern}
%%
%% Figures using additional raster images can only be included by \input if
%% they are in the same directory as the main LaTeX file. For loading figures
%% from other directories you can use the `import` package
%%   \usepackage{import}
%%
%% and then include the figures with
%%   \import{<path to file>}{<filename>.pgf}
%%
%% Matplotlib used the following preamble
%%   
%%   \usepackage{fontspec}
%%   \makeatletter\@ifpackageloaded{underscore}{}{\usepackage[strings]{underscore}}\makeatother
%%
\begingroup%
\makeatletter%
\begin{pgfpicture}%
\pgfpathrectangle{\pgfpointorigin}{\pgfqpoint{2.221861in}{1.754444in}}%
\pgfusepath{use as bounding box, clip}%
\begin{pgfscope}%
\pgfsetbuttcap%
\pgfsetmiterjoin%
\definecolor{currentfill}{rgb}{1.000000,1.000000,1.000000}%
\pgfsetfillcolor{currentfill}%
\pgfsetlinewidth{0.000000pt}%
\definecolor{currentstroke}{rgb}{1.000000,1.000000,1.000000}%
\pgfsetstrokecolor{currentstroke}%
\pgfsetdash{}{0pt}%
\pgfpathmoveto{\pgfqpoint{0.000000in}{0.000000in}}%
\pgfpathlineto{\pgfqpoint{2.221861in}{0.000000in}}%
\pgfpathlineto{\pgfqpoint{2.221861in}{1.754444in}}%
\pgfpathlineto{\pgfqpoint{0.000000in}{1.754444in}}%
\pgfpathlineto{\pgfqpoint{0.000000in}{0.000000in}}%
\pgfpathclose%
\pgfusepath{fill}%
\end{pgfscope}%
\begin{pgfscope}%
\pgfsetbuttcap%
\pgfsetmiterjoin%
\definecolor{currentfill}{rgb}{1.000000,1.000000,1.000000}%
\pgfsetfillcolor{currentfill}%
\pgfsetlinewidth{0.000000pt}%
\definecolor{currentstroke}{rgb}{0.000000,0.000000,0.000000}%
\pgfsetstrokecolor{currentstroke}%
\pgfsetstrokeopacity{0.000000}%
\pgfsetdash{}{0pt}%
\pgfpathmoveto{\pgfqpoint{0.553581in}{0.499444in}}%
\pgfpathlineto{\pgfqpoint{2.103581in}{0.499444in}}%
\pgfpathlineto{\pgfqpoint{2.103581in}{1.654444in}}%
\pgfpathlineto{\pgfqpoint{0.553581in}{1.654444in}}%
\pgfpathlineto{\pgfqpoint{0.553581in}{0.499444in}}%
\pgfpathclose%
\pgfusepath{fill}%
\end{pgfscope}%
\begin{pgfscope}%
\pgfsetbuttcap%
\pgfsetroundjoin%
\definecolor{currentfill}{rgb}{0.000000,0.000000,0.000000}%
\pgfsetfillcolor{currentfill}%
\pgfsetlinewidth{0.803000pt}%
\definecolor{currentstroke}{rgb}{0.000000,0.000000,0.000000}%
\pgfsetstrokecolor{currentstroke}%
\pgfsetdash{}{0pt}%
\pgfsys@defobject{currentmarker}{\pgfqpoint{0.000000in}{-0.048611in}}{\pgfqpoint{0.000000in}{0.000000in}}{%
\pgfpathmoveto{\pgfqpoint{0.000000in}{0.000000in}}%
\pgfpathlineto{\pgfqpoint{0.000000in}{-0.048611in}}%
\pgfusepath{stroke,fill}%
}%
\begin{pgfscope}%
\pgfsys@transformshift{0.624035in}{0.499444in}%
\pgfsys@useobject{currentmarker}{}%
\end{pgfscope}%
\end{pgfscope}%
\begin{pgfscope}%
\definecolor{textcolor}{rgb}{0.000000,0.000000,0.000000}%
\pgfsetstrokecolor{textcolor}%
\pgfsetfillcolor{textcolor}%
\pgftext[x=0.624035in,y=0.402222in,,top]{\color{textcolor}\rmfamily\fontsize{10.000000}{12.000000}\selectfont \(\displaystyle {0.0}\)}%
\end{pgfscope}%
\begin{pgfscope}%
\pgfsetbuttcap%
\pgfsetroundjoin%
\definecolor{currentfill}{rgb}{0.000000,0.000000,0.000000}%
\pgfsetfillcolor{currentfill}%
\pgfsetlinewidth{0.803000pt}%
\definecolor{currentstroke}{rgb}{0.000000,0.000000,0.000000}%
\pgfsetstrokecolor{currentstroke}%
\pgfsetdash{}{0pt}%
\pgfsys@defobject{currentmarker}{\pgfqpoint{0.000000in}{-0.048611in}}{\pgfqpoint{0.000000in}{0.000000in}}{%
\pgfpathmoveto{\pgfqpoint{0.000000in}{0.000000in}}%
\pgfpathlineto{\pgfqpoint{0.000000in}{-0.048611in}}%
\pgfusepath{stroke,fill}%
}%
\begin{pgfscope}%
\pgfsys@transformshift{1.328581in}{0.499444in}%
\pgfsys@useobject{currentmarker}{}%
\end{pgfscope}%
\end{pgfscope}%
\begin{pgfscope}%
\definecolor{textcolor}{rgb}{0.000000,0.000000,0.000000}%
\pgfsetstrokecolor{textcolor}%
\pgfsetfillcolor{textcolor}%
\pgftext[x=1.328581in,y=0.402222in,,top]{\color{textcolor}\rmfamily\fontsize{10.000000}{12.000000}\selectfont \(\displaystyle {0.5}\)}%
\end{pgfscope}%
\begin{pgfscope}%
\pgfsetbuttcap%
\pgfsetroundjoin%
\definecolor{currentfill}{rgb}{0.000000,0.000000,0.000000}%
\pgfsetfillcolor{currentfill}%
\pgfsetlinewidth{0.803000pt}%
\definecolor{currentstroke}{rgb}{0.000000,0.000000,0.000000}%
\pgfsetstrokecolor{currentstroke}%
\pgfsetdash{}{0pt}%
\pgfsys@defobject{currentmarker}{\pgfqpoint{0.000000in}{-0.048611in}}{\pgfqpoint{0.000000in}{0.000000in}}{%
\pgfpathmoveto{\pgfqpoint{0.000000in}{0.000000in}}%
\pgfpathlineto{\pgfqpoint{0.000000in}{-0.048611in}}%
\pgfusepath{stroke,fill}%
}%
\begin{pgfscope}%
\pgfsys@transformshift{2.033126in}{0.499444in}%
\pgfsys@useobject{currentmarker}{}%
\end{pgfscope}%
\end{pgfscope}%
\begin{pgfscope}%
\definecolor{textcolor}{rgb}{0.000000,0.000000,0.000000}%
\pgfsetstrokecolor{textcolor}%
\pgfsetfillcolor{textcolor}%
\pgftext[x=2.033126in,y=0.402222in,,top]{\color{textcolor}\rmfamily\fontsize{10.000000}{12.000000}\selectfont \(\displaystyle {1.0}\)}%
\end{pgfscope}%
\begin{pgfscope}%
\definecolor{textcolor}{rgb}{0.000000,0.000000,0.000000}%
\pgfsetstrokecolor{textcolor}%
\pgfsetfillcolor{textcolor}%
\pgftext[x=1.328581in,y=0.223333in,,top]{\color{textcolor}\rmfamily\fontsize{10.000000}{12.000000}\selectfont False positive rate}%
\end{pgfscope}%
\begin{pgfscope}%
\pgfsetbuttcap%
\pgfsetroundjoin%
\definecolor{currentfill}{rgb}{0.000000,0.000000,0.000000}%
\pgfsetfillcolor{currentfill}%
\pgfsetlinewidth{0.803000pt}%
\definecolor{currentstroke}{rgb}{0.000000,0.000000,0.000000}%
\pgfsetstrokecolor{currentstroke}%
\pgfsetdash{}{0pt}%
\pgfsys@defobject{currentmarker}{\pgfqpoint{-0.048611in}{0.000000in}}{\pgfqpoint{-0.000000in}{0.000000in}}{%
\pgfpathmoveto{\pgfqpoint{-0.000000in}{0.000000in}}%
\pgfpathlineto{\pgfqpoint{-0.048611in}{0.000000in}}%
\pgfusepath{stroke,fill}%
}%
\begin{pgfscope}%
\pgfsys@transformshift{0.553581in}{0.551944in}%
\pgfsys@useobject{currentmarker}{}%
\end{pgfscope}%
\end{pgfscope}%
\begin{pgfscope}%
\definecolor{textcolor}{rgb}{0.000000,0.000000,0.000000}%
\pgfsetstrokecolor{textcolor}%
\pgfsetfillcolor{textcolor}%
\pgftext[x=0.278889in, y=0.503750in, left, base]{\color{textcolor}\rmfamily\fontsize{10.000000}{12.000000}\selectfont \(\displaystyle {0.0}\)}%
\end{pgfscope}%
\begin{pgfscope}%
\pgfsetbuttcap%
\pgfsetroundjoin%
\definecolor{currentfill}{rgb}{0.000000,0.000000,0.000000}%
\pgfsetfillcolor{currentfill}%
\pgfsetlinewidth{0.803000pt}%
\definecolor{currentstroke}{rgb}{0.000000,0.000000,0.000000}%
\pgfsetstrokecolor{currentstroke}%
\pgfsetdash{}{0pt}%
\pgfsys@defobject{currentmarker}{\pgfqpoint{-0.048611in}{0.000000in}}{\pgfqpoint{-0.000000in}{0.000000in}}{%
\pgfpathmoveto{\pgfqpoint{-0.000000in}{0.000000in}}%
\pgfpathlineto{\pgfqpoint{-0.048611in}{0.000000in}}%
\pgfusepath{stroke,fill}%
}%
\begin{pgfscope}%
\pgfsys@transformshift{0.553581in}{1.076944in}%
\pgfsys@useobject{currentmarker}{}%
\end{pgfscope}%
\end{pgfscope}%
\begin{pgfscope}%
\definecolor{textcolor}{rgb}{0.000000,0.000000,0.000000}%
\pgfsetstrokecolor{textcolor}%
\pgfsetfillcolor{textcolor}%
\pgftext[x=0.278889in, y=1.028750in, left, base]{\color{textcolor}\rmfamily\fontsize{10.000000}{12.000000}\selectfont \(\displaystyle {0.5}\)}%
\end{pgfscope}%
\begin{pgfscope}%
\pgfsetbuttcap%
\pgfsetroundjoin%
\definecolor{currentfill}{rgb}{0.000000,0.000000,0.000000}%
\pgfsetfillcolor{currentfill}%
\pgfsetlinewidth{0.803000pt}%
\definecolor{currentstroke}{rgb}{0.000000,0.000000,0.000000}%
\pgfsetstrokecolor{currentstroke}%
\pgfsetdash{}{0pt}%
\pgfsys@defobject{currentmarker}{\pgfqpoint{-0.048611in}{0.000000in}}{\pgfqpoint{-0.000000in}{0.000000in}}{%
\pgfpathmoveto{\pgfqpoint{-0.000000in}{0.000000in}}%
\pgfpathlineto{\pgfqpoint{-0.048611in}{0.000000in}}%
\pgfusepath{stroke,fill}%
}%
\begin{pgfscope}%
\pgfsys@transformshift{0.553581in}{1.601944in}%
\pgfsys@useobject{currentmarker}{}%
\end{pgfscope}%
\end{pgfscope}%
\begin{pgfscope}%
\definecolor{textcolor}{rgb}{0.000000,0.000000,0.000000}%
\pgfsetstrokecolor{textcolor}%
\pgfsetfillcolor{textcolor}%
\pgftext[x=0.278889in, y=1.553750in, left, base]{\color{textcolor}\rmfamily\fontsize{10.000000}{12.000000}\selectfont \(\displaystyle {1.0}\)}%
\end{pgfscope}%
\begin{pgfscope}%
\definecolor{textcolor}{rgb}{0.000000,0.000000,0.000000}%
\pgfsetstrokecolor{textcolor}%
\pgfsetfillcolor{textcolor}%
\pgftext[x=0.223333in,y=1.076944in,,bottom,rotate=90.000000]{\color{textcolor}\rmfamily\fontsize{10.000000}{12.000000}\selectfont True positive rate}%
\end{pgfscope}%
\begin{pgfscope}%
\pgfpathrectangle{\pgfqpoint{0.553581in}{0.499444in}}{\pgfqpoint{1.550000in}{1.155000in}}%
\pgfusepath{clip}%
\pgfsetbuttcap%
\pgfsetroundjoin%
\pgfsetlinewidth{1.505625pt}%
\definecolor{currentstroke}{rgb}{0.000000,0.000000,0.000000}%
\pgfsetstrokecolor{currentstroke}%
\pgfsetdash{{5.550000pt}{2.400000pt}}{0.000000pt}%
\pgfpathmoveto{\pgfqpoint{0.624035in}{0.551944in}}%
\pgfpathlineto{\pgfqpoint{2.033126in}{1.601944in}}%
\pgfusepath{stroke}%
\end{pgfscope}%
\begin{pgfscope}%
\pgfpathrectangle{\pgfqpoint{0.553581in}{0.499444in}}{\pgfqpoint{1.550000in}{1.155000in}}%
\pgfusepath{clip}%
\pgfsetrectcap%
\pgfsetroundjoin%
\pgfsetlinewidth{1.505625pt}%
\definecolor{currentstroke}{rgb}{0.000000,0.000000,0.000000}%
\pgfsetstrokecolor{currentstroke}%
\pgfsetdash{}{0pt}%
\pgfpathmoveto{\pgfqpoint{0.624035in}{0.551944in}}%
\pgfpathlineto{\pgfqpoint{0.624793in}{0.559478in}}%
\pgfpathlineto{\pgfqpoint{0.625893in}{0.567803in}}%
\pgfpathlineto{\pgfqpoint{0.626087in}{0.568893in}}%
\pgfpathlineto{\pgfqpoint{0.627197in}{0.575952in}}%
\pgfpathlineto{\pgfqpoint{0.627422in}{0.577051in}}%
\pgfpathlineto{\pgfqpoint{0.628529in}{0.583635in}}%
\pgfpathlineto{\pgfqpoint{0.628745in}{0.584734in}}%
\pgfpathlineto{\pgfqpoint{0.629849in}{0.590340in}}%
\pgfpathlineto{\pgfqpoint{0.630110in}{0.591439in}}%
\pgfpathlineto{\pgfqpoint{0.631219in}{0.597324in}}%
\pgfpathlineto{\pgfqpoint{0.631407in}{0.598423in}}%
\pgfpathlineto{\pgfqpoint{0.632516in}{0.603331in}}%
\pgfpathlineto{\pgfqpoint{0.632746in}{0.604439in}}%
\pgfpathlineto{\pgfqpoint{0.633843in}{0.609365in}}%
\pgfpathlineto{\pgfqpoint{0.634143in}{0.610464in}}%
\pgfpathlineto{\pgfqpoint{0.635243in}{0.614878in}}%
\pgfpathlineto{\pgfqpoint{0.635497in}{0.615987in}}%
\pgfpathlineto{\pgfqpoint{0.636606in}{0.620438in}}%
\pgfpathlineto{\pgfqpoint{0.636902in}{0.621546in}}%
\pgfpathlineto{\pgfqpoint{0.638011in}{0.626463in}}%
\pgfpathlineto{\pgfqpoint{0.638259in}{0.627562in}}%
\pgfpathlineto{\pgfqpoint{0.639364in}{0.632312in}}%
\pgfpathlineto{\pgfqpoint{0.639624in}{0.633410in}}%
\pgfpathlineto{\pgfqpoint{0.640729in}{0.637974in}}%
\pgfpathlineto{\pgfqpoint{0.641083in}{0.639063in}}%
\pgfpathlineto{\pgfqpoint{0.642190in}{0.643505in}}%
\pgfpathlineto{\pgfqpoint{0.642439in}{0.644604in}}%
\pgfpathlineto{\pgfqpoint{0.643543in}{0.648832in}}%
\pgfpathlineto{\pgfqpoint{0.643886in}{0.649940in}}%
\pgfpathlineto{\pgfqpoint{0.644986in}{0.653619in}}%
\pgfpathlineto{\pgfqpoint{0.645267in}{0.654680in}}%
\pgfpathlineto{\pgfqpoint{0.646372in}{0.659132in}}%
\pgfpathlineto{\pgfqpoint{0.646667in}{0.660231in}}%
\pgfpathlineto{\pgfqpoint{0.647772in}{0.664170in}}%
\pgfpathlineto{\pgfqpoint{0.648140in}{0.665269in}}%
\pgfpathlineto{\pgfqpoint{0.649245in}{0.668919in}}%
\pgfpathlineto{\pgfqpoint{0.649571in}{0.670018in}}%
\pgfpathlineto{\pgfqpoint{0.650678in}{0.673929in}}%
\pgfpathlineto{\pgfqpoint{0.650943in}{0.675038in}}%
\pgfpathlineto{\pgfqpoint{0.652052in}{0.679172in}}%
\pgfpathlineto{\pgfqpoint{0.652390in}{0.680281in}}%
\pgfpathlineto{\pgfqpoint{0.653483in}{0.684182in}}%
\pgfpathlineto{\pgfqpoint{0.653832in}{0.685281in}}%
\pgfpathlineto{\pgfqpoint{0.654942in}{0.689332in}}%
\pgfpathlineto{\pgfqpoint{0.655246in}{0.690440in}}%
\pgfpathlineto{\pgfqpoint{0.656353in}{0.694166in}}%
\pgfpathlineto{\pgfqpoint{0.656694in}{0.695274in}}%
\pgfpathlineto{\pgfqpoint{0.657798in}{0.698310in}}%
\pgfpathlineto{\pgfqpoint{0.657803in}{0.698310in}}%
\pgfpathlineto{\pgfqpoint{0.658190in}{0.699418in}}%
\pgfpathlineto{\pgfqpoint{0.659299in}{0.703404in}}%
\pgfpathlineto{\pgfqpoint{0.659684in}{0.704512in}}%
\pgfpathlineto{\pgfqpoint{0.660793in}{0.707892in}}%
\pgfpathlineto{\pgfqpoint{0.661098in}{0.709000in}}%
\pgfpathlineto{\pgfqpoint{0.662205in}{0.712856in}}%
\pgfpathlineto{\pgfqpoint{0.662536in}{0.713955in}}%
\pgfpathlineto{\pgfqpoint{0.663645in}{0.717587in}}%
\pgfpathlineto{\pgfqpoint{0.664084in}{0.718685in}}%
\pgfpathlineto{\pgfqpoint{0.665193in}{0.721908in}}%
\pgfpathlineto{\pgfqpoint{0.665474in}{0.723016in}}%
\pgfpathlineto{\pgfqpoint{0.666584in}{0.726061in}}%
\pgfpathlineto{\pgfqpoint{0.666896in}{0.727160in}}%
\pgfpathlineto{\pgfqpoint{0.668003in}{0.730354in}}%
\pgfpathlineto{\pgfqpoint{0.668350in}{0.731462in}}%
\pgfpathlineto{\pgfqpoint{0.669452in}{0.734750in}}%
\pgfpathlineto{\pgfqpoint{0.669893in}{0.735858in}}%
\pgfpathlineto{\pgfqpoint{0.670986in}{0.738931in}}%
\pgfpathlineto{\pgfqpoint{0.671382in}{0.740030in}}%
\pgfpathlineto{\pgfqpoint{0.672489in}{0.743233in}}%
\pgfpathlineto{\pgfqpoint{0.672817in}{0.744342in}}%
\pgfpathlineto{\pgfqpoint{0.673927in}{0.746968in}}%
\pgfpathlineto{\pgfqpoint{0.674311in}{0.748076in}}%
\pgfpathlineto{\pgfqpoint{0.675414in}{0.751065in}}%
\pgfpathlineto{\pgfqpoint{0.675862in}{0.752173in}}%
\pgfpathlineto{\pgfqpoint{0.676964in}{0.755284in}}%
\pgfpathlineto{\pgfqpoint{0.677316in}{0.756392in}}%
\pgfpathlineto{\pgfqpoint{0.678413in}{0.759307in}}%
\pgfpathlineto{\pgfqpoint{0.678868in}{0.760415in}}%
\pgfpathlineto{\pgfqpoint{0.679978in}{0.763078in}}%
\pgfpathlineto{\pgfqpoint{0.680329in}{0.764187in}}%
\pgfpathlineto{\pgfqpoint{0.681436in}{0.767316in}}%
\pgfpathlineto{\pgfqpoint{0.681852in}{0.768396in}}%
\pgfpathlineto{\pgfqpoint{0.682956in}{0.771395in}}%
\pgfpathlineto{\pgfqpoint{0.683313in}{0.772493in}}%
\pgfpathlineto{\pgfqpoint{0.684422in}{0.775138in}}%
\pgfpathlineto{\pgfqpoint{0.684863in}{0.776246in}}%
\pgfpathlineto{\pgfqpoint{0.685970in}{0.778668in}}%
\pgfpathlineto{\pgfqpoint{0.686416in}{0.779776in}}%
\pgfpathlineto{\pgfqpoint{0.687513in}{0.782411in}}%
\pgfpathlineto{\pgfqpoint{0.687945in}{0.783519in}}%
\pgfpathlineto{\pgfqpoint{0.689049in}{0.786509in}}%
\pgfpathlineto{\pgfqpoint{0.689568in}{0.787617in}}%
\pgfpathlineto{\pgfqpoint{0.690675in}{0.790206in}}%
\pgfpathlineto{\pgfqpoint{0.691167in}{0.791314in}}%
\pgfpathlineto{\pgfqpoint{0.692276in}{0.793968in}}%
\pgfpathlineto{\pgfqpoint{0.692715in}{0.795076in}}%
\pgfpathlineto{\pgfqpoint{0.693824in}{0.797600in}}%
\pgfpathlineto{\pgfqpoint{0.694298in}{0.798708in}}%
\pgfpathlineto{\pgfqpoint{0.695407in}{0.801437in}}%
\pgfpathlineto{\pgfqpoint{0.695853in}{0.802545in}}%
\pgfpathlineto{\pgfqpoint{0.696958in}{0.805208in}}%
\pgfpathlineto{\pgfqpoint{0.696962in}{0.805208in}}%
\pgfpathlineto{\pgfqpoint{0.697436in}{0.806317in}}%
\pgfpathlineto{\pgfqpoint{0.698545in}{0.808933in}}%
\pgfpathlineto{\pgfqpoint{0.699047in}{0.810032in}}%
\pgfpathlineto{\pgfqpoint{0.700150in}{0.812500in}}%
\pgfpathlineto{\pgfqpoint{0.700607in}{0.813608in}}%
\pgfpathlineto{\pgfqpoint{0.701716in}{0.815955in}}%
\pgfpathlineto{\pgfqpoint{0.702138in}{0.817054in}}%
\pgfpathlineto{\pgfqpoint{0.703248in}{0.819354in}}%
\pgfpathlineto{\pgfqpoint{0.703843in}{0.820462in}}%
\pgfpathlineto{\pgfqpoint{0.704953in}{0.822949in}}%
\pgfpathlineto{\pgfqpoint{0.705450in}{0.824057in}}%
\pgfpathlineto{\pgfqpoint{0.706559in}{0.826739in}}%
\pgfpathlineto{\pgfqpoint{0.707141in}{0.827847in}}%
\pgfpathlineto{\pgfqpoint{0.708250in}{0.830380in}}%
\pgfpathlineto{\pgfqpoint{0.708783in}{0.831488in}}%
\pgfpathlineto{\pgfqpoint{0.709892in}{0.833994in}}%
\pgfpathlineto{\pgfqpoint{0.710354in}{0.835083in}}%
\pgfpathlineto{\pgfqpoint{0.711461in}{0.837691in}}%
\pgfpathlineto{\pgfqpoint{0.712000in}{0.838799in}}%
\pgfpathlineto{\pgfqpoint{0.713100in}{0.841229in}}%
\pgfpathlineto{\pgfqpoint{0.713583in}{0.842338in}}%
\pgfpathlineto{\pgfqpoint{0.714693in}{0.845020in}}%
\pgfpathlineto{\pgfqpoint{0.715253in}{0.846128in}}%
\pgfpathlineto{\pgfqpoint{0.716334in}{0.848111in}}%
\pgfpathlineto{\pgfqpoint{0.716865in}{0.849210in}}%
\pgfpathlineto{\pgfqpoint{0.717964in}{0.851594in}}%
\pgfpathlineto{\pgfqpoint{0.718459in}{0.852702in}}%
\pgfpathlineto{\pgfqpoint{0.719569in}{0.854947in}}%
\pgfpathlineto{\pgfqpoint{0.720103in}{0.856055in}}%
\pgfpathlineto{\pgfqpoint{0.721206in}{0.858374in}}%
\pgfpathlineto{\pgfqpoint{0.721773in}{0.859482in}}%
\pgfpathlineto{\pgfqpoint{0.722883in}{0.861875in}}%
\pgfpathlineto{\pgfqpoint{0.723436in}{0.862984in}}%
\pgfpathlineto{\pgfqpoint{0.724545in}{0.865004in}}%
\pgfpathlineto{\pgfqpoint{0.725085in}{0.866113in}}%
\pgfpathlineto{\pgfqpoint{0.726185in}{0.868059in}}%
\pgfpathlineto{\pgfqpoint{0.726773in}{0.869167in}}%
\pgfpathlineto{\pgfqpoint{0.727878in}{0.871365in}}%
\pgfpathlineto{\pgfqpoint{0.728474in}{0.872473in}}%
\pgfpathlineto{\pgfqpoint{0.729583in}{0.874699in}}%
\pgfpathlineto{\pgfqpoint{0.730214in}{0.875807in}}%
\pgfpathlineto{\pgfqpoint{0.731321in}{0.878098in}}%
\pgfpathlineto{\pgfqpoint{0.731919in}{0.879197in}}%
\pgfpathlineto{\pgfqpoint{0.733028in}{0.881413in}}%
\pgfpathlineto{\pgfqpoint{0.733516in}{0.882521in}}%
\pgfpathlineto{\pgfqpoint{0.734625in}{0.884468in}}%
\pgfpathlineto{\pgfqpoint{0.735235in}{0.885566in}}%
\pgfpathlineto{\pgfqpoint{0.736316in}{0.887503in}}%
\pgfpathlineto{\pgfqpoint{0.736886in}{0.888602in}}%
\pgfpathlineto{\pgfqpoint{0.737989in}{0.890446in}}%
\pgfpathlineto{\pgfqpoint{0.737996in}{0.890446in}}%
\pgfpathlineto{\pgfqpoint{0.738631in}{0.891527in}}%
\pgfpathlineto{\pgfqpoint{0.739738in}{0.893603in}}%
\pgfpathlineto{\pgfqpoint{0.740367in}{0.894711in}}%
\pgfpathlineto{\pgfqpoint{0.741467in}{0.896984in}}%
\pgfpathlineto{\pgfqpoint{0.742098in}{0.898092in}}%
\pgfpathlineto{\pgfqpoint{0.743202in}{0.900252in}}%
\pgfpathlineto{\pgfqpoint{0.743913in}{0.901351in}}%
\pgfpathlineto{\pgfqpoint{0.745022in}{0.903353in}}%
\pgfpathlineto{\pgfqpoint{0.745627in}{0.904462in}}%
\pgfpathlineto{\pgfqpoint{0.746706in}{0.906436in}}%
\pgfpathlineto{\pgfqpoint{0.746732in}{0.906436in}}%
\pgfpathlineto{\pgfqpoint{0.747410in}{0.907544in}}%
\pgfpathlineto{\pgfqpoint{0.748514in}{0.909546in}}%
\pgfpathlineto{\pgfqpoint{0.749098in}{0.910645in}}%
\pgfpathlineto{\pgfqpoint{0.750208in}{0.912610in}}%
\pgfpathlineto{\pgfqpoint{0.750829in}{0.913700in}}%
\pgfpathlineto{\pgfqpoint{0.751936in}{0.915627in}}%
\pgfpathlineto{\pgfqpoint{0.752628in}{0.916726in}}%
\pgfpathlineto{\pgfqpoint{0.753737in}{0.918281in}}%
\pgfpathlineto{\pgfqpoint{0.754382in}{0.919390in}}%
\pgfpathlineto{\pgfqpoint{0.755492in}{0.921429in}}%
\pgfpathlineto{\pgfqpoint{0.756176in}{0.922519in}}%
\pgfpathlineto{\pgfqpoint{0.757281in}{0.924391in}}%
\pgfpathlineto{\pgfqpoint{0.758001in}{0.925489in}}%
\pgfpathlineto{\pgfqpoint{0.759106in}{0.927743in}}%
\pgfpathlineto{\pgfqpoint{0.759868in}{0.928851in}}%
\pgfpathlineto{\pgfqpoint{0.760977in}{0.930947in}}%
\pgfpathlineto{\pgfqpoint{0.761657in}{0.932045in}}%
\pgfpathlineto{\pgfqpoint{0.762755in}{0.934113in}}%
\pgfpathlineto{\pgfqpoint{0.763294in}{0.935221in}}%
\pgfpathlineto{\pgfqpoint{0.764399in}{0.937056in}}%
\pgfpathlineto{\pgfqpoint{0.765039in}{0.938164in}}%
\pgfpathlineto{\pgfqpoint{0.766139in}{0.939803in}}%
\pgfpathlineto{\pgfqpoint{0.766149in}{0.939803in}}%
\pgfpathlineto{\pgfqpoint{0.766606in}{0.940911in}}%
\pgfpathlineto{\pgfqpoint{0.767715in}{0.942327in}}%
\pgfpathlineto{\pgfqpoint{0.768438in}{0.943435in}}%
\pgfpathlineto{\pgfqpoint{0.769545in}{0.945176in}}%
\pgfpathlineto{\pgfqpoint{0.770201in}{0.946266in}}%
\pgfpathlineto{\pgfqpoint{0.771308in}{0.948016in}}%
\pgfpathlineto{\pgfqpoint{0.771974in}{0.949115in}}%
\pgfpathlineto{\pgfqpoint{0.773084in}{0.950689in}}%
\pgfpathlineto{\pgfqpoint{0.773715in}{0.951788in}}%
\pgfpathlineto{\pgfqpoint{0.774824in}{0.953585in}}%
\pgfpathlineto{\pgfqpoint{0.775593in}{0.954675in}}%
\pgfpathlineto{\pgfqpoint{0.776695in}{0.956454in}}%
\pgfpathlineto{\pgfqpoint{0.777364in}{0.957553in}}%
\pgfpathlineto{\pgfqpoint{0.778473in}{0.959368in}}%
\pgfpathlineto{\pgfqpoint{0.779149in}{0.960477in}}%
\pgfpathlineto{\pgfqpoint{0.780256in}{0.962097in}}%
\pgfpathlineto{\pgfqpoint{0.780858in}{0.963205in}}%
\pgfpathlineto{\pgfqpoint{0.781951in}{0.964984in}}%
\pgfpathlineto{\pgfqpoint{0.782610in}{0.966092in}}%
\pgfpathlineto{\pgfqpoint{0.783720in}{0.967834in}}%
\pgfpathlineto{\pgfqpoint{0.784477in}{0.968942in}}%
\pgfpathlineto{\pgfqpoint{0.785584in}{0.970730in}}%
\pgfpathlineto{\pgfqpoint{0.786220in}{0.971838in}}%
\pgfpathlineto{\pgfqpoint{0.787329in}{0.973701in}}%
\pgfpathlineto{\pgfqpoint{0.788056in}{0.974809in}}%
\pgfpathlineto{\pgfqpoint{0.789165in}{0.976345in}}%
\pgfpathlineto{\pgfqpoint{0.789881in}{0.977453in}}%
\pgfpathlineto{\pgfqpoint{0.790985in}{0.978990in}}%
\pgfpathlineto{\pgfqpoint{0.791717in}{0.980089in}}%
\pgfpathlineto{\pgfqpoint{0.792826in}{0.981849in}}%
\pgfpathlineto{\pgfqpoint{0.793560in}{0.982957in}}%
\pgfpathlineto{\pgfqpoint{0.794665in}{0.984782in}}%
\pgfpathlineto{\pgfqpoint{0.795357in}{0.985881in}}%
\pgfpathlineto{\pgfqpoint{0.796464in}{0.987483in}}%
\pgfpathlineto{\pgfqpoint{0.797207in}{0.988591in}}%
\pgfpathlineto{\pgfqpoint{0.798312in}{0.990240in}}%
\pgfpathlineto{\pgfqpoint{0.799107in}{0.991348in}}%
\pgfpathlineto{\pgfqpoint{0.800216in}{0.993024in}}%
\pgfpathlineto{\pgfqpoint{0.800990in}{0.994132in}}%
\pgfpathlineto{\pgfqpoint{0.802093in}{0.995622in}}%
\pgfpathlineto{\pgfqpoint{0.802799in}{0.996703in}}%
\pgfpathlineto{\pgfqpoint{0.803899in}{0.998397in}}%
\pgfpathlineto{\pgfqpoint{0.804715in}{0.999506in}}%
\pgfpathlineto{\pgfqpoint{0.805817in}{1.001107in}}%
\pgfpathlineto{\pgfqpoint{0.806765in}{1.002216in}}%
\pgfpathlineto{\pgfqpoint{0.807874in}{1.003780in}}%
\pgfpathlineto{\pgfqpoint{0.808622in}{1.004879in}}%
\pgfpathlineto{\pgfqpoint{0.809731in}{1.006350in}}%
\pgfpathlineto{\pgfqpoint{0.810512in}{1.007459in}}%
\pgfpathlineto{\pgfqpoint{0.811622in}{1.008976in}}%
\pgfpathlineto{\pgfqpoint{0.812522in}{1.010085in}}%
\pgfpathlineto{\pgfqpoint{0.813632in}{1.011351in}}%
\pgfpathlineto{\pgfqpoint{0.814485in}{1.012459in}}%
\pgfpathlineto{\pgfqpoint{0.815581in}{1.014061in}}%
\pgfpathlineto{\pgfqpoint{0.816406in}{1.015169in}}%
\pgfpathlineto{\pgfqpoint{0.817499in}{1.016920in}}%
\pgfpathlineto{\pgfqpoint{0.818371in}{1.018028in}}%
\pgfpathlineto{\pgfqpoint{0.819478in}{1.019304in}}%
\pgfpathlineto{\pgfqpoint{0.820325in}{1.020412in}}%
\pgfpathlineto{\pgfqpoint{0.821434in}{1.022182in}}%
\pgfpathlineto{\pgfqpoint{0.822405in}{1.023290in}}%
\pgfpathlineto{\pgfqpoint{0.823515in}{1.024743in}}%
\pgfpathlineto{\pgfqpoint{0.824289in}{1.025851in}}%
\pgfpathlineto{\pgfqpoint{0.825396in}{1.027294in}}%
\pgfpathlineto{\pgfqpoint{0.826212in}{1.028402in}}%
\pgfpathlineto{\pgfqpoint{0.827319in}{1.029809in}}%
\pgfpathlineto{\pgfqpoint{0.828184in}{1.030917in}}%
\pgfpathlineto{\pgfqpoint{0.829282in}{1.032528in}}%
\pgfpathlineto{\pgfqpoint{0.830018in}{1.033627in}}%
\pgfpathlineto{\pgfqpoint{0.831125in}{1.035294in}}%
\pgfpathlineto{\pgfqpoint{0.831946in}{1.036393in}}%
\pgfpathlineto{\pgfqpoint{0.833046in}{1.037873in}}%
\pgfpathlineto{\pgfqpoint{0.834000in}{1.038982in}}%
\pgfpathlineto{\pgfqpoint{0.835105in}{1.040388in}}%
\pgfpathlineto{\pgfqpoint{0.835879in}{1.041468in}}%
\pgfpathlineto{\pgfqpoint{0.836986in}{1.042753in}}%
\pgfpathlineto{\pgfqpoint{0.837870in}{1.043852in}}%
\pgfpathlineto{\pgfqpoint{0.838970in}{1.045240in}}%
\pgfpathlineto{\pgfqpoint{0.839807in}{1.046348in}}%
\pgfpathlineto{\pgfqpoint{0.840917in}{1.047717in}}%
\pgfpathlineto{\pgfqpoint{0.841740in}{1.048825in}}%
\pgfpathlineto{\pgfqpoint{0.842845in}{1.049905in}}%
\pgfpathlineto{\pgfqpoint{0.843785in}{1.051004in}}%
\pgfpathlineto{\pgfqpoint{0.844883in}{1.052503in}}%
\pgfpathlineto{\pgfqpoint{0.845811in}{1.053612in}}%
\pgfpathlineto{\pgfqpoint{0.846904in}{1.055334in}}%
\pgfpathlineto{\pgfqpoint{0.847847in}{1.056433in}}%
\pgfpathlineto{\pgfqpoint{0.848952in}{1.057923in}}%
\pgfpathlineto{\pgfqpoint{0.849876in}{1.059031in}}%
\pgfpathlineto{\pgfqpoint{0.850980in}{1.060289in}}%
\pgfpathlineto{\pgfqpoint{0.851776in}{1.061397in}}%
\pgfpathlineto{\pgfqpoint{0.852873in}{1.062905in}}%
\pgfpathlineto{\pgfqpoint{0.853839in}{1.064004in}}%
\pgfpathlineto{\pgfqpoint{0.854949in}{1.065606in}}%
\pgfpathlineto{\pgfqpoint{0.855920in}{1.066714in}}%
\pgfpathlineto{\pgfqpoint{0.857015in}{1.068139in}}%
\pgfpathlineto{\pgfqpoint{0.857883in}{1.069238in}}%
\pgfpathlineto{\pgfqpoint{0.858980in}{1.070830in}}%
\pgfpathlineto{\pgfqpoint{0.859871in}{1.071939in}}%
\pgfpathlineto{\pgfqpoint{0.860981in}{1.073475in}}%
\pgfpathlineto{\pgfqpoint{0.862095in}{1.074574in}}%
\pgfpathlineto{\pgfqpoint{0.863195in}{1.075897in}}%
\pgfpathlineto{\pgfqpoint{0.864156in}{1.077005in}}%
\pgfpathlineto{\pgfqpoint{0.865266in}{1.078364in}}%
\pgfpathlineto{\pgfqpoint{0.866023in}{1.079473in}}%
\pgfpathlineto{\pgfqpoint{0.867133in}{1.080814in}}%
\pgfpathlineto{\pgfqpoint{0.868186in}{1.081922in}}%
\pgfpathlineto{\pgfqpoint{0.869288in}{1.083123in}}%
\pgfpathlineto{\pgfqpoint{0.870242in}{1.084222in}}%
\pgfpathlineto{\pgfqpoint{0.871349in}{1.085805in}}%
\pgfpathlineto{\pgfqpoint{0.872290in}{1.086895in}}%
\pgfpathlineto{\pgfqpoint{0.873397in}{1.088347in}}%
\pgfpathlineto{\pgfqpoint{0.874377in}{1.089456in}}%
\pgfpathlineto{\pgfqpoint{0.875435in}{1.090862in}}%
\pgfpathlineto{\pgfqpoint{0.876418in}{1.091970in}}%
\pgfpathlineto{\pgfqpoint{0.877499in}{1.093078in}}%
\pgfpathlineto{\pgfqpoint{0.878578in}{1.094186in}}%
\pgfpathlineto{\pgfqpoint{0.879687in}{1.095341in}}%
\pgfpathlineto{\pgfqpoint{0.880559in}{1.096449in}}%
\pgfpathlineto{\pgfqpoint{0.881659in}{1.097623in}}%
\pgfpathlineto{\pgfqpoint{0.881669in}{1.097623in}}%
\pgfpathlineto{\pgfqpoint{0.882680in}{1.098731in}}%
\pgfpathlineto{\pgfqpoint{0.883768in}{1.100016in}}%
\pgfpathlineto{\pgfqpoint{0.883784in}{1.100016in}}%
\pgfpathlineto{\pgfqpoint{0.884659in}{1.101115in}}%
\pgfpathlineto{\pgfqpoint{0.885745in}{1.102512in}}%
\pgfpathlineto{\pgfqpoint{0.886889in}{1.103620in}}%
\pgfpathlineto{\pgfqpoint{0.887987in}{1.104840in}}%
\pgfpathlineto{\pgfqpoint{0.889052in}{1.105948in}}%
\pgfpathlineto{\pgfqpoint{0.890147in}{1.106945in}}%
\pgfpathlineto{\pgfqpoint{0.891144in}{1.108043in}}%
\pgfpathlineto{\pgfqpoint{0.892239in}{1.109366in}}%
\pgfpathlineto{\pgfqpoint{0.893177in}{1.110474in}}%
\pgfpathlineto{\pgfqpoint{0.894270in}{1.111694in}}%
\pgfpathlineto{\pgfqpoint{0.895241in}{1.112802in}}%
\pgfpathlineto{\pgfqpoint{0.896350in}{1.114013in}}%
\pgfpathlineto{\pgfqpoint{0.897302in}{1.115121in}}%
\pgfpathlineto{\pgfqpoint{0.898409in}{1.116415in}}%
\pgfpathlineto{\pgfqpoint{0.899409in}{1.117514in}}%
\pgfpathlineto{\pgfqpoint{0.900513in}{1.118809in}}%
\pgfpathlineto{\pgfqpoint{0.901522in}{1.119908in}}%
\pgfpathlineto{\pgfqpoint{0.902612in}{1.121323in}}%
\pgfpathlineto{\pgfqpoint{0.903787in}{1.122431in}}%
\pgfpathlineto{\pgfqpoint{0.904875in}{1.123782in}}%
\pgfpathlineto{\pgfqpoint{0.906090in}{1.124890in}}%
\pgfpathlineto{\pgfqpoint{0.907179in}{1.126249in}}%
\pgfpathlineto{\pgfqpoint{0.908023in}{1.127348in}}%
\pgfpathlineto{\pgfqpoint{0.909120in}{1.128568in}}%
\pgfpathlineto{\pgfqpoint{0.909130in}{1.128568in}}%
\pgfpathlineto{\pgfqpoint{0.910298in}{1.129667in}}%
\pgfpathlineto{\pgfqpoint{0.911402in}{1.131055in}}%
\pgfpathlineto{\pgfqpoint{0.912502in}{1.132163in}}%
\pgfpathlineto{\pgfqpoint{0.913602in}{1.133355in}}%
\pgfpathlineto{\pgfqpoint{0.913612in}{1.133355in}}%
\pgfpathlineto{\pgfqpoint{0.914700in}{1.134463in}}%
\pgfpathlineto{\pgfqpoint{0.915807in}{1.135664in}}%
\pgfpathlineto{\pgfqpoint{0.916862in}{1.136773in}}%
\pgfpathlineto{\pgfqpoint{0.917967in}{1.137704in}}%
\pgfpathlineto{\pgfqpoint{0.918900in}{1.138812in}}%
\pgfpathlineto{\pgfqpoint{0.920010in}{1.140051in}}%
\pgfpathlineto{\pgfqpoint{0.921035in}{1.141159in}}%
\pgfpathlineto{\pgfqpoint{0.922144in}{1.142183in}}%
\pgfpathlineto{\pgfqpoint{0.923335in}{1.143291in}}%
\pgfpathlineto{\pgfqpoint{0.924445in}{1.144558in}}%
\pgfpathlineto{\pgfqpoint{0.925636in}{1.145666in}}%
\pgfpathlineto{\pgfqpoint{0.926743in}{1.146663in}}%
\pgfpathlineto{\pgfqpoint{0.927770in}{1.147771in}}%
\pgfpathlineto{\pgfqpoint{0.928873in}{1.148860in}}%
\pgfpathlineto{\pgfqpoint{0.929832in}{1.149969in}}%
\pgfpathlineto{\pgfqpoint{0.930918in}{1.151067in}}%
\pgfpathlineto{\pgfqpoint{0.930932in}{1.151067in}}%
\pgfpathlineto{\pgfqpoint{0.932069in}{1.152176in}}%
\pgfpathlineto{\pgfqpoint{0.933171in}{1.153368in}}%
\pgfpathlineto{\pgfqpoint{0.933179in}{1.153368in}}%
\pgfpathlineto{\pgfqpoint{0.934215in}{1.154457in}}%
\pgfpathlineto{\pgfqpoint{0.935313in}{1.155714in}}%
\pgfpathlineto{\pgfqpoint{0.936438in}{1.156823in}}%
\pgfpathlineto{\pgfqpoint{0.937545in}{1.157968in}}%
\pgfpathlineto{\pgfqpoint{0.938643in}{1.159058in}}%
\pgfpathlineto{\pgfqpoint{0.938643in}{1.159067in}}%
\pgfpathlineto{\pgfqpoint{0.939750in}{1.160333in}}%
\pgfpathlineto{\pgfqpoint{0.940754in}{1.161442in}}%
\pgfpathlineto{\pgfqpoint{0.941842in}{1.162503in}}%
\pgfpathlineto{\pgfqpoint{0.942914in}{1.163602in}}%
\pgfpathlineto{\pgfqpoint{0.944014in}{1.164701in}}%
\pgfpathlineto{\pgfqpoint{0.945449in}{1.165809in}}%
\pgfpathlineto{\pgfqpoint{0.946558in}{1.166703in}}%
\pgfpathlineto{\pgfqpoint{0.947783in}{1.167811in}}%
\pgfpathlineto{\pgfqpoint{0.948890in}{1.168854in}}%
\pgfpathlineto{\pgfqpoint{0.950004in}{1.169963in}}%
\pgfpathlineto{\pgfqpoint{0.951104in}{1.170903in}}%
\pgfpathlineto{\pgfqpoint{0.952368in}{1.171974in}}%
\pgfpathlineto{\pgfqpoint{0.953475in}{1.173036in}}%
\pgfpathlineto{\pgfqpoint{0.954633in}{1.174144in}}%
\pgfpathlineto{\pgfqpoint{0.955712in}{1.175224in}}%
\pgfpathlineto{\pgfqpoint{0.956887in}{1.176323in}}%
\pgfpathlineto{\pgfqpoint{0.957996in}{1.177655in}}%
\pgfpathlineto{\pgfqpoint{0.958986in}{1.178763in}}%
\pgfpathlineto{\pgfqpoint{0.960067in}{1.179862in}}%
\pgfpathlineto{\pgfqpoint{0.961310in}{1.180970in}}%
\pgfpathlineto{\pgfqpoint{0.962415in}{1.182106in}}%
\pgfpathlineto{\pgfqpoint{0.963503in}{1.183214in}}%
\pgfpathlineto{\pgfqpoint{0.964613in}{1.184434in}}%
\pgfpathlineto{\pgfqpoint{0.965818in}{1.185543in}}%
\pgfpathlineto{\pgfqpoint{0.966918in}{1.186632in}}%
\pgfpathlineto{\pgfqpoint{0.967955in}{1.187740in}}%
\pgfpathlineto{\pgfqpoint{0.969052in}{1.188737in}}%
\pgfpathlineto{\pgfqpoint{0.970276in}{1.189845in}}%
\pgfpathlineto{\pgfqpoint{0.971372in}{1.190767in}}%
\pgfpathlineto{\pgfqpoint{0.972720in}{1.191875in}}%
\pgfpathlineto{\pgfqpoint{0.973827in}{1.192955in}}%
\pgfpathlineto{\pgfqpoint{0.975105in}{1.194064in}}%
\pgfpathlineto{\pgfqpoint{0.976212in}{1.195293in}}%
\pgfpathlineto{\pgfqpoint{0.977720in}{1.196401in}}%
\pgfpathlineto{\pgfqpoint{0.979203in}{1.197779in}}%
\pgfpathlineto{\pgfqpoint{0.980589in}{1.198887in}}%
\pgfpathlineto{\pgfqpoint{0.981698in}{1.199996in}}%
\pgfpathlineto{\pgfqpoint{0.982857in}{1.201095in}}%
\pgfpathlineto{\pgfqpoint{0.983964in}{1.202287in}}%
\pgfpathlineto{\pgfqpoint{0.985319in}{1.203385in}}%
\pgfpathlineto{\pgfqpoint{0.986410in}{1.204242in}}%
\pgfpathlineto{\pgfqpoint{0.987730in}{1.205350in}}%
\pgfpathlineto{\pgfqpoint{0.988837in}{1.206300in}}%
\pgfpathlineto{\pgfqpoint{0.990017in}{1.207408in}}%
\pgfpathlineto{\pgfqpoint{0.991089in}{1.208144in}}%
\pgfpathlineto{\pgfqpoint{0.992522in}{1.209252in}}%
\pgfpathlineto{\pgfqpoint{0.993631in}{1.210277in}}%
\pgfpathlineto{\pgfqpoint{0.995038in}{1.211385in}}%
\pgfpathlineto{\pgfqpoint{0.996147in}{1.212512in}}%
\pgfpathlineto{\pgfqpoint{0.997440in}{1.213611in}}%
\pgfpathlineto{\pgfqpoint{0.998542in}{1.214607in}}%
\pgfpathlineto{\pgfqpoint{0.999853in}{1.215715in}}%
\pgfpathlineto{\pgfqpoint{1.000962in}{1.216646in}}%
\pgfpathlineto{\pgfqpoint{1.002161in}{1.217755in}}%
\pgfpathlineto{\pgfqpoint{1.003270in}{1.218472in}}%
\pgfpathlineto{\pgfqpoint{1.004337in}{1.219580in}}%
\pgfpathlineto{\pgfqpoint{1.005432in}{1.220688in}}%
\pgfpathlineto{\pgfqpoint{1.006406in}{1.221796in}}%
\pgfpathlineto{\pgfqpoint{1.007513in}{1.222858in}}%
\pgfpathlineto{\pgfqpoint{1.008721in}{1.223966in}}%
\pgfpathlineto{\pgfqpoint{1.009825in}{1.225000in}}%
\pgfpathlineto{\pgfqpoint{1.011289in}{1.226108in}}%
\pgfpathlineto{\pgfqpoint{1.012384in}{1.227170in}}%
\pgfpathlineto{\pgfqpoint{1.013920in}{1.228259in}}%
\pgfpathlineto{\pgfqpoint{1.015027in}{1.229470in}}%
\pgfpathlineto{\pgfqpoint{1.016392in}{1.230569in}}%
\pgfpathlineto{\pgfqpoint{1.017494in}{1.231584in}}%
\pgfpathlineto{\pgfqpoint{1.018883in}{1.232692in}}%
\pgfpathlineto{\pgfqpoint{1.019966in}{1.233558in}}%
\pgfpathlineto{\pgfqpoint{1.021383in}{1.234666in}}%
\pgfpathlineto{\pgfqpoint{1.022487in}{1.235728in}}%
\pgfpathlineto{\pgfqpoint{1.023784in}{1.236836in}}%
\pgfpathlineto{\pgfqpoint{1.024873in}{1.237488in}}%
\pgfpathlineto{\pgfqpoint{1.026188in}{1.238596in}}%
\pgfpathlineto{\pgfqpoint{1.027286in}{1.239378in}}%
\pgfpathlineto{\pgfqpoint{1.028834in}{1.240487in}}%
\pgfpathlineto{\pgfqpoint{1.029936in}{1.241446in}}%
\pgfpathlineto{\pgfqpoint{1.031313in}{1.242554in}}%
\pgfpathlineto{\pgfqpoint{1.032378in}{1.243383in}}%
\pgfpathlineto{\pgfqpoint{1.033904in}{1.244491in}}%
\pgfpathlineto{\pgfqpoint{1.035011in}{1.245487in}}%
\pgfpathlineto{\pgfqpoint{1.036322in}{1.246596in}}%
\pgfpathlineto{\pgfqpoint{1.037420in}{1.247518in}}%
\pgfpathlineto{\pgfqpoint{1.038954in}{1.248626in}}%
\pgfpathlineto{\pgfqpoint{1.040063in}{1.249687in}}%
\pgfpathlineto{\pgfqpoint{1.041280in}{1.250786in}}%
\pgfpathlineto{\pgfqpoint{1.042369in}{1.251745in}}%
\pgfpathlineto{\pgfqpoint{1.043682in}{1.252854in}}%
\pgfpathlineto{\pgfqpoint{1.044787in}{1.253766in}}%
\pgfpathlineto{\pgfqpoint{1.046128in}{1.254875in}}%
\pgfpathlineto{\pgfqpoint{1.047212in}{1.255703in}}%
\pgfpathlineto{\pgfqpoint{1.048642in}{1.256812in}}%
\pgfpathlineto{\pgfqpoint{1.049747in}{1.257706in}}%
\pgfpathlineto{\pgfqpoint{1.051006in}{1.258814in}}%
\pgfpathlineto{\pgfqpoint{1.052102in}{1.259708in}}%
\pgfpathlineto{\pgfqpoint{1.053657in}{1.260816in}}%
\pgfpathlineto{\pgfqpoint{1.054747in}{1.261915in}}%
\pgfpathlineto{\pgfqpoint{1.056464in}{1.263023in}}%
\pgfpathlineto{\pgfqpoint{1.057568in}{1.263936in}}%
\pgfpathlineto{\pgfqpoint{1.059039in}{1.265044in}}%
\pgfpathlineto{\pgfqpoint{1.060148in}{1.265863in}}%
\pgfpathlineto{\pgfqpoint{1.061731in}{1.266972in}}%
\pgfpathlineto{\pgfqpoint{1.062838in}{1.267856in}}%
\pgfpathlineto{\pgfqpoint{1.064351in}{1.268964in}}%
\pgfpathlineto{\pgfqpoint{1.065458in}{1.269803in}}%
\pgfpathlineto{\pgfqpoint{1.066771in}{1.270892in}}%
\pgfpathlineto{\pgfqpoint{1.067867in}{1.271758in}}%
\pgfpathlineto{\pgfqpoint{1.067881in}{1.271758in}}%
\pgfpathlineto{\pgfqpoint{1.069408in}{1.272866in}}%
\pgfpathlineto{\pgfqpoint{1.070505in}{1.273695in}}%
\pgfpathlineto{\pgfqpoint{1.071884in}{1.274785in}}%
\pgfpathlineto{\pgfqpoint{1.072991in}{1.275725in}}%
\pgfpathlineto{\pgfqpoint{1.074276in}{1.276834in}}%
\pgfpathlineto{\pgfqpoint{1.075379in}{1.277700in}}%
\pgfpathlineto{\pgfqpoint{1.076842in}{1.278808in}}%
\pgfpathlineto{\pgfqpoint{1.077951in}{1.279609in}}%
\pgfpathlineto{\pgfqpoint{1.079598in}{1.280717in}}%
\pgfpathlineto{\pgfqpoint{1.080707in}{1.281508in}}%
\pgfpathlineto{\pgfqpoint{1.081863in}{1.282617in}}%
\pgfpathlineto{\pgfqpoint{1.082956in}{1.283539in}}%
\pgfpathlineto{\pgfqpoint{1.084448in}{1.284647in}}%
\pgfpathlineto{\pgfqpoint{1.085522in}{1.285280in}}%
\pgfpathlineto{\pgfqpoint{1.086929in}{1.286388in}}%
\pgfpathlineto{\pgfqpoint{1.088039in}{1.287273in}}%
\pgfpathlineto{\pgfqpoint{1.089615in}{1.288381in}}%
\pgfpathlineto{\pgfqpoint{1.090710in}{1.289154in}}%
\pgfpathlineto{\pgfqpoint{1.092162in}{1.290262in}}%
\pgfpathlineto{\pgfqpoint{1.093240in}{1.291091in}}%
\pgfpathlineto{\pgfqpoint{1.093271in}{1.291091in}}%
\pgfpathlineto{\pgfqpoint{1.094906in}{1.292199in}}%
\pgfpathlineto{\pgfqpoint{1.096003in}{1.292972in}}%
\pgfpathlineto{\pgfqpoint{1.097697in}{1.294080in}}%
\pgfpathlineto{\pgfqpoint{1.098796in}{1.294751in}}%
\pgfpathlineto{\pgfqpoint{1.100328in}{1.295859in}}%
\pgfpathlineto{\pgfqpoint{1.101426in}{1.296595in}}%
\pgfpathlineto{\pgfqpoint{1.102908in}{1.297703in}}%
\pgfpathlineto{\pgfqpoint{1.104012in}{1.298476in}}%
\pgfpathlineto{\pgfqpoint{1.105596in}{1.299584in}}%
\pgfpathlineto{\pgfqpoint{1.106686in}{1.300301in}}%
\pgfpathlineto{\pgfqpoint{1.108147in}{1.301409in}}%
\pgfpathlineto{\pgfqpoint{1.109254in}{1.302136in}}%
\pgfpathlineto{\pgfqpoint{1.110999in}{1.303244in}}%
\pgfpathlineto{\pgfqpoint{1.112106in}{1.304110in}}%
\pgfpathlineto{\pgfqpoint{1.113574in}{1.305218in}}%
\pgfpathlineto{\pgfqpoint{1.114679in}{1.306094in}}%
\pgfpathlineto{\pgfqpoint{1.116255in}{1.307202in}}%
\pgfpathlineto{\pgfqpoint{1.117357in}{1.307919in}}%
\pgfpathlineto{\pgfqpoint{1.119184in}{1.309027in}}%
\pgfpathlineto{\pgfqpoint{1.120270in}{1.309912in}}%
\pgfpathlineto{\pgfqpoint{1.121938in}{1.311020in}}%
\pgfpathlineto{\pgfqpoint{1.123026in}{1.311746in}}%
\pgfpathlineto{\pgfqpoint{1.124822in}{1.312854in}}%
\pgfpathlineto{\pgfqpoint{1.125925in}{1.313655in}}%
\pgfpathlineto{\pgfqpoint{1.127533in}{1.314764in}}%
\pgfpathlineto{\pgfqpoint{1.128608in}{1.315453in}}%
\pgfpathlineto{\pgfqpoint{1.130195in}{1.316561in}}%
\pgfpathlineto{\pgfqpoint{1.131300in}{1.317259in}}%
\pgfpathlineto{\pgfqpoint{1.133207in}{1.318368in}}%
\pgfpathlineto{\pgfqpoint{1.134316in}{1.319038in}}%
\pgfpathlineto{\pgfqpoint{1.136131in}{1.320137in}}%
\pgfpathlineto{\pgfqpoint{1.137236in}{1.320938in}}%
\pgfpathlineto{\pgfqpoint{1.138983in}{1.322046in}}%
\pgfpathlineto{\pgfqpoint{1.140081in}{1.322893in}}%
\pgfpathlineto{\pgfqpoint{1.140088in}{1.322893in}}%
\pgfpathlineto{\pgfqpoint{1.141854in}{1.324002in}}%
\pgfpathlineto{\pgfqpoint{1.142930in}{1.324784in}}%
\pgfpathlineto{\pgfqpoint{1.144403in}{1.325892in}}%
\pgfpathlineto{\pgfqpoint{1.145503in}{1.326674in}}%
\pgfpathlineto{\pgfqpoint{1.145508in}{1.326674in}}%
\pgfpathlineto{\pgfqpoint{1.147304in}{1.327773in}}%
\pgfpathlineto{\pgfqpoint{1.148407in}{1.328481in}}%
\pgfpathlineto{\pgfqpoint{1.150140in}{1.329589in}}%
\pgfpathlineto{\pgfqpoint{1.151240in}{1.330483in}}%
\pgfpathlineto{\pgfqpoint{1.152935in}{1.331591in}}%
\pgfpathlineto{\pgfqpoint{1.153979in}{1.332411in}}%
\pgfpathlineto{\pgfqpoint{1.153993in}{1.332411in}}%
\pgfpathlineto{\pgfqpoint{1.155654in}{1.333519in}}%
\pgfpathlineto{\pgfqpoint{1.156754in}{1.334301in}}%
\pgfpathlineto{\pgfqpoint{1.158688in}{1.335410in}}%
\pgfpathlineto{\pgfqpoint{1.159774in}{1.336201in}}%
\pgfpathlineto{\pgfqpoint{1.161477in}{1.337300in}}%
\pgfpathlineto{\pgfqpoint{1.162572in}{1.337905in}}%
\pgfpathlineto{\pgfqpoint{1.164448in}{1.339013in}}%
\pgfpathlineto{\pgfqpoint{1.165558in}{1.339637in}}%
\pgfpathlineto{\pgfqpoint{1.167321in}{1.340746in}}%
\pgfpathlineto{\pgfqpoint{1.168424in}{1.341491in}}%
\pgfpathlineto{\pgfqpoint{1.170401in}{1.342599in}}%
\pgfpathlineto{\pgfqpoint{1.171501in}{1.343316in}}%
\pgfpathlineto{\pgfqpoint{1.173255in}{1.344424in}}%
\pgfpathlineto{\pgfqpoint{1.174355in}{1.345281in}}%
\pgfpathlineto{\pgfqpoint{1.176421in}{1.346389in}}%
\pgfpathlineto{\pgfqpoint{1.177528in}{1.347162in}}%
\pgfpathlineto{\pgfqpoint{1.178980in}{1.348270in}}%
\pgfpathlineto{\pgfqpoint{1.180080in}{1.348969in}}%
\pgfpathlineto{\pgfqpoint{1.181773in}{1.350077in}}%
\pgfpathlineto{\pgfqpoint{1.182840in}{1.350747in}}%
\pgfpathlineto{\pgfqpoint{1.184656in}{1.351855in}}%
\pgfpathlineto{\pgfqpoint{1.185751in}{1.352489in}}%
\pgfpathlineto{\pgfqpoint{1.187604in}{1.353597in}}%
\pgfpathlineto{\pgfqpoint{1.188659in}{1.354267in}}%
\pgfpathlineto{\pgfqpoint{1.190545in}{1.355376in}}%
\pgfpathlineto{\pgfqpoint{1.191647in}{1.355972in}}%
\pgfpathlineto{\pgfqpoint{1.193586in}{1.357080in}}%
\pgfpathlineto{\pgfqpoint{1.194658in}{1.357769in}}%
\pgfpathlineto{\pgfqpoint{1.196689in}{1.358877in}}%
\pgfpathlineto{\pgfqpoint{1.197759in}{1.359482in}}%
\pgfpathlineto{\pgfqpoint{1.199656in}{1.360591in}}%
\pgfpathlineto{\pgfqpoint{1.200763in}{1.361429in}}%
\pgfpathlineto{\pgfqpoint{1.203087in}{1.362537in}}%
\pgfpathlineto{\pgfqpoint{1.204190in}{1.363245in}}%
\pgfpathlineto{\pgfqpoint{1.204197in}{1.363245in}}%
\pgfpathlineto{\pgfqpoint{1.205822in}{1.364353in}}%
\pgfpathlineto{\pgfqpoint{1.206917in}{1.365023in}}%
\pgfpathlineto{\pgfqpoint{1.209009in}{1.366132in}}%
\pgfpathlineto{\pgfqpoint{1.210107in}{1.366895in}}%
\pgfpathlineto{\pgfqpoint{1.211842in}{1.367985in}}%
\pgfpathlineto{\pgfqpoint{1.212933in}{1.368618in}}%
\pgfpathlineto{\pgfqpoint{1.214856in}{1.369717in}}%
\pgfpathlineto{\pgfqpoint{1.215932in}{1.370341in}}%
\pgfpathlineto{\pgfqpoint{1.218034in}{1.371449in}}%
\pgfpathlineto{\pgfqpoint{1.219131in}{1.372101in}}%
\pgfpathlineto{\pgfqpoint{1.220778in}{1.373209in}}%
\pgfpathlineto{\pgfqpoint{1.221875in}{1.373852in}}%
\pgfpathlineto{\pgfqpoint{1.223970in}{1.374951in}}%
\pgfpathlineto{\pgfqpoint{1.225067in}{1.375621in}}%
\pgfpathlineto{\pgfqpoint{1.226707in}{1.376729in}}%
\pgfpathlineto{\pgfqpoint{1.227790in}{1.377307in}}%
\pgfpathlineto{\pgfqpoint{1.229697in}{1.378415in}}%
\pgfpathlineto{\pgfqpoint{1.230785in}{1.379039in}}%
\pgfpathlineto{\pgfqpoint{1.232805in}{1.380147in}}%
\pgfpathlineto{\pgfqpoint{1.233886in}{1.380948in}}%
\pgfpathlineto{\pgfqpoint{1.235778in}{1.382056in}}%
\pgfpathlineto{\pgfqpoint{1.236888in}{1.382643in}}%
\pgfpathlineto{\pgfqpoint{1.238891in}{1.383751in}}%
\pgfpathlineto{\pgfqpoint{1.239991in}{1.384412in}}%
\pgfpathlineto{\pgfqpoint{1.242104in}{1.385511in}}%
\pgfpathlineto{\pgfqpoint{1.243211in}{1.386144in}}%
\pgfpathlineto{\pgfqpoint{1.245361in}{1.387253in}}%
\pgfpathlineto{\pgfqpoint{1.246419in}{1.387802in}}%
\pgfpathlineto{\pgfqpoint{1.249041in}{1.388910in}}%
\pgfpathlineto{\pgfqpoint{1.250143in}{1.389627in}}%
\pgfpathlineto{\pgfqpoint{1.252414in}{1.390735in}}%
\pgfpathlineto{\pgfqpoint{1.253521in}{1.391425in}}%
\pgfpathlineto{\pgfqpoint{1.255481in}{1.392533in}}%
\pgfpathlineto{\pgfqpoint{1.256576in}{1.393129in}}%
\pgfpathlineto{\pgfqpoint{1.258790in}{1.394237in}}%
\pgfpathlineto{\pgfqpoint{1.259900in}{1.394945in}}%
\pgfpathlineto{\pgfqpoint{1.261910in}{1.396053in}}%
\pgfpathlineto{\pgfqpoint{1.263019in}{1.396695in}}%
\pgfpathlineto{\pgfqpoint{1.265184in}{1.397794in}}%
\pgfpathlineto{\pgfqpoint{1.266293in}{1.398362in}}%
\pgfpathlineto{\pgfqpoint{1.268732in}{1.399471in}}%
\pgfpathlineto{\pgfqpoint{1.269842in}{1.400216in}}%
\pgfpathlineto{\pgfqpoint{1.271936in}{1.401324in}}%
\pgfpathlineto{\pgfqpoint{1.273038in}{1.401957in}}%
\pgfpathlineto{\pgfqpoint{1.275527in}{1.403065in}}%
\pgfpathlineto{\pgfqpoint{1.276636in}{1.403736in}}%
\pgfpathlineto{\pgfqpoint{1.278897in}{1.404844in}}%
\pgfpathlineto{\pgfqpoint{1.280004in}{1.405580in}}%
\pgfpathlineto{\pgfqpoint{1.282218in}{1.406688in}}%
\pgfpathlineto{\pgfqpoint{1.283299in}{1.407247in}}%
\pgfpathlineto{\pgfqpoint{1.285757in}{1.408355in}}%
\pgfpathlineto{\pgfqpoint{1.286859in}{1.408867in}}%
\pgfpathlineto{\pgfqpoint{1.288951in}{1.409966in}}%
\pgfpathlineto{\pgfqpoint{1.290056in}{1.410636in}}%
\pgfpathlineto{\pgfqpoint{1.292422in}{1.411745in}}%
\pgfpathlineto{\pgfqpoint{1.293508in}{1.412313in}}%
\pgfpathlineto{\pgfqpoint{1.295954in}{1.413421in}}%
\pgfpathlineto{\pgfqpoint{1.297054in}{1.413970in}}%
\pgfpathlineto{\pgfqpoint{1.299343in}{1.415078in}}%
\pgfpathlineto{\pgfqpoint{1.300434in}{1.415591in}}%
\pgfpathlineto{\pgfqpoint{1.302957in}{1.416699in}}%
\pgfpathlineto{\pgfqpoint{1.304036in}{1.417360in}}%
\pgfpathlineto{\pgfqpoint{1.306123in}{1.418468in}}%
\pgfpathlineto{\pgfqpoint{1.307214in}{1.419018in}}%
\pgfpathlineto{\pgfqpoint{1.309524in}{1.420117in}}%
\pgfpathlineto{\pgfqpoint{1.310633in}{1.420666in}}%
\pgfpathlineto{\pgfqpoint{1.312571in}{1.421774in}}%
\pgfpathlineto{\pgfqpoint{1.313670in}{1.422389in}}%
\pgfpathlineto{\pgfqpoint{1.316018in}{1.423497in}}%
\pgfpathlineto{\pgfqpoint{1.317127in}{1.424186in}}%
\pgfpathlineto{\pgfqpoint{1.319287in}{1.425294in}}%
\pgfpathlineto{\pgfqpoint{1.320390in}{1.425853in}}%
\pgfpathlineto{\pgfqpoint{1.322151in}{1.426961in}}%
\pgfpathlineto{\pgfqpoint{1.323258in}{1.427408in}}%
\pgfpathlineto{\pgfqpoint{1.325207in}{1.428507in}}%
\pgfpathlineto{\pgfqpoint{1.326309in}{1.429029in}}%
\pgfpathlineto{\pgfqpoint{1.328901in}{1.430137in}}%
\pgfpathlineto{\pgfqpoint{1.330003in}{1.430770in}}%
\pgfpathlineto{\pgfqpoint{1.332386in}{1.431878in}}%
\pgfpathlineto{\pgfqpoint{1.333495in}{1.432567in}}%
\pgfpathlineto{\pgfqpoint{1.335658in}{1.433676in}}%
\pgfpathlineto{\pgfqpoint{1.336767in}{1.434123in}}%
\pgfpathlineto{\pgfqpoint{1.339047in}{1.435231in}}%
\pgfpathlineto{\pgfqpoint{1.340114in}{1.435901in}}%
\pgfpathlineto{\pgfqpoint{1.342907in}{1.437009in}}%
\pgfpathlineto{\pgfqpoint{1.343979in}{1.437587in}}%
\pgfpathlineto{\pgfqpoint{1.346106in}{1.438695in}}%
\pgfpathlineto{\pgfqpoint{1.347215in}{1.439217in}}%
\pgfpathlineto{\pgfqpoint{1.349305in}{1.440325in}}%
\pgfpathlineto{\pgfqpoint{1.350384in}{1.440865in}}%
\pgfpathlineto{\pgfqpoint{1.352380in}{1.441973in}}%
\pgfpathlineto{\pgfqpoint{1.353484in}{1.442448in}}%
\pgfpathlineto{\pgfqpoint{1.355635in}{1.443556in}}%
\pgfpathlineto{\pgfqpoint{1.356740in}{1.444050in}}%
\pgfpathlineto{\pgfqpoint{1.359495in}{1.445158in}}%
\pgfpathlineto{\pgfqpoint{1.360600in}{1.445596in}}%
\pgfpathlineto{\pgfqpoint{1.363191in}{1.446704in}}%
\pgfpathlineto{\pgfqpoint{1.364268in}{1.447300in}}%
\pgfpathlineto{\pgfqpoint{1.366620in}{1.448408in}}%
\pgfpathlineto{\pgfqpoint{1.367713in}{1.449041in}}%
\pgfpathlineto{\pgfqpoint{1.370729in}{1.450150in}}%
\pgfpathlineto{\pgfqpoint{1.371839in}{1.450578in}}%
\pgfpathlineto{\pgfqpoint{1.374343in}{1.451686in}}%
\pgfpathlineto{\pgfqpoint{1.375446in}{1.452245in}}%
\pgfpathlineto{\pgfqpoint{1.378072in}{1.453344in}}%
\pgfpathlineto{\pgfqpoint{1.379123in}{1.453875in}}%
\pgfpathlineto{\pgfqpoint{1.381635in}{1.454983in}}%
\pgfpathlineto{\pgfqpoint{1.382728in}{1.455541in}}%
\pgfpathlineto{\pgfqpoint{1.385725in}{1.456650in}}%
\pgfpathlineto{\pgfqpoint{1.386780in}{1.457022in}}%
\pgfpathlineto{\pgfqpoint{1.389323in}{1.458121in}}%
\pgfpathlineto{\pgfqpoint{1.390376in}{1.458447in}}%
\pgfpathlineto{\pgfqpoint{1.390411in}{1.458447in}}%
\pgfpathlineto{\pgfqpoint{1.392721in}{1.459555in}}%
\pgfpathlineto{\pgfqpoint{1.393830in}{1.460002in}}%
\pgfpathlineto{\pgfqpoint{1.396565in}{1.461110in}}%
\pgfpathlineto{\pgfqpoint{1.397649in}{1.461511in}}%
\pgfpathlineto{\pgfqpoint{1.400242in}{1.462619in}}%
\pgfpathlineto{\pgfqpoint{1.401328in}{1.463075in}}%
\pgfpathlineto{\pgfqpoint{1.401335in}{1.463075in}}%
\pgfpathlineto{\pgfqpoint{1.403906in}{1.464184in}}%
\pgfpathlineto{\pgfqpoint{1.404973in}{1.464742in}}%
\pgfpathlineto{\pgfqpoint{1.408153in}{1.465850in}}%
\pgfpathlineto{\pgfqpoint{1.409213in}{1.466223in}}%
\pgfpathlineto{\pgfqpoint{1.411866in}{1.467331in}}%
\pgfpathlineto{\pgfqpoint{1.412935in}{1.467806in}}%
\pgfpathlineto{\pgfqpoint{1.415982in}{1.468914in}}%
\pgfpathlineto{\pgfqpoint{1.417075in}{1.469277in}}%
\pgfpathlineto{\pgfqpoint{1.419561in}{1.470386in}}%
\pgfpathlineto{\pgfqpoint{1.420661in}{1.470786in}}%
\pgfpathlineto{\pgfqpoint{1.423496in}{1.471894in}}%
\pgfpathlineto{\pgfqpoint{1.424591in}{1.472360in}}%
\pgfpathlineto{\pgfqpoint{1.427256in}{1.473468in}}%
\pgfpathlineto{\pgfqpoint{1.428339in}{1.473906in}}%
\pgfpathlineto{\pgfqpoint{1.431693in}{1.475005in}}%
\pgfpathlineto{\pgfqpoint{1.432779in}{1.475480in}}%
\pgfpathlineto{\pgfqpoint{1.432800in}{1.475480in}}%
\pgfpathlineto{\pgfqpoint{1.435520in}{1.476569in}}%
\pgfpathlineto{\pgfqpoint{1.436588in}{1.476877in}}%
\pgfpathlineto{\pgfqpoint{1.436627in}{1.476877in}}%
\pgfpathlineto{\pgfqpoint{1.439716in}{1.477985in}}%
\pgfpathlineto{\pgfqpoint{1.440802in}{1.478432in}}%
\pgfpathlineto{\pgfqpoint{1.443551in}{1.479540in}}%
\pgfpathlineto{\pgfqpoint{1.444634in}{1.479903in}}%
\pgfpathlineto{\pgfqpoint{1.447730in}{1.481011in}}%
\pgfpathlineto{\pgfqpoint{1.448764in}{1.481365in}}%
\pgfpathlineto{\pgfqpoint{1.451513in}{1.482473in}}%
\pgfpathlineto{\pgfqpoint{1.452618in}{1.482874in}}%
\pgfpathlineto{\pgfqpoint{1.455298in}{1.483982in}}%
\pgfpathlineto{\pgfqpoint{1.456398in}{1.484364in}}%
\pgfpathlineto{\pgfqpoint{1.458392in}{1.485463in}}%
\pgfpathlineto{\pgfqpoint{1.459482in}{1.486012in}}%
\pgfpathlineto{\pgfqpoint{1.462278in}{1.487120in}}%
\pgfpathlineto{\pgfqpoint{1.463354in}{1.487484in}}%
\pgfpathlineto{\pgfqpoint{1.466959in}{1.488592in}}%
\pgfpathlineto{\pgfqpoint{1.468054in}{1.489039in}}%
\pgfpathlineto{\pgfqpoint{1.470866in}{1.490147in}}%
\pgfpathlineto{\pgfqpoint{1.471945in}{1.490557in}}%
\pgfpathlineto{\pgfqpoint{1.475247in}{1.491665in}}%
\pgfpathlineto{\pgfqpoint{1.476354in}{1.492149in}}%
\pgfpathlineto{\pgfqpoint{1.479495in}{1.493257in}}%
\pgfpathlineto{\pgfqpoint{1.480571in}{1.493658in}}%
\pgfpathlineto{\pgfqpoint{1.483878in}{1.494766in}}%
\pgfpathlineto{\pgfqpoint{1.484959in}{1.495232in}}%
\pgfpathlineto{\pgfqpoint{1.488627in}{1.496340in}}%
\pgfpathlineto{\pgfqpoint{1.489727in}{1.496833in}}%
\pgfpathlineto{\pgfqpoint{1.493271in}{1.497942in}}%
\pgfpathlineto{\pgfqpoint{1.494345in}{1.498361in}}%
\pgfpathlineto{\pgfqpoint{1.497603in}{1.499469in}}%
\pgfpathlineto{\pgfqpoint{1.498654in}{1.499832in}}%
\pgfpathlineto{\pgfqpoint{1.502465in}{1.500940in}}%
\pgfpathlineto{\pgfqpoint{1.503553in}{1.501331in}}%
\pgfpathlineto{\pgfqpoint{1.506623in}{1.502440in}}%
\pgfpathlineto{\pgfqpoint{1.507695in}{1.502812in}}%
\pgfpathlineto{\pgfqpoint{1.510779in}{1.503920in}}%
\pgfpathlineto{\pgfqpoint{1.511883in}{1.504311in}}%
\pgfpathlineto{\pgfqpoint{1.515828in}{1.505420in}}%
\pgfpathlineto{\pgfqpoint{1.516930in}{1.505801in}}%
\pgfpathlineto{\pgfqpoint{1.520019in}{1.506910in}}%
\pgfpathlineto{\pgfqpoint{1.521129in}{1.507263in}}%
\pgfpathlineto{\pgfqpoint{1.524281in}{1.508372in}}%
\pgfpathlineto{\pgfqpoint{1.525322in}{1.508809in}}%
\pgfpathlineto{\pgfqpoint{1.528582in}{1.509917in}}%
\pgfpathlineto{\pgfqpoint{1.529658in}{1.510374in}}%
\pgfpathlineto{\pgfqpoint{1.533561in}{1.511482in}}%
\pgfpathlineto{\pgfqpoint{1.534659in}{1.511780in}}%
\pgfpathlineto{\pgfqpoint{1.538847in}{1.512888in}}%
\pgfpathlineto{\pgfqpoint{1.539945in}{1.513354in}}%
\pgfpathlineto{\pgfqpoint{1.543383in}{1.514462in}}%
\pgfpathlineto{\pgfqpoint{1.544467in}{1.514937in}}%
\pgfpathlineto{\pgfqpoint{1.547933in}{1.516045in}}%
\pgfpathlineto{\pgfqpoint{1.549002in}{1.516399in}}%
\pgfpathlineto{\pgfqpoint{1.552931in}{1.517507in}}%
\pgfpathlineto{\pgfqpoint{1.554024in}{1.517842in}}%
\pgfpathlineto{\pgfqpoint{1.558370in}{1.518951in}}%
\pgfpathlineto{\pgfqpoint{1.559434in}{1.519407in}}%
\pgfpathlineto{\pgfqpoint{1.559479in}{1.519407in}}%
\pgfpathlineto{\pgfqpoint{1.563072in}{1.520506in}}%
\pgfpathlineto{\pgfqpoint{1.564172in}{1.520822in}}%
\pgfpathlineto{\pgfqpoint{1.568147in}{1.521931in}}%
\pgfpathlineto{\pgfqpoint{1.569238in}{1.522303in}}%
\pgfpathlineto{\pgfqpoint{1.572709in}{1.523411in}}%
\pgfpathlineto{\pgfqpoint{1.573816in}{1.523924in}}%
\pgfpathlineto{\pgfqpoint{1.578448in}{1.525022in}}%
\pgfpathlineto{\pgfqpoint{1.579543in}{1.525311in}}%
\pgfpathlineto{\pgfqpoint{1.583575in}{1.526419in}}%
\pgfpathlineto{\pgfqpoint{1.584658in}{1.526792in}}%
\pgfpathlineto{\pgfqpoint{1.588389in}{1.527900in}}%
\pgfpathlineto{\pgfqpoint{1.589492in}{1.528161in}}%
\pgfpathlineto{\pgfqpoint{1.593514in}{1.529269in}}%
\pgfpathlineto{\pgfqpoint{1.594621in}{1.529641in}}%
\pgfpathlineto{\pgfqpoint{1.598481in}{1.530750in}}%
\pgfpathlineto{\pgfqpoint{1.599548in}{1.531085in}}%
\pgfpathlineto{\pgfqpoint{1.604169in}{1.532193in}}%
\pgfpathlineto{\pgfqpoint{1.605224in}{1.532435in}}%
\pgfpathlineto{\pgfqpoint{1.605259in}{1.532435in}}%
\pgfpathlineto{\pgfqpoint{1.609980in}{1.533543in}}%
\pgfpathlineto{\pgfqpoint{1.611068in}{1.533935in}}%
\pgfpathlineto{\pgfqpoint{1.615328in}{1.535043in}}%
\pgfpathlineto{\pgfqpoint{1.616413in}{1.535387in}}%
\pgfpathlineto{\pgfqpoint{1.621081in}{1.536496in}}%
\pgfpathlineto{\pgfqpoint{1.622173in}{1.536784in}}%
\pgfpathlineto{\pgfqpoint{1.622190in}{1.536784in}}%
\pgfpathlineto{\pgfqpoint{1.626388in}{1.537892in}}%
\pgfpathlineto{\pgfqpoint{1.627464in}{1.538172in}}%
\pgfpathlineto{\pgfqpoint{1.627488in}{1.538172in}}%
\pgfpathlineto{\pgfqpoint{1.631663in}{1.539280in}}%
\pgfpathlineto{\pgfqpoint{1.632758in}{1.539597in}}%
\pgfpathlineto{\pgfqpoint{1.637064in}{1.540705in}}%
\pgfpathlineto{\pgfqpoint{1.638168in}{1.540975in}}%
\pgfpathlineto{\pgfqpoint{1.642327in}{1.542083in}}%
\pgfpathlineto{\pgfqpoint{1.643368in}{1.542400in}}%
\pgfpathlineto{\pgfqpoint{1.643403in}{1.542400in}}%
\pgfpathlineto{\pgfqpoint{1.647634in}{1.543508in}}%
\pgfpathlineto{\pgfqpoint{1.648725in}{1.543871in}}%
\pgfpathlineto{\pgfqpoint{1.648741in}{1.543871in}}%
\pgfpathlineto{\pgfqpoint{1.652768in}{1.544979in}}%
\pgfpathlineto{\pgfqpoint{1.653823in}{1.545231in}}%
\pgfpathlineto{\pgfqpoint{1.653847in}{1.545231in}}%
\pgfpathlineto{\pgfqpoint{1.658171in}{1.546339in}}%
\pgfpathlineto{\pgfqpoint{1.659180in}{1.546656in}}%
\pgfpathlineto{\pgfqpoint{1.663767in}{1.547764in}}%
\pgfpathlineto{\pgfqpoint{1.664870in}{1.548099in}}%
\pgfpathlineto{\pgfqpoint{1.669190in}{1.549207in}}%
\pgfpathlineto{\pgfqpoint{1.670212in}{1.549440in}}%
\pgfpathlineto{\pgfqpoint{1.670243in}{1.549440in}}%
\pgfpathlineto{\pgfqpoint{1.675738in}{1.550548in}}%
\pgfpathlineto{\pgfqpoint{1.676821in}{1.550828in}}%
\pgfpathlineto{\pgfqpoint{1.681601in}{1.551936in}}%
\pgfpathlineto{\pgfqpoint{1.682544in}{1.552159in}}%
\pgfpathlineto{\pgfqpoint{1.687049in}{1.553267in}}%
\pgfpathlineto{\pgfqpoint{1.688140in}{1.553565in}}%
\pgfpathlineto{\pgfqpoint{1.692793in}{1.554674in}}%
\pgfpathlineto{\pgfqpoint{1.693897in}{1.554897in}}%
\pgfpathlineto{\pgfqpoint{1.698403in}{1.556005in}}%
\pgfpathlineto{\pgfqpoint{1.699503in}{1.556285in}}%
\pgfpathlineto{\pgfqpoint{1.704275in}{1.557393in}}%
\pgfpathlineto{\pgfqpoint{1.705342in}{1.557533in}}%
\pgfpathlineto{\pgfqpoint{1.711018in}{1.558641in}}%
\pgfpathlineto{\pgfqpoint{1.712064in}{1.558948in}}%
\pgfpathlineto{\pgfqpoint{1.712118in}{1.558948in}}%
\pgfpathlineto{\pgfqpoint{1.717686in}{1.560056in}}%
\pgfpathlineto{\pgfqpoint{1.718786in}{1.560392in}}%
\pgfpathlineto{\pgfqpoint{1.724466in}{1.561500in}}%
\pgfpathlineto{\pgfqpoint{1.725571in}{1.561835in}}%
\pgfpathlineto{\pgfqpoint{1.731169in}{1.562943in}}%
\pgfpathlineto{\pgfqpoint{1.732079in}{1.563204in}}%
\pgfpathlineto{\pgfqpoint{1.732116in}{1.563204in}}%
\pgfpathlineto{\pgfqpoint{1.737951in}{1.564312in}}%
\pgfpathlineto{\pgfqpoint{1.739009in}{1.564489in}}%
\pgfpathlineto{\pgfqpoint{1.739016in}{1.564489in}}%
\pgfpathlineto{\pgfqpoint{1.744312in}{1.565597in}}%
\pgfpathlineto{\pgfqpoint{1.745414in}{1.565830in}}%
\pgfpathlineto{\pgfqpoint{1.750663in}{1.566938in}}%
\pgfpathlineto{\pgfqpoint{1.751683in}{1.567218in}}%
\pgfpathlineto{\pgfqpoint{1.751718in}{1.567218in}}%
\pgfpathlineto{\pgfqpoint{1.758496in}{1.568326in}}%
\pgfpathlineto{\pgfqpoint{1.759538in}{1.568624in}}%
\pgfpathlineto{\pgfqpoint{1.765668in}{1.569732in}}%
\pgfpathlineto{\pgfqpoint{1.766726in}{1.569974in}}%
\pgfpathlineto{\pgfqpoint{1.772953in}{1.571082in}}%
\pgfpathlineto{\pgfqpoint{1.774060in}{1.571315in}}%
\pgfpathlineto{\pgfqpoint{1.779531in}{1.572423in}}%
\pgfpathlineto{\pgfqpoint{1.780495in}{1.572638in}}%
\pgfpathlineto{\pgfqpoint{1.780638in}{1.572638in}}%
\pgfpathlineto{\pgfqpoint{1.787144in}{1.573746in}}%
\pgfpathlineto{\pgfqpoint{1.788007in}{1.573857in}}%
\pgfpathlineto{\pgfqpoint{1.788035in}{1.573857in}}%
\pgfpathlineto{\pgfqpoint{1.794342in}{1.574966in}}%
\pgfpathlineto{\pgfqpoint{1.795449in}{1.575096in}}%
\pgfpathlineto{\pgfqpoint{1.801071in}{1.576204in}}%
\pgfpathlineto{\pgfqpoint{1.802163in}{1.576428in}}%
\pgfpathlineto{\pgfqpoint{1.808852in}{1.577536in}}%
\pgfpathlineto{\pgfqpoint{1.809849in}{1.577759in}}%
\pgfpathlineto{\pgfqpoint{1.809922in}{1.577759in}}%
\pgfpathlineto{\pgfqpoint{1.816887in}{1.578868in}}%
\pgfpathlineto{\pgfqpoint{1.817872in}{1.579082in}}%
\pgfpathlineto{\pgfqpoint{1.825445in}{1.580190in}}%
\pgfpathlineto{\pgfqpoint{1.826484in}{1.580441in}}%
\pgfpathlineto{\pgfqpoint{1.833192in}{1.581550in}}%
\pgfpathlineto{\pgfqpoint{1.834278in}{1.581773in}}%
\pgfpathlineto{\pgfqpoint{1.841630in}{1.582881in}}%
\pgfpathlineto{\pgfqpoint{1.842613in}{1.583012in}}%
\pgfpathlineto{\pgfqpoint{1.849963in}{1.584120in}}%
\pgfpathlineto{\pgfqpoint{1.851039in}{1.584288in}}%
\pgfpathlineto{\pgfqpoint{1.858268in}{1.585396in}}%
\pgfpathlineto{\pgfqpoint{1.859335in}{1.585563in}}%
\pgfpathlineto{\pgfqpoint{1.866934in}{1.586672in}}%
\pgfpathlineto{\pgfqpoint{1.867958in}{1.586774in}}%
\pgfpathlineto{\pgfqpoint{1.867987in}{1.586774in}}%
\pgfpathlineto{\pgfqpoint{1.876423in}{1.587882in}}%
\pgfpathlineto{\pgfqpoint{1.877523in}{1.588050in}}%
\pgfpathlineto{\pgfqpoint{1.884816in}{1.589158in}}%
\pgfpathlineto{\pgfqpoint{1.885919in}{1.589363in}}%
\pgfpathlineto{\pgfqpoint{1.895839in}{1.590471in}}%
\pgfpathlineto{\pgfqpoint{1.896592in}{1.590555in}}%
\pgfpathlineto{\pgfqpoint{1.896780in}{1.590555in}}%
\pgfpathlineto{\pgfqpoint{1.906890in}{1.591663in}}%
\pgfpathlineto{\pgfqpoint{1.907901in}{1.591803in}}%
\pgfpathlineto{\pgfqpoint{1.917719in}{1.592911in}}%
\pgfpathlineto{\pgfqpoint{1.918765in}{1.593060in}}%
\pgfpathlineto{\pgfqpoint{1.929058in}{1.594168in}}%
\pgfpathlineto{\pgfqpoint{1.930130in}{1.594280in}}%
\pgfpathlineto{\pgfqpoint{1.930158in}{1.594280in}}%
\pgfpathlineto{\pgfqpoint{1.940895in}{1.595388in}}%
\pgfpathlineto{\pgfqpoint{1.941917in}{1.595509in}}%
\pgfpathlineto{\pgfqpoint{1.941974in}{1.595509in}}%
\pgfpathlineto{\pgfqpoint{1.953541in}{1.596617in}}%
\pgfpathlineto{\pgfqpoint{1.954561in}{1.596720in}}%
\pgfpathlineto{\pgfqpoint{1.954627in}{1.596720in}}%
\pgfpathlineto{\pgfqpoint{1.968248in}{1.597828in}}%
\pgfpathlineto{\pgfqpoint{1.969343in}{1.597930in}}%
\pgfpathlineto{\pgfqpoint{1.983413in}{1.599039in}}%
\pgfpathlineto{\pgfqpoint{1.984506in}{1.599150in}}%
\pgfpathlineto{\pgfqpoint{1.999921in}{1.600259in}}%
\pgfpathlineto{\pgfqpoint{2.001005in}{1.600361in}}%
\pgfpathlineto{\pgfqpoint{2.020056in}{1.601469in}}%
\pgfpathlineto{\pgfqpoint{2.021067in}{1.601525in}}%
\pgfpathlineto{\pgfqpoint{2.021109in}{1.601525in}}%
\pgfpathlineto{\pgfqpoint{2.033126in}{1.601944in}}%
\pgfpathlineto{\pgfqpoint{2.033126in}{1.601944in}}%
\pgfusepath{stroke}%
\end{pgfscope}%
\begin{pgfscope}%
\pgfsetrectcap%
\pgfsetmiterjoin%
\pgfsetlinewidth{0.803000pt}%
\definecolor{currentstroke}{rgb}{0.000000,0.000000,0.000000}%
\pgfsetstrokecolor{currentstroke}%
\pgfsetdash{}{0pt}%
\pgfpathmoveto{\pgfqpoint{0.553581in}{0.499444in}}%
\pgfpathlineto{\pgfqpoint{0.553581in}{1.654444in}}%
\pgfusepath{stroke}%
\end{pgfscope}%
\begin{pgfscope}%
\pgfsetrectcap%
\pgfsetmiterjoin%
\pgfsetlinewidth{0.803000pt}%
\definecolor{currentstroke}{rgb}{0.000000,0.000000,0.000000}%
\pgfsetstrokecolor{currentstroke}%
\pgfsetdash{}{0pt}%
\pgfpathmoveto{\pgfqpoint{2.103581in}{0.499444in}}%
\pgfpathlineto{\pgfqpoint{2.103581in}{1.654444in}}%
\pgfusepath{stroke}%
\end{pgfscope}%
\begin{pgfscope}%
\pgfsetrectcap%
\pgfsetmiterjoin%
\pgfsetlinewidth{0.803000pt}%
\definecolor{currentstroke}{rgb}{0.000000,0.000000,0.000000}%
\pgfsetstrokecolor{currentstroke}%
\pgfsetdash{}{0pt}%
\pgfpathmoveto{\pgfqpoint{0.553581in}{0.499444in}}%
\pgfpathlineto{\pgfqpoint{2.103581in}{0.499444in}}%
\pgfusepath{stroke}%
\end{pgfscope}%
\begin{pgfscope}%
\pgfsetrectcap%
\pgfsetmiterjoin%
\pgfsetlinewidth{0.803000pt}%
\definecolor{currentstroke}{rgb}{0.000000,0.000000,0.000000}%
\pgfsetstrokecolor{currentstroke}%
\pgfsetdash{}{0pt}%
\pgfpathmoveto{\pgfqpoint{0.553581in}{1.654444in}}%
\pgfpathlineto{\pgfqpoint{2.103581in}{1.654444in}}%
\pgfusepath{stroke}%
\end{pgfscope}%
\begin{pgfscope}%
\pgfsetbuttcap%
\pgfsetmiterjoin%
\definecolor{currentfill}{rgb}{1.000000,1.000000,1.000000}%
\pgfsetfillcolor{currentfill}%
\pgfsetfillopacity{0.800000}%
\pgfsetlinewidth{1.003750pt}%
\definecolor{currentstroke}{rgb}{0.800000,0.800000,0.800000}%
\pgfsetstrokecolor{currentstroke}%
\pgfsetstrokeopacity{0.800000}%
\pgfsetdash{}{0pt}%
\pgfpathmoveto{\pgfqpoint{0.832747in}{0.568889in}}%
\pgfpathlineto{\pgfqpoint{2.006358in}{0.568889in}}%
\pgfpathquadraticcurveto{\pgfqpoint{2.034136in}{0.568889in}}{\pgfqpoint{2.034136in}{0.596666in}}%
\pgfpathlineto{\pgfqpoint{2.034136in}{0.776388in}}%
\pgfpathquadraticcurveto{\pgfqpoint{2.034136in}{0.804166in}}{\pgfqpoint{2.006358in}{0.804166in}}%
\pgfpathlineto{\pgfqpoint{0.832747in}{0.804166in}}%
\pgfpathquadraticcurveto{\pgfqpoint{0.804970in}{0.804166in}}{\pgfqpoint{0.804970in}{0.776388in}}%
\pgfpathlineto{\pgfqpoint{0.804970in}{0.596666in}}%
\pgfpathquadraticcurveto{\pgfqpoint{0.804970in}{0.568889in}}{\pgfqpoint{0.832747in}{0.568889in}}%
\pgfpathlineto{\pgfqpoint{0.832747in}{0.568889in}}%
\pgfpathclose%
\pgfusepath{stroke,fill}%
\end{pgfscope}%
\begin{pgfscope}%
\pgfsetrectcap%
\pgfsetroundjoin%
\pgfsetlinewidth{1.505625pt}%
\definecolor{currentstroke}{rgb}{0.000000,0.000000,0.000000}%
\pgfsetstrokecolor{currentstroke}%
\pgfsetdash{}{0pt}%
\pgfpathmoveto{\pgfqpoint{0.860525in}{0.700000in}}%
\pgfpathlineto{\pgfqpoint{0.999414in}{0.700000in}}%
\pgfpathlineto{\pgfqpoint{1.138303in}{0.700000in}}%
\pgfusepath{stroke}%
\end{pgfscope}%
\begin{pgfscope}%
\definecolor{textcolor}{rgb}{0.000000,0.000000,0.000000}%
\pgfsetstrokecolor{textcolor}%
\pgfsetfillcolor{textcolor}%
\pgftext[x=1.249414in,y=0.651388in,left,base]{\color{textcolor}\rmfamily\fontsize{10.000000}{12.000000}\selectfont AUC=0.752}%
\end{pgfscope}%
\end{pgfpicture}%
\makeatother%
\endgroup%

\end{tabular}

The distribution has long tails, so we can make a more useful visualization by truncating the ends.  For this graph we mapped the 0.01 quantile to 0 and the 0.99 quantile to 1 leaving the center 98\% of the distribution and truncated the ends.  Our goal in clipping the tails is to make all of the models' results have approximately the same granularity when we choose the decision thresholds that give us the (politically) desired results.  


\

\verb|AdaBoost_5_Fold_Hard_Test_Transformed_98|

%
\noindent\begin{tabular}{@{\hspace{-6pt}}p{4.3in} @{\hspace{-6pt}}p{2.0in}}
	\vskip 0pt
	\hfil Raw Model Output
	
	%% Creator: Matplotlib, PGF backend
%%
%% To include the figure in your LaTeX document, write
%%   \input{<filename>.pgf}
%%
%% Make sure the required packages are loaded in your preamble
%%   \usepackage{pgf}
%%
%% Also ensure that all the required font packages are loaded; for instance,
%% the lmodern package is sometimes necessary when using math font.
%%   \usepackage{lmodern}
%%
%% Figures using additional raster images can only be included by \input if
%% they are in the same directory as the main LaTeX file. For loading figures
%% from other directories you can use the `import` package
%%   \usepackage{import}
%%
%% and then include the figures with
%%   \import{<path to file>}{<filename>.pgf}
%%
%% Matplotlib used the following preamble
%%   
%%   \usepackage{fontspec}
%%   \makeatletter\@ifpackageloaded{underscore}{}{\usepackage[strings]{underscore}}\makeatother
%%
\begingroup%
\makeatletter%
\begin{pgfpicture}%
\pgfpathrectangle{\pgfpointorigin}{\pgfqpoint{4.102500in}{1.775223in}}%
\pgfusepath{use as bounding box, clip}%
\begin{pgfscope}%
\pgfsetbuttcap%
\pgfsetmiterjoin%
\definecolor{currentfill}{rgb}{1.000000,1.000000,1.000000}%
\pgfsetfillcolor{currentfill}%
\pgfsetlinewidth{0.000000pt}%
\definecolor{currentstroke}{rgb}{1.000000,1.000000,1.000000}%
\pgfsetstrokecolor{currentstroke}%
\pgfsetdash{}{0pt}%
\pgfpathmoveto{\pgfqpoint{0.000000in}{0.000000in}}%
\pgfpathlineto{\pgfqpoint{4.102500in}{0.000000in}}%
\pgfpathlineto{\pgfqpoint{4.102500in}{1.775223in}}%
\pgfpathlineto{\pgfqpoint{0.000000in}{1.775223in}}%
\pgfpathlineto{\pgfqpoint{0.000000in}{0.000000in}}%
\pgfpathclose%
\pgfusepath{fill}%
\end{pgfscope}%
\begin{pgfscope}%
\pgfsetbuttcap%
\pgfsetmiterjoin%
\definecolor{currentfill}{rgb}{1.000000,1.000000,1.000000}%
\pgfsetfillcolor{currentfill}%
\pgfsetlinewidth{0.000000pt}%
\definecolor{currentstroke}{rgb}{0.000000,0.000000,0.000000}%
\pgfsetstrokecolor{currentstroke}%
\pgfsetstrokeopacity{0.000000}%
\pgfsetdash{}{0pt}%
\pgfpathmoveto{\pgfqpoint{0.515000in}{0.499444in}}%
\pgfpathlineto{\pgfqpoint{4.002500in}{0.499444in}}%
\pgfpathlineto{\pgfqpoint{4.002500in}{1.654444in}}%
\pgfpathlineto{\pgfqpoint{0.515000in}{1.654444in}}%
\pgfpathlineto{\pgfqpoint{0.515000in}{0.499444in}}%
\pgfpathclose%
\pgfusepath{fill}%
\end{pgfscope}%
\begin{pgfscope}%
\pgfpathrectangle{\pgfqpoint{0.515000in}{0.499444in}}{\pgfqpoint{3.487500in}{1.155000in}}%
\pgfusepath{clip}%
\pgfsetbuttcap%
\pgfsetmiterjoin%
\pgfsetlinewidth{1.003750pt}%
\definecolor{currentstroke}{rgb}{0.000000,0.000000,0.000000}%
\pgfsetstrokecolor{currentstroke}%
\pgfsetdash{}{0pt}%
\pgfpathmoveto{\pgfqpoint{0.610114in}{0.499444in}}%
\pgfpathlineto{\pgfqpoint{0.673523in}{0.499444in}}%
\pgfpathlineto{\pgfqpoint{0.673523in}{0.594399in}}%
\pgfpathlineto{\pgfqpoint{0.610114in}{0.594399in}}%
\pgfpathlineto{\pgfqpoint{0.610114in}{0.499444in}}%
\pgfpathclose%
\pgfusepath{stroke}%
\end{pgfscope}%
\begin{pgfscope}%
\pgfpathrectangle{\pgfqpoint{0.515000in}{0.499444in}}{\pgfqpoint{3.487500in}{1.155000in}}%
\pgfusepath{clip}%
\pgfsetbuttcap%
\pgfsetmiterjoin%
\pgfsetlinewidth{1.003750pt}%
\definecolor{currentstroke}{rgb}{0.000000,0.000000,0.000000}%
\pgfsetstrokecolor{currentstroke}%
\pgfsetdash{}{0pt}%
\pgfpathmoveto{\pgfqpoint{0.768637in}{0.499444in}}%
\pgfpathlineto{\pgfqpoint{0.832046in}{0.499444in}}%
\pgfpathlineto{\pgfqpoint{0.832046in}{0.632150in}}%
\pgfpathlineto{\pgfqpoint{0.768637in}{0.632150in}}%
\pgfpathlineto{\pgfqpoint{0.768637in}{0.499444in}}%
\pgfpathclose%
\pgfusepath{stroke}%
\end{pgfscope}%
\begin{pgfscope}%
\pgfpathrectangle{\pgfqpoint{0.515000in}{0.499444in}}{\pgfqpoint{3.487500in}{1.155000in}}%
\pgfusepath{clip}%
\pgfsetbuttcap%
\pgfsetmiterjoin%
\pgfsetlinewidth{1.003750pt}%
\definecolor{currentstroke}{rgb}{0.000000,0.000000,0.000000}%
\pgfsetstrokecolor{currentstroke}%
\pgfsetdash{}{0pt}%
\pgfpathmoveto{\pgfqpoint{0.927159in}{0.499444in}}%
\pgfpathlineto{\pgfqpoint{0.990568in}{0.499444in}}%
\pgfpathlineto{\pgfqpoint{0.990568in}{0.745925in}}%
\pgfpathlineto{\pgfqpoint{0.927159in}{0.745925in}}%
\pgfpathlineto{\pgfqpoint{0.927159in}{0.499444in}}%
\pgfpathclose%
\pgfusepath{stroke}%
\end{pgfscope}%
\begin{pgfscope}%
\pgfpathrectangle{\pgfqpoint{0.515000in}{0.499444in}}{\pgfqpoint{3.487500in}{1.155000in}}%
\pgfusepath{clip}%
\pgfsetbuttcap%
\pgfsetmiterjoin%
\pgfsetlinewidth{1.003750pt}%
\definecolor{currentstroke}{rgb}{0.000000,0.000000,0.000000}%
\pgfsetstrokecolor{currentstroke}%
\pgfsetdash{}{0pt}%
\pgfpathmoveto{\pgfqpoint{1.085682in}{0.499444in}}%
\pgfpathlineto{\pgfqpoint{1.149091in}{0.499444in}}%
\pgfpathlineto{\pgfqpoint{1.149091in}{0.905179in}}%
\pgfpathlineto{\pgfqpoint{1.085682in}{0.905179in}}%
\pgfpathlineto{\pgfqpoint{1.085682in}{0.499444in}}%
\pgfpathclose%
\pgfusepath{stroke}%
\end{pgfscope}%
\begin{pgfscope}%
\pgfpathrectangle{\pgfqpoint{0.515000in}{0.499444in}}{\pgfqpoint{3.487500in}{1.155000in}}%
\pgfusepath{clip}%
\pgfsetbuttcap%
\pgfsetmiterjoin%
\pgfsetlinewidth{1.003750pt}%
\definecolor{currentstroke}{rgb}{0.000000,0.000000,0.000000}%
\pgfsetstrokecolor{currentstroke}%
\pgfsetdash{}{0pt}%
\pgfpathmoveto{\pgfqpoint{1.244205in}{0.499444in}}%
\pgfpathlineto{\pgfqpoint{1.307614in}{0.499444in}}%
\pgfpathlineto{\pgfqpoint{1.307614in}{1.092055in}}%
\pgfpathlineto{\pgfqpoint{1.244205in}{1.092055in}}%
\pgfpathlineto{\pgfqpoint{1.244205in}{0.499444in}}%
\pgfpathclose%
\pgfusepath{stroke}%
\end{pgfscope}%
\begin{pgfscope}%
\pgfpathrectangle{\pgfqpoint{0.515000in}{0.499444in}}{\pgfqpoint{3.487500in}{1.155000in}}%
\pgfusepath{clip}%
\pgfsetbuttcap%
\pgfsetmiterjoin%
\pgfsetlinewidth{1.003750pt}%
\definecolor{currentstroke}{rgb}{0.000000,0.000000,0.000000}%
\pgfsetstrokecolor{currentstroke}%
\pgfsetdash{}{0pt}%
\pgfpathmoveto{\pgfqpoint{1.402728in}{0.499444in}}%
\pgfpathlineto{\pgfqpoint{1.466137in}{0.499444in}}%
\pgfpathlineto{\pgfqpoint{1.466137in}{1.276465in}}%
\pgfpathlineto{\pgfqpoint{1.402728in}{1.276465in}}%
\pgfpathlineto{\pgfqpoint{1.402728in}{0.499444in}}%
\pgfpathclose%
\pgfusepath{stroke}%
\end{pgfscope}%
\begin{pgfscope}%
\pgfpathrectangle{\pgfqpoint{0.515000in}{0.499444in}}{\pgfqpoint{3.487500in}{1.155000in}}%
\pgfusepath{clip}%
\pgfsetbuttcap%
\pgfsetmiterjoin%
\pgfsetlinewidth{1.003750pt}%
\definecolor{currentstroke}{rgb}{0.000000,0.000000,0.000000}%
\pgfsetstrokecolor{currentstroke}%
\pgfsetdash{}{0pt}%
\pgfpathmoveto{\pgfqpoint{1.561250in}{0.499444in}}%
\pgfpathlineto{\pgfqpoint{1.624659in}{0.499444in}}%
\pgfpathlineto{\pgfqpoint{1.624659in}{1.456388in}}%
\pgfpathlineto{\pgfqpoint{1.561250in}{1.456388in}}%
\pgfpathlineto{\pgfqpoint{1.561250in}{0.499444in}}%
\pgfpathclose%
\pgfusepath{stroke}%
\end{pgfscope}%
\begin{pgfscope}%
\pgfpathrectangle{\pgfqpoint{0.515000in}{0.499444in}}{\pgfqpoint{3.487500in}{1.155000in}}%
\pgfusepath{clip}%
\pgfsetbuttcap%
\pgfsetmiterjoin%
\pgfsetlinewidth{1.003750pt}%
\definecolor{currentstroke}{rgb}{0.000000,0.000000,0.000000}%
\pgfsetstrokecolor{currentstroke}%
\pgfsetdash{}{0pt}%
\pgfpathmoveto{\pgfqpoint{1.719773in}{0.499444in}}%
\pgfpathlineto{\pgfqpoint{1.783182in}{0.499444in}}%
\pgfpathlineto{\pgfqpoint{1.783182in}{1.581493in}}%
\pgfpathlineto{\pgfqpoint{1.719773in}{1.581493in}}%
\pgfpathlineto{\pgfqpoint{1.719773in}{0.499444in}}%
\pgfpathclose%
\pgfusepath{stroke}%
\end{pgfscope}%
\begin{pgfscope}%
\pgfpathrectangle{\pgfqpoint{0.515000in}{0.499444in}}{\pgfqpoint{3.487500in}{1.155000in}}%
\pgfusepath{clip}%
\pgfsetbuttcap%
\pgfsetmiterjoin%
\pgfsetlinewidth{1.003750pt}%
\definecolor{currentstroke}{rgb}{0.000000,0.000000,0.000000}%
\pgfsetstrokecolor{currentstroke}%
\pgfsetdash{}{0pt}%
\pgfpathmoveto{\pgfqpoint{1.878296in}{0.499444in}}%
\pgfpathlineto{\pgfqpoint{1.941705in}{0.499444in}}%
\pgfpathlineto{\pgfqpoint{1.941705in}{1.599444in}}%
\pgfpathlineto{\pgfqpoint{1.878296in}{1.599444in}}%
\pgfpathlineto{\pgfqpoint{1.878296in}{0.499444in}}%
\pgfpathclose%
\pgfusepath{stroke}%
\end{pgfscope}%
\begin{pgfscope}%
\pgfpathrectangle{\pgfqpoint{0.515000in}{0.499444in}}{\pgfqpoint{3.487500in}{1.155000in}}%
\pgfusepath{clip}%
\pgfsetbuttcap%
\pgfsetmiterjoin%
\pgfsetlinewidth{1.003750pt}%
\definecolor{currentstroke}{rgb}{0.000000,0.000000,0.000000}%
\pgfsetstrokecolor{currentstroke}%
\pgfsetdash{}{0pt}%
\pgfpathmoveto{\pgfqpoint{2.036818in}{0.499444in}}%
\pgfpathlineto{\pgfqpoint{2.100228in}{0.499444in}}%
\pgfpathlineto{\pgfqpoint{2.100228in}{1.524416in}}%
\pgfpathlineto{\pgfqpoint{2.036818in}{1.524416in}}%
\pgfpathlineto{\pgfqpoint{2.036818in}{0.499444in}}%
\pgfpathclose%
\pgfusepath{stroke}%
\end{pgfscope}%
\begin{pgfscope}%
\pgfpathrectangle{\pgfqpoint{0.515000in}{0.499444in}}{\pgfqpoint{3.487500in}{1.155000in}}%
\pgfusepath{clip}%
\pgfsetbuttcap%
\pgfsetmiterjoin%
\pgfsetlinewidth{1.003750pt}%
\definecolor{currentstroke}{rgb}{0.000000,0.000000,0.000000}%
\pgfsetstrokecolor{currentstroke}%
\pgfsetdash{}{0pt}%
\pgfpathmoveto{\pgfqpoint{2.195341in}{0.499444in}}%
\pgfpathlineto{\pgfqpoint{2.258750in}{0.499444in}}%
\pgfpathlineto{\pgfqpoint{2.258750in}{1.385879in}}%
\pgfpathlineto{\pgfqpoint{2.195341in}{1.385879in}}%
\pgfpathlineto{\pgfqpoint{2.195341in}{0.499444in}}%
\pgfpathclose%
\pgfusepath{stroke}%
\end{pgfscope}%
\begin{pgfscope}%
\pgfpathrectangle{\pgfqpoint{0.515000in}{0.499444in}}{\pgfqpoint{3.487500in}{1.155000in}}%
\pgfusepath{clip}%
\pgfsetbuttcap%
\pgfsetmiterjoin%
\pgfsetlinewidth{1.003750pt}%
\definecolor{currentstroke}{rgb}{0.000000,0.000000,0.000000}%
\pgfsetstrokecolor{currentstroke}%
\pgfsetdash{}{0pt}%
\pgfpathmoveto{\pgfqpoint{2.353864in}{0.499444in}}%
\pgfpathlineto{\pgfqpoint{2.417273in}{0.499444in}}%
\pgfpathlineto{\pgfqpoint{2.417273in}{1.207900in}}%
\pgfpathlineto{\pgfqpoint{2.353864in}{1.207900in}}%
\pgfpathlineto{\pgfqpoint{2.353864in}{0.499444in}}%
\pgfpathclose%
\pgfusepath{stroke}%
\end{pgfscope}%
\begin{pgfscope}%
\pgfpathrectangle{\pgfqpoint{0.515000in}{0.499444in}}{\pgfqpoint{3.487500in}{1.155000in}}%
\pgfusepath{clip}%
\pgfsetbuttcap%
\pgfsetmiterjoin%
\pgfsetlinewidth{1.003750pt}%
\definecolor{currentstroke}{rgb}{0.000000,0.000000,0.000000}%
\pgfsetstrokecolor{currentstroke}%
\pgfsetdash{}{0pt}%
\pgfpathmoveto{\pgfqpoint{2.512387in}{0.499444in}}%
\pgfpathlineto{\pgfqpoint{2.575796in}{0.499444in}}%
\pgfpathlineto{\pgfqpoint{2.575796in}{1.023189in}}%
\pgfpathlineto{\pgfqpoint{2.512387in}{1.023189in}}%
\pgfpathlineto{\pgfqpoint{2.512387in}{0.499444in}}%
\pgfpathclose%
\pgfusepath{stroke}%
\end{pgfscope}%
\begin{pgfscope}%
\pgfpathrectangle{\pgfqpoint{0.515000in}{0.499444in}}{\pgfqpoint{3.487500in}{1.155000in}}%
\pgfusepath{clip}%
\pgfsetbuttcap%
\pgfsetmiterjoin%
\pgfsetlinewidth{1.003750pt}%
\definecolor{currentstroke}{rgb}{0.000000,0.000000,0.000000}%
\pgfsetstrokecolor{currentstroke}%
\pgfsetdash{}{0pt}%
\pgfpathmoveto{\pgfqpoint{2.670909in}{0.499444in}}%
\pgfpathlineto{\pgfqpoint{2.734318in}{0.499444in}}%
\pgfpathlineto{\pgfqpoint{2.734318in}{0.860222in}}%
\pgfpathlineto{\pgfqpoint{2.670909in}{0.860222in}}%
\pgfpathlineto{\pgfqpoint{2.670909in}{0.499444in}}%
\pgfpathclose%
\pgfusepath{stroke}%
\end{pgfscope}%
\begin{pgfscope}%
\pgfpathrectangle{\pgfqpoint{0.515000in}{0.499444in}}{\pgfqpoint{3.487500in}{1.155000in}}%
\pgfusepath{clip}%
\pgfsetbuttcap%
\pgfsetmiterjoin%
\pgfsetlinewidth{1.003750pt}%
\definecolor{currentstroke}{rgb}{0.000000,0.000000,0.000000}%
\pgfsetstrokecolor{currentstroke}%
\pgfsetdash{}{0pt}%
\pgfpathmoveto{\pgfqpoint{2.829432in}{0.499444in}}%
\pgfpathlineto{\pgfqpoint{2.892841in}{0.499444in}}%
\pgfpathlineto{\pgfqpoint{2.892841in}{0.738593in}}%
\pgfpathlineto{\pgfqpoint{2.829432in}{0.738593in}}%
\pgfpathlineto{\pgfqpoint{2.829432in}{0.499444in}}%
\pgfpathclose%
\pgfusepath{stroke}%
\end{pgfscope}%
\begin{pgfscope}%
\pgfpathrectangle{\pgfqpoint{0.515000in}{0.499444in}}{\pgfqpoint{3.487500in}{1.155000in}}%
\pgfusepath{clip}%
\pgfsetbuttcap%
\pgfsetmiterjoin%
\pgfsetlinewidth{1.003750pt}%
\definecolor{currentstroke}{rgb}{0.000000,0.000000,0.000000}%
\pgfsetstrokecolor{currentstroke}%
\pgfsetdash{}{0pt}%
\pgfpathmoveto{\pgfqpoint{2.987955in}{0.499444in}}%
\pgfpathlineto{\pgfqpoint{3.051364in}{0.499444in}}%
\pgfpathlineto{\pgfqpoint{3.051364in}{0.651982in}}%
\pgfpathlineto{\pgfqpoint{2.987955in}{0.651982in}}%
\pgfpathlineto{\pgfqpoint{2.987955in}{0.499444in}}%
\pgfpathclose%
\pgfusepath{stroke}%
\end{pgfscope}%
\begin{pgfscope}%
\pgfpathrectangle{\pgfqpoint{0.515000in}{0.499444in}}{\pgfqpoint{3.487500in}{1.155000in}}%
\pgfusepath{clip}%
\pgfsetbuttcap%
\pgfsetmiterjoin%
\pgfsetlinewidth{1.003750pt}%
\definecolor{currentstroke}{rgb}{0.000000,0.000000,0.000000}%
\pgfsetstrokecolor{currentstroke}%
\pgfsetdash{}{0pt}%
\pgfpathmoveto{\pgfqpoint{3.146478in}{0.499444in}}%
\pgfpathlineto{\pgfqpoint{3.209887in}{0.499444in}}%
\pgfpathlineto{\pgfqpoint{3.209887in}{0.591302in}}%
\pgfpathlineto{\pgfqpoint{3.146478in}{0.591302in}}%
\pgfpathlineto{\pgfqpoint{3.146478in}{0.499444in}}%
\pgfpathclose%
\pgfusepath{stroke}%
\end{pgfscope}%
\begin{pgfscope}%
\pgfpathrectangle{\pgfqpoint{0.515000in}{0.499444in}}{\pgfqpoint{3.487500in}{1.155000in}}%
\pgfusepath{clip}%
\pgfsetbuttcap%
\pgfsetmiterjoin%
\pgfsetlinewidth{1.003750pt}%
\definecolor{currentstroke}{rgb}{0.000000,0.000000,0.000000}%
\pgfsetstrokecolor{currentstroke}%
\pgfsetdash{}{0pt}%
\pgfpathmoveto{\pgfqpoint{3.305000in}{0.499444in}}%
\pgfpathlineto{\pgfqpoint{3.368409in}{0.499444in}}%
\pgfpathlineto{\pgfqpoint{3.368409in}{0.553772in}}%
\pgfpathlineto{\pgfqpoint{3.305000in}{0.553772in}}%
\pgfpathlineto{\pgfqpoint{3.305000in}{0.499444in}}%
\pgfpathclose%
\pgfusepath{stroke}%
\end{pgfscope}%
\begin{pgfscope}%
\pgfpathrectangle{\pgfqpoint{0.515000in}{0.499444in}}{\pgfqpoint{3.487500in}{1.155000in}}%
\pgfusepath{clip}%
\pgfsetbuttcap%
\pgfsetmiterjoin%
\pgfsetlinewidth{1.003750pt}%
\definecolor{currentstroke}{rgb}{0.000000,0.000000,0.000000}%
\pgfsetstrokecolor{currentstroke}%
\pgfsetdash{}{0pt}%
\pgfpathmoveto{\pgfqpoint{3.463523in}{0.499444in}}%
\pgfpathlineto{\pgfqpoint{3.526932in}{0.499444in}}%
\pgfpathlineto{\pgfqpoint{3.526932in}{0.530290in}}%
\pgfpathlineto{\pgfqpoint{3.463523in}{0.530290in}}%
\pgfpathlineto{\pgfqpoint{3.463523in}{0.499444in}}%
\pgfpathclose%
\pgfusepath{stroke}%
\end{pgfscope}%
\begin{pgfscope}%
\pgfpathrectangle{\pgfqpoint{0.515000in}{0.499444in}}{\pgfqpoint{3.487500in}{1.155000in}}%
\pgfusepath{clip}%
\pgfsetbuttcap%
\pgfsetmiterjoin%
\pgfsetlinewidth{1.003750pt}%
\definecolor{currentstroke}{rgb}{0.000000,0.000000,0.000000}%
\pgfsetstrokecolor{currentstroke}%
\pgfsetdash{}{0pt}%
\pgfpathmoveto{\pgfqpoint{3.622046in}{0.499444in}}%
\pgfpathlineto{\pgfqpoint{3.685455in}{0.499444in}}%
\pgfpathlineto{\pgfqpoint{3.685455in}{0.515705in}}%
\pgfpathlineto{\pgfqpoint{3.622046in}{0.515705in}}%
\pgfpathlineto{\pgfqpoint{3.622046in}{0.499444in}}%
\pgfpathclose%
\pgfusepath{stroke}%
\end{pgfscope}%
\begin{pgfscope}%
\pgfpathrectangle{\pgfqpoint{0.515000in}{0.499444in}}{\pgfqpoint{3.487500in}{1.155000in}}%
\pgfusepath{clip}%
\pgfsetbuttcap%
\pgfsetmiterjoin%
\pgfsetlinewidth{1.003750pt}%
\definecolor{currentstroke}{rgb}{0.000000,0.000000,0.000000}%
\pgfsetstrokecolor{currentstroke}%
\pgfsetdash{}{0pt}%
\pgfpathmoveto{\pgfqpoint{3.780568in}{0.499444in}}%
\pgfpathlineto{\pgfqpoint{3.843978in}{0.499444in}}%
\pgfpathlineto{\pgfqpoint{3.843978in}{0.515720in}}%
\pgfpathlineto{\pgfqpoint{3.780568in}{0.515720in}}%
\pgfpathlineto{\pgfqpoint{3.780568in}{0.499444in}}%
\pgfpathclose%
\pgfusepath{stroke}%
\end{pgfscope}%
\begin{pgfscope}%
\pgfpathrectangle{\pgfqpoint{0.515000in}{0.499444in}}{\pgfqpoint{3.487500in}{1.155000in}}%
\pgfusepath{clip}%
\pgfsetbuttcap%
\pgfsetmiterjoin%
\definecolor{currentfill}{rgb}{0.000000,0.000000,0.000000}%
\pgfsetfillcolor{currentfill}%
\pgfsetlinewidth{0.000000pt}%
\definecolor{currentstroke}{rgb}{0.000000,0.000000,0.000000}%
\pgfsetstrokecolor{currentstroke}%
\pgfsetstrokeopacity{0.000000}%
\pgfsetdash{}{0pt}%
\pgfpathmoveto{\pgfqpoint{0.673523in}{0.499444in}}%
\pgfpathlineto{\pgfqpoint{0.736932in}{0.499444in}}%
\pgfpathlineto{\pgfqpoint{0.736932in}{0.500329in}}%
\pgfpathlineto{\pgfqpoint{0.673523in}{0.500329in}}%
\pgfpathlineto{\pgfqpoint{0.673523in}{0.499444in}}%
\pgfpathclose%
\pgfusepath{fill}%
\end{pgfscope}%
\begin{pgfscope}%
\pgfpathrectangle{\pgfqpoint{0.515000in}{0.499444in}}{\pgfqpoint{3.487500in}{1.155000in}}%
\pgfusepath{clip}%
\pgfsetbuttcap%
\pgfsetmiterjoin%
\definecolor{currentfill}{rgb}{0.000000,0.000000,0.000000}%
\pgfsetfillcolor{currentfill}%
\pgfsetlinewidth{0.000000pt}%
\definecolor{currentstroke}{rgb}{0.000000,0.000000,0.000000}%
\pgfsetstrokecolor{currentstroke}%
\pgfsetstrokeopacity{0.000000}%
\pgfsetdash{}{0pt}%
\pgfpathmoveto{\pgfqpoint{0.832046in}{0.499444in}}%
\pgfpathlineto{\pgfqpoint{0.895455in}{0.499444in}}%
\pgfpathlineto{\pgfqpoint{0.895455in}{0.501451in}}%
\pgfpathlineto{\pgfqpoint{0.832046in}{0.501451in}}%
\pgfpathlineto{\pgfqpoint{0.832046in}{0.499444in}}%
\pgfpathclose%
\pgfusepath{fill}%
\end{pgfscope}%
\begin{pgfscope}%
\pgfpathrectangle{\pgfqpoint{0.515000in}{0.499444in}}{\pgfqpoint{3.487500in}{1.155000in}}%
\pgfusepath{clip}%
\pgfsetbuttcap%
\pgfsetmiterjoin%
\definecolor{currentfill}{rgb}{0.000000,0.000000,0.000000}%
\pgfsetfillcolor{currentfill}%
\pgfsetlinewidth{0.000000pt}%
\definecolor{currentstroke}{rgb}{0.000000,0.000000,0.000000}%
\pgfsetstrokecolor{currentstroke}%
\pgfsetstrokeopacity{0.000000}%
\pgfsetdash{}{0pt}%
\pgfpathmoveto{\pgfqpoint{0.990568in}{0.499444in}}%
\pgfpathlineto{\pgfqpoint{1.053978in}{0.499444in}}%
\pgfpathlineto{\pgfqpoint{1.053978in}{0.504422in}}%
\pgfpathlineto{\pgfqpoint{0.990568in}{0.504422in}}%
\pgfpathlineto{\pgfqpoint{0.990568in}{0.499444in}}%
\pgfpathclose%
\pgfusepath{fill}%
\end{pgfscope}%
\begin{pgfscope}%
\pgfpathrectangle{\pgfqpoint{0.515000in}{0.499444in}}{\pgfqpoint{3.487500in}{1.155000in}}%
\pgfusepath{clip}%
\pgfsetbuttcap%
\pgfsetmiterjoin%
\definecolor{currentfill}{rgb}{0.000000,0.000000,0.000000}%
\pgfsetfillcolor{currentfill}%
\pgfsetlinewidth{0.000000pt}%
\definecolor{currentstroke}{rgb}{0.000000,0.000000,0.000000}%
\pgfsetstrokecolor{currentstroke}%
\pgfsetstrokeopacity{0.000000}%
\pgfsetdash{}{0pt}%
\pgfpathmoveto{\pgfqpoint{1.149091in}{0.499444in}}%
\pgfpathlineto{\pgfqpoint{1.212500in}{0.499444in}}%
\pgfpathlineto{\pgfqpoint{1.212500in}{0.509810in}}%
\pgfpathlineto{\pgfqpoint{1.149091in}{0.509810in}}%
\pgfpathlineto{\pgfqpoint{1.149091in}{0.499444in}}%
\pgfpathclose%
\pgfusepath{fill}%
\end{pgfscope}%
\begin{pgfscope}%
\pgfpathrectangle{\pgfqpoint{0.515000in}{0.499444in}}{\pgfqpoint{3.487500in}{1.155000in}}%
\pgfusepath{clip}%
\pgfsetbuttcap%
\pgfsetmiterjoin%
\definecolor{currentfill}{rgb}{0.000000,0.000000,0.000000}%
\pgfsetfillcolor{currentfill}%
\pgfsetlinewidth{0.000000pt}%
\definecolor{currentstroke}{rgb}{0.000000,0.000000,0.000000}%
\pgfsetstrokecolor{currentstroke}%
\pgfsetstrokeopacity{0.000000}%
\pgfsetdash{}{0pt}%
\pgfpathmoveto{\pgfqpoint{1.307614in}{0.499444in}}%
\pgfpathlineto{\pgfqpoint{1.371023in}{0.499444in}}%
\pgfpathlineto{\pgfqpoint{1.371023in}{0.521109in}}%
\pgfpathlineto{\pgfqpoint{1.307614in}{0.521109in}}%
\pgfpathlineto{\pgfqpoint{1.307614in}{0.499444in}}%
\pgfpathclose%
\pgfusepath{fill}%
\end{pgfscope}%
\begin{pgfscope}%
\pgfpathrectangle{\pgfqpoint{0.515000in}{0.499444in}}{\pgfqpoint{3.487500in}{1.155000in}}%
\pgfusepath{clip}%
\pgfsetbuttcap%
\pgfsetmiterjoin%
\definecolor{currentfill}{rgb}{0.000000,0.000000,0.000000}%
\pgfsetfillcolor{currentfill}%
\pgfsetlinewidth{0.000000pt}%
\definecolor{currentstroke}{rgb}{0.000000,0.000000,0.000000}%
\pgfsetstrokecolor{currentstroke}%
\pgfsetstrokeopacity{0.000000}%
\pgfsetdash{}{0pt}%
\pgfpathmoveto{\pgfqpoint{1.466137in}{0.499444in}}%
\pgfpathlineto{\pgfqpoint{1.529546in}{0.499444in}}%
\pgfpathlineto{\pgfqpoint{1.529546in}{0.537148in}}%
\pgfpathlineto{\pgfqpoint{1.466137in}{0.537148in}}%
\pgfpathlineto{\pgfqpoint{1.466137in}{0.499444in}}%
\pgfpathclose%
\pgfusepath{fill}%
\end{pgfscope}%
\begin{pgfscope}%
\pgfpathrectangle{\pgfqpoint{0.515000in}{0.499444in}}{\pgfqpoint{3.487500in}{1.155000in}}%
\pgfusepath{clip}%
\pgfsetbuttcap%
\pgfsetmiterjoin%
\definecolor{currentfill}{rgb}{0.000000,0.000000,0.000000}%
\pgfsetfillcolor{currentfill}%
\pgfsetlinewidth{0.000000pt}%
\definecolor{currentstroke}{rgb}{0.000000,0.000000,0.000000}%
\pgfsetstrokecolor{currentstroke}%
\pgfsetstrokeopacity{0.000000}%
\pgfsetdash{}{0pt}%
\pgfpathmoveto{\pgfqpoint{1.624659in}{0.499444in}}%
\pgfpathlineto{\pgfqpoint{1.688068in}{0.499444in}}%
\pgfpathlineto{\pgfqpoint{1.688068in}{0.563190in}}%
\pgfpathlineto{\pgfqpoint{1.624659in}{0.563190in}}%
\pgfpathlineto{\pgfqpoint{1.624659in}{0.499444in}}%
\pgfpathclose%
\pgfusepath{fill}%
\end{pgfscope}%
\begin{pgfscope}%
\pgfpathrectangle{\pgfqpoint{0.515000in}{0.499444in}}{\pgfqpoint{3.487500in}{1.155000in}}%
\pgfusepath{clip}%
\pgfsetbuttcap%
\pgfsetmiterjoin%
\definecolor{currentfill}{rgb}{0.000000,0.000000,0.000000}%
\pgfsetfillcolor{currentfill}%
\pgfsetlinewidth{0.000000pt}%
\definecolor{currentstroke}{rgb}{0.000000,0.000000,0.000000}%
\pgfsetstrokecolor{currentstroke}%
\pgfsetstrokeopacity{0.000000}%
\pgfsetdash{}{0pt}%
\pgfpathmoveto{\pgfqpoint{1.783182in}{0.499444in}}%
\pgfpathlineto{\pgfqpoint{1.846591in}{0.499444in}}%
\pgfpathlineto{\pgfqpoint{1.846591in}{0.597006in}}%
\pgfpathlineto{\pgfqpoint{1.783182in}{0.597006in}}%
\pgfpathlineto{\pgfqpoint{1.783182in}{0.499444in}}%
\pgfpathclose%
\pgfusepath{fill}%
\end{pgfscope}%
\begin{pgfscope}%
\pgfpathrectangle{\pgfqpoint{0.515000in}{0.499444in}}{\pgfqpoint{3.487500in}{1.155000in}}%
\pgfusepath{clip}%
\pgfsetbuttcap%
\pgfsetmiterjoin%
\definecolor{currentfill}{rgb}{0.000000,0.000000,0.000000}%
\pgfsetfillcolor{currentfill}%
\pgfsetlinewidth{0.000000pt}%
\definecolor{currentstroke}{rgb}{0.000000,0.000000,0.000000}%
\pgfsetstrokecolor{currentstroke}%
\pgfsetstrokeopacity{0.000000}%
\pgfsetdash{}{0pt}%
\pgfpathmoveto{\pgfqpoint{1.941705in}{0.499444in}}%
\pgfpathlineto{\pgfqpoint{2.005114in}{0.499444in}}%
\pgfpathlineto{\pgfqpoint{2.005114in}{0.636164in}}%
\pgfpathlineto{\pgfqpoint{1.941705in}{0.636164in}}%
\pgfpathlineto{\pgfqpoint{1.941705in}{0.499444in}}%
\pgfpathclose%
\pgfusepath{fill}%
\end{pgfscope}%
\begin{pgfscope}%
\pgfpathrectangle{\pgfqpoint{0.515000in}{0.499444in}}{\pgfqpoint{3.487500in}{1.155000in}}%
\pgfusepath{clip}%
\pgfsetbuttcap%
\pgfsetmiterjoin%
\definecolor{currentfill}{rgb}{0.000000,0.000000,0.000000}%
\pgfsetfillcolor{currentfill}%
\pgfsetlinewidth{0.000000pt}%
\definecolor{currentstroke}{rgb}{0.000000,0.000000,0.000000}%
\pgfsetstrokecolor{currentstroke}%
\pgfsetstrokeopacity{0.000000}%
\pgfsetdash{}{0pt}%
\pgfpathmoveto{\pgfqpoint{2.100228in}{0.499444in}}%
\pgfpathlineto{\pgfqpoint{2.163637in}{0.499444in}}%
\pgfpathlineto{\pgfqpoint{2.163637in}{0.666678in}}%
\pgfpathlineto{\pgfqpoint{2.100228in}{0.666678in}}%
\pgfpathlineto{\pgfqpoint{2.100228in}{0.499444in}}%
\pgfpathclose%
\pgfusepath{fill}%
\end{pgfscope}%
\begin{pgfscope}%
\pgfpathrectangle{\pgfqpoint{0.515000in}{0.499444in}}{\pgfqpoint{3.487500in}{1.155000in}}%
\pgfusepath{clip}%
\pgfsetbuttcap%
\pgfsetmiterjoin%
\definecolor{currentfill}{rgb}{0.000000,0.000000,0.000000}%
\pgfsetfillcolor{currentfill}%
\pgfsetlinewidth{0.000000pt}%
\definecolor{currentstroke}{rgb}{0.000000,0.000000,0.000000}%
\pgfsetstrokecolor{currentstroke}%
\pgfsetstrokeopacity{0.000000}%
\pgfsetdash{}{0pt}%
\pgfpathmoveto{\pgfqpoint{2.258750in}{0.499444in}}%
\pgfpathlineto{\pgfqpoint{2.322159in}{0.499444in}}%
\pgfpathlineto{\pgfqpoint{2.322159in}{0.690017in}}%
\pgfpathlineto{\pgfqpoint{2.258750in}{0.690017in}}%
\pgfpathlineto{\pgfqpoint{2.258750in}{0.499444in}}%
\pgfpathclose%
\pgfusepath{fill}%
\end{pgfscope}%
\begin{pgfscope}%
\pgfpathrectangle{\pgfqpoint{0.515000in}{0.499444in}}{\pgfqpoint{3.487500in}{1.155000in}}%
\pgfusepath{clip}%
\pgfsetbuttcap%
\pgfsetmiterjoin%
\definecolor{currentfill}{rgb}{0.000000,0.000000,0.000000}%
\pgfsetfillcolor{currentfill}%
\pgfsetlinewidth{0.000000pt}%
\definecolor{currentstroke}{rgb}{0.000000,0.000000,0.000000}%
\pgfsetstrokecolor{currentstroke}%
\pgfsetstrokeopacity{0.000000}%
\pgfsetdash{}{0pt}%
\pgfpathmoveto{\pgfqpoint{2.417273in}{0.499444in}}%
\pgfpathlineto{\pgfqpoint{2.480682in}{0.499444in}}%
\pgfpathlineto{\pgfqpoint{2.480682in}{0.699104in}}%
\pgfpathlineto{\pgfqpoint{2.417273in}{0.699104in}}%
\pgfpathlineto{\pgfqpoint{2.417273in}{0.499444in}}%
\pgfpathclose%
\pgfusepath{fill}%
\end{pgfscope}%
\begin{pgfscope}%
\pgfpathrectangle{\pgfqpoint{0.515000in}{0.499444in}}{\pgfqpoint{3.487500in}{1.155000in}}%
\pgfusepath{clip}%
\pgfsetbuttcap%
\pgfsetmiterjoin%
\definecolor{currentfill}{rgb}{0.000000,0.000000,0.000000}%
\pgfsetfillcolor{currentfill}%
\pgfsetlinewidth{0.000000pt}%
\definecolor{currentstroke}{rgb}{0.000000,0.000000,0.000000}%
\pgfsetstrokecolor{currentstroke}%
\pgfsetstrokeopacity{0.000000}%
\pgfsetdash{}{0pt}%
\pgfpathmoveto{\pgfqpoint{2.575796in}{0.499444in}}%
\pgfpathlineto{\pgfqpoint{2.639205in}{0.499444in}}%
\pgfpathlineto{\pgfqpoint{2.639205in}{0.687616in}}%
\pgfpathlineto{\pgfqpoint{2.575796in}{0.687616in}}%
\pgfpathlineto{\pgfqpoint{2.575796in}{0.499444in}}%
\pgfpathclose%
\pgfusepath{fill}%
\end{pgfscope}%
\begin{pgfscope}%
\pgfpathrectangle{\pgfqpoint{0.515000in}{0.499444in}}{\pgfqpoint{3.487500in}{1.155000in}}%
\pgfusepath{clip}%
\pgfsetbuttcap%
\pgfsetmiterjoin%
\definecolor{currentfill}{rgb}{0.000000,0.000000,0.000000}%
\pgfsetfillcolor{currentfill}%
\pgfsetlinewidth{0.000000pt}%
\definecolor{currentstroke}{rgb}{0.000000,0.000000,0.000000}%
\pgfsetstrokecolor{currentstroke}%
\pgfsetstrokeopacity{0.000000}%
\pgfsetdash{}{0pt}%
\pgfpathmoveto{\pgfqpoint{2.734318in}{0.499444in}}%
\pgfpathlineto{\pgfqpoint{2.797728in}{0.499444in}}%
\pgfpathlineto{\pgfqpoint{2.797728in}{0.668163in}}%
\pgfpathlineto{\pgfqpoint{2.734318in}{0.668163in}}%
\pgfpathlineto{\pgfqpoint{2.734318in}{0.499444in}}%
\pgfpathclose%
\pgfusepath{fill}%
\end{pgfscope}%
\begin{pgfscope}%
\pgfpathrectangle{\pgfqpoint{0.515000in}{0.499444in}}{\pgfqpoint{3.487500in}{1.155000in}}%
\pgfusepath{clip}%
\pgfsetbuttcap%
\pgfsetmiterjoin%
\definecolor{currentfill}{rgb}{0.000000,0.000000,0.000000}%
\pgfsetfillcolor{currentfill}%
\pgfsetlinewidth{0.000000pt}%
\definecolor{currentstroke}{rgb}{0.000000,0.000000,0.000000}%
\pgfsetstrokecolor{currentstroke}%
\pgfsetstrokeopacity{0.000000}%
\pgfsetdash{}{0pt}%
\pgfpathmoveto{\pgfqpoint{2.892841in}{0.499444in}}%
\pgfpathlineto{\pgfqpoint{2.956250in}{0.499444in}}%
\pgfpathlineto{\pgfqpoint{2.956250in}{0.642042in}}%
\pgfpathlineto{\pgfqpoint{2.892841in}{0.642042in}}%
\pgfpathlineto{\pgfqpoint{2.892841in}{0.499444in}}%
\pgfpathclose%
\pgfusepath{fill}%
\end{pgfscope}%
\begin{pgfscope}%
\pgfpathrectangle{\pgfqpoint{0.515000in}{0.499444in}}{\pgfqpoint{3.487500in}{1.155000in}}%
\pgfusepath{clip}%
\pgfsetbuttcap%
\pgfsetmiterjoin%
\definecolor{currentfill}{rgb}{0.000000,0.000000,0.000000}%
\pgfsetfillcolor{currentfill}%
\pgfsetlinewidth{0.000000pt}%
\definecolor{currentstroke}{rgb}{0.000000,0.000000,0.000000}%
\pgfsetstrokecolor{currentstroke}%
\pgfsetstrokeopacity{0.000000}%
\pgfsetdash{}{0pt}%
\pgfpathmoveto{\pgfqpoint{3.051364in}{0.499444in}}%
\pgfpathlineto{\pgfqpoint{3.114773in}{0.499444in}}%
\pgfpathlineto{\pgfqpoint{3.114773in}{0.614325in}}%
\pgfpathlineto{\pgfqpoint{3.051364in}{0.614325in}}%
\pgfpathlineto{\pgfqpoint{3.051364in}{0.499444in}}%
\pgfpathclose%
\pgfusepath{fill}%
\end{pgfscope}%
\begin{pgfscope}%
\pgfpathrectangle{\pgfqpoint{0.515000in}{0.499444in}}{\pgfqpoint{3.487500in}{1.155000in}}%
\pgfusepath{clip}%
\pgfsetbuttcap%
\pgfsetmiterjoin%
\definecolor{currentfill}{rgb}{0.000000,0.000000,0.000000}%
\pgfsetfillcolor{currentfill}%
\pgfsetlinewidth{0.000000pt}%
\definecolor{currentstroke}{rgb}{0.000000,0.000000,0.000000}%
\pgfsetstrokecolor{currentstroke}%
\pgfsetstrokeopacity{0.000000}%
\pgfsetdash{}{0pt}%
\pgfpathmoveto{\pgfqpoint{3.209887in}{0.499444in}}%
\pgfpathlineto{\pgfqpoint{3.273296in}{0.499444in}}%
\pgfpathlineto{\pgfqpoint{3.273296in}{0.583053in}}%
\pgfpathlineto{\pgfqpoint{3.209887in}{0.583053in}}%
\pgfpathlineto{\pgfqpoint{3.209887in}{0.499444in}}%
\pgfpathclose%
\pgfusepath{fill}%
\end{pgfscope}%
\begin{pgfscope}%
\pgfpathrectangle{\pgfqpoint{0.515000in}{0.499444in}}{\pgfqpoint{3.487500in}{1.155000in}}%
\pgfusepath{clip}%
\pgfsetbuttcap%
\pgfsetmiterjoin%
\definecolor{currentfill}{rgb}{0.000000,0.000000,0.000000}%
\pgfsetfillcolor{currentfill}%
\pgfsetlinewidth{0.000000pt}%
\definecolor{currentstroke}{rgb}{0.000000,0.000000,0.000000}%
\pgfsetstrokecolor{currentstroke}%
\pgfsetstrokeopacity{0.000000}%
\pgfsetdash{}{0pt}%
\pgfpathmoveto{\pgfqpoint{3.368409in}{0.499444in}}%
\pgfpathlineto{\pgfqpoint{3.431818in}{0.499444in}}%
\pgfpathlineto{\pgfqpoint{3.431818in}{0.555984in}}%
\pgfpathlineto{\pgfqpoint{3.368409in}{0.555984in}}%
\pgfpathlineto{\pgfqpoint{3.368409in}{0.499444in}}%
\pgfpathclose%
\pgfusepath{fill}%
\end{pgfscope}%
\begin{pgfscope}%
\pgfpathrectangle{\pgfqpoint{0.515000in}{0.499444in}}{\pgfqpoint{3.487500in}{1.155000in}}%
\pgfusepath{clip}%
\pgfsetbuttcap%
\pgfsetmiterjoin%
\definecolor{currentfill}{rgb}{0.000000,0.000000,0.000000}%
\pgfsetfillcolor{currentfill}%
\pgfsetlinewidth{0.000000pt}%
\definecolor{currentstroke}{rgb}{0.000000,0.000000,0.000000}%
\pgfsetstrokecolor{currentstroke}%
\pgfsetstrokeopacity{0.000000}%
\pgfsetdash{}{0pt}%
\pgfpathmoveto{\pgfqpoint{3.526932in}{0.499444in}}%
\pgfpathlineto{\pgfqpoint{3.590341in}{0.499444in}}%
\pgfpathlineto{\pgfqpoint{3.590341in}{0.536627in}}%
\pgfpathlineto{\pgfqpoint{3.526932in}{0.536627in}}%
\pgfpathlineto{\pgfqpoint{3.526932in}{0.499444in}}%
\pgfpathclose%
\pgfusepath{fill}%
\end{pgfscope}%
\begin{pgfscope}%
\pgfpathrectangle{\pgfqpoint{0.515000in}{0.499444in}}{\pgfqpoint{3.487500in}{1.155000in}}%
\pgfusepath{clip}%
\pgfsetbuttcap%
\pgfsetmiterjoin%
\definecolor{currentfill}{rgb}{0.000000,0.000000,0.000000}%
\pgfsetfillcolor{currentfill}%
\pgfsetlinewidth{0.000000pt}%
\definecolor{currentstroke}{rgb}{0.000000,0.000000,0.000000}%
\pgfsetstrokecolor{currentstroke}%
\pgfsetstrokeopacity{0.000000}%
\pgfsetdash{}{0pt}%
\pgfpathmoveto{\pgfqpoint{3.685455in}{0.499444in}}%
\pgfpathlineto{\pgfqpoint{3.748864in}{0.499444in}}%
\pgfpathlineto{\pgfqpoint{3.748864in}{0.523037in}}%
\pgfpathlineto{\pgfqpoint{3.685455in}{0.523037in}}%
\pgfpathlineto{\pgfqpoint{3.685455in}{0.499444in}}%
\pgfpathclose%
\pgfusepath{fill}%
\end{pgfscope}%
\begin{pgfscope}%
\pgfpathrectangle{\pgfqpoint{0.515000in}{0.499444in}}{\pgfqpoint{3.487500in}{1.155000in}}%
\pgfusepath{clip}%
\pgfsetbuttcap%
\pgfsetmiterjoin%
\definecolor{currentfill}{rgb}{0.000000,0.000000,0.000000}%
\pgfsetfillcolor{currentfill}%
\pgfsetlinewidth{0.000000pt}%
\definecolor{currentstroke}{rgb}{0.000000,0.000000,0.000000}%
\pgfsetstrokecolor{currentstroke}%
\pgfsetstrokeopacity{0.000000}%
\pgfsetdash{}{0pt}%
\pgfpathmoveto{\pgfqpoint{3.843978in}{0.499444in}}%
\pgfpathlineto{\pgfqpoint{3.907387in}{0.499444in}}%
\pgfpathlineto{\pgfqpoint{3.907387in}{0.532755in}}%
\pgfpathlineto{\pgfqpoint{3.843978in}{0.532755in}}%
\pgfpathlineto{\pgfqpoint{3.843978in}{0.499444in}}%
\pgfpathclose%
\pgfusepath{fill}%
\end{pgfscope}%
\begin{pgfscope}%
\pgfsetbuttcap%
\pgfsetroundjoin%
\definecolor{currentfill}{rgb}{0.000000,0.000000,0.000000}%
\pgfsetfillcolor{currentfill}%
\pgfsetlinewidth{0.803000pt}%
\definecolor{currentstroke}{rgb}{0.000000,0.000000,0.000000}%
\pgfsetstrokecolor{currentstroke}%
\pgfsetdash{}{0pt}%
\pgfsys@defobject{currentmarker}{\pgfqpoint{0.000000in}{-0.048611in}}{\pgfqpoint{0.000000in}{0.000000in}}{%
\pgfpathmoveto{\pgfqpoint{0.000000in}{0.000000in}}%
\pgfpathlineto{\pgfqpoint{0.000000in}{-0.048611in}}%
\pgfusepath{stroke,fill}%
}%
\begin{pgfscope}%
\pgfsys@transformshift{0.515000in}{0.499444in}%
\pgfsys@useobject{currentmarker}{}%
\end{pgfscope}%
\end{pgfscope}%
\begin{pgfscope}%
\pgfsetbuttcap%
\pgfsetroundjoin%
\definecolor{currentfill}{rgb}{0.000000,0.000000,0.000000}%
\pgfsetfillcolor{currentfill}%
\pgfsetlinewidth{0.803000pt}%
\definecolor{currentstroke}{rgb}{0.000000,0.000000,0.000000}%
\pgfsetstrokecolor{currentstroke}%
\pgfsetdash{}{0pt}%
\pgfsys@defobject{currentmarker}{\pgfqpoint{0.000000in}{-0.048611in}}{\pgfqpoint{0.000000in}{0.000000in}}{%
\pgfpathmoveto{\pgfqpoint{0.000000in}{0.000000in}}%
\pgfpathlineto{\pgfqpoint{0.000000in}{-0.048611in}}%
\pgfusepath{stroke,fill}%
}%
\begin{pgfscope}%
\pgfsys@transformshift{0.673523in}{0.499444in}%
\pgfsys@useobject{currentmarker}{}%
\end{pgfscope}%
\end{pgfscope}%
\begin{pgfscope}%
\definecolor{textcolor}{rgb}{0.000000,0.000000,0.000000}%
\pgfsetstrokecolor{textcolor}%
\pgfsetfillcolor{textcolor}%
\pgftext[x=0.673523in,y=0.402222in,,top]{\color{textcolor}\rmfamily\fontsize{10.000000}{12.000000}\selectfont 0.0}%
\end{pgfscope}%
\begin{pgfscope}%
\pgfsetbuttcap%
\pgfsetroundjoin%
\definecolor{currentfill}{rgb}{0.000000,0.000000,0.000000}%
\pgfsetfillcolor{currentfill}%
\pgfsetlinewidth{0.803000pt}%
\definecolor{currentstroke}{rgb}{0.000000,0.000000,0.000000}%
\pgfsetstrokecolor{currentstroke}%
\pgfsetdash{}{0pt}%
\pgfsys@defobject{currentmarker}{\pgfqpoint{0.000000in}{-0.048611in}}{\pgfqpoint{0.000000in}{0.000000in}}{%
\pgfpathmoveto{\pgfqpoint{0.000000in}{0.000000in}}%
\pgfpathlineto{\pgfqpoint{0.000000in}{-0.048611in}}%
\pgfusepath{stroke,fill}%
}%
\begin{pgfscope}%
\pgfsys@transformshift{0.832046in}{0.499444in}%
\pgfsys@useobject{currentmarker}{}%
\end{pgfscope}%
\end{pgfscope}%
\begin{pgfscope}%
\pgfsetbuttcap%
\pgfsetroundjoin%
\definecolor{currentfill}{rgb}{0.000000,0.000000,0.000000}%
\pgfsetfillcolor{currentfill}%
\pgfsetlinewidth{0.803000pt}%
\definecolor{currentstroke}{rgb}{0.000000,0.000000,0.000000}%
\pgfsetstrokecolor{currentstroke}%
\pgfsetdash{}{0pt}%
\pgfsys@defobject{currentmarker}{\pgfqpoint{0.000000in}{-0.048611in}}{\pgfqpoint{0.000000in}{0.000000in}}{%
\pgfpathmoveto{\pgfqpoint{0.000000in}{0.000000in}}%
\pgfpathlineto{\pgfqpoint{0.000000in}{-0.048611in}}%
\pgfusepath{stroke,fill}%
}%
\begin{pgfscope}%
\pgfsys@transformshift{0.990568in}{0.499444in}%
\pgfsys@useobject{currentmarker}{}%
\end{pgfscope}%
\end{pgfscope}%
\begin{pgfscope}%
\definecolor{textcolor}{rgb}{0.000000,0.000000,0.000000}%
\pgfsetstrokecolor{textcolor}%
\pgfsetfillcolor{textcolor}%
\pgftext[x=0.990568in,y=0.402222in,,top]{\color{textcolor}\rmfamily\fontsize{10.000000}{12.000000}\selectfont 0.1}%
\end{pgfscope}%
\begin{pgfscope}%
\pgfsetbuttcap%
\pgfsetroundjoin%
\definecolor{currentfill}{rgb}{0.000000,0.000000,0.000000}%
\pgfsetfillcolor{currentfill}%
\pgfsetlinewidth{0.803000pt}%
\definecolor{currentstroke}{rgb}{0.000000,0.000000,0.000000}%
\pgfsetstrokecolor{currentstroke}%
\pgfsetdash{}{0pt}%
\pgfsys@defobject{currentmarker}{\pgfqpoint{0.000000in}{-0.048611in}}{\pgfqpoint{0.000000in}{0.000000in}}{%
\pgfpathmoveto{\pgfqpoint{0.000000in}{0.000000in}}%
\pgfpathlineto{\pgfqpoint{0.000000in}{-0.048611in}}%
\pgfusepath{stroke,fill}%
}%
\begin{pgfscope}%
\pgfsys@transformshift{1.149091in}{0.499444in}%
\pgfsys@useobject{currentmarker}{}%
\end{pgfscope}%
\end{pgfscope}%
\begin{pgfscope}%
\pgfsetbuttcap%
\pgfsetroundjoin%
\definecolor{currentfill}{rgb}{0.000000,0.000000,0.000000}%
\pgfsetfillcolor{currentfill}%
\pgfsetlinewidth{0.803000pt}%
\definecolor{currentstroke}{rgb}{0.000000,0.000000,0.000000}%
\pgfsetstrokecolor{currentstroke}%
\pgfsetdash{}{0pt}%
\pgfsys@defobject{currentmarker}{\pgfqpoint{0.000000in}{-0.048611in}}{\pgfqpoint{0.000000in}{0.000000in}}{%
\pgfpathmoveto{\pgfqpoint{0.000000in}{0.000000in}}%
\pgfpathlineto{\pgfqpoint{0.000000in}{-0.048611in}}%
\pgfusepath{stroke,fill}%
}%
\begin{pgfscope}%
\pgfsys@transformshift{1.307614in}{0.499444in}%
\pgfsys@useobject{currentmarker}{}%
\end{pgfscope}%
\end{pgfscope}%
\begin{pgfscope}%
\definecolor{textcolor}{rgb}{0.000000,0.000000,0.000000}%
\pgfsetstrokecolor{textcolor}%
\pgfsetfillcolor{textcolor}%
\pgftext[x=1.307614in,y=0.402222in,,top]{\color{textcolor}\rmfamily\fontsize{10.000000}{12.000000}\selectfont 0.2}%
\end{pgfscope}%
\begin{pgfscope}%
\pgfsetbuttcap%
\pgfsetroundjoin%
\definecolor{currentfill}{rgb}{0.000000,0.000000,0.000000}%
\pgfsetfillcolor{currentfill}%
\pgfsetlinewidth{0.803000pt}%
\definecolor{currentstroke}{rgb}{0.000000,0.000000,0.000000}%
\pgfsetstrokecolor{currentstroke}%
\pgfsetdash{}{0pt}%
\pgfsys@defobject{currentmarker}{\pgfqpoint{0.000000in}{-0.048611in}}{\pgfqpoint{0.000000in}{0.000000in}}{%
\pgfpathmoveto{\pgfqpoint{0.000000in}{0.000000in}}%
\pgfpathlineto{\pgfqpoint{0.000000in}{-0.048611in}}%
\pgfusepath{stroke,fill}%
}%
\begin{pgfscope}%
\pgfsys@transformshift{1.466137in}{0.499444in}%
\pgfsys@useobject{currentmarker}{}%
\end{pgfscope}%
\end{pgfscope}%
\begin{pgfscope}%
\pgfsetbuttcap%
\pgfsetroundjoin%
\definecolor{currentfill}{rgb}{0.000000,0.000000,0.000000}%
\pgfsetfillcolor{currentfill}%
\pgfsetlinewidth{0.803000pt}%
\definecolor{currentstroke}{rgb}{0.000000,0.000000,0.000000}%
\pgfsetstrokecolor{currentstroke}%
\pgfsetdash{}{0pt}%
\pgfsys@defobject{currentmarker}{\pgfqpoint{0.000000in}{-0.048611in}}{\pgfqpoint{0.000000in}{0.000000in}}{%
\pgfpathmoveto{\pgfqpoint{0.000000in}{0.000000in}}%
\pgfpathlineto{\pgfqpoint{0.000000in}{-0.048611in}}%
\pgfusepath{stroke,fill}%
}%
\begin{pgfscope}%
\pgfsys@transformshift{1.624659in}{0.499444in}%
\pgfsys@useobject{currentmarker}{}%
\end{pgfscope}%
\end{pgfscope}%
\begin{pgfscope}%
\definecolor{textcolor}{rgb}{0.000000,0.000000,0.000000}%
\pgfsetstrokecolor{textcolor}%
\pgfsetfillcolor{textcolor}%
\pgftext[x=1.624659in,y=0.402222in,,top]{\color{textcolor}\rmfamily\fontsize{10.000000}{12.000000}\selectfont 0.3}%
\end{pgfscope}%
\begin{pgfscope}%
\pgfsetbuttcap%
\pgfsetroundjoin%
\definecolor{currentfill}{rgb}{0.000000,0.000000,0.000000}%
\pgfsetfillcolor{currentfill}%
\pgfsetlinewidth{0.803000pt}%
\definecolor{currentstroke}{rgb}{0.000000,0.000000,0.000000}%
\pgfsetstrokecolor{currentstroke}%
\pgfsetdash{}{0pt}%
\pgfsys@defobject{currentmarker}{\pgfqpoint{0.000000in}{-0.048611in}}{\pgfqpoint{0.000000in}{0.000000in}}{%
\pgfpathmoveto{\pgfqpoint{0.000000in}{0.000000in}}%
\pgfpathlineto{\pgfqpoint{0.000000in}{-0.048611in}}%
\pgfusepath{stroke,fill}%
}%
\begin{pgfscope}%
\pgfsys@transformshift{1.783182in}{0.499444in}%
\pgfsys@useobject{currentmarker}{}%
\end{pgfscope}%
\end{pgfscope}%
\begin{pgfscope}%
\pgfsetbuttcap%
\pgfsetroundjoin%
\definecolor{currentfill}{rgb}{0.000000,0.000000,0.000000}%
\pgfsetfillcolor{currentfill}%
\pgfsetlinewidth{0.803000pt}%
\definecolor{currentstroke}{rgb}{0.000000,0.000000,0.000000}%
\pgfsetstrokecolor{currentstroke}%
\pgfsetdash{}{0pt}%
\pgfsys@defobject{currentmarker}{\pgfqpoint{0.000000in}{-0.048611in}}{\pgfqpoint{0.000000in}{0.000000in}}{%
\pgfpathmoveto{\pgfqpoint{0.000000in}{0.000000in}}%
\pgfpathlineto{\pgfqpoint{0.000000in}{-0.048611in}}%
\pgfusepath{stroke,fill}%
}%
\begin{pgfscope}%
\pgfsys@transformshift{1.941705in}{0.499444in}%
\pgfsys@useobject{currentmarker}{}%
\end{pgfscope}%
\end{pgfscope}%
\begin{pgfscope}%
\definecolor{textcolor}{rgb}{0.000000,0.000000,0.000000}%
\pgfsetstrokecolor{textcolor}%
\pgfsetfillcolor{textcolor}%
\pgftext[x=1.941705in,y=0.402222in,,top]{\color{textcolor}\rmfamily\fontsize{10.000000}{12.000000}\selectfont 0.4}%
\end{pgfscope}%
\begin{pgfscope}%
\pgfsetbuttcap%
\pgfsetroundjoin%
\definecolor{currentfill}{rgb}{0.000000,0.000000,0.000000}%
\pgfsetfillcolor{currentfill}%
\pgfsetlinewidth{0.803000pt}%
\definecolor{currentstroke}{rgb}{0.000000,0.000000,0.000000}%
\pgfsetstrokecolor{currentstroke}%
\pgfsetdash{}{0pt}%
\pgfsys@defobject{currentmarker}{\pgfqpoint{0.000000in}{-0.048611in}}{\pgfqpoint{0.000000in}{0.000000in}}{%
\pgfpathmoveto{\pgfqpoint{0.000000in}{0.000000in}}%
\pgfpathlineto{\pgfqpoint{0.000000in}{-0.048611in}}%
\pgfusepath{stroke,fill}%
}%
\begin{pgfscope}%
\pgfsys@transformshift{2.100228in}{0.499444in}%
\pgfsys@useobject{currentmarker}{}%
\end{pgfscope}%
\end{pgfscope}%
\begin{pgfscope}%
\pgfsetbuttcap%
\pgfsetroundjoin%
\definecolor{currentfill}{rgb}{0.000000,0.000000,0.000000}%
\pgfsetfillcolor{currentfill}%
\pgfsetlinewidth{0.803000pt}%
\definecolor{currentstroke}{rgb}{0.000000,0.000000,0.000000}%
\pgfsetstrokecolor{currentstroke}%
\pgfsetdash{}{0pt}%
\pgfsys@defobject{currentmarker}{\pgfqpoint{0.000000in}{-0.048611in}}{\pgfqpoint{0.000000in}{0.000000in}}{%
\pgfpathmoveto{\pgfqpoint{0.000000in}{0.000000in}}%
\pgfpathlineto{\pgfqpoint{0.000000in}{-0.048611in}}%
\pgfusepath{stroke,fill}%
}%
\begin{pgfscope}%
\pgfsys@transformshift{2.258750in}{0.499444in}%
\pgfsys@useobject{currentmarker}{}%
\end{pgfscope}%
\end{pgfscope}%
\begin{pgfscope}%
\definecolor{textcolor}{rgb}{0.000000,0.000000,0.000000}%
\pgfsetstrokecolor{textcolor}%
\pgfsetfillcolor{textcolor}%
\pgftext[x=2.258750in,y=0.402222in,,top]{\color{textcolor}\rmfamily\fontsize{10.000000}{12.000000}\selectfont 0.5}%
\end{pgfscope}%
\begin{pgfscope}%
\pgfsetbuttcap%
\pgfsetroundjoin%
\definecolor{currentfill}{rgb}{0.000000,0.000000,0.000000}%
\pgfsetfillcolor{currentfill}%
\pgfsetlinewidth{0.803000pt}%
\definecolor{currentstroke}{rgb}{0.000000,0.000000,0.000000}%
\pgfsetstrokecolor{currentstroke}%
\pgfsetdash{}{0pt}%
\pgfsys@defobject{currentmarker}{\pgfqpoint{0.000000in}{-0.048611in}}{\pgfqpoint{0.000000in}{0.000000in}}{%
\pgfpathmoveto{\pgfqpoint{0.000000in}{0.000000in}}%
\pgfpathlineto{\pgfqpoint{0.000000in}{-0.048611in}}%
\pgfusepath{stroke,fill}%
}%
\begin{pgfscope}%
\pgfsys@transformshift{2.417273in}{0.499444in}%
\pgfsys@useobject{currentmarker}{}%
\end{pgfscope}%
\end{pgfscope}%
\begin{pgfscope}%
\pgfsetbuttcap%
\pgfsetroundjoin%
\definecolor{currentfill}{rgb}{0.000000,0.000000,0.000000}%
\pgfsetfillcolor{currentfill}%
\pgfsetlinewidth{0.803000pt}%
\definecolor{currentstroke}{rgb}{0.000000,0.000000,0.000000}%
\pgfsetstrokecolor{currentstroke}%
\pgfsetdash{}{0pt}%
\pgfsys@defobject{currentmarker}{\pgfqpoint{0.000000in}{-0.048611in}}{\pgfqpoint{0.000000in}{0.000000in}}{%
\pgfpathmoveto{\pgfqpoint{0.000000in}{0.000000in}}%
\pgfpathlineto{\pgfqpoint{0.000000in}{-0.048611in}}%
\pgfusepath{stroke,fill}%
}%
\begin{pgfscope}%
\pgfsys@transformshift{2.575796in}{0.499444in}%
\pgfsys@useobject{currentmarker}{}%
\end{pgfscope}%
\end{pgfscope}%
\begin{pgfscope}%
\definecolor{textcolor}{rgb}{0.000000,0.000000,0.000000}%
\pgfsetstrokecolor{textcolor}%
\pgfsetfillcolor{textcolor}%
\pgftext[x=2.575796in,y=0.402222in,,top]{\color{textcolor}\rmfamily\fontsize{10.000000}{12.000000}\selectfont 0.6}%
\end{pgfscope}%
\begin{pgfscope}%
\pgfsetbuttcap%
\pgfsetroundjoin%
\definecolor{currentfill}{rgb}{0.000000,0.000000,0.000000}%
\pgfsetfillcolor{currentfill}%
\pgfsetlinewidth{0.803000pt}%
\definecolor{currentstroke}{rgb}{0.000000,0.000000,0.000000}%
\pgfsetstrokecolor{currentstroke}%
\pgfsetdash{}{0pt}%
\pgfsys@defobject{currentmarker}{\pgfqpoint{0.000000in}{-0.048611in}}{\pgfqpoint{0.000000in}{0.000000in}}{%
\pgfpathmoveto{\pgfqpoint{0.000000in}{0.000000in}}%
\pgfpathlineto{\pgfqpoint{0.000000in}{-0.048611in}}%
\pgfusepath{stroke,fill}%
}%
\begin{pgfscope}%
\pgfsys@transformshift{2.734318in}{0.499444in}%
\pgfsys@useobject{currentmarker}{}%
\end{pgfscope}%
\end{pgfscope}%
\begin{pgfscope}%
\pgfsetbuttcap%
\pgfsetroundjoin%
\definecolor{currentfill}{rgb}{0.000000,0.000000,0.000000}%
\pgfsetfillcolor{currentfill}%
\pgfsetlinewidth{0.803000pt}%
\definecolor{currentstroke}{rgb}{0.000000,0.000000,0.000000}%
\pgfsetstrokecolor{currentstroke}%
\pgfsetdash{}{0pt}%
\pgfsys@defobject{currentmarker}{\pgfqpoint{0.000000in}{-0.048611in}}{\pgfqpoint{0.000000in}{0.000000in}}{%
\pgfpathmoveto{\pgfqpoint{0.000000in}{0.000000in}}%
\pgfpathlineto{\pgfqpoint{0.000000in}{-0.048611in}}%
\pgfusepath{stroke,fill}%
}%
\begin{pgfscope}%
\pgfsys@transformshift{2.892841in}{0.499444in}%
\pgfsys@useobject{currentmarker}{}%
\end{pgfscope}%
\end{pgfscope}%
\begin{pgfscope}%
\definecolor{textcolor}{rgb}{0.000000,0.000000,0.000000}%
\pgfsetstrokecolor{textcolor}%
\pgfsetfillcolor{textcolor}%
\pgftext[x=2.892841in,y=0.402222in,,top]{\color{textcolor}\rmfamily\fontsize{10.000000}{12.000000}\selectfont 0.7}%
\end{pgfscope}%
\begin{pgfscope}%
\pgfsetbuttcap%
\pgfsetroundjoin%
\definecolor{currentfill}{rgb}{0.000000,0.000000,0.000000}%
\pgfsetfillcolor{currentfill}%
\pgfsetlinewidth{0.803000pt}%
\definecolor{currentstroke}{rgb}{0.000000,0.000000,0.000000}%
\pgfsetstrokecolor{currentstroke}%
\pgfsetdash{}{0pt}%
\pgfsys@defobject{currentmarker}{\pgfqpoint{0.000000in}{-0.048611in}}{\pgfqpoint{0.000000in}{0.000000in}}{%
\pgfpathmoveto{\pgfqpoint{0.000000in}{0.000000in}}%
\pgfpathlineto{\pgfqpoint{0.000000in}{-0.048611in}}%
\pgfusepath{stroke,fill}%
}%
\begin{pgfscope}%
\pgfsys@transformshift{3.051364in}{0.499444in}%
\pgfsys@useobject{currentmarker}{}%
\end{pgfscope}%
\end{pgfscope}%
\begin{pgfscope}%
\pgfsetbuttcap%
\pgfsetroundjoin%
\definecolor{currentfill}{rgb}{0.000000,0.000000,0.000000}%
\pgfsetfillcolor{currentfill}%
\pgfsetlinewidth{0.803000pt}%
\definecolor{currentstroke}{rgb}{0.000000,0.000000,0.000000}%
\pgfsetstrokecolor{currentstroke}%
\pgfsetdash{}{0pt}%
\pgfsys@defobject{currentmarker}{\pgfqpoint{0.000000in}{-0.048611in}}{\pgfqpoint{0.000000in}{0.000000in}}{%
\pgfpathmoveto{\pgfqpoint{0.000000in}{0.000000in}}%
\pgfpathlineto{\pgfqpoint{0.000000in}{-0.048611in}}%
\pgfusepath{stroke,fill}%
}%
\begin{pgfscope}%
\pgfsys@transformshift{3.209887in}{0.499444in}%
\pgfsys@useobject{currentmarker}{}%
\end{pgfscope}%
\end{pgfscope}%
\begin{pgfscope}%
\definecolor{textcolor}{rgb}{0.000000,0.000000,0.000000}%
\pgfsetstrokecolor{textcolor}%
\pgfsetfillcolor{textcolor}%
\pgftext[x=3.209887in,y=0.402222in,,top]{\color{textcolor}\rmfamily\fontsize{10.000000}{12.000000}\selectfont 0.8}%
\end{pgfscope}%
\begin{pgfscope}%
\pgfsetbuttcap%
\pgfsetroundjoin%
\definecolor{currentfill}{rgb}{0.000000,0.000000,0.000000}%
\pgfsetfillcolor{currentfill}%
\pgfsetlinewidth{0.803000pt}%
\definecolor{currentstroke}{rgb}{0.000000,0.000000,0.000000}%
\pgfsetstrokecolor{currentstroke}%
\pgfsetdash{}{0pt}%
\pgfsys@defobject{currentmarker}{\pgfqpoint{0.000000in}{-0.048611in}}{\pgfqpoint{0.000000in}{0.000000in}}{%
\pgfpathmoveto{\pgfqpoint{0.000000in}{0.000000in}}%
\pgfpathlineto{\pgfqpoint{0.000000in}{-0.048611in}}%
\pgfusepath{stroke,fill}%
}%
\begin{pgfscope}%
\pgfsys@transformshift{3.368409in}{0.499444in}%
\pgfsys@useobject{currentmarker}{}%
\end{pgfscope}%
\end{pgfscope}%
\begin{pgfscope}%
\pgfsetbuttcap%
\pgfsetroundjoin%
\definecolor{currentfill}{rgb}{0.000000,0.000000,0.000000}%
\pgfsetfillcolor{currentfill}%
\pgfsetlinewidth{0.803000pt}%
\definecolor{currentstroke}{rgb}{0.000000,0.000000,0.000000}%
\pgfsetstrokecolor{currentstroke}%
\pgfsetdash{}{0pt}%
\pgfsys@defobject{currentmarker}{\pgfqpoint{0.000000in}{-0.048611in}}{\pgfqpoint{0.000000in}{0.000000in}}{%
\pgfpathmoveto{\pgfqpoint{0.000000in}{0.000000in}}%
\pgfpathlineto{\pgfqpoint{0.000000in}{-0.048611in}}%
\pgfusepath{stroke,fill}%
}%
\begin{pgfscope}%
\pgfsys@transformshift{3.526932in}{0.499444in}%
\pgfsys@useobject{currentmarker}{}%
\end{pgfscope}%
\end{pgfscope}%
\begin{pgfscope}%
\definecolor{textcolor}{rgb}{0.000000,0.000000,0.000000}%
\pgfsetstrokecolor{textcolor}%
\pgfsetfillcolor{textcolor}%
\pgftext[x=3.526932in,y=0.402222in,,top]{\color{textcolor}\rmfamily\fontsize{10.000000}{12.000000}\selectfont 0.9}%
\end{pgfscope}%
\begin{pgfscope}%
\pgfsetbuttcap%
\pgfsetroundjoin%
\definecolor{currentfill}{rgb}{0.000000,0.000000,0.000000}%
\pgfsetfillcolor{currentfill}%
\pgfsetlinewidth{0.803000pt}%
\definecolor{currentstroke}{rgb}{0.000000,0.000000,0.000000}%
\pgfsetstrokecolor{currentstroke}%
\pgfsetdash{}{0pt}%
\pgfsys@defobject{currentmarker}{\pgfqpoint{0.000000in}{-0.048611in}}{\pgfqpoint{0.000000in}{0.000000in}}{%
\pgfpathmoveto{\pgfqpoint{0.000000in}{0.000000in}}%
\pgfpathlineto{\pgfqpoint{0.000000in}{-0.048611in}}%
\pgfusepath{stroke,fill}%
}%
\begin{pgfscope}%
\pgfsys@transformshift{3.685455in}{0.499444in}%
\pgfsys@useobject{currentmarker}{}%
\end{pgfscope}%
\end{pgfscope}%
\begin{pgfscope}%
\pgfsetbuttcap%
\pgfsetroundjoin%
\definecolor{currentfill}{rgb}{0.000000,0.000000,0.000000}%
\pgfsetfillcolor{currentfill}%
\pgfsetlinewidth{0.803000pt}%
\definecolor{currentstroke}{rgb}{0.000000,0.000000,0.000000}%
\pgfsetstrokecolor{currentstroke}%
\pgfsetdash{}{0pt}%
\pgfsys@defobject{currentmarker}{\pgfqpoint{0.000000in}{-0.048611in}}{\pgfqpoint{0.000000in}{0.000000in}}{%
\pgfpathmoveto{\pgfqpoint{0.000000in}{0.000000in}}%
\pgfpathlineto{\pgfqpoint{0.000000in}{-0.048611in}}%
\pgfusepath{stroke,fill}%
}%
\begin{pgfscope}%
\pgfsys@transformshift{3.843978in}{0.499444in}%
\pgfsys@useobject{currentmarker}{}%
\end{pgfscope}%
\end{pgfscope}%
\begin{pgfscope}%
\definecolor{textcolor}{rgb}{0.000000,0.000000,0.000000}%
\pgfsetstrokecolor{textcolor}%
\pgfsetfillcolor{textcolor}%
\pgftext[x=3.843978in,y=0.402222in,,top]{\color{textcolor}\rmfamily\fontsize{10.000000}{12.000000}\selectfont 1.0}%
\end{pgfscope}%
\begin{pgfscope}%
\pgfsetbuttcap%
\pgfsetroundjoin%
\definecolor{currentfill}{rgb}{0.000000,0.000000,0.000000}%
\pgfsetfillcolor{currentfill}%
\pgfsetlinewidth{0.803000pt}%
\definecolor{currentstroke}{rgb}{0.000000,0.000000,0.000000}%
\pgfsetstrokecolor{currentstroke}%
\pgfsetdash{}{0pt}%
\pgfsys@defobject{currentmarker}{\pgfqpoint{0.000000in}{-0.048611in}}{\pgfqpoint{0.000000in}{0.000000in}}{%
\pgfpathmoveto{\pgfqpoint{0.000000in}{0.000000in}}%
\pgfpathlineto{\pgfqpoint{0.000000in}{-0.048611in}}%
\pgfusepath{stroke,fill}%
}%
\begin{pgfscope}%
\pgfsys@transformshift{4.002500in}{0.499444in}%
\pgfsys@useobject{currentmarker}{}%
\end{pgfscope}%
\end{pgfscope}%
\begin{pgfscope}%
\definecolor{textcolor}{rgb}{0.000000,0.000000,0.000000}%
\pgfsetstrokecolor{textcolor}%
\pgfsetfillcolor{textcolor}%
\pgftext[x=2.258750in,y=0.223333in,,top]{\color{textcolor}\rmfamily\fontsize{10.000000}{12.000000}\selectfont \(\displaystyle p\)}%
\end{pgfscope}%
\begin{pgfscope}%
\pgfsetbuttcap%
\pgfsetroundjoin%
\definecolor{currentfill}{rgb}{0.000000,0.000000,0.000000}%
\pgfsetfillcolor{currentfill}%
\pgfsetlinewidth{0.803000pt}%
\definecolor{currentstroke}{rgb}{0.000000,0.000000,0.000000}%
\pgfsetstrokecolor{currentstroke}%
\pgfsetdash{}{0pt}%
\pgfsys@defobject{currentmarker}{\pgfqpoint{-0.048611in}{0.000000in}}{\pgfqpoint{-0.000000in}{0.000000in}}{%
\pgfpathmoveto{\pgfqpoint{-0.000000in}{0.000000in}}%
\pgfpathlineto{\pgfqpoint{-0.048611in}{0.000000in}}%
\pgfusepath{stroke,fill}%
}%
\begin{pgfscope}%
\pgfsys@transformshift{0.515000in}{0.499444in}%
\pgfsys@useobject{currentmarker}{}%
\end{pgfscope}%
\end{pgfscope}%
\begin{pgfscope}%
\definecolor{textcolor}{rgb}{0.000000,0.000000,0.000000}%
\pgfsetstrokecolor{textcolor}%
\pgfsetfillcolor{textcolor}%
\pgftext[x=0.348333in, y=0.451250in, left, base]{\color{textcolor}\rmfamily\fontsize{10.000000}{12.000000}\selectfont \(\displaystyle {0}\)}%
\end{pgfscope}%
\begin{pgfscope}%
\pgfsetbuttcap%
\pgfsetroundjoin%
\definecolor{currentfill}{rgb}{0.000000,0.000000,0.000000}%
\pgfsetfillcolor{currentfill}%
\pgfsetlinewidth{0.803000pt}%
\definecolor{currentstroke}{rgb}{0.000000,0.000000,0.000000}%
\pgfsetstrokecolor{currentstroke}%
\pgfsetdash{}{0pt}%
\pgfsys@defobject{currentmarker}{\pgfqpoint{-0.048611in}{0.000000in}}{\pgfqpoint{-0.000000in}{0.000000in}}{%
\pgfpathmoveto{\pgfqpoint{-0.000000in}{0.000000in}}%
\pgfpathlineto{\pgfqpoint{-0.048611in}{0.000000in}}%
\pgfusepath{stroke,fill}%
}%
\begin{pgfscope}%
\pgfsys@transformshift{0.515000in}{1.063236in}%
\pgfsys@useobject{currentmarker}{}%
\end{pgfscope}%
\end{pgfscope}%
\begin{pgfscope}%
\definecolor{textcolor}{rgb}{0.000000,0.000000,0.000000}%
\pgfsetstrokecolor{textcolor}%
\pgfsetfillcolor{textcolor}%
\pgftext[x=0.348333in, y=1.015042in, left, base]{\color{textcolor}\rmfamily\fontsize{10.000000}{12.000000}\selectfont \(\displaystyle {5}\)}%
\end{pgfscope}%
\begin{pgfscope}%
\pgfsetbuttcap%
\pgfsetroundjoin%
\definecolor{currentfill}{rgb}{0.000000,0.000000,0.000000}%
\pgfsetfillcolor{currentfill}%
\pgfsetlinewidth{0.803000pt}%
\definecolor{currentstroke}{rgb}{0.000000,0.000000,0.000000}%
\pgfsetstrokecolor{currentstroke}%
\pgfsetdash{}{0pt}%
\pgfsys@defobject{currentmarker}{\pgfqpoint{-0.048611in}{0.000000in}}{\pgfqpoint{-0.000000in}{0.000000in}}{%
\pgfpathmoveto{\pgfqpoint{-0.000000in}{0.000000in}}%
\pgfpathlineto{\pgfqpoint{-0.048611in}{0.000000in}}%
\pgfusepath{stroke,fill}%
}%
\begin{pgfscope}%
\pgfsys@transformshift{0.515000in}{1.627028in}%
\pgfsys@useobject{currentmarker}{}%
\end{pgfscope}%
\end{pgfscope}%
\begin{pgfscope}%
\definecolor{textcolor}{rgb}{0.000000,0.000000,0.000000}%
\pgfsetstrokecolor{textcolor}%
\pgfsetfillcolor{textcolor}%
\pgftext[x=0.278889in, y=1.578834in, left, base]{\color{textcolor}\rmfamily\fontsize{10.000000}{12.000000}\selectfont \(\displaystyle {10}\)}%
\end{pgfscope}%
\begin{pgfscope}%
\definecolor{textcolor}{rgb}{0.000000,0.000000,0.000000}%
\pgfsetstrokecolor{textcolor}%
\pgfsetfillcolor{textcolor}%
\pgftext[x=0.223333in,y=1.076944in,,bottom,rotate=90.000000]{\color{textcolor}\rmfamily\fontsize{10.000000}{12.000000}\selectfont Percent of Data Set}%
\end{pgfscope}%
\begin{pgfscope}%
\pgfsetrectcap%
\pgfsetmiterjoin%
\pgfsetlinewidth{0.803000pt}%
\definecolor{currentstroke}{rgb}{0.000000,0.000000,0.000000}%
\pgfsetstrokecolor{currentstroke}%
\pgfsetdash{}{0pt}%
\pgfpathmoveto{\pgfqpoint{0.515000in}{0.499444in}}%
\pgfpathlineto{\pgfqpoint{0.515000in}{1.654444in}}%
\pgfusepath{stroke}%
\end{pgfscope}%
\begin{pgfscope}%
\pgfsetrectcap%
\pgfsetmiterjoin%
\pgfsetlinewidth{0.803000pt}%
\definecolor{currentstroke}{rgb}{0.000000,0.000000,0.000000}%
\pgfsetstrokecolor{currentstroke}%
\pgfsetdash{}{0pt}%
\pgfpathmoveto{\pgfqpoint{4.002500in}{0.499444in}}%
\pgfpathlineto{\pgfqpoint{4.002500in}{1.654444in}}%
\pgfusepath{stroke}%
\end{pgfscope}%
\begin{pgfscope}%
\pgfsetrectcap%
\pgfsetmiterjoin%
\pgfsetlinewidth{0.803000pt}%
\definecolor{currentstroke}{rgb}{0.000000,0.000000,0.000000}%
\pgfsetstrokecolor{currentstroke}%
\pgfsetdash{}{0pt}%
\pgfpathmoveto{\pgfqpoint{0.515000in}{0.499444in}}%
\pgfpathlineto{\pgfqpoint{4.002500in}{0.499444in}}%
\pgfusepath{stroke}%
\end{pgfscope}%
\begin{pgfscope}%
\pgfsetrectcap%
\pgfsetmiterjoin%
\pgfsetlinewidth{0.803000pt}%
\definecolor{currentstroke}{rgb}{0.000000,0.000000,0.000000}%
\pgfsetstrokecolor{currentstroke}%
\pgfsetdash{}{0pt}%
\pgfpathmoveto{\pgfqpoint{0.515000in}{1.654444in}}%
\pgfpathlineto{\pgfqpoint{4.002500in}{1.654444in}}%
\pgfusepath{stroke}%
\end{pgfscope}%
\begin{pgfscope}%
\pgfsetbuttcap%
\pgfsetmiterjoin%
\definecolor{currentfill}{rgb}{1.000000,1.000000,1.000000}%
\pgfsetfillcolor{currentfill}%
\pgfsetfillopacity{0.800000}%
\pgfsetlinewidth{1.003750pt}%
\definecolor{currentstroke}{rgb}{0.800000,0.800000,0.800000}%
\pgfsetstrokecolor{currentstroke}%
\pgfsetstrokeopacity{0.800000}%
\pgfsetdash{}{0pt}%
\pgfpathmoveto{\pgfqpoint{3.225556in}{1.154445in}}%
\pgfpathlineto{\pgfqpoint{3.905278in}{1.154445in}}%
\pgfpathquadraticcurveto{\pgfqpoint{3.933056in}{1.154445in}}{\pgfqpoint{3.933056in}{1.182222in}}%
\pgfpathlineto{\pgfqpoint{3.933056in}{1.557222in}}%
\pgfpathquadraticcurveto{\pgfqpoint{3.933056in}{1.585000in}}{\pgfqpoint{3.905278in}{1.585000in}}%
\pgfpathlineto{\pgfqpoint{3.225556in}{1.585000in}}%
\pgfpathquadraticcurveto{\pgfqpoint{3.197778in}{1.585000in}}{\pgfqpoint{3.197778in}{1.557222in}}%
\pgfpathlineto{\pgfqpoint{3.197778in}{1.182222in}}%
\pgfpathquadraticcurveto{\pgfqpoint{3.197778in}{1.154445in}}{\pgfqpoint{3.225556in}{1.154445in}}%
\pgfpathlineto{\pgfqpoint{3.225556in}{1.154445in}}%
\pgfpathclose%
\pgfusepath{stroke,fill}%
\end{pgfscope}%
\begin{pgfscope}%
\pgfsetbuttcap%
\pgfsetmiterjoin%
\pgfsetlinewidth{1.003750pt}%
\definecolor{currentstroke}{rgb}{0.000000,0.000000,0.000000}%
\pgfsetstrokecolor{currentstroke}%
\pgfsetdash{}{0pt}%
\pgfpathmoveto{\pgfqpoint{3.253334in}{1.432222in}}%
\pgfpathlineto{\pgfqpoint{3.531111in}{1.432222in}}%
\pgfpathlineto{\pgfqpoint{3.531111in}{1.529444in}}%
\pgfpathlineto{\pgfqpoint{3.253334in}{1.529444in}}%
\pgfpathlineto{\pgfqpoint{3.253334in}{1.432222in}}%
\pgfpathclose%
\pgfusepath{stroke}%
\end{pgfscope}%
\begin{pgfscope}%
\definecolor{textcolor}{rgb}{0.000000,0.000000,0.000000}%
\pgfsetstrokecolor{textcolor}%
\pgfsetfillcolor{textcolor}%
\pgftext[x=3.642223in,y=1.432222in,left,base]{\color{textcolor}\rmfamily\fontsize{10.000000}{12.000000}\selectfont Neg}%
\end{pgfscope}%
\begin{pgfscope}%
\pgfsetbuttcap%
\pgfsetmiterjoin%
\definecolor{currentfill}{rgb}{0.000000,0.000000,0.000000}%
\pgfsetfillcolor{currentfill}%
\pgfsetlinewidth{0.000000pt}%
\definecolor{currentstroke}{rgb}{0.000000,0.000000,0.000000}%
\pgfsetstrokecolor{currentstroke}%
\pgfsetstrokeopacity{0.000000}%
\pgfsetdash{}{0pt}%
\pgfpathmoveto{\pgfqpoint{3.253334in}{1.236944in}}%
\pgfpathlineto{\pgfqpoint{3.531111in}{1.236944in}}%
\pgfpathlineto{\pgfqpoint{3.531111in}{1.334167in}}%
\pgfpathlineto{\pgfqpoint{3.253334in}{1.334167in}}%
\pgfpathlineto{\pgfqpoint{3.253334in}{1.236944in}}%
\pgfpathclose%
\pgfusepath{fill}%
\end{pgfscope}%
\begin{pgfscope}%
\definecolor{textcolor}{rgb}{0.000000,0.000000,0.000000}%
\pgfsetstrokecolor{textcolor}%
\pgfsetfillcolor{textcolor}%
\pgftext[x=3.642223in,y=1.236944in,left,base]{\color{textcolor}\rmfamily\fontsize{10.000000}{12.000000}\selectfont Pos}%
\end{pgfscope}%
\end{pgfpicture}%
\makeatother%
\endgroup%
	
&
	\vskip 0pt
	\hfil ROC Curve
	
	%% Creator: Matplotlib, PGF backend
%%
%% To include the figure in your LaTeX document, write
%%   \input{<filename>.pgf}
%%
%% Make sure the required packages are loaded in your preamble
%%   \usepackage{pgf}
%%
%% Also ensure that all the required font packages are loaded; for instance,
%% the lmodern package is sometimes necessary when using math font.
%%   \usepackage{lmodern}
%%
%% Figures using additional raster images can only be included by \input if
%% they are in the same directory as the main LaTeX file. For loading figures
%% from other directories you can use the `import` package
%%   \usepackage{import}
%%
%% and then include the figures with
%%   \import{<path to file>}{<filename>.pgf}
%%
%% Matplotlib used the following preamble
%%   
%%   \usepackage{fontspec}
%%   \makeatletter\@ifpackageloaded{underscore}{}{\usepackage[strings]{underscore}}\makeatother
%%
\begingroup%
\makeatletter%
\begin{pgfpicture}%
\pgfpathrectangle{\pgfpointorigin}{\pgfqpoint{2.221861in}{1.754444in}}%
\pgfusepath{use as bounding box, clip}%
\begin{pgfscope}%
\pgfsetbuttcap%
\pgfsetmiterjoin%
\definecolor{currentfill}{rgb}{1.000000,1.000000,1.000000}%
\pgfsetfillcolor{currentfill}%
\pgfsetlinewidth{0.000000pt}%
\definecolor{currentstroke}{rgb}{1.000000,1.000000,1.000000}%
\pgfsetstrokecolor{currentstroke}%
\pgfsetdash{}{0pt}%
\pgfpathmoveto{\pgfqpoint{0.000000in}{0.000000in}}%
\pgfpathlineto{\pgfqpoint{2.221861in}{0.000000in}}%
\pgfpathlineto{\pgfqpoint{2.221861in}{1.754444in}}%
\pgfpathlineto{\pgfqpoint{0.000000in}{1.754444in}}%
\pgfpathlineto{\pgfqpoint{0.000000in}{0.000000in}}%
\pgfpathclose%
\pgfusepath{fill}%
\end{pgfscope}%
\begin{pgfscope}%
\pgfsetbuttcap%
\pgfsetmiterjoin%
\definecolor{currentfill}{rgb}{1.000000,1.000000,1.000000}%
\pgfsetfillcolor{currentfill}%
\pgfsetlinewidth{0.000000pt}%
\definecolor{currentstroke}{rgb}{0.000000,0.000000,0.000000}%
\pgfsetstrokecolor{currentstroke}%
\pgfsetstrokeopacity{0.000000}%
\pgfsetdash{}{0pt}%
\pgfpathmoveto{\pgfqpoint{0.553581in}{0.499444in}}%
\pgfpathlineto{\pgfqpoint{2.103581in}{0.499444in}}%
\pgfpathlineto{\pgfqpoint{2.103581in}{1.654444in}}%
\pgfpathlineto{\pgfqpoint{0.553581in}{1.654444in}}%
\pgfpathlineto{\pgfqpoint{0.553581in}{0.499444in}}%
\pgfpathclose%
\pgfusepath{fill}%
\end{pgfscope}%
\begin{pgfscope}%
\pgfsetbuttcap%
\pgfsetroundjoin%
\definecolor{currentfill}{rgb}{0.000000,0.000000,0.000000}%
\pgfsetfillcolor{currentfill}%
\pgfsetlinewidth{0.803000pt}%
\definecolor{currentstroke}{rgb}{0.000000,0.000000,0.000000}%
\pgfsetstrokecolor{currentstroke}%
\pgfsetdash{}{0pt}%
\pgfsys@defobject{currentmarker}{\pgfqpoint{0.000000in}{-0.048611in}}{\pgfqpoint{0.000000in}{0.000000in}}{%
\pgfpathmoveto{\pgfqpoint{0.000000in}{0.000000in}}%
\pgfpathlineto{\pgfqpoint{0.000000in}{-0.048611in}}%
\pgfusepath{stroke,fill}%
}%
\begin{pgfscope}%
\pgfsys@transformshift{0.624035in}{0.499444in}%
\pgfsys@useobject{currentmarker}{}%
\end{pgfscope}%
\end{pgfscope}%
\begin{pgfscope}%
\definecolor{textcolor}{rgb}{0.000000,0.000000,0.000000}%
\pgfsetstrokecolor{textcolor}%
\pgfsetfillcolor{textcolor}%
\pgftext[x=0.624035in,y=0.402222in,,top]{\color{textcolor}\rmfamily\fontsize{10.000000}{12.000000}\selectfont \(\displaystyle {0.0}\)}%
\end{pgfscope}%
\begin{pgfscope}%
\pgfsetbuttcap%
\pgfsetroundjoin%
\definecolor{currentfill}{rgb}{0.000000,0.000000,0.000000}%
\pgfsetfillcolor{currentfill}%
\pgfsetlinewidth{0.803000pt}%
\definecolor{currentstroke}{rgb}{0.000000,0.000000,0.000000}%
\pgfsetstrokecolor{currentstroke}%
\pgfsetdash{}{0pt}%
\pgfsys@defobject{currentmarker}{\pgfqpoint{0.000000in}{-0.048611in}}{\pgfqpoint{0.000000in}{0.000000in}}{%
\pgfpathmoveto{\pgfqpoint{0.000000in}{0.000000in}}%
\pgfpathlineto{\pgfqpoint{0.000000in}{-0.048611in}}%
\pgfusepath{stroke,fill}%
}%
\begin{pgfscope}%
\pgfsys@transformshift{1.328581in}{0.499444in}%
\pgfsys@useobject{currentmarker}{}%
\end{pgfscope}%
\end{pgfscope}%
\begin{pgfscope}%
\definecolor{textcolor}{rgb}{0.000000,0.000000,0.000000}%
\pgfsetstrokecolor{textcolor}%
\pgfsetfillcolor{textcolor}%
\pgftext[x=1.328581in,y=0.402222in,,top]{\color{textcolor}\rmfamily\fontsize{10.000000}{12.000000}\selectfont \(\displaystyle {0.5}\)}%
\end{pgfscope}%
\begin{pgfscope}%
\pgfsetbuttcap%
\pgfsetroundjoin%
\definecolor{currentfill}{rgb}{0.000000,0.000000,0.000000}%
\pgfsetfillcolor{currentfill}%
\pgfsetlinewidth{0.803000pt}%
\definecolor{currentstroke}{rgb}{0.000000,0.000000,0.000000}%
\pgfsetstrokecolor{currentstroke}%
\pgfsetdash{}{0pt}%
\pgfsys@defobject{currentmarker}{\pgfqpoint{0.000000in}{-0.048611in}}{\pgfqpoint{0.000000in}{0.000000in}}{%
\pgfpathmoveto{\pgfqpoint{0.000000in}{0.000000in}}%
\pgfpathlineto{\pgfqpoint{0.000000in}{-0.048611in}}%
\pgfusepath{stroke,fill}%
}%
\begin{pgfscope}%
\pgfsys@transformshift{2.033126in}{0.499444in}%
\pgfsys@useobject{currentmarker}{}%
\end{pgfscope}%
\end{pgfscope}%
\begin{pgfscope}%
\definecolor{textcolor}{rgb}{0.000000,0.000000,0.000000}%
\pgfsetstrokecolor{textcolor}%
\pgfsetfillcolor{textcolor}%
\pgftext[x=2.033126in,y=0.402222in,,top]{\color{textcolor}\rmfamily\fontsize{10.000000}{12.000000}\selectfont \(\displaystyle {1.0}\)}%
\end{pgfscope}%
\begin{pgfscope}%
\definecolor{textcolor}{rgb}{0.000000,0.000000,0.000000}%
\pgfsetstrokecolor{textcolor}%
\pgfsetfillcolor{textcolor}%
\pgftext[x=1.328581in,y=0.223333in,,top]{\color{textcolor}\rmfamily\fontsize{10.000000}{12.000000}\selectfont False positive rate}%
\end{pgfscope}%
\begin{pgfscope}%
\pgfsetbuttcap%
\pgfsetroundjoin%
\definecolor{currentfill}{rgb}{0.000000,0.000000,0.000000}%
\pgfsetfillcolor{currentfill}%
\pgfsetlinewidth{0.803000pt}%
\definecolor{currentstroke}{rgb}{0.000000,0.000000,0.000000}%
\pgfsetstrokecolor{currentstroke}%
\pgfsetdash{}{0pt}%
\pgfsys@defobject{currentmarker}{\pgfqpoint{-0.048611in}{0.000000in}}{\pgfqpoint{-0.000000in}{0.000000in}}{%
\pgfpathmoveto{\pgfqpoint{-0.000000in}{0.000000in}}%
\pgfpathlineto{\pgfqpoint{-0.048611in}{0.000000in}}%
\pgfusepath{stroke,fill}%
}%
\begin{pgfscope}%
\pgfsys@transformshift{0.553581in}{0.551944in}%
\pgfsys@useobject{currentmarker}{}%
\end{pgfscope}%
\end{pgfscope}%
\begin{pgfscope}%
\definecolor{textcolor}{rgb}{0.000000,0.000000,0.000000}%
\pgfsetstrokecolor{textcolor}%
\pgfsetfillcolor{textcolor}%
\pgftext[x=0.278889in, y=0.503750in, left, base]{\color{textcolor}\rmfamily\fontsize{10.000000}{12.000000}\selectfont \(\displaystyle {0.0}\)}%
\end{pgfscope}%
\begin{pgfscope}%
\pgfsetbuttcap%
\pgfsetroundjoin%
\definecolor{currentfill}{rgb}{0.000000,0.000000,0.000000}%
\pgfsetfillcolor{currentfill}%
\pgfsetlinewidth{0.803000pt}%
\definecolor{currentstroke}{rgb}{0.000000,0.000000,0.000000}%
\pgfsetstrokecolor{currentstroke}%
\pgfsetdash{}{0pt}%
\pgfsys@defobject{currentmarker}{\pgfqpoint{-0.048611in}{0.000000in}}{\pgfqpoint{-0.000000in}{0.000000in}}{%
\pgfpathmoveto{\pgfqpoint{-0.000000in}{0.000000in}}%
\pgfpathlineto{\pgfqpoint{-0.048611in}{0.000000in}}%
\pgfusepath{stroke,fill}%
}%
\begin{pgfscope}%
\pgfsys@transformshift{0.553581in}{1.076944in}%
\pgfsys@useobject{currentmarker}{}%
\end{pgfscope}%
\end{pgfscope}%
\begin{pgfscope}%
\definecolor{textcolor}{rgb}{0.000000,0.000000,0.000000}%
\pgfsetstrokecolor{textcolor}%
\pgfsetfillcolor{textcolor}%
\pgftext[x=0.278889in, y=1.028750in, left, base]{\color{textcolor}\rmfamily\fontsize{10.000000}{12.000000}\selectfont \(\displaystyle {0.5}\)}%
\end{pgfscope}%
\begin{pgfscope}%
\pgfsetbuttcap%
\pgfsetroundjoin%
\definecolor{currentfill}{rgb}{0.000000,0.000000,0.000000}%
\pgfsetfillcolor{currentfill}%
\pgfsetlinewidth{0.803000pt}%
\definecolor{currentstroke}{rgb}{0.000000,0.000000,0.000000}%
\pgfsetstrokecolor{currentstroke}%
\pgfsetdash{}{0pt}%
\pgfsys@defobject{currentmarker}{\pgfqpoint{-0.048611in}{0.000000in}}{\pgfqpoint{-0.000000in}{0.000000in}}{%
\pgfpathmoveto{\pgfqpoint{-0.000000in}{0.000000in}}%
\pgfpathlineto{\pgfqpoint{-0.048611in}{0.000000in}}%
\pgfusepath{stroke,fill}%
}%
\begin{pgfscope}%
\pgfsys@transformshift{0.553581in}{1.601944in}%
\pgfsys@useobject{currentmarker}{}%
\end{pgfscope}%
\end{pgfscope}%
\begin{pgfscope}%
\definecolor{textcolor}{rgb}{0.000000,0.000000,0.000000}%
\pgfsetstrokecolor{textcolor}%
\pgfsetfillcolor{textcolor}%
\pgftext[x=0.278889in, y=1.553750in, left, base]{\color{textcolor}\rmfamily\fontsize{10.000000}{12.000000}\selectfont \(\displaystyle {1.0}\)}%
\end{pgfscope}%
\begin{pgfscope}%
\definecolor{textcolor}{rgb}{0.000000,0.000000,0.000000}%
\pgfsetstrokecolor{textcolor}%
\pgfsetfillcolor{textcolor}%
\pgftext[x=0.223333in,y=1.076944in,,bottom,rotate=90.000000]{\color{textcolor}\rmfamily\fontsize{10.000000}{12.000000}\selectfont True positive rate}%
\end{pgfscope}%
\begin{pgfscope}%
\pgfpathrectangle{\pgfqpoint{0.553581in}{0.499444in}}{\pgfqpoint{1.550000in}{1.155000in}}%
\pgfusepath{clip}%
\pgfsetbuttcap%
\pgfsetroundjoin%
\pgfsetlinewidth{1.505625pt}%
\definecolor{currentstroke}{rgb}{0.000000,0.000000,0.000000}%
\pgfsetstrokecolor{currentstroke}%
\pgfsetdash{{5.550000pt}{2.400000pt}}{0.000000pt}%
\pgfpathmoveto{\pgfqpoint{0.624035in}{0.551944in}}%
\pgfpathlineto{\pgfqpoint{2.033126in}{1.601944in}}%
\pgfusepath{stroke}%
\end{pgfscope}%
\begin{pgfscope}%
\pgfpathrectangle{\pgfqpoint{0.553581in}{0.499444in}}{\pgfqpoint{1.550000in}{1.155000in}}%
\pgfusepath{clip}%
\pgfsetrectcap%
\pgfsetroundjoin%
\pgfsetlinewidth{1.505625pt}%
\definecolor{currentstroke}{rgb}{0.000000,0.000000,0.000000}%
\pgfsetstrokecolor{currentstroke}%
\pgfsetdash{}{0pt}%
\pgfpathmoveto{\pgfqpoint{0.624035in}{0.551944in}}%
\pgfpathlineto{\pgfqpoint{0.628637in}{0.584231in}}%
\pgfpathlineto{\pgfqpoint{0.628843in}{0.585339in}}%
\pgfpathlineto{\pgfqpoint{0.629941in}{0.590703in}}%
\pgfpathlineto{\pgfqpoint{0.630194in}{0.591811in}}%
\pgfpathlineto{\pgfqpoint{0.631299in}{0.597846in}}%
\pgfpathlineto{\pgfqpoint{0.631477in}{0.598954in}}%
\pgfpathlineto{\pgfqpoint{0.632586in}{0.603629in}}%
\pgfpathlineto{\pgfqpoint{0.632823in}{0.604728in}}%
\pgfpathlineto{\pgfqpoint{0.633930in}{0.609757in}}%
\pgfpathlineto{\pgfqpoint{0.634275in}{0.610865in}}%
\pgfpathlineto{\pgfqpoint{0.635384in}{0.615502in}}%
\pgfpathlineto{\pgfqpoint{0.635652in}{0.616592in}}%
\pgfpathlineto{\pgfqpoint{0.636758in}{0.621099in}}%
\pgfpathlineto{\pgfqpoint{0.637059in}{0.622198in}}%
\pgfpathlineto{\pgfqpoint{0.638166in}{0.627115in}}%
\pgfpathlineto{\pgfqpoint{0.638424in}{0.628195in}}%
\pgfpathlineto{\pgfqpoint{0.639533in}{0.633075in}}%
\pgfpathlineto{\pgfqpoint{0.639852in}{0.634183in}}%
\pgfpathlineto{\pgfqpoint{0.640959in}{0.638765in}}%
\pgfpathlineto{\pgfqpoint{0.641341in}{0.639873in}}%
\pgfpathlineto{\pgfqpoint{0.642451in}{0.644679in}}%
\pgfpathlineto{\pgfqpoint{0.642694in}{0.645787in}}%
\pgfpathlineto{\pgfqpoint{0.643801in}{0.649707in}}%
\pgfpathlineto{\pgfqpoint{0.644174in}{0.650778in}}%
\pgfpathlineto{\pgfqpoint{0.645281in}{0.654820in}}%
\pgfpathlineto{\pgfqpoint{0.645586in}{0.655910in}}%
\pgfpathlineto{\pgfqpoint{0.646693in}{0.660398in}}%
\pgfpathlineto{\pgfqpoint{0.647014in}{0.661497in}}%
\pgfpathlineto{\pgfqpoint{0.648124in}{0.665259in}}%
\pgfpathlineto{\pgfqpoint{0.648485in}{0.666368in}}%
\pgfpathlineto{\pgfqpoint{0.649594in}{0.670130in}}%
\pgfpathlineto{\pgfqpoint{0.649909in}{0.671229in}}%
\pgfpathlineto{\pgfqpoint{0.651018in}{0.675429in}}%
\pgfpathlineto{\pgfqpoint{0.651339in}{0.676537in}}%
\pgfpathlineto{\pgfqpoint{0.652449in}{0.680541in}}%
\pgfpathlineto{\pgfqpoint{0.652742in}{0.681649in}}%
\pgfpathlineto{\pgfqpoint{0.653846in}{0.685365in}}%
\pgfpathlineto{\pgfqpoint{0.654156in}{0.686464in}}%
\pgfpathlineto{\pgfqpoint{0.655265in}{0.690496in}}%
\pgfpathlineto{\pgfqpoint{0.655617in}{0.691605in}}%
\pgfpathlineto{\pgfqpoint{0.656726in}{0.695460in}}%
\pgfpathlineto{\pgfqpoint{0.657167in}{0.696559in}}%
\pgfpathlineto{\pgfqpoint{0.658277in}{0.699744in}}%
\pgfpathlineto{\pgfqpoint{0.658598in}{0.700852in}}%
\pgfpathlineto{\pgfqpoint{0.659707in}{0.704568in}}%
\pgfpathlineto{\pgfqpoint{0.660014in}{0.705657in}}%
\pgfpathlineto{\pgfqpoint{0.661121in}{0.709094in}}%
\pgfpathlineto{\pgfqpoint{0.661384in}{0.710202in}}%
\pgfpathlineto{\pgfqpoint{0.662489in}{0.713880in}}%
\pgfpathlineto{\pgfqpoint{0.662493in}{0.713880in}}%
\pgfpathlineto{\pgfqpoint{0.662902in}{0.714988in}}%
\pgfpathlineto{\pgfqpoint{0.664011in}{0.718592in}}%
\pgfpathlineto{\pgfqpoint{0.664405in}{0.719691in}}%
\pgfpathlineto{\pgfqpoint{0.665512in}{0.723128in}}%
\pgfpathlineto{\pgfqpoint{0.665972in}{0.724236in}}%
\pgfpathlineto{\pgfqpoint{0.667081in}{0.727737in}}%
\pgfpathlineto{\pgfqpoint{0.667475in}{0.728845in}}%
\pgfpathlineto{\pgfqpoint{0.668584in}{0.732217in}}%
\pgfpathlineto{\pgfqpoint{0.668976in}{0.733325in}}%
\pgfpathlineto{\pgfqpoint{0.670083in}{0.736323in}}%
\pgfpathlineto{\pgfqpoint{0.670449in}{0.737432in}}%
\pgfpathlineto{\pgfqpoint{0.671549in}{0.740412in}}%
\pgfpathlineto{\pgfqpoint{0.671954in}{0.741520in}}%
\pgfpathlineto{\pgfqpoint{0.673050in}{0.744844in}}%
\pgfpathlineto{\pgfqpoint{0.673061in}{0.744844in}}%
\pgfpathlineto{\pgfqpoint{0.673516in}{0.745934in}}%
\pgfpathlineto{\pgfqpoint{0.674621in}{0.748961in}}%
\pgfpathlineto{\pgfqpoint{0.674961in}{0.750069in}}%
\pgfpathlineto{\pgfqpoint{0.676068in}{0.752658in}}%
\pgfpathlineto{\pgfqpoint{0.676464in}{0.753757in}}%
\pgfpathlineto{\pgfqpoint{0.677567in}{0.757137in}}%
\pgfpathlineto{\pgfqpoint{0.678022in}{0.758245in}}%
\pgfpathlineto{\pgfqpoint{0.679131in}{0.761095in}}%
\pgfpathlineto{\pgfqpoint{0.679546in}{0.762203in}}%
\pgfpathlineto{\pgfqpoint{0.683200in}{0.772177in}}%
\pgfpathlineto{\pgfqpoint{0.683704in}{0.773276in}}%
\pgfpathlineto{\pgfqpoint{0.684800in}{0.776116in}}%
\pgfpathlineto{\pgfqpoint{0.685304in}{0.777224in}}%
\pgfpathlineto{\pgfqpoint{0.686394in}{0.779776in}}%
\pgfpathlineto{\pgfqpoint{0.686896in}{0.780884in}}%
\pgfpathlineto{\pgfqpoint{0.687989in}{0.783631in}}%
\pgfpathlineto{\pgfqpoint{0.688416in}{0.784739in}}%
\pgfpathlineto{\pgfqpoint{0.689525in}{0.787459in}}%
\pgfpathlineto{\pgfqpoint{0.689971in}{0.788567in}}%
\pgfpathlineto{\pgfqpoint{0.691080in}{0.791119in}}%
\pgfpathlineto{\pgfqpoint{0.691556in}{0.792227in}}%
\pgfpathlineto{\pgfqpoint{0.692663in}{0.794937in}}%
\pgfpathlineto{\pgfqpoint{0.693207in}{0.796036in}}%
\pgfpathlineto{\pgfqpoint{0.694314in}{0.798764in}}%
\pgfpathlineto{\pgfqpoint{0.694737in}{0.799872in}}%
\pgfpathlineto{\pgfqpoint{0.695839in}{0.802480in}}%
\pgfpathlineto{\pgfqpoint{0.696317in}{0.803579in}}%
\pgfpathlineto{\pgfqpoint{0.697410in}{0.806298in}}%
\pgfpathlineto{\pgfqpoint{0.697961in}{0.807406in}}%
\pgfpathlineto{\pgfqpoint{0.699061in}{0.810098in}}%
\pgfpathlineto{\pgfqpoint{0.699577in}{0.811206in}}%
\pgfpathlineto{\pgfqpoint{0.700687in}{0.813795in}}%
\pgfpathlineto{\pgfqpoint{0.701158in}{0.814903in}}%
\pgfpathlineto{\pgfqpoint{0.702265in}{0.817268in}}%
\pgfpathlineto{\pgfqpoint{0.702734in}{0.818376in}}%
\pgfpathlineto{\pgfqpoint{0.703843in}{0.820472in}}%
\pgfpathlineto{\pgfqpoint{0.704411in}{0.821580in}}%
\pgfpathlineto{\pgfqpoint{0.705520in}{0.824197in}}%
\pgfpathlineto{\pgfqpoint{0.705973in}{0.825305in}}%
\pgfpathlineto{\pgfqpoint{0.707082in}{0.827708in}}%
\pgfpathlineto{\pgfqpoint{0.707467in}{0.828816in}}%
\pgfpathlineto{\pgfqpoint{0.708576in}{0.830976in}}%
\pgfpathlineto{\pgfqpoint{0.709043in}{0.832075in}}%
\pgfpathlineto{\pgfqpoint{0.710150in}{0.834673in}}%
\pgfpathlineto{\pgfqpoint{0.710657in}{0.835763in}}%
\pgfpathlineto{\pgfqpoint{0.711763in}{0.838343in}}%
\pgfpathlineto{\pgfqpoint{0.712202in}{0.839432in}}%
\pgfpathlineto{\pgfqpoint{0.713302in}{0.841788in}}%
\pgfpathlineto{\pgfqpoint{0.713830in}{0.842887in}}%
\pgfpathlineto{\pgfqpoint{0.714932in}{0.845457in}}%
\pgfpathlineto{\pgfqpoint{0.715481in}{0.846566in}}%
\pgfpathlineto{\pgfqpoint{0.716562in}{0.848558in}}%
\pgfpathlineto{\pgfqpoint{0.716569in}{0.848558in}}%
\pgfpathlineto{\pgfqpoint{0.717057in}{0.849657in}}%
\pgfpathlineto{\pgfqpoint{0.718161in}{0.852023in}}%
\pgfpathlineto{\pgfqpoint{0.718616in}{0.853122in}}%
\pgfpathlineto{\pgfqpoint{0.719723in}{0.855301in}}%
\pgfpathlineto{\pgfqpoint{0.720246in}{0.856390in}}%
\pgfpathlineto{\pgfqpoint{0.721346in}{0.858690in}}%
\pgfpathlineto{\pgfqpoint{0.721891in}{0.859789in}}%
\pgfpathlineto{\pgfqpoint{0.723000in}{0.862117in}}%
\pgfpathlineto{\pgfqpoint{0.723553in}{0.863207in}}%
\pgfpathlineto{\pgfqpoint{0.724658in}{0.865163in}}%
\pgfpathlineto{\pgfqpoint{0.725162in}{0.866262in}}%
\pgfpathlineto{\pgfqpoint{0.726269in}{0.868208in}}%
\pgfpathlineto{\pgfqpoint{0.726858in}{0.869307in}}%
\pgfpathlineto{\pgfqpoint{0.727960in}{0.871505in}}%
\pgfpathlineto{\pgfqpoint{0.728504in}{0.872603in}}%
\pgfpathlineto{\pgfqpoint{0.729602in}{0.874717in}}%
\pgfpathlineto{\pgfqpoint{0.730237in}{0.875826in}}%
\pgfpathlineto{\pgfqpoint{0.731340in}{0.878135in}}%
\pgfpathlineto{\pgfqpoint{0.731940in}{0.879243in}}%
\pgfpathlineto{\pgfqpoint{0.733049in}{0.881450in}}%
\pgfpathlineto{\pgfqpoint{0.733521in}{0.882549in}}%
\pgfpathlineto{\pgfqpoint{0.734630in}{0.884486in}}%
\pgfpathlineto{\pgfqpoint{0.735254in}{0.885594in}}%
\pgfpathlineto{\pgfqpoint{0.736361in}{0.887541in}}%
\pgfpathlineto{\pgfqpoint{0.736896in}{0.888621in}}%
\pgfpathlineto{\pgfqpoint{0.738005in}{0.890465in}}%
\pgfpathlineto{\pgfqpoint{0.738636in}{0.891573in}}%
\pgfpathlineto{\pgfqpoint{0.739738in}{0.893603in}}%
\pgfpathlineto{\pgfqpoint{0.740367in}{0.894711in}}%
\pgfpathlineto{\pgfqpoint{0.741467in}{0.896984in}}%
\pgfpathlineto{\pgfqpoint{0.742098in}{0.898092in}}%
\pgfpathlineto{\pgfqpoint{0.743202in}{0.900252in}}%
\pgfpathlineto{\pgfqpoint{0.743913in}{0.901351in}}%
\pgfpathlineto{\pgfqpoint{0.745022in}{0.903353in}}%
\pgfpathlineto{\pgfqpoint{0.745627in}{0.904462in}}%
\pgfpathlineto{\pgfqpoint{0.746706in}{0.906436in}}%
\pgfpathlineto{\pgfqpoint{0.746732in}{0.906436in}}%
\pgfpathlineto{\pgfqpoint{0.747410in}{0.907544in}}%
\pgfpathlineto{\pgfqpoint{0.748514in}{0.909546in}}%
\pgfpathlineto{\pgfqpoint{0.749098in}{0.910645in}}%
\pgfpathlineto{\pgfqpoint{0.750208in}{0.912610in}}%
\pgfpathlineto{\pgfqpoint{0.750829in}{0.913700in}}%
\pgfpathlineto{\pgfqpoint{0.751936in}{0.915627in}}%
\pgfpathlineto{\pgfqpoint{0.752628in}{0.916726in}}%
\pgfpathlineto{\pgfqpoint{0.753737in}{0.918281in}}%
\pgfpathlineto{\pgfqpoint{0.754382in}{0.919390in}}%
\pgfpathlineto{\pgfqpoint{0.755492in}{0.921429in}}%
\pgfpathlineto{\pgfqpoint{0.756176in}{0.922519in}}%
\pgfpathlineto{\pgfqpoint{0.757281in}{0.924391in}}%
\pgfpathlineto{\pgfqpoint{0.758001in}{0.925489in}}%
\pgfpathlineto{\pgfqpoint{0.759106in}{0.927743in}}%
\pgfpathlineto{\pgfqpoint{0.759868in}{0.928851in}}%
\pgfpathlineto{\pgfqpoint{0.760977in}{0.930947in}}%
\pgfpathlineto{\pgfqpoint{0.761657in}{0.932045in}}%
\pgfpathlineto{\pgfqpoint{0.762755in}{0.934113in}}%
\pgfpathlineto{\pgfqpoint{0.763294in}{0.935221in}}%
\pgfpathlineto{\pgfqpoint{0.764399in}{0.937056in}}%
\pgfpathlineto{\pgfqpoint{0.765039in}{0.938164in}}%
\pgfpathlineto{\pgfqpoint{0.766139in}{0.939803in}}%
\pgfpathlineto{\pgfqpoint{0.766149in}{0.939803in}}%
\pgfpathlineto{\pgfqpoint{0.766606in}{0.940911in}}%
\pgfpathlineto{\pgfqpoint{0.767715in}{0.942327in}}%
\pgfpathlineto{\pgfqpoint{0.768438in}{0.943435in}}%
\pgfpathlineto{\pgfqpoint{0.769545in}{0.945176in}}%
\pgfpathlineto{\pgfqpoint{0.770201in}{0.946266in}}%
\pgfpathlineto{\pgfqpoint{0.771308in}{0.948016in}}%
\pgfpathlineto{\pgfqpoint{0.771974in}{0.949115in}}%
\pgfpathlineto{\pgfqpoint{0.773084in}{0.950689in}}%
\pgfpathlineto{\pgfqpoint{0.773715in}{0.951788in}}%
\pgfpathlineto{\pgfqpoint{0.774824in}{0.953585in}}%
\pgfpathlineto{\pgfqpoint{0.775593in}{0.954675in}}%
\pgfpathlineto{\pgfqpoint{0.776695in}{0.956454in}}%
\pgfpathlineto{\pgfqpoint{0.777364in}{0.957553in}}%
\pgfpathlineto{\pgfqpoint{0.778473in}{0.959368in}}%
\pgfpathlineto{\pgfqpoint{0.779149in}{0.960477in}}%
\pgfpathlineto{\pgfqpoint{0.780256in}{0.962097in}}%
\pgfpathlineto{\pgfqpoint{0.780858in}{0.963205in}}%
\pgfpathlineto{\pgfqpoint{0.781951in}{0.964984in}}%
\pgfpathlineto{\pgfqpoint{0.782610in}{0.966092in}}%
\pgfpathlineto{\pgfqpoint{0.783720in}{0.967834in}}%
\pgfpathlineto{\pgfqpoint{0.784477in}{0.968942in}}%
\pgfpathlineto{\pgfqpoint{0.785584in}{0.970730in}}%
\pgfpathlineto{\pgfqpoint{0.786220in}{0.971838in}}%
\pgfpathlineto{\pgfqpoint{0.787329in}{0.973701in}}%
\pgfpathlineto{\pgfqpoint{0.788056in}{0.974809in}}%
\pgfpathlineto{\pgfqpoint{0.789165in}{0.976345in}}%
\pgfpathlineto{\pgfqpoint{0.789881in}{0.977453in}}%
\pgfpathlineto{\pgfqpoint{0.790985in}{0.978990in}}%
\pgfpathlineto{\pgfqpoint{0.791717in}{0.980089in}}%
\pgfpathlineto{\pgfqpoint{0.792826in}{0.981849in}}%
\pgfpathlineto{\pgfqpoint{0.793560in}{0.982957in}}%
\pgfpathlineto{\pgfqpoint{0.794665in}{0.984782in}}%
\pgfpathlineto{\pgfqpoint{0.795357in}{0.985881in}}%
\pgfpathlineto{\pgfqpoint{0.796464in}{0.987483in}}%
\pgfpathlineto{\pgfqpoint{0.797207in}{0.988591in}}%
\pgfpathlineto{\pgfqpoint{0.798312in}{0.990240in}}%
\pgfpathlineto{\pgfqpoint{0.799107in}{0.991348in}}%
\pgfpathlineto{\pgfqpoint{0.800216in}{0.993024in}}%
\pgfpathlineto{\pgfqpoint{0.800990in}{0.994132in}}%
\pgfpathlineto{\pgfqpoint{0.802093in}{0.995622in}}%
\pgfpathlineto{\pgfqpoint{0.802799in}{0.996703in}}%
\pgfpathlineto{\pgfqpoint{0.803899in}{0.998397in}}%
\pgfpathlineto{\pgfqpoint{0.804715in}{0.999506in}}%
\pgfpathlineto{\pgfqpoint{0.805817in}{1.001107in}}%
\pgfpathlineto{\pgfqpoint{0.806765in}{1.002216in}}%
\pgfpathlineto{\pgfqpoint{0.807874in}{1.003780in}}%
\pgfpathlineto{\pgfqpoint{0.808622in}{1.004879in}}%
\pgfpathlineto{\pgfqpoint{0.809731in}{1.006350in}}%
\pgfpathlineto{\pgfqpoint{0.810512in}{1.007459in}}%
\pgfpathlineto{\pgfqpoint{0.811622in}{1.008976in}}%
\pgfpathlineto{\pgfqpoint{0.812522in}{1.010085in}}%
\pgfpathlineto{\pgfqpoint{0.813632in}{1.011351in}}%
\pgfpathlineto{\pgfqpoint{0.814485in}{1.012459in}}%
\pgfpathlineto{\pgfqpoint{0.815581in}{1.014061in}}%
\pgfpathlineto{\pgfqpoint{0.816406in}{1.015169in}}%
\pgfpathlineto{\pgfqpoint{0.817499in}{1.016920in}}%
\pgfpathlineto{\pgfqpoint{0.818371in}{1.018028in}}%
\pgfpathlineto{\pgfqpoint{0.819478in}{1.019304in}}%
\pgfpathlineto{\pgfqpoint{0.820325in}{1.020412in}}%
\pgfpathlineto{\pgfqpoint{0.821434in}{1.022182in}}%
\pgfpathlineto{\pgfqpoint{0.822405in}{1.023290in}}%
\pgfpathlineto{\pgfqpoint{0.823515in}{1.024743in}}%
\pgfpathlineto{\pgfqpoint{0.824289in}{1.025851in}}%
\pgfpathlineto{\pgfqpoint{0.825396in}{1.027294in}}%
\pgfpathlineto{\pgfqpoint{0.826212in}{1.028402in}}%
\pgfpathlineto{\pgfqpoint{0.827319in}{1.029809in}}%
\pgfpathlineto{\pgfqpoint{0.828184in}{1.030917in}}%
\pgfpathlineto{\pgfqpoint{0.829282in}{1.032528in}}%
\pgfpathlineto{\pgfqpoint{0.830018in}{1.033627in}}%
\pgfpathlineto{\pgfqpoint{0.831125in}{1.035294in}}%
\pgfpathlineto{\pgfqpoint{0.831946in}{1.036393in}}%
\pgfpathlineto{\pgfqpoint{0.833046in}{1.037873in}}%
\pgfpathlineto{\pgfqpoint{0.834000in}{1.038982in}}%
\pgfpathlineto{\pgfqpoint{0.835105in}{1.040388in}}%
\pgfpathlineto{\pgfqpoint{0.835879in}{1.041468in}}%
\pgfpathlineto{\pgfqpoint{0.836986in}{1.042753in}}%
\pgfpathlineto{\pgfqpoint{0.837870in}{1.043852in}}%
\pgfpathlineto{\pgfqpoint{0.838970in}{1.045240in}}%
\pgfpathlineto{\pgfqpoint{0.839807in}{1.046348in}}%
\pgfpathlineto{\pgfqpoint{0.840917in}{1.047717in}}%
\pgfpathlineto{\pgfqpoint{0.841740in}{1.048825in}}%
\pgfpathlineto{\pgfqpoint{0.842845in}{1.049905in}}%
\pgfpathlineto{\pgfqpoint{0.843785in}{1.051004in}}%
\pgfpathlineto{\pgfqpoint{0.844883in}{1.052503in}}%
\pgfpathlineto{\pgfqpoint{0.845811in}{1.053612in}}%
\pgfpathlineto{\pgfqpoint{0.846904in}{1.055334in}}%
\pgfpathlineto{\pgfqpoint{0.847847in}{1.056433in}}%
\pgfpathlineto{\pgfqpoint{0.848952in}{1.057923in}}%
\pgfpathlineto{\pgfqpoint{0.849876in}{1.059031in}}%
\pgfpathlineto{\pgfqpoint{0.850980in}{1.060289in}}%
\pgfpathlineto{\pgfqpoint{0.851776in}{1.061397in}}%
\pgfpathlineto{\pgfqpoint{0.852873in}{1.062905in}}%
\pgfpathlineto{\pgfqpoint{0.853839in}{1.064004in}}%
\pgfpathlineto{\pgfqpoint{0.854949in}{1.065606in}}%
\pgfpathlineto{\pgfqpoint{0.855920in}{1.066714in}}%
\pgfpathlineto{\pgfqpoint{0.857015in}{1.068139in}}%
\pgfpathlineto{\pgfqpoint{0.857883in}{1.069238in}}%
\pgfpathlineto{\pgfqpoint{0.858980in}{1.070830in}}%
\pgfpathlineto{\pgfqpoint{0.859871in}{1.071939in}}%
\pgfpathlineto{\pgfqpoint{0.860981in}{1.073475in}}%
\pgfpathlineto{\pgfqpoint{0.862095in}{1.074574in}}%
\pgfpathlineto{\pgfqpoint{0.863195in}{1.075897in}}%
\pgfpathlineto{\pgfqpoint{0.864156in}{1.077005in}}%
\pgfpathlineto{\pgfqpoint{0.865266in}{1.078364in}}%
\pgfpathlineto{\pgfqpoint{0.866023in}{1.079473in}}%
\pgfpathlineto{\pgfqpoint{0.867133in}{1.080814in}}%
\pgfpathlineto{\pgfqpoint{0.868186in}{1.081922in}}%
\pgfpathlineto{\pgfqpoint{0.869288in}{1.083123in}}%
\pgfpathlineto{\pgfqpoint{0.870242in}{1.084222in}}%
\pgfpathlineto{\pgfqpoint{0.871349in}{1.085805in}}%
\pgfpathlineto{\pgfqpoint{0.872290in}{1.086895in}}%
\pgfpathlineto{\pgfqpoint{0.873397in}{1.088347in}}%
\pgfpathlineto{\pgfqpoint{0.874377in}{1.089456in}}%
\pgfpathlineto{\pgfqpoint{0.875435in}{1.090862in}}%
\pgfpathlineto{\pgfqpoint{0.876418in}{1.091970in}}%
\pgfpathlineto{\pgfqpoint{0.877499in}{1.093078in}}%
\pgfpathlineto{\pgfqpoint{0.878578in}{1.094186in}}%
\pgfpathlineto{\pgfqpoint{0.879687in}{1.095341in}}%
\pgfpathlineto{\pgfqpoint{0.880559in}{1.096449in}}%
\pgfpathlineto{\pgfqpoint{0.881659in}{1.097623in}}%
\pgfpathlineto{\pgfqpoint{0.881669in}{1.097623in}}%
\pgfpathlineto{\pgfqpoint{0.882680in}{1.098731in}}%
\pgfpathlineto{\pgfqpoint{0.883768in}{1.100016in}}%
\pgfpathlineto{\pgfqpoint{0.883784in}{1.100016in}}%
\pgfpathlineto{\pgfqpoint{0.884659in}{1.101115in}}%
\pgfpathlineto{\pgfqpoint{0.885745in}{1.102512in}}%
\pgfpathlineto{\pgfqpoint{0.886889in}{1.103620in}}%
\pgfpathlineto{\pgfqpoint{0.887987in}{1.104840in}}%
\pgfpathlineto{\pgfqpoint{0.889052in}{1.105948in}}%
\pgfpathlineto{\pgfqpoint{0.890147in}{1.106945in}}%
\pgfpathlineto{\pgfqpoint{0.891144in}{1.108043in}}%
\pgfpathlineto{\pgfqpoint{0.892239in}{1.109366in}}%
\pgfpathlineto{\pgfqpoint{0.893177in}{1.110474in}}%
\pgfpathlineto{\pgfqpoint{0.894270in}{1.111694in}}%
\pgfpathlineto{\pgfqpoint{0.895241in}{1.112802in}}%
\pgfpathlineto{\pgfqpoint{0.896350in}{1.114013in}}%
\pgfpathlineto{\pgfqpoint{0.897302in}{1.115121in}}%
\pgfpathlineto{\pgfqpoint{0.898409in}{1.116415in}}%
\pgfpathlineto{\pgfqpoint{0.899409in}{1.117514in}}%
\pgfpathlineto{\pgfqpoint{0.900513in}{1.118809in}}%
\pgfpathlineto{\pgfqpoint{0.901522in}{1.119908in}}%
\pgfpathlineto{\pgfqpoint{0.902612in}{1.121323in}}%
\pgfpathlineto{\pgfqpoint{0.903787in}{1.122431in}}%
\pgfpathlineto{\pgfqpoint{0.904875in}{1.123782in}}%
\pgfpathlineto{\pgfqpoint{0.906090in}{1.124890in}}%
\pgfpathlineto{\pgfqpoint{0.907179in}{1.126249in}}%
\pgfpathlineto{\pgfqpoint{0.908023in}{1.127348in}}%
\pgfpathlineto{\pgfqpoint{0.909120in}{1.128568in}}%
\pgfpathlineto{\pgfqpoint{0.909130in}{1.128568in}}%
\pgfpathlineto{\pgfqpoint{0.910298in}{1.129667in}}%
\pgfpathlineto{\pgfqpoint{0.911402in}{1.131055in}}%
\pgfpathlineto{\pgfqpoint{0.912502in}{1.132163in}}%
\pgfpathlineto{\pgfqpoint{0.913602in}{1.133355in}}%
\pgfpathlineto{\pgfqpoint{0.913612in}{1.133355in}}%
\pgfpathlineto{\pgfqpoint{0.914700in}{1.134463in}}%
\pgfpathlineto{\pgfqpoint{0.915807in}{1.135664in}}%
\pgfpathlineto{\pgfqpoint{0.916862in}{1.136773in}}%
\pgfpathlineto{\pgfqpoint{0.917967in}{1.137704in}}%
\pgfpathlineto{\pgfqpoint{0.918900in}{1.138812in}}%
\pgfpathlineto{\pgfqpoint{0.920010in}{1.140051in}}%
\pgfpathlineto{\pgfqpoint{0.921035in}{1.141159in}}%
\pgfpathlineto{\pgfqpoint{0.922144in}{1.142183in}}%
\pgfpathlineto{\pgfqpoint{0.923335in}{1.143291in}}%
\pgfpathlineto{\pgfqpoint{0.924445in}{1.144558in}}%
\pgfpathlineto{\pgfqpoint{0.925636in}{1.145666in}}%
\pgfpathlineto{\pgfqpoint{0.926743in}{1.146663in}}%
\pgfpathlineto{\pgfqpoint{0.927770in}{1.147771in}}%
\pgfpathlineto{\pgfqpoint{0.928873in}{1.148860in}}%
\pgfpathlineto{\pgfqpoint{0.929832in}{1.149969in}}%
\pgfpathlineto{\pgfqpoint{0.930918in}{1.151067in}}%
\pgfpathlineto{\pgfqpoint{0.930932in}{1.151067in}}%
\pgfpathlineto{\pgfqpoint{0.932069in}{1.152176in}}%
\pgfpathlineto{\pgfqpoint{0.933171in}{1.153368in}}%
\pgfpathlineto{\pgfqpoint{0.933179in}{1.153368in}}%
\pgfpathlineto{\pgfqpoint{0.934215in}{1.154457in}}%
\pgfpathlineto{\pgfqpoint{0.935313in}{1.155714in}}%
\pgfpathlineto{\pgfqpoint{0.936438in}{1.156823in}}%
\pgfpathlineto{\pgfqpoint{0.937545in}{1.157968in}}%
\pgfpathlineto{\pgfqpoint{0.938643in}{1.159058in}}%
\pgfpathlineto{\pgfqpoint{0.938643in}{1.159067in}}%
\pgfpathlineto{\pgfqpoint{0.939750in}{1.160333in}}%
\pgfpathlineto{\pgfqpoint{0.940754in}{1.161442in}}%
\pgfpathlineto{\pgfqpoint{0.941842in}{1.162503in}}%
\pgfpathlineto{\pgfqpoint{0.942914in}{1.163602in}}%
\pgfpathlineto{\pgfqpoint{0.944014in}{1.164701in}}%
\pgfpathlineto{\pgfqpoint{0.945449in}{1.165809in}}%
\pgfpathlineto{\pgfqpoint{0.946558in}{1.166703in}}%
\pgfpathlineto{\pgfqpoint{0.947783in}{1.167811in}}%
\pgfpathlineto{\pgfqpoint{0.948890in}{1.168854in}}%
\pgfpathlineto{\pgfqpoint{0.950004in}{1.169963in}}%
\pgfpathlineto{\pgfqpoint{0.951104in}{1.170903in}}%
\pgfpathlineto{\pgfqpoint{0.952368in}{1.171974in}}%
\pgfpathlineto{\pgfqpoint{0.953475in}{1.173036in}}%
\pgfpathlineto{\pgfqpoint{0.954633in}{1.174144in}}%
\pgfpathlineto{\pgfqpoint{0.955712in}{1.175224in}}%
\pgfpathlineto{\pgfqpoint{0.956887in}{1.176323in}}%
\pgfpathlineto{\pgfqpoint{0.957996in}{1.177655in}}%
\pgfpathlineto{\pgfqpoint{0.958986in}{1.178763in}}%
\pgfpathlineto{\pgfqpoint{0.960067in}{1.179862in}}%
\pgfpathlineto{\pgfqpoint{0.961310in}{1.180970in}}%
\pgfpathlineto{\pgfqpoint{0.962415in}{1.182106in}}%
\pgfpathlineto{\pgfqpoint{0.963503in}{1.183214in}}%
\pgfpathlineto{\pgfqpoint{0.964613in}{1.184434in}}%
\pgfpathlineto{\pgfqpoint{0.965818in}{1.185543in}}%
\pgfpathlineto{\pgfqpoint{0.966918in}{1.186632in}}%
\pgfpathlineto{\pgfqpoint{0.967955in}{1.187740in}}%
\pgfpathlineto{\pgfqpoint{0.969052in}{1.188737in}}%
\pgfpathlineto{\pgfqpoint{0.970276in}{1.189845in}}%
\pgfpathlineto{\pgfqpoint{0.971372in}{1.190767in}}%
\pgfpathlineto{\pgfqpoint{0.972720in}{1.191875in}}%
\pgfpathlineto{\pgfqpoint{0.973827in}{1.192955in}}%
\pgfpathlineto{\pgfqpoint{0.975105in}{1.194064in}}%
\pgfpathlineto{\pgfqpoint{0.976212in}{1.195293in}}%
\pgfpathlineto{\pgfqpoint{0.977720in}{1.196401in}}%
\pgfpathlineto{\pgfqpoint{0.979203in}{1.197779in}}%
\pgfpathlineto{\pgfqpoint{0.980589in}{1.198887in}}%
\pgfpathlineto{\pgfqpoint{0.981698in}{1.199996in}}%
\pgfpathlineto{\pgfqpoint{0.982857in}{1.201095in}}%
\pgfpathlineto{\pgfqpoint{0.983964in}{1.202287in}}%
\pgfpathlineto{\pgfqpoint{0.985319in}{1.203385in}}%
\pgfpathlineto{\pgfqpoint{0.986410in}{1.204242in}}%
\pgfpathlineto{\pgfqpoint{0.987730in}{1.205350in}}%
\pgfpathlineto{\pgfqpoint{0.988837in}{1.206300in}}%
\pgfpathlineto{\pgfqpoint{0.990017in}{1.207408in}}%
\pgfpathlineto{\pgfqpoint{0.991089in}{1.208144in}}%
\pgfpathlineto{\pgfqpoint{0.992522in}{1.209252in}}%
\pgfpathlineto{\pgfqpoint{0.993631in}{1.210277in}}%
\pgfpathlineto{\pgfqpoint{0.995038in}{1.211385in}}%
\pgfpathlineto{\pgfqpoint{0.996147in}{1.212512in}}%
\pgfpathlineto{\pgfqpoint{0.997440in}{1.213611in}}%
\pgfpathlineto{\pgfqpoint{0.998542in}{1.214607in}}%
\pgfpathlineto{\pgfqpoint{0.999853in}{1.215715in}}%
\pgfpathlineto{\pgfqpoint{1.000962in}{1.216646in}}%
\pgfpathlineto{\pgfqpoint{1.002161in}{1.217755in}}%
\pgfpathlineto{\pgfqpoint{1.003270in}{1.218472in}}%
\pgfpathlineto{\pgfqpoint{1.004337in}{1.219580in}}%
\pgfpathlineto{\pgfqpoint{1.005432in}{1.220688in}}%
\pgfpathlineto{\pgfqpoint{1.006406in}{1.221796in}}%
\pgfpathlineto{\pgfqpoint{1.007513in}{1.222858in}}%
\pgfpathlineto{\pgfqpoint{1.008721in}{1.223966in}}%
\pgfpathlineto{\pgfqpoint{1.009825in}{1.225000in}}%
\pgfpathlineto{\pgfqpoint{1.011289in}{1.226108in}}%
\pgfpathlineto{\pgfqpoint{1.012384in}{1.227170in}}%
\pgfpathlineto{\pgfqpoint{1.013920in}{1.228259in}}%
\pgfpathlineto{\pgfqpoint{1.015027in}{1.229470in}}%
\pgfpathlineto{\pgfqpoint{1.016392in}{1.230569in}}%
\pgfpathlineto{\pgfqpoint{1.017494in}{1.231584in}}%
\pgfpathlineto{\pgfqpoint{1.018883in}{1.232692in}}%
\pgfpathlineto{\pgfqpoint{1.019966in}{1.233558in}}%
\pgfpathlineto{\pgfqpoint{1.021383in}{1.234666in}}%
\pgfpathlineto{\pgfqpoint{1.022487in}{1.235728in}}%
\pgfpathlineto{\pgfqpoint{1.023784in}{1.236836in}}%
\pgfpathlineto{\pgfqpoint{1.024873in}{1.237488in}}%
\pgfpathlineto{\pgfqpoint{1.026188in}{1.238596in}}%
\pgfpathlineto{\pgfqpoint{1.027286in}{1.239378in}}%
\pgfpathlineto{\pgfqpoint{1.028834in}{1.240487in}}%
\pgfpathlineto{\pgfqpoint{1.029936in}{1.241446in}}%
\pgfpathlineto{\pgfqpoint{1.031313in}{1.242554in}}%
\pgfpathlineto{\pgfqpoint{1.032378in}{1.243383in}}%
\pgfpathlineto{\pgfqpoint{1.033904in}{1.244491in}}%
\pgfpathlineto{\pgfqpoint{1.035011in}{1.245487in}}%
\pgfpathlineto{\pgfqpoint{1.036322in}{1.246596in}}%
\pgfpathlineto{\pgfqpoint{1.037420in}{1.247518in}}%
\pgfpathlineto{\pgfqpoint{1.038954in}{1.248626in}}%
\pgfpathlineto{\pgfqpoint{1.040063in}{1.249687in}}%
\pgfpathlineto{\pgfqpoint{1.041280in}{1.250786in}}%
\pgfpathlineto{\pgfqpoint{1.042369in}{1.251745in}}%
\pgfpathlineto{\pgfqpoint{1.043682in}{1.252854in}}%
\pgfpathlineto{\pgfqpoint{1.044787in}{1.253766in}}%
\pgfpathlineto{\pgfqpoint{1.046128in}{1.254875in}}%
\pgfpathlineto{\pgfqpoint{1.047212in}{1.255703in}}%
\pgfpathlineto{\pgfqpoint{1.048642in}{1.256812in}}%
\pgfpathlineto{\pgfqpoint{1.049747in}{1.257706in}}%
\pgfpathlineto{\pgfqpoint{1.051006in}{1.258814in}}%
\pgfpathlineto{\pgfqpoint{1.052102in}{1.259708in}}%
\pgfpathlineto{\pgfqpoint{1.053657in}{1.260816in}}%
\pgfpathlineto{\pgfqpoint{1.054747in}{1.261915in}}%
\pgfpathlineto{\pgfqpoint{1.056464in}{1.263023in}}%
\pgfpathlineto{\pgfqpoint{1.057568in}{1.263936in}}%
\pgfpathlineto{\pgfqpoint{1.059039in}{1.265044in}}%
\pgfpathlineto{\pgfqpoint{1.060148in}{1.265863in}}%
\pgfpathlineto{\pgfqpoint{1.061731in}{1.266972in}}%
\pgfpathlineto{\pgfqpoint{1.062838in}{1.267856in}}%
\pgfpathlineto{\pgfqpoint{1.064351in}{1.268964in}}%
\pgfpathlineto{\pgfqpoint{1.065458in}{1.269803in}}%
\pgfpathlineto{\pgfqpoint{1.066771in}{1.270892in}}%
\pgfpathlineto{\pgfqpoint{1.067867in}{1.271758in}}%
\pgfpathlineto{\pgfqpoint{1.067881in}{1.271758in}}%
\pgfpathlineto{\pgfqpoint{1.069408in}{1.272866in}}%
\pgfpathlineto{\pgfqpoint{1.070505in}{1.273695in}}%
\pgfpathlineto{\pgfqpoint{1.071884in}{1.274785in}}%
\pgfpathlineto{\pgfqpoint{1.072991in}{1.275725in}}%
\pgfpathlineto{\pgfqpoint{1.074276in}{1.276834in}}%
\pgfpathlineto{\pgfqpoint{1.075379in}{1.277700in}}%
\pgfpathlineto{\pgfqpoint{1.076842in}{1.278808in}}%
\pgfpathlineto{\pgfqpoint{1.077951in}{1.279609in}}%
\pgfpathlineto{\pgfqpoint{1.079598in}{1.280717in}}%
\pgfpathlineto{\pgfqpoint{1.080707in}{1.281508in}}%
\pgfpathlineto{\pgfqpoint{1.081863in}{1.282617in}}%
\pgfpathlineto{\pgfqpoint{1.082956in}{1.283539in}}%
\pgfpathlineto{\pgfqpoint{1.084448in}{1.284647in}}%
\pgfpathlineto{\pgfqpoint{1.085522in}{1.285280in}}%
\pgfpathlineto{\pgfqpoint{1.086929in}{1.286388in}}%
\pgfpathlineto{\pgfqpoint{1.088039in}{1.287273in}}%
\pgfpathlineto{\pgfqpoint{1.089615in}{1.288381in}}%
\pgfpathlineto{\pgfqpoint{1.090710in}{1.289154in}}%
\pgfpathlineto{\pgfqpoint{1.092162in}{1.290262in}}%
\pgfpathlineto{\pgfqpoint{1.093240in}{1.291091in}}%
\pgfpathlineto{\pgfqpoint{1.093271in}{1.291091in}}%
\pgfpathlineto{\pgfqpoint{1.094906in}{1.292199in}}%
\pgfpathlineto{\pgfqpoint{1.096003in}{1.292972in}}%
\pgfpathlineto{\pgfqpoint{1.097697in}{1.294080in}}%
\pgfpathlineto{\pgfqpoint{1.098796in}{1.294751in}}%
\pgfpathlineto{\pgfqpoint{1.100328in}{1.295859in}}%
\pgfpathlineto{\pgfqpoint{1.101426in}{1.296595in}}%
\pgfpathlineto{\pgfqpoint{1.102908in}{1.297703in}}%
\pgfpathlineto{\pgfqpoint{1.104012in}{1.298476in}}%
\pgfpathlineto{\pgfqpoint{1.105596in}{1.299584in}}%
\pgfpathlineto{\pgfqpoint{1.106686in}{1.300301in}}%
\pgfpathlineto{\pgfqpoint{1.108147in}{1.301409in}}%
\pgfpathlineto{\pgfqpoint{1.109254in}{1.302136in}}%
\pgfpathlineto{\pgfqpoint{1.110999in}{1.303244in}}%
\pgfpathlineto{\pgfqpoint{1.112106in}{1.304110in}}%
\pgfpathlineto{\pgfqpoint{1.113574in}{1.305218in}}%
\pgfpathlineto{\pgfqpoint{1.114679in}{1.306094in}}%
\pgfpathlineto{\pgfqpoint{1.116255in}{1.307202in}}%
\pgfpathlineto{\pgfqpoint{1.117357in}{1.307919in}}%
\pgfpathlineto{\pgfqpoint{1.119184in}{1.309027in}}%
\pgfpathlineto{\pgfqpoint{1.120270in}{1.309912in}}%
\pgfpathlineto{\pgfqpoint{1.121938in}{1.311020in}}%
\pgfpathlineto{\pgfqpoint{1.123026in}{1.311746in}}%
\pgfpathlineto{\pgfqpoint{1.124822in}{1.312854in}}%
\pgfpathlineto{\pgfqpoint{1.125925in}{1.313655in}}%
\pgfpathlineto{\pgfqpoint{1.127533in}{1.314764in}}%
\pgfpathlineto{\pgfqpoint{1.128608in}{1.315453in}}%
\pgfpathlineto{\pgfqpoint{1.130195in}{1.316561in}}%
\pgfpathlineto{\pgfqpoint{1.131300in}{1.317259in}}%
\pgfpathlineto{\pgfqpoint{1.133207in}{1.318368in}}%
\pgfpathlineto{\pgfqpoint{1.134316in}{1.319038in}}%
\pgfpathlineto{\pgfqpoint{1.136131in}{1.320137in}}%
\pgfpathlineto{\pgfqpoint{1.137236in}{1.320938in}}%
\pgfpathlineto{\pgfqpoint{1.138983in}{1.322046in}}%
\pgfpathlineto{\pgfqpoint{1.140081in}{1.322893in}}%
\pgfpathlineto{\pgfqpoint{1.140088in}{1.322893in}}%
\pgfpathlineto{\pgfqpoint{1.141854in}{1.324002in}}%
\pgfpathlineto{\pgfqpoint{1.142930in}{1.324784in}}%
\pgfpathlineto{\pgfqpoint{1.144403in}{1.325892in}}%
\pgfpathlineto{\pgfqpoint{1.145503in}{1.326674in}}%
\pgfpathlineto{\pgfqpoint{1.145508in}{1.326674in}}%
\pgfpathlineto{\pgfqpoint{1.147304in}{1.327773in}}%
\pgfpathlineto{\pgfqpoint{1.148407in}{1.328481in}}%
\pgfpathlineto{\pgfqpoint{1.150140in}{1.329589in}}%
\pgfpathlineto{\pgfqpoint{1.151240in}{1.330483in}}%
\pgfpathlineto{\pgfqpoint{1.152935in}{1.331591in}}%
\pgfpathlineto{\pgfqpoint{1.153979in}{1.332411in}}%
\pgfpathlineto{\pgfqpoint{1.153993in}{1.332411in}}%
\pgfpathlineto{\pgfqpoint{1.155654in}{1.333519in}}%
\pgfpathlineto{\pgfqpoint{1.156754in}{1.334301in}}%
\pgfpathlineto{\pgfqpoint{1.158688in}{1.335410in}}%
\pgfpathlineto{\pgfqpoint{1.159774in}{1.336201in}}%
\pgfpathlineto{\pgfqpoint{1.161477in}{1.337300in}}%
\pgfpathlineto{\pgfqpoint{1.162572in}{1.337905in}}%
\pgfpathlineto{\pgfqpoint{1.164448in}{1.339013in}}%
\pgfpathlineto{\pgfqpoint{1.165558in}{1.339637in}}%
\pgfpathlineto{\pgfqpoint{1.167321in}{1.340746in}}%
\pgfpathlineto{\pgfqpoint{1.168424in}{1.341491in}}%
\pgfpathlineto{\pgfqpoint{1.170401in}{1.342599in}}%
\pgfpathlineto{\pgfqpoint{1.171501in}{1.343316in}}%
\pgfpathlineto{\pgfqpoint{1.173255in}{1.344424in}}%
\pgfpathlineto{\pgfqpoint{1.174355in}{1.345281in}}%
\pgfpathlineto{\pgfqpoint{1.176421in}{1.346389in}}%
\pgfpathlineto{\pgfqpoint{1.177528in}{1.347162in}}%
\pgfpathlineto{\pgfqpoint{1.178980in}{1.348270in}}%
\pgfpathlineto{\pgfqpoint{1.180080in}{1.348969in}}%
\pgfpathlineto{\pgfqpoint{1.181773in}{1.350077in}}%
\pgfpathlineto{\pgfqpoint{1.182840in}{1.350747in}}%
\pgfpathlineto{\pgfqpoint{1.184656in}{1.351855in}}%
\pgfpathlineto{\pgfqpoint{1.185751in}{1.352489in}}%
\pgfpathlineto{\pgfqpoint{1.187604in}{1.353597in}}%
\pgfpathlineto{\pgfqpoint{1.188659in}{1.354267in}}%
\pgfpathlineto{\pgfqpoint{1.190545in}{1.355376in}}%
\pgfpathlineto{\pgfqpoint{1.191647in}{1.355972in}}%
\pgfpathlineto{\pgfqpoint{1.193586in}{1.357080in}}%
\pgfpathlineto{\pgfqpoint{1.194658in}{1.357769in}}%
\pgfpathlineto{\pgfqpoint{1.196689in}{1.358877in}}%
\pgfpathlineto{\pgfqpoint{1.197759in}{1.359482in}}%
\pgfpathlineto{\pgfqpoint{1.199656in}{1.360591in}}%
\pgfpathlineto{\pgfqpoint{1.200763in}{1.361429in}}%
\pgfpathlineto{\pgfqpoint{1.203087in}{1.362537in}}%
\pgfpathlineto{\pgfqpoint{1.204190in}{1.363245in}}%
\pgfpathlineto{\pgfqpoint{1.204197in}{1.363245in}}%
\pgfpathlineto{\pgfqpoint{1.205822in}{1.364353in}}%
\pgfpathlineto{\pgfqpoint{1.206917in}{1.365023in}}%
\pgfpathlineto{\pgfqpoint{1.209009in}{1.366132in}}%
\pgfpathlineto{\pgfqpoint{1.210107in}{1.366895in}}%
\pgfpathlineto{\pgfqpoint{1.211842in}{1.367985in}}%
\pgfpathlineto{\pgfqpoint{1.212933in}{1.368618in}}%
\pgfpathlineto{\pgfqpoint{1.214856in}{1.369717in}}%
\pgfpathlineto{\pgfqpoint{1.215932in}{1.370341in}}%
\pgfpathlineto{\pgfqpoint{1.218034in}{1.371449in}}%
\pgfpathlineto{\pgfqpoint{1.219131in}{1.372101in}}%
\pgfpathlineto{\pgfqpoint{1.220778in}{1.373209in}}%
\pgfpathlineto{\pgfqpoint{1.221875in}{1.373852in}}%
\pgfpathlineto{\pgfqpoint{1.223970in}{1.374951in}}%
\pgfpathlineto{\pgfqpoint{1.225067in}{1.375621in}}%
\pgfpathlineto{\pgfqpoint{1.226707in}{1.376729in}}%
\pgfpathlineto{\pgfqpoint{1.227790in}{1.377307in}}%
\pgfpathlineto{\pgfqpoint{1.229697in}{1.378415in}}%
\pgfpathlineto{\pgfqpoint{1.230785in}{1.379039in}}%
\pgfpathlineto{\pgfqpoint{1.232805in}{1.380147in}}%
\pgfpathlineto{\pgfqpoint{1.233886in}{1.380948in}}%
\pgfpathlineto{\pgfqpoint{1.235778in}{1.382056in}}%
\pgfpathlineto{\pgfqpoint{1.236888in}{1.382643in}}%
\pgfpathlineto{\pgfqpoint{1.238891in}{1.383751in}}%
\pgfpathlineto{\pgfqpoint{1.239991in}{1.384412in}}%
\pgfpathlineto{\pgfqpoint{1.242104in}{1.385511in}}%
\pgfpathlineto{\pgfqpoint{1.243211in}{1.386144in}}%
\pgfpathlineto{\pgfqpoint{1.245361in}{1.387253in}}%
\pgfpathlineto{\pgfqpoint{1.246419in}{1.387802in}}%
\pgfpathlineto{\pgfqpoint{1.249041in}{1.388910in}}%
\pgfpathlineto{\pgfqpoint{1.250143in}{1.389627in}}%
\pgfpathlineto{\pgfqpoint{1.252414in}{1.390735in}}%
\pgfpathlineto{\pgfqpoint{1.253521in}{1.391425in}}%
\pgfpathlineto{\pgfqpoint{1.255481in}{1.392533in}}%
\pgfpathlineto{\pgfqpoint{1.256576in}{1.393129in}}%
\pgfpathlineto{\pgfqpoint{1.258790in}{1.394237in}}%
\pgfpathlineto{\pgfqpoint{1.259900in}{1.394945in}}%
\pgfpathlineto{\pgfqpoint{1.261910in}{1.396053in}}%
\pgfpathlineto{\pgfqpoint{1.263019in}{1.396695in}}%
\pgfpathlineto{\pgfqpoint{1.265184in}{1.397794in}}%
\pgfpathlineto{\pgfqpoint{1.266293in}{1.398362in}}%
\pgfpathlineto{\pgfqpoint{1.268732in}{1.399471in}}%
\pgfpathlineto{\pgfqpoint{1.269842in}{1.400216in}}%
\pgfpathlineto{\pgfqpoint{1.271936in}{1.401324in}}%
\pgfpathlineto{\pgfqpoint{1.273038in}{1.401957in}}%
\pgfpathlineto{\pgfqpoint{1.275527in}{1.403065in}}%
\pgfpathlineto{\pgfqpoint{1.276636in}{1.403736in}}%
\pgfpathlineto{\pgfqpoint{1.278897in}{1.404844in}}%
\pgfpathlineto{\pgfqpoint{1.280004in}{1.405580in}}%
\pgfpathlineto{\pgfqpoint{1.282218in}{1.406688in}}%
\pgfpathlineto{\pgfqpoint{1.283299in}{1.407247in}}%
\pgfpathlineto{\pgfqpoint{1.285757in}{1.408355in}}%
\pgfpathlineto{\pgfqpoint{1.286859in}{1.408867in}}%
\pgfpathlineto{\pgfqpoint{1.288951in}{1.409966in}}%
\pgfpathlineto{\pgfqpoint{1.290056in}{1.410636in}}%
\pgfpathlineto{\pgfqpoint{1.292422in}{1.411745in}}%
\pgfpathlineto{\pgfqpoint{1.293508in}{1.412313in}}%
\pgfpathlineto{\pgfqpoint{1.295954in}{1.413421in}}%
\pgfpathlineto{\pgfqpoint{1.297054in}{1.413970in}}%
\pgfpathlineto{\pgfqpoint{1.299343in}{1.415078in}}%
\pgfpathlineto{\pgfqpoint{1.300434in}{1.415591in}}%
\pgfpathlineto{\pgfqpoint{1.302957in}{1.416699in}}%
\pgfpathlineto{\pgfqpoint{1.304036in}{1.417360in}}%
\pgfpathlineto{\pgfqpoint{1.306123in}{1.418468in}}%
\pgfpathlineto{\pgfqpoint{1.307214in}{1.419018in}}%
\pgfpathlineto{\pgfqpoint{1.309524in}{1.420117in}}%
\pgfpathlineto{\pgfqpoint{1.310633in}{1.420666in}}%
\pgfpathlineto{\pgfqpoint{1.312571in}{1.421774in}}%
\pgfpathlineto{\pgfqpoint{1.313670in}{1.422389in}}%
\pgfpathlineto{\pgfqpoint{1.316018in}{1.423497in}}%
\pgfpathlineto{\pgfqpoint{1.317127in}{1.424186in}}%
\pgfpathlineto{\pgfqpoint{1.319287in}{1.425294in}}%
\pgfpathlineto{\pgfqpoint{1.320390in}{1.425853in}}%
\pgfpathlineto{\pgfqpoint{1.322151in}{1.426961in}}%
\pgfpathlineto{\pgfqpoint{1.323258in}{1.427408in}}%
\pgfpathlineto{\pgfqpoint{1.325207in}{1.428507in}}%
\pgfpathlineto{\pgfqpoint{1.326309in}{1.429029in}}%
\pgfpathlineto{\pgfqpoint{1.328901in}{1.430137in}}%
\pgfpathlineto{\pgfqpoint{1.330003in}{1.430770in}}%
\pgfpathlineto{\pgfqpoint{1.332386in}{1.431878in}}%
\pgfpathlineto{\pgfqpoint{1.333495in}{1.432567in}}%
\pgfpathlineto{\pgfqpoint{1.335658in}{1.433676in}}%
\pgfpathlineto{\pgfqpoint{1.336767in}{1.434123in}}%
\pgfpathlineto{\pgfqpoint{1.339047in}{1.435231in}}%
\pgfpathlineto{\pgfqpoint{1.340114in}{1.435901in}}%
\pgfpathlineto{\pgfqpoint{1.342907in}{1.437009in}}%
\pgfpathlineto{\pgfqpoint{1.343979in}{1.437587in}}%
\pgfpathlineto{\pgfqpoint{1.346106in}{1.438695in}}%
\pgfpathlineto{\pgfqpoint{1.347215in}{1.439217in}}%
\pgfpathlineto{\pgfqpoint{1.349305in}{1.440325in}}%
\pgfpathlineto{\pgfqpoint{1.350384in}{1.440865in}}%
\pgfpathlineto{\pgfqpoint{1.352380in}{1.441973in}}%
\pgfpathlineto{\pgfqpoint{1.353484in}{1.442448in}}%
\pgfpathlineto{\pgfqpoint{1.355635in}{1.443556in}}%
\pgfpathlineto{\pgfqpoint{1.356740in}{1.444050in}}%
\pgfpathlineto{\pgfqpoint{1.359495in}{1.445158in}}%
\pgfpathlineto{\pgfqpoint{1.360600in}{1.445596in}}%
\pgfpathlineto{\pgfqpoint{1.363191in}{1.446704in}}%
\pgfpathlineto{\pgfqpoint{1.364268in}{1.447300in}}%
\pgfpathlineto{\pgfqpoint{1.366620in}{1.448408in}}%
\pgfpathlineto{\pgfqpoint{1.367713in}{1.449041in}}%
\pgfpathlineto{\pgfqpoint{1.370729in}{1.450150in}}%
\pgfpathlineto{\pgfqpoint{1.371839in}{1.450578in}}%
\pgfpathlineto{\pgfqpoint{1.374343in}{1.451686in}}%
\pgfpathlineto{\pgfqpoint{1.375446in}{1.452245in}}%
\pgfpathlineto{\pgfqpoint{1.378072in}{1.453344in}}%
\pgfpathlineto{\pgfqpoint{1.379123in}{1.453875in}}%
\pgfpathlineto{\pgfqpoint{1.381635in}{1.454983in}}%
\pgfpathlineto{\pgfqpoint{1.382728in}{1.455541in}}%
\pgfpathlineto{\pgfqpoint{1.385725in}{1.456650in}}%
\pgfpathlineto{\pgfqpoint{1.386780in}{1.457022in}}%
\pgfpathlineto{\pgfqpoint{1.389323in}{1.458121in}}%
\pgfpathlineto{\pgfqpoint{1.390376in}{1.458447in}}%
\pgfpathlineto{\pgfqpoint{1.390411in}{1.458447in}}%
\pgfpathlineto{\pgfqpoint{1.392721in}{1.459555in}}%
\pgfpathlineto{\pgfqpoint{1.393830in}{1.460002in}}%
\pgfpathlineto{\pgfqpoint{1.396565in}{1.461110in}}%
\pgfpathlineto{\pgfqpoint{1.397649in}{1.461511in}}%
\pgfpathlineto{\pgfqpoint{1.400242in}{1.462619in}}%
\pgfpathlineto{\pgfqpoint{1.401328in}{1.463075in}}%
\pgfpathlineto{\pgfqpoint{1.401335in}{1.463075in}}%
\pgfpathlineto{\pgfqpoint{1.403906in}{1.464184in}}%
\pgfpathlineto{\pgfqpoint{1.404973in}{1.464742in}}%
\pgfpathlineto{\pgfqpoint{1.408153in}{1.465850in}}%
\pgfpathlineto{\pgfqpoint{1.409213in}{1.466223in}}%
\pgfpathlineto{\pgfqpoint{1.411866in}{1.467331in}}%
\pgfpathlineto{\pgfqpoint{1.412935in}{1.467806in}}%
\pgfpathlineto{\pgfqpoint{1.415982in}{1.468914in}}%
\pgfpathlineto{\pgfqpoint{1.417075in}{1.469277in}}%
\pgfpathlineto{\pgfqpoint{1.419561in}{1.470386in}}%
\pgfpathlineto{\pgfqpoint{1.420661in}{1.470786in}}%
\pgfpathlineto{\pgfqpoint{1.423496in}{1.471894in}}%
\pgfpathlineto{\pgfqpoint{1.424591in}{1.472360in}}%
\pgfpathlineto{\pgfqpoint{1.427256in}{1.473468in}}%
\pgfpathlineto{\pgfqpoint{1.428339in}{1.473906in}}%
\pgfpathlineto{\pgfqpoint{1.431693in}{1.475005in}}%
\pgfpathlineto{\pgfqpoint{1.432779in}{1.475480in}}%
\pgfpathlineto{\pgfqpoint{1.432800in}{1.475480in}}%
\pgfpathlineto{\pgfqpoint{1.435520in}{1.476569in}}%
\pgfpathlineto{\pgfqpoint{1.436588in}{1.476877in}}%
\pgfpathlineto{\pgfqpoint{1.436627in}{1.476877in}}%
\pgfpathlineto{\pgfqpoint{1.439716in}{1.477985in}}%
\pgfpathlineto{\pgfqpoint{1.440802in}{1.478432in}}%
\pgfpathlineto{\pgfqpoint{1.443551in}{1.479540in}}%
\pgfpathlineto{\pgfqpoint{1.444634in}{1.479903in}}%
\pgfpathlineto{\pgfqpoint{1.447730in}{1.481011in}}%
\pgfpathlineto{\pgfqpoint{1.448764in}{1.481365in}}%
\pgfpathlineto{\pgfqpoint{1.451513in}{1.482473in}}%
\pgfpathlineto{\pgfqpoint{1.452618in}{1.482874in}}%
\pgfpathlineto{\pgfqpoint{1.455298in}{1.483982in}}%
\pgfpathlineto{\pgfqpoint{1.456398in}{1.484364in}}%
\pgfpathlineto{\pgfqpoint{1.458392in}{1.485463in}}%
\pgfpathlineto{\pgfqpoint{1.459482in}{1.486012in}}%
\pgfpathlineto{\pgfqpoint{1.462278in}{1.487120in}}%
\pgfpathlineto{\pgfqpoint{1.463354in}{1.487484in}}%
\pgfpathlineto{\pgfqpoint{1.466959in}{1.488592in}}%
\pgfpathlineto{\pgfqpoint{1.468054in}{1.489039in}}%
\pgfpathlineto{\pgfqpoint{1.470866in}{1.490147in}}%
\pgfpathlineto{\pgfqpoint{1.471945in}{1.490557in}}%
\pgfpathlineto{\pgfqpoint{1.475247in}{1.491665in}}%
\pgfpathlineto{\pgfqpoint{1.476354in}{1.492149in}}%
\pgfpathlineto{\pgfqpoint{1.479495in}{1.493257in}}%
\pgfpathlineto{\pgfqpoint{1.480571in}{1.493658in}}%
\pgfpathlineto{\pgfqpoint{1.483878in}{1.494766in}}%
\pgfpathlineto{\pgfqpoint{1.484959in}{1.495232in}}%
\pgfpathlineto{\pgfqpoint{1.488627in}{1.496340in}}%
\pgfpathlineto{\pgfqpoint{1.489727in}{1.496833in}}%
\pgfpathlineto{\pgfqpoint{1.493271in}{1.497942in}}%
\pgfpathlineto{\pgfqpoint{1.494345in}{1.498361in}}%
\pgfpathlineto{\pgfqpoint{1.497603in}{1.499469in}}%
\pgfpathlineto{\pgfqpoint{1.498654in}{1.499832in}}%
\pgfpathlineto{\pgfqpoint{1.502465in}{1.500940in}}%
\pgfpathlineto{\pgfqpoint{1.503553in}{1.501331in}}%
\pgfpathlineto{\pgfqpoint{1.506623in}{1.502440in}}%
\pgfpathlineto{\pgfqpoint{1.507695in}{1.502812in}}%
\pgfpathlineto{\pgfqpoint{1.510779in}{1.503920in}}%
\pgfpathlineto{\pgfqpoint{1.511883in}{1.504311in}}%
\pgfpathlineto{\pgfqpoint{1.515828in}{1.505420in}}%
\pgfpathlineto{\pgfqpoint{1.516930in}{1.505801in}}%
\pgfpathlineto{\pgfqpoint{1.520019in}{1.506910in}}%
\pgfpathlineto{\pgfqpoint{1.521129in}{1.507263in}}%
\pgfpathlineto{\pgfqpoint{1.524281in}{1.508372in}}%
\pgfpathlineto{\pgfqpoint{1.525322in}{1.508809in}}%
\pgfpathlineto{\pgfqpoint{1.528582in}{1.509917in}}%
\pgfpathlineto{\pgfqpoint{1.529658in}{1.510374in}}%
\pgfpathlineto{\pgfqpoint{1.533561in}{1.511482in}}%
\pgfpathlineto{\pgfqpoint{1.534659in}{1.511780in}}%
\pgfpathlineto{\pgfqpoint{1.538847in}{1.512888in}}%
\pgfpathlineto{\pgfqpoint{1.539945in}{1.513354in}}%
\pgfpathlineto{\pgfqpoint{1.543383in}{1.514462in}}%
\pgfpathlineto{\pgfqpoint{1.544467in}{1.514937in}}%
\pgfpathlineto{\pgfqpoint{1.547933in}{1.516045in}}%
\pgfpathlineto{\pgfqpoint{1.549002in}{1.516399in}}%
\pgfpathlineto{\pgfqpoint{1.552931in}{1.517507in}}%
\pgfpathlineto{\pgfqpoint{1.554024in}{1.517842in}}%
\pgfpathlineto{\pgfqpoint{1.558370in}{1.518951in}}%
\pgfpathlineto{\pgfqpoint{1.559434in}{1.519407in}}%
\pgfpathlineto{\pgfqpoint{1.559479in}{1.519407in}}%
\pgfpathlineto{\pgfqpoint{1.563072in}{1.520506in}}%
\pgfpathlineto{\pgfqpoint{1.564172in}{1.520822in}}%
\pgfpathlineto{\pgfqpoint{1.568147in}{1.521931in}}%
\pgfpathlineto{\pgfqpoint{1.569238in}{1.522303in}}%
\pgfpathlineto{\pgfqpoint{1.572709in}{1.523411in}}%
\pgfpathlineto{\pgfqpoint{1.573816in}{1.523924in}}%
\pgfpathlineto{\pgfqpoint{1.578448in}{1.525022in}}%
\pgfpathlineto{\pgfqpoint{1.579543in}{1.525311in}}%
\pgfpathlineto{\pgfqpoint{1.583575in}{1.526419in}}%
\pgfpathlineto{\pgfqpoint{1.584658in}{1.526792in}}%
\pgfpathlineto{\pgfqpoint{1.588389in}{1.527900in}}%
\pgfpathlineto{\pgfqpoint{1.589492in}{1.528161in}}%
\pgfpathlineto{\pgfqpoint{1.593514in}{1.529269in}}%
\pgfpathlineto{\pgfqpoint{1.594621in}{1.529641in}}%
\pgfpathlineto{\pgfqpoint{1.598481in}{1.530750in}}%
\pgfpathlineto{\pgfqpoint{1.599548in}{1.531085in}}%
\pgfpathlineto{\pgfqpoint{1.604169in}{1.532193in}}%
\pgfpathlineto{\pgfqpoint{1.605224in}{1.532435in}}%
\pgfpathlineto{\pgfqpoint{1.605259in}{1.532435in}}%
\pgfpathlineto{\pgfqpoint{1.609980in}{1.533543in}}%
\pgfpathlineto{\pgfqpoint{1.611068in}{1.533935in}}%
\pgfpathlineto{\pgfqpoint{1.615328in}{1.535043in}}%
\pgfpathlineto{\pgfqpoint{1.616413in}{1.535387in}}%
\pgfpathlineto{\pgfqpoint{1.621081in}{1.536496in}}%
\pgfpathlineto{\pgfqpoint{1.622173in}{1.536784in}}%
\pgfpathlineto{\pgfqpoint{1.622190in}{1.536784in}}%
\pgfpathlineto{\pgfqpoint{1.626388in}{1.537892in}}%
\pgfpathlineto{\pgfqpoint{1.627464in}{1.538172in}}%
\pgfpathlineto{\pgfqpoint{1.627488in}{1.538172in}}%
\pgfpathlineto{\pgfqpoint{1.631663in}{1.539280in}}%
\pgfpathlineto{\pgfqpoint{1.632758in}{1.539597in}}%
\pgfpathlineto{\pgfqpoint{1.637064in}{1.540705in}}%
\pgfpathlineto{\pgfqpoint{1.638168in}{1.540975in}}%
\pgfpathlineto{\pgfqpoint{1.642327in}{1.542083in}}%
\pgfpathlineto{\pgfqpoint{1.643368in}{1.542400in}}%
\pgfpathlineto{\pgfqpoint{1.643403in}{1.542400in}}%
\pgfpathlineto{\pgfqpoint{1.647634in}{1.543508in}}%
\pgfpathlineto{\pgfqpoint{1.648725in}{1.543871in}}%
\pgfpathlineto{\pgfqpoint{1.648741in}{1.543871in}}%
\pgfpathlineto{\pgfqpoint{1.652768in}{1.544979in}}%
\pgfpathlineto{\pgfqpoint{1.653823in}{1.545231in}}%
\pgfpathlineto{\pgfqpoint{1.653847in}{1.545231in}}%
\pgfpathlineto{\pgfqpoint{1.658171in}{1.546339in}}%
\pgfpathlineto{\pgfqpoint{1.659180in}{1.546656in}}%
\pgfpathlineto{\pgfqpoint{1.663767in}{1.547764in}}%
\pgfpathlineto{\pgfqpoint{1.664870in}{1.548099in}}%
\pgfpathlineto{\pgfqpoint{1.669190in}{1.549207in}}%
\pgfpathlineto{\pgfqpoint{1.670212in}{1.549440in}}%
\pgfpathlineto{\pgfqpoint{1.670243in}{1.549440in}}%
\pgfpathlineto{\pgfqpoint{1.675738in}{1.550548in}}%
\pgfpathlineto{\pgfqpoint{1.676821in}{1.550828in}}%
\pgfpathlineto{\pgfqpoint{1.681601in}{1.551936in}}%
\pgfpathlineto{\pgfqpoint{1.682544in}{1.552159in}}%
\pgfpathlineto{\pgfqpoint{1.687049in}{1.553267in}}%
\pgfpathlineto{\pgfqpoint{1.688140in}{1.553565in}}%
\pgfpathlineto{\pgfqpoint{1.692793in}{1.554674in}}%
\pgfpathlineto{\pgfqpoint{1.693897in}{1.554897in}}%
\pgfpathlineto{\pgfqpoint{1.698403in}{1.556005in}}%
\pgfpathlineto{\pgfqpoint{1.699503in}{1.556285in}}%
\pgfpathlineto{\pgfqpoint{1.704275in}{1.557393in}}%
\pgfpathlineto{\pgfqpoint{1.705342in}{1.557533in}}%
\pgfpathlineto{\pgfqpoint{1.711018in}{1.558641in}}%
\pgfpathlineto{\pgfqpoint{1.712064in}{1.558948in}}%
\pgfpathlineto{\pgfqpoint{1.712118in}{1.558948in}}%
\pgfpathlineto{\pgfqpoint{1.717686in}{1.560056in}}%
\pgfpathlineto{\pgfqpoint{1.718786in}{1.560392in}}%
\pgfpathlineto{\pgfqpoint{1.724466in}{1.561500in}}%
\pgfpathlineto{\pgfqpoint{1.725571in}{1.561835in}}%
\pgfpathlineto{\pgfqpoint{1.731169in}{1.562943in}}%
\pgfpathlineto{\pgfqpoint{1.732079in}{1.563204in}}%
\pgfpathlineto{\pgfqpoint{1.732116in}{1.563204in}}%
\pgfpathlineto{\pgfqpoint{1.737951in}{1.564312in}}%
\pgfpathlineto{\pgfqpoint{1.739009in}{1.564489in}}%
\pgfpathlineto{\pgfqpoint{1.739016in}{1.564489in}}%
\pgfpathlineto{\pgfqpoint{1.744312in}{1.565597in}}%
\pgfpathlineto{\pgfqpoint{1.745414in}{1.565830in}}%
\pgfpathlineto{\pgfqpoint{1.750663in}{1.566938in}}%
\pgfpathlineto{\pgfqpoint{1.751683in}{1.567218in}}%
\pgfpathlineto{\pgfqpoint{1.751718in}{1.567218in}}%
\pgfpathlineto{\pgfqpoint{1.758496in}{1.568326in}}%
\pgfpathlineto{\pgfqpoint{1.759538in}{1.568624in}}%
\pgfpathlineto{\pgfqpoint{1.765668in}{1.569732in}}%
\pgfpathlineto{\pgfqpoint{1.766726in}{1.569974in}}%
\pgfpathlineto{\pgfqpoint{1.772953in}{1.571082in}}%
\pgfpathlineto{\pgfqpoint{1.774060in}{1.571315in}}%
\pgfpathlineto{\pgfqpoint{1.779531in}{1.572423in}}%
\pgfpathlineto{\pgfqpoint{1.780495in}{1.572638in}}%
\pgfpathlineto{\pgfqpoint{1.780638in}{1.572638in}}%
\pgfpathlineto{\pgfqpoint{1.787144in}{1.573746in}}%
\pgfpathlineto{\pgfqpoint{1.788007in}{1.573857in}}%
\pgfpathlineto{\pgfqpoint{1.788035in}{1.573857in}}%
\pgfpathlineto{\pgfqpoint{1.794342in}{1.574966in}}%
\pgfpathlineto{\pgfqpoint{1.795449in}{1.575096in}}%
\pgfpathlineto{\pgfqpoint{1.801071in}{1.576204in}}%
\pgfpathlineto{\pgfqpoint{1.802163in}{1.576428in}}%
\pgfpathlineto{\pgfqpoint{1.808852in}{1.577536in}}%
\pgfpathlineto{\pgfqpoint{1.809849in}{1.577759in}}%
\pgfpathlineto{\pgfqpoint{1.809922in}{1.577759in}}%
\pgfpathlineto{\pgfqpoint{1.816887in}{1.578868in}}%
\pgfpathlineto{\pgfqpoint{1.817872in}{1.579082in}}%
\pgfpathlineto{\pgfqpoint{1.825445in}{1.580190in}}%
\pgfpathlineto{\pgfqpoint{1.826484in}{1.580441in}}%
\pgfpathlineto{\pgfqpoint{1.833192in}{1.581550in}}%
\pgfpathlineto{\pgfqpoint{1.834278in}{1.581773in}}%
\pgfpathlineto{\pgfqpoint{1.841630in}{1.582881in}}%
\pgfpathlineto{\pgfqpoint{1.842613in}{1.583012in}}%
\pgfpathlineto{\pgfqpoint{1.849963in}{1.584120in}}%
\pgfpathlineto{\pgfqpoint{1.851039in}{1.584288in}}%
\pgfpathlineto{\pgfqpoint{1.858268in}{1.585396in}}%
\pgfpathlineto{\pgfqpoint{1.859335in}{1.585563in}}%
\pgfpathlineto{\pgfqpoint{1.866934in}{1.586672in}}%
\pgfpathlineto{\pgfqpoint{1.867958in}{1.586774in}}%
\pgfpathlineto{\pgfqpoint{1.867987in}{1.586774in}}%
\pgfpathlineto{\pgfqpoint{1.876423in}{1.587882in}}%
\pgfpathlineto{\pgfqpoint{1.877523in}{1.588050in}}%
\pgfpathlineto{\pgfqpoint{1.884816in}{1.589158in}}%
\pgfpathlineto{\pgfqpoint{1.885919in}{1.589363in}}%
\pgfpathlineto{\pgfqpoint{1.895839in}{1.590471in}}%
\pgfpathlineto{\pgfqpoint{1.896592in}{1.590555in}}%
\pgfpathlineto{\pgfqpoint{1.896780in}{1.590555in}}%
\pgfpathlineto{\pgfqpoint{1.906890in}{1.591663in}}%
\pgfpathlineto{\pgfqpoint{1.907901in}{1.591803in}}%
\pgfpathlineto{\pgfqpoint{1.917719in}{1.592911in}}%
\pgfpathlineto{\pgfqpoint{1.918765in}{1.593060in}}%
\pgfpathlineto{\pgfqpoint{1.929058in}{1.594168in}}%
\pgfpathlineto{\pgfqpoint{1.930130in}{1.594280in}}%
\pgfpathlineto{\pgfqpoint{1.930158in}{1.594280in}}%
\pgfpathlineto{\pgfqpoint{1.940895in}{1.595388in}}%
\pgfpathlineto{\pgfqpoint{1.941917in}{1.595509in}}%
\pgfpathlineto{\pgfqpoint{1.941974in}{1.595509in}}%
\pgfpathlineto{\pgfqpoint{1.953541in}{1.596617in}}%
\pgfpathlineto{\pgfqpoint{1.954561in}{1.596720in}}%
\pgfpathlineto{\pgfqpoint{1.954627in}{1.596720in}}%
\pgfpathlineto{\pgfqpoint{1.968248in}{1.597828in}}%
\pgfpathlineto{\pgfqpoint{1.969343in}{1.597930in}}%
\pgfpathlineto{\pgfqpoint{1.983413in}{1.599039in}}%
\pgfpathlineto{\pgfqpoint{1.984506in}{1.599150in}}%
\pgfpathlineto{\pgfqpoint{1.999921in}{1.600259in}}%
\pgfpathlineto{\pgfqpoint{2.001005in}{1.600361in}}%
\pgfpathlineto{\pgfqpoint{2.019033in}{1.601423in}}%
\pgfpathlineto{\pgfqpoint{2.033126in}{1.601944in}}%
\pgfpathlineto{\pgfqpoint{2.033126in}{1.601944in}}%
\pgfusepath{stroke}%
\end{pgfscope}%
\begin{pgfscope}%
\pgfsetrectcap%
\pgfsetmiterjoin%
\pgfsetlinewidth{0.803000pt}%
\definecolor{currentstroke}{rgb}{0.000000,0.000000,0.000000}%
\pgfsetstrokecolor{currentstroke}%
\pgfsetdash{}{0pt}%
\pgfpathmoveto{\pgfqpoint{0.553581in}{0.499444in}}%
\pgfpathlineto{\pgfqpoint{0.553581in}{1.654444in}}%
\pgfusepath{stroke}%
\end{pgfscope}%
\begin{pgfscope}%
\pgfsetrectcap%
\pgfsetmiterjoin%
\pgfsetlinewidth{0.803000pt}%
\definecolor{currentstroke}{rgb}{0.000000,0.000000,0.000000}%
\pgfsetstrokecolor{currentstroke}%
\pgfsetdash{}{0pt}%
\pgfpathmoveto{\pgfqpoint{2.103581in}{0.499444in}}%
\pgfpathlineto{\pgfqpoint{2.103581in}{1.654444in}}%
\pgfusepath{stroke}%
\end{pgfscope}%
\begin{pgfscope}%
\pgfsetrectcap%
\pgfsetmiterjoin%
\pgfsetlinewidth{0.803000pt}%
\definecolor{currentstroke}{rgb}{0.000000,0.000000,0.000000}%
\pgfsetstrokecolor{currentstroke}%
\pgfsetdash{}{0pt}%
\pgfpathmoveto{\pgfqpoint{0.553581in}{0.499444in}}%
\pgfpathlineto{\pgfqpoint{2.103581in}{0.499444in}}%
\pgfusepath{stroke}%
\end{pgfscope}%
\begin{pgfscope}%
\pgfsetrectcap%
\pgfsetmiterjoin%
\pgfsetlinewidth{0.803000pt}%
\definecolor{currentstroke}{rgb}{0.000000,0.000000,0.000000}%
\pgfsetstrokecolor{currentstroke}%
\pgfsetdash{}{0pt}%
\pgfpathmoveto{\pgfqpoint{0.553581in}{1.654444in}}%
\pgfpathlineto{\pgfqpoint{2.103581in}{1.654444in}}%
\pgfusepath{stroke}%
\end{pgfscope}%
\begin{pgfscope}%
\pgfsetbuttcap%
\pgfsetmiterjoin%
\definecolor{currentfill}{rgb}{1.000000,1.000000,1.000000}%
\pgfsetfillcolor{currentfill}%
\pgfsetfillopacity{0.800000}%
\pgfsetlinewidth{1.003750pt}%
\definecolor{currentstroke}{rgb}{0.800000,0.800000,0.800000}%
\pgfsetstrokecolor{currentstroke}%
\pgfsetstrokeopacity{0.800000}%
\pgfsetdash{}{0pt}%
\pgfpathmoveto{\pgfqpoint{0.832747in}{0.568889in}}%
\pgfpathlineto{\pgfqpoint{2.006358in}{0.568889in}}%
\pgfpathquadraticcurveto{\pgfqpoint{2.034136in}{0.568889in}}{\pgfqpoint{2.034136in}{0.596666in}}%
\pgfpathlineto{\pgfqpoint{2.034136in}{0.776388in}}%
\pgfpathquadraticcurveto{\pgfqpoint{2.034136in}{0.804166in}}{\pgfqpoint{2.006358in}{0.804166in}}%
\pgfpathlineto{\pgfqpoint{0.832747in}{0.804166in}}%
\pgfpathquadraticcurveto{\pgfqpoint{0.804970in}{0.804166in}}{\pgfqpoint{0.804970in}{0.776388in}}%
\pgfpathlineto{\pgfqpoint{0.804970in}{0.596666in}}%
\pgfpathquadraticcurveto{\pgfqpoint{0.804970in}{0.568889in}}{\pgfqpoint{0.832747in}{0.568889in}}%
\pgfpathlineto{\pgfqpoint{0.832747in}{0.568889in}}%
\pgfpathclose%
\pgfusepath{stroke,fill}%
\end{pgfscope}%
\begin{pgfscope}%
\pgfsetrectcap%
\pgfsetroundjoin%
\pgfsetlinewidth{1.505625pt}%
\definecolor{currentstroke}{rgb}{0.000000,0.000000,0.000000}%
\pgfsetstrokecolor{currentstroke}%
\pgfsetdash{}{0pt}%
\pgfpathmoveto{\pgfqpoint{0.860525in}{0.700000in}}%
\pgfpathlineto{\pgfqpoint{0.999414in}{0.700000in}}%
\pgfpathlineto{\pgfqpoint{1.138303in}{0.700000in}}%
\pgfusepath{stroke}%
\end{pgfscope}%
\begin{pgfscope}%
\definecolor{textcolor}{rgb}{0.000000,0.000000,0.000000}%
\pgfsetstrokecolor{textcolor}%
\pgfsetfillcolor{textcolor}%
\pgftext[x=1.249414in,y=0.651388in,left,base]{\color{textcolor}\rmfamily\fontsize{10.000000}{12.000000}\selectfont AUC=0.752}%
\end{pgfscope}%
\end{pgfpicture}%
\makeatother%
\endgroup%

\end{tabular}

\



The model below is as effective at separating the two classes ($\text{ROC}=0.778$), but the distribution is skewed to the left.  Its results were nearly continuous, with the 214,070 samples returning 210,157 unique values of $p$, so we can fine tune the decision threshold.  

\

%
\verb|KBFC_5_Fold_alpha_0_5_gamma_0_0_Hard_Test|

\

\noindent\begin{tabular}{@{\hspace{-6pt}}p{4.3in} @{\hspace{-6pt}}p{2.0in}}
	\vskip 0pt
	\hfil Raw Model Output
	
	%% Creator: Matplotlib, PGF backend
%%
%% To include the figure in your LaTeX document, write
%%   \input{<filename>.pgf}
%%
%% Make sure the required packages are loaded in your preamble
%%   \usepackage{pgf}
%%
%% Also ensure that all the required font packages are loaded; for instance,
%% the lmodern package is sometimes necessary when using math font.
%%   \usepackage{lmodern}
%%
%% Figures using additional raster images can only be included by \input if
%% they are in the same directory as the main LaTeX file. For loading figures
%% from other directories you can use the `import` package
%%   \usepackage{import}
%%
%% and then include the figures with
%%   \import{<path to file>}{<filename>.pgf}
%%
%% Matplotlib used the following preamble
%%   
%%   \usepackage{fontspec}
%%   \makeatletter\@ifpackageloaded{underscore}{}{\usepackage[strings]{underscore}}\makeatother
%%
\begingroup%
\makeatletter%
\begin{pgfpicture}%
\pgfpathrectangle{\pgfpointorigin}{\pgfqpoint{4.102500in}{1.754444in}}%
\pgfusepath{use as bounding box, clip}%
\begin{pgfscope}%
\pgfsetbuttcap%
\pgfsetmiterjoin%
\definecolor{currentfill}{rgb}{1.000000,1.000000,1.000000}%
\pgfsetfillcolor{currentfill}%
\pgfsetlinewidth{0.000000pt}%
\definecolor{currentstroke}{rgb}{1.000000,1.000000,1.000000}%
\pgfsetstrokecolor{currentstroke}%
\pgfsetdash{}{0pt}%
\pgfpathmoveto{\pgfqpoint{0.000000in}{0.000000in}}%
\pgfpathlineto{\pgfqpoint{4.102500in}{0.000000in}}%
\pgfpathlineto{\pgfqpoint{4.102500in}{1.754444in}}%
\pgfpathlineto{\pgfqpoint{0.000000in}{1.754444in}}%
\pgfpathlineto{\pgfqpoint{0.000000in}{0.000000in}}%
\pgfpathclose%
\pgfusepath{fill}%
\end{pgfscope}%
\begin{pgfscope}%
\pgfsetbuttcap%
\pgfsetmiterjoin%
\definecolor{currentfill}{rgb}{1.000000,1.000000,1.000000}%
\pgfsetfillcolor{currentfill}%
\pgfsetlinewidth{0.000000pt}%
\definecolor{currentstroke}{rgb}{0.000000,0.000000,0.000000}%
\pgfsetstrokecolor{currentstroke}%
\pgfsetstrokeopacity{0.000000}%
\pgfsetdash{}{0pt}%
\pgfpathmoveto{\pgfqpoint{0.515000in}{0.499444in}}%
\pgfpathlineto{\pgfqpoint{4.002500in}{0.499444in}}%
\pgfpathlineto{\pgfqpoint{4.002500in}{1.654444in}}%
\pgfpathlineto{\pgfqpoint{0.515000in}{1.654444in}}%
\pgfpathlineto{\pgfqpoint{0.515000in}{0.499444in}}%
\pgfpathclose%
\pgfusepath{fill}%
\end{pgfscope}%
\begin{pgfscope}%
\pgfpathrectangle{\pgfqpoint{0.515000in}{0.499444in}}{\pgfqpoint{3.487500in}{1.155000in}}%
\pgfusepath{clip}%
\pgfsetbuttcap%
\pgfsetmiterjoin%
\pgfsetlinewidth{1.003750pt}%
\definecolor{currentstroke}{rgb}{0.000000,0.000000,0.000000}%
\pgfsetstrokecolor{currentstroke}%
\pgfsetdash{}{0pt}%
\pgfpathmoveto{\pgfqpoint{0.610114in}{0.499444in}}%
\pgfpathlineto{\pgfqpoint{0.673523in}{0.499444in}}%
\pgfpathlineto{\pgfqpoint{0.673523in}{0.499444in}}%
\pgfpathlineto{\pgfqpoint{0.610114in}{0.499444in}}%
\pgfpathlineto{\pgfqpoint{0.610114in}{0.499444in}}%
\pgfpathclose%
\pgfusepath{stroke}%
\end{pgfscope}%
\begin{pgfscope}%
\pgfpathrectangle{\pgfqpoint{0.515000in}{0.499444in}}{\pgfqpoint{3.487500in}{1.155000in}}%
\pgfusepath{clip}%
\pgfsetbuttcap%
\pgfsetmiterjoin%
\pgfsetlinewidth{1.003750pt}%
\definecolor{currentstroke}{rgb}{0.000000,0.000000,0.000000}%
\pgfsetstrokecolor{currentstroke}%
\pgfsetdash{}{0pt}%
\pgfpathmoveto{\pgfqpoint{0.768637in}{0.499444in}}%
\pgfpathlineto{\pgfqpoint{0.832046in}{0.499444in}}%
\pgfpathlineto{\pgfqpoint{0.832046in}{1.599444in}}%
\pgfpathlineto{\pgfqpoint{0.768637in}{1.599444in}}%
\pgfpathlineto{\pgfqpoint{0.768637in}{0.499444in}}%
\pgfpathclose%
\pgfusepath{stroke}%
\end{pgfscope}%
\begin{pgfscope}%
\pgfpathrectangle{\pgfqpoint{0.515000in}{0.499444in}}{\pgfqpoint{3.487500in}{1.155000in}}%
\pgfusepath{clip}%
\pgfsetbuttcap%
\pgfsetmiterjoin%
\pgfsetlinewidth{1.003750pt}%
\definecolor{currentstroke}{rgb}{0.000000,0.000000,0.000000}%
\pgfsetstrokecolor{currentstroke}%
\pgfsetdash{}{0pt}%
\pgfpathmoveto{\pgfqpoint{0.927159in}{0.499444in}}%
\pgfpathlineto{\pgfqpoint{0.990568in}{0.499444in}}%
\pgfpathlineto{\pgfqpoint{0.990568in}{1.374325in}}%
\pgfpathlineto{\pgfqpoint{0.927159in}{1.374325in}}%
\pgfpathlineto{\pgfqpoint{0.927159in}{0.499444in}}%
\pgfpathclose%
\pgfusepath{stroke}%
\end{pgfscope}%
\begin{pgfscope}%
\pgfpathrectangle{\pgfqpoint{0.515000in}{0.499444in}}{\pgfqpoint{3.487500in}{1.155000in}}%
\pgfusepath{clip}%
\pgfsetbuttcap%
\pgfsetmiterjoin%
\pgfsetlinewidth{1.003750pt}%
\definecolor{currentstroke}{rgb}{0.000000,0.000000,0.000000}%
\pgfsetstrokecolor{currentstroke}%
\pgfsetdash{}{0pt}%
\pgfpathmoveto{\pgfqpoint{1.085682in}{0.499444in}}%
\pgfpathlineto{\pgfqpoint{1.149091in}{0.499444in}}%
\pgfpathlineto{\pgfqpoint{1.149091in}{1.049348in}}%
\pgfpathlineto{\pgfqpoint{1.085682in}{1.049348in}}%
\pgfpathlineto{\pgfqpoint{1.085682in}{0.499444in}}%
\pgfpathclose%
\pgfusepath{stroke}%
\end{pgfscope}%
\begin{pgfscope}%
\pgfpathrectangle{\pgfqpoint{0.515000in}{0.499444in}}{\pgfqpoint{3.487500in}{1.155000in}}%
\pgfusepath{clip}%
\pgfsetbuttcap%
\pgfsetmiterjoin%
\pgfsetlinewidth{1.003750pt}%
\definecolor{currentstroke}{rgb}{0.000000,0.000000,0.000000}%
\pgfsetstrokecolor{currentstroke}%
\pgfsetdash{}{0pt}%
\pgfpathmoveto{\pgfqpoint{1.244205in}{0.499444in}}%
\pgfpathlineto{\pgfqpoint{1.307614in}{0.499444in}}%
\pgfpathlineto{\pgfqpoint{1.307614in}{0.850879in}}%
\pgfpathlineto{\pgfqpoint{1.244205in}{0.850879in}}%
\pgfpathlineto{\pgfqpoint{1.244205in}{0.499444in}}%
\pgfpathclose%
\pgfusepath{stroke}%
\end{pgfscope}%
\begin{pgfscope}%
\pgfpathrectangle{\pgfqpoint{0.515000in}{0.499444in}}{\pgfqpoint{3.487500in}{1.155000in}}%
\pgfusepath{clip}%
\pgfsetbuttcap%
\pgfsetmiterjoin%
\pgfsetlinewidth{1.003750pt}%
\definecolor{currentstroke}{rgb}{0.000000,0.000000,0.000000}%
\pgfsetstrokecolor{currentstroke}%
\pgfsetdash{}{0pt}%
\pgfpathmoveto{\pgfqpoint{1.402728in}{0.499444in}}%
\pgfpathlineto{\pgfqpoint{1.466137in}{0.499444in}}%
\pgfpathlineto{\pgfqpoint{1.466137in}{0.734107in}}%
\pgfpathlineto{\pgfqpoint{1.402728in}{0.734107in}}%
\pgfpathlineto{\pgfqpoint{1.402728in}{0.499444in}}%
\pgfpathclose%
\pgfusepath{stroke}%
\end{pgfscope}%
\begin{pgfscope}%
\pgfpathrectangle{\pgfqpoint{0.515000in}{0.499444in}}{\pgfqpoint{3.487500in}{1.155000in}}%
\pgfusepath{clip}%
\pgfsetbuttcap%
\pgfsetmiterjoin%
\pgfsetlinewidth{1.003750pt}%
\definecolor{currentstroke}{rgb}{0.000000,0.000000,0.000000}%
\pgfsetstrokecolor{currentstroke}%
\pgfsetdash{}{0pt}%
\pgfpathmoveto{\pgfqpoint{1.561250in}{0.499444in}}%
\pgfpathlineto{\pgfqpoint{1.624659in}{0.499444in}}%
\pgfpathlineto{\pgfqpoint{1.624659in}{0.656370in}}%
\pgfpathlineto{\pgfqpoint{1.561250in}{0.656370in}}%
\pgfpathlineto{\pgfqpoint{1.561250in}{0.499444in}}%
\pgfpathclose%
\pgfusepath{stroke}%
\end{pgfscope}%
\begin{pgfscope}%
\pgfpathrectangle{\pgfqpoint{0.515000in}{0.499444in}}{\pgfqpoint{3.487500in}{1.155000in}}%
\pgfusepath{clip}%
\pgfsetbuttcap%
\pgfsetmiterjoin%
\pgfsetlinewidth{1.003750pt}%
\definecolor{currentstroke}{rgb}{0.000000,0.000000,0.000000}%
\pgfsetstrokecolor{currentstroke}%
\pgfsetdash{}{0pt}%
\pgfpathmoveto{\pgfqpoint{1.719773in}{0.499444in}}%
\pgfpathlineto{\pgfqpoint{1.783182in}{0.499444in}}%
\pgfpathlineto{\pgfqpoint{1.783182in}{0.603029in}}%
\pgfpathlineto{\pgfqpoint{1.719773in}{0.603029in}}%
\pgfpathlineto{\pgfqpoint{1.719773in}{0.499444in}}%
\pgfpathclose%
\pgfusepath{stroke}%
\end{pgfscope}%
\begin{pgfscope}%
\pgfpathrectangle{\pgfqpoint{0.515000in}{0.499444in}}{\pgfqpoint{3.487500in}{1.155000in}}%
\pgfusepath{clip}%
\pgfsetbuttcap%
\pgfsetmiterjoin%
\pgfsetlinewidth{1.003750pt}%
\definecolor{currentstroke}{rgb}{0.000000,0.000000,0.000000}%
\pgfsetstrokecolor{currentstroke}%
\pgfsetdash{}{0pt}%
\pgfpathmoveto{\pgfqpoint{1.878296in}{0.499444in}}%
\pgfpathlineto{\pgfqpoint{1.941705in}{0.499444in}}%
\pgfpathlineto{\pgfqpoint{1.941705in}{0.569603in}}%
\pgfpathlineto{\pgfqpoint{1.878296in}{0.569603in}}%
\pgfpathlineto{\pgfqpoint{1.878296in}{0.499444in}}%
\pgfpathclose%
\pgfusepath{stroke}%
\end{pgfscope}%
\begin{pgfscope}%
\pgfpathrectangle{\pgfqpoint{0.515000in}{0.499444in}}{\pgfqpoint{3.487500in}{1.155000in}}%
\pgfusepath{clip}%
\pgfsetbuttcap%
\pgfsetmiterjoin%
\pgfsetlinewidth{1.003750pt}%
\definecolor{currentstroke}{rgb}{0.000000,0.000000,0.000000}%
\pgfsetstrokecolor{currentstroke}%
\pgfsetdash{}{0pt}%
\pgfpathmoveto{\pgfqpoint{2.036818in}{0.499444in}}%
\pgfpathlineto{\pgfqpoint{2.100228in}{0.499444in}}%
\pgfpathlineto{\pgfqpoint{2.100228in}{0.548513in}}%
\pgfpathlineto{\pgfqpoint{2.036818in}{0.548513in}}%
\pgfpathlineto{\pgfqpoint{2.036818in}{0.499444in}}%
\pgfpathclose%
\pgfusepath{stroke}%
\end{pgfscope}%
\begin{pgfscope}%
\pgfpathrectangle{\pgfqpoint{0.515000in}{0.499444in}}{\pgfqpoint{3.487500in}{1.155000in}}%
\pgfusepath{clip}%
\pgfsetbuttcap%
\pgfsetmiterjoin%
\pgfsetlinewidth{1.003750pt}%
\definecolor{currentstroke}{rgb}{0.000000,0.000000,0.000000}%
\pgfsetstrokecolor{currentstroke}%
\pgfsetdash{}{0pt}%
\pgfpathmoveto{\pgfqpoint{2.195341in}{0.499444in}}%
\pgfpathlineto{\pgfqpoint{2.258750in}{0.499444in}}%
\pgfpathlineto{\pgfqpoint{2.258750in}{0.532906in}}%
\pgfpathlineto{\pgfqpoint{2.195341in}{0.532906in}}%
\pgfpathlineto{\pgfqpoint{2.195341in}{0.499444in}}%
\pgfpathclose%
\pgfusepath{stroke}%
\end{pgfscope}%
\begin{pgfscope}%
\pgfpathrectangle{\pgfqpoint{0.515000in}{0.499444in}}{\pgfqpoint{3.487500in}{1.155000in}}%
\pgfusepath{clip}%
\pgfsetbuttcap%
\pgfsetmiterjoin%
\pgfsetlinewidth{1.003750pt}%
\definecolor{currentstroke}{rgb}{0.000000,0.000000,0.000000}%
\pgfsetstrokecolor{currentstroke}%
\pgfsetdash{}{0pt}%
\pgfpathmoveto{\pgfqpoint{2.353864in}{0.499444in}}%
\pgfpathlineto{\pgfqpoint{2.417273in}{0.499444in}}%
\pgfpathlineto{\pgfqpoint{2.417273in}{0.522984in}}%
\pgfpathlineto{\pgfqpoint{2.353864in}{0.522984in}}%
\pgfpathlineto{\pgfqpoint{2.353864in}{0.499444in}}%
\pgfpathclose%
\pgfusepath{stroke}%
\end{pgfscope}%
\begin{pgfscope}%
\pgfpathrectangle{\pgfqpoint{0.515000in}{0.499444in}}{\pgfqpoint{3.487500in}{1.155000in}}%
\pgfusepath{clip}%
\pgfsetbuttcap%
\pgfsetmiterjoin%
\pgfsetlinewidth{1.003750pt}%
\definecolor{currentstroke}{rgb}{0.000000,0.000000,0.000000}%
\pgfsetstrokecolor{currentstroke}%
\pgfsetdash{}{0pt}%
\pgfpathmoveto{\pgfqpoint{2.512387in}{0.499444in}}%
\pgfpathlineto{\pgfqpoint{2.575796in}{0.499444in}}%
\pgfpathlineto{\pgfqpoint{2.575796in}{0.515597in}}%
\pgfpathlineto{\pgfqpoint{2.512387in}{0.515597in}}%
\pgfpathlineto{\pgfqpoint{2.512387in}{0.499444in}}%
\pgfpathclose%
\pgfusepath{stroke}%
\end{pgfscope}%
\begin{pgfscope}%
\pgfpathrectangle{\pgfqpoint{0.515000in}{0.499444in}}{\pgfqpoint{3.487500in}{1.155000in}}%
\pgfusepath{clip}%
\pgfsetbuttcap%
\pgfsetmiterjoin%
\pgfsetlinewidth{1.003750pt}%
\definecolor{currentstroke}{rgb}{0.000000,0.000000,0.000000}%
\pgfsetstrokecolor{currentstroke}%
\pgfsetdash{}{0pt}%
\pgfpathmoveto{\pgfqpoint{2.670909in}{0.499444in}}%
\pgfpathlineto{\pgfqpoint{2.734318in}{0.499444in}}%
\pgfpathlineto{\pgfqpoint{2.734318in}{0.511457in}}%
\pgfpathlineto{\pgfqpoint{2.670909in}{0.511457in}}%
\pgfpathlineto{\pgfqpoint{2.670909in}{0.499444in}}%
\pgfpathclose%
\pgfusepath{stroke}%
\end{pgfscope}%
\begin{pgfscope}%
\pgfpathrectangle{\pgfqpoint{0.515000in}{0.499444in}}{\pgfqpoint{3.487500in}{1.155000in}}%
\pgfusepath{clip}%
\pgfsetbuttcap%
\pgfsetmiterjoin%
\pgfsetlinewidth{1.003750pt}%
\definecolor{currentstroke}{rgb}{0.000000,0.000000,0.000000}%
\pgfsetstrokecolor{currentstroke}%
\pgfsetdash{}{0pt}%
\pgfpathmoveto{\pgfqpoint{2.829432in}{0.499444in}}%
\pgfpathlineto{\pgfqpoint{2.892841in}{0.499444in}}%
\pgfpathlineto{\pgfqpoint{2.892841in}{0.508060in}}%
\pgfpathlineto{\pgfqpoint{2.829432in}{0.508060in}}%
\pgfpathlineto{\pgfqpoint{2.829432in}{0.499444in}}%
\pgfpathclose%
\pgfusepath{stroke}%
\end{pgfscope}%
\begin{pgfscope}%
\pgfpathrectangle{\pgfqpoint{0.515000in}{0.499444in}}{\pgfqpoint{3.487500in}{1.155000in}}%
\pgfusepath{clip}%
\pgfsetbuttcap%
\pgfsetmiterjoin%
\pgfsetlinewidth{1.003750pt}%
\definecolor{currentstroke}{rgb}{0.000000,0.000000,0.000000}%
\pgfsetstrokecolor{currentstroke}%
\pgfsetdash{}{0pt}%
\pgfpathmoveto{\pgfqpoint{2.987955in}{0.499444in}}%
\pgfpathlineto{\pgfqpoint{3.051364in}{0.499444in}}%
\pgfpathlineto{\pgfqpoint{3.051364in}{0.505753in}}%
\pgfpathlineto{\pgfqpoint{2.987955in}{0.505753in}}%
\pgfpathlineto{\pgfqpoint{2.987955in}{0.499444in}}%
\pgfpathclose%
\pgfusepath{stroke}%
\end{pgfscope}%
\begin{pgfscope}%
\pgfpathrectangle{\pgfqpoint{0.515000in}{0.499444in}}{\pgfqpoint{3.487500in}{1.155000in}}%
\pgfusepath{clip}%
\pgfsetbuttcap%
\pgfsetmiterjoin%
\pgfsetlinewidth{1.003750pt}%
\definecolor{currentstroke}{rgb}{0.000000,0.000000,0.000000}%
\pgfsetstrokecolor{currentstroke}%
\pgfsetdash{}{0pt}%
\pgfpathmoveto{\pgfqpoint{3.146478in}{0.499444in}}%
\pgfpathlineto{\pgfqpoint{3.209887in}{0.499444in}}%
\pgfpathlineto{\pgfqpoint{3.209887in}{0.504052in}}%
\pgfpathlineto{\pgfqpoint{3.146478in}{0.504052in}}%
\pgfpathlineto{\pgfqpoint{3.146478in}{0.499444in}}%
\pgfpathclose%
\pgfusepath{stroke}%
\end{pgfscope}%
\begin{pgfscope}%
\pgfpathrectangle{\pgfqpoint{0.515000in}{0.499444in}}{\pgfqpoint{3.487500in}{1.155000in}}%
\pgfusepath{clip}%
\pgfsetbuttcap%
\pgfsetmiterjoin%
\pgfsetlinewidth{1.003750pt}%
\definecolor{currentstroke}{rgb}{0.000000,0.000000,0.000000}%
\pgfsetstrokecolor{currentstroke}%
\pgfsetdash{}{0pt}%
\pgfpathmoveto{\pgfqpoint{3.305000in}{0.499444in}}%
\pgfpathlineto{\pgfqpoint{3.368409in}{0.499444in}}%
\pgfpathlineto{\pgfqpoint{3.368409in}{0.502194in}}%
\pgfpathlineto{\pgfqpoint{3.305000in}{0.502194in}}%
\pgfpathlineto{\pgfqpoint{3.305000in}{0.499444in}}%
\pgfpathclose%
\pgfusepath{stroke}%
\end{pgfscope}%
\begin{pgfscope}%
\pgfpathrectangle{\pgfqpoint{0.515000in}{0.499444in}}{\pgfqpoint{3.487500in}{1.155000in}}%
\pgfusepath{clip}%
\pgfsetbuttcap%
\pgfsetmiterjoin%
\pgfsetlinewidth{1.003750pt}%
\definecolor{currentstroke}{rgb}{0.000000,0.000000,0.000000}%
\pgfsetstrokecolor{currentstroke}%
\pgfsetdash{}{0pt}%
\pgfpathmoveto{\pgfqpoint{3.463523in}{0.499444in}}%
\pgfpathlineto{\pgfqpoint{3.526932in}{0.499444in}}%
\pgfpathlineto{\pgfqpoint{3.526932in}{0.500702in}}%
\pgfpathlineto{\pgfqpoint{3.463523in}{0.500702in}}%
\pgfpathlineto{\pgfqpoint{3.463523in}{0.499444in}}%
\pgfpathclose%
\pgfusepath{stroke}%
\end{pgfscope}%
\begin{pgfscope}%
\pgfpathrectangle{\pgfqpoint{0.515000in}{0.499444in}}{\pgfqpoint{3.487500in}{1.155000in}}%
\pgfusepath{clip}%
\pgfsetbuttcap%
\pgfsetmiterjoin%
\pgfsetlinewidth{1.003750pt}%
\definecolor{currentstroke}{rgb}{0.000000,0.000000,0.000000}%
\pgfsetstrokecolor{currentstroke}%
\pgfsetdash{}{0pt}%
\pgfpathmoveto{\pgfqpoint{3.622046in}{0.499444in}}%
\pgfpathlineto{\pgfqpoint{3.685455in}{0.499444in}}%
\pgfpathlineto{\pgfqpoint{3.685455in}{0.499768in}}%
\pgfpathlineto{\pgfqpoint{3.622046in}{0.499768in}}%
\pgfpathlineto{\pgfqpoint{3.622046in}{0.499444in}}%
\pgfpathclose%
\pgfusepath{stroke}%
\end{pgfscope}%
\begin{pgfscope}%
\pgfpathrectangle{\pgfqpoint{0.515000in}{0.499444in}}{\pgfqpoint{3.487500in}{1.155000in}}%
\pgfusepath{clip}%
\pgfsetbuttcap%
\pgfsetmiterjoin%
\pgfsetlinewidth{1.003750pt}%
\definecolor{currentstroke}{rgb}{0.000000,0.000000,0.000000}%
\pgfsetstrokecolor{currentstroke}%
\pgfsetdash{}{0pt}%
\pgfpathmoveto{\pgfqpoint{3.780568in}{0.499444in}}%
\pgfpathlineto{\pgfqpoint{3.843978in}{0.499444in}}%
\pgfpathlineto{\pgfqpoint{3.843978in}{0.499486in}}%
\pgfpathlineto{\pgfqpoint{3.780568in}{0.499486in}}%
\pgfpathlineto{\pgfqpoint{3.780568in}{0.499444in}}%
\pgfpathclose%
\pgfusepath{stroke}%
\end{pgfscope}%
\begin{pgfscope}%
\pgfpathrectangle{\pgfqpoint{0.515000in}{0.499444in}}{\pgfqpoint{3.487500in}{1.155000in}}%
\pgfusepath{clip}%
\pgfsetbuttcap%
\pgfsetmiterjoin%
\definecolor{currentfill}{rgb}{0.000000,0.000000,0.000000}%
\pgfsetfillcolor{currentfill}%
\pgfsetlinewidth{0.000000pt}%
\definecolor{currentstroke}{rgb}{0.000000,0.000000,0.000000}%
\pgfsetstrokecolor{currentstroke}%
\pgfsetstrokeopacity{0.000000}%
\pgfsetdash{}{0pt}%
\pgfpathmoveto{\pgfqpoint{0.673523in}{0.499444in}}%
\pgfpathlineto{\pgfqpoint{0.736932in}{0.499444in}}%
\pgfpathlineto{\pgfqpoint{0.736932in}{0.499444in}}%
\pgfpathlineto{\pgfqpoint{0.673523in}{0.499444in}}%
\pgfpathlineto{\pgfqpoint{0.673523in}{0.499444in}}%
\pgfpathclose%
\pgfusepath{fill}%
\end{pgfscope}%
\begin{pgfscope}%
\pgfpathrectangle{\pgfqpoint{0.515000in}{0.499444in}}{\pgfqpoint{3.487500in}{1.155000in}}%
\pgfusepath{clip}%
\pgfsetbuttcap%
\pgfsetmiterjoin%
\definecolor{currentfill}{rgb}{0.000000,0.000000,0.000000}%
\pgfsetfillcolor{currentfill}%
\pgfsetlinewidth{0.000000pt}%
\definecolor{currentstroke}{rgb}{0.000000,0.000000,0.000000}%
\pgfsetstrokecolor{currentstroke}%
\pgfsetstrokeopacity{0.000000}%
\pgfsetdash{}{0pt}%
\pgfpathmoveto{\pgfqpoint{0.832046in}{0.499444in}}%
\pgfpathlineto{\pgfqpoint{0.895455in}{0.499444in}}%
\pgfpathlineto{\pgfqpoint{0.895455in}{0.535512in}}%
\pgfpathlineto{\pgfqpoint{0.832046in}{0.535512in}}%
\pgfpathlineto{\pgfqpoint{0.832046in}{0.499444in}}%
\pgfpathclose%
\pgfusepath{fill}%
\end{pgfscope}%
\begin{pgfscope}%
\pgfpathrectangle{\pgfqpoint{0.515000in}{0.499444in}}{\pgfqpoint{3.487500in}{1.155000in}}%
\pgfusepath{clip}%
\pgfsetbuttcap%
\pgfsetmiterjoin%
\definecolor{currentfill}{rgb}{0.000000,0.000000,0.000000}%
\pgfsetfillcolor{currentfill}%
\pgfsetlinewidth{0.000000pt}%
\definecolor{currentstroke}{rgb}{0.000000,0.000000,0.000000}%
\pgfsetstrokecolor{currentstroke}%
\pgfsetstrokeopacity{0.000000}%
\pgfsetdash{}{0pt}%
\pgfpathmoveto{\pgfqpoint{0.990568in}{0.499444in}}%
\pgfpathlineto{\pgfqpoint{1.053978in}{0.499444in}}%
\pgfpathlineto{\pgfqpoint{1.053978in}{0.577565in}}%
\pgfpathlineto{\pgfqpoint{0.990568in}{0.577565in}}%
\pgfpathlineto{\pgfqpoint{0.990568in}{0.499444in}}%
\pgfpathclose%
\pgfusepath{fill}%
\end{pgfscope}%
\begin{pgfscope}%
\pgfpathrectangle{\pgfqpoint{0.515000in}{0.499444in}}{\pgfqpoint{3.487500in}{1.155000in}}%
\pgfusepath{clip}%
\pgfsetbuttcap%
\pgfsetmiterjoin%
\definecolor{currentfill}{rgb}{0.000000,0.000000,0.000000}%
\pgfsetfillcolor{currentfill}%
\pgfsetlinewidth{0.000000pt}%
\definecolor{currentstroke}{rgb}{0.000000,0.000000,0.000000}%
\pgfsetstrokecolor{currentstroke}%
\pgfsetstrokeopacity{0.000000}%
\pgfsetdash{}{0pt}%
\pgfpathmoveto{\pgfqpoint{1.149091in}{0.499444in}}%
\pgfpathlineto{\pgfqpoint{1.212500in}{0.499444in}}%
\pgfpathlineto{\pgfqpoint{1.212500in}{0.582754in}}%
\pgfpathlineto{\pgfqpoint{1.149091in}{0.582754in}}%
\pgfpathlineto{\pgfqpoint{1.149091in}{0.499444in}}%
\pgfpathclose%
\pgfusepath{fill}%
\end{pgfscope}%
\begin{pgfscope}%
\pgfpathrectangle{\pgfqpoint{0.515000in}{0.499444in}}{\pgfqpoint{3.487500in}{1.155000in}}%
\pgfusepath{clip}%
\pgfsetbuttcap%
\pgfsetmiterjoin%
\definecolor{currentfill}{rgb}{0.000000,0.000000,0.000000}%
\pgfsetfillcolor{currentfill}%
\pgfsetlinewidth{0.000000pt}%
\definecolor{currentstroke}{rgb}{0.000000,0.000000,0.000000}%
\pgfsetstrokecolor{currentstroke}%
\pgfsetstrokeopacity{0.000000}%
\pgfsetdash{}{0pt}%
\pgfpathmoveto{\pgfqpoint{1.307614in}{0.499444in}}%
\pgfpathlineto{\pgfqpoint{1.371023in}{0.499444in}}%
\pgfpathlineto{\pgfqpoint{1.371023in}{0.577661in}}%
\pgfpathlineto{\pgfqpoint{1.307614in}{0.577661in}}%
\pgfpathlineto{\pgfqpoint{1.307614in}{0.499444in}}%
\pgfpathclose%
\pgfusepath{fill}%
\end{pgfscope}%
\begin{pgfscope}%
\pgfpathrectangle{\pgfqpoint{0.515000in}{0.499444in}}{\pgfqpoint{3.487500in}{1.155000in}}%
\pgfusepath{clip}%
\pgfsetbuttcap%
\pgfsetmiterjoin%
\definecolor{currentfill}{rgb}{0.000000,0.000000,0.000000}%
\pgfsetfillcolor{currentfill}%
\pgfsetlinewidth{0.000000pt}%
\definecolor{currentstroke}{rgb}{0.000000,0.000000,0.000000}%
\pgfsetstrokecolor{currentstroke}%
\pgfsetstrokeopacity{0.000000}%
\pgfsetdash{}{0pt}%
\pgfpathmoveto{\pgfqpoint{1.466137in}{0.499444in}}%
\pgfpathlineto{\pgfqpoint{1.529546in}{0.499444in}}%
\pgfpathlineto{\pgfqpoint{1.529546in}{0.568662in}}%
\pgfpathlineto{\pgfqpoint{1.466137in}{0.568662in}}%
\pgfpathlineto{\pgfqpoint{1.466137in}{0.499444in}}%
\pgfpathclose%
\pgfusepath{fill}%
\end{pgfscope}%
\begin{pgfscope}%
\pgfpathrectangle{\pgfqpoint{0.515000in}{0.499444in}}{\pgfqpoint{3.487500in}{1.155000in}}%
\pgfusepath{clip}%
\pgfsetbuttcap%
\pgfsetmiterjoin%
\definecolor{currentfill}{rgb}{0.000000,0.000000,0.000000}%
\pgfsetfillcolor{currentfill}%
\pgfsetlinewidth{0.000000pt}%
\definecolor{currentstroke}{rgb}{0.000000,0.000000,0.000000}%
\pgfsetstrokecolor{currentstroke}%
\pgfsetstrokeopacity{0.000000}%
\pgfsetdash{}{0pt}%
\pgfpathmoveto{\pgfqpoint{1.624659in}{0.499444in}}%
\pgfpathlineto{\pgfqpoint{1.688068in}{0.499444in}}%
\pgfpathlineto{\pgfqpoint{1.688068in}{0.558393in}}%
\pgfpathlineto{\pgfqpoint{1.624659in}{0.558393in}}%
\pgfpathlineto{\pgfqpoint{1.624659in}{0.499444in}}%
\pgfpathclose%
\pgfusepath{fill}%
\end{pgfscope}%
\begin{pgfscope}%
\pgfpathrectangle{\pgfqpoint{0.515000in}{0.499444in}}{\pgfqpoint{3.487500in}{1.155000in}}%
\pgfusepath{clip}%
\pgfsetbuttcap%
\pgfsetmiterjoin%
\definecolor{currentfill}{rgb}{0.000000,0.000000,0.000000}%
\pgfsetfillcolor{currentfill}%
\pgfsetlinewidth{0.000000pt}%
\definecolor{currentstroke}{rgb}{0.000000,0.000000,0.000000}%
\pgfsetstrokecolor{currentstroke}%
\pgfsetstrokeopacity{0.000000}%
\pgfsetdash{}{0pt}%
\pgfpathmoveto{\pgfqpoint{1.783182in}{0.499444in}}%
\pgfpathlineto{\pgfqpoint{1.846591in}{0.499444in}}%
\pgfpathlineto{\pgfqpoint{1.846591in}{0.549035in}}%
\pgfpathlineto{\pgfqpoint{1.783182in}{0.549035in}}%
\pgfpathlineto{\pgfqpoint{1.783182in}{0.499444in}}%
\pgfpathclose%
\pgfusepath{fill}%
\end{pgfscope}%
\begin{pgfscope}%
\pgfpathrectangle{\pgfqpoint{0.515000in}{0.499444in}}{\pgfqpoint{3.487500in}{1.155000in}}%
\pgfusepath{clip}%
\pgfsetbuttcap%
\pgfsetmiterjoin%
\definecolor{currentfill}{rgb}{0.000000,0.000000,0.000000}%
\pgfsetfillcolor{currentfill}%
\pgfsetlinewidth{0.000000pt}%
\definecolor{currentstroke}{rgb}{0.000000,0.000000,0.000000}%
\pgfsetstrokecolor{currentstroke}%
\pgfsetstrokeopacity{0.000000}%
\pgfsetdash{}{0pt}%
\pgfpathmoveto{\pgfqpoint{1.941705in}{0.499444in}}%
\pgfpathlineto{\pgfqpoint{2.005114in}{0.499444in}}%
\pgfpathlineto{\pgfqpoint{2.005114in}{0.540491in}}%
\pgfpathlineto{\pgfqpoint{1.941705in}{0.540491in}}%
\pgfpathlineto{\pgfqpoint{1.941705in}{0.499444in}}%
\pgfpathclose%
\pgfusepath{fill}%
\end{pgfscope}%
\begin{pgfscope}%
\pgfpathrectangle{\pgfqpoint{0.515000in}{0.499444in}}{\pgfqpoint{3.487500in}{1.155000in}}%
\pgfusepath{clip}%
\pgfsetbuttcap%
\pgfsetmiterjoin%
\definecolor{currentfill}{rgb}{0.000000,0.000000,0.000000}%
\pgfsetfillcolor{currentfill}%
\pgfsetlinewidth{0.000000pt}%
\definecolor{currentstroke}{rgb}{0.000000,0.000000,0.000000}%
\pgfsetstrokecolor{currentstroke}%
\pgfsetstrokeopacity{0.000000}%
\pgfsetdash{}{0pt}%
\pgfpathmoveto{\pgfqpoint{2.100228in}{0.499444in}}%
\pgfpathlineto{\pgfqpoint{2.163637in}{0.499444in}}%
\pgfpathlineto{\pgfqpoint{2.163637in}{0.533517in}}%
\pgfpathlineto{\pgfqpoint{2.100228in}{0.533517in}}%
\pgfpathlineto{\pgfqpoint{2.100228in}{0.499444in}}%
\pgfpathclose%
\pgfusepath{fill}%
\end{pgfscope}%
\begin{pgfscope}%
\pgfpathrectangle{\pgfqpoint{0.515000in}{0.499444in}}{\pgfqpoint{3.487500in}{1.155000in}}%
\pgfusepath{clip}%
\pgfsetbuttcap%
\pgfsetmiterjoin%
\definecolor{currentfill}{rgb}{0.000000,0.000000,0.000000}%
\pgfsetfillcolor{currentfill}%
\pgfsetlinewidth{0.000000pt}%
\definecolor{currentstroke}{rgb}{0.000000,0.000000,0.000000}%
\pgfsetstrokecolor{currentstroke}%
\pgfsetstrokeopacity{0.000000}%
\pgfsetdash{}{0pt}%
\pgfpathmoveto{\pgfqpoint{2.258750in}{0.499444in}}%
\pgfpathlineto{\pgfqpoint{2.322159in}{0.499444in}}%
\pgfpathlineto{\pgfqpoint{2.322159in}{0.527933in}}%
\pgfpathlineto{\pgfqpoint{2.258750in}{0.527933in}}%
\pgfpathlineto{\pgfqpoint{2.258750in}{0.499444in}}%
\pgfpathclose%
\pgfusepath{fill}%
\end{pgfscope}%
\begin{pgfscope}%
\pgfpathrectangle{\pgfqpoint{0.515000in}{0.499444in}}{\pgfqpoint{3.487500in}{1.155000in}}%
\pgfusepath{clip}%
\pgfsetbuttcap%
\pgfsetmiterjoin%
\definecolor{currentfill}{rgb}{0.000000,0.000000,0.000000}%
\pgfsetfillcolor{currentfill}%
\pgfsetlinewidth{0.000000pt}%
\definecolor{currentstroke}{rgb}{0.000000,0.000000,0.000000}%
\pgfsetstrokecolor{currentstroke}%
\pgfsetstrokeopacity{0.000000}%
\pgfsetdash{}{0pt}%
\pgfpathmoveto{\pgfqpoint{2.417273in}{0.499444in}}%
\pgfpathlineto{\pgfqpoint{2.480682in}{0.499444in}}%
\pgfpathlineto{\pgfqpoint{2.480682in}{0.522763in}}%
\pgfpathlineto{\pgfqpoint{2.417273in}{0.522763in}}%
\pgfpathlineto{\pgfqpoint{2.417273in}{0.499444in}}%
\pgfpathclose%
\pgfusepath{fill}%
\end{pgfscope}%
\begin{pgfscope}%
\pgfpathrectangle{\pgfqpoint{0.515000in}{0.499444in}}{\pgfqpoint{3.487500in}{1.155000in}}%
\pgfusepath{clip}%
\pgfsetbuttcap%
\pgfsetmiterjoin%
\definecolor{currentfill}{rgb}{0.000000,0.000000,0.000000}%
\pgfsetfillcolor{currentfill}%
\pgfsetlinewidth{0.000000pt}%
\definecolor{currentstroke}{rgb}{0.000000,0.000000,0.000000}%
\pgfsetstrokecolor{currentstroke}%
\pgfsetstrokeopacity{0.000000}%
\pgfsetdash{}{0pt}%
\pgfpathmoveto{\pgfqpoint{2.575796in}{0.499444in}}%
\pgfpathlineto{\pgfqpoint{2.639205in}{0.499444in}}%
\pgfpathlineto{\pgfqpoint{2.639205in}{0.518730in}}%
\pgfpathlineto{\pgfqpoint{2.575796in}{0.518730in}}%
\pgfpathlineto{\pgfqpoint{2.575796in}{0.499444in}}%
\pgfpathclose%
\pgfusepath{fill}%
\end{pgfscope}%
\begin{pgfscope}%
\pgfpathrectangle{\pgfqpoint{0.515000in}{0.499444in}}{\pgfqpoint{3.487500in}{1.155000in}}%
\pgfusepath{clip}%
\pgfsetbuttcap%
\pgfsetmiterjoin%
\definecolor{currentfill}{rgb}{0.000000,0.000000,0.000000}%
\pgfsetfillcolor{currentfill}%
\pgfsetlinewidth{0.000000pt}%
\definecolor{currentstroke}{rgb}{0.000000,0.000000,0.000000}%
\pgfsetstrokecolor{currentstroke}%
\pgfsetstrokeopacity{0.000000}%
\pgfsetdash{}{0pt}%
\pgfpathmoveto{\pgfqpoint{2.734318in}{0.499444in}}%
\pgfpathlineto{\pgfqpoint{2.797728in}{0.499444in}}%
\pgfpathlineto{\pgfqpoint{2.797728in}{0.517574in}}%
\pgfpathlineto{\pgfqpoint{2.734318in}{0.517574in}}%
\pgfpathlineto{\pgfqpoint{2.734318in}{0.499444in}}%
\pgfpathclose%
\pgfusepath{fill}%
\end{pgfscope}%
\begin{pgfscope}%
\pgfpathrectangle{\pgfqpoint{0.515000in}{0.499444in}}{\pgfqpoint{3.487500in}{1.155000in}}%
\pgfusepath{clip}%
\pgfsetbuttcap%
\pgfsetmiterjoin%
\definecolor{currentfill}{rgb}{0.000000,0.000000,0.000000}%
\pgfsetfillcolor{currentfill}%
\pgfsetlinewidth{0.000000pt}%
\definecolor{currentstroke}{rgb}{0.000000,0.000000,0.000000}%
\pgfsetstrokecolor{currentstroke}%
\pgfsetstrokeopacity{0.000000}%
\pgfsetdash{}{0pt}%
\pgfpathmoveto{\pgfqpoint{2.892841in}{0.499444in}}%
\pgfpathlineto{\pgfqpoint{2.956250in}{0.499444in}}%
\pgfpathlineto{\pgfqpoint{2.956250in}{0.515273in}}%
\pgfpathlineto{\pgfqpoint{2.892841in}{0.515273in}}%
\pgfpathlineto{\pgfqpoint{2.892841in}{0.499444in}}%
\pgfpathclose%
\pgfusepath{fill}%
\end{pgfscope}%
\begin{pgfscope}%
\pgfpathrectangle{\pgfqpoint{0.515000in}{0.499444in}}{\pgfqpoint{3.487500in}{1.155000in}}%
\pgfusepath{clip}%
\pgfsetbuttcap%
\pgfsetmiterjoin%
\definecolor{currentfill}{rgb}{0.000000,0.000000,0.000000}%
\pgfsetfillcolor{currentfill}%
\pgfsetlinewidth{0.000000pt}%
\definecolor{currentstroke}{rgb}{0.000000,0.000000,0.000000}%
\pgfsetstrokecolor{currentstroke}%
\pgfsetstrokeopacity{0.000000}%
\pgfsetdash{}{0pt}%
\pgfpathmoveto{\pgfqpoint{3.051364in}{0.499444in}}%
\pgfpathlineto{\pgfqpoint{3.114773in}{0.499444in}}%
\pgfpathlineto{\pgfqpoint{3.114773in}{0.513835in}}%
\pgfpathlineto{\pgfqpoint{3.051364in}{0.513835in}}%
\pgfpathlineto{\pgfqpoint{3.051364in}{0.499444in}}%
\pgfpathclose%
\pgfusepath{fill}%
\end{pgfscope}%
\begin{pgfscope}%
\pgfpathrectangle{\pgfqpoint{0.515000in}{0.499444in}}{\pgfqpoint{3.487500in}{1.155000in}}%
\pgfusepath{clip}%
\pgfsetbuttcap%
\pgfsetmiterjoin%
\definecolor{currentfill}{rgb}{0.000000,0.000000,0.000000}%
\pgfsetfillcolor{currentfill}%
\pgfsetlinewidth{0.000000pt}%
\definecolor{currentstroke}{rgb}{0.000000,0.000000,0.000000}%
\pgfsetstrokecolor{currentstroke}%
\pgfsetstrokeopacity{0.000000}%
\pgfsetdash{}{0pt}%
\pgfpathmoveto{\pgfqpoint{3.209887in}{0.499444in}}%
\pgfpathlineto{\pgfqpoint{3.273296in}{0.499444in}}%
\pgfpathlineto{\pgfqpoint{3.273296in}{0.511427in}}%
\pgfpathlineto{\pgfqpoint{3.209887in}{0.511427in}}%
\pgfpathlineto{\pgfqpoint{3.209887in}{0.499444in}}%
\pgfpathclose%
\pgfusepath{fill}%
\end{pgfscope}%
\begin{pgfscope}%
\pgfpathrectangle{\pgfqpoint{0.515000in}{0.499444in}}{\pgfqpoint{3.487500in}{1.155000in}}%
\pgfusepath{clip}%
\pgfsetbuttcap%
\pgfsetmiterjoin%
\definecolor{currentfill}{rgb}{0.000000,0.000000,0.000000}%
\pgfsetfillcolor{currentfill}%
\pgfsetlinewidth{0.000000pt}%
\definecolor{currentstroke}{rgb}{0.000000,0.000000,0.000000}%
\pgfsetstrokecolor{currentstroke}%
\pgfsetstrokeopacity{0.000000}%
\pgfsetdash{}{0pt}%
\pgfpathmoveto{\pgfqpoint{3.368409in}{0.499444in}}%
\pgfpathlineto{\pgfqpoint{3.431818in}{0.499444in}}%
\pgfpathlineto{\pgfqpoint{3.431818in}{0.508593in}}%
\pgfpathlineto{\pgfqpoint{3.368409in}{0.508593in}}%
\pgfpathlineto{\pgfqpoint{3.368409in}{0.499444in}}%
\pgfpathclose%
\pgfusepath{fill}%
\end{pgfscope}%
\begin{pgfscope}%
\pgfpathrectangle{\pgfqpoint{0.515000in}{0.499444in}}{\pgfqpoint{3.487500in}{1.155000in}}%
\pgfusepath{clip}%
\pgfsetbuttcap%
\pgfsetmiterjoin%
\definecolor{currentfill}{rgb}{0.000000,0.000000,0.000000}%
\pgfsetfillcolor{currentfill}%
\pgfsetlinewidth{0.000000pt}%
\definecolor{currentstroke}{rgb}{0.000000,0.000000,0.000000}%
\pgfsetstrokecolor{currentstroke}%
\pgfsetstrokeopacity{0.000000}%
\pgfsetdash{}{0pt}%
\pgfpathmoveto{\pgfqpoint{3.526932in}{0.499444in}}%
\pgfpathlineto{\pgfqpoint{3.590341in}{0.499444in}}%
\pgfpathlineto{\pgfqpoint{3.590341in}{0.504189in}}%
\pgfpathlineto{\pgfqpoint{3.526932in}{0.504189in}}%
\pgfpathlineto{\pgfqpoint{3.526932in}{0.499444in}}%
\pgfpathclose%
\pgfusepath{fill}%
\end{pgfscope}%
\begin{pgfscope}%
\pgfpathrectangle{\pgfqpoint{0.515000in}{0.499444in}}{\pgfqpoint{3.487500in}{1.155000in}}%
\pgfusepath{clip}%
\pgfsetbuttcap%
\pgfsetmiterjoin%
\definecolor{currentfill}{rgb}{0.000000,0.000000,0.000000}%
\pgfsetfillcolor{currentfill}%
\pgfsetlinewidth{0.000000pt}%
\definecolor{currentstroke}{rgb}{0.000000,0.000000,0.000000}%
\pgfsetstrokecolor{currentstroke}%
\pgfsetstrokeopacity{0.000000}%
\pgfsetdash{}{0pt}%
\pgfpathmoveto{\pgfqpoint{3.685455in}{0.499444in}}%
\pgfpathlineto{\pgfqpoint{3.748864in}{0.499444in}}%
\pgfpathlineto{\pgfqpoint{3.748864in}{0.500816in}}%
\pgfpathlineto{\pgfqpoint{3.685455in}{0.500816in}}%
\pgfpathlineto{\pgfqpoint{3.685455in}{0.499444in}}%
\pgfpathclose%
\pgfusepath{fill}%
\end{pgfscope}%
\begin{pgfscope}%
\pgfpathrectangle{\pgfqpoint{0.515000in}{0.499444in}}{\pgfqpoint{3.487500in}{1.155000in}}%
\pgfusepath{clip}%
\pgfsetbuttcap%
\pgfsetmiterjoin%
\definecolor{currentfill}{rgb}{0.000000,0.000000,0.000000}%
\pgfsetfillcolor{currentfill}%
\pgfsetlinewidth{0.000000pt}%
\definecolor{currentstroke}{rgb}{0.000000,0.000000,0.000000}%
\pgfsetstrokecolor{currentstroke}%
\pgfsetstrokeopacity{0.000000}%
\pgfsetdash{}{0pt}%
\pgfpathmoveto{\pgfqpoint{3.843978in}{0.499444in}}%
\pgfpathlineto{\pgfqpoint{3.907387in}{0.499444in}}%
\pgfpathlineto{\pgfqpoint{3.907387in}{0.499690in}}%
\pgfpathlineto{\pgfqpoint{3.843978in}{0.499690in}}%
\pgfpathlineto{\pgfqpoint{3.843978in}{0.499444in}}%
\pgfpathclose%
\pgfusepath{fill}%
\end{pgfscope}%
\begin{pgfscope}%
\pgfsetbuttcap%
\pgfsetroundjoin%
\definecolor{currentfill}{rgb}{0.000000,0.000000,0.000000}%
\pgfsetfillcolor{currentfill}%
\pgfsetlinewidth{0.803000pt}%
\definecolor{currentstroke}{rgb}{0.000000,0.000000,0.000000}%
\pgfsetstrokecolor{currentstroke}%
\pgfsetdash{}{0pt}%
\pgfsys@defobject{currentmarker}{\pgfqpoint{0.000000in}{-0.048611in}}{\pgfqpoint{0.000000in}{0.000000in}}{%
\pgfpathmoveto{\pgfqpoint{0.000000in}{0.000000in}}%
\pgfpathlineto{\pgfqpoint{0.000000in}{-0.048611in}}%
\pgfusepath{stroke,fill}%
}%
\begin{pgfscope}%
\pgfsys@transformshift{0.515000in}{0.499444in}%
\pgfsys@useobject{currentmarker}{}%
\end{pgfscope}%
\end{pgfscope}%
\begin{pgfscope}%
\pgfsetbuttcap%
\pgfsetroundjoin%
\definecolor{currentfill}{rgb}{0.000000,0.000000,0.000000}%
\pgfsetfillcolor{currentfill}%
\pgfsetlinewidth{0.803000pt}%
\definecolor{currentstroke}{rgb}{0.000000,0.000000,0.000000}%
\pgfsetstrokecolor{currentstroke}%
\pgfsetdash{}{0pt}%
\pgfsys@defobject{currentmarker}{\pgfqpoint{0.000000in}{-0.048611in}}{\pgfqpoint{0.000000in}{0.000000in}}{%
\pgfpathmoveto{\pgfqpoint{0.000000in}{0.000000in}}%
\pgfpathlineto{\pgfqpoint{0.000000in}{-0.048611in}}%
\pgfusepath{stroke,fill}%
}%
\begin{pgfscope}%
\pgfsys@transformshift{0.673523in}{0.499444in}%
\pgfsys@useobject{currentmarker}{}%
\end{pgfscope}%
\end{pgfscope}%
\begin{pgfscope}%
\definecolor{textcolor}{rgb}{0.000000,0.000000,0.000000}%
\pgfsetstrokecolor{textcolor}%
\pgfsetfillcolor{textcolor}%
\pgftext[x=0.673523in,y=0.402222in,,top]{\color{textcolor}\rmfamily\fontsize{10.000000}{12.000000}\selectfont 0.0}%
\end{pgfscope}%
\begin{pgfscope}%
\pgfsetbuttcap%
\pgfsetroundjoin%
\definecolor{currentfill}{rgb}{0.000000,0.000000,0.000000}%
\pgfsetfillcolor{currentfill}%
\pgfsetlinewidth{0.803000pt}%
\definecolor{currentstroke}{rgb}{0.000000,0.000000,0.000000}%
\pgfsetstrokecolor{currentstroke}%
\pgfsetdash{}{0pt}%
\pgfsys@defobject{currentmarker}{\pgfqpoint{0.000000in}{-0.048611in}}{\pgfqpoint{0.000000in}{0.000000in}}{%
\pgfpathmoveto{\pgfqpoint{0.000000in}{0.000000in}}%
\pgfpathlineto{\pgfqpoint{0.000000in}{-0.048611in}}%
\pgfusepath{stroke,fill}%
}%
\begin{pgfscope}%
\pgfsys@transformshift{0.832046in}{0.499444in}%
\pgfsys@useobject{currentmarker}{}%
\end{pgfscope}%
\end{pgfscope}%
\begin{pgfscope}%
\pgfsetbuttcap%
\pgfsetroundjoin%
\definecolor{currentfill}{rgb}{0.000000,0.000000,0.000000}%
\pgfsetfillcolor{currentfill}%
\pgfsetlinewidth{0.803000pt}%
\definecolor{currentstroke}{rgb}{0.000000,0.000000,0.000000}%
\pgfsetstrokecolor{currentstroke}%
\pgfsetdash{}{0pt}%
\pgfsys@defobject{currentmarker}{\pgfqpoint{0.000000in}{-0.048611in}}{\pgfqpoint{0.000000in}{0.000000in}}{%
\pgfpathmoveto{\pgfqpoint{0.000000in}{0.000000in}}%
\pgfpathlineto{\pgfqpoint{0.000000in}{-0.048611in}}%
\pgfusepath{stroke,fill}%
}%
\begin{pgfscope}%
\pgfsys@transformshift{0.990568in}{0.499444in}%
\pgfsys@useobject{currentmarker}{}%
\end{pgfscope}%
\end{pgfscope}%
\begin{pgfscope}%
\definecolor{textcolor}{rgb}{0.000000,0.000000,0.000000}%
\pgfsetstrokecolor{textcolor}%
\pgfsetfillcolor{textcolor}%
\pgftext[x=0.990568in,y=0.402222in,,top]{\color{textcolor}\rmfamily\fontsize{10.000000}{12.000000}\selectfont 0.1}%
\end{pgfscope}%
\begin{pgfscope}%
\pgfsetbuttcap%
\pgfsetroundjoin%
\definecolor{currentfill}{rgb}{0.000000,0.000000,0.000000}%
\pgfsetfillcolor{currentfill}%
\pgfsetlinewidth{0.803000pt}%
\definecolor{currentstroke}{rgb}{0.000000,0.000000,0.000000}%
\pgfsetstrokecolor{currentstroke}%
\pgfsetdash{}{0pt}%
\pgfsys@defobject{currentmarker}{\pgfqpoint{0.000000in}{-0.048611in}}{\pgfqpoint{0.000000in}{0.000000in}}{%
\pgfpathmoveto{\pgfqpoint{0.000000in}{0.000000in}}%
\pgfpathlineto{\pgfqpoint{0.000000in}{-0.048611in}}%
\pgfusepath{stroke,fill}%
}%
\begin{pgfscope}%
\pgfsys@transformshift{1.149091in}{0.499444in}%
\pgfsys@useobject{currentmarker}{}%
\end{pgfscope}%
\end{pgfscope}%
\begin{pgfscope}%
\pgfsetbuttcap%
\pgfsetroundjoin%
\definecolor{currentfill}{rgb}{0.000000,0.000000,0.000000}%
\pgfsetfillcolor{currentfill}%
\pgfsetlinewidth{0.803000pt}%
\definecolor{currentstroke}{rgb}{0.000000,0.000000,0.000000}%
\pgfsetstrokecolor{currentstroke}%
\pgfsetdash{}{0pt}%
\pgfsys@defobject{currentmarker}{\pgfqpoint{0.000000in}{-0.048611in}}{\pgfqpoint{0.000000in}{0.000000in}}{%
\pgfpathmoveto{\pgfqpoint{0.000000in}{0.000000in}}%
\pgfpathlineto{\pgfqpoint{0.000000in}{-0.048611in}}%
\pgfusepath{stroke,fill}%
}%
\begin{pgfscope}%
\pgfsys@transformshift{1.307614in}{0.499444in}%
\pgfsys@useobject{currentmarker}{}%
\end{pgfscope}%
\end{pgfscope}%
\begin{pgfscope}%
\definecolor{textcolor}{rgb}{0.000000,0.000000,0.000000}%
\pgfsetstrokecolor{textcolor}%
\pgfsetfillcolor{textcolor}%
\pgftext[x=1.307614in,y=0.402222in,,top]{\color{textcolor}\rmfamily\fontsize{10.000000}{12.000000}\selectfont 0.2}%
\end{pgfscope}%
\begin{pgfscope}%
\pgfsetbuttcap%
\pgfsetroundjoin%
\definecolor{currentfill}{rgb}{0.000000,0.000000,0.000000}%
\pgfsetfillcolor{currentfill}%
\pgfsetlinewidth{0.803000pt}%
\definecolor{currentstroke}{rgb}{0.000000,0.000000,0.000000}%
\pgfsetstrokecolor{currentstroke}%
\pgfsetdash{}{0pt}%
\pgfsys@defobject{currentmarker}{\pgfqpoint{0.000000in}{-0.048611in}}{\pgfqpoint{0.000000in}{0.000000in}}{%
\pgfpathmoveto{\pgfqpoint{0.000000in}{0.000000in}}%
\pgfpathlineto{\pgfqpoint{0.000000in}{-0.048611in}}%
\pgfusepath{stroke,fill}%
}%
\begin{pgfscope}%
\pgfsys@transformshift{1.466137in}{0.499444in}%
\pgfsys@useobject{currentmarker}{}%
\end{pgfscope}%
\end{pgfscope}%
\begin{pgfscope}%
\pgfsetbuttcap%
\pgfsetroundjoin%
\definecolor{currentfill}{rgb}{0.000000,0.000000,0.000000}%
\pgfsetfillcolor{currentfill}%
\pgfsetlinewidth{0.803000pt}%
\definecolor{currentstroke}{rgb}{0.000000,0.000000,0.000000}%
\pgfsetstrokecolor{currentstroke}%
\pgfsetdash{}{0pt}%
\pgfsys@defobject{currentmarker}{\pgfqpoint{0.000000in}{-0.048611in}}{\pgfqpoint{0.000000in}{0.000000in}}{%
\pgfpathmoveto{\pgfqpoint{0.000000in}{0.000000in}}%
\pgfpathlineto{\pgfqpoint{0.000000in}{-0.048611in}}%
\pgfusepath{stroke,fill}%
}%
\begin{pgfscope}%
\pgfsys@transformshift{1.624659in}{0.499444in}%
\pgfsys@useobject{currentmarker}{}%
\end{pgfscope}%
\end{pgfscope}%
\begin{pgfscope}%
\definecolor{textcolor}{rgb}{0.000000,0.000000,0.000000}%
\pgfsetstrokecolor{textcolor}%
\pgfsetfillcolor{textcolor}%
\pgftext[x=1.624659in,y=0.402222in,,top]{\color{textcolor}\rmfamily\fontsize{10.000000}{12.000000}\selectfont 0.3}%
\end{pgfscope}%
\begin{pgfscope}%
\pgfsetbuttcap%
\pgfsetroundjoin%
\definecolor{currentfill}{rgb}{0.000000,0.000000,0.000000}%
\pgfsetfillcolor{currentfill}%
\pgfsetlinewidth{0.803000pt}%
\definecolor{currentstroke}{rgb}{0.000000,0.000000,0.000000}%
\pgfsetstrokecolor{currentstroke}%
\pgfsetdash{}{0pt}%
\pgfsys@defobject{currentmarker}{\pgfqpoint{0.000000in}{-0.048611in}}{\pgfqpoint{0.000000in}{0.000000in}}{%
\pgfpathmoveto{\pgfqpoint{0.000000in}{0.000000in}}%
\pgfpathlineto{\pgfqpoint{0.000000in}{-0.048611in}}%
\pgfusepath{stroke,fill}%
}%
\begin{pgfscope}%
\pgfsys@transformshift{1.783182in}{0.499444in}%
\pgfsys@useobject{currentmarker}{}%
\end{pgfscope}%
\end{pgfscope}%
\begin{pgfscope}%
\pgfsetbuttcap%
\pgfsetroundjoin%
\definecolor{currentfill}{rgb}{0.000000,0.000000,0.000000}%
\pgfsetfillcolor{currentfill}%
\pgfsetlinewidth{0.803000pt}%
\definecolor{currentstroke}{rgb}{0.000000,0.000000,0.000000}%
\pgfsetstrokecolor{currentstroke}%
\pgfsetdash{}{0pt}%
\pgfsys@defobject{currentmarker}{\pgfqpoint{0.000000in}{-0.048611in}}{\pgfqpoint{0.000000in}{0.000000in}}{%
\pgfpathmoveto{\pgfqpoint{0.000000in}{0.000000in}}%
\pgfpathlineto{\pgfqpoint{0.000000in}{-0.048611in}}%
\pgfusepath{stroke,fill}%
}%
\begin{pgfscope}%
\pgfsys@transformshift{1.941705in}{0.499444in}%
\pgfsys@useobject{currentmarker}{}%
\end{pgfscope}%
\end{pgfscope}%
\begin{pgfscope}%
\definecolor{textcolor}{rgb}{0.000000,0.000000,0.000000}%
\pgfsetstrokecolor{textcolor}%
\pgfsetfillcolor{textcolor}%
\pgftext[x=1.941705in,y=0.402222in,,top]{\color{textcolor}\rmfamily\fontsize{10.000000}{12.000000}\selectfont 0.4}%
\end{pgfscope}%
\begin{pgfscope}%
\pgfsetbuttcap%
\pgfsetroundjoin%
\definecolor{currentfill}{rgb}{0.000000,0.000000,0.000000}%
\pgfsetfillcolor{currentfill}%
\pgfsetlinewidth{0.803000pt}%
\definecolor{currentstroke}{rgb}{0.000000,0.000000,0.000000}%
\pgfsetstrokecolor{currentstroke}%
\pgfsetdash{}{0pt}%
\pgfsys@defobject{currentmarker}{\pgfqpoint{0.000000in}{-0.048611in}}{\pgfqpoint{0.000000in}{0.000000in}}{%
\pgfpathmoveto{\pgfqpoint{0.000000in}{0.000000in}}%
\pgfpathlineto{\pgfqpoint{0.000000in}{-0.048611in}}%
\pgfusepath{stroke,fill}%
}%
\begin{pgfscope}%
\pgfsys@transformshift{2.100228in}{0.499444in}%
\pgfsys@useobject{currentmarker}{}%
\end{pgfscope}%
\end{pgfscope}%
\begin{pgfscope}%
\pgfsetbuttcap%
\pgfsetroundjoin%
\definecolor{currentfill}{rgb}{0.000000,0.000000,0.000000}%
\pgfsetfillcolor{currentfill}%
\pgfsetlinewidth{0.803000pt}%
\definecolor{currentstroke}{rgb}{0.000000,0.000000,0.000000}%
\pgfsetstrokecolor{currentstroke}%
\pgfsetdash{}{0pt}%
\pgfsys@defobject{currentmarker}{\pgfqpoint{0.000000in}{-0.048611in}}{\pgfqpoint{0.000000in}{0.000000in}}{%
\pgfpathmoveto{\pgfqpoint{0.000000in}{0.000000in}}%
\pgfpathlineto{\pgfqpoint{0.000000in}{-0.048611in}}%
\pgfusepath{stroke,fill}%
}%
\begin{pgfscope}%
\pgfsys@transformshift{2.258750in}{0.499444in}%
\pgfsys@useobject{currentmarker}{}%
\end{pgfscope}%
\end{pgfscope}%
\begin{pgfscope}%
\definecolor{textcolor}{rgb}{0.000000,0.000000,0.000000}%
\pgfsetstrokecolor{textcolor}%
\pgfsetfillcolor{textcolor}%
\pgftext[x=2.258750in,y=0.402222in,,top]{\color{textcolor}\rmfamily\fontsize{10.000000}{12.000000}\selectfont 0.5}%
\end{pgfscope}%
\begin{pgfscope}%
\pgfsetbuttcap%
\pgfsetroundjoin%
\definecolor{currentfill}{rgb}{0.000000,0.000000,0.000000}%
\pgfsetfillcolor{currentfill}%
\pgfsetlinewidth{0.803000pt}%
\definecolor{currentstroke}{rgb}{0.000000,0.000000,0.000000}%
\pgfsetstrokecolor{currentstroke}%
\pgfsetdash{}{0pt}%
\pgfsys@defobject{currentmarker}{\pgfqpoint{0.000000in}{-0.048611in}}{\pgfqpoint{0.000000in}{0.000000in}}{%
\pgfpathmoveto{\pgfqpoint{0.000000in}{0.000000in}}%
\pgfpathlineto{\pgfqpoint{0.000000in}{-0.048611in}}%
\pgfusepath{stroke,fill}%
}%
\begin{pgfscope}%
\pgfsys@transformshift{2.417273in}{0.499444in}%
\pgfsys@useobject{currentmarker}{}%
\end{pgfscope}%
\end{pgfscope}%
\begin{pgfscope}%
\pgfsetbuttcap%
\pgfsetroundjoin%
\definecolor{currentfill}{rgb}{0.000000,0.000000,0.000000}%
\pgfsetfillcolor{currentfill}%
\pgfsetlinewidth{0.803000pt}%
\definecolor{currentstroke}{rgb}{0.000000,0.000000,0.000000}%
\pgfsetstrokecolor{currentstroke}%
\pgfsetdash{}{0pt}%
\pgfsys@defobject{currentmarker}{\pgfqpoint{0.000000in}{-0.048611in}}{\pgfqpoint{0.000000in}{0.000000in}}{%
\pgfpathmoveto{\pgfqpoint{0.000000in}{0.000000in}}%
\pgfpathlineto{\pgfqpoint{0.000000in}{-0.048611in}}%
\pgfusepath{stroke,fill}%
}%
\begin{pgfscope}%
\pgfsys@transformshift{2.575796in}{0.499444in}%
\pgfsys@useobject{currentmarker}{}%
\end{pgfscope}%
\end{pgfscope}%
\begin{pgfscope}%
\definecolor{textcolor}{rgb}{0.000000,0.000000,0.000000}%
\pgfsetstrokecolor{textcolor}%
\pgfsetfillcolor{textcolor}%
\pgftext[x=2.575796in,y=0.402222in,,top]{\color{textcolor}\rmfamily\fontsize{10.000000}{12.000000}\selectfont 0.6}%
\end{pgfscope}%
\begin{pgfscope}%
\pgfsetbuttcap%
\pgfsetroundjoin%
\definecolor{currentfill}{rgb}{0.000000,0.000000,0.000000}%
\pgfsetfillcolor{currentfill}%
\pgfsetlinewidth{0.803000pt}%
\definecolor{currentstroke}{rgb}{0.000000,0.000000,0.000000}%
\pgfsetstrokecolor{currentstroke}%
\pgfsetdash{}{0pt}%
\pgfsys@defobject{currentmarker}{\pgfqpoint{0.000000in}{-0.048611in}}{\pgfqpoint{0.000000in}{0.000000in}}{%
\pgfpathmoveto{\pgfqpoint{0.000000in}{0.000000in}}%
\pgfpathlineto{\pgfqpoint{0.000000in}{-0.048611in}}%
\pgfusepath{stroke,fill}%
}%
\begin{pgfscope}%
\pgfsys@transformshift{2.734318in}{0.499444in}%
\pgfsys@useobject{currentmarker}{}%
\end{pgfscope}%
\end{pgfscope}%
\begin{pgfscope}%
\pgfsetbuttcap%
\pgfsetroundjoin%
\definecolor{currentfill}{rgb}{0.000000,0.000000,0.000000}%
\pgfsetfillcolor{currentfill}%
\pgfsetlinewidth{0.803000pt}%
\definecolor{currentstroke}{rgb}{0.000000,0.000000,0.000000}%
\pgfsetstrokecolor{currentstroke}%
\pgfsetdash{}{0pt}%
\pgfsys@defobject{currentmarker}{\pgfqpoint{0.000000in}{-0.048611in}}{\pgfqpoint{0.000000in}{0.000000in}}{%
\pgfpathmoveto{\pgfqpoint{0.000000in}{0.000000in}}%
\pgfpathlineto{\pgfqpoint{0.000000in}{-0.048611in}}%
\pgfusepath{stroke,fill}%
}%
\begin{pgfscope}%
\pgfsys@transformshift{2.892841in}{0.499444in}%
\pgfsys@useobject{currentmarker}{}%
\end{pgfscope}%
\end{pgfscope}%
\begin{pgfscope}%
\definecolor{textcolor}{rgb}{0.000000,0.000000,0.000000}%
\pgfsetstrokecolor{textcolor}%
\pgfsetfillcolor{textcolor}%
\pgftext[x=2.892841in,y=0.402222in,,top]{\color{textcolor}\rmfamily\fontsize{10.000000}{12.000000}\selectfont 0.7}%
\end{pgfscope}%
\begin{pgfscope}%
\pgfsetbuttcap%
\pgfsetroundjoin%
\definecolor{currentfill}{rgb}{0.000000,0.000000,0.000000}%
\pgfsetfillcolor{currentfill}%
\pgfsetlinewidth{0.803000pt}%
\definecolor{currentstroke}{rgb}{0.000000,0.000000,0.000000}%
\pgfsetstrokecolor{currentstroke}%
\pgfsetdash{}{0pt}%
\pgfsys@defobject{currentmarker}{\pgfqpoint{0.000000in}{-0.048611in}}{\pgfqpoint{0.000000in}{0.000000in}}{%
\pgfpathmoveto{\pgfqpoint{0.000000in}{0.000000in}}%
\pgfpathlineto{\pgfqpoint{0.000000in}{-0.048611in}}%
\pgfusepath{stroke,fill}%
}%
\begin{pgfscope}%
\pgfsys@transformshift{3.051364in}{0.499444in}%
\pgfsys@useobject{currentmarker}{}%
\end{pgfscope}%
\end{pgfscope}%
\begin{pgfscope}%
\pgfsetbuttcap%
\pgfsetroundjoin%
\definecolor{currentfill}{rgb}{0.000000,0.000000,0.000000}%
\pgfsetfillcolor{currentfill}%
\pgfsetlinewidth{0.803000pt}%
\definecolor{currentstroke}{rgb}{0.000000,0.000000,0.000000}%
\pgfsetstrokecolor{currentstroke}%
\pgfsetdash{}{0pt}%
\pgfsys@defobject{currentmarker}{\pgfqpoint{0.000000in}{-0.048611in}}{\pgfqpoint{0.000000in}{0.000000in}}{%
\pgfpathmoveto{\pgfqpoint{0.000000in}{0.000000in}}%
\pgfpathlineto{\pgfqpoint{0.000000in}{-0.048611in}}%
\pgfusepath{stroke,fill}%
}%
\begin{pgfscope}%
\pgfsys@transformshift{3.209887in}{0.499444in}%
\pgfsys@useobject{currentmarker}{}%
\end{pgfscope}%
\end{pgfscope}%
\begin{pgfscope}%
\definecolor{textcolor}{rgb}{0.000000,0.000000,0.000000}%
\pgfsetstrokecolor{textcolor}%
\pgfsetfillcolor{textcolor}%
\pgftext[x=3.209887in,y=0.402222in,,top]{\color{textcolor}\rmfamily\fontsize{10.000000}{12.000000}\selectfont 0.8}%
\end{pgfscope}%
\begin{pgfscope}%
\pgfsetbuttcap%
\pgfsetroundjoin%
\definecolor{currentfill}{rgb}{0.000000,0.000000,0.000000}%
\pgfsetfillcolor{currentfill}%
\pgfsetlinewidth{0.803000pt}%
\definecolor{currentstroke}{rgb}{0.000000,0.000000,0.000000}%
\pgfsetstrokecolor{currentstroke}%
\pgfsetdash{}{0pt}%
\pgfsys@defobject{currentmarker}{\pgfqpoint{0.000000in}{-0.048611in}}{\pgfqpoint{0.000000in}{0.000000in}}{%
\pgfpathmoveto{\pgfqpoint{0.000000in}{0.000000in}}%
\pgfpathlineto{\pgfqpoint{0.000000in}{-0.048611in}}%
\pgfusepath{stroke,fill}%
}%
\begin{pgfscope}%
\pgfsys@transformshift{3.368409in}{0.499444in}%
\pgfsys@useobject{currentmarker}{}%
\end{pgfscope}%
\end{pgfscope}%
\begin{pgfscope}%
\pgfsetbuttcap%
\pgfsetroundjoin%
\definecolor{currentfill}{rgb}{0.000000,0.000000,0.000000}%
\pgfsetfillcolor{currentfill}%
\pgfsetlinewidth{0.803000pt}%
\definecolor{currentstroke}{rgb}{0.000000,0.000000,0.000000}%
\pgfsetstrokecolor{currentstroke}%
\pgfsetdash{}{0pt}%
\pgfsys@defobject{currentmarker}{\pgfqpoint{0.000000in}{-0.048611in}}{\pgfqpoint{0.000000in}{0.000000in}}{%
\pgfpathmoveto{\pgfqpoint{0.000000in}{0.000000in}}%
\pgfpathlineto{\pgfqpoint{0.000000in}{-0.048611in}}%
\pgfusepath{stroke,fill}%
}%
\begin{pgfscope}%
\pgfsys@transformshift{3.526932in}{0.499444in}%
\pgfsys@useobject{currentmarker}{}%
\end{pgfscope}%
\end{pgfscope}%
\begin{pgfscope}%
\definecolor{textcolor}{rgb}{0.000000,0.000000,0.000000}%
\pgfsetstrokecolor{textcolor}%
\pgfsetfillcolor{textcolor}%
\pgftext[x=3.526932in,y=0.402222in,,top]{\color{textcolor}\rmfamily\fontsize{10.000000}{12.000000}\selectfont 0.9}%
\end{pgfscope}%
\begin{pgfscope}%
\pgfsetbuttcap%
\pgfsetroundjoin%
\definecolor{currentfill}{rgb}{0.000000,0.000000,0.000000}%
\pgfsetfillcolor{currentfill}%
\pgfsetlinewidth{0.803000pt}%
\definecolor{currentstroke}{rgb}{0.000000,0.000000,0.000000}%
\pgfsetstrokecolor{currentstroke}%
\pgfsetdash{}{0pt}%
\pgfsys@defobject{currentmarker}{\pgfqpoint{0.000000in}{-0.048611in}}{\pgfqpoint{0.000000in}{0.000000in}}{%
\pgfpathmoveto{\pgfqpoint{0.000000in}{0.000000in}}%
\pgfpathlineto{\pgfqpoint{0.000000in}{-0.048611in}}%
\pgfusepath{stroke,fill}%
}%
\begin{pgfscope}%
\pgfsys@transformshift{3.685455in}{0.499444in}%
\pgfsys@useobject{currentmarker}{}%
\end{pgfscope}%
\end{pgfscope}%
\begin{pgfscope}%
\pgfsetbuttcap%
\pgfsetroundjoin%
\definecolor{currentfill}{rgb}{0.000000,0.000000,0.000000}%
\pgfsetfillcolor{currentfill}%
\pgfsetlinewidth{0.803000pt}%
\definecolor{currentstroke}{rgb}{0.000000,0.000000,0.000000}%
\pgfsetstrokecolor{currentstroke}%
\pgfsetdash{}{0pt}%
\pgfsys@defobject{currentmarker}{\pgfqpoint{0.000000in}{-0.048611in}}{\pgfqpoint{0.000000in}{0.000000in}}{%
\pgfpathmoveto{\pgfqpoint{0.000000in}{0.000000in}}%
\pgfpathlineto{\pgfqpoint{0.000000in}{-0.048611in}}%
\pgfusepath{stroke,fill}%
}%
\begin{pgfscope}%
\pgfsys@transformshift{3.843978in}{0.499444in}%
\pgfsys@useobject{currentmarker}{}%
\end{pgfscope}%
\end{pgfscope}%
\begin{pgfscope}%
\definecolor{textcolor}{rgb}{0.000000,0.000000,0.000000}%
\pgfsetstrokecolor{textcolor}%
\pgfsetfillcolor{textcolor}%
\pgftext[x=3.843978in,y=0.402222in,,top]{\color{textcolor}\rmfamily\fontsize{10.000000}{12.000000}\selectfont 1.0}%
\end{pgfscope}%
\begin{pgfscope}%
\pgfsetbuttcap%
\pgfsetroundjoin%
\definecolor{currentfill}{rgb}{0.000000,0.000000,0.000000}%
\pgfsetfillcolor{currentfill}%
\pgfsetlinewidth{0.803000pt}%
\definecolor{currentstroke}{rgb}{0.000000,0.000000,0.000000}%
\pgfsetstrokecolor{currentstroke}%
\pgfsetdash{}{0pt}%
\pgfsys@defobject{currentmarker}{\pgfqpoint{0.000000in}{-0.048611in}}{\pgfqpoint{0.000000in}{0.000000in}}{%
\pgfpathmoveto{\pgfqpoint{0.000000in}{0.000000in}}%
\pgfpathlineto{\pgfqpoint{0.000000in}{-0.048611in}}%
\pgfusepath{stroke,fill}%
}%
\begin{pgfscope}%
\pgfsys@transformshift{4.002500in}{0.499444in}%
\pgfsys@useobject{currentmarker}{}%
\end{pgfscope}%
\end{pgfscope}%
\begin{pgfscope}%
\definecolor{textcolor}{rgb}{0.000000,0.000000,0.000000}%
\pgfsetstrokecolor{textcolor}%
\pgfsetfillcolor{textcolor}%
\pgftext[x=2.258750in,y=0.223333in,,top]{\color{textcolor}\rmfamily\fontsize{10.000000}{12.000000}\selectfont \(\displaystyle p\)}%
\end{pgfscope}%
\begin{pgfscope}%
\pgfsetbuttcap%
\pgfsetroundjoin%
\definecolor{currentfill}{rgb}{0.000000,0.000000,0.000000}%
\pgfsetfillcolor{currentfill}%
\pgfsetlinewidth{0.803000pt}%
\definecolor{currentstroke}{rgb}{0.000000,0.000000,0.000000}%
\pgfsetstrokecolor{currentstroke}%
\pgfsetdash{}{0pt}%
\pgfsys@defobject{currentmarker}{\pgfqpoint{-0.048611in}{0.000000in}}{\pgfqpoint{-0.000000in}{0.000000in}}{%
\pgfpathmoveto{\pgfqpoint{-0.000000in}{0.000000in}}%
\pgfpathlineto{\pgfqpoint{-0.048611in}{0.000000in}}%
\pgfusepath{stroke,fill}%
}%
\begin{pgfscope}%
\pgfsys@transformshift{0.515000in}{0.499444in}%
\pgfsys@useobject{currentmarker}{}%
\end{pgfscope}%
\end{pgfscope}%
\begin{pgfscope}%
\definecolor{textcolor}{rgb}{0.000000,0.000000,0.000000}%
\pgfsetstrokecolor{textcolor}%
\pgfsetfillcolor{textcolor}%
\pgftext[x=0.348333in, y=0.451250in, left, base]{\color{textcolor}\rmfamily\fontsize{10.000000}{12.000000}\selectfont \(\displaystyle {0}\)}%
\end{pgfscope}%
\begin{pgfscope}%
\pgfsetbuttcap%
\pgfsetroundjoin%
\definecolor{currentfill}{rgb}{0.000000,0.000000,0.000000}%
\pgfsetfillcolor{currentfill}%
\pgfsetlinewidth{0.803000pt}%
\definecolor{currentstroke}{rgb}{0.000000,0.000000,0.000000}%
\pgfsetstrokecolor{currentstroke}%
\pgfsetdash{}{0pt}%
\pgfsys@defobject{currentmarker}{\pgfqpoint{-0.048611in}{0.000000in}}{\pgfqpoint{-0.000000in}{0.000000in}}{%
\pgfpathmoveto{\pgfqpoint{-0.000000in}{0.000000in}}%
\pgfpathlineto{\pgfqpoint{-0.048611in}{0.000000in}}%
\pgfusepath{stroke,fill}%
}%
\begin{pgfscope}%
\pgfsys@transformshift{0.515000in}{0.926967in}%
\pgfsys@useobject{currentmarker}{}%
\end{pgfscope}%
\end{pgfscope}%
\begin{pgfscope}%
\definecolor{textcolor}{rgb}{0.000000,0.000000,0.000000}%
\pgfsetstrokecolor{textcolor}%
\pgfsetfillcolor{textcolor}%
\pgftext[x=0.278889in, y=0.878772in, left, base]{\color{textcolor}\rmfamily\fontsize{10.000000}{12.000000}\selectfont \(\displaystyle {10}\)}%
\end{pgfscope}%
\begin{pgfscope}%
\pgfsetbuttcap%
\pgfsetroundjoin%
\definecolor{currentfill}{rgb}{0.000000,0.000000,0.000000}%
\pgfsetfillcolor{currentfill}%
\pgfsetlinewidth{0.803000pt}%
\definecolor{currentstroke}{rgb}{0.000000,0.000000,0.000000}%
\pgfsetstrokecolor{currentstroke}%
\pgfsetdash{}{0pt}%
\pgfsys@defobject{currentmarker}{\pgfqpoint{-0.048611in}{0.000000in}}{\pgfqpoint{-0.000000in}{0.000000in}}{%
\pgfpathmoveto{\pgfqpoint{-0.000000in}{0.000000in}}%
\pgfpathlineto{\pgfqpoint{-0.048611in}{0.000000in}}%
\pgfusepath{stroke,fill}%
}%
\begin{pgfscope}%
\pgfsys@transformshift{0.515000in}{1.354489in}%
\pgfsys@useobject{currentmarker}{}%
\end{pgfscope}%
\end{pgfscope}%
\begin{pgfscope}%
\definecolor{textcolor}{rgb}{0.000000,0.000000,0.000000}%
\pgfsetstrokecolor{textcolor}%
\pgfsetfillcolor{textcolor}%
\pgftext[x=0.278889in, y=1.306295in, left, base]{\color{textcolor}\rmfamily\fontsize{10.000000}{12.000000}\selectfont \(\displaystyle {20}\)}%
\end{pgfscope}%
\begin{pgfscope}%
\definecolor{textcolor}{rgb}{0.000000,0.000000,0.000000}%
\pgfsetstrokecolor{textcolor}%
\pgfsetfillcolor{textcolor}%
\pgftext[x=0.223333in,y=1.076944in,,bottom,rotate=90.000000]{\color{textcolor}\rmfamily\fontsize{10.000000}{12.000000}\selectfont Percent of Data Set}%
\end{pgfscope}%
\begin{pgfscope}%
\pgfsetrectcap%
\pgfsetmiterjoin%
\pgfsetlinewidth{0.803000pt}%
\definecolor{currentstroke}{rgb}{0.000000,0.000000,0.000000}%
\pgfsetstrokecolor{currentstroke}%
\pgfsetdash{}{0pt}%
\pgfpathmoveto{\pgfqpoint{0.515000in}{0.499444in}}%
\pgfpathlineto{\pgfqpoint{0.515000in}{1.654444in}}%
\pgfusepath{stroke}%
\end{pgfscope}%
\begin{pgfscope}%
\pgfsetrectcap%
\pgfsetmiterjoin%
\pgfsetlinewidth{0.803000pt}%
\definecolor{currentstroke}{rgb}{0.000000,0.000000,0.000000}%
\pgfsetstrokecolor{currentstroke}%
\pgfsetdash{}{0pt}%
\pgfpathmoveto{\pgfqpoint{4.002500in}{0.499444in}}%
\pgfpathlineto{\pgfqpoint{4.002500in}{1.654444in}}%
\pgfusepath{stroke}%
\end{pgfscope}%
\begin{pgfscope}%
\pgfsetrectcap%
\pgfsetmiterjoin%
\pgfsetlinewidth{0.803000pt}%
\definecolor{currentstroke}{rgb}{0.000000,0.000000,0.000000}%
\pgfsetstrokecolor{currentstroke}%
\pgfsetdash{}{0pt}%
\pgfpathmoveto{\pgfqpoint{0.515000in}{0.499444in}}%
\pgfpathlineto{\pgfqpoint{4.002500in}{0.499444in}}%
\pgfusepath{stroke}%
\end{pgfscope}%
\begin{pgfscope}%
\pgfsetrectcap%
\pgfsetmiterjoin%
\pgfsetlinewidth{0.803000pt}%
\definecolor{currentstroke}{rgb}{0.000000,0.000000,0.000000}%
\pgfsetstrokecolor{currentstroke}%
\pgfsetdash{}{0pt}%
\pgfpathmoveto{\pgfqpoint{0.515000in}{1.654444in}}%
\pgfpathlineto{\pgfqpoint{4.002500in}{1.654444in}}%
\pgfusepath{stroke}%
\end{pgfscope}%
\begin{pgfscope}%
\pgfsetbuttcap%
\pgfsetmiterjoin%
\definecolor{currentfill}{rgb}{1.000000,1.000000,1.000000}%
\pgfsetfillcolor{currentfill}%
\pgfsetfillopacity{0.800000}%
\pgfsetlinewidth{1.003750pt}%
\definecolor{currentstroke}{rgb}{0.800000,0.800000,0.800000}%
\pgfsetstrokecolor{currentstroke}%
\pgfsetstrokeopacity{0.800000}%
\pgfsetdash{}{0pt}%
\pgfpathmoveto{\pgfqpoint{3.225556in}{1.154445in}}%
\pgfpathlineto{\pgfqpoint{3.905278in}{1.154445in}}%
\pgfpathquadraticcurveto{\pgfqpoint{3.933056in}{1.154445in}}{\pgfqpoint{3.933056in}{1.182222in}}%
\pgfpathlineto{\pgfqpoint{3.933056in}{1.557222in}}%
\pgfpathquadraticcurveto{\pgfqpoint{3.933056in}{1.585000in}}{\pgfqpoint{3.905278in}{1.585000in}}%
\pgfpathlineto{\pgfqpoint{3.225556in}{1.585000in}}%
\pgfpathquadraticcurveto{\pgfqpoint{3.197778in}{1.585000in}}{\pgfqpoint{3.197778in}{1.557222in}}%
\pgfpathlineto{\pgfqpoint{3.197778in}{1.182222in}}%
\pgfpathquadraticcurveto{\pgfqpoint{3.197778in}{1.154445in}}{\pgfqpoint{3.225556in}{1.154445in}}%
\pgfpathlineto{\pgfqpoint{3.225556in}{1.154445in}}%
\pgfpathclose%
\pgfusepath{stroke,fill}%
\end{pgfscope}%
\begin{pgfscope}%
\pgfsetbuttcap%
\pgfsetmiterjoin%
\pgfsetlinewidth{1.003750pt}%
\definecolor{currentstroke}{rgb}{0.000000,0.000000,0.000000}%
\pgfsetstrokecolor{currentstroke}%
\pgfsetdash{}{0pt}%
\pgfpathmoveto{\pgfqpoint{3.253334in}{1.432222in}}%
\pgfpathlineto{\pgfqpoint{3.531111in}{1.432222in}}%
\pgfpathlineto{\pgfqpoint{3.531111in}{1.529444in}}%
\pgfpathlineto{\pgfqpoint{3.253334in}{1.529444in}}%
\pgfpathlineto{\pgfqpoint{3.253334in}{1.432222in}}%
\pgfpathclose%
\pgfusepath{stroke}%
\end{pgfscope}%
\begin{pgfscope}%
\definecolor{textcolor}{rgb}{0.000000,0.000000,0.000000}%
\pgfsetstrokecolor{textcolor}%
\pgfsetfillcolor{textcolor}%
\pgftext[x=3.642223in,y=1.432222in,left,base]{\color{textcolor}\rmfamily\fontsize{10.000000}{12.000000}\selectfont Neg}%
\end{pgfscope}%
\begin{pgfscope}%
\pgfsetbuttcap%
\pgfsetmiterjoin%
\definecolor{currentfill}{rgb}{0.000000,0.000000,0.000000}%
\pgfsetfillcolor{currentfill}%
\pgfsetlinewidth{0.000000pt}%
\definecolor{currentstroke}{rgb}{0.000000,0.000000,0.000000}%
\pgfsetstrokecolor{currentstroke}%
\pgfsetstrokeopacity{0.000000}%
\pgfsetdash{}{0pt}%
\pgfpathmoveto{\pgfqpoint{3.253334in}{1.236944in}}%
\pgfpathlineto{\pgfqpoint{3.531111in}{1.236944in}}%
\pgfpathlineto{\pgfqpoint{3.531111in}{1.334167in}}%
\pgfpathlineto{\pgfqpoint{3.253334in}{1.334167in}}%
\pgfpathlineto{\pgfqpoint{3.253334in}{1.236944in}}%
\pgfpathclose%
\pgfusepath{fill}%
\end{pgfscope}%
\begin{pgfscope}%
\definecolor{textcolor}{rgb}{0.000000,0.000000,0.000000}%
\pgfsetstrokecolor{textcolor}%
\pgfsetfillcolor{textcolor}%
\pgftext[x=3.642223in,y=1.236944in,left,base]{\color{textcolor}\rmfamily\fontsize{10.000000}{12.000000}\selectfont Pos}%
\end{pgfscope}%
\end{pgfpicture}%
\makeatother%
\endgroup%
	
&
	\vskip 0pt
	\hfil ROC Curve
	
	%% Creator: Matplotlib, PGF backend
%%
%% To include the figure in your LaTeX document, write
%%   \input{<filename>.pgf}
%%
%% Make sure the required packages are loaded in your preamble
%%   \usepackage{pgf}
%%
%% Also ensure that all the required font packages are loaded; for instance,
%% the lmodern package is sometimes necessary when using math font.
%%   \usepackage{lmodern}
%%
%% Figures using additional raster images can only be included by \input if
%% they are in the same directory as the main LaTeX file. For loading figures
%% from other directories you can use the `import` package
%%   \usepackage{import}
%%
%% and then include the figures with
%%   \import{<path to file>}{<filename>.pgf}
%%
%% Matplotlib used the following preamble
%%   
%%   \usepackage{fontspec}
%%   \makeatletter\@ifpackageloaded{underscore}{}{\usepackage[strings]{underscore}}\makeatother
%%
\begingroup%
\makeatletter%
\begin{pgfpicture}%
\pgfpathrectangle{\pgfpointorigin}{\pgfqpoint{2.221861in}{1.754444in}}%
\pgfusepath{use as bounding box, clip}%
\begin{pgfscope}%
\pgfsetbuttcap%
\pgfsetmiterjoin%
\definecolor{currentfill}{rgb}{1.000000,1.000000,1.000000}%
\pgfsetfillcolor{currentfill}%
\pgfsetlinewidth{0.000000pt}%
\definecolor{currentstroke}{rgb}{1.000000,1.000000,1.000000}%
\pgfsetstrokecolor{currentstroke}%
\pgfsetdash{}{0pt}%
\pgfpathmoveto{\pgfqpoint{0.000000in}{0.000000in}}%
\pgfpathlineto{\pgfqpoint{2.221861in}{0.000000in}}%
\pgfpathlineto{\pgfqpoint{2.221861in}{1.754444in}}%
\pgfpathlineto{\pgfqpoint{0.000000in}{1.754444in}}%
\pgfpathlineto{\pgfqpoint{0.000000in}{0.000000in}}%
\pgfpathclose%
\pgfusepath{fill}%
\end{pgfscope}%
\begin{pgfscope}%
\pgfsetbuttcap%
\pgfsetmiterjoin%
\definecolor{currentfill}{rgb}{1.000000,1.000000,1.000000}%
\pgfsetfillcolor{currentfill}%
\pgfsetlinewidth{0.000000pt}%
\definecolor{currentstroke}{rgb}{0.000000,0.000000,0.000000}%
\pgfsetstrokecolor{currentstroke}%
\pgfsetstrokeopacity{0.000000}%
\pgfsetdash{}{0pt}%
\pgfpathmoveto{\pgfqpoint{0.553581in}{0.499444in}}%
\pgfpathlineto{\pgfqpoint{2.103581in}{0.499444in}}%
\pgfpathlineto{\pgfqpoint{2.103581in}{1.654444in}}%
\pgfpathlineto{\pgfqpoint{0.553581in}{1.654444in}}%
\pgfpathlineto{\pgfqpoint{0.553581in}{0.499444in}}%
\pgfpathclose%
\pgfusepath{fill}%
\end{pgfscope}%
\begin{pgfscope}%
\pgfsetbuttcap%
\pgfsetroundjoin%
\definecolor{currentfill}{rgb}{0.000000,0.000000,0.000000}%
\pgfsetfillcolor{currentfill}%
\pgfsetlinewidth{0.803000pt}%
\definecolor{currentstroke}{rgb}{0.000000,0.000000,0.000000}%
\pgfsetstrokecolor{currentstroke}%
\pgfsetdash{}{0pt}%
\pgfsys@defobject{currentmarker}{\pgfqpoint{0.000000in}{-0.048611in}}{\pgfqpoint{0.000000in}{0.000000in}}{%
\pgfpathmoveto{\pgfqpoint{0.000000in}{0.000000in}}%
\pgfpathlineto{\pgfqpoint{0.000000in}{-0.048611in}}%
\pgfusepath{stroke,fill}%
}%
\begin{pgfscope}%
\pgfsys@transformshift{0.624035in}{0.499444in}%
\pgfsys@useobject{currentmarker}{}%
\end{pgfscope}%
\end{pgfscope}%
\begin{pgfscope}%
\definecolor{textcolor}{rgb}{0.000000,0.000000,0.000000}%
\pgfsetstrokecolor{textcolor}%
\pgfsetfillcolor{textcolor}%
\pgftext[x=0.624035in,y=0.402222in,,top]{\color{textcolor}\rmfamily\fontsize{10.000000}{12.000000}\selectfont \(\displaystyle {0.0}\)}%
\end{pgfscope}%
\begin{pgfscope}%
\pgfsetbuttcap%
\pgfsetroundjoin%
\definecolor{currentfill}{rgb}{0.000000,0.000000,0.000000}%
\pgfsetfillcolor{currentfill}%
\pgfsetlinewidth{0.803000pt}%
\definecolor{currentstroke}{rgb}{0.000000,0.000000,0.000000}%
\pgfsetstrokecolor{currentstroke}%
\pgfsetdash{}{0pt}%
\pgfsys@defobject{currentmarker}{\pgfqpoint{0.000000in}{-0.048611in}}{\pgfqpoint{0.000000in}{0.000000in}}{%
\pgfpathmoveto{\pgfqpoint{0.000000in}{0.000000in}}%
\pgfpathlineto{\pgfqpoint{0.000000in}{-0.048611in}}%
\pgfusepath{stroke,fill}%
}%
\begin{pgfscope}%
\pgfsys@transformshift{1.328581in}{0.499444in}%
\pgfsys@useobject{currentmarker}{}%
\end{pgfscope}%
\end{pgfscope}%
\begin{pgfscope}%
\definecolor{textcolor}{rgb}{0.000000,0.000000,0.000000}%
\pgfsetstrokecolor{textcolor}%
\pgfsetfillcolor{textcolor}%
\pgftext[x=1.328581in,y=0.402222in,,top]{\color{textcolor}\rmfamily\fontsize{10.000000}{12.000000}\selectfont \(\displaystyle {0.5}\)}%
\end{pgfscope}%
\begin{pgfscope}%
\pgfsetbuttcap%
\pgfsetroundjoin%
\definecolor{currentfill}{rgb}{0.000000,0.000000,0.000000}%
\pgfsetfillcolor{currentfill}%
\pgfsetlinewidth{0.803000pt}%
\definecolor{currentstroke}{rgb}{0.000000,0.000000,0.000000}%
\pgfsetstrokecolor{currentstroke}%
\pgfsetdash{}{0pt}%
\pgfsys@defobject{currentmarker}{\pgfqpoint{0.000000in}{-0.048611in}}{\pgfqpoint{0.000000in}{0.000000in}}{%
\pgfpathmoveto{\pgfqpoint{0.000000in}{0.000000in}}%
\pgfpathlineto{\pgfqpoint{0.000000in}{-0.048611in}}%
\pgfusepath{stroke,fill}%
}%
\begin{pgfscope}%
\pgfsys@transformshift{2.033126in}{0.499444in}%
\pgfsys@useobject{currentmarker}{}%
\end{pgfscope}%
\end{pgfscope}%
\begin{pgfscope}%
\definecolor{textcolor}{rgb}{0.000000,0.000000,0.000000}%
\pgfsetstrokecolor{textcolor}%
\pgfsetfillcolor{textcolor}%
\pgftext[x=2.033126in,y=0.402222in,,top]{\color{textcolor}\rmfamily\fontsize{10.000000}{12.000000}\selectfont \(\displaystyle {1.0}\)}%
\end{pgfscope}%
\begin{pgfscope}%
\definecolor{textcolor}{rgb}{0.000000,0.000000,0.000000}%
\pgfsetstrokecolor{textcolor}%
\pgfsetfillcolor{textcolor}%
\pgftext[x=1.328581in,y=0.223333in,,top]{\color{textcolor}\rmfamily\fontsize{10.000000}{12.000000}\selectfont False positive rate}%
\end{pgfscope}%
\begin{pgfscope}%
\pgfsetbuttcap%
\pgfsetroundjoin%
\definecolor{currentfill}{rgb}{0.000000,0.000000,0.000000}%
\pgfsetfillcolor{currentfill}%
\pgfsetlinewidth{0.803000pt}%
\definecolor{currentstroke}{rgb}{0.000000,0.000000,0.000000}%
\pgfsetstrokecolor{currentstroke}%
\pgfsetdash{}{0pt}%
\pgfsys@defobject{currentmarker}{\pgfqpoint{-0.048611in}{0.000000in}}{\pgfqpoint{-0.000000in}{0.000000in}}{%
\pgfpathmoveto{\pgfqpoint{-0.000000in}{0.000000in}}%
\pgfpathlineto{\pgfqpoint{-0.048611in}{0.000000in}}%
\pgfusepath{stroke,fill}%
}%
\begin{pgfscope}%
\pgfsys@transformshift{0.553581in}{0.551944in}%
\pgfsys@useobject{currentmarker}{}%
\end{pgfscope}%
\end{pgfscope}%
\begin{pgfscope}%
\definecolor{textcolor}{rgb}{0.000000,0.000000,0.000000}%
\pgfsetstrokecolor{textcolor}%
\pgfsetfillcolor{textcolor}%
\pgftext[x=0.278889in, y=0.503750in, left, base]{\color{textcolor}\rmfamily\fontsize{10.000000}{12.000000}\selectfont \(\displaystyle {0.0}\)}%
\end{pgfscope}%
\begin{pgfscope}%
\pgfsetbuttcap%
\pgfsetroundjoin%
\definecolor{currentfill}{rgb}{0.000000,0.000000,0.000000}%
\pgfsetfillcolor{currentfill}%
\pgfsetlinewidth{0.803000pt}%
\definecolor{currentstroke}{rgb}{0.000000,0.000000,0.000000}%
\pgfsetstrokecolor{currentstroke}%
\pgfsetdash{}{0pt}%
\pgfsys@defobject{currentmarker}{\pgfqpoint{-0.048611in}{0.000000in}}{\pgfqpoint{-0.000000in}{0.000000in}}{%
\pgfpathmoveto{\pgfqpoint{-0.000000in}{0.000000in}}%
\pgfpathlineto{\pgfqpoint{-0.048611in}{0.000000in}}%
\pgfusepath{stroke,fill}%
}%
\begin{pgfscope}%
\pgfsys@transformshift{0.553581in}{1.076944in}%
\pgfsys@useobject{currentmarker}{}%
\end{pgfscope}%
\end{pgfscope}%
\begin{pgfscope}%
\definecolor{textcolor}{rgb}{0.000000,0.000000,0.000000}%
\pgfsetstrokecolor{textcolor}%
\pgfsetfillcolor{textcolor}%
\pgftext[x=0.278889in, y=1.028750in, left, base]{\color{textcolor}\rmfamily\fontsize{10.000000}{12.000000}\selectfont \(\displaystyle {0.5}\)}%
\end{pgfscope}%
\begin{pgfscope}%
\pgfsetbuttcap%
\pgfsetroundjoin%
\definecolor{currentfill}{rgb}{0.000000,0.000000,0.000000}%
\pgfsetfillcolor{currentfill}%
\pgfsetlinewidth{0.803000pt}%
\definecolor{currentstroke}{rgb}{0.000000,0.000000,0.000000}%
\pgfsetstrokecolor{currentstroke}%
\pgfsetdash{}{0pt}%
\pgfsys@defobject{currentmarker}{\pgfqpoint{-0.048611in}{0.000000in}}{\pgfqpoint{-0.000000in}{0.000000in}}{%
\pgfpathmoveto{\pgfqpoint{-0.000000in}{0.000000in}}%
\pgfpathlineto{\pgfqpoint{-0.048611in}{0.000000in}}%
\pgfusepath{stroke,fill}%
}%
\begin{pgfscope}%
\pgfsys@transformshift{0.553581in}{1.601944in}%
\pgfsys@useobject{currentmarker}{}%
\end{pgfscope}%
\end{pgfscope}%
\begin{pgfscope}%
\definecolor{textcolor}{rgb}{0.000000,0.000000,0.000000}%
\pgfsetstrokecolor{textcolor}%
\pgfsetfillcolor{textcolor}%
\pgftext[x=0.278889in, y=1.553750in, left, base]{\color{textcolor}\rmfamily\fontsize{10.000000}{12.000000}\selectfont \(\displaystyle {1.0}\)}%
\end{pgfscope}%
\begin{pgfscope}%
\definecolor{textcolor}{rgb}{0.000000,0.000000,0.000000}%
\pgfsetstrokecolor{textcolor}%
\pgfsetfillcolor{textcolor}%
\pgftext[x=0.223333in,y=1.076944in,,bottom,rotate=90.000000]{\color{textcolor}\rmfamily\fontsize{10.000000}{12.000000}\selectfont True positive rate}%
\end{pgfscope}%
\begin{pgfscope}%
\pgfpathrectangle{\pgfqpoint{0.553581in}{0.499444in}}{\pgfqpoint{1.550000in}{1.155000in}}%
\pgfusepath{clip}%
\pgfsetbuttcap%
\pgfsetroundjoin%
\pgfsetlinewidth{1.505625pt}%
\definecolor{currentstroke}{rgb}{0.000000,0.000000,0.000000}%
\pgfsetstrokecolor{currentstroke}%
\pgfsetdash{{5.550000pt}{2.400000pt}}{0.000000pt}%
\pgfpathmoveto{\pgfqpoint{0.624035in}{0.551944in}}%
\pgfpathlineto{\pgfqpoint{2.033126in}{1.601944in}}%
\pgfusepath{stroke}%
\end{pgfscope}%
\begin{pgfscope}%
\pgfpathrectangle{\pgfqpoint{0.553581in}{0.499444in}}{\pgfqpoint{1.550000in}{1.155000in}}%
\pgfusepath{clip}%
\pgfsetrectcap%
\pgfsetroundjoin%
\pgfsetlinewidth{1.505625pt}%
\definecolor{currentstroke}{rgb}{0.000000,0.000000,0.000000}%
\pgfsetstrokecolor{currentstroke}%
\pgfsetdash{}{0pt}%
\pgfpathmoveto{\pgfqpoint{0.624035in}{0.551944in}}%
\pgfpathlineto{\pgfqpoint{0.624103in}{0.553034in}}%
\pgfpathlineto{\pgfqpoint{0.625210in}{0.569908in}}%
\pgfpathlineto{\pgfqpoint{0.625299in}{0.571007in}}%
\pgfpathlineto{\pgfqpoint{0.626409in}{0.583514in}}%
\pgfpathlineto{\pgfqpoint{0.626526in}{0.584594in}}%
\pgfpathlineto{\pgfqpoint{0.627633in}{0.595546in}}%
\pgfpathlineto{\pgfqpoint{0.627750in}{0.596644in}}%
\pgfpathlineto{\pgfqpoint{0.628857in}{0.607130in}}%
\pgfpathlineto{\pgfqpoint{0.628963in}{0.608155in}}%
\pgfpathlineto{\pgfqpoint{0.630067in}{0.617486in}}%
\pgfpathlineto{\pgfqpoint{0.630220in}{0.618585in}}%
\pgfpathlineto{\pgfqpoint{0.630220in}{0.618594in}}%
\pgfpathlineto{\pgfqpoint{0.631460in}{0.628046in}}%
\pgfpathlineto{\pgfqpoint{0.632570in}{0.635720in}}%
\pgfpathlineto{\pgfqpoint{0.632722in}{0.636754in}}%
\pgfpathlineto{\pgfqpoint{0.633832in}{0.644371in}}%
\pgfpathlineto{\pgfqpoint{0.634045in}{0.645433in}}%
\pgfpathlineto{\pgfqpoint{0.635154in}{0.652808in}}%
\pgfpathlineto{\pgfqpoint{0.635342in}{0.653907in}}%
\pgfpathlineto{\pgfqpoint{0.636451in}{0.660668in}}%
\pgfpathlineto{\pgfqpoint{0.636620in}{0.661776in}}%
\pgfpathlineto{\pgfqpoint{0.637729in}{0.667913in}}%
\pgfpathlineto{\pgfqpoint{0.637905in}{0.669022in}}%
\pgfpathlineto{\pgfqpoint{0.639012in}{0.674311in}}%
\pgfpathlineto{\pgfqpoint{0.639193in}{0.675419in}}%
\pgfpathlineto{\pgfqpoint{0.640300in}{0.680979in}}%
\pgfpathlineto{\pgfqpoint{0.640302in}{0.680979in}}%
\pgfpathlineto{\pgfqpoint{0.640513in}{0.682087in}}%
\pgfpathlineto{\pgfqpoint{0.641623in}{0.687386in}}%
\pgfpathlineto{\pgfqpoint{0.641867in}{0.688485in}}%
\pgfpathlineto{\pgfqpoint{0.642974in}{0.693365in}}%
\pgfpathlineto{\pgfqpoint{0.643217in}{0.694473in}}%
\pgfpathlineto{\pgfqpoint{0.644327in}{0.699343in}}%
\pgfpathlineto{\pgfqpoint{0.644528in}{0.700442in}}%
\pgfpathlineto{\pgfqpoint{0.645638in}{0.705387in}}%
\pgfpathlineto{\pgfqpoint{0.645898in}{0.706495in}}%
\pgfpathlineto{\pgfqpoint{0.647007in}{0.711124in}}%
\pgfpathlineto{\pgfqpoint{0.647265in}{0.712232in}}%
\pgfpathlineto{\pgfqpoint{0.659836in}{0.757621in}}%
\pgfpathlineto{\pgfqpoint{0.660167in}{0.758711in}}%
\pgfpathlineto{\pgfqpoint{0.660167in}{0.758729in}}%
\pgfpathlineto{\pgfqpoint{0.661276in}{0.762920in}}%
\pgfpathlineto{\pgfqpoint{0.661640in}{0.764028in}}%
\pgfpathlineto{\pgfqpoint{0.662749in}{0.767558in}}%
\pgfpathlineto{\pgfqpoint{0.663131in}{0.768666in}}%
\pgfpathlineto{\pgfqpoint{0.664238in}{0.772056in}}%
\pgfpathlineto{\pgfqpoint{0.664609in}{0.773145in}}%
\pgfpathlineto{\pgfqpoint{0.665714in}{0.776787in}}%
\pgfpathlineto{\pgfqpoint{0.666075in}{0.777885in}}%
\pgfpathlineto{\pgfqpoint{0.667182in}{0.782113in}}%
\pgfpathlineto{\pgfqpoint{0.667557in}{0.783221in}}%
\pgfpathlineto{\pgfqpoint{0.668666in}{0.786183in}}%
\pgfpathlineto{\pgfqpoint{0.669037in}{0.787263in}}%
\pgfpathlineto{\pgfqpoint{0.670132in}{0.790839in}}%
\pgfpathlineto{\pgfqpoint{0.670146in}{0.790839in}}%
\pgfpathlineto{\pgfqpoint{0.670486in}{0.791947in}}%
\pgfpathlineto{\pgfqpoint{0.671596in}{0.795104in}}%
\pgfpathlineto{\pgfqpoint{0.671975in}{0.796203in}}%
\pgfpathlineto{\pgfqpoint{0.673080in}{0.799444in}}%
\pgfpathlineto{\pgfqpoint{0.673507in}{0.800552in}}%
\pgfpathlineto{\pgfqpoint{0.674609in}{0.803532in}}%
\pgfpathlineto{\pgfqpoint{0.675050in}{0.804640in}}%
\pgfpathlineto{\pgfqpoint{0.676152in}{0.807881in}}%
\pgfpathlineto{\pgfqpoint{0.676565in}{0.808989in}}%
\pgfpathlineto{\pgfqpoint{0.677670in}{0.812146in}}%
\pgfpathlineto{\pgfqpoint{0.678162in}{0.813236in}}%
\pgfpathlineto{\pgfqpoint{0.679260in}{0.816430in}}%
\pgfpathlineto{\pgfqpoint{0.679703in}{0.817538in}}%
\pgfpathlineto{\pgfqpoint{0.680813in}{0.820351in}}%
\pgfpathlineto{\pgfqpoint{0.681319in}{0.821440in}}%
\pgfpathlineto{\pgfqpoint{0.682428in}{0.824420in}}%
\pgfpathlineto{\pgfqpoint{0.682790in}{0.825528in}}%
\pgfpathlineto{\pgfqpoint{0.683897in}{0.828527in}}%
\pgfpathlineto{\pgfqpoint{0.684342in}{0.829626in}}%
\pgfpathlineto{\pgfqpoint{0.685452in}{0.832476in}}%
\pgfpathlineto{\pgfqpoint{0.686031in}{0.833574in}}%
\pgfpathlineto{\pgfqpoint{0.687140in}{0.836387in}}%
\pgfpathlineto{\pgfqpoint{0.687569in}{0.837495in}}%
\pgfpathlineto{\pgfqpoint{0.688679in}{0.840429in}}%
\pgfpathlineto{\pgfqpoint{0.689134in}{0.841527in}}%
\pgfpathlineto{\pgfqpoint{0.690238in}{0.844265in}}%
\pgfpathlineto{\pgfqpoint{0.690628in}{0.845373in}}%
\pgfpathlineto{\pgfqpoint{0.691735in}{0.848111in}}%
\pgfpathlineto{\pgfqpoint{0.692218in}{0.849220in}}%
\pgfpathlineto{\pgfqpoint{0.693322in}{0.852200in}}%
\pgfpathlineto{\pgfqpoint{0.693859in}{0.853308in}}%
\pgfpathlineto{\pgfqpoint{0.696707in}{0.860106in}}%
\pgfpathlineto{\pgfqpoint{0.697155in}{0.861214in}}%
\pgfpathlineto{\pgfqpoint{0.698248in}{0.863589in}}%
\pgfpathlineto{\pgfqpoint{0.698843in}{0.864688in}}%
\pgfpathlineto{\pgfqpoint{0.699953in}{0.866979in}}%
\pgfpathlineto{\pgfqpoint{0.700464in}{0.868087in}}%
\pgfpathlineto{\pgfqpoint{0.701564in}{0.870732in}}%
\pgfpathlineto{\pgfqpoint{0.702091in}{0.871840in}}%
\pgfpathlineto{\pgfqpoint{0.703201in}{0.874317in}}%
\pgfpathlineto{\pgfqpoint{0.703665in}{0.875416in}}%
\pgfpathlineto{\pgfqpoint{0.704770in}{0.878303in}}%
\pgfpathlineto{\pgfqpoint{0.705201in}{0.879411in}}%
\pgfpathlineto{\pgfqpoint{0.706311in}{0.882112in}}%
\pgfpathlineto{\pgfqpoint{0.706768in}{0.883192in}}%
\pgfpathlineto{\pgfqpoint{0.707877in}{0.885809in}}%
\pgfpathlineto{\pgfqpoint{0.708452in}{0.886917in}}%
\pgfpathlineto{\pgfqpoint{0.709559in}{0.889170in}}%
\pgfpathlineto{\pgfqpoint{0.710138in}{0.890279in}}%
\pgfpathlineto{\pgfqpoint{0.711240in}{0.892607in}}%
\pgfpathlineto{\pgfqpoint{0.711817in}{0.893715in}}%
\pgfpathlineto{\pgfqpoint{0.712920in}{0.896192in}}%
\pgfpathlineto{\pgfqpoint{0.713541in}{0.897300in}}%
\pgfpathlineto{\pgfqpoint{0.714646in}{0.899936in}}%
\pgfpathlineto{\pgfqpoint{0.715145in}{0.901035in}}%
\pgfpathlineto{\pgfqpoint{0.716250in}{0.903549in}}%
\pgfpathlineto{\pgfqpoint{0.716857in}{0.904657in}}%
\pgfpathlineto{\pgfqpoint{0.717960in}{0.906920in}}%
\pgfpathlineto{\pgfqpoint{0.718635in}{0.908019in}}%
\pgfpathlineto{\pgfqpoint{0.719742in}{0.910235in}}%
\pgfpathlineto{\pgfqpoint{0.720216in}{0.911325in}}%
\pgfpathlineto{\pgfqpoint{0.721318in}{0.913849in}}%
\pgfpathlineto{\pgfqpoint{0.721954in}{0.914938in}}%
\pgfpathlineto{\pgfqpoint{0.723063in}{0.917332in}}%
\pgfpathlineto{\pgfqpoint{0.723640in}{0.918412in}}%
\pgfpathlineto{\pgfqpoint{0.724749in}{0.920442in}}%
\pgfpathlineto{\pgfqpoint{0.725498in}{0.921541in}}%
\pgfpathlineto{\pgfqpoint{0.726602in}{0.923739in}}%
\pgfpathlineto{\pgfqpoint{0.727158in}{0.924828in}}%
\pgfpathlineto{\pgfqpoint{0.728265in}{0.926914in}}%
\pgfpathlineto{\pgfqpoint{0.728821in}{0.928022in}}%
\pgfpathlineto{\pgfqpoint{0.729911in}{0.930323in}}%
\pgfpathlineto{\pgfqpoint{0.730549in}{0.931431in}}%
\pgfpathlineto{\pgfqpoint{0.731633in}{0.933414in}}%
\pgfpathlineto{\pgfqpoint{0.732304in}{0.934523in}}%
\pgfpathlineto{\pgfqpoint{0.733401in}{0.936795in}}%
\pgfpathlineto{\pgfqpoint{0.734018in}{0.937903in}}%
\pgfpathlineto{\pgfqpoint{0.735118in}{0.939924in}}%
\pgfpathlineto{\pgfqpoint{0.735822in}{0.941032in}}%
\pgfpathlineto{\pgfqpoint{0.736924in}{0.943053in}}%
\pgfpathlineto{\pgfqpoint{0.737480in}{0.944161in}}%
\pgfpathlineto{\pgfqpoint{0.738582in}{0.946499in}}%
\pgfpathlineto{\pgfqpoint{0.739171in}{0.947607in}}%
\pgfpathlineto{\pgfqpoint{0.740278in}{0.949860in}}%
\pgfpathlineto{\pgfqpoint{0.740904in}{0.950969in}}%
\pgfpathlineto{\pgfqpoint{0.742008in}{0.953036in}}%
\pgfpathlineto{\pgfqpoint{0.742789in}{0.954144in}}%
\pgfpathlineto{\pgfqpoint{0.743896in}{0.956370in}}%
\pgfpathlineto{\pgfqpoint{0.744391in}{0.957478in}}%
\pgfpathlineto{\pgfqpoint{0.746683in}{0.962041in}}%
\pgfpathlineto{\pgfqpoint{0.747358in}{0.963149in}}%
\pgfpathlineto{\pgfqpoint{0.748453in}{0.965273in}}%
\pgfpathlineto{\pgfqpoint{0.749157in}{0.966381in}}%
\pgfpathlineto{\pgfqpoint{0.750266in}{0.968215in}}%
\pgfpathlineto{\pgfqpoint{0.750855in}{0.969324in}}%
\pgfpathlineto{\pgfqpoint{0.751964in}{0.970972in}}%
\pgfpathlineto{\pgfqpoint{0.752602in}{0.972080in}}%
\pgfpathlineto{\pgfqpoint{0.753712in}{0.973812in}}%
\pgfpathlineto{\pgfqpoint{0.754413in}{0.974920in}}%
\pgfpathlineto{\pgfqpoint{0.755522in}{0.976746in}}%
\pgfpathlineto{\pgfqpoint{0.756113in}{0.977854in}}%
\pgfpathlineto{\pgfqpoint{0.757222in}{0.979875in}}%
\pgfpathlineto{\pgfqpoint{0.758022in}{0.980974in}}%
\pgfpathlineto{\pgfqpoint{0.759132in}{0.982920in}}%
\pgfpathlineto{\pgfqpoint{0.759905in}{0.984028in}}%
\pgfpathlineto{\pgfqpoint{0.761012in}{0.986030in}}%
\pgfpathlineto{\pgfqpoint{0.761681in}{0.987129in}}%
\pgfpathlineto{\pgfqpoint{0.762783in}{0.988917in}}%
\pgfpathlineto{\pgfqpoint{0.763487in}{0.990025in}}%
\pgfpathlineto{\pgfqpoint{0.764589in}{0.992000in}}%
\pgfpathlineto{\pgfqpoint{0.765391in}{0.993108in}}%
\pgfpathlineto{\pgfqpoint{0.766498in}{0.995315in}}%
\pgfpathlineto{\pgfqpoint{0.767056in}{0.996414in}}%
\pgfpathlineto{\pgfqpoint{0.768159in}{0.998165in}}%
\pgfpathlineto{\pgfqpoint{0.768933in}{0.999273in}}%
\pgfpathlineto{\pgfqpoint{0.770771in}{1.002104in}}%
\pgfpathlineto{\pgfqpoint{0.771510in}{1.003212in}}%
\pgfpathlineto{\pgfqpoint{0.772612in}{1.004786in}}%
\pgfpathlineto{\pgfqpoint{0.773480in}{1.005894in}}%
\pgfpathlineto{\pgfqpoint{0.774585in}{1.007719in}}%
\pgfpathlineto{\pgfqpoint{0.775220in}{1.008818in}}%
\pgfpathlineto{\pgfqpoint{0.776330in}{1.010746in}}%
\pgfpathlineto{\pgfqpoint{0.777183in}{1.011854in}}%
\pgfpathlineto{\pgfqpoint{0.778293in}{1.013363in}}%
\pgfpathlineto{\pgfqpoint{0.778865in}{1.014471in}}%
\pgfpathlineto{\pgfqpoint{0.779962in}{1.016222in}}%
\pgfpathlineto{\pgfqpoint{0.780861in}{1.017330in}}%
\pgfpathlineto{\pgfqpoint{0.781963in}{1.019127in}}%
\pgfpathlineto{\pgfqpoint{0.781970in}{1.019127in}}%
\pgfpathlineto{\pgfqpoint{0.782683in}{1.020226in}}%
\pgfpathlineto{\pgfqpoint{0.783792in}{1.021735in}}%
\pgfpathlineto{\pgfqpoint{0.784594in}{1.022843in}}%
\pgfpathlineto{\pgfqpoint{0.785704in}{1.024575in}}%
\pgfpathlineto{\pgfqpoint{0.786379in}{1.025683in}}%
\pgfpathlineto{\pgfqpoint{0.787489in}{1.027220in}}%
\pgfpathlineto{\pgfqpoint{0.788209in}{1.028319in}}%
\pgfpathlineto{\pgfqpoint{0.789301in}{1.029967in}}%
\pgfpathlineto{\pgfqpoint{0.789316in}{1.029967in}}%
\pgfpathlineto{\pgfqpoint{0.790075in}{1.031075in}}%
\pgfpathlineto{\pgfqpoint{0.791171in}{1.032388in}}%
\pgfpathlineto{\pgfqpoint{0.791863in}{1.033496in}}%
\pgfpathlineto{\pgfqpoint{0.792969in}{1.035145in}}%
\pgfpathlineto{\pgfqpoint{0.793753in}{1.036253in}}%
\pgfpathlineto{\pgfqpoint{0.794862in}{1.037957in}}%
\pgfpathlineto{\pgfqpoint{0.795746in}{1.039056in}}%
\pgfpathlineto{\pgfqpoint{0.796853in}{1.040639in}}%
\pgfpathlineto{\pgfqpoint{0.797608in}{1.041738in}}%
\pgfpathlineto{\pgfqpoint{0.798715in}{1.043358in}}%
\pgfpathlineto{\pgfqpoint{0.799541in}{1.044457in}}%
\pgfpathlineto{\pgfqpoint{0.800646in}{1.045966in}}%
\pgfpathlineto{\pgfqpoint{0.801518in}{1.047074in}}%
\pgfpathlineto{\pgfqpoint{0.802627in}{1.048592in}}%
\pgfpathlineto{\pgfqpoint{0.803483in}{1.049691in}}%
\pgfpathlineto{\pgfqpoint{0.804593in}{1.051116in}}%
\pgfpathlineto{\pgfqpoint{0.805503in}{1.052224in}}%
\pgfpathlineto{\pgfqpoint{0.806612in}{1.053723in}}%
\pgfpathlineto{\pgfqpoint{0.807466in}{1.054832in}}%
\pgfpathlineto{\pgfqpoint{0.808573in}{1.056359in}}%
\pgfpathlineto{\pgfqpoint{0.809258in}{1.057467in}}%
\pgfpathlineto{\pgfqpoint{0.810358in}{1.058882in}}%
\pgfpathlineto{\pgfqpoint{0.811195in}{1.059991in}}%
\pgfpathlineto{\pgfqpoint{0.812295in}{1.061453in}}%
\pgfpathlineto{\pgfqpoint{0.813163in}{1.062552in}}%
\pgfpathlineto{\pgfqpoint{0.814272in}{1.064293in}}%
\pgfpathlineto{\pgfqpoint{0.815093in}{1.065373in}}%
\pgfpathlineto{\pgfqpoint{0.816202in}{1.066947in}}%
\pgfpathlineto{\pgfqpoint{0.817044in}{1.068055in}}%
\pgfpathlineto{\pgfqpoint{0.818149in}{1.069732in}}%
\pgfpathlineto{\pgfqpoint{0.819030in}{1.070840in}}%
\pgfpathlineto{\pgfqpoint{0.820130in}{1.072320in}}%
\pgfpathlineto{\pgfqpoint{0.820940in}{1.073410in}}%
\pgfpathlineto{\pgfqpoint{0.822049in}{1.074658in}}%
\pgfpathlineto{\pgfqpoint{0.822973in}{1.075766in}}%
\pgfpathlineto{\pgfqpoint{0.824080in}{1.077126in}}%
\pgfpathlineto{\pgfqpoint{0.824948in}{1.078234in}}%
\pgfpathlineto{\pgfqpoint{0.826055in}{1.079556in}}%
\pgfpathlineto{\pgfqpoint{0.826836in}{1.080665in}}%
\pgfpathlineto{\pgfqpoint{0.827931in}{1.082266in}}%
\pgfpathlineto{\pgfqpoint{0.828848in}{1.083374in}}%
\pgfpathlineto{\pgfqpoint{0.829941in}{1.084697in}}%
\pgfpathlineto{\pgfqpoint{0.830839in}{1.085796in}}%
\pgfpathlineto{\pgfqpoint{0.831946in}{1.087351in}}%
\pgfpathlineto{\pgfqpoint{0.832945in}{1.088459in}}%
\pgfpathlineto{\pgfqpoint{0.834050in}{1.089977in}}%
\pgfpathlineto{\pgfqpoint{0.835002in}{1.091085in}}%
\pgfpathlineto{\pgfqpoint{0.836109in}{1.092482in}}%
\pgfpathlineto{\pgfqpoint{0.836979in}{1.093590in}}%
\pgfpathlineto{\pgfqpoint{0.838081in}{1.095043in}}%
\pgfpathlineto{\pgfqpoint{0.838933in}{1.096151in}}%
\pgfpathlineto{\pgfqpoint{0.840026in}{1.097548in}}%
\pgfpathlineto{\pgfqpoint{0.841114in}{1.098656in}}%
\pgfpathlineto{\pgfqpoint{0.842223in}{1.100165in}}%
\pgfpathlineto{\pgfqpoint{0.842957in}{1.101273in}}%
\pgfpathlineto{\pgfqpoint{0.844057in}{1.102977in}}%
\pgfpathlineto{\pgfqpoint{0.845098in}{1.104086in}}%
\pgfpathlineto{\pgfqpoint{0.846189in}{1.105455in}}%
\pgfpathlineto{\pgfqpoint{0.847064in}{1.106544in}}%
\pgfpathlineto{\pgfqpoint{0.848171in}{1.108006in}}%
\pgfpathlineto{\pgfqpoint{0.849358in}{1.109114in}}%
\pgfpathlineto{\pgfqpoint{0.850464in}{1.110390in}}%
\pgfpathlineto{\pgfqpoint{0.851438in}{1.111498in}}%
\pgfpathlineto{\pgfqpoint{0.852547in}{1.112960in}}%
\pgfpathlineto{\pgfqpoint{0.853509in}{1.114059in}}%
\pgfpathlineto{\pgfqpoint{0.853509in}{1.114069in}}%
\pgfpathlineto{\pgfqpoint{0.855103in}{1.115913in}}%
\pgfpathlineto{\pgfqpoint{0.856023in}{1.117021in}}%
\pgfpathlineto{\pgfqpoint{0.857123in}{1.118371in}}%
\pgfpathlineto{\pgfqpoint{0.858103in}{1.119470in}}%
\pgfpathlineto{\pgfqpoint{0.859210in}{1.120820in}}%
\pgfpathlineto{\pgfqpoint{0.859212in}{1.120820in}}%
\pgfpathlineto{\pgfqpoint{0.860209in}{1.121928in}}%
\pgfpathlineto{\pgfqpoint{0.861316in}{1.123251in}}%
\pgfpathlineto{\pgfqpoint{0.862390in}{1.124359in}}%
\pgfpathlineto{\pgfqpoint{0.863497in}{1.125709in}}%
\pgfpathlineto{\pgfqpoint{0.864583in}{1.126818in}}%
\pgfpathlineto{\pgfqpoint{0.865676in}{1.128112in}}%
\pgfpathlineto{\pgfqpoint{0.866633in}{1.129220in}}%
\pgfpathlineto{\pgfqpoint{0.867714in}{1.130505in}}%
\pgfpathlineto{\pgfqpoint{0.868645in}{1.131604in}}%
\pgfpathlineto{\pgfqpoint{0.869752in}{1.132908in}}%
\pgfpathlineto{\pgfqpoint{0.870791in}{1.134007in}}%
\pgfpathlineto{\pgfqpoint{0.871882in}{1.135366in}}%
\pgfpathlineto{\pgfqpoint{0.872792in}{1.136475in}}%
\pgfpathlineto{\pgfqpoint{0.873894in}{1.137974in}}%
\pgfpathlineto{\pgfqpoint{0.874816in}{1.139082in}}%
\pgfpathlineto{\pgfqpoint{0.875904in}{1.140311in}}%
\pgfpathlineto{\pgfqpoint{0.875913in}{1.140311in}}%
\pgfpathlineto{\pgfqpoint{0.876964in}{1.141420in}}%
\pgfpathlineto{\pgfqpoint{0.878055in}{1.142556in}}%
\pgfpathlineto{\pgfqpoint{0.878986in}{1.143664in}}%
\pgfpathlineto{\pgfqpoint{0.880079in}{1.144716in}}%
\pgfpathlineto{\pgfqpoint{0.881164in}{1.145824in}}%
\pgfpathlineto{\pgfqpoint{0.882269in}{1.147212in}}%
\pgfpathlineto{\pgfqpoint{0.883425in}{1.148320in}}%
\pgfpathlineto{\pgfqpoint{0.884521in}{1.149615in}}%
\pgfpathlineto{\pgfqpoint{0.885740in}{1.150723in}}%
\pgfpathlineto{\pgfqpoint{0.886842in}{1.151943in}}%
\pgfpathlineto{\pgfqpoint{0.887741in}{1.153051in}}%
\pgfpathlineto{\pgfqpoint{0.888848in}{1.154075in}}%
\pgfpathlineto{\pgfqpoint{0.890006in}{1.155184in}}%
\pgfpathlineto{\pgfqpoint{0.891099in}{1.156133in}}%
\pgfpathlineto{\pgfqpoint{0.892026in}{1.157242in}}%
\pgfpathlineto{\pgfqpoint{0.893121in}{1.158285in}}%
\pgfpathlineto{\pgfqpoint{0.894153in}{1.159393in}}%
\pgfpathlineto{\pgfqpoint{0.895257in}{1.160734in}}%
\pgfpathlineto{\pgfqpoint{0.896437in}{1.161842in}}%
\pgfpathlineto{\pgfqpoint{0.897546in}{1.163034in}}%
\pgfpathlineto{\pgfqpoint{0.898531in}{1.164142in}}%
\pgfpathlineto{\pgfqpoint{0.899636in}{1.165111in}}%
\pgfpathlineto{\pgfqpoint{0.900720in}{1.166210in}}%
\pgfpathlineto{\pgfqpoint{0.901815in}{1.167327in}}%
\pgfpathlineto{\pgfqpoint{0.902950in}{1.168435in}}%
\pgfpathlineto{\pgfqpoint{0.904055in}{1.169860in}}%
\pgfpathlineto{\pgfqpoint{0.905140in}{1.170959in}}%
\pgfpathlineto{\pgfqpoint{0.906250in}{1.172328in}}%
\pgfpathlineto{\pgfqpoint{0.907282in}{1.173427in}}%
\pgfpathlineto{\pgfqpoint{0.908382in}{1.174842in}}%
\pgfpathlineto{\pgfqpoint{0.909453in}{1.175951in}}%
\pgfpathlineto{\pgfqpoint{0.910563in}{1.177198in}}%
\pgfpathlineto{\pgfqpoint{0.911768in}{1.178307in}}%
\pgfpathlineto{\pgfqpoint{0.912864in}{1.179312in}}%
\pgfpathlineto{\pgfqpoint{0.913952in}{1.180402in}}%
\pgfpathlineto{\pgfqpoint{0.915061in}{1.181613in}}%
\pgfpathlineto{\pgfqpoint{0.915882in}{1.182721in}}%
\pgfpathlineto{\pgfqpoint{0.916975in}{1.183838in}}%
\pgfpathlineto{\pgfqpoint{0.918291in}{1.184947in}}%
\pgfpathlineto{\pgfqpoint{0.919393in}{1.186166in}}%
\pgfpathlineto{\pgfqpoint{0.920622in}{1.187275in}}%
\pgfpathlineto{\pgfqpoint{0.921731in}{1.188374in}}%
\pgfpathlineto{\pgfqpoint{0.922817in}{1.189463in}}%
\pgfpathlineto{\pgfqpoint{0.923924in}{1.190832in}}%
\pgfpathlineto{\pgfqpoint{0.923926in}{1.190832in}}%
\pgfpathlineto{\pgfqpoint{0.925416in}{1.192266in}}%
\pgfpathlineto{\pgfqpoint{0.926382in}{1.193365in}}%
\pgfpathlineto{\pgfqpoint{0.927486in}{1.194464in}}%
\pgfpathlineto{\pgfqpoint{0.928633in}{1.195563in}}%
\pgfpathlineto{\pgfqpoint{0.929733in}{1.196429in}}%
\pgfpathlineto{\pgfqpoint{0.931042in}{1.197537in}}%
\pgfpathlineto{\pgfqpoint{0.932140in}{1.198534in}}%
\pgfpathlineto{\pgfqpoint{0.932147in}{1.198534in}}%
\pgfpathlineto{\pgfqpoint{0.933408in}{1.199632in}}%
\pgfpathlineto{\pgfqpoint{0.934497in}{1.200620in}}%
\pgfpathlineto{\pgfqpoint{0.935779in}{1.201728in}}%
\pgfpathlineto{\pgfqpoint{0.936875in}{1.202938in}}%
\pgfpathlineto{\pgfqpoint{0.936879in}{1.202938in}}%
\pgfpathlineto{\pgfqpoint{0.937881in}{1.204047in}}%
\pgfpathlineto{\pgfqpoint{0.938971in}{1.205052in}}%
\pgfpathlineto{\pgfqpoint{0.938976in}{1.205052in}}%
\pgfpathlineto{\pgfqpoint{0.940282in}{1.206161in}}%
\pgfpathlineto{\pgfqpoint{0.941378in}{1.207380in}}%
\pgfpathlineto{\pgfqpoint{0.942639in}{1.208489in}}%
\pgfpathlineto{\pgfqpoint{0.943744in}{1.209578in}}%
\pgfpathlineto{\pgfqpoint{0.944947in}{1.210686in}}%
\pgfpathlineto{\pgfqpoint{0.946057in}{1.211813in}}%
\pgfpathlineto{\pgfqpoint{0.947300in}{1.212921in}}%
\pgfpathlineto{\pgfqpoint{0.948409in}{1.213974in}}%
\pgfpathlineto{\pgfqpoint{0.949488in}{1.215082in}}%
\pgfpathlineto{\pgfqpoint{0.950595in}{1.216190in}}%
\pgfpathlineto{\pgfqpoint{0.951819in}{1.217289in}}%
\pgfpathlineto{\pgfqpoint{0.952926in}{1.218341in}}%
\pgfpathlineto{\pgfqpoint{0.954113in}{1.219450in}}%
\pgfpathlineto{\pgfqpoint{0.955215in}{1.220362in}}%
\pgfpathlineto{\pgfqpoint{0.956411in}{1.221470in}}%
\pgfpathlineto{\pgfqpoint{0.957513in}{1.222513in}}%
\pgfpathlineto{\pgfqpoint{0.958489in}{1.223622in}}%
\pgfpathlineto{\pgfqpoint{0.960020in}{1.225177in}}%
\pgfpathlineto{\pgfqpoint{0.961242in}{1.226285in}}%
\pgfpathlineto{\pgfqpoint{0.962349in}{1.227216in}}%
\pgfpathlineto{\pgfqpoint{0.963567in}{1.228324in}}%
\pgfpathlineto{\pgfqpoint{0.964666in}{1.229358in}}%
\pgfpathlineto{\pgfqpoint{0.965738in}{1.230438in}}%
\pgfpathlineto{\pgfqpoint{0.966838in}{1.231286in}}%
\pgfpathlineto{\pgfqpoint{0.968027in}{1.232385in}}%
\pgfpathlineto{\pgfqpoint{0.969132in}{1.233316in}}%
\pgfpathlineto{\pgfqpoint{0.970370in}{1.234424in}}%
\pgfpathlineto{\pgfqpoint{0.971468in}{1.235430in}}%
\pgfpathlineto{\pgfqpoint{0.972800in}{1.236538in}}%
\pgfpathlineto{\pgfqpoint{0.973902in}{1.237320in}}%
\pgfpathlineto{\pgfqpoint{0.975028in}{1.238429in}}%
\pgfpathlineto{\pgfqpoint{0.976130in}{1.239630in}}%
\pgfpathlineto{\pgfqpoint{0.976137in}{1.239630in}}%
\pgfpathlineto{\pgfqpoint{0.977559in}{1.240738in}}%
\pgfpathlineto{\pgfqpoint{0.978659in}{1.241604in}}%
\pgfpathlineto{\pgfqpoint{0.979927in}{1.242712in}}%
\pgfpathlineto{\pgfqpoint{0.981034in}{1.243727in}}%
\pgfpathlineto{\pgfqpoint{0.982245in}{1.244836in}}%
\pgfpathlineto{\pgfqpoint{0.983337in}{1.245860in}}%
\pgfpathlineto{\pgfqpoint{0.984536in}{1.246968in}}%
\pgfpathlineto{\pgfqpoint{0.985626in}{1.247788in}}%
\pgfpathlineto{\pgfqpoint{0.985640in}{1.247788in}}%
\pgfpathlineto{\pgfqpoint{0.986839in}{1.248896in}}%
\pgfpathlineto{\pgfqpoint{0.987934in}{1.249781in}}%
\pgfpathlineto{\pgfqpoint{0.989168in}{1.250889in}}%
\pgfpathlineto{\pgfqpoint{0.990275in}{1.251876in}}%
\pgfpathlineto{\pgfqpoint{0.991595in}{1.252984in}}%
\pgfpathlineto{\pgfqpoint{0.992665in}{1.253869in}}%
\pgfpathlineto{\pgfqpoint{0.993847in}{1.254977in}}%
\pgfpathlineto{\pgfqpoint{0.994954in}{1.255890in}}%
\pgfpathlineto{\pgfqpoint{0.996635in}{1.256979in}}%
\pgfpathlineto{\pgfqpoint{0.997738in}{1.258050in}}%
\pgfpathlineto{\pgfqpoint{0.999168in}{1.259158in}}%
\pgfpathlineto{\pgfqpoint{1.000270in}{1.259996in}}%
\pgfpathlineto{\pgfqpoint{1.001572in}{1.261105in}}%
\pgfpathlineto{\pgfqpoint{1.002674in}{1.261924in}}%
\pgfpathlineto{\pgfqpoint{1.003880in}{1.263032in}}%
\pgfpathlineto{\pgfqpoint{1.005266in}{1.264057in}}%
\pgfpathlineto{\pgfqpoint{1.006816in}{1.265165in}}%
\pgfpathlineto{\pgfqpoint{1.007918in}{1.265994in}}%
\pgfpathlineto{\pgfqpoint{1.009122in}{1.267102in}}%
\pgfpathlineto{\pgfqpoint{1.010229in}{1.268145in}}%
\pgfpathlineto{\pgfqpoint{1.011868in}{1.269253in}}%
\pgfpathlineto{\pgfqpoint{1.012975in}{1.270175in}}%
\pgfpathlineto{\pgfqpoint{1.014284in}{1.271283in}}%
\pgfpathlineto{\pgfqpoint{1.015391in}{1.272242in}}%
\pgfpathlineto{\pgfqpoint{1.016685in}{1.273341in}}%
\pgfpathlineto{\pgfqpoint{1.017790in}{1.274226in}}%
\pgfpathlineto{\pgfqpoint{1.018955in}{1.275334in}}%
\pgfpathlineto{\pgfqpoint{1.020060in}{1.276070in}}%
\pgfpathlineto{\pgfqpoint{1.021500in}{1.277169in}}%
\pgfpathlineto{\pgfqpoint{1.022591in}{1.278100in}}%
\pgfpathlineto{\pgfqpoint{1.022598in}{1.278100in}}%
\pgfpathlineto{\pgfqpoint{1.023913in}{1.279208in}}%
\pgfpathlineto{\pgfqpoint{1.025002in}{1.280037in}}%
\pgfpathlineto{\pgfqpoint{1.026543in}{1.281145in}}%
\pgfpathlineto{\pgfqpoint{1.027640in}{1.282058in}}%
\pgfpathlineto{\pgfqpoint{1.029158in}{1.283166in}}%
\pgfpathlineto{\pgfqpoint{1.030248in}{1.283846in}}%
\pgfpathlineto{\pgfqpoint{1.031608in}{1.284954in}}%
\pgfpathlineto{\pgfqpoint{1.032701in}{1.285829in}}%
\pgfpathlineto{\pgfqpoint{1.034026in}{1.286938in}}%
\pgfpathlineto{\pgfqpoint{1.035129in}{1.287878in}}%
\pgfpathlineto{\pgfqpoint{1.036667in}{1.288986in}}%
\pgfpathlineto{\pgfqpoint{1.038072in}{1.290048in}}%
\pgfpathlineto{\pgfqpoint{1.039514in}{1.291156in}}%
\pgfpathlineto{\pgfqpoint{1.040619in}{1.291994in}}%
\pgfpathlineto{\pgfqpoint{1.041850in}{1.293093in}}%
\pgfpathlineto{\pgfqpoint{1.042955in}{1.293894in}}%
\pgfpathlineto{\pgfqpoint{1.044890in}{1.295002in}}%
\pgfpathlineto{\pgfqpoint{1.046281in}{1.296017in}}%
\pgfpathlineto{\pgfqpoint{1.048065in}{1.297126in}}%
\pgfpathlineto{\pgfqpoint{1.049170in}{1.297964in}}%
\pgfpathlineto{\pgfqpoint{1.050694in}{1.299072in}}%
\pgfpathlineto{\pgfqpoint{1.051790in}{1.300003in}}%
\pgfpathlineto{\pgfqpoint{1.053340in}{1.301111in}}%
\pgfpathlineto{\pgfqpoint{1.054445in}{1.301773in}}%
\pgfpathlineto{\pgfqpoint{1.056159in}{1.302881in}}%
\pgfpathlineto{\pgfqpoint{1.057250in}{1.303728in}}%
\pgfpathlineto{\pgfqpoint{1.058872in}{1.304836in}}%
\pgfpathlineto{\pgfqpoint{1.059979in}{1.305684in}}%
\pgfpathlineto{\pgfqpoint{1.061647in}{1.306783in}}%
\pgfpathlineto{\pgfqpoint{1.062749in}{1.307621in}}%
\pgfpathlineto{\pgfqpoint{1.064180in}{1.308729in}}%
\pgfpathlineto{\pgfqpoint{1.065249in}{1.309642in}}%
\pgfpathlineto{\pgfqpoint{1.065287in}{1.309642in}}%
\pgfpathlineto{\pgfqpoint{1.066931in}{1.310750in}}%
\pgfpathlineto{\pgfqpoint{1.068036in}{1.311728in}}%
\pgfpathlineto{\pgfqpoint{1.069525in}{1.312836in}}%
\pgfpathlineto{\pgfqpoint{1.070911in}{1.313814in}}%
\pgfpathlineto{\pgfqpoint{1.072412in}{1.314922in}}%
\pgfpathlineto{\pgfqpoint{1.073500in}{1.315658in}}%
\pgfpathlineto{\pgfqpoint{1.073507in}{1.315658in}}%
\pgfpathlineto{\pgfqpoint{1.075186in}{1.316766in}}%
\pgfpathlineto{\pgfqpoint{1.076296in}{1.317576in}}%
\pgfpathlineto{\pgfqpoint{1.077754in}{1.318647in}}%
\pgfpathlineto{\pgfqpoint{1.077754in}{1.318666in}}%
\pgfpathlineto{\pgfqpoint{1.078857in}{1.319364in}}%
\pgfpathlineto{\pgfqpoint{1.080334in}{1.320472in}}%
\pgfpathlineto{\pgfqpoint{1.081444in}{1.321217in}}%
\pgfpathlineto{\pgfqpoint{1.082771in}{1.322325in}}%
\pgfpathlineto{\pgfqpoint{1.083880in}{1.322931in}}%
\pgfpathlineto{\pgfqpoint{1.085112in}{1.324039in}}%
\pgfpathlineto{\pgfqpoint{1.086221in}{1.324775in}}%
\pgfpathlineto{\pgfqpoint{1.087694in}{1.325883in}}%
\pgfpathlineto{\pgfqpoint{1.088798in}{1.326665in}}%
\pgfpathlineto{\pgfqpoint{1.090393in}{1.327773in}}%
\pgfpathlineto{\pgfqpoint{1.091496in}{1.328565in}}%
\pgfpathlineto{\pgfqpoint{1.093163in}{1.329673in}}%
\pgfpathlineto{\pgfqpoint{1.094272in}{1.330381in}}%
\pgfpathlineto{\pgfqpoint{1.096273in}{1.331489in}}%
\pgfpathlineto{\pgfqpoint{1.097324in}{1.332197in}}%
\pgfpathlineto{\pgfqpoint{1.097368in}{1.332197in}}%
\pgfpathlineto{\pgfqpoint{1.099106in}{1.333286in}}%
\pgfpathlineto{\pgfqpoint{1.099106in}{1.333296in}}%
\pgfpathlineto{\pgfqpoint{1.100206in}{1.334162in}}%
\pgfpathlineto{\pgfqpoint{1.102028in}{1.335270in}}%
\pgfpathlineto{\pgfqpoint{1.103114in}{1.335968in}}%
\pgfpathlineto{\pgfqpoint{1.104803in}{1.337076in}}%
\pgfpathlineto{\pgfqpoint{1.105877in}{1.337766in}}%
\pgfpathlineto{\pgfqpoint{1.107507in}{1.338864in}}%
\pgfpathlineto{\pgfqpoint{1.108616in}{1.339731in}}%
\pgfpathlineto{\pgfqpoint{1.110394in}{1.340829in}}%
\pgfpathlineto{\pgfqpoint{1.111468in}{1.341453in}}%
\pgfpathlineto{\pgfqpoint{1.113164in}{1.342562in}}%
\pgfpathlineto{\pgfqpoint{1.114259in}{1.343334in}}%
\pgfpathlineto{\pgfqpoint{1.115861in}{1.344443in}}%
\pgfpathlineto{\pgfqpoint{1.116951in}{1.345290in}}%
\pgfpathlineto{\pgfqpoint{1.116968in}{1.345290in}}%
\pgfpathlineto{\pgfqpoint{1.118474in}{1.346398in}}%
\pgfpathlineto{\pgfqpoint{1.119581in}{1.347069in}}%
\pgfpathlineto{\pgfqpoint{1.121314in}{1.348177in}}%
\pgfpathlineto{\pgfqpoint{1.122397in}{1.348969in}}%
\pgfpathlineto{\pgfqpoint{1.124271in}{1.350077in}}%
\pgfpathlineto{\pgfqpoint{1.125371in}{1.350570in}}%
\pgfpathlineto{\pgfqpoint{1.126865in}{1.351679in}}%
\pgfpathlineto{\pgfqpoint{1.127927in}{1.352489in}}%
\pgfpathlineto{\pgfqpoint{1.127967in}{1.352489in}}%
\pgfpathlineto{\pgfqpoint{1.129745in}{1.353588in}}%
\pgfpathlineto{\pgfqpoint{1.130850in}{1.354295in}}%
\pgfpathlineto{\pgfqpoint{1.132836in}{1.355404in}}%
\pgfpathlineto{\pgfqpoint{1.133943in}{1.356028in}}%
\pgfpathlineto{\pgfqpoint{1.136089in}{1.357136in}}%
\pgfpathlineto{\pgfqpoint{1.137191in}{1.357704in}}%
\pgfpathlineto{\pgfqpoint{1.137196in}{1.357704in}}%
\pgfpathlineto{\pgfqpoint{1.139098in}{1.358812in}}%
\pgfpathlineto{\pgfqpoint{1.140193in}{1.359482in}}%
\pgfpathlineto{\pgfqpoint{1.140207in}{1.359482in}}%
\pgfpathlineto{\pgfqpoint{1.141884in}{1.360591in}}%
\pgfpathlineto{\pgfqpoint{1.142989in}{1.361401in}}%
\pgfpathlineto{\pgfqpoint{1.144748in}{1.362509in}}%
\pgfpathlineto{\pgfqpoint{1.145843in}{1.363049in}}%
\pgfpathlineto{\pgfqpoint{1.147790in}{1.364157in}}%
\pgfpathlineto{\pgfqpoint{1.148897in}{1.364716in}}%
\pgfpathlineto{\pgfqpoint{1.150869in}{1.365824in}}%
\pgfpathlineto{\pgfqpoint{1.151875in}{1.366420in}}%
\pgfpathlineto{\pgfqpoint{1.154033in}{1.367519in}}%
\pgfpathlineto{\pgfqpoint{1.155121in}{1.368143in}}%
\pgfpathlineto{\pgfqpoint{1.157033in}{1.369251in}}%
\pgfpathlineto{\pgfqpoint{1.158137in}{1.369940in}}%
\pgfpathlineto{\pgfqpoint{1.159875in}{1.371049in}}%
\pgfpathlineto{\pgfqpoint{1.160984in}{1.371579in}}%
\pgfpathlineto{\pgfqpoint{1.162847in}{1.372688in}}%
\pgfpathlineto{\pgfqpoint{1.163949in}{1.373451in}}%
\pgfpathlineto{\pgfqpoint{1.165478in}{1.374559in}}%
\pgfpathlineto{\pgfqpoint{1.166576in}{1.375100in}}%
\pgfpathlineto{\pgfqpoint{1.166580in}{1.375100in}}%
\pgfpathlineto{\pgfqpoint{1.168403in}{1.376208in}}%
\pgfpathlineto{\pgfqpoint{1.169505in}{1.376822in}}%
\pgfpathlineto{\pgfqpoint{1.171222in}{1.377931in}}%
\pgfpathlineto{\pgfqpoint{1.172331in}{1.378555in}}%
\pgfpathlineto{\pgfqpoint{1.174317in}{1.379663in}}%
\pgfpathlineto{\pgfqpoint{1.175422in}{1.380296in}}%
\pgfpathlineto{\pgfqpoint{1.177554in}{1.381404in}}%
\pgfpathlineto{\pgfqpoint{1.178659in}{1.382103in}}%
\pgfpathlineto{\pgfqpoint{1.180741in}{1.383211in}}%
\pgfpathlineto{\pgfqpoint{1.181836in}{1.383825in}}%
\pgfpathlineto{\pgfqpoint{1.183846in}{1.384924in}}%
\pgfpathlineto{\pgfqpoint{1.184956in}{1.385446in}}%
\pgfpathlineto{\pgfqpoint{1.187198in}{1.386554in}}%
\pgfpathlineto{\pgfqpoint{1.188298in}{1.387355in}}%
\pgfpathlineto{\pgfqpoint{1.190491in}{1.388463in}}%
\pgfpathlineto{\pgfqpoint{1.191541in}{1.388985in}}%
\pgfpathlineto{\pgfqpoint{1.191598in}{1.388985in}}%
\pgfpathlineto{\pgfqpoint{1.193795in}{1.390084in}}%
\pgfpathlineto{\pgfqpoint{1.194902in}{1.390819in}}%
\pgfpathlineto{\pgfqpoint{1.196919in}{1.391927in}}%
\pgfpathlineto{\pgfqpoint{1.198017in}{1.392495in}}%
\pgfpathlineto{\pgfqpoint{1.200102in}{1.393604in}}%
\pgfpathlineto{\pgfqpoint{1.201190in}{1.394237in}}%
\pgfpathlineto{\pgfqpoint{1.203258in}{1.395345in}}%
\pgfpathlineto{\pgfqpoint{1.204335in}{1.395839in}}%
\pgfpathlineto{\pgfqpoint{1.206300in}{1.396947in}}%
\pgfpathlineto{\pgfqpoint{1.207386in}{1.397627in}}%
\pgfpathlineto{\pgfqpoint{1.207400in}{1.397627in}}%
\pgfpathlineto{\pgfqpoint{1.209556in}{1.398735in}}%
\pgfpathlineto{\pgfqpoint{1.210658in}{1.399228in}}%
\pgfpathlineto{\pgfqpoint{1.212440in}{1.400337in}}%
\pgfpathlineto{\pgfqpoint{1.213543in}{1.400840in}}%
\pgfpathlineto{\pgfqpoint{1.215356in}{1.401948in}}%
\pgfpathlineto{\pgfqpoint{1.216427in}{1.402451in}}%
\pgfpathlineto{\pgfqpoint{1.218876in}{1.403559in}}%
\pgfpathlineto{\pgfqpoint{1.219980in}{1.404127in}}%
\pgfpathlineto{\pgfqpoint{1.221868in}{1.405235in}}%
\pgfpathlineto{\pgfqpoint{1.222959in}{1.405710in}}%
\pgfpathlineto{\pgfqpoint{1.225166in}{1.406818in}}%
\pgfpathlineto{\pgfqpoint{1.226259in}{1.407451in}}%
\pgfpathlineto{\pgfqpoint{1.228553in}{1.408560in}}%
\pgfpathlineto{\pgfqpoint{1.229638in}{1.409230in}}%
\pgfpathlineto{\pgfqpoint{1.231566in}{1.410329in}}%
\pgfpathlineto{\pgfqpoint{1.231566in}{1.410338in}}%
\pgfpathlineto{\pgfqpoint{1.232851in}{1.410962in}}%
\pgfpathlineto{\pgfqpoint{1.234922in}{1.412070in}}%
\pgfpathlineto{\pgfqpoint{1.236001in}{1.412639in}}%
\pgfpathlineto{\pgfqpoint{1.236013in}{1.412639in}}%
\pgfpathlineto{\pgfqpoint{1.238281in}{1.413747in}}%
\pgfpathlineto{\pgfqpoint{1.239371in}{1.414296in}}%
\pgfpathlineto{\pgfqpoint{1.241689in}{1.415404in}}%
\pgfpathlineto{\pgfqpoint{1.242796in}{1.416010in}}%
\pgfpathlineto{\pgfqpoint{1.245206in}{1.417118in}}%
\pgfpathlineto{\pgfqpoint{1.246311in}{1.417649in}}%
\pgfpathlineto{\pgfqpoint{1.248413in}{1.418757in}}%
\pgfpathlineto{\pgfqpoint{1.249517in}{1.419362in}}%
\pgfpathlineto{\pgfqpoint{1.252062in}{1.420470in}}%
\pgfpathlineto{\pgfqpoint{1.253169in}{1.420973in}}%
\pgfpathlineto{\pgfqpoint{1.255329in}{1.422081in}}%
\pgfpathlineto{\pgfqpoint{1.256419in}{1.422640in}}%
\pgfpathlineto{\pgfqpoint{1.258699in}{1.423739in}}%
\pgfpathlineto{\pgfqpoint{1.258699in}{1.423748in}}%
\pgfpathlineto{\pgfqpoint{1.259991in}{1.424419in}}%
\pgfpathlineto{\pgfqpoint{1.262224in}{1.425527in}}%
\pgfpathlineto{\pgfqpoint{1.263333in}{1.426151in}}%
\pgfpathlineto{\pgfqpoint{1.265775in}{1.427259in}}%
\pgfpathlineto{\pgfqpoint{1.266851in}{1.427771in}}%
\pgfpathlineto{\pgfqpoint{1.269220in}{1.428880in}}%
\pgfpathlineto{\pgfqpoint{1.270327in}{1.429401in}}%
\pgfpathlineto{\pgfqpoint{1.273144in}{1.430509in}}%
\pgfpathlineto{\pgfqpoint{1.274244in}{1.431040in}}%
\pgfpathlineto{\pgfqpoint{1.276427in}{1.432148in}}%
\pgfpathlineto{\pgfqpoint{1.277534in}{1.432605in}}%
\pgfpathlineto{\pgfqpoint{1.279760in}{1.433713in}}%
\pgfpathlineto{\pgfqpoint{1.280843in}{1.434188in}}%
\pgfpathlineto{\pgfqpoint{1.280850in}{1.434188in}}%
\pgfpathlineto{\pgfqpoint{1.283268in}{1.435287in}}%
\pgfpathlineto{\pgfqpoint{1.284378in}{1.435817in}}%
\pgfpathlineto{\pgfqpoint{1.286798in}{1.436926in}}%
\pgfpathlineto{\pgfqpoint{1.287905in}{1.437466in}}%
\pgfpathlineto{\pgfqpoint{1.290353in}{1.438574in}}%
\pgfpathlineto{\pgfqpoint{1.291439in}{1.439123in}}%
\pgfpathlineto{\pgfqpoint{1.291453in}{1.439123in}}%
\pgfpathlineto{\pgfqpoint{1.293785in}{1.440232in}}%
\pgfpathlineto{\pgfqpoint{1.294892in}{1.440772in}}%
\pgfpathlineto{\pgfqpoint{1.297626in}{1.441871in}}%
\pgfpathlineto{\pgfqpoint{1.298714in}{1.442457in}}%
\pgfpathlineto{\pgfqpoint{1.301236in}{1.443566in}}%
\pgfpathlineto{\pgfqpoint{1.302336in}{1.444040in}}%
\pgfpathlineto{\pgfqpoint{1.302340in}{1.444040in}}%
\pgfpathlineto{\pgfqpoint{1.304693in}{1.445149in}}%
\pgfpathlineto{\pgfqpoint{1.305750in}{1.445707in}}%
\pgfpathlineto{\pgfqpoint{1.305793in}{1.445707in}}%
\pgfpathlineto{\pgfqpoint{1.308189in}{1.446816in}}%
\pgfpathlineto{\pgfqpoint{1.309285in}{1.447216in}}%
\pgfpathlineto{\pgfqpoint{1.311822in}{1.448324in}}%
\pgfpathlineto{\pgfqpoint{1.312922in}{1.448911in}}%
\pgfpathlineto{\pgfqpoint{1.315418in}{1.450019in}}%
\pgfpathlineto{\pgfqpoint{1.316511in}{1.450354in}}%
\pgfpathlineto{\pgfqpoint{1.316525in}{1.450354in}}%
\pgfpathlineto{\pgfqpoint{1.318711in}{1.451463in}}%
\pgfpathlineto{\pgfqpoint{1.319820in}{1.451938in}}%
\pgfpathlineto{\pgfqpoint{1.322409in}{1.453046in}}%
\pgfpathlineto{\pgfqpoint{1.323488in}{1.453586in}}%
\pgfpathlineto{\pgfqpoint{1.326626in}{1.454685in}}%
\pgfpathlineto{\pgfqpoint{1.327716in}{1.455234in}}%
\pgfpathlineto{\pgfqpoint{1.330953in}{1.456342in}}%
\pgfpathlineto{\pgfqpoint{1.332051in}{1.456827in}}%
\pgfpathlineto{\pgfqpoint{1.334698in}{1.457935in}}%
\pgfpathlineto{\pgfqpoint{1.335808in}{1.458428in}}%
\pgfpathlineto{\pgfqpoint{1.338193in}{1.459537in}}%
\pgfpathlineto{\pgfqpoint{1.339274in}{1.459984in}}%
\pgfpathlineto{\pgfqpoint{1.342030in}{1.461092in}}%
\pgfpathlineto{\pgfqpoint{1.343123in}{1.461557in}}%
\pgfpathlineto{\pgfqpoint{1.345806in}{1.462666in}}%
\pgfpathlineto{\pgfqpoint{1.346906in}{1.463122in}}%
\pgfpathlineto{\pgfqpoint{1.349394in}{1.464211in}}%
\pgfpathlineto{\pgfqpoint{1.350503in}{1.464696in}}%
\pgfpathlineto{\pgfqpoint{1.353224in}{1.465804in}}%
\pgfpathlineto{\pgfqpoint{1.354540in}{1.466521in}}%
\pgfpathlineto{\pgfqpoint{1.357657in}{1.467629in}}%
\pgfpathlineto{\pgfqpoint{1.358759in}{1.467974in}}%
\pgfpathlineto{\pgfqpoint{1.358766in}{1.467974in}}%
\pgfpathlineto{\pgfqpoint{1.361564in}{1.469082in}}%
\pgfpathlineto{\pgfqpoint{1.362657in}{1.469613in}}%
\pgfpathlineto{\pgfqpoint{1.365434in}{1.470721in}}%
\pgfpathlineto{\pgfqpoint{1.366519in}{1.471093in}}%
\pgfpathlineto{\pgfqpoint{1.366536in}{1.471093in}}%
\pgfpathlineto{\pgfqpoint{1.369772in}{1.472202in}}%
\pgfpathlineto{\pgfqpoint{1.370868in}{1.472723in}}%
\pgfpathlineto{\pgfqpoint{1.373736in}{1.473831in}}%
\pgfpathlineto{\pgfqpoint{1.374826in}{1.474371in}}%
\pgfpathlineto{\pgfqpoint{1.377746in}{1.475480in}}%
\pgfpathlineto{\pgfqpoint{1.378839in}{1.475917in}}%
\pgfpathlineto{\pgfqpoint{1.381898in}{1.477026in}}%
\pgfpathlineto{\pgfqpoint{1.382997in}{1.477426in}}%
\pgfpathlineto{\pgfqpoint{1.385624in}{1.478525in}}%
\pgfpathlineto{\pgfqpoint{1.386722in}{1.479028in}}%
\pgfpathlineto{\pgfqpoint{1.389651in}{1.480136in}}%
\pgfpathlineto{\pgfqpoint{1.390704in}{1.480527in}}%
\pgfpathlineto{\pgfqpoint{1.390744in}{1.480527in}}%
\pgfpathlineto{\pgfqpoint{1.394077in}{1.481635in}}%
\pgfpathlineto{\pgfqpoint{1.395186in}{1.482026in}}%
\pgfpathlineto{\pgfqpoint{1.398352in}{1.483135in}}%
\pgfpathlineto{\pgfqpoint{1.399454in}{1.483600in}}%
\pgfpathlineto{\pgfqpoint{1.402475in}{1.484708in}}%
\pgfpathlineto{\pgfqpoint{1.403542in}{1.485016in}}%
\pgfpathlineto{\pgfqpoint{1.403556in}{1.485016in}}%
\pgfpathlineto{\pgfqpoint{1.406995in}{1.486124in}}%
\pgfpathlineto{\pgfqpoint{1.408078in}{1.486543in}}%
\pgfpathlineto{\pgfqpoint{1.410820in}{1.487651in}}%
\pgfpathlineto{\pgfqpoint{1.411896in}{1.488080in}}%
\pgfpathlineto{\pgfqpoint{1.415581in}{1.489188in}}%
\pgfpathlineto{\pgfqpoint{1.416688in}{1.489644in}}%
\pgfpathlineto{\pgfqpoint{1.420170in}{1.490752in}}%
\pgfpathlineto{\pgfqpoint{1.421240in}{1.491181in}}%
\pgfpathlineto{\pgfqpoint{1.424071in}{1.492289in}}%
\pgfpathlineto{\pgfqpoint{1.425168in}{1.492671in}}%
\pgfpathlineto{\pgfqpoint{1.428637in}{1.493779in}}%
\pgfpathlineto{\pgfqpoint{1.429746in}{1.494179in}}%
\pgfpathlineto{\pgfqpoint{1.433135in}{1.495287in}}%
\pgfpathlineto{\pgfqpoint{1.434198in}{1.495707in}}%
\pgfpathlineto{\pgfqpoint{1.434238in}{1.495707in}}%
\pgfpathlineto{\pgfqpoint{1.437643in}{1.496815in}}%
\pgfpathlineto{\pgfqpoint{1.438717in}{1.497234in}}%
\pgfpathlineto{\pgfqpoint{1.441616in}{1.498333in}}%
\pgfpathlineto{\pgfqpoint{1.442723in}{1.498696in}}%
\pgfpathlineto{\pgfqpoint{1.442725in}{1.498696in}}%
\pgfpathlineto{\pgfqpoint{1.446046in}{1.499804in}}%
\pgfpathlineto{\pgfqpoint{1.447099in}{1.500177in}}%
\pgfpathlineto{\pgfqpoint{1.447144in}{1.500177in}}%
\pgfpathlineto{\pgfqpoint{1.450308in}{1.501285in}}%
\pgfpathlineto{\pgfqpoint{1.451403in}{1.501639in}}%
\pgfpathlineto{\pgfqpoint{1.454703in}{1.502747in}}%
\pgfpathlineto{\pgfqpoint{1.455807in}{1.503129in}}%
\pgfpathlineto{\pgfqpoint{1.458720in}{1.504237in}}%
\pgfpathlineto{\pgfqpoint{1.459815in}{1.504665in}}%
\pgfpathlineto{\pgfqpoint{1.462860in}{1.505773in}}%
\pgfpathlineto{\pgfqpoint{1.464089in}{1.506192in}}%
\pgfpathlineto{\pgfqpoint{1.468111in}{1.507301in}}%
\pgfpathlineto{\pgfqpoint{1.469197in}{1.507664in}}%
\pgfpathlineto{\pgfqpoint{1.472593in}{1.508772in}}%
\pgfpathlineto{\pgfqpoint{1.473671in}{1.509145in}}%
\pgfpathlineto{\pgfqpoint{1.477173in}{1.510253in}}%
\pgfpathlineto{\pgfqpoint{1.478212in}{1.510616in}}%
\pgfpathlineto{\pgfqpoint{1.478280in}{1.510616in}}%
\pgfpathlineto{\pgfqpoint{1.481624in}{1.511724in}}%
\pgfpathlineto{\pgfqpoint{1.482724in}{1.512115in}}%
\pgfpathlineto{\pgfqpoint{1.486104in}{1.513223in}}%
\pgfpathlineto{\pgfqpoint{1.487211in}{1.513549in}}%
\pgfpathlineto{\pgfqpoint{1.491062in}{1.514658in}}%
\pgfpathlineto{\pgfqpoint{1.492152in}{1.515002in}}%
\pgfpathlineto{\pgfqpoint{1.492169in}{1.515002in}}%
\pgfpathlineto{\pgfqpoint{1.496250in}{1.516110in}}%
\pgfpathlineto{\pgfqpoint{1.497359in}{1.516446in}}%
\pgfpathlineto{\pgfqpoint{1.501604in}{1.517554in}}%
\pgfpathlineto{\pgfqpoint{1.502683in}{1.518019in}}%
\pgfpathlineto{\pgfqpoint{1.502711in}{1.518019in}}%
\pgfpathlineto{\pgfqpoint{1.506918in}{1.519128in}}%
\pgfpathlineto{\pgfqpoint{1.508164in}{1.519537in}}%
\pgfpathlineto{\pgfqpoint{1.511937in}{1.520646in}}%
\pgfpathlineto{\pgfqpoint{1.512990in}{1.521074in}}%
\pgfpathlineto{\pgfqpoint{1.512997in}{1.521074in}}%
\pgfpathlineto{\pgfqpoint{1.516748in}{1.522182in}}%
\pgfpathlineto{\pgfqpoint{1.517965in}{1.522499in}}%
\pgfpathlineto{\pgfqpoint{1.522074in}{1.523607in}}%
\pgfpathlineto{\pgfqpoint{1.523162in}{1.524026in}}%
\pgfpathlineto{\pgfqpoint{1.526903in}{1.525134in}}%
\pgfpathlineto{\pgfqpoint{1.527991in}{1.525497in}}%
\pgfpathlineto{\pgfqpoint{1.532055in}{1.526606in}}%
\pgfpathlineto{\pgfqpoint{1.533003in}{1.526978in}}%
\pgfpathlineto{\pgfqpoint{1.533113in}{1.526978in}}%
\pgfpathlineto{\pgfqpoint{1.536612in}{1.528086in}}%
\pgfpathlineto{\pgfqpoint{1.537705in}{1.528375in}}%
\pgfpathlineto{\pgfqpoint{1.541420in}{1.529483in}}%
\pgfpathlineto{\pgfqpoint{1.542492in}{1.529688in}}%
\pgfpathlineto{\pgfqpoint{1.542529in}{1.529688in}}%
\pgfpathlineto{\pgfqpoint{1.546451in}{1.530796in}}%
\pgfpathlineto{\pgfqpoint{1.547389in}{1.531104in}}%
\pgfpathlineto{\pgfqpoint{1.547537in}{1.531104in}}%
\pgfpathlineto{\pgfqpoint{1.551263in}{1.532212in}}%
\pgfpathlineto{\pgfqpoint{1.552281in}{1.532519in}}%
\pgfpathlineto{\pgfqpoint{1.552370in}{1.532519in}}%
\pgfpathlineto{\pgfqpoint{1.556768in}{1.533627in}}%
\pgfpathlineto{\pgfqpoint{1.557858in}{1.533897in}}%
\pgfpathlineto{\pgfqpoint{1.561773in}{1.535006in}}%
\pgfpathlineto{\pgfqpoint{1.562776in}{1.535276in}}%
\pgfpathlineto{\pgfqpoint{1.567507in}{1.536384in}}%
\pgfpathlineto{\pgfqpoint{1.568375in}{1.536766in}}%
\pgfpathlineto{\pgfqpoint{1.568565in}{1.536766in}}%
\pgfpathlineto{\pgfqpoint{1.572573in}{1.537874in}}%
\pgfpathlineto{\pgfqpoint{1.573677in}{1.538265in}}%
\pgfpathlineto{\pgfqpoint{1.578117in}{1.539373in}}%
\pgfpathlineto{\pgfqpoint{1.579208in}{1.539746in}}%
\pgfpathlineto{\pgfqpoint{1.583542in}{1.540854in}}%
\pgfpathlineto{\pgfqpoint{1.584548in}{1.541161in}}%
\pgfpathlineto{\pgfqpoint{1.584632in}{1.541161in}}%
\pgfpathlineto{\pgfqpoint{1.588903in}{1.542269in}}%
\pgfpathlineto{\pgfqpoint{1.589902in}{1.542660in}}%
\pgfpathlineto{\pgfqpoint{1.589989in}{1.542660in}}%
\pgfpathlineto{\pgfqpoint{1.594869in}{1.543769in}}%
\pgfpathlineto{\pgfqpoint{1.595976in}{1.544085in}}%
\pgfpathlineto{\pgfqpoint{1.600144in}{1.545184in}}%
\pgfpathlineto{\pgfqpoint{1.601242in}{1.545538in}}%
\pgfpathlineto{\pgfqpoint{1.605850in}{1.546646in}}%
\pgfpathlineto{\pgfqpoint{1.606849in}{1.546954in}}%
\pgfpathlineto{\pgfqpoint{1.606955in}{1.546954in}}%
\pgfpathlineto{\pgfqpoint{1.611814in}{1.548062in}}%
\pgfpathlineto{\pgfqpoint{1.612849in}{1.548304in}}%
\pgfpathlineto{\pgfqpoint{1.612924in}{1.548304in}}%
\pgfpathlineto{\pgfqpoint{1.617994in}{1.549412in}}%
\pgfpathlineto{\pgfqpoint{1.619047in}{1.549729in}}%
\pgfpathlineto{\pgfqpoint{1.619103in}{1.549729in}}%
\pgfpathlineto{\pgfqpoint{1.624237in}{1.550837in}}%
\pgfpathlineto{\pgfqpoint{1.625344in}{1.551163in}}%
\pgfpathlineto{\pgfqpoint{1.630316in}{1.552271in}}%
\pgfpathlineto{\pgfqpoint{1.631313in}{1.552504in}}%
\pgfpathlineto{\pgfqpoint{1.631426in}{1.552504in}}%
\pgfpathlineto{\pgfqpoint{1.636979in}{1.553612in}}%
\pgfpathlineto{\pgfqpoint{1.638065in}{1.553863in}}%
\pgfpathlineto{\pgfqpoint{1.644290in}{1.554972in}}%
\pgfpathlineto{\pgfqpoint{1.645359in}{1.555251in}}%
\pgfpathlineto{\pgfqpoint{1.645371in}{1.555251in}}%
\pgfpathlineto{\pgfqpoint{1.650645in}{1.556359in}}%
\pgfpathlineto{\pgfqpoint{1.651752in}{1.556592in}}%
\pgfpathlineto{\pgfqpoint{1.656898in}{1.557700in}}%
\pgfpathlineto{\pgfqpoint{1.658000in}{1.557877in}}%
\pgfpathlineto{\pgfqpoint{1.664321in}{1.558985in}}%
\pgfpathlineto{\pgfqpoint{1.665304in}{1.559200in}}%
\pgfpathlineto{\pgfqpoint{1.665364in}{1.559200in}}%
\pgfpathlineto{\pgfqpoint{1.671704in}{1.560308in}}%
\pgfpathlineto{\pgfqpoint{1.672808in}{1.560485in}}%
\pgfpathlineto{\pgfqpoint{1.678461in}{1.561593in}}%
\pgfpathlineto{\pgfqpoint{1.679525in}{1.561891in}}%
\pgfpathlineto{\pgfqpoint{1.684669in}{1.562999in}}%
\pgfpathlineto{\pgfqpoint{1.685654in}{1.563167in}}%
\pgfpathlineto{\pgfqpoint{1.692875in}{1.564275in}}%
\pgfpathlineto{\pgfqpoint{1.693872in}{1.564508in}}%
\pgfpathlineto{\pgfqpoint{1.693937in}{1.564508in}}%
\pgfpathlineto{\pgfqpoint{1.699517in}{1.565616in}}%
\pgfpathlineto{\pgfqpoint{1.700450in}{1.565793in}}%
\pgfpathlineto{\pgfqpoint{1.700614in}{1.565793in}}%
\pgfpathlineto{\pgfqpoint{1.706942in}{1.566901in}}%
\pgfpathlineto{\pgfqpoint{1.707941in}{1.567106in}}%
\pgfpathlineto{\pgfqpoint{1.713612in}{1.568214in}}%
\pgfpathlineto{\pgfqpoint{1.714651in}{1.568428in}}%
\pgfpathlineto{\pgfqpoint{1.720932in}{1.569536in}}%
\pgfpathlineto{\pgfqpoint{1.722039in}{1.569685in}}%
\pgfpathlineto{\pgfqpoint{1.728392in}{1.570794in}}%
\pgfpathlineto{\pgfqpoint{1.729365in}{1.570943in}}%
\pgfpathlineto{\pgfqpoint{1.729459in}{1.570943in}}%
\pgfpathlineto{\pgfqpoint{1.735817in}{1.572051in}}%
\pgfpathlineto{\pgfqpoint{1.736737in}{1.572284in}}%
\pgfpathlineto{\pgfqpoint{1.736859in}{1.572284in}}%
\pgfpathlineto{\pgfqpoint{1.743683in}{1.573392in}}%
\pgfpathlineto{\pgfqpoint{1.744793in}{1.573634in}}%
\pgfpathlineto{\pgfqpoint{1.751170in}{1.574742in}}%
\pgfpathlineto{\pgfqpoint{1.752270in}{1.574966in}}%
\pgfpathlineto{\pgfqpoint{1.759662in}{1.576074in}}%
\pgfpathlineto{\pgfqpoint{1.760689in}{1.576269in}}%
\pgfpathlineto{\pgfqpoint{1.767608in}{1.577378in}}%
\pgfpathlineto{\pgfqpoint{1.768698in}{1.577583in}}%
\pgfpathlineto{\pgfqpoint{1.768717in}{1.577583in}}%
\pgfpathlineto{\pgfqpoint{1.776698in}{1.578691in}}%
\pgfpathlineto{\pgfqpoint{1.777808in}{1.578840in}}%
\pgfpathlineto{\pgfqpoint{1.786356in}{1.579948in}}%
\pgfpathlineto{\pgfqpoint{1.787433in}{1.580125in}}%
\pgfpathlineto{\pgfqpoint{1.795078in}{1.581233in}}%
\pgfpathlineto{\pgfqpoint{1.796183in}{1.581419in}}%
\pgfpathlineto{\pgfqpoint{1.804201in}{1.582527in}}%
\pgfpathlineto{\pgfqpoint{1.805297in}{1.582649in}}%
\pgfpathlineto{\pgfqpoint{1.813369in}{1.583757in}}%
\pgfpathlineto{\pgfqpoint{1.814443in}{1.583971in}}%
\pgfpathlineto{\pgfqpoint{1.822786in}{1.585079in}}%
\pgfpathlineto{\pgfqpoint{1.823895in}{1.585200in}}%
\pgfpathlineto{\pgfqpoint{1.832054in}{1.586308in}}%
\pgfpathlineto{\pgfqpoint{1.833082in}{1.586457in}}%
\pgfpathlineto{\pgfqpoint{1.842822in}{1.587566in}}%
\pgfpathlineto{\pgfqpoint{1.843882in}{1.587696in}}%
\pgfpathlineto{\pgfqpoint{1.853291in}{1.588804in}}%
\pgfpathlineto{\pgfqpoint{1.854234in}{1.588935in}}%
\pgfpathlineto{\pgfqpoint{1.854321in}{1.588935in}}%
\pgfpathlineto{\pgfqpoint{1.864724in}{1.590043in}}%
\pgfpathlineto{\pgfqpoint{1.865705in}{1.590182in}}%
\pgfpathlineto{\pgfqpoint{1.865791in}{1.590182in}}%
\pgfpathlineto{\pgfqpoint{1.875144in}{1.591291in}}%
\pgfpathlineto{\pgfqpoint{1.876247in}{1.591440in}}%
\pgfpathlineto{\pgfqpoint{1.886545in}{1.592548in}}%
\pgfpathlineto{\pgfqpoint{1.887298in}{1.592613in}}%
\pgfpathlineto{\pgfqpoint{1.887497in}{1.592613in}}%
\pgfpathlineto{\pgfqpoint{1.900919in}{1.593721in}}%
\pgfpathlineto{\pgfqpoint{1.902022in}{1.593805in}}%
\pgfpathlineto{\pgfqpoint{1.911801in}{1.594913in}}%
\pgfpathlineto{\pgfqpoint{1.912857in}{1.595034in}}%
\pgfpathlineto{\pgfqpoint{1.926952in}{1.596142in}}%
\pgfpathlineto{\pgfqpoint{1.927925in}{1.596226in}}%
\pgfpathlineto{\pgfqpoint{1.928057in}{1.596226in}}%
\pgfpathlineto{\pgfqpoint{1.941073in}{1.597334in}}%
\pgfpathlineto{\pgfqpoint{1.942143in}{1.597437in}}%
\pgfpathlineto{\pgfqpoint{1.942154in}{1.597437in}}%
\pgfpathlineto{\pgfqpoint{1.959587in}{1.598545in}}%
\pgfpathlineto{\pgfqpoint{1.960537in}{1.598601in}}%
\pgfpathlineto{\pgfqpoint{1.979175in}{1.599709in}}%
\pgfpathlineto{\pgfqpoint{1.980265in}{1.599756in}}%
\pgfpathlineto{\pgfqpoint{2.004596in}{1.600864in}}%
\pgfpathlineto{\pgfqpoint{2.005039in}{1.600910in}}%
\pgfpathlineto{\pgfqpoint{2.005574in}{1.600910in}}%
\pgfpathlineto{\pgfqpoint{2.033126in}{1.601944in}}%
\pgfpathlineto{\pgfqpoint{2.033126in}{1.601944in}}%
\pgfusepath{stroke}%
\end{pgfscope}%
\begin{pgfscope}%
\pgfsetrectcap%
\pgfsetmiterjoin%
\pgfsetlinewidth{0.803000pt}%
\definecolor{currentstroke}{rgb}{0.000000,0.000000,0.000000}%
\pgfsetstrokecolor{currentstroke}%
\pgfsetdash{}{0pt}%
\pgfpathmoveto{\pgfqpoint{0.553581in}{0.499444in}}%
\pgfpathlineto{\pgfqpoint{0.553581in}{1.654444in}}%
\pgfusepath{stroke}%
\end{pgfscope}%
\begin{pgfscope}%
\pgfsetrectcap%
\pgfsetmiterjoin%
\pgfsetlinewidth{0.803000pt}%
\definecolor{currentstroke}{rgb}{0.000000,0.000000,0.000000}%
\pgfsetstrokecolor{currentstroke}%
\pgfsetdash{}{0pt}%
\pgfpathmoveto{\pgfqpoint{2.103581in}{0.499444in}}%
\pgfpathlineto{\pgfqpoint{2.103581in}{1.654444in}}%
\pgfusepath{stroke}%
\end{pgfscope}%
\begin{pgfscope}%
\pgfsetrectcap%
\pgfsetmiterjoin%
\pgfsetlinewidth{0.803000pt}%
\definecolor{currentstroke}{rgb}{0.000000,0.000000,0.000000}%
\pgfsetstrokecolor{currentstroke}%
\pgfsetdash{}{0pt}%
\pgfpathmoveto{\pgfqpoint{0.553581in}{0.499444in}}%
\pgfpathlineto{\pgfqpoint{2.103581in}{0.499444in}}%
\pgfusepath{stroke}%
\end{pgfscope}%
\begin{pgfscope}%
\pgfsetrectcap%
\pgfsetmiterjoin%
\pgfsetlinewidth{0.803000pt}%
\definecolor{currentstroke}{rgb}{0.000000,0.000000,0.000000}%
\pgfsetstrokecolor{currentstroke}%
\pgfsetdash{}{0pt}%
\pgfpathmoveto{\pgfqpoint{0.553581in}{1.654444in}}%
\pgfpathlineto{\pgfqpoint{2.103581in}{1.654444in}}%
\pgfusepath{stroke}%
\end{pgfscope}%
\begin{pgfscope}%
\pgfsetbuttcap%
\pgfsetmiterjoin%
\definecolor{currentfill}{rgb}{1.000000,1.000000,1.000000}%
\pgfsetfillcolor{currentfill}%
\pgfsetfillopacity{0.800000}%
\pgfsetlinewidth{1.003750pt}%
\definecolor{currentstroke}{rgb}{0.800000,0.800000,0.800000}%
\pgfsetstrokecolor{currentstroke}%
\pgfsetstrokeopacity{0.800000}%
\pgfsetdash{}{0pt}%
\pgfpathmoveto{\pgfqpoint{0.832747in}{0.568889in}}%
\pgfpathlineto{\pgfqpoint{2.006358in}{0.568889in}}%
\pgfpathquadraticcurveto{\pgfqpoint{2.034136in}{0.568889in}}{\pgfqpoint{2.034136in}{0.596666in}}%
\pgfpathlineto{\pgfqpoint{2.034136in}{0.776388in}}%
\pgfpathquadraticcurveto{\pgfqpoint{2.034136in}{0.804166in}}{\pgfqpoint{2.006358in}{0.804166in}}%
\pgfpathlineto{\pgfqpoint{0.832747in}{0.804166in}}%
\pgfpathquadraticcurveto{\pgfqpoint{0.804970in}{0.804166in}}{\pgfqpoint{0.804970in}{0.776388in}}%
\pgfpathlineto{\pgfqpoint{0.804970in}{0.596666in}}%
\pgfpathquadraticcurveto{\pgfqpoint{0.804970in}{0.568889in}}{\pgfqpoint{0.832747in}{0.568889in}}%
\pgfpathlineto{\pgfqpoint{0.832747in}{0.568889in}}%
\pgfpathclose%
\pgfusepath{stroke,fill}%
\end{pgfscope}%
\begin{pgfscope}%
\pgfsetrectcap%
\pgfsetroundjoin%
\pgfsetlinewidth{1.505625pt}%
\definecolor{currentstroke}{rgb}{0.000000,0.000000,0.000000}%
\pgfsetstrokecolor{currentstroke}%
\pgfsetdash{}{0pt}%
\pgfpathmoveto{\pgfqpoint{0.860525in}{0.700000in}}%
\pgfpathlineto{\pgfqpoint{0.999414in}{0.700000in}}%
\pgfpathlineto{\pgfqpoint{1.138303in}{0.700000in}}%
\pgfusepath{stroke}%
\end{pgfscope}%
\begin{pgfscope}%
\definecolor{textcolor}{rgb}{0.000000,0.000000,0.000000}%
\pgfsetstrokecolor{textcolor}%
\pgfsetfillcolor{textcolor}%
\pgftext[x=1.249414in,y=0.651388in,left,base]{\color{textcolor}\rmfamily\fontsize{10.000000}{12.000000}\selectfont AUC=0.778}%
\end{pgfscope}%
\end{pgfpicture}%
\makeatother%
\endgroup%

\end{tabular}

\

The second method we will use to modify the model outputs' distribution is to employ class weights in the model building process.  Here we employed class weights proportional to the class imbalance.  The motivation behind class weights is to better separate the positive and negative classes, but note that the area under the ROC curve does not change.  We have not investigated whether the model using class weights does a better job at separating the classes in some intervals, but overall the effect is negligible.  One effect using class weights did have here is shifting the distribution.  


\

\verb|KBFC_5_Fold_alpha_balanced_gamma_0_0_Hard|



\noindent\begin{tabular}{@{\hspace{-6pt}}p{4.3in} @{\hspace{-6pt}}p{2.0in}}
	\vskip 0pt
	\hfil Raw Model Output
	
	%% Creator: Matplotlib, PGF backend
%%
%% To include the figure in your LaTeX document, write
%%   \input{<filename>.pgf}
%%
%% Make sure the required packages are loaded in your preamble
%%   \usepackage{pgf}
%%
%% Also ensure that all the required font packages are loaded; for instance,
%% the lmodern package is sometimes necessary when using math font.
%%   \usepackage{lmodern}
%%
%% Figures using additional raster images can only be included by \input if
%% they are in the same directory as the main LaTeX file. For loading figures
%% from other directories you can use the `import` package
%%   \usepackage{import}
%%
%% and then include the figures with
%%   \import{<path to file>}{<filename>.pgf}
%%
%% Matplotlib used the following preamble
%%   
%%   \usepackage{fontspec}
%%   \makeatletter\@ifpackageloaded{underscore}{}{\usepackage[strings]{underscore}}\makeatother
%%
\begingroup%
\makeatletter%
\begin{pgfpicture}%
\pgfpathrectangle{\pgfpointorigin}{\pgfqpoint{4.033056in}{1.754444in}}%
\pgfusepath{use as bounding box, clip}%
\begin{pgfscope}%
\pgfsetbuttcap%
\pgfsetmiterjoin%
\definecolor{currentfill}{rgb}{1.000000,1.000000,1.000000}%
\pgfsetfillcolor{currentfill}%
\pgfsetlinewidth{0.000000pt}%
\definecolor{currentstroke}{rgb}{1.000000,1.000000,1.000000}%
\pgfsetstrokecolor{currentstroke}%
\pgfsetdash{}{0pt}%
\pgfpathmoveto{\pgfqpoint{0.000000in}{0.000000in}}%
\pgfpathlineto{\pgfqpoint{4.033056in}{0.000000in}}%
\pgfpathlineto{\pgfqpoint{4.033056in}{1.754444in}}%
\pgfpathlineto{\pgfqpoint{0.000000in}{1.754444in}}%
\pgfpathlineto{\pgfqpoint{0.000000in}{0.000000in}}%
\pgfpathclose%
\pgfusepath{fill}%
\end{pgfscope}%
\begin{pgfscope}%
\pgfsetbuttcap%
\pgfsetmiterjoin%
\definecolor{currentfill}{rgb}{1.000000,1.000000,1.000000}%
\pgfsetfillcolor{currentfill}%
\pgfsetlinewidth{0.000000pt}%
\definecolor{currentstroke}{rgb}{0.000000,0.000000,0.000000}%
\pgfsetstrokecolor{currentstroke}%
\pgfsetstrokeopacity{0.000000}%
\pgfsetdash{}{0pt}%
\pgfpathmoveto{\pgfqpoint{0.445556in}{0.499444in}}%
\pgfpathlineto{\pgfqpoint{3.933056in}{0.499444in}}%
\pgfpathlineto{\pgfqpoint{3.933056in}{1.654444in}}%
\pgfpathlineto{\pgfqpoint{0.445556in}{1.654444in}}%
\pgfpathlineto{\pgfqpoint{0.445556in}{0.499444in}}%
\pgfpathclose%
\pgfusepath{fill}%
\end{pgfscope}%
\begin{pgfscope}%
\pgfpathrectangle{\pgfqpoint{0.445556in}{0.499444in}}{\pgfqpoint{3.487500in}{1.155000in}}%
\pgfusepath{clip}%
\pgfsetbuttcap%
\pgfsetmiterjoin%
\pgfsetlinewidth{1.003750pt}%
\definecolor{currentstroke}{rgb}{0.000000,0.000000,0.000000}%
\pgfsetstrokecolor{currentstroke}%
\pgfsetdash{}{0pt}%
\pgfpathmoveto{\pgfqpoint{0.540669in}{0.499444in}}%
\pgfpathlineto{\pgfqpoint{0.604078in}{0.499444in}}%
\pgfpathlineto{\pgfqpoint{0.604078in}{0.499444in}}%
\pgfpathlineto{\pgfqpoint{0.540669in}{0.499444in}}%
\pgfpathlineto{\pgfqpoint{0.540669in}{0.499444in}}%
\pgfpathclose%
\pgfusepath{stroke}%
\end{pgfscope}%
\begin{pgfscope}%
\pgfpathrectangle{\pgfqpoint{0.445556in}{0.499444in}}{\pgfqpoint{3.487500in}{1.155000in}}%
\pgfusepath{clip}%
\pgfsetbuttcap%
\pgfsetmiterjoin%
\pgfsetlinewidth{1.003750pt}%
\definecolor{currentstroke}{rgb}{0.000000,0.000000,0.000000}%
\pgfsetstrokecolor{currentstroke}%
\pgfsetdash{}{0pt}%
\pgfpathmoveto{\pgfqpoint{0.699192in}{0.499444in}}%
\pgfpathlineto{\pgfqpoint{0.762601in}{0.499444in}}%
\pgfpathlineto{\pgfqpoint{0.762601in}{1.026981in}}%
\pgfpathlineto{\pgfqpoint{0.699192in}{1.026981in}}%
\pgfpathlineto{\pgfqpoint{0.699192in}{0.499444in}}%
\pgfpathclose%
\pgfusepath{stroke}%
\end{pgfscope}%
\begin{pgfscope}%
\pgfpathrectangle{\pgfqpoint{0.445556in}{0.499444in}}{\pgfqpoint{3.487500in}{1.155000in}}%
\pgfusepath{clip}%
\pgfsetbuttcap%
\pgfsetmiterjoin%
\pgfsetlinewidth{1.003750pt}%
\definecolor{currentstroke}{rgb}{0.000000,0.000000,0.000000}%
\pgfsetstrokecolor{currentstroke}%
\pgfsetdash{}{0pt}%
\pgfpathmoveto{\pgfqpoint{0.857715in}{0.499444in}}%
\pgfpathlineto{\pgfqpoint{0.921124in}{0.499444in}}%
\pgfpathlineto{\pgfqpoint{0.921124in}{1.461428in}}%
\pgfpathlineto{\pgfqpoint{0.857715in}{1.461428in}}%
\pgfpathlineto{\pgfqpoint{0.857715in}{0.499444in}}%
\pgfpathclose%
\pgfusepath{stroke}%
\end{pgfscope}%
\begin{pgfscope}%
\pgfpathrectangle{\pgfqpoint{0.445556in}{0.499444in}}{\pgfqpoint{3.487500in}{1.155000in}}%
\pgfusepath{clip}%
\pgfsetbuttcap%
\pgfsetmiterjoin%
\pgfsetlinewidth{1.003750pt}%
\definecolor{currentstroke}{rgb}{0.000000,0.000000,0.000000}%
\pgfsetstrokecolor{currentstroke}%
\pgfsetdash{}{0pt}%
\pgfpathmoveto{\pgfqpoint{1.016238in}{0.499444in}}%
\pgfpathlineto{\pgfqpoint{1.079647in}{0.499444in}}%
\pgfpathlineto{\pgfqpoint{1.079647in}{1.570492in}}%
\pgfpathlineto{\pgfqpoint{1.016238in}{1.570492in}}%
\pgfpathlineto{\pgfqpoint{1.016238in}{0.499444in}}%
\pgfpathclose%
\pgfusepath{stroke}%
\end{pgfscope}%
\begin{pgfscope}%
\pgfpathrectangle{\pgfqpoint{0.445556in}{0.499444in}}{\pgfqpoint{3.487500in}{1.155000in}}%
\pgfusepath{clip}%
\pgfsetbuttcap%
\pgfsetmiterjoin%
\pgfsetlinewidth{1.003750pt}%
\definecolor{currentstroke}{rgb}{0.000000,0.000000,0.000000}%
\pgfsetstrokecolor{currentstroke}%
\pgfsetdash{}{0pt}%
\pgfpathmoveto{\pgfqpoint{1.174760in}{0.499444in}}%
\pgfpathlineto{\pgfqpoint{1.238169in}{0.499444in}}%
\pgfpathlineto{\pgfqpoint{1.238169in}{1.599444in}}%
\pgfpathlineto{\pgfqpoint{1.174760in}{1.599444in}}%
\pgfpathlineto{\pgfqpoint{1.174760in}{0.499444in}}%
\pgfpathclose%
\pgfusepath{stroke}%
\end{pgfscope}%
\begin{pgfscope}%
\pgfpathrectangle{\pgfqpoint{0.445556in}{0.499444in}}{\pgfqpoint{3.487500in}{1.155000in}}%
\pgfusepath{clip}%
\pgfsetbuttcap%
\pgfsetmiterjoin%
\pgfsetlinewidth{1.003750pt}%
\definecolor{currentstroke}{rgb}{0.000000,0.000000,0.000000}%
\pgfsetstrokecolor{currentstroke}%
\pgfsetdash{}{0pt}%
\pgfpathmoveto{\pgfqpoint{1.333283in}{0.499444in}}%
\pgfpathlineto{\pgfqpoint{1.396692in}{0.499444in}}%
\pgfpathlineto{\pgfqpoint{1.396692in}{1.590575in}}%
\pgfpathlineto{\pgfqpoint{1.333283in}{1.590575in}}%
\pgfpathlineto{\pgfqpoint{1.333283in}{0.499444in}}%
\pgfpathclose%
\pgfusepath{stroke}%
\end{pgfscope}%
\begin{pgfscope}%
\pgfpathrectangle{\pgfqpoint{0.445556in}{0.499444in}}{\pgfqpoint{3.487500in}{1.155000in}}%
\pgfusepath{clip}%
\pgfsetbuttcap%
\pgfsetmiterjoin%
\pgfsetlinewidth{1.003750pt}%
\definecolor{currentstroke}{rgb}{0.000000,0.000000,0.000000}%
\pgfsetstrokecolor{currentstroke}%
\pgfsetdash{}{0pt}%
\pgfpathmoveto{\pgfqpoint{1.491806in}{0.499444in}}%
\pgfpathlineto{\pgfqpoint{1.555215in}{0.499444in}}%
\pgfpathlineto{\pgfqpoint{1.555215in}{1.557043in}}%
\pgfpathlineto{\pgfqpoint{1.491806in}{1.557043in}}%
\pgfpathlineto{\pgfqpoint{1.491806in}{0.499444in}}%
\pgfpathclose%
\pgfusepath{stroke}%
\end{pgfscope}%
\begin{pgfscope}%
\pgfpathrectangle{\pgfqpoint{0.445556in}{0.499444in}}{\pgfqpoint{3.487500in}{1.155000in}}%
\pgfusepath{clip}%
\pgfsetbuttcap%
\pgfsetmiterjoin%
\pgfsetlinewidth{1.003750pt}%
\definecolor{currentstroke}{rgb}{0.000000,0.000000,0.000000}%
\pgfsetstrokecolor{currentstroke}%
\pgfsetdash{}{0pt}%
\pgfpathmoveto{\pgfqpoint{1.650328in}{0.499444in}}%
\pgfpathlineto{\pgfqpoint{1.713738in}{0.499444in}}%
\pgfpathlineto{\pgfqpoint{1.713738in}{1.518707in}}%
\pgfpathlineto{\pgfqpoint{1.650328in}{1.518707in}}%
\pgfpathlineto{\pgfqpoint{1.650328in}{0.499444in}}%
\pgfpathclose%
\pgfusepath{stroke}%
\end{pgfscope}%
\begin{pgfscope}%
\pgfpathrectangle{\pgfqpoint{0.445556in}{0.499444in}}{\pgfqpoint{3.487500in}{1.155000in}}%
\pgfusepath{clip}%
\pgfsetbuttcap%
\pgfsetmiterjoin%
\pgfsetlinewidth{1.003750pt}%
\definecolor{currentstroke}{rgb}{0.000000,0.000000,0.000000}%
\pgfsetstrokecolor{currentstroke}%
\pgfsetdash{}{0pt}%
\pgfpathmoveto{\pgfqpoint{1.808851in}{0.499444in}}%
\pgfpathlineto{\pgfqpoint{1.872260in}{0.499444in}}%
\pgfpathlineto{\pgfqpoint{1.872260in}{1.470699in}}%
\pgfpathlineto{\pgfqpoint{1.808851in}{1.470699in}}%
\pgfpathlineto{\pgfqpoint{1.808851in}{0.499444in}}%
\pgfpathclose%
\pgfusepath{stroke}%
\end{pgfscope}%
\begin{pgfscope}%
\pgfpathrectangle{\pgfqpoint{0.445556in}{0.499444in}}{\pgfqpoint{3.487500in}{1.155000in}}%
\pgfusepath{clip}%
\pgfsetbuttcap%
\pgfsetmiterjoin%
\pgfsetlinewidth{1.003750pt}%
\definecolor{currentstroke}{rgb}{0.000000,0.000000,0.000000}%
\pgfsetstrokecolor{currentstroke}%
\pgfsetdash{}{0pt}%
\pgfpathmoveto{\pgfqpoint{1.967374in}{0.499444in}}%
\pgfpathlineto{\pgfqpoint{2.030783in}{0.499444in}}%
\pgfpathlineto{\pgfqpoint{2.030783in}{1.406650in}}%
\pgfpathlineto{\pgfqpoint{1.967374in}{1.406650in}}%
\pgfpathlineto{\pgfqpoint{1.967374in}{0.499444in}}%
\pgfpathclose%
\pgfusepath{stroke}%
\end{pgfscope}%
\begin{pgfscope}%
\pgfpathrectangle{\pgfqpoint{0.445556in}{0.499444in}}{\pgfqpoint{3.487500in}{1.155000in}}%
\pgfusepath{clip}%
\pgfsetbuttcap%
\pgfsetmiterjoin%
\pgfsetlinewidth{1.003750pt}%
\definecolor{currentstroke}{rgb}{0.000000,0.000000,0.000000}%
\pgfsetstrokecolor{currentstroke}%
\pgfsetdash{}{0pt}%
\pgfpathmoveto{\pgfqpoint{2.125897in}{0.499444in}}%
\pgfpathlineto{\pgfqpoint{2.189306in}{0.499444in}}%
\pgfpathlineto{\pgfqpoint{2.189306in}{1.335676in}}%
\pgfpathlineto{\pgfqpoint{2.125897in}{1.335676in}}%
\pgfpathlineto{\pgfqpoint{2.125897in}{0.499444in}}%
\pgfpathclose%
\pgfusepath{stroke}%
\end{pgfscope}%
\begin{pgfscope}%
\pgfpathrectangle{\pgfqpoint{0.445556in}{0.499444in}}{\pgfqpoint{3.487500in}{1.155000in}}%
\pgfusepath{clip}%
\pgfsetbuttcap%
\pgfsetmiterjoin%
\pgfsetlinewidth{1.003750pt}%
\definecolor{currentstroke}{rgb}{0.000000,0.000000,0.000000}%
\pgfsetstrokecolor{currentstroke}%
\pgfsetdash{}{0pt}%
\pgfpathmoveto{\pgfqpoint{2.284419in}{0.499444in}}%
\pgfpathlineto{\pgfqpoint{2.347828in}{0.499444in}}%
\pgfpathlineto{\pgfqpoint{2.347828in}{1.267315in}}%
\pgfpathlineto{\pgfqpoint{2.284419in}{1.267315in}}%
\pgfpathlineto{\pgfqpoint{2.284419in}{0.499444in}}%
\pgfpathclose%
\pgfusepath{stroke}%
\end{pgfscope}%
\begin{pgfscope}%
\pgfpathrectangle{\pgfqpoint{0.445556in}{0.499444in}}{\pgfqpoint{3.487500in}{1.155000in}}%
\pgfusepath{clip}%
\pgfsetbuttcap%
\pgfsetmiterjoin%
\pgfsetlinewidth{1.003750pt}%
\definecolor{currentstroke}{rgb}{0.000000,0.000000,0.000000}%
\pgfsetstrokecolor{currentstroke}%
\pgfsetdash{}{0pt}%
\pgfpathmoveto{\pgfqpoint{2.442942in}{0.499444in}}%
\pgfpathlineto{\pgfqpoint{2.506351in}{0.499444in}}%
\pgfpathlineto{\pgfqpoint{2.506351in}{1.195202in}}%
\pgfpathlineto{\pgfqpoint{2.442942in}{1.195202in}}%
\pgfpathlineto{\pgfqpoint{2.442942in}{0.499444in}}%
\pgfpathclose%
\pgfusepath{stroke}%
\end{pgfscope}%
\begin{pgfscope}%
\pgfpathrectangle{\pgfqpoint{0.445556in}{0.499444in}}{\pgfqpoint{3.487500in}{1.155000in}}%
\pgfusepath{clip}%
\pgfsetbuttcap%
\pgfsetmiterjoin%
\pgfsetlinewidth{1.003750pt}%
\definecolor{currentstroke}{rgb}{0.000000,0.000000,0.000000}%
\pgfsetstrokecolor{currentstroke}%
\pgfsetdash{}{0pt}%
\pgfpathmoveto{\pgfqpoint{2.601465in}{0.499444in}}%
\pgfpathlineto{\pgfqpoint{2.664874in}{0.499444in}}%
\pgfpathlineto{\pgfqpoint{2.664874in}{1.113080in}}%
\pgfpathlineto{\pgfqpoint{2.601465in}{1.113080in}}%
\pgfpathlineto{\pgfqpoint{2.601465in}{0.499444in}}%
\pgfpathclose%
\pgfusepath{stroke}%
\end{pgfscope}%
\begin{pgfscope}%
\pgfpathrectangle{\pgfqpoint{0.445556in}{0.499444in}}{\pgfqpoint{3.487500in}{1.155000in}}%
\pgfusepath{clip}%
\pgfsetbuttcap%
\pgfsetmiterjoin%
\pgfsetlinewidth{1.003750pt}%
\definecolor{currentstroke}{rgb}{0.000000,0.000000,0.000000}%
\pgfsetstrokecolor{currentstroke}%
\pgfsetdash{}{0pt}%
\pgfpathmoveto{\pgfqpoint{2.759988in}{0.499444in}}%
\pgfpathlineto{\pgfqpoint{2.823397in}{0.499444in}}%
\pgfpathlineto{\pgfqpoint{2.823397in}{1.023318in}}%
\pgfpathlineto{\pgfqpoint{2.759988in}{1.023318in}}%
\pgfpathlineto{\pgfqpoint{2.759988in}{0.499444in}}%
\pgfpathclose%
\pgfusepath{stroke}%
\end{pgfscope}%
\begin{pgfscope}%
\pgfpathrectangle{\pgfqpoint{0.445556in}{0.499444in}}{\pgfqpoint{3.487500in}{1.155000in}}%
\pgfusepath{clip}%
\pgfsetbuttcap%
\pgfsetmiterjoin%
\pgfsetlinewidth{1.003750pt}%
\definecolor{currentstroke}{rgb}{0.000000,0.000000,0.000000}%
\pgfsetstrokecolor{currentstroke}%
\pgfsetdash{}{0pt}%
\pgfpathmoveto{\pgfqpoint{2.918510in}{0.499444in}}%
\pgfpathlineto{\pgfqpoint{2.981919in}{0.499444in}}%
\pgfpathlineto{\pgfqpoint{2.981919in}{0.926205in}}%
\pgfpathlineto{\pgfqpoint{2.918510in}{0.926205in}}%
\pgfpathlineto{\pgfqpoint{2.918510in}{0.499444in}}%
\pgfpathclose%
\pgfusepath{stroke}%
\end{pgfscope}%
\begin{pgfscope}%
\pgfpathrectangle{\pgfqpoint{0.445556in}{0.499444in}}{\pgfqpoint{3.487500in}{1.155000in}}%
\pgfusepath{clip}%
\pgfsetbuttcap%
\pgfsetmiterjoin%
\pgfsetlinewidth{1.003750pt}%
\definecolor{currentstroke}{rgb}{0.000000,0.000000,0.000000}%
\pgfsetstrokecolor{currentstroke}%
\pgfsetdash{}{0pt}%
\pgfpathmoveto{\pgfqpoint{3.077033in}{0.499444in}}%
\pgfpathlineto{\pgfqpoint{3.140442in}{0.499444in}}%
\pgfpathlineto{\pgfqpoint{3.140442in}{0.826859in}}%
\pgfpathlineto{\pgfqpoint{3.077033in}{0.826859in}}%
\pgfpathlineto{\pgfqpoint{3.077033in}{0.499444in}}%
\pgfpathclose%
\pgfusepath{stroke}%
\end{pgfscope}%
\begin{pgfscope}%
\pgfpathrectangle{\pgfqpoint{0.445556in}{0.499444in}}{\pgfqpoint{3.487500in}{1.155000in}}%
\pgfusepath{clip}%
\pgfsetbuttcap%
\pgfsetmiterjoin%
\pgfsetlinewidth{1.003750pt}%
\definecolor{currentstroke}{rgb}{0.000000,0.000000,0.000000}%
\pgfsetstrokecolor{currentstroke}%
\pgfsetdash{}{0pt}%
\pgfpathmoveto{\pgfqpoint{3.235556in}{0.499444in}}%
\pgfpathlineto{\pgfqpoint{3.298965in}{0.499444in}}%
\pgfpathlineto{\pgfqpoint{3.298965in}{0.743464in}}%
\pgfpathlineto{\pgfqpoint{3.235556in}{0.743464in}}%
\pgfpathlineto{\pgfqpoint{3.235556in}{0.499444in}}%
\pgfpathclose%
\pgfusepath{stroke}%
\end{pgfscope}%
\begin{pgfscope}%
\pgfpathrectangle{\pgfqpoint{0.445556in}{0.499444in}}{\pgfqpoint{3.487500in}{1.155000in}}%
\pgfusepath{clip}%
\pgfsetbuttcap%
\pgfsetmiterjoin%
\pgfsetlinewidth{1.003750pt}%
\definecolor{currentstroke}{rgb}{0.000000,0.000000,0.000000}%
\pgfsetstrokecolor{currentstroke}%
\pgfsetdash{}{0pt}%
\pgfpathmoveto{\pgfqpoint{3.394078in}{0.499444in}}%
\pgfpathlineto{\pgfqpoint{3.457488in}{0.499444in}}%
\pgfpathlineto{\pgfqpoint{3.457488in}{0.662973in}}%
\pgfpathlineto{\pgfqpoint{3.394078in}{0.662973in}}%
\pgfpathlineto{\pgfqpoint{3.394078in}{0.499444in}}%
\pgfpathclose%
\pgfusepath{stroke}%
\end{pgfscope}%
\begin{pgfscope}%
\pgfpathrectangle{\pgfqpoint{0.445556in}{0.499444in}}{\pgfqpoint{3.487500in}{1.155000in}}%
\pgfusepath{clip}%
\pgfsetbuttcap%
\pgfsetmiterjoin%
\pgfsetlinewidth{1.003750pt}%
\definecolor{currentstroke}{rgb}{0.000000,0.000000,0.000000}%
\pgfsetstrokecolor{currentstroke}%
\pgfsetdash{}{0pt}%
\pgfpathmoveto{\pgfqpoint{3.552601in}{0.499444in}}%
\pgfpathlineto{\pgfqpoint{3.616010in}{0.499444in}}%
\pgfpathlineto{\pgfqpoint{3.616010in}{0.591284in}}%
\pgfpathlineto{\pgfqpoint{3.552601in}{0.591284in}}%
\pgfpathlineto{\pgfqpoint{3.552601in}{0.499444in}}%
\pgfpathclose%
\pgfusepath{stroke}%
\end{pgfscope}%
\begin{pgfscope}%
\pgfpathrectangle{\pgfqpoint{0.445556in}{0.499444in}}{\pgfqpoint{3.487500in}{1.155000in}}%
\pgfusepath{clip}%
\pgfsetbuttcap%
\pgfsetmiterjoin%
\pgfsetlinewidth{1.003750pt}%
\definecolor{currentstroke}{rgb}{0.000000,0.000000,0.000000}%
\pgfsetstrokecolor{currentstroke}%
\pgfsetdash{}{0pt}%
\pgfpathmoveto{\pgfqpoint{3.711124in}{0.499444in}}%
\pgfpathlineto{\pgfqpoint{3.774533in}{0.499444in}}%
\pgfpathlineto{\pgfqpoint{3.774533in}{0.523705in}}%
\pgfpathlineto{\pgfqpoint{3.711124in}{0.523705in}}%
\pgfpathlineto{\pgfqpoint{3.711124in}{0.499444in}}%
\pgfpathclose%
\pgfusepath{stroke}%
\end{pgfscope}%
\begin{pgfscope}%
\pgfpathrectangle{\pgfqpoint{0.445556in}{0.499444in}}{\pgfqpoint{3.487500in}{1.155000in}}%
\pgfusepath{clip}%
\pgfsetbuttcap%
\pgfsetmiterjoin%
\definecolor{currentfill}{rgb}{0.000000,0.000000,0.000000}%
\pgfsetfillcolor{currentfill}%
\pgfsetlinewidth{0.000000pt}%
\definecolor{currentstroke}{rgb}{0.000000,0.000000,0.000000}%
\pgfsetstrokecolor{currentstroke}%
\pgfsetstrokeopacity{0.000000}%
\pgfsetdash{}{0pt}%
\pgfpathmoveto{\pgfqpoint{0.604078in}{0.499444in}}%
\pgfpathlineto{\pgfqpoint{0.667488in}{0.499444in}}%
\pgfpathlineto{\pgfqpoint{0.667488in}{0.499444in}}%
\pgfpathlineto{\pgfqpoint{0.604078in}{0.499444in}}%
\pgfpathlineto{\pgfqpoint{0.604078in}{0.499444in}}%
\pgfpathclose%
\pgfusepath{fill}%
\end{pgfscope}%
\begin{pgfscope}%
\pgfpathrectangle{\pgfqpoint{0.445556in}{0.499444in}}{\pgfqpoint{3.487500in}{1.155000in}}%
\pgfusepath{clip}%
\pgfsetbuttcap%
\pgfsetmiterjoin%
\definecolor{currentfill}{rgb}{0.000000,0.000000,0.000000}%
\pgfsetfillcolor{currentfill}%
\pgfsetlinewidth{0.000000pt}%
\definecolor{currentstroke}{rgb}{0.000000,0.000000,0.000000}%
\pgfsetstrokecolor{currentstroke}%
\pgfsetstrokeopacity{0.000000}%
\pgfsetdash{}{0pt}%
\pgfpathmoveto{\pgfqpoint{0.762601in}{0.499444in}}%
\pgfpathlineto{\pgfqpoint{0.826010in}{0.499444in}}%
\pgfpathlineto{\pgfqpoint{0.826010in}{0.504649in}}%
\pgfpathlineto{\pgfqpoint{0.762601in}{0.504649in}}%
\pgfpathlineto{\pgfqpoint{0.762601in}{0.499444in}}%
\pgfpathclose%
\pgfusepath{fill}%
\end{pgfscope}%
\begin{pgfscope}%
\pgfpathrectangle{\pgfqpoint{0.445556in}{0.499444in}}{\pgfqpoint{3.487500in}{1.155000in}}%
\pgfusepath{clip}%
\pgfsetbuttcap%
\pgfsetmiterjoin%
\definecolor{currentfill}{rgb}{0.000000,0.000000,0.000000}%
\pgfsetfillcolor{currentfill}%
\pgfsetlinewidth{0.000000pt}%
\definecolor{currentstroke}{rgb}{0.000000,0.000000,0.000000}%
\pgfsetstrokecolor{currentstroke}%
\pgfsetstrokeopacity{0.000000}%
\pgfsetdash{}{0pt}%
\pgfpathmoveto{\pgfqpoint{0.921124in}{0.499444in}}%
\pgfpathlineto{\pgfqpoint{0.984533in}{0.499444in}}%
\pgfpathlineto{\pgfqpoint{0.984533in}{0.517919in}}%
\pgfpathlineto{\pgfqpoint{0.921124in}{0.517919in}}%
\pgfpathlineto{\pgfqpoint{0.921124in}{0.499444in}}%
\pgfpathclose%
\pgfusepath{fill}%
\end{pgfscope}%
\begin{pgfscope}%
\pgfpathrectangle{\pgfqpoint{0.445556in}{0.499444in}}{\pgfqpoint{3.487500in}{1.155000in}}%
\pgfusepath{clip}%
\pgfsetbuttcap%
\pgfsetmiterjoin%
\definecolor{currentfill}{rgb}{0.000000,0.000000,0.000000}%
\pgfsetfillcolor{currentfill}%
\pgfsetlinewidth{0.000000pt}%
\definecolor{currentstroke}{rgb}{0.000000,0.000000,0.000000}%
\pgfsetstrokecolor{currentstroke}%
\pgfsetstrokeopacity{0.000000}%
\pgfsetdash{}{0pt}%
\pgfpathmoveto{\pgfqpoint{1.079647in}{0.499444in}}%
\pgfpathlineto{\pgfqpoint{1.143056in}{0.499444in}}%
\pgfpathlineto{\pgfqpoint{1.143056in}{0.533490in}}%
\pgfpathlineto{\pgfqpoint{1.079647in}{0.533490in}}%
\pgfpathlineto{\pgfqpoint{1.079647in}{0.499444in}}%
\pgfpathclose%
\pgfusepath{fill}%
\end{pgfscope}%
\begin{pgfscope}%
\pgfpathrectangle{\pgfqpoint{0.445556in}{0.499444in}}{\pgfqpoint{3.487500in}{1.155000in}}%
\pgfusepath{clip}%
\pgfsetbuttcap%
\pgfsetmiterjoin%
\definecolor{currentfill}{rgb}{0.000000,0.000000,0.000000}%
\pgfsetfillcolor{currentfill}%
\pgfsetlinewidth{0.000000pt}%
\definecolor{currentstroke}{rgb}{0.000000,0.000000,0.000000}%
\pgfsetstrokecolor{currentstroke}%
\pgfsetstrokeopacity{0.000000}%
\pgfsetdash{}{0pt}%
\pgfpathmoveto{\pgfqpoint{1.238169in}{0.499444in}}%
\pgfpathlineto{\pgfqpoint{1.301578in}{0.499444in}}%
\pgfpathlineto{\pgfqpoint{1.301578in}{0.550111in}}%
\pgfpathlineto{\pgfqpoint{1.238169in}{0.550111in}}%
\pgfpathlineto{\pgfqpoint{1.238169in}{0.499444in}}%
\pgfpathclose%
\pgfusepath{fill}%
\end{pgfscope}%
\begin{pgfscope}%
\pgfpathrectangle{\pgfqpoint{0.445556in}{0.499444in}}{\pgfqpoint{3.487500in}{1.155000in}}%
\pgfusepath{clip}%
\pgfsetbuttcap%
\pgfsetmiterjoin%
\definecolor{currentfill}{rgb}{0.000000,0.000000,0.000000}%
\pgfsetfillcolor{currentfill}%
\pgfsetlinewidth{0.000000pt}%
\definecolor{currentstroke}{rgb}{0.000000,0.000000,0.000000}%
\pgfsetstrokecolor{currentstroke}%
\pgfsetstrokeopacity{0.000000}%
\pgfsetdash{}{0pt}%
\pgfpathmoveto{\pgfqpoint{1.396692in}{0.499444in}}%
\pgfpathlineto{\pgfqpoint{1.460101in}{0.499444in}}%
\pgfpathlineto{\pgfqpoint{1.460101in}{0.567157in}}%
\pgfpathlineto{\pgfqpoint{1.396692in}{0.567157in}}%
\pgfpathlineto{\pgfqpoint{1.396692in}{0.499444in}}%
\pgfpathclose%
\pgfusepath{fill}%
\end{pgfscope}%
\begin{pgfscope}%
\pgfpathrectangle{\pgfqpoint{0.445556in}{0.499444in}}{\pgfqpoint{3.487500in}{1.155000in}}%
\pgfusepath{clip}%
\pgfsetbuttcap%
\pgfsetmiterjoin%
\definecolor{currentfill}{rgb}{0.000000,0.000000,0.000000}%
\pgfsetfillcolor{currentfill}%
\pgfsetlinewidth{0.000000pt}%
\definecolor{currentstroke}{rgb}{0.000000,0.000000,0.000000}%
\pgfsetstrokecolor{currentstroke}%
\pgfsetstrokeopacity{0.000000}%
\pgfsetdash{}{0pt}%
\pgfpathmoveto{\pgfqpoint{1.555215in}{0.499444in}}%
\pgfpathlineto{\pgfqpoint{1.618624in}{0.499444in}}%
\pgfpathlineto{\pgfqpoint{1.618624in}{0.582795in}}%
\pgfpathlineto{\pgfqpoint{1.555215in}{0.582795in}}%
\pgfpathlineto{\pgfqpoint{1.555215in}{0.499444in}}%
\pgfpathclose%
\pgfusepath{fill}%
\end{pgfscope}%
\begin{pgfscope}%
\pgfpathrectangle{\pgfqpoint{0.445556in}{0.499444in}}{\pgfqpoint{3.487500in}{1.155000in}}%
\pgfusepath{clip}%
\pgfsetbuttcap%
\pgfsetmiterjoin%
\definecolor{currentfill}{rgb}{0.000000,0.000000,0.000000}%
\pgfsetfillcolor{currentfill}%
\pgfsetlinewidth{0.000000pt}%
\definecolor{currentstroke}{rgb}{0.000000,0.000000,0.000000}%
\pgfsetstrokecolor{currentstroke}%
\pgfsetstrokeopacity{0.000000}%
\pgfsetdash{}{0pt}%
\pgfpathmoveto{\pgfqpoint{1.713738in}{0.499444in}}%
\pgfpathlineto{\pgfqpoint{1.777147in}{0.499444in}}%
\pgfpathlineto{\pgfqpoint{1.777147in}{0.598924in}}%
\pgfpathlineto{\pgfqpoint{1.713738in}{0.598924in}}%
\pgfpathlineto{\pgfqpoint{1.713738in}{0.499444in}}%
\pgfpathclose%
\pgfusepath{fill}%
\end{pgfscope}%
\begin{pgfscope}%
\pgfpathrectangle{\pgfqpoint{0.445556in}{0.499444in}}{\pgfqpoint{3.487500in}{1.155000in}}%
\pgfusepath{clip}%
\pgfsetbuttcap%
\pgfsetmiterjoin%
\definecolor{currentfill}{rgb}{0.000000,0.000000,0.000000}%
\pgfsetfillcolor{currentfill}%
\pgfsetlinewidth{0.000000pt}%
\definecolor{currentstroke}{rgb}{0.000000,0.000000,0.000000}%
\pgfsetstrokecolor{currentstroke}%
\pgfsetstrokeopacity{0.000000}%
\pgfsetdash{}{0pt}%
\pgfpathmoveto{\pgfqpoint{1.872260in}{0.499444in}}%
\pgfpathlineto{\pgfqpoint{1.935669in}{0.499444in}}%
\pgfpathlineto{\pgfqpoint{1.935669in}{0.618405in}}%
\pgfpathlineto{\pgfqpoint{1.872260in}{0.618405in}}%
\pgfpathlineto{\pgfqpoint{1.872260in}{0.499444in}}%
\pgfpathclose%
\pgfusepath{fill}%
\end{pgfscope}%
\begin{pgfscope}%
\pgfpathrectangle{\pgfqpoint{0.445556in}{0.499444in}}{\pgfqpoint{3.487500in}{1.155000in}}%
\pgfusepath{clip}%
\pgfsetbuttcap%
\pgfsetmiterjoin%
\definecolor{currentfill}{rgb}{0.000000,0.000000,0.000000}%
\pgfsetfillcolor{currentfill}%
\pgfsetlinewidth{0.000000pt}%
\definecolor{currentstroke}{rgb}{0.000000,0.000000,0.000000}%
\pgfsetstrokecolor{currentstroke}%
\pgfsetstrokeopacity{0.000000}%
\pgfsetdash{}{0pt}%
\pgfpathmoveto{\pgfqpoint{2.030783in}{0.499444in}}%
\pgfpathlineto{\pgfqpoint{2.094192in}{0.499444in}}%
\pgfpathlineto{\pgfqpoint{2.094192in}{0.634691in}}%
\pgfpathlineto{\pgfqpoint{2.030783in}{0.634691in}}%
\pgfpathlineto{\pgfqpoint{2.030783in}{0.499444in}}%
\pgfpathclose%
\pgfusepath{fill}%
\end{pgfscope}%
\begin{pgfscope}%
\pgfpathrectangle{\pgfqpoint{0.445556in}{0.499444in}}{\pgfqpoint{3.487500in}{1.155000in}}%
\pgfusepath{clip}%
\pgfsetbuttcap%
\pgfsetmiterjoin%
\definecolor{currentfill}{rgb}{0.000000,0.000000,0.000000}%
\pgfsetfillcolor{currentfill}%
\pgfsetlinewidth{0.000000pt}%
\definecolor{currentstroke}{rgb}{0.000000,0.000000,0.000000}%
\pgfsetstrokecolor{currentstroke}%
\pgfsetstrokeopacity{0.000000}%
\pgfsetdash{}{0pt}%
\pgfpathmoveto{\pgfqpoint{2.189306in}{0.499444in}}%
\pgfpathlineto{\pgfqpoint{2.252715in}{0.499444in}}%
\pgfpathlineto{\pgfqpoint{2.252715in}{0.649636in}}%
\pgfpathlineto{\pgfqpoint{2.189306in}{0.649636in}}%
\pgfpathlineto{\pgfqpoint{2.189306in}{0.499444in}}%
\pgfpathclose%
\pgfusepath{fill}%
\end{pgfscope}%
\begin{pgfscope}%
\pgfpathrectangle{\pgfqpoint{0.445556in}{0.499444in}}{\pgfqpoint{3.487500in}{1.155000in}}%
\pgfusepath{clip}%
\pgfsetbuttcap%
\pgfsetmiterjoin%
\definecolor{currentfill}{rgb}{0.000000,0.000000,0.000000}%
\pgfsetfillcolor{currentfill}%
\pgfsetlinewidth{0.000000pt}%
\definecolor{currentstroke}{rgb}{0.000000,0.000000,0.000000}%
\pgfsetstrokecolor{currentstroke}%
\pgfsetstrokeopacity{0.000000}%
\pgfsetdash{}{0pt}%
\pgfpathmoveto{\pgfqpoint{2.347828in}{0.499444in}}%
\pgfpathlineto{\pgfqpoint{2.411238in}{0.499444in}}%
\pgfpathlineto{\pgfqpoint{2.411238in}{0.664671in}}%
\pgfpathlineto{\pgfqpoint{2.347828in}{0.664671in}}%
\pgfpathlineto{\pgfqpoint{2.347828in}{0.499444in}}%
\pgfpathclose%
\pgfusepath{fill}%
\end{pgfscope}%
\begin{pgfscope}%
\pgfpathrectangle{\pgfqpoint{0.445556in}{0.499444in}}{\pgfqpoint{3.487500in}{1.155000in}}%
\pgfusepath{clip}%
\pgfsetbuttcap%
\pgfsetmiterjoin%
\definecolor{currentfill}{rgb}{0.000000,0.000000,0.000000}%
\pgfsetfillcolor{currentfill}%
\pgfsetlinewidth{0.000000pt}%
\definecolor{currentstroke}{rgb}{0.000000,0.000000,0.000000}%
\pgfsetstrokecolor{currentstroke}%
\pgfsetstrokeopacity{0.000000}%
\pgfsetdash{}{0pt}%
\pgfpathmoveto{\pgfqpoint{2.506351in}{0.499444in}}%
\pgfpathlineto{\pgfqpoint{2.569760in}{0.499444in}}%
\pgfpathlineto{\pgfqpoint{2.569760in}{0.679371in}}%
\pgfpathlineto{\pgfqpoint{2.506351in}{0.679371in}}%
\pgfpathlineto{\pgfqpoint{2.506351in}{0.499444in}}%
\pgfpathclose%
\pgfusepath{fill}%
\end{pgfscope}%
\begin{pgfscope}%
\pgfpathrectangle{\pgfqpoint{0.445556in}{0.499444in}}{\pgfqpoint{3.487500in}{1.155000in}}%
\pgfusepath{clip}%
\pgfsetbuttcap%
\pgfsetmiterjoin%
\definecolor{currentfill}{rgb}{0.000000,0.000000,0.000000}%
\pgfsetfillcolor{currentfill}%
\pgfsetlinewidth{0.000000pt}%
\definecolor{currentstroke}{rgb}{0.000000,0.000000,0.000000}%
\pgfsetstrokecolor{currentstroke}%
\pgfsetstrokeopacity{0.000000}%
\pgfsetdash{}{0pt}%
\pgfpathmoveto{\pgfqpoint{2.664874in}{0.499444in}}%
\pgfpathlineto{\pgfqpoint{2.728283in}{0.499444in}}%
\pgfpathlineto{\pgfqpoint{2.728283in}{0.693087in}}%
\pgfpathlineto{\pgfqpoint{2.664874in}{0.693087in}}%
\pgfpathlineto{\pgfqpoint{2.664874in}{0.499444in}}%
\pgfpathclose%
\pgfusepath{fill}%
\end{pgfscope}%
\begin{pgfscope}%
\pgfpathrectangle{\pgfqpoint{0.445556in}{0.499444in}}{\pgfqpoint{3.487500in}{1.155000in}}%
\pgfusepath{clip}%
\pgfsetbuttcap%
\pgfsetmiterjoin%
\definecolor{currentfill}{rgb}{0.000000,0.000000,0.000000}%
\pgfsetfillcolor{currentfill}%
\pgfsetlinewidth{0.000000pt}%
\definecolor{currentstroke}{rgb}{0.000000,0.000000,0.000000}%
\pgfsetstrokecolor{currentstroke}%
\pgfsetstrokeopacity{0.000000}%
\pgfsetdash{}{0pt}%
\pgfpathmoveto{\pgfqpoint{2.823397in}{0.499444in}}%
\pgfpathlineto{\pgfqpoint{2.886806in}{0.499444in}}%
\pgfpathlineto{\pgfqpoint{2.886806in}{0.699075in}}%
\pgfpathlineto{\pgfqpoint{2.823397in}{0.699075in}}%
\pgfpathlineto{\pgfqpoint{2.823397in}{0.499444in}}%
\pgfpathclose%
\pgfusepath{fill}%
\end{pgfscope}%
\begin{pgfscope}%
\pgfpathrectangle{\pgfqpoint{0.445556in}{0.499444in}}{\pgfqpoint{3.487500in}{1.155000in}}%
\pgfusepath{clip}%
\pgfsetbuttcap%
\pgfsetmiterjoin%
\definecolor{currentfill}{rgb}{0.000000,0.000000,0.000000}%
\pgfsetfillcolor{currentfill}%
\pgfsetlinewidth{0.000000pt}%
\definecolor{currentstroke}{rgb}{0.000000,0.000000,0.000000}%
\pgfsetstrokecolor{currentstroke}%
\pgfsetstrokeopacity{0.000000}%
\pgfsetdash{}{0pt}%
\pgfpathmoveto{\pgfqpoint{2.981919in}{0.499444in}}%
\pgfpathlineto{\pgfqpoint{3.045328in}{0.499444in}}%
\pgfpathlineto{\pgfqpoint{3.045328in}{0.698226in}}%
\pgfpathlineto{\pgfqpoint{2.981919in}{0.698226in}}%
\pgfpathlineto{\pgfqpoint{2.981919in}{0.499444in}}%
\pgfpathclose%
\pgfusepath{fill}%
\end{pgfscope}%
\begin{pgfscope}%
\pgfpathrectangle{\pgfqpoint{0.445556in}{0.499444in}}{\pgfqpoint{3.487500in}{1.155000in}}%
\pgfusepath{clip}%
\pgfsetbuttcap%
\pgfsetmiterjoin%
\definecolor{currentfill}{rgb}{0.000000,0.000000,0.000000}%
\pgfsetfillcolor{currentfill}%
\pgfsetlinewidth{0.000000pt}%
\definecolor{currentstroke}{rgb}{0.000000,0.000000,0.000000}%
\pgfsetstrokecolor{currentstroke}%
\pgfsetstrokeopacity{0.000000}%
\pgfsetdash{}{0pt}%
\pgfpathmoveto{\pgfqpoint{3.140442in}{0.499444in}}%
\pgfpathlineto{\pgfqpoint{3.203851in}{0.499444in}}%
\pgfpathlineto{\pgfqpoint{3.203851in}{0.695053in}}%
\pgfpathlineto{\pgfqpoint{3.140442in}{0.695053in}}%
\pgfpathlineto{\pgfqpoint{3.140442in}{0.499444in}}%
\pgfpathclose%
\pgfusepath{fill}%
\end{pgfscope}%
\begin{pgfscope}%
\pgfpathrectangle{\pgfqpoint{0.445556in}{0.499444in}}{\pgfqpoint{3.487500in}{1.155000in}}%
\pgfusepath{clip}%
\pgfsetbuttcap%
\pgfsetmiterjoin%
\definecolor{currentfill}{rgb}{0.000000,0.000000,0.000000}%
\pgfsetfillcolor{currentfill}%
\pgfsetlinewidth{0.000000pt}%
\definecolor{currentstroke}{rgb}{0.000000,0.000000,0.000000}%
\pgfsetstrokecolor{currentstroke}%
\pgfsetstrokeopacity{0.000000}%
\pgfsetdash{}{0pt}%
\pgfpathmoveto{\pgfqpoint{3.298965in}{0.499444in}}%
\pgfpathlineto{\pgfqpoint{3.362374in}{0.499444in}}%
\pgfpathlineto{\pgfqpoint{3.362374in}{0.695589in}}%
\pgfpathlineto{\pgfqpoint{3.298965in}{0.695589in}}%
\pgfpathlineto{\pgfqpoint{3.298965in}{0.499444in}}%
\pgfpathclose%
\pgfusepath{fill}%
\end{pgfscope}%
\begin{pgfscope}%
\pgfpathrectangle{\pgfqpoint{0.445556in}{0.499444in}}{\pgfqpoint{3.487500in}{1.155000in}}%
\pgfusepath{clip}%
\pgfsetbuttcap%
\pgfsetmiterjoin%
\definecolor{currentfill}{rgb}{0.000000,0.000000,0.000000}%
\pgfsetfillcolor{currentfill}%
\pgfsetlinewidth{0.000000pt}%
\definecolor{currentstroke}{rgb}{0.000000,0.000000,0.000000}%
\pgfsetstrokecolor{currentstroke}%
\pgfsetstrokeopacity{0.000000}%
\pgfsetdash{}{0pt}%
\pgfpathmoveto{\pgfqpoint{3.457488in}{0.499444in}}%
\pgfpathlineto{\pgfqpoint{3.520897in}{0.499444in}}%
\pgfpathlineto{\pgfqpoint{3.520897in}{0.681493in}}%
\pgfpathlineto{\pgfqpoint{3.457488in}{0.681493in}}%
\pgfpathlineto{\pgfqpoint{3.457488in}{0.499444in}}%
\pgfpathclose%
\pgfusepath{fill}%
\end{pgfscope}%
\begin{pgfscope}%
\pgfpathrectangle{\pgfqpoint{0.445556in}{0.499444in}}{\pgfqpoint{3.487500in}{1.155000in}}%
\pgfusepath{clip}%
\pgfsetbuttcap%
\pgfsetmiterjoin%
\definecolor{currentfill}{rgb}{0.000000,0.000000,0.000000}%
\pgfsetfillcolor{currentfill}%
\pgfsetlinewidth{0.000000pt}%
\definecolor{currentstroke}{rgb}{0.000000,0.000000,0.000000}%
\pgfsetstrokecolor{currentstroke}%
\pgfsetstrokeopacity{0.000000}%
\pgfsetdash{}{0pt}%
\pgfpathmoveto{\pgfqpoint{3.616010in}{0.499444in}}%
\pgfpathlineto{\pgfqpoint{3.679419in}{0.499444in}}%
\pgfpathlineto{\pgfqpoint{3.679419in}{0.669407in}}%
\pgfpathlineto{\pgfqpoint{3.616010in}{0.669407in}}%
\pgfpathlineto{\pgfqpoint{3.616010in}{0.499444in}}%
\pgfpathclose%
\pgfusepath{fill}%
\end{pgfscope}%
\begin{pgfscope}%
\pgfpathrectangle{\pgfqpoint{0.445556in}{0.499444in}}{\pgfqpoint{3.487500in}{1.155000in}}%
\pgfusepath{clip}%
\pgfsetbuttcap%
\pgfsetmiterjoin%
\definecolor{currentfill}{rgb}{0.000000,0.000000,0.000000}%
\pgfsetfillcolor{currentfill}%
\pgfsetlinewidth{0.000000pt}%
\definecolor{currentstroke}{rgb}{0.000000,0.000000,0.000000}%
\pgfsetstrokecolor{currentstroke}%
\pgfsetstrokeopacity{0.000000}%
\pgfsetdash{}{0pt}%
\pgfpathmoveto{\pgfqpoint{3.774533in}{0.499444in}}%
\pgfpathlineto{\pgfqpoint{3.837942in}{0.499444in}}%
\pgfpathlineto{\pgfqpoint{3.837942in}{0.573993in}}%
\pgfpathlineto{\pgfqpoint{3.774533in}{0.573993in}}%
\pgfpathlineto{\pgfqpoint{3.774533in}{0.499444in}}%
\pgfpathclose%
\pgfusepath{fill}%
\end{pgfscope}%
\begin{pgfscope}%
\pgfsetbuttcap%
\pgfsetroundjoin%
\definecolor{currentfill}{rgb}{0.000000,0.000000,0.000000}%
\pgfsetfillcolor{currentfill}%
\pgfsetlinewidth{0.803000pt}%
\definecolor{currentstroke}{rgb}{0.000000,0.000000,0.000000}%
\pgfsetstrokecolor{currentstroke}%
\pgfsetdash{}{0pt}%
\pgfsys@defobject{currentmarker}{\pgfqpoint{0.000000in}{-0.048611in}}{\pgfqpoint{0.000000in}{0.000000in}}{%
\pgfpathmoveto{\pgfqpoint{0.000000in}{0.000000in}}%
\pgfpathlineto{\pgfqpoint{0.000000in}{-0.048611in}}%
\pgfusepath{stroke,fill}%
}%
\begin{pgfscope}%
\pgfsys@transformshift{0.445556in}{0.499444in}%
\pgfsys@useobject{currentmarker}{}%
\end{pgfscope}%
\end{pgfscope}%
\begin{pgfscope}%
\pgfsetbuttcap%
\pgfsetroundjoin%
\definecolor{currentfill}{rgb}{0.000000,0.000000,0.000000}%
\pgfsetfillcolor{currentfill}%
\pgfsetlinewidth{0.803000pt}%
\definecolor{currentstroke}{rgb}{0.000000,0.000000,0.000000}%
\pgfsetstrokecolor{currentstroke}%
\pgfsetdash{}{0pt}%
\pgfsys@defobject{currentmarker}{\pgfqpoint{0.000000in}{-0.048611in}}{\pgfqpoint{0.000000in}{0.000000in}}{%
\pgfpathmoveto{\pgfqpoint{0.000000in}{0.000000in}}%
\pgfpathlineto{\pgfqpoint{0.000000in}{-0.048611in}}%
\pgfusepath{stroke,fill}%
}%
\begin{pgfscope}%
\pgfsys@transformshift{0.604078in}{0.499444in}%
\pgfsys@useobject{currentmarker}{}%
\end{pgfscope}%
\end{pgfscope}%
\begin{pgfscope}%
\definecolor{textcolor}{rgb}{0.000000,0.000000,0.000000}%
\pgfsetstrokecolor{textcolor}%
\pgfsetfillcolor{textcolor}%
\pgftext[x=0.604078in,y=0.402222in,,top]{\color{textcolor}\rmfamily\fontsize{10.000000}{12.000000}\selectfont 0.0}%
\end{pgfscope}%
\begin{pgfscope}%
\pgfsetbuttcap%
\pgfsetroundjoin%
\definecolor{currentfill}{rgb}{0.000000,0.000000,0.000000}%
\pgfsetfillcolor{currentfill}%
\pgfsetlinewidth{0.803000pt}%
\definecolor{currentstroke}{rgb}{0.000000,0.000000,0.000000}%
\pgfsetstrokecolor{currentstroke}%
\pgfsetdash{}{0pt}%
\pgfsys@defobject{currentmarker}{\pgfqpoint{0.000000in}{-0.048611in}}{\pgfqpoint{0.000000in}{0.000000in}}{%
\pgfpathmoveto{\pgfqpoint{0.000000in}{0.000000in}}%
\pgfpathlineto{\pgfqpoint{0.000000in}{-0.048611in}}%
\pgfusepath{stroke,fill}%
}%
\begin{pgfscope}%
\pgfsys@transformshift{0.762601in}{0.499444in}%
\pgfsys@useobject{currentmarker}{}%
\end{pgfscope}%
\end{pgfscope}%
\begin{pgfscope}%
\pgfsetbuttcap%
\pgfsetroundjoin%
\definecolor{currentfill}{rgb}{0.000000,0.000000,0.000000}%
\pgfsetfillcolor{currentfill}%
\pgfsetlinewidth{0.803000pt}%
\definecolor{currentstroke}{rgb}{0.000000,0.000000,0.000000}%
\pgfsetstrokecolor{currentstroke}%
\pgfsetdash{}{0pt}%
\pgfsys@defobject{currentmarker}{\pgfqpoint{0.000000in}{-0.048611in}}{\pgfqpoint{0.000000in}{0.000000in}}{%
\pgfpathmoveto{\pgfqpoint{0.000000in}{0.000000in}}%
\pgfpathlineto{\pgfqpoint{0.000000in}{-0.048611in}}%
\pgfusepath{stroke,fill}%
}%
\begin{pgfscope}%
\pgfsys@transformshift{0.921124in}{0.499444in}%
\pgfsys@useobject{currentmarker}{}%
\end{pgfscope}%
\end{pgfscope}%
\begin{pgfscope}%
\definecolor{textcolor}{rgb}{0.000000,0.000000,0.000000}%
\pgfsetstrokecolor{textcolor}%
\pgfsetfillcolor{textcolor}%
\pgftext[x=0.921124in,y=0.402222in,,top]{\color{textcolor}\rmfamily\fontsize{10.000000}{12.000000}\selectfont 0.1}%
\end{pgfscope}%
\begin{pgfscope}%
\pgfsetbuttcap%
\pgfsetroundjoin%
\definecolor{currentfill}{rgb}{0.000000,0.000000,0.000000}%
\pgfsetfillcolor{currentfill}%
\pgfsetlinewidth{0.803000pt}%
\definecolor{currentstroke}{rgb}{0.000000,0.000000,0.000000}%
\pgfsetstrokecolor{currentstroke}%
\pgfsetdash{}{0pt}%
\pgfsys@defobject{currentmarker}{\pgfqpoint{0.000000in}{-0.048611in}}{\pgfqpoint{0.000000in}{0.000000in}}{%
\pgfpathmoveto{\pgfqpoint{0.000000in}{0.000000in}}%
\pgfpathlineto{\pgfqpoint{0.000000in}{-0.048611in}}%
\pgfusepath{stroke,fill}%
}%
\begin{pgfscope}%
\pgfsys@transformshift{1.079647in}{0.499444in}%
\pgfsys@useobject{currentmarker}{}%
\end{pgfscope}%
\end{pgfscope}%
\begin{pgfscope}%
\pgfsetbuttcap%
\pgfsetroundjoin%
\definecolor{currentfill}{rgb}{0.000000,0.000000,0.000000}%
\pgfsetfillcolor{currentfill}%
\pgfsetlinewidth{0.803000pt}%
\definecolor{currentstroke}{rgb}{0.000000,0.000000,0.000000}%
\pgfsetstrokecolor{currentstroke}%
\pgfsetdash{}{0pt}%
\pgfsys@defobject{currentmarker}{\pgfqpoint{0.000000in}{-0.048611in}}{\pgfqpoint{0.000000in}{0.000000in}}{%
\pgfpathmoveto{\pgfqpoint{0.000000in}{0.000000in}}%
\pgfpathlineto{\pgfqpoint{0.000000in}{-0.048611in}}%
\pgfusepath{stroke,fill}%
}%
\begin{pgfscope}%
\pgfsys@transformshift{1.238169in}{0.499444in}%
\pgfsys@useobject{currentmarker}{}%
\end{pgfscope}%
\end{pgfscope}%
\begin{pgfscope}%
\definecolor{textcolor}{rgb}{0.000000,0.000000,0.000000}%
\pgfsetstrokecolor{textcolor}%
\pgfsetfillcolor{textcolor}%
\pgftext[x=1.238169in,y=0.402222in,,top]{\color{textcolor}\rmfamily\fontsize{10.000000}{12.000000}\selectfont 0.2}%
\end{pgfscope}%
\begin{pgfscope}%
\pgfsetbuttcap%
\pgfsetroundjoin%
\definecolor{currentfill}{rgb}{0.000000,0.000000,0.000000}%
\pgfsetfillcolor{currentfill}%
\pgfsetlinewidth{0.803000pt}%
\definecolor{currentstroke}{rgb}{0.000000,0.000000,0.000000}%
\pgfsetstrokecolor{currentstroke}%
\pgfsetdash{}{0pt}%
\pgfsys@defobject{currentmarker}{\pgfqpoint{0.000000in}{-0.048611in}}{\pgfqpoint{0.000000in}{0.000000in}}{%
\pgfpathmoveto{\pgfqpoint{0.000000in}{0.000000in}}%
\pgfpathlineto{\pgfqpoint{0.000000in}{-0.048611in}}%
\pgfusepath{stroke,fill}%
}%
\begin{pgfscope}%
\pgfsys@transformshift{1.396692in}{0.499444in}%
\pgfsys@useobject{currentmarker}{}%
\end{pgfscope}%
\end{pgfscope}%
\begin{pgfscope}%
\pgfsetbuttcap%
\pgfsetroundjoin%
\definecolor{currentfill}{rgb}{0.000000,0.000000,0.000000}%
\pgfsetfillcolor{currentfill}%
\pgfsetlinewidth{0.803000pt}%
\definecolor{currentstroke}{rgb}{0.000000,0.000000,0.000000}%
\pgfsetstrokecolor{currentstroke}%
\pgfsetdash{}{0pt}%
\pgfsys@defobject{currentmarker}{\pgfqpoint{0.000000in}{-0.048611in}}{\pgfqpoint{0.000000in}{0.000000in}}{%
\pgfpathmoveto{\pgfqpoint{0.000000in}{0.000000in}}%
\pgfpathlineto{\pgfqpoint{0.000000in}{-0.048611in}}%
\pgfusepath{stroke,fill}%
}%
\begin{pgfscope}%
\pgfsys@transformshift{1.555215in}{0.499444in}%
\pgfsys@useobject{currentmarker}{}%
\end{pgfscope}%
\end{pgfscope}%
\begin{pgfscope}%
\definecolor{textcolor}{rgb}{0.000000,0.000000,0.000000}%
\pgfsetstrokecolor{textcolor}%
\pgfsetfillcolor{textcolor}%
\pgftext[x=1.555215in,y=0.402222in,,top]{\color{textcolor}\rmfamily\fontsize{10.000000}{12.000000}\selectfont 0.3}%
\end{pgfscope}%
\begin{pgfscope}%
\pgfsetbuttcap%
\pgfsetroundjoin%
\definecolor{currentfill}{rgb}{0.000000,0.000000,0.000000}%
\pgfsetfillcolor{currentfill}%
\pgfsetlinewidth{0.803000pt}%
\definecolor{currentstroke}{rgb}{0.000000,0.000000,0.000000}%
\pgfsetstrokecolor{currentstroke}%
\pgfsetdash{}{0pt}%
\pgfsys@defobject{currentmarker}{\pgfqpoint{0.000000in}{-0.048611in}}{\pgfqpoint{0.000000in}{0.000000in}}{%
\pgfpathmoveto{\pgfqpoint{0.000000in}{0.000000in}}%
\pgfpathlineto{\pgfqpoint{0.000000in}{-0.048611in}}%
\pgfusepath{stroke,fill}%
}%
\begin{pgfscope}%
\pgfsys@transformshift{1.713738in}{0.499444in}%
\pgfsys@useobject{currentmarker}{}%
\end{pgfscope}%
\end{pgfscope}%
\begin{pgfscope}%
\pgfsetbuttcap%
\pgfsetroundjoin%
\definecolor{currentfill}{rgb}{0.000000,0.000000,0.000000}%
\pgfsetfillcolor{currentfill}%
\pgfsetlinewidth{0.803000pt}%
\definecolor{currentstroke}{rgb}{0.000000,0.000000,0.000000}%
\pgfsetstrokecolor{currentstroke}%
\pgfsetdash{}{0pt}%
\pgfsys@defobject{currentmarker}{\pgfqpoint{0.000000in}{-0.048611in}}{\pgfqpoint{0.000000in}{0.000000in}}{%
\pgfpathmoveto{\pgfqpoint{0.000000in}{0.000000in}}%
\pgfpathlineto{\pgfqpoint{0.000000in}{-0.048611in}}%
\pgfusepath{stroke,fill}%
}%
\begin{pgfscope}%
\pgfsys@transformshift{1.872260in}{0.499444in}%
\pgfsys@useobject{currentmarker}{}%
\end{pgfscope}%
\end{pgfscope}%
\begin{pgfscope}%
\definecolor{textcolor}{rgb}{0.000000,0.000000,0.000000}%
\pgfsetstrokecolor{textcolor}%
\pgfsetfillcolor{textcolor}%
\pgftext[x=1.872260in,y=0.402222in,,top]{\color{textcolor}\rmfamily\fontsize{10.000000}{12.000000}\selectfont 0.4}%
\end{pgfscope}%
\begin{pgfscope}%
\pgfsetbuttcap%
\pgfsetroundjoin%
\definecolor{currentfill}{rgb}{0.000000,0.000000,0.000000}%
\pgfsetfillcolor{currentfill}%
\pgfsetlinewidth{0.803000pt}%
\definecolor{currentstroke}{rgb}{0.000000,0.000000,0.000000}%
\pgfsetstrokecolor{currentstroke}%
\pgfsetdash{}{0pt}%
\pgfsys@defobject{currentmarker}{\pgfqpoint{0.000000in}{-0.048611in}}{\pgfqpoint{0.000000in}{0.000000in}}{%
\pgfpathmoveto{\pgfqpoint{0.000000in}{0.000000in}}%
\pgfpathlineto{\pgfqpoint{0.000000in}{-0.048611in}}%
\pgfusepath{stroke,fill}%
}%
\begin{pgfscope}%
\pgfsys@transformshift{2.030783in}{0.499444in}%
\pgfsys@useobject{currentmarker}{}%
\end{pgfscope}%
\end{pgfscope}%
\begin{pgfscope}%
\pgfsetbuttcap%
\pgfsetroundjoin%
\definecolor{currentfill}{rgb}{0.000000,0.000000,0.000000}%
\pgfsetfillcolor{currentfill}%
\pgfsetlinewidth{0.803000pt}%
\definecolor{currentstroke}{rgb}{0.000000,0.000000,0.000000}%
\pgfsetstrokecolor{currentstroke}%
\pgfsetdash{}{0pt}%
\pgfsys@defobject{currentmarker}{\pgfqpoint{0.000000in}{-0.048611in}}{\pgfqpoint{0.000000in}{0.000000in}}{%
\pgfpathmoveto{\pgfqpoint{0.000000in}{0.000000in}}%
\pgfpathlineto{\pgfqpoint{0.000000in}{-0.048611in}}%
\pgfusepath{stroke,fill}%
}%
\begin{pgfscope}%
\pgfsys@transformshift{2.189306in}{0.499444in}%
\pgfsys@useobject{currentmarker}{}%
\end{pgfscope}%
\end{pgfscope}%
\begin{pgfscope}%
\definecolor{textcolor}{rgb}{0.000000,0.000000,0.000000}%
\pgfsetstrokecolor{textcolor}%
\pgfsetfillcolor{textcolor}%
\pgftext[x=2.189306in,y=0.402222in,,top]{\color{textcolor}\rmfamily\fontsize{10.000000}{12.000000}\selectfont 0.5}%
\end{pgfscope}%
\begin{pgfscope}%
\pgfsetbuttcap%
\pgfsetroundjoin%
\definecolor{currentfill}{rgb}{0.000000,0.000000,0.000000}%
\pgfsetfillcolor{currentfill}%
\pgfsetlinewidth{0.803000pt}%
\definecolor{currentstroke}{rgb}{0.000000,0.000000,0.000000}%
\pgfsetstrokecolor{currentstroke}%
\pgfsetdash{}{0pt}%
\pgfsys@defobject{currentmarker}{\pgfqpoint{0.000000in}{-0.048611in}}{\pgfqpoint{0.000000in}{0.000000in}}{%
\pgfpathmoveto{\pgfqpoint{0.000000in}{0.000000in}}%
\pgfpathlineto{\pgfqpoint{0.000000in}{-0.048611in}}%
\pgfusepath{stroke,fill}%
}%
\begin{pgfscope}%
\pgfsys@transformshift{2.347828in}{0.499444in}%
\pgfsys@useobject{currentmarker}{}%
\end{pgfscope}%
\end{pgfscope}%
\begin{pgfscope}%
\pgfsetbuttcap%
\pgfsetroundjoin%
\definecolor{currentfill}{rgb}{0.000000,0.000000,0.000000}%
\pgfsetfillcolor{currentfill}%
\pgfsetlinewidth{0.803000pt}%
\definecolor{currentstroke}{rgb}{0.000000,0.000000,0.000000}%
\pgfsetstrokecolor{currentstroke}%
\pgfsetdash{}{0pt}%
\pgfsys@defobject{currentmarker}{\pgfqpoint{0.000000in}{-0.048611in}}{\pgfqpoint{0.000000in}{0.000000in}}{%
\pgfpathmoveto{\pgfqpoint{0.000000in}{0.000000in}}%
\pgfpathlineto{\pgfqpoint{0.000000in}{-0.048611in}}%
\pgfusepath{stroke,fill}%
}%
\begin{pgfscope}%
\pgfsys@transformshift{2.506351in}{0.499444in}%
\pgfsys@useobject{currentmarker}{}%
\end{pgfscope}%
\end{pgfscope}%
\begin{pgfscope}%
\definecolor{textcolor}{rgb}{0.000000,0.000000,0.000000}%
\pgfsetstrokecolor{textcolor}%
\pgfsetfillcolor{textcolor}%
\pgftext[x=2.506351in,y=0.402222in,,top]{\color{textcolor}\rmfamily\fontsize{10.000000}{12.000000}\selectfont 0.6}%
\end{pgfscope}%
\begin{pgfscope}%
\pgfsetbuttcap%
\pgfsetroundjoin%
\definecolor{currentfill}{rgb}{0.000000,0.000000,0.000000}%
\pgfsetfillcolor{currentfill}%
\pgfsetlinewidth{0.803000pt}%
\definecolor{currentstroke}{rgb}{0.000000,0.000000,0.000000}%
\pgfsetstrokecolor{currentstroke}%
\pgfsetdash{}{0pt}%
\pgfsys@defobject{currentmarker}{\pgfqpoint{0.000000in}{-0.048611in}}{\pgfqpoint{0.000000in}{0.000000in}}{%
\pgfpathmoveto{\pgfqpoint{0.000000in}{0.000000in}}%
\pgfpathlineto{\pgfqpoint{0.000000in}{-0.048611in}}%
\pgfusepath{stroke,fill}%
}%
\begin{pgfscope}%
\pgfsys@transformshift{2.664874in}{0.499444in}%
\pgfsys@useobject{currentmarker}{}%
\end{pgfscope}%
\end{pgfscope}%
\begin{pgfscope}%
\pgfsetbuttcap%
\pgfsetroundjoin%
\definecolor{currentfill}{rgb}{0.000000,0.000000,0.000000}%
\pgfsetfillcolor{currentfill}%
\pgfsetlinewidth{0.803000pt}%
\definecolor{currentstroke}{rgb}{0.000000,0.000000,0.000000}%
\pgfsetstrokecolor{currentstroke}%
\pgfsetdash{}{0pt}%
\pgfsys@defobject{currentmarker}{\pgfqpoint{0.000000in}{-0.048611in}}{\pgfqpoint{0.000000in}{0.000000in}}{%
\pgfpathmoveto{\pgfqpoint{0.000000in}{0.000000in}}%
\pgfpathlineto{\pgfqpoint{0.000000in}{-0.048611in}}%
\pgfusepath{stroke,fill}%
}%
\begin{pgfscope}%
\pgfsys@transformshift{2.823397in}{0.499444in}%
\pgfsys@useobject{currentmarker}{}%
\end{pgfscope}%
\end{pgfscope}%
\begin{pgfscope}%
\definecolor{textcolor}{rgb}{0.000000,0.000000,0.000000}%
\pgfsetstrokecolor{textcolor}%
\pgfsetfillcolor{textcolor}%
\pgftext[x=2.823397in,y=0.402222in,,top]{\color{textcolor}\rmfamily\fontsize{10.000000}{12.000000}\selectfont 0.7}%
\end{pgfscope}%
\begin{pgfscope}%
\pgfsetbuttcap%
\pgfsetroundjoin%
\definecolor{currentfill}{rgb}{0.000000,0.000000,0.000000}%
\pgfsetfillcolor{currentfill}%
\pgfsetlinewidth{0.803000pt}%
\definecolor{currentstroke}{rgb}{0.000000,0.000000,0.000000}%
\pgfsetstrokecolor{currentstroke}%
\pgfsetdash{}{0pt}%
\pgfsys@defobject{currentmarker}{\pgfqpoint{0.000000in}{-0.048611in}}{\pgfqpoint{0.000000in}{0.000000in}}{%
\pgfpathmoveto{\pgfqpoint{0.000000in}{0.000000in}}%
\pgfpathlineto{\pgfqpoint{0.000000in}{-0.048611in}}%
\pgfusepath{stroke,fill}%
}%
\begin{pgfscope}%
\pgfsys@transformshift{2.981919in}{0.499444in}%
\pgfsys@useobject{currentmarker}{}%
\end{pgfscope}%
\end{pgfscope}%
\begin{pgfscope}%
\pgfsetbuttcap%
\pgfsetroundjoin%
\definecolor{currentfill}{rgb}{0.000000,0.000000,0.000000}%
\pgfsetfillcolor{currentfill}%
\pgfsetlinewidth{0.803000pt}%
\definecolor{currentstroke}{rgb}{0.000000,0.000000,0.000000}%
\pgfsetstrokecolor{currentstroke}%
\pgfsetdash{}{0pt}%
\pgfsys@defobject{currentmarker}{\pgfqpoint{0.000000in}{-0.048611in}}{\pgfqpoint{0.000000in}{0.000000in}}{%
\pgfpathmoveto{\pgfqpoint{0.000000in}{0.000000in}}%
\pgfpathlineto{\pgfqpoint{0.000000in}{-0.048611in}}%
\pgfusepath{stroke,fill}%
}%
\begin{pgfscope}%
\pgfsys@transformshift{3.140442in}{0.499444in}%
\pgfsys@useobject{currentmarker}{}%
\end{pgfscope}%
\end{pgfscope}%
\begin{pgfscope}%
\definecolor{textcolor}{rgb}{0.000000,0.000000,0.000000}%
\pgfsetstrokecolor{textcolor}%
\pgfsetfillcolor{textcolor}%
\pgftext[x=3.140442in,y=0.402222in,,top]{\color{textcolor}\rmfamily\fontsize{10.000000}{12.000000}\selectfont 0.8}%
\end{pgfscope}%
\begin{pgfscope}%
\pgfsetbuttcap%
\pgfsetroundjoin%
\definecolor{currentfill}{rgb}{0.000000,0.000000,0.000000}%
\pgfsetfillcolor{currentfill}%
\pgfsetlinewidth{0.803000pt}%
\definecolor{currentstroke}{rgb}{0.000000,0.000000,0.000000}%
\pgfsetstrokecolor{currentstroke}%
\pgfsetdash{}{0pt}%
\pgfsys@defobject{currentmarker}{\pgfqpoint{0.000000in}{-0.048611in}}{\pgfqpoint{0.000000in}{0.000000in}}{%
\pgfpathmoveto{\pgfqpoint{0.000000in}{0.000000in}}%
\pgfpathlineto{\pgfqpoint{0.000000in}{-0.048611in}}%
\pgfusepath{stroke,fill}%
}%
\begin{pgfscope}%
\pgfsys@transformshift{3.298965in}{0.499444in}%
\pgfsys@useobject{currentmarker}{}%
\end{pgfscope}%
\end{pgfscope}%
\begin{pgfscope}%
\pgfsetbuttcap%
\pgfsetroundjoin%
\definecolor{currentfill}{rgb}{0.000000,0.000000,0.000000}%
\pgfsetfillcolor{currentfill}%
\pgfsetlinewidth{0.803000pt}%
\definecolor{currentstroke}{rgb}{0.000000,0.000000,0.000000}%
\pgfsetstrokecolor{currentstroke}%
\pgfsetdash{}{0pt}%
\pgfsys@defobject{currentmarker}{\pgfqpoint{0.000000in}{-0.048611in}}{\pgfqpoint{0.000000in}{0.000000in}}{%
\pgfpathmoveto{\pgfqpoint{0.000000in}{0.000000in}}%
\pgfpathlineto{\pgfqpoint{0.000000in}{-0.048611in}}%
\pgfusepath{stroke,fill}%
}%
\begin{pgfscope}%
\pgfsys@transformshift{3.457488in}{0.499444in}%
\pgfsys@useobject{currentmarker}{}%
\end{pgfscope}%
\end{pgfscope}%
\begin{pgfscope}%
\definecolor{textcolor}{rgb}{0.000000,0.000000,0.000000}%
\pgfsetstrokecolor{textcolor}%
\pgfsetfillcolor{textcolor}%
\pgftext[x=3.457488in,y=0.402222in,,top]{\color{textcolor}\rmfamily\fontsize{10.000000}{12.000000}\selectfont 0.9}%
\end{pgfscope}%
\begin{pgfscope}%
\pgfsetbuttcap%
\pgfsetroundjoin%
\definecolor{currentfill}{rgb}{0.000000,0.000000,0.000000}%
\pgfsetfillcolor{currentfill}%
\pgfsetlinewidth{0.803000pt}%
\definecolor{currentstroke}{rgb}{0.000000,0.000000,0.000000}%
\pgfsetstrokecolor{currentstroke}%
\pgfsetdash{}{0pt}%
\pgfsys@defobject{currentmarker}{\pgfqpoint{0.000000in}{-0.048611in}}{\pgfqpoint{0.000000in}{0.000000in}}{%
\pgfpathmoveto{\pgfqpoint{0.000000in}{0.000000in}}%
\pgfpathlineto{\pgfqpoint{0.000000in}{-0.048611in}}%
\pgfusepath{stroke,fill}%
}%
\begin{pgfscope}%
\pgfsys@transformshift{3.616010in}{0.499444in}%
\pgfsys@useobject{currentmarker}{}%
\end{pgfscope}%
\end{pgfscope}%
\begin{pgfscope}%
\pgfsetbuttcap%
\pgfsetroundjoin%
\definecolor{currentfill}{rgb}{0.000000,0.000000,0.000000}%
\pgfsetfillcolor{currentfill}%
\pgfsetlinewidth{0.803000pt}%
\definecolor{currentstroke}{rgb}{0.000000,0.000000,0.000000}%
\pgfsetstrokecolor{currentstroke}%
\pgfsetdash{}{0pt}%
\pgfsys@defobject{currentmarker}{\pgfqpoint{0.000000in}{-0.048611in}}{\pgfqpoint{0.000000in}{0.000000in}}{%
\pgfpathmoveto{\pgfqpoint{0.000000in}{0.000000in}}%
\pgfpathlineto{\pgfqpoint{0.000000in}{-0.048611in}}%
\pgfusepath{stroke,fill}%
}%
\begin{pgfscope}%
\pgfsys@transformshift{3.774533in}{0.499444in}%
\pgfsys@useobject{currentmarker}{}%
\end{pgfscope}%
\end{pgfscope}%
\begin{pgfscope}%
\definecolor{textcolor}{rgb}{0.000000,0.000000,0.000000}%
\pgfsetstrokecolor{textcolor}%
\pgfsetfillcolor{textcolor}%
\pgftext[x=3.774533in,y=0.402222in,,top]{\color{textcolor}\rmfamily\fontsize{10.000000}{12.000000}\selectfont 1.0}%
\end{pgfscope}%
\begin{pgfscope}%
\pgfsetbuttcap%
\pgfsetroundjoin%
\definecolor{currentfill}{rgb}{0.000000,0.000000,0.000000}%
\pgfsetfillcolor{currentfill}%
\pgfsetlinewidth{0.803000pt}%
\definecolor{currentstroke}{rgb}{0.000000,0.000000,0.000000}%
\pgfsetstrokecolor{currentstroke}%
\pgfsetdash{}{0pt}%
\pgfsys@defobject{currentmarker}{\pgfqpoint{0.000000in}{-0.048611in}}{\pgfqpoint{0.000000in}{0.000000in}}{%
\pgfpathmoveto{\pgfqpoint{0.000000in}{0.000000in}}%
\pgfpathlineto{\pgfqpoint{0.000000in}{-0.048611in}}%
\pgfusepath{stroke,fill}%
}%
\begin{pgfscope}%
\pgfsys@transformshift{3.933056in}{0.499444in}%
\pgfsys@useobject{currentmarker}{}%
\end{pgfscope}%
\end{pgfscope}%
\begin{pgfscope}%
\definecolor{textcolor}{rgb}{0.000000,0.000000,0.000000}%
\pgfsetstrokecolor{textcolor}%
\pgfsetfillcolor{textcolor}%
\pgftext[x=2.189306in,y=0.223333in,,top]{\color{textcolor}\rmfamily\fontsize{10.000000}{12.000000}\selectfont \(\displaystyle p\)}%
\end{pgfscope}%
\begin{pgfscope}%
\pgfsetbuttcap%
\pgfsetroundjoin%
\definecolor{currentfill}{rgb}{0.000000,0.000000,0.000000}%
\pgfsetfillcolor{currentfill}%
\pgfsetlinewidth{0.803000pt}%
\definecolor{currentstroke}{rgb}{0.000000,0.000000,0.000000}%
\pgfsetstrokecolor{currentstroke}%
\pgfsetdash{}{0pt}%
\pgfsys@defobject{currentmarker}{\pgfqpoint{-0.048611in}{0.000000in}}{\pgfqpoint{-0.000000in}{0.000000in}}{%
\pgfpathmoveto{\pgfqpoint{-0.000000in}{0.000000in}}%
\pgfpathlineto{\pgfqpoint{-0.048611in}{0.000000in}}%
\pgfusepath{stroke,fill}%
}%
\begin{pgfscope}%
\pgfsys@transformshift{0.445556in}{0.499444in}%
\pgfsys@useobject{currentmarker}{}%
\end{pgfscope}%
\end{pgfscope}%
\begin{pgfscope}%
\definecolor{textcolor}{rgb}{0.000000,0.000000,0.000000}%
\pgfsetstrokecolor{textcolor}%
\pgfsetfillcolor{textcolor}%
\pgftext[x=0.278889in, y=0.451250in, left, base]{\color{textcolor}\rmfamily\fontsize{10.000000}{12.000000}\selectfont \(\displaystyle {0}\)}%
\end{pgfscope}%
\begin{pgfscope}%
\pgfsetbuttcap%
\pgfsetroundjoin%
\definecolor{currentfill}{rgb}{0.000000,0.000000,0.000000}%
\pgfsetfillcolor{currentfill}%
\pgfsetlinewidth{0.803000pt}%
\definecolor{currentstroke}{rgb}{0.000000,0.000000,0.000000}%
\pgfsetstrokecolor{currentstroke}%
\pgfsetdash{}{0pt}%
\pgfsys@defobject{currentmarker}{\pgfqpoint{-0.048611in}{0.000000in}}{\pgfqpoint{-0.000000in}{0.000000in}}{%
\pgfpathmoveto{\pgfqpoint{-0.000000in}{0.000000in}}%
\pgfpathlineto{\pgfqpoint{-0.048611in}{0.000000in}}%
\pgfusepath{stroke,fill}%
}%
\begin{pgfscope}%
\pgfsys@transformshift{0.445556in}{0.818266in}%
\pgfsys@useobject{currentmarker}{}%
\end{pgfscope}%
\end{pgfscope}%
\begin{pgfscope}%
\definecolor{textcolor}{rgb}{0.000000,0.000000,0.000000}%
\pgfsetstrokecolor{textcolor}%
\pgfsetfillcolor{textcolor}%
\pgftext[x=0.278889in, y=0.770071in, left, base]{\color{textcolor}\rmfamily\fontsize{10.000000}{12.000000}\selectfont \(\displaystyle {2}\)}%
\end{pgfscope}%
\begin{pgfscope}%
\pgfsetbuttcap%
\pgfsetroundjoin%
\definecolor{currentfill}{rgb}{0.000000,0.000000,0.000000}%
\pgfsetfillcolor{currentfill}%
\pgfsetlinewidth{0.803000pt}%
\definecolor{currentstroke}{rgb}{0.000000,0.000000,0.000000}%
\pgfsetstrokecolor{currentstroke}%
\pgfsetdash{}{0pt}%
\pgfsys@defobject{currentmarker}{\pgfqpoint{-0.048611in}{0.000000in}}{\pgfqpoint{-0.000000in}{0.000000in}}{%
\pgfpathmoveto{\pgfqpoint{-0.000000in}{0.000000in}}%
\pgfpathlineto{\pgfqpoint{-0.048611in}{0.000000in}}%
\pgfusepath{stroke,fill}%
}%
\begin{pgfscope}%
\pgfsys@transformshift{0.445556in}{1.137087in}%
\pgfsys@useobject{currentmarker}{}%
\end{pgfscope}%
\end{pgfscope}%
\begin{pgfscope}%
\definecolor{textcolor}{rgb}{0.000000,0.000000,0.000000}%
\pgfsetstrokecolor{textcolor}%
\pgfsetfillcolor{textcolor}%
\pgftext[x=0.278889in, y=1.088893in, left, base]{\color{textcolor}\rmfamily\fontsize{10.000000}{12.000000}\selectfont \(\displaystyle {4}\)}%
\end{pgfscope}%
\begin{pgfscope}%
\pgfsetbuttcap%
\pgfsetroundjoin%
\definecolor{currentfill}{rgb}{0.000000,0.000000,0.000000}%
\pgfsetfillcolor{currentfill}%
\pgfsetlinewidth{0.803000pt}%
\definecolor{currentstroke}{rgb}{0.000000,0.000000,0.000000}%
\pgfsetstrokecolor{currentstroke}%
\pgfsetdash{}{0pt}%
\pgfsys@defobject{currentmarker}{\pgfqpoint{-0.048611in}{0.000000in}}{\pgfqpoint{-0.000000in}{0.000000in}}{%
\pgfpathmoveto{\pgfqpoint{-0.000000in}{0.000000in}}%
\pgfpathlineto{\pgfqpoint{-0.048611in}{0.000000in}}%
\pgfusepath{stroke,fill}%
}%
\begin{pgfscope}%
\pgfsys@transformshift{0.445556in}{1.455909in}%
\pgfsys@useobject{currentmarker}{}%
\end{pgfscope}%
\end{pgfscope}%
\begin{pgfscope}%
\definecolor{textcolor}{rgb}{0.000000,0.000000,0.000000}%
\pgfsetstrokecolor{textcolor}%
\pgfsetfillcolor{textcolor}%
\pgftext[x=0.278889in, y=1.407714in, left, base]{\color{textcolor}\rmfamily\fontsize{10.000000}{12.000000}\selectfont \(\displaystyle {6}\)}%
\end{pgfscope}%
\begin{pgfscope}%
\definecolor{textcolor}{rgb}{0.000000,0.000000,0.000000}%
\pgfsetstrokecolor{textcolor}%
\pgfsetfillcolor{textcolor}%
\pgftext[x=0.223333in,y=1.076944in,,bottom,rotate=90.000000]{\color{textcolor}\rmfamily\fontsize{10.000000}{12.000000}\selectfont Percent of Data Set}%
\end{pgfscope}%
\begin{pgfscope}%
\pgfsetrectcap%
\pgfsetmiterjoin%
\pgfsetlinewidth{0.803000pt}%
\definecolor{currentstroke}{rgb}{0.000000,0.000000,0.000000}%
\pgfsetstrokecolor{currentstroke}%
\pgfsetdash{}{0pt}%
\pgfpathmoveto{\pgfqpoint{0.445556in}{0.499444in}}%
\pgfpathlineto{\pgfqpoint{0.445556in}{1.654444in}}%
\pgfusepath{stroke}%
\end{pgfscope}%
\begin{pgfscope}%
\pgfsetrectcap%
\pgfsetmiterjoin%
\pgfsetlinewidth{0.803000pt}%
\definecolor{currentstroke}{rgb}{0.000000,0.000000,0.000000}%
\pgfsetstrokecolor{currentstroke}%
\pgfsetdash{}{0pt}%
\pgfpathmoveto{\pgfqpoint{3.933056in}{0.499444in}}%
\pgfpathlineto{\pgfqpoint{3.933056in}{1.654444in}}%
\pgfusepath{stroke}%
\end{pgfscope}%
\begin{pgfscope}%
\pgfsetrectcap%
\pgfsetmiterjoin%
\pgfsetlinewidth{0.803000pt}%
\definecolor{currentstroke}{rgb}{0.000000,0.000000,0.000000}%
\pgfsetstrokecolor{currentstroke}%
\pgfsetdash{}{0pt}%
\pgfpathmoveto{\pgfqpoint{0.445556in}{0.499444in}}%
\pgfpathlineto{\pgfqpoint{3.933056in}{0.499444in}}%
\pgfusepath{stroke}%
\end{pgfscope}%
\begin{pgfscope}%
\pgfsetrectcap%
\pgfsetmiterjoin%
\pgfsetlinewidth{0.803000pt}%
\definecolor{currentstroke}{rgb}{0.000000,0.000000,0.000000}%
\pgfsetstrokecolor{currentstroke}%
\pgfsetdash{}{0pt}%
\pgfpathmoveto{\pgfqpoint{0.445556in}{1.654444in}}%
\pgfpathlineto{\pgfqpoint{3.933056in}{1.654444in}}%
\pgfusepath{stroke}%
\end{pgfscope}%
\begin{pgfscope}%
\pgfsetbuttcap%
\pgfsetmiterjoin%
\definecolor{currentfill}{rgb}{1.000000,1.000000,1.000000}%
\pgfsetfillcolor{currentfill}%
\pgfsetfillopacity{0.800000}%
\pgfsetlinewidth{1.003750pt}%
\definecolor{currentstroke}{rgb}{0.800000,0.800000,0.800000}%
\pgfsetstrokecolor{currentstroke}%
\pgfsetstrokeopacity{0.800000}%
\pgfsetdash{}{0pt}%
\pgfpathmoveto{\pgfqpoint{3.156111in}{1.154445in}}%
\pgfpathlineto{\pgfqpoint{3.835833in}{1.154445in}}%
\pgfpathquadraticcurveto{\pgfqpoint{3.863611in}{1.154445in}}{\pgfqpoint{3.863611in}{1.182222in}}%
\pgfpathlineto{\pgfqpoint{3.863611in}{1.557222in}}%
\pgfpathquadraticcurveto{\pgfqpoint{3.863611in}{1.585000in}}{\pgfqpoint{3.835833in}{1.585000in}}%
\pgfpathlineto{\pgfqpoint{3.156111in}{1.585000in}}%
\pgfpathquadraticcurveto{\pgfqpoint{3.128333in}{1.585000in}}{\pgfqpoint{3.128333in}{1.557222in}}%
\pgfpathlineto{\pgfqpoint{3.128333in}{1.182222in}}%
\pgfpathquadraticcurveto{\pgfqpoint{3.128333in}{1.154445in}}{\pgfqpoint{3.156111in}{1.154445in}}%
\pgfpathlineto{\pgfqpoint{3.156111in}{1.154445in}}%
\pgfpathclose%
\pgfusepath{stroke,fill}%
\end{pgfscope}%
\begin{pgfscope}%
\pgfsetbuttcap%
\pgfsetmiterjoin%
\pgfsetlinewidth{1.003750pt}%
\definecolor{currentstroke}{rgb}{0.000000,0.000000,0.000000}%
\pgfsetstrokecolor{currentstroke}%
\pgfsetdash{}{0pt}%
\pgfpathmoveto{\pgfqpoint{3.183889in}{1.432222in}}%
\pgfpathlineto{\pgfqpoint{3.461667in}{1.432222in}}%
\pgfpathlineto{\pgfqpoint{3.461667in}{1.529444in}}%
\pgfpathlineto{\pgfqpoint{3.183889in}{1.529444in}}%
\pgfpathlineto{\pgfqpoint{3.183889in}{1.432222in}}%
\pgfpathclose%
\pgfusepath{stroke}%
\end{pgfscope}%
\begin{pgfscope}%
\definecolor{textcolor}{rgb}{0.000000,0.000000,0.000000}%
\pgfsetstrokecolor{textcolor}%
\pgfsetfillcolor{textcolor}%
\pgftext[x=3.572778in,y=1.432222in,left,base]{\color{textcolor}\rmfamily\fontsize{10.000000}{12.000000}\selectfont Neg}%
\end{pgfscope}%
\begin{pgfscope}%
\pgfsetbuttcap%
\pgfsetmiterjoin%
\definecolor{currentfill}{rgb}{0.000000,0.000000,0.000000}%
\pgfsetfillcolor{currentfill}%
\pgfsetlinewidth{0.000000pt}%
\definecolor{currentstroke}{rgb}{0.000000,0.000000,0.000000}%
\pgfsetstrokecolor{currentstroke}%
\pgfsetstrokeopacity{0.000000}%
\pgfsetdash{}{0pt}%
\pgfpathmoveto{\pgfqpoint{3.183889in}{1.236944in}}%
\pgfpathlineto{\pgfqpoint{3.461667in}{1.236944in}}%
\pgfpathlineto{\pgfqpoint{3.461667in}{1.334167in}}%
\pgfpathlineto{\pgfqpoint{3.183889in}{1.334167in}}%
\pgfpathlineto{\pgfqpoint{3.183889in}{1.236944in}}%
\pgfpathclose%
\pgfusepath{fill}%
\end{pgfscope}%
\begin{pgfscope}%
\definecolor{textcolor}{rgb}{0.000000,0.000000,0.000000}%
\pgfsetstrokecolor{textcolor}%
\pgfsetfillcolor{textcolor}%
\pgftext[x=3.572778in,y=1.236944in,left,base]{\color{textcolor}\rmfamily\fontsize{10.000000}{12.000000}\selectfont Pos}%
\end{pgfscope}%
\end{pgfpicture}%
\makeatother%
\endgroup%
	
&
	\vskip 0pt
	\hfil ROC Curve
	
	%% Creator: Matplotlib, PGF backend
%%
%% To include the figure in your LaTeX document, write
%%   \input{<filename>.pgf}
%%
%% Make sure the required packages are loaded in your preamble
%%   \usepackage{pgf}
%%
%% Also ensure that all the required font packages are loaded; for instance,
%% the lmodern package is sometimes necessary when using math font.
%%   \usepackage{lmodern}
%%
%% Figures using additional raster images can only be included by \input if
%% they are in the same directory as the main LaTeX file. For loading figures
%% from other directories you can use the `import` package
%%   \usepackage{import}
%%
%% and then include the figures with
%%   \import{<path to file>}{<filename>.pgf}
%%
%% Matplotlib used the following preamble
%%   
%%   \usepackage{fontspec}
%%   \makeatletter\@ifpackageloaded{underscore}{}{\usepackage[strings]{underscore}}\makeatother
%%
\begingroup%
\makeatletter%
\begin{pgfpicture}%
\pgfpathrectangle{\pgfpointorigin}{\pgfqpoint{2.221861in}{1.754444in}}%
\pgfusepath{use as bounding box, clip}%
\begin{pgfscope}%
\pgfsetbuttcap%
\pgfsetmiterjoin%
\definecolor{currentfill}{rgb}{1.000000,1.000000,1.000000}%
\pgfsetfillcolor{currentfill}%
\pgfsetlinewidth{0.000000pt}%
\definecolor{currentstroke}{rgb}{1.000000,1.000000,1.000000}%
\pgfsetstrokecolor{currentstroke}%
\pgfsetdash{}{0pt}%
\pgfpathmoveto{\pgfqpoint{0.000000in}{0.000000in}}%
\pgfpathlineto{\pgfqpoint{2.221861in}{0.000000in}}%
\pgfpathlineto{\pgfqpoint{2.221861in}{1.754444in}}%
\pgfpathlineto{\pgfqpoint{0.000000in}{1.754444in}}%
\pgfpathlineto{\pgfqpoint{0.000000in}{0.000000in}}%
\pgfpathclose%
\pgfusepath{fill}%
\end{pgfscope}%
\begin{pgfscope}%
\pgfsetbuttcap%
\pgfsetmiterjoin%
\definecolor{currentfill}{rgb}{1.000000,1.000000,1.000000}%
\pgfsetfillcolor{currentfill}%
\pgfsetlinewidth{0.000000pt}%
\definecolor{currentstroke}{rgb}{0.000000,0.000000,0.000000}%
\pgfsetstrokecolor{currentstroke}%
\pgfsetstrokeopacity{0.000000}%
\pgfsetdash{}{0pt}%
\pgfpathmoveto{\pgfqpoint{0.553581in}{0.499444in}}%
\pgfpathlineto{\pgfqpoint{2.103581in}{0.499444in}}%
\pgfpathlineto{\pgfqpoint{2.103581in}{1.654444in}}%
\pgfpathlineto{\pgfqpoint{0.553581in}{1.654444in}}%
\pgfpathlineto{\pgfqpoint{0.553581in}{0.499444in}}%
\pgfpathclose%
\pgfusepath{fill}%
\end{pgfscope}%
\begin{pgfscope}%
\pgfsetbuttcap%
\pgfsetroundjoin%
\definecolor{currentfill}{rgb}{0.000000,0.000000,0.000000}%
\pgfsetfillcolor{currentfill}%
\pgfsetlinewidth{0.803000pt}%
\definecolor{currentstroke}{rgb}{0.000000,0.000000,0.000000}%
\pgfsetstrokecolor{currentstroke}%
\pgfsetdash{}{0pt}%
\pgfsys@defobject{currentmarker}{\pgfqpoint{0.000000in}{-0.048611in}}{\pgfqpoint{0.000000in}{0.000000in}}{%
\pgfpathmoveto{\pgfqpoint{0.000000in}{0.000000in}}%
\pgfpathlineto{\pgfqpoint{0.000000in}{-0.048611in}}%
\pgfusepath{stroke,fill}%
}%
\begin{pgfscope}%
\pgfsys@transformshift{0.624035in}{0.499444in}%
\pgfsys@useobject{currentmarker}{}%
\end{pgfscope}%
\end{pgfscope}%
\begin{pgfscope}%
\definecolor{textcolor}{rgb}{0.000000,0.000000,0.000000}%
\pgfsetstrokecolor{textcolor}%
\pgfsetfillcolor{textcolor}%
\pgftext[x=0.624035in,y=0.402222in,,top]{\color{textcolor}\rmfamily\fontsize{10.000000}{12.000000}\selectfont \(\displaystyle {0.0}\)}%
\end{pgfscope}%
\begin{pgfscope}%
\pgfsetbuttcap%
\pgfsetroundjoin%
\definecolor{currentfill}{rgb}{0.000000,0.000000,0.000000}%
\pgfsetfillcolor{currentfill}%
\pgfsetlinewidth{0.803000pt}%
\definecolor{currentstroke}{rgb}{0.000000,0.000000,0.000000}%
\pgfsetstrokecolor{currentstroke}%
\pgfsetdash{}{0pt}%
\pgfsys@defobject{currentmarker}{\pgfqpoint{0.000000in}{-0.048611in}}{\pgfqpoint{0.000000in}{0.000000in}}{%
\pgfpathmoveto{\pgfqpoint{0.000000in}{0.000000in}}%
\pgfpathlineto{\pgfqpoint{0.000000in}{-0.048611in}}%
\pgfusepath{stroke,fill}%
}%
\begin{pgfscope}%
\pgfsys@transformshift{1.328581in}{0.499444in}%
\pgfsys@useobject{currentmarker}{}%
\end{pgfscope}%
\end{pgfscope}%
\begin{pgfscope}%
\definecolor{textcolor}{rgb}{0.000000,0.000000,0.000000}%
\pgfsetstrokecolor{textcolor}%
\pgfsetfillcolor{textcolor}%
\pgftext[x=1.328581in,y=0.402222in,,top]{\color{textcolor}\rmfamily\fontsize{10.000000}{12.000000}\selectfont \(\displaystyle {0.5}\)}%
\end{pgfscope}%
\begin{pgfscope}%
\pgfsetbuttcap%
\pgfsetroundjoin%
\definecolor{currentfill}{rgb}{0.000000,0.000000,0.000000}%
\pgfsetfillcolor{currentfill}%
\pgfsetlinewidth{0.803000pt}%
\definecolor{currentstroke}{rgb}{0.000000,0.000000,0.000000}%
\pgfsetstrokecolor{currentstroke}%
\pgfsetdash{}{0pt}%
\pgfsys@defobject{currentmarker}{\pgfqpoint{0.000000in}{-0.048611in}}{\pgfqpoint{0.000000in}{0.000000in}}{%
\pgfpathmoveto{\pgfqpoint{0.000000in}{0.000000in}}%
\pgfpathlineto{\pgfqpoint{0.000000in}{-0.048611in}}%
\pgfusepath{stroke,fill}%
}%
\begin{pgfscope}%
\pgfsys@transformshift{2.033126in}{0.499444in}%
\pgfsys@useobject{currentmarker}{}%
\end{pgfscope}%
\end{pgfscope}%
\begin{pgfscope}%
\definecolor{textcolor}{rgb}{0.000000,0.000000,0.000000}%
\pgfsetstrokecolor{textcolor}%
\pgfsetfillcolor{textcolor}%
\pgftext[x=2.033126in,y=0.402222in,,top]{\color{textcolor}\rmfamily\fontsize{10.000000}{12.000000}\selectfont \(\displaystyle {1.0}\)}%
\end{pgfscope}%
\begin{pgfscope}%
\definecolor{textcolor}{rgb}{0.000000,0.000000,0.000000}%
\pgfsetstrokecolor{textcolor}%
\pgfsetfillcolor{textcolor}%
\pgftext[x=1.328581in,y=0.223333in,,top]{\color{textcolor}\rmfamily\fontsize{10.000000}{12.000000}\selectfont False positive rate}%
\end{pgfscope}%
\begin{pgfscope}%
\pgfsetbuttcap%
\pgfsetroundjoin%
\definecolor{currentfill}{rgb}{0.000000,0.000000,0.000000}%
\pgfsetfillcolor{currentfill}%
\pgfsetlinewidth{0.803000pt}%
\definecolor{currentstroke}{rgb}{0.000000,0.000000,0.000000}%
\pgfsetstrokecolor{currentstroke}%
\pgfsetdash{}{0pt}%
\pgfsys@defobject{currentmarker}{\pgfqpoint{-0.048611in}{0.000000in}}{\pgfqpoint{-0.000000in}{0.000000in}}{%
\pgfpathmoveto{\pgfqpoint{-0.000000in}{0.000000in}}%
\pgfpathlineto{\pgfqpoint{-0.048611in}{0.000000in}}%
\pgfusepath{stroke,fill}%
}%
\begin{pgfscope}%
\pgfsys@transformshift{0.553581in}{0.551944in}%
\pgfsys@useobject{currentmarker}{}%
\end{pgfscope}%
\end{pgfscope}%
\begin{pgfscope}%
\definecolor{textcolor}{rgb}{0.000000,0.000000,0.000000}%
\pgfsetstrokecolor{textcolor}%
\pgfsetfillcolor{textcolor}%
\pgftext[x=0.278889in, y=0.503750in, left, base]{\color{textcolor}\rmfamily\fontsize{10.000000}{12.000000}\selectfont \(\displaystyle {0.0}\)}%
\end{pgfscope}%
\begin{pgfscope}%
\pgfsetbuttcap%
\pgfsetroundjoin%
\definecolor{currentfill}{rgb}{0.000000,0.000000,0.000000}%
\pgfsetfillcolor{currentfill}%
\pgfsetlinewidth{0.803000pt}%
\definecolor{currentstroke}{rgb}{0.000000,0.000000,0.000000}%
\pgfsetstrokecolor{currentstroke}%
\pgfsetdash{}{0pt}%
\pgfsys@defobject{currentmarker}{\pgfqpoint{-0.048611in}{0.000000in}}{\pgfqpoint{-0.000000in}{0.000000in}}{%
\pgfpathmoveto{\pgfqpoint{-0.000000in}{0.000000in}}%
\pgfpathlineto{\pgfqpoint{-0.048611in}{0.000000in}}%
\pgfusepath{stroke,fill}%
}%
\begin{pgfscope}%
\pgfsys@transformshift{0.553581in}{1.076944in}%
\pgfsys@useobject{currentmarker}{}%
\end{pgfscope}%
\end{pgfscope}%
\begin{pgfscope}%
\definecolor{textcolor}{rgb}{0.000000,0.000000,0.000000}%
\pgfsetstrokecolor{textcolor}%
\pgfsetfillcolor{textcolor}%
\pgftext[x=0.278889in, y=1.028750in, left, base]{\color{textcolor}\rmfamily\fontsize{10.000000}{12.000000}\selectfont \(\displaystyle {0.5}\)}%
\end{pgfscope}%
\begin{pgfscope}%
\pgfsetbuttcap%
\pgfsetroundjoin%
\definecolor{currentfill}{rgb}{0.000000,0.000000,0.000000}%
\pgfsetfillcolor{currentfill}%
\pgfsetlinewidth{0.803000pt}%
\definecolor{currentstroke}{rgb}{0.000000,0.000000,0.000000}%
\pgfsetstrokecolor{currentstroke}%
\pgfsetdash{}{0pt}%
\pgfsys@defobject{currentmarker}{\pgfqpoint{-0.048611in}{0.000000in}}{\pgfqpoint{-0.000000in}{0.000000in}}{%
\pgfpathmoveto{\pgfqpoint{-0.000000in}{0.000000in}}%
\pgfpathlineto{\pgfqpoint{-0.048611in}{0.000000in}}%
\pgfusepath{stroke,fill}%
}%
\begin{pgfscope}%
\pgfsys@transformshift{0.553581in}{1.601944in}%
\pgfsys@useobject{currentmarker}{}%
\end{pgfscope}%
\end{pgfscope}%
\begin{pgfscope}%
\definecolor{textcolor}{rgb}{0.000000,0.000000,0.000000}%
\pgfsetstrokecolor{textcolor}%
\pgfsetfillcolor{textcolor}%
\pgftext[x=0.278889in, y=1.553750in, left, base]{\color{textcolor}\rmfamily\fontsize{10.000000}{12.000000}\selectfont \(\displaystyle {1.0}\)}%
\end{pgfscope}%
\begin{pgfscope}%
\definecolor{textcolor}{rgb}{0.000000,0.000000,0.000000}%
\pgfsetstrokecolor{textcolor}%
\pgfsetfillcolor{textcolor}%
\pgftext[x=0.223333in,y=1.076944in,,bottom,rotate=90.000000]{\color{textcolor}\rmfamily\fontsize{10.000000}{12.000000}\selectfont True positive rate}%
\end{pgfscope}%
\begin{pgfscope}%
\pgfpathrectangle{\pgfqpoint{0.553581in}{0.499444in}}{\pgfqpoint{1.550000in}{1.155000in}}%
\pgfusepath{clip}%
\pgfsetbuttcap%
\pgfsetroundjoin%
\pgfsetlinewidth{1.505625pt}%
\definecolor{currentstroke}{rgb}{0.000000,0.000000,0.000000}%
\pgfsetstrokecolor{currentstroke}%
\pgfsetdash{{5.550000pt}{2.400000pt}}{0.000000pt}%
\pgfpathmoveto{\pgfqpoint{0.624035in}{0.551944in}}%
\pgfpathlineto{\pgfqpoint{2.033126in}{1.601944in}}%
\pgfusepath{stroke}%
\end{pgfscope}%
\begin{pgfscope}%
\pgfpathrectangle{\pgfqpoint{0.553581in}{0.499444in}}{\pgfqpoint{1.550000in}{1.155000in}}%
\pgfusepath{clip}%
\pgfsetrectcap%
\pgfsetroundjoin%
\pgfsetlinewidth{1.505625pt}%
\definecolor{currentstroke}{rgb}{0.000000,0.000000,0.000000}%
\pgfsetstrokecolor{currentstroke}%
\pgfsetdash{}{0pt}%
\pgfpathmoveto{\pgfqpoint{0.624035in}{0.551944in}}%
\pgfpathlineto{\pgfqpoint{0.625145in}{0.567152in}}%
\pgfpathlineto{\pgfqpoint{0.625236in}{0.568195in}}%
\pgfpathlineto{\pgfqpoint{0.626345in}{0.580133in}}%
\pgfpathlineto{\pgfqpoint{0.626430in}{0.581241in}}%
\pgfpathlineto{\pgfqpoint{0.627539in}{0.592566in}}%
\pgfpathlineto{\pgfqpoint{0.627635in}{0.593674in}}%
\pgfpathlineto{\pgfqpoint{0.628745in}{0.604355in}}%
\pgfpathlineto{\pgfqpoint{0.628857in}{0.605463in}}%
\pgfpathlineto{\pgfqpoint{0.629967in}{0.614515in}}%
\pgfpathlineto{\pgfqpoint{0.630079in}{0.615623in}}%
\pgfpathlineto{\pgfqpoint{0.631188in}{0.623902in}}%
\pgfpathlineto{\pgfqpoint{0.631341in}{0.624955in}}%
\pgfpathlineto{\pgfqpoint{0.632450in}{0.631976in}}%
\pgfpathlineto{\pgfqpoint{0.632600in}{0.633038in}}%
\pgfpathlineto{\pgfqpoint{0.633707in}{0.639305in}}%
\pgfpathlineto{\pgfqpoint{0.633907in}{0.640386in}}%
\pgfpathlineto{\pgfqpoint{0.635016in}{0.646625in}}%
\pgfpathlineto{\pgfqpoint{0.635185in}{0.647714in}}%
\pgfpathlineto{\pgfqpoint{0.636294in}{0.654317in}}%
\pgfpathlineto{\pgfqpoint{0.636489in}{0.655425in}}%
\pgfpathlineto{\pgfqpoint{0.637596in}{0.660966in}}%
\pgfpathlineto{\pgfqpoint{0.637847in}{0.662037in}}%
\pgfpathlineto{\pgfqpoint{0.638951in}{0.667951in}}%
\pgfpathlineto{\pgfqpoint{0.638954in}{0.667951in}}%
\pgfpathlineto{\pgfqpoint{0.639270in}{0.669050in}}%
\pgfpathlineto{\pgfqpoint{0.640377in}{0.675038in}}%
\pgfpathlineto{\pgfqpoint{0.640624in}{0.676136in}}%
\pgfpathlineto{\pgfqpoint{0.641728in}{0.681659in}}%
\pgfpathlineto{\pgfqpoint{0.641970in}{0.682758in}}%
\pgfpathlineto{\pgfqpoint{0.643079in}{0.688103in}}%
\pgfpathlineto{\pgfqpoint{0.643295in}{0.689202in}}%
\pgfpathlineto{\pgfqpoint{0.644404in}{0.694044in}}%
\pgfpathlineto{\pgfqpoint{0.644653in}{0.695134in}}%
\pgfpathlineto{\pgfqpoint{0.645762in}{0.699688in}}%
\pgfpathlineto{\pgfqpoint{0.645999in}{0.700796in}}%
\pgfpathlineto{\pgfqpoint{0.647106in}{0.705732in}}%
\pgfpathlineto{\pgfqpoint{0.647324in}{0.706831in}}%
\pgfpathlineto{\pgfqpoint{0.648429in}{0.711394in}}%
\pgfpathlineto{\pgfqpoint{0.648743in}{0.712483in}}%
\pgfpathlineto{\pgfqpoint{0.649852in}{0.717130in}}%
\pgfpathlineto{\pgfqpoint{0.650174in}{0.718238in}}%
\pgfpathlineto{\pgfqpoint{0.651281in}{0.722038in}}%
\pgfpathlineto{\pgfqpoint{0.651621in}{0.723146in}}%
\pgfpathlineto{\pgfqpoint{0.652730in}{0.727365in}}%
\pgfpathlineto{\pgfqpoint{0.653044in}{0.728464in}}%
\pgfpathlineto{\pgfqpoint{0.654154in}{0.732366in}}%
\pgfpathlineto{\pgfqpoint{0.654426in}{0.733474in}}%
\pgfpathlineto{\pgfqpoint{0.655533in}{0.737562in}}%
\pgfpathlineto{\pgfqpoint{0.655896in}{0.738670in}}%
\pgfpathlineto{\pgfqpoint{0.657005in}{0.742470in}}%
\pgfpathlineto{\pgfqpoint{0.657353in}{0.743569in}}%
\pgfpathlineto{\pgfqpoint{0.658457in}{0.747731in}}%
\pgfpathlineto{\pgfqpoint{0.658771in}{0.748830in}}%
\pgfpathlineto{\pgfqpoint{0.659881in}{0.753235in}}%
\pgfpathlineto{\pgfqpoint{0.660197in}{0.754325in}}%
\pgfpathlineto{\pgfqpoint{0.661307in}{0.758264in}}%
\pgfpathlineto{\pgfqpoint{0.661654in}{0.759372in}}%
\pgfpathlineto{\pgfqpoint{0.662763in}{0.763386in}}%
\pgfpathlineto{\pgfqpoint{0.668504in}{0.781648in}}%
\pgfpathlineto{\pgfqpoint{0.668910in}{0.782719in}}%
\pgfpathlineto{\pgfqpoint{0.670008in}{0.786285in}}%
\pgfpathlineto{\pgfqpoint{0.670015in}{0.786285in}}%
\pgfpathlineto{\pgfqpoint{0.670369in}{0.787384in}}%
\pgfpathlineto{\pgfqpoint{0.671476in}{0.790523in}}%
\pgfpathlineto{\pgfqpoint{0.671809in}{0.791603in}}%
\pgfpathlineto{\pgfqpoint{0.671809in}{0.791621in}}%
\pgfpathlineto{\pgfqpoint{0.672918in}{0.794760in}}%
\pgfpathlineto{\pgfqpoint{0.673315in}{0.795868in}}%
\pgfpathlineto{\pgfqpoint{0.674419in}{0.799044in}}%
\pgfpathlineto{\pgfqpoint{0.674865in}{0.800133in}}%
\pgfpathlineto{\pgfqpoint{0.674865in}{0.800142in}}%
\pgfpathlineto{\pgfqpoint{0.675974in}{0.803104in}}%
\pgfpathlineto{\pgfqpoint{0.676368in}{0.804212in}}%
\pgfpathlineto{\pgfqpoint{0.677478in}{0.807192in}}%
\pgfpathlineto{\pgfqpoint{0.677822in}{0.808254in}}%
\pgfpathlineto{\pgfqpoint{0.677822in}{0.808300in}}%
\pgfpathlineto{\pgfqpoint{0.678927in}{0.811262in}}%
\pgfpathlineto{\pgfqpoint{0.679391in}{0.812360in}}%
\pgfpathlineto{\pgfqpoint{0.680482in}{0.815685in}}%
\pgfpathlineto{\pgfqpoint{0.680949in}{0.816793in}}%
\pgfpathlineto{\pgfqpoint{0.682056in}{0.819857in}}%
\pgfpathlineto{\pgfqpoint{0.682459in}{0.820947in}}%
\pgfpathlineto{\pgfqpoint{0.683566in}{0.823647in}}%
\pgfpathlineto{\pgfqpoint{0.683918in}{0.824737in}}%
\pgfpathlineto{\pgfqpoint{0.683918in}{0.824755in}}%
\pgfpathlineto{\pgfqpoint{0.685020in}{0.827754in}}%
\pgfpathlineto{\pgfqpoint{0.685569in}{0.828862in}}%
\pgfpathlineto{\pgfqpoint{0.686678in}{0.831870in}}%
\pgfpathlineto{\pgfqpoint{0.687025in}{0.832960in}}%
\pgfpathlineto{\pgfqpoint{0.688123in}{0.835949in}}%
\pgfpathlineto{\pgfqpoint{0.688573in}{0.837048in}}%
\pgfpathlineto{\pgfqpoint{0.689673in}{0.839618in}}%
\pgfpathlineto{\pgfqpoint{0.690133in}{0.840717in}}%
\pgfpathlineto{\pgfqpoint{0.691242in}{0.843194in}}%
\pgfpathlineto{\pgfqpoint{0.691681in}{0.844303in}}%
\pgfpathlineto{\pgfqpoint{0.692790in}{0.847096in}}%
\pgfpathlineto{\pgfqpoint{0.693193in}{0.848205in}}%
\pgfpathlineto{\pgfqpoint{0.694298in}{0.850905in}}%
\pgfpathlineto{\pgfqpoint{0.694746in}{0.851985in}}%
\pgfpathlineto{\pgfqpoint{0.694746in}{0.852004in}}%
\pgfpathlineto{\pgfqpoint{0.695832in}{0.854798in}}%
\pgfpathlineto{\pgfqpoint{0.695844in}{0.854798in}}%
\pgfpathlineto{\pgfqpoint{0.696336in}{0.855906in}}%
\pgfpathlineto{\pgfqpoint{0.697438in}{0.858635in}}%
\pgfpathlineto{\pgfqpoint{0.697959in}{0.859743in}}%
\pgfpathlineto{\pgfqpoint{0.699068in}{0.862192in}}%
\pgfpathlineto{\pgfqpoint{0.699577in}{0.863300in}}%
\pgfpathlineto{\pgfqpoint{0.700687in}{0.865991in}}%
\pgfpathlineto{\pgfqpoint{0.701289in}{0.867100in}}%
\pgfpathlineto{\pgfqpoint{0.702399in}{0.869400in}}%
\pgfpathlineto{\pgfqpoint{0.702901in}{0.870508in}}%
\pgfpathlineto{\pgfqpoint{0.704008in}{0.873097in}}%
\pgfpathlineto{\pgfqpoint{0.704535in}{0.874205in}}%
\pgfpathlineto{\pgfqpoint{0.705645in}{0.876543in}}%
\pgfpathlineto{\pgfqpoint{0.706121in}{0.877641in}}%
\pgfpathlineto{\pgfqpoint{0.707223in}{0.879700in}}%
\pgfpathlineto{\pgfqpoint{0.707772in}{0.880808in}}%
\pgfpathlineto{\pgfqpoint{0.708876in}{0.883406in}}%
\pgfpathlineto{\pgfqpoint{0.709324in}{0.884514in}}%
\pgfpathlineto{\pgfqpoint{0.710431in}{0.886712in}}%
\pgfpathlineto{\pgfqpoint{0.710933in}{0.887820in}}%
\pgfpathlineto{\pgfqpoint{0.712031in}{0.890241in}}%
\pgfpathlineto{\pgfqpoint{0.712591in}{0.891350in}}%
\pgfpathlineto{\pgfqpoint{0.713701in}{0.893706in}}%
\pgfpathlineto{\pgfqpoint{0.714254in}{0.894805in}}%
\pgfpathlineto{\pgfqpoint{0.715354in}{0.897142in}}%
\pgfpathlineto{\pgfqpoint{0.715854in}{0.898250in}}%
\pgfpathlineto{\pgfqpoint{0.716963in}{0.900671in}}%
\pgfpathlineto{\pgfqpoint{0.717535in}{0.901752in}}%
\pgfpathlineto{\pgfqpoint{0.718645in}{0.904201in}}%
\pgfpathlineto{\pgfqpoint{0.719186in}{0.905309in}}%
\pgfpathlineto{\pgfqpoint{0.720293in}{0.907991in}}%
\pgfpathlineto{\pgfqpoint{0.720880in}{0.909081in}}%
\pgfpathlineto{\pgfqpoint{0.720880in}{0.909099in}}%
\pgfpathlineto{\pgfqpoint{0.721984in}{0.911232in}}%
\pgfpathlineto{\pgfqpoint{0.722554in}{0.912321in}}%
\pgfpathlineto{\pgfqpoint{0.723664in}{0.914603in}}%
\pgfpathlineto{\pgfqpoint{0.724285in}{0.915702in}}%
\pgfpathlineto{\pgfqpoint{0.724285in}{0.915711in}}%
\pgfpathlineto{\pgfqpoint{0.725394in}{0.917741in}}%
\pgfpathlineto{\pgfqpoint{0.725927in}{0.918840in}}%
\pgfpathlineto{\pgfqpoint{0.727029in}{0.921280in}}%
\pgfpathlineto{\pgfqpoint{0.727704in}{0.922388in}}%
\pgfpathlineto{\pgfqpoint{0.728811in}{0.924716in}}%
\pgfpathlineto{\pgfqpoint{0.729370in}{0.925825in}}%
\pgfpathlineto{\pgfqpoint{0.731635in}{0.930313in}}%
\pgfpathlineto{\pgfqpoint{0.732229in}{0.931412in}}%
\pgfpathlineto{\pgfqpoint{0.733331in}{0.933610in}}%
\pgfpathlineto{\pgfqpoint{0.734070in}{0.934718in}}%
\pgfpathlineto{\pgfqpoint{0.735174in}{0.936888in}}%
\pgfpathlineto{\pgfqpoint{0.735864in}{0.937996in}}%
\pgfpathlineto{\pgfqpoint{0.736973in}{0.940026in}}%
\pgfpathlineto{\pgfqpoint{0.737510in}{0.941135in}}%
\pgfpathlineto{\pgfqpoint{0.738620in}{0.943258in}}%
\pgfpathlineto{\pgfqpoint{0.739309in}{0.944366in}}%
\pgfpathlineto{\pgfqpoint{0.740416in}{0.946350in}}%
\pgfpathlineto{\pgfqpoint{0.741066in}{0.947458in}}%
\pgfpathlineto{\pgfqpoint{0.742152in}{0.949506in}}%
\pgfpathlineto{\pgfqpoint{0.742168in}{0.949506in}}%
\pgfpathlineto{\pgfqpoint{0.742804in}{0.950615in}}%
\pgfpathlineto{\pgfqpoint{0.743911in}{0.952514in}}%
\pgfpathlineto{\pgfqpoint{0.744520in}{0.953623in}}%
\pgfpathlineto{\pgfqpoint{0.745616in}{0.955392in}}%
\pgfpathlineto{\pgfqpoint{0.746256in}{0.956500in}}%
\pgfpathlineto{\pgfqpoint{0.747339in}{0.958586in}}%
\pgfpathlineto{\pgfqpoint{0.747982in}{0.959657in}}%
\pgfpathlineto{\pgfqpoint{0.749075in}{0.961352in}}%
\pgfpathlineto{\pgfqpoint{0.749727in}{0.962451in}}%
\pgfpathlineto{\pgfqpoint{0.750829in}{0.964034in}}%
\pgfpathlineto{\pgfqpoint{0.751505in}{0.965142in}}%
\pgfpathlineto{\pgfqpoint{0.752593in}{0.966856in}}%
\pgfpathlineto{\pgfqpoint{0.753221in}{0.967955in}}%
\pgfpathlineto{\pgfqpoint{0.754328in}{0.969817in}}%
\pgfpathlineto{\pgfqpoint{0.755027in}{0.970925in}}%
\pgfpathlineto{\pgfqpoint{0.756113in}{0.972760in}}%
\pgfpathlineto{\pgfqpoint{0.756805in}{0.973859in}}%
\pgfpathlineto{\pgfqpoint{0.757907in}{0.975489in}}%
\pgfpathlineto{\pgfqpoint{0.758583in}{0.976597in}}%
\pgfpathlineto{\pgfqpoint{0.759678in}{0.978226in}}%
\pgfpathlineto{\pgfqpoint{0.760450in}{0.979335in}}%
\pgfpathlineto{\pgfqpoint{0.761554in}{0.980974in}}%
\pgfpathlineto{\pgfqpoint{0.762251in}{0.982073in}}%
\pgfpathlineto{\pgfqpoint{0.763339in}{0.983982in}}%
\pgfpathlineto{\pgfqpoint{0.764024in}{0.985090in}}%
\pgfpathlineto{\pgfqpoint{0.765131in}{0.987241in}}%
\pgfpathlineto{\pgfqpoint{0.765724in}{0.988340in}}%
\pgfpathlineto{\pgfqpoint{0.766829in}{0.990184in}}%
\pgfpathlineto{\pgfqpoint{0.767598in}{0.991292in}}%
\pgfpathlineto{\pgfqpoint{0.769524in}{0.994402in}}%
\pgfpathlineto{\pgfqpoint{0.770375in}{0.995501in}}%
\pgfpathlineto{\pgfqpoint{0.771480in}{0.997308in}}%
\pgfpathlineto{\pgfqpoint{0.772075in}{0.998397in}}%
\pgfpathlineto{\pgfqpoint{0.773178in}{1.000325in}}%
\pgfpathlineto{\pgfqpoint{0.773900in}{1.001433in}}%
\pgfpathlineto{\pgfqpoint{0.775000in}{1.003203in}}%
\pgfpathlineto{\pgfqpoint{0.775009in}{1.003203in}}%
\pgfpathlineto{\pgfqpoint{0.775793in}{1.004302in}}%
\pgfpathlineto{\pgfqpoint{0.776897in}{1.006276in}}%
\pgfpathlineto{\pgfqpoint{0.777516in}{1.007384in}}%
\pgfpathlineto{\pgfqpoint{0.778621in}{1.009284in}}%
\pgfpathlineto{\pgfqpoint{0.779327in}{1.010383in}}%
\pgfpathlineto{\pgfqpoint{0.780436in}{1.011956in}}%
\pgfpathlineto{\pgfqpoint{0.781224in}{1.013055in}}%
\pgfpathlineto{\pgfqpoint{0.782334in}{1.014611in}}%
\pgfpathlineto{\pgfqpoint{0.783023in}{1.015719in}}%
\pgfpathlineto{\pgfqpoint{0.784130in}{1.017432in}}%
\pgfpathlineto{\pgfqpoint{0.785108in}{1.018540in}}%
\pgfpathlineto{\pgfqpoint{0.786217in}{1.020133in}}%
\pgfpathlineto{\pgfqpoint{0.787008in}{1.021241in}}%
\pgfpathlineto{\pgfqpoint{0.788101in}{1.023122in}}%
\pgfpathlineto{\pgfqpoint{0.788793in}{1.024230in}}%
\pgfpathlineto{\pgfqpoint{0.789897in}{1.025842in}}%
\pgfpathlineto{\pgfqpoint{0.790519in}{1.026950in}}%
\pgfpathlineto{\pgfqpoint{0.791619in}{1.028663in}}%
\pgfpathlineto{\pgfqpoint{0.792477in}{1.029771in}}%
\pgfpathlineto{\pgfqpoint{0.793584in}{1.031224in}}%
\pgfpathlineto{\pgfqpoint{0.794367in}{1.032332in}}%
\pgfpathlineto{\pgfqpoint{0.795453in}{1.034037in}}%
\pgfpathlineto{\pgfqpoint{0.796131in}{1.035145in}}%
\pgfpathlineto{\pgfqpoint{0.797236in}{1.036765in}}%
\pgfpathlineto{\pgfqpoint{0.797869in}{1.037864in}}%
\pgfpathlineto{\pgfqpoint{0.798976in}{1.039484in}}%
\pgfpathlineto{\pgfqpoint{0.799783in}{1.040555in}}%
\pgfpathlineto{\pgfqpoint{0.800890in}{1.042008in}}%
\pgfpathlineto{\pgfqpoint{0.801626in}{1.043116in}}%
\pgfpathlineto{\pgfqpoint{0.802728in}{1.044765in}}%
\pgfpathlineto{\pgfqpoint{0.803570in}{1.045873in}}%
\pgfpathlineto{\pgfqpoint{0.804670in}{1.047605in}}%
\pgfpathlineto{\pgfqpoint{0.805472in}{1.048713in}}%
\pgfpathlineto{\pgfqpoint{0.806570in}{1.050278in}}%
\pgfpathlineto{\pgfqpoint{0.807384in}{1.051386in}}%
\pgfpathlineto{\pgfqpoint{0.808486in}{1.052894in}}%
\pgfpathlineto{\pgfqpoint{0.809321in}{1.054003in}}%
\pgfpathlineto{\pgfqpoint{0.810428in}{1.055614in}}%
\pgfpathlineto{\pgfqpoint{0.811120in}{1.056722in}}%
\pgfpathlineto{\pgfqpoint{0.812227in}{1.057923in}}%
\pgfpathlineto{\pgfqpoint{0.813111in}{1.059031in}}%
\pgfpathlineto{\pgfqpoint{0.814213in}{1.060428in}}%
\pgfpathlineto{\pgfqpoint{0.814982in}{1.061537in}}%
\pgfpathlineto{\pgfqpoint{0.816089in}{1.063287in}}%
\pgfpathlineto{\pgfqpoint{0.816920in}{1.064377in}}%
\pgfpathlineto{\pgfqpoint{0.818027in}{1.066128in}}%
\pgfpathlineto{\pgfqpoint{0.818794in}{1.067236in}}%
\pgfpathlineto{\pgfqpoint{0.819894in}{1.068623in}}%
\pgfpathlineto{\pgfqpoint{0.819903in}{1.068623in}}%
\pgfpathlineto{\pgfqpoint{0.820630in}{1.069713in}}%
\pgfpathlineto{\pgfqpoint{0.821735in}{1.071156in}}%
\pgfpathlineto{\pgfqpoint{0.822588in}{1.072265in}}%
\pgfpathlineto{\pgfqpoint{0.823695in}{1.073531in}}%
\pgfpathlineto{\pgfqpoint{0.824589in}{1.074639in}}%
\pgfpathlineto{\pgfqpoint{0.825696in}{1.076083in}}%
\pgfpathlineto{\pgfqpoint{0.826730in}{1.077182in}}%
\pgfpathlineto{\pgfqpoint{0.827835in}{1.078560in}}%
\pgfpathlineto{\pgfqpoint{0.828749in}{1.079668in}}%
\pgfpathlineto{\pgfqpoint{0.829856in}{1.080869in}}%
\pgfpathlineto{\pgfqpoint{0.830794in}{1.081968in}}%
\pgfpathlineto{\pgfqpoint{0.831904in}{1.083561in}}%
\pgfpathlineto{\pgfqpoint{0.832905in}{1.084669in}}%
\pgfpathlineto{\pgfqpoint{0.834549in}{1.086662in}}%
\pgfpathlineto{\pgfqpoint{0.835516in}{1.087770in}}%
\pgfpathlineto{\pgfqpoint{0.836620in}{1.089241in}}%
\pgfpathlineto{\pgfqpoint{0.837483in}{1.090350in}}%
\pgfpathlineto{\pgfqpoint{0.838588in}{1.091812in}}%
\pgfpathlineto{\pgfqpoint{0.839364in}{1.092911in}}%
\pgfpathlineto{\pgfqpoint{0.840469in}{1.094428in}}%
\pgfpathlineto{\pgfqpoint{0.841489in}{1.095537in}}%
\pgfpathlineto{\pgfqpoint{0.842591in}{1.096980in}}%
\pgfpathlineto{\pgfqpoint{0.843450in}{1.098079in}}%
\pgfpathlineto{\pgfqpoint{0.844547in}{1.099737in}}%
\pgfpathlineto{\pgfqpoint{0.844557in}{1.099737in}}%
\pgfpathlineto{\pgfqpoint{0.845560in}{1.100845in}}%
\pgfpathlineto{\pgfqpoint{0.846642in}{1.102353in}}%
\pgfpathlineto{\pgfqpoint{0.847575in}{1.103462in}}%
\pgfpathlineto{\pgfqpoint{0.848675in}{1.104710in}}%
\pgfpathlineto{\pgfqpoint{0.849465in}{1.105818in}}%
\pgfpathlineto{\pgfqpoint{0.850572in}{1.107038in}}%
\pgfpathlineto{\pgfqpoint{0.851635in}{1.108146in}}%
\pgfpathlineto{\pgfqpoint{0.852721in}{1.109310in}}%
\pgfpathlineto{\pgfqpoint{0.853755in}{1.110409in}}%
\pgfpathlineto{\pgfqpoint{0.854864in}{1.111973in}}%
\pgfpathlineto{\pgfqpoint{0.855741in}{1.113072in}}%
\pgfpathlineto{\pgfqpoint{0.856848in}{1.114618in}}%
\pgfpathlineto{\pgfqpoint{0.857937in}{1.115726in}}%
\pgfpathlineto{\pgfqpoint{0.859041in}{1.117151in}}%
\pgfpathlineto{\pgfqpoint{0.859044in}{1.117151in}}%
\pgfpathlineto{\pgfqpoint{0.859942in}{1.118259in}}%
\pgfpathlineto{\pgfqpoint{0.861044in}{1.119591in}}%
\pgfpathlineto{\pgfqpoint{0.862095in}{1.120699in}}%
\pgfpathlineto{\pgfqpoint{0.863199in}{1.122022in}}%
\pgfpathlineto{\pgfqpoint{0.863202in}{1.122022in}}%
\pgfpathlineto{\pgfqpoint{0.864114in}{1.123130in}}%
\pgfpathlineto{\pgfqpoint{0.865221in}{1.124415in}}%
\pgfpathlineto{\pgfqpoint{0.866035in}{1.125523in}}%
\pgfpathlineto{\pgfqpoint{0.867142in}{1.127013in}}%
\pgfpathlineto{\pgfqpoint{0.867944in}{1.128121in}}%
\pgfpathlineto{\pgfqpoint{0.869046in}{1.129369in}}%
\pgfpathlineto{\pgfqpoint{0.869973in}{1.130477in}}%
\pgfpathlineto{\pgfqpoint{0.871077in}{1.132051in}}%
\pgfpathlineto{\pgfqpoint{0.872030in}{1.133159in}}%
\pgfpathlineto{\pgfqpoint{0.873122in}{1.134640in}}%
\pgfpathlineto{\pgfqpoint{0.873978in}{1.135739in}}%
\pgfpathlineto{\pgfqpoint{0.875081in}{1.137015in}}%
\pgfpathlineto{\pgfqpoint{0.876056in}{1.138123in}}%
\pgfpathlineto{\pgfqpoint{0.877163in}{1.139380in}}%
\pgfpathlineto{\pgfqpoint{0.878242in}{1.140488in}}%
\pgfpathlineto{\pgfqpoint{0.879345in}{1.141774in}}%
\pgfpathlineto{\pgfqpoint{0.880264in}{1.142882in}}%
\pgfpathlineto{\pgfqpoint{0.881347in}{1.144018in}}%
\pgfpathlineto{\pgfqpoint{0.881373in}{1.144018in}}%
\pgfpathlineto{\pgfqpoint{0.882379in}{1.145098in}}%
\pgfpathlineto{\pgfqpoint{0.883449in}{1.146458in}}%
\pgfpathlineto{\pgfqpoint{0.883475in}{1.146458in}}%
\pgfpathlineto{\pgfqpoint{0.884631in}{1.147566in}}%
\pgfpathlineto{\pgfqpoint{0.885731in}{1.148907in}}%
\pgfpathlineto{\pgfqpoint{0.886767in}{1.150015in}}%
\pgfpathlineto{\pgfqpoint{0.887877in}{1.151179in}}%
\pgfpathlineto{\pgfqpoint{0.888759in}{1.152287in}}%
\pgfpathlineto{\pgfqpoint{0.889868in}{1.153433in}}%
\pgfpathlineto{\pgfqpoint{0.890909in}{1.154541in}}%
\pgfpathlineto{\pgfqpoint{0.892469in}{1.156282in}}%
\pgfpathlineto{\pgfqpoint{0.893707in}{1.157391in}}%
\pgfpathlineto{\pgfqpoint{0.894816in}{1.158694in}}%
\pgfpathlineto{\pgfqpoint{0.895837in}{1.159793in}}%
\pgfpathlineto{\pgfqpoint{0.896920in}{1.160948in}}%
\pgfpathlineto{\pgfqpoint{0.897825in}{1.162056in}}%
\pgfpathlineto{\pgfqpoint{0.898935in}{1.163174in}}%
\pgfpathlineto{\pgfqpoint{0.899810in}{1.164273in}}%
\pgfpathlineto{\pgfqpoint{0.900917in}{1.165344in}}%
\pgfpathlineto{\pgfqpoint{0.902085in}{1.166452in}}%
\pgfpathlineto{\pgfqpoint{0.903189in}{1.167560in}}%
\pgfpathlineto{\pgfqpoint{0.904376in}{1.168668in}}%
\pgfpathlineto{\pgfqpoint{0.905476in}{1.169907in}}%
\pgfpathlineto{\pgfqpoint{0.906423in}{1.171006in}}%
\pgfpathlineto{\pgfqpoint{0.907533in}{1.172179in}}%
\pgfpathlineto{\pgfqpoint{0.908649in}{1.173287in}}%
\pgfpathlineto{\pgfqpoint{0.909758in}{1.174433in}}%
\pgfpathlineto{\pgfqpoint{0.910626in}{1.175541in}}%
\pgfpathlineto{\pgfqpoint{0.911735in}{1.176761in}}%
\pgfpathlineto{\pgfqpoint{0.912859in}{1.177869in}}%
\pgfpathlineto{\pgfqpoint{0.913947in}{1.178893in}}%
\pgfpathlineto{\pgfqpoint{0.915232in}{1.180002in}}%
\pgfpathlineto{\pgfqpoint{0.916339in}{1.181277in}}%
\pgfpathlineto{\pgfqpoint{0.917432in}{1.182386in}}%
\pgfpathlineto{\pgfqpoint{0.918539in}{1.183447in}}%
\pgfpathlineto{\pgfqpoint{0.919677in}{1.184555in}}%
\pgfpathlineto{\pgfqpoint{0.920784in}{1.185654in}}%
\pgfpathlineto{\pgfqpoint{0.921928in}{1.186753in}}%
\pgfpathlineto{\pgfqpoint{0.923037in}{1.187871in}}%
\pgfpathlineto{\pgfqpoint{0.923959in}{1.188979in}}%
\pgfpathlineto{\pgfqpoint{0.925066in}{1.190087in}}%
\pgfpathlineto{\pgfqpoint{0.925985in}{1.191186in}}%
\pgfpathlineto{\pgfqpoint{0.925985in}{1.191195in}}%
\pgfpathlineto{\pgfqpoint{0.927547in}{1.192862in}}%
\pgfpathlineto{\pgfqpoint{0.928720in}{1.193952in}}%
\pgfpathlineto{\pgfqpoint{0.928720in}{1.193961in}}%
\pgfpathlineto{\pgfqpoint{0.929820in}{1.194920in}}%
\pgfpathlineto{\pgfqpoint{0.931072in}{1.196028in}}%
\pgfpathlineto{\pgfqpoint{0.932175in}{1.197286in}}%
\pgfpathlineto{\pgfqpoint{0.933354in}{1.198394in}}%
\pgfpathlineto{\pgfqpoint{0.934447in}{1.199334in}}%
\pgfpathlineto{\pgfqpoint{0.935786in}{1.200443in}}%
\pgfpathlineto{\pgfqpoint{0.936884in}{1.201514in}}%
\pgfpathlineto{\pgfqpoint{0.937914in}{1.202622in}}%
\pgfpathlineto{\pgfqpoint{0.939021in}{1.203665in}}%
\pgfpathlineto{\pgfqpoint{0.940226in}{1.204773in}}%
\pgfpathlineto{\pgfqpoint{0.941333in}{1.205779in}}%
\pgfpathlineto{\pgfqpoint{0.942438in}{1.206887in}}%
\pgfpathlineto{\pgfqpoint{0.943542in}{1.208116in}}%
\pgfpathlineto{\pgfqpoint{0.943547in}{1.208116in}}%
\pgfpathlineto{\pgfqpoint{0.944811in}{1.209224in}}%
\pgfpathlineto{\pgfqpoint{0.945921in}{1.210277in}}%
\pgfpathlineto{\pgfqpoint{0.947295in}{1.211385in}}%
\pgfpathlineto{\pgfqpoint{0.948397in}{1.212540in}}%
\pgfpathlineto{\pgfqpoint{0.949377in}{1.213639in}}%
\pgfpathlineto{\pgfqpoint{0.950480in}{1.214737in}}%
\pgfpathlineto{\pgfqpoint{0.950484in}{1.214737in}}%
\pgfpathlineto{\pgfqpoint{0.951648in}{1.215846in}}%
\pgfpathlineto{\pgfqpoint{0.952748in}{1.216907in}}%
\pgfpathlineto{\pgfqpoint{0.954040in}{1.218006in}}%
\pgfpathlineto{\pgfqpoint{0.955149in}{1.219030in}}%
\pgfpathlineto{\pgfqpoint{0.956336in}{1.220139in}}%
\pgfpathlineto{\pgfqpoint{0.957401in}{1.221033in}}%
\pgfpathlineto{\pgfqpoint{0.957445in}{1.221033in}}%
\pgfpathlineto{\pgfqpoint{0.958714in}{1.222141in}}%
\pgfpathlineto{\pgfqpoint{0.959800in}{1.223109in}}%
\pgfpathlineto{\pgfqpoint{0.961001in}{1.224218in}}%
\pgfpathlineto{\pgfqpoint{0.962101in}{1.225158in}}%
\pgfpathlineto{\pgfqpoint{0.963306in}{1.226266in}}%
\pgfpathlineto{\pgfqpoint{0.964397in}{1.227188in}}%
\pgfpathlineto{\pgfqpoint{0.964409in}{1.227188in}}%
\pgfpathlineto{\pgfqpoint{0.965637in}{1.228296in}}%
\pgfpathlineto{\pgfqpoint{0.966742in}{1.229433in}}%
\pgfpathlineto{\pgfqpoint{0.966747in}{1.229433in}}%
\pgfpathlineto{\pgfqpoint{0.967957in}{1.230541in}}%
\pgfpathlineto{\pgfqpoint{0.969064in}{1.231537in}}%
\pgfpathlineto{\pgfqpoint{0.970086in}{1.232645in}}%
\pgfpathlineto{\pgfqpoint{0.971144in}{1.233633in}}%
\pgfpathlineto{\pgfqpoint{0.971172in}{1.233633in}}%
\pgfpathlineto{\pgfqpoint{0.972699in}{1.234731in}}%
\pgfpathlineto{\pgfqpoint{0.973785in}{1.235625in}}%
\pgfpathlineto{\pgfqpoint{0.973790in}{1.235625in}}%
\pgfpathlineto{\pgfqpoint{0.975176in}{1.236734in}}%
\pgfpathlineto{\pgfqpoint{0.976285in}{1.237600in}}%
\pgfpathlineto{\pgfqpoint{0.977580in}{1.238708in}}%
\pgfpathlineto{\pgfqpoint{0.978675in}{1.239760in}}%
\pgfpathlineto{\pgfqpoint{0.980099in}{1.240859in}}%
\pgfpathlineto{\pgfqpoint{0.981208in}{1.241837in}}%
\pgfpathlineto{\pgfqpoint{0.982486in}{1.242945in}}%
\pgfpathlineto{\pgfqpoint{0.983565in}{1.244007in}}%
\pgfpathlineto{\pgfqpoint{0.983595in}{1.244007in}}%
\pgfpathlineto{\pgfqpoint{0.984796in}{1.245115in}}%
\pgfpathlineto{\pgfqpoint{0.985896in}{1.245981in}}%
\pgfpathlineto{\pgfqpoint{0.987210in}{1.247089in}}%
\pgfpathlineto{\pgfqpoint{0.988302in}{1.248011in}}%
\pgfpathlineto{\pgfqpoint{0.988319in}{1.248011in}}%
\pgfpathlineto{\pgfqpoint{0.989499in}{1.249119in}}%
\pgfpathlineto{\pgfqpoint{0.990573in}{1.249948in}}%
\pgfpathlineto{\pgfqpoint{0.991722in}{1.251047in}}%
\pgfpathlineto{\pgfqpoint{0.991722in}{1.251056in}}%
\pgfpathlineto{\pgfqpoint{0.993232in}{1.252518in}}%
\pgfpathlineto{\pgfqpoint{0.994536in}{1.253627in}}%
\pgfpathlineto{\pgfqpoint{0.995610in}{1.254595in}}%
\pgfpathlineto{\pgfqpoint{0.996713in}{1.255703in}}%
\pgfpathlineto{\pgfqpoint{0.997817in}{1.256746in}}%
\pgfpathlineto{\pgfqpoint{0.999142in}{1.257855in}}%
\pgfpathlineto{\pgfqpoint{1.000249in}{1.258730in}}%
\pgfpathlineto{\pgfqpoint{1.001642in}{1.259838in}}%
\pgfpathlineto{\pgfqpoint{1.002752in}{1.260825in}}%
\pgfpathlineto{\pgfqpoint{1.004131in}{1.261933in}}%
\pgfpathlineto{\pgfqpoint{1.005235in}{1.263032in}}%
\pgfpathlineto{\pgfqpoint{1.006647in}{1.264141in}}%
\pgfpathlineto{\pgfqpoint{1.007719in}{1.264988in}}%
\pgfpathlineto{\pgfqpoint{1.009051in}{1.266096in}}%
\pgfpathlineto{\pgfqpoint{1.010147in}{1.266906in}}%
\pgfpathlineto{\pgfqpoint{1.011737in}{1.268005in}}%
\pgfpathlineto{\pgfqpoint{1.012841in}{1.268806in}}%
\pgfpathlineto{\pgfqpoint{1.014185in}{1.269905in}}%
\pgfpathlineto{\pgfqpoint{1.015285in}{1.270715in}}%
\pgfpathlineto{\pgfqpoint{1.016568in}{1.271823in}}%
\pgfpathlineto{\pgfqpoint{1.017663in}{1.272755in}}%
\pgfpathlineto{\pgfqpoint{1.018859in}{1.273863in}}%
\pgfpathlineto{\pgfqpoint{1.019964in}{1.274599in}}%
\pgfpathlineto{\pgfqpoint{1.019969in}{1.274599in}}%
\pgfpathlineto{\pgfqpoint{1.021357in}{1.275623in}}%
\pgfpathlineto{\pgfqpoint{1.022647in}{1.276731in}}%
\pgfpathlineto{\pgfqpoint{1.023756in}{1.277569in}}%
\pgfpathlineto{\pgfqpoint{1.025013in}{1.278668in}}%
\pgfpathlineto{\pgfqpoint{1.026118in}{1.279711in}}%
\pgfpathlineto{\pgfqpoint{1.027455in}{1.280810in}}%
\pgfpathlineto{\pgfqpoint{1.028564in}{1.281667in}}%
\pgfpathlineto{\pgfqpoint{1.029845in}{1.282775in}}%
\pgfpathlineto{\pgfqpoint{1.030933in}{1.283548in}}%
\pgfpathlineto{\pgfqpoint{1.032251in}{1.284656in}}%
\pgfpathlineto{\pgfqpoint{1.033309in}{1.285466in}}%
\pgfpathlineto{\pgfqpoint{1.034838in}{1.286574in}}%
\pgfpathlineto{\pgfqpoint{1.035942in}{1.287319in}}%
\pgfpathlineto{\pgfqpoint{1.037387in}{1.288428in}}%
\pgfpathlineto{\pgfqpoint{1.038457in}{1.289387in}}%
\pgfpathlineto{\pgfqpoint{1.038494in}{1.289387in}}%
\pgfpathlineto{\pgfqpoint{1.039669in}{1.290495in}}%
\pgfpathlineto{\pgfqpoint{1.040748in}{1.291352in}}%
\pgfpathlineto{\pgfqpoint{1.040769in}{1.291352in}}%
\pgfpathlineto{\pgfqpoint{1.042040in}{1.292460in}}%
\pgfpathlineto{\pgfqpoint{1.043145in}{1.293289in}}%
\pgfpathlineto{\pgfqpoint{1.044697in}{1.294397in}}%
\pgfpathlineto{\pgfqpoint{1.045807in}{1.295244in}}%
\pgfpathlineto{\pgfqpoint{1.047641in}{1.296353in}}%
\pgfpathlineto{\pgfqpoint{1.048724in}{1.297181in}}%
\pgfpathlineto{\pgfqpoint{1.050124in}{1.298280in}}%
\pgfpathlineto{\pgfqpoint{1.051180in}{1.299156in}}%
\pgfpathlineto{\pgfqpoint{1.052784in}{1.300264in}}%
\pgfpathlineto{\pgfqpoint{1.054243in}{1.301540in}}%
\pgfpathlineto{\pgfqpoint{1.055681in}{1.302639in}}%
\pgfpathlineto{\pgfqpoint{1.056790in}{1.303579in}}%
\pgfpathlineto{\pgfqpoint{1.058242in}{1.304678in}}%
\pgfpathlineto{\pgfqpoint{1.059339in}{1.305479in}}%
\pgfpathlineto{\pgfqpoint{1.060889in}{1.306587in}}%
\pgfpathlineto{\pgfqpoint{1.061987in}{1.307537in}}%
\pgfpathlineto{\pgfqpoint{1.063493in}{1.308645in}}%
\pgfpathlineto{\pgfqpoint{1.064600in}{1.309576in}}%
\pgfpathlineto{\pgfqpoint{1.066251in}{1.310685in}}%
\pgfpathlineto{\pgfqpoint{1.067353in}{1.311458in}}%
\pgfpathlineto{\pgfqpoint{1.069079in}{1.312566in}}%
\pgfpathlineto{\pgfqpoint{1.070156in}{1.313441in}}%
\pgfpathlineto{\pgfqpoint{1.070181in}{1.313441in}}%
\pgfpathlineto{\pgfqpoint{1.071894in}{1.314549in}}%
\pgfpathlineto{\pgfqpoint{1.073001in}{1.315462in}}%
\pgfpathlineto{\pgfqpoint{1.074586in}{1.316570in}}%
\pgfpathlineto{\pgfqpoint{1.075669in}{1.317390in}}%
\pgfpathlineto{\pgfqpoint{1.076924in}{1.318498in}}%
\pgfpathlineto{\pgfqpoint{1.078031in}{1.319364in}}%
\pgfpathlineto{\pgfqpoint{1.079338in}{1.320472in}}%
\pgfpathlineto{\pgfqpoint{1.080445in}{1.321096in}}%
\pgfpathlineto{\pgfqpoint{1.082110in}{1.322204in}}%
\pgfpathlineto{\pgfqpoint{1.083181in}{1.322847in}}%
\pgfpathlineto{\pgfqpoint{1.084711in}{1.323955in}}%
\pgfpathlineto{\pgfqpoint{1.085818in}{1.324905in}}%
\pgfpathlineto{\pgfqpoint{1.087584in}{1.326004in}}%
\pgfpathlineto{\pgfqpoint{1.088653in}{1.326777in}}%
\pgfpathlineto{\pgfqpoint{1.088665in}{1.326777in}}%
\pgfpathlineto{\pgfqpoint{1.090396in}{1.327885in}}%
\pgfpathlineto{\pgfqpoint{1.091503in}{1.328583in}}%
\pgfpathlineto{\pgfqpoint{1.093130in}{1.329692in}}%
\pgfpathlineto{\pgfqpoint{1.094226in}{1.330409in}}%
\pgfpathlineto{\pgfqpoint{1.096055in}{1.331517in}}%
\pgfpathlineto{\pgfqpoint{1.097143in}{1.332197in}}%
\pgfpathlineto{\pgfqpoint{1.098611in}{1.333305in}}%
\pgfpathlineto{\pgfqpoint{1.099709in}{1.334115in}}%
\pgfpathlineto{\pgfqpoint{1.099716in}{1.334115in}}%
\pgfpathlineto{\pgfqpoint{1.101320in}{1.335214in}}%
\pgfpathlineto{\pgfqpoint{1.101320in}{1.335223in}}%
\pgfpathlineto{\pgfqpoint{1.102645in}{1.336043in}}%
\pgfpathlineto{\pgfqpoint{1.104378in}{1.337151in}}%
\pgfpathlineto{\pgfqpoint{1.105488in}{1.337887in}}%
\pgfpathlineto{\pgfqpoint{1.107230in}{1.338995in}}%
\pgfpathlineto{\pgfqpoint{1.108337in}{1.339591in}}%
\pgfpathlineto{\pgfqpoint{1.110143in}{1.340690in}}%
\pgfpathlineto{\pgfqpoint{1.111234in}{1.341574in}}%
\pgfpathlineto{\pgfqpoint{1.111250in}{1.341574in}}%
\pgfpathlineto{\pgfqpoint{1.113152in}{1.342683in}}%
\pgfpathlineto{\pgfqpoint{1.114210in}{1.343307in}}%
\pgfpathlineto{\pgfqpoint{1.116100in}{1.344396in}}%
\pgfpathlineto{\pgfqpoint{1.117193in}{1.345020in}}%
\pgfpathlineto{\pgfqpoint{1.117200in}{1.345020in}}%
\pgfpathlineto{\pgfqpoint{1.118755in}{1.346128in}}%
\pgfpathlineto{\pgfqpoint{1.119855in}{1.346799in}}%
\pgfpathlineto{\pgfqpoint{1.121623in}{1.347907in}}%
\pgfpathlineto{\pgfqpoint{1.122723in}{1.348652in}}%
\pgfpathlineto{\pgfqpoint{1.124421in}{1.349760in}}%
\pgfpathlineto{\pgfqpoint{1.125524in}{1.350775in}}%
\pgfpathlineto{\pgfqpoint{1.127285in}{1.351883in}}%
\pgfpathlineto{\pgfqpoint{1.128394in}{1.352582in}}%
\pgfpathlineto{\pgfqpoint{1.130155in}{1.353690in}}%
\pgfpathlineto{\pgfqpoint{1.131223in}{1.354314in}}%
\pgfpathlineto{\pgfqpoint{1.131230in}{1.354314in}}%
\pgfpathlineto{\pgfqpoint{1.132982in}{1.355422in}}%
\pgfpathlineto{\pgfqpoint{1.134042in}{1.356214in}}%
\pgfpathlineto{\pgfqpoint{1.135934in}{1.357322in}}%
\pgfpathlineto{\pgfqpoint{1.137034in}{1.358002in}}%
\pgfpathlineto{\pgfqpoint{1.138908in}{1.359101in}}%
\pgfpathlineto{\pgfqpoint{1.140015in}{1.359883in}}%
\pgfpathlineto{\pgfqpoint{1.141619in}{1.360991in}}%
\pgfpathlineto{\pgfqpoint{1.142722in}{1.361690in}}%
\pgfpathlineto{\pgfqpoint{1.144624in}{1.362798in}}%
\pgfpathlineto{\pgfqpoint{1.145728in}{1.363422in}}%
\pgfpathlineto{\pgfqpoint{1.147464in}{1.364530in}}%
\pgfpathlineto{\pgfqpoint{1.148561in}{1.365219in}}%
\pgfpathlineto{\pgfqpoint{1.150646in}{1.366327in}}%
\pgfpathlineto{\pgfqpoint{1.151753in}{1.366998in}}%
\pgfpathlineto{\pgfqpoint{1.153550in}{1.368106in}}%
\pgfpathlineto{\pgfqpoint{1.154650in}{1.368711in}}%
\pgfpathlineto{\pgfqpoint{1.156357in}{1.369810in}}%
\pgfpathlineto{\pgfqpoint{1.157427in}{1.370462in}}%
\pgfpathlineto{\pgfqpoint{1.159317in}{1.371570in}}%
\pgfpathlineto{\pgfqpoint{1.160412in}{1.372306in}}%
\pgfpathlineto{\pgfqpoint{1.162525in}{1.373414in}}%
\pgfpathlineto{\pgfqpoint{1.163628in}{1.373991in}}%
\pgfpathlineto{\pgfqpoint{1.165532in}{1.375100in}}%
\pgfpathlineto{\pgfqpoint{1.166630in}{1.375770in}}%
\pgfpathlineto{\pgfqpoint{1.168370in}{1.376878in}}%
\pgfpathlineto{\pgfqpoint{1.169449in}{1.377474in}}%
\pgfpathlineto{\pgfqpoint{1.171266in}{1.378583in}}%
\pgfpathlineto{\pgfqpoint{1.172343in}{1.379328in}}%
\pgfpathlineto{\pgfqpoint{1.174517in}{1.380436in}}%
\pgfpathlineto{\pgfqpoint{1.175570in}{1.380985in}}%
\pgfpathlineto{\pgfqpoint{1.177493in}{1.382084in}}%
\pgfpathlineto{\pgfqpoint{1.178581in}{1.382745in}}%
\pgfpathlineto{\pgfqpoint{1.179953in}{1.383853in}}%
\pgfpathlineto{\pgfqpoint{1.181063in}{1.384570in}}%
\pgfpathlineto{\pgfqpoint{1.182646in}{1.385669in}}%
\pgfpathlineto{\pgfqpoint{1.183717in}{1.386256in}}%
\pgfpathlineto{\pgfqpoint{1.183750in}{1.386256in}}%
\pgfpathlineto{\pgfqpoint{1.185760in}{1.387364in}}%
\pgfpathlineto{\pgfqpoint{1.186841in}{1.388156in}}%
\pgfpathlineto{\pgfqpoint{1.186858in}{1.388156in}}%
\pgfpathlineto{\pgfqpoint{1.189130in}{1.389264in}}%
\pgfpathlineto{\pgfqpoint{1.190237in}{1.389841in}}%
\pgfpathlineto{\pgfqpoint{1.192353in}{1.390950in}}%
\pgfpathlineto{\pgfqpoint{1.193427in}{1.391629in}}%
\pgfpathlineto{\pgfqpoint{1.195542in}{1.392738in}}%
\pgfpathlineto{\pgfqpoint{1.196882in}{1.393529in}}%
\pgfpathlineto{\pgfqpoint{1.198950in}{1.394609in}}%
\pgfpathlineto{\pgfqpoint{1.198950in}{1.394637in}}%
\pgfpathlineto{\pgfqpoint{1.200057in}{1.395317in}}%
\pgfpathlineto{\pgfqpoint{1.201980in}{1.396425in}}%
\pgfpathlineto{\pgfqpoint{1.203073in}{1.396854in}}%
\pgfpathlineto{\pgfqpoint{1.205315in}{1.397962in}}%
\pgfpathlineto{\pgfqpoint{1.206389in}{1.398502in}}%
\pgfpathlineto{\pgfqpoint{1.206406in}{1.398502in}}%
\pgfpathlineto{\pgfqpoint{1.208336in}{1.399601in}}%
\pgfpathlineto{\pgfqpoint{1.209429in}{1.400281in}}%
\pgfpathlineto{\pgfqpoint{1.209445in}{1.400281in}}%
\pgfpathlineto{\pgfqpoint{1.211763in}{1.401389in}}%
\pgfpathlineto{\pgfqpoint{1.212872in}{1.402181in}}%
\pgfpathlineto{\pgfqpoint{1.215442in}{1.403289in}}%
\pgfpathlineto{\pgfqpoint{1.216538in}{1.403894in}}%
\pgfpathlineto{\pgfqpoint{1.218913in}{1.405002in}}%
\pgfpathlineto{\pgfqpoint{1.219962in}{1.405524in}}%
\pgfpathlineto{\pgfqpoint{1.219976in}{1.405524in}}%
\pgfpathlineto{\pgfqpoint{1.222068in}{1.406632in}}%
\pgfpathlineto{\pgfqpoint{1.223158in}{1.407237in}}%
\pgfpathlineto{\pgfqpoint{1.225084in}{1.408336in}}%
\pgfpathlineto{\pgfqpoint{1.226186in}{1.408960in}}%
\pgfpathlineto{\pgfqpoint{1.228487in}{1.410068in}}%
\pgfpathlineto{\pgfqpoint{1.229561in}{1.410729in}}%
\pgfpathlineto{\pgfqpoint{1.229591in}{1.410729in}}%
\pgfpathlineto{\pgfqpoint{1.232009in}{1.411838in}}%
\pgfpathlineto{\pgfqpoint{1.233114in}{1.412341in}}%
\pgfpathlineto{\pgfqpoint{1.235009in}{1.413439in}}%
\pgfpathlineto{\pgfqpoint{1.236111in}{1.413905in}}%
\pgfpathlineto{\pgfqpoint{1.238546in}{1.415013in}}%
\pgfpathlineto{\pgfqpoint{1.239653in}{1.415507in}}%
\pgfpathlineto{\pgfqpoint{1.241956in}{1.416615in}}%
\pgfpathlineto{\pgfqpoint{1.243042in}{1.417202in}}%
\pgfpathlineto{\pgfqpoint{1.245040in}{1.418310in}}%
\pgfpathlineto{\pgfqpoint{1.246142in}{1.418803in}}%
\pgfpathlineto{\pgfqpoint{1.248366in}{1.419912in}}%
\pgfpathlineto{\pgfqpoint{1.249466in}{1.420396in}}%
\pgfpathlineto{\pgfqpoint{1.251410in}{1.421504in}}%
\pgfpathlineto{\pgfqpoint{1.252512in}{1.422119in}}%
\pgfpathlineto{\pgfqpoint{1.254775in}{1.423227in}}%
\pgfpathlineto{\pgfqpoint{1.255772in}{1.423804in}}%
\pgfpathlineto{\pgfqpoint{1.258568in}{1.424912in}}%
\pgfpathlineto{\pgfqpoint{1.259843in}{1.425508in}}%
\pgfpathlineto{\pgfqpoint{1.262149in}{1.426617in}}%
\pgfpathlineto{\pgfqpoint{1.263251in}{1.427259in}}%
\pgfpathlineto{\pgfqpoint{1.265418in}{1.428367in}}%
\pgfpathlineto{\pgfqpoint{1.266506in}{1.429066in}}%
\pgfpathlineto{\pgfqpoint{1.268746in}{1.430174in}}%
\pgfpathlineto{\pgfqpoint{1.269820in}{1.430742in}}%
\pgfpathlineto{\pgfqpoint{1.272445in}{1.431850in}}%
\pgfpathlineto{\pgfqpoint{1.273554in}{1.432372in}}%
\pgfpathlineto{\pgfqpoint{1.275935in}{1.433471in}}%
\pgfpathlineto{\pgfqpoint{1.277044in}{1.433946in}}%
\pgfpathlineto{\pgfqpoint{1.279401in}{1.435054in}}%
\pgfpathlineto{\pgfqpoint{1.280412in}{1.435473in}}%
\pgfpathlineto{\pgfqpoint{1.280494in}{1.435473in}}%
\pgfpathlineto{\pgfqpoint{1.282605in}{1.436581in}}%
\pgfpathlineto{\pgfqpoint{1.283705in}{1.437168in}}%
\pgfpathlineto{\pgfqpoint{1.285595in}{1.438276in}}%
\pgfpathlineto{\pgfqpoint{1.286700in}{1.438900in}}%
\pgfpathlineto{\pgfqpoint{1.288974in}{1.440008in}}%
\pgfpathlineto{\pgfqpoint{1.290077in}{1.440511in}}%
\pgfpathlineto{\pgfqpoint{1.292861in}{1.441619in}}%
\pgfpathlineto{\pgfqpoint{1.293951in}{1.442122in}}%
\pgfpathlineto{\pgfqpoint{1.296292in}{1.443230in}}%
\pgfpathlineto{\pgfqpoint{1.297396in}{1.443826in}}%
\pgfpathlineto{\pgfqpoint{1.299918in}{1.444934in}}%
\pgfpathlineto{\pgfqpoint{1.301018in}{1.445530in}}%
\pgfpathlineto{\pgfqpoint{1.303447in}{1.446639in}}%
\pgfpathlineto{\pgfqpoint{1.304545in}{1.447095in}}%
\pgfpathlineto{\pgfqpoint{1.307451in}{1.448194in}}%
\pgfpathlineto{\pgfqpoint{1.308541in}{1.448790in}}%
\pgfpathlineto{\pgfqpoint{1.311109in}{1.449898in}}%
\pgfpathlineto{\pgfqpoint{1.312216in}{1.450475in}}%
\pgfpathlineto{\pgfqpoint{1.314780in}{1.451584in}}%
\pgfpathlineto{\pgfqpoint{1.315877in}{1.452114in}}%
\pgfpathlineto{\pgfqpoint{1.318258in}{1.453223in}}%
\pgfpathlineto{\pgfqpoint{1.319355in}{1.453679in}}%
\pgfpathlineto{\pgfqpoint{1.321844in}{1.454787in}}%
\pgfpathlineto{\pgfqpoint{1.322951in}{1.455216in}}%
\pgfpathlineto{\pgfqpoint{1.325181in}{1.456324in}}%
\pgfpathlineto{\pgfqpoint{1.326281in}{1.456780in}}%
\pgfpathlineto{\pgfqpoint{1.329074in}{1.457888in}}%
\pgfpathlineto{\pgfqpoint{1.330172in}{1.458335in}}%
\pgfpathlineto{\pgfqpoint{1.330184in}{1.458335in}}%
\pgfpathlineto{\pgfqpoint{1.333186in}{1.459443in}}%
\pgfpathlineto{\pgfqpoint{1.334293in}{1.459965in}}%
\pgfpathlineto{\pgfqpoint{1.336985in}{1.461073in}}%
\pgfpathlineto{\pgfqpoint{1.338040in}{1.461557in}}%
\pgfpathlineto{\pgfqpoint{1.340918in}{1.462666in}}%
\pgfpathlineto{\pgfqpoint{1.342011in}{1.463094in}}%
\pgfpathlineto{\pgfqpoint{1.344438in}{1.464202in}}%
\pgfpathlineto{\pgfqpoint{1.345543in}{1.464621in}}%
\pgfpathlineto{\pgfqpoint{1.345548in}{1.464621in}}%
\pgfpathlineto{\pgfqpoint{1.348425in}{1.465729in}}%
\pgfpathlineto{\pgfqpoint{1.349530in}{1.466307in}}%
\pgfpathlineto{\pgfqpoint{1.352321in}{1.467406in}}%
\pgfpathlineto{\pgfqpoint{1.353419in}{1.467899in}}%
\pgfpathlineto{\pgfqpoint{1.356592in}{1.469007in}}%
\pgfpathlineto{\pgfqpoint{1.357696in}{1.469426in}}%
\pgfpathlineto{\pgfqpoint{1.360729in}{1.470535in}}%
\pgfpathlineto{\pgfqpoint{1.361819in}{1.471121in}}%
\pgfpathlineto{\pgfqpoint{1.364735in}{1.472230in}}%
\pgfpathlineto{\pgfqpoint{1.366036in}{1.472863in}}%
\pgfpathlineto{\pgfqpoint{1.369024in}{1.473971in}}%
\pgfpathlineto{\pgfqpoint{1.370134in}{1.474437in}}%
\pgfpathlineto{\pgfqpoint{1.373215in}{1.475545in}}%
\pgfpathlineto{\pgfqpoint{1.374311in}{1.476048in}}%
\pgfpathlineto{\pgfqpoint{1.374325in}{1.476048in}}%
\pgfpathlineto{\pgfqpoint{1.377043in}{1.477156in}}%
\pgfpathlineto{\pgfqpoint{1.378147in}{1.477491in}}%
\pgfpathlineto{\pgfqpoint{1.381527in}{1.478590in}}%
\pgfpathlineto{\pgfqpoint{1.382587in}{1.478990in}}%
\pgfpathlineto{\pgfqpoint{1.385831in}{1.480099in}}%
\pgfpathlineto{\pgfqpoint{1.386938in}{1.480480in}}%
\pgfpathlineto{\pgfqpoint{1.389682in}{1.481589in}}%
\pgfpathlineto{\pgfqpoint{1.390786in}{1.482157in}}%
\pgfpathlineto{\pgfqpoint{1.393884in}{1.483265in}}%
\pgfpathlineto{\pgfqpoint{1.394954in}{1.483703in}}%
\pgfpathlineto{\pgfqpoint{1.398111in}{1.484811in}}%
\pgfpathlineto{\pgfqpoint{1.399199in}{1.485202in}}%
\pgfpathlineto{\pgfqpoint{1.401964in}{1.486310in}}%
\pgfpathlineto{\pgfqpoint{1.403071in}{1.486655in}}%
\pgfpathlineto{\pgfqpoint{1.406263in}{1.487763in}}%
\pgfpathlineto{\pgfqpoint{1.407274in}{1.488145in}}%
\pgfpathlineto{\pgfqpoint{1.410165in}{1.489253in}}%
\pgfpathlineto{\pgfqpoint{1.411275in}{1.489597in}}%
\pgfpathlineto{\pgfqpoint{1.414368in}{1.490706in}}%
\pgfpathlineto{\pgfqpoint{1.415623in}{1.491171in}}%
\pgfpathlineto{\pgfqpoint{1.418524in}{1.492280in}}%
\pgfpathlineto{\pgfqpoint{1.419626in}{1.492652in}}%
\pgfpathlineto{\pgfqpoint{1.423184in}{1.493760in}}%
\pgfpathlineto{\pgfqpoint{1.424240in}{1.494189in}}%
\pgfpathlineto{\pgfqpoint{1.427833in}{1.495297in}}%
\pgfpathlineto{\pgfqpoint{1.428900in}{1.495716in}}%
\pgfpathlineto{\pgfqpoint{1.431669in}{1.496824in}}%
\pgfpathlineto{\pgfqpoint{1.432769in}{1.497262in}}%
\pgfpathlineto{\pgfqpoint{1.436372in}{1.498370in}}%
\pgfpathlineto{\pgfqpoint{1.437474in}{1.498789in}}%
\pgfpathlineto{\pgfqpoint{1.440802in}{1.499897in}}%
\pgfpathlineto{\pgfqpoint{1.441869in}{1.500251in}}%
\pgfpathlineto{\pgfqpoint{1.445636in}{1.501359in}}%
\pgfpathlineto{\pgfqpoint{1.446740in}{1.501788in}}%
\pgfpathlineto{\pgfqpoint{1.449932in}{1.502896in}}%
\pgfpathlineto{\pgfqpoint{1.451018in}{1.503324in}}%
\pgfpathlineto{\pgfqpoint{1.454187in}{1.504432in}}%
\pgfpathlineto{\pgfqpoint{1.455294in}{1.504786in}}%
\pgfpathlineto{\pgfqpoint{1.458910in}{1.505894in}}%
\pgfpathlineto{\pgfqpoint{1.460015in}{1.506286in}}%
\pgfpathlineto{\pgfqpoint{1.460017in}{1.506286in}}%
\pgfpathlineto{\pgfqpoint{1.463899in}{1.507394in}}%
\pgfpathlineto{\pgfqpoint{1.465008in}{1.507766in}}%
\pgfpathlineto{\pgfqpoint{1.468451in}{1.508874in}}%
\pgfpathlineto{\pgfqpoint{1.469499in}{1.509154in}}%
\pgfpathlineto{\pgfqpoint{1.473868in}{1.510262in}}%
\pgfpathlineto{\pgfqpoint{1.474954in}{1.510690in}}%
\pgfpathlineto{\pgfqpoint{1.474978in}{1.510690in}}%
\pgfpathlineto{\pgfqpoint{1.478913in}{1.511799in}}%
\pgfpathlineto{\pgfqpoint{1.480018in}{1.512069in}}%
\pgfpathlineto{\pgfqpoint{1.483036in}{1.513177in}}%
\pgfpathlineto{\pgfqpoint{1.484120in}{1.513540in}}%
\pgfpathlineto{\pgfqpoint{1.487635in}{1.514648in}}%
\pgfpathlineto{\pgfqpoint{1.488684in}{1.514918in}}%
\pgfpathlineto{\pgfqpoint{1.492061in}{1.516008in}}%
\pgfpathlineto{\pgfqpoint{1.492061in}{1.516017in}}%
\pgfpathlineto{\pgfqpoint{1.493144in}{1.516334in}}%
\pgfpathlineto{\pgfqpoint{1.493156in}{1.516334in}}%
\pgfpathlineto{\pgfqpoint{1.496510in}{1.517442in}}%
\pgfpathlineto{\pgfqpoint{1.497615in}{1.517768in}}%
\pgfpathlineto{\pgfqpoint{1.501060in}{1.518876in}}%
\pgfpathlineto{\pgfqpoint{1.502157in}{1.519239in}}%
\pgfpathlineto{\pgfqpoint{1.505692in}{1.520348in}}%
\pgfpathlineto{\pgfqpoint{1.506745in}{1.520580in}}%
\pgfpathlineto{\pgfqpoint{1.506787in}{1.520580in}}%
\pgfpathlineto{\pgfqpoint{1.510849in}{1.521689in}}%
\pgfpathlineto{\pgfqpoint{1.511954in}{1.521959in}}%
\pgfpathlineto{\pgfqpoint{1.515486in}{1.523067in}}%
\pgfpathlineto{\pgfqpoint{1.516544in}{1.523439in}}%
\pgfpathlineto{\pgfqpoint{1.516572in}{1.523439in}}%
\pgfpathlineto{\pgfqpoint{1.520678in}{1.524548in}}%
\pgfpathlineto{\pgfqpoint{1.521741in}{1.524873in}}%
\pgfpathlineto{\pgfqpoint{1.521752in}{1.524873in}}%
\pgfpathlineto{\pgfqpoint{1.525643in}{1.525982in}}%
\pgfpathlineto{\pgfqpoint{1.526743in}{1.526363in}}%
\pgfpathlineto{\pgfqpoint{1.530498in}{1.527472in}}%
\pgfpathlineto{\pgfqpoint{1.531570in}{1.527751in}}%
\pgfpathlineto{\pgfqpoint{1.535205in}{1.528859in}}%
\pgfpathlineto{\pgfqpoint{1.536307in}{1.529073in}}%
\pgfpathlineto{\pgfqpoint{1.540580in}{1.530182in}}%
\pgfpathlineto{\pgfqpoint{1.541648in}{1.530386in}}%
\pgfpathlineto{\pgfqpoint{1.541680in}{1.530386in}}%
\pgfpathlineto{\pgfqpoint{1.545691in}{1.531495in}}%
\pgfpathlineto{\pgfqpoint{1.546760in}{1.531774in}}%
\pgfpathlineto{\pgfqpoint{1.546793in}{1.531774in}}%
\pgfpathlineto{\pgfqpoint{1.550822in}{1.532873in}}%
\pgfpathlineto{\pgfqpoint{1.551897in}{1.533171in}}%
\pgfpathlineto{\pgfqpoint{1.555785in}{1.534279in}}%
\pgfpathlineto{\pgfqpoint{1.556887in}{1.534577in}}%
\pgfpathlineto{\pgfqpoint{1.561191in}{1.535685in}}%
\pgfpathlineto{\pgfqpoint{1.562221in}{1.535946in}}%
\pgfpathlineto{\pgfqpoint{1.562225in}{1.535946in}}%
\pgfpathlineto{\pgfqpoint{1.567005in}{1.537054in}}%
\pgfpathlineto{\pgfqpoint{1.568072in}{1.537371in}}%
\pgfpathlineto{\pgfqpoint{1.572591in}{1.538479in}}%
\pgfpathlineto{\pgfqpoint{1.573513in}{1.538749in}}%
\pgfpathlineto{\pgfqpoint{1.573701in}{1.538749in}}%
\pgfpathlineto{\pgfqpoint{1.574890in}{1.539029in}}%
\pgfpathlineto{\pgfqpoint{1.579095in}{1.540137in}}%
\pgfpathlineto{\pgfqpoint{1.580179in}{1.540491in}}%
\pgfpathlineto{\pgfqpoint{1.584276in}{1.541599in}}%
\pgfpathlineto{\pgfqpoint{1.585366in}{1.541915in}}%
\pgfpathlineto{\pgfqpoint{1.589860in}{1.543024in}}%
\pgfpathlineto{\pgfqpoint{1.590941in}{1.543322in}}%
\pgfpathlineto{\pgfqpoint{1.594539in}{1.544430in}}%
\pgfpathlineto{\pgfqpoint{1.595618in}{1.544635in}}%
\pgfpathlineto{\pgfqpoint{1.595646in}{1.544635in}}%
\pgfpathlineto{\pgfqpoint{1.599830in}{1.545743in}}%
\pgfpathlineto{\pgfqpoint{1.600834in}{1.546022in}}%
\pgfpathlineto{\pgfqpoint{1.600852in}{1.546022in}}%
\pgfpathlineto{\pgfqpoint{1.605449in}{1.547130in}}%
\pgfpathlineto{\pgfqpoint{1.606533in}{1.547391in}}%
\pgfpathlineto{\pgfqpoint{1.606544in}{1.547391in}}%
\pgfpathlineto{\pgfqpoint{1.611204in}{1.548499in}}%
\pgfpathlineto{\pgfqpoint{1.612180in}{1.548714in}}%
\pgfpathlineto{\pgfqpoint{1.612227in}{1.548714in}}%
\pgfpathlineto{\pgfqpoint{1.617354in}{1.549822in}}%
\pgfpathlineto{\pgfqpoint{1.618458in}{1.550101in}}%
\pgfpathlineto{\pgfqpoint{1.623158in}{1.551209in}}%
\pgfpathlineto{\pgfqpoint{1.624202in}{1.551563in}}%
\pgfpathlineto{\pgfqpoint{1.624256in}{1.551563in}}%
\pgfpathlineto{\pgfqpoint{1.629594in}{1.552671in}}%
\pgfpathlineto{\pgfqpoint{1.630696in}{1.552932in}}%
\pgfpathlineto{\pgfqpoint{1.635931in}{1.554040in}}%
\pgfpathlineto{\pgfqpoint{1.637029in}{1.554357in}}%
\pgfpathlineto{\pgfqpoint{1.642334in}{1.555465in}}%
\pgfpathlineto{\pgfqpoint{1.643373in}{1.555661in}}%
\pgfpathlineto{\pgfqpoint{1.643389in}{1.555661in}}%
\pgfpathlineto{\pgfqpoint{1.648879in}{1.556760in}}%
\pgfpathlineto{\pgfqpoint{1.649958in}{1.556983in}}%
\pgfpathlineto{\pgfqpoint{1.655631in}{1.558091in}}%
\pgfpathlineto{\pgfqpoint{1.656687in}{1.558343in}}%
\pgfpathlineto{\pgfqpoint{1.662097in}{1.559451in}}%
\pgfpathlineto{\pgfqpoint{1.663195in}{1.559758in}}%
\pgfpathlineto{\pgfqpoint{1.668273in}{1.560866in}}%
\pgfpathlineto{\pgfqpoint{1.669333in}{1.561155in}}%
\pgfpathlineto{\pgfqpoint{1.669382in}{1.561155in}}%
\pgfpathlineto{\pgfqpoint{1.675123in}{1.562263in}}%
\pgfpathlineto{\pgfqpoint{1.676127in}{1.562515in}}%
\pgfpathlineto{\pgfqpoint{1.676197in}{1.562515in}}%
\pgfpathlineto{\pgfqpoint{1.682183in}{1.563623in}}%
\pgfpathlineto{\pgfqpoint{1.683278in}{1.563856in}}%
\pgfpathlineto{\pgfqpoint{1.688853in}{1.564964in}}%
\pgfpathlineto{\pgfqpoint{1.689871in}{1.565150in}}%
\pgfpathlineto{\pgfqpoint{1.695792in}{1.566258in}}%
\pgfpathlineto{\pgfqpoint{1.696967in}{1.566454in}}%
\pgfpathlineto{\pgfqpoint{1.703241in}{1.567562in}}%
\pgfpathlineto{\pgfqpoint{1.704437in}{1.567804in}}%
\pgfpathlineto{\pgfqpoint{1.710481in}{1.568913in}}%
\pgfpathlineto{\pgfqpoint{1.711471in}{1.569136in}}%
\pgfpathlineto{\pgfqpoint{1.711590in}{1.569136in}}%
\pgfpathlineto{\pgfqpoint{1.718497in}{1.570244in}}%
\pgfpathlineto{\pgfqpoint{1.719590in}{1.570440in}}%
\pgfpathlineto{\pgfqpoint{1.719597in}{1.570440in}}%
\pgfpathlineto{\pgfqpoint{1.726729in}{1.571548in}}%
\pgfpathlineto{\pgfqpoint{1.727824in}{1.571799in}}%
\pgfpathlineto{\pgfqpoint{1.733589in}{1.572908in}}%
\pgfpathlineto{\pgfqpoint{1.734612in}{1.573019in}}%
\pgfpathlineto{\pgfqpoint{1.734642in}{1.573019in}}%
\pgfpathlineto{\pgfqpoint{1.741805in}{1.574128in}}%
\pgfpathlineto{\pgfqpoint{1.742891in}{1.574323in}}%
\pgfpathlineto{\pgfqpoint{1.749903in}{1.575431in}}%
\pgfpathlineto{\pgfqpoint{1.750989in}{1.575562in}}%
\pgfpathlineto{\pgfqpoint{1.751010in}{1.575562in}}%
\pgfpathlineto{\pgfqpoint{1.757786in}{1.576670in}}%
\pgfpathlineto{\pgfqpoint{1.758883in}{1.576912in}}%
\pgfpathlineto{\pgfqpoint{1.765635in}{1.578020in}}%
\pgfpathlineto{\pgfqpoint{1.766634in}{1.578179in}}%
\pgfpathlineto{\pgfqpoint{1.766681in}{1.578179in}}%
\pgfpathlineto{\pgfqpoint{1.773645in}{1.579287in}}%
\pgfpathlineto{\pgfqpoint{1.774688in}{1.579501in}}%
\pgfpathlineto{\pgfqpoint{1.783410in}{1.580609in}}%
\pgfpathlineto{\pgfqpoint{1.784419in}{1.580777in}}%
\pgfpathlineto{\pgfqpoint{1.784517in}{1.580777in}}%
\pgfpathlineto{\pgfqpoint{1.792747in}{1.581885in}}%
\pgfpathlineto{\pgfqpoint{1.793833in}{1.582108in}}%
\pgfpathlineto{\pgfqpoint{1.802651in}{1.583217in}}%
\pgfpathlineto{\pgfqpoint{1.803608in}{1.583338in}}%
\pgfpathlineto{\pgfqpoint{1.803702in}{1.583338in}}%
\pgfpathlineto{\pgfqpoint{1.812075in}{1.584446in}}%
\pgfpathlineto{\pgfqpoint{1.813142in}{1.584567in}}%
\pgfpathlineto{\pgfqpoint{1.821965in}{1.585675in}}%
\pgfpathlineto{\pgfqpoint{1.822936in}{1.585806in}}%
\pgfpathlineto{\pgfqpoint{1.832734in}{1.586914in}}%
\pgfpathlineto{\pgfqpoint{1.833816in}{1.587063in}}%
\pgfpathlineto{\pgfqpoint{1.833827in}{1.587063in}}%
\pgfpathlineto{\pgfqpoint{1.844039in}{1.588171in}}%
\pgfpathlineto{\pgfqpoint{1.845106in}{1.588273in}}%
\pgfpathlineto{\pgfqpoint{1.845132in}{1.588273in}}%
\pgfpathlineto{\pgfqpoint{1.853652in}{1.589382in}}%
\pgfpathlineto{\pgfqpoint{1.854590in}{1.589475in}}%
\pgfpathlineto{\pgfqpoint{1.854647in}{1.589475in}}%
\pgfpathlineto{\pgfqpoint{1.864039in}{1.590583in}}%
\pgfpathlineto{\pgfqpoint{1.865043in}{1.590713in}}%
\pgfpathlineto{\pgfqpoint{1.865083in}{1.590713in}}%
\pgfpathlineto{\pgfqpoint{1.874141in}{1.591821in}}%
\pgfpathlineto{\pgfqpoint{1.875234in}{1.591961in}}%
\pgfpathlineto{\pgfqpoint{1.888391in}{1.593069in}}%
\pgfpathlineto{\pgfqpoint{1.889467in}{1.593172in}}%
\pgfpathlineto{\pgfqpoint{1.901344in}{1.594280in}}%
\pgfpathlineto{\pgfqpoint{1.902423in}{1.594401in}}%
\pgfpathlineto{\pgfqpoint{1.915695in}{1.595509in}}%
\pgfpathlineto{\pgfqpoint{1.916642in}{1.595612in}}%
\pgfpathlineto{\pgfqpoint{1.916776in}{1.595612in}}%
\pgfpathlineto{\pgfqpoint{1.929640in}{1.596720in}}%
\pgfpathlineto{\pgfqpoint{1.930529in}{1.596822in}}%
\pgfpathlineto{\pgfqpoint{1.930615in}{1.596822in}}%
\pgfpathlineto{\pgfqpoint{1.947070in}{1.597930in}}%
\pgfpathlineto{\pgfqpoint{1.948142in}{1.598033in}}%
\pgfpathlineto{\pgfqpoint{1.966700in}{1.599141in}}%
\pgfpathlineto{\pgfqpoint{1.967793in}{1.599216in}}%
\pgfpathlineto{\pgfqpoint{1.967805in}{1.599216in}}%
\pgfpathlineto{\pgfqpoint{1.988106in}{1.600324in}}%
\pgfpathlineto{\pgfqpoint{1.989065in}{1.600408in}}%
\pgfpathlineto{\pgfqpoint{1.989159in}{1.600408in}}%
\pgfpathlineto{\pgfqpoint{2.016179in}{1.601516in}}%
\pgfpathlineto{\pgfqpoint{2.017070in}{1.601553in}}%
\pgfpathlineto{\pgfqpoint{2.017253in}{1.601553in}}%
\pgfpathlineto{\pgfqpoint{2.033126in}{1.601944in}}%
\pgfpathlineto{\pgfqpoint{2.033126in}{1.601944in}}%
\pgfusepath{stroke}%
\end{pgfscope}%
\begin{pgfscope}%
\pgfsetrectcap%
\pgfsetmiterjoin%
\pgfsetlinewidth{0.803000pt}%
\definecolor{currentstroke}{rgb}{0.000000,0.000000,0.000000}%
\pgfsetstrokecolor{currentstroke}%
\pgfsetdash{}{0pt}%
\pgfpathmoveto{\pgfqpoint{0.553581in}{0.499444in}}%
\pgfpathlineto{\pgfqpoint{0.553581in}{1.654444in}}%
\pgfusepath{stroke}%
\end{pgfscope}%
\begin{pgfscope}%
\pgfsetrectcap%
\pgfsetmiterjoin%
\pgfsetlinewidth{0.803000pt}%
\definecolor{currentstroke}{rgb}{0.000000,0.000000,0.000000}%
\pgfsetstrokecolor{currentstroke}%
\pgfsetdash{}{0pt}%
\pgfpathmoveto{\pgfqpoint{2.103581in}{0.499444in}}%
\pgfpathlineto{\pgfqpoint{2.103581in}{1.654444in}}%
\pgfusepath{stroke}%
\end{pgfscope}%
\begin{pgfscope}%
\pgfsetrectcap%
\pgfsetmiterjoin%
\pgfsetlinewidth{0.803000pt}%
\definecolor{currentstroke}{rgb}{0.000000,0.000000,0.000000}%
\pgfsetstrokecolor{currentstroke}%
\pgfsetdash{}{0pt}%
\pgfpathmoveto{\pgfqpoint{0.553581in}{0.499444in}}%
\pgfpathlineto{\pgfqpoint{2.103581in}{0.499444in}}%
\pgfusepath{stroke}%
\end{pgfscope}%
\begin{pgfscope}%
\pgfsetrectcap%
\pgfsetmiterjoin%
\pgfsetlinewidth{0.803000pt}%
\definecolor{currentstroke}{rgb}{0.000000,0.000000,0.000000}%
\pgfsetstrokecolor{currentstroke}%
\pgfsetdash{}{0pt}%
\pgfpathmoveto{\pgfqpoint{0.553581in}{1.654444in}}%
\pgfpathlineto{\pgfqpoint{2.103581in}{1.654444in}}%
\pgfusepath{stroke}%
\end{pgfscope}%
\begin{pgfscope}%
\pgfsetbuttcap%
\pgfsetmiterjoin%
\definecolor{currentfill}{rgb}{1.000000,1.000000,1.000000}%
\pgfsetfillcolor{currentfill}%
\pgfsetfillopacity{0.800000}%
\pgfsetlinewidth{1.003750pt}%
\definecolor{currentstroke}{rgb}{0.800000,0.800000,0.800000}%
\pgfsetstrokecolor{currentstroke}%
\pgfsetstrokeopacity{0.800000}%
\pgfsetdash{}{0pt}%
\pgfpathmoveto{\pgfqpoint{0.832747in}{0.568889in}}%
\pgfpathlineto{\pgfqpoint{2.006358in}{0.568889in}}%
\pgfpathquadraticcurveto{\pgfqpoint{2.034136in}{0.568889in}}{\pgfqpoint{2.034136in}{0.596666in}}%
\pgfpathlineto{\pgfqpoint{2.034136in}{0.776388in}}%
\pgfpathquadraticcurveto{\pgfqpoint{2.034136in}{0.804166in}}{\pgfqpoint{2.006358in}{0.804166in}}%
\pgfpathlineto{\pgfqpoint{0.832747in}{0.804166in}}%
\pgfpathquadraticcurveto{\pgfqpoint{0.804970in}{0.804166in}}{\pgfqpoint{0.804970in}{0.776388in}}%
\pgfpathlineto{\pgfqpoint{0.804970in}{0.596666in}}%
\pgfpathquadraticcurveto{\pgfqpoint{0.804970in}{0.568889in}}{\pgfqpoint{0.832747in}{0.568889in}}%
\pgfpathlineto{\pgfqpoint{0.832747in}{0.568889in}}%
\pgfpathclose%
\pgfusepath{stroke,fill}%
\end{pgfscope}%
\begin{pgfscope}%
\pgfsetrectcap%
\pgfsetroundjoin%
\pgfsetlinewidth{1.505625pt}%
\definecolor{currentstroke}{rgb}{0.000000,0.000000,0.000000}%
\pgfsetstrokecolor{currentstroke}%
\pgfsetdash{}{0pt}%
\pgfpathmoveto{\pgfqpoint{0.860525in}{0.700000in}}%
\pgfpathlineto{\pgfqpoint{0.999414in}{0.700000in}}%
\pgfpathlineto{\pgfqpoint{1.138303in}{0.700000in}}%
\pgfusepath{stroke}%
\end{pgfscope}%
\begin{pgfscope}%
\definecolor{textcolor}{rgb}{0.000000,0.000000,0.000000}%
\pgfsetstrokecolor{textcolor}%
\pgfsetfillcolor{textcolor}%
\pgftext[x=1.249414in,y=0.651388in,left,base]{\color{textcolor}\rmfamily\fontsize{10.000000}{12.000000}\selectfont AUC=0.777}%
\end{pgfscope}%
\end{pgfpicture}%
\makeatother%
\endgroup%

\end{tabular}

\



%
\verb|Bagging_Hard_Tomek_0_v1_Test|

\

This model returned 217 different values, but most of them were rare.  Taking out the 5\% of the data set with the least frequent values, 95\% of the samples had only 10 values of $p$.  It may be a useful model, but we will not be able to fine tune the decision threshold.  

\noindent\begin{tabular}{@{\hspace{-6pt}}p{4.3in} @{\hspace{-6pt}}p{2.0in}}
	\vskip 0pt
	\hfil Raw Model Output
	
	%% Creator: Matplotlib, PGF backend
%%
%% To include the figure in your LaTeX document, write
%%   \input{<filename>.pgf}
%%
%% Make sure the required packages are loaded in your preamble
%%   \usepackage{pgf}
%%
%% Also ensure that all the required font packages are loaded; for instance,
%% the lmodern package is sometimes necessary when using math font.
%%   \usepackage{lmodern}
%%
%% Figures using additional raster images can only be included by \input if
%% they are in the same directory as the main LaTeX file. For loading figures
%% from other directories you can use the `import` package
%%   \usepackage{import}
%%
%% and then include the figures with
%%   \import{<path to file>}{<filename>.pgf}
%%
%% Matplotlib used the following preamble
%%   
%%   \usepackage{fontspec}
%%   \makeatletter\@ifpackageloaded{underscore}{}{\usepackage[strings]{underscore}}\makeatother
%%
\begingroup%
\makeatletter%
\begin{pgfpicture}%
\pgfpathrectangle{\pgfpointorigin}{\pgfqpoint{4.102500in}{1.754444in}}%
\pgfusepath{use as bounding box, clip}%
\begin{pgfscope}%
\pgfsetbuttcap%
\pgfsetmiterjoin%
\definecolor{currentfill}{rgb}{1.000000,1.000000,1.000000}%
\pgfsetfillcolor{currentfill}%
\pgfsetlinewidth{0.000000pt}%
\definecolor{currentstroke}{rgb}{1.000000,1.000000,1.000000}%
\pgfsetstrokecolor{currentstroke}%
\pgfsetdash{}{0pt}%
\pgfpathmoveto{\pgfqpoint{0.000000in}{0.000000in}}%
\pgfpathlineto{\pgfqpoint{4.102500in}{0.000000in}}%
\pgfpathlineto{\pgfqpoint{4.102500in}{1.754444in}}%
\pgfpathlineto{\pgfqpoint{0.000000in}{1.754444in}}%
\pgfpathlineto{\pgfqpoint{0.000000in}{0.000000in}}%
\pgfpathclose%
\pgfusepath{fill}%
\end{pgfscope}%
\begin{pgfscope}%
\pgfsetbuttcap%
\pgfsetmiterjoin%
\definecolor{currentfill}{rgb}{1.000000,1.000000,1.000000}%
\pgfsetfillcolor{currentfill}%
\pgfsetlinewidth{0.000000pt}%
\definecolor{currentstroke}{rgb}{0.000000,0.000000,0.000000}%
\pgfsetstrokecolor{currentstroke}%
\pgfsetstrokeopacity{0.000000}%
\pgfsetdash{}{0pt}%
\pgfpathmoveto{\pgfqpoint{0.515000in}{0.499444in}}%
\pgfpathlineto{\pgfqpoint{4.002500in}{0.499444in}}%
\pgfpathlineto{\pgfqpoint{4.002500in}{1.654444in}}%
\pgfpathlineto{\pgfqpoint{0.515000in}{1.654444in}}%
\pgfpathlineto{\pgfqpoint{0.515000in}{0.499444in}}%
\pgfpathclose%
\pgfusepath{fill}%
\end{pgfscope}%
\begin{pgfscope}%
\pgfpathrectangle{\pgfqpoint{0.515000in}{0.499444in}}{\pgfqpoint{3.487500in}{1.155000in}}%
\pgfusepath{clip}%
\pgfsetbuttcap%
\pgfsetmiterjoin%
\pgfsetlinewidth{1.003750pt}%
\definecolor{currentstroke}{rgb}{0.000000,0.000000,0.000000}%
\pgfsetstrokecolor{currentstroke}%
\pgfsetdash{}{0pt}%
\pgfpathmoveto{\pgfqpoint{0.610114in}{0.499444in}}%
\pgfpathlineto{\pgfqpoint{0.673523in}{0.499444in}}%
\pgfpathlineto{\pgfqpoint{0.673523in}{1.338346in}}%
\pgfpathlineto{\pgfqpoint{0.610114in}{1.338346in}}%
\pgfpathlineto{\pgfqpoint{0.610114in}{0.499444in}}%
\pgfpathclose%
\pgfusepath{stroke}%
\end{pgfscope}%
\begin{pgfscope}%
\pgfpathrectangle{\pgfqpoint{0.515000in}{0.499444in}}{\pgfqpoint{3.487500in}{1.155000in}}%
\pgfusepath{clip}%
\pgfsetbuttcap%
\pgfsetmiterjoin%
\pgfsetlinewidth{1.003750pt}%
\definecolor{currentstroke}{rgb}{0.000000,0.000000,0.000000}%
\pgfsetstrokecolor{currentstroke}%
\pgfsetdash{}{0pt}%
\pgfpathmoveto{\pgfqpoint{0.768637in}{0.499444in}}%
\pgfpathlineto{\pgfqpoint{0.832046in}{0.499444in}}%
\pgfpathlineto{\pgfqpoint{0.832046in}{0.502174in}}%
\pgfpathlineto{\pgfqpoint{0.768637in}{0.502174in}}%
\pgfpathlineto{\pgfqpoint{0.768637in}{0.499444in}}%
\pgfpathclose%
\pgfusepath{stroke}%
\end{pgfscope}%
\begin{pgfscope}%
\pgfpathrectangle{\pgfqpoint{0.515000in}{0.499444in}}{\pgfqpoint{3.487500in}{1.155000in}}%
\pgfusepath{clip}%
\pgfsetbuttcap%
\pgfsetmiterjoin%
\pgfsetlinewidth{1.003750pt}%
\definecolor{currentstroke}{rgb}{0.000000,0.000000,0.000000}%
\pgfsetstrokecolor{currentstroke}%
\pgfsetdash{}{0pt}%
\pgfpathmoveto{\pgfqpoint{0.927159in}{0.499444in}}%
\pgfpathlineto{\pgfqpoint{0.990568in}{0.499444in}}%
\pgfpathlineto{\pgfqpoint{0.990568in}{1.565567in}}%
\pgfpathlineto{\pgfqpoint{0.927159in}{1.565567in}}%
\pgfpathlineto{\pgfqpoint{0.927159in}{0.499444in}}%
\pgfpathclose%
\pgfusepath{stroke}%
\end{pgfscope}%
\begin{pgfscope}%
\pgfpathrectangle{\pgfqpoint{0.515000in}{0.499444in}}{\pgfqpoint{3.487500in}{1.155000in}}%
\pgfusepath{clip}%
\pgfsetbuttcap%
\pgfsetmiterjoin%
\pgfsetlinewidth{1.003750pt}%
\definecolor{currentstroke}{rgb}{0.000000,0.000000,0.000000}%
\pgfsetstrokecolor{currentstroke}%
\pgfsetdash{}{0pt}%
\pgfpathmoveto{\pgfqpoint{1.085682in}{0.499444in}}%
\pgfpathlineto{\pgfqpoint{1.149091in}{0.499444in}}%
\pgfpathlineto{\pgfqpoint{1.149091in}{0.504325in}}%
\pgfpathlineto{\pgfqpoint{1.085682in}{0.504325in}}%
\pgfpathlineto{\pgfqpoint{1.085682in}{0.499444in}}%
\pgfpathclose%
\pgfusepath{stroke}%
\end{pgfscope}%
\begin{pgfscope}%
\pgfpathrectangle{\pgfqpoint{0.515000in}{0.499444in}}{\pgfqpoint{3.487500in}{1.155000in}}%
\pgfusepath{clip}%
\pgfsetbuttcap%
\pgfsetmiterjoin%
\pgfsetlinewidth{1.003750pt}%
\definecolor{currentstroke}{rgb}{0.000000,0.000000,0.000000}%
\pgfsetstrokecolor{currentstroke}%
\pgfsetdash{}{0pt}%
\pgfpathmoveto{\pgfqpoint{1.244205in}{0.499444in}}%
\pgfpathlineto{\pgfqpoint{1.307614in}{0.499444in}}%
\pgfpathlineto{\pgfqpoint{1.307614in}{1.599444in}}%
\pgfpathlineto{\pgfqpoint{1.244205in}{1.599444in}}%
\pgfpathlineto{\pgfqpoint{1.244205in}{0.499444in}}%
\pgfpathclose%
\pgfusepath{stroke}%
\end{pgfscope}%
\begin{pgfscope}%
\pgfpathrectangle{\pgfqpoint{0.515000in}{0.499444in}}{\pgfqpoint{3.487500in}{1.155000in}}%
\pgfusepath{clip}%
\pgfsetbuttcap%
\pgfsetmiterjoin%
\pgfsetlinewidth{1.003750pt}%
\definecolor{currentstroke}{rgb}{0.000000,0.000000,0.000000}%
\pgfsetstrokecolor{currentstroke}%
\pgfsetdash{}{0pt}%
\pgfpathmoveto{\pgfqpoint{1.402728in}{0.499444in}}%
\pgfpathlineto{\pgfqpoint{1.466137in}{0.499444in}}%
\pgfpathlineto{\pgfqpoint{1.466137in}{0.505774in}}%
\pgfpathlineto{\pgfqpoint{1.402728in}{0.505774in}}%
\pgfpathlineto{\pgfqpoint{1.402728in}{0.499444in}}%
\pgfpathclose%
\pgfusepath{stroke}%
\end{pgfscope}%
\begin{pgfscope}%
\pgfpathrectangle{\pgfqpoint{0.515000in}{0.499444in}}{\pgfqpoint{3.487500in}{1.155000in}}%
\pgfusepath{clip}%
\pgfsetbuttcap%
\pgfsetmiterjoin%
\pgfsetlinewidth{1.003750pt}%
\definecolor{currentstroke}{rgb}{0.000000,0.000000,0.000000}%
\pgfsetstrokecolor{currentstroke}%
\pgfsetdash{}{0pt}%
\pgfpathmoveto{\pgfqpoint{1.561250in}{0.499444in}}%
\pgfpathlineto{\pgfqpoint{1.624659in}{0.499444in}}%
\pgfpathlineto{\pgfqpoint{1.624659in}{1.519343in}}%
\pgfpathlineto{\pgfqpoint{1.561250in}{1.519343in}}%
\pgfpathlineto{\pgfqpoint{1.561250in}{0.499444in}}%
\pgfpathclose%
\pgfusepath{stroke}%
\end{pgfscope}%
\begin{pgfscope}%
\pgfpathrectangle{\pgfqpoint{0.515000in}{0.499444in}}{\pgfqpoint{3.487500in}{1.155000in}}%
\pgfusepath{clip}%
\pgfsetbuttcap%
\pgfsetmiterjoin%
\pgfsetlinewidth{1.003750pt}%
\definecolor{currentstroke}{rgb}{0.000000,0.000000,0.000000}%
\pgfsetstrokecolor{currentstroke}%
\pgfsetdash{}{0pt}%
\pgfpathmoveto{\pgfqpoint{1.719773in}{0.499444in}}%
\pgfpathlineto{\pgfqpoint{1.783182in}{0.499444in}}%
\pgfpathlineto{\pgfqpoint{1.783182in}{0.506197in}}%
\pgfpathlineto{\pgfqpoint{1.719773in}{0.506197in}}%
\pgfpathlineto{\pgfqpoint{1.719773in}{0.499444in}}%
\pgfpathclose%
\pgfusepath{stroke}%
\end{pgfscope}%
\begin{pgfscope}%
\pgfpathrectangle{\pgfqpoint{0.515000in}{0.499444in}}{\pgfqpoint{3.487500in}{1.155000in}}%
\pgfusepath{clip}%
\pgfsetbuttcap%
\pgfsetmiterjoin%
\pgfsetlinewidth{1.003750pt}%
\definecolor{currentstroke}{rgb}{0.000000,0.000000,0.000000}%
\pgfsetstrokecolor{currentstroke}%
\pgfsetdash{}{0pt}%
\pgfpathmoveto{\pgfqpoint{1.878296in}{0.499444in}}%
\pgfpathlineto{\pgfqpoint{1.941705in}{0.499444in}}%
\pgfpathlineto{\pgfqpoint{1.941705in}{1.360243in}}%
\pgfpathlineto{\pgfqpoint{1.878296in}{1.360243in}}%
\pgfpathlineto{\pgfqpoint{1.878296in}{0.499444in}}%
\pgfpathclose%
\pgfusepath{stroke}%
\end{pgfscope}%
\begin{pgfscope}%
\pgfpathrectangle{\pgfqpoint{0.515000in}{0.499444in}}{\pgfqpoint{3.487500in}{1.155000in}}%
\pgfusepath{clip}%
\pgfsetbuttcap%
\pgfsetmiterjoin%
\pgfsetlinewidth{1.003750pt}%
\definecolor{currentstroke}{rgb}{0.000000,0.000000,0.000000}%
\pgfsetstrokecolor{currentstroke}%
\pgfsetdash{}{0pt}%
\pgfpathmoveto{\pgfqpoint{2.036818in}{0.499444in}}%
\pgfpathlineto{\pgfqpoint{2.100228in}{0.499444in}}%
\pgfpathlineto{\pgfqpoint{2.100228in}{0.505908in}}%
\pgfpathlineto{\pgfqpoint{2.036818in}{0.505908in}}%
\pgfpathlineto{\pgfqpoint{2.036818in}{0.499444in}}%
\pgfpathclose%
\pgfusepath{stroke}%
\end{pgfscope}%
\begin{pgfscope}%
\pgfpathrectangle{\pgfqpoint{0.515000in}{0.499444in}}{\pgfqpoint{3.487500in}{1.155000in}}%
\pgfusepath{clip}%
\pgfsetbuttcap%
\pgfsetmiterjoin%
\pgfsetlinewidth{1.003750pt}%
\definecolor{currentstroke}{rgb}{0.000000,0.000000,0.000000}%
\pgfsetstrokecolor{currentstroke}%
\pgfsetdash{}{0pt}%
\pgfpathmoveto{\pgfqpoint{2.195341in}{0.499444in}}%
\pgfpathlineto{\pgfqpoint{2.258750in}{0.499444in}}%
\pgfpathlineto{\pgfqpoint{2.258750in}{1.162431in}}%
\pgfpathlineto{\pgfqpoint{2.195341in}{1.162431in}}%
\pgfpathlineto{\pgfqpoint{2.195341in}{0.499444in}}%
\pgfpathclose%
\pgfusepath{stroke}%
\end{pgfscope}%
\begin{pgfscope}%
\pgfpathrectangle{\pgfqpoint{0.515000in}{0.499444in}}{\pgfqpoint{3.487500in}{1.155000in}}%
\pgfusepath{clip}%
\pgfsetbuttcap%
\pgfsetmiterjoin%
\pgfsetlinewidth{1.003750pt}%
\definecolor{currentstroke}{rgb}{0.000000,0.000000,0.000000}%
\pgfsetstrokecolor{currentstroke}%
\pgfsetdash{}{0pt}%
\pgfpathmoveto{\pgfqpoint{2.353864in}{0.499444in}}%
\pgfpathlineto{\pgfqpoint{2.417273in}{0.499444in}}%
\pgfpathlineto{\pgfqpoint{2.417273in}{0.503969in}}%
\pgfpathlineto{\pgfqpoint{2.353864in}{0.503969in}}%
\pgfpathlineto{\pgfqpoint{2.353864in}{0.499444in}}%
\pgfpathclose%
\pgfusepath{stroke}%
\end{pgfscope}%
\begin{pgfscope}%
\pgfpathrectangle{\pgfqpoint{0.515000in}{0.499444in}}{\pgfqpoint{3.487500in}{1.155000in}}%
\pgfusepath{clip}%
\pgfsetbuttcap%
\pgfsetmiterjoin%
\pgfsetlinewidth{1.003750pt}%
\definecolor{currentstroke}{rgb}{0.000000,0.000000,0.000000}%
\pgfsetstrokecolor{currentstroke}%
\pgfsetdash{}{0pt}%
\pgfpathmoveto{\pgfqpoint{2.512387in}{0.499444in}}%
\pgfpathlineto{\pgfqpoint{2.575796in}{0.499444in}}%
\pgfpathlineto{\pgfqpoint{2.575796in}{0.975595in}}%
\pgfpathlineto{\pgfqpoint{2.512387in}{0.975595in}}%
\pgfpathlineto{\pgfqpoint{2.512387in}{0.499444in}}%
\pgfpathclose%
\pgfusepath{stroke}%
\end{pgfscope}%
\begin{pgfscope}%
\pgfpathrectangle{\pgfqpoint{0.515000in}{0.499444in}}{\pgfqpoint{3.487500in}{1.155000in}}%
\pgfusepath{clip}%
\pgfsetbuttcap%
\pgfsetmiterjoin%
\pgfsetlinewidth{1.003750pt}%
\definecolor{currentstroke}{rgb}{0.000000,0.000000,0.000000}%
\pgfsetstrokecolor{currentstroke}%
\pgfsetdash{}{0pt}%
\pgfpathmoveto{\pgfqpoint{2.670909in}{0.499444in}}%
\pgfpathlineto{\pgfqpoint{2.734318in}{0.499444in}}%
\pgfpathlineto{\pgfqpoint{2.734318in}{0.502988in}}%
\pgfpathlineto{\pgfqpoint{2.670909in}{0.502988in}}%
\pgfpathlineto{\pgfqpoint{2.670909in}{0.499444in}}%
\pgfpathclose%
\pgfusepath{stroke}%
\end{pgfscope}%
\begin{pgfscope}%
\pgfpathrectangle{\pgfqpoint{0.515000in}{0.499444in}}{\pgfqpoint{3.487500in}{1.155000in}}%
\pgfusepath{clip}%
\pgfsetbuttcap%
\pgfsetmiterjoin%
\pgfsetlinewidth{1.003750pt}%
\definecolor{currentstroke}{rgb}{0.000000,0.000000,0.000000}%
\pgfsetstrokecolor{currentstroke}%
\pgfsetdash{}{0pt}%
\pgfpathmoveto{\pgfqpoint{2.829432in}{0.499444in}}%
\pgfpathlineto{\pgfqpoint{2.892841in}{0.499444in}}%
\pgfpathlineto{\pgfqpoint{2.892841in}{0.810121in}}%
\pgfpathlineto{\pgfqpoint{2.829432in}{0.810121in}}%
\pgfpathlineto{\pgfqpoint{2.829432in}{0.499444in}}%
\pgfpathclose%
\pgfusepath{stroke}%
\end{pgfscope}%
\begin{pgfscope}%
\pgfpathrectangle{\pgfqpoint{0.515000in}{0.499444in}}{\pgfqpoint{3.487500in}{1.155000in}}%
\pgfusepath{clip}%
\pgfsetbuttcap%
\pgfsetmiterjoin%
\pgfsetlinewidth{1.003750pt}%
\definecolor{currentstroke}{rgb}{0.000000,0.000000,0.000000}%
\pgfsetstrokecolor{currentstroke}%
\pgfsetdash{}{0pt}%
\pgfpathmoveto{\pgfqpoint{2.987955in}{0.499444in}}%
\pgfpathlineto{\pgfqpoint{3.051364in}{0.499444in}}%
\pgfpathlineto{\pgfqpoint{3.051364in}{0.501974in}}%
\pgfpathlineto{\pgfqpoint{2.987955in}{0.501974in}}%
\pgfpathlineto{\pgfqpoint{2.987955in}{0.499444in}}%
\pgfpathclose%
\pgfusepath{stroke}%
\end{pgfscope}%
\begin{pgfscope}%
\pgfpathrectangle{\pgfqpoint{0.515000in}{0.499444in}}{\pgfqpoint{3.487500in}{1.155000in}}%
\pgfusepath{clip}%
\pgfsetbuttcap%
\pgfsetmiterjoin%
\pgfsetlinewidth{1.003750pt}%
\definecolor{currentstroke}{rgb}{0.000000,0.000000,0.000000}%
\pgfsetstrokecolor{currentstroke}%
\pgfsetdash{}{0pt}%
\pgfpathmoveto{\pgfqpoint{3.146478in}{0.499444in}}%
\pgfpathlineto{\pgfqpoint{3.209887in}{0.499444in}}%
\pgfpathlineto{\pgfqpoint{3.209887in}{0.682692in}}%
\pgfpathlineto{\pgfqpoint{3.146478in}{0.682692in}}%
\pgfpathlineto{\pgfqpoint{3.146478in}{0.499444in}}%
\pgfpathclose%
\pgfusepath{stroke}%
\end{pgfscope}%
\begin{pgfscope}%
\pgfpathrectangle{\pgfqpoint{0.515000in}{0.499444in}}{\pgfqpoint{3.487500in}{1.155000in}}%
\pgfusepath{clip}%
\pgfsetbuttcap%
\pgfsetmiterjoin%
\pgfsetlinewidth{1.003750pt}%
\definecolor{currentstroke}{rgb}{0.000000,0.000000,0.000000}%
\pgfsetstrokecolor{currentstroke}%
\pgfsetdash{}{0pt}%
\pgfpathmoveto{\pgfqpoint{3.305000in}{0.499444in}}%
\pgfpathlineto{\pgfqpoint{3.368409in}{0.499444in}}%
\pgfpathlineto{\pgfqpoint{3.368409in}{0.501060in}}%
\pgfpathlineto{\pgfqpoint{3.305000in}{0.501060in}}%
\pgfpathlineto{\pgfqpoint{3.305000in}{0.499444in}}%
\pgfpathclose%
\pgfusepath{stroke}%
\end{pgfscope}%
\begin{pgfscope}%
\pgfpathrectangle{\pgfqpoint{0.515000in}{0.499444in}}{\pgfqpoint{3.487500in}{1.155000in}}%
\pgfusepath{clip}%
\pgfsetbuttcap%
\pgfsetmiterjoin%
\pgfsetlinewidth{1.003750pt}%
\definecolor{currentstroke}{rgb}{0.000000,0.000000,0.000000}%
\pgfsetstrokecolor{currentstroke}%
\pgfsetdash{}{0pt}%
\pgfpathmoveto{\pgfqpoint{3.463523in}{0.499444in}}%
\pgfpathlineto{\pgfqpoint{3.526932in}{0.499444in}}%
\pgfpathlineto{\pgfqpoint{3.526932in}{0.598011in}}%
\pgfpathlineto{\pgfqpoint{3.463523in}{0.598011in}}%
\pgfpathlineto{\pgfqpoint{3.463523in}{0.499444in}}%
\pgfpathclose%
\pgfusepath{stroke}%
\end{pgfscope}%
\begin{pgfscope}%
\pgfpathrectangle{\pgfqpoint{0.515000in}{0.499444in}}{\pgfqpoint{3.487500in}{1.155000in}}%
\pgfusepath{clip}%
\pgfsetbuttcap%
\pgfsetmiterjoin%
\pgfsetlinewidth{1.003750pt}%
\definecolor{currentstroke}{rgb}{0.000000,0.000000,0.000000}%
\pgfsetstrokecolor{currentstroke}%
\pgfsetdash{}{0pt}%
\pgfpathmoveto{\pgfqpoint{3.622046in}{0.499444in}}%
\pgfpathlineto{\pgfqpoint{3.685455in}{0.499444in}}%
\pgfpathlineto{\pgfqpoint{3.685455in}{0.500046in}}%
\pgfpathlineto{\pgfqpoint{3.622046in}{0.500046in}}%
\pgfpathlineto{\pgfqpoint{3.622046in}{0.499444in}}%
\pgfpathclose%
\pgfusepath{stroke}%
\end{pgfscope}%
\begin{pgfscope}%
\pgfpathrectangle{\pgfqpoint{0.515000in}{0.499444in}}{\pgfqpoint{3.487500in}{1.155000in}}%
\pgfusepath{clip}%
\pgfsetbuttcap%
\pgfsetmiterjoin%
\pgfsetlinewidth{1.003750pt}%
\definecolor{currentstroke}{rgb}{0.000000,0.000000,0.000000}%
\pgfsetstrokecolor{currentstroke}%
\pgfsetdash{}{0pt}%
\pgfpathmoveto{\pgfqpoint{3.780568in}{0.499444in}}%
\pgfpathlineto{\pgfqpoint{3.843978in}{0.499444in}}%
\pgfpathlineto{\pgfqpoint{3.843978in}{0.537456in}}%
\pgfpathlineto{\pgfqpoint{3.780568in}{0.537456in}}%
\pgfpathlineto{\pgfqpoint{3.780568in}{0.499444in}}%
\pgfpathclose%
\pgfusepath{stroke}%
\end{pgfscope}%
\begin{pgfscope}%
\pgfpathrectangle{\pgfqpoint{0.515000in}{0.499444in}}{\pgfqpoint{3.487500in}{1.155000in}}%
\pgfusepath{clip}%
\pgfsetbuttcap%
\pgfsetmiterjoin%
\definecolor{currentfill}{rgb}{0.000000,0.000000,0.000000}%
\pgfsetfillcolor{currentfill}%
\pgfsetlinewidth{0.000000pt}%
\definecolor{currentstroke}{rgb}{0.000000,0.000000,0.000000}%
\pgfsetstrokecolor{currentstroke}%
\pgfsetstrokeopacity{0.000000}%
\pgfsetdash{}{0pt}%
\pgfpathmoveto{\pgfqpoint{0.673523in}{0.499444in}}%
\pgfpathlineto{\pgfqpoint{0.736932in}{0.499444in}}%
\pgfpathlineto{\pgfqpoint{0.736932in}{0.524852in}}%
\pgfpathlineto{\pgfqpoint{0.673523in}{0.524852in}}%
\pgfpathlineto{\pgfqpoint{0.673523in}{0.499444in}}%
\pgfpathclose%
\pgfusepath{fill}%
\end{pgfscope}%
\begin{pgfscope}%
\pgfpathrectangle{\pgfqpoint{0.515000in}{0.499444in}}{\pgfqpoint{3.487500in}{1.155000in}}%
\pgfusepath{clip}%
\pgfsetbuttcap%
\pgfsetmiterjoin%
\definecolor{currentfill}{rgb}{0.000000,0.000000,0.000000}%
\pgfsetfillcolor{currentfill}%
\pgfsetlinewidth{0.000000pt}%
\definecolor{currentstroke}{rgb}{0.000000,0.000000,0.000000}%
\pgfsetstrokecolor{currentstroke}%
\pgfsetstrokeopacity{0.000000}%
\pgfsetdash{}{0pt}%
\pgfpathmoveto{\pgfqpoint{0.832046in}{0.499444in}}%
\pgfpathlineto{\pgfqpoint{0.895455in}{0.499444in}}%
\pgfpathlineto{\pgfqpoint{0.895455in}{0.499578in}}%
\pgfpathlineto{\pgfqpoint{0.832046in}{0.499578in}}%
\pgfpathlineto{\pgfqpoint{0.832046in}{0.499444in}}%
\pgfpathclose%
\pgfusepath{fill}%
\end{pgfscope}%
\begin{pgfscope}%
\pgfpathrectangle{\pgfqpoint{0.515000in}{0.499444in}}{\pgfqpoint{3.487500in}{1.155000in}}%
\pgfusepath{clip}%
\pgfsetbuttcap%
\pgfsetmiterjoin%
\definecolor{currentfill}{rgb}{0.000000,0.000000,0.000000}%
\pgfsetfillcolor{currentfill}%
\pgfsetlinewidth{0.000000pt}%
\definecolor{currentstroke}{rgb}{0.000000,0.000000,0.000000}%
\pgfsetstrokecolor{currentstroke}%
\pgfsetstrokeopacity{0.000000}%
\pgfsetdash{}{0pt}%
\pgfpathmoveto{\pgfqpoint{0.990568in}{0.499444in}}%
\pgfpathlineto{\pgfqpoint{1.053978in}{0.499444in}}%
\pgfpathlineto{\pgfqpoint{1.053978in}{0.555208in}}%
\pgfpathlineto{\pgfqpoint{0.990568in}{0.555208in}}%
\pgfpathlineto{\pgfqpoint{0.990568in}{0.499444in}}%
\pgfpathclose%
\pgfusepath{fill}%
\end{pgfscope}%
\begin{pgfscope}%
\pgfpathrectangle{\pgfqpoint{0.515000in}{0.499444in}}{\pgfqpoint{3.487500in}{1.155000in}}%
\pgfusepath{clip}%
\pgfsetbuttcap%
\pgfsetmiterjoin%
\definecolor{currentfill}{rgb}{0.000000,0.000000,0.000000}%
\pgfsetfillcolor{currentfill}%
\pgfsetlinewidth{0.000000pt}%
\definecolor{currentstroke}{rgb}{0.000000,0.000000,0.000000}%
\pgfsetstrokecolor{currentstroke}%
\pgfsetstrokeopacity{0.000000}%
\pgfsetdash{}{0pt}%
\pgfpathmoveto{\pgfqpoint{1.149091in}{0.499444in}}%
\pgfpathlineto{\pgfqpoint{1.212500in}{0.499444in}}%
\pgfpathlineto{\pgfqpoint{1.212500in}{0.499745in}}%
\pgfpathlineto{\pgfqpoint{1.149091in}{0.499745in}}%
\pgfpathlineto{\pgfqpoint{1.149091in}{0.499444in}}%
\pgfpathclose%
\pgfusepath{fill}%
\end{pgfscope}%
\begin{pgfscope}%
\pgfpathrectangle{\pgfqpoint{0.515000in}{0.499444in}}{\pgfqpoint{3.487500in}{1.155000in}}%
\pgfusepath{clip}%
\pgfsetbuttcap%
\pgfsetmiterjoin%
\definecolor{currentfill}{rgb}{0.000000,0.000000,0.000000}%
\pgfsetfillcolor{currentfill}%
\pgfsetlinewidth{0.000000pt}%
\definecolor{currentstroke}{rgb}{0.000000,0.000000,0.000000}%
\pgfsetstrokecolor{currentstroke}%
\pgfsetstrokeopacity{0.000000}%
\pgfsetdash{}{0pt}%
\pgfpathmoveto{\pgfqpoint{1.307614in}{0.499444in}}%
\pgfpathlineto{\pgfqpoint{1.371023in}{0.499444in}}%
\pgfpathlineto{\pgfqpoint{1.371023in}{0.585039in}}%
\pgfpathlineto{\pgfqpoint{1.307614in}{0.585039in}}%
\pgfpathlineto{\pgfqpoint{1.307614in}{0.499444in}}%
\pgfpathclose%
\pgfusepath{fill}%
\end{pgfscope}%
\begin{pgfscope}%
\pgfpathrectangle{\pgfqpoint{0.515000in}{0.499444in}}{\pgfqpoint{3.487500in}{1.155000in}}%
\pgfusepath{clip}%
\pgfsetbuttcap%
\pgfsetmiterjoin%
\definecolor{currentfill}{rgb}{0.000000,0.000000,0.000000}%
\pgfsetfillcolor{currentfill}%
\pgfsetlinewidth{0.000000pt}%
\definecolor{currentstroke}{rgb}{0.000000,0.000000,0.000000}%
\pgfsetstrokecolor{currentstroke}%
\pgfsetstrokeopacity{0.000000}%
\pgfsetdash{}{0pt}%
\pgfpathmoveto{\pgfqpoint{1.466137in}{0.499444in}}%
\pgfpathlineto{\pgfqpoint{1.529546in}{0.499444in}}%
\pgfpathlineto{\pgfqpoint{1.529546in}{0.500079in}}%
\pgfpathlineto{\pgfqpoint{1.466137in}{0.500079in}}%
\pgfpathlineto{\pgfqpoint{1.466137in}{0.499444in}}%
\pgfpathclose%
\pgfusepath{fill}%
\end{pgfscope}%
\begin{pgfscope}%
\pgfpathrectangle{\pgfqpoint{0.515000in}{0.499444in}}{\pgfqpoint{3.487500in}{1.155000in}}%
\pgfusepath{clip}%
\pgfsetbuttcap%
\pgfsetmiterjoin%
\definecolor{currentfill}{rgb}{0.000000,0.000000,0.000000}%
\pgfsetfillcolor{currentfill}%
\pgfsetlinewidth{0.000000pt}%
\definecolor{currentstroke}{rgb}{0.000000,0.000000,0.000000}%
\pgfsetstrokecolor{currentstroke}%
\pgfsetstrokeopacity{0.000000}%
\pgfsetdash{}{0pt}%
\pgfpathmoveto{\pgfqpoint{1.624659in}{0.499444in}}%
\pgfpathlineto{\pgfqpoint{1.688068in}{0.499444in}}%
\pgfpathlineto{\pgfqpoint{1.688068in}{0.615239in}}%
\pgfpathlineto{\pgfqpoint{1.624659in}{0.615239in}}%
\pgfpathlineto{\pgfqpoint{1.624659in}{0.499444in}}%
\pgfpathclose%
\pgfusepath{fill}%
\end{pgfscope}%
\begin{pgfscope}%
\pgfpathrectangle{\pgfqpoint{0.515000in}{0.499444in}}{\pgfqpoint{3.487500in}{1.155000in}}%
\pgfusepath{clip}%
\pgfsetbuttcap%
\pgfsetmiterjoin%
\definecolor{currentfill}{rgb}{0.000000,0.000000,0.000000}%
\pgfsetfillcolor{currentfill}%
\pgfsetlinewidth{0.000000pt}%
\definecolor{currentstroke}{rgb}{0.000000,0.000000,0.000000}%
\pgfsetstrokecolor{currentstroke}%
\pgfsetstrokeopacity{0.000000}%
\pgfsetdash{}{0pt}%
\pgfpathmoveto{\pgfqpoint{1.783182in}{0.499444in}}%
\pgfpathlineto{\pgfqpoint{1.846591in}{0.499444in}}%
\pgfpathlineto{\pgfqpoint{1.846591in}{0.500403in}}%
\pgfpathlineto{\pgfqpoint{1.783182in}{0.500403in}}%
\pgfpathlineto{\pgfqpoint{1.783182in}{0.499444in}}%
\pgfpathclose%
\pgfusepath{fill}%
\end{pgfscope}%
\begin{pgfscope}%
\pgfpathrectangle{\pgfqpoint{0.515000in}{0.499444in}}{\pgfqpoint{3.487500in}{1.155000in}}%
\pgfusepath{clip}%
\pgfsetbuttcap%
\pgfsetmiterjoin%
\definecolor{currentfill}{rgb}{0.000000,0.000000,0.000000}%
\pgfsetfillcolor{currentfill}%
\pgfsetlinewidth{0.000000pt}%
\definecolor{currentstroke}{rgb}{0.000000,0.000000,0.000000}%
\pgfsetstrokecolor{currentstroke}%
\pgfsetstrokeopacity{0.000000}%
\pgfsetdash{}{0pt}%
\pgfpathmoveto{\pgfqpoint{1.941705in}{0.499444in}}%
\pgfpathlineto{\pgfqpoint{2.005114in}{0.499444in}}%
\pgfpathlineto{\pgfqpoint{2.005114in}{0.638373in}}%
\pgfpathlineto{\pgfqpoint{1.941705in}{0.638373in}}%
\pgfpathlineto{\pgfqpoint{1.941705in}{0.499444in}}%
\pgfpathclose%
\pgfusepath{fill}%
\end{pgfscope}%
\begin{pgfscope}%
\pgfpathrectangle{\pgfqpoint{0.515000in}{0.499444in}}{\pgfqpoint{3.487500in}{1.155000in}}%
\pgfusepath{clip}%
\pgfsetbuttcap%
\pgfsetmiterjoin%
\definecolor{currentfill}{rgb}{0.000000,0.000000,0.000000}%
\pgfsetfillcolor{currentfill}%
\pgfsetlinewidth{0.000000pt}%
\definecolor{currentstroke}{rgb}{0.000000,0.000000,0.000000}%
\pgfsetstrokecolor{currentstroke}%
\pgfsetstrokeopacity{0.000000}%
\pgfsetdash{}{0pt}%
\pgfpathmoveto{\pgfqpoint{2.100228in}{0.499444in}}%
\pgfpathlineto{\pgfqpoint{2.163637in}{0.499444in}}%
\pgfpathlineto{\pgfqpoint{2.163637in}{0.500648in}}%
\pgfpathlineto{\pgfqpoint{2.100228in}{0.500648in}}%
\pgfpathlineto{\pgfqpoint{2.100228in}{0.499444in}}%
\pgfpathclose%
\pgfusepath{fill}%
\end{pgfscope}%
\begin{pgfscope}%
\pgfpathrectangle{\pgfqpoint{0.515000in}{0.499444in}}{\pgfqpoint{3.487500in}{1.155000in}}%
\pgfusepath{clip}%
\pgfsetbuttcap%
\pgfsetmiterjoin%
\definecolor{currentfill}{rgb}{0.000000,0.000000,0.000000}%
\pgfsetfillcolor{currentfill}%
\pgfsetlinewidth{0.000000pt}%
\definecolor{currentstroke}{rgb}{0.000000,0.000000,0.000000}%
\pgfsetstrokecolor{currentstroke}%
\pgfsetstrokeopacity{0.000000}%
\pgfsetdash{}{0pt}%
\pgfpathmoveto{\pgfqpoint{2.258750in}{0.499444in}}%
\pgfpathlineto{\pgfqpoint{2.322159in}{0.499444in}}%
\pgfpathlineto{\pgfqpoint{2.322159in}{0.653651in}}%
\pgfpathlineto{\pgfqpoint{2.258750in}{0.653651in}}%
\pgfpathlineto{\pgfqpoint{2.258750in}{0.499444in}}%
\pgfpathclose%
\pgfusepath{fill}%
\end{pgfscope}%
\begin{pgfscope}%
\pgfpathrectangle{\pgfqpoint{0.515000in}{0.499444in}}{\pgfqpoint{3.487500in}{1.155000in}}%
\pgfusepath{clip}%
\pgfsetbuttcap%
\pgfsetmiterjoin%
\definecolor{currentfill}{rgb}{0.000000,0.000000,0.000000}%
\pgfsetfillcolor{currentfill}%
\pgfsetlinewidth{0.000000pt}%
\definecolor{currentstroke}{rgb}{0.000000,0.000000,0.000000}%
\pgfsetstrokecolor{currentstroke}%
\pgfsetstrokeopacity{0.000000}%
\pgfsetdash{}{0pt}%
\pgfpathmoveto{\pgfqpoint{2.417273in}{0.499444in}}%
\pgfpathlineto{\pgfqpoint{2.480682in}{0.499444in}}%
\pgfpathlineto{\pgfqpoint{2.480682in}{0.500971in}}%
\pgfpathlineto{\pgfqpoint{2.417273in}{0.500971in}}%
\pgfpathlineto{\pgfqpoint{2.417273in}{0.499444in}}%
\pgfpathclose%
\pgfusepath{fill}%
\end{pgfscope}%
\begin{pgfscope}%
\pgfpathrectangle{\pgfqpoint{0.515000in}{0.499444in}}{\pgfqpoint{3.487500in}{1.155000in}}%
\pgfusepath{clip}%
\pgfsetbuttcap%
\pgfsetmiterjoin%
\definecolor{currentfill}{rgb}{0.000000,0.000000,0.000000}%
\pgfsetfillcolor{currentfill}%
\pgfsetlinewidth{0.000000pt}%
\definecolor{currentstroke}{rgb}{0.000000,0.000000,0.000000}%
\pgfsetstrokecolor{currentstroke}%
\pgfsetstrokeopacity{0.000000}%
\pgfsetdash{}{0pt}%
\pgfpathmoveto{\pgfqpoint{2.575796in}{0.499444in}}%
\pgfpathlineto{\pgfqpoint{2.639205in}{0.499444in}}%
\pgfpathlineto{\pgfqpoint{2.639205in}{0.660070in}}%
\pgfpathlineto{\pgfqpoint{2.575796in}{0.660070in}}%
\pgfpathlineto{\pgfqpoint{2.575796in}{0.499444in}}%
\pgfpathclose%
\pgfusepath{fill}%
\end{pgfscope}%
\begin{pgfscope}%
\pgfpathrectangle{\pgfqpoint{0.515000in}{0.499444in}}{\pgfqpoint{3.487500in}{1.155000in}}%
\pgfusepath{clip}%
\pgfsetbuttcap%
\pgfsetmiterjoin%
\definecolor{currentfill}{rgb}{0.000000,0.000000,0.000000}%
\pgfsetfillcolor{currentfill}%
\pgfsetlinewidth{0.000000pt}%
\definecolor{currentstroke}{rgb}{0.000000,0.000000,0.000000}%
\pgfsetstrokecolor{currentstroke}%
\pgfsetstrokeopacity{0.000000}%
\pgfsetdash{}{0pt}%
\pgfpathmoveto{\pgfqpoint{2.734318in}{0.499444in}}%
\pgfpathlineto{\pgfqpoint{2.797728in}{0.499444in}}%
\pgfpathlineto{\pgfqpoint{2.797728in}{0.500859in}}%
\pgfpathlineto{\pgfqpoint{2.734318in}{0.500859in}}%
\pgfpathlineto{\pgfqpoint{2.734318in}{0.499444in}}%
\pgfpathclose%
\pgfusepath{fill}%
\end{pgfscope}%
\begin{pgfscope}%
\pgfpathrectangle{\pgfqpoint{0.515000in}{0.499444in}}{\pgfqpoint{3.487500in}{1.155000in}}%
\pgfusepath{clip}%
\pgfsetbuttcap%
\pgfsetmiterjoin%
\definecolor{currentfill}{rgb}{0.000000,0.000000,0.000000}%
\pgfsetfillcolor{currentfill}%
\pgfsetlinewidth{0.000000pt}%
\definecolor{currentstroke}{rgb}{0.000000,0.000000,0.000000}%
\pgfsetstrokecolor{currentstroke}%
\pgfsetstrokeopacity{0.000000}%
\pgfsetdash{}{0pt}%
\pgfpathmoveto{\pgfqpoint{2.892841in}{0.499444in}}%
\pgfpathlineto{\pgfqpoint{2.956250in}{0.499444in}}%
\pgfpathlineto{\pgfqpoint{2.956250in}{0.656939in}}%
\pgfpathlineto{\pgfqpoint{2.892841in}{0.656939in}}%
\pgfpathlineto{\pgfqpoint{2.892841in}{0.499444in}}%
\pgfpathclose%
\pgfusepath{fill}%
\end{pgfscope}%
\begin{pgfscope}%
\pgfpathrectangle{\pgfqpoint{0.515000in}{0.499444in}}{\pgfqpoint{3.487500in}{1.155000in}}%
\pgfusepath{clip}%
\pgfsetbuttcap%
\pgfsetmiterjoin%
\definecolor{currentfill}{rgb}{0.000000,0.000000,0.000000}%
\pgfsetfillcolor{currentfill}%
\pgfsetlinewidth{0.000000pt}%
\definecolor{currentstroke}{rgb}{0.000000,0.000000,0.000000}%
\pgfsetstrokecolor{currentstroke}%
\pgfsetstrokeopacity{0.000000}%
\pgfsetdash{}{0pt}%
\pgfpathmoveto{\pgfqpoint{3.051364in}{0.499444in}}%
\pgfpathlineto{\pgfqpoint{3.114773in}{0.499444in}}%
\pgfpathlineto{\pgfqpoint{3.114773in}{0.500826in}}%
\pgfpathlineto{\pgfqpoint{3.051364in}{0.500826in}}%
\pgfpathlineto{\pgfqpoint{3.051364in}{0.499444in}}%
\pgfpathclose%
\pgfusepath{fill}%
\end{pgfscope}%
\begin{pgfscope}%
\pgfpathrectangle{\pgfqpoint{0.515000in}{0.499444in}}{\pgfqpoint{3.487500in}{1.155000in}}%
\pgfusepath{clip}%
\pgfsetbuttcap%
\pgfsetmiterjoin%
\definecolor{currentfill}{rgb}{0.000000,0.000000,0.000000}%
\pgfsetfillcolor{currentfill}%
\pgfsetlinewidth{0.000000pt}%
\definecolor{currentstroke}{rgb}{0.000000,0.000000,0.000000}%
\pgfsetstrokecolor{currentstroke}%
\pgfsetstrokeopacity{0.000000}%
\pgfsetdash{}{0pt}%
\pgfpathmoveto{\pgfqpoint{3.209887in}{0.499444in}}%
\pgfpathlineto{\pgfqpoint{3.273296in}{0.499444in}}%
\pgfpathlineto{\pgfqpoint{3.273296in}{0.645773in}}%
\pgfpathlineto{\pgfqpoint{3.209887in}{0.645773in}}%
\pgfpathlineto{\pgfqpoint{3.209887in}{0.499444in}}%
\pgfpathclose%
\pgfusepath{fill}%
\end{pgfscope}%
\begin{pgfscope}%
\pgfpathrectangle{\pgfqpoint{0.515000in}{0.499444in}}{\pgfqpoint{3.487500in}{1.155000in}}%
\pgfusepath{clip}%
\pgfsetbuttcap%
\pgfsetmiterjoin%
\definecolor{currentfill}{rgb}{0.000000,0.000000,0.000000}%
\pgfsetfillcolor{currentfill}%
\pgfsetlinewidth{0.000000pt}%
\definecolor{currentstroke}{rgb}{0.000000,0.000000,0.000000}%
\pgfsetstrokecolor{currentstroke}%
\pgfsetstrokeopacity{0.000000}%
\pgfsetdash{}{0pt}%
\pgfpathmoveto{\pgfqpoint{3.368409in}{0.499444in}}%
\pgfpathlineto{\pgfqpoint{3.431818in}{0.499444in}}%
\pgfpathlineto{\pgfqpoint{3.431818in}{0.500793in}}%
\pgfpathlineto{\pgfqpoint{3.368409in}{0.500793in}}%
\pgfpathlineto{\pgfqpoint{3.368409in}{0.499444in}}%
\pgfpathclose%
\pgfusepath{fill}%
\end{pgfscope}%
\begin{pgfscope}%
\pgfpathrectangle{\pgfqpoint{0.515000in}{0.499444in}}{\pgfqpoint{3.487500in}{1.155000in}}%
\pgfusepath{clip}%
\pgfsetbuttcap%
\pgfsetmiterjoin%
\definecolor{currentfill}{rgb}{0.000000,0.000000,0.000000}%
\pgfsetfillcolor{currentfill}%
\pgfsetlinewidth{0.000000pt}%
\definecolor{currentstroke}{rgb}{0.000000,0.000000,0.000000}%
\pgfsetstrokecolor{currentstroke}%
\pgfsetstrokeopacity{0.000000}%
\pgfsetdash{}{0pt}%
\pgfpathmoveto{\pgfqpoint{3.526932in}{0.499444in}}%
\pgfpathlineto{\pgfqpoint{3.590341in}{0.499444in}}%
\pgfpathlineto{\pgfqpoint{3.590341in}{0.625101in}}%
\pgfpathlineto{\pgfqpoint{3.526932in}{0.625101in}}%
\pgfpathlineto{\pgfqpoint{3.526932in}{0.499444in}}%
\pgfpathclose%
\pgfusepath{fill}%
\end{pgfscope}%
\begin{pgfscope}%
\pgfpathrectangle{\pgfqpoint{0.515000in}{0.499444in}}{\pgfqpoint{3.487500in}{1.155000in}}%
\pgfusepath{clip}%
\pgfsetbuttcap%
\pgfsetmiterjoin%
\definecolor{currentfill}{rgb}{0.000000,0.000000,0.000000}%
\pgfsetfillcolor{currentfill}%
\pgfsetlinewidth{0.000000pt}%
\definecolor{currentstroke}{rgb}{0.000000,0.000000,0.000000}%
\pgfsetstrokecolor{currentstroke}%
\pgfsetstrokeopacity{0.000000}%
\pgfsetdash{}{0pt}%
\pgfpathmoveto{\pgfqpoint{3.685455in}{0.499444in}}%
\pgfpathlineto{\pgfqpoint{3.748864in}{0.499444in}}%
\pgfpathlineto{\pgfqpoint{3.748864in}{0.500258in}}%
\pgfpathlineto{\pgfqpoint{3.685455in}{0.500258in}}%
\pgfpathlineto{\pgfqpoint{3.685455in}{0.499444in}}%
\pgfpathclose%
\pgfusepath{fill}%
\end{pgfscope}%
\begin{pgfscope}%
\pgfpathrectangle{\pgfqpoint{0.515000in}{0.499444in}}{\pgfqpoint{3.487500in}{1.155000in}}%
\pgfusepath{clip}%
\pgfsetbuttcap%
\pgfsetmiterjoin%
\definecolor{currentfill}{rgb}{0.000000,0.000000,0.000000}%
\pgfsetfillcolor{currentfill}%
\pgfsetlinewidth{0.000000pt}%
\definecolor{currentstroke}{rgb}{0.000000,0.000000,0.000000}%
\pgfsetstrokecolor{currentstroke}%
\pgfsetstrokeopacity{0.000000}%
\pgfsetdash{}{0pt}%
\pgfpathmoveto{\pgfqpoint{3.843978in}{0.499444in}}%
\pgfpathlineto{\pgfqpoint{3.907387in}{0.499444in}}%
\pgfpathlineto{\pgfqpoint{3.907387in}{0.580392in}}%
\pgfpathlineto{\pgfqpoint{3.843978in}{0.580392in}}%
\pgfpathlineto{\pgfqpoint{3.843978in}{0.499444in}}%
\pgfpathclose%
\pgfusepath{fill}%
\end{pgfscope}%
\begin{pgfscope}%
\pgfsetbuttcap%
\pgfsetroundjoin%
\definecolor{currentfill}{rgb}{0.000000,0.000000,0.000000}%
\pgfsetfillcolor{currentfill}%
\pgfsetlinewidth{0.803000pt}%
\definecolor{currentstroke}{rgb}{0.000000,0.000000,0.000000}%
\pgfsetstrokecolor{currentstroke}%
\pgfsetdash{}{0pt}%
\pgfsys@defobject{currentmarker}{\pgfqpoint{0.000000in}{-0.048611in}}{\pgfqpoint{0.000000in}{0.000000in}}{%
\pgfpathmoveto{\pgfqpoint{0.000000in}{0.000000in}}%
\pgfpathlineto{\pgfqpoint{0.000000in}{-0.048611in}}%
\pgfusepath{stroke,fill}%
}%
\begin{pgfscope}%
\pgfsys@transformshift{0.515000in}{0.499444in}%
\pgfsys@useobject{currentmarker}{}%
\end{pgfscope}%
\end{pgfscope}%
\begin{pgfscope}%
\pgfsetbuttcap%
\pgfsetroundjoin%
\definecolor{currentfill}{rgb}{0.000000,0.000000,0.000000}%
\pgfsetfillcolor{currentfill}%
\pgfsetlinewidth{0.803000pt}%
\definecolor{currentstroke}{rgb}{0.000000,0.000000,0.000000}%
\pgfsetstrokecolor{currentstroke}%
\pgfsetdash{}{0pt}%
\pgfsys@defobject{currentmarker}{\pgfqpoint{0.000000in}{-0.048611in}}{\pgfqpoint{0.000000in}{0.000000in}}{%
\pgfpathmoveto{\pgfqpoint{0.000000in}{0.000000in}}%
\pgfpathlineto{\pgfqpoint{0.000000in}{-0.048611in}}%
\pgfusepath{stroke,fill}%
}%
\begin{pgfscope}%
\pgfsys@transformshift{0.673523in}{0.499444in}%
\pgfsys@useobject{currentmarker}{}%
\end{pgfscope}%
\end{pgfscope}%
\begin{pgfscope}%
\definecolor{textcolor}{rgb}{0.000000,0.000000,0.000000}%
\pgfsetstrokecolor{textcolor}%
\pgfsetfillcolor{textcolor}%
\pgftext[x=0.673523in,y=0.402222in,,top]{\color{textcolor}\rmfamily\fontsize{10.000000}{12.000000}\selectfont 0.0}%
\end{pgfscope}%
\begin{pgfscope}%
\pgfsetbuttcap%
\pgfsetroundjoin%
\definecolor{currentfill}{rgb}{0.000000,0.000000,0.000000}%
\pgfsetfillcolor{currentfill}%
\pgfsetlinewidth{0.803000pt}%
\definecolor{currentstroke}{rgb}{0.000000,0.000000,0.000000}%
\pgfsetstrokecolor{currentstroke}%
\pgfsetdash{}{0pt}%
\pgfsys@defobject{currentmarker}{\pgfqpoint{0.000000in}{-0.048611in}}{\pgfqpoint{0.000000in}{0.000000in}}{%
\pgfpathmoveto{\pgfqpoint{0.000000in}{0.000000in}}%
\pgfpathlineto{\pgfqpoint{0.000000in}{-0.048611in}}%
\pgfusepath{stroke,fill}%
}%
\begin{pgfscope}%
\pgfsys@transformshift{0.832046in}{0.499444in}%
\pgfsys@useobject{currentmarker}{}%
\end{pgfscope}%
\end{pgfscope}%
\begin{pgfscope}%
\pgfsetbuttcap%
\pgfsetroundjoin%
\definecolor{currentfill}{rgb}{0.000000,0.000000,0.000000}%
\pgfsetfillcolor{currentfill}%
\pgfsetlinewidth{0.803000pt}%
\definecolor{currentstroke}{rgb}{0.000000,0.000000,0.000000}%
\pgfsetstrokecolor{currentstroke}%
\pgfsetdash{}{0pt}%
\pgfsys@defobject{currentmarker}{\pgfqpoint{0.000000in}{-0.048611in}}{\pgfqpoint{0.000000in}{0.000000in}}{%
\pgfpathmoveto{\pgfqpoint{0.000000in}{0.000000in}}%
\pgfpathlineto{\pgfqpoint{0.000000in}{-0.048611in}}%
\pgfusepath{stroke,fill}%
}%
\begin{pgfscope}%
\pgfsys@transformshift{0.990568in}{0.499444in}%
\pgfsys@useobject{currentmarker}{}%
\end{pgfscope}%
\end{pgfscope}%
\begin{pgfscope}%
\definecolor{textcolor}{rgb}{0.000000,0.000000,0.000000}%
\pgfsetstrokecolor{textcolor}%
\pgfsetfillcolor{textcolor}%
\pgftext[x=0.990568in,y=0.402222in,,top]{\color{textcolor}\rmfamily\fontsize{10.000000}{12.000000}\selectfont 0.1}%
\end{pgfscope}%
\begin{pgfscope}%
\pgfsetbuttcap%
\pgfsetroundjoin%
\definecolor{currentfill}{rgb}{0.000000,0.000000,0.000000}%
\pgfsetfillcolor{currentfill}%
\pgfsetlinewidth{0.803000pt}%
\definecolor{currentstroke}{rgb}{0.000000,0.000000,0.000000}%
\pgfsetstrokecolor{currentstroke}%
\pgfsetdash{}{0pt}%
\pgfsys@defobject{currentmarker}{\pgfqpoint{0.000000in}{-0.048611in}}{\pgfqpoint{0.000000in}{0.000000in}}{%
\pgfpathmoveto{\pgfqpoint{0.000000in}{0.000000in}}%
\pgfpathlineto{\pgfqpoint{0.000000in}{-0.048611in}}%
\pgfusepath{stroke,fill}%
}%
\begin{pgfscope}%
\pgfsys@transformshift{1.149091in}{0.499444in}%
\pgfsys@useobject{currentmarker}{}%
\end{pgfscope}%
\end{pgfscope}%
\begin{pgfscope}%
\pgfsetbuttcap%
\pgfsetroundjoin%
\definecolor{currentfill}{rgb}{0.000000,0.000000,0.000000}%
\pgfsetfillcolor{currentfill}%
\pgfsetlinewidth{0.803000pt}%
\definecolor{currentstroke}{rgb}{0.000000,0.000000,0.000000}%
\pgfsetstrokecolor{currentstroke}%
\pgfsetdash{}{0pt}%
\pgfsys@defobject{currentmarker}{\pgfqpoint{0.000000in}{-0.048611in}}{\pgfqpoint{0.000000in}{0.000000in}}{%
\pgfpathmoveto{\pgfqpoint{0.000000in}{0.000000in}}%
\pgfpathlineto{\pgfqpoint{0.000000in}{-0.048611in}}%
\pgfusepath{stroke,fill}%
}%
\begin{pgfscope}%
\pgfsys@transformshift{1.307614in}{0.499444in}%
\pgfsys@useobject{currentmarker}{}%
\end{pgfscope}%
\end{pgfscope}%
\begin{pgfscope}%
\definecolor{textcolor}{rgb}{0.000000,0.000000,0.000000}%
\pgfsetstrokecolor{textcolor}%
\pgfsetfillcolor{textcolor}%
\pgftext[x=1.307614in,y=0.402222in,,top]{\color{textcolor}\rmfamily\fontsize{10.000000}{12.000000}\selectfont 0.2}%
\end{pgfscope}%
\begin{pgfscope}%
\pgfsetbuttcap%
\pgfsetroundjoin%
\definecolor{currentfill}{rgb}{0.000000,0.000000,0.000000}%
\pgfsetfillcolor{currentfill}%
\pgfsetlinewidth{0.803000pt}%
\definecolor{currentstroke}{rgb}{0.000000,0.000000,0.000000}%
\pgfsetstrokecolor{currentstroke}%
\pgfsetdash{}{0pt}%
\pgfsys@defobject{currentmarker}{\pgfqpoint{0.000000in}{-0.048611in}}{\pgfqpoint{0.000000in}{0.000000in}}{%
\pgfpathmoveto{\pgfqpoint{0.000000in}{0.000000in}}%
\pgfpathlineto{\pgfqpoint{0.000000in}{-0.048611in}}%
\pgfusepath{stroke,fill}%
}%
\begin{pgfscope}%
\pgfsys@transformshift{1.466137in}{0.499444in}%
\pgfsys@useobject{currentmarker}{}%
\end{pgfscope}%
\end{pgfscope}%
\begin{pgfscope}%
\pgfsetbuttcap%
\pgfsetroundjoin%
\definecolor{currentfill}{rgb}{0.000000,0.000000,0.000000}%
\pgfsetfillcolor{currentfill}%
\pgfsetlinewidth{0.803000pt}%
\definecolor{currentstroke}{rgb}{0.000000,0.000000,0.000000}%
\pgfsetstrokecolor{currentstroke}%
\pgfsetdash{}{0pt}%
\pgfsys@defobject{currentmarker}{\pgfqpoint{0.000000in}{-0.048611in}}{\pgfqpoint{0.000000in}{0.000000in}}{%
\pgfpathmoveto{\pgfqpoint{0.000000in}{0.000000in}}%
\pgfpathlineto{\pgfqpoint{0.000000in}{-0.048611in}}%
\pgfusepath{stroke,fill}%
}%
\begin{pgfscope}%
\pgfsys@transformshift{1.624659in}{0.499444in}%
\pgfsys@useobject{currentmarker}{}%
\end{pgfscope}%
\end{pgfscope}%
\begin{pgfscope}%
\definecolor{textcolor}{rgb}{0.000000,0.000000,0.000000}%
\pgfsetstrokecolor{textcolor}%
\pgfsetfillcolor{textcolor}%
\pgftext[x=1.624659in,y=0.402222in,,top]{\color{textcolor}\rmfamily\fontsize{10.000000}{12.000000}\selectfont 0.3}%
\end{pgfscope}%
\begin{pgfscope}%
\pgfsetbuttcap%
\pgfsetroundjoin%
\definecolor{currentfill}{rgb}{0.000000,0.000000,0.000000}%
\pgfsetfillcolor{currentfill}%
\pgfsetlinewidth{0.803000pt}%
\definecolor{currentstroke}{rgb}{0.000000,0.000000,0.000000}%
\pgfsetstrokecolor{currentstroke}%
\pgfsetdash{}{0pt}%
\pgfsys@defobject{currentmarker}{\pgfqpoint{0.000000in}{-0.048611in}}{\pgfqpoint{0.000000in}{0.000000in}}{%
\pgfpathmoveto{\pgfqpoint{0.000000in}{0.000000in}}%
\pgfpathlineto{\pgfqpoint{0.000000in}{-0.048611in}}%
\pgfusepath{stroke,fill}%
}%
\begin{pgfscope}%
\pgfsys@transformshift{1.783182in}{0.499444in}%
\pgfsys@useobject{currentmarker}{}%
\end{pgfscope}%
\end{pgfscope}%
\begin{pgfscope}%
\pgfsetbuttcap%
\pgfsetroundjoin%
\definecolor{currentfill}{rgb}{0.000000,0.000000,0.000000}%
\pgfsetfillcolor{currentfill}%
\pgfsetlinewidth{0.803000pt}%
\definecolor{currentstroke}{rgb}{0.000000,0.000000,0.000000}%
\pgfsetstrokecolor{currentstroke}%
\pgfsetdash{}{0pt}%
\pgfsys@defobject{currentmarker}{\pgfqpoint{0.000000in}{-0.048611in}}{\pgfqpoint{0.000000in}{0.000000in}}{%
\pgfpathmoveto{\pgfqpoint{0.000000in}{0.000000in}}%
\pgfpathlineto{\pgfqpoint{0.000000in}{-0.048611in}}%
\pgfusepath{stroke,fill}%
}%
\begin{pgfscope}%
\pgfsys@transformshift{1.941705in}{0.499444in}%
\pgfsys@useobject{currentmarker}{}%
\end{pgfscope}%
\end{pgfscope}%
\begin{pgfscope}%
\definecolor{textcolor}{rgb}{0.000000,0.000000,0.000000}%
\pgfsetstrokecolor{textcolor}%
\pgfsetfillcolor{textcolor}%
\pgftext[x=1.941705in,y=0.402222in,,top]{\color{textcolor}\rmfamily\fontsize{10.000000}{12.000000}\selectfont 0.4}%
\end{pgfscope}%
\begin{pgfscope}%
\pgfsetbuttcap%
\pgfsetroundjoin%
\definecolor{currentfill}{rgb}{0.000000,0.000000,0.000000}%
\pgfsetfillcolor{currentfill}%
\pgfsetlinewidth{0.803000pt}%
\definecolor{currentstroke}{rgb}{0.000000,0.000000,0.000000}%
\pgfsetstrokecolor{currentstroke}%
\pgfsetdash{}{0pt}%
\pgfsys@defobject{currentmarker}{\pgfqpoint{0.000000in}{-0.048611in}}{\pgfqpoint{0.000000in}{0.000000in}}{%
\pgfpathmoveto{\pgfqpoint{0.000000in}{0.000000in}}%
\pgfpathlineto{\pgfqpoint{0.000000in}{-0.048611in}}%
\pgfusepath{stroke,fill}%
}%
\begin{pgfscope}%
\pgfsys@transformshift{2.100228in}{0.499444in}%
\pgfsys@useobject{currentmarker}{}%
\end{pgfscope}%
\end{pgfscope}%
\begin{pgfscope}%
\pgfsetbuttcap%
\pgfsetroundjoin%
\definecolor{currentfill}{rgb}{0.000000,0.000000,0.000000}%
\pgfsetfillcolor{currentfill}%
\pgfsetlinewidth{0.803000pt}%
\definecolor{currentstroke}{rgb}{0.000000,0.000000,0.000000}%
\pgfsetstrokecolor{currentstroke}%
\pgfsetdash{}{0pt}%
\pgfsys@defobject{currentmarker}{\pgfqpoint{0.000000in}{-0.048611in}}{\pgfqpoint{0.000000in}{0.000000in}}{%
\pgfpathmoveto{\pgfqpoint{0.000000in}{0.000000in}}%
\pgfpathlineto{\pgfqpoint{0.000000in}{-0.048611in}}%
\pgfusepath{stroke,fill}%
}%
\begin{pgfscope}%
\pgfsys@transformshift{2.258750in}{0.499444in}%
\pgfsys@useobject{currentmarker}{}%
\end{pgfscope}%
\end{pgfscope}%
\begin{pgfscope}%
\definecolor{textcolor}{rgb}{0.000000,0.000000,0.000000}%
\pgfsetstrokecolor{textcolor}%
\pgfsetfillcolor{textcolor}%
\pgftext[x=2.258750in,y=0.402222in,,top]{\color{textcolor}\rmfamily\fontsize{10.000000}{12.000000}\selectfont 0.5}%
\end{pgfscope}%
\begin{pgfscope}%
\pgfsetbuttcap%
\pgfsetroundjoin%
\definecolor{currentfill}{rgb}{0.000000,0.000000,0.000000}%
\pgfsetfillcolor{currentfill}%
\pgfsetlinewidth{0.803000pt}%
\definecolor{currentstroke}{rgb}{0.000000,0.000000,0.000000}%
\pgfsetstrokecolor{currentstroke}%
\pgfsetdash{}{0pt}%
\pgfsys@defobject{currentmarker}{\pgfqpoint{0.000000in}{-0.048611in}}{\pgfqpoint{0.000000in}{0.000000in}}{%
\pgfpathmoveto{\pgfqpoint{0.000000in}{0.000000in}}%
\pgfpathlineto{\pgfqpoint{0.000000in}{-0.048611in}}%
\pgfusepath{stroke,fill}%
}%
\begin{pgfscope}%
\pgfsys@transformshift{2.417273in}{0.499444in}%
\pgfsys@useobject{currentmarker}{}%
\end{pgfscope}%
\end{pgfscope}%
\begin{pgfscope}%
\pgfsetbuttcap%
\pgfsetroundjoin%
\definecolor{currentfill}{rgb}{0.000000,0.000000,0.000000}%
\pgfsetfillcolor{currentfill}%
\pgfsetlinewidth{0.803000pt}%
\definecolor{currentstroke}{rgb}{0.000000,0.000000,0.000000}%
\pgfsetstrokecolor{currentstroke}%
\pgfsetdash{}{0pt}%
\pgfsys@defobject{currentmarker}{\pgfqpoint{0.000000in}{-0.048611in}}{\pgfqpoint{0.000000in}{0.000000in}}{%
\pgfpathmoveto{\pgfqpoint{0.000000in}{0.000000in}}%
\pgfpathlineto{\pgfqpoint{0.000000in}{-0.048611in}}%
\pgfusepath{stroke,fill}%
}%
\begin{pgfscope}%
\pgfsys@transformshift{2.575796in}{0.499444in}%
\pgfsys@useobject{currentmarker}{}%
\end{pgfscope}%
\end{pgfscope}%
\begin{pgfscope}%
\definecolor{textcolor}{rgb}{0.000000,0.000000,0.000000}%
\pgfsetstrokecolor{textcolor}%
\pgfsetfillcolor{textcolor}%
\pgftext[x=2.575796in,y=0.402222in,,top]{\color{textcolor}\rmfamily\fontsize{10.000000}{12.000000}\selectfont 0.6}%
\end{pgfscope}%
\begin{pgfscope}%
\pgfsetbuttcap%
\pgfsetroundjoin%
\definecolor{currentfill}{rgb}{0.000000,0.000000,0.000000}%
\pgfsetfillcolor{currentfill}%
\pgfsetlinewidth{0.803000pt}%
\definecolor{currentstroke}{rgb}{0.000000,0.000000,0.000000}%
\pgfsetstrokecolor{currentstroke}%
\pgfsetdash{}{0pt}%
\pgfsys@defobject{currentmarker}{\pgfqpoint{0.000000in}{-0.048611in}}{\pgfqpoint{0.000000in}{0.000000in}}{%
\pgfpathmoveto{\pgfqpoint{0.000000in}{0.000000in}}%
\pgfpathlineto{\pgfqpoint{0.000000in}{-0.048611in}}%
\pgfusepath{stroke,fill}%
}%
\begin{pgfscope}%
\pgfsys@transformshift{2.734318in}{0.499444in}%
\pgfsys@useobject{currentmarker}{}%
\end{pgfscope}%
\end{pgfscope}%
\begin{pgfscope}%
\pgfsetbuttcap%
\pgfsetroundjoin%
\definecolor{currentfill}{rgb}{0.000000,0.000000,0.000000}%
\pgfsetfillcolor{currentfill}%
\pgfsetlinewidth{0.803000pt}%
\definecolor{currentstroke}{rgb}{0.000000,0.000000,0.000000}%
\pgfsetstrokecolor{currentstroke}%
\pgfsetdash{}{0pt}%
\pgfsys@defobject{currentmarker}{\pgfqpoint{0.000000in}{-0.048611in}}{\pgfqpoint{0.000000in}{0.000000in}}{%
\pgfpathmoveto{\pgfqpoint{0.000000in}{0.000000in}}%
\pgfpathlineto{\pgfqpoint{0.000000in}{-0.048611in}}%
\pgfusepath{stroke,fill}%
}%
\begin{pgfscope}%
\pgfsys@transformshift{2.892841in}{0.499444in}%
\pgfsys@useobject{currentmarker}{}%
\end{pgfscope}%
\end{pgfscope}%
\begin{pgfscope}%
\definecolor{textcolor}{rgb}{0.000000,0.000000,0.000000}%
\pgfsetstrokecolor{textcolor}%
\pgfsetfillcolor{textcolor}%
\pgftext[x=2.892841in,y=0.402222in,,top]{\color{textcolor}\rmfamily\fontsize{10.000000}{12.000000}\selectfont 0.7}%
\end{pgfscope}%
\begin{pgfscope}%
\pgfsetbuttcap%
\pgfsetroundjoin%
\definecolor{currentfill}{rgb}{0.000000,0.000000,0.000000}%
\pgfsetfillcolor{currentfill}%
\pgfsetlinewidth{0.803000pt}%
\definecolor{currentstroke}{rgb}{0.000000,0.000000,0.000000}%
\pgfsetstrokecolor{currentstroke}%
\pgfsetdash{}{0pt}%
\pgfsys@defobject{currentmarker}{\pgfqpoint{0.000000in}{-0.048611in}}{\pgfqpoint{0.000000in}{0.000000in}}{%
\pgfpathmoveto{\pgfqpoint{0.000000in}{0.000000in}}%
\pgfpathlineto{\pgfqpoint{0.000000in}{-0.048611in}}%
\pgfusepath{stroke,fill}%
}%
\begin{pgfscope}%
\pgfsys@transformshift{3.051364in}{0.499444in}%
\pgfsys@useobject{currentmarker}{}%
\end{pgfscope}%
\end{pgfscope}%
\begin{pgfscope}%
\pgfsetbuttcap%
\pgfsetroundjoin%
\definecolor{currentfill}{rgb}{0.000000,0.000000,0.000000}%
\pgfsetfillcolor{currentfill}%
\pgfsetlinewidth{0.803000pt}%
\definecolor{currentstroke}{rgb}{0.000000,0.000000,0.000000}%
\pgfsetstrokecolor{currentstroke}%
\pgfsetdash{}{0pt}%
\pgfsys@defobject{currentmarker}{\pgfqpoint{0.000000in}{-0.048611in}}{\pgfqpoint{0.000000in}{0.000000in}}{%
\pgfpathmoveto{\pgfqpoint{0.000000in}{0.000000in}}%
\pgfpathlineto{\pgfqpoint{0.000000in}{-0.048611in}}%
\pgfusepath{stroke,fill}%
}%
\begin{pgfscope}%
\pgfsys@transformshift{3.209887in}{0.499444in}%
\pgfsys@useobject{currentmarker}{}%
\end{pgfscope}%
\end{pgfscope}%
\begin{pgfscope}%
\definecolor{textcolor}{rgb}{0.000000,0.000000,0.000000}%
\pgfsetstrokecolor{textcolor}%
\pgfsetfillcolor{textcolor}%
\pgftext[x=3.209887in,y=0.402222in,,top]{\color{textcolor}\rmfamily\fontsize{10.000000}{12.000000}\selectfont 0.8}%
\end{pgfscope}%
\begin{pgfscope}%
\pgfsetbuttcap%
\pgfsetroundjoin%
\definecolor{currentfill}{rgb}{0.000000,0.000000,0.000000}%
\pgfsetfillcolor{currentfill}%
\pgfsetlinewidth{0.803000pt}%
\definecolor{currentstroke}{rgb}{0.000000,0.000000,0.000000}%
\pgfsetstrokecolor{currentstroke}%
\pgfsetdash{}{0pt}%
\pgfsys@defobject{currentmarker}{\pgfqpoint{0.000000in}{-0.048611in}}{\pgfqpoint{0.000000in}{0.000000in}}{%
\pgfpathmoveto{\pgfqpoint{0.000000in}{0.000000in}}%
\pgfpathlineto{\pgfqpoint{0.000000in}{-0.048611in}}%
\pgfusepath{stroke,fill}%
}%
\begin{pgfscope}%
\pgfsys@transformshift{3.368409in}{0.499444in}%
\pgfsys@useobject{currentmarker}{}%
\end{pgfscope}%
\end{pgfscope}%
\begin{pgfscope}%
\pgfsetbuttcap%
\pgfsetroundjoin%
\definecolor{currentfill}{rgb}{0.000000,0.000000,0.000000}%
\pgfsetfillcolor{currentfill}%
\pgfsetlinewidth{0.803000pt}%
\definecolor{currentstroke}{rgb}{0.000000,0.000000,0.000000}%
\pgfsetstrokecolor{currentstroke}%
\pgfsetdash{}{0pt}%
\pgfsys@defobject{currentmarker}{\pgfqpoint{0.000000in}{-0.048611in}}{\pgfqpoint{0.000000in}{0.000000in}}{%
\pgfpathmoveto{\pgfqpoint{0.000000in}{0.000000in}}%
\pgfpathlineto{\pgfqpoint{0.000000in}{-0.048611in}}%
\pgfusepath{stroke,fill}%
}%
\begin{pgfscope}%
\pgfsys@transformshift{3.526932in}{0.499444in}%
\pgfsys@useobject{currentmarker}{}%
\end{pgfscope}%
\end{pgfscope}%
\begin{pgfscope}%
\definecolor{textcolor}{rgb}{0.000000,0.000000,0.000000}%
\pgfsetstrokecolor{textcolor}%
\pgfsetfillcolor{textcolor}%
\pgftext[x=3.526932in,y=0.402222in,,top]{\color{textcolor}\rmfamily\fontsize{10.000000}{12.000000}\selectfont 0.9}%
\end{pgfscope}%
\begin{pgfscope}%
\pgfsetbuttcap%
\pgfsetroundjoin%
\definecolor{currentfill}{rgb}{0.000000,0.000000,0.000000}%
\pgfsetfillcolor{currentfill}%
\pgfsetlinewidth{0.803000pt}%
\definecolor{currentstroke}{rgb}{0.000000,0.000000,0.000000}%
\pgfsetstrokecolor{currentstroke}%
\pgfsetdash{}{0pt}%
\pgfsys@defobject{currentmarker}{\pgfqpoint{0.000000in}{-0.048611in}}{\pgfqpoint{0.000000in}{0.000000in}}{%
\pgfpathmoveto{\pgfqpoint{0.000000in}{0.000000in}}%
\pgfpathlineto{\pgfqpoint{0.000000in}{-0.048611in}}%
\pgfusepath{stroke,fill}%
}%
\begin{pgfscope}%
\pgfsys@transformshift{3.685455in}{0.499444in}%
\pgfsys@useobject{currentmarker}{}%
\end{pgfscope}%
\end{pgfscope}%
\begin{pgfscope}%
\pgfsetbuttcap%
\pgfsetroundjoin%
\definecolor{currentfill}{rgb}{0.000000,0.000000,0.000000}%
\pgfsetfillcolor{currentfill}%
\pgfsetlinewidth{0.803000pt}%
\definecolor{currentstroke}{rgb}{0.000000,0.000000,0.000000}%
\pgfsetstrokecolor{currentstroke}%
\pgfsetdash{}{0pt}%
\pgfsys@defobject{currentmarker}{\pgfqpoint{0.000000in}{-0.048611in}}{\pgfqpoint{0.000000in}{0.000000in}}{%
\pgfpathmoveto{\pgfqpoint{0.000000in}{0.000000in}}%
\pgfpathlineto{\pgfqpoint{0.000000in}{-0.048611in}}%
\pgfusepath{stroke,fill}%
}%
\begin{pgfscope}%
\pgfsys@transformshift{3.843978in}{0.499444in}%
\pgfsys@useobject{currentmarker}{}%
\end{pgfscope}%
\end{pgfscope}%
\begin{pgfscope}%
\definecolor{textcolor}{rgb}{0.000000,0.000000,0.000000}%
\pgfsetstrokecolor{textcolor}%
\pgfsetfillcolor{textcolor}%
\pgftext[x=3.843978in,y=0.402222in,,top]{\color{textcolor}\rmfamily\fontsize{10.000000}{12.000000}\selectfont 1.0}%
\end{pgfscope}%
\begin{pgfscope}%
\pgfsetbuttcap%
\pgfsetroundjoin%
\definecolor{currentfill}{rgb}{0.000000,0.000000,0.000000}%
\pgfsetfillcolor{currentfill}%
\pgfsetlinewidth{0.803000pt}%
\definecolor{currentstroke}{rgb}{0.000000,0.000000,0.000000}%
\pgfsetstrokecolor{currentstroke}%
\pgfsetdash{}{0pt}%
\pgfsys@defobject{currentmarker}{\pgfqpoint{0.000000in}{-0.048611in}}{\pgfqpoint{0.000000in}{0.000000in}}{%
\pgfpathmoveto{\pgfqpoint{0.000000in}{0.000000in}}%
\pgfpathlineto{\pgfqpoint{0.000000in}{-0.048611in}}%
\pgfusepath{stroke,fill}%
}%
\begin{pgfscope}%
\pgfsys@transformshift{4.002500in}{0.499444in}%
\pgfsys@useobject{currentmarker}{}%
\end{pgfscope}%
\end{pgfscope}%
\begin{pgfscope}%
\definecolor{textcolor}{rgb}{0.000000,0.000000,0.000000}%
\pgfsetstrokecolor{textcolor}%
\pgfsetfillcolor{textcolor}%
\pgftext[x=2.258750in,y=0.223333in,,top]{\color{textcolor}\rmfamily\fontsize{10.000000}{12.000000}\selectfont \(\displaystyle p\)}%
\end{pgfscope}%
\begin{pgfscope}%
\pgfsetbuttcap%
\pgfsetroundjoin%
\definecolor{currentfill}{rgb}{0.000000,0.000000,0.000000}%
\pgfsetfillcolor{currentfill}%
\pgfsetlinewidth{0.803000pt}%
\definecolor{currentstroke}{rgb}{0.000000,0.000000,0.000000}%
\pgfsetstrokecolor{currentstroke}%
\pgfsetdash{}{0pt}%
\pgfsys@defobject{currentmarker}{\pgfqpoint{-0.048611in}{0.000000in}}{\pgfqpoint{-0.000000in}{0.000000in}}{%
\pgfpathmoveto{\pgfqpoint{-0.000000in}{0.000000in}}%
\pgfpathlineto{\pgfqpoint{-0.048611in}{0.000000in}}%
\pgfusepath{stroke,fill}%
}%
\begin{pgfscope}%
\pgfsys@transformshift{0.515000in}{0.499444in}%
\pgfsys@useobject{currentmarker}{}%
\end{pgfscope}%
\end{pgfscope}%
\begin{pgfscope}%
\definecolor{textcolor}{rgb}{0.000000,0.000000,0.000000}%
\pgfsetstrokecolor{textcolor}%
\pgfsetfillcolor{textcolor}%
\pgftext[x=0.348333in, y=0.451250in, left, base]{\color{textcolor}\rmfamily\fontsize{10.000000}{12.000000}\selectfont \(\displaystyle {0}\)}%
\end{pgfscope}%
\begin{pgfscope}%
\pgfsetbuttcap%
\pgfsetroundjoin%
\definecolor{currentfill}{rgb}{0.000000,0.000000,0.000000}%
\pgfsetfillcolor{currentfill}%
\pgfsetlinewidth{0.803000pt}%
\definecolor{currentstroke}{rgb}{0.000000,0.000000,0.000000}%
\pgfsetstrokecolor{currentstroke}%
\pgfsetdash{}{0pt}%
\pgfsys@defobject{currentmarker}{\pgfqpoint{-0.048611in}{0.000000in}}{\pgfqpoint{-0.000000in}{0.000000in}}{%
\pgfpathmoveto{\pgfqpoint{-0.000000in}{0.000000in}}%
\pgfpathlineto{\pgfqpoint{-0.048611in}{0.000000in}}%
\pgfusepath{stroke,fill}%
}%
\begin{pgfscope}%
\pgfsys@transformshift{0.515000in}{0.897034in}%
\pgfsys@useobject{currentmarker}{}%
\end{pgfscope}%
\end{pgfscope}%
\begin{pgfscope}%
\definecolor{textcolor}{rgb}{0.000000,0.000000,0.000000}%
\pgfsetstrokecolor{textcolor}%
\pgfsetfillcolor{textcolor}%
\pgftext[x=0.348333in, y=0.848840in, left, base]{\color{textcolor}\rmfamily\fontsize{10.000000}{12.000000}\selectfont \(\displaystyle {5}\)}%
\end{pgfscope}%
\begin{pgfscope}%
\pgfsetbuttcap%
\pgfsetroundjoin%
\definecolor{currentfill}{rgb}{0.000000,0.000000,0.000000}%
\pgfsetfillcolor{currentfill}%
\pgfsetlinewidth{0.803000pt}%
\definecolor{currentstroke}{rgb}{0.000000,0.000000,0.000000}%
\pgfsetstrokecolor{currentstroke}%
\pgfsetdash{}{0pt}%
\pgfsys@defobject{currentmarker}{\pgfqpoint{-0.048611in}{0.000000in}}{\pgfqpoint{-0.000000in}{0.000000in}}{%
\pgfpathmoveto{\pgfqpoint{-0.000000in}{0.000000in}}%
\pgfpathlineto{\pgfqpoint{-0.048611in}{0.000000in}}%
\pgfusepath{stroke,fill}%
}%
\begin{pgfscope}%
\pgfsys@transformshift{0.515000in}{1.294625in}%
\pgfsys@useobject{currentmarker}{}%
\end{pgfscope}%
\end{pgfscope}%
\begin{pgfscope}%
\definecolor{textcolor}{rgb}{0.000000,0.000000,0.000000}%
\pgfsetstrokecolor{textcolor}%
\pgfsetfillcolor{textcolor}%
\pgftext[x=0.278889in, y=1.246430in, left, base]{\color{textcolor}\rmfamily\fontsize{10.000000}{12.000000}\selectfont \(\displaystyle {10}\)}%
\end{pgfscope}%
\begin{pgfscope}%
\definecolor{textcolor}{rgb}{0.000000,0.000000,0.000000}%
\pgfsetstrokecolor{textcolor}%
\pgfsetfillcolor{textcolor}%
\pgftext[x=0.223333in,y=1.076944in,,bottom,rotate=90.000000]{\color{textcolor}\rmfamily\fontsize{10.000000}{12.000000}\selectfont Percent of Data Set}%
\end{pgfscope}%
\begin{pgfscope}%
\pgfsetrectcap%
\pgfsetmiterjoin%
\pgfsetlinewidth{0.803000pt}%
\definecolor{currentstroke}{rgb}{0.000000,0.000000,0.000000}%
\pgfsetstrokecolor{currentstroke}%
\pgfsetdash{}{0pt}%
\pgfpathmoveto{\pgfqpoint{0.515000in}{0.499444in}}%
\pgfpathlineto{\pgfqpoint{0.515000in}{1.654444in}}%
\pgfusepath{stroke}%
\end{pgfscope}%
\begin{pgfscope}%
\pgfsetrectcap%
\pgfsetmiterjoin%
\pgfsetlinewidth{0.803000pt}%
\definecolor{currentstroke}{rgb}{0.000000,0.000000,0.000000}%
\pgfsetstrokecolor{currentstroke}%
\pgfsetdash{}{0pt}%
\pgfpathmoveto{\pgfqpoint{4.002500in}{0.499444in}}%
\pgfpathlineto{\pgfqpoint{4.002500in}{1.654444in}}%
\pgfusepath{stroke}%
\end{pgfscope}%
\begin{pgfscope}%
\pgfsetrectcap%
\pgfsetmiterjoin%
\pgfsetlinewidth{0.803000pt}%
\definecolor{currentstroke}{rgb}{0.000000,0.000000,0.000000}%
\pgfsetstrokecolor{currentstroke}%
\pgfsetdash{}{0pt}%
\pgfpathmoveto{\pgfqpoint{0.515000in}{0.499444in}}%
\pgfpathlineto{\pgfqpoint{4.002500in}{0.499444in}}%
\pgfusepath{stroke}%
\end{pgfscope}%
\begin{pgfscope}%
\pgfsetrectcap%
\pgfsetmiterjoin%
\pgfsetlinewidth{0.803000pt}%
\definecolor{currentstroke}{rgb}{0.000000,0.000000,0.000000}%
\pgfsetstrokecolor{currentstroke}%
\pgfsetdash{}{0pt}%
\pgfpathmoveto{\pgfqpoint{0.515000in}{1.654444in}}%
\pgfpathlineto{\pgfqpoint{4.002500in}{1.654444in}}%
\pgfusepath{stroke}%
\end{pgfscope}%
\begin{pgfscope}%
\pgfsetbuttcap%
\pgfsetmiterjoin%
\definecolor{currentfill}{rgb}{1.000000,1.000000,1.000000}%
\pgfsetfillcolor{currentfill}%
\pgfsetfillopacity{0.800000}%
\pgfsetlinewidth{1.003750pt}%
\definecolor{currentstroke}{rgb}{0.800000,0.800000,0.800000}%
\pgfsetstrokecolor{currentstroke}%
\pgfsetstrokeopacity{0.800000}%
\pgfsetdash{}{0pt}%
\pgfpathmoveto{\pgfqpoint{3.225556in}{1.154445in}}%
\pgfpathlineto{\pgfqpoint{3.905278in}{1.154445in}}%
\pgfpathquadraticcurveto{\pgfqpoint{3.933056in}{1.154445in}}{\pgfqpoint{3.933056in}{1.182222in}}%
\pgfpathlineto{\pgfqpoint{3.933056in}{1.557222in}}%
\pgfpathquadraticcurveto{\pgfqpoint{3.933056in}{1.585000in}}{\pgfqpoint{3.905278in}{1.585000in}}%
\pgfpathlineto{\pgfqpoint{3.225556in}{1.585000in}}%
\pgfpathquadraticcurveto{\pgfqpoint{3.197778in}{1.585000in}}{\pgfqpoint{3.197778in}{1.557222in}}%
\pgfpathlineto{\pgfqpoint{3.197778in}{1.182222in}}%
\pgfpathquadraticcurveto{\pgfqpoint{3.197778in}{1.154445in}}{\pgfqpoint{3.225556in}{1.154445in}}%
\pgfpathlineto{\pgfqpoint{3.225556in}{1.154445in}}%
\pgfpathclose%
\pgfusepath{stroke,fill}%
\end{pgfscope}%
\begin{pgfscope}%
\pgfsetbuttcap%
\pgfsetmiterjoin%
\pgfsetlinewidth{1.003750pt}%
\definecolor{currentstroke}{rgb}{0.000000,0.000000,0.000000}%
\pgfsetstrokecolor{currentstroke}%
\pgfsetdash{}{0pt}%
\pgfpathmoveto{\pgfqpoint{3.253334in}{1.432222in}}%
\pgfpathlineto{\pgfqpoint{3.531111in}{1.432222in}}%
\pgfpathlineto{\pgfqpoint{3.531111in}{1.529444in}}%
\pgfpathlineto{\pgfqpoint{3.253334in}{1.529444in}}%
\pgfpathlineto{\pgfqpoint{3.253334in}{1.432222in}}%
\pgfpathclose%
\pgfusepath{stroke}%
\end{pgfscope}%
\begin{pgfscope}%
\definecolor{textcolor}{rgb}{0.000000,0.000000,0.000000}%
\pgfsetstrokecolor{textcolor}%
\pgfsetfillcolor{textcolor}%
\pgftext[x=3.642223in,y=1.432222in,left,base]{\color{textcolor}\rmfamily\fontsize{10.000000}{12.000000}\selectfont Neg}%
\end{pgfscope}%
\begin{pgfscope}%
\pgfsetbuttcap%
\pgfsetmiterjoin%
\definecolor{currentfill}{rgb}{0.000000,0.000000,0.000000}%
\pgfsetfillcolor{currentfill}%
\pgfsetlinewidth{0.000000pt}%
\definecolor{currentstroke}{rgb}{0.000000,0.000000,0.000000}%
\pgfsetstrokecolor{currentstroke}%
\pgfsetstrokeopacity{0.000000}%
\pgfsetdash{}{0pt}%
\pgfpathmoveto{\pgfqpoint{3.253334in}{1.236944in}}%
\pgfpathlineto{\pgfqpoint{3.531111in}{1.236944in}}%
\pgfpathlineto{\pgfqpoint{3.531111in}{1.334167in}}%
\pgfpathlineto{\pgfqpoint{3.253334in}{1.334167in}}%
\pgfpathlineto{\pgfqpoint{3.253334in}{1.236944in}}%
\pgfpathclose%
\pgfusepath{fill}%
\end{pgfscope}%
\begin{pgfscope}%
\definecolor{textcolor}{rgb}{0.000000,0.000000,0.000000}%
\pgfsetstrokecolor{textcolor}%
\pgfsetfillcolor{textcolor}%
\pgftext[x=3.642223in,y=1.236944in,left,base]{\color{textcolor}\rmfamily\fontsize{10.000000}{12.000000}\selectfont Pos}%
\end{pgfscope}%
\end{pgfpicture}%
\makeatother%
\endgroup%
	
&
	\vskip 0pt
	\hfil ROC Curve
	
	%% Creator: Matplotlib, PGF backend
%%
%% To include the figure in your LaTeX document, write
%%   \input{<filename>.pgf}
%%
%% Make sure the required packages are loaded in your preamble
%%   \usepackage{pgf}
%%
%% Also ensure that all the required font packages are loaded; for instance,
%% the lmodern package is sometimes necessary when using math font.
%%   \usepackage{lmodern}
%%
%% Figures using additional raster images can only be included by \input if
%% they are in the same directory as the main LaTeX file. For loading figures
%% from other directories you can use the `import` package
%%   \usepackage{import}
%%
%% and then include the figures with
%%   \import{<path to file>}{<filename>.pgf}
%%
%% Matplotlib used the following preamble
%%   
%%   \usepackage{fontspec}
%%   \makeatletter\@ifpackageloaded{underscore}{}{\usepackage[strings]{underscore}}\makeatother
%%
\begingroup%
\makeatletter%
\begin{pgfpicture}%
\pgfpathrectangle{\pgfpointorigin}{\pgfqpoint{2.221861in}{1.754444in}}%
\pgfusepath{use as bounding box, clip}%
\begin{pgfscope}%
\pgfsetbuttcap%
\pgfsetmiterjoin%
\definecolor{currentfill}{rgb}{1.000000,1.000000,1.000000}%
\pgfsetfillcolor{currentfill}%
\pgfsetlinewidth{0.000000pt}%
\definecolor{currentstroke}{rgb}{1.000000,1.000000,1.000000}%
\pgfsetstrokecolor{currentstroke}%
\pgfsetdash{}{0pt}%
\pgfpathmoveto{\pgfqpoint{0.000000in}{0.000000in}}%
\pgfpathlineto{\pgfqpoint{2.221861in}{0.000000in}}%
\pgfpathlineto{\pgfqpoint{2.221861in}{1.754444in}}%
\pgfpathlineto{\pgfqpoint{0.000000in}{1.754444in}}%
\pgfpathlineto{\pgfqpoint{0.000000in}{0.000000in}}%
\pgfpathclose%
\pgfusepath{fill}%
\end{pgfscope}%
\begin{pgfscope}%
\pgfsetbuttcap%
\pgfsetmiterjoin%
\definecolor{currentfill}{rgb}{1.000000,1.000000,1.000000}%
\pgfsetfillcolor{currentfill}%
\pgfsetlinewidth{0.000000pt}%
\definecolor{currentstroke}{rgb}{0.000000,0.000000,0.000000}%
\pgfsetstrokecolor{currentstroke}%
\pgfsetstrokeopacity{0.000000}%
\pgfsetdash{}{0pt}%
\pgfpathmoveto{\pgfqpoint{0.553581in}{0.499444in}}%
\pgfpathlineto{\pgfqpoint{2.103581in}{0.499444in}}%
\pgfpathlineto{\pgfqpoint{2.103581in}{1.654444in}}%
\pgfpathlineto{\pgfqpoint{0.553581in}{1.654444in}}%
\pgfpathlineto{\pgfqpoint{0.553581in}{0.499444in}}%
\pgfpathclose%
\pgfusepath{fill}%
\end{pgfscope}%
\begin{pgfscope}%
\pgfsetbuttcap%
\pgfsetroundjoin%
\definecolor{currentfill}{rgb}{0.000000,0.000000,0.000000}%
\pgfsetfillcolor{currentfill}%
\pgfsetlinewidth{0.803000pt}%
\definecolor{currentstroke}{rgb}{0.000000,0.000000,0.000000}%
\pgfsetstrokecolor{currentstroke}%
\pgfsetdash{}{0pt}%
\pgfsys@defobject{currentmarker}{\pgfqpoint{0.000000in}{-0.048611in}}{\pgfqpoint{0.000000in}{0.000000in}}{%
\pgfpathmoveto{\pgfqpoint{0.000000in}{0.000000in}}%
\pgfpathlineto{\pgfqpoint{0.000000in}{-0.048611in}}%
\pgfusepath{stroke,fill}%
}%
\begin{pgfscope}%
\pgfsys@transformshift{0.624035in}{0.499444in}%
\pgfsys@useobject{currentmarker}{}%
\end{pgfscope}%
\end{pgfscope}%
\begin{pgfscope}%
\definecolor{textcolor}{rgb}{0.000000,0.000000,0.000000}%
\pgfsetstrokecolor{textcolor}%
\pgfsetfillcolor{textcolor}%
\pgftext[x=0.624035in,y=0.402222in,,top]{\color{textcolor}\rmfamily\fontsize{10.000000}{12.000000}\selectfont \(\displaystyle {0.0}\)}%
\end{pgfscope}%
\begin{pgfscope}%
\pgfsetbuttcap%
\pgfsetroundjoin%
\definecolor{currentfill}{rgb}{0.000000,0.000000,0.000000}%
\pgfsetfillcolor{currentfill}%
\pgfsetlinewidth{0.803000pt}%
\definecolor{currentstroke}{rgb}{0.000000,0.000000,0.000000}%
\pgfsetstrokecolor{currentstroke}%
\pgfsetdash{}{0pt}%
\pgfsys@defobject{currentmarker}{\pgfqpoint{0.000000in}{-0.048611in}}{\pgfqpoint{0.000000in}{0.000000in}}{%
\pgfpathmoveto{\pgfqpoint{0.000000in}{0.000000in}}%
\pgfpathlineto{\pgfqpoint{0.000000in}{-0.048611in}}%
\pgfusepath{stroke,fill}%
}%
\begin{pgfscope}%
\pgfsys@transformshift{1.328581in}{0.499444in}%
\pgfsys@useobject{currentmarker}{}%
\end{pgfscope}%
\end{pgfscope}%
\begin{pgfscope}%
\definecolor{textcolor}{rgb}{0.000000,0.000000,0.000000}%
\pgfsetstrokecolor{textcolor}%
\pgfsetfillcolor{textcolor}%
\pgftext[x=1.328581in,y=0.402222in,,top]{\color{textcolor}\rmfamily\fontsize{10.000000}{12.000000}\selectfont \(\displaystyle {0.5}\)}%
\end{pgfscope}%
\begin{pgfscope}%
\pgfsetbuttcap%
\pgfsetroundjoin%
\definecolor{currentfill}{rgb}{0.000000,0.000000,0.000000}%
\pgfsetfillcolor{currentfill}%
\pgfsetlinewidth{0.803000pt}%
\definecolor{currentstroke}{rgb}{0.000000,0.000000,0.000000}%
\pgfsetstrokecolor{currentstroke}%
\pgfsetdash{}{0pt}%
\pgfsys@defobject{currentmarker}{\pgfqpoint{0.000000in}{-0.048611in}}{\pgfqpoint{0.000000in}{0.000000in}}{%
\pgfpathmoveto{\pgfqpoint{0.000000in}{0.000000in}}%
\pgfpathlineto{\pgfqpoint{0.000000in}{-0.048611in}}%
\pgfusepath{stroke,fill}%
}%
\begin{pgfscope}%
\pgfsys@transformshift{2.033126in}{0.499444in}%
\pgfsys@useobject{currentmarker}{}%
\end{pgfscope}%
\end{pgfscope}%
\begin{pgfscope}%
\definecolor{textcolor}{rgb}{0.000000,0.000000,0.000000}%
\pgfsetstrokecolor{textcolor}%
\pgfsetfillcolor{textcolor}%
\pgftext[x=2.033126in,y=0.402222in,,top]{\color{textcolor}\rmfamily\fontsize{10.000000}{12.000000}\selectfont \(\displaystyle {1.0}\)}%
\end{pgfscope}%
\begin{pgfscope}%
\definecolor{textcolor}{rgb}{0.000000,0.000000,0.000000}%
\pgfsetstrokecolor{textcolor}%
\pgfsetfillcolor{textcolor}%
\pgftext[x=1.328581in,y=0.223333in,,top]{\color{textcolor}\rmfamily\fontsize{10.000000}{12.000000}\selectfont False positive rate}%
\end{pgfscope}%
\begin{pgfscope}%
\pgfsetbuttcap%
\pgfsetroundjoin%
\definecolor{currentfill}{rgb}{0.000000,0.000000,0.000000}%
\pgfsetfillcolor{currentfill}%
\pgfsetlinewidth{0.803000pt}%
\definecolor{currentstroke}{rgb}{0.000000,0.000000,0.000000}%
\pgfsetstrokecolor{currentstroke}%
\pgfsetdash{}{0pt}%
\pgfsys@defobject{currentmarker}{\pgfqpoint{-0.048611in}{0.000000in}}{\pgfqpoint{-0.000000in}{0.000000in}}{%
\pgfpathmoveto{\pgfqpoint{-0.000000in}{0.000000in}}%
\pgfpathlineto{\pgfqpoint{-0.048611in}{0.000000in}}%
\pgfusepath{stroke,fill}%
}%
\begin{pgfscope}%
\pgfsys@transformshift{0.553581in}{0.551944in}%
\pgfsys@useobject{currentmarker}{}%
\end{pgfscope}%
\end{pgfscope}%
\begin{pgfscope}%
\definecolor{textcolor}{rgb}{0.000000,0.000000,0.000000}%
\pgfsetstrokecolor{textcolor}%
\pgfsetfillcolor{textcolor}%
\pgftext[x=0.278889in, y=0.503750in, left, base]{\color{textcolor}\rmfamily\fontsize{10.000000}{12.000000}\selectfont \(\displaystyle {0.0}\)}%
\end{pgfscope}%
\begin{pgfscope}%
\pgfsetbuttcap%
\pgfsetroundjoin%
\definecolor{currentfill}{rgb}{0.000000,0.000000,0.000000}%
\pgfsetfillcolor{currentfill}%
\pgfsetlinewidth{0.803000pt}%
\definecolor{currentstroke}{rgb}{0.000000,0.000000,0.000000}%
\pgfsetstrokecolor{currentstroke}%
\pgfsetdash{}{0pt}%
\pgfsys@defobject{currentmarker}{\pgfqpoint{-0.048611in}{0.000000in}}{\pgfqpoint{-0.000000in}{0.000000in}}{%
\pgfpathmoveto{\pgfqpoint{-0.000000in}{0.000000in}}%
\pgfpathlineto{\pgfqpoint{-0.048611in}{0.000000in}}%
\pgfusepath{stroke,fill}%
}%
\begin{pgfscope}%
\pgfsys@transformshift{0.553581in}{1.076944in}%
\pgfsys@useobject{currentmarker}{}%
\end{pgfscope}%
\end{pgfscope}%
\begin{pgfscope}%
\definecolor{textcolor}{rgb}{0.000000,0.000000,0.000000}%
\pgfsetstrokecolor{textcolor}%
\pgfsetfillcolor{textcolor}%
\pgftext[x=0.278889in, y=1.028750in, left, base]{\color{textcolor}\rmfamily\fontsize{10.000000}{12.000000}\selectfont \(\displaystyle {0.5}\)}%
\end{pgfscope}%
\begin{pgfscope}%
\pgfsetbuttcap%
\pgfsetroundjoin%
\definecolor{currentfill}{rgb}{0.000000,0.000000,0.000000}%
\pgfsetfillcolor{currentfill}%
\pgfsetlinewidth{0.803000pt}%
\definecolor{currentstroke}{rgb}{0.000000,0.000000,0.000000}%
\pgfsetstrokecolor{currentstroke}%
\pgfsetdash{}{0pt}%
\pgfsys@defobject{currentmarker}{\pgfqpoint{-0.048611in}{0.000000in}}{\pgfqpoint{-0.000000in}{0.000000in}}{%
\pgfpathmoveto{\pgfqpoint{-0.000000in}{0.000000in}}%
\pgfpathlineto{\pgfqpoint{-0.048611in}{0.000000in}}%
\pgfusepath{stroke,fill}%
}%
\begin{pgfscope}%
\pgfsys@transformshift{0.553581in}{1.601944in}%
\pgfsys@useobject{currentmarker}{}%
\end{pgfscope}%
\end{pgfscope}%
\begin{pgfscope}%
\definecolor{textcolor}{rgb}{0.000000,0.000000,0.000000}%
\pgfsetstrokecolor{textcolor}%
\pgfsetfillcolor{textcolor}%
\pgftext[x=0.278889in, y=1.553750in, left, base]{\color{textcolor}\rmfamily\fontsize{10.000000}{12.000000}\selectfont \(\displaystyle {1.0}\)}%
\end{pgfscope}%
\begin{pgfscope}%
\definecolor{textcolor}{rgb}{0.000000,0.000000,0.000000}%
\pgfsetstrokecolor{textcolor}%
\pgfsetfillcolor{textcolor}%
\pgftext[x=0.223333in,y=1.076944in,,bottom,rotate=90.000000]{\color{textcolor}\rmfamily\fontsize{10.000000}{12.000000}\selectfont True positive rate}%
\end{pgfscope}%
\begin{pgfscope}%
\pgfpathrectangle{\pgfqpoint{0.553581in}{0.499444in}}{\pgfqpoint{1.550000in}{1.155000in}}%
\pgfusepath{clip}%
\pgfsetbuttcap%
\pgfsetroundjoin%
\pgfsetlinewidth{1.505625pt}%
\definecolor{currentstroke}{rgb}{0.000000,0.000000,0.000000}%
\pgfsetstrokecolor{currentstroke}%
\pgfsetdash{{5.550000pt}{2.400000pt}}{0.000000pt}%
\pgfpathmoveto{\pgfqpoint{0.624035in}{0.551944in}}%
\pgfpathlineto{\pgfqpoint{2.033126in}{1.601944in}}%
\pgfusepath{stroke}%
\end{pgfscope}%
\begin{pgfscope}%
\pgfpathrectangle{\pgfqpoint{0.553581in}{0.499444in}}{\pgfqpoint{1.550000in}{1.155000in}}%
\pgfusepath{clip}%
\pgfsetrectcap%
\pgfsetroundjoin%
\pgfsetlinewidth{1.505625pt}%
\definecolor{currentstroke}{rgb}{0.000000,0.000000,0.000000}%
\pgfsetstrokecolor{currentstroke}%
\pgfsetdash{}{0pt}%
\pgfpathmoveto{\pgfqpoint{0.624035in}{0.551944in}}%
\pgfpathlineto{\pgfqpoint{0.632162in}{0.620270in}}%
\pgfpathlineto{\pgfqpoint{0.653246in}{0.726406in}}%
\pgfpathlineto{\pgfqpoint{0.692344in}{0.849844in}}%
\pgfpathlineto{\pgfqpoint{0.758475in}{0.982641in}}%
\pgfpathlineto{\pgfqpoint{0.859637in}{1.118148in}}%
\pgfpathlineto{\pgfqpoint{1.000528in}{1.248020in}}%
\pgfpathlineto{\pgfqpoint{1.183112in}{1.364921in}}%
\pgfpathlineto{\pgfqpoint{1.399091in}{1.462219in}}%
\pgfpathlineto{\pgfqpoint{1.631623in}{1.534000in}}%
\pgfpathlineto{\pgfqpoint{1.856572in}{1.580712in}}%
\pgfpathlineto{\pgfqpoint{2.033126in}{1.601944in}}%
\pgfpathlineto{\pgfqpoint{2.033126in}{1.601944in}}%
\pgfusepath{stroke}%
\end{pgfscope}%
\begin{pgfscope}%
\pgfsetrectcap%
\pgfsetmiterjoin%
\pgfsetlinewidth{0.803000pt}%
\definecolor{currentstroke}{rgb}{0.000000,0.000000,0.000000}%
\pgfsetstrokecolor{currentstroke}%
\pgfsetdash{}{0pt}%
\pgfpathmoveto{\pgfqpoint{0.553581in}{0.499444in}}%
\pgfpathlineto{\pgfqpoint{0.553581in}{1.654444in}}%
\pgfusepath{stroke}%
\end{pgfscope}%
\begin{pgfscope}%
\pgfsetrectcap%
\pgfsetmiterjoin%
\pgfsetlinewidth{0.803000pt}%
\definecolor{currentstroke}{rgb}{0.000000,0.000000,0.000000}%
\pgfsetstrokecolor{currentstroke}%
\pgfsetdash{}{0pt}%
\pgfpathmoveto{\pgfqpoint{2.103581in}{0.499444in}}%
\pgfpathlineto{\pgfqpoint{2.103581in}{1.654444in}}%
\pgfusepath{stroke}%
\end{pgfscope}%
\begin{pgfscope}%
\pgfsetrectcap%
\pgfsetmiterjoin%
\pgfsetlinewidth{0.803000pt}%
\definecolor{currentstroke}{rgb}{0.000000,0.000000,0.000000}%
\pgfsetstrokecolor{currentstroke}%
\pgfsetdash{}{0pt}%
\pgfpathmoveto{\pgfqpoint{0.553581in}{0.499444in}}%
\pgfpathlineto{\pgfqpoint{2.103581in}{0.499444in}}%
\pgfusepath{stroke}%
\end{pgfscope}%
\begin{pgfscope}%
\pgfsetrectcap%
\pgfsetmiterjoin%
\pgfsetlinewidth{0.803000pt}%
\definecolor{currentstroke}{rgb}{0.000000,0.000000,0.000000}%
\pgfsetstrokecolor{currentstroke}%
\pgfsetdash{}{0pt}%
\pgfpathmoveto{\pgfqpoint{0.553581in}{1.654444in}}%
\pgfpathlineto{\pgfqpoint{2.103581in}{1.654444in}}%
\pgfusepath{stroke}%
\end{pgfscope}%
\begin{pgfscope}%
\pgfsetbuttcap%
\pgfsetmiterjoin%
\definecolor{currentfill}{rgb}{1.000000,1.000000,1.000000}%
\pgfsetfillcolor{currentfill}%
\pgfsetfillopacity{0.800000}%
\pgfsetlinewidth{1.003750pt}%
\definecolor{currentstroke}{rgb}{0.800000,0.800000,0.800000}%
\pgfsetstrokecolor{currentstroke}%
\pgfsetstrokeopacity{0.800000}%
\pgfsetdash{}{0pt}%
\pgfpathmoveto{\pgfqpoint{0.832747in}{0.568889in}}%
\pgfpathlineto{\pgfqpoint{2.006358in}{0.568889in}}%
\pgfpathquadraticcurveto{\pgfqpoint{2.034136in}{0.568889in}}{\pgfqpoint{2.034136in}{0.596666in}}%
\pgfpathlineto{\pgfqpoint{2.034136in}{0.776388in}}%
\pgfpathquadraticcurveto{\pgfqpoint{2.034136in}{0.804166in}}{\pgfqpoint{2.006358in}{0.804166in}}%
\pgfpathlineto{\pgfqpoint{0.832747in}{0.804166in}}%
\pgfpathquadraticcurveto{\pgfqpoint{0.804970in}{0.804166in}}{\pgfqpoint{0.804970in}{0.776388in}}%
\pgfpathlineto{\pgfqpoint{0.804970in}{0.596666in}}%
\pgfpathquadraticcurveto{\pgfqpoint{0.804970in}{0.568889in}}{\pgfqpoint{0.832747in}{0.568889in}}%
\pgfpathlineto{\pgfqpoint{0.832747in}{0.568889in}}%
\pgfpathclose%
\pgfusepath{stroke,fill}%
\end{pgfscope}%
\begin{pgfscope}%
\pgfsetrectcap%
\pgfsetroundjoin%
\pgfsetlinewidth{1.505625pt}%
\definecolor{currentstroke}{rgb}{0.000000,0.000000,0.000000}%
\pgfsetstrokecolor{currentstroke}%
\pgfsetdash{}{0pt}%
\pgfpathmoveto{\pgfqpoint{0.860525in}{0.700000in}}%
\pgfpathlineto{\pgfqpoint{0.999414in}{0.700000in}}%
\pgfpathlineto{\pgfqpoint{1.138303in}{0.700000in}}%
\pgfusepath{stroke}%
\end{pgfscope}%
\begin{pgfscope}%
\definecolor{textcolor}{rgb}{0.000000,0.000000,0.000000}%
\pgfsetstrokecolor{textcolor}%
\pgfsetfillcolor{textcolor}%
\pgftext[x=1.249414in,y=0.651388in,left,base]{\color{textcolor}\rmfamily\fontsize{10.000000}{12.000000}\selectfont AUC=0.763}%
\end{pgfscope}%
\end{pgfpicture}%
\makeatother%
\endgroup%

\end{tabular}

\

\

Other stuff

\


%%%
\begin{comment}
If we set the discrimination threshold about $0.7$, the model would classify almost all of the samples, both positive and negative class, correctly, with about the same number of false positives (sending an ambulance when one is not needed, negative class samples with $p > 0.7$) and false negatives (not sending an ambulance when one is needed, positive class samples with $p < 0.7$).  If we (as a society) were willing to tolerate more false positives, we could set the discrimination threshold lower, and if budgets were tighter we could increase the $p$ threshold.  

The table below gives the number of true negatives (TN), false positives (FP), false negatives (FN), and true positives (TP) for the 499,496 samples in the test set, along with the precision and recall values, for different discrimination thresholds $p$.  The precision is the proportion of ambulances we sent that were needed, and the recall is the proportion of ambulances needed that we sent.  

$$\text{Precision} = \frac{TP}{FP+TP}, \qquad \text{Recall} = \frac{TP}{FN + TP}$$

\begin{center}
\begin{tabular}{rrrrrrrrrrrrrr}
\toprule
$p$ &   TN &       FP &      FN &      TP &  Precision &   Recall \\
\midrule
0.50 &  346,776 &   73,794 &       1 &  78,925 &  0.52 &  1.00       \\
0.60 &  390,335 &   30,235 &      89 &  78,837 &  0.72 &  1.00  \\
0.70 & 411,040 &    9,530 &   2,838 &  76,088 &  0.89 &  0.96 \\
0.80 & 418,739 &    1,831 &  19,174 &  59,752 &  0.97 &  0.76  \\
0.90 & 420,496 &       74 &  53,736 &  25,190 &  1.00 &  0.32 & \\
\bottomrule
\end{tabular}
\end{center}

\end{comment}
%%%



