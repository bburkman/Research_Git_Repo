% Analysis of Results

We would like results like in the graphs below, where the machine learning (ML) algorithm nearly perfectly separates the two classes.  A ML algorithm assigns to each sample (feature vector, crash person) a probability that the person needs an ambulance.  The histogram shows the percentage of the dataset in each range of $p$, showing the percentages for the negative class (``Does not need an ambulance'') and the positive class (``Needs an ambulance'').  

On the right, the Receiver Operating Characteristic (ROC) curve, and particularly the area under the curve (AUC), is a metric for how well the model separates the two classes, with $AUC=1.0$ being perfect and $AUC=0.5$ (the dashed line) being just random assignment with no insight.  Having an AUC of 0.996 is amazing.  

\noindent\begin{tabular}{@{\hspace{-6pt}}p{4.3in} @{\hspace{-6pt}}p{2.0in}}
	\vskip 0pt
	\hfil Raw Model Output
	
	%% Creator: Matplotlib, PGF backend
%%
%% To include the figure in your LaTeX document, write
%%   \input{<filename>.pgf}
%%
%% Make sure the required packages are loaded in your preamble
%%   \usepackage{pgf}
%%
%% Also ensure that all the required font packages are loaded; for instance,
%% the lmodern package is sometimes necessary when using math font.
%%   \usepackage{lmodern}
%%
%% Figures using additional raster images can only be included by \input if
%% they are in the same directory as the main LaTeX file. For loading figures
%% from other directories you can use the `import` package
%%   \usepackage{import}
%%
%% and then include the figures with
%%   \import{<path to file>}{<filename>.pgf}
%%
%% Matplotlib used the following preamble
%%   
%%   \usepackage{fontspec}
%%   \makeatletter\@ifpackageloaded{underscore}{}{\usepackage[strings]{underscore}}\makeatother
%%
\begingroup%
\makeatletter%
\begin{pgfpicture}%
\pgfpathrectangle{\pgfpointorigin}{\pgfqpoint{4.509306in}{1.754444in}}%
\pgfusepath{use as bounding box, clip}%
\begin{pgfscope}%
\pgfsetbuttcap%
\pgfsetmiterjoin%
\definecolor{currentfill}{rgb}{1.000000,1.000000,1.000000}%
\pgfsetfillcolor{currentfill}%
\pgfsetlinewidth{0.000000pt}%
\definecolor{currentstroke}{rgb}{1.000000,1.000000,1.000000}%
\pgfsetstrokecolor{currentstroke}%
\pgfsetdash{}{0pt}%
\pgfpathmoveto{\pgfqpoint{0.000000in}{0.000000in}}%
\pgfpathlineto{\pgfqpoint{4.509306in}{0.000000in}}%
\pgfpathlineto{\pgfqpoint{4.509306in}{1.754444in}}%
\pgfpathlineto{\pgfqpoint{0.000000in}{1.754444in}}%
\pgfpathlineto{\pgfqpoint{0.000000in}{0.000000in}}%
\pgfpathclose%
\pgfusepath{fill}%
\end{pgfscope}%
\begin{pgfscope}%
\pgfsetbuttcap%
\pgfsetmiterjoin%
\definecolor{currentfill}{rgb}{1.000000,1.000000,1.000000}%
\pgfsetfillcolor{currentfill}%
\pgfsetlinewidth{0.000000pt}%
\definecolor{currentstroke}{rgb}{0.000000,0.000000,0.000000}%
\pgfsetstrokecolor{currentstroke}%
\pgfsetstrokeopacity{0.000000}%
\pgfsetdash{}{0pt}%
\pgfpathmoveto{\pgfqpoint{0.445556in}{0.499444in}}%
\pgfpathlineto{\pgfqpoint{4.320556in}{0.499444in}}%
\pgfpathlineto{\pgfqpoint{4.320556in}{1.654444in}}%
\pgfpathlineto{\pgfqpoint{0.445556in}{1.654444in}}%
\pgfpathlineto{\pgfqpoint{0.445556in}{0.499444in}}%
\pgfpathclose%
\pgfusepath{fill}%
\end{pgfscope}%
\begin{pgfscope}%
\pgfpathrectangle{\pgfqpoint{0.445556in}{0.499444in}}{\pgfqpoint{3.875000in}{1.155000in}}%
\pgfusepath{clip}%
\pgfsetbuttcap%
\pgfsetmiterjoin%
\pgfsetlinewidth{1.003750pt}%
\definecolor{currentstroke}{rgb}{0.000000,0.000000,0.000000}%
\pgfsetstrokecolor{currentstroke}%
\pgfsetdash{}{0pt}%
\pgfpathmoveto{\pgfqpoint{0.435556in}{0.499444in}}%
\pgfpathlineto{\pgfqpoint{0.483922in}{0.499444in}}%
\pgfpathlineto{\pgfqpoint{0.483922in}{0.632510in}}%
\pgfpathlineto{\pgfqpoint{0.435556in}{0.632510in}}%
\pgfusepath{stroke}%
\end{pgfscope}%
\begin{pgfscope}%
\pgfpathrectangle{\pgfqpoint{0.445556in}{0.499444in}}{\pgfqpoint{3.875000in}{1.155000in}}%
\pgfusepath{clip}%
\pgfsetbuttcap%
\pgfsetmiterjoin%
\pgfsetlinewidth{1.003750pt}%
\definecolor{currentstroke}{rgb}{0.000000,0.000000,0.000000}%
\pgfsetstrokecolor{currentstroke}%
\pgfsetdash{}{0pt}%
\pgfpathmoveto{\pgfqpoint{0.576001in}{0.499444in}}%
\pgfpathlineto{\pgfqpoint{0.637387in}{0.499444in}}%
\pgfpathlineto{\pgfqpoint{0.637387in}{0.855505in}}%
\pgfpathlineto{\pgfqpoint{0.576001in}{0.855505in}}%
\pgfpathlineto{\pgfqpoint{0.576001in}{0.499444in}}%
\pgfpathclose%
\pgfusepath{stroke}%
\end{pgfscope}%
\begin{pgfscope}%
\pgfpathrectangle{\pgfqpoint{0.445556in}{0.499444in}}{\pgfqpoint{3.875000in}{1.155000in}}%
\pgfusepath{clip}%
\pgfsetbuttcap%
\pgfsetmiterjoin%
\pgfsetlinewidth{1.003750pt}%
\definecolor{currentstroke}{rgb}{0.000000,0.000000,0.000000}%
\pgfsetstrokecolor{currentstroke}%
\pgfsetdash{}{0pt}%
\pgfpathmoveto{\pgfqpoint{0.729467in}{0.499444in}}%
\pgfpathlineto{\pgfqpoint{0.790853in}{0.499444in}}%
\pgfpathlineto{\pgfqpoint{0.790853in}{1.087595in}}%
\pgfpathlineto{\pgfqpoint{0.729467in}{1.087595in}}%
\pgfpathlineto{\pgfqpoint{0.729467in}{0.499444in}}%
\pgfpathclose%
\pgfusepath{stroke}%
\end{pgfscope}%
\begin{pgfscope}%
\pgfpathrectangle{\pgfqpoint{0.445556in}{0.499444in}}{\pgfqpoint{3.875000in}{1.155000in}}%
\pgfusepath{clip}%
\pgfsetbuttcap%
\pgfsetmiterjoin%
\pgfsetlinewidth{1.003750pt}%
\definecolor{currentstroke}{rgb}{0.000000,0.000000,0.000000}%
\pgfsetstrokecolor{currentstroke}%
\pgfsetdash{}{0pt}%
\pgfpathmoveto{\pgfqpoint{0.882932in}{0.499444in}}%
\pgfpathlineto{\pgfqpoint{0.944318in}{0.499444in}}%
\pgfpathlineto{\pgfqpoint{0.944318in}{1.269324in}}%
\pgfpathlineto{\pgfqpoint{0.882932in}{1.269324in}}%
\pgfpathlineto{\pgfqpoint{0.882932in}{0.499444in}}%
\pgfpathclose%
\pgfusepath{stroke}%
\end{pgfscope}%
\begin{pgfscope}%
\pgfpathrectangle{\pgfqpoint{0.445556in}{0.499444in}}{\pgfqpoint{3.875000in}{1.155000in}}%
\pgfusepath{clip}%
\pgfsetbuttcap%
\pgfsetmiterjoin%
\pgfsetlinewidth{1.003750pt}%
\definecolor{currentstroke}{rgb}{0.000000,0.000000,0.000000}%
\pgfsetstrokecolor{currentstroke}%
\pgfsetdash{}{0pt}%
\pgfpathmoveto{\pgfqpoint{1.036397in}{0.499444in}}%
\pgfpathlineto{\pgfqpoint{1.097783in}{0.499444in}}%
\pgfpathlineto{\pgfqpoint{1.097783in}{1.428301in}}%
\pgfpathlineto{\pgfqpoint{1.036397in}{1.428301in}}%
\pgfpathlineto{\pgfqpoint{1.036397in}{0.499444in}}%
\pgfpathclose%
\pgfusepath{stroke}%
\end{pgfscope}%
\begin{pgfscope}%
\pgfpathrectangle{\pgfqpoint{0.445556in}{0.499444in}}{\pgfqpoint{3.875000in}{1.155000in}}%
\pgfusepath{clip}%
\pgfsetbuttcap%
\pgfsetmiterjoin%
\pgfsetlinewidth{1.003750pt}%
\definecolor{currentstroke}{rgb}{0.000000,0.000000,0.000000}%
\pgfsetstrokecolor{currentstroke}%
\pgfsetdash{}{0pt}%
\pgfpathmoveto{\pgfqpoint{1.189863in}{0.499444in}}%
\pgfpathlineto{\pgfqpoint{1.251249in}{0.499444in}}%
\pgfpathlineto{\pgfqpoint{1.251249in}{1.521593in}}%
\pgfpathlineto{\pgfqpoint{1.189863in}{1.521593in}}%
\pgfpathlineto{\pgfqpoint{1.189863in}{0.499444in}}%
\pgfpathclose%
\pgfusepath{stroke}%
\end{pgfscope}%
\begin{pgfscope}%
\pgfpathrectangle{\pgfqpoint{0.445556in}{0.499444in}}{\pgfqpoint{3.875000in}{1.155000in}}%
\pgfusepath{clip}%
\pgfsetbuttcap%
\pgfsetmiterjoin%
\pgfsetlinewidth{1.003750pt}%
\definecolor{currentstroke}{rgb}{0.000000,0.000000,0.000000}%
\pgfsetstrokecolor{currentstroke}%
\pgfsetdash{}{0pt}%
\pgfpathmoveto{\pgfqpoint{1.343328in}{0.499444in}}%
\pgfpathlineto{\pgfqpoint{1.404714in}{0.499444in}}%
\pgfpathlineto{\pgfqpoint{1.404714in}{1.590875in}}%
\pgfpathlineto{\pgfqpoint{1.343328in}{1.590875in}}%
\pgfpathlineto{\pgfqpoint{1.343328in}{0.499444in}}%
\pgfpathclose%
\pgfusepath{stroke}%
\end{pgfscope}%
\begin{pgfscope}%
\pgfpathrectangle{\pgfqpoint{0.445556in}{0.499444in}}{\pgfqpoint{3.875000in}{1.155000in}}%
\pgfusepath{clip}%
\pgfsetbuttcap%
\pgfsetmiterjoin%
\pgfsetlinewidth{1.003750pt}%
\definecolor{currentstroke}{rgb}{0.000000,0.000000,0.000000}%
\pgfsetstrokecolor{currentstroke}%
\pgfsetdash{}{0pt}%
\pgfpathmoveto{\pgfqpoint{1.496793in}{0.499444in}}%
\pgfpathlineto{\pgfqpoint{1.558179in}{0.499444in}}%
\pgfpathlineto{\pgfqpoint{1.558179in}{1.599444in}}%
\pgfpathlineto{\pgfqpoint{1.496793in}{1.599444in}}%
\pgfpathlineto{\pgfqpoint{1.496793in}{0.499444in}}%
\pgfpathclose%
\pgfusepath{stroke}%
\end{pgfscope}%
\begin{pgfscope}%
\pgfpathrectangle{\pgfqpoint{0.445556in}{0.499444in}}{\pgfqpoint{3.875000in}{1.155000in}}%
\pgfusepath{clip}%
\pgfsetbuttcap%
\pgfsetmiterjoin%
\pgfsetlinewidth{1.003750pt}%
\definecolor{currentstroke}{rgb}{0.000000,0.000000,0.000000}%
\pgfsetstrokecolor{currentstroke}%
\pgfsetdash{}{0pt}%
\pgfpathmoveto{\pgfqpoint{1.650259in}{0.499444in}}%
\pgfpathlineto{\pgfqpoint{1.711645in}{0.499444in}}%
\pgfpathlineto{\pgfqpoint{1.711645in}{1.566953in}}%
\pgfpathlineto{\pgfqpoint{1.650259in}{1.566953in}}%
\pgfpathlineto{\pgfqpoint{1.650259in}{0.499444in}}%
\pgfpathclose%
\pgfusepath{stroke}%
\end{pgfscope}%
\begin{pgfscope}%
\pgfpathrectangle{\pgfqpoint{0.445556in}{0.499444in}}{\pgfqpoint{3.875000in}{1.155000in}}%
\pgfusepath{clip}%
\pgfsetbuttcap%
\pgfsetmiterjoin%
\pgfsetlinewidth{1.003750pt}%
\definecolor{currentstroke}{rgb}{0.000000,0.000000,0.000000}%
\pgfsetstrokecolor{currentstroke}%
\pgfsetdash{}{0pt}%
\pgfpathmoveto{\pgfqpoint{1.803724in}{0.499444in}}%
\pgfpathlineto{\pgfqpoint{1.865110in}{0.499444in}}%
\pgfpathlineto{\pgfqpoint{1.865110in}{1.506942in}}%
\pgfpathlineto{\pgfqpoint{1.803724in}{1.506942in}}%
\pgfpathlineto{\pgfqpoint{1.803724in}{0.499444in}}%
\pgfpathclose%
\pgfusepath{stroke}%
\end{pgfscope}%
\begin{pgfscope}%
\pgfpathrectangle{\pgfqpoint{0.445556in}{0.499444in}}{\pgfqpoint{3.875000in}{1.155000in}}%
\pgfusepath{clip}%
\pgfsetbuttcap%
\pgfsetmiterjoin%
\pgfsetlinewidth{1.003750pt}%
\definecolor{currentstroke}{rgb}{0.000000,0.000000,0.000000}%
\pgfsetstrokecolor{currentstroke}%
\pgfsetdash{}{0pt}%
\pgfpathmoveto{\pgfqpoint{1.957189in}{0.499444in}}%
\pgfpathlineto{\pgfqpoint{2.018575in}{0.499444in}}%
\pgfpathlineto{\pgfqpoint{2.018575in}{1.416545in}}%
\pgfpathlineto{\pgfqpoint{1.957189in}{1.416545in}}%
\pgfpathlineto{\pgfqpoint{1.957189in}{0.499444in}}%
\pgfpathclose%
\pgfusepath{stroke}%
\end{pgfscope}%
\begin{pgfscope}%
\pgfpathrectangle{\pgfqpoint{0.445556in}{0.499444in}}{\pgfqpoint{3.875000in}{1.155000in}}%
\pgfusepath{clip}%
\pgfsetbuttcap%
\pgfsetmiterjoin%
\pgfsetlinewidth{1.003750pt}%
\definecolor{currentstroke}{rgb}{0.000000,0.000000,0.000000}%
\pgfsetstrokecolor{currentstroke}%
\pgfsetdash{}{0pt}%
\pgfpathmoveto{\pgfqpoint{2.110655in}{0.499444in}}%
\pgfpathlineto{\pgfqpoint{2.172041in}{0.499444in}}%
\pgfpathlineto{\pgfqpoint{2.172041in}{1.298482in}}%
\pgfpathlineto{\pgfqpoint{2.110655in}{1.298482in}}%
\pgfpathlineto{\pgfqpoint{2.110655in}{0.499444in}}%
\pgfpathclose%
\pgfusepath{stroke}%
\end{pgfscope}%
\begin{pgfscope}%
\pgfpathrectangle{\pgfqpoint{0.445556in}{0.499444in}}{\pgfqpoint{3.875000in}{1.155000in}}%
\pgfusepath{clip}%
\pgfsetbuttcap%
\pgfsetmiterjoin%
\pgfsetlinewidth{1.003750pt}%
\definecolor{currentstroke}{rgb}{0.000000,0.000000,0.000000}%
\pgfsetstrokecolor{currentstroke}%
\pgfsetdash{}{0pt}%
\pgfpathmoveto{\pgfqpoint{2.264120in}{0.499444in}}%
\pgfpathlineto{\pgfqpoint{2.325506in}{0.499444in}}%
\pgfpathlineto{\pgfqpoint{2.325506in}{1.165884in}}%
\pgfpathlineto{\pgfqpoint{2.264120in}{1.165884in}}%
\pgfpathlineto{\pgfqpoint{2.264120in}{0.499444in}}%
\pgfpathclose%
\pgfusepath{stroke}%
\end{pgfscope}%
\begin{pgfscope}%
\pgfpathrectangle{\pgfqpoint{0.445556in}{0.499444in}}{\pgfqpoint{3.875000in}{1.155000in}}%
\pgfusepath{clip}%
\pgfsetbuttcap%
\pgfsetmiterjoin%
\pgfsetlinewidth{1.003750pt}%
\definecolor{currentstroke}{rgb}{0.000000,0.000000,0.000000}%
\pgfsetstrokecolor{currentstroke}%
\pgfsetdash{}{0pt}%
\pgfpathmoveto{\pgfqpoint{2.417585in}{0.499444in}}%
\pgfpathlineto{\pgfqpoint{2.478972in}{0.499444in}}%
\pgfpathlineto{\pgfqpoint{2.478972in}{1.041650in}}%
\pgfpathlineto{\pgfqpoint{2.417585in}{1.041650in}}%
\pgfpathlineto{\pgfqpoint{2.417585in}{0.499444in}}%
\pgfpathclose%
\pgfusepath{stroke}%
\end{pgfscope}%
\begin{pgfscope}%
\pgfpathrectangle{\pgfqpoint{0.445556in}{0.499444in}}{\pgfqpoint{3.875000in}{1.155000in}}%
\pgfusepath{clip}%
\pgfsetbuttcap%
\pgfsetmiterjoin%
\pgfsetlinewidth{1.003750pt}%
\definecolor{currentstroke}{rgb}{0.000000,0.000000,0.000000}%
\pgfsetstrokecolor{currentstroke}%
\pgfsetdash{}{0pt}%
\pgfpathmoveto{\pgfqpoint{2.571051in}{0.499444in}}%
\pgfpathlineto{\pgfqpoint{2.632437in}{0.499444in}}%
\pgfpathlineto{\pgfqpoint{2.632437in}{0.916218in}}%
\pgfpathlineto{\pgfqpoint{2.571051in}{0.916218in}}%
\pgfpathlineto{\pgfqpoint{2.571051in}{0.499444in}}%
\pgfpathclose%
\pgfusepath{stroke}%
\end{pgfscope}%
\begin{pgfscope}%
\pgfpathrectangle{\pgfqpoint{0.445556in}{0.499444in}}{\pgfqpoint{3.875000in}{1.155000in}}%
\pgfusepath{clip}%
\pgfsetbuttcap%
\pgfsetmiterjoin%
\pgfsetlinewidth{1.003750pt}%
\definecolor{currentstroke}{rgb}{0.000000,0.000000,0.000000}%
\pgfsetstrokecolor{currentstroke}%
\pgfsetdash{}{0pt}%
\pgfpathmoveto{\pgfqpoint{2.724516in}{0.499444in}}%
\pgfpathlineto{\pgfqpoint{2.785902in}{0.499444in}}%
\pgfpathlineto{\pgfqpoint{2.785902in}{0.806958in}}%
\pgfpathlineto{\pgfqpoint{2.724516in}{0.806958in}}%
\pgfpathlineto{\pgfqpoint{2.724516in}{0.499444in}}%
\pgfpathclose%
\pgfusepath{stroke}%
\end{pgfscope}%
\begin{pgfscope}%
\pgfpathrectangle{\pgfqpoint{0.445556in}{0.499444in}}{\pgfqpoint{3.875000in}{1.155000in}}%
\pgfusepath{clip}%
\pgfsetbuttcap%
\pgfsetmiterjoin%
\pgfsetlinewidth{1.003750pt}%
\definecolor{currentstroke}{rgb}{0.000000,0.000000,0.000000}%
\pgfsetstrokecolor{currentstroke}%
\pgfsetdash{}{0pt}%
\pgfpathmoveto{\pgfqpoint{2.877981in}{0.499444in}}%
\pgfpathlineto{\pgfqpoint{2.939368in}{0.499444in}}%
\pgfpathlineto{\pgfqpoint{2.939368in}{0.716063in}}%
\pgfpathlineto{\pgfqpoint{2.877981in}{0.716063in}}%
\pgfpathlineto{\pgfqpoint{2.877981in}{0.499444in}}%
\pgfpathclose%
\pgfusepath{stroke}%
\end{pgfscope}%
\begin{pgfscope}%
\pgfpathrectangle{\pgfqpoint{0.445556in}{0.499444in}}{\pgfqpoint{3.875000in}{1.155000in}}%
\pgfusepath{clip}%
\pgfsetbuttcap%
\pgfsetmiterjoin%
\pgfsetlinewidth{1.003750pt}%
\definecolor{currentstroke}{rgb}{0.000000,0.000000,0.000000}%
\pgfsetstrokecolor{currentstroke}%
\pgfsetdash{}{0pt}%
\pgfpathmoveto{\pgfqpoint{3.031447in}{0.499444in}}%
\pgfpathlineto{\pgfqpoint{3.092833in}{0.499444in}}%
\pgfpathlineto{\pgfqpoint{3.092833in}{0.645904in}}%
\pgfpathlineto{\pgfqpoint{3.031447in}{0.645904in}}%
\pgfpathlineto{\pgfqpoint{3.031447in}{0.499444in}}%
\pgfpathclose%
\pgfusepath{stroke}%
\end{pgfscope}%
\begin{pgfscope}%
\pgfpathrectangle{\pgfqpoint{0.445556in}{0.499444in}}{\pgfqpoint{3.875000in}{1.155000in}}%
\pgfusepath{clip}%
\pgfsetbuttcap%
\pgfsetmiterjoin%
\pgfsetlinewidth{1.003750pt}%
\definecolor{currentstroke}{rgb}{0.000000,0.000000,0.000000}%
\pgfsetstrokecolor{currentstroke}%
\pgfsetdash{}{0pt}%
\pgfpathmoveto{\pgfqpoint{3.184912in}{0.499444in}}%
\pgfpathlineto{\pgfqpoint{3.246298in}{0.499444in}}%
\pgfpathlineto{\pgfqpoint{3.246298in}{0.598936in}}%
\pgfpathlineto{\pgfqpoint{3.184912in}{0.598936in}}%
\pgfpathlineto{\pgfqpoint{3.184912in}{0.499444in}}%
\pgfpathclose%
\pgfusepath{stroke}%
\end{pgfscope}%
\begin{pgfscope}%
\pgfpathrectangle{\pgfqpoint{0.445556in}{0.499444in}}{\pgfqpoint{3.875000in}{1.155000in}}%
\pgfusepath{clip}%
\pgfsetbuttcap%
\pgfsetmiterjoin%
\pgfsetlinewidth{1.003750pt}%
\definecolor{currentstroke}{rgb}{0.000000,0.000000,0.000000}%
\pgfsetstrokecolor{currentstroke}%
\pgfsetdash{}{0pt}%
\pgfpathmoveto{\pgfqpoint{3.338377in}{0.499444in}}%
\pgfpathlineto{\pgfqpoint{3.399764in}{0.499444in}}%
\pgfpathlineto{\pgfqpoint{3.399764in}{0.560040in}}%
\pgfpathlineto{\pgfqpoint{3.338377in}{0.560040in}}%
\pgfpathlineto{\pgfqpoint{3.338377in}{0.499444in}}%
\pgfpathclose%
\pgfusepath{stroke}%
\end{pgfscope}%
\begin{pgfscope}%
\pgfpathrectangle{\pgfqpoint{0.445556in}{0.499444in}}{\pgfqpoint{3.875000in}{1.155000in}}%
\pgfusepath{clip}%
\pgfsetbuttcap%
\pgfsetmiterjoin%
\pgfsetlinewidth{1.003750pt}%
\definecolor{currentstroke}{rgb}{0.000000,0.000000,0.000000}%
\pgfsetstrokecolor{currentstroke}%
\pgfsetdash{}{0pt}%
\pgfpathmoveto{\pgfqpoint{3.491843in}{0.499444in}}%
\pgfpathlineto{\pgfqpoint{3.553229in}{0.499444in}}%
\pgfpathlineto{\pgfqpoint{3.553229in}{0.535591in}}%
\pgfpathlineto{\pgfqpoint{3.491843in}{0.535591in}}%
\pgfpathlineto{\pgfqpoint{3.491843in}{0.499444in}}%
\pgfpathclose%
\pgfusepath{stroke}%
\end{pgfscope}%
\begin{pgfscope}%
\pgfpathrectangle{\pgfqpoint{0.445556in}{0.499444in}}{\pgfqpoint{3.875000in}{1.155000in}}%
\pgfusepath{clip}%
\pgfsetbuttcap%
\pgfsetmiterjoin%
\pgfsetlinewidth{1.003750pt}%
\definecolor{currentstroke}{rgb}{0.000000,0.000000,0.000000}%
\pgfsetstrokecolor{currentstroke}%
\pgfsetdash{}{0pt}%
\pgfpathmoveto{\pgfqpoint{3.645308in}{0.499444in}}%
\pgfpathlineto{\pgfqpoint{3.706694in}{0.499444in}}%
\pgfpathlineto{\pgfqpoint{3.706694in}{0.512137in}}%
\pgfpathlineto{\pgfqpoint{3.645308in}{0.512137in}}%
\pgfpathlineto{\pgfqpoint{3.645308in}{0.499444in}}%
\pgfpathclose%
\pgfusepath{stroke}%
\end{pgfscope}%
\begin{pgfscope}%
\pgfpathrectangle{\pgfqpoint{0.445556in}{0.499444in}}{\pgfqpoint{3.875000in}{1.155000in}}%
\pgfusepath{clip}%
\pgfsetbuttcap%
\pgfsetmiterjoin%
\pgfsetlinewidth{1.003750pt}%
\definecolor{currentstroke}{rgb}{0.000000,0.000000,0.000000}%
\pgfsetstrokecolor{currentstroke}%
\pgfsetdash{}{0pt}%
\pgfpathmoveto{\pgfqpoint{3.798774in}{0.499444in}}%
\pgfpathlineto{\pgfqpoint{3.860160in}{0.499444in}}%
\pgfpathlineto{\pgfqpoint{3.860160in}{0.503305in}}%
\pgfpathlineto{\pgfqpoint{3.798774in}{0.503305in}}%
\pgfpathlineto{\pgfqpoint{3.798774in}{0.499444in}}%
\pgfpathclose%
\pgfusepath{stroke}%
\end{pgfscope}%
\begin{pgfscope}%
\pgfpathrectangle{\pgfqpoint{0.445556in}{0.499444in}}{\pgfqpoint{3.875000in}{1.155000in}}%
\pgfusepath{clip}%
\pgfsetbuttcap%
\pgfsetmiterjoin%
\pgfsetlinewidth{1.003750pt}%
\definecolor{currentstroke}{rgb}{0.000000,0.000000,0.000000}%
\pgfsetstrokecolor{currentstroke}%
\pgfsetdash{}{0pt}%
\pgfpathmoveto{\pgfqpoint{3.952239in}{0.499444in}}%
\pgfpathlineto{\pgfqpoint{4.013625in}{0.499444in}}%
\pgfpathlineto{\pgfqpoint{4.013625in}{0.500234in}}%
\pgfpathlineto{\pgfqpoint{3.952239in}{0.500234in}}%
\pgfpathlineto{\pgfqpoint{3.952239in}{0.499444in}}%
\pgfpathclose%
\pgfusepath{stroke}%
\end{pgfscope}%
\begin{pgfscope}%
\pgfpathrectangle{\pgfqpoint{0.445556in}{0.499444in}}{\pgfqpoint{3.875000in}{1.155000in}}%
\pgfusepath{clip}%
\pgfsetbuttcap%
\pgfsetmiterjoin%
\pgfsetlinewidth{1.003750pt}%
\definecolor{currentstroke}{rgb}{0.000000,0.000000,0.000000}%
\pgfsetstrokecolor{currentstroke}%
\pgfsetdash{}{0pt}%
\pgfpathmoveto{\pgfqpoint{4.105704in}{0.499444in}}%
\pgfpathlineto{\pgfqpoint{4.167090in}{0.499444in}}%
\pgfpathlineto{\pgfqpoint{4.167090in}{0.499503in}}%
\pgfpathlineto{\pgfqpoint{4.105704in}{0.499503in}}%
\pgfpathlineto{\pgfqpoint{4.105704in}{0.499444in}}%
\pgfpathclose%
\pgfusepath{stroke}%
\end{pgfscope}%
\begin{pgfscope}%
\pgfpathrectangle{\pgfqpoint{0.445556in}{0.499444in}}{\pgfqpoint{3.875000in}{1.155000in}}%
\pgfusepath{clip}%
\pgfsetbuttcap%
\pgfsetmiterjoin%
\definecolor{currentfill}{rgb}{0.000000,0.000000,0.000000}%
\pgfsetfillcolor{currentfill}%
\pgfsetlinewidth{0.000000pt}%
\definecolor{currentstroke}{rgb}{0.000000,0.000000,0.000000}%
\pgfsetstrokecolor{currentstroke}%
\pgfsetstrokeopacity{0.000000}%
\pgfsetdash{}{0pt}%
\pgfpathmoveto{\pgfqpoint{0.483922in}{0.499444in}}%
\pgfpathlineto{\pgfqpoint{0.545308in}{0.499444in}}%
\pgfpathlineto{\pgfqpoint{0.545308in}{0.499444in}}%
\pgfpathlineto{\pgfqpoint{0.483922in}{0.499444in}}%
\pgfpathlineto{\pgfqpoint{0.483922in}{0.499444in}}%
\pgfpathclose%
\pgfusepath{fill}%
\end{pgfscope}%
\begin{pgfscope}%
\pgfpathrectangle{\pgfqpoint{0.445556in}{0.499444in}}{\pgfqpoint{3.875000in}{1.155000in}}%
\pgfusepath{clip}%
\pgfsetbuttcap%
\pgfsetmiterjoin%
\definecolor{currentfill}{rgb}{0.000000,0.000000,0.000000}%
\pgfsetfillcolor{currentfill}%
\pgfsetlinewidth{0.000000pt}%
\definecolor{currentstroke}{rgb}{0.000000,0.000000,0.000000}%
\pgfsetstrokecolor{currentstroke}%
\pgfsetstrokeopacity{0.000000}%
\pgfsetdash{}{0pt}%
\pgfpathmoveto{\pgfqpoint{0.637387in}{0.499444in}}%
\pgfpathlineto{\pgfqpoint{0.698774in}{0.499444in}}%
\pgfpathlineto{\pgfqpoint{0.698774in}{0.499444in}}%
\pgfpathlineto{\pgfqpoint{0.637387in}{0.499444in}}%
\pgfpathlineto{\pgfqpoint{0.637387in}{0.499444in}}%
\pgfpathclose%
\pgfusepath{fill}%
\end{pgfscope}%
\begin{pgfscope}%
\pgfpathrectangle{\pgfqpoint{0.445556in}{0.499444in}}{\pgfqpoint{3.875000in}{1.155000in}}%
\pgfusepath{clip}%
\pgfsetbuttcap%
\pgfsetmiterjoin%
\definecolor{currentfill}{rgb}{0.000000,0.000000,0.000000}%
\pgfsetfillcolor{currentfill}%
\pgfsetlinewidth{0.000000pt}%
\definecolor{currentstroke}{rgb}{0.000000,0.000000,0.000000}%
\pgfsetstrokecolor{currentstroke}%
\pgfsetstrokeopacity{0.000000}%
\pgfsetdash{}{0pt}%
\pgfpathmoveto{\pgfqpoint{0.790853in}{0.499444in}}%
\pgfpathlineto{\pgfqpoint{0.852239in}{0.499444in}}%
\pgfpathlineto{\pgfqpoint{0.852239in}{0.499444in}}%
\pgfpathlineto{\pgfqpoint{0.790853in}{0.499444in}}%
\pgfpathlineto{\pgfqpoint{0.790853in}{0.499444in}}%
\pgfpathclose%
\pgfusepath{fill}%
\end{pgfscope}%
\begin{pgfscope}%
\pgfpathrectangle{\pgfqpoint{0.445556in}{0.499444in}}{\pgfqpoint{3.875000in}{1.155000in}}%
\pgfusepath{clip}%
\pgfsetbuttcap%
\pgfsetmiterjoin%
\definecolor{currentfill}{rgb}{0.000000,0.000000,0.000000}%
\pgfsetfillcolor{currentfill}%
\pgfsetlinewidth{0.000000pt}%
\definecolor{currentstroke}{rgb}{0.000000,0.000000,0.000000}%
\pgfsetstrokecolor{currentstroke}%
\pgfsetstrokeopacity{0.000000}%
\pgfsetdash{}{0pt}%
\pgfpathmoveto{\pgfqpoint{0.944318in}{0.499444in}}%
\pgfpathlineto{\pgfqpoint{1.005704in}{0.499444in}}%
\pgfpathlineto{\pgfqpoint{1.005704in}{0.499444in}}%
\pgfpathlineto{\pgfqpoint{0.944318in}{0.499444in}}%
\pgfpathlineto{\pgfqpoint{0.944318in}{0.499444in}}%
\pgfpathclose%
\pgfusepath{fill}%
\end{pgfscope}%
\begin{pgfscope}%
\pgfpathrectangle{\pgfqpoint{0.445556in}{0.499444in}}{\pgfqpoint{3.875000in}{1.155000in}}%
\pgfusepath{clip}%
\pgfsetbuttcap%
\pgfsetmiterjoin%
\definecolor{currentfill}{rgb}{0.000000,0.000000,0.000000}%
\pgfsetfillcolor{currentfill}%
\pgfsetlinewidth{0.000000pt}%
\definecolor{currentstroke}{rgb}{0.000000,0.000000,0.000000}%
\pgfsetstrokecolor{currentstroke}%
\pgfsetstrokeopacity{0.000000}%
\pgfsetdash{}{0pt}%
\pgfpathmoveto{\pgfqpoint{1.097783in}{0.499444in}}%
\pgfpathlineto{\pgfqpoint{1.159170in}{0.499444in}}%
\pgfpathlineto{\pgfqpoint{1.159170in}{0.499444in}}%
\pgfpathlineto{\pgfqpoint{1.097783in}{0.499444in}}%
\pgfpathlineto{\pgfqpoint{1.097783in}{0.499444in}}%
\pgfpathclose%
\pgfusepath{fill}%
\end{pgfscope}%
\begin{pgfscope}%
\pgfpathrectangle{\pgfqpoint{0.445556in}{0.499444in}}{\pgfqpoint{3.875000in}{1.155000in}}%
\pgfusepath{clip}%
\pgfsetbuttcap%
\pgfsetmiterjoin%
\definecolor{currentfill}{rgb}{0.000000,0.000000,0.000000}%
\pgfsetfillcolor{currentfill}%
\pgfsetlinewidth{0.000000pt}%
\definecolor{currentstroke}{rgb}{0.000000,0.000000,0.000000}%
\pgfsetstrokecolor{currentstroke}%
\pgfsetstrokeopacity{0.000000}%
\pgfsetdash{}{0pt}%
\pgfpathmoveto{\pgfqpoint{1.251249in}{0.499444in}}%
\pgfpathlineto{\pgfqpoint{1.312635in}{0.499444in}}%
\pgfpathlineto{\pgfqpoint{1.312635in}{0.499444in}}%
\pgfpathlineto{\pgfqpoint{1.251249in}{0.499444in}}%
\pgfpathlineto{\pgfqpoint{1.251249in}{0.499444in}}%
\pgfpathclose%
\pgfusepath{fill}%
\end{pgfscope}%
\begin{pgfscope}%
\pgfpathrectangle{\pgfqpoint{0.445556in}{0.499444in}}{\pgfqpoint{3.875000in}{1.155000in}}%
\pgfusepath{clip}%
\pgfsetbuttcap%
\pgfsetmiterjoin%
\definecolor{currentfill}{rgb}{0.000000,0.000000,0.000000}%
\pgfsetfillcolor{currentfill}%
\pgfsetlinewidth{0.000000pt}%
\definecolor{currentstroke}{rgb}{0.000000,0.000000,0.000000}%
\pgfsetstrokecolor{currentstroke}%
\pgfsetstrokeopacity{0.000000}%
\pgfsetdash{}{0pt}%
\pgfpathmoveto{\pgfqpoint{1.404714in}{0.499444in}}%
\pgfpathlineto{\pgfqpoint{1.466100in}{0.499444in}}%
\pgfpathlineto{\pgfqpoint{1.466100in}{0.499444in}}%
\pgfpathlineto{\pgfqpoint{1.404714in}{0.499444in}}%
\pgfpathlineto{\pgfqpoint{1.404714in}{0.499444in}}%
\pgfpathclose%
\pgfusepath{fill}%
\end{pgfscope}%
\begin{pgfscope}%
\pgfpathrectangle{\pgfqpoint{0.445556in}{0.499444in}}{\pgfqpoint{3.875000in}{1.155000in}}%
\pgfusepath{clip}%
\pgfsetbuttcap%
\pgfsetmiterjoin%
\definecolor{currentfill}{rgb}{0.000000,0.000000,0.000000}%
\pgfsetfillcolor{currentfill}%
\pgfsetlinewidth{0.000000pt}%
\definecolor{currentstroke}{rgb}{0.000000,0.000000,0.000000}%
\pgfsetstrokecolor{currentstroke}%
\pgfsetstrokeopacity{0.000000}%
\pgfsetdash{}{0pt}%
\pgfpathmoveto{\pgfqpoint{1.558179in}{0.499444in}}%
\pgfpathlineto{\pgfqpoint{1.619566in}{0.499444in}}%
\pgfpathlineto{\pgfqpoint{1.619566in}{0.499444in}}%
\pgfpathlineto{\pgfqpoint{1.558179in}{0.499444in}}%
\pgfpathlineto{\pgfqpoint{1.558179in}{0.499444in}}%
\pgfpathclose%
\pgfusepath{fill}%
\end{pgfscope}%
\begin{pgfscope}%
\pgfpathrectangle{\pgfqpoint{0.445556in}{0.499444in}}{\pgfqpoint{3.875000in}{1.155000in}}%
\pgfusepath{clip}%
\pgfsetbuttcap%
\pgfsetmiterjoin%
\definecolor{currentfill}{rgb}{0.000000,0.000000,0.000000}%
\pgfsetfillcolor{currentfill}%
\pgfsetlinewidth{0.000000pt}%
\definecolor{currentstroke}{rgb}{0.000000,0.000000,0.000000}%
\pgfsetstrokecolor{currentstroke}%
\pgfsetstrokeopacity{0.000000}%
\pgfsetdash{}{0pt}%
\pgfpathmoveto{\pgfqpoint{1.711645in}{0.499444in}}%
\pgfpathlineto{\pgfqpoint{1.773031in}{0.499444in}}%
\pgfpathlineto{\pgfqpoint{1.773031in}{0.499444in}}%
\pgfpathlineto{\pgfqpoint{1.711645in}{0.499444in}}%
\pgfpathlineto{\pgfqpoint{1.711645in}{0.499444in}}%
\pgfpathclose%
\pgfusepath{fill}%
\end{pgfscope}%
\begin{pgfscope}%
\pgfpathrectangle{\pgfqpoint{0.445556in}{0.499444in}}{\pgfqpoint{3.875000in}{1.155000in}}%
\pgfusepath{clip}%
\pgfsetbuttcap%
\pgfsetmiterjoin%
\definecolor{currentfill}{rgb}{0.000000,0.000000,0.000000}%
\pgfsetfillcolor{currentfill}%
\pgfsetlinewidth{0.000000pt}%
\definecolor{currentstroke}{rgb}{0.000000,0.000000,0.000000}%
\pgfsetstrokecolor{currentstroke}%
\pgfsetstrokeopacity{0.000000}%
\pgfsetdash{}{0pt}%
\pgfpathmoveto{\pgfqpoint{1.865110in}{0.499444in}}%
\pgfpathlineto{\pgfqpoint{1.926496in}{0.499444in}}%
\pgfpathlineto{\pgfqpoint{1.926496in}{0.499444in}}%
\pgfpathlineto{\pgfqpoint{1.865110in}{0.499444in}}%
\pgfpathlineto{\pgfqpoint{1.865110in}{0.499444in}}%
\pgfpathclose%
\pgfusepath{fill}%
\end{pgfscope}%
\begin{pgfscope}%
\pgfpathrectangle{\pgfqpoint{0.445556in}{0.499444in}}{\pgfqpoint{3.875000in}{1.155000in}}%
\pgfusepath{clip}%
\pgfsetbuttcap%
\pgfsetmiterjoin%
\definecolor{currentfill}{rgb}{0.000000,0.000000,0.000000}%
\pgfsetfillcolor{currentfill}%
\pgfsetlinewidth{0.000000pt}%
\definecolor{currentstroke}{rgb}{0.000000,0.000000,0.000000}%
\pgfsetstrokecolor{currentstroke}%
\pgfsetstrokeopacity{0.000000}%
\pgfsetdash{}{0pt}%
\pgfpathmoveto{\pgfqpoint{2.018575in}{0.499444in}}%
\pgfpathlineto{\pgfqpoint{2.079962in}{0.499444in}}%
\pgfpathlineto{\pgfqpoint{2.079962in}{0.499444in}}%
\pgfpathlineto{\pgfqpoint{2.018575in}{0.499444in}}%
\pgfpathlineto{\pgfqpoint{2.018575in}{0.499444in}}%
\pgfpathclose%
\pgfusepath{fill}%
\end{pgfscope}%
\begin{pgfscope}%
\pgfpathrectangle{\pgfqpoint{0.445556in}{0.499444in}}{\pgfqpoint{3.875000in}{1.155000in}}%
\pgfusepath{clip}%
\pgfsetbuttcap%
\pgfsetmiterjoin%
\definecolor{currentfill}{rgb}{0.000000,0.000000,0.000000}%
\pgfsetfillcolor{currentfill}%
\pgfsetlinewidth{0.000000pt}%
\definecolor{currentstroke}{rgb}{0.000000,0.000000,0.000000}%
\pgfsetstrokecolor{currentstroke}%
\pgfsetstrokeopacity{0.000000}%
\pgfsetdash{}{0pt}%
\pgfpathmoveto{\pgfqpoint{2.172041in}{0.499444in}}%
\pgfpathlineto{\pgfqpoint{2.233427in}{0.499444in}}%
\pgfpathlineto{\pgfqpoint{2.233427in}{0.499444in}}%
\pgfpathlineto{\pgfqpoint{2.172041in}{0.499444in}}%
\pgfpathlineto{\pgfqpoint{2.172041in}{0.499444in}}%
\pgfpathclose%
\pgfusepath{fill}%
\end{pgfscope}%
\begin{pgfscope}%
\pgfpathrectangle{\pgfqpoint{0.445556in}{0.499444in}}{\pgfqpoint{3.875000in}{1.155000in}}%
\pgfusepath{clip}%
\pgfsetbuttcap%
\pgfsetmiterjoin%
\definecolor{currentfill}{rgb}{0.000000,0.000000,0.000000}%
\pgfsetfillcolor{currentfill}%
\pgfsetlinewidth{0.000000pt}%
\definecolor{currentstroke}{rgb}{0.000000,0.000000,0.000000}%
\pgfsetstrokecolor{currentstroke}%
\pgfsetstrokeopacity{0.000000}%
\pgfsetdash{}{0pt}%
\pgfpathmoveto{\pgfqpoint{2.325506in}{0.499444in}}%
\pgfpathlineto{\pgfqpoint{2.386892in}{0.499444in}}%
\pgfpathlineto{\pgfqpoint{2.386892in}{0.499561in}}%
\pgfpathlineto{\pgfqpoint{2.325506in}{0.499561in}}%
\pgfpathlineto{\pgfqpoint{2.325506in}{0.499444in}}%
\pgfpathclose%
\pgfusepath{fill}%
\end{pgfscope}%
\begin{pgfscope}%
\pgfpathrectangle{\pgfqpoint{0.445556in}{0.499444in}}{\pgfqpoint{3.875000in}{1.155000in}}%
\pgfusepath{clip}%
\pgfsetbuttcap%
\pgfsetmiterjoin%
\definecolor{currentfill}{rgb}{0.000000,0.000000,0.000000}%
\pgfsetfillcolor{currentfill}%
\pgfsetlinewidth{0.000000pt}%
\definecolor{currentstroke}{rgb}{0.000000,0.000000,0.000000}%
\pgfsetstrokecolor{currentstroke}%
\pgfsetstrokeopacity{0.000000}%
\pgfsetdash{}{0pt}%
\pgfpathmoveto{\pgfqpoint{2.478972in}{0.499444in}}%
\pgfpathlineto{\pgfqpoint{2.540358in}{0.499444in}}%
\pgfpathlineto{\pgfqpoint{2.540358in}{0.499854in}}%
\pgfpathlineto{\pgfqpoint{2.478972in}{0.499854in}}%
\pgfpathlineto{\pgfqpoint{2.478972in}{0.499444in}}%
\pgfpathclose%
\pgfusepath{fill}%
\end{pgfscope}%
\begin{pgfscope}%
\pgfpathrectangle{\pgfqpoint{0.445556in}{0.499444in}}{\pgfqpoint{3.875000in}{1.155000in}}%
\pgfusepath{clip}%
\pgfsetbuttcap%
\pgfsetmiterjoin%
\definecolor{currentfill}{rgb}{0.000000,0.000000,0.000000}%
\pgfsetfillcolor{currentfill}%
\pgfsetlinewidth{0.000000pt}%
\definecolor{currentstroke}{rgb}{0.000000,0.000000,0.000000}%
\pgfsetstrokecolor{currentstroke}%
\pgfsetstrokeopacity{0.000000}%
\pgfsetdash{}{0pt}%
\pgfpathmoveto{\pgfqpoint{2.632437in}{0.499444in}}%
\pgfpathlineto{\pgfqpoint{2.693823in}{0.499444in}}%
\pgfpathlineto{\pgfqpoint{2.693823in}{0.501521in}}%
\pgfpathlineto{\pgfqpoint{2.632437in}{0.501521in}}%
\pgfpathlineto{\pgfqpoint{2.632437in}{0.499444in}}%
\pgfpathclose%
\pgfusepath{fill}%
\end{pgfscope}%
\begin{pgfscope}%
\pgfpathrectangle{\pgfqpoint{0.445556in}{0.499444in}}{\pgfqpoint{3.875000in}{1.155000in}}%
\pgfusepath{clip}%
\pgfsetbuttcap%
\pgfsetmiterjoin%
\definecolor{currentfill}{rgb}{0.000000,0.000000,0.000000}%
\pgfsetfillcolor{currentfill}%
\pgfsetlinewidth{0.000000pt}%
\definecolor{currentstroke}{rgb}{0.000000,0.000000,0.000000}%
\pgfsetstrokecolor{currentstroke}%
\pgfsetstrokeopacity{0.000000}%
\pgfsetdash{}{0pt}%
\pgfpathmoveto{\pgfqpoint{2.785902in}{0.499444in}}%
\pgfpathlineto{\pgfqpoint{2.847288in}{0.499444in}}%
\pgfpathlineto{\pgfqpoint{2.847288in}{0.511171in}}%
\pgfpathlineto{\pgfqpoint{2.785902in}{0.511171in}}%
\pgfpathlineto{\pgfqpoint{2.785902in}{0.499444in}}%
\pgfpathclose%
\pgfusepath{fill}%
\end{pgfscope}%
\begin{pgfscope}%
\pgfpathrectangle{\pgfqpoint{0.445556in}{0.499444in}}{\pgfqpoint{3.875000in}{1.155000in}}%
\pgfusepath{clip}%
\pgfsetbuttcap%
\pgfsetmiterjoin%
\definecolor{currentfill}{rgb}{0.000000,0.000000,0.000000}%
\pgfsetfillcolor{currentfill}%
\pgfsetlinewidth{0.000000pt}%
\definecolor{currentstroke}{rgb}{0.000000,0.000000,0.000000}%
\pgfsetstrokecolor{currentstroke}%
\pgfsetstrokeopacity{0.000000}%
\pgfsetdash{}{0pt}%
\pgfpathmoveto{\pgfqpoint{2.939368in}{0.499444in}}%
\pgfpathlineto{\pgfqpoint{3.000754in}{0.499444in}}%
\pgfpathlineto{\pgfqpoint{3.000754in}{0.534889in}}%
\pgfpathlineto{\pgfqpoint{2.939368in}{0.534889in}}%
\pgfpathlineto{\pgfqpoint{2.939368in}{0.499444in}}%
\pgfpathclose%
\pgfusepath{fill}%
\end{pgfscope}%
\begin{pgfscope}%
\pgfpathrectangle{\pgfqpoint{0.445556in}{0.499444in}}{\pgfqpoint{3.875000in}{1.155000in}}%
\pgfusepath{clip}%
\pgfsetbuttcap%
\pgfsetmiterjoin%
\definecolor{currentfill}{rgb}{0.000000,0.000000,0.000000}%
\pgfsetfillcolor{currentfill}%
\pgfsetlinewidth{0.000000pt}%
\definecolor{currentstroke}{rgb}{0.000000,0.000000,0.000000}%
\pgfsetstrokecolor{currentstroke}%
\pgfsetstrokeopacity{0.000000}%
\pgfsetdash{}{0pt}%
\pgfpathmoveto{\pgfqpoint{3.092833in}{0.499444in}}%
\pgfpathlineto{\pgfqpoint{3.154219in}{0.499444in}}%
\pgfpathlineto{\pgfqpoint{3.154219in}{0.582998in}}%
\pgfpathlineto{\pgfqpoint{3.092833in}{0.582998in}}%
\pgfpathlineto{\pgfqpoint{3.092833in}{0.499444in}}%
\pgfpathclose%
\pgfusepath{fill}%
\end{pgfscope}%
\begin{pgfscope}%
\pgfpathrectangle{\pgfqpoint{0.445556in}{0.499444in}}{\pgfqpoint{3.875000in}{1.155000in}}%
\pgfusepath{clip}%
\pgfsetbuttcap%
\pgfsetmiterjoin%
\definecolor{currentfill}{rgb}{0.000000,0.000000,0.000000}%
\pgfsetfillcolor{currentfill}%
\pgfsetlinewidth{0.000000pt}%
\definecolor{currentstroke}{rgb}{0.000000,0.000000,0.000000}%
\pgfsetstrokecolor{currentstroke}%
\pgfsetstrokeopacity{0.000000}%
\pgfsetdash{}{0pt}%
\pgfpathmoveto{\pgfqpoint{3.246298in}{0.499444in}}%
\pgfpathlineto{\pgfqpoint{3.307684in}{0.499444in}}%
\pgfpathlineto{\pgfqpoint{3.307684in}{0.662340in}}%
\pgfpathlineto{\pgfqpoint{3.246298in}{0.662340in}}%
\pgfpathlineto{\pgfqpoint{3.246298in}{0.499444in}}%
\pgfpathclose%
\pgfusepath{fill}%
\end{pgfscope}%
\begin{pgfscope}%
\pgfpathrectangle{\pgfqpoint{0.445556in}{0.499444in}}{\pgfqpoint{3.875000in}{1.155000in}}%
\pgfusepath{clip}%
\pgfsetbuttcap%
\pgfsetmiterjoin%
\definecolor{currentfill}{rgb}{0.000000,0.000000,0.000000}%
\pgfsetfillcolor{currentfill}%
\pgfsetlinewidth{0.000000pt}%
\definecolor{currentstroke}{rgb}{0.000000,0.000000,0.000000}%
\pgfsetstrokecolor{currentstroke}%
\pgfsetstrokeopacity{0.000000}%
\pgfsetdash{}{0pt}%
\pgfpathmoveto{\pgfqpoint{3.399764in}{0.499444in}}%
\pgfpathlineto{\pgfqpoint{3.461150in}{0.499444in}}%
\pgfpathlineto{\pgfqpoint{3.461150in}{0.763967in}}%
\pgfpathlineto{\pgfqpoint{3.399764in}{0.763967in}}%
\pgfpathlineto{\pgfqpoint{3.399764in}{0.499444in}}%
\pgfpathclose%
\pgfusepath{fill}%
\end{pgfscope}%
\begin{pgfscope}%
\pgfpathrectangle{\pgfqpoint{0.445556in}{0.499444in}}{\pgfqpoint{3.875000in}{1.155000in}}%
\pgfusepath{clip}%
\pgfsetbuttcap%
\pgfsetmiterjoin%
\definecolor{currentfill}{rgb}{0.000000,0.000000,0.000000}%
\pgfsetfillcolor{currentfill}%
\pgfsetlinewidth{0.000000pt}%
\definecolor{currentstroke}{rgb}{0.000000,0.000000,0.000000}%
\pgfsetstrokecolor{currentstroke}%
\pgfsetstrokeopacity{0.000000}%
\pgfsetdash{}{0pt}%
\pgfpathmoveto{\pgfqpoint{3.553229in}{0.499444in}}%
\pgfpathlineto{\pgfqpoint{3.614615in}{0.499444in}}%
\pgfpathlineto{\pgfqpoint{3.614615in}{0.871794in}}%
\pgfpathlineto{\pgfqpoint{3.553229in}{0.871794in}}%
\pgfpathlineto{\pgfqpoint{3.553229in}{0.499444in}}%
\pgfpathclose%
\pgfusepath{fill}%
\end{pgfscope}%
\begin{pgfscope}%
\pgfpathrectangle{\pgfqpoint{0.445556in}{0.499444in}}{\pgfqpoint{3.875000in}{1.155000in}}%
\pgfusepath{clip}%
\pgfsetbuttcap%
\pgfsetmiterjoin%
\definecolor{currentfill}{rgb}{0.000000,0.000000,0.000000}%
\pgfsetfillcolor{currentfill}%
\pgfsetlinewidth{0.000000pt}%
\definecolor{currentstroke}{rgb}{0.000000,0.000000,0.000000}%
\pgfsetstrokecolor{currentstroke}%
\pgfsetstrokeopacity{0.000000}%
\pgfsetdash{}{0pt}%
\pgfpathmoveto{\pgfqpoint{3.706694in}{0.499444in}}%
\pgfpathlineto{\pgfqpoint{3.768080in}{0.499444in}}%
\pgfpathlineto{\pgfqpoint{3.768080in}{0.925810in}}%
\pgfpathlineto{\pgfqpoint{3.706694in}{0.925810in}}%
\pgfpathlineto{\pgfqpoint{3.706694in}{0.499444in}}%
\pgfpathclose%
\pgfusepath{fill}%
\end{pgfscope}%
\begin{pgfscope}%
\pgfpathrectangle{\pgfqpoint{0.445556in}{0.499444in}}{\pgfqpoint{3.875000in}{1.155000in}}%
\pgfusepath{clip}%
\pgfsetbuttcap%
\pgfsetmiterjoin%
\definecolor{currentfill}{rgb}{0.000000,0.000000,0.000000}%
\pgfsetfillcolor{currentfill}%
\pgfsetlinewidth{0.000000pt}%
\definecolor{currentstroke}{rgb}{0.000000,0.000000,0.000000}%
\pgfsetstrokecolor{currentstroke}%
\pgfsetstrokeopacity{0.000000}%
\pgfsetdash{}{0pt}%
\pgfpathmoveto{\pgfqpoint{3.860160in}{0.499444in}}%
\pgfpathlineto{\pgfqpoint{3.921546in}{0.499444in}}%
\pgfpathlineto{\pgfqpoint{3.921546in}{0.913586in}}%
\pgfpathlineto{\pgfqpoint{3.860160in}{0.913586in}}%
\pgfpathlineto{\pgfqpoint{3.860160in}{0.499444in}}%
\pgfpathclose%
\pgfusepath{fill}%
\end{pgfscope}%
\begin{pgfscope}%
\pgfpathrectangle{\pgfqpoint{0.445556in}{0.499444in}}{\pgfqpoint{3.875000in}{1.155000in}}%
\pgfusepath{clip}%
\pgfsetbuttcap%
\pgfsetmiterjoin%
\definecolor{currentfill}{rgb}{0.000000,0.000000,0.000000}%
\pgfsetfillcolor{currentfill}%
\pgfsetlinewidth{0.000000pt}%
\definecolor{currentstroke}{rgb}{0.000000,0.000000,0.000000}%
\pgfsetstrokecolor{currentstroke}%
\pgfsetstrokeopacity{0.000000}%
\pgfsetdash{}{0pt}%
\pgfpathmoveto{\pgfqpoint{4.013625in}{0.499444in}}%
\pgfpathlineto{\pgfqpoint{4.075011in}{0.499444in}}%
\pgfpathlineto{\pgfqpoint{4.075011in}{0.851907in}}%
\pgfpathlineto{\pgfqpoint{4.013625in}{0.851907in}}%
\pgfpathlineto{\pgfqpoint{4.013625in}{0.499444in}}%
\pgfpathclose%
\pgfusepath{fill}%
\end{pgfscope}%
\begin{pgfscope}%
\pgfpathrectangle{\pgfqpoint{0.445556in}{0.499444in}}{\pgfqpoint{3.875000in}{1.155000in}}%
\pgfusepath{clip}%
\pgfsetbuttcap%
\pgfsetmiterjoin%
\definecolor{currentfill}{rgb}{0.000000,0.000000,0.000000}%
\pgfsetfillcolor{currentfill}%
\pgfsetlinewidth{0.000000pt}%
\definecolor{currentstroke}{rgb}{0.000000,0.000000,0.000000}%
\pgfsetstrokecolor{currentstroke}%
\pgfsetstrokeopacity{0.000000}%
\pgfsetdash{}{0pt}%
\pgfpathmoveto{\pgfqpoint{4.167090in}{0.499444in}}%
\pgfpathlineto{\pgfqpoint{4.228476in}{0.499444in}}%
\pgfpathlineto{\pgfqpoint{4.228476in}{0.681583in}}%
\pgfpathlineto{\pgfqpoint{4.167090in}{0.681583in}}%
\pgfpathlineto{\pgfqpoint{4.167090in}{0.499444in}}%
\pgfpathclose%
\pgfusepath{fill}%
\end{pgfscope}%
\begin{pgfscope}%
\pgfsetbuttcap%
\pgfsetroundjoin%
\definecolor{currentfill}{rgb}{0.000000,0.000000,0.000000}%
\pgfsetfillcolor{currentfill}%
\pgfsetlinewidth{0.803000pt}%
\definecolor{currentstroke}{rgb}{0.000000,0.000000,0.000000}%
\pgfsetstrokecolor{currentstroke}%
\pgfsetdash{}{0pt}%
\pgfsys@defobject{currentmarker}{\pgfqpoint{0.000000in}{-0.048611in}}{\pgfqpoint{0.000000in}{0.000000in}}{%
\pgfpathmoveto{\pgfqpoint{0.000000in}{0.000000in}}%
\pgfpathlineto{\pgfqpoint{0.000000in}{-0.048611in}}%
\pgfusepath{stroke,fill}%
}%
\begin{pgfscope}%
\pgfsys@transformshift{0.483922in}{0.499444in}%
\pgfsys@useobject{currentmarker}{}%
\end{pgfscope}%
\end{pgfscope}%
\begin{pgfscope}%
\definecolor{textcolor}{rgb}{0.000000,0.000000,0.000000}%
\pgfsetstrokecolor{textcolor}%
\pgfsetfillcolor{textcolor}%
\pgftext[x=0.483922in,y=0.402222in,,top]{\color{textcolor}\rmfamily\fontsize{10.000000}{12.000000}\selectfont 0.0}%
\end{pgfscope}%
\begin{pgfscope}%
\pgfsetbuttcap%
\pgfsetroundjoin%
\definecolor{currentfill}{rgb}{0.000000,0.000000,0.000000}%
\pgfsetfillcolor{currentfill}%
\pgfsetlinewidth{0.803000pt}%
\definecolor{currentstroke}{rgb}{0.000000,0.000000,0.000000}%
\pgfsetstrokecolor{currentstroke}%
\pgfsetdash{}{0pt}%
\pgfsys@defobject{currentmarker}{\pgfqpoint{0.000000in}{-0.048611in}}{\pgfqpoint{0.000000in}{0.000000in}}{%
\pgfpathmoveto{\pgfqpoint{0.000000in}{0.000000in}}%
\pgfpathlineto{\pgfqpoint{0.000000in}{-0.048611in}}%
\pgfusepath{stroke,fill}%
}%
\begin{pgfscope}%
\pgfsys@transformshift{0.867585in}{0.499444in}%
\pgfsys@useobject{currentmarker}{}%
\end{pgfscope}%
\end{pgfscope}%
\begin{pgfscope}%
\definecolor{textcolor}{rgb}{0.000000,0.000000,0.000000}%
\pgfsetstrokecolor{textcolor}%
\pgfsetfillcolor{textcolor}%
\pgftext[x=0.867585in,y=0.402222in,,top]{\color{textcolor}\rmfamily\fontsize{10.000000}{12.000000}\selectfont 0.1}%
\end{pgfscope}%
\begin{pgfscope}%
\pgfsetbuttcap%
\pgfsetroundjoin%
\definecolor{currentfill}{rgb}{0.000000,0.000000,0.000000}%
\pgfsetfillcolor{currentfill}%
\pgfsetlinewidth{0.803000pt}%
\definecolor{currentstroke}{rgb}{0.000000,0.000000,0.000000}%
\pgfsetstrokecolor{currentstroke}%
\pgfsetdash{}{0pt}%
\pgfsys@defobject{currentmarker}{\pgfqpoint{0.000000in}{-0.048611in}}{\pgfqpoint{0.000000in}{0.000000in}}{%
\pgfpathmoveto{\pgfqpoint{0.000000in}{0.000000in}}%
\pgfpathlineto{\pgfqpoint{0.000000in}{-0.048611in}}%
\pgfusepath{stroke,fill}%
}%
\begin{pgfscope}%
\pgfsys@transformshift{1.251249in}{0.499444in}%
\pgfsys@useobject{currentmarker}{}%
\end{pgfscope}%
\end{pgfscope}%
\begin{pgfscope}%
\definecolor{textcolor}{rgb}{0.000000,0.000000,0.000000}%
\pgfsetstrokecolor{textcolor}%
\pgfsetfillcolor{textcolor}%
\pgftext[x=1.251249in,y=0.402222in,,top]{\color{textcolor}\rmfamily\fontsize{10.000000}{12.000000}\selectfont 0.2}%
\end{pgfscope}%
\begin{pgfscope}%
\pgfsetbuttcap%
\pgfsetroundjoin%
\definecolor{currentfill}{rgb}{0.000000,0.000000,0.000000}%
\pgfsetfillcolor{currentfill}%
\pgfsetlinewidth{0.803000pt}%
\definecolor{currentstroke}{rgb}{0.000000,0.000000,0.000000}%
\pgfsetstrokecolor{currentstroke}%
\pgfsetdash{}{0pt}%
\pgfsys@defobject{currentmarker}{\pgfqpoint{0.000000in}{-0.048611in}}{\pgfqpoint{0.000000in}{0.000000in}}{%
\pgfpathmoveto{\pgfqpoint{0.000000in}{0.000000in}}%
\pgfpathlineto{\pgfqpoint{0.000000in}{-0.048611in}}%
\pgfusepath{stroke,fill}%
}%
\begin{pgfscope}%
\pgfsys@transformshift{1.634912in}{0.499444in}%
\pgfsys@useobject{currentmarker}{}%
\end{pgfscope}%
\end{pgfscope}%
\begin{pgfscope}%
\definecolor{textcolor}{rgb}{0.000000,0.000000,0.000000}%
\pgfsetstrokecolor{textcolor}%
\pgfsetfillcolor{textcolor}%
\pgftext[x=1.634912in,y=0.402222in,,top]{\color{textcolor}\rmfamily\fontsize{10.000000}{12.000000}\selectfont 0.3}%
\end{pgfscope}%
\begin{pgfscope}%
\pgfsetbuttcap%
\pgfsetroundjoin%
\definecolor{currentfill}{rgb}{0.000000,0.000000,0.000000}%
\pgfsetfillcolor{currentfill}%
\pgfsetlinewidth{0.803000pt}%
\definecolor{currentstroke}{rgb}{0.000000,0.000000,0.000000}%
\pgfsetstrokecolor{currentstroke}%
\pgfsetdash{}{0pt}%
\pgfsys@defobject{currentmarker}{\pgfqpoint{0.000000in}{-0.048611in}}{\pgfqpoint{0.000000in}{0.000000in}}{%
\pgfpathmoveto{\pgfqpoint{0.000000in}{0.000000in}}%
\pgfpathlineto{\pgfqpoint{0.000000in}{-0.048611in}}%
\pgfusepath{stroke,fill}%
}%
\begin{pgfscope}%
\pgfsys@transformshift{2.018575in}{0.499444in}%
\pgfsys@useobject{currentmarker}{}%
\end{pgfscope}%
\end{pgfscope}%
\begin{pgfscope}%
\definecolor{textcolor}{rgb}{0.000000,0.000000,0.000000}%
\pgfsetstrokecolor{textcolor}%
\pgfsetfillcolor{textcolor}%
\pgftext[x=2.018575in,y=0.402222in,,top]{\color{textcolor}\rmfamily\fontsize{10.000000}{12.000000}\selectfont 0.4}%
\end{pgfscope}%
\begin{pgfscope}%
\pgfsetbuttcap%
\pgfsetroundjoin%
\definecolor{currentfill}{rgb}{0.000000,0.000000,0.000000}%
\pgfsetfillcolor{currentfill}%
\pgfsetlinewidth{0.803000pt}%
\definecolor{currentstroke}{rgb}{0.000000,0.000000,0.000000}%
\pgfsetstrokecolor{currentstroke}%
\pgfsetdash{}{0pt}%
\pgfsys@defobject{currentmarker}{\pgfqpoint{0.000000in}{-0.048611in}}{\pgfqpoint{0.000000in}{0.000000in}}{%
\pgfpathmoveto{\pgfqpoint{0.000000in}{0.000000in}}%
\pgfpathlineto{\pgfqpoint{0.000000in}{-0.048611in}}%
\pgfusepath{stroke,fill}%
}%
\begin{pgfscope}%
\pgfsys@transformshift{2.402239in}{0.499444in}%
\pgfsys@useobject{currentmarker}{}%
\end{pgfscope}%
\end{pgfscope}%
\begin{pgfscope}%
\definecolor{textcolor}{rgb}{0.000000,0.000000,0.000000}%
\pgfsetstrokecolor{textcolor}%
\pgfsetfillcolor{textcolor}%
\pgftext[x=2.402239in,y=0.402222in,,top]{\color{textcolor}\rmfamily\fontsize{10.000000}{12.000000}\selectfont 0.5}%
\end{pgfscope}%
\begin{pgfscope}%
\pgfsetbuttcap%
\pgfsetroundjoin%
\definecolor{currentfill}{rgb}{0.000000,0.000000,0.000000}%
\pgfsetfillcolor{currentfill}%
\pgfsetlinewidth{0.803000pt}%
\definecolor{currentstroke}{rgb}{0.000000,0.000000,0.000000}%
\pgfsetstrokecolor{currentstroke}%
\pgfsetdash{}{0pt}%
\pgfsys@defobject{currentmarker}{\pgfqpoint{0.000000in}{-0.048611in}}{\pgfqpoint{0.000000in}{0.000000in}}{%
\pgfpathmoveto{\pgfqpoint{0.000000in}{0.000000in}}%
\pgfpathlineto{\pgfqpoint{0.000000in}{-0.048611in}}%
\pgfusepath{stroke,fill}%
}%
\begin{pgfscope}%
\pgfsys@transformshift{2.785902in}{0.499444in}%
\pgfsys@useobject{currentmarker}{}%
\end{pgfscope}%
\end{pgfscope}%
\begin{pgfscope}%
\definecolor{textcolor}{rgb}{0.000000,0.000000,0.000000}%
\pgfsetstrokecolor{textcolor}%
\pgfsetfillcolor{textcolor}%
\pgftext[x=2.785902in,y=0.402222in,,top]{\color{textcolor}\rmfamily\fontsize{10.000000}{12.000000}\selectfont 0.6}%
\end{pgfscope}%
\begin{pgfscope}%
\pgfsetbuttcap%
\pgfsetroundjoin%
\definecolor{currentfill}{rgb}{0.000000,0.000000,0.000000}%
\pgfsetfillcolor{currentfill}%
\pgfsetlinewidth{0.803000pt}%
\definecolor{currentstroke}{rgb}{0.000000,0.000000,0.000000}%
\pgfsetstrokecolor{currentstroke}%
\pgfsetdash{}{0pt}%
\pgfsys@defobject{currentmarker}{\pgfqpoint{0.000000in}{-0.048611in}}{\pgfqpoint{0.000000in}{0.000000in}}{%
\pgfpathmoveto{\pgfqpoint{0.000000in}{0.000000in}}%
\pgfpathlineto{\pgfqpoint{0.000000in}{-0.048611in}}%
\pgfusepath{stroke,fill}%
}%
\begin{pgfscope}%
\pgfsys@transformshift{3.169566in}{0.499444in}%
\pgfsys@useobject{currentmarker}{}%
\end{pgfscope}%
\end{pgfscope}%
\begin{pgfscope}%
\definecolor{textcolor}{rgb}{0.000000,0.000000,0.000000}%
\pgfsetstrokecolor{textcolor}%
\pgfsetfillcolor{textcolor}%
\pgftext[x=3.169566in,y=0.402222in,,top]{\color{textcolor}\rmfamily\fontsize{10.000000}{12.000000}\selectfont 0.7}%
\end{pgfscope}%
\begin{pgfscope}%
\pgfsetbuttcap%
\pgfsetroundjoin%
\definecolor{currentfill}{rgb}{0.000000,0.000000,0.000000}%
\pgfsetfillcolor{currentfill}%
\pgfsetlinewidth{0.803000pt}%
\definecolor{currentstroke}{rgb}{0.000000,0.000000,0.000000}%
\pgfsetstrokecolor{currentstroke}%
\pgfsetdash{}{0pt}%
\pgfsys@defobject{currentmarker}{\pgfqpoint{0.000000in}{-0.048611in}}{\pgfqpoint{0.000000in}{0.000000in}}{%
\pgfpathmoveto{\pgfqpoint{0.000000in}{0.000000in}}%
\pgfpathlineto{\pgfqpoint{0.000000in}{-0.048611in}}%
\pgfusepath{stroke,fill}%
}%
\begin{pgfscope}%
\pgfsys@transformshift{3.553229in}{0.499444in}%
\pgfsys@useobject{currentmarker}{}%
\end{pgfscope}%
\end{pgfscope}%
\begin{pgfscope}%
\definecolor{textcolor}{rgb}{0.000000,0.000000,0.000000}%
\pgfsetstrokecolor{textcolor}%
\pgfsetfillcolor{textcolor}%
\pgftext[x=3.553229in,y=0.402222in,,top]{\color{textcolor}\rmfamily\fontsize{10.000000}{12.000000}\selectfont 0.8}%
\end{pgfscope}%
\begin{pgfscope}%
\pgfsetbuttcap%
\pgfsetroundjoin%
\definecolor{currentfill}{rgb}{0.000000,0.000000,0.000000}%
\pgfsetfillcolor{currentfill}%
\pgfsetlinewidth{0.803000pt}%
\definecolor{currentstroke}{rgb}{0.000000,0.000000,0.000000}%
\pgfsetstrokecolor{currentstroke}%
\pgfsetdash{}{0pt}%
\pgfsys@defobject{currentmarker}{\pgfqpoint{0.000000in}{-0.048611in}}{\pgfqpoint{0.000000in}{0.000000in}}{%
\pgfpathmoveto{\pgfqpoint{0.000000in}{0.000000in}}%
\pgfpathlineto{\pgfqpoint{0.000000in}{-0.048611in}}%
\pgfusepath{stroke,fill}%
}%
\begin{pgfscope}%
\pgfsys@transformshift{3.936892in}{0.499444in}%
\pgfsys@useobject{currentmarker}{}%
\end{pgfscope}%
\end{pgfscope}%
\begin{pgfscope}%
\definecolor{textcolor}{rgb}{0.000000,0.000000,0.000000}%
\pgfsetstrokecolor{textcolor}%
\pgfsetfillcolor{textcolor}%
\pgftext[x=3.936892in,y=0.402222in,,top]{\color{textcolor}\rmfamily\fontsize{10.000000}{12.000000}\selectfont 0.9}%
\end{pgfscope}%
\begin{pgfscope}%
\pgfsetbuttcap%
\pgfsetroundjoin%
\definecolor{currentfill}{rgb}{0.000000,0.000000,0.000000}%
\pgfsetfillcolor{currentfill}%
\pgfsetlinewidth{0.803000pt}%
\definecolor{currentstroke}{rgb}{0.000000,0.000000,0.000000}%
\pgfsetstrokecolor{currentstroke}%
\pgfsetdash{}{0pt}%
\pgfsys@defobject{currentmarker}{\pgfqpoint{0.000000in}{-0.048611in}}{\pgfqpoint{0.000000in}{0.000000in}}{%
\pgfpathmoveto{\pgfqpoint{0.000000in}{0.000000in}}%
\pgfpathlineto{\pgfqpoint{0.000000in}{-0.048611in}}%
\pgfusepath{stroke,fill}%
}%
\begin{pgfscope}%
\pgfsys@transformshift{4.320556in}{0.499444in}%
\pgfsys@useobject{currentmarker}{}%
\end{pgfscope}%
\end{pgfscope}%
\begin{pgfscope}%
\definecolor{textcolor}{rgb}{0.000000,0.000000,0.000000}%
\pgfsetstrokecolor{textcolor}%
\pgfsetfillcolor{textcolor}%
\pgftext[x=4.320556in,y=0.402222in,,top]{\color{textcolor}\rmfamily\fontsize{10.000000}{12.000000}\selectfont 1.0}%
\end{pgfscope}%
\begin{pgfscope}%
\definecolor{textcolor}{rgb}{0.000000,0.000000,0.000000}%
\pgfsetstrokecolor{textcolor}%
\pgfsetfillcolor{textcolor}%
\pgftext[x=2.383056in,y=0.223333in,,top]{\color{textcolor}\rmfamily\fontsize{10.000000}{12.000000}\selectfont \(\displaystyle p\)}%
\end{pgfscope}%
\begin{pgfscope}%
\pgfsetbuttcap%
\pgfsetroundjoin%
\definecolor{currentfill}{rgb}{0.000000,0.000000,0.000000}%
\pgfsetfillcolor{currentfill}%
\pgfsetlinewidth{0.803000pt}%
\definecolor{currentstroke}{rgb}{0.000000,0.000000,0.000000}%
\pgfsetstrokecolor{currentstroke}%
\pgfsetdash{}{0pt}%
\pgfsys@defobject{currentmarker}{\pgfqpoint{-0.048611in}{0.000000in}}{\pgfqpoint{-0.000000in}{0.000000in}}{%
\pgfpathmoveto{\pgfqpoint{-0.000000in}{0.000000in}}%
\pgfpathlineto{\pgfqpoint{-0.048611in}{0.000000in}}%
\pgfusepath{stroke,fill}%
}%
\begin{pgfscope}%
\pgfsys@transformshift{0.445556in}{0.499444in}%
\pgfsys@useobject{currentmarker}{}%
\end{pgfscope}%
\end{pgfscope}%
\begin{pgfscope}%
\definecolor{textcolor}{rgb}{0.000000,0.000000,0.000000}%
\pgfsetstrokecolor{textcolor}%
\pgfsetfillcolor{textcolor}%
\pgftext[x=0.278889in, y=0.451250in, left, base]{\color{textcolor}\rmfamily\fontsize{10.000000}{12.000000}\selectfont \(\displaystyle {0}\)}%
\end{pgfscope}%
\begin{pgfscope}%
\pgfsetbuttcap%
\pgfsetroundjoin%
\definecolor{currentfill}{rgb}{0.000000,0.000000,0.000000}%
\pgfsetfillcolor{currentfill}%
\pgfsetlinewidth{0.803000pt}%
\definecolor{currentstroke}{rgb}{0.000000,0.000000,0.000000}%
\pgfsetstrokecolor{currentstroke}%
\pgfsetdash{}{0pt}%
\pgfsys@defobject{currentmarker}{\pgfqpoint{-0.048611in}{0.000000in}}{\pgfqpoint{-0.000000in}{0.000000in}}{%
\pgfpathmoveto{\pgfqpoint{-0.000000in}{0.000000in}}%
\pgfpathlineto{\pgfqpoint{-0.048611in}{0.000000in}}%
\pgfusepath{stroke,fill}%
}%
\begin{pgfscope}%
\pgfsys@transformshift{0.445556in}{0.791601in}%
\pgfsys@useobject{currentmarker}{}%
\end{pgfscope}%
\end{pgfscope}%
\begin{pgfscope}%
\definecolor{textcolor}{rgb}{0.000000,0.000000,0.000000}%
\pgfsetstrokecolor{textcolor}%
\pgfsetfillcolor{textcolor}%
\pgftext[x=0.278889in, y=0.743407in, left, base]{\color{textcolor}\rmfamily\fontsize{10.000000}{12.000000}\selectfont \(\displaystyle {2}\)}%
\end{pgfscope}%
\begin{pgfscope}%
\pgfsetbuttcap%
\pgfsetroundjoin%
\definecolor{currentfill}{rgb}{0.000000,0.000000,0.000000}%
\pgfsetfillcolor{currentfill}%
\pgfsetlinewidth{0.803000pt}%
\definecolor{currentstroke}{rgb}{0.000000,0.000000,0.000000}%
\pgfsetstrokecolor{currentstroke}%
\pgfsetdash{}{0pt}%
\pgfsys@defobject{currentmarker}{\pgfqpoint{-0.048611in}{0.000000in}}{\pgfqpoint{-0.000000in}{0.000000in}}{%
\pgfpathmoveto{\pgfqpoint{-0.000000in}{0.000000in}}%
\pgfpathlineto{\pgfqpoint{-0.048611in}{0.000000in}}%
\pgfusepath{stroke,fill}%
}%
\begin{pgfscope}%
\pgfsys@transformshift{0.445556in}{1.083759in}%
\pgfsys@useobject{currentmarker}{}%
\end{pgfscope}%
\end{pgfscope}%
\begin{pgfscope}%
\definecolor{textcolor}{rgb}{0.000000,0.000000,0.000000}%
\pgfsetstrokecolor{textcolor}%
\pgfsetfillcolor{textcolor}%
\pgftext[x=0.278889in, y=1.035564in, left, base]{\color{textcolor}\rmfamily\fontsize{10.000000}{12.000000}\selectfont \(\displaystyle {4}\)}%
\end{pgfscope}%
\begin{pgfscope}%
\pgfsetbuttcap%
\pgfsetroundjoin%
\definecolor{currentfill}{rgb}{0.000000,0.000000,0.000000}%
\pgfsetfillcolor{currentfill}%
\pgfsetlinewidth{0.803000pt}%
\definecolor{currentstroke}{rgb}{0.000000,0.000000,0.000000}%
\pgfsetstrokecolor{currentstroke}%
\pgfsetdash{}{0pt}%
\pgfsys@defobject{currentmarker}{\pgfqpoint{-0.048611in}{0.000000in}}{\pgfqpoint{-0.000000in}{0.000000in}}{%
\pgfpathmoveto{\pgfqpoint{-0.000000in}{0.000000in}}%
\pgfpathlineto{\pgfqpoint{-0.048611in}{0.000000in}}%
\pgfusepath{stroke,fill}%
}%
\begin{pgfscope}%
\pgfsys@transformshift{0.445556in}{1.375916in}%
\pgfsys@useobject{currentmarker}{}%
\end{pgfscope}%
\end{pgfscope}%
\begin{pgfscope}%
\definecolor{textcolor}{rgb}{0.000000,0.000000,0.000000}%
\pgfsetstrokecolor{textcolor}%
\pgfsetfillcolor{textcolor}%
\pgftext[x=0.278889in, y=1.327722in, left, base]{\color{textcolor}\rmfamily\fontsize{10.000000}{12.000000}\selectfont \(\displaystyle {6}\)}%
\end{pgfscope}%
\begin{pgfscope}%
\definecolor{textcolor}{rgb}{0.000000,0.000000,0.000000}%
\pgfsetstrokecolor{textcolor}%
\pgfsetfillcolor{textcolor}%
\pgftext[x=0.223333in,y=1.076944in,,bottom,rotate=90.000000]{\color{textcolor}\rmfamily\fontsize{10.000000}{12.000000}\selectfont Percent of Data Set}%
\end{pgfscope}%
\begin{pgfscope}%
\pgfsetrectcap%
\pgfsetmiterjoin%
\pgfsetlinewidth{0.803000pt}%
\definecolor{currentstroke}{rgb}{0.000000,0.000000,0.000000}%
\pgfsetstrokecolor{currentstroke}%
\pgfsetdash{}{0pt}%
\pgfpathmoveto{\pgfqpoint{0.445556in}{0.499444in}}%
\pgfpathlineto{\pgfqpoint{0.445556in}{1.654444in}}%
\pgfusepath{stroke}%
\end{pgfscope}%
\begin{pgfscope}%
\pgfsetrectcap%
\pgfsetmiterjoin%
\pgfsetlinewidth{0.803000pt}%
\definecolor{currentstroke}{rgb}{0.000000,0.000000,0.000000}%
\pgfsetstrokecolor{currentstroke}%
\pgfsetdash{}{0pt}%
\pgfpathmoveto{\pgfqpoint{4.320556in}{0.499444in}}%
\pgfpathlineto{\pgfqpoint{4.320556in}{1.654444in}}%
\pgfusepath{stroke}%
\end{pgfscope}%
\begin{pgfscope}%
\pgfsetrectcap%
\pgfsetmiterjoin%
\pgfsetlinewidth{0.803000pt}%
\definecolor{currentstroke}{rgb}{0.000000,0.000000,0.000000}%
\pgfsetstrokecolor{currentstroke}%
\pgfsetdash{}{0pt}%
\pgfpathmoveto{\pgfqpoint{0.445556in}{0.499444in}}%
\pgfpathlineto{\pgfqpoint{4.320556in}{0.499444in}}%
\pgfusepath{stroke}%
\end{pgfscope}%
\begin{pgfscope}%
\pgfsetrectcap%
\pgfsetmiterjoin%
\pgfsetlinewidth{0.803000pt}%
\definecolor{currentstroke}{rgb}{0.000000,0.000000,0.000000}%
\pgfsetstrokecolor{currentstroke}%
\pgfsetdash{}{0pt}%
\pgfpathmoveto{\pgfqpoint{0.445556in}{1.654444in}}%
\pgfpathlineto{\pgfqpoint{4.320556in}{1.654444in}}%
\pgfusepath{stroke}%
\end{pgfscope}%
\begin{pgfscope}%
\pgfsetbuttcap%
\pgfsetmiterjoin%
\definecolor{currentfill}{rgb}{1.000000,1.000000,1.000000}%
\pgfsetfillcolor{currentfill}%
\pgfsetfillopacity{0.800000}%
\pgfsetlinewidth{1.003750pt}%
\definecolor{currentstroke}{rgb}{0.800000,0.800000,0.800000}%
\pgfsetstrokecolor{currentstroke}%
\pgfsetstrokeopacity{0.800000}%
\pgfsetdash{}{0pt}%
\pgfpathmoveto{\pgfqpoint{3.543611in}{1.154445in}}%
\pgfpathlineto{\pgfqpoint{4.223333in}{1.154445in}}%
\pgfpathquadraticcurveto{\pgfqpoint{4.251111in}{1.154445in}}{\pgfqpoint{4.251111in}{1.182222in}}%
\pgfpathlineto{\pgfqpoint{4.251111in}{1.557222in}}%
\pgfpathquadraticcurveto{\pgfqpoint{4.251111in}{1.585000in}}{\pgfqpoint{4.223333in}{1.585000in}}%
\pgfpathlineto{\pgfqpoint{3.543611in}{1.585000in}}%
\pgfpathquadraticcurveto{\pgfqpoint{3.515833in}{1.585000in}}{\pgfqpoint{3.515833in}{1.557222in}}%
\pgfpathlineto{\pgfqpoint{3.515833in}{1.182222in}}%
\pgfpathquadraticcurveto{\pgfqpoint{3.515833in}{1.154445in}}{\pgfqpoint{3.543611in}{1.154445in}}%
\pgfpathlineto{\pgfqpoint{3.543611in}{1.154445in}}%
\pgfpathclose%
\pgfusepath{stroke,fill}%
\end{pgfscope}%
\begin{pgfscope}%
\pgfsetbuttcap%
\pgfsetmiterjoin%
\pgfsetlinewidth{1.003750pt}%
\definecolor{currentstroke}{rgb}{0.000000,0.000000,0.000000}%
\pgfsetstrokecolor{currentstroke}%
\pgfsetdash{}{0pt}%
\pgfpathmoveto{\pgfqpoint{3.571389in}{1.432222in}}%
\pgfpathlineto{\pgfqpoint{3.849167in}{1.432222in}}%
\pgfpathlineto{\pgfqpoint{3.849167in}{1.529444in}}%
\pgfpathlineto{\pgfqpoint{3.571389in}{1.529444in}}%
\pgfpathlineto{\pgfqpoint{3.571389in}{1.432222in}}%
\pgfpathclose%
\pgfusepath{stroke}%
\end{pgfscope}%
\begin{pgfscope}%
\definecolor{textcolor}{rgb}{0.000000,0.000000,0.000000}%
\pgfsetstrokecolor{textcolor}%
\pgfsetfillcolor{textcolor}%
\pgftext[x=3.960278in,y=1.432222in,left,base]{\color{textcolor}\rmfamily\fontsize{10.000000}{12.000000}\selectfont Neg}%
\end{pgfscope}%
\begin{pgfscope}%
\pgfsetbuttcap%
\pgfsetmiterjoin%
\definecolor{currentfill}{rgb}{0.000000,0.000000,0.000000}%
\pgfsetfillcolor{currentfill}%
\pgfsetlinewidth{0.000000pt}%
\definecolor{currentstroke}{rgb}{0.000000,0.000000,0.000000}%
\pgfsetstrokecolor{currentstroke}%
\pgfsetstrokeopacity{0.000000}%
\pgfsetdash{}{0pt}%
\pgfpathmoveto{\pgfqpoint{3.571389in}{1.236944in}}%
\pgfpathlineto{\pgfqpoint{3.849167in}{1.236944in}}%
\pgfpathlineto{\pgfqpoint{3.849167in}{1.334167in}}%
\pgfpathlineto{\pgfqpoint{3.571389in}{1.334167in}}%
\pgfpathlineto{\pgfqpoint{3.571389in}{1.236944in}}%
\pgfpathclose%
\pgfusepath{fill}%
\end{pgfscope}%
\begin{pgfscope}%
\definecolor{textcolor}{rgb}{0.000000,0.000000,0.000000}%
\pgfsetstrokecolor{textcolor}%
\pgfsetfillcolor{textcolor}%
\pgftext[x=3.960278in,y=1.236944in,left,base]{\color{textcolor}\rmfamily\fontsize{10.000000}{12.000000}\selectfont Pos}%
\end{pgfscope}%
\end{pgfpicture}%
\makeatother%
\endgroup%
	
&
	\vskip 0pt
	\hfil ROC Curve
	
	%% Creator: Matplotlib, PGF backend
%%
%% To include the figure in your LaTeX document, write
%%   \input{<filename>.pgf}
%%
%% Make sure the required packages are loaded in your preamble
%%   \usepackage{pgf}
%%
%% Also ensure that all the required font packages are loaded; for instance,
%% the lmodern package is sometimes necessary when using math font.
%%   \usepackage{lmodern}
%%
%% Figures using additional raster images can only be included by \input if
%% they are in the same directory as the main LaTeX file. For loading figures
%% from other directories you can use the `import` package
%%   \usepackage{import}
%%
%% and then include the figures with
%%   \import{<path to file>}{<filename>.pgf}
%%
%% Matplotlib used the following preamble
%%   
%%   \usepackage{fontspec}
%%   \makeatletter\@ifpackageloaded{underscore}{}{\usepackage[strings]{underscore}}\makeatother
%%
\begingroup%
\makeatletter%
\begin{pgfpicture}%
\pgfpathrectangle{\pgfpointorigin}{\pgfqpoint{2.221861in}{1.754444in}}%
\pgfusepath{use as bounding box, clip}%
\begin{pgfscope}%
\pgfsetbuttcap%
\pgfsetmiterjoin%
\definecolor{currentfill}{rgb}{1.000000,1.000000,1.000000}%
\pgfsetfillcolor{currentfill}%
\pgfsetlinewidth{0.000000pt}%
\definecolor{currentstroke}{rgb}{1.000000,1.000000,1.000000}%
\pgfsetstrokecolor{currentstroke}%
\pgfsetdash{}{0pt}%
\pgfpathmoveto{\pgfqpoint{0.000000in}{0.000000in}}%
\pgfpathlineto{\pgfqpoint{2.221861in}{0.000000in}}%
\pgfpathlineto{\pgfqpoint{2.221861in}{1.754444in}}%
\pgfpathlineto{\pgfqpoint{0.000000in}{1.754444in}}%
\pgfpathlineto{\pgfqpoint{0.000000in}{0.000000in}}%
\pgfpathclose%
\pgfusepath{fill}%
\end{pgfscope}%
\begin{pgfscope}%
\pgfsetbuttcap%
\pgfsetmiterjoin%
\definecolor{currentfill}{rgb}{1.000000,1.000000,1.000000}%
\pgfsetfillcolor{currentfill}%
\pgfsetlinewidth{0.000000pt}%
\definecolor{currentstroke}{rgb}{0.000000,0.000000,0.000000}%
\pgfsetstrokecolor{currentstroke}%
\pgfsetstrokeopacity{0.000000}%
\pgfsetdash{}{0pt}%
\pgfpathmoveto{\pgfqpoint{0.553581in}{0.499444in}}%
\pgfpathlineto{\pgfqpoint{2.103581in}{0.499444in}}%
\pgfpathlineto{\pgfqpoint{2.103581in}{1.654444in}}%
\pgfpathlineto{\pgfqpoint{0.553581in}{1.654444in}}%
\pgfpathlineto{\pgfqpoint{0.553581in}{0.499444in}}%
\pgfpathclose%
\pgfusepath{fill}%
\end{pgfscope}%
\begin{pgfscope}%
\pgfsetbuttcap%
\pgfsetroundjoin%
\definecolor{currentfill}{rgb}{0.000000,0.000000,0.000000}%
\pgfsetfillcolor{currentfill}%
\pgfsetlinewidth{0.803000pt}%
\definecolor{currentstroke}{rgb}{0.000000,0.000000,0.000000}%
\pgfsetstrokecolor{currentstroke}%
\pgfsetdash{}{0pt}%
\pgfsys@defobject{currentmarker}{\pgfqpoint{0.000000in}{-0.048611in}}{\pgfqpoint{0.000000in}{0.000000in}}{%
\pgfpathmoveto{\pgfqpoint{0.000000in}{0.000000in}}%
\pgfpathlineto{\pgfqpoint{0.000000in}{-0.048611in}}%
\pgfusepath{stroke,fill}%
}%
\begin{pgfscope}%
\pgfsys@transformshift{0.624035in}{0.499444in}%
\pgfsys@useobject{currentmarker}{}%
\end{pgfscope}%
\end{pgfscope}%
\begin{pgfscope}%
\definecolor{textcolor}{rgb}{0.000000,0.000000,0.000000}%
\pgfsetstrokecolor{textcolor}%
\pgfsetfillcolor{textcolor}%
\pgftext[x=0.624035in,y=0.402222in,,top]{\color{textcolor}\rmfamily\fontsize{10.000000}{12.000000}\selectfont \(\displaystyle {0.0}\)}%
\end{pgfscope}%
\begin{pgfscope}%
\pgfsetbuttcap%
\pgfsetroundjoin%
\definecolor{currentfill}{rgb}{0.000000,0.000000,0.000000}%
\pgfsetfillcolor{currentfill}%
\pgfsetlinewidth{0.803000pt}%
\definecolor{currentstroke}{rgb}{0.000000,0.000000,0.000000}%
\pgfsetstrokecolor{currentstroke}%
\pgfsetdash{}{0pt}%
\pgfsys@defobject{currentmarker}{\pgfqpoint{0.000000in}{-0.048611in}}{\pgfqpoint{0.000000in}{0.000000in}}{%
\pgfpathmoveto{\pgfqpoint{0.000000in}{0.000000in}}%
\pgfpathlineto{\pgfqpoint{0.000000in}{-0.048611in}}%
\pgfusepath{stroke,fill}%
}%
\begin{pgfscope}%
\pgfsys@transformshift{1.328581in}{0.499444in}%
\pgfsys@useobject{currentmarker}{}%
\end{pgfscope}%
\end{pgfscope}%
\begin{pgfscope}%
\definecolor{textcolor}{rgb}{0.000000,0.000000,0.000000}%
\pgfsetstrokecolor{textcolor}%
\pgfsetfillcolor{textcolor}%
\pgftext[x=1.328581in,y=0.402222in,,top]{\color{textcolor}\rmfamily\fontsize{10.000000}{12.000000}\selectfont \(\displaystyle {0.5}\)}%
\end{pgfscope}%
\begin{pgfscope}%
\pgfsetbuttcap%
\pgfsetroundjoin%
\definecolor{currentfill}{rgb}{0.000000,0.000000,0.000000}%
\pgfsetfillcolor{currentfill}%
\pgfsetlinewidth{0.803000pt}%
\definecolor{currentstroke}{rgb}{0.000000,0.000000,0.000000}%
\pgfsetstrokecolor{currentstroke}%
\pgfsetdash{}{0pt}%
\pgfsys@defobject{currentmarker}{\pgfqpoint{0.000000in}{-0.048611in}}{\pgfqpoint{0.000000in}{0.000000in}}{%
\pgfpathmoveto{\pgfqpoint{0.000000in}{0.000000in}}%
\pgfpathlineto{\pgfqpoint{0.000000in}{-0.048611in}}%
\pgfusepath{stroke,fill}%
}%
\begin{pgfscope}%
\pgfsys@transformshift{2.033126in}{0.499444in}%
\pgfsys@useobject{currentmarker}{}%
\end{pgfscope}%
\end{pgfscope}%
\begin{pgfscope}%
\definecolor{textcolor}{rgb}{0.000000,0.000000,0.000000}%
\pgfsetstrokecolor{textcolor}%
\pgfsetfillcolor{textcolor}%
\pgftext[x=2.033126in,y=0.402222in,,top]{\color{textcolor}\rmfamily\fontsize{10.000000}{12.000000}\selectfont \(\displaystyle {1.0}\)}%
\end{pgfscope}%
\begin{pgfscope}%
\definecolor{textcolor}{rgb}{0.000000,0.000000,0.000000}%
\pgfsetstrokecolor{textcolor}%
\pgfsetfillcolor{textcolor}%
\pgftext[x=1.328581in,y=0.223333in,,top]{\color{textcolor}\rmfamily\fontsize{10.000000}{12.000000}\selectfont False positive rate}%
\end{pgfscope}%
\begin{pgfscope}%
\pgfsetbuttcap%
\pgfsetroundjoin%
\definecolor{currentfill}{rgb}{0.000000,0.000000,0.000000}%
\pgfsetfillcolor{currentfill}%
\pgfsetlinewidth{0.803000pt}%
\definecolor{currentstroke}{rgb}{0.000000,0.000000,0.000000}%
\pgfsetstrokecolor{currentstroke}%
\pgfsetdash{}{0pt}%
\pgfsys@defobject{currentmarker}{\pgfqpoint{-0.048611in}{0.000000in}}{\pgfqpoint{-0.000000in}{0.000000in}}{%
\pgfpathmoveto{\pgfqpoint{-0.000000in}{0.000000in}}%
\pgfpathlineto{\pgfqpoint{-0.048611in}{0.000000in}}%
\pgfusepath{stroke,fill}%
}%
\begin{pgfscope}%
\pgfsys@transformshift{0.553581in}{0.551944in}%
\pgfsys@useobject{currentmarker}{}%
\end{pgfscope}%
\end{pgfscope}%
\begin{pgfscope}%
\definecolor{textcolor}{rgb}{0.000000,0.000000,0.000000}%
\pgfsetstrokecolor{textcolor}%
\pgfsetfillcolor{textcolor}%
\pgftext[x=0.278889in, y=0.503750in, left, base]{\color{textcolor}\rmfamily\fontsize{10.000000}{12.000000}\selectfont \(\displaystyle {0.0}\)}%
\end{pgfscope}%
\begin{pgfscope}%
\pgfsetbuttcap%
\pgfsetroundjoin%
\definecolor{currentfill}{rgb}{0.000000,0.000000,0.000000}%
\pgfsetfillcolor{currentfill}%
\pgfsetlinewidth{0.803000pt}%
\definecolor{currentstroke}{rgb}{0.000000,0.000000,0.000000}%
\pgfsetstrokecolor{currentstroke}%
\pgfsetdash{}{0pt}%
\pgfsys@defobject{currentmarker}{\pgfqpoint{-0.048611in}{0.000000in}}{\pgfqpoint{-0.000000in}{0.000000in}}{%
\pgfpathmoveto{\pgfqpoint{-0.000000in}{0.000000in}}%
\pgfpathlineto{\pgfqpoint{-0.048611in}{0.000000in}}%
\pgfusepath{stroke,fill}%
}%
\begin{pgfscope}%
\pgfsys@transformshift{0.553581in}{1.076944in}%
\pgfsys@useobject{currentmarker}{}%
\end{pgfscope}%
\end{pgfscope}%
\begin{pgfscope}%
\definecolor{textcolor}{rgb}{0.000000,0.000000,0.000000}%
\pgfsetstrokecolor{textcolor}%
\pgfsetfillcolor{textcolor}%
\pgftext[x=0.278889in, y=1.028750in, left, base]{\color{textcolor}\rmfamily\fontsize{10.000000}{12.000000}\selectfont \(\displaystyle {0.5}\)}%
\end{pgfscope}%
\begin{pgfscope}%
\pgfsetbuttcap%
\pgfsetroundjoin%
\definecolor{currentfill}{rgb}{0.000000,0.000000,0.000000}%
\pgfsetfillcolor{currentfill}%
\pgfsetlinewidth{0.803000pt}%
\definecolor{currentstroke}{rgb}{0.000000,0.000000,0.000000}%
\pgfsetstrokecolor{currentstroke}%
\pgfsetdash{}{0pt}%
\pgfsys@defobject{currentmarker}{\pgfqpoint{-0.048611in}{0.000000in}}{\pgfqpoint{-0.000000in}{0.000000in}}{%
\pgfpathmoveto{\pgfqpoint{-0.000000in}{0.000000in}}%
\pgfpathlineto{\pgfqpoint{-0.048611in}{0.000000in}}%
\pgfusepath{stroke,fill}%
}%
\begin{pgfscope}%
\pgfsys@transformshift{0.553581in}{1.601944in}%
\pgfsys@useobject{currentmarker}{}%
\end{pgfscope}%
\end{pgfscope}%
\begin{pgfscope}%
\definecolor{textcolor}{rgb}{0.000000,0.000000,0.000000}%
\pgfsetstrokecolor{textcolor}%
\pgfsetfillcolor{textcolor}%
\pgftext[x=0.278889in, y=1.553750in, left, base]{\color{textcolor}\rmfamily\fontsize{10.000000}{12.000000}\selectfont \(\displaystyle {1.0}\)}%
\end{pgfscope}%
\begin{pgfscope}%
\definecolor{textcolor}{rgb}{0.000000,0.000000,0.000000}%
\pgfsetstrokecolor{textcolor}%
\pgfsetfillcolor{textcolor}%
\pgftext[x=0.223333in,y=1.076944in,,bottom,rotate=90.000000]{\color{textcolor}\rmfamily\fontsize{10.000000}{12.000000}\selectfont True positive rate}%
\end{pgfscope}%
\begin{pgfscope}%
\pgfpathrectangle{\pgfqpoint{0.553581in}{0.499444in}}{\pgfqpoint{1.550000in}{1.155000in}}%
\pgfusepath{clip}%
\pgfsetbuttcap%
\pgfsetroundjoin%
\pgfsetlinewidth{1.505625pt}%
\definecolor{currentstroke}{rgb}{0.000000,0.000000,0.000000}%
\pgfsetstrokecolor{currentstroke}%
\pgfsetdash{{5.550000pt}{2.400000pt}}{0.000000pt}%
\pgfpathmoveto{\pgfqpoint{0.624035in}{0.551944in}}%
\pgfpathlineto{\pgfqpoint{2.033126in}{1.601944in}}%
\pgfusepath{stroke}%
\end{pgfscope}%
\begin{pgfscope}%
\pgfpathrectangle{\pgfqpoint{0.553581in}{0.499444in}}{\pgfqpoint{1.550000in}{1.155000in}}%
\pgfusepath{clip}%
\pgfsetrectcap%
\pgfsetroundjoin%
\pgfsetlinewidth{1.505625pt}%
\definecolor{currentstroke}{rgb}{0.000000,0.000000,0.000000}%
\pgfsetstrokecolor{currentstroke}%
\pgfsetdash{}{0pt}%
\pgfpathmoveto{\pgfqpoint{0.624035in}{0.551944in}}%
\pgfpathlineto{\pgfqpoint{0.625087in}{1.081068in}}%
\pgfpathlineto{\pgfqpoint{0.626766in}{1.222951in}}%
\pgfpathlineto{\pgfqpoint{0.630063in}{1.346143in}}%
\pgfpathlineto{\pgfqpoint{0.630170in}{1.346861in}}%
\pgfpathlineto{\pgfqpoint{0.633316in}{1.414483in}}%
\pgfpathlineto{\pgfqpoint{0.637112in}{1.467192in}}%
\pgfpathlineto{\pgfqpoint{0.642134in}{1.509564in}}%
\pgfpathlineto{\pgfqpoint{0.648493in}{1.541240in}}%
\pgfpathlineto{\pgfqpoint{0.648510in}{1.541293in}}%
\pgfpathlineto{\pgfqpoint{0.655596in}{1.563550in}}%
\pgfpathlineto{\pgfqpoint{0.660310in}{1.572583in}}%
\pgfpathlineto{\pgfqpoint{0.670656in}{1.585341in}}%
\pgfpathlineto{\pgfqpoint{0.682708in}{1.593071in}}%
\pgfpathlineto{\pgfqpoint{0.683120in}{1.593230in}}%
\pgfpathlineto{\pgfqpoint{0.697859in}{1.597421in}}%
\pgfpathlineto{\pgfqpoint{0.715458in}{1.600108in}}%
\pgfpathlineto{\pgfqpoint{0.747388in}{1.601439in}}%
\pgfpathlineto{\pgfqpoint{0.871277in}{1.601931in}}%
\pgfpathlineto{\pgfqpoint{2.033126in}{1.601944in}}%
\pgfpathlineto{\pgfqpoint{2.033126in}{1.601944in}}%
\pgfusepath{stroke}%
\end{pgfscope}%
\begin{pgfscope}%
\pgfsetrectcap%
\pgfsetmiterjoin%
\pgfsetlinewidth{0.803000pt}%
\definecolor{currentstroke}{rgb}{0.000000,0.000000,0.000000}%
\pgfsetstrokecolor{currentstroke}%
\pgfsetdash{}{0pt}%
\pgfpathmoveto{\pgfqpoint{0.553581in}{0.499444in}}%
\pgfpathlineto{\pgfqpoint{0.553581in}{1.654444in}}%
\pgfusepath{stroke}%
\end{pgfscope}%
\begin{pgfscope}%
\pgfsetrectcap%
\pgfsetmiterjoin%
\pgfsetlinewidth{0.803000pt}%
\definecolor{currentstroke}{rgb}{0.000000,0.000000,0.000000}%
\pgfsetstrokecolor{currentstroke}%
\pgfsetdash{}{0pt}%
\pgfpathmoveto{\pgfqpoint{2.103581in}{0.499444in}}%
\pgfpathlineto{\pgfqpoint{2.103581in}{1.654444in}}%
\pgfusepath{stroke}%
\end{pgfscope}%
\begin{pgfscope}%
\pgfsetrectcap%
\pgfsetmiterjoin%
\pgfsetlinewidth{0.803000pt}%
\definecolor{currentstroke}{rgb}{0.000000,0.000000,0.000000}%
\pgfsetstrokecolor{currentstroke}%
\pgfsetdash{}{0pt}%
\pgfpathmoveto{\pgfqpoint{0.553581in}{0.499444in}}%
\pgfpathlineto{\pgfqpoint{2.103581in}{0.499444in}}%
\pgfusepath{stroke}%
\end{pgfscope}%
\begin{pgfscope}%
\pgfsetrectcap%
\pgfsetmiterjoin%
\pgfsetlinewidth{0.803000pt}%
\definecolor{currentstroke}{rgb}{0.000000,0.000000,0.000000}%
\pgfsetstrokecolor{currentstroke}%
\pgfsetdash{}{0pt}%
\pgfpathmoveto{\pgfqpoint{0.553581in}{1.654444in}}%
\pgfpathlineto{\pgfqpoint{2.103581in}{1.654444in}}%
\pgfusepath{stroke}%
\end{pgfscope}%
\begin{pgfscope}%
\pgfsetbuttcap%
\pgfsetmiterjoin%
\definecolor{currentfill}{rgb}{1.000000,1.000000,1.000000}%
\pgfsetfillcolor{currentfill}%
\pgfsetfillopacity{0.800000}%
\pgfsetlinewidth{1.003750pt}%
\definecolor{currentstroke}{rgb}{0.800000,0.800000,0.800000}%
\pgfsetstrokecolor{currentstroke}%
\pgfsetstrokeopacity{0.800000}%
\pgfsetdash{}{0pt}%
\pgfpathmoveto{\pgfqpoint{0.832747in}{1.349722in}}%
\pgfpathlineto{\pgfqpoint{2.006358in}{1.349722in}}%
\pgfpathquadraticcurveto{\pgfqpoint{2.034136in}{1.349722in}}{\pgfqpoint{2.034136in}{1.377500in}}%
\pgfpathlineto{\pgfqpoint{2.034136in}{1.557222in}}%
\pgfpathquadraticcurveto{\pgfqpoint{2.034136in}{1.585000in}}{\pgfqpoint{2.006358in}{1.585000in}}%
\pgfpathlineto{\pgfqpoint{0.832747in}{1.585000in}}%
\pgfpathquadraticcurveto{\pgfqpoint{0.804970in}{1.585000in}}{\pgfqpoint{0.804970in}{1.557222in}}%
\pgfpathlineto{\pgfqpoint{0.804970in}{1.377500in}}%
\pgfpathquadraticcurveto{\pgfqpoint{0.804970in}{1.349722in}}{\pgfqpoint{0.832747in}{1.349722in}}%
\pgfpathlineto{\pgfqpoint{0.832747in}{1.349722in}}%
\pgfpathclose%
\pgfusepath{stroke,fill}%
\end{pgfscope}%
\begin{pgfscope}%
\pgfsetrectcap%
\pgfsetroundjoin%
\pgfsetlinewidth{1.505625pt}%
\definecolor{currentstroke}{rgb}{0.000000,0.000000,0.000000}%
\pgfsetstrokecolor{currentstroke}%
\pgfsetdash{}{0pt}%
\pgfpathmoveto{\pgfqpoint{0.860525in}{1.480833in}}%
\pgfpathlineto{\pgfqpoint{0.999414in}{1.480833in}}%
\pgfpathlineto{\pgfqpoint{1.138303in}{1.480833in}}%
\pgfusepath{stroke}%
\end{pgfscope}%
\begin{pgfscope}%
\definecolor{textcolor}{rgb}{0.000000,0.000000,0.000000}%
\pgfsetstrokecolor{textcolor}%
\pgfsetfillcolor{textcolor}%
\pgftext[x=1.249414in,y=1.432222in,left,base]{\color{textcolor}\rmfamily\fontsize{10.000000}{12.000000}\selectfont AUC=0.996}%
\end{pgfscope}%
\end{pgfpicture}%
\makeatother%
\endgroup%

	
\end{tabular}

Unfortunately, our test results do not look quite that nice.  They do not separate the two classes as well, some are clustered to one side or in the middle, and some look more discrete than continuous.

\

\verb|BRFC_Hard_Tomek_0_alpha_0_5_v1_Test|

\noindent\begin{tabular}{@{\hspace{-6pt}}p{4.3in} @{\hspace{-6pt}}p{2.0in}}
	\vskip 0pt
	\hfil Raw Model Output
	
	%% Creator: Matplotlib, PGF backend
%%
%% To include the figure in your LaTeX document, write
%%   \input{<filename>.pgf}
%%
%% Make sure the required packages are loaded in your preamble
%%   \usepackage{pgf}
%%
%% Also ensure that all the required font packages are loaded; for instance,
%% the lmodern package is sometimes necessary when using math font.
%%   \usepackage{lmodern}
%%
%% Figures using additional raster images can only be included by \input if
%% they are in the same directory as the main LaTeX file. For loading figures
%% from other directories you can use the `import` package
%%   \usepackage{import}
%%
%% and then include the figures with
%%   \import{<path to file>}{<filename>.pgf}
%%
%% Matplotlib used the following preamble
%%   
%%   \usepackage{fontspec}
%%   \makeatletter\@ifpackageloaded{underscore}{}{\usepackage[strings]{underscore}}\makeatother
%%
\begingroup%
\makeatletter%
\begin{pgfpicture}%
\pgfpathrectangle{\pgfpointorigin}{\pgfqpoint{4.509306in}{1.754444in}}%
\pgfusepath{use as bounding box, clip}%
\begin{pgfscope}%
\pgfsetbuttcap%
\pgfsetmiterjoin%
\definecolor{currentfill}{rgb}{1.000000,1.000000,1.000000}%
\pgfsetfillcolor{currentfill}%
\pgfsetlinewidth{0.000000pt}%
\definecolor{currentstroke}{rgb}{1.000000,1.000000,1.000000}%
\pgfsetstrokecolor{currentstroke}%
\pgfsetdash{}{0pt}%
\pgfpathmoveto{\pgfqpoint{0.000000in}{0.000000in}}%
\pgfpathlineto{\pgfqpoint{4.509306in}{0.000000in}}%
\pgfpathlineto{\pgfqpoint{4.509306in}{1.754444in}}%
\pgfpathlineto{\pgfqpoint{0.000000in}{1.754444in}}%
\pgfpathlineto{\pgfqpoint{0.000000in}{0.000000in}}%
\pgfpathclose%
\pgfusepath{fill}%
\end{pgfscope}%
\begin{pgfscope}%
\pgfsetbuttcap%
\pgfsetmiterjoin%
\definecolor{currentfill}{rgb}{1.000000,1.000000,1.000000}%
\pgfsetfillcolor{currentfill}%
\pgfsetlinewidth{0.000000pt}%
\definecolor{currentstroke}{rgb}{0.000000,0.000000,0.000000}%
\pgfsetstrokecolor{currentstroke}%
\pgfsetstrokeopacity{0.000000}%
\pgfsetdash{}{0pt}%
\pgfpathmoveto{\pgfqpoint{0.445556in}{0.499444in}}%
\pgfpathlineto{\pgfqpoint{4.320556in}{0.499444in}}%
\pgfpathlineto{\pgfqpoint{4.320556in}{1.654444in}}%
\pgfpathlineto{\pgfqpoint{0.445556in}{1.654444in}}%
\pgfpathlineto{\pgfqpoint{0.445556in}{0.499444in}}%
\pgfpathclose%
\pgfusepath{fill}%
\end{pgfscope}%
\begin{pgfscope}%
\pgfpathrectangle{\pgfqpoint{0.445556in}{0.499444in}}{\pgfqpoint{3.875000in}{1.155000in}}%
\pgfusepath{clip}%
\pgfsetbuttcap%
\pgfsetmiterjoin%
\pgfsetlinewidth{1.003750pt}%
\definecolor{currentstroke}{rgb}{0.000000,0.000000,0.000000}%
\pgfsetstrokecolor{currentstroke}%
\pgfsetdash{}{0pt}%
\pgfpathmoveto{\pgfqpoint{0.435556in}{0.499444in}}%
\pgfpathlineto{\pgfqpoint{0.483922in}{0.499444in}}%
\pgfpathlineto{\pgfqpoint{0.483922in}{0.618288in}}%
\pgfpathlineto{\pgfqpoint{0.435556in}{0.618288in}}%
\pgfusepath{stroke}%
\end{pgfscope}%
\begin{pgfscope}%
\pgfpathrectangle{\pgfqpoint{0.445556in}{0.499444in}}{\pgfqpoint{3.875000in}{1.155000in}}%
\pgfusepath{clip}%
\pgfsetbuttcap%
\pgfsetmiterjoin%
\pgfsetlinewidth{1.003750pt}%
\definecolor{currentstroke}{rgb}{0.000000,0.000000,0.000000}%
\pgfsetstrokecolor{currentstroke}%
\pgfsetdash{}{0pt}%
\pgfpathmoveto{\pgfqpoint{0.576001in}{0.499444in}}%
\pgfpathlineto{\pgfqpoint{0.637387in}{0.499444in}}%
\pgfpathlineto{\pgfqpoint{0.637387in}{0.801196in}}%
\pgfpathlineto{\pgfqpoint{0.576001in}{0.801196in}}%
\pgfpathlineto{\pgfqpoint{0.576001in}{0.499444in}}%
\pgfpathclose%
\pgfusepath{stroke}%
\end{pgfscope}%
\begin{pgfscope}%
\pgfpathrectangle{\pgfqpoint{0.445556in}{0.499444in}}{\pgfqpoint{3.875000in}{1.155000in}}%
\pgfusepath{clip}%
\pgfsetbuttcap%
\pgfsetmiterjoin%
\pgfsetlinewidth{1.003750pt}%
\definecolor{currentstroke}{rgb}{0.000000,0.000000,0.000000}%
\pgfsetstrokecolor{currentstroke}%
\pgfsetdash{}{0pt}%
\pgfpathmoveto{\pgfqpoint{0.729467in}{0.499444in}}%
\pgfpathlineto{\pgfqpoint{0.790853in}{0.499444in}}%
\pgfpathlineto{\pgfqpoint{0.790853in}{1.026194in}}%
\pgfpathlineto{\pgfqpoint{0.729467in}{1.026194in}}%
\pgfpathlineto{\pgfqpoint{0.729467in}{0.499444in}}%
\pgfpathclose%
\pgfusepath{stroke}%
\end{pgfscope}%
\begin{pgfscope}%
\pgfpathrectangle{\pgfqpoint{0.445556in}{0.499444in}}{\pgfqpoint{3.875000in}{1.155000in}}%
\pgfusepath{clip}%
\pgfsetbuttcap%
\pgfsetmiterjoin%
\pgfsetlinewidth{1.003750pt}%
\definecolor{currentstroke}{rgb}{0.000000,0.000000,0.000000}%
\pgfsetstrokecolor{currentstroke}%
\pgfsetdash{}{0pt}%
\pgfpathmoveto{\pgfqpoint{0.882932in}{0.499444in}}%
\pgfpathlineto{\pgfqpoint{0.944318in}{0.499444in}}%
\pgfpathlineto{\pgfqpoint{0.944318in}{1.203686in}}%
\pgfpathlineto{\pgfqpoint{0.882932in}{1.203686in}}%
\pgfpathlineto{\pgfqpoint{0.882932in}{0.499444in}}%
\pgfpathclose%
\pgfusepath{stroke}%
\end{pgfscope}%
\begin{pgfscope}%
\pgfpathrectangle{\pgfqpoint{0.445556in}{0.499444in}}{\pgfqpoint{3.875000in}{1.155000in}}%
\pgfusepath{clip}%
\pgfsetbuttcap%
\pgfsetmiterjoin%
\pgfsetlinewidth{1.003750pt}%
\definecolor{currentstroke}{rgb}{0.000000,0.000000,0.000000}%
\pgfsetstrokecolor{currentstroke}%
\pgfsetdash{}{0pt}%
\pgfpathmoveto{\pgfqpoint{1.036397in}{0.499444in}}%
\pgfpathlineto{\pgfqpoint{1.097783in}{0.499444in}}%
\pgfpathlineto{\pgfqpoint{1.097783in}{1.348758in}}%
\pgfpathlineto{\pgfqpoint{1.036397in}{1.348758in}}%
\pgfpathlineto{\pgfqpoint{1.036397in}{0.499444in}}%
\pgfpathclose%
\pgfusepath{stroke}%
\end{pgfscope}%
\begin{pgfscope}%
\pgfpathrectangle{\pgfqpoint{0.445556in}{0.499444in}}{\pgfqpoint{3.875000in}{1.155000in}}%
\pgfusepath{clip}%
\pgfsetbuttcap%
\pgfsetmiterjoin%
\pgfsetlinewidth{1.003750pt}%
\definecolor{currentstroke}{rgb}{0.000000,0.000000,0.000000}%
\pgfsetstrokecolor{currentstroke}%
\pgfsetdash{}{0pt}%
\pgfpathmoveto{\pgfqpoint{1.189863in}{0.499444in}}%
\pgfpathlineto{\pgfqpoint{1.251249in}{0.499444in}}%
\pgfpathlineto{\pgfqpoint{1.251249in}{1.455609in}}%
\pgfpathlineto{\pgfqpoint{1.189863in}{1.455609in}}%
\pgfpathlineto{\pgfqpoint{1.189863in}{0.499444in}}%
\pgfpathclose%
\pgfusepath{stroke}%
\end{pgfscope}%
\begin{pgfscope}%
\pgfpathrectangle{\pgfqpoint{0.445556in}{0.499444in}}{\pgfqpoint{3.875000in}{1.155000in}}%
\pgfusepath{clip}%
\pgfsetbuttcap%
\pgfsetmiterjoin%
\pgfsetlinewidth{1.003750pt}%
\definecolor{currentstroke}{rgb}{0.000000,0.000000,0.000000}%
\pgfsetstrokecolor{currentstroke}%
\pgfsetdash{}{0pt}%
\pgfpathmoveto{\pgfqpoint{1.343328in}{0.499444in}}%
\pgfpathlineto{\pgfqpoint{1.404714in}{0.499444in}}%
\pgfpathlineto{\pgfqpoint{1.404714in}{1.549075in}}%
\pgfpathlineto{\pgfqpoint{1.343328in}{1.549075in}}%
\pgfpathlineto{\pgfqpoint{1.343328in}{0.499444in}}%
\pgfpathclose%
\pgfusepath{stroke}%
\end{pgfscope}%
\begin{pgfscope}%
\pgfpathrectangle{\pgfqpoint{0.445556in}{0.499444in}}{\pgfqpoint{3.875000in}{1.155000in}}%
\pgfusepath{clip}%
\pgfsetbuttcap%
\pgfsetmiterjoin%
\pgfsetlinewidth{1.003750pt}%
\definecolor{currentstroke}{rgb}{0.000000,0.000000,0.000000}%
\pgfsetstrokecolor{currentstroke}%
\pgfsetdash{}{0pt}%
\pgfpathmoveto{\pgfqpoint{1.496793in}{0.499444in}}%
\pgfpathlineto{\pgfqpoint{1.558179in}{0.499444in}}%
\pgfpathlineto{\pgfqpoint{1.558179in}{1.599444in}}%
\pgfpathlineto{\pgfqpoint{1.496793in}{1.599444in}}%
\pgfpathlineto{\pgfqpoint{1.496793in}{0.499444in}}%
\pgfpathclose%
\pgfusepath{stroke}%
\end{pgfscope}%
\begin{pgfscope}%
\pgfpathrectangle{\pgfqpoint{0.445556in}{0.499444in}}{\pgfqpoint{3.875000in}{1.155000in}}%
\pgfusepath{clip}%
\pgfsetbuttcap%
\pgfsetmiterjoin%
\pgfsetlinewidth{1.003750pt}%
\definecolor{currentstroke}{rgb}{0.000000,0.000000,0.000000}%
\pgfsetstrokecolor{currentstroke}%
\pgfsetdash{}{0pt}%
\pgfpathmoveto{\pgfqpoint{1.650259in}{0.499444in}}%
\pgfpathlineto{\pgfqpoint{1.711645in}{0.499444in}}%
\pgfpathlineto{\pgfqpoint{1.711645in}{1.597742in}}%
\pgfpathlineto{\pgfqpoint{1.650259in}{1.597742in}}%
\pgfpathlineto{\pgfqpoint{1.650259in}{0.499444in}}%
\pgfpathclose%
\pgfusepath{stroke}%
\end{pgfscope}%
\begin{pgfscope}%
\pgfpathrectangle{\pgfqpoint{0.445556in}{0.499444in}}{\pgfqpoint{3.875000in}{1.155000in}}%
\pgfusepath{clip}%
\pgfsetbuttcap%
\pgfsetmiterjoin%
\pgfsetlinewidth{1.003750pt}%
\definecolor{currentstroke}{rgb}{0.000000,0.000000,0.000000}%
\pgfsetstrokecolor{currentstroke}%
\pgfsetdash{}{0pt}%
\pgfpathmoveto{\pgfqpoint{1.803724in}{0.499444in}}%
\pgfpathlineto{\pgfqpoint{1.865110in}{0.499444in}}%
\pgfpathlineto{\pgfqpoint{1.865110in}{1.571977in}}%
\pgfpathlineto{\pgfqpoint{1.803724in}{1.571977in}}%
\pgfpathlineto{\pgfqpoint{1.803724in}{0.499444in}}%
\pgfpathclose%
\pgfusepath{stroke}%
\end{pgfscope}%
\begin{pgfscope}%
\pgfpathrectangle{\pgfqpoint{0.445556in}{0.499444in}}{\pgfqpoint{3.875000in}{1.155000in}}%
\pgfusepath{clip}%
\pgfsetbuttcap%
\pgfsetmiterjoin%
\pgfsetlinewidth{1.003750pt}%
\definecolor{currentstroke}{rgb}{0.000000,0.000000,0.000000}%
\pgfsetstrokecolor{currentstroke}%
\pgfsetdash{}{0pt}%
\pgfpathmoveto{\pgfqpoint{1.957189in}{0.499444in}}%
\pgfpathlineto{\pgfqpoint{2.018575in}{0.499444in}}%
\pgfpathlineto{\pgfqpoint{2.018575in}{1.535535in}}%
\pgfpathlineto{\pgfqpoint{1.957189in}{1.535535in}}%
\pgfpathlineto{\pgfqpoint{1.957189in}{0.499444in}}%
\pgfpathclose%
\pgfusepath{stroke}%
\end{pgfscope}%
\begin{pgfscope}%
\pgfpathrectangle{\pgfqpoint{0.445556in}{0.499444in}}{\pgfqpoint{3.875000in}{1.155000in}}%
\pgfusepath{clip}%
\pgfsetbuttcap%
\pgfsetmiterjoin%
\pgfsetlinewidth{1.003750pt}%
\definecolor{currentstroke}{rgb}{0.000000,0.000000,0.000000}%
\pgfsetstrokecolor{currentstroke}%
\pgfsetdash{}{0pt}%
\pgfpathmoveto{\pgfqpoint{2.110655in}{0.499444in}}%
\pgfpathlineto{\pgfqpoint{2.172041in}{0.499444in}}%
\pgfpathlineto{\pgfqpoint{2.172041in}{1.450812in}}%
\pgfpathlineto{\pgfqpoint{2.110655in}{1.450812in}}%
\pgfpathlineto{\pgfqpoint{2.110655in}{0.499444in}}%
\pgfpathclose%
\pgfusepath{stroke}%
\end{pgfscope}%
\begin{pgfscope}%
\pgfpathrectangle{\pgfqpoint{0.445556in}{0.499444in}}{\pgfqpoint{3.875000in}{1.155000in}}%
\pgfusepath{clip}%
\pgfsetbuttcap%
\pgfsetmiterjoin%
\pgfsetlinewidth{1.003750pt}%
\definecolor{currentstroke}{rgb}{0.000000,0.000000,0.000000}%
\pgfsetstrokecolor{currentstroke}%
\pgfsetdash{}{0pt}%
\pgfpathmoveto{\pgfqpoint{2.264120in}{0.499444in}}%
\pgfpathlineto{\pgfqpoint{2.325506in}{0.499444in}}%
\pgfpathlineto{\pgfqpoint{2.325506in}{1.379398in}}%
\pgfpathlineto{\pgfqpoint{2.264120in}{1.379398in}}%
\pgfpathlineto{\pgfqpoint{2.264120in}{0.499444in}}%
\pgfpathclose%
\pgfusepath{stroke}%
\end{pgfscope}%
\begin{pgfscope}%
\pgfpathrectangle{\pgfqpoint{0.445556in}{0.499444in}}{\pgfqpoint{3.875000in}{1.155000in}}%
\pgfusepath{clip}%
\pgfsetbuttcap%
\pgfsetmiterjoin%
\pgfsetlinewidth{1.003750pt}%
\definecolor{currentstroke}{rgb}{0.000000,0.000000,0.000000}%
\pgfsetstrokecolor{currentstroke}%
\pgfsetdash{}{0pt}%
\pgfpathmoveto{\pgfqpoint{2.417585in}{0.499444in}}%
\pgfpathlineto{\pgfqpoint{2.478972in}{0.499444in}}%
\pgfpathlineto{\pgfqpoint{2.478972in}{1.248329in}}%
\pgfpathlineto{\pgfqpoint{2.417585in}{1.248329in}}%
\pgfpathlineto{\pgfqpoint{2.417585in}{0.499444in}}%
\pgfpathclose%
\pgfusepath{stroke}%
\end{pgfscope}%
\begin{pgfscope}%
\pgfpathrectangle{\pgfqpoint{0.445556in}{0.499444in}}{\pgfqpoint{3.875000in}{1.155000in}}%
\pgfusepath{clip}%
\pgfsetbuttcap%
\pgfsetmiterjoin%
\pgfsetlinewidth{1.003750pt}%
\definecolor{currentstroke}{rgb}{0.000000,0.000000,0.000000}%
\pgfsetstrokecolor{currentstroke}%
\pgfsetdash{}{0pt}%
\pgfpathmoveto{\pgfqpoint{2.571051in}{0.499444in}}%
\pgfpathlineto{\pgfqpoint{2.632437in}{0.499444in}}%
\pgfpathlineto{\pgfqpoint{2.632437in}{1.129640in}}%
\pgfpathlineto{\pgfqpoint{2.571051in}{1.129640in}}%
\pgfpathlineto{\pgfqpoint{2.571051in}{0.499444in}}%
\pgfpathclose%
\pgfusepath{stroke}%
\end{pgfscope}%
\begin{pgfscope}%
\pgfpathrectangle{\pgfqpoint{0.445556in}{0.499444in}}{\pgfqpoint{3.875000in}{1.155000in}}%
\pgfusepath{clip}%
\pgfsetbuttcap%
\pgfsetmiterjoin%
\pgfsetlinewidth{1.003750pt}%
\definecolor{currentstroke}{rgb}{0.000000,0.000000,0.000000}%
\pgfsetstrokecolor{currentstroke}%
\pgfsetdash{}{0pt}%
\pgfpathmoveto{\pgfqpoint{2.724516in}{0.499444in}}%
\pgfpathlineto{\pgfqpoint{2.785902in}{0.499444in}}%
\pgfpathlineto{\pgfqpoint{2.785902in}{1.011880in}}%
\pgfpathlineto{\pgfqpoint{2.724516in}{1.011880in}}%
\pgfpathlineto{\pgfqpoint{2.724516in}{0.499444in}}%
\pgfpathclose%
\pgfusepath{stroke}%
\end{pgfscope}%
\begin{pgfscope}%
\pgfpathrectangle{\pgfqpoint{0.445556in}{0.499444in}}{\pgfqpoint{3.875000in}{1.155000in}}%
\pgfusepath{clip}%
\pgfsetbuttcap%
\pgfsetmiterjoin%
\pgfsetlinewidth{1.003750pt}%
\definecolor{currentstroke}{rgb}{0.000000,0.000000,0.000000}%
\pgfsetstrokecolor{currentstroke}%
\pgfsetdash{}{0pt}%
\pgfpathmoveto{\pgfqpoint{2.877981in}{0.499444in}}%
\pgfpathlineto{\pgfqpoint{2.939368in}{0.499444in}}%
\pgfpathlineto{\pgfqpoint{2.939368in}{0.912921in}}%
\pgfpathlineto{\pgfqpoint{2.877981in}{0.912921in}}%
\pgfpathlineto{\pgfqpoint{2.877981in}{0.499444in}}%
\pgfpathclose%
\pgfusepath{stroke}%
\end{pgfscope}%
\begin{pgfscope}%
\pgfpathrectangle{\pgfqpoint{0.445556in}{0.499444in}}{\pgfqpoint{3.875000in}{1.155000in}}%
\pgfusepath{clip}%
\pgfsetbuttcap%
\pgfsetmiterjoin%
\pgfsetlinewidth{1.003750pt}%
\definecolor{currentstroke}{rgb}{0.000000,0.000000,0.000000}%
\pgfsetstrokecolor{currentstroke}%
\pgfsetdash{}{0pt}%
\pgfpathmoveto{\pgfqpoint{3.031447in}{0.499444in}}%
\pgfpathlineto{\pgfqpoint{3.092833in}{0.499444in}}%
\pgfpathlineto{\pgfqpoint{3.092833in}{0.805374in}}%
\pgfpathlineto{\pgfqpoint{3.031447in}{0.805374in}}%
\pgfpathlineto{\pgfqpoint{3.031447in}{0.499444in}}%
\pgfpathclose%
\pgfusepath{stroke}%
\end{pgfscope}%
\begin{pgfscope}%
\pgfpathrectangle{\pgfqpoint{0.445556in}{0.499444in}}{\pgfqpoint{3.875000in}{1.155000in}}%
\pgfusepath{clip}%
\pgfsetbuttcap%
\pgfsetmiterjoin%
\pgfsetlinewidth{1.003750pt}%
\definecolor{currentstroke}{rgb}{0.000000,0.000000,0.000000}%
\pgfsetstrokecolor{currentstroke}%
\pgfsetdash{}{0pt}%
\pgfpathmoveto{\pgfqpoint{3.184912in}{0.499444in}}%
\pgfpathlineto{\pgfqpoint{3.246298in}{0.499444in}}%
\pgfpathlineto{\pgfqpoint{3.246298in}{0.730710in}}%
\pgfpathlineto{\pgfqpoint{3.184912in}{0.730710in}}%
\pgfpathlineto{\pgfqpoint{3.184912in}{0.499444in}}%
\pgfpathclose%
\pgfusepath{stroke}%
\end{pgfscope}%
\begin{pgfscope}%
\pgfpathrectangle{\pgfqpoint{0.445556in}{0.499444in}}{\pgfqpoint{3.875000in}{1.155000in}}%
\pgfusepath{clip}%
\pgfsetbuttcap%
\pgfsetmiterjoin%
\pgfsetlinewidth{1.003750pt}%
\definecolor{currentstroke}{rgb}{0.000000,0.000000,0.000000}%
\pgfsetstrokecolor{currentstroke}%
\pgfsetdash{}{0pt}%
\pgfpathmoveto{\pgfqpoint{3.338377in}{0.499444in}}%
\pgfpathlineto{\pgfqpoint{3.399764in}{0.499444in}}%
\pgfpathlineto{\pgfqpoint{3.399764in}{0.672603in}}%
\pgfpathlineto{\pgfqpoint{3.338377in}{0.672603in}}%
\pgfpathlineto{\pgfqpoint{3.338377in}{0.499444in}}%
\pgfpathclose%
\pgfusepath{stroke}%
\end{pgfscope}%
\begin{pgfscope}%
\pgfpathrectangle{\pgfqpoint{0.445556in}{0.499444in}}{\pgfqpoint{3.875000in}{1.155000in}}%
\pgfusepath{clip}%
\pgfsetbuttcap%
\pgfsetmiterjoin%
\pgfsetlinewidth{1.003750pt}%
\definecolor{currentstroke}{rgb}{0.000000,0.000000,0.000000}%
\pgfsetstrokecolor{currentstroke}%
\pgfsetdash{}{0pt}%
\pgfpathmoveto{\pgfqpoint{3.491843in}{0.499444in}}%
\pgfpathlineto{\pgfqpoint{3.553229in}{0.499444in}}%
\pgfpathlineto{\pgfqpoint{3.553229in}{0.616663in}}%
\pgfpathlineto{\pgfqpoint{3.491843in}{0.616663in}}%
\pgfpathlineto{\pgfqpoint{3.491843in}{0.499444in}}%
\pgfpathclose%
\pgfusepath{stroke}%
\end{pgfscope}%
\begin{pgfscope}%
\pgfpathrectangle{\pgfqpoint{0.445556in}{0.499444in}}{\pgfqpoint{3.875000in}{1.155000in}}%
\pgfusepath{clip}%
\pgfsetbuttcap%
\pgfsetmiterjoin%
\pgfsetlinewidth{1.003750pt}%
\definecolor{currentstroke}{rgb}{0.000000,0.000000,0.000000}%
\pgfsetstrokecolor{currentstroke}%
\pgfsetdash{}{0pt}%
\pgfpathmoveto{\pgfqpoint{3.645308in}{0.499444in}}%
\pgfpathlineto{\pgfqpoint{3.706694in}{0.499444in}}%
\pgfpathlineto{\pgfqpoint{3.706694in}{0.574031in}}%
\pgfpathlineto{\pgfqpoint{3.645308in}{0.574031in}}%
\pgfpathlineto{\pgfqpoint{3.645308in}{0.499444in}}%
\pgfpathclose%
\pgfusepath{stroke}%
\end{pgfscope}%
\begin{pgfscope}%
\pgfpathrectangle{\pgfqpoint{0.445556in}{0.499444in}}{\pgfqpoint{3.875000in}{1.155000in}}%
\pgfusepath{clip}%
\pgfsetbuttcap%
\pgfsetmiterjoin%
\pgfsetlinewidth{1.003750pt}%
\definecolor{currentstroke}{rgb}{0.000000,0.000000,0.000000}%
\pgfsetstrokecolor{currentstroke}%
\pgfsetdash{}{0pt}%
\pgfpathmoveto{\pgfqpoint{3.798774in}{0.499444in}}%
\pgfpathlineto{\pgfqpoint{3.860160in}{0.499444in}}%
\pgfpathlineto{\pgfqpoint{3.860160in}{0.550587in}}%
\pgfpathlineto{\pgfqpoint{3.798774in}{0.550587in}}%
\pgfpathlineto{\pgfqpoint{3.798774in}{0.499444in}}%
\pgfpathclose%
\pgfusepath{stroke}%
\end{pgfscope}%
\begin{pgfscope}%
\pgfpathrectangle{\pgfqpoint{0.445556in}{0.499444in}}{\pgfqpoint{3.875000in}{1.155000in}}%
\pgfusepath{clip}%
\pgfsetbuttcap%
\pgfsetmiterjoin%
\pgfsetlinewidth{1.003750pt}%
\definecolor{currentstroke}{rgb}{0.000000,0.000000,0.000000}%
\pgfsetstrokecolor{currentstroke}%
\pgfsetdash{}{0pt}%
\pgfpathmoveto{\pgfqpoint{3.952239in}{0.499444in}}%
\pgfpathlineto{\pgfqpoint{4.013625in}{0.499444in}}%
\pgfpathlineto{\pgfqpoint{4.013625in}{0.526679in}}%
\pgfpathlineto{\pgfqpoint{3.952239in}{0.526679in}}%
\pgfpathlineto{\pgfqpoint{3.952239in}{0.499444in}}%
\pgfpathclose%
\pgfusepath{stroke}%
\end{pgfscope}%
\begin{pgfscope}%
\pgfpathrectangle{\pgfqpoint{0.445556in}{0.499444in}}{\pgfqpoint{3.875000in}{1.155000in}}%
\pgfusepath{clip}%
\pgfsetbuttcap%
\pgfsetmiterjoin%
\pgfsetlinewidth{1.003750pt}%
\definecolor{currentstroke}{rgb}{0.000000,0.000000,0.000000}%
\pgfsetstrokecolor{currentstroke}%
\pgfsetdash{}{0pt}%
\pgfpathmoveto{\pgfqpoint{4.105704in}{0.499444in}}%
\pgfpathlineto{\pgfqpoint{4.167090in}{0.499444in}}%
\pgfpathlineto{\pgfqpoint{4.167090in}{0.506640in}}%
\pgfpathlineto{\pgfqpoint{4.105704in}{0.506640in}}%
\pgfpathlineto{\pgfqpoint{4.105704in}{0.499444in}}%
\pgfpathclose%
\pgfusepath{stroke}%
\end{pgfscope}%
\begin{pgfscope}%
\pgfpathrectangle{\pgfqpoint{0.445556in}{0.499444in}}{\pgfqpoint{3.875000in}{1.155000in}}%
\pgfusepath{clip}%
\pgfsetbuttcap%
\pgfsetmiterjoin%
\definecolor{currentfill}{rgb}{0.000000,0.000000,0.000000}%
\pgfsetfillcolor{currentfill}%
\pgfsetlinewidth{0.000000pt}%
\definecolor{currentstroke}{rgb}{0.000000,0.000000,0.000000}%
\pgfsetstrokecolor{currentstroke}%
\pgfsetstrokeopacity{0.000000}%
\pgfsetdash{}{0pt}%
\pgfpathmoveto{\pgfqpoint{0.483922in}{0.499444in}}%
\pgfpathlineto{\pgfqpoint{0.545308in}{0.499444in}}%
\pgfpathlineto{\pgfqpoint{0.545308in}{0.500527in}}%
\pgfpathlineto{\pgfqpoint{0.483922in}{0.500527in}}%
\pgfpathlineto{\pgfqpoint{0.483922in}{0.499444in}}%
\pgfpathclose%
\pgfusepath{fill}%
\end{pgfscope}%
\begin{pgfscope}%
\pgfpathrectangle{\pgfqpoint{0.445556in}{0.499444in}}{\pgfqpoint{3.875000in}{1.155000in}}%
\pgfusepath{clip}%
\pgfsetbuttcap%
\pgfsetmiterjoin%
\definecolor{currentfill}{rgb}{0.000000,0.000000,0.000000}%
\pgfsetfillcolor{currentfill}%
\pgfsetlinewidth{0.000000pt}%
\definecolor{currentstroke}{rgb}{0.000000,0.000000,0.000000}%
\pgfsetstrokecolor{currentstroke}%
\pgfsetstrokeopacity{0.000000}%
\pgfsetdash{}{0pt}%
\pgfpathmoveto{\pgfqpoint{0.637387in}{0.499444in}}%
\pgfpathlineto{\pgfqpoint{0.698774in}{0.499444in}}%
\pgfpathlineto{\pgfqpoint{0.698774in}{0.502230in}}%
\pgfpathlineto{\pgfqpoint{0.637387in}{0.502230in}}%
\pgfpathlineto{\pgfqpoint{0.637387in}{0.499444in}}%
\pgfpathclose%
\pgfusepath{fill}%
\end{pgfscope}%
\begin{pgfscope}%
\pgfpathrectangle{\pgfqpoint{0.445556in}{0.499444in}}{\pgfqpoint{3.875000in}{1.155000in}}%
\pgfusepath{clip}%
\pgfsetbuttcap%
\pgfsetmiterjoin%
\definecolor{currentfill}{rgb}{0.000000,0.000000,0.000000}%
\pgfsetfillcolor{currentfill}%
\pgfsetlinewidth{0.000000pt}%
\definecolor{currentstroke}{rgb}{0.000000,0.000000,0.000000}%
\pgfsetstrokecolor{currentstroke}%
\pgfsetstrokeopacity{0.000000}%
\pgfsetdash{}{0pt}%
\pgfpathmoveto{\pgfqpoint{0.790853in}{0.499444in}}%
\pgfpathlineto{\pgfqpoint{0.852239in}{0.499444in}}%
\pgfpathlineto{\pgfqpoint{0.852239in}{0.506330in}}%
\pgfpathlineto{\pgfqpoint{0.790853in}{0.506330in}}%
\pgfpathlineto{\pgfqpoint{0.790853in}{0.499444in}}%
\pgfpathclose%
\pgfusepath{fill}%
\end{pgfscope}%
\begin{pgfscope}%
\pgfpathrectangle{\pgfqpoint{0.445556in}{0.499444in}}{\pgfqpoint{3.875000in}{1.155000in}}%
\pgfusepath{clip}%
\pgfsetbuttcap%
\pgfsetmiterjoin%
\definecolor{currentfill}{rgb}{0.000000,0.000000,0.000000}%
\pgfsetfillcolor{currentfill}%
\pgfsetlinewidth{0.000000pt}%
\definecolor{currentstroke}{rgb}{0.000000,0.000000,0.000000}%
\pgfsetstrokecolor{currentstroke}%
\pgfsetstrokeopacity{0.000000}%
\pgfsetdash{}{0pt}%
\pgfpathmoveto{\pgfqpoint{0.944318in}{0.499444in}}%
\pgfpathlineto{\pgfqpoint{1.005704in}{0.499444in}}%
\pgfpathlineto{\pgfqpoint{1.005704in}{0.514068in}}%
\pgfpathlineto{\pgfqpoint{0.944318in}{0.514068in}}%
\pgfpathlineto{\pgfqpoint{0.944318in}{0.499444in}}%
\pgfpathclose%
\pgfusepath{fill}%
\end{pgfscope}%
\begin{pgfscope}%
\pgfpathrectangle{\pgfqpoint{0.445556in}{0.499444in}}{\pgfqpoint{3.875000in}{1.155000in}}%
\pgfusepath{clip}%
\pgfsetbuttcap%
\pgfsetmiterjoin%
\definecolor{currentfill}{rgb}{0.000000,0.000000,0.000000}%
\pgfsetfillcolor{currentfill}%
\pgfsetlinewidth{0.000000pt}%
\definecolor{currentstroke}{rgb}{0.000000,0.000000,0.000000}%
\pgfsetstrokecolor{currentstroke}%
\pgfsetstrokeopacity{0.000000}%
\pgfsetdash{}{0pt}%
\pgfpathmoveto{\pgfqpoint{1.097783in}{0.499444in}}%
\pgfpathlineto{\pgfqpoint{1.159170in}{0.499444in}}%
\pgfpathlineto{\pgfqpoint{1.159170in}{0.524977in}}%
\pgfpathlineto{\pgfqpoint{1.097783in}{0.524977in}}%
\pgfpathlineto{\pgfqpoint{1.097783in}{0.499444in}}%
\pgfpathclose%
\pgfusepath{fill}%
\end{pgfscope}%
\begin{pgfscope}%
\pgfpathrectangle{\pgfqpoint{0.445556in}{0.499444in}}{\pgfqpoint{3.875000in}{1.155000in}}%
\pgfusepath{clip}%
\pgfsetbuttcap%
\pgfsetmiterjoin%
\definecolor{currentfill}{rgb}{0.000000,0.000000,0.000000}%
\pgfsetfillcolor{currentfill}%
\pgfsetlinewidth{0.000000pt}%
\definecolor{currentstroke}{rgb}{0.000000,0.000000,0.000000}%
\pgfsetstrokecolor{currentstroke}%
\pgfsetstrokeopacity{0.000000}%
\pgfsetdash{}{0pt}%
\pgfpathmoveto{\pgfqpoint{1.251249in}{0.499444in}}%
\pgfpathlineto{\pgfqpoint{1.312635in}{0.499444in}}%
\pgfpathlineto{\pgfqpoint{1.312635in}{0.535267in}}%
\pgfpathlineto{\pgfqpoint{1.251249in}{0.535267in}}%
\pgfpathlineto{\pgfqpoint{1.251249in}{0.499444in}}%
\pgfpathclose%
\pgfusepath{fill}%
\end{pgfscope}%
\begin{pgfscope}%
\pgfpathrectangle{\pgfqpoint{0.445556in}{0.499444in}}{\pgfqpoint{3.875000in}{1.155000in}}%
\pgfusepath{clip}%
\pgfsetbuttcap%
\pgfsetmiterjoin%
\definecolor{currentfill}{rgb}{0.000000,0.000000,0.000000}%
\pgfsetfillcolor{currentfill}%
\pgfsetlinewidth{0.000000pt}%
\definecolor{currentstroke}{rgb}{0.000000,0.000000,0.000000}%
\pgfsetstrokecolor{currentstroke}%
\pgfsetstrokeopacity{0.000000}%
\pgfsetdash{}{0pt}%
\pgfpathmoveto{\pgfqpoint{1.404714in}{0.499444in}}%
\pgfpathlineto{\pgfqpoint{1.466100in}{0.499444in}}%
\pgfpathlineto{\pgfqpoint{1.466100in}{0.549659in}}%
\pgfpathlineto{\pgfqpoint{1.404714in}{0.549659in}}%
\pgfpathlineto{\pgfqpoint{1.404714in}{0.499444in}}%
\pgfpathclose%
\pgfusepath{fill}%
\end{pgfscope}%
\begin{pgfscope}%
\pgfpathrectangle{\pgfqpoint{0.445556in}{0.499444in}}{\pgfqpoint{3.875000in}{1.155000in}}%
\pgfusepath{clip}%
\pgfsetbuttcap%
\pgfsetmiterjoin%
\definecolor{currentfill}{rgb}{0.000000,0.000000,0.000000}%
\pgfsetfillcolor{currentfill}%
\pgfsetlinewidth{0.000000pt}%
\definecolor{currentstroke}{rgb}{0.000000,0.000000,0.000000}%
\pgfsetstrokecolor{currentstroke}%
\pgfsetstrokeopacity{0.000000}%
\pgfsetdash{}{0pt}%
\pgfpathmoveto{\pgfqpoint{1.558179in}{0.499444in}}%
\pgfpathlineto{\pgfqpoint{1.619566in}{0.499444in}}%
\pgfpathlineto{\pgfqpoint{1.619566in}{0.568228in}}%
\pgfpathlineto{\pgfqpoint{1.558179in}{0.568228in}}%
\pgfpathlineto{\pgfqpoint{1.558179in}{0.499444in}}%
\pgfpathclose%
\pgfusepath{fill}%
\end{pgfscope}%
\begin{pgfscope}%
\pgfpathrectangle{\pgfqpoint{0.445556in}{0.499444in}}{\pgfqpoint{3.875000in}{1.155000in}}%
\pgfusepath{clip}%
\pgfsetbuttcap%
\pgfsetmiterjoin%
\definecolor{currentfill}{rgb}{0.000000,0.000000,0.000000}%
\pgfsetfillcolor{currentfill}%
\pgfsetlinewidth{0.000000pt}%
\definecolor{currentstroke}{rgb}{0.000000,0.000000,0.000000}%
\pgfsetstrokecolor{currentstroke}%
\pgfsetstrokeopacity{0.000000}%
\pgfsetdash{}{0pt}%
\pgfpathmoveto{\pgfqpoint{1.711645in}{0.499444in}}%
\pgfpathlineto{\pgfqpoint{1.773031in}{0.499444in}}%
\pgfpathlineto{\pgfqpoint{1.773031in}{0.590743in}}%
\pgfpathlineto{\pgfqpoint{1.711645in}{0.590743in}}%
\pgfpathlineto{\pgfqpoint{1.711645in}{0.499444in}}%
\pgfpathclose%
\pgfusepath{fill}%
\end{pgfscope}%
\begin{pgfscope}%
\pgfpathrectangle{\pgfqpoint{0.445556in}{0.499444in}}{\pgfqpoint{3.875000in}{1.155000in}}%
\pgfusepath{clip}%
\pgfsetbuttcap%
\pgfsetmiterjoin%
\definecolor{currentfill}{rgb}{0.000000,0.000000,0.000000}%
\pgfsetfillcolor{currentfill}%
\pgfsetlinewidth{0.000000pt}%
\definecolor{currentstroke}{rgb}{0.000000,0.000000,0.000000}%
\pgfsetstrokecolor{currentstroke}%
\pgfsetstrokeopacity{0.000000}%
\pgfsetdash{}{0pt}%
\pgfpathmoveto{\pgfqpoint{1.865110in}{0.499444in}}%
\pgfpathlineto{\pgfqpoint{1.926496in}{0.499444in}}%
\pgfpathlineto{\pgfqpoint{1.926496in}{0.605521in}}%
\pgfpathlineto{\pgfqpoint{1.865110in}{0.605521in}}%
\pgfpathlineto{\pgfqpoint{1.865110in}{0.499444in}}%
\pgfpathclose%
\pgfusepath{fill}%
\end{pgfscope}%
\begin{pgfscope}%
\pgfpathrectangle{\pgfqpoint{0.445556in}{0.499444in}}{\pgfqpoint{3.875000in}{1.155000in}}%
\pgfusepath{clip}%
\pgfsetbuttcap%
\pgfsetmiterjoin%
\definecolor{currentfill}{rgb}{0.000000,0.000000,0.000000}%
\pgfsetfillcolor{currentfill}%
\pgfsetlinewidth{0.000000pt}%
\definecolor{currentstroke}{rgb}{0.000000,0.000000,0.000000}%
\pgfsetstrokecolor{currentstroke}%
\pgfsetstrokeopacity{0.000000}%
\pgfsetdash{}{0pt}%
\pgfpathmoveto{\pgfqpoint{2.018575in}{0.499444in}}%
\pgfpathlineto{\pgfqpoint{2.079962in}{0.499444in}}%
\pgfpathlineto{\pgfqpoint{2.079962in}{0.624632in}}%
\pgfpathlineto{\pgfqpoint{2.018575in}{0.624632in}}%
\pgfpathlineto{\pgfqpoint{2.018575in}{0.499444in}}%
\pgfpathclose%
\pgfusepath{fill}%
\end{pgfscope}%
\begin{pgfscope}%
\pgfpathrectangle{\pgfqpoint{0.445556in}{0.499444in}}{\pgfqpoint{3.875000in}{1.155000in}}%
\pgfusepath{clip}%
\pgfsetbuttcap%
\pgfsetmiterjoin%
\definecolor{currentfill}{rgb}{0.000000,0.000000,0.000000}%
\pgfsetfillcolor{currentfill}%
\pgfsetlinewidth{0.000000pt}%
\definecolor{currentstroke}{rgb}{0.000000,0.000000,0.000000}%
\pgfsetstrokecolor{currentstroke}%
\pgfsetstrokeopacity{0.000000}%
\pgfsetdash{}{0pt}%
\pgfpathmoveto{\pgfqpoint{2.172041in}{0.499444in}}%
\pgfpathlineto{\pgfqpoint{2.233427in}{0.499444in}}%
\pgfpathlineto{\pgfqpoint{2.233427in}{0.643356in}}%
\pgfpathlineto{\pgfqpoint{2.172041in}{0.643356in}}%
\pgfpathlineto{\pgfqpoint{2.172041in}{0.499444in}}%
\pgfpathclose%
\pgfusepath{fill}%
\end{pgfscope}%
\begin{pgfscope}%
\pgfpathrectangle{\pgfqpoint{0.445556in}{0.499444in}}{\pgfqpoint{3.875000in}{1.155000in}}%
\pgfusepath{clip}%
\pgfsetbuttcap%
\pgfsetmiterjoin%
\definecolor{currentfill}{rgb}{0.000000,0.000000,0.000000}%
\pgfsetfillcolor{currentfill}%
\pgfsetlinewidth{0.000000pt}%
\definecolor{currentstroke}{rgb}{0.000000,0.000000,0.000000}%
\pgfsetstrokecolor{currentstroke}%
\pgfsetstrokeopacity{0.000000}%
\pgfsetdash{}{0pt}%
\pgfpathmoveto{\pgfqpoint{2.325506in}{0.499444in}}%
\pgfpathlineto{\pgfqpoint{2.386892in}{0.499444in}}%
\pgfpathlineto{\pgfqpoint{2.386892in}{0.662622in}}%
\pgfpathlineto{\pgfqpoint{2.325506in}{0.662622in}}%
\pgfpathlineto{\pgfqpoint{2.325506in}{0.499444in}}%
\pgfpathclose%
\pgfusepath{fill}%
\end{pgfscope}%
\begin{pgfscope}%
\pgfpathrectangle{\pgfqpoint{0.445556in}{0.499444in}}{\pgfqpoint{3.875000in}{1.155000in}}%
\pgfusepath{clip}%
\pgfsetbuttcap%
\pgfsetmiterjoin%
\definecolor{currentfill}{rgb}{0.000000,0.000000,0.000000}%
\pgfsetfillcolor{currentfill}%
\pgfsetlinewidth{0.000000pt}%
\definecolor{currentstroke}{rgb}{0.000000,0.000000,0.000000}%
\pgfsetstrokecolor{currentstroke}%
\pgfsetstrokeopacity{0.000000}%
\pgfsetdash{}{0pt}%
\pgfpathmoveto{\pgfqpoint{2.478972in}{0.499444in}}%
\pgfpathlineto{\pgfqpoint{2.540358in}{0.499444in}}%
\pgfpathlineto{\pgfqpoint{2.540358in}{0.672216in}}%
\pgfpathlineto{\pgfqpoint{2.478972in}{0.672216in}}%
\pgfpathlineto{\pgfqpoint{2.478972in}{0.499444in}}%
\pgfpathclose%
\pgfusepath{fill}%
\end{pgfscope}%
\begin{pgfscope}%
\pgfpathrectangle{\pgfqpoint{0.445556in}{0.499444in}}{\pgfqpoint{3.875000in}{1.155000in}}%
\pgfusepath{clip}%
\pgfsetbuttcap%
\pgfsetmiterjoin%
\definecolor{currentfill}{rgb}{0.000000,0.000000,0.000000}%
\pgfsetfillcolor{currentfill}%
\pgfsetlinewidth{0.000000pt}%
\definecolor{currentstroke}{rgb}{0.000000,0.000000,0.000000}%
\pgfsetstrokecolor{currentstroke}%
\pgfsetstrokeopacity{0.000000}%
\pgfsetdash{}{0pt}%
\pgfpathmoveto{\pgfqpoint{2.632437in}{0.499444in}}%
\pgfpathlineto{\pgfqpoint{2.693823in}{0.499444in}}%
\pgfpathlineto{\pgfqpoint{2.693823in}{0.684286in}}%
\pgfpathlineto{\pgfqpoint{2.632437in}{0.684286in}}%
\pgfpathlineto{\pgfqpoint{2.632437in}{0.499444in}}%
\pgfpathclose%
\pgfusepath{fill}%
\end{pgfscope}%
\begin{pgfscope}%
\pgfpathrectangle{\pgfqpoint{0.445556in}{0.499444in}}{\pgfqpoint{3.875000in}{1.155000in}}%
\pgfusepath{clip}%
\pgfsetbuttcap%
\pgfsetmiterjoin%
\definecolor{currentfill}{rgb}{0.000000,0.000000,0.000000}%
\pgfsetfillcolor{currentfill}%
\pgfsetlinewidth{0.000000pt}%
\definecolor{currentstroke}{rgb}{0.000000,0.000000,0.000000}%
\pgfsetstrokecolor{currentstroke}%
\pgfsetstrokeopacity{0.000000}%
\pgfsetdash{}{0pt}%
\pgfpathmoveto{\pgfqpoint{2.785902in}{0.499444in}}%
\pgfpathlineto{\pgfqpoint{2.847288in}{0.499444in}}%
\pgfpathlineto{\pgfqpoint{2.847288in}{0.686607in}}%
\pgfpathlineto{\pgfqpoint{2.785902in}{0.686607in}}%
\pgfpathlineto{\pgfqpoint{2.785902in}{0.499444in}}%
\pgfpathclose%
\pgfusepath{fill}%
\end{pgfscope}%
\begin{pgfscope}%
\pgfpathrectangle{\pgfqpoint{0.445556in}{0.499444in}}{\pgfqpoint{3.875000in}{1.155000in}}%
\pgfusepath{clip}%
\pgfsetbuttcap%
\pgfsetmiterjoin%
\definecolor{currentfill}{rgb}{0.000000,0.000000,0.000000}%
\pgfsetfillcolor{currentfill}%
\pgfsetlinewidth{0.000000pt}%
\definecolor{currentstroke}{rgb}{0.000000,0.000000,0.000000}%
\pgfsetstrokecolor{currentstroke}%
\pgfsetstrokeopacity{0.000000}%
\pgfsetdash{}{0pt}%
\pgfpathmoveto{\pgfqpoint{2.939368in}{0.499444in}}%
\pgfpathlineto{\pgfqpoint{3.000754in}{0.499444in}}%
\pgfpathlineto{\pgfqpoint{3.000754in}{0.682816in}}%
\pgfpathlineto{\pgfqpoint{2.939368in}{0.682816in}}%
\pgfpathlineto{\pgfqpoint{2.939368in}{0.499444in}}%
\pgfpathclose%
\pgfusepath{fill}%
\end{pgfscope}%
\begin{pgfscope}%
\pgfpathrectangle{\pgfqpoint{0.445556in}{0.499444in}}{\pgfqpoint{3.875000in}{1.155000in}}%
\pgfusepath{clip}%
\pgfsetbuttcap%
\pgfsetmiterjoin%
\definecolor{currentfill}{rgb}{0.000000,0.000000,0.000000}%
\pgfsetfillcolor{currentfill}%
\pgfsetlinewidth{0.000000pt}%
\definecolor{currentstroke}{rgb}{0.000000,0.000000,0.000000}%
\pgfsetstrokecolor{currentstroke}%
\pgfsetstrokeopacity{0.000000}%
\pgfsetdash{}{0pt}%
\pgfpathmoveto{\pgfqpoint{3.092833in}{0.499444in}}%
\pgfpathlineto{\pgfqpoint{3.154219in}{0.499444in}}%
\pgfpathlineto{\pgfqpoint{3.154219in}{0.681037in}}%
\pgfpathlineto{\pgfqpoint{3.092833in}{0.681037in}}%
\pgfpathlineto{\pgfqpoint{3.092833in}{0.499444in}}%
\pgfpathclose%
\pgfusepath{fill}%
\end{pgfscope}%
\begin{pgfscope}%
\pgfpathrectangle{\pgfqpoint{0.445556in}{0.499444in}}{\pgfqpoint{3.875000in}{1.155000in}}%
\pgfusepath{clip}%
\pgfsetbuttcap%
\pgfsetmiterjoin%
\definecolor{currentfill}{rgb}{0.000000,0.000000,0.000000}%
\pgfsetfillcolor{currentfill}%
\pgfsetlinewidth{0.000000pt}%
\definecolor{currentstroke}{rgb}{0.000000,0.000000,0.000000}%
\pgfsetstrokecolor{currentstroke}%
\pgfsetstrokeopacity{0.000000}%
\pgfsetdash{}{0pt}%
\pgfpathmoveto{\pgfqpoint{3.246298in}{0.499444in}}%
\pgfpathlineto{\pgfqpoint{3.307684in}{0.499444in}}%
\pgfpathlineto{\pgfqpoint{3.307684in}{0.677787in}}%
\pgfpathlineto{\pgfqpoint{3.246298in}{0.677787in}}%
\pgfpathlineto{\pgfqpoint{3.246298in}{0.499444in}}%
\pgfpathclose%
\pgfusepath{fill}%
\end{pgfscope}%
\begin{pgfscope}%
\pgfpathrectangle{\pgfqpoint{0.445556in}{0.499444in}}{\pgfqpoint{3.875000in}{1.155000in}}%
\pgfusepath{clip}%
\pgfsetbuttcap%
\pgfsetmiterjoin%
\definecolor{currentfill}{rgb}{0.000000,0.000000,0.000000}%
\pgfsetfillcolor{currentfill}%
\pgfsetlinewidth{0.000000pt}%
\definecolor{currentstroke}{rgb}{0.000000,0.000000,0.000000}%
\pgfsetstrokecolor{currentstroke}%
\pgfsetstrokeopacity{0.000000}%
\pgfsetdash{}{0pt}%
\pgfpathmoveto{\pgfqpoint{3.399764in}{0.499444in}}%
\pgfpathlineto{\pgfqpoint{3.461150in}{0.499444in}}%
\pgfpathlineto{\pgfqpoint{3.461150in}{0.659527in}}%
\pgfpathlineto{\pgfqpoint{3.399764in}{0.659527in}}%
\pgfpathlineto{\pgfqpoint{3.399764in}{0.499444in}}%
\pgfpathclose%
\pgfusepath{fill}%
\end{pgfscope}%
\begin{pgfscope}%
\pgfpathrectangle{\pgfqpoint{0.445556in}{0.499444in}}{\pgfqpoint{3.875000in}{1.155000in}}%
\pgfusepath{clip}%
\pgfsetbuttcap%
\pgfsetmiterjoin%
\definecolor{currentfill}{rgb}{0.000000,0.000000,0.000000}%
\pgfsetfillcolor{currentfill}%
\pgfsetlinewidth{0.000000pt}%
\definecolor{currentstroke}{rgb}{0.000000,0.000000,0.000000}%
\pgfsetstrokecolor{currentstroke}%
\pgfsetstrokeopacity{0.000000}%
\pgfsetdash{}{0pt}%
\pgfpathmoveto{\pgfqpoint{3.553229in}{0.499444in}}%
\pgfpathlineto{\pgfqpoint{3.614615in}{0.499444in}}%
\pgfpathlineto{\pgfqpoint{3.614615in}{0.657206in}}%
\pgfpathlineto{\pgfqpoint{3.553229in}{0.657206in}}%
\pgfpathlineto{\pgfqpoint{3.553229in}{0.499444in}}%
\pgfpathclose%
\pgfusepath{fill}%
\end{pgfscope}%
\begin{pgfscope}%
\pgfpathrectangle{\pgfqpoint{0.445556in}{0.499444in}}{\pgfqpoint{3.875000in}{1.155000in}}%
\pgfusepath{clip}%
\pgfsetbuttcap%
\pgfsetmiterjoin%
\definecolor{currentfill}{rgb}{0.000000,0.000000,0.000000}%
\pgfsetfillcolor{currentfill}%
\pgfsetlinewidth{0.000000pt}%
\definecolor{currentstroke}{rgb}{0.000000,0.000000,0.000000}%
\pgfsetstrokecolor{currentstroke}%
\pgfsetstrokeopacity{0.000000}%
\pgfsetdash{}{0pt}%
\pgfpathmoveto{\pgfqpoint{3.706694in}{0.499444in}}%
\pgfpathlineto{\pgfqpoint{3.768080in}{0.499444in}}%
\pgfpathlineto{\pgfqpoint{3.768080in}{0.639333in}}%
\pgfpathlineto{\pgfqpoint{3.706694in}{0.639333in}}%
\pgfpathlineto{\pgfqpoint{3.706694in}{0.499444in}}%
\pgfpathclose%
\pgfusepath{fill}%
\end{pgfscope}%
\begin{pgfscope}%
\pgfpathrectangle{\pgfqpoint{0.445556in}{0.499444in}}{\pgfqpoint{3.875000in}{1.155000in}}%
\pgfusepath{clip}%
\pgfsetbuttcap%
\pgfsetmiterjoin%
\definecolor{currentfill}{rgb}{0.000000,0.000000,0.000000}%
\pgfsetfillcolor{currentfill}%
\pgfsetlinewidth{0.000000pt}%
\definecolor{currentstroke}{rgb}{0.000000,0.000000,0.000000}%
\pgfsetstrokecolor{currentstroke}%
\pgfsetstrokeopacity{0.000000}%
\pgfsetdash{}{0pt}%
\pgfpathmoveto{\pgfqpoint{3.860160in}{0.499444in}}%
\pgfpathlineto{\pgfqpoint{3.921546in}{0.499444in}}%
\pgfpathlineto{\pgfqpoint{3.921546in}{0.623626in}}%
\pgfpathlineto{\pgfqpoint{3.860160in}{0.623626in}}%
\pgfpathlineto{\pgfqpoint{3.860160in}{0.499444in}}%
\pgfpathclose%
\pgfusepath{fill}%
\end{pgfscope}%
\begin{pgfscope}%
\pgfpathrectangle{\pgfqpoint{0.445556in}{0.499444in}}{\pgfqpoint{3.875000in}{1.155000in}}%
\pgfusepath{clip}%
\pgfsetbuttcap%
\pgfsetmiterjoin%
\definecolor{currentfill}{rgb}{0.000000,0.000000,0.000000}%
\pgfsetfillcolor{currentfill}%
\pgfsetlinewidth{0.000000pt}%
\definecolor{currentstroke}{rgb}{0.000000,0.000000,0.000000}%
\pgfsetstrokecolor{currentstroke}%
\pgfsetstrokeopacity{0.000000}%
\pgfsetdash{}{0pt}%
\pgfpathmoveto{\pgfqpoint{4.013625in}{0.499444in}}%
\pgfpathlineto{\pgfqpoint{4.075011in}{0.499444in}}%
\pgfpathlineto{\pgfqpoint{4.075011in}{0.580840in}}%
\pgfpathlineto{\pgfqpoint{4.013625in}{0.580840in}}%
\pgfpathlineto{\pgfqpoint{4.013625in}{0.499444in}}%
\pgfpathclose%
\pgfusepath{fill}%
\end{pgfscope}%
\begin{pgfscope}%
\pgfpathrectangle{\pgfqpoint{0.445556in}{0.499444in}}{\pgfqpoint{3.875000in}{1.155000in}}%
\pgfusepath{clip}%
\pgfsetbuttcap%
\pgfsetmiterjoin%
\definecolor{currentfill}{rgb}{0.000000,0.000000,0.000000}%
\pgfsetfillcolor{currentfill}%
\pgfsetlinewidth{0.000000pt}%
\definecolor{currentstroke}{rgb}{0.000000,0.000000,0.000000}%
\pgfsetstrokecolor{currentstroke}%
\pgfsetstrokeopacity{0.000000}%
\pgfsetdash{}{0pt}%
\pgfpathmoveto{\pgfqpoint{4.167090in}{0.499444in}}%
\pgfpathlineto{\pgfqpoint{4.228476in}{0.499444in}}%
\pgfpathlineto{\pgfqpoint{4.228476in}{0.529774in}}%
\pgfpathlineto{\pgfqpoint{4.167090in}{0.529774in}}%
\pgfpathlineto{\pgfqpoint{4.167090in}{0.499444in}}%
\pgfpathclose%
\pgfusepath{fill}%
\end{pgfscope}%
\begin{pgfscope}%
\pgfsetbuttcap%
\pgfsetroundjoin%
\definecolor{currentfill}{rgb}{0.000000,0.000000,0.000000}%
\pgfsetfillcolor{currentfill}%
\pgfsetlinewidth{0.803000pt}%
\definecolor{currentstroke}{rgb}{0.000000,0.000000,0.000000}%
\pgfsetstrokecolor{currentstroke}%
\pgfsetdash{}{0pt}%
\pgfsys@defobject{currentmarker}{\pgfqpoint{0.000000in}{-0.048611in}}{\pgfqpoint{0.000000in}{0.000000in}}{%
\pgfpathmoveto{\pgfqpoint{0.000000in}{0.000000in}}%
\pgfpathlineto{\pgfqpoint{0.000000in}{-0.048611in}}%
\pgfusepath{stroke,fill}%
}%
\begin{pgfscope}%
\pgfsys@transformshift{0.483922in}{0.499444in}%
\pgfsys@useobject{currentmarker}{}%
\end{pgfscope}%
\end{pgfscope}%
\begin{pgfscope}%
\definecolor{textcolor}{rgb}{0.000000,0.000000,0.000000}%
\pgfsetstrokecolor{textcolor}%
\pgfsetfillcolor{textcolor}%
\pgftext[x=0.483922in,y=0.402222in,,top]{\color{textcolor}\rmfamily\fontsize{10.000000}{12.000000}\selectfont 0.0}%
\end{pgfscope}%
\begin{pgfscope}%
\pgfsetbuttcap%
\pgfsetroundjoin%
\definecolor{currentfill}{rgb}{0.000000,0.000000,0.000000}%
\pgfsetfillcolor{currentfill}%
\pgfsetlinewidth{0.803000pt}%
\definecolor{currentstroke}{rgb}{0.000000,0.000000,0.000000}%
\pgfsetstrokecolor{currentstroke}%
\pgfsetdash{}{0pt}%
\pgfsys@defobject{currentmarker}{\pgfqpoint{0.000000in}{-0.048611in}}{\pgfqpoint{0.000000in}{0.000000in}}{%
\pgfpathmoveto{\pgfqpoint{0.000000in}{0.000000in}}%
\pgfpathlineto{\pgfqpoint{0.000000in}{-0.048611in}}%
\pgfusepath{stroke,fill}%
}%
\begin{pgfscope}%
\pgfsys@transformshift{0.867585in}{0.499444in}%
\pgfsys@useobject{currentmarker}{}%
\end{pgfscope}%
\end{pgfscope}%
\begin{pgfscope}%
\definecolor{textcolor}{rgb}{0.000000,0.000000,0.000000}%
\pgfsetstrokecolor{textcolor}%
\pgfsetfillcolor{textcolor}%
\pgftext[x=0.867585in,y=0.402222in,,top]{\color{textcolor}\rmfamily\fontsize{10.000000}{12.000000}\selectfont 0.1}%
\end{pgfscope}%
\begin{pgfscope}%
\pgfsetbuttcap%
\pgfsetroundjoin%
\definecolor{currentfill}{rgb}{0.000000,0.000000,0.000000}%
\pgfsetfillcolor{currentfill}%
\pgfsetlinewidth{0.803000pt}%
\definecolor{currentstroke}{rgb}{0.000000,0.000000,0.000000}%
\pgfsetstrokecolor{currentstroke}%
\pgfsetdash{}{0pt}%
\pgfsys@defobject{currentmarker}{\pgfqpoint{0.000000in}{-0.048611in}}{\pgfqpoint{0.000000in}{0.000000in}}{%
\pgfpathmoveto{\pgfqpoint{0.000000in}{0.000000in}}%
\pgfpathlineto{\pgfqpoint{0.000000in}{-0.048611in}}%
\pgfusepath{stroke,fill}%
}%
\begin{pgfscope}%
\pgfsys@transformshift{1.251249in}{0.499444in}%
\pgfsys@useobject{currentmarker}{}%
\end{pgfscope}%
\end{pgfscope}%
\begin{pgfscope}%
\definecolor{textcolor}{rgb}{0.000000,0.000000,0.000000}%
\pgfsetstrokecolor{textcolor}%
\pgfsetfillcolor{textcolor}%
\pgftext[x=1.251249in,y=0.402222in,,top]{\color{textcolor}\rmfamily\fontsize{10.000000}{12.000000}\selectfont 0.2}%
\end{pgfscope}%
\begin{pgfscope}%
\pgfsetbuttcap%
\pgfsetroundjoin%
\definecolor{currentfill}{rgb}{0.000000,0.000000,0.000000}%
\pgfsetfillcolor{currentfill}%
\pgfsetlinewidth{0.803000pt}%
\definecolor{currentstroke}{rgb}{0.000000,0.000000,0.000000}%
\pgfsetstrokecolor{currentstroke}%
\pgfsetdash{}{0pt}%
\pgfsys@defobject{currentmarker}{\pgfqpoint{0.000000in}{-0.048611in}}{\pgfqpoint{0.000000in}{0.000000in}}{%
\pgfpathmoveto{\pgfqpoint{0.000000in}{0.000000in}}%
\pgfpathlineto{\pgfqpoint{0.000000in}{-0.048611in}}%
\pgfusepath{stroke,fill}%
}%
\begin{pgfscope}%
\pgfsys@transformshift{1.634912in}{0.499444in}%
\pgfsys@useobject{currentmarker}{}%
\end{pgfscope}%
\end{pgfscope}%
\begin{pgfscope}%
\definecolor{textcolor}{rgb}{0.000000,0.000000,0.000000}%
\pgfsetstrokecolor{textcolor}%
\pgfsetfillcolor{textcolor}%
\pgftext[x=1.634912in,y=0.402222in,,top]{\color{textcolor}\rmfamily\fontsize{10.000000}{12.000000}\selectfont 0.3}%
\end{pgfscope}%
\begin{pgfscope}%
\pgfsetbuttcap%
\pgfsetroundjoin%
\definecolor{currentfill}{rgb}{0.000000,0.000000,0.000000}%
\pgfsetfillcolor{currentfill}%
\pgfsetlinewidth{0.803000pt}%
\definecolor{currentstroke}{rgb}{0.000000,0.000000,0.000000}%
\pgfsetstrokecolor{currentstroke}%
\pgfsetdash{}{0pt}%
\pgfsys@defobject{currentmarker}{\pgfqpoint{0.000000in}{-0.048611in}}{\pgfqpoint{0.000000in}{0.000000in}}{%
\pgfpathmoveto{\pgfqpoint{0.000000in}{0.000000in}}%
\pgfpathlineto{\pgfqpoint{0.000000in}{-0.048611in}}%
\pgfusepath{stroke,fill}%
}%
\begin{pgfscope}%
\pgfsys@transformshift{2.018575in}{0.499444in}%
\pgfsys@useobject{currentmarker}{}%
\end{pgfscope}%
\end{pgfscope}%
\begin{pgfscope}%
\definecolor{textcolor}{rgb}{0.000000,0.000000,0.000000}%
\pgfsetstrokecolor{textcolor}%
\pgfsetfillcolor{textcolor}%
\pgftext[x=2.018575in,y=0.402222in,,top]{\color{textcolor}\rmfamily\fontsize{10.000000}{12.000000}\selectfont 0.4}%
\end{pgfscope}%
\begin{pgfscope}%
\pgfsetbuttcap%
\pgfsetroundjoin%
\definecolor{currentfill}{rgb}{0.000000,0.000000,0.000000}%
\pgfsetfillcolor{currentfill}%
\pgfsetlinewidth{0.803000pt}%
\definecolor{currentstroke}{rgb}{0.000000,0.000000,0.000000}%
\pgfsetstrokecolor{currentstroke}%
\pgfsetdash{}{0pt}%
\pgfsys@defobject{currentmarker}{\pgfqpoint{0.000000in}{-0.048611in}}{\pgfqpoint{0.000000in}{0.000000in}}{%
\pgfpathmoveto{\pgfqpoint{0.000000in}{0.000000in}}%
\pgfpathlineto{\pgfqpoint{0.000000in}{-0.048611in}}%
\pgfusepath{stroke,fill}%
}%
\begin{pgfscope}%
\pgfsys@transformshift{2.402239in}{0.499444in}%
\pgfsys@useobject{currentmarker}{}%
\end{pgfscope}%
\end{pgfscope}%
\begin{pgfscope}%
\definecolor{textcolor}{rgb}{0.000000,0.000000,0.000000}%
\pgfsetstrokecolor{textcolor}%
\pgfsetfillcolor{textcolor}%
\pgftext[x=2.402239in,y=0.402222in,,top]{\color{textcolor}\rmfamily\fontsize{10.000000}{12.000000}\selectfont 0.5}%
\end{pgfscope}%
\begin{pgfscope}%
\pgfsetbuttcap%
\pgfsetroundjoin%
\definecolor{currentfill}{rgb}{0.000000,0.000000,0.000000}%
\pgfsetfillcolor{currentfill}%
\pgfsetlinewidth{0.803000pt}%
\definecolor{currentstroke}{rgb}{0.000000,0.000000,0.000000}%
\pgfsetstrokecolor{currentstroke}%
\pgfsetdash{}{0pt}%
\pgfsys@defobject{currentmarker}{\pgfqpoint{0.000000in}{-0.048611in}}{\pgfqpoint{0.000000in}{0.000000in}}{%
\pgfpathmoveto{\pgfqpoint{0.000000in}{0.000000in}}%
\pgfpathlineto{\pgfqpoint{0.000000in}{-0.048611in}}%
\pgfusepath{stroke,fill}%
}%
\begin{pgfscope}%
\pgfsys@transformshift{2.785902in}{0.499444in}%
\pgfsys@useobject{currentmarker}{}%
\end{pgfscope}%
\end{pgfscope}%
\begin{pgfscope}%
\definecolor{textcolor}{rgb}{0.000000,0.000000,0.000000}%
\pgfsetstrokecolor{textcolor}%
\pgfsetfillcolor{textcolor}%
\pgftext[x=2.785902in,y=0.402222in,,top]{\color{textcolor}\rmfamily\fontsize{10.000000}{12.000000}\selectfont 0.6}%
\end{pgfscope}%
\begin{pgfscope}%
\pgfsetbuttcap%
\pgfsetroundjoin%
\definecolor{currentfill}{rgb}{0.000000,0.000000,0.000000}%
\pgfsetfillcolor{currentfill}%
\pgfsetlinewidth{0.803000pt}%
\definecolor{currentstroke}{rgb}{0.000000,0.000000,0.000000}%
\pgfsetstrokecolor{currentstroke}%
\pgfsetdash{}{0pt}%
\pgfsys@defobject{currentmarker}{\pgfqpoint{0.000000in}{-0.048611in}}{\pgfqpoint{0.000000in}{0.000000in}}{%
\pgfpathmoveto{\pgfqpoint{0.000000in}{0.000000in}}%
\pgfpathlineto{\pgfqpoint{0.000000in}{-0.048611in}}%
\pgfusepath{stroke,fill}%
}%
\begin{pgfscope}%
\pgfsys@transformshift{3.169566in}{0.499444in}%
\pgfsys@useobject{currentmarker}{}%
\end{pgfscope}%
\end{pgfscope}%
\begin{pgfscope}%
\definecolor{textcolor}{rgb}{0.000000,0.000000,0.000000}%
\pgfsetstrokecolor{textcolor}%
\pgfsetfillcolor{textcolor}%
\pgftext[x=3.169566in,y=0.402222in,,top]{\color{textcolor}\rmfamily\fontsize{10.000000}{12.000000}\selectfont 0.7}%
\end{pgfscope}%
\begin{pgfscope}%
\pgfsetbuttcap%
\pgfsetroundjoin%
\definecolor{currentfill}{rgb}{0.000000,0.000000,0.000000}%
\pgfsetfillcolor{currentfill}%
\pgfsetlinewidth{0.803000pt}%
\definecolor{currentstroke}{rgb}{0.000000,0.000000,0.000000}%
\pgfsetstrokecolor{currentstroke}%
\pgfsetdash{}{0pt}%
\pgfsys@defobject{currentmarker}{\pgfqpoint{0.000000in}{-0.048611in}}{\pgfqpoint{0.000000in}{0.000000in}}{%
\pgfpathmoveto{\pgfqpoint{0.000000in}{0.000000in}}%
\pgfpathlineto{\pgfqpoint{0.000000in}{-0.048611in}}%
\pgfusepath{stroke,fill}%
}%
\begin{pgfscope}%
\pgfsys@transformshift{3.553229in}{0.499444in}%
\pgfsys@useobject{currentmarker}{}%
\end{pgfscope}%
\end{pgfscope}%
\begin{pgfscope}%
\definecolor{textcolor}{rgb}{0.000000,0.000000,0.000000}%
\pgfsetstrokecolor{textcolor}%
\pgfsetfillcolor{textcolor}%
\pgftext[x=3.553229in,y=0.402222in,,top]{\color{textcolor}\rmfamily\fontsize{10.000000}{12.000000}\selectfont 0.8}%
\end{pgfscope}%
\begin{pgfscope}%
\pgfsetbuttcap%
\pgfsetroundjoin%
\definecolor{currentfill}{rgb}{0.000000,0.000000,0.000000}%
\pgfsetfillcolor{currentfill}%
\pgfsetlinewidth{0.803000pt}%
\definecolor{currentstroke}{rgb}{0.000000,0.000000,0.000000}%
\pgfsetstrokecolor{currentstroke}%
\pgfsetdash{}{0pt}%
\pgfsys@defobject{currentmarker}{\pgfqpoint{0.000000in}{-0.048611in}}{\pgfqpoint{0.000000in}{0.000000in}}{%
\pgfpathmoveto{\pgfqpoint{0.000000in}{0.000000in}}%
\pgfpathlineto{\pgfqpoint{0.000000in}{-0.048611in}}%
\pgfusepath{stroke,fill}%
}%
\begin{pgfscope}%
\pgfsys@transformshift{3.936892in}{0.499444in}%
\pgfsys@useobject{currentmarker}{}%
\end{pgfscope}%
\end{pgfscope}%
\begin{pgfscope}%
\definecolor{textcolor}{rgb}{0.000000,0.000000,0.000000}%
\pgfsetstrokecolor{textcolor}%
\pgfsetfillcolor{textcolor}%
\pgftext[x=3.936892in,y=0.402222in,,top]{\color{textcolor}\rmfamily\fontsize{10.000000}{12.000000}\selectfont 0.9}%
\end{pgfscope}%
\begin{pgfscope}%
\pgfsetbuttcap%
\pgfsetroundjoin%
\definecolor{currentfill}{rgb}{0.000000,0.000000,0.000000}%
\pgfsetfillcolor{currentfill}%
\pgfsetlinewidth{0.803000pt}%
\definecolor{currentstroke}{rgb}{0.000000,0.000000,0.000000}%
\pgfsetstrokecolor{currentstroke}%
\pgfsetdash{}{0pt}%
\pgfsys@defobject{currentmarker}{\pgfqpoint{0.000000in}{-0.048611in}}{\pgfqpoint{0.000000in}{0.000000in}}{%
\pgfpathmoveto{\pgfqpoint{0.000000in}{0.000000in}}%
\pgfpathlineto{\pgfqpoint{0.000000in}{-0.048611in}}%
\pgfusepath{stroke,fill}%
}%
\begin{pgfscope}%
\pgfsys@transformshift{4.320556in}{0.499444in}%
\pgfsys@useobject{currentmarker}{}%
\end{pgfscope}%
\end{pgfscope}%
\begin{pgfscope}%
\definecolor{textcolor}{rgb}{0.000000,0.000000,0.000000}%
\pgfsetstrokecolor{textcolor}%
\pgfsetfillcolor{textcolor}%
\pgftext[x=4.320556in,y=0.402222in,,top]{\color{textcolor}\rmfamily\fontsize{10.000000}{12.000000}\selectfont 1.0}%
\end{pgfscope}%
\begin{pgfscope}%
\definecolor{textcolor}{rgb}{0.000000,0.000000,0.000000}%
\pgfsetstrokecolor{textcolor}%
\pgfsetfillcolor{textcolor}%
\pgftext[x=2.383056in,y=0.223333in,,top]{\color{textcolor}\rmfamily\fontsize{10.000000}{12.000000}\selectfont \(\displaystyle p\)}%
\end{pgfscope}%
\begin{pgfscope}%
\pgfsetbuttcap%
\pgfsetroundjoin%
\definecolor{currentfill}{rgb}{0.000000,0.000000,0.000000}%
\pgfsetfillcolor{currentfill}%
\pgfsetlinewidth{0.803000pt}%
\definecolor{currentstroke}{rgb}{0.000000,0.000000,0.000000}%
\pgfsetstrokecolor{currentstroke}%
\pgfsetdash{}{0pt}%
\pgfsys@defobject{currentmarker}{\pgfqpoint{-0.048611in}{0.000000in}}{\pgfqpoint{-0.000000in}{0.000000in}}{%
\pgfpathmoveto{\pgfqpoint{-0.000000in}{0.000000in}}%
\pgfpathlineto{\pgfqpoint{-0.048611in}{0.000000in}}%
\pgfusepath{stroke,fill}%
}%
\begin{pgfscope}%
\pgfsys@transformshift{0.445556in}{0.499444in}%
\pgfsys@useobject{currentmarker}{}%
\end{pgfscope}%
\end{pgfscope}%
\begin{pgfscope}%
\definecolor{textcolor}{rgb}{0.000000,0.000000,0.000000}%
\pgfsetstrokecolor{textcolor}%
\pgfsetfillcolor{textcolor}%
\pgftext[x=0.278889in, y=0.451250in, left, base]{\color{textcolor}\rmfamily\fontsize{10.000000}{12.000000}\selectfont \(\displaystyle {0}\)}%
\end{pgfscope}%
\begin{pgfscope}%
\pgfsetbuttcap%
\pgfsetroundjoin%
\definecolor{currentfill}{rgb}{0.000000,0.000000,0.000000}%
\pgfsetfillcolor{currentfill}%
\pgfsetlinewidth{0.803000pt}%
\definecolor{currentstroke}{rgb}{0.000000,0.000000,0.000000}%
\pgfsetstrokecolor{currentstroke}%
\pgfsetdash{}{0pt}%
\pgfsys@defobject{currentmarker}{\pgfqpoint{-0.048611in}{0.000000in}}{\pgfqpoint{-0.000000in}{0.000000in}}{%
\pgfpathmoveto{\pgfqpoint{-0.000000in}{0.000000in}}%
\pgfpathlineto{\pgfqpoint{-0.048611in}{0.000000in}}%
\pgfusepath{stroke,fill}%
}%
\begin{pgfscope}%
\pgfsys@transformshift{0.445556in}{0.830705in}%
\pgfsys@useobject{currentmarker}{}%
\end{pgfscope}%
\end{pgfscope}%
\begin{pgfscope}%
\definecolor{textcolor}{rgb}{0.000000,0.000000,0.000000}%
\pgfsetstrokecolor{textcolor}%
\pgfsetfillcolor{textcolor}%
\pgftext[x=0.278889in, y=0.782511in, left, base]{\color{textcolor}\rmfamily\fontsize{10.000000}{12.000000}\selectfont \(\displaystyle {2}\)}%
\end{pgfscope}%
\begin{pgfscope}%
\pgfsetbuttcap%
\pgfsetroundjoin%
\definecolor{currentfill}{rgb}{0.000000,0.000000,0.000000}%
\pgfsetfillcolor{currentfill}%
\pgfsetlinewidth{0.803000pt}%
\definecolor{currentstroke}{rgb}{0.000000,0.000000,0.000000}%
\pgfsetstrokecolor{currentstroke}%
\pgfsetdash{}{0pt}%
\pgfsys@defobject{currentmarker}{\pgfqpoint{-0.048611in}{0.000000in}}{\pgfqpoint{-0.000000in}{0.000000in}}{%
\pgfpathmoveto{\pgfqpoint{-0.000000in}{0.000000in}}%
\pgfpathlineto{\pgfqpoint{-0.048611in}{0.000000in}}%
\pgfusepath{stroke,fill}%
}%
\begin{pgfscope}%
\pgfsys@transformshift{0.445556in}{1.161966in}%
\pgfsys@useobject{currentmarker}{}%
\end{pgfscope}%
\end{pgfscope}%
\begin{pgfscope}%
\definecolor{textcolor}{rgb}{0.000000,0.000000,0.000000}%
\pgfsetstrokecolor{textcolor}%
\pgfsetfillcolor{textcolor}%
\pgftext[x=0.278889in, y=1.113772in, left, base]{\color{textcolor}\rmfamily\fontsize{10.000000}{12.000000}\selectfont \(\displaystyle {4}\)}%
\end{pgfscope}%
\begin{pgfscope}%
\pgfsetbuttcap%
\pgfsetroundjoin%
\definecolor{currentfill}{rgb}{0.000000,0.000000,0.000000}%
\pgfsetfillcolor{currentfill}%
\pgfsetlinewidth{0.803000pt}%
\definecolor{currentstroke}{rgb}{0.000000,0.000000,0.000000}%
\pgfsetstrokecolor{currentstroke}%
\pgfsetdash{}{0pt}%
\pgfsys@defobject{currentmarker}{\pgfqpoint{-0.048611in}{0.000000in}}{\pgfqpoint{-0.000000in}{0.000000in}}{%
\pgfpathmoveto{\pgfqpoint{-0.000000in}{0.000000in}}%
\pgfpathlineto{\pgfqpoint{-0.048611in}{0.000000in}}%
\pgfusepath{stroke,fill}%
}%
\begin{pgfscope}%
\pgfsys@transformshift{0.445556in}{1.493228in}%
\pgfsys@useobject{currentmarker}{}%
\end{pgfscope}%
\end{pgfscope}%
\begin{pgfscope}%
\definecolor{textcolor}{rgb}{0.000000,0.000000,0.000000}%
\pgfsetstrokecolor{textcolor}%
\pgfsetfillcolor{textcolor}%
\pgftext[x=0.278889in, y=1.445033in, left, base]{\color{textcolor}\rmfamily\fontsize{10.000000}{12.000000}\selectfont \(\displaystyle {6}\)}%
\end{pgfscope}%
\begin{pgfscope}%
\definecolor{textcolor}{rgb}{0.000000,0.000000,0.000000}%
\pgfsetstrokecolor{textcolor}%
\pgfsetfillcolor{textcolor}%
\pgftext[x=0.223333in,y=1.076944in,,bottom,rotate=90.000000]{\color{textcolor}\rmfamily\fontsize{10.000000}{12.000000}\selectfont Percent of Data Set}%
\end{pgfscope}%
\begin{pgfscope}%
\pgfsetrectcap%
\pgfsetmiterjoin%
\pgfsetlinewidth{0.803000pt}%
\definecolor{currentstroke}{rgb}{0.000000,0.000000,0.000000}%
\pgfsetstrokecolor{currentstroke}%
\pgfsetdash{}{0pt}%
\pgfpathmoveto{\pgfqpoint{0.445556in}{0.499444in}}%
\pgfpathlineto{\pgfqpoint{0.445556in}{1.654444in}}%
\pgfusepath{stroke}%
\end{pgfscope}%
\begin{pgfscope}%
\pgfsetrectcap%
\pgfsetmiterjoin%
\pgfsetlinewidth{0.803000pt}%
\definecolor{currentstroke}{rgb}{0.000000,0.000000,0.000000}%
\pgfsetstrokecolor{currentstroke}%
\pgfsetdash{}{0pt}%
\pgfpathmoveto{\pgfqpoint{4.320556in}{0.499444in}}%
\pgfpathlineto{\pgfqpoint{4.320556in}{1.654444in}}%
\pgfusepath{stroke}%
\end{pgfscope}%
\begin{pgfscope}%
\pgfsetrectcap%
\pgfsetmiterjoin%
\pgfsetlinewidth{0.803000pt}%
\definecolor{currentstroke}{rgb}{0.000000,0.000000,0.000000}%
\pgfsetstrokecolor{currentstroke}%
\pgfsetdash{}{0pt}%
\pgfpathmoveto{\pgfqpoint{0.445556in}{0.499444in}}%
\pgfpathlineto{\pgfqpoint{4.320556in}{0.499444in}}%
\pgfusepath{stroke}%
\end{pgfscope}%
\begin{pgfscope}%
\pgfsetrectcap%
\pgfsetmiterjoin%
\pgfsetlinewidth{0.803000pt}%
\definecolor{currentstroke}{rgb}{0.000000,0.000000,0.000000}%
\pgfsetstrokecolor{currentstroke}%
\pgfsetdash{}{0pt}%
\pgfpathmoveto{\pgfqpoint{0.445556in}{1.654444in}}%
\pgfpathlineto{\pgfqpoint{4.320556in}{1.654444in}}%
\pgfusepath{stroke}%
\end{pgfscope}%
\begin{pgfscope}%
\pgfsetbuttcap%
\pgfsetmiterjoin%
\definecolor{currentfill}{rgb}{1.000000,1.000000,1.000000}%
\pgfsetfillcolor{currentfill}%
\pgfsetfillopacity{0.800000}%
\pgfsetlinewidth{1.003750pt}%
\definecolor{currentstroke}{rgb}{0.800000,0.800000,0.800000}%
\pgfsetstrokecolor{currentstroke}%
\pgfsetstrokeopacity{0.800000}%
\pgfsetdash{}{0pt}%
\pgfpathmoveto{\pgfqpoint{3.543611in}{1.154445in}}%
\pgfpathlineto{\pgfqpoint{4.223333in}{1.154445in}}%
\pgfpathquadraticcurveto{\pgfqpoint{4.251111in}{1.154445in}}{\pgfqpoint{4.251111in}{1.182222in}}%
\pgfpathlineto{\pgfqpoint{4.251111in}{1.557222in}}%
\pgfpathquadraticcurveto{\pgfqpoint{4.251111in}{1.585000in}}{\pgfqpoint{4.223333in}{1.585000in}}%
\pgfpathlineto{\pgfqpoint{3.543611in}{1.585000in}}%
\pgfpathquadraticcurveto{\pgfqpoint{3.515833in}{1.585000in}}{\pgfqpoint{3.515833in}{1.557222in}}%
\pgfpathlineto{\pgfqpoint{3.515833in}{1.182222in}}%
\pgfpathquadraticcurveto{\pgfqpoint{3.515833in}{1.154445in}}{\pgfqpoint{3.543611in}{1.154445in}}%
\pgfpathlineto{\pgfqpoint{3.543611in}{1.154445in}}%
\pgfpathclose%
\pgfusepath{stroke,fill}%
\end{pgfscope}%
\begin{pgfscope}%
\pgfsetbuttcap%
\pgfsetmiterjoin%
\pgfsetlinewidth{1.003750pt}%
\definecolor{currentstroke}{rgb}{0.000000,0.000000,0.000000}%
\pgfsetstrokecolor{currentstroke}%
\pgfsetdash{}{0pt}%
\pgfpathmoveto{\pgfqpoint{3.571389in}{1.432222in}}%
\pgfpathlineto{\pgfqpoint{3.849167in}{1.432222in}}%
\pgfpathlineto{\pgfqpoint{3.849167in}{1.529444in}}%
\pgfpathlineto{\pgfqpoint{3.571389in}{1.529444in}}%
\pgfpathlineto{\pgfqpoint{3.571389in}{1.432222in}}%
\pgfpathclose%
\pgfusepath{stroke}%
\end{pgfscope}%
\begin{pgfscope}%
\definecolor{textcolor}{rgb}{0.000000,0.000000,0.000000}%
\pgfsetstrokecolor{textcolor}%
\pgfsetfillcolor{textcolor}%
\pgftext[x=3.960278in,y=1.432222in,left,base]{\color{textcolor}\rmfamily\fontsize{10.000000}{12.000000}\selectfont Neg}%
\end{pgfscope}%
\begin{pgfscope}%
\pgfsetbuttcap%
\pgfsetmiterjoin%
\definecolor{currentfill}{rgb}{0.000000,0.000000,0.000000}%
\pgfsetfillcolor{currentfill}%
\pgfsetlinewidth{0.000000pt}%
\definecolor{currentstroke}{rgb}{0.000000,0.000000,0.000000}%
\pgfsetstrokecolor{currentstroke}%
\pgfsetstrokeopacity{0.000000}%
\pgfsetdash{}{0pt}%
\pgfpathmoveto{\pgfqpoint{3.571389in}{1.236944in}}%
\pgfpathlineto{\pgfqpoint{3.849167in}{1.236944in}}%
\pgfpathlineto{\pgfqpoint{3.849167in}{1.334167in}}%
\pgfpathlineto{\pgfqpoint{3.571389in}{1.334167in}}%
\pgfpathlineto{\pgfqpoint{3.571389in}{1.236944in}}%
\pgfpathclose%
\pgfusepath{fill}%
\end{pgfscope}%
\begin{pgfscope}%
\definecolor{textcolor}{rgb}{0.000000,0.000000,0.000000}%
\pgfsetstrokecolor{textcolor}%
\pgfsetfillcolor{textcolor}%
\pgftext[x=3.960278in,y=1.236944in,left,base]{\color{textcolor}\rmfamily\fontsize{10.000000}{12.000000}\selectfont Pos}%
\end{pgfscope}%
\end{pgfpicture}%
\makeatother%
\endgroup%
	
&
	\vskip 0pt
	\hfil ROC Curve
	
	%% Creator: Matplotlib, PGF backend
%%
%% To include the figure in your LaTeX document, write
%%   \input{<filename>.pgf}
%%
%% Make sure the required packages are loaded in your preamble
%%   \usepackage{pgf}
%%
%% Also ensure that all the required font packages are loaded; for instance,
%% the lmodern package is sometimes necessary when using math font.
%%   \usepackage{lmodern}
%%
%% Figures using additional raster images can only be included by \input if
%% they are in the same directory as the main LaTeX file. For loading figures
%% from other directories you can use the `import` package
%%   \usepackage{import}
%%
%% and then include the figures with
%%   \import{<path to file>}{<filename>.pgf}
%%
%% Matplotlib used the following preamble
%%   
%%   \usepackage{fontspec}
%%   \makeatletter\@ifpackageloaded{underscore}{}{\usepackage[strings]{underscore}}\makeatother
%%
\begingroup%
\makeatletter%
\begin{pgfpicture}%
\pgfpathrectangle{\pgfpointorigin}{\pgfqpoint{2.221861in}{1.754444in}}%
\pgfusepath{use as bounding box, clip}%
\begin{pgfscope}%
\pgfsetbuttcap%
\pgfsetmiterjoin%
\definecolor{currentfill}{rgb}{1.000000,1.000000,1.000000}%
\pgfsetfillcolor{currentfill}%
\pgfsetlinewidth{0.000000pt}%
\definecolor{currentstroke}{rgb}{1.000000,1.000000,1.000000}%
\pgfsetstrokecolor{currentstroke}%
\pgfsetdash{}{0pt}%
\pgfpathmoveto{\pgfqpoint{0.000000in}{0.000000in}}%
\pgfpathlineto{\pgfqpoint{2.221861in}{0.000000in}}%
\pgfpathlineto{\pgfqpoint{2.221861in}{1.754444in}}%
\pgfpathlineto{\pgfqpoint{0.000000in}{1.754444in}}%
\pgfpathlineto{\pgfqpoint{0.000000in}{0.000000in}}%
\pgfpathclose%
\pgfusepath{fill}%
\end{pgfscope}%
\begin{pgfscope}%
\pgfsetbuttcap%
\pgfsetmiterjoin%
\definecolor{currentfill}{rgb}{1.000000,1.000000,1.000000}%
\pgfsetfillcolor{currentfill}%
\pgfsetlinewidth{0.000000pt}%
\definecolor{currentstroke}{rgb}{0.000000,0.000000,0.000000}%
\pgfsetstrokecolor{currentstroke}%
\pgfsetstrokeopacity{0.000000}%
\pgfsetdash{}{0pt}%
\pgfpathmoveto{\pgfqpoint{0.553581in}{0.499444in}}%
\pgfpathlineto{\pgfqpoint{2.103581in}{0.499444in}}%
\pgfpathlineto{\pgfqpoint{2.103581in}{1.654444in}}%
\pgfpathlineto{\pgfqpoint{0.553581in}{1.654444in}}%
\pgfpathlineto{\pgfqpoint{0.553581in}{0.499444in}}%
\pgfpathclose%
\pgfusepath{fill}%
\end{pgfscope}%
\begin{pgfscope}%
\pgfsetbuttcap%
\pgfsetroundjoin%
\definecolor{currentfill}{rgb}{0.000000,0.000000,0.000000}%
\pgfsetfillcolor{currentfill}%
\pgfsetlinewidth{0.803000pt}%
\definecolor{currentstroke}{rgb}{0.000000,0.000000,0.000000}%
\pgfsetstrokecolor{currentstroke}%
\pgfsetdash{}{0pt}%
\pgfsys@defobject{currentmarker}{\pgfqpoint{0.000000in}{-0.048611in}}{\pgfqpoint{0.000000in}{0.000000in}}{%
\pgfpathmoveto{\pgfqpoint{0.000000in}{0.000000in}}%
\pgfpathlineto{\pgfqpoint{0.000000in}{-0.048611in}}%
\pgfusepath{stroke,fill}%
}%
\begin{pgfscope}%
\pgfsys@transformshift{0.624035in}{0.499444in}%
\pgfsys@useobject{currentmarker}{}%
\end{pgfscope}%
\end{pgfscope}%
\begin{pgfscope}%
\definecolor{textcolor}{rgb}{0.000000,0.000000,0.000000}%
\pgfsetstrokecolor{textcolor}%
\pgfsetfillcolor{textcolor}%
\pgftext[x=0.624035in,y=0.402222in,,top]{\color{textcolor}\rmfamily\fontsize{10.000000}{12.000000}\selectfont \(\displaystyle {0.0}\)}%
\end{pgfscope}%
\begin{pgfscope}%
\pgfsetbuttcap%
\pgfsetroundjoin%
\definecolor{currentfill}{rgb}{0.000000,0.000000,0.000000}%
\pgfsetfillcolor{currentfill}%
\pgfsetlinewidth{0.803000pt}%
\definecolor{currentstroke}{rgb}{0.000000,0.000000,0.000000}%
\pgfsetstrokecolor{currentstroke}%
\pgfsetdash{}{0pt}%
\pgfsys@defobject{currentmarker}{\pgfqpoint{0.000000in}{-0.048611in}}{\pgfqpoint{0.000000in}{0.000000in}}{%
\pgfpathmoveto{\pgfqpoint{0.000000in}{0.000000in}}%
\pgfpathlineto{\pgfqpoint{0.000000in}{-0.048611in}}%
\pgfusepath{stroke,fill}%
}%
\begin{pgfscope}%
\pgfsys@transformshift{1.328581in}{0.499444in}%
\pgfsys@useobject{currentmarker}{}%
\end{pgfscope}%
\end{pgfscope}%
\begin{pgfscope}%
\definecolor{textcolor}{rgb}{0.000000,0.000000,0.000000}%
\pgfsetstrokecolor{textcolor}%
\pgfsetfillcolor{textcolor}%
\pgftext[x=1.328581in,y=0.402222in,,top]{\color{textcolor}\rmfamily\fontsize{10.000000}{12.000000}\selectfont \(\displaystyle {0.5}\)}%
\end{pgfscope}%
\begin{pgfscope}%
\pgfsetbuttcap%
\pgfsetroundjoin%
\definecolor{currentfill}{rgb}{0.000000,0.000000,0.000000}%
\pgfsetfillcolor{currentfill}%
\pgfsetlinewidth{0.803000pt}%
\definecolor{currentstroke}{rgb}{0.000000,0.000000,0.000000}%
\pgfsetstrokecolor{currentstroke}%
\pgfsetdash{}{0pt}%
\pgfsys@defobject{currentmarker}{\pgfqpoint{0.000000in}{-0.048611in}}{\pgfqpoint{0.000000in}{0.000000in}}{%
\pgfpathmoveto{\pgfqpoint{0.000000in}{0.000000in}}%
\pgfpathlineto{\pgfqpoint{0.000000in}{-0.048611in}}%
\pgfusepath{stroke,fill}%
}%
\begin{pgfscope}%
\pgfsys@transformshift{2.033126in}{0.499444in}%
\pgfsys@useobject{currentmarker}{}%
\end{pgfscope}%
\end{pgfscope}%
\begin{pgfscope}%
\definecolor{textcolor}{rgb}{0.000000,0.000000,0.000000}%
\pgfsetstrokecolor{textcolor}%
\pgfsetfillcolor{textcolor}%
\pgftext[x=2.033126in,y=0.402222in,,top]{\color{textcolor}\rmfamily\fontsize{10.000000}{12.000000}\selectfont \(\displaystyle {1.0}\)}%
\end{pgfscope}%
\begin{pgfscope}%
\definecolor{textcolor}{rgb}{0.000000,0.000000,0.000000}%
\pgfsetstrokecolor{textcolor}%
\pgfsetfillcolor{textcolor}%
\pgftext[x=1.328581in,y=0.223333in,,top]{\color{textcolor}\rmfamily\fontsize{10.000000}{12.000000}\selectfont False positive rate}%
\end{pgfscope}%
\begin{pgfscope}%
\pgfsetbuttcap%
\pgfsetroundjoin%
\definecolor{currentfill}{rgb}{0.000000,0.000000,0.000000}%
\pgfsetfillcolor{currentfill}%
\pgfsetlinewidth{0.803000pt}%
\definecolor{currentstroke}{rgb}{0.000000,0.000000,0.000000}%
\pgfsetstrokecolor{currentstroke}%
\pgfsetdash{}{0pt}%
\pgfsys@defobject{currentmarker}{\pgfqpoint{-0.048611in}{0.000000in}}{\pgfqpoint{-0.000000in}{0.000000in}}{%
\pgfpathmoveto{\pgfqpoint{-0.000000in}{0.000000in}}%
\pgfpathlineto{\pgfqpoint{-0.048611in}{0.000000in}}%
\pgfusepath{stroke,fill}%
}%
\begin{pgfscope}%
\pgfsys@transformshift{0.553581in}{0.551944in}%
\pgfsys@useobject{currentmarker}{}%
\end{pgfscope}%
\end{pgfscope}%
\begin{pgfscope}%
\definecolor{textcolor}{rgb}{0.000000,0.000000,0.000000}%
\pgfsetstrokecolor{textcolor}%
\pgfsetfillcolor{textcolor}%
\pgftext[x=0.278889in, y=0.503750in, left, base]{\color{textcolor}\rmfamily\fontsize{10.000000}{12.000000}\selectfont \(\displaystyle {0.0}\)}%
\end{pgfscope}%
\begin{pgfscope}%
\pgfsetbuttcap%
\pgfsetroundjoin%
\definecolor{currentfill}{rgb}{0.000000,0.000000,0.000000}%
\pgfsetfillcolor{currentfill}%
\pgfsetlinewidth{0.803000pt}%
\definecolor{currentstroke}{rgb}{0.000000,0.000000,0.000000}%
\pgfsetstrokecolor{currentstroke}%
\pgfsetdash{}{0pt}%
\pgfsys@defobject{currentmarker}{\pgfqpoint{-0.048611in}{0.000000in}}{\pgfqpoint{-0.000000in}{0.000000in}}{%
\pgfpathmoveto{\pgfqpoint{-0.000000in}{0.000000in}}%
\pgfpathlineto{\pgfqpoint{-0.048611in}{0.000000in}}%
\pgfusepath{stroke,fill}%
}%
\begin{pgfscope}%
\pgfsys@transformshift{0.553581in}{1.076944in}%
\pgfsys@useobject{currentmarker}{}%
\end{pgfscope}%
\end{pgfscope}%
\begin{pgfscope}%
\definecolor{textcolor}{rgb}{0.000000,0.000000,0.000000}%
\pgfsetstrokecolor{textcolor}%
\pgfsetfillcolor{textcolor}%
\pgftext[x=0.278889in, y=1.028750in, left, base]{\color{textcolor}\rmfamily\fontsize{10.000000}{12.000000}\selectfont \(\displaystyle {0.5}\)}%
\end{pgfscope}%
\begin{pgfscope}%
\pgfsetbuttcap%
\pgfsetroundjoin%
\definecolor{currentfill}{rgb}{0.000000,0.000000,0.000000}%
\pgfsetfillcolor{currentfill}%
\pgfsetlinewidth{0.803000pt}%
\definecolor{currentstroke}{rgb}{0.000000,0.000000,0.000000}%
\pgfsetstrokecolor{currentstroke}%
\pgfsetdash{}{0pt}%
\pgfsys@defobject{currentmarker}{\pgfqpoint{-0.048611in}{0.000000in}}{\pgfqpoint{-0.000000in}{0.000000in}}{%
\pgfpathmoveto{\pgfqpoint{-0.000000in}{0.000000in}}%
\pgfpathlineto{\pgfqpoint{-0.048611in}{0.000000in}}%
\pgfusepath{stroke,fill}%
}%
\begin{pgfscope}%
\pgfsys@transformshift{0.553581in}{1.601944in}%
\pgfsys@useobject{currentmarker}{}%
\end{pgfscope}%
\end{pgfscope}%
\begin{pgfscope}%
\definecolor{textcolor}{rgb}{0.000000,0.000000,0.000000}%
\pgfsetstrokecolor{textcolor}%
\pgfsetfillcolor{textcolor}%
\pgftext[x=0.278889in, y=1.553750in, left, base]{\color{textcolor}\rmfamily\fontsize{10.000000}{12.000000}\selectfont \(\displaystyle {1.0}\)}%
\end{pgfscope}%
\begin{pgfscope}%
\definecolor{textcolor}{rgb}{0.000000,0.000000,0.000000}%
\pgfsetstrokecolor{textcolor}%
\pgfsetfillcolor{textcolor}%
\pgftext[x=0.223333in,y=1.076944in,,bottom,rotate=90.000000]{\color{textcolor}\rmfamily\fontsize{10.000000}{12.000000}\selectfont True positive rate}%
\end{pgfscope}%
\begin{pgfscope}%
\pgfpathrectangle{\pgfqpoint{0.553581in}{0.499444in}}{\pgfqpoint{1.550000in}{1.155000in}}%
\pgfusepath{clip}%
\pgfsetbuttcap%
\pgfsetroundjoin%
\pgfsetlinewidth{1.505625pt}%
\definecolor{currentstroke}{rgb}{0.000000,0.000000,0.000000}%
\pgfsetstrokecolor{currentstroke}%
\pgfsetdash{{5.550000pt}{2.400000pt}}{0.000000pt}%
\pgfpathmoveto{\pgfqpoint{0.624035in}{0.551944in}}%
\pgfpathlineto{\pgfqpoint{2.033126in}{1.601944in}}%
\pgfusepath{stroke}%
\end{pgfscope}%
\begin{pgfscope}%
\pgfpathrectangle{\pgfqpoint{0.553581in}{0.499444in}}{\pgfqpoint{1.550000in}{1.155000in}}%
\pgfusepath{clip}%
\pgfsetrectcap%
\pgfsetroundjoin%
\pgfsetlinewidth{1.505625pt}%
\definecolor{currentstroke}{rgb}{0.000000,0.000000,0.000000}%
\pgfsetstrokecolor{currentstroke}%
\pgfsetdash{}{0pt}%
\pgfpathmoveto{\pgfqpoint{0.624035in}{0.551944in}}%
\pgfpathlineto{\pgfqpoint{0.625818in}{0.578454in}}%
\pgfpathlineto{\pgfqpoint{0.635863in}{0.672977in}}%
\pgfpathlineto{\pgfqpoint{0.642954in}{0.717523in}}%
\pgfpathlineto{\pgfqpoint{0.652061in}{0.766011in}}%
\pgfpathlineto{\pgfqpoint{0.660082in}{0.797456in}}%
\pgfpathlineto{\pgfqpoint{0.680307in}{0.866246in}}%
\pgfpathlineto{\pgfqpoint{0.692854in}{0.901510in}}%
\pgfpathlineto{\pgfqpoint{0.692924in}{0.901789in}}%
\pgfpathlineto{\pgfqpoint{0.715377in}{0.955057in}}%
\pgfpathlineto{\pgfqpoint{0.723835in}{0.974645in}}%
\pgfpathlineto{\pgfqpoint{0.765613in}{1.048215in}}%
\pgfpathlineto{\pgfqpoint{0.817389in}{1.123306in}}%
\pgfpathlineto{\pgfqpoint{0.847119in}{1.160680in}}%
\pgfpathlineto{\pgfqpoint{0.879485in}{1.196068in}}%
\pgfpathlineto{\pgfqpoint{0.880837in}{1.197248in}}%
\pgfpathlineto{\pgfqpoint{0.881064in}{1.197465in}}%
\pgfpathlineto{\pgfqpoint{0.936850in}{1.250454in}}%
\pgfpathlineto{\pgfqpoint{0.977643in}{1.283607in}}%
\pgfpathlineto{\pgfqpoint{0.999563in}{1.300153in}}%
\pgfpathlineto{\pgfqpoint{1.092633in}{1.363292in}}%
\pgfpathlineto{\pgfqpoint{1.116703in}{1.376827in}}%
\pgfpathlineto{\pgfqpoint{1.141766in}{1.389989in}}%
\pgfpathlineto{\pgfqpoint{1.167416in}{1.404609in}}%
\pgfpathlineto{\pgfqpoint{1.246452in}{1.440215in}}%
\pgfpathlineto{\pgfqpoint{1.273478in}{1.452507in}}%
\pgfpathlineto{\pgfqpoint{1.299753in}{1.462689in}}%
\pgfpathlineto{\pgfqpoint{1.438524in}{1.510618in}}%
\pgfpathlineto{\pgfqpoint{1.465792in}{1.519403in}}%
\pgfpathlineto{\pgfqpoint{1.521602in}{1.533651in}}%
\pgfpathlineto{\pgfqpoint{1.576935in}{1.547000in}}%
\pgfpathlineto{\pgfqpoint{1.707928in}{1.571678in}}%
\pgfpathlineto{\pgfqpoint{1.802459in}{1.584933in}}%
\pgfpathlineto{\pgfqpoint{1.865414in}{1.591762in}}%
\pgfpathlineto{\pgfqpoint{1.936570in}{1.597629in}}%
\pgfpathlineto{\pgfqpoint{1.989793in}{1.600392in}}%
\pgfpathlineto{\pgfqpoint{2.033126in}{1.601944in}}%
\pgfpathlineto{\pgfqpoint{2.033126in}{1.601944in}}%
\pgfusepath{stroke}%
\end{pgfscope}%
\begin{pgfscope}%
\pgfsetrectcap%
\pgfsetmiterjoin%
\pgfsetlinewidth{0.803000pt}%
\definecolor{currentstroke}{rgb}{0.000000,0.000000,0.000000}%
\pgfsetstrokecolor{currentstroke}%
\pgfsetdash{}{0pt}%
\pgfpathmoveto{\pgfqpoint{0.553581in}{0.499444in}}%
\pgfpathlineto{\pgfqpoint{0.553581in}{1.654444in}}%
\pgfusepath{stroke}%
\end{pgfscope}%
\begin{pgfscope}%
\pgfsetrectcap%
\pgfsetmiterjoin%
\pgfsetlinewidth{0.803000pt}%
\definecolor{currentstroke}{rgb}{0.000000,0.000000,0.000000}%
\pgfsetstrokecolor{currentstroke}%
\pgfsetdash{}{0pt}%
\pgfpathmoveto{\pgfqpoint{2.103581in}{0.499444in}}%
\pgfpathlineto{\pgfqpoint{2.103581in}{1.654444in}}%
\pgfusepath{stroke}%
\end{pgfscope}%
\begin{pgfscope}%
\pgfsetrectcap%
\pgfsetmiterjoin%
\pgfsetlinewidth{0.803000pt}%
\definecolor{currentstroke}{rgb}{0.000000,0.000000,0.000000}%
\pgfsetstrokecolor{currentstroke}%
\pgfsetdash{}{0pt}%
\pgfpathmoveto{\pgfqpoint{0.553581in}{0.499444in}}%
\pgfpathlineto{\pgfqpoint{2.103581in}{0.499444in}}%
\pgfusepath{stroke}%
\end{pgfscope}%
\begin{pgfscope}%
\pgfsetrectcap%
\pgfsetmiterjoin%
\pgfsetlinewidth{0.803000pt}%
\definecolor{currentstroke}{rgb}{0.000000,0.000000,0.000000}%
\pgfsetstrokecolor{currentstroke}%
\pgfsetdash{}{0pt}%
\pgfpathmoveto{\pgfqpoint{0.553581in}{1.654444in}}%
\pgfpathlineto{\pgfqpoint{2.103581in}{1.654444in}}%
\pgfusepath{stroke}%
\end{pgfscope}%
\begin{pgfscope}%
\pgfsetbuttcap%
\pgfsetmiterjoin%
\definecolor{currentfill}{rgb}{1.000000,1.000000,1.000000}%
\pgfsetfillcolor{currentfill}%
\pgfsetfillopacity{0.800000}%
\pgfsetlinewidth{1.003750pt}%
\definecolor{currentstroke}{rgb}{0.800000,0.800000,0.800000}%
\pgfsetstrokecolor{currentstroke}%
\pgfsetstrokeopacity{0.800000}%
\pgfsetdash{}{0pt}%
\pgfpathmoveto{\pgfqpoint{0.832747in}{0.568889in}}%
\pgfpathlineto{\pgfqpoint{2.006358in}{0.568889in}}%
\pgfpathquadraticcurveto{\pgfqpoint{2.034136in}{0.568889in}}{\pgfqpoint{2.034136in}{0.596666in}}%
\pgfpathlineto{\pgfqpoint{2.034136in}{0.776388in}}%
\pgfpathquadraticcurveto{\pgfqpoint{2.034136in}{0.804166in}}{\pgfqpoint{2.006358in}{0.804166in}}%
\pgfpathlineto{\pgfqpoint{0.832747in}{0.804166in}}%
\pgfpathquadraticcurveto{\pgfqpoint{0.804970in}{0.804166in}}{\pgfqpoint{0.804970in}{0.776388in}}%
\pgfpathlineto{\pgfqpoint{0.804970in}{0.596666in}}%
\pgfpathquadraticcurveto{\pgfqpoint{0.804970in}{0.568889in}}{\pgfqpoint{0.832747in}{0.568889in}}%
\pgfpathlineto{\pgfqpoint{0.832747in}{0.568889in}}%
\pgfpathclose%
\pgfusepath{stroke,fill}%
\end{pgfscope}%
\begin{pgfscope}%
\pgfsetrectcap%
\pgfsetroundjoin%
\pgfsetlinewidth{1.505625pt}%
\definecolor{currentstroke}{rgb}{0.000000,0.000000,0.000000}%
\pgfsetstrokecolor{currentstroke}%
\pgfsetdash{}{0pt}%
\pgfpathmoveto{\pgfqpoint{0.860525in}{0.700000in}}%
\pgfpathlineto{\pgfqpoint{0.999414in}{0.700000in}}%
\pgfpathlineto{\pgfqpoint{1.138303in}{0.700000in}}%
\pgfusepath{stroke}%
\end{pgfscope}%
\begin{pgfscope}%
\definecolor{textcolor}{rgb}{0.000000,0.000000,0.000000}%
\pgfsetstrokecolor{textcolor}%
\pgfsetfillcolor{textcolor}%
\pgftext[x=1.249414in,y=0.651388in,left,base]{\color{textcolor}\rmfamily\fontsize{10.000000}{12.000000}\selectfont AUC=0.799}%
\end{pgfscope}%
\end{pgfpicture}%
\makeatother%
\endgroup%

	
\end{tabular}

\verb|BRFC_Hard_Tomek_0_alpha_balanced_v1_Test|

\noindent\begin{tabular}{@{\hspace{-6pt}}p{4.3in} @{\hspace{-6pt}}p{2.0in}}
	\vskip 0pt
	\hfil Raw Model Output
	
	%% Creator: Matplotlib, PGF backend
%%
%% To include the figure in your LaTeX document, write
%%   \input{<filename>.pgf}
%%
%% Make sure the required packages are loaded in your preamble
%%   \usepackage{pgf}
%%
%% Also ensure that all the required font packages are loaded; for instance,
%% the lmodern package is sometimes necessary when using math font.
%%   \usepackage{lmodern}
%%
%% Figures using additional raster images can only be included by \input if
%% they are in the same directory as the main LaTeX file. For loading figures
%% from other directories you can use the `import` package
%%   \usepackage{import}
%%
%% and then include the figures with
%%   \import{<path to file>}{<filename>.pgf}
%%
%% Matplotlib used the following preamble
%%   
%%   \usepackage{fontspec}
%%   \makeatletter\@ifpackageloaded{underscore}{}{\usepackage[strings]{underscore}}\makeatother
%%
\begingroup%
\makeatletter%
\begin{pgfpicture}%
\pgfpathrectangle{\pgfpointorigin}{\pgfqpoint{4.509306in}{1.754444in}}%
\pgfusepath{use as bounding box, clip}%
\begin{pgfscope}%
\pgfsetbuttcap%
\pgfsetmiterjoin%
\definecolor{currentfill}{rgb}{1.000000,1.000000,1.000000}%
\pgfsetfillcolor{currentfill}%
\pgfsetlinewidth{0.000000pt}%
\definecolor{currentstroke}{rgb}{1.000000,1.000000,1.000000}%
\pgfsetstrokecolor{currentstroke}%
\pgfsetdash{}{0pt}%
\pgfpathmoveto{\pgfqpoint{0.000000in}{0.000000in}}%
\pgfpathlineto{\pgfqpoint{4.509306in}{0.000000in}}%
\pgfpathlineto{\pgfqpoint{4.509306in}{1.754444in}}%
\pgfpathlineto{\pgfqpoint{0.000000in}{1.754444in}}%
\pgfpathlineto{\pgfqpoint{0.000000in}{0.000000in}}%
\pgfpathclose%
\pgfusepath{fill}%
\end{pgfscope}%
\begin{pgfscope}%
\pgfsetbuttcap%
\pgfsetmiterjoin%
\definecolor{currentfill}{rgb}{1.000000,1.000000,1.000000}%
\pgfsetfillcolor{currentfill}%
\pgfsetlinewidth{0.000000pt}%
\definecolor{currentstroke}{rgb}{0.000000,0.000000,0.000000}%
\pgfsetstrokecolor{currentstroke}%
\pgfsetstrokeopacity{0.000000}%
\pgfsetdash{}{0pt}%
\pgfpathmoveto{\pgfqpoint{0.445556in}{0.499444in}}%
\pgfpathlineto{\pgfqpoint{4.320556in}{0.499444in}}%
\pgfpathlineto{\pgfqpoint{4.320556in}{1.654444in}}%
\pgfpathlineto{\pgfqpoint{0.445556in}{1.654444in}}%
\pgfpathlineto{\pgfqpoint{0.445556in}{0.499444in}}%
\pgfpathclose%
\pgfusepath{fill}%
\end{pgfscope}%
\begin{pgfscope}%
\pgfpathrectangle{\pgfqpoint{0.445556in}{0.499444in}}{\pgfqpoint{3.875000in}{1.155000in}}%
\pgfusepath{clip}%
\pgfsetbuttcap%
\pgfsetmiterjoin%
\pgfsetlinewidth{1.003750pt}%
\definecolor{currentstroke}{rgb}{0.000000,0.000000,0.000000}%
\pgfsetstrokecolor{currentstroke}%
\pgfsetdash{}{0pt}%
\pgfpathmoveto{\pgfqpoint{0.435556in}{0.499444in}}%
\pgfpathlineto{\pgfqpoint{0.483922in}{0.499444in}}%
\pgfpathlineto{\pgfqpoint{0.483922in}{0.614967in}}%
\pgfpathlineto{\pgfqpoint{0.435556in}{0.614967in}}%
\pgfusepath{stroke}%
\end{pgfscope}%
\begin{pgfscope}%
\pgfpathrectangle{\pgfqpoint{0.445556in}{0.499444in}}{\pgfqpoint{3.875000in}{1.155000in}}%
\pgfusepath{clip}%
\pgfsetbuttcap%
\pgfsetmiterjoin%
\pgfsetlinewidth{1.003750pt}%
\definecolor{currentstroke}{rgb}{0.000000,0.000000,0.000000}%
\pgfsetstrokecolor{currentstroke}%
\pgfsetdash{}{0pt}%
\pgfpathmoveto{\pgfqpoint{0.576001in}{0.499444in}}%
\pgfpathlineto{\pgfqpoint{0.637387in}{0.499444in}}%
\pgfpathlineto{\pgfqpoint{0.637387in}{0.814006in}}%
\pgfpathlineto{\pgfqpoint{0.576001in}{0.814006in}}%
\pgfpathlineto{\pgfqpoint{0.576001in}{0.499444in}}%
\pgfpathclose%
\pgfusepath{stroke}%
\end{pgfscope}%
\begin{pgfscope}%
\pgfpathrectangle{\pgfqpoint{0.445556in}{0.499444in}}{\pgfqpoint{3.875000in}{1.155000in}}%
\pgfusepath{clip}%
\pgfsetbuttcap%
\pgfsetmiterjoin%
\pgfsetlinewidth{1.003750pt}%
\definecolor{currentstroke}{rgb}{0.000000,0.000000,0.000000}%
\pgfsetstrokecolor{currentstroke}%
\pgfsetdash{}{0pt}%
\pgfpathmoveto{\pgfqpoint{0.729467in}{0.499444in}}%
\pgfpathlineto{\pgfqpoint{0.790853in}{0.499444in}}%
\pgfpathlineto{\pgfqpoint{0.790853in}{1.060386in}}%
\pgfpathlineto{\pgfqpoint{0.729467in}{1.060386in}}%
\pgfpathlineto{\pgfqpoint{0.729467in}{0.499444in}}%
\pgfpathclose%
\pgfusepath{stroke}%
\end{pgfscope}%
\begin{pgfscope}%
\pgfpathrectangle{\pgfqpoint{0.445556in}{0.499444in}}{\pgfqpoint{3.875000in}{1.155000in}}%
\pgfusepath{clip}%
\pgfsetbuttcap%
\pgfsetmiterjoin%
\pgfsetlinewidth{1.003750pt}%
\definecolor{currentstroke}{rgb}{0.000000,0.000000,0.000000}%
\pgfsetstrokecolor{currentstroke}%
\pgfsetdash{}{0pt}%
\pgfpathmoveto{\pgfqpoint{0.882932in}{0.499444in}}%
\pgfpathlineto{\pgfqpoint{0.944318in}{0.499444in}}%
\pgfpathlineto{\pgfqpoint{0.944318in}{1.257192in}}%
\pgfpathlineto{\pgfqpoint{0.882932in}{1.257192in}}%
\pgfpathlineto{\pgfqpoint{0.882932in}{0.499444in}}%
\pgfpathclose%
\pgfusepath{stroke}%
\end{pgfscope}%
\begin{pgfscope}%
\pgfpathrectangle{\pgfqpoint{0.445556in}{0.499444in}}{\pgfqpoint{3.875000in}{1.155000in}}%
\pgfusepath{clip}%
\pgfsetbuttcap%
\pgfsetmiterjoin%
\pgfsetlinewidth{1.003750pt}%
\definecolor{currentstroke}{rgb}{0.000000,0.000000,0.000000}%
\pgfsetstrokecolor{currentstroke}%
\pgfsetdash{}{0pt}%
\pgfpathmoveto{\pgfqpoint{1.036397in}{0.499444in}}%
\pgfpathlineto{\pgfqpoint{1.097783in}{0.499444in}}%
\pgfpathlineto{\pgfqpoint{1.097783in}{1.418046in}}%
\pgfpathlineto{\pgfqpoint{1.036397in}{1.418046in}}%
\pgfpathlineto{\pgfqpoint{1.036397in}{0.499444in}}%
\pgfpathclose%
\pgfusepath{stroke}%
\end{pgfscope}%
\begin{pgfscope}%
\pgfpathrectangle{\pgfqpoint{0.445556in}{0.499444in}}{\pgfqpoint{3.875000in}{1.155000in}}%
\pgfusepath{clip}%
\pgfsetbuttcap%
\pgfsetmiterjoin%
\pgfsetlinewidth{1.003750pt}%
\definecolor{currentstroke}{rgb}{0.000000,0.000000,0.000000}%
\pgfsetstrokecolor{currentstroke}%
\pgfsetdash{}{0pt}%
\pgfpathmoveto{\pgfqpoint{1.189863in}{0.499444in}}%
\pgfpathlineto{\pgfqpoint{1.251249in}{0.499444in}}%
\pgfpathlineto{\pgfqpoint{1.251249in}{1.518831in}}%
\pgfpathlineto{\pgfqpoint{1.189863in}{1.518831in}}%
\pgfpathlineto{\pgfqpoint{1.189863in}{0.499444in}}%
\pgfpathclose%
\pgfusepath{stroke}%
\end{pgfscope}%
\begin{pgfscope}%
\pgfpathrectangle{\pgfqpoint{0.445556in}{0.499444in}}{\pgfqpoint{3.875000in}{1.155000in}}%
\pgfusepath{clip}%
\pgfsetbuttcap%
\pgfsetmiterjoin%
\pgfsetlinewidth{1.003750pt}%
\definecolor{currentstroke}{rgb}{0.000000,0.000000,0.000000}%
\pgfsetstrokecolor{currentstroke}%
\pgfsetdash{}{0pt}%
\pgfpathmoveto{\pgfqpoint{1.343328in}{0.499444in}}%
\pgfpathlineto{\pgfqpoint{1.404714in}{0.499444in}}%
\pgfpathlineto{\pgfqpoint{1.404714in}{1.587832in}}%
\pgfpathlineto{\pgfqpoint{1.343328in}{1.587832in}}%
\pgfpathlineto{\pgfqpoint{1.343328in}{0.499444in}}%
\pgfpathclose%
\pgfusepath{stroke}%
\end{pgfscope}%
\begin{pgfscope}%
\pgfpathrectangle{\pgfqpoint{0.445556in}{0.499444in}}{\pgfqpoint{3.875000in}{1.155000in}}%
\pgfusepath{clip}%
\pgfsetbuttcap%
\pgfsetmiterjoin%
\pgfsetlinewidth{1.003750pt}%
\definecolor{currentstroke}{rgb}{0.000000,0.000000,0.000000}%
\pgfsetstrokecolor{currentstroke}%
\pgfsetdash{}{0pt}%
\pgfpathmoveto{\pgfqpoint{1.496793in}{0.499444in}}%
\pgfpathlineto{\pgfqpoint{1.558179in}{0.499444in}}%
\pgfpathlineto{\pgfqpoint{1.558179in}{1.599444in}}%
\pgfpathlineto{\pgfqpoint{1.496793in}{1.599444in}}%
\pgfpathlineto{\pgfqpoint{1.496793in}{0.499444in}}%
\pgfpathclose%
\pgfusepath{stroke}%
\end{pgfscope}%
\begin{pgfscope}%
\pgfpathrectangle{\pgfqpoint{0.445556in}{0.499444in}}{\pgfqpoint{3.875000in}{1.155000in}}%
\pgfusepath{clip}%
\pgfsetbuttcap%
\pgfsetmiterjoin%
\pgfsetlinewidth{1.003750pt}%
\definecolor{currentstroke}{rgb}{0.000000,0.000000,0.000000}%
\pgfsetstrokecolor{currentstroke}%
\pgfsetdash{}{0pt}%
\pgfpathmoveto{\pgfqpoint{1.650259in}{0.499444in}}%
\pgfpathlineto{\pgfqpoint{1.711645in}{0.499444in}}%
\pgfpathlineto{\pgfqpoint{1.711645in}{1.582324in}}%
\pgfpathlineto{\pgfqpoint{1.650259in}{1.582324in}}%
\pgfpathlineto{\pgfqpoint{1.650259in}{0.499444in}}%
\pgfpathclose%
\pgfusepath{stroke}%
\end{pgfscope}%
\begin{pgfscope}%
\pgfpathrectangle{\pgfqpoint{0.445556in}{0.499444in}}{\pgfqpoint{3.875000in}{1.155000in}}%
\pgfusepath{clip}%
\pgfsetbuttcap%
\pgfsetmiterjoin%
\pgfsetlinewidth{1.003750pt}%
\definecolor{currentstroke}{rgb}{0.000000,0.000000,0.000000}%
\pgfsetstrokecolor{currentstroke}%
\pgfsetdash{}{0pt}%
\pgfpathmoveto{\pgfqpoint{1.803724in}{0.499444in}}%
\pgfpathlineto{\pgfqpoint{1.865110in}{0.499444in}}%
\pgfpathlineto{\pgfqpoint{1.865110in}{1.524860in}}%
\pgfpathlineto{\pgfqpoint{1.803724in}{1.524860in}}%
\pgfpathlineto{\pgfqpoint{1.803724in}{0.499444in}}%
\pgfpathclose%
\pgfusepath{stroke}%
\end{pgfscope}%
\begin{pgfscope}%
\pgfpathrectangle{\pgfqpoint{0.445556in}{0.499444in}}{\pgfqpoint{3.875000in}{1.155000in}}%
\pgfusepath{clip}%
\pgfsetbuttcap%
\pgfsetmiterjoin%
\pgfsetlinewidth{1.003750pt}%
\definecolor{currentstroke}{rgb}{0.000000,0.000000,0.000000}%
\pgfsetstrokecolor{currentstroke}%
\pgfsetdash{}{0pt}%
\pgfpathmoveto{\pgfqpoint{1.957189in}{0.499444in}}%
\pgfpathlineto{\pgfqpoint{2.018575in}{0.499444in}}%
\pgfpathlineto{\pgfqpoint{2.018575in}{1.467173in}}%
\pgfpathlineto{\pgfqpoint{1.957189in}{1.467173in}}%
\pgfpathlineto{\pgfqpoint{1.957189in}{0.499444in}}%
\pgfpathclose%
\pgfusepath{stroke}%
\end{pgfscope}%
\begin{pgfscope}%
\pgfpathrectangle{\pgfqpoint{0.445556in}{0.499444in}}{\pgfqpoint{3.875000in}{1.155000in}}%
\pgfusepath{clip}%
\pgfsetbuttcap%
\pgfsetmiterjoin%
\pgfsetlinewidth{1.003750pt}%
\definecolor{currentstroke}{rgb}{0.000000,0.000000,0.000000}%
\pgfsetstrokecolor{currentstroke}%
\pgfsetdash{}{0pt}%
\pgfpathmoveto{\pgfqpoint{2.110655in}{0.499444in}}%
\pgfpathlineto{\pgfqpoint{2.172041in}{0.499444in}}%
\pgfpathlineto{\pgfqpoint{2.172041in}{1.355595in}}%
\pgfpathlineto{\pgfqpoint{2.110655in}{1.355595in}}%
\pgfpathlineto{\pgfqpoint{2.110655in}{0.499444in}}%
\pgfpathclose%
\pgfusepath{stroke}%
\end{pgfscope}%
\begin{pgfscope}%
\pgfpathrectangle{\pgfqpoint{0.445556in}{0.499444in}}{\pgfqpoint{3.875000in}{1.155000in}}%
\pgfusepath{clip}%
\pgfsetbuttcap%
\pgfsetmiterjoin%
\pgfsetlinewidth{1.003750pt}%
\definecolor{currentstroke}{rgb}{0.000000,0.000000,0.000000}%
\pgfsetstrokecolor{currentstroke}%
\pgfsetdash{}{0pt}%
\pgfpathmoveto{\pgfqpoint{2.264120in}{0.499444in}}%
\pgfpathlineto{\pgfqpoint{2.325506in}{0.499444in}}%
\pgfpathlineto{\pgfqpoint{2.325506in}{1.248186in}}%
\pgfpathlineto{\pgfqpoint{2.264120in}{1.248186in}}%
\pgfpathlineto{\pgfqpoint{2.264120in}{0.499444in}}%
\pgfpathclose%
\pgfusepath{stroke}%
\end{pgfscope}%
\begin{pgfscope}%
\pgfpathrectangle{\pgfqpoint{0.445556in}{0.499444in}}{\pgfqpoint{3.875000in}{1.155000in}}%
\pgfusepath{clip}%
\pgfsetbuttcap%
\pgfsetmiterjoin%
\pgfsetlinewidth{1.003750pt}%
\definecolor{currentstroke}{rgb}{0.000000,0.000000,0.000000}%
\pgfsetstrokecolor{currentstroke}%
\pgfsetdash{}{0pt}%
\pgfpathmoveto{\pgfqpoint{2.417585in}{0.499444in}}%
\pgfpathlineto{\pgfqpoint{2.478972in}{0.499444in}}%
\pgfpathlineto{\pgfqpoint{2.478972in}{1.146284in}}%
\pgfpathlineto{\pgfqpoint{2.417585in}{1.146284in}}%
\pgfpathlineto{\pgfqpoint{2.417585in}{0.499444in}}%
\pgfpathclose%
\pgfusepath{stroke}%
\end{pgfscope}%
\begin{pgfscope}%
\pgfpathrectangle{\pgfqpoint{0.445556in}{0.499444in}}{\pgfqpoint{3.875000in}{1.155000in}}%
\pgfusepath{clip}%
\pgfsetbuttcap%
\pgfsetmiterjoin%
\pgfsetlinewidth{1.003750pt}%
\definecolor{currentstroke}{rgb}{0.000000,0.000000,0.000000}%
\pgfsetstrokecolor{currentstroke}%
\pgfsetdash{}{0pt}%
\pgfpathmoveto{\pgfqpoint{2.571051in}{0.499444in}}%
\pgfpathlineto{\pgfqpoint{2.632437in}{0.499444in}}%
\pgfpathlineto{\pgfqpoint{2.632437in}{1.051528in}}%
\pgfpathlineto{\pgfqpoint{2.571051in}{1.051528in}}%
\pgfpathlineto{\pgfqpoint{2.571051in}{0.499444in}}%
\pgfpathclose%
\pgfusepath{stroke}%
\end{pgfscope}%
\begin{pgfscope}%
\pgfpathrectangle{\pgfqpoint{0.445556in}{0.499444in}}{\pgfqpoint{3.875000in}{1.155000in}}%
\pgfusepath{clip}%
\pgfsetbuttcap%
\pgfsetmiterjoin%
\pgfsetlinewidth{1.003750pt}%
\definecolor{currentstroke}{rgb}{0.000000,0.000000,0.000000}%
\pgfsetstrokecolor{currentstroke}%
\pgfsetdash{}{0pt}%
\pgfpathmoveto{\pgfqpoint{2.724516in}{0.499444in}}%
\pgfpathlineto{\pgfqpoint{2.785902in}{0.499444in}}%
\pgfpathlineto{\pgfqpoint{2.785902in}{0.940471in}}%
\pgfpathlineto{\pgfqpoint{2.724516in}{0.940471in}}%
\pgfpathlineto{\pgfqpoint{2.724516in}{0.499444in}}%
\pgfpathclose%
\pgfusepath{stroke}%
\end{pgfscope}%
\begin{pgfscope}%
\pgfpathrectangle{\pgfqpoint{0.445556in}{0.499444in}}{\pgfqpoint{3.875000in}{1.155000in}}%
\pgfusepath{clip}%
\pgfsetbuttcap%
\pgfsetmiterjoin%
\pgfsetlinewidth{1.003750pt}%
\definecolor{currentstroke}{rgb}{0.000000,0.000000,0.000000}%
\pgfsetstrokecolor{currentstroke}%
\pgfsetdash{}{0pt}%
\pgfpathmoveto{\pgfqpoint{2.877981in}{0.499444in}}%
\pgfpathlineto{\pgfqpoint{2.939368in}{0.499444in}}%
\pgfpathlineto{\pgfqpoint{2.939368in}{0.839314in}}%
\pgfpathlineto{\pgfqpoint{2.877981in}{0.839314in}}%
\pgfpathlineto{\pgfqpoint{2.877981in}{0.499444in}}%
\pgfpathclose%
\pgfusepath{stroke}%
\end{pgfscope}%
\begin{pgfscope}%
\pgfpathrectangle{\pgfqpoint{0.445556in}{0.499444in}}{\pgfqpoint{3.875000in}{1.155000in}}%
\pgfusepath{clip}%
\pgfsetbuttcap%
\pgfsetmiterjoin%
\pgfsetlinewidth{1.003750pt}%
\definecolor{currentstroke}{rgb}{0.000000,0.000000,0.000000}%
\pgfsetstrokecolor{currentstroke}%
\pgfsetdash{}{0pt}%
\pgfpathmoveto{\pgfqpoint{3.031447in}{0.499444in}}%
\pgfpathlineto{\pgfqpoint{3.092833in}{0.499444in}}%
\pgfpathlineto{\pgfqpoint{3.092833in}{0.764507in}}%
\pgfpathlineto{\pgfqpoint{3.031447in}{0.764507in}}%
\pgfpathlineto{\pgfqpoint{3.031447in}{0.499444in}}%
\pgfpathclose%
\pgfusepath{stroke}%
\end{pgfscope}%
\begin{pgfscope}%
\pgfpathrectangle{\pgfqpoint{0.445556in}{0.499444in}}{\pgfqpoint{3.875000in}{1.155000in}}%
\pgfusepath{clip}%
\pgfsetbuttcap%
\pgfsetmiterjoin%
\pgfsetlinewidth{1.003750pt}%
\definecolor{currentstroke}{rgb}{0.000000,0.000000,0.000000}%
\pgfsetstrokecolor{currentstroke}%
\pgfsetdash{}{0pt}%
\pgfpathmoveto{\pgfqpoint{3.184912in}{0.499444in}}%
\pgfpathlineto{\pgfqpoint{3.246298in}{0.499444in}}%
\pgfpathlineto{\pgfqpoint{3.246298in}{0.702949in}}%
\pgfpathlineto{\pgfqpoint{3.184912in}{0.702949in}}%
\pgfpathlineto{\pgfqpoint{3.184912in}{0.499444in}}%
\pgfpathclose%
\pgfusepath{stroke}%
\end{pgfscope}%
\begin{pgfscope}%
\pgfpathrectangle{\pgfqpoint{0.445556in}{0.499444in}}{\pgfqpoint{3.875000in}{1.155000in}}%
\pgfusepath{clip}%
\pgfsetbuttcap%
\pgfsetmiterjoin%
\pgfsetlinewidth{1.003750pt}%
\definecolor{currentstroke}{rgb}{0.000000,0.000000,0.000000}%
\pgfsetstrokecolor{currentstroke}%
\pgfsetdash{}{0pt}%
\pgfpathmoveto{\pgfqpoint{3.338377in}{0.499444in}}%
\pgfpathlineto{\pgfqpoint{3.399764in}{0.499444in}}%
\pgfpathlineto{\pgfqpoint{3.399764in}{0.649133in}}%
\pgfpathlineto{\pgfqpoint{3.338377in}{0.649133in}}%
\pgfpathlineto{\pgfqpoint{3.338377in}{0.499444in}}%
\pgfpathclose%
\pgfusepath{stroke}%
\end{pgfscope}%
\begin{pgfscope}%
\pgfpathrectangle{\pgfqpoint{0.445556in}{0.499444in}}{\pgfqpoint{3.875000in}{1.155000in}}%
\pgfusepath{clip}%
\pgfsetbuttcap%
\pgfsetmiterjoin%
\pgfsetlinewidth{1.003750pt}%
\definecolor{currentstroke}{rgb}{0.000000,0.000000,0.000000}%
\pgfsetstrokecolor{currentstroke}%
\pgfsetdash{}{0pt}%
\pgfpathmoveto{\pgfqpoint{3.491843in}{0.499444in}}%
\pgfpathlineto{\pgfqpoint{3.553229in}{0.499444in}}%
\pgfpathlineto{\pgfqpoint{3.553229in}{0.604770in}}%
\pgfpathlineto{\pgfqpoint{3.491843in}{0.604770in}}%
\pgfpathlineto{\pgfqpoint{3.491843in}{0.499444in}}%
\pgfpathclose%
\pgfusepath{stroke}%
\end{pgfscope}%
\begin{pgfscope}%
\pgfpathrectangle{\pgfqpoint{0.445556in}{0.499444in}}{\pgfqpoint{3.875000in}{1.155000in}}%
\pgfusepath{clip}%
\pgfsetbuttcap%
\pgfsetmiterjoin%
\pgfsetlinewidth{1.003750pt}%
\definecolor{currentstroke}{rgb}{0.000000,0.000000,0.000000}%
\pgfsetstrokecolor{currentstroke}%
\pgfsetdash{}{0pt}%
\pgfpathmoveto{\pgfqpoint{3.645308in}{0.499444in}}%
\pgfpathlineto{\pgfqpoint{3.706694in}{0.499444in}}%
\pgfpathlineto{\pgfqpoint{3.706694in}{0.572390in}}%
\pgfpathlineto{\pgfqpoint{3.645308in}{0.572390in}}%
\pgfpathlineto{\pgfqpoint{3.645308in}{0.499444in}}%
\pgfpathclose%
\pgfusepath{stroke}%
\end{pgfscope}%
\begin{pgfscope}%
\pgfpathrectangle{\pgfqpoint{0.445556in}{0.499444in}}{\pgfqpoint{3.875000in}{1.155000in}}%
\pgfusepath{clip}%
\pgfsetbuttcap%
\pgfsetmiterjoin%
\pgfsetlinewidth{1.003750pt}%
\definecolor{currentstroke}{rgb}{0.000000,0.000000,0.000000}%
\pgfsetstrokecolor{currentstroke}%
\pgfsetdash{}{0pt}%
\pgfpathmoveto{\pgfqpoint{3.798774in}{0.499444in}}%
\pgfpathlineto{\pgfqpoint{3.860160in}{0.499444in}}%
\pgfpathlineto{\pgfqpoint{3.860160in}{0.544105in}}%
\pgfpathlineto{\pgfqpoint{3.798774in}{0.544105in}}%
\pgfpathlineto{\pgfqpoint{3.798774in}{0.499444in}}%
\pgfpathclose%
\pgfusepath{stroke}%
\end{pgfscope}%
\begin{pgfscope}%
\pgfpathrectangle{\pgfqpoint{0.445556in}{0.499444in}}{\pgfqpoint{3.875000in}{1.155000in}}%
\pgfusepath{clip}%
\pgfsetbuttcap%
\pgfsetmiterjoin%
\pgfsetlinewidth{1.003750pt}%
\definecolor{currentstroke}{rgb}{0.000000,0.000000,0.000000}%
\pgfsetstrokecolor{currentstroke}%
\pgfsetdash{}{0pt}%
\pgfpathmoveto{\pgfqpoint{3.952239in}{0.499444in}}%
\pgfpathlineto{\pgfqpoint{4.013625in}{0.499444in}}%
\pgfpathlineto{\pgfqpoint{4.013625in}{0.526241in}}%
\pgfpathlineto{\pgfqpoint{3.952239in}{0.526241in}}%
\pgfpathlineto{\pgfqpoint{3.952239in}{0.499444in}}%
\pgfpathclose%
\pgfusepath{stroke}%
\end{pgfscope}%
\begin{pgfscope}%
\pgfpathrectangle{\pgfqpoint{0.445556in}{0.499444in}}{\pgfqpoint{3.875000in}{1.155000in}}%
\pgfusepath{clip}%
\pgfsetbuttcap%
\pgfsetmiterjoin%
\pgfsetlinewidth{1.003750pt}%
\definecolor{currentstroke}{rgb}{0.000000,0.000000,0.000000}%
\pgfsetstrokecolor{currentstroke}%
\pgfsetdash{}{0pt}%
\pgfpathmoveto{\pgfqpoint{4.105704in}{0.499444in}}%
\pgfpathlineto{\pgfqpoint{4.167090in}{0.499444in}}%
\pgfpathlineto{\pgfqpoint{4.167090in}{0.503910in}}%
\pgfpathlineto{\pgfqpoint{4.105704in}{0.503910in}}%
\pgfpathlineto{\pgfqpoint{4.105704in}{0.499444in}}%
\pgfpathclose%
\pgfusepath{stroke}%
\end{pgfscope}%
\begin{pgfscope}%
\pgfpathrectangle{\pgfqpoint{0.445556in}{0.499444in}}{\pgfqpoint{3.875000in}{1.155000in}}%
\pgfusepath{clip}%
\pgfsetbuttcap%
\pgfsetmiterjoin%
\definecolor{currentfill}{rgb}{0.000000,0.000000,0.000000}%
\pgfsetfillcolor{currentfill}%
\pgfsetlinewidth{0.000000pt}%
\definecolor{currentstroke}{rgb}{0.000000,0.000000,0.000000}%
\pgfsetstrokecolor{currentstroke}%
\pgfsetstrokeopacity{0.000000}%
\pgfsetdash{}{0pt}%
\pgfpathmoveto{\pgfqpoint{0.483922in}{0.499444in}}%
\pgfpathlineto{\pgfqpoint{0.545308in}{0.499444in}}%
\pgfpathlineto{\pgfqpoint{0.545308in}{0.500337in}}%
\pgfpathlineto{\pgfqpoint{0.483922in}{0.500337in}}%
\pgfpathlineto{\pgfqpoint{0.483922in}{0.499444in}}%
\pgfpathclose%
\pgfusepath{fill}%
\end{pgfscope}%
\begin{pgfscope}%
\pgfpathrectangle{\pgfqpoint{0.445556in}{0.499444in}}{\pgfqpoint{3.875000in}{1.155000in}}%
\pgfusepath{clip}%
\pgfsetbuttcap%
\pgfsetmiterjoin%
\definecolor{currentfill}{rgb}{0.000000,0.000000,0.000000}%
\pgfsetfillcolor{currentfill}%
\pgfsetlinewidth{0.000000pt}%
\definecolor{currentstroke}{rgb}{0.000000,0.000000,0.000000}%
\pgfsetstrokecolor{currentstroke}%
\pgfsetstrokeopacity{0.000000}%
\pgfsetdash{}{0pt}%
\pgfpathmoveto{\pgfqpoint{0.637387in}{0.499444in}}%
\pgfpathlineto{\pgfqpoint{0.698774in}{0.499444in}}%
\pgfpathlineto{\pgfqpoint{0.698774in}{0.502645in}}%
\pgfpathlineto{\pgfqpoint{0.637387in}{0.502645in}}%
\pgfpathlineto{\pgfqpoint{0.637387in}{0.499444in}}%
\pgfpathclose%
\pgfusepath{fill}%
\end{pgfscope}%
\begin{pgfscope}%
\pgfpathrectangle{\pgfqpoint{0.445556in}{0.499444in}}{\pgfqpoint{3.875000in}{1.155000in}}%
\pgfusepath{clip}%
\pgfsetbuttcap%
\pgfsetmiterjoin%
\definecolor{currentfill}{rgb}{0.000000,0.000000,0.000000}%
\pgfsetfillcolor{currentfill}%
\pgfsetlinewidth{0.000000pt}%
\definecolor{currentstroke}{rgb}{0.000000,0.000000,0.000000}%
\pgfsetstrokecolor{currentstroke}%
\pgfsetstrokeopacity{0.000000}%
\pgfsetdash{}{0pt}%
\pgfpathmoveto{\pgfqpoint{0.790853in}{0.499444in}}%
\pgfpathlineto{\pgfqpoint{0.852239in}{0.499444in}}%
\pgfpathlineto{\pgfqpoint{0.852239in}{0.506962in}}%
\pgfpathlineto{\pgfqpoint{0.790853in}{0.506962in}}%
\pgfpathlineto{\pgfqpoint{0.790853in}{0.499444in}}%
\pgfpathclose%
\pgfusepath{fill}%
\end{pgfscope}%
\begin{pgfscope}%
\pgfpathrectangle{\pgfqpoint{0.445556in}{0.499444in}}{\pgfqpoint{3.875000in}{1.155000in}}%
\pgfusepath{clip}%
\pgfsetbuttcap%
\pgfsetmiterjoin%
\definecolor{currentfill}{rgb}{0.000000,0.000000,0.000000}%
\pgfsetfillcolor{currentfill}%
\pgfsetlinewidth{0.000000pt}%
\definecolor{currentstroke}{rgb}{0.000000,0.000000,0.000000}%
\pgfsetstrokecolor{currentstroke}%
\pgfsetstrokeopacity{0.000000}%
\pgfsetdash{}{0pt}%
\pgfpathmoveto{\pgfqpoint{0.944318in}{0.499444in}}%
\pgfpathlineto{\pgfqpoint{1.005704in}{0.499444in}}%
\pgfpathlineto{\pgfqpoint{1.005704in}{0.515597in}}%
\pgfpathlineto{\pgfqpoint{0.944318in}{0.515597in}}%
\pgfpathlineto{\pgfqpoint{0.944318in}{0.499444in}}%
\pgfpathclose%
\pgfusepath{fill}%
\end{pgfscope}%
\begin{pgfscope}%
\pgfpathrectangle{\pgfqpoint{0.445556in}{0.499444in}}{\pgfqpoint{3.875000in}{1.155000in}}%
\pgfusepath{clip}%
\pgfsetbuttcap%
\pgfsetmiterjoin%
\definecolor{currentfill}{rgb}{0.000000,0.000000,0.000000}%
\pgfsetfillcolor{currentfill}%
\pgfsetlinewidth{0.000000pt}%
\definecolor{currentstroke}{rgb}{0.000000,0.000000,0.000000}%
\pgfsetstrokecolor{currentstroke}%
\pgfsetstrokeopacity{0.000000}%
\pgfsetdash{}{0pt}%
\pgfpathmoveto{\pgfqpoint{1.097783in}{0.499444in}}%
\pgfpathlineto{\pgfqpoint{1.159170in}{0.499444in}}%
\pgfpathlineto{\pgfqpoint{1.159170in}{0.527283in}}%
\pgfpathlineto{\pgfqpoint{1.097783in}{0.527283in}}%
\pgfpathlineto{\pgfqpoint{1.097783in}{0.499444in}}%
\pgfpathclose%
\pgfusepath{fill}%
\end{pgfscope}%
\begin{pgfscope}%
\pgfpathrectangle{\pgfqpoint{0.445556in}{0.499444in}}{\pgfqpoint{3.875000in}{1.155000in}}%
\pgfusepath{clip}%
\pgfsetbuttcap%
\pgfsetmiterjoin%
\definecolor{currentfill}{rgb}{0.000000,0.000000,0.000000}%
\pgfsetfillcolor{currentfill}%
\pgfsetlinewidth{0.000000pt}%
\definecolor{currentstroke}{rgb}{0.000000,0.000000,0.000000}%
\pgfsetstrokecolor{currentstroke}%
\pgfsetstrokeopacity{0.000000}%
\pgfsetdash{}{0pt}%
\pgfpathmoveto{\pgfqpoint{1.251249in}{0.499444in}}%
\pgfpathlineto{\pgfqpoint{1.312635in}{0.499444in}}%
\pgfpathlineto{\pgfqpoint{1.312635in}{0.540607in}}%
\pgfpathlineto{\pgfqpoint{1.251249in}{0.540607in}}%
\pgfpathlineto{\pgfqpoint{1.251249in}{0.499444in}}%
\pgfpathclose%
\pgfusepath{fill}%
\end{pgfscope}%
\begin{pgfscope}%
\pgfpathrectangle{\pgfqpoint{0.445556in}{0.499444in}}{\pgfqpoint{3.875000in}{1.155000in}}%
\pgfusepath{clip}%
\pgfsetbuttcap%
\pgfsetmiterjoin%
\definecolor{currentfill}{rgb}{0.000000,0.000000,0.000000}%
\pgfsetfillcolor{currentfill}%
\pgfsetlinewidth{0.000000pt}%
\definecolor{currentstroke}{rgb}{0.000000,0.000000,0.000000}%
\pgfsetstrokecolor{currentstroke}%
\pgfsetstrokeopacity{0.000000}%
\pgfsetdash{}{0pt}%
\pgfpathmoveto{\pgfqpoint{1.404714in}{0.499444in}}%
\pgfpathlineto{\pgfqpoint{1.466100in}{0.499444in}}%
\pgfpathlineto{\pgfqpoint{1.466100in}{0.556908in}}%
\pgfpathlineto{\pgfqpoint{1.404714in}{0.556908in}}%
\pgfpathlineto{\pgfqpoint{1.404714in}{0.499444in}}%
\pgfpathclose%
\pgfusepath{fill}%
\end{pgfscope}%
\begin{pgfscope}%
\pgfpathrectangle{\pgfqpoint{0.445556in}{0.499444in}}{\pgfqpoint{3.875000in}{1.155000in}}%
\pgfusepath{clip}%
\pgfsetbuttcap%
\pgfsetmiterjoin%
\definecolor{currentfill}{rgb}{0.000000,0.000000,0.000000}%
\pgfsetfillcolor{currentfill}%
\pgfsetlinewidth{0.000000pt}%
\definecolor{currentstroke}{rgb}{0.000000,0.000000,0.000000}%
\pgfsetstrokecolor{currentstroke}%
\pgfsetstrokeopacity{0.000000}%
\pgfsetdash{}{0pt}%
\pgfpathmoveto{\pgfqpoint{1.558179in}{0.499444in}}%
\pgfpathlineto{\pgfqpoint{1.619566in}{0.499444in}}%
\pgfpathlineto{\pgfqpoint{1.619566in}{0.574847in}}%
\pgfpathlineto{\pgfqpoint{1.558179in}{0.574847in}}%
\pgfpathlineto{\pgfqpoint{1.558179in}{0.499444in}}%
\pgfpathclose%
\pgfusepath{fill}%
\end{pgfscope}%
\begin{pgfscope}%
\pgfpathrectangle{\pgfqpoint{0.445556in}{0.499444in}}{\pgfqpoint{3.875000in}{1.155000in}}%
\pgfusepath{clip}%
\pgfsetbuttcap%
\pgfsetmiterjoin%
\definecolor{currentfill}{rgb}{0.000000,0.000000,0.000000}%
\pgfsetfillcolor{currentfill}%
\pgfsetlinewidth{0.000000pt}%
\definecolor{currentstroke}{rgb}{0.000000,0.000000,0.000000}%
\pgfsetstrokecolor{currentstroke}%
\pgfsetstrokeopacity{0.000000}%
\pgfsetdash{}{0pt}%
\pgfpathmoveto{\pgfqpoint{1.711645in}{0.499444in}}%
\pgfpathlineto{\pgfqpoint{1.773031in}{0.499444in}}%
\pgfpathlineto{\pgfqpoint{1.773031in}{0.593381in}}%
\pgfpathlineto{\pgfqpoint{1.711645in}{0.593381in}}%
\pgfpathlineto{\pgfqpoint{1.711645in}{0.499444in}}%
\pgfpathclose%
\pgfusepath{fill}%
\end{pgfscope}%
\begin{pgfscope}%
\pgfpathrectangle{\pgfqpoint{0.445556in}{0.499444in}}{\pgfqpoint{3.875000in}{1.155000in}}%
\pgfusepath{clip}%
\pgfsetbuttcap%
\pgfsetmiterjoin%
\definecolor{currentfill}{rgb}{0.000000,0.000000,0.000000}%
\pgfsetfillcolor{currentfill}%
\pgfsetlinewidth{0.000000pt}%
\definecolor{currentstroke}{rgb}{0.000000,0.000000,0.000000}%
\pgfsetstrokecolor{currentstroke}%
\pgfsetstrokeopacity{0.000000}%
\pgfsetdash{}{0pt}%
\pgfpathmoveto{\pgfqpoint{1.865110in}{0.499444in}}%
\pgfpathlineto{\pgfqpoint{1.926496in}{0.499444in}}%
\pgfpathlineto{\pgfqpoint{1.926496in}{0.618689in}}%
\pgfpathlineto{\pgfqpoint{1.865110in}{0.618689in}}%
\pgfpathlineto{\pgfqpoint{1.865110in}{0.499444in}}%
\pgfpathclose%
\pgfusepath{fill}%
\end{pgfscope}%
\begin{pgfscope}%
\pgfpathrectangle{\pgfqpoint{0.445556in}{0.499444in}}{\pgfqpoint{3.875000in}{1.155000in}}%
\pgfusepath{clip}%
\pgfsetbuttcap%
\pgfsetmiterjoin%
\definecolor{currentfill}{rgb}{0.000000,0.000000,0.000000}%
\pgfsetfillcolor{currentfill}%
\pgfsetlinewidth{0.000000pt}%
\definecolor{currentstroke}{rgb}{0.000000,0.000000,0.000000}%
\pgfsetstrokecolor{currentstroke}%
\pgfsetstrokeopacity{0.000000}%
\pgfsetdash{}{0pt}%
\pgfpathmoveto{\pgfqpoint{2.018575in}{0.499444in}}%
\pgfpathlineto{\pgfqpoint{2.079962in}{0.499444in}}%
\pgfpathlineto{\pgfqpoint{2.079962in}{0.633650in}}%
\pgfpathlineto{\pgfqpoint{2.018575in}{0.633650in}}%
\pgfpathlineto{\pgfqpoint{2.018575in}{0.499444in}}%
\pgfpathclose%
\pgfusepath{fill}%
\end{pgfscope}%
\begin{pgfscope}%
\pgfpathrectangle{\pgfqpoint{0.445556in}{0.499444in}}{\pgfqpoint{3.875000in}{1.155000in}}%
\pgfusepath{clip}%
\pgfsetbuttcap%
\pgfsetmiterjoin%
\definecolor{currentfill}{rgb}{0.000000,0.000000,0.000000}%
\pgfsetfillcolor{currentfill}%
\pgfsetlinewidth{0.000000pt}%
\definecolor{currentstroke}{rgb}{0.000000,0.000000,0.000000}%
\pgfsetstrokecolor{currentstroke}%
\pgfsetstrokeopacity{0.000000}%
\pgfsetdash{}{0pt}%
\pgfpathmoveto{\pgfqpoint{2.172041in}{0.499444in}}%
\pgfpathlineto{\pgfqpoint{2.233427in}{0.499444in}}%
\pgfpathlineto{\pgfqpoint{2.233427in}{0.644667in}}%
\pgfpathlineto{\pgfqpoint{2.172041in}{0.644667in}}%
\pgfpathlineto{\pgfqpoint{2.172041in}{0.499444in}}%
\pgfpathclose%
\pgfusepath{fill}%
\end{pgfscope}%
\begin{pgfscope}%
\pgfpathrectangle{\pgfqpoint{0.445556in}{0.499444in}}{\pgfqpoint{3.875000in}{1.155000in}}%
\pgfusepath{clip}%
\pgfsetbuttcap%
\pgfsetmiterjoin%
\definecolor{currentfill}{rgb}{0.000000,0.000000,0.000000}%
\pgfsetfillcolor{currentfill}%
\pgfsetlinewidth{0.000000pt}%
\definecolor{currentstroke}{rgb}{0.000000,0.000000,0.000000}%
\pgfsetstrokecolor{currentstroke}%
\pgfsetstrokeopacity{0.000000}%
\pgfsetdash{}{0pt}%
\pgfpathmoveto{\pgfqpoint{2.325506in}{0.499444in}}%
\pgfpathlineto{\pgfqpoint{2.386892in}{0.499444in}}%
\pgfpathlineto{\pgfqpoint{2.386892in}{0.652631in}}%
\pgfpathlineto{\pgfqpoint{2.325506in}{0.652631in}}%
\pgfpathlineto{\pgfqpoint{2.325506in}{0.499444in}}%
\pgfpathclose%
\pgfusepath{fill}%
\end{pgfscope}%
\begin{pgfscope}%
\pgfpathrectangle{\pgfqpoint{0.445556in}{0.499444in}}{\pgfqpoint{3.875000in}{1.155000in}}%
\pgfusepath{clip}%
\pgfsetbuttcap%
\pgfsetmiterjoin%
\definecolor{currentfill}{rgb}{0.000000,0.000000,0.000000}%
\pgfsetfillcolor{currentfill}%
\pgfsetlinewidth{0.000000pt}%
\definecolor{currentstroke}{rgb}{0.000000,0.000000,0.000000}%
\pgfsetstrokecolor{currentstroke}%
\pgfsetstrokeopacity{0.000000}%
\pgfsetdash{}{0pt}%
\pgfpathmoveto{\pgfqpoint{2.478972in}{0.499444in}}%
\pgfpathlineto{\pgfqpoint{2.540358in}{0.499444in}}%
\pgfpathlineto{\pgfqpoint{2.540358in}{0.677046in}}%
\pgfpathlineto{\pgfqpoint{2.478972in}{0.677046in}}%
\pgfpathlineto{\pgfqpoint{2.478972in}{0.499444in}}%
\pgfpathclose%
\pgfusepath{fill}%
\end{pgfscope}%
\begin{pgfscope}%
\pgfpathrectangle{\pgfqpoint{0.445556in}{0.499444in}}{\pgfqpoint{3.875000in}{1.155000in}}%
\pgfusepath{clip}%
\pgfsetbuttcap%
\pgfsetmiterjoin%
\definecolor{currentfill}{rgb}{0.000000,0.000000,0.000000}%
\pgfsetfillcolor{currentfill}%
\pgfsetlinewidth{0.000000pt}%
\definecolor{currentstroke}{rgb}{0.000000,0.000000,0.000000}%
\pgfsetstrokecolor{currentstroke}%
\pgfsetstrokeopacity{0.000000}%
\pgfsetdash{}{0pt}%
\pgfpathmoveto{\pgfqpoint{2.632437in}{0.499444in}}%
\pgfpathlineto{\pgfqpoint{2.693823in}{0.499444in}}%
\pgfpathlineto{\pgfqpoint{2.693823in}{0.665062in}}%
\pgfpathlineto{\pgfqpoint{2.632437in}{0.665062in}}%
\pgfpathlineto{\pgfqpoint{2.632437in}{0.499444in}}%
\pgfpathclose%
\pgfusepath{fill}%
\end{pgfscope}%
\begin{pgfscope}%
\pgfpathrectangle{\pgfqpoint{0.445556in}{0.499444in}}{\pgfqpoint{3.875000in}{1.155000in}}%
\pgfusepath{clip}%
\pgfsetbuttcap%
\pgfsetmiterjoin%
\definecolor{currentfill}{rgb}{0.000000,0.000000,0.000000}%
\pgfsetfillcolor{currentfill}%
\pgfsetlinewidth{0.000000pt}%
\definecolor{currentstroke}{rgb}{0.000000,0.000000,0.000000}%
\pgfsetstrokecolor{currentstroke}%
\pgfsetstrokeopacity{0.000000}%
\pgfsetdash{}{0pt}%
\pgfpathmoveto{\pgfqpoint{2.785902in}{0.499444in}}%
\pgfpathlineto{\pgfqpoint{2.847288in}{0.499444in}}%
\pgfpathlineto{\pgfqpoint{2.847288in}{0.670942in}}%
\pgfpathlineto{\pgfqpoint{2.785902in}{0.670942in}}%
\pgfpathlineto{\pgfqpoint{2.785902in}{0.499444in}}%
\pgfpathclose%
\pgfusepath{fill}%
\end{pgfscope}%
\begin{pgfscope}%
\pgfpathrectangle{\pgfqpoint{0.445556in}{0.499444in}}{\pgfqpoint{3.875000in}{1.155000in}}%
\pgfusepath{clip}%
\pgfsetbuttcap%
\pgfsetmiterjoin%
\definecolor{currentfill}{rgb}{0.000000,0.000000,0.000000}%
\pgfsetfillcolor{currentfill}%
\pgfsetlinewidth{0.000000pt}%
\definecolor{currentstroke}{rgb}{0.000000,0.000000,0.000000}%
\pgfsetstrokecolor{currentstroke}%
\pgfsetstrokeopacity{0.000000}%
\pgfsetdash{}{0pt}%
\pgfpathmoveto{\pgfqpoint{2.939368in}{0.499444in}}%
\pgfpathlineto{\pgfqpoint{3.000754in}{0.499444in}}%
\pgfpathlineto{\pgfqpoint{3.000754in}{0.663722in}}%
\pgfpathlineto{\pgfqpoint{2.939368in}{0.663722in}}%
\pgfpathlineto{\pgfqpoint{2.939368in}{0.499444in}}%
\pgfpathclose%
\pgfusepath{fill}%
\end{pgfscope}%
\begin{pgfscope}%
\pgfpathrectangle{\pgfqpoint{0.445556in}{0.499444in}}{\pgfqpoint{3.875000in}{1.155000in}}%
\pgfusepath{clip}%
\pgfsetbuttcap%
\pgfsetmiterjoin%
\definecolor{currentfill}{rgb}{0.000000,0.000000,0.000000}%
\pgfsetfillcolor{currentfill}%
\pgfsetlinewidth{0.000000pt}%
\definecolor{currentstroke}{rgb}{0.000000,0.000000,0.000000}%
\pgfsetstrokecolor{currentstroke}%
\pgfsetstrokeopacity{0.000000}%
\pgfsetdash{}{0pt}%
\pgfpathmoveto{\pgfqpoint{3.092833in}{0.499444in}}%
\pgfpathlineto{\pgfqpoint{3.154219in}{0.499444in}}%
\pgfpathlineto{\pgfqpoint{3.154219in}{0.663945in}}%
\pgfpathlineto{\pgfqpoint{3.092833in}{0.663945in}}%
\pgfpathlineto{\pgfqpoint{3.092833in}{0.499444in}}%
\pgfpathclose%
\pgfusepath{fill}%
\end{pgfscope}%
\begin{pgfscope}%
\pgfpathrectangle{\pgfqpoint{0.445556in}{0.499444in}}{\pgfqpoint{3.875000in}{1.155000in}}%
\pgfusepath{clip}%
\pgfsetbuttcap%
\pgfsetmiterjoin%
\definecolor{currentfill}{rgb}{0.000000,0.000000,0.000000}%
\pgfsetfillcolor{currentfill}%
\pgfsetlinewidth{0.000000pt}%
\definecolor{currentstroke}{rgb}{0.000000,0.000000,0.000000}%
\pgfsetstrokecolor{currentstroke}%
\pgfsetstrokeopacity{0.000000}%
\pgfsetdash{}{0pt}%
\pgfpathmoveto{\pgfqpoint{3.246298in}{0.499444in}}%
\pgfpathlineto{\pgfqpoint{3.307684in}{0.499444in}}%
\pgfpathlineto{\pgfqpoint{3.307684in}{0.661936in}}%
\pgfpathlineto{\pgfqpoint{3.246298in}{0.661936in}}%
\pgfpathlineto{\pgfqpoint{3.246298in}{0.499444in}}%
\pgfpathclose%
\pgfusepath{fill}%
\end{pgfscope}%
\begin{pgfscope}%
\pgfpathrectangle{\pgfqpoint{0.445556in}{0.499444in}}{\pgfqpoint{3.875000in}{1.155000in}}%
\pgfusepath{clip}%
\pgfsetbuttcap%
\pgfsetmiterjoin%
\definecolor{currentfill}{rgb}{0.000000,0.000000,0.000000}%
\pgfsetfillcolor{currentfill}%
\pgfsetlinewidth{0.000000pt}%
\definecolor{currentstroke}{rgb}{0.000000,0.000000,0.000000}%
\pgfsetstrokecolor{currentstroke}%
\pgfsetstrokeopacity{0.000000}%
\pgfsetdash{}{0pt}%
\pgfpathmoveto{\pgfqpoint{3.399764in}{0.499444in}}%
\pgfpathlineto{\pgfqpoint{3.461150in}{0.499444in}}%
\pgfpathlineto{\pgfqpoint{3.461150in}{0.648314in}}%
\pgfpathlineto{\pgfqpoint{3.399764in}{0.648314in}}%
\pgfpathlineto{\pgfqpoint{3.399764in}{0.499444in}}%
\pgfpathclose%
\pgfusepath{fill}%
\end{pgfscope}%
\begin{pgfscope}%
\pgfpathrectangle{\pgfqpoint{0.445556in}{0.499444in}}{\pgfqpoint{3.875000in}{1.155000in}}%
\pgfusepath{clip}%
\pgfsetbuttcap%
\pgfsetmiterjoin%
\definecolor{currentfill}{rgb}{0.000000,0.000000,0.000000}%
\pgfsetfillcolor{currentfill}%
\pgfsetlinewidth{0.000000pt}%
\definecolor{currentstroke}{rgb}{0.000000,0.000000,0.000000}%
\pgfsetstrokecolor{currentstroke}%
\pgfsetstrokeopacity{0.000000}%
\pgfsetdash{}{0pt}%
\pgfpathmoveto{\pgfqpoint{3.553229in}{0.499444in}}%
\pgfpathlineto{\pgfqpoint{3.614615in}{0.499444in}}%
\pgfpathlineto{\pgfqpoint{3.614615in}{0.642136in}}%
\pgfpathlineto{\pgfqpoint{3.553229in}{0.642136in}}%
\pgfpathlineto{\pgfqpoint{3.553229in}{0.499444in}}%
\pgfpathclose%
\pgfusepath{fill}%
\end{pgfscope}%
\begin{pgfscope}%
\pgfpathrectangle{\pgfqpoint{0.445556in}{0.499444in}}{\pgfqpoint{3.875000in}{1.155000in}}%
\pgfusepath{clip}%
\pgfsetbuttcap%
\pgfsetmiterjoin%
\definecolor{currentfill}{rgb}{0.000000,0.000000,0.000000}%
\pgfsetfillcolor{currentfill}%
\pgfsetlinewidth{0.000000pt}%
\definecolor{currentstroke}{rgb}{0.000000,0.000000,0.000000}%
\pgfsetstrokecolor{currentstroke}%
\pgfsetstrokeopacity{0.000000}%
\pgfsetdash{}{0pt}%
\pgfpathmoveto{\pgfqpoint{3.706694in}{0.499444in}}%
\pgfpathlineto{\pgfqpoint{3.768080in}{0.499444in}}%
\pgfpathlineto{\pgfqpoint{3.768080in}{0.630673in}}%
\pgfpathlineto{\pgfqpoint{3.706694in}{0.630673in}}%
\pgfpathlineto{\pgfqpoint{3.706694in}{0.499444in}}%
\pgfpathclose%
\pgfusepath{fill}%
\end{pgfscope}%
\begin{pgfscope}%
\pgfpathrectangle{\pgfqpoint{0.445556in}{0.499444in}}{\pgfqpoint{3.875000in}{1.155000in}}%
\pgfusepath{clip}%
\pgfsetbuttcap%
\pgfsetmiterjoin%
\definecolor{currentfill}{rgb}{0.000000,0.000000,0.000000}%
\pgfsetfillcolor{currentfill}%
\pgfsetlinewidth{0.000000pt}%
\definecolor{currentstroke}{rgb}{0.000000,0.000000,0.000000}%
\pgfsetstrokecolor{currentstroke}%
\pgfsetstrokeopacity{0.000000}%
\pgfsetdash{}{0pt}%
\pgfpathmoveto{\pgfqpoint{3.860160in}{0.499444in}}%
\pgfpathlineto{\pgfqpoint{3.921546in}{0.499444in}}%
\pgfpathlineto{\pgfqpoint{3.921546in}{0.611394in}}%
\pgfpathlineto{\pgfqpoint{3.860160in}{0.611394in}}%
\pgfpathlineto{\pgfqpoint{3.860160in}{0.499444in}}%
\pgfpathclose%
\pgfusepath{fill}%
\end{pgfscope}%
\begin{pgfscope}%
\pgfpathrectangle{\pgfqpoint{0.445556in}{0.499444in}}{\pgfqpoint{3.875000in}{1.155000in}}%
\pgfusepath{clip}%
\pgfsetbuttcap%
\pgfsetmiterjoin%
\definecolor{currentfill}{rgb}{0.000000,0.000000,0.000000}%
\pgfsetfillcolor{currentfill}%
\pgfsetlinewidth{0.000000pt}%
\definecolor{currentstroke}{rgb}{0.000000,0.000000,0.000000}%
\pgfsetstrokecolor{currentstroke}%
\pgfsetstrokeopacity{0.000000}%
\pgfsetdash{}{0pt}%
\pgfpathmoveto{\pgfqpoint{4.013625in}{0.499444in}}%
\pgfpathlineto{\pgfqpoint{4.075011in}{0.499444in}}%
\pgfpathlineto{\pgfqpoint{4.075011in}{0.575740in}}%
\pgfpathlineto{\pgfqpoint{4.013625in}{0.575740in}}%
\pgfpathlineto{\pgfqpoint{4.013625in}{0.499444in}}%
\pgfpathclose%
\pgfusepath{fill}%
\end{pgfscope}%
\begin{pgfscope}%
\pgfpathrectangle{\pgfqpoint{0.445556in}{0.499444in}}{\pgfqpoint{3.875000in}{1.155000in}}%
\pgfusepath{clip}%
\pgfsetbuttcap%
\pgfsetmiterjoin%
\definecolor{currentfill}{rgb}{0.000000,0.000000,0.000000}%
\pgfsetfillcolor{currentfill}%
\pgfsetlinewidth{0.000000pt}%
\definecolor{currentstroke}{rgb}{0.000000,0.000000,0.000000}%
\pgfsetstrokecolor{currentstroke}%
\pgfsetstrokeopacity{0.000000}%
\pgfsetdash{}{0pt}%
\pgfpathmoveto{\pgfqpoint{4.167090in}{0.499444in}}%
\pgfpathlineto{\pgfqpoint{4.228476in}{0.499444in}}%
\pgfpathlineto{\pgfqpoint{4.228476in}{0.524752in}}%
\pgfpathlineto{\pgfqpoint{4.167090in}{0.524752in}}%
\pgfpathlineto{\pgfqpoint{4.167090in}{0.499444in}}%
\pgfpathclose%
\pgfusepath{fill}%
\end{pgfscope}%
\begin{pgfscope}%
\pgfsetbuttcap%
\pgfsetroundjoin%
\definecolor{currentfill}{rgb}{0.000000,0.000000,0.000000}%
\pgfsetfillcolor{currentfill}%
\pgfsetlinewidth{0.803000pt}%
\definecolor{currentstroke}{rgb}{0.000000,0.000000,0.000000}%
\pgfsetstrokecolor{currentstroke}%
\pgfsetdash{}{0pt}%
\pgfsys@defobject{currentmarker}{\pgfqpoint{0.000000in}{-0.048611in}}{\pgfqpoint{0.000000in}{0.000000in}}{%
\pgfpathmoveto{\pgfqpoint{0.000000in}{0.000000in}}%
\pgfpathlineto{\pgfqpoint{0.000000in}{-0.048611in}}%
\pgfusepath{stroke,fill}%
}%
\begin{pgfscope}%
\pgfsys@transformshift{0.483922in}{0.499444in}%
\pgfsys@useobject{currentmarker}{}%
\end{pgfscope}%
\end{pgfscope}%
\begin{pgfscope}%
\definecolor{textcolor}{rgb}{0.000000,0.000000,0.000000}%
\pgfsetstrokecolor{textcolor}%
\pgfsetfillcolor{textcolor}%
\pgftext[x=0.483922in,y=0.402222in,,top]{\color{textcolor}\rmfamily\fontsize{10.000000}{12.000000}\selectfont 0.0}%
\end{pgfscope}%
\begin{pgfscope}%
\pgfsetbuttcap%
\pgfsetroundjoin%
\definecolor{currentfill}{rgb}{0.000000,0.000000,0.000000}%
\pgfsetfillcolor{currentfill}%
\pgfsetlinewidth{0.803000pt}%
\definecolor{currentstroke}{rgb}{0.000000,0.000000,0.000000}%
\pgfsetstrokecolor{currentstroke}%
\pgfsetdash{}{0pt}%
\pgfsys@defobject{currentmarker}{\pgfqpoint{0.000000in}{-0.048611in}}{\pgfqpoint{0.000000in}{0.000000in}}{%
\pgfpathmoveto{\pgfqpoint{0.000000in}{0.000000in}}%
\pgfpathlineto{\pgfqpoint{0.000000in}{-0.048611in}}%
\pgfusepath{stroke,fill}%
}%
\begin{pgfscope}%
\pgfsys@transformshift{0.867585in}{0.499444in}%
\pgfsys@useobject{currentmarker}{}%
\end{pgfscope}%
\end{pgfscope}%
\begin{pgfscope}%
\definecolor{textcolor}{rgb}{0.000000,0.000000,0.000000}%
\pgfsetstrokecolor{textcolor}%
\pgfsetfillcolor{textcolor}%
\pgftext[x=0.867585in,y=0.402222in,,top]{\color{textcolor}\rmfamily\fontsize{10.000000}{12.000000}\selectfont 0.1}%
\end{pgfscope}%
\begin{pgfscope}%
\pgfsetbuttcap%
\pgfsetroundjoin%
\definecolor{currentfill}{rgb}{0.000000,0.000000,0.000000}%
\pgfsetfillcolor{currentfill}%
\pgfsetlinewidth{0.803000pt}%
\definecolor{currentstroke}{rgb}{0.000000,0.000000,0.000000}%
\pgfsetstrokecolor{currentstroke}%
\pgfsetdash{}{0pt}%
\pgfsys@defobject{currentmarker}{\pgfqpoint{0.000000in}{-0.048611in}}{\pgfqpoint{0.000000in}{0.000000in}}{%
\pgfpathmoveto{\pgfqpoint{0.000000in}{0.000000in}}%
\pgfpathlineto{\pgfqpoint{0.000000in}{-0.048611in}}%
\pgfusepath{stroke,fill}%
}%
\begin{pgfscope}%
\pgfsys@transformshift{1.251249in}{0.499444in}%
\pgfsys@useobject{currentmarker}{}%
\end{pgfscope}%
\end{pgfscope}%
\begin{pgfscope}%
\definecolor{textcolor}{rgb}{0.000000,0.000000,0.000000}%
\pgfsetstrokecolor{textcolor}%
\pgfsetfillcolor{textcolor}%
\pgftext[x=1.251249in,y=0.402222in,,top]{\color{textcolor}\rmfamily\fontsize{10.000000}{12.000000}\selectfont 0.2}%
\end{pgfscope}%
\begin{pgfscope}%
\pgfsetbuttcap%
\pgfsetroundjoin%
\definecolor{currentfill}{rgb}{0.000000,0.000000,0.000000}%
\pgfsetfillcolor{currentfill}%
\pgfsetlinewidth{0.803000pt}%
\definecolor{currentstroke}{rgb}{0.000000,0.000000,0.000000}%
\pgfsetstrokecolor{currentstroke}%
\pgfsetdash{}{0pt}%
\pgfsys@defobject{currentmarker}{\pgfqpoint{0.000000in}{-0.048611in}}{\pgfqpoint{0.000000in}{0.000000in}}{%
\pgfpathmoveto{\pgfqpoint{0.000000in}{0.000000in}}%
\pgfpathlineto{\pgfqpoint{0.000000in}{-0.048611in}}%
\pgfusepath{stroke,fill}%
}%
\begin{pgfscope}%
\pgfsys@transformshift{1.634912in}{0.499444in}%
\pgfsys@useobject{currentmarker}{}%
\end{pgfscope}%
\end{pgfscope}%
\begin{pgfscope}%
\definecolor{textcolor}{rgb}{0.000000,0.000000,0.000000}%
\pgfsetstrokecolor{textcolor}%
\pgfsetfillcolor{textcolor}%
\pgftext[x=1.634912in,y=0.402222in,,top]{\color{textcolor}\rmfamily\fontsize{10.000000}{12.000000}\selectfont 0.3}%
\end{pgfscope}%
\begin{pgfscope}%
\pgfsetbuttcap%
\pgfsetroundjoin%
\definecolor{currentfill}{rgb}{0.000000,0.000000,0.000000}%
\pgfsetfillcolor{currentfill}%
\pgfsetlinewidth{0.803000pt}%
\definecolor{currentstroke}{rgb}{0.000000,0.000000,0.000000}%
\pgfsetstrokecolor{currentstroke}%
\pgfsetdash{}{0pt}%
\pgfsys@defobject{currentmarker}{\pgfqpoint{0.000000in}{-0.048611in}}{\pgfqpoint{0.000000in}{0.000000in}}{%
\pgfpathmoveto{\pgfqpoint{0.000000in}{0.000000in}}%
\pgfpathlineto{\pgfqpoint{0.000000in}{-0.048611in}}%
\pgfusepath{stroke,fill}%
}%
\begin{pgfscope}%
\pgfsys@transformshift{2.018575in}{0.499444in}%
\pgfsys@useobject{currentmarker}{}%
\end{pgfscope}%
\end{pgfscope}%
\begin{pgfscope}%
\definecolor{textcolor}{rgb}{0.000000,0.000000,0.000000}%
\pgfsetstrokecolor{textcolor}%
\pgfsetfillcolor{textcolor}%
\pgftext[x=2.018575in,y=0.402222in,,top]{\color{textcolor}\rmfamily\fontsize{10.000000}{12.000000}\selectfont 0.4}%
\end{pgfscope}%
\begin{pgfscope}%
\pgfsetbuttcap%
\pgfsetroundjoin%
\definecolor{currentfill}{rgb}{0.000000,0.000000,0.000000}%
\pgfsetfillcolor{currentfill}%
\pgfsetlinewidth{0.803000pt}%
\definecolor{currentstroke}{rgb}{0.000000,0.000000,0.000000}%
\pgfsetstrokecolor{currentstroke}%
\pgfsetdash{}{0pt}%
\pgfsys@defobject{currentmarker}{\pgfqpoint{0.000000in}{-0.048611in}}{\pgfqpoint{0.000000in}{0.000000in}}{%
\pgfpathmoveto{\pgfqpoint{0.000000in}{0.000000in}}%
\pgfpathlineto{\pgfqpoint{0.000000in}{-0.048611in}}%
\pgfusepath{stroke,fill}%
}%
\begin{pgfscope}%
\pgfsys@transformshift{2.402239in}{0.499444in}%
\pgfsys@useobject{currentmarker}{}%
\end{pgfscope}%
\end{pgfscope}%
\begin{pgfscope}%
\definecolor{textcolor}{rgb}{0.000000,0.000000,0.000000}%
\pgfsetstrokecolor{textcolor}%
\pgfsetfillcolor{textcolor}%
\pgftext[x=2.402239in,y=0.402222in,,top]{\color{textcolor}\rmfamily\fontsize{10.000000}{12.000000}\selectfont 0.5}%
\end{pgfscope}%
\begin{pgfscope}%
\pgfsetbuttcap%
\pgfsetroundjoin%
\definecolor{currentfill}{rgb}{0.000000,0.000000,0.000000}%
\pgfsetfillcolor{currentfill}%
\pgfsetlinewidth{0.803000pt}%
\definecolor{currentstroke}{rgb}{0.000000,0.000000,0.000000}%
\pgfsetstrokecolor{currentstroke}%
\pgfsetdash{}{0pt}%
\pgfsys@defobject{currentmarker}{\pgfqpoint{0.000000in}{-0.048611in}}{\pgfqpoint{0.000000in}{0.000000in}}{%
\pgfpathmoveto{\pgfqpoint{0.000000in}{0.000000in}}%
\pgfpathlineto{\pgfqpoint{0.000000in}{-0.048611in}}%
\pgfusepath{stroke,fill}%
}%
\begin{pgfscope}%
\pgfsys@transformshift{2.785902in}{0.499444in}%
\pgfsys@useobject{currentmarker}{}%
\end{pgfscope}%
\end{pgfscope}%
\begin{pgfscope}%
\definecolor{textcolor}{rgb}{0.000000,0.000000,0.000000}%
\pgfsetstrokecolor{textcolor}%
\pgfsetfillcolor{textcolor}%
\pgftext[x=2.785902in,y=0.402222in,,top]{\color{textcolor}\rmfamily\fontsize{10.000000}{12.000000}\selectfont 0.6}%
\end{pgfscope}%
\begin{pgfscope}%
\pgfsetbuttcap%
\pgfsetroundjoin%
\definecolor{currentfill}{rgb}{0.000000,0.000000,0.000000}%
\pgfsetfillcolor{currentfill}%
\pgfsetlinewidth{0.803000pt}%
\definecolor{currentstroke}{rgb}{0.000000,0.000000,0.000000}%
\pgfsetstrokecolor{currentstroke}%
\pgfsetdash{}{0pt}%
\pgfsys@defobject{currentmarker}{\pgfqpoint{0.000000in}{-0.048611in}}{\pgfqpoint{0.000000in}{0.000000in}}{%
\pgfpathmoveto{\pgfqpoint{0.000000in}{0.000000in}}%
\pgfpathlineto{\pgfqpoint{0.000000in}{-0.048611in}}%
\pgfusepath{stroke,fill}%
}%
\begin{pgfscope}%
\pgfsys@transformshift{3.169566in}{0.499444in}%
\pgfsys@useobject{currentmarker}{}%
\end{pgfscope}%
\end{pgfscope}%
\begin{pgfscope}%
\definecolor{textcolor}{rgb}{0.000000,0.000000,0.000000}%
\pgfsetstrokecolor{textcolor}%
\pgfsetfillcolor{textcolor}%
\pgftext[x=3.169566in,y=0.402222in,,top]{\color{textcolor}\rmfamily\fontsize{10.000000}{12.000000}\selectfont 0.7}%
\end{pgfscope}%
\begin{pgfscope}%
\pgfsetbuttcap%
\pgfsetroundjoin%
\definecolor{currentfill}{rgb}{0.000000,0.000000,0.000000}%
\pgfsetfillcolor{currentfill}%
\pgfsetlinewidth{0.803000pt}%
\definecolor{currentstroke}{rgb}{0.000000,0.000000,0.000000}%
\pgfsetstrokecolor{currentstroke}%
\pgfsetdash{}{0pt}%
\pgfsys@defobject{currentmarker}{\pgfqpoint{0.000000in}{-0.048611in}}{\pgfqpoint{0.000000in}{0.000000in}}{%
\pgfpathmoveto{\pgfqpoint{0.000000in}{0.000000in}}%
\pgfpathlineto{\pgfqpoint{0.000000in}{-0.048611in}}%
\pgfusepath{stroke,fill}%
}%
\begin{pgfscope}%
\pgfsys@transformshift{3.553229in}{0.499444in}%
\pgfsys@useobject{currentmarker}{}%
\end{pgfscope}%
\end{pgfscope}%
\begin{pgfscope}%
\definecolor{textcolor}{rgb}{0.000000,0.000000,0.000000}%
\pgfsetstrokecolor{textcolor}%
\pgfsetfillcolor{textcolor}%
\pgftext[x=3.553229in,y=0.402222in,,top]{\color{textcolor}\rmfamily\fontsize{10.000000}{12.000000}\selectfont 0.8}%
\end{pgfscope}%
\begin{pgfscope}%
\pgfsetbuttcap%
\pgfsetroundjoin%
\definecolor{currentfill}{rgb}{0.000000,0.000000,0.000000}%
\pgfsetfillcolor{currentfill}%
\pgfsetlinewidth{0.803000pt}%
\definecolor{currentstroke}{rgb}{0.000000,0.000000,0.000000}%
\pgfsetstrokecolor{currentstroke}%
\pgfsetdash{}{0pt}%
\pgfsys@defobject{currentmarker}{\pgfqpoint{0.000000in}{-0.048611in}}{\pgfqpoint{0.000000in}{0.000000in}}{%
\pgfpathmoveto{\pgfqpoint{0.000000in}{0.000000in}}%
\pgfpathlineto{\pgfqpoint{0.000000in}{-0.048611in}}%
\pgfusepath{stroke,fill}%
}%
\begin{pgfscope}%
\pgfsys@transformshift{3.936892in}{0.499444in}%
\pgfsys@useobject{currentmarker}{}%
\end{pgfscope}%
\end{pgfscope}%
\begin{pgfscope}%
\definecolor{textcolor}{rgb}{0.000000,0.000000,0.000000}%
\pgfsetstrokecolor{textcolor}%
\pgfsetfillcolor{textcolor}%
\pgftext[x=3.936892in,y=0.402222in,,top]{\color{textcolor}\rmfamily\fontsize{10.000000}{12.000000}\selectfont 0.9}%
\end{pgfscope}%
\begin{pgfscope}%
\pgfsetbuttcap%
\pgfsetroundjoin%
\definecolor{currentfill}{rgb}{0.000000,0.000000,0.000000}%
\pgfsetfillcolor{currentfill}%
\pgfsetlinewidth{0.803000pt}%
\definecolor{currentstroke}{rgb}{0.000000,0.000000,0.000000}%
\pgfsetstrokecolor{currentstroke}%
\pgfsetdash{}{0pt}%
\pgfsys@defobject{currentmarker}{\pgfqpoint{0.000000in}{-0.048611in}}{\pgfqpoint{0.000000in}{0.000000in}}{%
\pgfpathmoveto{\pgfqpoint{0.000000in}{0.000000in}}%
\pgfpathlineto{\pgfqpoint{0.000000in}{-0.048611in}}%
\pgfusepath{stroke,fill}%
}%
\begin{pgfscope}%
\pgfsys@transformshift{4.320556in}{0.499444in}%
\pgfsys@useobject{currentmarker}{}%
\end{pgfscope}%
\end{pgfscope}%
\begin{pgfscope}%
\definecolor{textcolor}{rgb}{0.000000,0.000000,0.000000}%
\pgfsetstrokecolor{textcolor}%
\pgfsetfillcolor{textcolor}%
\pgftext[x=4.320556in,y=0.402222in,,top]{\color{textcolor}\rmfamily\fontsize{10.000000}{12.000000}\selectfont 1.0}%
\end{pgfscope}%
\begin{pgfscope}%
\definecolor{textcolor}{rgb}{0.000000,0.000000,0.000000}%
\pgfsetstrokecolor{textcolor}%
\pgfsetfillcolor{textcolor}%
\pgftext[x=2.383056in,y=0.223333in,,top]{\color{textcolor}\rmfamily\fontsize{10.000000}{12.000000}\selectfont \(\displaystyle p\)}%
\end{pgfscope}%
\begin{pgfscope}%
\pgfsetbuttcap%
\pgfsetroundjoin%
\definecolor{currentfill}{rgb}{0.000000,0.000000,0.000000}%
\pgfsetfillcolor{currentfill}%
\pgfsetlinewidth{0.803000pt}%
\definecolor{currentstroke}{rgb}{0.000000,0.000000,0.000000}%
\pgfsetstrokecolor{currentstroke}%
\pgfsetdash{}{0pt}%
\pgfsys@defobject{currentmarker}{\pgfqpoint{-0.048611in}{0.000000in}}{\pgfqpoint{-0.000000in}{0.000000in}}{%
\pgfpathmoveto{\pgfqpoint{-0.000000in}{0.000000in}}%
\pgfpathlineto{\pgfqpoint{-0.048611in}{0.000000in}}%
\pgfusepath{stroke,fill}%
}%
\begin{pgfscope}%
\pgfsys@transformshift{0.445556in}{0.499444in}%
\pgfsys@useobject{currentmarker}{}%
\end{pgfscope}%
\end{pgfscope}%
\begin{pgfscope}%
\definecolor{textcolor}{rgb}{0.000000,0.000000,0.000000}%
\pgfsetstrokecolor{textcolor}%
\pgfsetfillcolor{textcolor}%
\pgftext[x=0.278889in, y=0.451250in, left, base]{\color{textcolor}\rmfamily\fontsize{10.000000}{12.000000}\selectfont \(\displaystyle {0}\)}%
\end{pgfscope}%
\begin{pgfscope}%
\pgfsetbuttcap%
\pgfsetroundjoin%
\definecolor{currentfill}{rgb}{0.000000,0.000000,0.000000}%
\pgfsetfillcolor{currentfill}%
\pgfsetlinewidth{0.803000pt}%
\definecolor{currentstroke}{rgb}{0.000000,0.000000,0.000000}%
\pgfsetstrokecolor{currentstroke}%
\pgfsetdash{}{0pt}%
\pgfsys@defobject{currentmarker}{\pgfqpoint{-0.048611in}{0.000000in}}{\pgfqpoint{-0.000000in}{0.000000in}}{%
\pgfpathmoveto{\pgfqpoint{-0.000000in}{0.000000in}}%
\pgfpathlineto{\pgfqpoint{-0.048611in}{0.000000in}}%
\pgfusepath{stroke,fill}%
}%
\begin{pgfscope}%
\pgfsys@transformshift{0.445556in}{0.818130in}%
\pgfsys@useobject{currentmarker}{}%
\end{pgfscope}%
\end{pgfscope}%
\begin{pgfscope}%
\definecolor{textcolor}{rgb}{0.000000,0.000000,0.000000}%
\pgfsetstrokecolor{textcolor}%
\pgfsetfillcolor{textcolor}%
\pgftext[x=0.278889in, y=0.769936in, left, base]{\color{textcolor}\rmfamily\fontsize{10.000000}{12.000000}\selectfont \(\displaystyle {2}\)}%
\end{pgfscope}%
\begin{pgfscope}%
\pgfsetbuttcap%
\pgfsetroundjoin%
\definecolor{currentfill}{rgb}{0.000000,0.000000,0.000000}%
\pgfsetfillcolor{currentfill}%
\pgfsetlinewidth{0.803000pt}%
\definecolor{currentstroke}{rgb}{0.000000,0.000000,0.000000}%
\pgfsetstrokecolor{currentstroke}%
\pgfsetdash{}{0pt}%
\pgfsys@defobject{currentmarker}{\pgfqpoint{-0.048611in}{0.000000in}}{\pgfqpoint{-0.000000in}{0.000000in}}{%
\pgfpathmoveto{\pgfqpoint{-0.000000in}{0.000000in}}%
\pgfpathlineto{\pgfqpoint{-0.048611in}{0.000000in}}%
\pgfusepath{stroke,fill}%
}%
\begin{pgfscope}%
\pgfsys@transformshift{0.445556in}{1.136816in}%
\pgfsys@useobject{currentmarker}{}%
\end{pgfscope}%
\end{pgfscope}%
\begin{pgfscope}%
\definecolor{textcolor}{rgb}{0.000000,0.000000,0.000000}%
\pgfsetstrokecolor{textcolor}%
\pgfsetfillcolor{textcolor}%
\pgftext[x=0.278889in, y=1.088621in, left, base]{\color{textcolor}\rmfamily\fontsize{10.000000}{12.000000}\selectfont \(\displaystyle {4}\)}%
\end{pgfscope}%
\begin{pgfscope}%
\pgfsetbuttcap%
\pgfsetroundjoin%
\definecolor{currentfill}{rgb}{0.000000,0.000000,0.000000}%
\pgfsetfillcolor{currentfill}%
\pgfsetlinewidth{0.803000pt}%
\definecolor{currentstroke}{rgb}{0.000000,0.000000,0.000000}%
\pgfsetstrokecolor{currentstroke}%
\pgfsetdash{}{0pt}%
\pgfsys@defobject{currentmarker}{\pgfqpoint{-0.048611in}{0.000000in}}{\pgfqpoint{-0.000000in}{0.000000in}}{%
\pgfpathmoveto{\pgfqpoint{-0.000000in}{0.000000in}}%
\pgfpathlineto{\pgfqpoint{-0.048611in}{0.000000in}}%
\pgfusepath{stroke,fill}%
}%
\begin{pgfscope}%
\pgfsys@transformshift{0.445556in}{1.455502in}%
\pgfsys@useobject{currentmarker}{}%
\end{pgfscope}%
\end{pgfscope}%
\begin{pgfscope}%
\definecolor{textcolor}{rgb}{0.000000,0.000000,0.000000}%
\pgfsetstrokecolor{textcolor}%
\pgfsetfillcolor{textcolor}%
\pgftext[x=0.278889in, y=1.407307in, left, base]{\color{textcolor}\rmfamily\fontsize{10.000000}{12.000000}\selectfont \(\displaystyle {6}\)}%
\end{pgfscope}%
\begin{pgfscope}%
\definecolor{textcolor}{rgb}{0.000000,0.000000,0.000000}%
\pgfsetstrokecolor{textcolor}%
\pgfsetfillcolor{textcolor}%
\pgftext[x=0.223333in,y=1.076944in,,bottom,rotate=90.000000]{\color{textcolor}\rmfamily\fontsize{10.000000}{12.000000}\selectfont Percent of Data Set}%
\end{pgfscope}%
\begin{pgfscope}%
\pgfsetrectcap%
\pgfsetmiterjoin%
\pgfsetlinewidth{0.803000pt}%
\definecolor{currentstroke}{rgb}{0.000000,0.000000,0.000000}%
\pgfsetstrokecolor{currentstroke}%
\pgfsetdash{}{0pt}%
\pgfpathmoveto{\pgfqpoint{0.445556in}{0.499444in}}%
\pgfpathlineto{\pgfqpoint{0.445556in}{1.654444in}}%
\pgfusepath{stroke}%
\end{pgfscope}%
\begin{pgfscope}%
\pgfsetrectcap%
\pgfsetmiterjoin%
\pgfsetlinewidth{0.803000pt}%
\definecolor{currentstroke}{rgb}{0.000000,0.000000,0.000000}%
\pgfsetstrokecolor{currentstroke}%
\pgfsetdash{}{0pt}%
\pgfpathmoveto{\pgfqpoint{4.320556in}{0.499444in}}%
\pgfpathlineto{\pgfqpoint{4.320556in}{1.654444in}}%
\pgfusepath{stroke}%
\end{pgfscope}%
\begin{pgfscope}%
\pgfsetrectcap%
\pgfsetmiterjoin%
\pgfsetlinewidth{0.803000pt}%
\definecolor{currentstroke}{rgb}{0.000000,0.000000,0.000000}%
\pgfsetstrokecolor{currentstroke}%
\pgfsetdash{}{0pt}%
\pgfpathmoveto{\pgfqpoint{0.445556in}{0.499444in}}%
\pgfpathlineto{\pgfqpoint{4.320556in}{0.499444in}}%
\pgfusepath{stroke}%
\end{pgfscope}%
\begin{pgfscope}%
\pgfsetrectcap%
\pgfsetmiterjoin%
\pgfsetlinewidth{0.803000pt}%
\definecolor{currentstroke}{rgb}{0.000000,0.000000,0.000000}%
\pgfsetstrokecolor{currentstroke}%
\pgfsetdash{}{0pt}%
\pgfpathmoveto{\pgfqpoint{0.445556in}{1.654444in}}%
\pgfpathlineto{\pgfqpoint{4.320556in}{1.654444in}}%
\pgfusepath{stroke}%
\end{pgfscope}%
\begin{pgfscope}%
\pgfsetbuttcap%
\pgfsetmiterjoin%
\definecolor{currentfill}{rgb}{1.000000,1.000000,1.000000}%
\pgfsetfillcolor{currentfill}%
\pgfsetfillopacity{0.800000}%
\pgfsetlinewidth{1.003750pt}%
\definecolor{currentstroke}{rgb}{0.800000,0.800000,0.800000}%
\pgfsetstrokecolor{currentstroke}%
\pgfsetstrokeopacity{0.800000}%
\pgfsetdash{}{0pt}%
\pgfpathmoveto{\pgfqpoint{3.543611in}{1.154445in}}%
\pgfpathlineto{\pgfqpoint{4.223333in}{1.154445in}}%
\pgfpathquadraticcurveto{\pgfqpoint{4.251111in}{1.154445in}}{\pgfqpoint{4.251111in}{1.182222in}}%
\pgfpathlineto{\pgfqpoint{4.251111in}{1.557222in}}%
\pgfpathquadraticcurveto{\pgfqpoint{4.251111in}{1.585000in}}{\pgfqpoint{4.223333in}{1.585000in}}%
\pgfpathlineto{\pgfqpoint{3.543611in}{1.585000in}}%
\pgfpathquadraticcurveto{\pgfqpoint{3.515833in}{1.585000in}}{\pgfqpoint{3.515833in}{1.557222in}}%
\pgfpathlineto{\pgfqpoint{3.515833in}{1.182222in}}%
\pgfpathquadraticcurveto{\pgfqpoint{3.515833in}{1.154445in}}{\pgfqpoint{3.543611in}{1.154445in}}%
\pgfpathlineto{\pgfqpoint{3.543611in}{1.154445in}}%
\pgfpathclose%
\pgfusepath{stroke,fill}%
\end{pgfscope}%
\begin{pgfscope}%
\pgfsetbuttcap%
\pgfsetmiterjoin%
\pgfsetlinewidth{1.003750pt}%
\definecolor{currentstroke}{rgb}{0.000000,0.000000,0.000000}%
\pgfsetstrokecolor{currentstroke}%
\pgfsetdash{}{0pt}%
\pgfpathmoveto{\pgfqpoint{3.571389in}{1.432222in}}%
\pgfpathlineto{\pgfqpoint{3.849167in}{1.432222in}}%
\pgfpathlineto{\pgfqpoint{3.849167in}{1.529444in}}%
\pgfpathlineto{\pgfqpoint{3.571389in}{1.529444in}}%
\pgfpathlineto{\pgfqpoint{3.571389in}{1.432222in}}%
\pgfpathclose%
\pgfusepath{stroke}%
\end{pgfscope}%
\begin{pgfscope}%
\definecolor{textcolor}{rgb}{0.000000,0.000000,0.000000}%
\pgfsetstrokecolor{textcolor}%
\pgfsetfillcolor{textcolor}%
\pgftext[x=3.960278in,y=1.432222in,left,base]{\color{textcolor}\rmfamily\fontsize{10.000000}{12.000000}\selectfont Neg}%
\end{pgfscope}%
\begin{pgfscope}%
\pgfsetbuttcap%
\pgfsetmiterjoin%
\definecolor{currentfill}{rgb}{0.000000,0.000000,0.000000}%
\pgfsetfillcolor{currentfill}%
\pgfsetlinewidth{0.000000pt}%
\definecolor{currentstroke}{rgb}{0.000000,0.000000,0.000000}%
\pgfsetstrokecolor{currentstroke}%
\pgfsetstrokeopacity{0.000000}%
\pgfsetdash{}{0pt}%
\pgfpathmoveto{\pgfqpoint{3.571389in}{1.236944in}}%
\pgfpathlineto{\pgfqpoint{3.849167in}{1.236944in}}%
\pgfpathlineto{\pgfqpoint{3.849167in}{1.334167in}}%
\pgfpathlineto{\pgfqpoint{3.571389in}{1.334167in}}%
\pgfpathlineto{\pgfqpoint{3.571389in}{1.236944in}}%
\pgfpathclose%
\pgfusepath{fill}%
\end{pgfscope}%
\begin{pgfscope}%
\definecolor{textcolor}{rgb}{0.000000,0.000000,0.000000}%
\pgfsetstrokecolor{textcolor}%
\pgfsetfillcolor{textcolor}%
\pgftext[x=3.960278in,y=1.236944in,left,base]{\color{textcolor}\rmfamily\fontsize{10.000000}{12.000000}\selectfont Pos}%
\end{pgfscope}%
\end{pgfpicture}%
\makeatother%
\endgroup%
	
&
	\vskip 0pt
	\hfil ROC Curve
	
	%% Creator: Matplotlib, PGF backend
%%
%% To include the figure in your LaTeX document, write
%%   \input{<filename>.pgf}
%%
%% Make sure the required packages are loaded in your preamble
%%   \usepackage{pgf}
%%
%% Also ensure that all the required font packages are loaded; for instance,
%% the lmodern package is sometimes necessary when using math font.
%%   \usepackage{lmodern}
%%
%% Figures using additional raster images can only be included by \input if
%% they are in the same directory as the main LaTeX file. For loading figures
%% from other directories you can use the `import` package
%%   \usepackage{import}
%%
%% and then include the figures with
%%   \import{<path to file>}{<filename>.pgf}
%%
%% Matplotlib used the following preamble
%%   
%%   \usepackage{fontspec}
%%   \makeatletter\@ifpackageloaded{underscore}{}{\usepackage[strings]{underscore}}\makeatother
%%
\begingroup%
\makeatletter%
\begin{pgfpicture}%
\pgfpathrectangle{\pgfpointorigin}{\pgfqpoint{2.221861in}{1.754444in}}%
\pgfusepath{use as bounding box, clip}%
\begin{pgfscope}%
\pgfsetbuttcap%
\pgfsetmiterjoin%
\definecolor{currentfill}{rgb}{1.000000,1.000000,1.000000}%
\pgfsetfillcolor{currentfill}%
\pgfsetlinewidth{0.000000pt}%
\definecolor{currentstroke}{rgb}{1.000000,1.000000,1.000000}%
\pgfsetstrokecolor{currentstroke}%
\pgfsetdash{}{0pt}%
\pgfpathmoveto{\pgfqpoint{0.000000in}{0.000000in}}%
\pgfpathlineto{\pgfqpoint{2.221861in}{0.000000in}}%
\pgfpathlineto{\pgfqpoint{2.221861in}{1.754444in}}%
\pgfpathlineto{\pgfqpoint{0.000000in}{1.754444in}}%
\pgfpathlineto{\pgfqpoint{0.000000in}{0.000000in}}%
\pgfpathclose%
\pgfusepath{fill}%
\end{pgfscope}%
\begin{pgfscope}%
\pgfsetbuttcap%
\pgfsetmiterjoin%
\definecolor{currentfill}{rgb}{1.000000,1.000000,1.000000}%
\pgfsetfillcolor{currentfill}%
\pgfsetlinewidth{0.000000pt}%
\definecolor{currentstroke}{rgb}{0.000000,0.000000,0.000000}%
\pgfsetstrokecolor{currentstroke}%
\pgfsetstrokeopacity{0.000000}%
\pgfsetdash{}{0pt}%
\pgfpathmoveto{\pgfqpoint{0.553581in}{0.499444in}}%
\pgfpathlineto{\pgfqpoint{2.103581in}{0.499444in}}%
\pgfpathlineto{\pgfqpoint{2.103581in}{1.654444in}}%
\pgfpathlineto{\pgfqpoint{0.553581in}{1.654444in}}%
\pgfpathlineto{\pgfqpoint{0.553581in}{0.499444in}}%
\pgfpathclose%
\pgfusepath{fill}%
\end{pgfscope}%
\begin{pgfscope}%
\pgfsetbuttcap%
\pgfsetroundjoin%
\definecolor{currentfill}{rgb}{0.000000,0.000000,0.000000}%
\pgfsetfillcolor{currentfill}%
\pgfsetlinewidth{0.803000pt}%
\definecolor{currentstroke}{rgb}{0.000000,0.000000,0.000000}%
\pgfsetstrokecolor{currentstroke}%
\pgfsetdash{}{0pt}%
\pgfsys@defobject{currentmarker}{\pgfqpoint{0.000000in}{-0.048611in}}{\pgfqpoint{0.000000in}{0.000000in}}{%
\pgfpathmoveto{\pgfqpoint{0.000000in}{0.000000in}}%
\pgfpathlineto{\pgfqpoint{0.000000in}{-0.048611in}}%
\pgfusepath{stroke,fill}%
}%
\begin{pgfscope}%
\pgfsys@transformshift{0.624035in}{0.499444in}%
\pgfsys@useobject{currentmarker}{}%
\end{pgfscope}%
\end{pgfscope}%
\begin{pgfscope}%
\definecolor{textcolor}{rgb}{0.000000,0.000000,0.000000}%
\pgfsetstrokecolor{textcolor}%
\pgfsetfillcolor{textcolor}%
\pgftext[x=0.624035in,y=0.402222in,,top]{\color{textcolor}\rmfamily\fontsize{10.000000}{12.000000}\selectfont \(\displaystyle {0.0}\)}%
\end{pgfscope}%
\begin{pgfscope}%
\pgfsetbuttcap%
\pgfsetroundjoin%
\definecolor{currentfill}{rgb}{0.000000,0.000000,0.000000}%
\pgfsetfillcolor{currentfill}%
\pgfsetlinewidth{0.803000pt}%
\definecolor{currentstroke}{rgb}{0.000000,0.000000,0.000000}%
\pgfsetstrokecolor{currentstroke}%
\pgfsetdash{}{0pt}%
\pgfsys@defobject{currentmarker}{\pgfqpoint{0.000000in}{-0.048611in}}{\pgfqpoint{0.000000in}{0.000000in}}{%
\pgfpathmoveto{\pgfqpoint{0.000000in}{0.000000in}}%
\pgfpathlineto{\pgfqpoint{0.000000in}{-0.048611in}}%
\pgfusepath{stroke,fill}%
}%
\begin{pgfscope}%
\pgfsys@transformshift{1.328581in}{0.499444in}%
\pgfsys@useobject{currentmarker}{}%
\end{pgfscope}%
\end{pgfscope}%
\begin{pgfscope}%
\definecolor{textcolor}{rgb}{0.000000,0.000000,0.000000}%
\pgfsetstrokecolor{textcolor}%
\pgfsetfillcolor{textcolor}%
\pgftext[x=1.328581in,y=0.402222in,,top]{\color{textcolor}\rmfamily\fontsize{10.000000}{12.000000}\selectfont \(\displaystyle {0.5}\)}%
\end{pgfscope}%
\begin{pgfscope}%
\pgfsetbuttcap%
\pgfsetroundjoin%
\definecolor{currentfill}{rgb}{0.000000,0.000000,0.000000}%
\pgfsetfillcolor{currentfill}%
\pgfsetlinewidth{0.803000pt}%
\definecolor{currentstroke}{rgb}{0.000000,0.000000,0.000000}%
\pgfsetstrokecolor{currentstroke}%
\pgfsetdash{}{0pt}%
\pgfsys@defobject{currentmarker}{\pgfqpoint{0.000000in}{-0.048611in}}{\pgfqpoint{0.000000in}{0.000000in}}{%
\pgfpathmoveto{\pgfqpoint{0.000000in}{0.000000in}}%
\pgfpathlineto{\pgfqpoint{0.000000in}{-0.048611in}}%
\pgfusepath{stroke,fill}%
}%
\begin{pgfscope}%
\pgfsys@transformshift{2.033126in}{0.499444in}%
\pgfsys@useobject{currentmarker}{}%
\end{pgfscope}%
\end{pgfscope}%
\begin{pgfscope}%
\definecolor{textcolor}{rgb}{0.000000,0.000000,0.000000}%
\pgfsetstrokecolor{textcolor}%
\pgfsetfillcolor{textcolor}%
\pgftext[x=2.033126in,y=0.402222in,,top]{\color{textcolor}\rmfamily\fontsize{10.000000}{12.000000}\selectfont \(\displaystyle {1.0}\)}%
\end{pgfscope}%
\begin{pgfscope}%
\definecolor{textcolor}{rgb}{0.000000,0.000000,0.000000}%
\pgfsetstrokecolor{textcolor}%
\pgfsetfillcolor{textcolor}%
\pgftext[x=1.328581in,y=0.223333in,,top]{\color{textcolor}\rmfamily\fontsize{10.000000}{12.000000}\selectfont False positive rate}%
\end{pgfscope}%
\begin{pgfscope}%
\pgfsetbuttcap%
\pgfsetroundjoin%
\definecolor{currentfill}{rgb}{0.000000,0.000000,0.000000}%
\pgfsetfillcolor{currentfill}%
\pgfsetlinewidth{0.803000pt}%
\definecolor{currentstroke}{rgb}{0.000000,0.000000,0.000000}%
\pgfsetstrokecolor{currentstroke}%
\pgfsetdash{}{0pt}%
\pgfsys@defobject{currentmarker}{\pgfqpoint{-0.048611in}{0.000000in}}{\pgfqpoint{-0.000000in}{0.000000in}}{%
\pgfpathmoveto{\pgfqpoint{-0.000000in}{0.000000in}}%
\pgfpathlineto{\pgfqpoint{-0.048611in}{0.000000in}}%
\pgfusepath{stroke,fill}%
}%
\begin{pgfscope}%
\pgfsys@transformshift{0.553581in}{0.551944in}%
\pgfsys@useobject{currentmarker}{}%
\end{pgfscope}%
\end{pgfscope}%
\begin{pgfscope}%
\definecolor{textcolor}{rgb}{0.000000,0.000000,0.000000}%
\pgfsetstrokecolor{textcolor}%
\pgfsetfillcolor{textcolor}%
\pgftext[x=0.278889in, y=0.503750in, left, base]{\color{textcolor}\rmfamily\fontsize{10.000000}{12.000000}\selectfont \(\displaystyle {0.0}\)}%
\end{pgfscope}%
\begin{pgfscope}%
\pgfsetbuttcap%
\pgfsetroundjoin%
\definecolor{currentfill}{rgb}{0.000000,0.000000,0.000000}%
\pgfsetfillcolor{currentfill}%
\pgfsetlinewidth{0.803000pt}%
\definecolor{currentstroke}{rgb}{0.000000,0.000000,0.000000}%
\pgfsetstrokecolor{currentstroke}%
\pgfsetdash{}{0pt}%
\pgfsys@defobject{currentmarker}{\pgfqpoint{-0.048611in}{0.000000in}}{\pgfqpoint{-0.000000in}{0.000000in}}{%
\pgfpathmoveto{\pgfqpoint{-0.000000in}{0.000000in}}%
\pgfpathlineto{\pgfqpoint{-0.048611in}{0.000000in}}%
\pgfusepath{stroke,fill}%
}%
\begin{pgfscope}%
\pgfsys@transformshift{0.553581in}{1.076944in}%
\pgfsys@useobject{currentmarker}{}%
\end{pgfscope}%
\end{pgfscope}%
\begin{pgfscope}%
\definecolor{textcolor}{rgb}{0.000000,0.000000,0.000000}%
\pgfsetstrokecolor{textcolor}%
\pgfsetfillcolor{textcolor}%
\pgftext[x=0.278889in, y=1.028750in, left, base]{\color{textcolor}\rmfamily\fontsize{10.000000}{12.000000}\selectfont \(\displaystyle {0.5}\)}%
\end{pgfscope}%
\begin{pgfscope}%
\pgfsetbuttcap%
\pgfsetroundjoin%
\definecolor{currentfill}{rgb}{0.000000,0.000000,0.000000}%
\pgfsetfillcolor{currentfill}%
\pgfsetlinewidth{0.803000pt}%
\definecolor{currentstroke}{rgb}{0.000000,0.000000,0.000000}%
\pgfsetstrokecolor{currentstroke}%
\pgfsetdash{}{0pt}%
\pgfsys@defobject{currentmarker}{\pgfqpoint{-0.048611in}{0.000000in}}{\pgfqpoint{-0.000000in}{0.000000in}}{%
\pgfpathmoveto{\pgfqpoint{-0.000000in}{0.000000in}}%
\pgfpathlineto{\pgfqpoint{-0.048611in}{0.000000in}}%
\pgfusepath{stroke,fill}%
}%
\begin{pgfscope}%
\pgfsys@transformshift{0.553581in}{1.601944in}%
\pgfsys@useobject{currentmarker}{}%
\end{pgfscope}%
\end{pgfscope}%
\begin{pgfscope}%
\definecolor{textcolor}{rgb}{0.000000,0.000000,0.000000}%
\pgfsetstrokecolor{textcolor}%
\pgfsetfillcolor{textcolor}%
\pgftext[x=0.278889in, y=1.553750in, left, base]{\color{textcolor}\rmfamily\fontsize{10.000000}{12.000000}\selectfont \(\displaystyle {1.0}\)}%
\end{pgfscope}%
\begin{pgfscope}%
\definecolor{textcolor}{rgb}{0.000000,0.000000,0.000000}%
\pgfsetstrokecolor{textcolor}%
\pgfsetfillcolor{textcolor}%
\pgftext[x=0.223333in,y=1.076944in,,bottom,rotate=90.000000]{\color{textcolor}\rmfamily\fontsize{10.000000}{12.000000}\selectfont True positive rate}%
\end{pgfscope}%
\begin{pgfscope}%
\pgfpathrectangle{\pgfqpoint{0.553581in}{0.499444in}}{\pgfqpoint{1.550000in}{1.155000in}}%
\pgfusepath{clip}%
\pgfsetbuttcap%
\pgfsetroundjoin%
\pgfsetlinewidth{1.505625pt}%
\definecolor{currentstroke}{rgb}{0.000000,0.000000,0.000000}%
\pgfsetstrokecolor{currentstroke}%
\pgfsetdash{{5.550000pt}{2.400000pt}}{0.000000pt}%
\pgfpathmoveto{\pgfqpoint{0.624035in}{0.551944in}}%
\pgfpathlineto{\pgfqpoint{2.033126in}{1.601944in}}%
\pgfusepath{stroke}%
\end{pgfscope}%
\begin{pgfscope}%
\pgfpathrectangle{\pgfqpoint{0.553581in}{0.499444in}}{\pgfqpoint{1.550000in}{1.155000in}}%
\pgfusepath{clip}%
\pgfsetrectcap%
\pgfsetroundjoin%
\pgfsetlinewidth{1.505625pt}%
\definecolor{currentstroke}{rgb}{0.000000,0.000000,0.000000}%
\pgfsetstrokecolor{currentstroke}%
\pgfsetdash{}{0pt}%
\pgfpathmoveto{\pgfqpoint{0.624035in}{0.551944in}}%
\pgfpathlineto{\pgfqpoint{0.625012in}{0.569079in}}%
\pgfpathlineto{\pgfqpoint{0.635394in}{0.667887in}}%
\pgfpathlineto{\pgfqpoint{0.647621in}{0.739439in}}%
\pgfpathlineto{\pgfqpoint{0.657940in}{0.785971in}}%
\pgfpathlineto{\pgfqpoint{0.671395in}{0.833962in}}%
\pgfpathlineto{\pgfqpoint{0.687827in}{0.885088in}}%
\pgfpathlineto{\pgfqpoint{0.707621in}{0.935097in}}%
\pgfpathlineto{\pgfqpoint{0.731911in}{0.988862in}}%
\pgfpathlineto{\pgfqpoint{0.772289in}{1.056317in}}%
\pgfpathlineto{\pgfqpoint{0.773125in}{1.057403in}}%
\pgfpathlineto{\pgfqpoint{0.773344in}{1.057931in}}%
\pgfpathlineto{\pgfqpoint{0.773360in}{1.057931in}}%
\pgfpathlineto{\pgfqpoint{0.797681in}{1.093722in}}%
\pgfpathlineto{\pgfqpoint{0.825598in}{1.128272in}}%
\pgfpathlineto{\pgfqpoint{0.855664in}{1.162791in}}%
\pgfpathlineto{\pgfqpoint{0.905689in}{1.219598in}}%
\pgfpathlineto{\pgfqpoint{0.923599in}{1.236858in}}%
\pgfpathlineto{\pgfqpoint{0.961953in}{1.269700in}}%
\pgfpathlineto{\pgfqpoint{1.045234in}{1.332933in}}%
\pgfpathlineto{\pgfqpoint{1.068898in}{1.347864in}}%
\pgfpathlineto{\pgfqpoint{1.092156in}{1.361306in}}%
\pgfpathlineto{\pgfqpoint{1.193793in}{1.417275in}}%
\pgfpathlineto{\pgfqpoint{1.246687in}{1.443567in}}%
\pgfpathlineto{\pgfqpoint{1.274416in}{1.455953in}}%
\pgfpathlineto{\pgfqpoint{1.329241in}{1.477589in}}%
\pgfpathlineto{\pgfqpoint{1.385075in}{1.497022in}}%
\pgfpathlineto{\pgfqpoint{1.472867in}{1.523035in}}%
\pgfpathlineto{\pgfqpoint{1.588678in}{1.550321in}}%
\pgfpathlineto{\pgfqpoint{1.645058in}{1.561589in}}%
\pgfpathlineto{\pgfqpoint{1.726198in}{1.575341in}}%
\pgfpathlineto{\pgfqpoint{1.802248in}{1.585368in}}%
\pgfpathlineto{\pgfqpoint{1.910030in}{1.595953in}}%
\pgfpathlineto{\pgfqpoint{1.997720in}{1.600609in}}%
\pgfpathlineto{\pgfqpoint{2.024660in}{1.601665in}}%
\pgfpathlineto{\pgfqpoint{2.033126in}{1.601944in}}%
\pgfpathlineto{\pgfqpoint{2.033126in}{1.601944in}}%
\pgfusepath{stroke}%
\end{pgfscope}%
\begin{pgfscope}%
\pgfsetrectcap%
\pgfsetmiterjoin%
\pgfsetlinewidth{0.803000pt}%
\definecolor{currentstroke}{rgb}{0.000000,0.000000,0.000000}%
\pgfsetstrokecolor{currentstroke}%
\pgfsetdash{}{0pt}%
\pgfpathmoveto{\pgfqpoint{0.553581in}{0.499444in}}%
\pgfpathlineto{\pgfqpoint{0.553581in}{1.654444in}}%
\pgfusepath{stroke}%
\end{pgfscope}%
\begin{pgfscope}%
\pgfsetrectcap%
\pgfsetmiterjoin%
\pgfsetlinewidth{0.803000pt}%
\definecolor{currentstroke}{rgb}{0.000000,0.000000,0.000000}%
\pgfsetstrokecolor{currentstroke}%
\pgfsetdash{}{0pt}%
\pgfpathmoveto{\pgfqpoint{2.103581in}{0.499444in}}%
\pgfpathlineto{\pgfqpoint{2.103581in}{1.654444in}}%
\pgfusepath{stroke}%
\end{pgfscope}%
\begin{pgfscope}%
\pgfsetrectcap%
\pgfsetmiterjoin%
\pgfsetlinewidth{0.803000pt}%
\definecolor{currentstroke}{rgb}{0.000000,0.000000,0.000000}%
\pgfsetstrokecolor{currentstroke}%
\pgfsetdash{}{0pt}%
\pgfpathmoveto{\pgfqpoint{0.553581in}{0.499444in}}%
\pgfpathlineto{\pgfqpoint{2.103581in}{0.499444in}}%
\pgfusepath{stroke}%
\end{pgfscope}%
\begin{pgfscope}%
\pgfsetrectcap%
\pgfsetmiterjoin%
\pgfsetlinewidth{0.803000pt}%
\definecolor{currentstroke}{rgb}{0.000000,0.000000,0.000000}%
\pgfsetstrokecolor{currentstroke}%
\pgfsetdash{}{0pt}%
\pgfpathmoveto{\pgfqpoint{0.553581in}{1.654444in}}%
\pgfpathlineto{\pgfqpoint{2.103581in}{1.654444in}}%
\pgfusepath{stroke}%
\end{pgfscope}%
\begin{pgfscope}%
\pgfsetbuttcap%
\pgfsetmiterjoin%
\definecolor{currentfill}{rgb}{1.000000,1.000000,1.000000}%
\pgfsetfillcolor{currentfill}%
\pgfsetfillopacity{0.800000}%
\pgfsetlinewidth{1.003750pt}%
\definecolor{currentstroke}{rgb}{0.800000,0.800000,0.800000}%
\pgfsetstrokecolor{currentstroke}%
\pgfsetstrokeopacity{0.800000}%
\pgfsetdash{}{0pt}%
\pgfpathmoveto{\pgfqpoint{0.832747in}{0.568889in}}%
\pgfpathlineto{\pgfqpoint{2.006358in}{0.568889in}}%
\pgfpathquadraticcurveto{\pgfqpoint{2.034136in}{0.568889in}}{\pgfqpoint{2.034136in}{0.596666in}}%
\pgfpathlineto{\pgfqpoint{2.034136in}{0.776388in}}%
\pgfpathquadraticcurveto{\pgfqpoint{2.034136in}{0.804166in}}{\pgfqpoint{2.006358in}{0.804166in}}%
\pgfpathlineto{\pgfqpoint{0.832747in}{0.804166in}}%
\pgfpathquadraticcurveto{\pgfqpoint{0.804970in}{0.804166in}}{\pgfqpoint{0.804970in}{0.776388in}}%
\pgfpathlineto{\pgfqpoint{0.804970in}{0.596666in}}%
\pgfpathquadraticcurveto{\pgfqpoint{0.804970in}{0.568889in}}{\pgfqpoint{0.832747in}{0.568889in}}%
\pgfpathlineto{\pgfqpoint{0.832747in}{0.568889in}}%
\pgfpathclose%
\pgfusepath{stroke,fill}%
\end{pgfscope}%
\begin{pgfscope}%
\pgfsetrectcap%
\pgfsetroundjoin%
\pgfsetlinewidth{1.505625pt}%
\definecolor{currentstroke}{rgb}{0.000000,0.000000,0.000000}%
\pgfsetstrokecolor{currentstroke}%
\pgfsetdash{}{0pt}%
\pgfpathmoveto{\pgfqpoint{0.860525in}{0.700000in}}%
\pgfpathlineto{\pgfqpoint{0.999414in}{0.700000in}}%
\pgfpathlineto{\pgfqpoint{1.138303in}{0.700000in}}%
\pgfusepath{stroke}%
\end{pgfscope}%
\begin{pgfscope}%
\definecolor{textcolor}{rgb}{0.000000,0.000000,0.000000}%
\pgfsetstrokecolor{textcolor}%
\pgfsetfillcolor{textcolor}%
\pgftext[x=1.249414in,y=0.651388in,left,base]{\color{textcolor}\rmfamily\fontsize{10.000000}{12.000000}\selectfont AUC=0.799}%
\end{pgfscope}%
\end{pgfpicture}%
\makeatother%
\endgroup%

	
\end{tabular}

\verb|KBFC_Hard_Tomek_0_alpha_0_5_gamma_0_0_v1_Test|

\noindent\begin{tabular}{@{\hspace{-6pt}}p{4.3in} @{\hspace{-6pt}}p{2.0in}}
	\vskip 0pt
	\hfil Raw Model Output
	
	%% Creator: Matplotlib, PGF backend
%%
%% To include the figure in your LaTeX document, write
%%   \input{<filename>.pgf}
%%
%% Make sure the required packages are loaded in your preamble
%%   \usepackage{pgf}
%%
%% Also ensure that all the required font packages are loaded; for instance,
%% the lmodern package is sometimes necessary when using math font.
%%   \usepackage{lmodern}
%%
%% Figures using additional raster images can only be included by \input if
%% they are in the same directory as the main LaTeX file. For loading figures
%% from other directories you can use the `import` package
%%   \usepackage{import}
%%
%% and then include the figures with
%%   \import{<path to file>}{<filename>.pgf}
%%
%% Matplotlib used the following preamble
%%   
%%   \usepackage{fontspec}
%%   \makeatletter\@ifpackageloaded{underscore}{}{\usepackage[strings]{underscore}}\makeatother
%%
\begingroup%
\makeatletter%
\begin{pgfpicture}%
\pgfpathrectangle{\pgfpointorigin}{\pgfqpoint{4.102500in}{1.754444in}}%
\pgfusepath{use as bounding box, clip}%
\begin{pgfscope}%
\pgfsetbuttcap%
\pgfsetmiterjoin%
\definecolor{currentfill}{rgb}{1.000000,1.000000,1.000000}%
\pgfsetfillcolor{currentfill}%
\pgfsetlinewidth{0.000000pt}%
\definecolor{currentstroke}{rgb}{1.000000,1.000000,1.000000}%
\pgfsetstrokecolor{currentstroke}%
\pgfsetdash{}{0pt}%
\pgfpathmoveto{\pgfqpoint{0.000000in}{0.000000in}}%
\pgfpathlineto{\pgfqpoint{4.102500in}{0.000000in}}%
\pgfpathlineto{\pgfqpoint{4.102500in}{1.754444in}}%
\pgfpathlineto{\pgfqpoint{0.000000in}{1.754444in}}%
\pgfpathlineto{\pgfqpoint{0.000000in}{0.000000in}}%
\pgfpathclose%
\pgfusepath{fill}%
\end{pgfscope}%
\begin{pgfscope}%
\pgfsetbuttcap%
\pgfsetmiterjoin%
\definecolor{currentfill}{rgb}{1.000000,1.000000,1.000000}%
\pgfsetfillcolor{currentfill}%
\pgfsetlinewidth{0.000000pt}%
\definecolor{currentstroke}{rgb}{0.000000,0.000000,0.000000}%
\pgfsetstrokecolor{currentstroke}%
\pgfsetstrokeopacity{0.000000}%
\pgfsetdash{}{0pt}%
\pgfpathmoveto{\pgfqpoint{0.515000in}{0.499444in}}%
\pgfpathlineto{\pgfqpoint{4.002500in}{0.499444in}}%
\pgfpathlineto{\pgfqpoint{4.002500in}{1.654444in}}%
\pgfpathlineto{\pgfqpoint{0.515000in}{1.654444in}}%
\pgfpathlineto{\pgfqpoint{0.515000in}{0.499444in}}%
\pgfpathclose%
\pgfusepath{fill}%
\end{pgfscope}%
\begin{pgfscope}%
\pgfpathrectangle{\pgfqpoint{0.515000in}{0.499444in}}{\pgfqpoint{3.487500in}{1.155000in}}%
\pgfusepath{clip}%
\pgfsetbuttcap%
\pgfsetmiterjoin%
\pgfsetlinewidth{1.003750pt}%
\definecolor{currentstroke}{rgb}{0.000000,0.000000,0.000000}%
\pgfsetstrokecolor{currentstroke}%
\pgfsetdash{}{0pt}%
\pgfpathmoveto{\pgfqpoint{0.610114in}{0.499444in}}%
\pgfpathlineto{\pgfqpoint{0.673523in}{0.499444in}}%
\pgfpathlineto{\pgfqpoint{0.673523in}{0.499444in}}%
\pgfpathlineto{\pgfqpoint{0.610114in}{0.499444in}}%
\pgfpathlineto{\pgfqpoint{0.610114in}{0.499444in}}%
\pgfpathclose%
\pgfusepath{stroke}%
\end{pgfscope}%
\begin{pgfscope}%
\pgfpathrectangle{\pgfqpoint{0.515000in}{0.499444in}}{\pgfqpoint{3.487500in}{1.155000in}}%
\pgfusepath{clip}%
\pgfsetbuttcap%
\pgfsetmiterjoin%
\pgfsetlinewidth{1.003750pt}%
\definecolor{currentstroke}{rgb}{0.000000,0.000000,0.000000}%
\pgfsetstrokecolor{currentstroke}%
\pgfsetdash{}{0pt}%
\pgfpathmoveto{\pgfqpoint{0.768637in}{0.499444in}}%
\pgfpathlineto{\pgfqpoint{0.832046in}{0.499444in}}%
\pgfpathlineto{\pgfqpoint{0.832046in}{1.599444in}}%
\pgfpathlineto{\pgfqpoint{0.768637in}{1.599444in}}%
\pgfpathlineto{\pgfqpoint{0.768637in}{0.499444in}}%
\pgfpathclose%
\pgfusepath{stroke}%
\end{pgfscope}%
\begin{pgfscope}%
\pgfpathrectangle{\pgfqpoint{0.515000in}{0.499444in}}{\pgfqpoint{3.487500in}{1.155000in}}%
\pgfusepath{clip}%
\pgfsetbuttcap%
\pgfsetmiterjoin%
\pgfsetlinewidth{1.003750pt}%
\definecolor{currentstroke}{rgb}{0.000000,0.000000,0.000000}%
\pgfsetstrokecolor{currentstroke}%
\pgfsetdash{}{0pt}%
\pgfpathmoveto{\pgfqpoint{0.927159in}{0.499444in}}%
\pgfpathlineto{\pgfqpoint{0.990568in}{0.499444in}}%
\pgfpathlineto{\pgfqpoint{0.990568in}{1.346951in}}%
\pgfpathlineto{\pgfqpoint{0.927159in}{1.346951in}}%
\pgfpathlineto{\pgfqpoint{0.927159in}{0.499444in}}%
\pgfpathclose%
\pgfusepath{stroke}%
\end{pgfscope}%
\begin{pgfscope}%
\pgfpathrectangle{\pgfqpoint{0.515000in}{0.499444in}}{\pgfqpoint{3.487500in}{1.155000in}}%
\pgfusepath{clip}%
\pgfsetbuttcap%
\pgfsetmiterjoin%
\pgfsetlinewidth{1.003750pt}%
\definecolor{currentstroke}{rgb}{0.000000,0.000000,0.000000}%
\pgfsetstrokecolor{currentstroke}%
\pgfsetdash{}{0pt}%
\pgfpathmoveto{\pgfqpoint{1.085682in}{0.499444in}}%
\pgfpathlineto{\pgfqpoint{1.149091in}{0.499444in}}%
\pgfpathlineto{\pgfqpoint{1.149091in}{1.032251in}}%
\pgfpathlineto{\pgfqpoint{1.085682in}{1.032251in}}%
\pgfpathlineto{\pgfqpoint{1.085682in}{0.499444in}}%
\pgfpathclose%
\pgfusepath{stroke}%
\end{pgfscope}%
\begin{pgfscope}%
\pgfpathrectangle{\pgfqpoint{0.515000in}{0.499444in}}{\pgfqpoint{3.487500in}{1.155000in}}%
\pgfusepath{clip}%
\pgfsetbuttcap%
\pgfsetmiterjoin%
\pgfsetlinewidth{1.003750pt}%
\definecolor{currentstroke}{rgb}{0.000000,0.000000,0.000000}%
\pgfsetstrokecolor{currentstroke}%
\pgfsetdash{}{0pt}%
\pgfpathmoveto{\pgfqpoint{1.244205in}{0.499444in}}%
\pgfpathlineto{\pgfqpoint{1.307614in}{0.499444in}}%
\pgfpathlineto{\pgfqpoint{1.307614in}{0.848431in}}%
\pgfpathlineto{\pgfqpoint{1.244205in}{0.848431in}}%
\pgfpathlineto{\pgfqpoint{1.244205in}{0.499444in}}%
\pgfpathclose%
\pgfusepath{stroke}%
\end{pgfscope}%
\begin{pgfscope}%
\pgfpathrectangle{\pgfqpoint{0.515000in}{0.499444in}}{\pgfqpoint{3.487500in}{1.155000in}}%
\pgfusepath{clip}%
\pgfsetbuttcap%
\pgfsetmiterjoin%
\pgfsetlinewidth{1.003750pt}%
\definecolor{currentstroke}{rgb}{0.000000,0.000000,0.000000}%
\pgfsetstrokecolor{currentstroke}%
\pgfsetdash{}{0pt}%
\pgfpathmoveto{\pgfqpoint{1.402728in}{0.499444in}}%
\pgfpathlineto{\pgfqpoint{1.466137in}{0.499444in}}%
\pgfpathlineto{\pgfqpoint{1.466137in}{0.740608in}}%
\pgfpathlineto{\pgfqpoint{1.402728in}{0.740608in}}%
\pgfpathlineto{\pgfqpoint{1.402728in}{0.499444in}}%
\pgfpathclose%
\pgfusepath{stroke}%
\end{pgfscope}%
\begin{pgfscope}%
\pgfpathrectangle{\pgfqpoint{0.515000in}{0.499444in}}{\pgfqpoint{3.487500in}{1.155000in}}%
\pgfusepath{clip}%
\pgfsetbuttcap%
\pgfsetmiterjoin%
\pgfsetlinewidth{1.003750pt}%
\definecolor{currentstroke}{rgb}{0.000000,0.000000,0.000000}%
\pgfsetstrokecolor{currentstroke}%
\pgfsetdash{}{0pt}%
\pgfpathmoveto{\pgfqpoint{1.561250in}{0.499444in}}%
\pgfpathlineto{\pgfqpoint{1.624659in}{0.499444in}}%
\pgfpathlineto{\pgfqpoint{1.624659in}{0.660907in}}%
\pgfpathlineto{\pgfqpoint{1.561250in}{0.660907in}}%
\pgfpathlineto{\pgfqpoint{1.561250in}{0.499444in}}%
\pgfpathclose%
\pgfusepath{stroke}%
\end{pgfscope}%
\begin{pgfscope}%
\pgfpathrectangle{\pgfqpoint{0.515000in}{0.499444in}}{\pgfqpoint{3.487500in}{1.155000in}}%
\pgfusepath{clip}%
\pgfsetbuttcap%
\pgfsetmiterjoin%
\pgfsetlinewidth{1.003750pt}%
\definecolor{currentstroke}{rgb}{0.000000,0.000000,0.000000}%
\pgfsetstrokecolor{currentstroke}%
\pgfsetdash{}{0pt}%
\pgfpathmoveto{\pgfqpoint{1.719773in}{0.499444in}}%
\pgfpathlineto{\pgfqpoint{1.783182in}{0.499444in}}%
\pgfpathlineto{\pgfqpoint{1.783182in}{0.611530in}}%
\pgfpathlineto{\pgfqpoint{1.719773in}{0.611530in}}%
\pgfpathlineto{\pgfqpoint{1.719773in}{0.499444in}}%
\pgfpathclose%
\pgfusepath{stroke}%
\end{pgfscope}%
\begin{pgfscope}%
\pgfpathrectangle{\pgfqpoint{0.515000in}{0.499444in}}{\pgfqpoint{3.487500in}{1.155000in}}%
\pgfusepath{clip}%
\pgfsetbuttcap%
\pgfsetmiterjoin%
\pgfsetlinewidth{1.003750pt}%
\definecolor{currentstroke}{rgb}{0.000000,0.000000,0.000000}%
\pgfsetstrokecolor{currentstroke}%
\pgfsetdash{}{0pt}%
\pgfpathmoveto{\pgfqpoint{1.878296in}{0.499444in}}%
\pgfpathlineto{\pgfqpoint{1.941705in}{0.499444in}}%
\pgfpathlineto{\pgfqpoint{1.941705in}{0.577063in}}%
\pgfpathlineto{\pgfqpoint{1.878296in}{0.577063in}}%
\pgfpathlineto{\pgfqpoint{1.878296in}{0.499444in}}%
\pgfpathclose%
\pgfusepath{stroke}%
\end{pgfscope}%
\begin{pgfscope}%
\pgfpathrectangle{\pgfqpoint{0.515000in}{0.499444in}}{\pgfqpoint{3.487500in}{1.155000in}}%
\pgfusepath{clip}%
\pgfsetbuttcap%
\pgfsetmiterjoin%
\pgfsetlinewidth{1.003750pt}%
\definecolor{currentstroke}{rgb}{0.000000,0.000000,0.000000}%
\pgfsetstrokecolor{currentstroke}%
\pgfsetdash{}{0pt}%
\pgfpathmoveto{\pgfqpoint{2.036818in}{0.499444in}}%
\pgfpathlineto{\pgfqpoint{2.100228in}{0.499444in}}%
\pgfpathlineto{\pgfqpoint{2.100228in}{0.554326in}}%
\pgfpathlineto{\pgfqpoint{2.036818in}{0.554326in}}%
\pgfpathlineto{\pgfqpoint{2.036818in}{0.499444in}}%
\pgfpathclose%
\pgfusepath{stroke}%
\end{pgfscope}%
\begin{pgfscope}%
\pgfpathrectangle{\pgfqpoint{0.515000in}{0.499444in}}{\pgfqpoint{3.487500in}{1.155000in}}%
\pgfusepath{clip}%
\pgfsetbuttcap%
\pgfsetmiterjoin%
\pgfsetlinewidth{1.003750pt}%
\definecolor{currentstroke}{rgb}{0.000000,0.000000,0.000000}%
\pgfsetstrokecolor{currentstroke}%
\pgfsetdash{}{0pt}%
\pgfpathmoveto{\pgfqpoint{2.195341in}{0.499444in}}%
\pgfpathlineto{\pgfqpoint{2.258750in}{0.499444in}}%
\pgfpathlineto{\pgfqpoint{2.258750in}{0.536953in}}%
\pgfpathlineto{\pgfqpoint{2.195341in}{0.536953in}}%
\pgfpathlineto{\pgfqpoint{2.195341in}{0.499444in}}%
\pgfpathclose%
\pgfusepath{stroke}%
\end{pgfscope}%
\begin{pgfscope}%
\pgfpathrectangle{\pgfqpoint{0.515000in}{0.499444in}}{\pgfqpoint{3.487500in}{1.155000in}}%
\pgfusepath{clip}%
\pgfsetbuttcap%
\pgfsetmiterjoin%
\pgfsetlinewidth{1.003750pt}%
\definecolor{currentstroke}{rgb}{0.000000,0.000000,0.000000}%
\pgfsetstrokecolor{currentstroke}%
\pgfsetdash{}{0pt}%
\pgfpathmoveto{\pgfqpoint{2.353864in}{0.499444in}}%
\pgfpathlineto{\pgfqpoint{2.417273in}{0.499444in}}%
\pgfpathlineto{\pgfqpoint{2.417273in}{0.528186in}}%
\pgfpathlineto{\pgfqpoint{2.353864in}{0.528186in}}%
\pgfpathlineto{\pgfqpoint{2.353864in}{0.499444in}}%
\pgfpathclose%
\pgfusepath{stroke}%
\end{pgfscope}%
\begin{pgfscope}%
\pgfpathrectangle{\pgfqpoint{0.515000in}{0.499444in}}{\pgfqpoint{3.487500in}{1.155000in}}%
\pgfusepath{clip}%
\pgfsetbuttcap%
\pgfsetmiterjoin%
\pgfsetlinewidth{1.003750pt}%
\definecolor{currentstroke}{rgb}{0.000000,0.000000,0.000000}%
\pgfsetstrokecolor{currentstroke}%
\pgfsetdash{}{0pt}%
\pgfpathmoveto{\pgfqpoint{2.512387in}{0.499444in}}%
\pgfpathlineto{\pgfqpoint{2.575796in}{0.499444in}}%
\pgfpathlineto{\pgfqpoint{2.575796in}{0.519540in}}%
\pgfpathlineto{\pgfqpoint{2.512387in}{0.519540in}}%
\pgfpathlineto{\pgfqpoint{2.512387in}{0.499444in}}%
\pgfpathclose%
\pgfusepath{stroke}%
\end{pgfscope}%
\begin{pgfscope}%
\pgfpathrectangle{\pgfqpoint{0.515000in}{0.499444in}}{\pgfqpoint{3.487500in}{1.155000in}}%
\pgfusepath{clip}%
\pgfsetbuttcap%
\pgfsetmiterjoin%
\pgfsetlinewidth{1.003750pt}%
\definecolor{currentstroke}{rgb}{0.000000,0.000000,0.000000}%
\pgfsetstrokecolor{currentstroke}%
\pgfsetdash{}{0pt}%
\pgfpathmoveto{\pgfqpoint{2.670909in}{0.499444in}}%
\pgfpathlineto{\pgfqpoint{2.734318in}{0.499444in}}%
\pgfpathlineto{\pgfqpoint{2.734318in}{0.513475in}}%
\pgfpathlineto{\pgfqpoint{2.670909in}{0.513475in}}%
\pgfpathlineto{\pgfqpoint{2.670909in}{0.499444in}}%
\pgfpathclose%
\pgfusepath{stroke}%
\end{pgfscope}%
\begin{pgfscope}%
\pgfpathrectangle{\pgfqpoint{0.515000in}{0.499444in}}{\pgfqpoint{3.487500in}{1.155000in}}%
\pgfusepath{clip}%
\pgfsetbuttcap%
\pgfsetmiterjoin%
\pgfsetlinewidth{1.003750pt}%
\definecolor{currentstroke}{rgb}{0.000000,0.000000,0.000000}%
\pgfsetstrokecolor{currentstroke}%
\pgfsetdash{}{0pt}%
\pgfpathmoveto{\pgfqpoint{2.829432in}{0.499444in}}%
\pgfpathlineto{\pgfqpoint{2.892841in}{0.499444in}}%
\pgfpathlineto{\pgfqpoint{2.892841in}{0.509932in}}%
\pgfpathlineto{\pgfqpoint{2.829432in}{0.509932in}}%
\pgfpathlineto{\pgfqpoint{2.829432in}{0.499444in}}%
\pgfpathclose%
\pgfusepath{stroke}%
\end{pgfscope}%
\begin{pgfscope}%
\pgfpathrectangle{\pgfqpoint{0.515000in}{0.499444in}}{\pgfqpoint{3.487500in}{1.155000in}}%
\pgfusepath{clip}%
\pgfsetbuttcap%
\pgfsetmiterjoin%
\pgfsetlinewidth{1.003750pt}%
\definecolor{currentstroke}{rgb}{0.000000,0.000000,0.000000}%
\pgfsetstrokecolor{currentstroke}%
\pgfsetdash{}{0pt}%
\pgfpathmoveto{\pgfqpoint{2.987955in}{0.499444in}}%
\pgfpathlineto{\pgfqpoint{3.051364in}{0.499444in}}%
\pgfpathlineto{\pgfqpoint{3.051364in}{0.507630in}}%
\pgfpathlineto{\pgfqpoint{2.987955in}{0.507630in}}%
\pgfpathlineto{\pgfqpoint{2.987955in}{0.499444in}}%
\pgfpathclose%
\pgfusepath{stroke}%
\end{pgfscope}%
\begin{pgfscope}%
\pgfpathrectangle{\pgfqpoint{0.515000in}{0.499444in}}{\pgfqpoint{3.487500in}{1.155000in}}%
\pgfusepath{clip}%
\pgfsetbuttcap%
\pgfsetmiterjoin%
\pgfsetlinewidth{1.003750pt}%
\definecolor{currentstroke}{rgb}{0.000000,0.000000,0.000000}%
\pgfsetstrokecolor{currentstroke}%
\pgfsetdash{}{0pt}%
\pgfpathmoveto{\pgfqpoint{3.146478in}{0.499444in}}%
\pgfpathlineto{\pgfqpoint{3.209887in}{0.499444in}}%
\pgfpathlineto{\pgfqpoint{3.209887in}{0.505709in}}%
\pgfpathlineto{\pgfqpoint{3.146478in}{0.505709in}}%
\pgfpathlineto{\pgfqpoint{3.146478in}{0.499444in}}%
\pgfpathclose%
\pgfusepath{stroke}%
\end{pgfscope}%
\begin{pgfscope}%
\pgfpathrectangle{\pgfqpoint{0.515000in}{0.499444in}}{\pgfqpoint{3.487500in}{1.155000in}}%
\pgfusepath{clip}%
\pgfsetbuttcap%
\pgfsetmiterjoin%
\pgfsetlinewidth{1.003750pt}%
\definecolor{currentstroke}{rgb}{0.000000,0.000000,0.000000}%
\pgfsetstrokecolor{currentstroke}%
\pgfsetdash{}{0pt}%
\pgfpathmoveto{\pgfqpoint{3.305000in}{0.499444in}}%
\pgfpathlineto{\pgfqpoint{3.368409in}{0.499444in}}%
\pgfpathlineto{\pgfqpoint{3.368409in}{0.503147in}}%
\pgfpathlineto{\pgfqpoint{3.305000in}{0.503147in}}%
\pgfpathlineto{\pgfqpoint{3.305000in}{0.499444in}}%
\pgfpathclose%
\pgfusepath{stroke}%
\end{pgfscope}%
\begin{pgfscope}%
\pgfpathrectangle{\pgfqpoint{0.515000in}{0.499444in}}{\pgfqpoint{3.487500in}{1.155000in}}%
\pgfusepath{clip}%
\pgfsetbuttcap%
\pgfsetmiterjoin%
\pgfsetlinewidth{1.003750pt}%
\definecolor{currentstroke}{rgb}{0.000000,0.000000,0.000000}%
\pgfsetstrokecolor{currentstroke}%
\pgfsetdash{}{0pt}%
\pgfpathmoveto{\pgfqpoint{3.463523in}{0.499444in}}%
\pgfpathlineto{\pgfqpoint{3.526932in}{0.499444in}}%
\pgfpathlineto{\pgfqpoint{3.526932in}{0.501266in}}%
\pgfpathlineto{\pgfqpoint{3.463523in}{0.501266in}}%
\pgfpathlineto{\pgfqpoint{3.463523in}{0.499444in}}%
\pgfpathclose%
\pgfusepath{stroke}%
\end{pgfscope}%
\begin{pgfscope}%
\pgfpathrectangle{\pgfqpoint{0.515000in}{0.499444in}}{\pgfqpoint{3.487500in}{1.155000in}}%
\pgfusepath{clip}%
\pgfsetbuttcap%
\pgfsetmiterjoin%
\pgfsetlinewidth{1.003750pt}%
\definecolor{currentstroke}{rgb}{0.000000,0.000000,0.000000}%
\pgfsetstrokecolor{currentstroke}%
\pgfsetdash{}{0pt}%
\pgfpathmoveto{\pgfqpoint{3.622046in}{0.499444in}}%
\pgfpathlineto{\pgfqpoint{3.685455in}{0.499444in}}%
\pgfpathlineto{\pgfqpoint{3.685455in}{0.499744in}}%
\pgfpathlineto{\pgfqpoint{3.622046in}{0.499744in}}%
\pgfpathlineto{\pgfqpoint{3.622046in}{0.499444in}}%
\pgfpathclose%
\pgfusepath{stroke}%
\end{pgfscope}%
\begin{pgfscope}%
\pgfpathrectangle{\pgfqpoint{0.515000in}{0.499444in}}{\pgfqpoint{3.487500in}{1.155000in}}%
\pgfusepath{clip}%
\pgfsetbuttcap%
\pgfsetmiterjoin%
\pgfsetlinewidth{1.003750pt}%
\definecolor{currentstroke}{rgb}{0.000000,0.000000,0.000000}%
\pgfsetstrokecolor{currentstroke}%
\pgfsetdash{}{0pt}%
\pgfpathmoveto{\pgfqpoint{3.780568in}{0.499444in}}%
\pgfpathlineto{\pgfqpoint{3.843978in}{0.499444in}}%
\pgfpathlineto{\pgfqpoint{3.843978in}{0.499444in}}%
\pgfpathlineto{\pgfqpoint{3.780568in}{0.499444in}}%
\pgfpathlineto{\pgfqpoint{3.780568in}{0.499444in}}%
\pgfpathclose%
\pgfusepath{stroke}%
\end{pgfscope}%
\begin{pgfscope}%
\pgfpathrectangle{\pgfqpoint{0.515000in}{0.499444in}}{\pgfqpoint{3.487500in}{1.155000in}}%
\pgfusepath{clip}%
\pgfsetbuttcap%
\pgfsetmiterjoin%
\definecolor{currentfill}{rgb}{0.000000,0.000000,0.000000}%
\pgfsetfillcolor{currentfill}%
\pgfsetlinewidth{0.000000pt}%
\definecolor{currentstroke}{rgb}{0.000000,0.000000,0.000000}%
\pgfsetstrokecolor{currentstroke}%
\pgfsetstrokeopacity{0.000000}%
\pgfsetdash{}{0pt}%
\pgfpathmoveto{\pgfqpoint{0.673523in}{0.499444in}}%
\pgfpathlineto{\pgfqpoint{0.736932in}{0.499444in}}%
\pgfpathlineto{\pgfqpoint{0.736932in}{0.499444in}}%
\pgfpathlineto{\pgfqpoint{0.673523in}{0.499444in}}%
\pgfpathlineto{\pgfqpoint{0.673523in}{0.499444in}}%
\pgfpathclose%
\pgfusepath{fill}%
\end{pgfscope}%
\begin{pgfscope}%
\pgfpathrectangle{\pgfqpoint{0.515000in}{0.499444in}}{\pgfqpoint{3.487500in}{1.155000in}}%
\pgfusepath{clip}%
\pgfsetbuttcap%
\pgfsetmiterjoin%
\definecolor{currentfill}{rgb}{0.000000,0.000000,0.000000}%
\pgfsetfillcolor{currentfill}%
\pgfsetlinewidth{0.000000pt}%
\definecolor{currentstroke}{rgb}{0.000000,0.000000,0.000000}%
\pgfsetstrokecolor{currentstroke}%
\pgfsetstrokeopacity{0.000000}%
\pgfsetdash{}{0pt}%
\pgfpathmoveto{\pgfqpoint{0.832046in}{0.499444in}}%
\pgfpathlineto{\pgfqpoint{0.895455in}{0.499444in}}%
\pgfpathlineto{\pgfqpoint{0.895455in}{0.535532in}}%
\pgfpathlineto{\pgfqpoint{0.832046in}{0.535532in}}%
\pgfpathlineto{\pgfqpoint{0.832046in}{0.499444in}}%
\pgfpathclose%
\pgfusepath{fill}%
\end{pgfscope}%
\begin{pgfscope}%
\pgfpathrectangle{\pgfqpoint{0.515000in}{0.499444in}}{\pgfqpoint{3.487500in}{1.155000in}}%
\pgfusepath{clip}%
\pgfsetbuttcap%
\pgfsetmiterjoin%
\definecolor{currentfill}{rgb}{0.000000,0.000000,0.000000}%
\pgfsetfillcolor{currentfill}%
\pgfsetlinewidth{0.000000pt}%
\definecolor{currentstroke}{rgb}{0.000000,0.000000,0.000000}%
\pgfsetstrokecolor{currentstroke}%
\pgfsetstrokeopacity{0.000000}%
\pgfsetdash{}{0pt}%
\pgfpathmoveto{\pgfqpoint{0.990568in}{0.499444in}}%
\pgfpathlineto{\pgfqpoint{1.053978in}{0.499444in}}%
\pgfpathlineto{\pgfqpoint{1.053978in}{0.575602in}}%
\pgfpathlineto{\pgfqpoint{0.990568in}{0.575602in}}%
\pgfpathlineto{\pgfqpoint{0.990568in}{0.499444in}}%
\pgfpathclose%
\pgfusepath{fill}%
\end{pgfscope}%
\begin{pgfscope}%
\pgfpathrectangle{\pgfqpoint{0.515000in}{0.499444in}}{\pgfqpoint{3.487500in}{1.155000in}}%
\pgfusepath{clip}%
\pgfsetbuttcap%
\pgfsetmiterjoin%
\definecolor{currentfill}{rgb}{0.000000,0.000000,0.000000}%
\pgfsetfillcolor{currentfill}%
\pgfsetlinewidth{0.000000pt}%
\definecolor{currentstroke}{rgb}{0.000000,0.000000,0.000000}%
\pgfsetstrokecolor{currentstroke}%
\pgfsetstrokeopacity{0.000000}%
\pgfsetdash{}{0pt}%
\pgfpathmoveto{\pgfqpoint{1.149091in}{0.499444in}}%
\pgfpathlineto{\pgfqpoint{1.212500in}{0.499444in}}%
\pgfpathlineto{\pgfqpoint{1.212500in}{0.576243in}}%
\pgfpathlineto{\pgfqpoint{1.149091in}{0.576243in}}%
\pgfpathlineto{\pgfqpoint{1.149091in}{0.499444in}}%
\pgfpathclose%
\pgfusepath{fill}%
\end{pgfscope}%
\begin{pgfscope}%
\pgfpathrectangle{\pgfqpoint{0.515000in}{0.499444in}}{\pgfqpoint{3.487500in}{1.155000in}}%
\pgfusepath{clip}%
\pgfsetbuttcap%
\pgfsetmiterjoin%
\definecolor{currentfill}{rgb}{0.000000,0.000000,0.000000}%
\pgfsetfillcolor{currentfill}%
\pgfsetlinewidth{0.000000pt}%
\definecolor{currentstroke}{rgb}{0.000000,0.000000,0.000000}%
\pgfsetstrokecolor{currentstroke}%
\pgfsetstrokeopacity{0.000000}%
\pgfsetdash{}{0pt}%
\pgfpathmoveto{\pgfqpoint{1.307614in}{0.499444in}}%
\pgfpathlineto{\pgfqpoint{1.371023in}{0.499444in}}%
\pgfpathlineto{\pgfqpoint{1.371023in}{0.571940in}}%
\pgfpathlineto{\pgfqpoint{1.307614in}{0.571940in}}%
\pgfpathlineto{\pgfqpoint{1.307614in}{0.499444in}}%
\pgfpathclose%
\pgfusepath{fill}%
\end{pgfscope}%
\begin{pgfscope}%
\pgfpathrectangle{\pgfqpoint{0.515000in}{0.499444in}}{\pgfqpoint{3.487500in}{1.155000in}}%
\pgfusepath{clip}%
\pgfsetbuttcap%
\pgfsetmiterjoin%
\definecolor{currentfill}{rgb}{0.000000,0.000000,0.000000}%
\pgfsetfillcolor{currentfill}%
\pgfsetlinewidth{0.000000pt}%
\definecolor{currentstroke}{rgb}{0.000000,0.000000,0.000000}%
\pgfsetstrokecolor{currentstroke}%
\pgfsetstrokeopacity{0.000000}%
\pgfsetdash{}{0pt}%
\pgfpathmoveto{\pgfqpoint{1.466137in}{0.499444in}}%
\pgfpathlineto{\pgfqpoint{1.529546in}{0.499444in}}%
\pgfpathlineto{\pgfqpoint{1.529546in}{0.564794in}}%
\pgfpathlineto{\pgfqpoint{1.466137in}{0.564794in}}%
\pgfpathlineto{\pgfqpoint{1.466137in}{0.499444in}}%
\pgfpathclose%
\pgfusepath{fill}%
\end{pgfscope}%
\begin{pgfscope}%
\pgfpathrectangle{\pgfqpoint{0.515000in}{0.499444in}}{\pgfqpoint{3.487500in}{1.155000in}}%
\pgfusepath{clip}%
\pgfsetbuttcap%
\pgfsetmiterjoin%
\definecolor{currentfill}{rgb}{0.000000,0.000000,0.000000}%
\pgfsetfillcolor{currentfill}%
\pgfsetlinewidth{0.000000pt}%
\definecolor{currentstroke}{rgb}{0.000000,0.000000,0.000000}%
\pgfsetstrokecolor{currentstroke}%
\pgfsetstrokeopacity{0.000000}%
\pgfsetdash{}{0pt}%
\pgfpathmoveto{\pgfqpoint{1.624659in}{0.499444in}}%
\pgfpathlineto{\pgfqpoint{1.688068in}{0.499444in}}%
\pgfpathlineto{\pgfqpoint{1.688068in}{0.557128in}}%
\pgfpathlineto{\pgfqpoint{1.624659in}{0.557128in}}%
\pgfpathlineto{\pgfqpoint{1.624659in}{0.499444in}}%
\pgfpathclose%
\pgfusepath{fill}%
\end{pgfscope}%
\begin{pgfscope}%
\pgfpathrectangle{\pgfqpoint{0.515000in}{0.499444in}}{\pgfqpoint{3.487500in}{1.155000in}}%
\pgfusepath{clip}%
\pgfsetbuttcap%
\pgfsetmiterjoin%
\definecolor{currentfill}{rgb}{0.000000,0.000000,0.000000}%
\pgfsetfillcolor{currentfill}%
\pgfsetlinewidth{0.000000pt}%
\definecolor{currentstroke}{rgb}{0.000000,0.000000,0.000000}%
\pgfsetstrokecolor{currentstroke}%
\pgfsetstrokeopacity{0.000000}%
\pgfsetdash{}{0pt}%
\pgfpathmoveto{\pgfqpoint{1.783182in}{0.499444in}}%
\pgfpathlineto{\pgfqpoint{1.846591in}{0.499444in}}%
\pgfpathlineto{\pgfqpoint{1.846591in}{0.548922in}}%
\pgfpathlineto{\pgfqpoint{1.783182in}{0.548922in}}%
\pgfpathlineto{\pgfqpoint{1.783182in}{0.499444in}}%
\pgfpathclose%
\pgfusepath{fill}%
\end{pgfscope}%
\begin{pgfscope}%
\pgfpathrectangle{\pgfqpoint{0.515000in}{0.499444in}}{\pgfqpoint{3.487500in}{1.155000in}}%
\pgfusepath{clip}%
\pgfsetbuttcap%
\pgfsetmiterjoin%
\definecolor{currentfill}{rgb}{0.000000,0.000000,0.000000}%
\pgfsetfillcolor{currentfill}%
\pgfsetlinewidth{0.000000pt}%
\definecolor{currentstroke}{rgb}{0.000000,0.000000,0.000000}%
\pgfsetstrokecolor{currentstroke}%
\pgfsetstrokeopacity{0.000000}%
\pgfsetdash{}{0pt}%
\pgfpathmoveto{\pgfqpoint{1.941705in}{0.499444in}}%
\pgfpathlineto{\pgfqpoint{2.005114in}{0.499444in}}%
\pgfpathlineto{\pgfqpoint{2.005114in}{0.542597in}}%
\pgfpathlineto{\pgfqpoint{1.941705in}{0.542597in}}%
\pgfpathlineto{\pgfqpoint{1.941705in}{0.499444in}}%
\pgfpathclose%
\pgfusepath{fill}%
\end{pgfscope}%
\begin{pgfscope}%
\pgfpathrectangle{\pgfqpoint{0.515000in}{0.499444in}}{\pgfqpoint{3.487500in}{1.155000in}}%
\pgfusepath{clip}%
\pgfsetbuttcap%
\pgfsetmiterjoin%
\definecolor{currentfill}{rgb}{0.000000,0.000000,0.000000}%
\pgfsetfillcolor{currentfill}%
\pgfsetlinewidth{0.000000pt}%
\definecolor{currentstroke}{rgb}{0.000000,0.000000,0.000000}%
\pgfsetstrokecolor{currentstroke}%
\pgfsetstrokeopacity{0.000000}%
\pgfsetdash{}{0pt}%
\pgfpathmoveto{\pgfqpoint{2.100228in}{0.499444in}}%
\pgfpathlineto{\pgfqpoint{2.163637in}{0.499444in}}%
\pgfpathlineto{\pgfqpoint{2.163637in}{0.534451in}}%
\pgfpathlineto{\pgfqpoint{2.100228in}{0.534451in}}%
\pgfpathlineto{\pgfqpoint{2.100228in}{0.499444in}}%
\pgfpathclose%
\pgfusepath{fill}%
\end{pgfscope}%
\begin{pgfscope}%
\pgfpathrectangle{\pgfqpoint{0.515000in}{0.499444in}}{\pgfqpoint{3.487500in}{1.155000in}}%
\pgfusepath{clip}%
\pgfsetbuttcap%
\pgfsetmiterjoin%
\definecolor{currentfill}{rgb}{0.000000,0.000000,0.000000}%
\pgfsetfillcolor{currentfill}%
\pgfsetlinewidth{0.000000pt}%
\definecolor{currentstroke}{rgb}{0.000000,0.000000,0.000000}%
\pgfsetstrokecolor{currentstroke}%
\pgfsetstrokeopacity{0.000000}%
\pgfsetdash{}{0pt}%
\pgfpathmoveto{\pgfqpoint{2.258750in}{0.499444in}}%
\pgfpathlineto{\pgfqpoint{2.322159in}{0.499444in}}%
\pgfpathlineto{\pgfqpoint{2.322159in}{0.528907in}}%
\pgfpathlineto{\pgfqpoint{2.258750in}{0.528907in}}%
\pgfpathlineto{\pgfqpoint{2.258750in}{0.499444in}}%
\pgfpathclose%
\pgfusepath{fill}%
\end{pgfscope}%
\begin{pgfscope}%
\pgfpathrectangle{\pgfqpoint{0.515000in}{0.499444in}}{\pgfqpoint{3.487500in}{1.155000in}}%
\pgfusepath{clip}%
\pgfsetbuttcap%
\pgfsetmiterjoin%
\definecolor{currentfill}{rgb}{0.000000,0.000000,0.000000}%
\pgfsetfillcolor{currentfill}%
\pgfsetlinewidth{0.000000pt}%
\definecolor{currentstroke}{rgb}{0.000000,0.000000,0.000000}%
\pgfsetstrokecolor{currentstroke}%
\pgfsetstrokeopacity{0.000000}%
\pgfsetdash{}{0pt}%
\pgfpathmoveto{\pgfqpoint{2.417273in}{0.499444in}}%
\pgfpathlineto{\pgfqpoint{2.480682in}{0.499444in}}%
\pgfpathlineto{\pgfqpoint{2.480682in}{0.525304in}}%
\pgfpathlineto{\pgfqpoint{2.417273in}{0.525304in}}%
\pgfpathlineto{\pgfqpoint{2.417273in}{0.499444in}}%
\pgfpathclose%
\pgfusepath{fill}%
\end{pgfscope}%
\begin{pgfscope}%
\pgfpathrectangle{\pgfqpoint{0.515000in}{0.499444in}}{\pgfqpoint{3.487500in}{1.155000in}}%
\pgfusepath{clip}%
\pgfsetbuttcap%
\pgfsetmiterjoin%
\definecolor{currentfill}{rgb}{0.000000,0.000000,0.000000}%
\pgfsetfillcolor{currentfill}%
\pgfsetlinewidth{0.000000pt}%
\definecolor{currentstroke}{rgb}{0.000000,0.000000,0.000000}%
\pgfsetstrokecolor{currentstroke}%
\pgfsetstrokeopacity{0.000000}%
\pgfsetdash{}{0pt}%
\pgfpathmoveto{\pgfqpoint{2.575796in}{0.499444in}}%
\pgfpathlineto{\pgfqpoint{2.639205in}{0.499444in}}%
\pgfpathlineto{\pgfqpoint{2.639205in}{0.521501in}}%
\pgfpathlineto{\pgfqpoint{2.575796in}{0.521501in}}%
\pgfpathlineto{\pgfqpoint{2.575796in}{0.499444in}}%
\pgfpathclose%
\pgfusepath{fill}%
\end{pgfscope}%
\begin{pgfscope}%
\pgfpathrectangle{\pgfqpoint{0.515000in}{0.499444in}}{\pgfqpoint{3.487500in}{1.155000in}}%
\pgfusepath{clip}%
\pgfsetbuttcap%
\pgfsetmiterjoin%
\definecolor{currentfill}{rgb}{0.000000,0.000000,0.000000}%
\pgfsetfillcolor{currentfill}%
\pgfsetlinewidth{0.000000pt}%
\definecolor{currentstroke}{rgb}{0.000000,0.000000,0.000000}%
\pgfsetstrokecolor{currentstroke}%
\pgfsetstrokeopacity{0.000000}%
\pgfsetdash{}{0pt}%
\pgfpathmoveto{\pgfqpoint{2.734318in}{0.499444in}}%
\pgfpathlineto{\pgfqpoint{2.797728in}{0.499444in}}%
\pgfpathlineto{\pgfqpoint{2.797728in}{0.518299in}}%
\pgfpathlineto{\pgfqpoint{2.734318in}{0.518299in}}%
\pgfpathlineto{\pgfqpoint{2.734318in}{0.499444in}}%
\pgfpathclose%
\pgfusepath{fill}%
\end{pgfscope}%
\begin{pgfscope}%
\pgfpathrectangle{\pgfqpoint{0.515000in}{0.499444in}}{\pgfqpoint{3.487500in}{1.155000in}}%
\pgfusepath{clip}%
\pgfsetbuttcap%
\pgfsetmiterjoin%
\definecolor{currentfill}{rgb}{0.000000,0.000000,0.000000}%
\pgfsetfillcolor{currentfill}%
\pgfsetlinewidth{0.000000pt}%
\definecolor{currentstroke}{rgb}{0.000000,0.000000,0.000000}%
\pgfsetstrokecolor{currentstroke}%
\pgfsetstrokeopacity{0.000000}%
\pgfsetdash{}{0pt}%
\pgfpathmoveto{\pgfqpoint{2.892841in}{0.499444in}}%
\pgfpathlineto{\pgfqpoint{2.956250in}{0.499444in}}%
\pgfpathlineto{\pgfqpoint{2.956250in}{0.517418in}}%
\pgfpathlineto{\pgfqpoint{2.892841in}{0.517418in}}%
\pgfpathlineto{\pgfqpoint{2.892841in}{0.499444in}}%
\pgfpathclose%
\pgfusepath{fill}%
\end{pgfscope}%
\begin{pgfscope}%
\pgfpathrectangle{\pgfqpoint{0.515000in}{0.499444in}}{\pgfqpoint{3.487500in}{1.155000in}}%
\pgfusepath{clip}%
\pgfsetbuttcap%
\pgfsetmiterjoin%
\definecolor{currentfill}{rgb}{0.000000,0.000000,0.000000}%
\pgfsetfillcolor{currentfill}%
\pgfsetlinewidth{0.000000pt}%
\definecolor{currentstroke}{rgb}{0.000000,0.000000,0.000000}%
\pgfsetstrokecolor{currentstroke}%
\pgfsetstrokeopacity{0.000000}%
\pgfsetdash{}{0pt}%
\pgfpathmoveto{\pgfqpoint{3.051364in}{0.499444in}}%
\pgfpathlineto{\pgfqpoint{3.114773in}{0.499444in}}%
\pgfpathlineto{\pgfqpoint{3.114773in}{0.515116in}}%
\pgfpathlineto{\pgfqpoint{3.051364in}{0.515116in}}%
\pgfpathlineto{\pgfqpoint{3.051364in}{0.499444in}}%
\pgfpathclose%
\pgfusepath{fill}%
\end{pgfscope}%
\begin{pgfscope}%
\pgfpathrectangle{\pgfqpoint{0.515000in}{0.499444in}}{\pgfqpoint{3.487500in}{1.155000in}}%
\pgfusepath{clip}%
\pgfsetbuttcap%
\pgfsetmiterjoin%
\definecolor{currentfill}{rgb}{0.000000,0.000000,0.000000}%
\pgfsetfillcolor{currentfill}%
\pgfsetlinewidth{0.000000pt}%
\definecolor{currentstroke}{rgb}{0.000000,0.000000,0.000000}%
\pgfsetstrokecolor{currentstroke}%
\pgfsetstrokeopacity{0.000000}%
\pgfsetdash{}{0pt}%
\pgfpathmoveto{\pgfqpoint{3.209887in}{0.499444in}}%
\pgfpathlineto{\pgfqpoint{3.273296in}{0.499444in}}%
\pgfpathlineto{\pgfqpoint{3.273296in}{0.513695in}}%
\pgfpathlineto{\pgfqpoint{3.209887in}{0.513695in}}%
\pgfpathlineto{\pgfqpoint{3.209887in}{0.499444in}}%
\pgfpathclose%
\pgfusepath{fill}%
\end{pgfscope}%
\begin{pgfscope}%
\pgfpathrectangle{\pgfqpoint{0.515000in}{0.499444in}}{\pgfqpoint{3.487500in}{1.155000in}}%
\pgfusepath{clip}%
\pgfsetbuttcap%
\pgfsetmiterjoin%
\definecolor{currentfill}{rgb}{0.000000,0.000000,0.000000}%
\pgfsetfillcolor{currentfill}%
\pgfsetlinewidth{0.000000pt}%
\definecolor{currentstroke}{rgb}{0.000000,0.000000,0.000000}%
\pgfsetstrokecolor{currentstroke}%
\pgfsetstrokeopacity{0.000000}%
\pgfsetdash{}{0pt}%
\pgfpathmoveto{\pgfqpoint{3.368409in}{0.499444in}}%
\pgfpathlineto{\pgfqpoint{3.431818in}{0.499444in}}%
\pgfpathlineto{\pgfqpoint{3.431818in}{0.511653in}}%
\pgfpathlineto{\pgfqpoint{3.368409in}{0.511653in}}%
\pgfpathlineto{\pgfqpoint{3.368409in}{0.499444in}}%
\pgfpathclose%
\pgfusepath{fill}%
\end{pgfscope}%
\begin{pgfscope}%
\pgfpathrectangle{\pgfqpoint{0.515000in}{0.499444in}}{\pgfqpoint{3.487500in}{1.155000in}}%
\pgfusepath{clip}%
\pgfsetbuttcap%
\pgfsetmiterjoin%
\definecolor{currentfill}{rgb}{0.000000,0.000000,0.000000}%
\pgfsetfillcolor{currentfill}%
\pgfsetlinewidth{0.000000pt}%
\definecolor{currentstroke}{rgb}{0.000000,0.000000,0.000000}%
\pgfsetstrokecolor{currentstroke}%
\pgfsetstrokeopacity{0.000000}%
\pgfsetdash{}{0pt}%
\pgfpathmoveto{\pgfqpoint{3.526932in}{0.499444in}}%
\pgfpathlineto{\pgfqpoint{3.590341in}{0.499444in}}%
\pgfpathlineto{\pgfqpoint{3.590341in}{0.505949in}}%
\pgfpathlineto{\pgfqpoint{3.526932in}{0.505949in}}%
\pgfpathlineto{\pgfqpoint{3.526932in}{0.499444in}}%
\pgfpathclose%
\pgfusepath{fill}%
\end{pgfscope}%
\begin{pgfscope}%
\pgfpathrectangle{\pgfqpoint{0.515000in}{0.499444in}}{\pgfqpoint{3.487500in}{1.155000in}}%
\pgfusepath{clip}%
\pgfsetbuttcap%
\pgfsetmiterjoin%
\definecolor{currentfill}{rgb}{0.000000,0.000000,0.000000}%
\pgfsetfillcolor{currentfill}%
\pgfsetlinewidth{0.000000pt}%
\definecolor{currentstroke}{rgb}{0.000000,0.000000,0.000000}%
\pgfsetstrokecolor{currentstroke}%
\pgfsetstrokeopacity{0.000000}%
\pgfsetdash{}{0pt}%
\pgfpathmoveto{\pgfqpoint{3.685455in}{0.499444in}}%
\pgfpathlineto{\pgfqpoint{3.748864in}{0.499444in}}%
\pgfpathlineto{\pgfqpoint{3.748864in}{0.501386in}}%
\pgfpathlineto{\pgfqpoint{3.685455in}{0.501386in}}%
\pgfpathlineto{\pgfqpoint{3.685455in}{0.499444in}}%
\pgfpathclose%
\pgfusepath{fill}%
\end{pgfscope}%
\begin{pgfscope}%
\pgfpathrectangle{\pgfqpoint{0.515000in}{0.499444in}}{\pgfqpoint{3.487500in}{1.155000in}}%
\pgfusepath{clip}%
\pgfsetbuttcap%
\pgfsetmiterjoin%
\definecolor{currentfill}{rgb}{0.000000,0.000000,0.000000}%
\pgfsetfillcolor{currentfill}%
\pgfsetlinewidth{0.000000pt}%
\definecolor{currentstroke}{rgb}{0.000000,0.000000,0.000000}%
\pgfsetstrokecolor{currentstroke}%
\pgfsetstrokeopacity{0.000000}%
\pgfsetdash{}{0pt}%
\pgfpathmoveto{\pgfqpoint{3.843978in}{0.499444in}}%
\pgfpathlineto{\pgfqpoint{3.907387in}{0.499444in}}%
\pgfpathlineto{\pgfqpoint{3.907387in}{0.499464in}}%
\pgfpathlineto{\pgfqpoint{3.843978in}{0.499464in}}%
\pgfpathlineto{\pgfqpoint{3.843978in}{0.499444in}}%
\pgfpathclose%
\pgfusepath{fill}%
\end{pgfscope}%
\begin{pgfscope}%
\pgfsetbuttcap%
\pgfsetroundjoin%
\definecolor{currentfill}{rgb}{0.000000,0.000000,0.000000}%
\pgfsetfillcolor{currentfill}%
\pgfsetlinewidth{0.803000pt}%
\definecolor{currentstroke}{rgb}{0.000000,0.000000,0.000000}%
\pgfsetstrokecolor{currentstroke}%
\pgfsetdash{}{0pt}%
\pgfsys@defobject{currentmarker}{\pgfqpoint{0.000000in}{-0.048611in}}{\pgfqpoint{0.000000in}{0.000000in}}{%
\pgfpathmoveto{\pgfqpoint{0.000000in}{0.000000in}}%
\pgfpathlineto{\pgfqpoint{0.000000in}{-0.048611in}}%
\pgfusepath{stroke,fill}%
}%
\begin{pgfscope}%
\pgfsys@transformshift{0.515000in}{0.499444in}%
\pgfsys@useobject{currentmarker}{}%
\end{pgfscope}%
\end{pgfscope}%
\begin{pgfscope}%
\pgfsetbuttcap%
\pgfsetroundjoin%
\definecolor{currentfill}{rgb}{0.000000,0.000000,0.000000}%
\pgfsetfillcolor{currentfill}%
\pgfsetlinewidth{0.803000pt}%
\definecolor{currentstroke}{rgb}{0.000000,0.000000,0.000000}%
\pgfsetstrokecolor{currentstroke}%
\pgfsetdash{}{0pt}%
\pgfsys@defobject{currentmarker}{\pgfqpoint{0.000000in}{-0.048611in}}{\pgfqpoint{0.000000in}{0.000000in}}{%
\pgfpathmoveto{\pgfqpoint{0.000000in}{0.000000in}}%
\pgfpathlineto{\pgfqpoint{0.000000in}{-0.048611in}}%
\pgfusepath{stroke,fill}%
}%
\begin{pgfscope}%
\pgfsys@transformshift{0.673523in}{0.499444in}%
\pgfsys@useobject{currentmarker}{}%
\end{pgfscope}%
\end{pgfscope}%
\begin{pgfscope}%
\definecolor{textcolor}{rgb}{0.000000,0.000000,0.000000}%
\pgfsetstrokecolor{textcolor}%
\pgfsetfillcolor{textcolor}%
\pgftext[x=0.673523in,y=0.402222in,,top]{\color{textcolor}\rmfamily\fontsize{10.000000}{12.000000}\selectfont 0.0}%
\end{pgfscope}%
\begin{pgfscope}%
\pgfsetbuttcap%
\pgfsetroundjoin%
\definecolor{currentfill}{rgb}{0.000000,0.000000,0.000000}%
\pgfsetfillcolor{currentfill}%
\pgfsetlinewidth{0.803000pt}%
\definecolor{currentstroke}{rgb}{0.000000,0.000000,0.000000}%
\pgfsetstrokecolor{currentstroke}%
\pgfsetdash{}{0pt}%
\pgfsys@defobject{currentmarker}{\pgfqpoint{0.000000in}{-0.048611in}}{\pgfqpoint{0.000000in}{0.000000in}}{%
\pgfpathmoveto{\pgfqpoint{0.000000in}{0.000000in}}%
\pgfpathlineto{\pgfqpoint{0.000000in}{-0.048611in}}%
\pgfusepath{stroke,fill}%
}%
\begin{pgfscope}%
\pgfsys@transformshift{0.832046in}{0.499444in}%
\pgfsys@useobject{currentmarker}{}%
\end{pgfscope}%
\end{pgfscope}%
\begin{pgfscope}%
\pgfsetbuttcap%
\pgfsetroundjoin%
\definecolor{currentfill}{rgb}{0.000000,0.000000,0.000000}%
\pgfsetfillcolor{currentfill}%
\pgfsetlinewidth{0.803000pt}%
\definecolor{currentstroke}{rgb}{0.000000,0.000000,0.000000}%
\pgfsetstrokecolor{currentstroke}%
\pgfsetdash{}{0pt}%
\pgfsys@defobject{currentmarker}{\pgfqpoint{0.000000in}{-0.048611in}}{\pgfqpoint{0.000000in}{0.000000in}}{%
\pgfpathmoveto{\pgfqpoint{0.000000in}{0.000000in}}%
\pgfpathlineto{\pgfqpoint{0.000000in}{-0.048611in}}%
\pgfusepath{stroke,fill}%
}%
\begin{pgfscope}%
\pgfsys@transformshift{0.990568in}{0.499444in}%
\pgfsys@useobject{currentmarker}{}%
\end{pgfscope}%
\end{pgfscope}%
\begin{pgfscope}%
\definecolor{textcolor}{rgb}{0.000000,0.000000,0.000000}%
\pgfsetstrokecolor{textcolor}%
\pgfsetfillcolor{textcolor}%
\pgftext[x=0.990568in,y=0.402222in,,top]{\color{textcolor}\rmfamily\fontsize{10.000000}{12.000000}\selectfont 0.1}%
\end{pgfscope}%
\begin{pgfscope}%
\pgfsetbuttcap%
\pgfsetroundjoin%
\definecolor{currentfill}{rgb}{0.000000,0.000000,0.000000}%
\pgfsetfillcolor{currentfill}%
\pgfsetlinewidth{0.803000pt}%
\definecolor{currentstroke}{rgb}{0.000000,0.000000,0.000000}%
\pgfsetstrokecolor{currentstroke}%
\pgfsetdash{}{0pt}%
\pgfsys@defobject{currentmarker}{\pgfqpoint{0.000000in}{-0.048611in}}{\pgfqpoint{0.000000in}{0.000000in}}{%
\pgfpathmoveto{\pgfqpoint{0.000000in}{0.000000in}}%
\pgfpathlineto{\pgfqpoint{0.000000in}{-0.048611in}}%
\pgfusepath{stroke,fill}%
}%
\begin{pgfscope}%
\pgfsys@transformshift{1.149091in}{0.499444in}%
\pgfsys@useobject{currentmarker}{}%
\end{pgfscope}%
\end{pgfscope}%
\begin{pgfscope}%
\pgfsetbuttcap%
\pgfsetroundjoin%
\definecolor{currentfill}{rgb}{0.000000,0.000000,0.000000}%
\pgfsetfillcolor{currentfill}%
\pgfsetlinewidth{0.803000pt}%
\definecolor{currentstroke}{rgb}{0.000000,0.000000,0.000000}%
\pgfsetstrokecolor{currentstroke}%
\pgfsetdash{}{0pt}%
\pgfsys@defobject{currentmarker}{\pgfqpoint{0.000000in}{-0.048611in}}{\pgfqpoint{0.000000in}{0.000000in}}{%
\pgfpathmoveto{\pgfqpoint{0.000000in}{0.000000in}}%
\pgfpathlineto{\pgfqpoint{0.000000in}{-0.048611in}}%
\pgfusepath{stroke,fill}%
}%
\begin{pgfscope}%
\pgfsys@transformshift{1.307614in}{0.499444in}%
\pgfsys@useobject{currentmarker}{}%
\end{pgfscope}%
\end{pgfscope}%
\begin{pgfscope}%
\definecolor{textcolor}{rgb}{0.000000,0.000000,0.000000}%
\pgfsetstrokecolor{textcolor}%
\pgfsetfillcolor{textcolor}%
\pgftext[x=1.307614in,y=0.402222in,,top]{\color{textcolor}\rmfamily\fontsize{10.000000}{12.000000}\selectfont 0.2}%
\end{pgfscope}%
\begin{pgfscope}%
\pgfsetbuttcap%
\pgfsetroundjoin%
\definecolor{currentfill}{rgb}{0.000000,0.000000,0.000000}%
\pgfsetfillcolor{currentfill}%
\pgfsetlinewidth{0.803000pt}%
\definecolor{currentstroke}{rgb}{0.000000,0.000000,0.000000}%
\pgfsetstrokecolor{currentstroke}%
\pgfsetdash{}{0pt}%
\pgfsys@defobject{currentmarker}{\pgfqpoint{0.000000in}{-0.048611in}}{\pgfqpoint{0.000000in}{0.000000in}}{%
\pgfpathmoveto{\pgfqpoint{0.000000in}{0.000000in}}%
\pgfpathlineto{\pgfqpoint{0.000000in}{-0.048611in}}%
\pgfusepath{stroke,fill}%
}%
\begin{pgfscope}%
\pgfsys@transformshift{1.466137in}{0.499444in}%
\pgfsys@useobject{currentmarker}{}%
\end{pgfscope}%
\end{pgfscope}%
\begin{pgfscope}%
\pgfsetbuttcap%
\pgfsetroundjoin%
\definecolor{currentfill}{rgb}{0.000000,0.000000,0.000000}%
\pgfsetfillcolor{currentfill}%
\pgfsetlinewidth{0.803000pt}%
\definecolor{currentstroke}{rgb}{0.000000,0.000000,0.000000}%
\pgfsetstrokecolor{currentstroke}%
\pgfsetdash{}{0pt}%
\pgfsys@defobject{currentmarker}{\pgfqpoint{0.000000in}{-0.048611in}}{\pgfqpoint{0.000000in}{0.000000in}}{%
\pgfpathmoveto{\pgfqpoint{0.000000in}{0.000000in}}%
\pgfpathlineto{\pgfqpoint{0.000000in}{-0.048611in}}%
\pgfusepath{stroke,fill}%
}%
\begin{pgfscope}%
\pgfsys@transformshift{1.624659in}{0.499444in}%
\pgfsys@useobject{currentmarker}{}%
\end{pgfscope}%
\end{pgfscope}%
\begin{pgfscope}%
\definecolor{textcolor}{rgb}{0.000000,0.000000,0.000000}%
\pgfsetstrokecolor{textcolor}%
\pgfsetfillcolor{textcolor}%
\pgftext[x=1.624659in,y=0.402222in,,top]{\color{textcolor}\rmfamily\fontsize{10.000000}{12.000000}\selectfont 0.3}%
\end{pgfscope}%
\begin{pgfscope}%
\pgfsetbuttcap%
\pgfsetroundjoin%
\definecolor{currentfill}{rgb}{0.000000,0.000000,0.000000}%
\pgfsetfillcolor{currentfill}%
\pgfsetlinewidth{0.803000pt}%
\definecolor{currentstroke}{rgb}{0.000000,0.000000,0.000000}%
\pgfsetstrokecolor{currentstroke}%
\pgfsetdash{}{0pt}%
\pgfsys@defobject{currentmarker}{\pgfqpoint{0.000000in}{-0.048611in}}{\pgfqpoint{0.000000in}{0.000000in}}{%
\pgfpathmoveto{\pgfqpoint{0.000000in}{0.000000in}}%
\pgfpathlineto{\pgfqpoint{0.000000in}{-0.048611in}}%
\pgfusepath{stroke,fill}%
}%
\begin{pgfscope}%
\pgfsys@transformshift{1.783182in}{0.499444in}%
\pgfsys@useobject{currentmarker}{}%
\end{pgfscope}%
\end{pgfscope}%
\begin{pgfscope}%
\pgfsetbuttcap%
\pgfsetroundjoin%
\definecolor{currentfill}{rgb}{0.000000,0.000000,0.000000}%
\pgfsetfillcolor{currentfill}%
\pgfsetlinewidth{0.803000pt}%
\definecolor{currentstroke}{rgb}{0.000000,0.000000,0.000000}%
\pgfsetstrokecolor{currentstroke}%
\pgfsetdash{}{0pt}%
\pgfsys@defobject{currentmarker}{\pgfqpoint{0.000000in}{-0.048611in}}{\pgfqpoint{0.000000in}{0.000000in}}{%
\pgfpathmoveto{\pgfqpoint{0.000000in}{0.000000in}}%
\pgfpathlineto{\pgfqpoint{0.000000in}{-0.048611in}}%
\pgfusepath{stroke,fill}%
}%
\begin{pgfscope}%
\pgfsys@transformshift{1.941705in}{0.499444in}%
\pgfsys@useobject{currentmarker}{}%
\end{pgfscope}%
\end{pgfscope}%
\begin{pgfscope}%
\definecolor{textcolor}{rgb}{0.000000,0.000000,0.000000}%
\pgfsetstrokecolor{textcolor}%
\pgfsetfillcolor{textcolor}%
\pgftext[x=1.941705in,y=0.402222in,,top]{\color{textcolor}\rmfamily\fontsize{10.000000}{12.000000}\selectfont 0.4}%
\end{pgfscope}%
\begin{pgfscope}%
\pgfsetbuttcap%
\pgfsetroundjoin%
\definecolor{currentfill}{rgb}{0.000000,0.000000,0.000000}%
\pgfsetfillcolor{currentfill}%
\pgfsetlinewidth{0.803000pt}%
\definecolor{currentstroke}{rgb}{0.000000,0.000000,0.000000}%
\pgfsetstrokecolor{currentstroke}%
\pgfsetdash{}{0pt}%
\pgfsys@defobject{currentmarker}{\pgfqpoint{0.000000in}{-0.048611in}}{\pgfqpoint{0.000000in}{0.000000in}}{%
\pgfpathmoveto{\pgfqpoint{0.000000in}{0.000000in}}%
\pgfpathlineto{\pgfqpoint{0.000000in}{-0.048611in}}%
\pgfusepath{stroke,fill}%
}%
\begin{pgfscope}%
\pgfsys@transformshift{2.100228in}{0.499444in}%
\pgfsys@useobject{currentmarker}{}%
\end{pgfscope}%
\end{pgfscope}%
\begin{pgfscope}%
\pgfsetbuttcap%
\pgfsetroundjoin%
\definecolor{currentfill}{rgb}{0.000000,0.000000,0.000000}%
\pgfsetfillcolor{currentfill}%
\pgfsetlinewidth{0.803000pt}%
\definecolor{currentstroke}{rgb}{0.000000,0.000000,0.000000}%
\pgfsetstrokecolor{currentstroke}%
\pgfsetdash{}{0pt}%
\pgfsys@defobject{currentmarker}{\pgfqpoint{0.000000in}{-0.048611in}}{\pgfqpoint{0.000000in}{0.000000in}}{%
\pgfpathmoveto{\pgfqpoint{0.000000in}{0.000000in}}%
\pgfpathlineto{\pgfqpoint{0.000000in}{-0.048611in}}%
\pgfusepath{stroke,fill}%
}%
\begin{pgfscope}%
\pgfsys@transformshift{2.258750in}{0.499444in}%
\pgfsys@useobject{currentmarker}{}%
\end{pgfscope}%
\end{pgfscope}%
\begin{pgfscope}%
\definecolor{textcolor}{rgb}{0.000000,0.000000,0.000000}%
\pgfsetstrokecolor{textcolor}%
\pgfsetfillcolor{textcolor}%
\pgftext[x=2.258750in,y=0.402222in,,top]{\color{textcolor}\rmfamily\fontsize{10.000000}{12.000000}\selectfont 0.5}%
\end{pgfscope}%
\begin{pgfscope}%
\pgfsetbuttcap%
\pgfsetroundjoin%
\definecolor{currentfill}{rgb}{0.000000,0.000000,0.000000}%
\pgfsetfillcolor{currentfill}%
\pgfsetlinewidth{0.803000pt}%
\definecolor{currentstroke}{rgb}{0.000000,0.000000,0.000000}%
\pgfsetstrokecolor{currentstroke}%
\pgfsetdash{}{0pt}%
\pgfsys@defobject{currentmarker}{\pgfqpoint{0.000000in}{-0.048611in}}{\pgfqpoint{0.000000in}{0.000000in}}{%
\pgfpathmoveto{\pgfqpoint{0.000000in}{0.000000in}}%
\pgfpathlineto{\pgfqpoint{0.000000in}{-0.048611in}}%
\pgfusepath{stroke,fill}%
}%
\begin{pgfscope}%
\pgfsys@transformshift{2.417273in}{0.499444in}%
\pgfsys@useobject{currentmarker}{}%
\end{pgfscope}%
\end{pgfscope}%
\begin{pgfscope}%
\pgfsetbuttcap%
\pgfsetroundjoin%
\definecolor{currentfill}{rgb}{0.000000,0.000000,0.000000}%
\pgfsetfillcolor{currentfill}%
\pgfsetlinewidth{0.803000pt}%
\definecolor{currentstroke}{rgb}{0.000000,0.000000,0.000000}%
\pgfsetstrokecolor{currentstroke}%
\pgfsetdash{}{0pt}%
\pgfsys@defobject{currentmarker}{\pgfqpoint{0.000000in}{-0.048611in}}{\pgfqpoint{0.000000in}{0.000000in}}{%
\pgfpathmoveto{\pgfqpoint{0.000000in}{0.000000in}}%
\pgfpathlineto{\pgfqpoint{0.000000in}{-0.048611in}}%
\pgfusepath{stroke,fill}%
}%
\begin{pgfscope}%
\pgfsys@transformshift{2.575796in}{0.499444in}%
\pgfsys@useobject{currentmarker}{}%
\end{pgfscope}%
\end{pgfscope}%
\begin{pgfscope}%
\definecolor{textcolor}{rgb}{0.000000,0.000000,0.000000}%
\pgfsetstrokecolor{textcolor}%
\pgfsetfillcolor{textcolor}%
\pgftext[x=2.575796in,y=0.402222in,,top]{\color{textcolor}\rmfamily\fontsize{10.000000}{12.000000}\selectfont 0.6}%
\end{pgfscope}%
\begin{pgfscope}%
\pgfsetbuttcap%
\pgfsetroundjoin%
\definecolor{currentfill}{rgb}{0.000000,0.000000,0.000000}%
\pgfsetfillcolor{currentfill}%
\pgfsetlinewidth{0.803000pt}%
\definecolor{currentstroke}{rgb}{0.000000,0.000000,0.000000}%
\pgfsetstrokecolor{currentstroke}%
\pgfsetdash{}{0pt}%
\pgfsys@defobject{currentmarker}{\pgfqpoint{0.000000in}{-0.048611in}}{\pgfqpoint{0.000000in}{0.000000in}}{%
\pgfpathmoveto{\pgfqpoint{0.000000in}{0.000000in}}%
\pgfpathlineto{\pgfqpoint{0.000000in}{-0.048611in}}%
\pgfusepath{stroke,fill}%
}%
\begin{pgfscope}%
\pgfsys@transformshift{2.734318in}{0.499444in}%
\pgfsys@useobject{currentmarker}{}%
\end{pgfscope}%
\end{pgfscope}%
\begin{pgfscope}%
\pgfsetbuttcap%
\pgfsetroundjoin%
\definecolor{currentfill}{rgb}{0.000000,0.000000,0.000000}%
\pgfsetfillcolor{currentfill}%
\pgfsetlinewidth{0.803000pt}%
\definecolor{currentstroke}{rgb}{0.000000,0.000000,0.000000}%
\pgfsetstrokecolor{currentstroke}%
\pgfsetdash{}{0pt}%
\pgfsys@defobject{currentmarker}{\pgfqpoint{0.000000in}{-0.048611in}}{\pgfqpoint{0.000000in}{0.000000in}}{%
\pgfpathmoveto{\pgfqpoint{0.000000in}{0.000000in}}%
\pgfpathlineto{\pgfqpoint{0.000000in}{-0.048611in}}%
\pgfusepath{stroke,fill}%
}%
\begin{pgfscope}%
\pgfsys@transformshift{2.892841in}{0.499444in}%
\pgfsys@useobject{currentmarker}{}%
\end{pgfscope}%
\end{pgfscope}%
\begin{pgfscope}%
\definecolor{textcolor}{rgb}{0.000000,0.000000,0.000000}%
\pgfsetstrokecolor{textcolor}%
\pgfsetfillcolor{textcolor}%
\pgftext[x=2.892841in,y=0.402222in,,top]{\color{textcolor}\rmfamily\fontsize{10.000000}{12.000000}\selectfont 0.7}%
\end{pgfscope}%
\begin{pgfscope}%
\pgfsetbuttcap%
\pgfsetroundjoin%
\definecolor{currentfill}{rgb}{0.000000,0.000000,0.000000}%
\pgfsetfillcolor{currentfill}%
\pgfsetlinewidth{0.803000pt}%
\definecolor{currentstroke}{rgb}{0.000000,0.000000,0.000000}%
\pgfsetstrokecolor{currentstroke}%
\pgfsetdash{}{0pt}%
\pgfsys@defobject{currentmarker}{\pgfqpoint{0.000000in}{-0.048611in}}{\pgfqpoint{0.000000in}{0.000000in}}{%
\pgfpathmoveto{\pgfqpoint{0.000000in}{0.000000in}}%
\pgfpathlineto{\pgfqpoint{0.000000in}{-0.048611in}}%
\pgfusepath{stroke,fill}%
}%
\begin{pgfscope}%
\pgfsys@transformshift{3.051364in}{0.499444in}%
\pgfsys@useobject{currentmarker}{}%
\end{pgfscope}%
\end{pgfscope}%
\begin{pgfscope}%
\pgfsetbuttcap%
\pgfsetroundjoin%
\definecolor{currentfill}{rgb}{0.000000,0.000000,0.000000}%
\pgfsetfillcolor{currentfill}%
\pgfsetlinewidth{0.803000pt}%
\definecolor{currentstroke}{rgb}{0.000000,0.000000,0.000000}%
\pgfsetstrokecolor{currentstroke}%
\pgfsetdash{}{0pt}%
\pgfsys@defobject{currentmarker}{\pgfqpoint{0.000000in}{-0.048611in}}{\pgfqpoint{0.000000in}{0.000000in}}{%
\pgfpathmoveto{\pgfqpoint{0.000000in}{0.000000in}}%
\pgfpathlineto{\pgfqpoint{0.000000in}{-0.048611in}}%
\pgfusepath{stroke,fill}%
}%
\begin{pgfscope}%
\pgfsys@transformshift{3.209887in}{0.499444in}%
\pgfsys@useobject{currentmarker}{}%
\end{pgfscope}%
\end{pgfscope}%
\begin{pgfscope}%
\definecolor{textcolor}{rgb}{0.000000,0.000000,0.000000}%
\pgfsetstrokecolor{textcolor}%
\pgfsetfillcolor{textcolor}%
\pgftext[x=3.209887in,y=0.402222in,,top]{\color{textcolor}\rmfamily\fontsize{10.000000}{12.000000}\selectfont 0.8}%
\end{pgfscope}%
\begin{pgfscope}%
\pgfsetbuttcap%
\pgfsetroundjoin%
\definecolor{currentfill}{rgb}{0.000000,0.000000,0.000000}%
\pgfsetfillcolor{currentfill}%
\pgfsetlinewidth{0.803000pt}%
\definecolor{currentstroke}{rgb}{0.000000,0.000000,0.000000}%
\pgfsetstrokecolor{currentstroke}%
\pgfsetdash{}{0pt}%
\pgfsys@defobject{currentmarker}{\pgfqpoint{0.000000in}{-0.048611in}}{\pgfqpoint{0.000000in}{0.000000in}}{%
\pgfpathmoveto{\pgfqpoint{0.000000in}{0.000000in}}%
\pgfpathlineto{\pgfqpoint{0.000000in}{-0.048611in}}%
\pgfusepath{stroke,fill}%
}%
\begin{pgfscope}%
\pgfsys@transformshift{3.368409in}{0.499444in}%
\pgfsys@useobject{currentmarker}{}%
\end{pgfscope}%
\end{pgfscope}%
\begin{pgfscope}%
\pgfsetbuttcap%
\pgfsetroundjoin%
\definecolor{currentfill}{rgb}{0.000000,0.000000,0.000000}%
\pgfsetfillcolor{currentfill}%
\pgfsetlinewidth{0.803000pt}%
\definecolor{currentstroke}{rgb}{0.000000,0.000000,0.000000}%
\pgfsetstrokecolor{currentstroke}%
\pgfsetdash{}{0pt}%
\pgfsys@defobject{currentmarker}{\pgfqpoint{0.000000in}{-0.048611in}}{\pgfqpoint{0.000000in}{0.000000in}}{%
\pgfpathmoveto{\pgfqpoint{0.000000in}{0.000000in}}%
\pgfpathlineto{\pgfqpoint{0.000000in}{-0.048611in}}%
\pgfusepath{stroke,fill}%
}%
\begin{pgfscope}%
\pgfsys@transformshift{3.526932in}{0.499444in}%
\pgfsys@useobject{currentmarker}{}%
\end{pgfscope}%
\end{pgfscope}%
\begin{pgfscope}%
\definecolor{textcolor}{rgb}{0.000000,0.000000,0.000000}%
\pgfsetstrokecolor{textcolor}%
\pgfsetfillcolor{textcolor}%
\pgftext[x=3.526932in,y=0.402222in,,top]{\color{textcolor}\rmfamily\fontsize{10.000000}{12.000000}\selectfont 0.9}%
\end{pgfscope}%
\begin{pgfscope}%
\pgfsetbuttcap%
\pgfsetroundjoin%
\definecolor{currentfill}{rgb}{0.000000,0.000000,0.000000}%
\pgfsetfillcolor{currentfill}%
\pgfsetlinewidth{0.803000pt}%
\definecolor{currentstroke}{rgb}{0.000000,0.000000,0.000000}%
\pgfsetstrokecolor{currentstroke}%
\pgfsetdash{}{0pt}%
\pgfsys@defobject{currentmarker}{\pgfqpoint{0.000000in}{-0.048611in}}{\pgfqpoint{0.000000in}{0.000000in}}{%
\pgfpathmoveto{\pgfqpoint{0.000000in}{0.000000in}}%
\pgfpathlineto{\pgfqpoint{0.000000in}{-0.048611in}}%
\pgfusepath{stroke,fill}%
}%
\begin{pgfscope}%
\pgfsys@transformshift{3.685455in}{0.499444in}%
\pgfsys@useobject{currentmarker}{}%
\end{pgfscope}%
\end{pgfscope}%
\begin{pgfscope}%
\pgfsetbuttcap%
\pgfsetroundjoin%
\definecolor{currentfill}{rgb}{0.000000,0.000000,0.000000}%
\pgfsetfillcolor{currentfill}%
\pgfsetlinewidth{0.803000pt}%
\definecolor{currentstroke}{rgb}{0.000000,0.000000,0.000000}%
\pgfsetstrokecolor{currentstroke}%
\pgfsetdash{}{0pt}%
\pgfsys@defobject{currentmarker}{\pgfqpoint{0.000000in}{-0.048611in}}{\pgfqpoint{0.000000in}{0.000000in}}{%
\pgfpathmoveto{\pgfqpoint{0.000000in}{0.000000in}}%
\pgfpathlineto{\pgfqpoint{0.000000in}{-0.048611in}}%
\pgfusepath{stroke,fill}%
}%
\begin{pgfscope}%
\pgfsys@transformshift{3.843978in}{0.499444in}%
\pgfsys@useobject{currentmarker}{}%
\end{pgfscope}%
\end{pgfscope}%
\begin{pgfscope}%
\definecolor{textcolor}{rgb}{0.000000,0.000000,0.000000}%
\pgfsetstrokecolor{textcolor}%
\pgfsetfillcolor{textcolor}%
\pgftext[x=3.843978in,y=0.402222in,,top]{\color{textcolor}\rmfamily\fontsize{10.000000}{12.000000}\selectfont 1.0}%
\end{pgfscope}%
\begin{pgfscope}%
\pgfsetbuttcap%
\pgfsetroundjoin%
\definecolor{currentfill}{rgb}{0.000000,0.000000,0.000000}%
\pgfsetfillcolor{currentfill}%
\pgfsetlinewidth{0.803000pt}%
\definecolor{currentstroke}{rgb}{0.000000,0.000000,0.000000}%
\pgfsetstrokecolor{currentstroke}%
\pgfsetdash{}{0pt}%
\pgfsys@defobject{currentmarker}{\pgfqpoint{0.000000in}{-0.048611in}}{\pgfqpoint{0.000000in}{0.000000in}}{%
\pgfpathmoveto{\pgfqpoint{0.000000in}{0.000000in}}%
\pgfpathlineto{\pgfqpoint{0.000000in}{-0.048611in}}%
\pgfusepath{stroke,fill}%
}%
\begin{pgfscope}%
\pgfsys@transformshift{4.002500in}{0.499444in}%
\pgfsys@useobject{currentmarker}{}%
\end{pgfscope}%
\end{pgfscope}%
\begin{pgfscope}%
\definecolor{textcolor}{rgb}{0.000000,0.000000,0.000000}%
\pgfsetstrokecolor{textcolor}%
\pgfsetfillcolor{textcolor}%
\pgftext[x=2.258750in,y=0.223333in,,top]{\color{textcolor}\rmfamily\fontsize{10.000000}{12.000000}\selectfont \(\displaystyle p\)}%
\end{pgfscope}%
\begin{pgfscope}%
\pgfsetbuttcap%
\pgfsetroundjoin%
\definecolor{currentfill}{rgb}{0.000000,0.000000,0.000000}%
\pgfsetfillcolor{currentfill}%
\pgfsetlinewidth{0.803000pt}%
\definecolor{currentstroke}{rgb}{0.000000,0.000000,0.000000}%
\pgfsetstrokecolor{currentstroke}%
\pgfsetdash{}{0pt}%
\pgfsys@defobject{currentmarker}{\pgfqpoint{-0.048611in}{0.000000in}}{\pgfqpoint{-0.000000in}{0.000000in}}{%
\pgfpathmoveto{\pgfqpoint{-0.000000in}{0.000000in}}%
\pgfpathlineto{\pgfqpoint{-0.048611in}{0.000000in}}%
\pgfusepath{stroke,fill}%
}%
\begin{pgfscope}%
\pgfsys@transformshift{0.515000in}{0.499444in}%
\pgfsys@useobject{currentmarker}{}%
\end{pgfscope}%
\end{pgfscope}%
\begin{pgfscope}%
\definecolor{textcolor}{rgb}{0.000000,0.000000,0.000000}%
\pgfsetstrokecolor{textcolor}%
\pgfsetfillcolor{textcolor}%
\pgftext[x=0.348333in, y=0.451250in, left, base]{\color{textcolor}\rmfamily\fontsize{10.000000}{12.000000}\selectfont \(\displaystyle {0}\)}%
\end{pgfscope}%
\begin{pgfscope}%
\pgfsetbuttcap%
\pgfsetroundjoin%
\definecolor{currentfill}{rgb}{0.000000,0.000000,0.000000}%
\pgfsetfillcolor{currentfill}%
\pgfsetlinewidth{0.803000pt}%
\definecolor{currentstroke}{rgb}{0.000000,0.000000,0.000000}%
\pgfsetstrokecolor{currentstroke}%
\pgfsetdash{}{0pt}%
\pgfsys@defobject{currentmarker}{\pgfqpoint{-0.048611in}{0.000000in}}{\pgfqpoint{-0.000000in}{0.000000in}}{%
\pgfpathmoveto{\pgfqpoint{-0.000000in}{0.000000in}}%
\pgfpathlineto{\pgfqpoint{-0.048611in}{0.000000in}}%
\pgfusepath{stroke,fill}%
}%
\begin{pgfscope}%
\pgfsys@transformshift{0.515000in}{0.927911in}%
\pgfsys@useobject{currentmarker}{}%
\end{pgfscope}%
\end{pgfscope}%
\begin{pgfscope}%
\definecolor{textcolor}{rgb}{0.000000,0.000000,0.000000}%
\pgfsetstrokecolor{textcolor}%
\pgfsetfillcolor{textcolor}%
\pgftext[x=0.278889in, y=0.879717in, left, base]{\color{textcolor}\rmfamily\fontsize{10.000000}{12.000000}\selectfont \(\displaystyle {10}\)}%
\end{pgfscope}%
\begin{pgfscope}%
\pgfsetbuttcap%
\pgfsetroundjoin%
\definecolor{currentfill}{rgb}{0.000000,0.000000,0.000000}%
\pgfsetfillcolor{currentfill}%
\pgfsetlinewidth{0.803000pt}%
\definecolor{currentstroke}{rgb}{0.000000,0.000000,0.000000}%
\pgfsetstrokecolor{currentstroke}%
\pgfsetdash{}{0pt}%
\pgfsys@defobject{currentmarker}{\pgfqpoint{-0.048611in}{0.000000in}}{\pgfqpoint{-0.000000in}{0.000000in}}{%
\pgfpathmoveto{\pgfqpoint{-0.000000in}{0.000000in}}%
\pgfpathlineto{\pgfqpoint{-0.048611in}{0.000000in}}%
\pgfusepath{stroke,fill}%
}%
\begin{pgfscope}%
\pgfsys@transformshift{0.515000in}{1.356379in}%
\pgfsys@useobject{currentmarker}{}%
\end{pgfscope}%
\end{pgfscope}%
\begin{pgfscope}%
\definecolor{textcolor}{rgb}{0.000000,0.000000,0.000000}%
\pgfsetstrokecolor{textcolor}%
\pgfsetfillcolor{textcolor}%
\pgftext[x=0.278889in, y=1.308184in, left, base]{\color{textcolor}\rmfamily\fontsize{10.000000}{12.000000}\selectfont \(\displaystyle {20}\)}%
\end{pgfscope}%
\begin{pgfscope}%
\definecolor{textcolor}{rgb}{0.000000,0.000000,0.000000}%
\pgfsetstrokecolor{textcolor}%
\pgfsetfillcolor{textcolor}%
\pgftext[x=0.223333in,y=1.076944in,,bottom,rotate=90.000000]{\color{textcolor}\rmfamily\fontsize{10.000000}{12.000000}\selectfont Percent of Data Set}%
\end{pgfscope}%
\begin{pgfscope}%
\pgfsetrectcap%
\pgfsetmiterjoin%
\pgfsetlinewidth{0.803000pt}%
\definecolor{currentstroke}{rgb}{0.000000,0.000000,0.000000}%
\pgfsetstrokecolor{currentstroke}%
\pgfsetdash{}{0pt}%
\pgfpathmoveto{\pgfqpoint{0.515000in}{0.499444in}}%
\pgfpathlineto{\pgfqpoint{0.515000in}{1.654444in}}%
\pgfusepath{stroke}%
\end{pgfscope}%
\begin{pgfscope}%
\pgfsetrectcap%
\pgfsetmiterjoin%
\pgfsetlinewidth{0.803000pt}%
\definecolor{currentstroke}{rgb}{0.000000,0.000000,0.000000}%
\pgfsetstrokecolor{currentstroke}%
\pgfsetdash{}{0pt}%
\pgfpathmoveto{\pgfqpoint{4.002500in}{0.499444in}}%
\pgfpathlineto{\pgfqpoint{4.002500in}{1.654444in}}%
\pgfusepath{stroke}%
\end{pgfscope}%
\begin{pgfscope}%
\pgfsetrectcap%
\pgfsetmiterjoin%
\pgfsetlinewidth{0.803000pt}%
\definecolor{currentstroke}{rgb}{0.000000,0.000000,0.000000}%
\pgfsetstrokecolor{currentstroke}%
\pgfsetdash{}{0pt}%
\pgfpathmoveto{\pgfqpoint{0.515000in}{0.499444in}}%
\pgfpathlineto{\pgfqpoint{4.002500in}{0.499444in}}%
\pgfusepath{stroke}%
\end{pgfscope}%
\begin{pgfscope}%
\pgfsetrectcap%
\pgfsetmiterjoin%
\pgfsetlinewidth{0.803000pt}%
\definecolor{currentstroke}{rgb}{0.000000,0.000000,0.000000}%
\pgfsetstrokecolor{currentstroke}%
\pgfsetdash{}{0pt}%
\pgfpathmoveto{\pgfqpoint{0.515000in}{1.654444in}}%
\pgfpathlineto{\pgfqpoint{4.002500in}{1.654444in}}%
\pgfusepath{stroke}%
\end{pgfscope}%
\begin{pgfscope}%
\pgfsetbuttcap%
\pgfsetmiterjoin%
\definecolor{currentfill}{rgb}{1.000000,1.000000,1.000000}%
\pgfsetfillcolor{currentfill}%
\pgfsetfillopacity{0.800000}%
\pgfsetlinewidth{1.003750pt}%
\definecolor{currentstroke}{rgb}{0.800000,0.800000,0.800000}%
\pgfsetstrokecolor{currentstroke}%
\pgfsetstrokeopacity{0.800000}%
\pgfsetdash{}{0pt}%
\pgfpathmoveto{\pgfqpoint{3.225556in}{1.154445in}}%
\pgfpathlineto{\pgfqpoint{3.905278in}{1.154445in}}%
\pgfpathquadraticcurveto{\pgfqpoint{3.933056in}{1.154445in}}{\pgfqpoint{3.933056in}{1.182222in}}%
\pgfpathlineto{\pgfqpoint{3.933056in}{1.557222in}}%
\pgfpathquadraticcurveto{\pgfqpoint{3.933056in}{1.585000in}}{\pgfqpoint{3.905278in}{1.585000in}}%
\pgfpathlineto{\pgfqpoint{3.225556in}{1.585000in}}%
\pgfpathquadraticcurveto{\pgfqpoint{3.197778in}{1.585000in}}{\pgfqpoint{3.197778in}{1.557222in}}%
\pgfpathlineto{\pgfqpoint{3.197778in}{1.182222in}}%
\pgfpathquadraticcurveto{\pgfqpoint{3.197778in}{1.154445in}}{\pgfqpoint{3.225556in}{1.154445in}}%
\pgfpathlineto{\pgfqpoint{3.225556in}{1.154445in}}%
\pgfpathclose%
\pgfusepath{stroke,fill}%
\end{pgfscope}%
\begin{pgfscope}%
\pgfsetbuttcap%
\pgfsetmiterjoin%
\pgfsetlinewidth{1.003750pt}%
\definecolor{currentstroke}{rgb}{0.000000,0.000000,0.000000}%
\pgfsetstrokecolor{currentstroke}%
\pgfsetdash{}{0pt}%
\pgfpathmoveto{\pgfqpoint{3.253334in}{1.432222in}}%
\pgfpathlineto{\pgfqpoint{3.531111in}{1.432222in}}%
\pgfpathlineto{\pgfqpoint{3.531111in}{1.529444in}}%
\pgfpathlineto{\pgfqpoint{3.253334in}{1.529444in}}%
\pgfpathlineto{\pgfqpoint{3.253334in}{1.432222in}}%
\pgfpathclose%
\pgfusepath{stroke}%
\end{pgfscope}%
\begin{pgfscope}%
\definecolor{textcolor}{rgb}{0.000000,0.000000,0.000000}%
\pgfsetstrokecolor{textcolor}%
\pgfsetfillcolor{textcolor}%
\pgftext[x=3.642223in,y=1.432222in,left,base]{\color{textcolor}\rmfamily\fontsize{10.000000}{12.000000}\selectfont Neg}%
\end{pgfscope}%
\begin{pgfscope}%
\pgfsetbuttcap%
\pgfsetmiterjoin%
\definecolor{currentfill}{rgb}{0.000000,0.000000,0.000000}%
\pgfsetfillcolor{currentfill}%
\pgfsetlinewidth{0.000000pt}%
\definecolor{currentstroke}{rgb}{0.000000,0.000000,0.000000}%
\pgfsetstrokecolor{currentstroke}%
\pgfsetstrokeopacity{0.000000}%
\pgfsetdash{}{0pt}%
\pgfpathmoveto{\pgfqpoint{3.253334in}{1.236944in}}%
\pgfpathlineto{\pgfqpoint{3.531111in}{1.236944in}}%
\pgfpathlineto{\pgfqpoint{3.531111in}{1.334167in}}%
\pgfpathlineto{\pgfqpoint{3.253334in}{1.334167in}}%
\pgfpathlineto{\pgfqpoint{3.253334in}{1.236944in}}%
\pgfpathclose%
\pgfusepath{fill}%
\end{pgfscope}%
\begin{pgfscope}%
\definecolor{textcolor}{rgb}{0.000000,0.000000,0.000000}%
\pgfsetstrokecolor{textcolor}%
\pgfsetfillcolor{textcolor}%
\pgftext[x=3.642223in,y=1.236944in,left,base]{\color{textcolor}\rmfamily\fontsize{10.000000}{12.000000}\selectfont Pos}%
\end{pgfscope}%
\end{pgfpicture}%
\makeatother%
\endgroup%
	
&
	\vskip 0pt
	\hfil ROC Curve
	
	%% Creator: Matplotlib, PGF backend
%%
%% To include the figure in your LaTeX document, write
%%   \input{<filename>.pgf}
%%
%% Make sure the required packages are loaded in your preamble
%%   \usepackage{pgf}
%%
%% Also ensure that all the required font packages are loaded; for instance,
%% the lmodern package is sometimes necessary when using math font.
%%   \usepackage{lmodern}
%%
%% Figures using additional raster images can only be included by \input if
%% they are in the same directory as the main LaTeX file. For loading figures
%% from other directories you can use the `import` package
%%   \usepackage{import}
%%
%% and then include the figures with
%%   \import{<path to file>}{<filename>.pgf}
%%
%% Matplotlib used the following preamble
%%   
%%   \usepackage{fontspec}
%%   \makeatletter\@ifpackageloaded{underscore}{}{\usepackage[strings]{underscore}}\makeatother
%%
\begingroup%
\makeatletter%
\begin{pgfpicture}%
\pgfpathrectangle{\pgfpointorigin}{\pgfqpoint{2.221861in}{1.754444in}}%
\pgfusepath{use as bounding box, clip}%
\begin{pgfscope}%
\pgfsetbuttcap%
\pgfsetmiterjoin%
\definecolor{currentfill}{rgb}{1.000000,1.000000,1.000000}%
\pgfsetfillcolor{currentfill}%
\pgfsetlinewidth{0.000000pt}%
\definecolor{currentstroke}{rgb}{1.000000,1.000000,1.000000}%
\pgfsetstrokecolor{currentstroke}%
\pgfsetdash{}{0pt}%
\pgfpathmoveto{\pgfqpoint{0.000000in}{0.000000in}}%
\pgfpathlineto{\pgfqpoint{2.221861in}{0.000000in}}%
\pgfpathlineto{\pgfqpoint{2.221861in}{1.754444in}}%
\pgfpathlineto{\pgfqpoint{0.000000in}{1.754444in}}%
\pgfpathlineto{\pgfqpoint{0.000000in}{0.000000in}}%
\pgfpathclose%
\pgfusepath{fill}%
\end{pgfscope}%
\begin{pgfscope}%
\pgfsetbuttcap%
\pgfsetmiterjoin%
\definecolor{currentfill}{rgb}{1.000000,1.000000,1.000000}%
\pgfsetfillcolor{currentfill}%
\pgfsetlinewidth{0.000000pt}%
\definecolor{currentstroke}{rgb}{0.000000,0.000000,0.000000}%
\pgfsetstrokecolor{currentstroke}%
\pgfsetstrokeopacity{0.000000}%
\pgfsetdash{}{0pt}%
\pgfpathmoveto{\pgfqpoint{0.553581in}{0.499444in}}%
\pgfpathlineto{\pgfqpoint{2.103581in}{0.499444in}}%
\pgfpathlineto{\pgfqpoint{2.103581in}{1.654444in}}%
\pgfpathlineto{\pgfqpoint{0.553581in}{1.654444in}}%
\pgfpathlineto{\pgfqpoint{0.553581in}{0.499444in}}%
\pgfpathclose%
\pgfusepath{fill}%
\end{pgfscope}%
\begin{pgfscope}%
\pgfsetbuttcap%
\pgfsetroundjoin%
\definecolor{currentfill}{rgb}{0.000000,0.000000,0.000000}%
\pgfsetfillcolor{currentfill}%
\pgfsetlinewidth{0.803000pt}%
\definecolor{currentstroke}{rgb}{0.000000,0.000000,0.000000}%
\pgfsetstrokecolor{currentstroke}%
\pgfsetdash{}{0pt}%
\pgfsys@defobject{currentmarker}{\pgfqpoint{0.000000in}{-0.048611in}}{\pgfqpoint{0.000000in}{0.000000in}}{%
\pgfpathmoveto{\pgfqpoint{0.000000in}{0.000000in}}%
\pgfpathlineto{\pgfqpoint{0.000000in}{-0.048611in}}%
\pgfusepath{stroke,fill}%
}%
\begin{pgfscope}%
\pgfsys@transformshift{0.624035in}{0.499444in}%
\pgfsys@useobject{currentmarker}{}%
\end{pgfscope}%
\end{pgfscope}%
\begin{pgfscope}%
\definecolor{textcolor}{rgb}{0.000000,0.000000,0.000000}%
\pgfsetstrokecolor{textcolor}%
\pgfsetfillcolor{textcolor}%
\pgftext[x=0.624035in,y=0.402222in,,top]{\color{textcolor}\rmfamily\fontsize{10.000000}{12.000000}\selectfont \(\displaystyle {0.0}\)}%
\end{pgfscope}%
\begin{pgfscope}%
\pgfsetbuttcap%
\pgfsetroundjoin%
\definecolor{currentfill}{rgb}{0.000000,0.000000,0.000000}%
\pgfsetfillcolor{currentfill}%
\pgfsetlinewidth{0.803000pt}%
\definecolor{currentstroke}{rgb}{0.000000,0.000000,0.000000}%
\pgfsetstrokecolor{currentstroke}%
\pgfsetdash{}{0pt}%
\pgfsys@defobject{currentmarker}{\pgfqpoint{0.000000in}{-0.048611in}}{\pgfqpoint{0.000000in}{0.000000in}}{%
\pgfpathmoveto{\pgfqpoint{0.000000in}{0.000000in}}%
\pgfpathlineto{\pgfqpoint{0.000000in}{-0.048611in}}%
\pgfusepath{stroke,fill}%
}%
\begin{pgfscope}%
\pgfsys@transformshift{1.328581in}{0.499444in}%
\pgfsys@useobject{currentmarker}{}%
\end{pgfscope}%
\end{pgfscope}%
\begin{pgfscope}%
\definecolor{textcolor}{rgb}{0.000000,0.000000,0.000000}%
\pgfsetstrokecolor{textcolor}%
\pgfsetfillcolor{textcolor}%
\pgftext[x=1.328581in,y=0.402222in,,top]{\color{textcolor}\rmfamily\fontsize{10.000000}{12.000000}\selectfont \(\displaystyle {0.5}\)}%
\end{pgfscope}%
\begin{pgfscope}%
\pgfsetbuttcap%
\pgfsetroundjoin%
\definecolor{currentfill}{rgb}{0.000000,0.000000,0.000000}%
\pgfsetfillcolor{currentfill}%
\pgfsetlinewidth{0.803000pt}%
\definecolor{currentstroke}{rgb}{0.000000,0.000000,0.000000}%
\pgfsetstrokecolor{currentstroke}%
\pgfsetdash{}{0pt}%
\pgfsys@defobject{currentmarker}{\pgfqpoint{0.000000in}{-0.048611in}}{\pgfqpoint{0.000000in}{0.000000in}}{%
\pgfpathmoveto{\pgfqpoint{0.000000in}{0.000000in}}%
\pgfpathlineto{\pgfqpoint{0.000000in}{-0.048611in}}%
\pgfusepath{stroke,fill}%
}%
\begin{pgfscope}%
\pgfsys@transformshift{2.033126in}{0.499444in}%
\pgfsys@useobject{currentmarker}{}%
\end{pgfscope}%
\end{pgfscope}%
\begin{pgfscope}%
\definecolor{textcolor}{rgb}{0.000000,0.000000,0.000000}%
\pgfsetstrokecolor{textcolor}%
\pgfsetfillcolor{textcolor}%
\pgftext[x=2.033126in,y=0.402222in,,top]{\color{textcolor}\rmfamily\fontsize{10.000000}{12.000000}\selectfont \(\displaystyle {1.0}\)}%
\end{pgfscope}%
\begin{pgfscope}%
\definecolor{textcolor}{rgb}{0.000000,0.000000,0.000000}%
\pgfsetstrokecolor{textcolor}%
\pgfsetfillcolor{textcolor}%
\pgftext[x=1.328581in,y=0.223333in,,top]{\color{textcolor}\rmfamily\fontsize{10.000000}{12.000000}\selectfont False positive rate}%
\end{pgfscope}%
\begin{pgfscope}%
\pgfsetbuttcap%
\pgfsetroundjoin%
\definecolor{currentfill}{rgb}{0.000000,0.000000,0.000000}%
\pgfsetfillcolor{currentfill}%
\pgfsetlinewidth{0.803000pt}%
\definecolor{currentstroke}{rgb}{0.000000,0.000000,0.000000}%
\pgfsetstrokecolor{currentstroke}%
\pgfsetdash{}{0pt}%
\pgfsys@defobject{currentmarker}{\pgfqpoint{-0.048611in}{0.000000in}}{\pgfqpoint{-0.000000in}{0.000000in}}{%
\pgfpathmoveto{\pgfqpoint{-0.000000in}{0.000000in}}%
\pgfpathlineto{\pgfqpoint{-0.048611in}{0.000000in}}%
\pgfusepath{stroke,fill}%
}%
\begin{pgfscope}%
\pgfsys@transformshift{0.553581in}{0.551944in}%
\pgfsys@useobject{currentmarker}{}%
\end{pgfscope}%
\end{pgfscope}%
\begin{pgfscope}%
\definecolor{textcolor}{rgb}{0.000000,0.000000,0.000000}%
\pgfsetstrokecolor{textcolor}%
\pgfsetfillcolor{textcolor}%
\pgftext[x=0.278889in, y=0.503750in, left, base]{\color{textcolor}\rmfamily\fontsize{10.000000}{12.000000}\selectfont \(\displaystyle {0.0}\)}%
\end{pgfscope}%
\begin{pgfscope}%
\pgfsetbuttcap%
\pgfsetroundjoin%
\definecolor{currentfill}{rgb}{0.000000,0.000000,0.000000}%
\pgfsetfillcolor{currentfill}%
\pgfsetlinewidth{0.803000pt}%
\definecolor{currentstroke}{rgb}{0.000000,0.000000,0.000000}%
\pgfsetstrokecolor{currentstroke}%
\pgfsetdash{}{0pt}%
\pgfsys@defobject{currentmarker}{\pgfqpoint{-0.048611in}{0.000000in}}{\pgfqpoint{-0.000000in}{0.000000in}}{%
\pgfpathmoveto{\pgfqpoint{-0.000000in}{0.000000in}}%
\pgfpathlineto{\pgfqpoint{-0.048611in}{0.000000in}}%
\pgfusepath{stroke,fill}%
}%
\begin{pgfscope}%
\pgfsys@transformshift{0.553581in}{1.076944in}%
\pgfsys@useobject{currentmarker}{}%
\end{pgfscope}%
\end{pgfscope}%
\begin{pgfscope}%
\definecolor{textcolor}{rgb}{0.000000,0.000000,0.000000}%
\pgfsetstrokecolor{textcolor}%
\pgfsetfillcolor{textcolor}%
\pgftext[x=0.278889in, y=1.028750in, left, base]{\color{textcolor}\rmfamily\fontsize{10.000000}{12.000000}\selectfont \(\displaystyle {0.5}\)}%
\end{pgfscope}%
\begin{pgfscope}%
\pgfsetbuttcap%
\pgfsetroundjoin%
\definecolor{currentfill}{rgb}{0.000000,0.000000,0.000000}%
\pgfsetfillcolor{currentfill}%
\pgfsetlinewidth{0.803000pt}%
\definecolor{currentstroke}{rgb}{0.000000,0.000000,0.000000}%
\pgfsetstrokecolor{currentstroke}%
\pgfsetdash{}{0pt}%
\pgfsys@defobject{currentmarker}{\pgfqpoint{-0.048611in}{0.000000in}}{\pgfqpoint{-0.000000in}{0.000000in}}{%
\pgfpathmoveto{\pgfqpoint{-0.000000in}{0.000000in}}%
\pgfpathlineto{\pgfqpoint{-0.048611in}{0.000000in}}%
\pgfusepath{stroke,fill}%
}%
\begin{pgfscope}%
\pgfsys@transformshift{0.553581in}{1.601944in}%
\pgfsys@useobject{currentmarker}{}%
\end{pgfscope}%
\end{pgfscope}%
\begin{pgfscope}%
\definecolor{textcolor}{rgb}{0.000000,0.000000,0.000000}%
\pgfsetstrokecolor{textcolor}%
\pgfsetfillcolor{textcolor}%
\pgftext[x=0.278889in, y=1.553750in, left, base]{\color{textcolor}\rmfamily\fontsize{10.000000}{12.000000}\selectfont \(\displaystyle {1.0}\)}%
\end{pgfscope}%
\begin{pgfscope}%
\definecolor{textcolor}{rgb}{0.000000,0.000000,0.000000}%
\pgfsetstrokecolor{textcolor}%
\pgfsetfillcolor{textcolor}%
\pgftext[x=0.223333in,y=1.076944in,,bottom,rotate=90.000000]{\color{textcolor}\rmfamily\fontsize{10.000000}{12.000000}\selectfont True positive rate}%
\end{pgfscope}%
\begin{pgfscope}%
\pgfpathrectangle{\pgfqpoint{0.553581in}{0.499444in}}{\pgfqpoint{1.550000in}{1.155000in}}%
\pgfusepath{clip}%
\pgfsetbuttcap%
\pgfsetroundjoin%
\pgfsetlinewidth{1.505625pt}%
\definecolor{currentstroke}{rgb}{0.000000,0.000000,0.000000}%
\pgfsetstrokecolor{currentstroke}%
\pgfsetdash{{5.550000pt}{2.400000pt}}{0.000000pt}%
\pgfpathmoveto{\pgfqpoint{0.624035in}{0.551944in}}%
\pgfpathlineto{\pgfqpoint{2.033126in}{1.601944in}}%
\pgfusepath{stroke}%
\end{pgfscope}%
\begin{pgfscope}%
\pgfpathrectangle{\pgfqpoint{0.553581in}{0.499444in}}{\pgfqpoint{1.550000in}{1.155000in}}%
\pgfusepath{clip}%
\pgfsetrectcap%
\pgfsetroundjoin%
\pgfsetlinewidth{1.505625pt}%
\definecolor{currentstroke}{rgb}{0.000000,0.000000,0.000000}%
\pgfsetstrokecolor{currentstroke}%
\pgfsetdash{}{0pt}%
\pgfpathmoveto{\pgfqpoint{0.624035in}{0.551944in}}%
\pgfpathlineto{\pgfqpoint{0.625145in}{0.568893in}}%
\pgfpathlineto{\pgfqpoint{0.625239in}{0.569949in}}%
\pgfpathlineto{\pgfqpoint{0.626341in}{0.584414in}}%
\pgfpathlineto{\pgfqpoint{0.626388in}{0.585345in}}%
\pgfpathlineto{\pgfqpoint{0.627498in}{0.595310in}}%
\pgfpathlineto{\pgfqpoint{0.627616in}{0.596396in}}%
\pgfpathlineto{\pgfqpoint{0.628726in}{0.605306in}}%
\pgfpathlineto{\pgfqpoint{0.628781in}{0.606237in}}%
\pgfpathlineto{\pgfqpoint{0.629891in}{0.616698in}}%
\pgfpathlineto{\pgfqpoint{0.629992in}{0.617785in}}%
\pgfpathlineto{\pgfqpoint{0.631102in}{0.624986in}}%
\pgfpathlineto{\pgfqpoint{0.631235in}{0.625887in}}%
\pgfpathlineto{\pgfqpoint{0.632330in}{0.633150in}}%
\pgfpathlineto{\pgfqpoint{0.632533in}{0.634175in}}%
\pgfpathlineto{\pgfqpoint{0.633643in}{0.643208in}}%
\pgfpathlineto{\pgfqpoint{0.633885in}{0.644263in}}%
\pgfpathlineto{\pgfqpoint{0.634996in}{0.651869in}}%
\pgfpathlineto{\pgfqpoint{0.635128in}{0.652924in}}%
\pgfpathlineto{\pgfqpoint{0.636239in}{0.659785in}}%
\pgfpathlineto{\pgfqpoint{0.636442in}{0.660840in}}%
\pgfpathlineto{\pgfqpoint{0.637552in}{0.666862in}}%
\pgfpathlineto{\pgfqpoint{0.637833in}{0.667949in}}%
\pgfpathlineto{\pgfqpoint{0.638943in}{0.674219in}}%
\pgfpathlineto{\pgfqpoint{0.639084in}{0.675306in}}%
\pgfpathlineto{\pgfqpoint{0.640194in}{0.680955in}}%
\pgfpathlineto{\pgfqpoint{0.640444in}{0.682042in}}%
\pgfpathlineto{\pgfqpoint{0.641547in}{0.687567in}}%
\pgfpathlineto{\pgfqpoint{0.641766in}{0.688654in}}%
\pgfpathlineto{\pgfqpoint{0.642868in}{0.693124in}}%
\pgfpathlineto{\pgfqpoint{0.643040in}{0.694210in}}%
\pgfpathlineto{\pgfqpoint{0.644111in}{0.699549in}}%
\pgfpathlineto{\pgfqpoint{0.644431in}{0.700574in}}%
\pgfpathlineto{\pgfqpoint{0.645542in}{0.705696in}}%
\pgfpathlineto{\pgfqpoint{0.645901in}{0.706782in}}%
\pgfpathlineto{\pgfqpoint{0.647003in}{0.711687in}}%
\pgfpathlineto{\pgfqpoint{0.647308in}{0.712773in}}%
\pgfpathlineto{\pgfqpoint{0.648418in}{0.717585in}}%
\pgfpathlineto{\pgfqpoint{0.648661in}{0.718609in}}%
\pgfpathlineto{\pgfqpoint{0.649771in}{0.723390in}}%
\pgfpathlineto{\pgfqpoint{0.650076in}{0.724476in}}%
\pgfpathlineto{\pgfqpoint{0.651186in}{0.727953in}}%
\pgfpathlineto{\pgfqpoint{0.651483in}{0.729040in}}%
\pgfpathlineto{\pgfqpoint{0.652593in}{0.733820in}}%
\pgfpathlineto{\pgfqpoint{0.652921in}{0.734813in}}%
\pgfpathlineto{\pgfqpoint{0.654016in}{0.740059in}}%
\pgfpathlineto{\pgfqpoint{0.654462in}{0.741115in}}%
\pgfpathlineto{\pgfqpoint{0.655556in}{0.745740in}}%
\pgfpathlineto{\pgfqpoint{0.655861in}{0.746827in}}%
\pgfpathlineto{\pgfqpoint{0.656971in}{0.750334in}}%
\pgfpathlineto{\pgfqpoint{0.657323in}{0.751390in}}%
\pgfpathlineto{\pgfqpoint{0.658425in}{0.754711in}}%
\pgfpathlineto{\pgfqpoint{0.658800in}{0.755736in}}%
\pgfpathlineto{\pgfqpoint{0.659895in}{0.759492in}}%
\pgfpathlineto{\pgfqpoint{0.660161in}{0.760516in}}%
\pgfpathlineto{\pgfqpoint{0.661271in}{0.765173in}}%
\pgfpathlineto{\pgfqpoint{0.661544in}{0.766228in}}%
\pgfpathlineto{\pgfqpoint{0.662647in}{0.769922in}}%
\pgfpathlineto{\pgfqpoint{0.663037in}{0.770853in}}%
\pgfpathlineto{\pgfqpoint{0.664148in}{0.774144in}}%
\pgfpathlineto{\pgfqpoint{0.664531in}{0.775199in}}%
\pgfpathlineto{\pgfqpoint{0.665641in}{0.778769in}}%
\pgfpathlineto{\pgfqpoint{0.666094in}{0.779855in}}%
\pgfpathlineto{\pgfqpoint{0.667204in}{0.783239in}}%
\pgfpathlineto{\pgfqpoint{0.667462in}{0.784294in}}%
\pgfpathlineto{\pgfqpoint{0.668565in}{0.787802in}}%
\pgfpathlineto{\pgfqpoint{0.669010in}{0.788889in}}%
\pgfpathlineto{\pgfqpoint{0.670120in}{0.792521in}}%
\pgfpathlineto{\pgfqpoint{0.670511in}{0.793607in}}%
\pgfpathlineto{\pgfqpoint{0.671582in}{0.797860in}}%
\pgfpathlineto{\pgfqpoint{0.671981in}{0.798915in}}%
\pgfpathlineto{\pgfqpoint{0.671981in}{0.798946in}}%
\pgfpathlineto{\pgfqpoint{0.673091in}{0.802113in}}%
\pgfpathlineto{\pgfqpoint{0.673568in}{0.803199in}}%
\pgfpathlineto{\pgfqpoint{0.674670in}{0.806241in}}%
\pgfpathlineto{\pgfqpoint{0.675163in}{0.807328in}}%
\pgfpathlineto{\pgfqpoint{0.676265in}{0.810370in}}%
\pgfpathlineto{\pgfqpoint{0.676562in}{0.811425in}}%
\pgfpathlineto{\pgfqpoint{0.677672in}{0.814467in}}%
\pgfpathlineto{\pgfqpoint{0.678079in}{0.815461in}}%
\pgfpathlineto{\pgfqpoint{0.679181in}{0.819093in}}%
\pgfpathlineto{\pgfqpoint{0.679689in}{0.820148in}}%
\pgfpathlineto{\pgfqpoint{0.680776in}{0.823066in}}%
\pgfpathlineto{\pgfqpoint{0.681104in}{0.824091in}}%
\pgfpathlineto{\pgfqpoint{0.682214in}{0.826760in}}%
\pgfpathlineto{\pgfqpoint{0.682668in}{0.827847in}}%
\pgfpathlineto{\pgfqpoint{0.683778in}{0.831013in}}%
\pgfpathlineto{\pgfqpoint{0.684239in}{0.832099in}}%
\pgfpathlineto{\pgfqpoint{0.685302in}{0.834831in}}%
\pgfpathlineto{\pgfqpoint{0.685888in}{0.835918in}}%
\pgfpathlineto{\pgfqpoint{0.686991in}{0.839332in}}%
\pgfpathlineto{\pgfqpoint{0.687475in}{0.840419in}}%
\pgfpathlineto{\pgfqpoint{0.688578in}{0.843523in}}%
\pgfpathlineto{\pgfqpoint{0.689039in}{0.844578in}}%
\pgfpathlineto{\pgfqpoint{0.690141in}{0.846844in}}%
\pgfpathlineto{\pgfqpoint{0.690790in}{0.847931in}}%
\pgfpathlineto{\pgfqpoint{0.691900in}{0.850414in}}%
\pgfpathlineto{\pgfqpoint{0.692393in}{0.851439in}}%
\pgfpathlineto{\pgfqpoint{0.693495in}{0.853922in}}%
\pgfpathlineto{\pgfqpoint{0.693792in}{0.854915in}}%
\pgfpathlineto{\pgfqpoint{0.694871in}{0.857926in}}%
\pgfpathlineto{\pgfqpoint{0.695301in}{0.859013in}}%
\pgfpathlineto{\pgfqpoint{0.696395in}{0.861372in}}%
\pgfpathlineto{\pgfqpoint{0.696880in}{0.862459in}}%
\pgfpathlineto{\pgfqpoint{0.697990in}{0.865345in}}%
\pgfpathlineto{\pgfqpoint{0.698475in}{0.866370in}}%
\pgfpathlineto{\pgfqpoint{0.699538in}{0.869381in}}%
\pgfpathlineto{\pgfqpoint{0.700023in}{0.870405in}}%
\pgfpathlineto{\pgfqpoint{0.701133in}{0.873013in}}%
\pgfpathlineto{\pgfqpoint{0.701672in}{0.874037in}}%
\pgfpathlineto{\pgfqpoint{0.702751in}{0.876521in}}%
\pgfpathlineto{\pgfqpoint{0.703181in}{0.877607in}}%
\pgfpathlineto{\pgfqpoint{0.704291in}{0.880494in}}%
\pgfpathlineto{\pgfqpoint{0.704838in}{0.881581in}}%
\pgfpathlineto{\pgfqpoint{0.705941in}{0.884561in}}%
\pgfpathlineto{\pgfqpoint{0.706543in}{0.885585in}}%
\pgfpathlineto{\pgfqpoint{0.707653in}{0.887975in}}%
\pgfpathlineto{\pgfqpoint{0.708075in}{0.889062in}}%
\pgfpathlineto{\pgfqpoint{0.709185in}{0.891762in}}%
\pgfpathlineto{\pgfqpoint{0.709842in}{0.892849in}}%
\pgfpathlineto{\pgfqpoint{0.710928in}{0.895860in}}%
\pgfpathlineto{\pgfqpoint{0.711476in}{0.896946in}}%
\pgfpathlineto{\pgfqpoint{0.712547in}{0.899554in}}%
\pgfpathlineto{\pgfqpoint{0.713297in}{0.900640in}}%
\pgfpathlineto{\pgfqpoint{0.714392in}{0.902844in}}%
\pgfpathlineto{\pgfqpoint{0.714853in}{0.903745in}}%
\pgfpathlineto{\pgfqpoint{0.714853in}{0.903900in}}%
\pgfpathlineto{\pgfqpoint{0.715955in}{0.906290in}}%
\pgfpathlineto{\pgfqpoint{0.716643in}{0.907377in}}%
\pgfpathlineto{\pgfqpoint{0.717753in}{0.909363in}}%
\pgfpathlineto{\pgfqpoint{0.718128in}{0.910450in}}%
\pgfpathlineto{\pgfqpoint{0.719238in}{0.912964in}}%
\pgfpathlineto{\pgfqpoint{0.719723in}{0.914051in}}%
\pgfpathlineto{\pgfqpoint{0.720825in}{0.916161in}}%
\pgfpathlineto{\pgfqpoint{0.721631in}{0.917248in}}%
\pgfpathlineto{\pgfqpoint{0.722725in}{0.919142in}}%
\pgfpathlineto{\pgfqpoint{0.723225in}{0.920228in}}%
\pgfpathlineto{\pgfqpoint{0.724320in}{0.923115in}}%
\pgfpathlineto{\pgfqpoint{0.724875in}{0.924201in}}%
\pgfpathlineto{\pgfqpoint{0.725954in}{0.926312in}}%
\pgfpathlineto{\pgfqpoint{0.726525in}{0.927399in}}%
\pgfpathlineto{\pgfqpoint{0.727627in}{0.929789in}}%
\pgfpathlineto{\pgfqpoint{0.728221in}{0.930875in}}%
\pgfpathlineto{\pgfqpoint{0.729323in}{0.933079in}}%
\pgfpathlineto{\pgfqpoint{0.729824in}{0.934166in}}%
\pgfpathlineto{\pgfqpoint{0.730887in}{0.936184in}}%
\pgfpathlineto{\pgfqpoint{0.731637in}{0.937270in}}%
\pgfpathlineto{\pgfqpoint{0.732732in}{0.940033in}}%
\pgfpathlineto{\pgfqpoint{0.733240in}{0.941119in}}%
\pgfpathlineto{\pgfqpoint{0.734334in}{0.943447in}}%
\pgfpathlineto{\pgfqpoint{0.734975in}{0.944534in}}%
\pgfpathlineto{\pgfqpoint{0.736078in}{0.947017in}}%
\pgfpathlineto{\pgfqpoint{0.736953in}{0.948104in}}%
\pgfpathlineto{\pgfqpoint{0.738040in}{0.949997in}}%
\pgfpathlineto{\pgfqpoint{0.738470in}{0.951084in}}%
\pgfpathlineto{\pgfqpoint{0.739549in}{0.953381in}}%
\pgfpathlineto{\pgfqpoint{0.739572in}{0.953381in}}%
\pgfpathlineto{\pgfqpoint{0.740049in}{0.954467in}}%
\pgfpathlineto{\pgfqpoint{0.741159in}{0.955957in}}%
\pgfpathlineto{\pgfqpoint{0.741706in}{0.957044in}}%
\pgfpathlineto{\pgfqpoint{0.742770in}{0.959031in}}%
\pgfpathlineto{\pgfqpoint{0.742801in}{0.959031in}}%
\pgfpathlineto{\pgfqpoint{0.743340in}{0.960117in}}%
\pgfpathlineto{\pgfqpoint{0.744450in}{0.961980in}}%
\pgfpathlineto{\pgfqpoint{0.745334in}{0.963066in}}%
\pgfpathlineto{\pgfqpoint{0.746413in}{0.965953in}}%
\pgfpathlineto{\pgfqpoint{0.746944in}{0.967040in}}%
\pgfpathlineto{\pgfqpoint{0.748046in}{0.969181in}}%
\pgfpathlineto{\pgfqpoint{0.748688in}{0.970268in}}%
\pgfpathlineto{\pgfqpoint{0.749790in}{0.971882in}}%
\pgfpathlineto{\pgfqpoint{0.750533in}{0.972969in}}%
\pgfpathlineto{\pgfqpoint{0.751627in}{0.974924in}}%
\pgfpathlineto{\pgfqpoint{0.752714in}{0.976011in}}%
\pgfpathlineto{\pgfqpoint{0.753808in}{0.977501in}}%
\pgfpathlineto{\pgfqpoint{0.754535in}{0.978587in}}%
\pgfpathlineto{\pgfqpoint{0.755630in}{0.980481in}}%
\pgfpathlineto{\pgfqpoint{0.756318in}{0.981567in}}%
\pgfpathlineto{\pgfqpoint{0.757412in}{0.983275in}}%
\pgfpathlineto{\pgfqpoint{0.757428in}{0.983275in}}%
\pgfpathlineto{\pgfqpoint{0.758092in}{0.984330in}}%
\pgfpathlineto{\pgfqpoint{0.759179in}{0.985975in}}%
\pgfpathlineto{\pgfqpoint{0.759945in}{0.987062in}}%
\pgfpathlineto{\pgfqpoint{0.761024in}{0.988738in}}%
\pgfpathlineto{\pgfqpoint{0.761884in}{0.989824in}}%
\pgfpathlineto{\pgfqpoint{0.762963in}{0.991749in}}%
\pgfpathlineto{\pgfqpoint{0.763705in}{0.992836in}}%
\pgfpathlineto{\pgfqpoint{0.764815in}{0.994543in}}%
\pgfpathlineto{\pgfqpoint{0.765433in}{0.995629in}}%
\pgfpathlineto{\pgfqpoint{0.766512in}{0.997647in}}%
\pgfpathlineto{\pgfqpoint{0.767270in}{0.998702in}}%
\pgfpathlineto{\pgfqpoint{0.768380in}{1.001062in}}%
\pgfpathlineto{\pgfqpoint{0.768998in}{1.002117in}}%
\pgfpathlineto{\pgfqpoint{0.770069in}{1.004197in}}%
\pgfpathlineto{\pgfqpoint{0.770600in}{1.005221in}}%
\pgfpathlineto{\pgfqpoint{0.771679in}{1.006804in}}%
\pgfpathlineto{\pgfqpoint{0.771695in}{1.006804in}}%
\pgfpathlineto{\pgfqpoint{0.772289in}{1.007891in}}%
\pgfpathlineto{\pgfqpoint{0.773383in}{1.009412in}}%
\pgfpathlineto{\pgfqpoint{0.774032in}{1.010498in}}%
\pgfpathlineto{\pgfqpoint{0.775142in}{1.011895in}}%
\pgfpathlineto{\pgfqpoint{0.775987in}{1.012951in}}%
\pgfpathlineto{\pgfqpoint{0.777097in}{1.014844in}}%
\pgfpathlineto{\pgfqpoint{0.777754in}{1.015931in}}%
\pgfpathlineto{\pgfqpoint{0.778840in}{1.017762in}}%
\pgfpathlineto{\pgfqpoint{0.779599in}{1.018849in}}%
\pgfpathlineto{\pgfqpoint{0.780709in}{1.021053in}}%
\pgfpathlineto{\pgfqpoint{0.781608in}{1.022139in}}%
\pgfpathlineto{\pgfqpoint{0.782640in}{1.023691in}}%
\pgfpathlineto{\pgfqpoint{0.783437in}{1.024778in}}%
\pgfpathlineto{\pgfqpoint{0.784508in}{1.026485in}}%
\pgfpathlineto{\pgfqpoint{0.785274in}{1.027572in}}%
\pgfpathlineto{\pgfqpoint{0.786376in}{1.029434in}}%
\pgfpathlineto{\pgfqpoint{0.786853in}{1.030490in}}%
\pgfpathlineto{\pgfqpoint{0.787948in}{1.032166in}}%
\pgfpathlineto{\pgfqpoint{0.788659in}{1.033252in}}%
\pgfpathlineto{\pgfqpoint{0.789769in}{1.035208in}}%
\pgfpathlineto{\pgfqpoint{0.790434in}{1.036294in}}%
\pgfpathlineto{\pgfqpoint{0.791520in}{1.038343in}}%
\pgfpathlineto{\pgfqpoint{0.791544in}{1.038343in}}%
\pgfpathlineto{\pgfqpoint{0.792068in}{1.039430in}}%
\pgfpathlineto{\pgfqpoint{0.793131in}{1.040951in}}%
\pgfpathlineto{\pgfqpoint{0.793960in}{1.042037in}}%
\pgfpathlineto{\pgfqpoint{0.795070in}{1.043527in}}%
\pgfpathlineto{\pgfqpoint{0.795914in}{1.044583in}}%
\pgfpathlineto{\pgfqpoint{0.797016in}{1.046166in}}%
\pgfpathlineto{\pgfqpoint{0.798236in}{1.047252in}}%
\pgfpathlineto{\pgfqpoint{0.799276in}{1.048711in}}%
\pgfpathlineto{\pgfqpoint{0.799322in}{1.048711in}}%
\pgfpathlineto{\pgfqpoint{0.800120in}{1.049798in}}%
\pgfpathlineto{\pgfqpoint{0.801167in}{1.051567in}}%
\pgfpathlineto{\pgfqpoint{0.801918in}{1.052654in}}%
\pgfpathlineto{\pgfqpoint{0.802997in}{1.054237in}}%
\pgfpathlineto{\pgfqpoint{0.803896in}{1.055323in}}%
\pgfpathlineto{\pgfqpoint{0.804935in}{1.056906in}}%
\pgfpathlineto{\pgfqpoint{0.805670in}{1.057962in}}%
\pgfpathlineto{\pgfqpoint{0.806749in}{1.059390in}}%
\pgfpathlineto{\pgfqpoint{0.808055in}{1.060445in}}%
\pgfpathlineto{\pgfqpoint{0.809165in}{1.061997in}}%
\pgfpathlineto{\pgfqpoint{0.810126in}{1.063084in}}%
\pgfpathlineto{\pgfqpoint{0.811236in}{1.065040in}}%
\pgfpathlineto{\pgfqpoint{0.812057in}{1.066064in}}%
\pgfpathlineto{\pgfqpoint{0.813152in}{1.067368in}}%
\pgfpathlineto{\pgfqpoint{0.814184in}{1.068454in}}%
\pgfpathlineto{\pgfqpoint{0.815270in}{1.070099in}}%
\pgfpathlineto{\pgfqpoint{0.816044in}{1.071155in}}%
\pgfpathlineto{\pgfqpoint{0.817154in}{1.072210in}}%
\pgfpathlineto{\pgfqpoint{0.817999in}{1.073266in}}%
\pgfpathlineto{\pgfqpoint{0.819101in}{1.074973in}}%
\pgfpathlineto{\pgfqpoint{0.819883in}{1.075997in}}%
\pgfpathlineto{\pgfqpoint{0.820962in}{1.078046in}}%
\pgfpathlineto{\pgfqpoint{0.821814in}{1.079102in}}%
\pgfpathlineto{\pgfqpoint{0.822924in}{1.080902in}}%
\pgfpathlineto{\pgfqpoint{0.823909in}{1.081989in}}%
\pgfpathlineto{\pgfqpoint{0.825011in}{1.083323in}}%
\pgfpathlineto{\pgfqpoint{0.825613in}{1.084410in}}%
\pgfpathlineto{\pgfqpoint{0.826692in}{1.086210in}}%
\pgfpathlineto{\pgfqpoint{0.827833in}{1.087297in}}%
\pgfpathlineto{\pgfqpoint{0.828936in}{1.088569in}}%
\pgfpathlineto{\pgfqpoint{0.830030in}{1.089656in}}%
\pgfpathlineto{\pgfqpoint{0.831132in}{1.091332in}}%
\pgfpathlineto{\pgfqpoint{0.832188in}{1.092419in}}%
\pgfpathlineto{\pgfqpoint{0.833235in}{1.094064in}}%
\pgfpathlineto{\pgfqpoint{0.833962in}{1.095119in}}%
\pgfpathlineto{\pgfqpoint{0.835065in}{1.096423in}}%
\pgfpathlineto{\pgfqpoint{0.836034in}{1.097479in}}%
\pgfpathlineto{\pgfqpoint{0.837136in}{1.099341in}}%
\pgfpathlineto{\pgfqpoint{0.838082in}{1.100428in}}%
\pgfpathlineto{\pgfqpoint{0.839192in}{1.102135in}}%
\pgfpathlineto{\pgfqpoint{0.839701in}{1.103221in}}%
\pgfpathlineto{\pgfqpoint{0.840748in}{1.104246in}}%
\pgfpathlineto{\pgfqpoint{0.841639in}{1.105332in}}%
\pgfpathlineto{\pgfqpoint{0.842726in}{1.106729in}}%
\pgfpathlineto{\pgfqpoint{0.843688in}{1.107753in}}%
\pgfpathlineto{\pgfqpoint{0.844782in}{1.108809in}}%
\pgfpathlineto{\pgfqpoint{0.845658in}{1.109895in}}%
\pgfpathlineto{\pgfqpoint{0.846768in}{1.111541in}}%
\pgfpathlineto{\pgfqpoint{0.847534in}{1.112627in}}%
\pgfpathlineto{\pgfqpoint{0.848636in}{1.113900in}}%
\pgfpathlineto{\pgfqpoint{0.849519in}{1.114986in}}%
\pgfpathlineto{\pgfqpoint{0.850630in}{1.116042in}}%
\pgfpathlineto{\pgfqpoint{0.851615in}{1.117128in}}%
\pgfpathlineto{\pgfqpoint{0.852748in}{1.118649in}}%
\pgfpathlineto{\pgfqpoint{0.853796in}{1.119705in}}%
\pgfpathlineto{\pgfqpoint{0.854875in}{1.120822in}}%
\pgfpathlineto{\pgfqpoint{0.855867in}{1.121909in}}%
\pgfpathlineto{\pgfqpoint{0.856962in}{1.123306in}}%
\pgfpathlineto{\pgfqpoint{0.857900in}{1.124392in}}%
\pgfpathlineto{\pgfqpoint{0.859002in}{1.126317in}}%
\pgfpathlineto{\pgfqpoint{0.860339in}{1.127372in}}%
\pgfpathlineto{\pgfqpoint{0.861441in}{1.128583in}}%
\pgfpathlineto{\pgfqpoint{0.862684in}{1.129669in}}%
\pgfpathlineto{\pgfqpoint{0.863771in}{1.130911in}}%
\pgfpathlineto{\pgfqpoint{0.864803in}{1.131966in}}%
\pgfpathlineto{\pgfqpoint{0.865874in}{1.133239in}}%
\pgfpathlineto{\pgfqpoint{0.867187in}{1.134326in}}%
\pgfpathlineto{\pgfqpoint{0.868282in}{1.135785in}}%
\pgfpathlineto{\pgfqpoint{0.869759in}{1.136871in}}%
\pgfpathlineto{\pgfqpoint{0.870869in}{1.138082in}}%
\pgfpathlineto{\pgfqpoint{0.871987in}{1.139137in}}%
\pgfpathlineto{\pgfqpoint{0.873098in}{1.140348in}}%
\pgfpathlineto{\pgfqpoint{0.874200in}{1.141434in}}%
\pgfpathlineto{\pgfqpoint{0.875310in}{1.142521in}}%
\pgfpathlineto{\pgfqpoint{0.876506in}{1.143607in}}%
\pgfpathlineto{\pgfqpoint{0.877616in}{1.144787in}}%
\pgfpathlineto{\pgfqpoint{0.878523in}{1.145842in}}%
\pgfpathlineto{\pgfqpoint{0.879633in}{1.147332in}}%
\pgfpathlineto{\pgfqpoint{0.880696in}{1.148419in}}%
\pgfpathlineto{\pgfqpoint{0.881720in}{1.149226in}}%
\pgfpathlineto{\pgfqpoint{0.883143in}{1.150312in}}%
\pgfpathlineto{\pgfqpoint{0.884253in}{1.151616in}}%
\pgfpathlineto{\pgfqpoint{0.885418in}{1.152702in}}%
\pgfpathlineto{\pgfqpoint{0.886481in}{1.153944in}}%
\pgfpathlineto{\pgfqpoint{0.886520in}{1.153944in}}%
\pgfpathlineto{\pgfqpoint{0.887818in}{1.155031in}}%
\pgfpathlineto{\pgfqpoint{0.888928in}{1.156521in}}%
\pgfpathlineto{\pgfqpoint{0.889929in}{1.157607in}}%
\pgfpathlineto{\pgfqpoint{0.891039in}{1.158600in}}%
\pgfpathlineto{\pgfqpoint{0.891891in}{1.159687in}}%
\pgfpathlineto{\pgfqpoint{0.892962in}{1.160680in}}%
\pgfpathlineto{\pgfqpoint{0.894002in}{1.161767in}}%
\pgfpathlineto{\pgfqpoint{0.895089in}{1.162698in}}%
\pgfpathlineto{\pgfqpoint{0.896034in}{1.163785in}}%
\pgfpathlineto{\pgfqpoint{0.897113in}{1.165119in}}%
\pgfpathlineto{\pgfqpoint{0.898067in}{1.166206in}}%
\pgfpathlineto{\pgfqpoint{0.899130in}{1.167572in}}%
\pgfpathlineto{\pgfqpoint{0.900279in}{1.168658in}}%
\pgfpathlineto{\pgfqpoint{0.901390in}{1.169838in}}%
\pgfpathlineto{\pgfqpoint{0.902672in}{1.170924in}}%
\pgfpathlineto{\pgfqpoint{0.903774in}{1.172042in}}%
\pgfpathlineto{\pgfqpoint{0.904853in}{1.173128in}}%
\pgfpathlineto{\pgfqpoint{0.905900in}{1.174122in}}%
\pgfpathlineto{\pgfqpoint{0.906721in}{1.175208in}}%
\pgfpathlineto{\pgfqpoint{0.907800in}{1.176388in}}%
\pgfpathlineto{\pgfqpoint{0.909090in}{1.177474in}}%
\pgfpathlineto{\pgfqpoint{0.910169in}{1.178064in}}%
\pgfpathlineto{\pgfqpoint{0.911310in}{1.179119in}}%
\pgfpathlineto{\pgfqpoint{0.912412in}{1.180082in}}%
\pgfpathlineto{\pgfqpoint{0.913499in}{1.181106in}}%
\pgfpathlineto{\pgfqpoint{0.914609in}{1.182503in}}%
\pgfpathlineto{\pgfqpoint{0.916079in}{1.183558in}}%
\pgfpathlineto{\pgfqpoint{0.917189in}{1.185017in}}%
\pgfpathlineto{\pgfqpoint{0.918080in}{1.186104in}}%
\pgfpathlineto{\pgfqpoint{0.919175in}{1.187625in}}%
\pgfpathlineto{\pgfqpoint{0.920183in}{1.188711in}}%
\pgfpathlineto{\pgfqpoint{0.921207in}{1.189643in}}%
\pgfpathlineto{\pgfqpoint{0.922450in}{1.190698in}}%
\pgfpathlineto{\pgfqpoint{0.923545in}{1.191722in}}%
\pgfpathlineto{\pgfqpoint{0.924889in}{1.192809in}}%
\pgfpathlineto{\pgfqpoint{0.925992in}{1.194299in}}%
\pgfpathlineto{\pgfqpoint{0.927149in}{1.195354in}}%
\pgfpathlineto{\pgfqpoint{0.928259in}{1.196534in}}%
\pgfpathlineto{\pgfqpoint{0.929486in}{1.197620in}}%
\pgfpathlineto{\pgfqpoint{0.930463in}{1.198428in}}%
\pgfpathlineto{\pgfqpoint{0.931878in}{1.199514in}}%
\pgfpathlineto{\pgfqpoint{0.932942in}{1.200414in}}%
\pgfpathlineto{\pgfqpoint{0.932981in}{1.200414in}}%
\pgfpathlineto{\pgfqpoint{0.934646in}{1.201439in}}%
\pgfpathlineto{\pgfqpoint{0.935646in}{1.202370in}}%
\pgfpathlineto{\pgfqpoint{0.937194in}{1.203456in}}%
\pgfpathlineto{\pgfqpoint{0.938289in}{1.204605in}}%
\pgfpathlineto{\pgfqpoint{0.939375in}{1.205691in}}%
\pgfpathlineto{\pgfqpoint{0.940478in}{1.206654in}}%
\pgfpathlineto{\pgfqpoint{0.941588in}{1.207709in}}%
\pgfpathlineto{\pgfqpoint{0.942682in}{1.208858in}}%
\pgfpathlineto{\pgfqpoint{0.944347in}{1.209944in}}%
\pgfpathlineto{\pgfqpoint{0.945426in}{1.210875in}}%
\pgfpathlineto{\pgfqpoint{0.946372in}{1.211962in}}%
\pgfpathlineto{\pgfqpoint{0.947467in}{1.213048in}}%
\pgfpathlineto{\pgfqpoint{0.948600in}{1.214135in}}%
\pgfpathlineto{\pgfqpoint{0.949695in}{1.215190in}}%
\pgfpathlineto{\pgfqpoint{0.950891in}{1.216277in}}%
\pgfpathlineto{\pgfqpoint{0.952001in}{1.217487in}}%
\pgfpathlineto{\pgfqpoint{0.953588in}{1.218574in}}%
\pgfpathlineto{\pgfqpoint{0.954659in}{1.219505in}}%
\pgfpathlineto{\pgfqpoint{0.954690in}{1.219505in}}%
\pgfpathlineto{\pgfqpoint{0.955417in}{1.220561in}}%
\pgfpathlineto{\pgfqpoint{0.956520in}{1.222020in}}%
\pgfpathlineto{\pgfqpoint{0.957778in}{1.223106in}}%
\pgfpathlineto{\pgfqpoint{0.958865in}{1.224037in}}%
\pgfpathlineto{\pgfqpoint{0.960014in}{1.225093in}}%
\pgfpathlineto{\pgfqpoint{0.961069in}{1.226086in}}%
\pgfpathlineto{\pgfqpoint{0.962461in}{1.227173in}}%
\pgfpathlineto{\pgfqpoint{0.963548in}{1.228073in}}%
\pgfpathlineto{\pgfqpoint{0.965111in}{1.229128in}}%
\pgfpathlineto{\pgfqpoint{0.966198in}{1.230153in}}%
\pgfpathlineto{\pgfqpoint{0.967347in}{1.231239in}}%
\pgfpathlineto{\pgfqpoint{0.968387in}{1.232294in}}%
\pgfpathlineto{\pgfqpoint{0.969880in}{1.233381in}}%
\pgfpathlineto{\pgfqpoint{0.970935in}{1.234467in}}%
\pgfpathlineto{\pgfqpoint{0.971952in}{1.235554in}}%
\pgfpathlineto{\pgfqpoint{0.973062in}{1.236392in}}%
\pgfpathlineto{\pgfqpoint{0.974438in}{1.237479in}}%
\pgfpathlineto{\pgfqpoint{0.975430in}{1.238255in}}%
\pgfpathlineto{\pgfqpoint{0.976908in}{1.239341in}}%
\pgfpathlineto{\pgfqpoint{0.978018in}{1.240396in}}%
\pgfpathlineto{\pgfqpoint{0.979644in}{1.241483in}}%
\pgfpathlineto{\pgfqpoint{0.980746in}{1.242694in}}%
\pgfpathlineto{\pgfqpoint{0.982193in}{1.243780in}}%
\pgfpathlineto{\pgfqpoint{0.983295in}{1.244960in}}%
\pgfpathlineto{\pgfqpoint{0.984444in}{1.246046in}}%
\pgfpathlineto{\pgfqpoint{0.985546in}{1.246946in}}%
\pgfpathlineto{\pgfqpoint{0.987352in}{1.248002in}}%
\pgfpathlineto{\pgfqpoint{0.988392in}{1.249026in}}%
\pgfpathlineto{\pgfqpoint{0.989971in}{1.250113in}}%
\pgfpathlineto{\pgfqpoint{0.991066in}{1.250982in}}%
\pgfpathlineto{\pgfqpoint{0.992184in}{1.252068in}}%
\pgfpathlineto{\pgfqpoint{0.993098in}{1.252751in}}%
\pgfpathlineto{\pgfqpoint{0.993176in}{1.252751in}}%
\pgfpathlineto{\pgfqpoint{0.994889in}{1.253838in}}%
\pgfpathlineto{\pgfqpoint{0.995967in}{1.254645in}}%
\pgfpathlineto{\pgfqpoint{0.997203in}{1.255731in}}%
\pgfpathlineto{\pgfqpoint{0.998258in}{1.256414in}}%
\pgfpathlineto{\pgfqpoint{0.999407in}{1.257501in}}%
\pgfpathlineto{\pgfqpoint{1.000455in}{1.258401in}}%
\pgfpathlineto{\pgfqpoint{1.000509in}{1.258401in}}%
\pgfpathlineto{\pgfqpoint{1.002143in}{1.259456in}}%
\pgfpathlineto{\pgfqpoint{1.003214in}{1.260481in}}%
\pgfpathlineto{\pgfqpoint{1.005020in}{1.261536in}}%
\pgfpathlineto{\pgfqpoint{1.006130in}{1.262498in}}%
\pgfpathlineto{\pgfqpoint{1.007451in}{1.263585in}}%
\pgfpathlineto{\pgfqpoint{1.008554in}{1.264392in}}%
\pgfpathlineto{\pgfqpoint{1.010188in}{1.265447in}}%
\pgfpathlineto{\pgfqpoint{1.011274in}{1.266317in}}%
\pgfpathlineto{\pgfqpoint{1.012861in}{1.267403in}}%
\pgfpathlineto{\pgfqpoint{1.013948in}{1.268490in}}%
\pgfpathlineto{\pgfqpoint{1.015379in}{1.269576in}}%
\pgfpathlineto{\pgfqpoint{1.016489in}{1.271035in}}%
\pgfpathlineto{\pgfqpoint{1.017505in}{1.272122in}}%
\pgfpathlineto{\pgfqpoint{1.018615in}{1.273208in}}%
\pgfpathlineto{\pgfqpoint{1.020116in}{1.274263in}}%
\pgfpathlineto{\pgfqpoint{1.021171in}{1.274915in}}%
\pgfpathlineto{\pgfqpoint{1.022719in}{1.276002in}}%
\pgfpathlineto{\pgfqpoint{1.023822in}{1.277306in}}%
\pgfpathlineto{\pgfqpoint{1.025096in}{1.278392in}}%
\pgfpathlineto{\pgfqpoint{1.026206in}{1.279572in}}%
\pgfpathlineto{\pgfqpoint{1.027472in}{1.280658in}}%
\pgfpathlineto{\pgfqpoint{1.028567in}{1.281745in}}%
\pgfpathlineto{\pgfqpoint{1.029974in}{1.282831in}}%
\pgfpathlineto{\pgfqpoint{1.031084in}{1.283980in}}%
\pgfpathlineto{\pgfqpoint{1.032812in}{1.285066in}}%
\pgfpathlineto{\pgfqpoint{1.033867in}{1.286401in}}%
\pgfpathlineto{\pgfqpoint{1.035384in}{1.287487in}}%
\pgfpathlineto{\pgfqpoint{1.036478in}{1.288388in}}%
\pgfpathlineto{\pgfqpoint{1.037901in}{1.289474in}}%
\pgfpathlineto{\pgfqpoint{1.038925in}{1.290095in}}%
\pgfpathlineto{\pgfqpoint{1.040137in}{1.291181in}}%
\pgfpathlineto{\pgfqpoint{1.041153in}{1.291833in}}%
\pgfpathlineto{\pgfqpoint{1.042686in}{1.292920in}}%
\pgfpathlineto{\pgfqpoint{1.043780in}{1.294161in}}%
\pgfpathlineto{\pgfqpoint{1.044953in}{1.295217in}}%
\pgfpathlineto{\pgfqpoint{1.045985in}{1.296086in}}%
\pgfpathlineto{\pgfqpoint{1.046008in}{1.296086in}}%
\pgfpathlineto{\pgfqpoint{1.047462in}{1.297142in}}%
\pgfpathlineto{\pgfqpoint{1.048400in}{1.297762in}}%
\pgfpathlineto{\pgfqpoint{1.050613in}{1.298849in}}%
\pgfpathlineto{\pgfqpoint{1.051684in}{1.299842in}}%
\pgfpathlineto{\pgfqpoint{1.053380in}{1.300929in}}%
\pgfpathlineto{\pgfqpoint{1.054412in}{1.301674in}}%
\pgfpathlineto{\pgfqpoint{1.055796in}{1.302760in}}%
\pgfpathlineto{\pgfqpoint{1.056804in}{1.303474in}}%
\pgfpathlineto{\pgfqpoint{1.058282in}{1.304561in}}%
\pgfpathlineto{\pgfqpoint{1.059314in}{1.305554in}}%
\pgfpathlineto{\pgfqpoint{1.061143in}{1.306640in}}%
\pgfpathlineto{\pgfqpoint{1.062245in}{1.307354in}}%
\pgfpathlineto{\pgfqpoint{1.064012in}{1.308441in}}%
\pgfpathlineto{\pgfqpoint{1.065107in}{1.309062in}}%
\pgfpathlineto{\pgfqpoint{1.066350in}{1.310117in}}%
\pgfpathlineto{\pgfqpoint{1.067421in}{1.311079in}}%
\pgfpathlineto{\pgfqpoint{1.068718in}{1.312166in}}%
\pgfpathlineto{\pgfqpoint{1.069797in}{1.313035in}}%
\pgfpathlineto{\pgfqpoint{1.071165in}{1.314122in}}%
\pgfpathlineto{\pgfqpoint{1.072197in}{1.314804in}}%
\pgfpathlineto{\pgfqpoint{1.073823in}{1.315891in}}%
\pgfpathlineto{\pgfqpoint{1.074910in}{1.316760in}}%
\pgfpathlineto{\pgfqpoint{1.076872in}{1.317816in}}%
\pgfpathlineto{\pgfqpoint{1.077982in}{1.318871in}}%
\pgfpathlineto{\pgfqpoint{1.079585in}{1.319957in}}%
\pgfpathlineto{\pgfqpoint{1.080625in}{1.320671in}}%
\pgfpathlineto{\pgfqpoint{1.080679in}{1.320671in}}%
\pgfpathlineto{\pgfqpoint{1.082133in}{1.321758in}}%
\pgfpathlineto{\pgfqpoint{1.083204in}{1.322627in}}%
\pgfpathlineto{\pgfqpoint{1.084823in}{1.323714in}}%
\pgfpathlineto{\pgfqpoint{1.085909in}{1.324490in}}%
\pgfpathlineto{\pgfqpoint{1.088427in}{1.325576in}}%
\pgfpathlineto{\pgfqpoint{1.089380in}{1.326197in}}%
\pgfpathlineto{\pgfqpoint{1.091468in}{1.327283in}}%
\pgfpathlineto{\pgfqpoint{1.092570in}{1.328277in}}%
\pgfpathlineto{\pgfqpoint{1.094454in}{1.329363in}}%
\pgfpathlineto{\pgfqpoint{1.095564in}{1.330015in}}%
\pgfpathlineto{\pgfqpoint{1.097135in}{1.331102in}}%
\pgfpathlineto{\pgfqpoint{1.098222in}{1.332033in}}%
\pgfpathlineto{\pgfqpoint{1.099762in}{1.333119in}}%
\pgfpathlineto{\pgfqpoint{1.100810in}{1.333926in}}%
\pgfpathlineto{\pgfqpoint{1.102663in}{1.334982in}}%
\pgfpathlineto{\pgfqpoint{1.103757in}{1.335851in}}%
\pgfpathlineto{\pgfqpoint{1.105352in}{1.336938in}}%
\pgfpathlineto{\pgfqpoint{1.106462in}{1.337496in}}%
\pgfpathlineto{\pgfqpoint{1.107642in}{1.338583in}}%
\pgfpathlineto{\pgfqpoint{1.108667in}{1.339576in}}%
\pgfpathlineto{\pgfqpoint{1.108690in}{1.339576in}}%
\pgfpathlineto{\pgfqpoint{1.110644in}{1.340663in}}%
\pgfpathlineto{\pgfqpoint{1.111747in}{1.341221in}}%
\pgfpathlineto{\pgfqpoint{1.113138in}{1.342308in}}%
\pgfpathlineto{\pgfqpoint{1.114233in}{1.342867in}}%
\pgfpathlineto{\pgfqpoint{1.115765in}{1.343953in}}%
\pgfpathlineto{\pgfqpoint{1.116805in}{1.344698in}}%
\pgfpathlineto{\pgfqpoint{1.118321in}{1.345753in}}%
\pgfpathlineto{\pgfqpoint{1.119338in}{1.346685in}}%
\pgfpathlineto{\pgfqpoint{1.122535in}{1.347771in}}%
\pgfpathlineto{\pgfqpoint{1.123622in}{1.348765in}}%
\pgfpathlineto{\pgfqpoint{1.125529in}{1.349820in}}%
\pgfpathlineto{\pgfqpoint{1.126553in}{1.350503in}}%
\pgfpathlineto{\pgfqpoint{1.128484in}{1.351589in}}%
\pgfpathlineto{\pgfqpoint{1.129469in}{1.352148in}}%
\pgfpathlineto{\pgfqpoint{1.132377in}{1.353235in}}%
\pgfpathlineto{\pgfqpoint{1.133441in}{1.354073in}}%
\pgfpathlineto{\pgfqpoint{1.135418in}{1.355159in}}%
\pgfpathlineto{\pgfqpoint{1.136466in}{1.355594in}}%
\pgfpathlineto{\pgfqpoint{1.138413in}{1.356680in}}%
\pgfpathlineto{\pgfqpoint{1.139515in}{1.357456in}}%
\pgfpathlineto{\pgfqpoint{1.141790in}{1.358543in}}%
\pgfpathlineto{\pgfqpoint{1.142869in}{1.359288in}}%
\pgfpathlineto{\pgfqpoint{1.144581in}{1.360374in}}%
\pgfpathlineto{\pgfqpoint{1.145660in}{1.361088in}}%
\pgfpathlineto{\pgfqpoint{1.147403in}{1.362144in}}%
\pgfpathlineto{\pgfqpoint{1.148458in}{1.363075in}}%
\pgfpathlineto{\pgfqpoint{1.148474in}{1.363075in}}%
\pgfpathlineto{\pgfqpoint{1.149647in}{1.364161in}}%
\pgfpathlineto{\pgfqpoint{1.150655in}{1.364813in}}%
\pgfpathlineto{\pgfqpoint{1.152508in}{1.365900in}}%
\pgfpathlineto{\pgfqpoint{1.153579in}{1.366614in}}%
\pgfpathlineto{\pgfqpoint{1.155103in}{1.367669in}}%
\pgfpathlineto{\pgfqpoint{1.156057in}{1.368259in}}%
\pgfpathlineto{\pgfqpoint{1.156143in}{1.368259in}}%
\pgfpathlineto{\pgfqpoint{1.157363in}{1.369345in}}%
\pgfpathlineto{\pgfqpoint{1.158426in}{1.369966in}}%
\pgfpathlineto{\pgfqpoint{1.158449in}{1.369966in}}%
\pgfpathlineto{\pgfqpoint{1.160239in}{1.371053in}}%
\pgfpathlineto{\pgfqpoint{1.161326in}{1.371798in}}%
\pgfpathlineto{\pgfqpoint{1.163023in}{1.372853in}}%
\pgfpathlineto{\pgfqpoint{1.164047in}{1.373722in}}%
\pgfpathlineto{\pgfqpoint{1.166048in}{1.374809in}}%
\pgfpathlineto{\pgfqpoint{1.167135in}{1.375461in}}%
\pgfpathlineto{\pgfqpoint{1.169480in}{1.376547in}}%
\pgfpathlineto{\pgfqpoint{1.170528in}{1.377230in}}%
\pgfpathlineto{\pgfqpoint{1.173272in}{1.378317in}}%
\pgfpathlineto{\pgfqpoint{1.174335in}{1.378969in}}%
\pgfpathlineto{\pgfqpoint{1.176375in}{1.380055in}}%
\pgfpathlineto{\pgfqpoint{1.177462in}{1.380893in}}%
\pgfpathlineto{\pgfqpoint{1.180221in}{1.381980in}}%
\pgfpathlineto{\pgfqpoint{1.181316in}{1.382476in}}%
\pgfpathlineto{\pgfqpoint{1.183497in}{1.383563in}}%
\pgfpathlineto{\pgfqpoint{1.184591in}{1.384091in}}%
\pgfpathlineto{\pgfqpoint{1.184599in}{1.384091in}}%
\pgfpathlineto{\pgfqpoint{1.186929in}{1.385177in}}%
\pgfpathlineto{\pgfqpoint{1.188039in}{1.385984in}}%
\pgfpathlineto{\pgfqpoint{1.190408in}{1.387071in}}%
\pgfpathlineto{\pgfqpoint{1.191494in}{1.387722in}}%
\pgfpathlineto{\pgfqpoint{1.194106in}{1.388809in}}%
\pgfpathlineto{\pgfqpoint{1.195177in}{1.389523in}}%
\pgfpathlineto{\pgfqpoint{1.197420in}{1.390609in}}%
\pgfpathlineto{\pgfqpoint{1.198436in}{1.391416in}}%
\pgfpathlineto{\pgfqpoint{1.200485in}{1.392472in}}%
\pgfpathlineto{\pgfqpoint{1.201587in}{1.393341in}}%
\pgfpathlineto{\pgfqpoint{1.203752in}{1.394428in}}%
\pgfpathlineto{\pgfqpoint{1.204777in}{1.394893in}}%
\pgfpathlineto{\pgfqpoint{1.206942in}{1.395980in}}%
\pgfpathlineto{\pgfqpoint{1.208044in}{1.396818in}}%
\pgfpathlineto{\pgfqpoint{1.210405in}{1.397904in}}%
\pgfpathlineto{\pgfqpoint{1.211422in}{1.398339in}}%
\pgfpathlineto{\pgfqpoint{1.213869in}{1.399394in}}%
\pgfpathlineto{\pgfqpoint{1.214971in}{1.400077in}}%
\pgfpathlineto{\pgfqpoint{1.217183in}{1.401164in}}%
\pgfpathlineto{\pgfqpoint{1.218285in}{1.401722in}}%
\pgfpathlineto{\pgfqpoint{1.220482in}{1.402809in}}%
\pgfpathlineto{\pgfqpoint{1.221444in}{1.403244in}}%
\pgfpathlineto{\pgfqpoint{1.223703in}{1.404330in}}%
\pgfpathlineto{\pgfqpoint{1.224329in}{1.404796in}}%
\pgfpathlineto{\pgfqpoint{1.224563in}{1.404796in}}%
\pgfpathlineto{\pgfqpoint{1.227800in}{1.405882in}}%
\pgfpathlineto{\pgfqpoint{1.228886in}{1.406534in}}%
\pgfpathlineto{\pgfqpoint{1.231357in}{1.407620in}}%
\pgfpathlineto{\pgfqpoint{1.232459in}{1.408241in}}%
\pgfpathlineto{\pgfqpoint{1.235117in}{1.409328in}}%
\pgfpathlineto{\pgfqpoint{1.236102in}{1.410011in}}%
\pgfpathlineto{\pgfqpoint{1.238385in}{1.411097in}}%
\pgfpathlineto{\pgfqpoint{1.239487in}{1.411532in}}%
\pgfpathlineto{\pgfqpoint{1.242028in}{1.412618in}}%
\pgfpathlineto{\pgfqpoint{1.243091in}{1.413301in}}%
\pgfpathlineto{\pgfqpoint{1.244826in}{1.414388in}}%
\pgfpathlineto{\pgfqpoint{1.245929in}{1.415102in}}%
\pgfpathlineto{\pgfqpoint{1.248032in}{1.416188in}}%
\pgfpathlineto{\pgfqpoint{1.249142in}{1.416871in}}%
\pgfpathlineto{\pgfqpoint{1.251088in}{1.417957in}}%
\pgfpathlineto{\pgfqpoint{1.252183in}{1.418454in}}%
\pgfpathlineto{\pgfqpoint{1.254810in}{1.419541in}}%
\pgfpathlineto{\pgfqpoint{1.255912in}{1.420068in}}%
\pgfpathlineto{\pgfqpoint{1.257593in}{1.421155in}}%
\pgfpathlineto{\pgfqpoint{1.258648in}{1.421714in}}%
\pgfpathlineto{\pgfqpoint{1.261064in}{1.422800in}}%
\pgfpathlineto{\pgfqpoint{1.262119in}{1.423204in}}%
\pgfpathlineto{\pgfqpoint{1.264988in}{1.424290in}}%
\pgfpathlineto{\pgfqpoint{1.266059in}{1.424756in}}%
\pgfpathlineto{\pgfqpoint{1.268529in}{1.425842in}}%
\pgfpathlineto{\pgfqpoint{1.269624in}{1.426401in}}%
\pgfpathlineto{\pgfqpoint{1.271649in}{1.427487in}}%
\pgfpathlineto{\pgfqpoint{1.272509in}{1.427891in}}%
\pgfpathlineto{\pgfqpoint{1.275112in}{1.428977in}}%
\pgfpathlineto{\pgfqpoint{1.276183in}{1.429722in}}%
\pgfpathlineto{\pgfqpoint{1.278497in}{1.430809in}}%
\pgfpathlineto{\pgfqpoint{1.279584in}{1.431368in}}%
\pgfpathlineto{\pgfqpoint{1.282070in}{1.432454in}}%
\pgfpathlineto{\pgfqpoint{1.283141in}{1.432951in}}%
\pgfpathlineto{\pgfqpoint{1.284603in}{1.434037in}}%
\pgfpathlineto{\pgfqpoint{1.285681in}{1.434813in}}%
\pgfpathlineto{\pgfqpoint{1.288550in}{1.435900in}}%
\pgfpathlineto{\pgfqpoint{1.289410in}{1.436272in}}%
\pgfpathlineto{\pgfqpoint{1.291466in}{1.437359in}}%
\pgfpathlineto{\pgfqpoint{1.292514in}{1.437855in}}%
\pgfpathlineto{\pgfqpoint{1.295203in}{1.438942in}}%
\pgfpathlineto{\pgfqpoint{1.296251in}{1.439314in}}%
\pgfpathlineto{\pgfqpoint{1.298987in}{1.440401in}}%
\pgfpathlineto{\pgfqpoint{1.299988in}{1.440960in}}%
\pgfpathlineto{\pgfqpoint{1.302216in}{1.442046in}}%
\pgfpathlineto{\pgfqpoint{1.303326in}{1.442543in}}%
\pgfpathlineto{\pgfqpoint{1.306164in}{1.443629in}}%
\pgfpathlineto{\pgfqpoint{1.307227in}{1.444064in}}%
\pgfpathlineto{\pgfqpoint{1.310878in}{1.445150in}}%
\pgfpathlineto{\pgfqpoint{1.311784in}{1.445492in}}%
\pgfpathlineto{\pgfqpoint{1.315084in}{1.446578in}}%
\pgfpathlineto{\pgfqpoint{1.316061in}{1.447137in}}%
\pgfpathlineto{\pgfqpoint{1.317820in}{1.448224in}}%
\pgfpathlineto{\pgfqpoint{1.318883in}{1.448938in}}%
\pgfpathlineto{\pgfqpoint{1.320751in}{1.450024in}}%
\pgfpathlineto{\pgfqpoint{1.321854in}{1.450334in}}%
\pgfpathlineto{\pgfqpoint{1.324332in}{1.451421in}}%
\pgfpathlineto{\pgfqpoint{1.325411in}{1.452011in}}%
\pgfpathlineto{\pgfqpoint{1.327850in}{1.453097in}}%
\pgfpathlineto{\pgfqpoint{1.328788in}{1.453563in}}%
\pgfpathlineto{\pgfqpoint{1.331813in}{1.454649in}}%
\pgfpathlineto{\pgfqpoint{1.332837in}{1.455053in}}%
\pgfpathlineto{\pgfqpoint{1.336043in}{1.456139in}}%
\pgfpathlineto{\pgfqpoint{1.337106in}{1.457040in}}%
\pgfpathlineto{\pgfqpoint{1.339240in}{1.458126in}}%
\pgfpathlineto{\pgfqpoint{1.340295in}{1.458716in}}%
\pgfpathlineto{\pgfqpoint{1.340311in}{1.458716in}}%
\pgfpathlineto{\pgfqpoint{1.342086in}{1.459802in}}%
\pgfpathlineto{\pgfqpoint{1.343188in}{1.460299in}}%
\pgfpathlineto{\pgfqpoint{1.345776in}{1.461385in}}%
\pgfpathlineto{\pgfqpoint{1.346800in}{1.461944in}}%
\pgfpathlineto{\pgfqpoint{1.350513in}{1.463031in}}%
\pgfpathlineto{\pgfqpoint{1.351576in}{1.463496in}}%
\pgfpathlineto{\pgfqpoint{1.354054in}{1.464552in}}%
\pgfpathlineto{\pgfqpoint{1.355094in}{1.465017in}}%
\pgfpathlineto{\pgfqpoint{1.358401in}{1.466104in}}%
\pgfpathlineto{\pgfqpoint{1.359308in}{1.466663in}}%
\pgfpathlineto{\pgfqpoint{1.359355in}{1.466663in}}%
\pgfpathlineto{\pgfqpoint{1.363068in}{1.467749in}}%
\pgfpathlineto{\pgfqpoint{1.363975in}{1.468153in}}%
\pgfpathlineto{\pgfqpoint{1.364085in}{1.468153in}}%
\pgfpathlineto{\pgfqpoint{1.366453in}{1.469239in}}%
\pgfpathlineto{\pgfqpoint{1.367438in}{1.469829in}}%
\pgfpathlineto{\pgfqpoint{1.370518in}{1.470915in}}%
\pgfpathlineto{\pgfqpoint{1.371402in}{1.471505in}}%
\pgfpathlineto{\pgfqpoint{1.371621in}{1.471505in}}%
\pgfpathlineto{\pgfqpoint{1.374107in}{1.472592in}}%
\pgfpathlineto{\pgfqpoint{1.375092in}{1.473026in}}%
\pgfpathlineto{\pgfqpoint{1.375107in}{1.473026in}}%
\pgfpathlineto{\pgfqpoint{1.377148in}{1.474113in}}%
\pgfpathlineto{\pgfqpoint{1.378172in}{1.474702in}}%
\pgfpathlineto{\pgfqpoint{1.380822in}{1.475789in}}%
\pgfpathlineto{\pgfqpoint{1.381877in}{1.476255in}}%
\pgfpathlineto{\pgfqpoint{1.384754in}{1.477341in}}%
\pgfpathlineto{\pgfqpoint{1.385841in}{1.477776in}}%
\pgfpathlineto{\pgfqpoint{1.385857in}{1.477776in}}%
\pgfpathlineto{\pgfqpoint{1.390047in}{1.478862in}}%
\pgfpathlineto{\pgfqpoint{1.391016in}{1.479142in}}%
\pgfpathlineto{\pgfqpoint{1.391141in}{1.479142in}}%
\pgfpathlineto{\pgfqpoint{1.394902in}{1.480228in}}%
\pgfpathlineto{\pgfqpoint{1.395965in}{1.480632in}}%
\pgfpathlineto{\pgfqpoint{1.399092in}{1.481718in}}%
\pgfpathlineto{\pgfqpoint{1.400179in}{1.482122in}}%
\pgfpathlineto{\pgfqpoint{1.400194in}{1.482122in}}%
\pgfpathlineto{\pgfqpoint{1.403251in}{1.483208in}}%
\pgfpathlineto{\pgfqpoint{1.404361in}{1.483767in}}%
\pgfpathlineto{\pgfqpoint{1.406988in}{1.484853in}}%
\pgfpathlineto{\pgfqpoint{1.408012in}{1.485474in}}%
\pgfpathlineto{\pgfqpoint{1.411741in}{1.486561in}}%
\pgfpathlineto{\pgfqpoint{1.412804in}{1.486933in}}%
\pgfpathlineto{\pgfqpoint{1.415579in}{1.488020in}}%
\pgfpathlineto{\pgfqpoint{1.416682in}{1.488578in}}%
\pgfpathlineto{\pgfqpoint{1.420544in}{1.489665in}}%
\pgfpathlineto{\pgfqpoint{1.421591in}{1.490130in}}%
\pgfpathlineto{\pgfqpoint{1.424468in}{1.491217in}}%
\pgfpathlineto{\pgfqpoint{1.425461in}{1.491714in}}%
\pgfpathlineto{\pgfqpoint{1.425531in}{1.491714in}}%
\pgfpathlineto{\pgfqpoint{1.428353in}{1.492800in}}%
\pgfpathlineto{\pgfqpoint{1.429370in}{1.493173in}}%
\pgfpathlineto{\pgfqpoint{1.429463in}{1.493173in}}%
\pgfpathlineto{\pgfqpoint{1.432442in}{1.494259in}}%
\pgfpathlineto{\pgfqpoint{1.433380in}{1.494600in}}%
\pgfpathlineto{\pgfqpoint{1.433466in}{1.494600in}}%
\pgfpathlineto{\pgfqpoint{1.437179in}{1.495687in}}%
\pgfpathlineto{\pgfqpoint{1.438282in}{1.496059in}}%
\pgfpathlineto{\pgfqpoint{1.441213in}{1.497146in}}%
\pgfpathlineto{\pgfqpoint{1.442300in}{1.497581in}}%
\pgfpathlineto{\pgfqpoint{1.445286in}{1.498667in}}%
\pgfpathlineto{\pgfqpoint{1.446310in}{1.499226in}}%
\pgfpathlineto{\pgfqpoint{1.449281in}{1.500312in}}%
\pgfpathlineto{\pgfqpoint{1.450368in}{1.500902in}}%
\pgfpathlineto{\pgfqpoint{1.453354in}{1.501989in}}%
\pgfpathlineto{\pgfqpoint{1.454433in}{1.502330in}}%
\pgfpathlineto{\pgfqpoint{1.458561in}{1.503385in}}%
\pgfpathlineto{\pgfqpoint{1.459632in}{1.503789in}}%
\pgfpathlineto{\pgfqpoint{1.462923in}{1.504875in}}%
\pgfpathlineto{\pgfqpoint{1.463822in}{1.505124in}}%
\pgfpathlineto{\pgfqpoint{1.463900in}{1.505124in}}%
\pgfpathlineto{\pgfqpoint{1.467301in}{1.506210in}}%
\pgfpathlineto{\pgfqpoint{1.468395in}{1.506645in}}%
\pgfpathlineto{\pgfqpoint{1.470874in}{1.507731in}}%
\pgfpathlineto{\pgfqpoint{1.471819in}{1.508042in}}%
\pgfpathlineto{\pgfqpoint{1.471945in}{1.508042in}}%
\pgfpathlineto{\pgfqpoint{1.474915in}{1.509097in}}%
\pgfpathlineto{\pgfqpoint{1.476018in}{1.509408in}}%
\pgfpathlineto{\pgfqpoint{1.476025in}{1.509408in}}%
\pgfpathlineto{\pgfqpoint{1.478425in}{1.510494in}}%
\pgfpathlineto{\pgfqpoint{1.479629in}{1.510898in}}%
\pgfpathlineto{\pgfqpoint{1.484038in}{1.511984in}}%
\pgfpathlineto{\pgfqpoint{1.485070in}{1.512481in}}%
\pgfpathlineto{\pgfqpoint{1.488815in}{1.513567in}}%
\pgfpathlineto{\pgfqpoint{1.489823in}{1.514033in}}%
\pgfpathlineto{\pgfqpoint{1.492786in}{1.515119in}}%
\pgfpathlineto{\pgfqpoint{1.493826in}{1.515399in}}%
\pgfpathlineto{\pgfqpoint{1.497696in}{1.516485in}}%
\pgfpathlineto{\pgfqpoint{1.498790in}{1.516796in}}%
\pgfpathlineto{\pgfqpoint{1.501761in}{1.517820in}}%
\pgfpathlineto{\pgfqpoint{1.501761in}{1.517851in}}%
\pgfpathlineto{\pgfqpoint{1.502801in}{1.518286in}}%
\pgfpathlineto{\pgfqpoint{1.507366in}{1.519372in}}%
\pgfpathlineto{\pgfqpoint{1.508429in}{1.519807in}}%
\pgfpathlineto{\pgfqpoint{1.511729in}{1.520893in}}%
\pgfpathlineto{\pgfqpoint{1.512768in}{1.521328in}}%
\pgfpathlineto{\pgfqpoint{1.516411in}{1.522383in}}%
\pgfpathlineto{\pgfqpoint{1.517467in}{1.522694in}}%
\pgfpathlineto{\pgfqpoint{1.521196in}{1.523749in}}%
\pgfpathlineto{\pgfqpoint{1.522228in}{1.524122in}}%
\pgfpathlineto{\pgfqpoint{1.527309in}{1.525208in}}%
\pgfpathlineto{\pgfqpoint{1.528396in}{1.525612in}}%
\pgfpathlineto{\pgfqpoint{1.532265in}{1.526698in}}%
\pgfpathlineto{\pgfqpoint{1.533250in}{1.526977in}}%
\pgfpathlineto{\pgfqpoint{1.536847in}{1.528064in}}%
\pgfpathlineto{\pgfqpoint{1.537957in}{1.528281in}}%
\pgfpathlineto{\pgfqpoint{1.541350in}{1.529337in}}%
\pgfpathlineto{\pgfqpoint{1.542374in}{1.529678in}}%
\pgfpathlineto{\pgfqpoint{1.546095in}{1.530765in}}%
\pgfpathlineto{\pgfqpoint{1.546994in}{1.531137in}}%
\pgfpathlineto{\pgfqpoint{1.551747in}{1.532224in}}%
\pgfpathlineto{\pgfqpoint{1.552787in}{1.532410in}}%
\pgfpathlineto{\pgfqpoint{1.556883in}{1.533496in}}%
\pgfpathlineto{\pgfqpoint{1.557962in}{1.533931in}}%
\pgfpathlineto{\pgfqpoint{1.557978in}{1.533931in}}%
\pgfpathlineto{\pgfqpoint{1.561902in}{1.535017in}}%
\pgfpathlineto{\pgfqpoint{1.562911in}{1.535452in}}%
\pgfpathlineto{\pgfqpoint{1.567234in}{1.536538in}}%
\pgfpathlineto{\pgfqpoint{1.568297in}{1.536911in}}%
\pgfpathlineto{\pgfqpoint{1.573019in}{1.537966in}}%
\pgfpathlineto{\pgfqpoint{1.574090in}{1.538308in}}%
\pgfpathlineto{\pgfqpoint{1.578655in}{1.539394in}}%
\pgfpathlineto{\pgfqpoint{1.579601in}{1.539643in}}%
\pgfpathlineto{\pgfqpoint{1.583651in}{1.540729in}}%
\pgfpathlineto{\pgfqpoint{1.584737in}{1.541008in}}%
\pgfpathlineto{\pgfqpoint{1.588779in}{1.542095in}}%
\pgfpathlineto{\pgfqpoint{1.589874in}{1.542312in}}%
\pgfpathlineto{\pgfqpoint{1.593048in}{1.543399in}}%
\pgfpathlineto{\pgfqpoint{1.594119in}{1.543492in}}%
\pgfpathlineto{\pgfqpoint{1.594134in}{1.543492in}}%
\pgfpathlineto{\pgfqpoint{1.597996in}{1.544547in}}%
\pgfpathlineto{\pgfqpoint{1.599012in}{1.544951in}}%
\pgfpathlineto{\pgfqpoint{1.603789in}{1.546037in}}%
\pgfpathlineto{\pgfqpoint{1.604852in}{1.546317in}}%
\pgfpathlineto{\pgfqpoint{1.610841in}{1.547403in}}%
\pgfpathlineto{\pgfqpoint{1.611388in}{1.547651in}}%
\pgfpathlineto{\pgfqpoint{1.611708in}{1.547651in}}%
\pgfpathlineto{\pgfqpoint{1.617306in}{1.548738in}}%
\pgfpathlineto{\pgfqpoint{1.618361in}{1.549110in}}%
\pgfpathlineto{\pgfqpoint{1.623091in}{1.550197in}}%
\pgfpathlineto{\pgfqpoint{1.623966in}{1.550569in}}%
\pgfpathlineto{\pgfqpoint{1.624193in}{1.550569in}}%
\pgfpathlineto{\pgfqpoint{1.630181in}{1.551656in}}%
\pgfpathlineto{\pgfqpoint{1.631135in}{1.551780in}}%
\pgfpathlineto{\pgfqpoint{1.631206in}{1.551780in}}%
\pgfpathlineto{\pgfqpoint{1.636084in}{1.552836in}}%
\pgfpathlineto{\pgfqpoint{1.637053in}{1.553208in}}%
\pgfpathlineto{\pgfqpoint{1.637170in}{1.553208in}}%
\pgfpathlineto{\pgfqpoint{1.641157in}{1.554294in}}%
\pgfpathlineto{\pgfqpoint{1.642252in}{1.554481in}}%
\pgfpathlineto{\pgfqpoint{1.646919in}{1.555567in}}%
\pgfpathlineto{\pgfqpoint{1.648013in}{1.555878in}}%
\pgfpathlineto{\pgfqpoint{1.652524in}{1.556964in}}%
\pgfpathlineto{\pgfqpoint{1.653603in}{1.557150in}}%
\pgfpathlineto{\pgfqpoint{1.660655in}{1.558237in}}%
\pgfpathlineto{\pgfqpoint{1.661710in}{1.558454in}}%
\pgfpathlineto{\pgfqpoint{1.661726in}{1.558454in}}%
\pgfpathlineto{\pgfqpoint{1.667956in}{1.559541in}}%
\pgfpathlineto{\pgfqpoint{1.668980in}{1.559727in}}%
\pgfpathlineto{\pgfqpoint{1.674851in}{1.560813in}}%
\pgfpathlineto{\pgfqpoint{1.675883in}{1.560938in}}%
\pgfpathlineto{\pgfqpoint{1.675946in}{1.560938in}}%
\pgfpathlineto{\pgfqpoint{1.681020in}{1.562024in}}%
\pgfpathlineto{\pgfqpoint{1.682020in}{1.562334in}}%
\pgfpathlineto{\pgfqpoint{1.686554in}{1.563421in}}%
\pgfpathlineto{\pgfqpoint{1.687625in}{1.563669in}}%
\pgfpathlineto{\pgfqpoint{1.694622in}{1.564756in}}%
\pgfpathlineto{\pgfqpoint{1.695513in}{1.565035in}}%
\pgfpathlineto{\pgfqpoint{1.700657in}{1.566122in}}%
\pgfpathlineto{\pgfqpoint{1.701760in}{1.566277in}}%
\pgfpathlineto{\pgfqpoint{1.706583in}{1.567363in}}%
\pgfpathlineto{\pgfqpoint{1.707428in}{1.567674in}}%
\pgfpathlineto{\pgfqpoint{1.713815in}{1.568760in}}%
\pgfpathlineto{\pgfqpoint{1.714917in}{1.569040in}}%
\pgfpathlineto{\pgfqpoint{1.720194in}{1.570126in}}%
\pgfpathlineto{\pgfqpoint{1.721241in}{1.570374in}}%
\pgfpathlineto{\pgfqpoint{1.728645in}{1.571461in}}%
\pgfpathlineto{\pgfqpoint{1.729614in}{1.571616in}}%
\pgfpathlineto{\pgfqpoint{1.735477in}{1.572702in}}%
\pgfpathlineto{\pgfqpoint{1.736501in}{1.573199in}}%
\pgfpathlineto{\pgfqpoint{1.742685in}{1.574286in}}%
\pgfpathlineto{\pgfqpoint{1.743709in}{1.574565in}}%
\pgfpathlineto{\pgfqpoint{1.750706in}{1.575651in}}%
\pgfpathlineto{\pgfqpoint{1.751761in}{1.575838in}}%
\pgfpathlineto{\pgfqpoint{1.756249in}{1.576924in}}%
\pgfpathlineto{\pgfqpoint{1.756819in}{1.577079in}}%
\pgfpathlineto{\pgfqpoint{1.763832in}{1.578166in}}%
\pgfpathlineto{\pgfqpoint{1.764848in}{1.578445in}}%
\pgfpathlineto{\pgfqpoint{1.764934in}{1.578445in}}%
\pgfpathlineto{\pgfqpoint{1.772150in}{1.579532in}}%
\pgfpathlineto{\pgfqpoint{1.773197in}{1.579780in}}%
\pgfpathlineto{\pgfqpoint{1.773260in}{1.579780in}}%
\pgfpathlineto{\pgfqpoint{1.780155in}{1.580867in}}%
\pgfpathlineto{\pgfqpoint{1.781031in}{1.581146in}}%
\pgfpathlineto{\pgfqpoint{1.790834in}{1.582232in}}%
\pgfpathlineto{\pgfqpoint{1.791897in}{1.582388in}}%
\pgfpathlineto{\pgfqpoint{1.800966in}{1.583474in}}%
\pgfpathlineto{\pgfqpoint{1.802013in}{1.583660in}}%
\pgfpathlineto{\pgfqpoint{1.812215in}{1.584747in}}%
\pgfpathlineto{\pgfqpoint{1.813192in}{1.585057in}}%
\pgfpathlineto{\pgfqpoint{1.822409in}{1.586144in}}%
\pgfpathlineto{\pgfqpoint{1.822409in}{1.586175in}}%
\pgfpathlineto{\pgfqpoint{1.823113in}{1.586175in}}%
\pgfpathlineto{\pgfqpoint{1.834066in}{1.587261in}}%
\pgfpathlineto{\pgfqpoint{1.834511in}{1.587447in}}%
\pgfpathlineto{\pgfqpoint{1.834965in}{1.587447in}}%
\pgfpathlineto{\pgfqpoint{1.842524in}{1.588534in}}%
\pgfpathlineto{\pgfqpoint{1.843580in}{1.588751in}}%
\pgfpathlineto{\pgfqpoint{1.851952in}{1.589838in}}%
\pgfpathlineto{\pgfqpoint{1.852937in}{1.589993in}}%
\pgfpathlineto{\pgfqpoint{1.853055in}{1.589993in}}%
\pgfpathlineto{\pgfqpoint{1.865453in}{1.591079in}}%
\pgfpathlineto{\pgfqpoint{1.866462in}{1.591266in}}%
\pgfpathlineto{\pgfqpoint{1.876656in}{1.592352in}}%
\pgfpathlineto{\pgfqpoint{1.877680in}{1.592507in}}%
\pgfpathlineto{\pgfqpoint{1.892565in}{1.593594in}}%
\pgfpathlineto{\pgfqpoint{1.893284in}{1.593687in}}%
\pgfpathlineto{\pgfqpoint{1.893628in}{1.593687in}}%
\pgfpathlineto{\pgfqpoint{1.909091in}{1.594773in}}%
\pgfpathlineto{\pgfqpoint{1.909865in}{1.594898in}}%
\pgfpathlineto{\pgfqpoint{1.910108in}{1.594898in}}%
\pgfpathlineto{\pgfqpoint{1.920310in}{1.595984in}}%
\pgfpathlineto{\pgfqpoint{1.920427in}{1.596046in}}%
\pgfpathlineto{\pgfqpoint{1.920497in}{1.596046in}}%
\pgfpathlineto{\pgfqpoint{1.935578in}{1.597133in}}%
\pgfpathlineto{\pgfqpoint{1.936484in}{1.597350in}}%
\pgfpathlineto{\pgfqpoint{1.951127in}{1.598436in}}%
\pgfpathlineto{\pgfqpoint{1.951987in}{1.598498in}}%
\pgfpathlineto{\pgfqpoint{1.968388in}{1.599585in}}%
\pgfpathlineto{\pgfqpoint{1.969451in}{1.599709in}}%
\pgfpathlineto{\pgfqpoint{1.991599in}{1.600796in}}%
\pgfpathlineto{\pgfqpoint{1.992334in}{1.600889in}}%
\pgfpathlineto{\pgfqpoint{1.992654in}{1.600889in}}%
\pgfpathlineto{\pgfqpoint{2.033126in}{1.601944in}}%
\pgfpathlineto{\pgfqpoint{2.033126in}{1.601944in}}%
\pgfusepath{stroke}%
\end{pgfscope}%
\begin{pgfscope}%
\pgfsetrectcap%
\pgfsetmiterjoin%
\pgfsetlinewidth{0.803000pt}%
\definecolor{currentstroke}{rgb}{0.000000,0.000000,0.000000}%
\pgfsetstrokecolor{currentstroke}%
\pgfsetdash{}{0pt}%
\pgfpathmoveto{\pgfqpoint{0.553581in}{0.499444in}}%
\pgfpathlineto{\pgfqpoint{0.553581in}{1.654444in}}%
\pgfusepath{stroke}%
\end{pgfscope}%
\begin{pgfscope}%
\pgfsetrectcap%
\pgfsetmiterjoin%
\pgfsetlinewidth{0.803000pt}%
\definecolor{currentstroke}{rgb}{0.000000,0.000000,0.000000}%
\pgfsetstrokecolor{currentstroke}%
\pgfsetdash{}{0pt}%
\pgfpathmoveto{\pgfqpoint{2.103581in}{0.499444in}}%
\pgfpathlineto{\pgfqpoint{2.103581in}{1.654444in}}%
\pgfusepath{stroke}%
\end{pgfscope}%
\begin{pgfscope}%
\pgfsetrectcap%
\pgfsetmiterjoin%
\pgfsetlinewidth{0.803000pt}%
\definecolor{currentstroke}{rgb}{0.000000,0.000000,0.000000}%
\pgfsetstrokecolor{currentstroke}%
\pgfsetdash{}{0pt}%
\pgfpathmoveto{\pgfqpoint{0.553581in}{0.499444in}}%
\pgfpathlineto{\pgfqpoint{2.103581in}{0.499444in}}%
\pgfusepath{stroke}%
\end{pgfscope}%
\begin{pgfscope}%
\pgfsetrectcap%
\pgfsetmiterjoin%
\pgfsetlinewidth{0.803000pt}%
\definecolor{currentstroke}{rgb}{0.000000,0.000000,0.000000}%
\pgfsetstrokecolor{currentstroke}%
\pgfsetdash{}{0pt}%
\pgfpathmoveto{\pgfqpoint{0.553581in}{1.654444in}}%
\pgfpathlineto{\pgfqpoint{2.103581in}{1.654444in}}%
\pgfusepath{stroke}%
\end{pgfscope}%
\begin{pgfscope}%
\pgfsetbuttcap%
\pgfsetmiterjoin%
\definecolor{currentfill}{rgb}{1.000000,1.000000,1.000000}%
\pgfsetfillcolor{currentfill}%
\pgfsetfillopacity{0.800000}%
\pgfsetlinewidth{1.003750pt}%
\definecolor{currentstroke}{rgb}{0.800000,0.800000,0.800000}%
\pgfsetstrokecolor{currentstroke}%
\pgfsetstrokeopacity{0.800000}%
\pgfsetdash{}{0pt}%
\pgfpathmoveto{\pgfqpoint{0.832747in}{0.568889in}}%
\pgfpathlineto{\pgfqpoint{2.006358in}{0.568889in}}%
\pgfpathquadraticcurveto{\pgfqpoint{2.034136in}{0.568889in}}{\pgfqpoint{2.034136in}{0.596666in}}%
\pgfpathlineto{\pgfqpoint{2.034136in}{0.776388in}}%
\pgfpathquadraticcurveto{\pgfqpoint{2.034136in}{0.804166in}}{\pgfqpoint{2.006358in}{0.804166in}}%
\pgfpathlineto{\pgfqpoint{0.832747in}{0.804166in}}%
\pgfpathquadraticcurveto{\pgfqpoint{0.804970in}{0.804166in}}{\pgfqpoint{0.804970in}{0.776388in}}%
\pgfpathlineto{\pgfqpoint{0.804970in}{0.596666in}}%
\pgfpathquadraticcurveto{\pgfqpoint{0.804970in}{0.568889in}}{\pgfqpoint{0.832747in}{0.568889in}}%
\pgfpathlineto{\pgfqpoint{0.832747in}{0.568889in}}%
\pgfpathclose%
\pgfusepath{stroke,fill}%
\end{pgfscope}%
\begin{pgfscope}%
\pgfsetrectcap%
\pgfsetroundjoin%
\pgfsetlinewidth{1.505625pt}%
\definecolor{currentstroke}{rgb}{0.000000,0.000000,0.000000}%
\pgfsetstrokecolor{currentstroke}%
\pgfsetdash{}{0pt}%
\pgfpathmoveto{\pgfqpoint{0.860525in}{0.700000in}}%
\pgfpathlineto{\pgfqpoint{0.999414in}{0.700000in}}%
\pgfpathlineto{\pgfqpoint{1.138303in}{0.700000in}}%
\pgfusepath{stroke}%
\end{pgfscope}%
\begin{pgfscope}%
\definecolor{textcolor}{rgb}{0.000000,0.000000,0.000000}%
\pgfsetstrokecolor{textcolor}%
\pgfsetfillcolor{textcolor}%
\pgftext[x=1.249414in,y=0.651388in,left,base]{\color{textcolor}\rmfamily\fontsize{10.000000}{12.000000}\selectfont AUC=0.778}%
\end{pgfscope}%
\end{pgfpicture}%
\makeatother%
\endgroup%

	
\end{tabular}

\verb|KBFC_Hard_Tomek_0_alpha_balanced_gamma_0_0_v1_Test|

\noindent\begin{tabular}{@{\hspace{-6pt}}p{4.3in} @{\hspace{-6pt}}p{2.0in}}
	\vskip 0pt
	\hfil Raw Model Output
	
	%% Creator: Matplotlib, PGF backend
%%
%% To include the figure in your LaTeX document, write
%%   \input{<filename>.pgf}
%%
%% Make sure the required packages are loaded in your preamble
%%   \usepackage{pgf}
%%
%% Also ensure that all the required font packages are loaded; for instance,
%% the lmodern package is sometimes necessary when using math font.
%%   \usepackage{lmodern}
%%
%% Figures using additional raster images can only be included by \input if
%% they are in the same directory as the main LaTeX file. For loading figures
%% from other directories you can use the `import` package
%%   \usepackage{import}
%%
%% and then include the figures with
%%   \import{<path to file>}{<filename>.pgf}
%%
%% Matplotlib used the following preamble
%%   
%%   \usepackage{fontspec}
%%   \makeatletter\@ifpackageloaded{underscore}{}{\usepackage[strings]{underscore}}\makeatother
%%
\begingroup%
\makeatletter%
\begin{pgfpicture}%
\pgfpathrectangle{\pgfpointorigin}{\pgfqpoint{4.509306in}{1.754444in}}%
\pgfusepath{use as bounding box, clip}%
\begin{pgfscope}%
\pgfsetbuttcap%
\pgfsetmiterjoin%
\definecolor{currentfill}{rgb}{1.000000,1.000000,1.000000}%
\pgfsetfillcolor{currentfill}%
\pgfsetlinewidth{0.000000pt}%
\definecolor{currentstroke}{rgb}{1.000000,1.000000,1.000000}%
\pgfsetstrokecolor{currentstroke}%
\pgfsetdash{}{0pt}%
\pgfpathmoveto{\pgfqpoint{0.000000in}{0.000000in}}%
\pgfpathlineto{\pgfqpoint{4.509306in}{0.000000in}}%
\pgfpathlineto{\pgfqpoint{4.509306in}{1.754444in}}%
\pgfpathlineto{\pgfqpoint{0.000000in}{1.754444in}}%
\pgfpathlineto{\pgfqpoint{0.000000in}{0.000000in}}%
\pgfpathclose%
\pgfusepath{fill}%
\end{pgfscope}%
\begin{pgfscope}%
\pgfsetbuttcap%
\pgfsetmiterjoin%
\definecolor{currentfill}{rgb}{1.000000,1.000000,1.000000}%
\pgfsetfillcolor{currentfill}%
\pgfsetlinewidth{0.000000pt}%
\definecolor{currentstroke}{rgb}{0.000000,0.000000,0.000000}%
\pgfsetstrokecolor{currentstroke}%
\pgfsetstrokeopacity{0.000000}%
\pgfsetdash{}{0pt}%
\pgfpathmoveto{\pgfqpoint{0.445556in}{0.499444in}}%
\pgfpathlineto{\pgfqpoint{4.320556in}{0.499444in}}%
\pgfpathlineto{\pgfqpoint{4.320556in}{1.654444in}}%
\pgfpathlineto{\pgfqpoint{0.445556in}{1.654444in}}%
\pgfpathlineto{\pgfqpoint{0.445556in}{0.499444in}}%
\pgfpathclose%
\pgfusepath{fill}%
\end{pgfscope}%
\begin{pgfscope}%
\pgfpathrectangle{\pgfqpoint{0.445556in}{0.499444in}}{\pgfqpoint{3.875000in}{1.155000in}}%
\pgfusepath{clip}%
\pgfsetbuttcap%
\pgfsetmiterjoin%
\pgfsetlinewidth{1.003750pt}%
\definecolor{currentstroke}{rgb}{0.000000,0.000000,0.000000}%
\pgfsetstrokecolor{currentstroke}%
\pgfsetdash{}{0pt}%
\pgfpathmoveto{\pgfqpoint{0.435556in}{0.499444in}}%
\pgfpathlineto{\pgfqpoint{0.483922in}{0.499444in}}%
\pgfpathlineto{\pgfqpoint{0.483922in}{0.905316in}}%
\pgfpathlineto{\pgfqpoint{0.435556in}{0.905316in}}%
\pgfusepath{stroke}%
\end{pgfscope}%
\begin{pgfscope}%
\pgfpathrectangle{\pgfqpoint{0.445556in}{0.499444in}}{\pgfqpoint{3.875000in}{1.155000in}}%
\pgfusepath{clip}%
\pgfsetbuttcap%
\pgfsetmiterjoin%
\pgfsetlinewidth{1.003750pt}%
\definecolor{currentstroke}{rgb}{0.000000,0.000000,0.000000}%
\pgfsetstrokecolor{currentstroke}%
\pgfsetdash{}{0pt}%
\pgfpathmoveto{\pgfqpoint{0.576001in}{0.499444in}}%
\pgfpathlineto{\pgfqpoint{0.637387in}{0.499444in}}%
\pgfpathlineto{\pgfqpoint{0.637387in}{1.095204in}}%
\pgfpathlineto{\pgfqpoint{0.576001in}{1.095204in}}%
\pgfpathlineto{\pgfqpoint{0.576001in}{0.499444in}}%
\pgfpathclose%
\pgfusepath{stroke}%
\end{pgfscope}%
\begin{pgfscope}%
\pgfpathrectangle{\pgfqpoint{0.445556in}{0.499444in}}{\pgfqpoint{3.875000in}{1.155000in}}%
\pgfusepath{clip}%
\pgfsetbuttcap%
\pgfsetmiterjoin%
\pgfsetlinewidth{1.003750pt}%
\definecolor{currentstroke}{rgb}{0.000000,0.000000,0.000000}%
\pgfsetstrokecolor{currentstroke}%
\pgfsetdash{}{0pt}%
\pgfpathmoveto{\pgfqpoint{0.729467in}{0.499444in}}%
\pgfpathlineto{\pgfqpoint{0.790853in}{0.499444in}}%
\pgfpathlineto{\pgfqpoint{0.790853in}{1.209750in}}%
\pgfpathlineto{\pgfqpoint{0.729467in}{1.209750in}}%
\pgfpathlineto{\pgfqpoint{0.729467in}{0.499444in}}%
\pgfpathclose%
\pgfusepath{stroke}%
\end{pgfscope}%
\begin{pgfscope}%
\pgfpathrectangle{\pgfqpoint{0.445556in}{0.499444in}}{\pgfqpoint{3.875000in}{1.155000in}}%
\pgfusepath{clip}%
\pgfsetbuttcap%
\pgfsetmiterjoin%
\pgfsetlinewidth{1.003750pt}%
\definecolor{currentstroke}{rgb}{0.000000,0.000000,0.000000}%
\pgfsetstrokecolor{currentstroke}%
\pgfsetdash{}{0pt}%
\pgfpathmoveto{\pgfqpoint{0.882932in}{0.499444in}}%
\pgfpathlineto{\pgfqpoint{0.944318in}{0.499444in}}%
\pgfpathlineto{\pgfqpoint{0.944318in}{1.285327in}}%
\pgfpathlineto{\pgfqpoint{0.882932in}{1.285327in}}%
\pgfpathlineto{\pgfqpoint{0.882932in}{0.499444in}}%
\pgfpathclose%
\pgfusepath{stroke}%
\end{pgfscope}%
\begin{pgfscope}%
\pgfpathrectangle{\pgfqpoint{0.445556in}{0.499444in}}{\pgfqpoint{3.875000in}{1.155000in}}%
\pgfusepath{clip}%
\pgfsetbuttcap%
\pgfsetmiterjoin%
\pgfsetlinewidth{1.003750pt}%
\definecolor{currentstroke}{rgb}{0.000000,0.000000,0.000000}%
\pgfsetstrokecolor{currentstroke}%
\pgfsetdash{}{0pt}%
\pgfpathmoveto{\pgfqpoint{1.036397in}{0.499444in}}%
\pgfpathlineto{\pgfqpoint{1.097783in}{0.499444in}}%
\pgfpathlineto{\pgfqpoint{1.097783in}{1.316621in}}%
\pgfpathlineto{\pgfqpoint{1.036397in}{1.316621in}}%
\pgfpathlineto{\pgfqpoint{1.036397in}{0.499444in}}%
\pgfpathclose%
\pgfusepath{stroke}%
\end{pgfscope}%
\begin{pgfscope}%
\pgfpathrectangle{\pgfqpoint{0.445556in}{0.499444in}}{\pgfqpoint{3.875000in}{1.155000in}}%
\pgfusepath{clip}%
\pgfsetbuttcap%
\pgfsetmiterjoin%
\pgfsetlinewidth{1.003750pt}%
\definecolor{currentstroke}{rgb}{0.000000,0.000000,0.000000}%
\pgfsetstrokecolor{currentstroke}%
\pgfsetdash{}{0pt}%
\pgfpathmoveto{\pgfqpoint{1.189863in}{0.499444in}}%
\pgfpathlineto{\pgfqpoint{1.251249in}{0.499444in}}%
\pgfpathlineto{\pgfqpoint{1.251249in}{1.329256in}}%
\pgfpathlineto{\pgfqpoint{1.189863in}{1.329256in}}%
\pgfpathlineto{\pgfqpoint{1.189863in}{0.499444in}}%
\pgfpathclose%
\pgfusepath{stroke}%
\end{pgfscope}%
\begin{pgfscope}%
\pgfpathrectangle{\pgfqpoint{0.445556in}{0.499444in}}{\pgfqpoint{3.875000in}{1.155000in}}%
\pgfusepath{clip}%
\pgfsetbuttcap%
\pgfsetmiterjoin%
\pgfsetlinewidth{1.003750pt}%
\definecolor{currentstroke}{rgb}{0.000000,0.000000,0.000000}%
\pgfsetstrokecolor{currentstroke}%
\pgfsetdash{}{0pt}%
\pgfpathmoveto{\pgfqpoint{1.343328in}{0.499444in}}%
\pgfpathlineto{\pgfqpoint{1.404714in}{0.499444in}}%
\pgfpathlineto{\pgfqpoint{1.404714in}{1.369643in}}%
\pgfpathlineto{\pgfqpoint{1.343328in}{1.369643in}}%
\pgfpathlineto{\pgfqpoint{1.343328in}{0.499444in}}%
\pgfpathclose%
\pgfusepath{stroke}%
\end{pgfscope}%
\begin{pgfscope}%
\pgfpathrectangle{\pgfqpoint{0.445556in}{0.499444in}}{\pgfqpoint{3.875000in}{1.155000in}}%
\pgfusepath{clip}%
\pgfsetbuttcap%
\pgfsetmiterjoin%
\pgfsetlinewidth{1.003750pt}%
\definecolor{currentstroke}{rgb}{0.000000,0.000000,0.000000}%
\pgfsetstrokecolor{currentstroke}%
\pgfsetdash{}{0pt}%
\pgfpathmoveto{\pgfqpoint{1.496793in}{0.499444in}}%
\pgfpathlineto{\pgfqpoint{1.558179in}{0.499444in}}%
\pgfpathlineto{\pgfqpoint{1.558179in}{1.389836in}}%
\pgfpathlineto{\pgfqpoint{1.496793in}{1.389836in}}%
\pgfpathlineto{\pgfqpoint{1.496793in}{0.499444in}}%
\pgfpathclose%
\pgfusepath{stroke}%
\end{pgfscope}%
\begin{pgfscope}%
\pgfpathrectangle{\pgfqpoint{0.445556in}{0.499444in}}{\pgfqpoint{3.875000in}{1.155000in}}%
\pgfusepath{clip}%
\pgfsetbuttcap%
\pgfsetmiterjoin%
\pgfsetlinewidth{1.003750pt}%
\definecolor{currentstroke}{rgb}{0.000000,0.000000,0.000000}%
\pgfsetstrokecolor{currentstroke}%
\pgfsetdash{}{0pt}%
\pgfpathmoveto{\pgfqpoint{1.650259in}{0.499444in}}%
\pgfpathlineto{\pgfqpoint{1.711645in}{0.499444in}}%
\pgfpathlineto{\pgfqpoint{1.711645in}{1.423728in}}%
\pgfpathlineto{\pgfqpoint{1.650259in}{1.423728in}}%
\pgfpathlineto{\pgfqpoint{1.650259in}{0.499444in}}%
\pgfpathclose%
\pgfusepath{stroke}%
\end{pgfscope}%
\begin{pgfscope}%
\pgfpathrectangle{\pgfqpoint{0.445556in}{0.499444in}}{\pgfqpoint{3.875000in}{1.155000in}}%
\pgfusepath{clip}%
\pgfsetbuttcap%
\pgfsetmiterjoin%
\pgfsetlinewidth{1.003750pt}%
\definecolor{currentstroke}{rgb}{0.000000,0.000000,0.000000}%
\pgfsetstrokecolor{currentstroke}%
\pgfsetdash{}{0pt}%
\pgfpathmoveto{\pgfqpoint{1.803724in}{0.499444in}}%
\pgfpathlineto{\pgfqpoint{1.865110in}{0.499444in}}%
\pgfpathlineto{\pgfqpoint{1.865110in}{1.450534in}}%
\pgfpathlineto{\pgfqpoint{1.803724in}{1.450534in}}%
\pgfpathlineto{\pgfqpoint{1.803724in}{0.499444in}}%
\pgfpathclose%
\pgfusepath{stroke}%
\end{pgfscope}%
\begin{pgfscope}%
\pgfpathrectangle{\pgfqpoint{0.445556in}{0.499444in}}{\pgfqpoint{3.875000in}{1.155000in}}%
\pgfusepath{clip}%
\pgfsetbuttcap%
\pgfsetmiterjoin%
\pgfsetlinewidth{1.003750pt}%
\definecolor{currentstroke}{rgb}{0.000000,0.000000,0.000000}%
\pgfsetstrokecolor{currentstroke}%
\pgfsetdash{}{0pt}%
\pgfpathmoveto{\pgfqpoint{1.957189in}{0.499444in}}%
\pgfpathlineto{\pgfqpoint{2.018575in}{0.499444in}}%
\pgfpathlineto{\pgfqpoint{2.018575in}{1.477104in}}%
\pgfpathlineto{\pgfqpoint{1.957189in}{1.477104in}}%
\pgfpathlineto{\pgfqpoint{1.957189in}{0.499444in}}%
\pgfpathclose%
\pgfusepath{stroke}%
\end{pgfscope}%
\begin{pgfscope}%
\pgfpathrectangle{\pgfqpoint{0.445556in}{0.499444in}}{\pgfqpoint{3.875000in}{1.155000in}}%
\pgfusepath{clip}%
\pgfsetbuttcap%
\pgfsetmiterjoin%
\pgfsetlinewidth{1.003750pt}%
\definecolor{currentstroke}{rgb}{0.000000,0.000000,0.000000}%
\pgfsetstrokecolor{currentstroke}%
\pgfsetdash{}{0pt}%
\pgfpathmoveto{\pgfqpoint{2.110655in}{0.499444in}}%
\pgfpathlineto{\pgfqpoint{2.172041in}{0.499444in}}%
\pgfpathlineto{\pgfqpoint{2.172041in}{1.501312in}}%
\pgfpathlineto{\pgfqpoint{2.110655in}{1.501312in}}%
\pgfpathlineto{\pgfqpoint{2.110655in}{0.499444in}}%
\pgfpathclose%
\pgfusepath{stroke}%
\end{pgfscope}%
\begin{pgfscope}%
\pgfpathrectangle{\pgfqpoint{0.445556in}{0.499444in}}{\pgfqpoint{3.875000in}{1.155000in}}%
\pgfusepath{clip}%
\pgfsetbuttcap%
\pgfsetmiterjoin%
\pgfsetlinewidth{1.003750pt}%
\definecolor{currentstroke}{rgb}{0.000000,0.000000,0.000000}%
\pgfsetstrokecolor{currentstroke}%
\pgfsetdash{}{0pt}%
\pgfpathmoveto{\pgfqpoint{2.264120in}{0.499444in}}%
\pgfpathlineto{\pgfqpoint{2.325506in}{0.499444in}}%
\pgfpathlineto{\pgfqpoint{2.325506in}{1.532960in}}%
\pgfpathlineto{\pgfqpoint{2.264120in}{1.532960in}}%
\pgfpathlineto{\pgfqpoint{2.264120in}{0.499444in}}%
\pgfpathclose%
\pgfusepath{stroke}%
\end{pgfscope}%
\begin{pgfscope}%
\pgfpathrectangle{\pgfqpoint{0.445556in}{0.499444in}}{\pgfqpoint{3.875000in}{1.155000in}}%
\pgfusepath{clip}%
\pgfsetbuttcap%
\pgfsetmiterjoin%
\pgfsetlinewidth{1.003750pt}%
\definecolor{currentstroke}{rgb}{0.000000,0.000000,0.000000}%
\pgfsetstrokecolor{currentstroke}%
\pgfsetdash{}{0pt}%
\pgfpathmoveto{\pgfqpoint{2.417585in}{0.499444in}}%
\pgfpathlineto{\pgfqpoint{2.478972in}{0.499444in}}%
\pgfpathlineto{\pgfqpoint{2.478972in}{1.543352in}}%
\pgfpathlineto{\pgfqpoint{2.417585in}{1.543352in}}%
\pgfpathlineto{\pgfqpoint{2.417585in}{0.499444in}}%
\pgfpathclose%
\pgfusepath{stroke}%
\end{pgfscope}%
\begin{pgfscope}%
\pgfpathrectangle{\pgfqpoint{0.445556in}{0.499444in}}{\pgfqpoint{3.875000in}{1.155000in}}%
\pgfusepath{clip}%
\pgfsetbuttcap%
\pgfsetmiterjoin%
\pgfsetlinewidth{1.003750pt}%
\definecolor{currentstroke}{rgb}{0.000000,0.000000,0.000000}%
\pgfsetstrokecolor{currentstroke}%
\pgfsetdash{}{0pt}%
\pgfpathmoveto{\pgfqpoint{2.571051in}{0.499444in}}%
\pgfpathlineto{\pgfqpoint{2.632437in}{0.499444in}}%
\pgfpathlineto{\pgfqpoint{2.632437in}{1.562718in}}%
\pgfpathlineto{\pgfqpoint{2.571051in}{1.562718in}}%
\pgfpathlineto{\pgfqpoint{2.571051in}{0.499444in}}%
\pgfpathclose%
\pgfusepath{stroke}%
\end{pgfscope}%
\begin{pgfscope}%
\pgfpathrectangle{\pgfqpoint{0.445556in}{0.499444in}}{\pgfqpoint{3.875000in}{1.155000in}}%
\pgfusepath{clip}%
\pgfsetbuttcap%
\pgfsetmiterjoin%
\pgfsetlinewidth{1.003750pt}%
\definecolor{currentstroke}{rgb}{0.000000,0.000000,0.000000}%
\pgfsetstrokecolor{currentstroke}%
\pgfsetdash{}{0pt}%
\pgfpathmoveto{\pgfqpoint{2.724516in}{0.499444in}}%
\pgfpathlineto{\pgfqpoint{2.785902in}{0.499444in}}%
\pgfpathlineto{\pgfqpoint{2.785902in}{1.589052in}}%
\pgfpathlineto{\pgfqpoint{2.724516in}{1.589052in}}%
\pgfpathlineto{\pgfqpoint{2.724516in}{0.499444in}}%
\pgfpathclose%
\pgfusepath{stroke}%
\end{pgfscope}%
\begin{pgfscope}%
\pgfpathrectangle{\pgfqpoint{0.445556in}{0.499444in}}{\pgfqpoint{3.875000in}{1.155000in}}%
\pgfusepath{clip}%
\pgfsetbuttcap%
\pgfsetmiterjoin%
\pgfsetlinewidth{1.003750pt}%
\definecolor{currentstroke}{rgb}{0.000000,0.000000,0.000000}%
\pgfsetstrokecolor{currentstroke}%
\pgfsetdash{}{0pt}%
\pgfpathmoveto{\pgfqpoint{2.877981in}{0.499444in}}%
\pgfpathlineto{\pgfqpoint{2.939368in}{0.499444in}}%
\pgfpathlineto{\pgfqpoint{2.939368in}{1.599444in}}%
\pgfpathlineto{\pgfqpoint{2.877981in}{1.599444in}}%
\pgfpathlineto{\pgfqpoint{2.877981in}{0.499444in}}%
\pgfpathclose%
\pgfusepath{stroke}%
\end{pgfscope}%
\begin{pgfscope}%
\pgfpathrectangle{\pgfqpoint{0.445556in}{0.499444in}}{\pgfqpoint{3.875000in}{1.155000in}}%
\pgfusepath{clip}%
\pgfsetbuttcap%
\pgfsetmiterjoin%
\pgfsetlinewidth{1.003750pt}%
\definecolor{currentstroke}{rgb}{0.000000,0.000000,0.000000}%
\pgfsetstrokecolor{currentstroke}%
\pgfsetdash{}{0pt}%
\pgfpathmoveto{\pgfqpoint{3.031447in}{0.499444in}}%
\pgfpathlineto{\pgfqpoint{3.092833in}{0.499444in}}%
\pgfpathlineto{\pgfqpoint{3.092833in}{1.557759in}}%
\pgfpathlineto{\pgfqpoint{3.031447in}{1.557759in}}%
\pgfpathlineto{\pgfqpoint{3.031447in}{0.499444in}}%
\pgfpathclose%
\pgfusepath{stroke}%
\end{pgfscope}%
\begin{pgfscope}%
\pgfpathrectangle{\pgfqpoint{0.445556in}{0.499444in}}{\pgfqpoint{3.875000in}{1.155000in}}%
\pgfusepath{clip}%
\pgfsetbuttcap%
\pgfsetmiterjoin%
\pgfsetlinewidth{1.003750pt}%
\definecolor{currentstroke}{rgb}{0.000000,0.000000,0.000000}%
\pgfsetstrokecolor{currentstroke}%
\pgfsetdash{}{0pt}%
\pgfpathmoveto{\pgfqpoint{3.184912in}{0.499444in}}%
\pgfpathlineto{\pgfqpoint{3.246298in}{0.499444in}}%
\pgfpathlineto{\pgfqpoint{3.246298in}{1.555043in}}%
\pgfpathlineto{\pgfqpoint{3.184912in}{1.555043in}}%
\pgfpathlineto{\pgfqpoint{3.184912in}{0.499444in}}%
\pgfpathclose%
\pgfusepath{stroke}%
\end{pgfscope}%
\begin{pgfscope}%
\pgfpathrectangle{\pgfqpoint{0.445556in}{0.499444in}}{\pgfqpoint{3.875000in}{1.155000in}}%
\pgfusepath{clip}%
\pgfsetbuttcap%
\pgfsetmiterjoin%
\pgfsetlinewidth{1.003750pt}%
\definecolor{currentstroke}{rgb}{0.000000,0.000000,0.000000}%
\pgfsetstrokecolor{currentstroke}%
\pgfsetdash{}{0pt}%
\pgfpathmoveto{\pgfqpoint{3.338377in}{0.499444in}}%
\pgfpathlineto{\pgfqpoint{3.399764in}{0.499444in}}%
\pgfpathlineto{\pgfqpoint{3.399764in}{1.526111in}}%
\pgfpathlineto{\pgfqpoint{3.338377in}{1.526111in}}%
\pgfpathlineto{\pgfqpoint{3.338377in}{0.499444in}}%
\pgfpathclose%
\pgfusepath{stroke}%
\end{pgfscope}%
\begin{pgfscope}%
\pgfpathrectangle{\pgfqpoint{0.445556in}{0.499444in}}{\pgfqpoint{3.875000in}{1.155000in}}%
\pgfusepath{clip}%
\pgfsetbuttcap%
\pgfsetmiterjoin%
\pgfsetlinewidth{1.003750pt}%
\definecolor{currentstroke}{rgb}{0.000000,0.000000,0.000000}%
\pgfsetstrokecolor{currentstroke}%
\pgfsetdash{}{0pt}%
\pgfpathmoveto{\pgfqpoint{3.491843in}{0.499444in}}%
\pgfpathlineto{\pgfqpoint{3.553229in}{0.499444in}}%
\pgfpathlineto{\pgfqpoint{3.553229in}{1.448999in}}%
\pgfpathlineto{\pgfqpoint{3.491843in}{1.448999in}}%
\pgfpathlineto{\pgfqpoint{3.491843in}{0.499444in}}%
\pgfpathclose%
\pgfusepath{stroke}%
\end{pgfscope}%
\begin{pgfscope}%
\pgfpathrectangle{\pgfqpoint{0.445556in}{0.499444in}}{\pgfqpoint{3.875000in}{1.155000in}}%
\pgfusepath{clip}%
\pgfsetbuttcap%
\pgfsetmiterjoin%
\pgfsetlinewidth{1.003750pt}%
\definecolor{currentstroke}{rgb}{0.000000,0.000000,0.000000}%
\pgfsetstrokecolor{currentstroke}%
\pgfsetdash{}{0pt}%
\pgfpathmoveto{\pgfqpoint{3.645308in}{0.499444in}}%
\pgfpathlineto{\pgfqpoint{3.706694in}{0.499444in}}%
\pgfpathlineto{\pgfqpoint{3.706694in}{1.348032in}}%
\pgfpathlineto{\pgfqpoint{3.645308in}{1.348032in}}%
\pgfpathlineto{\pgfqpoint{3.645308in}{0.499444in}}%
\pgfpathclose%
\pgfusepath{stroke}%
\end{pgfscope}%
\begin{pgfscope}%
\pgfpathrectangle{\pgfqpoint{0.445556in}{0.499444in}}{\pgfqpoint{3.875000in}{1.155000in}}%
\pgfusepath{clip}%
\pgfsetbuttcap%
\pgfsetmiterjoin%
\pgfsetlinewidth{1.003750pt}%
\definecolor{currentstroke}{rgb}{0.000000,0.000000,0.000000}%
\pgfsetstrokecolor{currentstroke}%
\pgfsetdash{}{0pt}%
\pgfpathmoveto{\pgfqpoint{3.798774in}{0.499444in}}%
\pgfpathlineto{\pgfqpoint{3.860160in}{0.499444in}}%
\pgfpathlineto{\pgfqpoint{3.860160in}{1.173143in}}%
\pgfpathlineto{\pgfqpoint{3.798774in}{1.173143in}}%
\pgfpathlineto{\pgfqpoint{3.798774in}{0.499444in}}%
\pgfpathclose%
\pgfusepath{stroke}%
\end{pgfscope}%
\begin{pgfscope}%
\pgfpathrectangle{\pgfqpoint{0.445556in}{0.499444in}}{\pgfqpoint{3.875000in}{1.155000in}}%
\pgfusepath{clip}%
\pgfsetbuttcap%
\pgfsetmiterjoin%
\pgfsetlinewidth{1.003750pt}%
\definecolor{currentstroke}{rgb}{0.000000,0.000000,0.000000}%
\pgfsetstrokecolor{currentstroke}%
\pgfsetdash{}{0pt}%
\pgfpathmoveto{\pgfqpoint{3.952239in}{0.499444in}}%
\pgfpathlineto{\pgfqpoint{4.013625in}{0.499444in}}%
\pgfpathlineto{\pgfqpoint{4.013625in}{0.920432in}}%
\pgfpathlineto{\pgfqpoint{3.952239in}{0.920432in}}%
\pgfpathlineto{\pgfqpoint{3.952239in}{0.499444in}}%
\pgfpathclose%
\pgfusepath{stroke}%
\end{pgfscope}%
\begin{pgfscope}%
\pgfpathrectangle{\pgfqpoint{0.445556in}{0.499444in}}{\pgfqpoint{3.875000in}{1.155000in}}%
\pgfusepath{clip}%
\pgfsetbuttcap%
\pgfsetmiterjoin%
\pgfsetlinewidth{1.003750pt}%
\definecolor{currentstroke}{rgb}{0.000000,0.000000,0.000000}%
\pgfsetstrokecolor{currentstroke}%
\pgfsetdash{}{0pt}%
\pgfpathmoveto{\pgfqpoint{4.105704in}{0.499444in}}%
\pgfpathlineto{\pgfqpoint{4.167090in}{0.499444in}}%
\pgfpathlineto{\pgfqpoint{4.167090in}{0.660400in}}%
\pgfpathlineto{\pgfqpoint{4.105704in}{0.660400in}}%
\pgfpathlineto{\pgfqpoint{4.105704in}{0.499444in}}%
\pgfpathclose%
\pgfusepath{stroke}%
\end{pgfscope}%
\begin{pgfscope}%
\pgfpathrectangle{\pgfqpoint{0.445556in}{0.499444in}}{\pgfqpoint{3.875000in}{1.155000in}}%
\pgfusepath{clip}%
\pgfsetbuttcap%
\pgfsetmiterjoin%
\definecolor{currentfill}{rgb}{0.000000,0.000000,0.000000}%
\pgfsetfillcolor{currentfill}%
\pgfsetlinewidth{0.000000pt}%
\definecolor{currentstroke}{rgb}{0.000000,0.000000,0.000000}%
\pgfsetstrokecolor{currentstroke}%
\pgfsetstrokeopacity{0.000000}%
\pgfsetdash{}{0pt}%
\pgfpathmoveto{\pgfqpoint{0.483922in}{0.499444in}}%
\pgfpathlineto{\pgfqpoint{0.545308in}{0.499444in}}%
\pgfpathlineto{\pgfqpoint{0.545308in}{0.502514in}}%
\pgfpathlineto{\pgfqpoint{0.483922in}{0.502514in}}%
\pgfpathlineto{\pgfqpoint{0.483922in}{0.499444in}}%
\pgfpathclose%
\pgfusepath{fill}%
\end{pgfscope}%
\begin{pgfscope}%
\pgfpathrectangle{\pgfqpoint{0.445556in}{0.499444in}}{\pgfqpoint{3.875000in}{1.155000in}}%
\pgfusepath{clip}%
\pgfsetbuttcap%
\pgfsetmiterjoin%
\definecolor{currentfill}{rgb}{0.000000,0.000000,0.000000}%
\pgfsetfillcolor{currentfill}%
\pgfsetlinewidth{0.000000pt}%
\definecolor{currentstroke}{rgb}{0.000000,0.000000,0.000000}%
\pgfsetstrokecolor{currentstroke}%
\pgfsetstrokeopacity{0.000000}%
\pgfsetdash{}{0pt}%
\pgfpathmoveto{\pgfqpoint{0.637387in}{0.499444in}}%
\pgfpathlineto{\pgfqpoint{0.698774in}{0.499444in}}%
\pgfpathlineto{\pgfqpoint{0.698774in}{0.506411in}}%
\pgfpathlineto{\pgfqpoint{0.637387in}{0.506411in}}%
\pgfpathlineto{\pgfqpoint{0.637387in}{0.499444in}}%
\pgfpathclose%
\pgfusepath{fill}%
\end{pgfscope}%
\begin{pgfscope}%
\pgfpathrectangle{\pgfqpoint{0.445556in}{0.499444in}}{\pgfqpoint{3.875000in}{1.155000in}}%
\pgfusepath{clip}%
\pgfsetbuttcap%
\pgfsetmiterjoin%
\definecolor{currentfill}{rgb}{0.000000,0.000000,0.000000}%
\pgfsetfillcolor{currentfill}%
\pgfsetlinewidth{0.000000pt}%
\definecolor{currentstroke}{rgb}{0.000000,0.000000,0.000000}%
\pgfsetstrokecolor{currentstroke}%
\pgfsetstrokeopacity{0.000000}%
\pgfsetdash{}{0pt}%
\pgfpathmoveto{\pgfqpoint{0.790853in}{0.499444in}}%
\pgfpathlineto{\pgfqpoint{0.852239in}{0.499444in}}%
\pgfpathlineto{\pgfqpoint{0.852239in}{0.511489in}}%
\pgfpathlineto{\pgfqpoint{0.790853in}{0.511489in}}%
\pgfpathlineto{\pgfqpoint{0.790853in}{0.499444in}}%
\pgfpathclose%
\pgfusepath{fill}%
\end{pgfscope}%
\begin{pgfscope}%
\pgfpathrectangle{\pgfqpoint{0.445556in}{0.499444in}}{\pgfqpoint{3.875000in}{1.155000in}}%
\pgfusepath{clip}%
\pgfsetbuttcap%
\pgfsetmiterjoin%
\definecolor{currentfill}{rgb}{0.000000,0.000000,0.000000}%
\pgfsetfillcolor{currentfill}%
\pgfsetlinewidth{0.000000pt}%
\definecolor{currentstroke}{rgb}{0.000000,0.000000,0.000000}%
\pgfsetstrokecolor{currentstroke}%
\pgfsetstrokeopacity{0.000000}%
\pgfsetdash{}{0pt}%
\pgfpathmoveto{\pgfqpoint{0.944318in}{0.499444in}}%
\pgfpathlineto{\pgfqpoint{1.005704in}{0.499444in}}%
\pgfpathlineto{\pgfqpoint{1.005704in}{0.515268in}}%
\pgfpathlineto{\pgfqpoint{0.944318in}{0.515268in}}%
\pgfpathlineto{\pgfqpoint{0.944318in}{0.499444in}}%
\pgfpathclose%
\pgfusepath{fill}%
\end{pgfscope}%
\begin{pgfscope}%
\pgfpathrectangle{\pgfqpoint{0.445556in}{0.499444in}}{\pgfqpoint{3.875000in}{1.155000in}}%
\pgfusepath{clip}%
\pgfsetbuttcap%
\pgfsetmiterjoin%
\definecolor{currentfill}{rgb}{0.000000,0.000000,0.000000}%
\pgfsetfillcolor{currentfill}%
\pgfsetlinewidth{0.000000pt}%
\definecolor{currentstroke}{rgb}{0.000000,0.000000,0.000000}%
\pgfsetstrokecolor{currentstroke}%
\pgfsetstrokeopacity{0.000000}%
\pgfsetdash{}{0pt}%
\pgfpathmoveto{\pgfqpoint{1.097783in}{0.499444in}}%
\pgfpathlineto{\pgfqpoint{1.159170in}{0.499444in}}%
\pgfpathlineto{\pgfqpoint{1.159170in}{0.523416in}}%
\pgfpathlineto{\pgfqpoint{1.097783in}{0.523416in}}%
\pgfpathlineto{\pgfqpoint{1.097783in}{0.499444in}}%
\pgfpathclose%
\pgfusepath{fill}%
\end{pgfscope}%
\begin{pgfscope}%
\pgfpathrectangle{\pgfqpoint{0.445556in}{0.499444in}}{\pgfqpoint{3.875000in}{1.155000in}}%
\pgfusepath{clip}%
\pgfsetbuttcap%
\pgfsetmiterjoin%
\definecolor{currentfill}{rgb}{0.000000,0.000000,0.000000}%
\pgfsetfillcolor{currentfill}%
\pgfsetlinewidth{0.000000pt}%
\definecolor{currentstroke}{rgb}{0.000000,0.000000,0.000000}%
\pgfsetstrokecolor{currentstroke}%
\pgfsetstrokeopacity{0.000000}%
\pgfsetdash{}{0pt}%
\pgfpathmoveto{\pgfqpoint{1.251249in}{0.499444in}}%
\pgfpathlineto{\pgfqpoint{1.312635in}{0.499444in}}%
\pgfpathlineto{\pgfqpoint{1.312635in}{0.530029in}}%
\pgfpathlineto{\pgfqpoint{1.251249in}{0.530029in}}%
\pgfpathlineto{\pgfqpoint{1.251249in}{0.499444in}}%
\pgfpathclose%
\pgfusepath{fill}%
\end{pgfscope}%
\begin{pgfscope}%
\pgfpathrectangle{\pgfqpoint{0.445556in}{0.499444in}}{\pgfqpoint{3.875000in}{1.155000in}}%
\pgfusepath{clip}%
\pgfsetbuttcap%
\pgfsetmiterjoin%
\definecolor{currentfill}{rgb}{0.000000,0.000000,0.000000}%
\pgfsetfillcolor{currentfill}%
\pgfsetlinewidth{0.000000pt}%
\definecolor{currentstroke}{rgb}{0.000000,0.000000,0.000000}%
\pgfsetstrokecolor{currentstroke}%
\pgfsetstrokeopacity{0.000000}%
\pgfsetdash{}{0pt}%
\pgfpathmoveto{\pgfqpoint{1.404714in}{0.499444in}}%
\pgfpathlineto{\pgfqpoint{1.466100in}{0.499444in}}%
\pgfpathlineto{\pgfqpoint{1.466100in}{0.542547in}}%
\pgfpathlineto{\pgfqpoint{1.404714in}{0.542547in}}%
\pgfpathlineto{\pgfqpoint{1.404714in}{0.499444in}}%
\pgfpathclose%
\pgfusepath{fill}%
\end{pgfscope}%
\begin{pgfscope}%
\pgfpathrectangle{\pgfqpoint{0.445556in}{0.499444in}}{\pgfqpoint{3.875000in}{1.155000in}}%
\pgfusepath{clip}%
\pgfsetbuttcap%
\pgfsetmiterjoin%
\definecolor{currentfill}{rgb}{0.000000,0.000000,0.000000}%
\pgfsetfillcolor{currentfill}%
\pgfsetlinewidth{0.000000pt}%
\definecolor{currentstroke}{rgb}{0.000000,0.000000,0.000000}%
\pgfsetstrokecolor{currentstroke}%
\pgfsetstrokeopacity{0.000000}%
\pgfsetdash{}{0pt}%
\pgfpathmoveto{\pgfqpoint{1.558179in}{0.499444in}}%
\pgfpathlineto{\pgfqpoint{1.619566in}{0.499444in}}%
\pgfpathlineto{\pgfqpoint{1.619566in}{0.543019in}}%
\pgfpathlineto{\pgfqpoint{1.558179in}{0.543019in}}%
\pgfpathlineto{\pgfqpoint{1.558179in}{0.499444in}}%
\pgfpathclose%
\pgfusepath{fill}%
\end{pgfscope}%
\begin{pgfscope}%
\pgfpathrectangle{\pgfqpoint{0.445556in}{0.499444in}}{\pgfqpoint{3.875000in}{1.155000in}}%
\pgfusepath{clip}%
\pgfsetbuttcap%
\pgfsetmiterjoin%
\definecolor{currentfill}{rgb}{0.000000,0.000000,0.000000}%
\pgfsetfillcolor{currentfill}%
\pgfsetlinewidth{0.000000pt}%
\definecolor{currentstroke}{rgb}{0.000000,0.000000,0.000000}%
\pgfsetstrokecolor{currentstroke}%
\pgfsetstrokeopacity{0.000000}%
\pgfsetdash{}{0pt}%
\pgfpathmoveto{\pgfqpoint{1.711645in}{0.499444in}}%
\pgfpathlineto{\pgfqpoint{1.773031in}{0.499444in}}%
\pgfpathlineto{\pgfqpoint{1.773031in}{0.552230in}}%
\pgfpathlineto{\pgfqpoint{1.711645in}{0.552230in}}%
\pgfpathlineto{\pgfqpoint{1.711645in}{0.499444in}}%
\pgfpathclose%
\pgfusepath{fill}%
\end{pgfscope}%
\begin{pgfscope}%
\pgfpathrectangle{\pgfqpoint{0.445556in}{0.499444in}}{\pgfqpoint{3.875000in}{1.155000in}}%
\pgfusepath{clip}%
\pgfsetbuttcap%
\pgfsetmiterjoin%
\definecolor{currentfill}{rgb}{0.000000,0.000000,0.000000}%
\pgfsetfillcolor{currentfill}%
\pgfsetlinewidth{0.000000pt}%
\definecolor{currentstroke}{rgb}{0.000000,0.000000,0.000000}%
\pgfsetstrokecolor{currentstroke}%
\pgfsetstrokeopacity{0.000000}%
\pgfsetdash{}{0pt}%
\pgfpathmoveto{\pgfqpoint{1.865110in}{0.499444in}}%
\pgfpathlineto{\pgfqpoint{1.926496in}{0.499444in}}%
\pgfpathlineto{\pgfqpoint{1.926496in}{0.567463in}}%
\pgfpathlineto{\pgfqpoint{1.865110in}{0.567463in}}%
\pgfpathlineto{\pgfqpoint{1.865110in}{0.499444in}}%
\pgfpathclose%
\pgfusepath{fill}%
\end{pgfscope}%
\begin{pgfscope}%
\pgfpathrectangle{\pgfqpoint{0.445556in}{0.499444in}}{\pgfqpoint{3.875000in}{1.155000in}}%
\pgfusepath{clip}%
\pgfsetbuttcap%
\pgfsetmiterjoin%
\definecolor{currentfill}{rgb}{0.000000,0.000000,0.000000}%
\pgfsetfillcolor{currentfill}%
\pgfsetlinewidth{0.000000pt}%
\definecolor{currentstroke}{rgb}{0.000000,0.000000,0.000000}%
\pgfsetstrokecolor{currentstroke}%
\pgfsetstrokeopacity{0.000000}%
\pgfsetdash{}{0pt}%
\pgfpathmoveto{\pgfqpoint{2.018575in}{0.499444in}}%
\pgfpathlineto{\pgfqpoint{2.079962in}{0.499444in}}%
\pgfpathlineto{\pgfqpoint{2.079962in}{0.581162in}}%
\pgfpathlineto{\pgfqpoint{2.018575in}{0.581162in}}%
\pgfpathlineto{\pgfqpoint{2.018575in}{0.499444in}}%
\pgfpathclose%
\pgfusepath{fill}%
\end{pgfscope}%
\begin{pgfscope}%
\pgfpathrectangle{\pgfqpoint{0.445556in}{0.499444in}}{\pgfqpoint{3.875000in}{1.155000in}}%
\pgfusepath{clip}%
\pgfsetbuttcap%
\pgfsetmiterjoin%
\definecolor{currentfill}{rgb}{0.000000,0.000000,0.000000}%
\pgfsetfillcolor{currentfill}%
\pgfsetlinewidth{0.000000pt}%
\definecolor{currentstroke}{rgb}{0.000000,0.000000,0.000000}%
\pgfsetstrokecolor{currentstroke}%
\pgfsetstrokeopacity{0.000000}%
\pgfsetdash{}{0pt}%
\pgfpathmoveto{\pgfqpoint{2.172041in}{0.499444in}}%
\pgfpathlineto{\pgfqpoint{2.233427in}{0.499444in}}%
\pgfpathlineto{\pgfqpoint{2.233427in}{0.590255in}}%
\pgfpathlineto{\pgfqpoint{2.172041in}{0.590255in}}%
\pgfpathlineto{\pgfqpoint{2.172041in}{0.499444in}}%
\pgfpathclose%
\pgfusepath{fill}%
\end{pgfscope}%
\begin{pgfscope}%
\pgfpathrectangle{\pgfqpoint{0.445556in}{0.499444in}}{\pgfqpoint{3.875000in}{1.155000in}}%
\pgfusepath{clip}%
\pgfsetbuttcap%
\pgfsetmiterjoin%
\definecolor{currentfill}{rgb}{0.000000,0.000000,0.000000}%
\pgfsetfillcolor{currentfill}%
\pgfsetlinewidth{0.000000pt}%
\definecolor{currentstroke}{rgb}{0.000000,0.000000,0.000000}%
\pgfsetstrokecolor{currentstroke}%
\pgfsetstrokeopacity{0.000000}%
\pgfsetdash{}{0pt}%
\pgfpathmoveto{\pgfqpoint{2.325506in}{0.499444in}}%
\pgfpathlineto{\pgfqpoint{2.386892in}{0.499444in}}%
\pgfpathlineto{\pgfqpoint{2.386892in}{0.604189in}}%
\pgfpathlineto{\pgfqpoint{2.325506in}{0.604189in}}%
\pgfpathlineto{\pgfqpoint{2.325506in}{0.499444in}}%
\pgfpathclose%
\pgfusepath{fill}%
\end{pgfscope}%
\begin{pgfscope}%
\pgfpathrectangle{\pgfqpoint{0.445556in}{0.499444in}}{\pgfqpoint{3.875000in}{1.155000in}}%
\pgfusepath{clip}%
\pgfsetbuttcap%
\pgfsetmiterjoin%
\definecolor{currentfill}{rgb}{0.000000,0.000000,0.000000}%
\pgfsetfillcolor{currentfill}%
\pgfsetlinewidth{0.000000pt}%
\definecolor{currentstroke}{rgb}{0.000000,0.000000,0.000000}%
\pgfsetstrokecolor{currentstroke}%
\pgfsetstrokeopacity{0.000000}%
\pgfsetdash{}{0pt}%
\pgfpathmoveto{\pgfqpoint{2.478972in}{0.499444in}}%
\pgfpathlineto{\pgfqpoint{2.540358in}{0.499444in}}%
\pgfpathlineto{\pgfqpoint{2.540358in}{0.622965in}}%
\pgfpathlineto{\pgfqpoint{2.478972in}{0.622965in}}%
\pgfpathlineto{\pgfqpoint{2.478972in}{0.499444in}}%
\pgfpathclose%
\pgfusepath{fill}%
\end{pgfscope}%
\begin{pgfscope}%
\pgfpathrectangle{\pgfqpoint{0.445556in}{0.499444in}}{\pgfqpoint{3.875000in}{1.155000in}}%
\pgfusepath{clip}%
\pgfsetbuttcap%
\pgfsetmiterjoin%
\definecolor{currentfill}{rgb}{0.000000,0.000000,0.000000}%
\pgfsetfillcolor{currentfill}%
\pgfsetlinewidth{0.000000pt}%
\definecolor{currentstroke}{rgb}{0.000000,0.000000,0.000000}%
\pgfsetstrokecolor{currentstroke}%
\pgfsetstrokeopacity{0.000000}%
\pgfsetdash{}{0pt}%
\pgfpathmoveto{\pgfqpoint{2.632437in}{0.499444in}}%
\pgfpathlineto{\pgfqpoint{2.693823in}{0.499444in}}%
\pgfpathlineto{\pgfqpoint{2.693823in}{0.647292in}}%
\pgfpathlineto{\pgfqpoint{2.632437in}{0.647292in}}%
\pgfpathlineto{\pgfqpoint{2.632437in}{0.499444in}}%
\pgfpathclose%
\pgfusepath{fill}%
\end{pgfscope}%
\begin{pgfscope}%
\pgfpathrectangle{\pgfqpoint{0.445556in}{0.499444in}}{\pgfqpoint{3.875000in}{1.155000in}}%
\pgfusepath{clip}%
\pgfsetbuttcap%
\pgfsetmiterjoin%
\definecolor{currentfill}{rgb}{0.000000,0.000000,0.000000}%
\pgfsetfillcolor{currentfill}%
\pgfsetlinewidth{0.000000pt}%
\definecolor{currentstroke}{rgb}{0.000000,0.000000,0.000000}%
\pgfsetstrokecolor{currentstroke}%
\pgfsetstrokeopacity{0.000000}%
\pgfsetdash{}{0pt}%
\pgfpathmoveto{\pgfqpoint{2.785902in}{0.499444in}}%
\pgfpathlineto{\pgfqpoint{2.847288in}{0.499444in}}%
\pgfpathlineto{\pgfqpoint{2.847288in}{0.671264in}}%
\pgfpathlineto{\pgfqpoint{2.785902in}{0.671264in}}%
\pgfpathlineto{\pgfqpoint{2.785902in}{0.499444in}}%
\pgfpathclose%
\pgfusepath{fill}%
\end{pgfscope}%
\begin{pgfscope}%
\pgfpathrectangle{\pgfqpoint{0.445556in}{0.499444in}}{\pgfqpoint{3.875000in}{1.155000in}}%
\pgfusepath{clip}%
\pgfsetbuttcap%
\pgfsetmiterjoin%
\definecolor{currentfill}{rgb}{0.000000,0.000000,0.000000}%
\pgfsetfillcolor{currentfill}%
\pgfsetlinewidth{0.000000pt}%
\definecolor{currentstroke}{rgb}{0.000000,0.000000,0.000000}%
\pgfsetstrokecolor{currentstroke}%
\pgfsetstrokeopacity{0.000000}%
\pgfsetdash{}{0pt}%
\pgfpathmoveto{\pgfqpoint{2.939368in}{0.499444in}}%
\pgfpathlineto{\pgfqpoint{3.000754in}{0.499444in}}%
\pgfpathlineto{\pgfqpoint{3.000754in}{0.687442in}}%
\pgfpathlineto{\pgfqpoint{2.939368in}{0.687442in}}%
\pgfpathlineto{\pgfqpoint{2.939368in}{0.499444in}}%
\pgfpathclose%
\pgfusepath{fill}%
\end{pgfscope}%
\begin{pgfscope}%
\pgfpathrectangle{\pgfqpoint{0.445556in}{0.499444in}}{\pgfqpoint{3.875000in}{1.155000in}}%
\pgfusepath{clip}%
\pgfsetbuttcap%
\pgfsetmiterjoin%
\definecolor{currentfill}{rgb}{0.000000,0.000000,0.000000}%
\pgfsetfillcolor{currentfill}%
\pgfsetlinewidth{0.000000pt}%
\definecolor{currentstroke}{rgb}{0.000000,0.000000,0.000000}%
\pgfsetstrokecolor{currentstroke}%
\pgfsetstrokeopacity{0.000000}%
\pgfsetdash{}{0pt}%
\pgfpathmoveto{\pgfqpoint{3.092833in}{0.499444in}}%
\pgfpathlineto{\pgfqpoint{3.154219in}{0.499444in}}%
\pgfpathlineto{\pgfqpoint{3.154219in}{0.722869in}}%
\pgfpathlineto{\pgfqpoint{3.092833in}{0.722869in}}%
\pgfpathlineto{\pgfqpoint{3.092833in}{0.499444in}}%
\pgfpathclose%
\pgfusepath{fill}%
\end{pgfscope}%
\begin{pgfscope}%
\pgfpathrectangle{\pgfqpoint{0.445556in}{0.499444in}}{\pgfqpoint{3.875000in}{1.155000in}}%
\pgfusepath{clip}%
\pgfsetbuttcap%
\pgfsetmiterjoin%
\definecolor{currentfill}{rgb}{0.000000,0.000000,0.000000}%
\pgfsetfillcolor{currentfill}%
\pgfsetlinewidth{0.000000pt}%
\definecolor{currentstroke}{rgb}{0.000000,0.000000,0.000000}%
\pgfsetstrokecolor{currentstroke}%
\pgfsetstrokeopacity{0.000000}%
\pgfsetdash{}{0pt}%
\pgfpathmoveto{\pgfqpoint{3.246298in}{0.499444in}}%
\pgfpathlineto{\pgfqpoint{3.307684in}{0.499444in}}%
\pgfpathlineto{\pgfqpoint{3.307684in}{0.758059in}}%
\pgfpathlineto{\pgfqpoint{3.246298in}{0.758059in}}%
\pgfpathlineto{\pgfqpoint{3.246298in}{0.499444in}}%
\pgfpathclose%
\pgfusepath{fill}%
\end{pgfscope}%
\begin{pgfscope}%
\pgfpathrectangle{\pgfqpoint{0.445556in}{0.499444in}}{\pgfqpoint{3.875000in}{1.155000in}}%
\pgfusepath{clip}%
\pgfsetbuttcap%
\pgfsetmiterjoin%
\definecolor{currentfill}{rgb}{0.000000,0.000000,0.000000}%
\pgfsetfillcolor{currentfill}%
\pgfsetlinewidth{0.000000pt}%
\definecolor{currentstroke}{rgb}{0.000000,0.000000,0.000000}%
\pgfsetstrokecolor{currentstroke}%
\pgfsetstrokeopacity{0.000000}%
\pgfsetdash{}{0pt}%
\pgfpathmoveto{\pgfqpoint{3.399764in}{0.499444in}}%
\pgfpathlineto{\pgfqpoint{3.461150in}{0.499444in}}%
\pgfpathlineto{\pgfqpoint{3.461150in}{0.806476in}}%
\pgfpathlineto{\pgfqpoint{3.399764in}{0.806476in}}%
\pgfpathlineto{\pgfqpoint{3.399764in}{0.499444in}}%
\pgfpathclose%
\pgfusepath{fill}%
\end{pgfscope}%
\begin{pgfscope}%
\pgfpathrectangle{\pgfqpoint{0.445556in}{0.499444in}}{\pgfqpoint{3.875000in}{1.155000in}}%
\pgfusepath{clip}%
\pgfsetbuttcap%
\pgfsetmiterjoin%
\definecolor{currentfill}{rgb}{0.000000,0.000000,0.000000}%
\pgfsetfillcolor{currentfill}%
\pgfsetlinewidth{0.000000pt}%
\definecolor{currentstroke}{rgb}{0.000000,0.000000,0.000000}%
\pgfsetstrokecolor{currentstroke}%
\pgfsetstrokeopacity{0.000000}%
\pgfsetdash{}{0pt}%
\pgfpathmoveto{\pgfqpoint{3.553229in}{0.499444in}}%
\pgfpathlineto{\pgfqpoint{3.614615in}{0.499444in}}%
\pgfpathlineto{\pgfqpoint{3.614615in}{0.846862in}}%
\pgfpathlineto{\pgfqpoint{3.553229in}{0.846862in}}%
\pgfpathlineto{\pgfqpoint{3.553229in}{0.499444in}}%
\pgfpathclose%
\pgfusepath{fill}%
\end{pgfscope}%
\begin{pgfscope}%
\pgfpathrectangle{\pgfqpoint{0.445556in}{0.499444in}}{\pgfqpoint{3.875000in}{1.155000in}}%
\pgfusepath{clip}%
\pgfsetbuttcap%
\pgfsetmiterjoin%
\definecolor{currentfill}{rgb}{0.000000,0.000000,0.000000}%
\pgfsetfillcolor{currentfill}%
\pgfsetlinewidth{0.000000pt}%
\definecolor{currentstroke}{rgb}{0.000000,0.000000,0.000000}%
\pgfsetstrokecolor{currentstroke}%
\pgfsetstrokeopacity{0.000000}%
\pgfsetdash{}{0pt}%
\pgfpathmoveto{\pgfqpoint{3.706694in}{0.499444in}}%
\pgfpathlineto{\pgfqpoint{3.768080in}{0.499444in}}%
\pgfpathlineto{\pgfqpoint{3.768080in}{0.900947in}}%
\pgfpathlineto{\pgfqpoint{3.706694in}{0.900947in}}%
\pgfpathlineto{\pgfqpoint{3.706694in}{0.499444in}}%
\pgfpathclose%
\pgfusepath{fill}%
\end{pgfscope}%
\begin{pgfscope}%
\pgfpathrectangle{\pgfqpoint{0.445556in}{0.499444in}}{\pgfqpoint{3.875000in}{1.155000in}}%
\pgfusepath{clip}%
\pgfsetbuttcap%
\pgfsetmiterjoin%
\definecolor{currentfill}{rgb}{0.000000,0.000000,0.000000}%
\pgfsetfillcolor{currentfill}%
\pgfsetlinewidth{0.000000pt}%
\definecolor{currentstroke}{rgb}{0.000000,0.000000,0.000000}%
\pgfsetstrokecolor{currentstroke}%
\pgfsetstrokeopacity{0.000000}%
\pgfsetdash{}{0pt}%
\pgfpathmoveto{\pgfqpoint{3.860160in}{0.499444in}}%
\pgfpathlineto{\pgfqpoint{3.921546in}{0.499444in}}%
\pgfpathlineto{\pgfqpoint{3.921546in}{0.937200in}}%
\pgfpathlineto{\pgfqpoint{3.860160in}{0.937200in}}%
\pgfpathlineto{\pgfqpoint{3.860160in}{0.499444in}}%
\pgfpathclose%
\pgfusepath{fill}%
\end{pgfscope}%
\begin{pgfscope}%
\pgfpathrectangle{\pgfqpoint{0.445556in}{0.499444in}}{\pgfqpoint{3.875000in}{1.155000in}}%
\pgfusepath{clip}%
\pgfsetbuttcap%
\pgfsetmiterjoin%
\definecolor{currentfill}{rgb}{0.000000,0.000000,0.000000}%
\pgfsetfillcolor{currentfill}%
\pgfsetlinewidth{0.000000pt}%
\definecolor{currentstroke}{rgb}{0.000000,0.000000,0.000000}%
\pgfsetstrokecolor{currentstroke}%
\pgfsetstrokeopacity{0.000000}%
\pgfsetdash{}{0pt}%
\pgfpathmoveto{\pgfqpoint{4.013625in}{0.499444in}}%
\pgfpathlineto{\pgfqpoint{4.075011in}{0.499444in}}%
\pgfpathlineto{\pgfqpoint{4.075011in}{0.945112in}}%
\pgfpathlineto{\pgfqpoint{4.013625in}{0.945112in}}%
\pgfpathlineto{\pgfqpoint{4.013625in}{0.499444in}}%
\pgfpathclose%
\pgfusepath{fill}%
\end{pgfscope}%
\begin{pgfscope}%
\pgfpathrectangle{\pgfqpoint{0.445556in}{0.499444in}}{\pgfqpoint{3.875000in}{1.155000in}}%
\pgfusepath{clip}%
\pgfsetbuttcap%
\pgfsetmiterjoin%
\definecolor{currentfill}{rgb}{0.000000,0.000000,0.000000}%
\pgfsetfillcolor{currentfill}%
\pgfsetlinewidth{0.000000pt}%
\definecolor{currentstroke}{rgb}{0.000000,0.000000,0.000000}%
\pgfsetstrokecolor{currentstroke}%
\pgfsetstrokeopacity{0.000000}%
\pgfsetdash{}{0pt}%
\pgfpathmoveto{\pgfqpoint{4.167090in}{0.499444in}}%
\pgfpathlineto{\pgfqpoint{4.228476in}{0.499444in}}%
\pgfpathlineto{\pgfqpoint{4.228476in}{0.863985in}}%
\pgfpathlineto{\pgfqpoint{4.167090in}{0.863985in}}%
\pgfpathlineto{\pgfqpoint{4.167090in}{0.499444in}}%
\pgfpathclose%
\pgfusepath{fill}%
\end{pgfscope}%
\begin{pgfscope}%
\pgfsetbuttcap%
\pgfsetroundjoin%
\definecolor{currentfill}{rgb}{0.000000,0.000000,0.000000}%
\pgfsetfillcolor{currentfill}%
\pgfsetlinewidth{0.803000pt}%
\definecolor{currentstroke}{rgb}{0.000000,0.000000,0.000000}%
\pgfsetstrokecolor{currentstroke}%
\pgfsetdash{}{0pt}%
\pgfsys@defobject{currentmarker}{\pgfqpoint{0.000000in}{-0.048611in}}{\pgfqpoint{0.000000in}{0.000000in}}{%
\pgfpathmoveto{\pgfqpoint{0.000000in}{0.000000in}}%
\pgfpathlineto{\pgfqpoint{0.000000in}{-0.048611in}}%
\pgfusepath{stroke,fill}%
}%
\begin{pgfscope}%
\pgfsys@transformshift{0.483922in}{0.499444in}%
\pgfsys@useobject{currentmarker}{}%
\end{pgfscope}%
\end{pgfscope}%
\begin{pgfscope}%
\definecolor{textcolor}{rgb}{0.000000,0.000000,0.000000}%
\pgfsetstrokecolor{textcolor}%
\pgfsetfillcolor{textcolor}%
\pgftext[x=0.483922in,y=0.402222in,,top]{\color{textcolor}\rmfamily\fontsize{10.000000}{12.000000}\selectfont 0.0}%
\end{pgfscope}%
\begin{pgfscope}%
\pgfsetbuttcap%
\pgfsetroundjoin%
\definecolor{currentfill}{rgb}{0.000000,0.000000,0.000000}%
\pgfsetfillcolor{currentfill}%
\pgfsetlinewidth{0.803000pt}%
\definecolor{currentstroke}{rgb}{0.000000,0.000000,0.000000}%
\pgfsetstrokecolor{currentstroke}%
\pgfsetdash{}{0pt}%
\pgfsys@defobject{currentmarker}{\pgfqpoint{0.000000in}{-0.048611in}}{\pgfqpoint{0.000000in}{0.000000in}}{%
\pgfpathmoveto{\pgfqpoint{0.000000in}{0.000000in}}%
\pgfpathlineto{\pgfqpoint{0.000000in}{-0.048611in}}%
\pgfusepath{stroke,fill}%
}%
\begin{pgfscope}%
\pgfsys@transformshift{0.867585in}{0.499444in}%
\pgfsys@useobject{currentmarker}{}%
\end{pgfscope}%
\end{pgfscope}%
\begin{pgfscope}%
\definecolor{textcolor}{rgb}{0.000000,0.000000,0.000000}%
\pgfsetstrokecolor{textcolor}%
\pgfsetfillcolor{textcolor}%
\pgftext[x=0.867585in,y=0.402222in,,top]{\color{textcolor}\rmfamily\fontsize{10.000000}{12.000000}\selectfont 0.1}%
\end{pgfscope}%
\begin{pgfscope}%
\pgfsetbuttcap%
\pgfsetroundjoin%
\definecolor{currentfill}{rgb}{0.000000,0.000000,0.000000}%
\pgfsetfillcolor{currentfill}%
\pgfsetlinewidth{0.803000pt}%
\definecolor{currentstroke}{rgb}{0.000000,0.000000,0.000000}%
\pgfsetstrokecolor{currentstroke}%
\pgfsetdash{}{0pt}%
\pgfsys@defobject{currentmarker}{\pgfqpoint{0.000000in}{-0.048611in}}{\pgfqpoint{0.000000in}{0.000000in}}{%
\pgfpathmoveto{\pgfqpoint{0.000000in}{0.000000in}}%
\pgfpathlineto{\pgfqpoint{0.000000in}{-0.048611in}}%
\pgfusepath{stroke,fill}%
}%
\begin{pgfscope}%
\pgfsys@transformshift{1.251249in}{0.499444in}%
\pgfsys@useobject{currentmarker}{}%
\end{pgfscope}%
\end{pgfscope}%
\begin{pgfscope}%
\definecolor{textcolor}{rgb}{0.000000,0.000000,0.000000}%
\pgfsetstrokecolor{textcolor}%
\pgfsetfillcolor{textcolor}%
\pgftext[x=1.251249in,y=0.402222in,,top]{\color{textcolor}\rmfamily\fontsize{10.000000}{12.000000}\selectfont 0.2}%
\end{pgfscope}%
\begin{pgfscope}%
\pgfsetbuttcap%
\pgfsetroundjoin%
\definecolor{currentfill}{rgb}{0.000000,0.000000,0.000000}%
\pgfsetfillcolor{currentfill}%
\pgfsetlinewidth{0.803000pt}%
\definecolor{currentstroke}{rgb}{0.000000,0.000000,0.000000}%
\pgfsetstrokecolor{currentstroke}%
\pgfsetdash{}{0pt}%
\pgfsys@defobject{currentmarker}{\pgfqpoint{0.000000in}{-0.048611in}}{\pgfqpoint{0.000000in}{0.000000in}}{%
\pgfpathmoveto{\pgfqpoint{0.000000in}{0.000000in}}%
\pgfpathlineto{\pgfqpoint{0.000000in}{-0.048611in}}%
\pgfusepath{stroke,fill}%
}%
\begin{pgfscope}%
\pgfsys@transformshift{1.634912in}{0.499444in}%
\pgfsys@useobject{currentmarker}{}%
\end{pgfscope}%
\end{pgfscope}%
\begin{pgfscope}%
\definecolor{textcolor}{rgb}{0.000000,0.000000,0.000000}%
\pgfsetstrokecolor{textcolor}%
\pgfsetfillcolor{textcolor}%
\pgftext[x=1.634912in,y=0.402222in,,top]{\color{textcolor}\rmfamily\fontsize{10.000000}{12.000000}\selectfont 0.3}%
\end{pgfscope}%
\begin{pgfscope}%
\pgfsetbuttcap%
\pgfsetroundjoin%
\definecolor{currentfill}{rgb}{0.000000,0.000000,0.000000}%
\pgfsetfillcolor{currentfill}%
\pgfsetlinewidth{0.803000pt}%
\definecolor{currentstroke}{rgb}{0.000000,0.000000,0.000000}%
\pgfsetstrokecolor{currentstroke}%
\pgfsetdash{}{0pt}%
\pgfsys@defobject{currentmarker}{\pgfqpoint{0.000000in}{-0.048611in}}{\pgfqpoint{0.000000in}{0.000000in}}{%
\pgfpathmoveto{\pgfqpoint{0.000000in}{0.000000in}}%
\pgfpathlineto{\pgfqpoint{0.000000in}{-0.048611in}}%
\pgfusepath{stroke,fill}%
}%
\begin{pgfscope}%
\pgfsys@transformshift{2.018575in}{0.499444in}%
\pgfsys@useobject{currentmarker}{}%
\end{pgfscope}%
\end{pgfscope}%
\begin{pgfscope}%
\definecolor{textcolor}{rgb}{0.000000,0.000000,0.000000}%
\pgfsetstrokecolor{textcolor}%
\pgfsetfillcolor{textcolor}%
\pgftext[x=2.018575in,y=0.402222in,,top]{\color{textcolor}\rmfamily\fontsize{10.000000}{12.000000}\selectfont 0.4}%
\end{pgfscope}%
\begin{pgfscope}%
\pgfsetbuttcap%
\pgfsetroundjoin%
\definecolor{currentfill}{rgb}{0.000000,0.000000,0.000000}%
\pgfsetfillcolor{currentfill}%
\pgfsetlinewidth{0.803000pt}%
\definecolor{currentstroke}{rgb}{0.000000,0.000000,0.000000}%
\pgfsetstrokecolor{currentstroke}%
\pgfsetdash{}{0pt}%
\pgfsys@defobject{currentmarker}{\pgfqpoint{0.000000in}{-0.048611in}}{\pgfqpoint{0.000000in}{0.000000in}}{%
\pgfpathmoveto{\pgfqpoint{0.000000in}{0.000000in}}%
\pgfpathlineto{\pgfqpoint{0.000000in}{-0.048611in}}%
\pgfusepath{stroke,fill}%
}%
\begin{pgfscope}%
\pgfsys@transformshift{2.402239in}{0.499444in}%
\pgfsys@useobject{currentmarker}{}%
\end{pgfscope}%
\end{pgfscope}%
\begin{pgfscope}%
\definecolor{textcolor}{rgb}{0.000000,0.000000,0.000000}%
\pgfsetstrokecolor{textcolor}%
\pgfsetfillcolor{textcolor}%
\pgftext[x=2.402239in,y=0.402222in,,top]{\color{textcolor}\rmfamily\fontsize{10.000000}{12.000000}\selectfont 0.5}%
\end{pgfscope}%
\begin{pgfscope}%
\pgfsetbuttcap%
\pgfsetroundjoin%
\definecolor{currentfill}{rgb}{0.000000,0.000000,0.000000}%
\pgfsetfillcolor{currentfill}%
\pgfsetlinewidth{0.803000pt}%
\definecolor{currentstroke}{rgb}{0.000000,0.000000,0.000000}%
\pgfsetstrokecolor{currentstroke}%
\pgfsetdash{}{0pt}%
\pgfsys@defobject{currentmarker}{\pgfqpoint{0.000000in}{-0.048611in}}{\pgfqpoint{0.000000in}{0.000000in}}{%
\pgfpathmoveto{\pgfqpoint{0.000000in}{0.000000in}}%
\pgfpathlineto{\pgfqpoint{0.000000in}{-0.048611in}}%
\pgfusepath{stroke,fill}%
}%
\begin{pgfscope}%
\pgfsys@transformshift{2.785902in}{0.499444in}%
\pgfsys@useobject{currentmarker}{}%
\end{pgfscope}%
\end{pgfscope}%
\begin{pgfscope}%
\definecolor{textcolor}{rgb}{0.000000,0.000000,0.000000}%
\pgfsetstrokecolor{textcolor}%
\pgfsetfillcolor{textcolor}%
\pgftext[x=2.785902in,y=0.402222in,,top]{\color{textcolor}\rmfamily\fontsize{10.000000}{12.000000}\selectfont 0.6}%
\end{pgfscope}%
\begin{pgfscope}%
\pgfsetbuttcap%
\pgfsetroundjoin%
\definecolor{currentfill}{rgb}{0.000000,0.000000,0.000000}%
\pgfsetfillcolor{currentfill}%
\pgfsetlinewidth{0.803000pt}%
\definecolor{currentstroke}{rgb}{0.000000,0.000000,0.000000}%
\pgfsetstrokecolor{currentstroke}%
\pgfsetdash{}{0pt}%
\pgfsys@defobject{currentmarker}{\pgfqpoint{0.000000in}{-0.048611in}}{\pgfqpoint{0.000000in}{0.000000in}}{%
\pgfpathmoveto{\pgfqpoint{0.000000in}{0.000000in}}%
\pgfpathlineto{\pgfqpoint{0.000000in}{-0.048611in}}%
\pgfusepath{stroke,fill}%
}%
\begin{pgfscope}%
\pgfsys@transformshift{3.169566in}{0.499444in}%
\pgfsys@useobject{currentmarker}{}%
\end{pgfscope}%
\end{pgfscope}%
\begin{pgfscope}%
\definecolor{textcolor}{rgb}{0.000000,0.000000,0.000000}%
\pgfsetstrokecolor{textcolor}%
\pgfsetfillcolor{textcolor}%
\pgftext[x=3.169566in,y=0.402222in,,top]{\color{textcolor}\rmfamily\fontsize{10.000000}{12.000000}\selectfont 0.7}%
\end{pgfscope}%
\begin{pgfscope}%
\pgfsetbuttcap%
\pgfsetroundjoin%
\definecolor{currentfill}{rgb}{0.000000,0.000000,0.000000}%
\pgfsetfillcolor{currentfill}%
\pgfsetlinewidth{0.803000pt}%
\definecolor{currentstroke}{rgb}{0.000000,0.000000,0.000000}%
\pgfsetstrokecolor{currentstroke}%
\pgfsetdash{}{0pt}%
\pgfsys@defobject{currentmarker}{\pgfqpoint{0.000000in}{-0.048611in}}{\pgfqpoint{0.000000in}{0.000000in}}{%
\pgfpathmoveto{\pgfqpoint{0.000000in}{0.000000in}}%
\pgfpathlineto{\pgfqpoint{0.000000in}{-0.048611in}}%
\pgfusepath{stroke,fill}%
}%
\begin{pgfscope}%
\pgfsys@transformshift{3.553229in}{0.499444in}%
\pgfsys@useobject{currentmarker}{}%
\end{pgfscope}%
\end{pgfscope}%
\begin{pgfscope}%
\definecolor{textcolor}{rgb}{0.000000,0.000000,0.000000}%
\pgfsetstrokecolor{textcolor}%
\pgfsetfillcolor{textcolor}%
\pgftext[x=3.553229in,y=0.402222in,,top]{\color{textcolor}\rmfamily\fontsize{10.000000}{12.000000}\selectfont 0.8}%
\end{pgfscope}%
\begin{pgfscope}%
\pgfsetbuttcap%
\pgfsetroundjoin%
\definecolor{currentfill}{rgb}{0.000000,0.000000,0.000000}%
\pgfsetfillcolor{currentfill}%
\pgfsetlinewidth{0.803000pt}%
\definecolor{currentstroke}{rgb}{0.000000,0.000000,0.000000}%
\pgfsetstrokecolor{currentstroke}%
\pgfsetdash{}{0pt}%
\pgfsys@defobject{currentmarker}{\pgfqpoint{0.000000in}{-0.048611in}}{\pgfqpoint{0.000000in}{0.000000in}}{%
\pgfpathmoveto{\pgfqpoint{0.000000in}{0.000000in}}%
\pgfpathlineto{\pgfqpoint{0.000000in}{-0.048611in}}%
\pgfusepath{stroke,fill}%
}%
\begin{pgfscope}%
\pgfsys@transformshift{3.936892in}{0.499444in}%
\pgfsys@useobject{currentmarker}{}%
\end{pgfscope}%
\end{pgfscope}%
\begin{pgfscope}%
\definecolor{textcolor}{rgb}{0.000000,0.000000,0.000000}%
\pgfsetstrokecolor{textcolor}%
\pgfsetfillcolor{textcolor}%
\pgftext[x=3.936892in,y=0.402222in,,top]{\color{textcolor}\rmfamily\fontsize{10.000000}{12.000000}\selectfont 0.9}%
\end{pgfscope}%
\begin{pgfscope}%
\pgfsetbuttcap%
\pgfsetroundjoin%
\definecolor{currentfill}{rgb}{0.000000,0.000000,0.000000}%
\pgfsetfillcolor{currentfill}%
\pgfsetlinewidth{0.803000pt}%
\definecolor{currentstroke}{rgb}{0.000000,0.000000,0.000000}%
\pgfsetstrokecolor{currentstroke}%
\pgfsetdash{}{0pt}%
\pgfsys@defobject{currentmarker}{\pgfqpoint{0.000000in}{-0.048611in}}{\pgfqpoint{0.000000in}{0.000000in}}{%
\pgfpathmoveto{\pgfqpoint{0.000000in}{0.000000in}}%
\pgfpathlineto{\pgfqpoint{0.000000in}{-0.048611in}}%
\pgfusepath{stroke,fill}%
}%
\begin{pgfscope}%
\pgfsys@transformshift{4.320556in}{0.499444in}%
\pgfsys@useobject{currentmarker}{}%
\end{pgfscope}%
\end{pgfscope}%
\begin{pgfscope}%
\definecolor{textcolor}{rgb}{0.000000,0.000000,0.000000}%
\pgfsetstrokecolor{textcolor}%
\pgfsetfillcolor{textcolor}%
\pgftext[x=4.320556in,y=0.402222in,,top]{\color{textcolor}\rmfamily\fontsize{10.000000}{12.000000}\selectfont 1.0}%
\end{pgfscope}%
\begin{pgfscope}%
\definecolor{textcolor}{rgb}{0.000000,0.000000,0.000000}%
\pgfsetstrokecolor{textcolor}%
\pgfsetfillcolor{textcolor}%
\pgftext[x=2.383056in,y=0.223333in,,top]{\color{textcolor}\rmfamily\fontsize{10.000000}{12.000000}\selectfont \(\displaystyle p\)}%
\end{pgfscope}%
\begin{pgfscope}%
\pgfsetbuttcap%
\pgfsetroundjoin%
\definecolor{currentfill}{rgb}{0.000000,0.000000,0.000000}%
\pgfsetfillcolor{currentfill}%
\pgfsetlinewidth{0.803000pt}%
\definecolor{currentstroke}{rgb}{0.000000,0.000000,0.000000}%
\pgfsetstrokecolor{currentstroke}%
\pgfsetdash{}{0pt}%
\pgfsys@defobject{currentmarker}{\pgfqpoint{-0.048611in}{0.000000in}}{\pgfqpoint{-0.000000in}{0.000000in}}{%
\pgfpathmoveto{\pgfqpoint{-0.000000in}{0.000000in}}%
\pgfpathlineto{\pgfqpoint{-0.048611in}{0.000000in}}%
\pgfusepath{stroke,fill}%
}%
\begin{pgfscope}%
\pgfsys@transformshift{0.445556in}{0.499444in}%
\pgfsys@useobject{currentmarker}{}%
\end{pgfscope}%
\end{pgfscope}%
\begin{pgfscope}%
\definecolor{textcolor}{rgb}{0.000000,0.000000,0.000000}%
\pgfsetstrokecolor{textcolor}%
\pgfsetfillcolor{textcolor}%
\pgftext[x=0.278889in, y=0.451250in, left, base]{\color{textcolor}\rmfamily\fontsize{10.000000}{12.000000}\selectfont \(\displaystyle {0}\)}%
\end{pgfscope}%
\begin{pgfscope}%
\pgfsetbuttcap%
\pgfsetroundjoin%
\definecolor{currentfill}{rgb}{0.000000,0.000000,0.000000}%
\pgfsetfillcolor{currentfill}%
\pgfsetlinewidth{0.803000pt}%
\definecolor{currentstroke}{rgb}{0.000000,0.000000,0.000000}%
\pgfsetstrokecolor{currentstroke}%
\pgfsetdash{}{0pt}%
\pgfsys@defobject{currentmarker}{\pgfqpoint{-0.048611in}{0.000000in}}{\pgfqpoint{-0.000000in}{0.000000in}}{%
\pgfpathmoveto{\pgfqpoint{-0.000000in}{0.000000in}}%
\pgfpathlineto{\pgfqpoint{-0.048611in}{0.000000in}}%
\pgfusepath{stroke,fill}%
}%
\begin{pgfscope}%
\pgfsys@transformshift{0.445556in}{1.005031in}%
\pgfsys@useobject{currentmarker}{}%
\end{pgfscope}%
\end{pgfscope}%
\begin{pgfscope}%
\definecolor{textcolor}{rgb}{0.000000,0.000000,0.000000}%
\pgfsetstrokecolor{textcolor}%
\pgfsetfillcolor{textcolor}%
\pgftext[x=0.278889in, y=0.956836in, left, base]{\color{textcolor}\rmfamily\fontsize{10.000000}{12.000000}\selectfont \(\displaystyle {2}\)}%
\end{pgfscope}%
\begin{pgfscope}%
\pgfsetbuttcap%
\pgfsetroundjoin%
\definecolor{currentfill}{rgb}{0.000000,0.000000,0.000000}%
\pgfsetfillcolor{currentfill}%
\pgfsetlinewidth{0.803000pt}%
\definecolor{currentstroke}{rgb}{0.000000,0.000000,0.000000}%
\pgfsetstrokecolor{currentstroke}%
\pgfsetdash{}{0pt}%
\pgfsys@defobject{currentmarker}{\pgfqpoint{-0.048611in}{0.000000in}}{\pgfqpoint{-0.000000in}{0.000000in}}{%
\pgfpathmoveto{\pgfqpoint{-0.000000in}{0.000000in}}%
\pgfpathlineto{\pgfqpoint{-0.048611in}{0.000000in}}%
\pgfusepath{stroke,fill}%
}%
\begin{pgfscope}%
\pgfsys@transformshift{0.445556in}{1.510618in}%
\pgfsys@useobject{currentmarker}{}%
\end{pgfscope}%
\end{pgfscope}%
\begin{pgfscope}%
\definecolor{textcolor}{rgb}{0.000000,0.000000,0.000000}%
\pgfsetstrokecolor{textcolor}%
\pgfsetfillcolor{textcolor}%
\pgftext[x=0.278889in, y=1.462423in, left, base]{\color{textcolor}\rmfamily\fontsize{10.000000}{12.000000}\selectfont \(\displaystyle {4}\)}%
\end{pgfscope}%
\begin{pgfscope}%
\definecolor{textcolor}{rgb}{0.000000,0.000000,0.000000}%
\pgfsetstrokecolor{textcolor}%
\pgfsetfillcolor{textcolor}%
\pgftext[x=0.223333in,y=1.076944in,,bottom,rotate=90.000000]{\color{textcolor}\rmfamily\fontsize{10.000000}{12.000000}\selectfont Percent of Data Set}%
\end{pgfscope}%
\begin{pgfscope}%
\pgfsetrectcap%
\pgfsetmiterjoin%
\pgfsetlinewidth{0.803000pt}%
\definecolor{currentstroke}{rgb}{0.000000,0.000000,0.000000}%
\pgfsetstrokecolor{currentstroke}%
\pgfsetdash{}{0pt}%
\pgfpathmoveto{\pgfqpoint{0.445556in}{0.499444in}}%
\pgfpathlineto{\pgfqpoint{0.445556in}{1.654444in}}%
\pgfusepath{stroke}%
\end{pgfscope}%
\begin{pgfscope}%
\pgfsetrectcap%
\pgfsetmiterjoin%
\pgfsetlinewidth{0.803000pt}%
\definecolor{currentstroke}{rgb}{0.000000,0.000000,0.000000}%
\pgfsetstrokecolor{currentstroke}%
\pgfsetdash{}{0pt}%
\pgfpathmoveto{\pgfqpoint{4.320556in}{0.499444in}}%
\pgfpathlineto{\pgfqpoint{4.320556in}{1.654444in}}%
\pgfusepath{stroke}%
\end{pgfscope}%
\begin{pgfscope}%
\pgfsetrectcap%
\pgfsetmiterjoin%
\pgfsetlinewidth{0.803000pt}%
\definecolor{currentstroke}{rgb}{0.000000,0.000000,0.000000}%
\pgfsetstrokecolor{currentstroke}%
\pgfsetdash{}{0pt}%
\pgfpathmoveto{\pgfqpoint{0.445556in}{0.499444in}}%
\pgfpathlineto{\pgfqpoint{4.320556in}{0.499444in}}%
\pgfusepath{stroke}%
\end{pgfscope}%
\begin{pgfscope}%
\pgfsetrectcap%
\pgfsetmiterjoin%
\pgfsetlinewidth{0.803000pt}%
\definecolor{currentstroke}{rgb}{0.000000,0.000000,0.000000}%
\pgfsetstrokecolor{currentstroke}%
\pgfsetdash{}{0pt}%
\pgfpathmoveto{\pgfqpoint{0.445556in}{1.654444in}}%
\pgfpathlineto{\pgfqpoint{4.320556in}{1.654444in}}%
\pgfusepath{stroke}%
\end{pgfscope}%
\begin{pgfscope}%
\pgfsetbuttcap%
\pgfsetmiterjoin%
\definecolor{currentfill}{rgb}{1.000000,1.000000,1.000000}%
\pgfsetfillcolor{currentfill}%
\pgfsetfillopacity{0.800000}%
\pgfsetlinewidth{1.003750pt}%
\definecolor{currentstroke}{rgb}{0.800000,0.800000,0.800000}%
\pgfsetstrokecolor{currentstroke}%
\pgfsetstrokeopacity{0.800000}%
\pgfsetdash{}{0pt}%
\pgfpathmoveto{\pgfqpoint{3.543611in}{1.154445in}}%
\pgfpathlineto{\pgfqpoint{4.223333in}{1.154445in}}%
\pgfpathquadraticcurveto{\pgfqpoint{4.251111in}{1.154445in}}{\pgfqpoint{4.251111in}{1.182222in}}%
\pgfpathlineto{\pgfqpoint{4.251111in}{1.557222in}}%
\pgfpathquadraticcurveto{\pgfqpoint{4.251111in}{1.585000in}}{\pgfqpoint{4.223333in}{1.585000in}}%
\pgfpathlineto{\pgfqpoint{3.543611in}{1.585000in}}%
\pgfpathquadraticcurveto{\pgfqpoint{3.515833in}{1.585000in}}{\pgfqpoint{3.515833in}{1.557222in}}%
\pgfpathlineto{\pgfqpoint{3.515833in}{1.182222in}}%
\pgfpathquadraticcurveto{\pgfqpoint{3.515833in}{1.154445in}}{\pgfqpoint{3.543611in}{1.154445in}}%
\pgfpathlineto{\pgfqpoint{3.543611in}{1.154445in}}%
\pgfpathclose%
\pgfusepath{stroke,fill}%
\end{pgfscope}%
\begin{pgfscope}%
\pgfsetbuttcap%
\pgfsetmiterjoin%
\pgfsetlinewidth{1.003750pt}%
\definecolor{currentstroke}{rgb}{0.000000,0.000000,0.000000}%
\pgfsetstrokecolor{currentstroke}%
\pgfsetdash{}{0pt}%
\pgfpathmoveto{\pgfqpoint{3.571389in}{1.432222in}}%
\pgfpathlineto{\pgfqpoint{3.849167in}{1.432222in}}%
\pgfpathlineto{\pgfqpoint{3.849167in}{1.529444in}}%
\pgfpathlineto{\pgfqpoint{3.571389in}{1.529444in}}%
\pgfpathlineto{\pgfqpoint{3.571389in}{1.432222in}}%
\pgfpathclose%
\pgfusepath{stroke}%
\end{pgfscope}%
\begin{pgfscope}%
\definecolor{textcolor}{rgb}{0.000000,0.000000,0.000000}%
\pgfsetstrokecolor{textcolor}%
\pgfsetfillcolor{textcolor}%
\pgftext[x=3.960278in,y=1.432222in,left,base]{\color{textcolor}\rmfamily\fontsize{10.000000}{12.000000}\selectfont Neg}%
\end{pgfscope}%
\begin{pgfscope}%
\pgfsetbuttcap%
\pgfsetmiterjoin%
\definecolor{currentfill}{rgb}{0.000000,0.000000,0.000000}%
\pgfsetfillcolor{currentfill}%
\pgfsetlinewidth{0.000000pt}%
\definecolor{currentstroke}{rgb}{0.000000,0.000000,0.000000}%
\pgfsetstrokecolor{currentstroke}%
\pgfsetstrokeopacity{0.000000}%
\pgfsetdash{}{0pt}%
\pgfpathmoveto{\pgfqpoint{3.571389in}{1.236944in}}%
\pgfpathlineto{\pgfqpoint{3.849167in}{1.236944in}}%
\pgfpathlineto{\pgfqpoint{3.849167in}{1.334167in}}%
\pgfpathlineto{\pgfqpoint{3.571389in}{1.334167in}}%
\pgfpathlineto{\pgfqpoint{3.571389in}{1.236944in}}%
\pgfpathclose%
\pgfusepath{fill}%
\end{pgfscope}%
\begin{pgfscope}%
\definecolor{textcolor}{rgb}{0.000000,0.000000,0.000000}%
\pgfsetstrokecolor{textcolor}%
\pgfsetfillcolor{textcolor}%
\pgftext[x=3.960278in,y=1.236944in,left,base]{\color{textcolor}\rmfamily\fontsize{10.000000}{12.000000}\selectfont Pos}%
\end{pgfscope}%
\end{pgfpicture}%
\makeatother%
\endgroup%
	
&
	\vskip 0pt
	\hfil ROC Curve
	
	%% Creator: Matplotlib, PGF backend
%%
%% To include the figure in your LaTeX document, write
%%   \input{<filename>.pgf}
%%
%% Make sure the required packages are loaded in your preamble
%%   \usepackage{pgf}
%%
%% Also ensure that all the required font packages are loaded; for instance,
%% the lmodern package is sometimes necessary when using math font.
%%   \usepackage{lmodern}
%%
%% Figures using additional raster images can only be included by \input if
%% they are in the same directory as the main LaTeX file. For loading figures
%% from other directories you can use the `import` package
%%   \usepackage{import}
%%
%% and then include the figures with
%%   \import{<path to file>}{<filename>.pgf}
%%
%% Matplotlib used the following preamble
%%   
%%   \usepackage{fontspec}
%%   \makeatletter\@ifpackageloaded{underscore}{}{\usepackage[strings]{underscore}}\makeatother
%%
\begingroup%
\makeatletter%
\begin{pgfpicture}%
\pgfpathrectangle{\pgfpointorigin}{\pgfqpoint{2.221861in}{1.754444in}}%
\pgfusepath{use as bounding box, clip}%
\begin{pgfscope}%
\pgfsetbuttcap%
\pgfsetmiterjoin%
\definecolor{currentfill}{rgb}{1.000000,1.000000,1.000000}%
\pgfsetfillcolor{currentfill}%
\pgfsetlinewidth{0.000000pt}%
\definecolor{currentstroke}{rgb}{1.000000,1.000000,1.000000}%
\pgfsetstrokecolor{currentstroke}%
\pgfsetdash{}{0pt}%
\pgfpathmoveto{\pgfqpoint{0.000000in}{0.000000in}}%
\pgfpathlineto{\pgfqpoint{2.221861in}{0.000000in}}%
\pgfpathlineto{\pgfqpoint{2.221861in}{1.754444in}}%
\pgfpathlineto{\pgfqpoint{0.000000in}{1.754444in}}%
\pgfpathlineto{\pgfqpoint{0.000000in}{0.000000in}}%
\pgfpathclose%
\pgfusepath{fill}%
\end{pgfscope}%
\begin{pgfscope}%
\pgfsetbuttcap%
\pgfsetmiterjoin%
\definecolor{currentfill}{rgb}{1.000000,1.000000,1.000000}%
\pgfsetfillcolor{currentfill}%
\pgfsetlinewidth{0.000000pt}%
\definecolor{currentstroke}{rgb}{0.000000,0.000000,0.000000}%
\pgfsetstrokecolor{currentstroke}%
\pgfsetstrokeopacity{0.000000}%
\pgfsetdash{}{0pt}%
\pgfpathmoveto{\pgfqpoint{0.553581in}{0.499444in}}%
\pgfpathlineto{\pgfqpoint{2.103581in}{0.499444in}}%
\pgfpathlineto{\pgfqpoint{2.103581in}{1.654444in}}%
\pgfpathlineto{\pgfqpoint{0.553581in}{1.654444in}}%
\pgfpathlineto{\pgfqpoint{0.553581in}{0.499444in}}%
\pgfpathclose%
\pgfusepath{fill}%
\end{pgfscope}%
\begin{pgfscope}%
\pgfsetbuttcap%
\pgfsetroundjoin%
\definecolor{currentfill}{rgb}{0.000000,0.000000,0.000000}%
\pgfsetfillcolor{currentfill}%
\pgfsetlinewidth{0.803000pt}%
\definecolor{currentstroke}{rgb}{0.000000,0.000000,0.000000}%
\pgfsetstrokecolor{currentstroke}%
\pgfsetdash{}{0pt}%
\pgfsys@defobject{currentmarker}{\pgfqpoint{0.000000in}{-0.048611in}}{\pgfqpoint{0.000000in}{0.000000in}}{%
\pgfpathmoveto{\pgfqpoint{0.000000in}{0.000000in}}%
\pgfpathlineto{\pgfqpoint{0.000000in}{-0.048611in}}%
\pgfusepath{stroke,fill}%
}%
\begin{pgfscope}%
\pgfsys@transformshift{0.624035in}{0.499444in}%
\pgfsys@useobject{currentmarker}{}%
\end{pgfscope}%
\end{pgfscope}%
\begin{pgfscope}%
\definecolor{textcolor}{rgb}{0.000000,0.000000,0.000000}%
\pgfsetstrokecolor{textcolor}%
\pgfsetfillcolor{textcolor}%
\pgftext[x=0.624035in,y=0.402222in,,top]{\color{textcolor}\rmfamily\fontsize{10.000000}{12.000000}\selectfont \(\displaystyle {0.0}\)}%
\end{pgfscope}%
\begin{pgfscope}%
\pgfsetbuttcap%
\pgfsetroundjoin%
\definecolor{currentfill}{rgb}{0.000000,0.000000,0.000000}%
\pgfsetfillcolor{currentfill}%
\pgfsetlinewidth{0.803000pt}%
\definecolor{currentstroke}{rgb}{0.000000,0.000000,0.000000}%
\pgfsetstrokecolor{currentstroke}%
\pgfsetdash{}{0pt}%
\pgfsys@defobject{currentmarker}{\pgfqpoint{0.000000in}{-0.048611in}}{\pgfqpoint{0.000000in}{0.000000in}}{%
\pgfpathmoveto{\pgfqpoint{0.000000in}{0.000000in}}%
\pgfpathlineto{\pgfqpoint{0.000000in}{-0.048611in}}%
\pgfusepath{stroke,fill}%
}%
\begin{pgfscope}%
\pgfsys@transformshift{1.328581in}{0.499444in}%
\pgfsys@useobject{currentmarker}{}%
\end{pgfscope}%
\end{pgfscope}%
\begin{pgfscope}%
\definecolor{textcolor}{rgb}{0.000000,0.000000,0.000000}%
\pgfsetstrokecolor{textcolor}%
\pgfsetfillcolor{textcolor}%
\pgftext[x=1.328581in,y=0.402222in,,top]{\color{textcolor}\rmfamily\fontsize{10.000000}{12.000000}\selectfont \(\displaystyle {0.5}\)}%
\end{pgfscope}%
\begin{pgfscope}%
\pgfsetbuttcap%
\pgfsetroundjoin%
\definecolor{currentfill}{rgb}{0.000000,0.000000,0.000000}%
\pgfsetfillcolor{currentfill}%
\pgfsetlinewidth{0.803000pt}%
\definecolor{currentstroke}{rgb}{0.000000,0.000000,0.000000}%
\pgfsetstrokecolor{currentstroke}%
\pgfsetdash{}{0pt}%
\pgfsys@defobject{currentmarker}{\pgfqpoint{0.000000in}{-0.048611in}}{\pgfqpoint{0.000000in}{0.000000in}}{%
\pgfpathmoveto{\pgfqpoint{0.000000in}{0.000000in}}%
\pgfpathlineto{\pgfqpoint{0.000000in}{-0.048611in}}%
\pgfusepath{stroke,fill}%
}%
\begin{pgfscope}%
\pgfsys@transformshift{2.033126in}{0.499444in}%
\pgfsys@useobject{currentmarker}{}%
\end{pgfscope}%
\end{pgfscope}%
\begin{pgfscope}%
\definecolor{textcolor}{rgb}{0.000000,0.000000,0.000000}%
\pgfsetstrokecolor{textcolor}%
\pgfsetfillcolor{textcolor}%
\pgftext[x=2.033126in,y=0.402222in,,top]{\color{textcolor}\rmfamily\fontsize{10.000000}{12.000000}\selectfont \(\displaystyle {1.0}\)}%
\end{pgfscope}%
\begin{pgfscope}%
\definecolor{textcolor}{rgb}{0.000000,0.000000,0.000000}%
\pgfsetstrokecolor{textcolor}%
\pgfsetfillcolor{textcolor}%
\pgftext[x=1.328581in,y=0.223333in,,top]{\color{textcolor}\rmfamily\fontsize{10.000000}{12.000000}\selectfont False positive rate}%
\end{pgfscope}%
\begin{pgfscope}%
\pgfsetbuttcap%
\pgfsetroundjoin%
\definecolor{currentfill}{rgb}{0.000000,0.000000,0.000000}%
\pgfsetfillcolor{currentfill}%
\pgfsetlinewidth{0.803000pt}%
\definecolor{currentstroke}{rgb}{0.000000,0.000000,0.000000}%
\pgfsetstrokecolor{currentstroke}%
\pgfsetdash{}{0pt}%
\pgfsys@defobject{currentmarker}{\pgfqpoint{-0.048611in}{0.000000in}}{\pgfqpoint{-0.000000in}{0.000000in}}{%
\pgfpathmoveto{\pgfqpoint{-0.000000in}{0.000000in}}%
\pgfpathlineto{\pgfqpoint{-0.048611in}{0.000000in}}%
\pgfusepath{stroke,fill}%
}%
\begin{pgfscope}%
\pgfsys@transformshift{0.553581in}{0.551944in}%
\pgfsys@useobject{currentmarker}{}%
\end{pgfscope}%
\end{pgfscope}%
\begin{pgfscope}%
\definecolor{textcolor}{rgb}{0.000000,0.000000,0.000000}%
\pgfsetstrokecolor{textcolor}%
\pgfsetfillcolor{textcolor}%
\pgftext[x=0.278889in, y=0.503750in, left, base]{\color{textcolor}\rmfamily\fontsize{10.000000}{12.000000}\selectfont \(\displaystyle {0.0}\)}%
\end{pgfscope}%
\begin{pgfscope}%
\pgfsetbuttcap%
\pgfsetroundjoin%
\definecolor{currentfill}{rgb}{0.000000,0.000000,0.000000}%
\pgfsetfillcolor{currentfill}%
\pgfsetlinewidth{0.803000pt}%
\definecolor{currentstroke}{rgb}{0.000000,0.000000,0.000000}%
\pgfsetstrokecolor{currentstroke}%
\pgfsetdash{}{0pt}%
\pgfsys@defobject{currentmarker}{\pgfqpoint{-0.048611in}{0.000000in}}{\pgfqpoint{-0.000000in}{0.000000in}}{%
\pgfpathmoveto{\pgfqpoint{-0.000000in}{0.000000in}}%
\pgfpathlineto{\pgfqpoint{-0.048611in}{0.000000in}}%
\pgfusepath{stroke,fill}%
}%
\begin{pgfscope}%
\pgfsys@transformshift{0.553581in}{1.076944in}%
\pgfsys@useobject{currentmarker}{}%
\end{pgfscope}%
\end{pgfscope}%
\begin{pgfscope}%
\definecolor{textcolor}{rgb}{0.000000,0.000000,0.000000}%
\pgfsetstrokecolor{textcolor}%
\pgfsetfillcolor{textcolor}%
\pgftext[x=0.278889in, y=1.028750in, left, base]{\color{textcolor}\rmfamily\fontsize{10.000000}{12.000000}\selectfont \(\displaystyle {0.5}\)}%
\end{pgfscope}%
\begin{pgfscope}%
\pgfsetbuttcap%
\pgfsetroundjoin%
\definecolor{currentfill}{rgb}{0.000000,0.000000,0.000000}%
\pgfsetfillcolor{currentfill}%
\pgfsetlinewidth{0.803000pt}%
\definecolor{currentstroke}{rgb}{0.000000,0.000000,0.000000}%
\pgfsetstrokecolor{currentstroke}%
\pgfsetdash{}{0pt}%
\pgfsys@defobject{currentmarker}{\pgfqpoint{-0.048611in}{0.000000in}}{\pgfqpoint{-0.000000in}{0.000000in}}{%
\pgfpathmoveto{\pgfqpoint{-0.000000in}{0.000000in}}%
\pgfpathlineto{\pgfqpoint{-0.048611in}{0.000000in}}%
\pgfusepath{stroke,fill}%
}%
\begin{pgfscope}%
\pgfsys@transformshift{0.553581in}{1.601944in}%
\pgfsys@useobject{currentmarker}{}%
\end{pgfscope}%
\end{pgfscope}%
\begin{pgfscope}%
\definecolor{textcolor}{rgb}{0.000000,0.000000,0.000000}%
\pgfsetstrokecolor{textcolor}%
\pgfsetfillcolor{textcolor}%
\pgftext[x=0.278889in, y=1.553750in, left, base]{\color{textcolor}\rmfamily\fontsize{10.000000}{12.000000}\selectfont \(\displaystyle {1.0}\)}%
\end{pgfscope}%
\begin{pgfscope}%
\definecolor{textcolor}{rgb}{0.000000,0.000000,0.000000}%
\pgfsetstrokecolor{textcolor}%
\pgfsetfillcolor{textcolor}%
\pgftext[x=0.223333in,y=1.076944in,,bottom,rotate=90.000000]{\color{textcolor}\rmfamily\fontsize{10.000000}{12.000000}\selectfont True positive rate}%
\end{pgfscope}%
\begin{pgfscope}%
\pgfpathrectangle{\pgfqpoint{0.553581in}{0.499444in}}{\pgfqpoint{1.550000in}{1.155000in}}%
\pgfusepath{clip}%
\pgfsetbuttcap%
\pgfsetroundjoin%
\pgfsetlinewidth{1.505625pt}%
\definecolor{currentstroke}{rgb}{0.000000,0.000000,0.000000}%
\pgfsetstrokecolor{currentstroke}%
\pgfsetdash{{5.550000pt}{2.400000pt}}{0.000000pt}%
\pgfpathmoveto{\pgfqpoint{0.624035in}{0.551944in}}%
\pgfpathlineto{\pgfqpoint{2.033126in}{1.601944in}}%
\pgfusepath{stroke}%
\end{pgfscope}%
\begin{pgfscope}%
\pgfpathrectangle{\pgfqpoint{0.553581in}{0.499444in}}{\pgfqpoint{1.550000in}{1.155000in}}%
\pgfusepath{clip}%
\pgfsetrectcap%
\pgfsetroundjoin%
\pgfsetlinewidth{1.505625pt}%
\definecolor{currentstroke}{rgb}{0.000000,0.000000,0.000000}%
\pgfsetstrokecolor{currentstroke}%
\pgfsetdash{}{0pt}%
\pgfpathmoveto{\pgfqpoint{0.624035in}{0.551944in}}%
\pgfpathlineto{\pgfqpoint{0.624098in}{0.552969in}}%
\pgfpathlineto{\pgfqpoint{0.625208in}{0.569887in}}%
\pgfpathlineto{\pgfqpoint{0.625310in}{0.570880in}}%
\pgfpathlineto{\pgfqpoint{0.625310in}{0.570973in}}%
\pgfpathlineto{\pgfqpoint{0.626420in}{0.585159in}}%
\pgfpathlineto{\pgfqpoint{0.626529in}{0.586091in}}%
\pgfpathlineto{\pgfqpoint{0.627639in}{0.597390in}}%
\pgfpathlineto{\pgfqpoint{0.627780in}{0.598383in}}%
\pgfpathlineto{\pgfqpoint{0.628882in}{0.607541in}}%
\pgfpathlineto{\pgfqpoint{0.629062in}{0.608627in}}%
\pgfpathlineto{\pgfqpoint{0.630172in}{0.617443in}}%
\pgfpathlineto{\pgfqpoint{0.630430in}{0.618530in}}%
\pgfpathlineto{\pgfqpoint{0.631540in}{0.627097in}}%
\pgfpathlineto{\pgfqpoint{0.631673in}{0.628122in}}%
\pgfpathlineto{\pgfqpoint{0.632783in}{0.636472in}}%
\pgfpathlineto{\pgfqpoint{0.632908in}{0.637403in}}%
\pgfpathlineto{\pgfqpoint{0.634011in}{0.643891in}}%
\pgfpathlineto{\pgfqpoint{0.634190in}{0.644853in}}%
\pgfpathlineto{\pgfqpoint{0.635277in}{0.651683in}}%
\pgfpathlineto{\pgfqpoint{0.635293in}{0.651683in}}%
\pgfpathlineto{\pgfqpoint{0.635480in}{0.652707in}}%
\pgfpathlineto{\pgfqpoint{0.636590in}{0.659040in}}%
\pgfpathlineto{\pgfqpoint{0.636825in}{0.660064in}}%
\pgfpathlineto{\pgfqpoint{0.637927in}{0.665962in}}%
\pgfpathlineto{\pgfqpoint{0.638201in}{0.667017in}}%
\pgfpathlineto{\pgfqpoint{0.639311in}{0.672419in}}%
\pgfpathlineto{\pgfqpoint{0.639553in}{0.673505in}}%
\pgfpathlineto{\pgfqpoint{0.640663in}{0.678782in}}%
\pgfpathlineto{\pgfqpoint{0.640976in}{0.679838in}}%
\pgfpathlineto{\pgfqpoint{0.642086in}{0.684339in}}%
\pgfpathlineto{\pgfqpoint{0.642297in}{0.685425in}}%
\pgfpathlineto{\pgfqpoint{0.643407in}{0.691510in}}%
\pgfpathlineto{\pgfqpoint{0.643720in}{0.692596in}}%
\pgfpathlineto{\pgfqpoint{0.644830in}{0.697687in}}%
\pgfpathlineto{\pgfqpoint{0.645026in}{0.698742in}}%
\pgfpathlineto{\pgfqpoint{0.646104in}{0.703616in}}%
\pgfpathlineto{\pgfqpoint{0.646542in}{0.704640in}}%
\pgfpathlineto{\pgfqpoint{0.647652in}{0.708862in}}%
\pgfpathlineto{\pgfqpoint{0.647871in}{0.709949in}}%
\pgfpathlineto{\pgfqpoint{0.648966in}{0.714574in}}%
\pgfpathlineto{\pgfqpoint{0.649200in}{0.715660in}}%
\pgfpathlineto{\pgfqpoint{0.650310in}{0.721403in}}%
\pgfpathlineto{\pgfqpoint{0.650654in}{0.722490in}}%
\pgfpathlineto{\pgfqpoint{0.651741in}{0.726463in}}%
\pgfpathlineto{\pgfqpoint{0.652069in}{0.727549in}}%
\pgfpathlineto{\pgfqpoint{0.653172in}{0.732423in}}%
\pgfpathlineto{\pgfqpoint{0.653453in}{0.733510in}}%
\pgfpathlineto{\pgfqpoint{0.654555in}{0.738073in}}%
\pgfpathlineto{\pgfqpoint{0.654845in}{0.739159in}}%
\pgfpathlineto{\pgfqpoint{0.655955in}{0.743008in}}%
\pgfpathlineto{\pgfqpoint{0.656338in}{0.744095in}}%
\pgfpathlineto{\pgfqpoint{0.657448in}{0.748037in}}%
\pgfpathlineto{\pgfqpoint{0.657792in}{0.749124in}}%
\pgfpathlineto{\pgfqpoint{0.658894in}{0.752694in}}%
\pgfpathlineto{\pgfqpoint{0.659222in}{0.753780in}}%
\pgfpathlineto{\pgfqpoint{0.660333in}{0.757381in}}%
\pgfpathlineto{\pgfqpoint{0.660700in}{0.758467in}}%
\pgfpathlineto{\pgfqpoint{0.661810in}{0.762255in}}%
\pgfpathlineto{\pgfqpoint{0.662092in}{0.763341in}}%
\pgfpathlineto{\pgfqpoint{0.667666in}{0.780973in}}%
\pgfpathlineto{\pgfqpoint{0.668017in}{0.782059in}}%
\pgfpathlineto{\pgfqpoint{0.669034in}{0.785847in}}%
\pgfpathlineto{\pgfqpoint{0.669479in}{0.786933in}}%
\pgfpathlineto{\pgfqpoint{0.670589in}{0.791372in}}%
\pgfpathlineto{\pgfqpoint{0.670855in}{0.792459in}}%
\pgfpathlineto{\pgfqpoint{0.671965in}{0.795780in}}%
\pgfpathlineto{\pgfqpoint{0.672387in}{0.796867in}}%
\pgfpathlineto{\pgfqpoint{0.673466in}{0.799629in}}%
\pgfpathlineto{\pgfqpoint{0.674021in}{0.800716in}}%
\pgfpathlineto{\pgfqpoint{0.675131in}{0.803975in}}%
\pgfpathlineto{\pgfqpoint{0.675811in}{0.805031in}}%
\pgfpathlineto{\pgfqpoint{0.676828in}{0.808228in}}%
\pgfpathlineto{\pgfqpoint{0.677297in}{0.809283in}}%
\pgfpathlineto{\pgfqpoint{0.678391in}{0.812201in}}%
\pgfpathlineto{\pgfqpoint{0.678939in}{0.813288in}}%
\pgfpathlineto{\pgfqpoint{0.680049in}{0.815957in}}%
\pgfpathlineto{\pgfqpoint{0.680744in}{0.817013in}}%
\pgfpathlineto{\pgfqpoint{0.681855in}{0.819683in}}%
\pgfpathlineto{\pgfqpoint{0.682433in}{0.820738in}}%
\pgfpathlineto{\pgfqpoint{0.683543in}{0.823377in}}%
\pgfpathlineto{\pgfqpoint{0.684012in}{0.824370in}}%
\pgfpathlineto{\pgfqpoint{0.685122in}{0.827412in}}%
\pgfpathlineto{\pgfqpoint{0.685560in}{0.828498in}}%
\pgfpathlineto{\pgfqpoint{0.686662in}{0.831385in}}%
\pgfpathlineto{\pgfqpoint{0.687045in}{0.832472in}}%
\pgfpathlineto{\pgfqpoint{0.688148in}{0.835731in}}%
\pgfpathlineto{\pgfqpoint{0.688859in}{0.836818in}}%
\pgfpathlineto{\pgfqpoint{0.689969in}{0.839456in}}%
\pgfpathlineto{\pgfqpoint{0.690454in}{0.840543in}}%
\pgfpathlineto{\pgfqpoint{0.691564in}{0.843430in}}%
\pgfpathlineto{\pgfqpoint{0.691838in}{0.844516in}}%
\pgfpathlineto{\pgfqpoint{0.692948in}{0.847900in}}%
\pgfpathlineto{\pgfqpoint{0.693253in}{0.848986in}}%
\pgfpathlineto{\pgfqpoint{0.694363in}{0.852308in}}%
\pgfpathlineto{\pgfqpoint{0.694847in}{0.853394in}}%
\pgfpathlineto{\pgfqpoint{0.695934in}{0.855785in}}%
\pgfpathlineto{\pgfqpoint{0.696481in}{0.856840in}}%
\pgfpathlineto{\pgfqpoint{0.697576in}{0.859789in}}%
\pgfpathlineto{\pgfqpoint{0.698186in}{0.860844in}}%
\pgfpathlineto{\pgfqpoint{0.699296in}{0.863887in}}%
\pgfpathlineto{\pgfqpoint{0.699859in}{0.864942in}}%
\pgfpathlineto{\pgfqpoint{0.700969in}{0.867922in}}%
\pgfpathlineto{\pgfqpoint{0.701578in}{0.868977in}}%
\pgfpathlineto{\pgfqpoint{0.702665in}{0.871119in}}%
\pgfpathlineto{\pgfqpoint{0.703111in}{0.872206in}}%
\pgfpathlineto{\pgfqpoint{0.704205in}{0.874037in}}%
\pgfpathlineto{\pgfqpoint{0.704690in}{0.875093in}}%
\pgfpathlineto{\pgfqpoint{0.705800in}{0.877855in}}%
\pgfpathlineto{\pgfqpoint{0.706574in}{0.878942in}}%
\pgfpathlineto{\pgfqpoint{0.707684in}{0.881425in}}%
\pgfpathlineto{\pgfqpoint{0.708294in}{0.882512in}}%
\pgfpathlineto{\pgfqpoint{0.709373in}{0.884933in}}%
\pgfpathlineto{\pgfqpoint{0.709396in}{0.884933in}}%
\pgfpathlineto{\pgfqpoint{0.710029in}{0.886020in}}%
\pgfpathlineto{\pgfqpoint{0.711116in}{0.888317in}}%
\pgfpathlineto{\pgfqpoint{0.711632in}{0.889403in}}%
\pgfpathlineto{\pgfqpoint{0.712719in}{0.892042in}}%
\pgfpathlineto{\pgfqpoint{0.713149in}{0.893128in}}%
\pgfpathlineto{\pgfqpoint{0.714235in}{0.895705in}}%
\pgfpathlineto{\pgfqpoint{0.714939in}{0.896791in}}%
\pgfpathlineto{\pgfqpoint{0.716041in}{0.899368in}}%
\pgfpathlineto{\pgfqpoint{0.716463in}{0.900454in}}%
\pgfpathlineto{\pgfqpoint{0.717573in}{0.902906in}}%
\pgfpathlineto{\pgfqpoint{0.718089in}{0.903993in}}%
\pgfpathlineto{\pgfqpoint{0.719168in}{0.905887in}}%
\pgfpathlineto{\pgfqpoint{0.719794in}{0.906942in}}%
\pgfpathlineto{\pgfqpoint{0.720896in}{0.909301in}}%
\pgfpathlineto{\pgfqpoint{0.721443in}{0.910388in}}%
\pgfpathlineto{\pgfqpoint{0.722553in}{0.912685in}}%
\pgfpathlineto{\pgfqpoint{0.723014in}{0.913771in}}%
\pgfpathlineto{\pgfqpoint{0.724117in}{0.916037in}}%
\pgfpathlineto{\pgfqpoint{0.724750in}{0.917124in}}%
\pgfpathlineto{\pgfqpoint{0.725852in}{0.918893in}}%
\pgfpathlineto{\pgfqpoint{0.726407in}{0.919918in}}%
\pgfpathlineto{\pgfqpoint{0.727517in}{0.922649in}}%
\pgfpathlineto{\pgfqpoint{0.728182in}{0.923736in}}%
\pgfpathlineto{\pgfqpoint{0.729292in}{0.926002in}}%
\pgfpathlineto{\pgfqpoint{0.729972in}{0.927057in}}%
\pgfpathlineto{\pgfqpoint{0.731082in}{0.928547in}}%
\pgfpathlineto{\pgfqpoint{0.731543in}{0.929634in}}%
\pgfpathlineto{\pgfqpoint{0.732646in}{0.932024in}}%
\pgfpathlineto{\pgfqpoint{0.733263in}{0.933110in}}%
\pgfpathlineto{\pgfqpoint{0.734366in}{0.934663in}}%
\pgfpathlineto{\pgfqpoint{0.735007in}{0.935749in}}%
\pgfpathlineto{\pgfqpoint{0.736117in}{0.937363in}}%
\pgfpathlineto{\pgfqpoint{0.736617in}{0.938450in}}%
\pgfpathlineto{\pgfqpoint{0.737688in}{0.940592in}}%
\pgfpathlineto{\pgfqpoint{0.738282in}{0.941585in}}%
\pgfpathlineto{\pgfqpoint{0.739392in}{0.943851in}}%
\pgfpathlineto{\pgfqpoint{0.740002in}{0.944938in}}%
\pgfpathlineto{\pgfqpoint{0.741089in}{0.947079in}}%
\pgfpathlineto{\pgfqpoint{0.741706in}{0.948166in}}%
\pgfpathlineto{\pgfqpoint{0.742816in}{0.949935in}}%
\pgfpathlineto{\pgfqpoint{0.743528in}{0.951022in}}%
\pgfpathlineto{\pgfqpoint{0.744583in}{0.953288in}}%
\pgfpathlineto{\pgfqpoint{0.745506in}{0.954374in}}%
\pgfpathlineto{\pgfqpoint{0.746616in}{0.956423in}}%
\pgfpathlineto{\pgfqpoint{0.747147in}{0.957510in}}%
\pgfpathlineto{\pgfqpoint{0.748187in}{0.959217in}}%
\pgfpathlineto{\pgfqpoint{0.748938in}{0.960303in}}%
\pgfpathlineto{\pgfqpoint{0.750048in}{0.962663in}}%
\pgfpathlineto{\pgfqpoint{0.750556in}{0.963749in}}%
\pgfpathlineto{\pgfqpoint{0.751666in}{0.965860in}}%
\pgfpathlineto{\pgfqpoint{0.752448in}{0.966946in}}%
\pgfpathlineto{\pgfqpoint{0.753550in}{0.968902in}}%
\pgfpathlineto{\pgfqpoint{0.754176in}{0.969926in}}%
\pgfpathlineto{\pgfqpoint{0.754176in}{0.969989in}}%
\pgfpathlineto{\pgfqpoint{0.755966in}{0.972658in}}%
\pgfpathlineto{\pgfqpoint{0.756763in}{0.973714in}}%
\pgfpathlineto{\pgfqpoint{0.757873in}{0.975793in}}%
\pgfpathlineto{\pgfqpoint{0.758421in}{0.976880in}}%
\pgfpathlineto{\pgfqpoint{0.759531in}{0.978773in}}%
\pgfpathlineto{\pgfqpoint{0.760140in}{0.979829in}}%
\pgfpathlineto{\pgfqpoint{0.761227in}{0.982033in}}%
\pgfpathlineto{\pgfqpoint{0.762017in}{0.983119in}}%
\pgfpathlineto{\pgfqpoint{0.763088in}{0.985106in}}%
\pgfpathlineto{\pgfqpoint{0.763729in}{0.986193in}}%
\pgfpathlineto{\pgfqpoint{0.764792in}{0.988210in}}%
\pgfpathlineto{\pgfqpoint{0.765636in}{0.989297in}}%
\pgfpathlineto{\pgfqpoint{0.766738in}{0.991594in}}%
\pgfpathlineto{\pgfqpoint{0.767466in}{0.992680in}}%
\pgfpathlineto{\pgfqpoint{0.768552in}{0.994636in}}%
\pgfpathlineto{\pgfqpoint{0.769021in}{0.995722in}}%
\pgfpathlineto{\pgfqpoint{0.770116in}{0.997088in}}%
\pgfpathlineto{\pgfqpoint{0.770475in}{0.998144in}}%
\pgfpathlineto{\pgfqpoint{0.771585in}{0.999696in}}%
\pgfpathlineto{\pgfqpoint{0.772391in}{1.000782in}}%
\pgfpathlineto{\pgfqpoint{0.773501in}{1.002800in}}%
\pgfpathlineto{\pgfqpoint{0.773993in}{1.003887in}}%
\pgfpathlineto{\pgfqpoint{0.775064in}{1.005159in}}%
\pgfpathlineto{\pgfqpoint{0.775823in}{1.006246in}}%
\pgfpathlineto{\pgfqpoint{0.776886in}{1.007581in}}%
\pgfpathlineto{\pgfqpoint{0.778004in}{1.008667in}}%
\pgfpathlineto{\pgfqpoint{0.779028in}{1.010405in}}%
\pgfpathlineto{\pgfqpoint{0.779114in}{1.010405in}}%
\pgfpathlineto{\pgfqpoint{0.779896in}{1.011492in}}%
\pgfpathlineto{\pgfqpoint{0.780990in}{1.012982in}}%
\pgfpathlineto{\pgfqpoint{0.781811in}{1.014037in}}%
\pgfpathlineto{\pgfqpoint{0.782890in}{1.015589in}}%
\pgfpathlineto{\pgfqpoint{0.782905in}{1.015589in}}%
\pgfpathlineto{\pgfqpoint{0.783757in}{1.016645in}}%
\pgfpathlineto{\pgfqpoint{0.784844in}{1.018880in}}%
\pgfpathlineto{\pgfqpoint{0.785900in}{1.019966in}}%
\pgfpathlineto{\pgfqpoint{0.787010in}{1.021705in}}%
\pgfpathlineto{\pgfqpoint{0.787963in}{1.022729in}}%
\pgfpathlineto{\pgfqpoint{0.787963in}{1.022760in}}%
\pgfpathlineto{\pgfqpoint{0.789019in}{1.024498in}}%
\pgfpathlineto{\pgfqpoint{0.789073in}{1.024498in}}%
\pgfpathlineto{\pgfqpoint{0.789801in}{1.025585in}}%
\pgfpathlineto{\pgfqpoint{0.790887in}{1.027106in}}%
\pgfpathlineto{\pgfqpoint{0.791591in}{1.028193in}}%
\pgfpathlineto{\pgfqpoint{0.792693in}{1.030303in}}%
\pgfpathlineto{\pgfqpoint{0.793404in}{1.031390in}}%
\pgfpathlineto{\pgfqpoint{0.794507in}{1.033035in}}%
\pgfpathlineto{\pgfqpoint{0.795226in}{1.034122in}}%
\pgfpathlineto{\pgfqpoint{0.796320in}{1.035798in}}%
\pgfpathlineto{\pgfqpoint{0.797024in}{1.036884in}}%
\pgfpathlineto{\pgfqpoint{0.798126in}{1.038467in}}%
\pgfpathlineto{\pgfqpoint{0.798939in}{1.039554in}}%
\pgfpathlineto{\pgfqpoint{0.800049in}{1.041230in}}%
\pgfpathlineto{\pgfqpoint{0.800808in}{1.042317in}}%
\pgfpathlineto{\pgfqpoint{0.801894in}{1.043993in}}%
\pgfpathlineto{\pgfqpoint{0.802457in}{1.045079in}}%
\pgfpathlineto{\pgfqpoint{0.803466in}{1.046507in}}%
\pgfpathlineto{\pgfqpoint{0.803536in}{1.046507in}}%
\pgfpathlineto{\pgfqpoint{0.804240in}{1.047594in}}%
\pgfpathlineto{\pgfqpoint{0.805303in}{1.049239in}}%
\pgfpathlineto{\pgfqpoint{0.806218in}{1.050294in}}%
\pgfpathlineto{\pgfqpoint{0.807312in}{1.051971in}}%
\pgfpathlineto{\pgfqpoint{0.808063in}{1.053057in}}%
\pgfpathlineto{\pgfqpoint{0.809165in}{1.054237in}}%
\pgfpathlineto{\pgfqpoint{0.809939in}{1.055323in}}%
\pgfpathlineto{\pgfqpoint{0.811494in}{1.057000in}}%
\pgfpathlineto{\pgfqpoint{0.812698in}{1.058086in}}%
\pgfpathlineto{\pgfqpoint{0.813746in}{1.059514in}}%
\pgfpathlineto{\pgfqpoint{0.814340in}{1.060600in}}%
\pgfpathlineto{\pgfqpoint{0.815419in}{1.062308in}}%
\pgfpathlineto{\pgfqpoint{0.816130in}{1.063394in}}%
\pgfpathlineto{\pgfqpoint{0.817240in}{1.064667in}}%
\pgfpathlineto{\pgfqpoint{0.818233in}{1.065753in}}%
\pgfpathlineto{\pgfqpoint{0.819281in}{1.066964in}}%
\pgfpathlineto{\pgfqpoint{0.820078in}{1.068051in}}%
\pgfpathlineto{\pgfqpoint{0.821165in}{1.069696in}}%
\pgfpathlineto{\pgfqpoint{0.821868in}{1.070782in}}%
\pgfpathlineto{\pgfqpoint{0.822947in}{1.072210in}}%
\pgfpathlineto{\pgfqpoint{0.822971in}{1.072210in}}%
\pgfpathlineto{\pgfqpoint{0.823745in}{1.073297in}}%
\pgfpathlineto{\pgfqpoint{0.824855in}{1.074787in}}%
\pgfpathlineto{\pgfqpoint{0.825379in}{1.075873in}}%
\pgfpathlineto{\pgfqpoint{0.826489in}{1.077456in}}%
\pgfpathlineto{\pgfqpoint{0.827098in}{1.078543in}}%
\pgfpathlineto{\pgfqpoint{0.828201in}{1.080126in}}%
\pgfpathlineto{\pgfqpoint{0.828990in}{1.081212in}}%
\pgfpathlineto{\pgfqpoint{0.830085in}{1.082454in}}%
\pgfpathlineto{\pgfqpoint{0.831046in}{1.083541in}}%
\pgfpathlineto{\pgfqpoint{0.832149in}{1.084534in}}%
\pgfpathlineto{\pgfqpoint{0.833321in}{1.085589in}}%
\pgfpathlineto{\pgfqpoint{0.834431in}{1.087204in}}%
\pgfpathlineto{\pgfqpoint{0.835229in}{1.088259in}}%
\pgfpathlineto{\pgfqpoint{0.836323in}{1.089718in}}%
\pgfpathlineto{\pgfqpoint{0.837410in}{1.090804in}}%
\pgfpathlineto{\pgfqpoint{0.838520in}{1.092574in}}%
\pgfpathlineto{\pgfqpoint{0.839231in}{1.093660in}}%
\pgfpathlineto{\pgfqpoint{0.840310in}{1.095275in}}%
\pgfpathlineto{\pgfqpoint{0.840342in}{1.095275in}}%
\pgfpathlineto{\pgfqpoint{0.841194in}{1.096361in}}%
\pgfpathlineto{\pgfqpoint{0.842280in}{1.097416in}}%
\pgfpathlineto{\pgfqpoint{0.843015in}{1.098503in}}%
\pgfpathlineto{\pgfqpoint{0.844071in}{1.100024in}}%
\pgfpathlineto{\pgfqpoint{0.844094in}{1.100024in}}%
\pgfpathlineto{\pgfqpoint{0.844899in}{1.101048in}}%
\pgfpathlineto{\pgfqpoint{0.845923in}{1.102383in}}%
\pgfpathlineto{\pgfqpoint{0.846971in}{1.103439in}}%
\pgfpathlineto{\pgfqpoint{0.848073in}{1.104773in}}%
\pgfpathlineto{\pgfqpoint{0.849105in}{1.105860in}}%
\pgfpathlineto{\pgfqpoint{0.850114in}{1.107381in}}%
\pgfpathlineto{\pgfqpoint{0.850974in}{1.108467in}}%
\pgfpathlineto{\pgfqpoint{0.852013in}{1.110237in}}%
\pgfpathlineto{\pgfqpoint{0.852967in}{1.111292in}}%
\pgfpathlineto{\pgfqpoint{0.854022in}{1.112379in}}%
\pgfpathlineto{\pgfqpoint{0.854781in}{1.113434in}}%
\pgfpathlineto{\pgfqpoint{0.855891in}{1.115017in}}%
\pgfpathlineto{\pgfqpoint{0.856892in}{1.116104in}}%
\pgfpathlineto{\pgfqpoint{0.857931in}{1.117128in}}%
\pgfpathlineto{\pgfqpoint{0.858002in}{1.117128in}}%
\pgfpathlineto{\pgfqpoint{0.858987in}{1.118215in}}%
\pgfpathlineto{\pgfqpoint{0.860073in}{1.119705in}}%
\pgfpathlineto{\pgfqpoint{0.860871in}{1.120760in}}%
\pgfpathlineto{\pgfqpoint{0.861942in}{1.122095in}}%
\pgfpathlineto{\pgfqpoint{0.861981in}{1.122095in}}%
\pgfpathlineto{\pgfqpoint{0.862880in}{1.123181in}}%
\pgfpathlineto{\pgfqpoint{0.863959in}{1.124578in}}%
\pgfpathlineto{\pgfqpoint{0.864834in}{1.125665in}}%
\pgfpathlineto{\pgfqpoint{0.865929in}{1.126720in}}%
\pgfpathlineto{\pgfqpoint{0.866781in}{1.127807in}}%
\pgfpathlineto{\pgfqpoint{0.867883in}{1.129110in}}%
\pgfpathlineto{\pgfqpoint{0.869001in}{1.130197in}}%
\pgfpathlineto{\pgfqpoint{0.870088in}{1.131687in}}%
\pgfpathlineto{\pgfqpoint{0.870103in}{1.131687in}}%
\pgfpathlineto{\pgfqpoint{0.870924in}{1.132773in}}%
\pgfpathlineto{\pgfqpoint{0.871980in}{1.133829in}}%
\pgfpathlineto{\pgfqpoint{0.873113in}{1.134915in}}%
\pgfpathlineto{\pgfqpoint{0.874176in}{1.136592in}}%
\pgfpathlineto{\pgfqpoint{0.875474in}{1.137678in}}%
\pgfpathlineto{\pgfqpoint{0.876576in}{1.139106in}}%
\pgfpathlineto{\pgfqpoint{0.877210in}{1.140193in}}%
\pgfpathlineto{\pgfqpoint{0.878296in}{1.141186in}}%
\pgfpathlineto{\pgfqpoint{0.879289in}{1.142272in}}%
\pgfpathlineto{\pgfqpoint{0.880368in}{1.143483in}}%
\pgfpathlineto{\pgfqpoint{0.881658in}{1.144569in}}%
\pgfpathlineto{\pgfqpoint{0.882729in}{1.146184in}}%
\pgfpathlineto{\pgfqpoint{0.884066in}{1.147239in}}%
\pgfpathlineto{\pgfqpoint{0.885113in}{1.148574in}}%
\pgfpathlineto{\pgfqpoint{0.886169in}{1.149660in}}%
\pgfpathlineto{\pgfqpoint{0.887263in}{1.150747in}}%
\pgfpathlineto{\pgfqpoint{0.888100in}{1.151833in}}%
\pgfpathlineto{\pgfqpoint{0.889124in}{1.152827in}}%
\pgfpathlineto{\pgfqpoint{0.890367in}{1.153913in}}%
\pgfpathlineto{\pgfqpoint{0.891438in}{1.155279in}}%
\pgfpathlineto{\pgfqpoint{0.892642in}{1.156365in}}%
\pgfpathlineto{\pgfqpoint{0.893744in}{1.157390in}}%
\pgfpathlineto{\pgfqpoint{0.894698in}{1.158476in}}%
\pgfpathlineto{\pgfqpoint{0.895792in}{1.159749in}}%
\pgfpathlineto{\pgfqpoint{0.896949in}{1.160836in}}%
\pgfpathlineto{\pgfqpoint{0.898044in}{1.161829in}}%
\pgfpathlineto{\pgfqpoint{0.899107in}{1.162915in}}%
\pgfpathlineto{\pgfqpoint{0.900193in}{1.164405in}}%
\pgfpathlineto{\pgfqpoint{0.900920in}{1.165461in}}%
\pgfpathlineto{\pgfqpoint{0.901984in}{1.166609in}}%
\pgfpathlineto{\pgfqpoint{0.903094in}{1.167603in}}%
\pgfpathlineto{\pgfqpoint{0.904102in}{1.168689in}}%
\pgfpathlineto{\pgfqpoint{0.904196in}{1.168689in}}%
\pgfpathlineto{\pgfqpoint{0.905267in}{1.169776in}}%
\pgfpathlineto{\pgfqpoint{0.906354in}{1.170893in}}%
\pgfpathlineto{\pgfqpoint{0.907558in}{1.171980in}}%
\pgfpathlineto{\pgfqpoint{0.908582in}{1.173066in}}%
\pgfpathlineto{\pgfqpoint{0.909778in}{1.174153in}}%
\pgfpathlineto{\pgfqpoint{0.910865in}{1.175053in}}%
\pgfpathlineto{\pgfqpoint{0.911834in}{1.176139in}}%
\pgfpathlineto{\pgfqpoint{0.912921in}{1.177164in}}%
\pgfpathlineto{\pgfqpoint{0.913843in}{1.178219in}}%
\pgfpathlineto{\pgfqpoint{0.914953in}{1.179150in}}%
\pgfpathlineto{\pgfqpoint{0.916071in}{1.180237in}}%
\pgfpathlineto{\pgfqpoint{0.917166in}{1.181510in}}%
\pgfpathlineto{\pgfqpoint{0.918237in}{1.182534in}}%
\pgfpathlineto{\pgfqpoint{0.919347in}{1.183683in}}%
\pgfpathlineto{\pgfqpoint{0.920128in}{1.184769in}}%
\pgfpathlineto{\pgfqpoint{0.921207in}{1.185887in}}%
\pgfpathlineto{\pgfqpoint{0.922489in}{1.186942in}}%
\pgfpathlineto{\pgfqpoint{0.923584in}{1.187966in}}%
\pgfpathlineto{\pgfqpoint{0.924631in}{1.189053in}}%
\pgfpathlineto{\pgfqpoint{0.925742in}{1.189922in}}%
\pgfpathlineto{\pgfqpoint{0.926781in}{1.191008in}}%
\pgfpathlineto{\pgfqpoint{0.927813in}{1.192374in}}%
\pgfpathlineto{\pgfqpoint{0.929150in}{1.193461in}}%
\pgfpathlineto{\pgfqpoint{0.930252in}{1.194609in}}%
\pgfpathlineto{\pgfqpoint{0.931097in}{1.195696in}}%
\pgfpathlineto{\pgfqpoint{0.932129in}{1.197248in}}%
\pgfpathlineto{\pgfqpoint{0.933192in}{1.198334in}}%
\pgfpathlineto{\pgfqpoint{0.934286in}{1.199204in}}%
\pgfpathlineto{\pgfqpoint{0.935482in}{1.200290in}}%
\pgfpathlineto{\pgfqpoint{0.936569in}{1.201345in}}%
\pgfpathlineto{\pgfqpoint{0.937820in}{1.202432in}}%
\pgfpathlineto{\pgfqpoint{0.938899in}{1.203674in}}%
\pgfpathlineto{\pgfqpoint{0.940517in}{1.204760in}}%
\pgfpathlineto{\pgfqpoint{0.941619in}{1.205722in}}%
\pgfpathlineto{\pgfqpoint{0.943042in}{1.206809in}}%
\pgfpathlineto{\pgfqpoint{0.944144in}{1.207771in}}%
\pgfpathlineto{\pgfqpoint{0.945489in}{1.208858in}}%
\pgfpathlineto{\pgfqpoint{0.946576in}{1.209913in}}%
\pgfpathlineto{\pgfqpoint{0.947662in}{1.211000in}}%
\pgfpathlineto{\pgfqpoint{0.948772in}{1.211807in}}%
\pgfpathlineto{\pgfqpoint{0.949984in}{1.212893in}}%
\pgfpathlineto{\pgfqpoint{0.951063in}{1.213514in}}%
\pgfpathlineto{\pgfqpoint{0.952736in}{1.214569in}}%
\pgfpathlineto{\pgfqpoint{0.953830in}{1.215594in}}%
\pgfpathlineto{\pgfqpoint{0.954893in}{1.216680in}}%
\pgfpathlineto{\pgfqpoint{0.956004in}{1.218077in}}%
\pgfpathlineto{\pgfqpoint{0.957270in}{1.219164in}}%
\pgfpathlineto{\pgfqpoint{0.958341in}{1.220281in}}%
\pgfpathlineto{\pgfqpoint{0.959600in}{1.221368in}}%
\pgfpathlineto{\pgfqpoint{0.960710in}{1.222547in}}%
\pgfpathlineto{\pgfqpoint{0.961750in}{1.223634in}}%
\pgfpathlineto{\pgfqpoint{0.962821in}{1.224472in}}%
\pgfpathlineto{\pgfqpoint{0.964079in}{1.225558in}}%
\pgfpathlineto{\pgfqpoint{0.965166in}{1.226769in}}%
\pgfpathlineto{\pgfqpoint{0.966432in}{1.227855in}}%
\pgfpathlineto{\pgfqpoint{0.967542in}{1.229128in}}%
\pgfpathlineto{\pgfqpoint{0.968355in}{1.230215in}}%
\pgfpathlineto{\pgfqpoint{0.969442in}{1.231239in}}%
\pgfpathlineto{\pgfqpoint{0.970669in}{1.232326in}}%
\pgfpathlineto{\pgfqpoint{0.971748in}{1.233350in}}%
\pgfpathlineto{\pgfqpoint{0.972991in}{1.234436in}}%
\pgfpathlineto{\pgfqpoint{0.974086in}{1.235430in}}%
\pgfpathlineto{\pgfqpoint{0.975290in}{1.236516in}}%
\pgfpathlineto{\pgfqpoint{0.976314in}{1.237447in}}%
\pgfpathlineto{\pgfqpoint{0.977666in}{1.238534in}}%
\pgfpathlineto{\pgfqpoint{0.978690in}{1.239589in}}%
\pgfpathlineto{\pgfqpoint{0.978706in}{1.239589in}}%
\pgfpathlineto{\pgfqpoint{0.980105in}{1.240676in}}%
\pgfpathlineto{\pgfqpoint{0.981200in}{1.241980in}}%
\pgfpathlineto{\pgfqpoint{0.982505in}{1.243066in}}%
\pgfpathlineto{\pgfqpoint{0.983569in}{1.243842in}}%
\pgfpathlineto{\pgfqpoint{0.983608in}{1.243842in}}%
\pgfpathlineto{\pgfqpoint{0.985328in}{1.244929in}}%
\pgfpathlineto{\pgfqpoint{0.986328in}{1.245860in}}%
\pgfpathlineto{\pgfqpoint{0.987626in}{1.246946in}}%
\pgfpathlineto{\pgfqpoint{0.988666in}{1.247722in}}%
\pgfpathlineto{\pgfqpoint{0.988728in}{1.247722in}}%
\pgfpathlineto{\pgfqpoint{0.989885in}{1.248809in}}%
\pgfpathlineto{\pgfqpoint{0.990956in}{1.249802in}}%
\pgfpathlineto{\pgfqpoint{0.992371in}{1.250889in}}%
\pgfpathlineto{\pgfqpoint{0.993474in}{1.252006in}}%
\pgfpathlineto{\pgfqpoint{0.994670in}{1.253093in}}%
\pgfpathlineto{\pgfqpoint{0.995764in}{1.254272in}}%
\pgfpathlineto{\pgfqpoint{0.997476in}{1.255359in}}%
\pgfpathlineto{\pgfqpoint{0.998586in}{1.256476in}}%
\pgfpathlineto{\pgfqpoint{1.000533in}{1.257563in}}%
\pgfpathlineto{\pgfqpoint{1.001643in}{1.258525in}}%
\pgfpathlineto{\pgfqpoint{1.003003in}{1.259581in}}%
\pgfpathlineto{\pgfqpoint{1.004027in}{1.260667in}}%
\pgfpathlineto{\pgfqpoint{1.005435in}{1.261753in}}%
\pgfpathlineto{\pgfqpoint{1.006529in}{1.262716in}}%
\pgfpathlineto{\pgfqpoint{1.007803in}{1.263802in}}%
\pgfpathlineto{\pgfqpoint{1.008906in}{1.264547in}}%
\pgfpathlineto{\pgfqpoint{1.010790in}{1.265634in}}%
\pgfpathlineto{\pgfqpoint{1.011822in}{1.266596in}}%
\pgfpathlineto{\pgfqpoint{1.013346in}{1.267683in}}%
\pgfpathlineto{\pgfqpoint{1.014394in}{1.268241in}}%
\pgfpathlineto{\pgfqpoint{1.015637in}{1.269328in}}%
\pgfpathlineto{\pgfqpoint{1.016747in}{1.270042in}}%
\pgfpathlineto{\pgfqpoint{1.018209in}{1.271097in}}%
\pgfpathlineto{\pgfqpoint{1.019311in}{1.271904in}}%
\pgfpathlineto{\pgfqpoint{1.020546in}{1.272991in}}%
\pgfpathlineto{\pgfqpoint{1.021648in}{1.274046in}}%
\pgfpathlineto{\pgfqpoint{1.022743in}{1.275133in}}%
\pgfpathlineto{\pgfqpoint{1.023853in}{1.276157in}}%
\pgfpathlineto{\pgfqpoint{1.025010in}{1.277212in}}%
\pgfpathlineto{\pgfqpoint{1.026089in}{1.278113in}}%
\pgfpathlineto{\pgfqpoint{1.027371in}{1.279199in}}%
\pgfpathlineto{\pgfqpoint{1.028309in}{1.279820in}}%
\pgfpathlineto{\pgfqpoint{1.029779in}{1.280906in}}%
\pgfpathlineto{\pgfqpoint{1.030889in}{1.281745in}}%
\pgfpathlineto{\pgfqpoint{1.032359in}{1.282831in}}%
\pgfpathlineto{\pgfqpoint{1.033461in}{1.283793in}}%
\pgfpathlineto{\pgfqpoint{1.034759in}{1.284849in}}%
\pgfpathlineto{\pgfqpoint{1.035822in}{1.285470in}}%
\pgfpathlineto{\pgfqpoint{1.037276in}{1.286556in}}%
\pgfpathlineto{\pgfqpoint{1.038378in}{1.287177in}}%
\pgfpathlineto{\pgfqpoint{1.040254in}{1.288263in}}%
\pgfpathlineto{\pgfqpoint{1.041318in}{1.288915in}}%
\pgfpathlineto{\pgfqpoint{1.042936in}{1.290002in}}%
\pgfpathlineto{\pgfqpoint{1.043968in}{1.290747in}}%
\pgfpathlineto{\pgfqpoint{1.045508in}{1.291833in}}%
\pgfpathlineto{\pgfqpoint{1.046540in}{1.292547in}}%
\pgfpathlineto{\pgfqpoint{1.048158in}{1.293634in}}%
\pgfpathlineto{\pgfqpoint{1.049206in}{1.294472in}}%
\pgfpathlineto{\pgfqpoint{1.050589in}{1.295558in}}%
\pgfpathlineto{\pgfqpoint{1.051692in}{1.296365in}}%
\pgfpathlineto{\pgfqpoint{1.053443in}{1.297452in}}%
\pgfpathlineto{\pgfqpoint{1.054506in}{1.298197in}}%
\pgfpathlineto{\pgfqpoint{1.056140in}{1.299283in}}%
\pgfpathlineto{\pgfqpoint{1.057234in}{1.299935in}}%
\pgfpathlineto{\pgfqpoint{1.058892in}{1.300991in}}%
\pgfpathlineto{\pgfqpoint{1.059994in}{1.301953in}}%
\pgfpathlineto{\pgfqpoint{1.061956in}{1.303040in}}%
\pgfpathlineto{\pgfqpoint{1.063019in}{1.303660in}}%
\pgfpathlineto{\pgfqpoint{1.064286in}{1.304747in}}%
\pgfpathlineto{\pgfqpoint{1.065279in}{1.305430in}}%
\pgfpathlineto{\pgfqpoint{1.065372in}{1.305430in}}%
\pgfpathlineto{\pgfqpoint{1.066944in}{1.306516in}}%
\pgfpathlineto{\pgfqpoint{1.068030in}{1.307385in}}%
\pgfpathlineto{\pgfqpoint{1.069750in}{1.308472in}}%
\pgfpathlineto{\pgfqpoint{1.070845in}{1.309031in}}%
\pgfpathlineto{\pgfqpoint{1.072479in}{1.310117in}}%
\pgfpathlineto{\pgfqpoint{1.073518in}{1.310831in}}%
\pgfpathlineto{\pgfqpoint{1.074949in}{1.311918in}}%
\pgfpathlineto{\pgfqpoint{1.076020in}{1.312756in}}%
\pgfpathlineto{\pgfqpoint{1.077490in}{1.313842in}}%
\pgfpathlineto{\pgfqpoint{1.078600in}{1.314867in}}%
\pgfpathlineto{\pgfqpoint{1.079835in}{1.315922in}}%
\pgfpathlineto{\pgfqpoint{1.080914in}{1.316729in}}%
\pgfpathlineto{\pgfqpoint{1.080945in}{1.316729in}}%
\pgfpathlineto{\pgfqpoint{1.082837in}{1.317816in}}%
\pgfpathlineto{\pgfqpoint{1.083908in}{1.318654in}}%
\pgfpathlineto{\pgfqpoint{1.085698in}{1.319740in}}%
\pgfpathlineto{\pgfqpoint{1.086761in}{1.320485in}}%
\pgfpathlineto{\pgfqpoint{1.088474in}{1.321572in}}%
\pgfpathlineto{\pgfqpoint{1.089560in}{1.322379in}}%
\pgfpathlineto{\pgfqpoint{1.091687in}{1.323434in}}%
\pgfpathlineto{\pgfqpoint{1.092797in}{1.324148in}}%
\pgfpathlineto{\pgfqpoint{1.094610in}{1.325235in}}%
\pgfpathlineto{\pgfqpoint{1.095619in}{1.326259in}}%
\pgfpathlineto{\pgfqpoint{1.097261in}{1.327345in}}%
\pgfpathlineto{\pgfqpoint{1.098316in}{1.328215in}}%
\pgfpathlineto{\pgfqpoint{1.100255in}{1.329301in}}%
\pgfpathlineto{\pgfqpoint{1.101341in}{1.330170in}}%
\pgfpathlineto{\pgfqpoint{1.103030in}{1.331257in}}%
\pgfpathlineto{\pgfqpoint{1.104031in}{1.332095in}}%
\pgfpathlineto{\pgfqpoint{1.106095in}{1.333181in}}%
\pgfpathlineto{\pgfqpoint{1.107158in}{1.333895in}}%
\pgfpathlineto{\pgfqpoint{1.108627in}{1.334982in}}%
\pgfpathlineto{\pgfqpoint{1.109706in}{1.335665in}}%
\pgfpathlineto{\pgfqpoint{1.110746in}{1.336751in}}%
\pgfpathlineto{\pgfqpoint{1.111840in}{1.337403in}}%
\pgfpathlineto{\pgfqpoint{1.113693in}{1.338490in}}%
\pgfpathlineto{\pgfqpoint{1.114733in}{1.339328in}}%
\pgfpathlineto{\pgfqpoint{1.117063in}{1.340414in}}%
\pgfpathlineto{\pgfqpoint{1.118040in}{1.341128in}}%
\pgfpathlineto{\pgfqpoint{1.118173in}{1.341128in}}%
\pgfpathlineto{\pgfqpoint{1.119947in}{1.342215in}}%
\pgfpathlineto{\pgfqpoint{1.121206in}{1.342804in}}%
\pgfpathlineto{\pgfqpoint{1.122543in}{1.343891in}}%
\pgfpathlineto{\pgfqpoint{1.123536in}{1.344481in}}%
\pgfpathlineto{\pgfqpoint{1.123606in}{1.344481in}}%
\pgfpathlineto{\pgfqpoint{1.125404in}{1.345567in}}%
\pgfpathlineto{\pgfqpoint{1.126491in}{1.346219in}}%
\pgfpathlineto{\pgfqpoint{1.128328in}{1.347306in}}%
\pgfpathlineto{\pgfqpoint{1.129422in}{1.347989in}}%
\pgfpathlineto{\pgfqpoint{1.131643in}{1.349075in}}%
\pgfpathlineto{\pgfqpoint{1.132682in}{1.349820in}}%
\pgfpathlineto{\pgfqpoint{1.134598in}{1.350875in}}%
\pgfpathlineto{\pgfqpoint{1.135684in}{1.351714in}}%
\pgfpathlineto{\pgfqpoint{1.137475in}{1.352800in}}%
\pgfpathlineto{\pgfqpoint{1.138561in}{1.353731in}}%
\pgfpathlineto{\pgfqpoint{1.140476in}{1.354818in}}%
\pgfpathlineto{\pgfqpoint{1.141469in}{1.355439in}}%
\pgfpathlineto{\pgfqpoint{1.143822in}{1.356525in}}%
\pgfpathlineto{\pgfqpoint{1.144925in}{1.357612in}}%
\pgfpathlineto{\pgfqpoint{1.146512in}{1.358667in}}%
\pgfpathlineto{\pgfqpoint{1.147606in}{1.359412in}}%
\pgfpathlineto{\pgfqpoint{1.149811in}{1.360498in}}%
\pgfpathlineto{\pgfqpoint{1.150843in}{1.361119in}}%
\pgfpathlineto{\pgfqpoint{1.152711in}{1.362206in}}%
\pgfpathlineto{\pgfqpoint{1.153821in}{1.363013in}}%
\pgfpathlineto{\pgfqpoint{1.155533in}{1.364099in}}%
\pgfpathlineto{\pgfqpoint{1.156589in}{1.364844in}}%
\pgfpathlineto{\pgfqpoint{1.158754in}{1.365931in}}%
\pgfpathlineto{\pgfqpoint{1.159559in}{1.366117in}}%
\pgfpathlineto{\pgfqpoint{1.159692in}{1.366117in}}%
\pgfpathlineto{\pgfqpoint{1.161287in}{1.367204in}}%
\pgfpathlineto{\pgfqpoint{1.162374in}{1.368259in}}%
\pgfpathlineto{\pgfqpoint{1.164336in}{1.369345in}}%
\pgfpathlineto{\pgfqpoint{1.165391in}{1.370339in}}%
\pgfpathlineto{\pgfqpoint{1.167103in}{1.371425in}}%
\pgfpathlineto{\pgfqpoint{1.168174in}{1.372419in}}%
\pgfpathlineto{\pgfqpoint{1.168198in}{1.372419in}}%
\pgfpathlineto{\pgfqpoint{1.169933in}{1.373474in}}%
\pgfpathlineto{\pgfqpoint{1.171028in}{1.374467in}}%
\pgfpathlineto{\pgfqpoint{1.173178in}{1.375554in}}%
\pgfpathlineto{\pgfqpoint{1.174280in}{1.376206in}}%
\pgfpathlineto{\pgfqpoint{1.176313in}{1.377292in}}%
\pgfpathlineto{\pgfqpoint{1.177360in}{1.377975in}}%
\pgfpathlineto{\pgfqpoint{1.177415in}{1.377975in}}%
\pgfpathlineto{\pgfqpoint{1.179393in}{1.379062in}}%
\pgfpathlineto{\pgfqpoint{1.180487in}{1.379745in}}%
\pgfpathlineto{\pgfqpoint{1.182426in}{1.380831in}}%
\pgfpathlineto{\pgfqpoint{1.183536in}{1.381421in}}%
\pgfpathlineto{\pgfqpoint{1.185850in}{1.382507in}}%
\pgfpathlineto{\pgfqpoint{1.186960in}{1.383221in}}%
\pgfpathlineto{\pgfqpoint{1.189032in}{1.384308in}}%
\pgfpathlineto{\pgfqpoint{1.190017in}{1.385177in}}%
\pgfpathlineto{\pgfqpoint{1.192104in}{1.386263in}}%
\pgfpathlineto{\pgfqpoint{1.193199in}{1.387164in}}%
\pgfpathlineto{\pgfqpoint{1.195200in}{1.388219in}}%
\pgfpathlineto{\pgfqpoint{1.196240in}{1.388964in}}%
\pgfpathlineto{\pgfqpoint{1.198343in}{1.390020in}}%
\pgfpathlineto{\pgfqpoint{1.199429in}{1.390765in}}%
\pgfpathlineto{\pgfqpoint{1.201947in}{1.391851in}}%
\pgfpathlineto{\pgfqpoint{1.203049in}{1.392534in}}%
\pgfpathlineto{\pgfqpoint{1.204487in}{1.393589in}}%
\pgfpathlineto{\pgfqpoint{1.205558in}{1.394148in}}%
\pgfpathlineto{\pgfqpoint{1.207583in}{1.395235in}}%
\pgfpathlineto{\pgfqpoint{1.208685in}{1.396011in}}%
\pgfpathlineto{\pgfqpoint{1.210796in}{1.397097in}}%
\pgfpathlineto{\pgfqpoint{1.211836in}{1.397935in}}%
\pgfpathlineto{\pgfqpoint{1.213853in}{1.399022in}}%
\pgfpathlineto{\pgfqpoint{1.214908in}{1.399798in}}%
\pgfpathlineto{\pgfqpoint{1.216902in}{1.400884in}}%
\pgfpathlineto{\pgfqpoint{1.218012in}{1.401350in}}%
\pgfpathlineto{\pgfqpoint{1.220076in}{1.402436in}}%
\pgfpathlineto{\pgfqpoint{1.221014in}{1.403057in}}%
\pgfpathlineto{\pgfqpoint{1.222929in}{1.404144in}}%
\pgfpathlineto{\pgfqpoint{1.224039in}{1.404796in}}%
\pgfpathlineto{\pgfqpoint{1.226799in}{1.405882in}}%
\pgfpathlineto{\pgfqpoint{1.227886in}{1.406689in}}%
\pgfpathlineto{\pgfqpoint{1.229926in}{1.407776in}}%
\pgfpathlineto{\pgfqpoint{1.230981in}{1.408583in}}%
\pgfpathlineto{\pgfqpoint{1.231036in}{1.408583in}}%
\pgfpathlineto{\pgfqpoint{1.233014in}{1.409638in}}%
\pgfpathlineto{\pgfqpoint{1.234108in}{1.410445in}}%
\pgfpathlineto{\pgfqpoint{1.236532in}{1.411532in}}%
\pgfpathlineto{\pgfqpoint{1.237603in}{1.412246in}}%
\pgfpathlineto{\pgfqpoint{1.239956in}{1.413332in}}%
\pgfpathlineto{\pgfqpoint{1.240925in}{1.413705in}}%
\pgfpathlineto{\pgfqpoint{1.241066in}{1.413705in}}%
\pgfpathlineto{\pgfqpoint{1.243013in}{1.414791in}}%
\pgfpathlineto{\pgfqpoint{1.244271in}{1.415412in}}%
\pgfpathlineto{\pgfqpoint{1.246726in}{1.416498in}}%
\pgfpathlineto{\pgfqpoint{1.247797in}{1.417026in}}%
\pgfpathlineto{\pgfqpoint{1.247836in}{1.417026in}}%
\pgfpathlineto{\pgfqpoint{1.250236in}{1.418113in}}%
\pgfpathlineto{\pgfqpoint{1.251276in}{1.418889in}}%
\pgfpathlineto{\pgfqpoint{1.253449in}{1.419975in}}%
\pgfpathlineto{\pgfqpoint{1.254544in}{1.420565in}}%
\pgfpathlineto{\pgfqpoint{1.257131in}{1.421651in}}%
\pgfpathlineto{\pgfqpoint{1.258241in}{1.422179in}}%
\pgfpathlineto{\pgfqpoint{1.260376in}{1.423266in}}%
\pgfpathlineto{\pgfqpoint{1.261369in}{1.423731in}}%
\pgfpathlineto{\pgfqpoint{1.263941in}{1.424818in}}%
\pgfpathlineto{\pgfqpoint{1.264941in}{1.425252in}}%
\pgfpathlineto{\pgfqpoint{1.267388in}{1.426339in}}%
\pgfpathlineto{\pgfqpoint{1.268373in}{1.426649in}}%
\pgfpathlineto{\pgfqpoint{1.271156in}{1.427705in}}%
\pgfpathlineto{\pgfqpoint{1.272258in}{1.428108in}}%
\pgfpathlineto{\pgfqpoint{1.274408in}{1.429195in}}%
\pgfpathlineto{\pgfqpoint{1.275393in}{1.429691in}}%
\pgfpathlineto{\pgfqpoint{1.278364in}{1.430778in}}%
\pgfpathlineto{\pgfqpoint{1.279412in}{1.431306in}}%
\pgfpathlineto{\pgfqpoint{1.279459in}{1.431306in}}%
\pgfpathlineto{\pgfqpoint{1.281155in}{1.432392in}}%
\pgfpathlineto{\pgfqpoint{1.282218in}{1.432951in}}%
\pgfpathlineto{\pgfqpoint{1.284047in}{1.434037in}}%
\pgfpathlineto{\pgfqpoint{1.285009in}{1.434379in}}%
\pgfpathlineto{\pgfqpoint{1.287464in}{1.435465in}}%
\pgfpathlineto{\pgfqpoint{1.288707in}{1.436055in}}%
\pgfpathlineto{\pgfqpoint{1.291459in}{1.437142in}}%
\pgfpathlineto{\pgfqpoint{1.292459in}{1.437793in}}%
\pgfpathlineto{\pgfqpoint{1.294664in}{1.438880in}}%
\pgfpathlineto{\pgfqpoint{1.295758in}{1.439439in}}%
\pgfpathlineto{\pgfqpoint{1.298377in}{1.440525in}}%
\pgfpathlineto{\pgfqpoint{1.299417in}{1.441084in}}%
\pgfpathlineto{\pgfqpoint{1.301059in}{1.442170in}}%
\pgfpathlineto{\pgfqpoint{1.302169in}{1.443102in}}%
\pgfpathlineto{\pgfqpoint{1.304975in}{1.444188in}}%
\pgfpathlineto{\pgfqpoint{1.306062in}{1.444995in}}%
\pgfpathlineto{\pgfqpoint{1.308697in}{1.446082in}}%
\pgfpathlineto{\pgfqpoint{1.309807in}{1.446734in}}%
\pgfpathlineto{\pgfqpoint{1.312566in}{1.447820in}}%
\pgfpathlineto{\pgfqpoint{1.313676in}{1.448348in}}%
\pgfpathlineto{\pgfqpoint{1.316147in}{1.449434in}}%
\pgfpathlineto{\pgfqpoint{1.317249in}{1.449900in}}%
\pgfpathlineto{\pgfqpoint{1.320407in}{1.450986in}}%
\pgfpathlineto{\pgfqpoint{1.321236in}{1.451421in}}%
\pgfpathlineto{\pgfqpoint{1.321431in}{1.451421in}}%
\pgfpathlineto{\pgfqpoint{1.324473in}{1.452507in}}%
\pgfpathlineto{\pgfqpoint{1.325379in}{1.453035in}}%
\pgfpathlineto{\pgfqpoint{1.328256in}{1.454091in}}%
\pgfpathlineto{\pgfqpoint{1.329171in}{1.454649in}}%
\pgfpathlineto{\pgfqpoint{1.331759in}{1.455736in}}%
\pgfpathlineto{\pgfqpoint{1.332845in}{1.456170in}}%
\pgfpathlineto{\pgfqpoint{1.335245in}{1.457257in}}%
\pgfpathlineto{\pgfqpoint{1.336293in}{1.457722in}}%
\pgfpathlineto{\pgfqpoint{1.339342in}{1.458809in}}%
\pgfpathlineto{\pgfqpoint{1.340436in}{1.459212in}}%
\pgfpathlineto{\pgfqpoint{1.342953in}{1.460268in}}%
\pgfpathlineto{\pgfqpoint{1.344048in}{1.460765in}}%
\pgfpathlineto{\pgfqpoint{1.347050in}{1.461851in}}%
\pgfpathlineto{\pgfqpoint{1.348082in}{1.462410in}}%
\pgfpathlineto{\pgfqpoint{1.351826in}{1.463496in}}%
\pgfpathlineto{\pgfqpoint{1.352905in}{1.463962in}}%
\pgfpathlineto{\pgfqpoint{1.354938in}{1.465048in}}%
\pgfpathlineto{\pgfqpoint{1.356032in}{1.465483in}}%
\pgfpathlineto{\pgfqpoint{1.358354in}{1.466569in}}%
\pgfpathlineto{\pgfqpoint{1.359386in}{1.466973in}}%
\pgfpathlineto{\pgfqpoint{1.359456in}{1.466973in}}%
\pgfpathlineto{\pgfqpoint{1.363717in}{1.468059in}}%
\pgfpathlineto{\pgfqpoint{1.364577in}{1.468308in}}%
\pgfpathlineto{\pgfqpoint{1.364827in}{1.468308in}}%
\pgfpathlineto{\pgfqpoint{1.367227in}{1.469394in}}%
\pgfpathlineto{\pgfqpoint{1.368283in}{1.470015in}}%
\pgfpathlineto{\pgfqpoint{1.371668in}{1.471102in}}%
\pgfpathlineto{\pgfqpoint{1.372762in}{1.471598in}}%
\pgfpathlineto{\pgfqpoint{1.375100in}{1.472685in}}%
\pgfpathlineto{\pgfqpoint{1.376053in}{1.473244in}}%
\pgfpathlineto{\pgfqpoint{1.378875in}{1.474330in}}%
\pgfpathlineto{\pgfqpoint{1.379853in}{1.474920in}}%
\pgfpathlineto{\pgfqpoint{1.382737in}{1.476006in}}%
\pgfpathlineto{\pgfqpoint{1.383793in}{1.476472in}}%
\pgfpathlineto{\pgfqpoint{1.386638in}{1.477527in}}%
\pgfpathlineto{\pgfqpoint{1.387756in}{1.478086in}}%
\pgfpathlineto{\pgfqpoint{1.389593in}{1.479173in}}%
\pgfpathlineto{\pgfqpoint{1.390704in}{1.479793in}}%
\pgfpathlineto{\pgfqpoint{1.393494in}{1.480880in}}%
\pgfpathlineto{\pgfqpoint{1.394597in}{1.481314in}}%
\pgfpathlineto{\pgfqpoint{1.397857in}{1.482401in}}%
\pgfpathlineto{\pgfqpoint{1.398732in}{1.482960in}}%
\pgfpathlineto{\pgfqpoint{1.402282in}{1.484046in}}%
\pgfpathlineto{\pgfqpoint{1.403368in}{1.484512in}}%
\pgfpathlineto{\pgfqpoint{1.407011in}{1.485598in}}%
\pgfpathlineto{\pgfqpoint{1.407957in}{1.486281in}}%
\pgfpathlineto{\pgfqpoint{1.411100in}{1.487368in}}%
\pgfpathlineto{\pgfqpoint{1.412108in}{1.487864in}}%
\pgfpathlineto{\pgfqpoint{1.415751in}{1.488858in}}%
\pgfpathlineto{\pgfqpoint{1.416736in}{1.489447in}}%
\pgfpathlineto{\pgfqpoint{1.416807in}{1.489447in}}%
\pgfpathlineto{\pgfqpoint{1.420473in}{1.490534in}}%
\pgfpathlineto{\pgfqpoint{1.421552in}{1.490938in}}%
\pgfpathlineto{\pgfqpoint{1.424359in}{1.492024in}}%
\pgfpathlineto{\pgfqpoint{1.425390in}{1.492521in}}%
\pgfpathlineto{\pgfqpoint{1.428533in}{1.493607in}}%
\pgfpathlineto{\pgfqpoint{1.429542in}{1.494135in}}%
\pgfpathlineto{\pgfqpoint{1.432747in}{1.495221in}}%
\pgfpathlineto{\pgfqpoint{1.433818in}{1.495780in}}%
\pgfpathlineto{\pgfqpoint{1.437273in}{1.496867in}}%
\pgfpathlineto{\pgfqpoint{1.438180in}{1.497270in}}%
\pgfpathlineto{\pgfqpoint{1.442378in}{1.498357in}}%
\pgfpathlineto{\pgfqpoint{1.443473in}{1.498636in}}%
\pgfpathlineto{\pgfqpoint{1.447905in}{1.499722in}}%
\pgfpathlineto{\pgfqpoint{1.448757in}{1.500157in}}%
\pgfpathlineto{\pgfqpoint{1.452111in}{1.501244in}}%
\pgfpathlineto{\pgfqpoint{1.453018in}{1.501554in}}%
\pgfpathlineto{\pgfqpoint{1.457154in}{1.502640in}}%
\pgfpathlineto{\pgfqpoint{1.458217in}{1.503106in}}%
\pgfpathlineto{\pgfqpoint{1.458264in}{1.503106in}}%
\pgfpathlineto{\pgfqpoint{1.460367in}{1.504193in}}%
\pgfpathlineto{\pgfqpoint{1.460984in}{1.504472in}}%
\pgfpathlineto{\pgfqpoint{1.461469in}{1.504472in}}%
\pgfpathlineto{\pgfqpoint{1.465956in}{1.505558in}}%
\pgfpathlineto{\pgfqpoint{1.466996in}{1.505900in}}%
\pgfpathlineto{\pgfqpoint{1.467027in}{1.505900in}}%
\pgfpathlineto{\pgfqpoint{1.470006in}{1.506986in}}%
\pgfpathlineto{\pgfqpoint{1.471092in}{1.507359in}}%
\pgfpathlineto{\pgfqpoint{1.474001in}{1.508414in}}%
\pgfpathlineto{\pgfqpoint{1.475040in}{1.508880in}}%
\pgfpathlineto{\pgfqpoint{1.475079in}{1.508880in}}%
\pgfpathlineto{\pgfqpoint{1.478496in}{1.509966in}}%
\pgfpathlineto{\pgfqpoint{1.479238in}{1.510246in}}%
\pgfpathlineto{\pgfqpoint{1.482475in}{1.511332in}}%
\pgfpathlineto{\pgfqpoint{1.483530in}{1.511829in}}%
\pgfpathlineto{\pgfqpoint{1.483585in}{1.511829in}}%
\pgfpathlineto{\pgfqpoint{1.486032in}{1.512915in}}%
\pgfpathlineto{\pgfqpoint{1.487040in}{1.513443in}}%
\pgfpathlineto{\pgfqpoint{1.490511in}{1.514530in}}%
\pgfpathlineto{\pgfqpoint{1.491582in}{1.514964in}}%
\pgfpathlineto{\pgfqpoint{1.495311in}{1.516051in}}%
\pgfpathlineto{\pgfqpoint{1.496367in}{1.516454in}}%
\pgfpathlineto{\pgfqpoint{1.500213in}{1.517541in}}%
\pgfpathlineto{\pgfqpoint{1.501151in}{1.517758in}}%
\pgfpathlineto{\pgfqpoint{1.501245in}{1.517758in}}%
\pgfpathlineto{\pgfqpoint{1.505639in}{1.518844in}}%
\pgfpathlineto{\pgfqpoint{1.506647in}{1.519279in}}%
\pgfpathlineto{\pgfqpoint{1.506717in}{1.519279in}}%
\pgfpathlineto{\pgfqpoint{1.510142in}{1.520334in}}%
\pgfpathlineto{\pgfqpoint{1.511173in}{1.520831in}}%
\pgfpathlineto{\pgfqpoint{1.515207in}{1.521918in}}%
\pgfpathlineto{\pgfqpoint{1.516317in}{1.522476in}}%
\pgfpathlineto{\pgfqpoint{1.520688in}{1.523563in}}%
\pgfpathlineto{\pgfqpoint{1.521790in}{1.523904in}}%
\pgfpathlineto{\pgfqpoint{1.525644in}{1.524991in}}%
\pgfpathlineto{\pgfqpoint{1.526754in}{1.525487in}}%
\pgfpathlineto{\pgfqpoint{1.531718in}{1.526574in}}%
\pgfpathlineto{\pgfqpoint{1.532609in}{1.526977in}}%
\pgfpathlineto{\pgfqpoint{1.536784in}{1.528064in}}%
\pgfpathlineto{\pgfqpoint{1.537839in}{1.528716in}}%
\pgfpathlineto{\pgfqpoint{1.537894in}{1.528716in}}%
\pgfpathlineto{\pgfqpoint{1.541482in}{1.529771in}}%
\pgfpathlineto{\pgfqpoint{1.542569in}{1.530051in}}%
\pgfpathlineto{\pgfqpoint{1.547400in}{1.531137in}}%
\pgfpathlineto{\pgfqpoint{1.548323in}{1.531510in}}%
\pgfpathlineto{\pgfqpoint{1.552912in}{1.532596in}}%
\pgfpathlineto{\pgfqpoint{1.553920in}{1.533000in}}%
\pgfpathlineto{\pgfqpoint{1.557516in}{1.534086in}}%
\pgfpathlineto{\pgfqpoint{1.558548in}{1.534428in}}%
\pgfpathlineto{\pgfqpoint{1.562770in}{1.535514in}}%
\pgfpathlineto{\pgfqpoint{1.563825in}{1.535855in}}%
\pgfpathlineto{\pgfqpoint{1.567930in}{1.536911in}}%
\pgfpathlineto{\pgfqpoint{1.568625in}{1.537252in}}%
\pgfpathlineto{\pgfqpoint{1.573253in}{1.538339in}}%
\pgfpathlineto{\pgfqpoint{1.574340in}{1.538804in}}%
\pgfpathlineto{\pgfqpoint{1.577881in}{1.539891in}}%
\pgfpathlineto{\pgfqpoint{1.578984in}{1.540326in}}%
\pgfpathlineto{\pgfqpoint{1.582197in}{1.541412in}}%
\pgfpathlineto{\pgfqpoint{1.583252in}{1.541660in}}%
\pgfpathlineto{\pgfqpoint{1.588482in}{1.542747in}}%
\pgfpathlineto{\pgfqpoint{1.588764in}{1.542809in}}%
\pgfpathlineto{\pgfqpoint{1.589592in}{1.542809in}}%
\pgfpathlineto{\pgfqpoint{1.594564in}{1.543895in}}%
\pgfpathlineto{\pgfqpoint{1.595604in}{1.544206in}}%
\pgfpathlineto{\pgfqpoint{1.601498in}{1.545292in}}%
\pgfpathlineto{\pgfqpoint{1.602585in}{1.545696in}}%
\pgfpathlineto{\pgfqpoint{1.607065in}{1.546782in}}%
\pgfpathlineto{\pgfqpoint{1.608151in}{1.547124in}}%
\pgfpathlineto{\pgfqpoint{1.613436in}{1.548210in}}%
\pgfpathlineto{\pgfqpoint{1.614546in}{1.548552in}}%
\pgfpathlineto{\pgfqpoint{1.619721in}{1.549638in}}%
\pgfpathlineto{\pgfqpoint{1.620753in}{1.550011in}}%
\pgfpathlineto{\pgfqpoint{1.625741in}{1.551097in}}%
\pgfpathlineto{\pgfqpoint{1.626945in}{1.551345in}}%
\pgfpathlineto{\pgfqpoint{1.632925in}{1.552432in}}%
\pgfpathlineto{\pgfqpoint{1.633926in}{1.552804in}}%
\pgfpathlineto{\pgfqpoint{1.638499in}{1.553891in}}%
\pgfpathlineto{\pgfqpoint{1.639258in}{1.554294in}}%
\pgfpathlineto{\pgfqpoint{1.645801in}{1.555381in}}%
\pgfpathlineto{\pgfqpoint{1.646817in}{1.555660in}}%
\pgfpathlineto{\pgfqpoint{1.649655in}{1.556747in}}%
\pgfpathlineto{\pgfqpoint{1.650687in}{1.556933in}}%
\pgfpathlineto{\pgfqpoint{1.654526in}{1.558020in}}%
\pgfpathlineto{\pgfqpoint{1.655557in}{1.558299in}}%
\pgfpathlineto{\pgfqpoint{1.661655in}{1.559385in}}%
\pgfpathlineto{\pgfqpoint{1.662726in}{1.559665in}}%
\pgfpathlineto{\pgfqpoint{1.669645in}{1.560751in}}%
\pgfpathlineto{\pgfqpoint{1.670114in}{1.560906in}}%
\pgfpathlineto{\pgfqpoint{1.670724in}{1.560906in}}%
\pgfpathlineto{\pgfqpoint{1.677392in}{1.561993in}}%
\pgfpathlineto{\pgfqpoint{1.678448in}{1.562241in}}%
\pgfpathlineto{\pgfqpoint{1.684092in}{1.563328in}}%
\pgfpathlineto{\pgfqpoint{1.685179in}{1.563700in}}%
\pgfpathlineto{\pgfqpoint{1.692418in}{1.564787in}}%
\pgfpathlineto{\pgfqpoint{1.693512in}{1.565035in}}%
\pgfpathlineto{\pgfqpoint{1.700134in}{1.566122in}}%
\pgfpathlineto{\pgfqpoint{1.700853in}{1.566308in}}%
\pgfpathlineto{\pgfqpoint{1.701158in}{1.566308in}}%
\pgfpathlineto{\pgfqpoint{1.706951in}{1.567394in}}%
\pgfpathlineto{\pgfqpoint{1.708061in}{1.567643in}}%
\pgfpathlineto{\pgfqpoint{1.713283in}{1.568729in}}%
\pgfpathlineto{\pgfqpoint{1.714284in}{1.569040in}}%
\pgfpathlineto{\pgfqpoint{1.719506in}{1.570126in}}%
\pgfpathlineto{\pgfqpoint{1.720366in}{1.570281in}}%
\pgfpathlineto{\pgfqpoint{1.726581in}{1.571368in}}%
\pgfpathlineto{\pgfqpoint{1.727409in}{1.571740in}}%
\pgfpathlineto{\pgfqpoint{1.727613in}{1.571740in}}%
\pgfpathlineto{\pgfqpoint{1.734375in}{1.572827in}}%
\pgfpathlineto{\pgfqpoint{1.735469in}{1.573075in}}%
\pgfpathlineto{\pgfqpoint{1.741309in}{1.574161in}}%
\pgfpathlineto{\pgfqpoint{1.742372in}{1.574534in}}%
\pgfpathlineto{\pgfqpoint{1.748752in}{1.575620in}}%
\pgfpathlineto{\pgfqpoint{1.749682in}{1.575931in}}%
\pgfpathlineto{\pgfqpoint{1.755858in}{1.577017in}}%
\pgfpathlineto{\pgfqpoint{1.756905in}{1.577173in}}%
\pgfpathlineto{\pgfqpoint{1.762933in}{1.578259in}}%
\pgfpathlineto{\pgfqpoint{1.763316in}{1.578383in}}%
\pgfpathlineto{\pgfqpoint{1.763973in}{1.578383in}}%
\pgfpathlineto{\pgfqpoint{1.769367in}{1.579470in}}%
\pgfpathlineto{\pgfqpoint{1.770446in}{1.579687in}}%
\pgfpathlineto{\pgfqpoint{1.777575in}{1.580773in}}%
\pgfpathlineto{\pgfqpoint{1.778529in}{1.580929in}}%
\pgfpathlineto{\pgfqpoint{1.786018in}{1.582015in}}%
\pgfpathlineto{\pgfqpoint{1.786425in}{1.582170in}}%
\pgfpathlineto{\pgfqpoint{1.787089in}{1.582170in}}%
\pgfpathlineto{\pgfqpoint{1.794805in}{1.583257in}}%
\pgfpathlineto{\pgfqpoint{1.795876in}{1.583412in}}%
\pgfpathlineto{\pgfqpoint{1.803561in}{1.584498in}}%
\pgfpathlineto{\pgfqpoint{1.804593in}{1.584716in}}%
\pgfpathlineto{\pgfqpoint{1.815585in}{1.585802in}}%
\pgfpathlineto{\pgfqpoint{1.816687in}{1.585989in}}%
\pgfpathlineto{\pgfqpoint{1.825529in}{1.587075in}}%
\pgfpathlineto{\pgfqpoint{1.826303in}{1.587168in}}%
\pgfpathlineto{\pgfqpoint{1.835981in}{1.588255in}}%
\pgfpathlineto{\pgfqpoint{1.837052in}{1.588410in}}%
\pgfpathlineto{\pgfqpoint{1.846105in}{1.589496in}}%
\pgfpathlineto{\pgfqpoint{1.847066in}{1.589589in}}%
\pgfpathlineto{\pgfqpoint{1.854626in}{1.590676in}}%
\pgfpathlineto{\pgfqpoint{1.855728in}{1.590769in}}%
\pgfpathlineto{\pgfqpoint{1.866782in}{1.591855in}}%
\pgfpathlineto{\pgfqpoint{1.867564in}{1.591980in}}%
\pgfpathlineto{\pgfqpoint{1.880190in}{1.593066in}}%
\pgfpathlineto{\pgfqpoint{1.881222in}{1.593159in}}%
\pgfpathlineto{\pgfqpoint{1.891940in}{1.594246in}}%
\pgfpathlineto{\pgfqpoint{1.892416in}{1.594339in}}%
\pgfpathlineto{\pgfqpoint{1.892823in}{1.594339in}}%
\pgfpathlineto{\pgfqpoint{1.907708in}{1.595425in}}%
\pgfpathlineto{\pgfqpoint{1.908787in}{1.595549in}}%
\pgfpathlineto{\pgfqpoint{1.924469in}{1.596636in}}%
\pgfpathlineto{\pgfqpoint{1.925305in}{1.596760in}}%
\pgfpathlineto{\pgfqpoint{1.925438in}{1.596760in}}%
\pgfpathlineto{\pgfqpoint{1.947414in}{1.597847in}}%
\pgfpathlineto{\pgfqpoint{1.947992in}{1.598033in}}%
\pgfpathlineto{\pgfqpoint{1.948375in}{1.598033in}}%
\pgfpathlineto{\pgfqpoint{1.964104in}{1.599119in}}%
\pgfpathlineto{\pgfqpoint{1.964221in}{1.599212in}}%
\pgfpathlineto{\pgfqpoint{1.965214in}{1.599212in}}%
\pgfpathlineto{\pgfqpoint{1.989472in}{1.600299in}}%
\pgfpathlineto{\pgfqpoint{1.990145in}{1.600423in}}%
\pgfpathlineto{\pgfqpoint{1.990285in}{1.600423in}}%
\pgfpathlineto{\pgfqpoint{2.016256in}{1.601510in}}%
\pgfpathlineto{\pgfqpoint{2.016357in}{1.601572in}}%
\pgfpathlineto{\pgfqpoint{2.016967in}{1.601572in}}%
\pgfpathlineto{\pgfqpoint{2.033126in}{1.601944in}}%
\pgfpathlineto{\pgfqpoint{2.033126in}{1.601944in}}%
\pgfusepath{stroke}%
\end{pgfscope}%
\begin{pgfscope}%
\pgfsetrectcap%
\pgfsetmiterjoin%
\pgfsetlinewidth{0.803000pt}%
\definecolor{currentstroke}{rgb}{0.000000,0.000000,0.000000}%
\pgfsetstrokecolor{currentstroke}%
\pgfsetdash{}{0pt}%
\pgfpathmoveto{\pgfqpoint{0.553581in}{0.499444in}}%
\pgfpathlineto{\pgfqpoint{0.553581in}{1.654444in}}%
\pgfusepath{stroke}%
\end{pgfscope}%
\begin{pgfscope}%
\pgfsetrectcap%
\pgfsetmiterjoin%
\pgfsetlinewidth{0.803000pt}%
\definecolor{currentstroke}{rgb}{0.000000,0.000000,0.000000}%
\pgfsetstrokecolor{currentstroke}%
\pgfsetdash{}{0pt}%
\pgfpathmoveto{\pgfqpoint{2.103581in}{0.499444in}}%
\pgfpathlineto{\pgfqpoint{2.103581in}{1.654444in}}%
\pgfusepath{stroke}%
\end{pgfscope}%
\begin{pgfscope}%
\pgfsetrectcap%
\pgfsetmiterjoin%
\pgfsetlinewidth{0.803000pt}%
\definecolor{currentstroke}{rgb}{0.000000,0.000000,0.000000}%
\pgfsetstrokecolor{currentstroke}%
\pgfsetdash{}{0pt}%
\pgfpathmoveto{\pgfqpoint{0.553581in}{0.499444in}}%
\pgfpathlineto{\pgfqpoint{2.103581in}{0.499444in}}%
\pgfusepath{stroke}%
\end{pgfscope}%
\begin{pgfscope}%
\pgfsetrectcap%
\pgfsetmiterjoin%
\pgfsetlinewidth{0.803000pt}%
\definecolor{currentstroke}{rgb}{0.000000,0.000000,0.000000}%
\pgfsetstrokecolor{currentstroke}%
\pgfsetdash{}{0pt}%
\pgfpathmoveto{\pgfqpoint{0.553581in}{1.654444in}}%
\pgfpathlineto{\pgfqpoint{2.103581in}{1.654444in}}%
\pgfusepath{stroke}%
\end{pgfscope}%
\begin{pgfscope}%
\pgfsetbuttcap%
\pgfsetmiterjoin%
\definecolor{currentfill}{rgb}{1.000000,1.000000,1.000000}%
\pgfsetfillcolor{currentfill}%
\pgfsetfillopacity{0.800000}%
\pgfsetlinewidth{1.003750pt}%
\definecolor{currentstroke}{rgb}{0.800000,0.800000,0.800000}%
\pgfsetstrokecolor{currentstroke}%
\pgfsetstrokeopacity{0.800000}%
\pgfsetdash{}{0pt}%
\pgfpathmoveto{\pgfqpoint{0.832747in}{0.568889in}}%
\pgfpathlineto{\pgfqpoint{2.006358in}{0.568889in}}%
\pgfpathquadraticcurveto{\pgfqpoint{2.034136in}{0.568889in}}{\pgfqpoint{2.034136in}{0.596666in}}%
\pgfpathlineto{\pgfqpoint{2.034136in}{0.776388in}}%
\pgfpathquadraticcurveto{\pgfqpoint{2.034136in}{0.804166in}}{\pgfqpoint{2.006358in}{0.804166in}}%
\pgfpathlineto{\pgfqpoint{0.832747in}{0.804166in}}%
\pgfpathquadraticcurveto{\pgfqpoint{0.804970in}{0.804166in}}{\pgfqpoint{0.804970in}{0.776388in}}%
\pgfpathlineto{\pgfqpoint{0.804970in}{0.596666in}}%
\pgfpathquadraticcurveto{\pgfqpoint{0.804970in}{0.568889in}}{\pgfqpoint{0.832747in}{0.568889in}}%
\pgfpathlineto{\pgfqpoint{0.832747in}{0.568889in}}%
\pgfpathclose%
\pgfusepath{stroke,fill}%
\end{pgfscope}%
\begin{pgfscope}%
\pgfsetrectcap%
\pgfsetroundjoin%
\pgfsetlinewidth{1.505625pt}%
\definecolor{currentstroke}{rgb}{0.000000,0.000000,0.000000}%
\pgfsetstrokecolor{currentstroke}%
\pgfsetdash{}{0pt}%
\pgfpathmoveto{\pgfqpoint{0.860525in}{0.700000in}}%
\pgfpathlineto{\pgfqpoint{0.999414in}{0.700000in}}%
\pgfpathlineto{\pgfqpoint{1.138303in}{0.700000in}}%
\pgfusepath{stroke}%
\end{pgfscope}%
\begin{pgfscope}%
\definecolor{textcolor}{rgb}{0.000000,0.000000,0.000000}%
\pgfsetstrokecolor{textcolor}%
\pgfsetfillcolor{textcolor}%
\pgftext[x=1.249414in,y=0.651388in,left,base]{\color{textcolor}\rmfamily\fontsize{10.000000}{12.000000}\selectfont AUC=0.776}%
\end{pgfscope}%
\end{pgfpicture}%
\makeatother%
\endgroup%

	
\end{tabular}

\verb|LRC_Hard_Tomek_0_alpha_0_5_v1_Test|

\noindent\begin{tabular}{@{\hspace{-6pt}}p{4.3in} @{\hspace{-6pt}}p{2.0in}}
	\vskip 0pt
	\hfil Raw Model Output
	
	%% Creator: Matplotlib, PGF backend
%%
%% To include the figure in your LaTeX document, write
%%   \input{<filename>.pgf}
%%
%% Make sure the required packages are loaded in your preamble
%%   \usepackage{pgf}
%%
%% Also ensure that all the required font packages are loaded; for instance,
%% the lmodern package is sometimes necessary when using math font.
%%   \usepackage{lmodern}
%%
%% Figures using additional raster images can only be included by \input if
%% they are in the same directory as the main LaTeX file. For loading figures
%% from other directories you can use the `import` package
%%   \usepackage{import}
%%
%% and then include the figures with
%%   \import{<path to file>}{<filename>.pgf}
%%
%% Matplotlib used the following preamble
%%   
%%   \usepackage{fontspec}
%%   \makeatletter\@ifpackageloaded{underscore}{}{\usepackage[strings]{underscore}}\makeatother
%%
\begingroup%
\makeatletter%
\begin{pgfpicture}%
\pgfpathrectangle{\pgfpointorigin}{\pgfqpoint{4.191250in}{1.754444in}}%
\pgfusepath{use as bounding box, clip}%
\begin{pgfscope}%
\pgfsetbuttcap%
\pgfsetmiterjoin%
\definecolor{currentfill}{rgb}{1.000000,1.000000,1.000000}%
\pgfsetfillcolor{currentfill}%
\pgfsetlinewidth{0.000000pt}%
\definecolor{currentstroke}{rgb}{1.000000,1.000000,1.000000}%
\pgfsetstrokecolor{currentstroke}%
\pgfsetdash{}{0pt}%
\pgfpathmoveto{\pgfqpoint{0.000000in}{0.000000in}}%
\pgfpathlineto{\pgfqpoint{4.191250in}{0.000000in}}%
\pgfpathlineto{\pgfqpoint{4.191250in}{1.754444in}}%
\pgfpathlineto{\pgfqpoint{0.000000in}{1.754444in}}%
\pgfpathlineto{\pgfqpoint{0.000000in}{0.000000in}}%
\pgfpathclose%
\pgfusepath{fill}%
\end{pgfscope}%
\begin{pgfscope}%
\pgfsetbuttcap%
\pgfsetmiterjoin%
\definecolor{currentfill}{rgb}{1.000000,1.000000,1.000000}%
\pgfsetfillcolor{currentfill}%
\pgfsetlinewidth{0.000000pt}%
\definecolor{currentstroke}{rgb}{0.000000,0.000000,0.000000}%
\pgfsetstrokecolor{currentstroke}%
\pgfsetstrokeopacity{0.000000}%
\pgfsetdash{}{0pt}%
\pgfpathmoveto{\pgfqpoint{0.515000in}{0.499444in}}%
\pgfpathlineto{\pgfqpoint{4.002500in}{0.499444in}}%
\pgfpathlineto{\pgfqpoint{4.002500in}{1.654444in}}%
\pgfpathlineto{\pgfqpoint{0.515000in}{1.654444in}}%
\pgfpathlineto{\pgfqpoint{0.515000in}{0.499444in}}%
\pgfpathclose%
\pgfusepath{fill}%
\end{pgfscope}%
\begin{pgfscope}%
\pgfpathrectangle{\pgfqpoint{0.515000in}{0.499444in}}{\pgfqpoint{3.487500in}{1.155000in}}%
\pgfusepath{clip}%
\pgfsetbuttcap%
\pgfsetmiterjoin%
\pgfsetlinewidth{1.003750pt}%
\definecolor{currentstroke}{rgb}{0.000000,0.000000,0.000000}%
\pgfsetstrokecolor{currentstroke}%
\pgfsetdash{}{0pt}%
\pgfpathmoveto{\pgfqpoint{0.505000in}{0.499444in}}%
\pgfpathlineto{\pgfqpoint{0.549530in}{0.499444in}}%
\pgfpathlineto{\pgfqpoint{0.549530in}{1.016392in}}%
\pgfpathlineto{\pgfqpoint{0.505000in}{1.016392in}}%
\pgfusepath{stroke}%
\end{pgfscope}%
\begin{pgfscope}%
\pgfpathrectangle{\pgfqpoint{0.515000in}{0.499444in}}{\pgfqpoint{3.487500in}{1.155000in}}%
\pgfusepath{clip}%
\pgfsetbuttcap%
\pgfsetmiterjoin%
\pgfsetlinewidth{1.003750pt}%
\definecolor{currentstroke}{rgb}{0.000000,0.000000,0.000000}%
\pgfsetstrokecolor{currentstroke}%
\pgfsetdash{}{0pt}%
\pgfpathmoveto{\pgfqpoint{0.632401in}{0.499444in}}%
\pgfpathlineto{\pgfqpoint{0.687649in}{0.499444in}}%
\pgfpathlineto{\pgfqpoint{0.687649in}{1.599444in}}%
\pgfpathlineto{\pgfqpoint{0.632401in}{1.599444in}}%
\pgfpathlineto{\pgfqpoint{0.632401in}{0.499444in}}%
\pgfpathclose%
\pgfusepath{stroke}%
\end{pgfscope}%
\begin{pgfscope}%
\pgfpathrectangle{\pgfqpoint{0.515000in}{0.499444in}}{\pgfqpoint{3.487500in}{1.155000in}}%
\pgfusepath{clip}%
\pgfsetbuttcap%
\pgfsetmiterjoin%
\pgfsetlinewidth{1.003750pt}%
\definecolor{currentstroke}{rgb}{0.000000,0.000000,0.000000}%
\pgfsetstrokecolor{currentstroke}%
\pgfsetdash{}{0pt}%
\pgfpathmoveto{\pgfqpoint{0.770520in}{0.499444in}}%
\pgfpathlineto{\pgfqpoint{0.825768in}{0.499444in}}%
\pgfpathlineto{\pgfqpoint{0.825768in}{1.378707in}}%
\pgfpathlineto{\pgfqpoint{0.770520in}{1.378707in}}%
\pgfpathlineto{\pgfqpoint{0.770520in}{0.499444in}}%
\pgfpathclose%
\pgfusepath{stroke}%
\end{pgfscope}%
\begin{pgfscope}%
\pgfpathrectangle{\pgfqpoint{0.515000in}{0.499444in}}{\pgfqpoint{3.487500in}{1.155000in}}%
\pgfusepath{clip}%
\pgfsetbuttcap%
\pgfsetmiterjoin%
\pgfsetlinewidth{1.003750pt}%
\definecolor{currentstroke}{rgb}{0.000000,0.000000,0.000000}%
\pgfsetstrokecolor{currentstroke}%
\pgfsetdash{}{0pt}%
\pgfpathmoveto{\pgfqpoint{0.908639in}{0.499444in}}%
\pgfpathlineto{\pgfqpoint{0.963886in}{0.499444in}}%
\pgfpathlineto{\pgfqpoint{0.963886in}{1.097454in}}%
\pgfpathlineto{\pgfqpoint{0.908639in}{1.097454in}}%
\pgfpathlineto{\pgfqpoint{0.908639in}{0.499444in}}%
\pgfpathclose%
\pgfusepath{stroke}%
\end{pgfscope}%
\begin{pgfscope}%
\pgfpathrectangle{\pgfqpoint{0.515000in}{0.499444in}}{\pgfqpoint{3.487500in}{1.155000in}}%
\pgfusepath{clip}%
\pgfsetbuttcap%
\pgfsetmiterjoin%
\pgfsetlinewidth{1.003750pt}%
\definecolor{currentstroke}{rgb}{0.000000,0.000000,0.000000}%
\pgfsetstrokecolor{currentstroke}%
\pgfsetdash{}{0pt}%
\pgfpathmoveto{\pgfqpoint{1.046758in}{0.499444in}}%
\pgfpathlineto{\pgfqpoint{1.102005in}{0.499444in}}%
\pgfpathlineto{\pgfqpoint{1.102005in}{0.901290in}}%
\pgfpathlineto{\pgfqpoint{1.046758in}{0.901290in}}%
\pgfpathlineto{\pgfqpoint{1.046758in}{0.499444in}}%
\pgfpathclose%
\pgfusepath{stroke}%
\end{pgfscope}%
\begin{pgfscope}%
\pgfpathrectangle{\pgfqpoint{0.515000in}{0.499444in}}{\pgfqpoint{3.487500in}{1.155000in}}%
\pgfusepath{clip}%
\pgfsetbuttcap%
\pgfsetmiterjoin%
\pgfsetlinewidth{1.003750pt}%
\definecolor{currentstroke}{rgb}{0.000000,0.000000,0.000000}%
\pgfsetstrokecolor{currentstroke}%
\pgfsetdash{}{0pt}%
\pgfpathmoveto{\pgfqpoint{1.184877in}{0.499444in}}%
\pgfpathlineto{\pgfqpoint{1.240124in}{0.499444in}}%
\pgfpathlineto{\pgfqpoint{1.240124in}{0.771961in}}%
\pgfpathlineto{\pgfqpoint{1.184877in}{0.771961in}}%
\pgfpathlineto{\pgfqpoint{1.184877in}{0.499444in}}%
\pgfpathclose%
\pgfusepath{stroke}%
\end{pgfscope}%
\begin{pgfscope}%
\pgfpathrectangle{\pgfqpoint{0.515000in}{0.499444in}}{\pgfqpoint{3.487500in}{1.155000in}}%
\pgfusepath{clip}%
\pgfsetbuttcap%
\pgfsetmiterjoin%
\pgfsetlinewidth{1.003750pt}%
\definecolor{currentstroke}{rgb}{0.000000,0.000000,0.000000}%
\pgfsetstrokecolor{currentstroke}%
\pgfsetdash{}{0pt}%
\pgfpathmoveto{\pgfqpoint{1.322995in}{0.499444in}}%
\pgfpathlineto{\pgfqpoint{1.378243in}{0.499444in}}%
\pgfpathlineto{\pgfqpoint{1.378243in}{0.683261in}}%
\pgfpathlineto{\pgfqpoint{1.322995in}{0.683261in}}%
\pgfpathlineto{\pgfqpoint{1.322995in}{0.499444in}}%
\pgfpathclose%
\pgfusepath{stroke}%
\end{pgfscope}%
\begin{pgfscope}%
\pgfpathrectangle{\pgfqpoint{0.515000in}{0.499444in}}{\pgfqpoint{3.487500in}{1.155000in}}%
\pgfusepath{clip}%
\pgfsetbuttcap%
\pgfsetmiterjoin%
\pgfsetlinewidth{1.003750pt}%
\definecolor{currentstroke}{rgb}{0.000000,0.000000,0.000000}%
\pgfsetstrokecolor{currentstroke}%
\pgfsetdash{}{0pt}%
\pgfpathmoveto{\pgfqpoint{1.461114in}{0.499444in}}%
\pgfpathlineto{\pgfqpoint{1.516362in}{0.499444in}}%
\pgfpathlineto{\pgfqpoint{1.516362in}{0.626821in}}%
\pgfpathlineto{\pgfqpoint{1.461114in}{0.626821in}}%
\pgfpathlineto{\pgfqpoint{1.461114in}{0.499444in}}%
\pgfpathclose%
\pgfusepath{stroke}%
\end{pgfscope}%
\begin{pgfscope}%
\pgfpathrectangle{\pgfqpoint{0.515000in}{0.499444in}}{\pgfqpoint{3.487500in}{1.155000in}}%
\pgfusepath{clip}%
\pgfsetbuttcap%
\pgfsetmiterjoin%
\pgfsetlinewidth{1.003750pt}%
\definecolor{currentstroke}{rgb}{0.000000,0.000000,0.000000}%
\pgfsetstrokecolor{currentstroke}%
\pgfsetdash{}{0pt}%
\pgfpathmoveto{\pgfqpoint{1.599233in}{0.499444in}}%
\pgfpathlineto{\pgfqpoint{1.654480in}{0.499444in}}%
\pgfpathlineto{\pgfqpoint{1.654480in}{0.588925in}}%
\pgfpathlineto{\pgfqpoint{1.599233in}{0.588925in}}%
\pgfpathlineto{\pgfqpoint{1.599233in}{0.499444in}}%
\pgfpathclose%
\pgfusepath{stroke}%
\end{pgfscope}%
\begin{pgfscope}%
\pgfpathrectangle{\pgfqpoint{0.515000in}{0.499444in}}{\pgfqpoint{3.487500in}{1.155000in}}%
\pgfusepath{clip}%
\pgfsetbuttcap%
\pgfsetmiterjoin%
\pgfsetlinewidth{1.003750pt}%
\definecolor{currentstroke}{rgb}{0.000000,0.000000,0.000000}%
\pgfsetstrokecolor{currentstroke}%
\pgfsetdash{}{0pt}%
\pgfpathmoveto{\pgfqpoint{1.737352in}{0.499444in}}%
\pgfpathlineto{\pgfqpoint{1.792599in}{0.499444in}}%
\pgfpathlineto{\pgfqpoint{1.792599in}{0.563986in}}%
\pgfpathlineto{\pgfqpoint{1.737352in}{0.563986in}}%
\pgfpathlineto{\pgfqpoint{1.737352in}{0.499444in}}%
\pgfpathclose%
\pgfusepath{stroke}%
\end{pgfscope}%
\begin{pgfscope}%
\pgfpathrectangle{\pgfqpoint{0.515000in}{0.499444in}}{\pgfqpoint{3.487500in}{1.155000in}}%
\pgfusepath{clip}%
\pgfsetbuttcap%
\pgfsetmiterjoin%
\pgfsetlinewidth{1.003750pt}%
\definecolor{currentstroke}{rgb}{0.000000,0.000000,0.000000}%
\pgfsetstrokecolor{currentstroke}%
\pgfsetdash{}{0pt}%
\pgfpathmoveto{\pgfqpoint{1.875471in}{0.499444in}}%
\pgfpathlineto{\pgfqpoint{1.930718in}{0.499444in}}%
\pgfpathlineto{\pgfqpoint{1.930718in}{0.546734in}}%
\pgfpathlineto{\pgfqpoint{1.875471in}{0.546734in}}%
\pgfpathlineto{\pgfqpoint{1.875471in}{0.499444in}}%
\pgfpathclose%
\pgfusepath{stroke}%
\end{pgfscope}%
\begin{pgfscope}%
\pgfpathrectangle{\pgfqpoint{0.515000in}{0.499444in}}{\pgfqpoint{3.487500in}{1.155000in}}%
\pgfusepath{clip}%
\pgfsetbuttcap%
\pgfsetmiterjoin%
\pgfsetlinewidth{1.003750pt}%
\definecolor{currentstroke}{rgb}{0.000000,0.000000,0.000000}%
\pgfsetstrokecolor{currentstroke}%
\pgfsetdash{}{0pt}%
\pgfpathmoveto{\pgfqpoint{2.013589in}{0.499444in}}%
\pgfpathlineto{\pgfqpoint{2.068837in}{0.499444in}}%
\pgfpathlineto{\pgfqpoint{2.068837in}{0.533606in}}%
\pgfpathlineto{\pgfqpoint{2.013589in}{0.533606in}}%
\pgfpathlineto{\pgfqpoint{2.013589in}{0.499444in}}%
\pgfpathclose%
\pgfusepath{stroke}%
\end{pgfscope}%
\begin{pgfscope}%
\pgfpathrectangle{\pgfqpoint{0.515000in}{0.499444in}}{\pgfqpoint{3.487500in}{1.155000in}}%
\pgfusepath{clip}%
\pgfsetbuttcap%
\pgfsetmiterjoin%
\pgfsetlinewidth{1.003750pt}%
\definecolor{currentstroke}{rgb}{0.000000,0.000000,0.000000}%
\pgfsetstrokecolor{currentstroke}%
\pgfsetdash{}{0pt}%
\pgfpathmoveto{\pgfqpoint{2.151708in}{0.499444in}}%
\pgfpathlineto{\pgfqpoint{2.206956in}{0.499444in}}%
\pgfpathlineto{\pgfqpoint{2.206956in}{0.523382in}}%
\pgfpathlineto{\pgfqpoint{2.151708in}{0.523382in}}%
\pgfpathlineto{\pgfqpoint{2.151708in}{0.499444in}}%
\pgfpathclose%
\pgfusepath{stroke}%
\end{pgfscope}%
\begin{pgfscope}%
\pgfpathrectangle{\pgfqpoint{0.515000in}{0.499444in}}{\pgfqpoint{3.487500in}{1.155000in}}%
\pgfusepath{clip}%
\pgfsetbuttcap%
\pgfsetmiterjoin%
\pgfsetlinewidth{1.003750pt}%
\definecolor{currentstroke}{rgb}{0.000000,0.000000,0.000000}%
\pgfsetstrokecolor{currentstroke}%
\pgfsetdash{}{0pt}%
\pgfpathmoveto{\pgfqpoint{2.289827in}{0.499444in}}%
\pgfpathlineto{\pgfqpoint{2.345075in}{0.499444in}}%
\pgfpathlineto{\pgfqpoint{2.345075in}{0.517331in}}%
\pgfpathlineto{\pgfqpoint{2.289827in}{0.517331in}}%
\pgfpathlineto{\pgfqpoint{2.289827in}{0.499444in}}%
\pgfpathclose%
\pgfusepath{stroke}%
\end{pgfscope}%
\begin{pgfscope}%
\pgfpathrectangle{\pgfqpoint{0.515000in}{0.499444in}}{\pgfqpoint{3.487500in}{1.155000in}}%
\pgfusepath{clip}%
\pgfsetbuttcap%
\pgfsetmiterjoin%
\pgfsetlinewidth{1.003750pt}%
\definecolor{currentstroke}{rgb}{0.000000,0.000000,0.000000}%
\pgfsetstrokecolor{currentstroke}%
\pgfsetdash{}{0pt}%
\pgfpathmoveto{\pgfqpoint{2.427946in}{0.499444in}}%
\pgfpathlineto{\pgfqpoint{2.483193in}{0.499444in}}%
\pgfpathlineto{\pgfqpoint{2.483193in}{0.512816in}}%
\pgfpathlineto{\pgfqpoint{2.427946in}{0.512816in}}%
\pgfpathlineto{\pgfqpoint{2.427946in}{0.499444in}}%
\pgfpathclose%
\pgfusepath{stroke}%
\end{pgfscope}%
\begin{pgfscope}%
\pgfpathrectangle{\pgfqpoint{0.515000in}{0.499444in}}{\pgfqpoint{3.487500in}{1.155000in}}%
\pgfusepath{clip}%
\pgfsetbuttcap%
\pgfsetmiterjoin%
\pgfsetlinewidth{1.003750pt}%
\definecolor{currentstroke}{rgb}{0.000000,0.000000,0.000000}%
\pgfsetstrokecolor{currentstroke}%
\pgfsetdash{}{0pt}%
\pgfpathmoveto{\pgfqpoint{2.566065in}{0.499444in}}%
\pgfpathlineto{\pgfqpoint{2.621312in}{0.499444in}}%
\pgfpathlineto{\pgfqpoint{2.621312in}{0.509400in}}%
\pgfpathlineto{\pgfqpoint{2.566065in}{0.509400in}}%
\pgfpathlineto{\pgfqpoint{2.566065in}{0.499444in}}%
\pgfpathclose%
\pgfusepath{stroke}%
\end{pgfscope}%
\begin{pgfscope}%
\pgfpathrectangle{\pgfqpoint{0.515000in}{0.499444in}}{\pgfqpoint{3.487500in}{1.155000in}}%
\pgfusepath{clip}%
\pgfsetbuttcap%
\pgfsetmiterjoin%
\pgfsetlinewidth{1.003750pt}%
\definecolor{currentstroke}{rgb}{0.000000,0.000000,0.000000}%
\pgfsetstrokecolor{currentstroke}%
\pgfsetdash{}{0pt}%
\pgfpathmoveto{\pgfqpoint{2.704183in}{0.499444in}}%
\pgfpathlineto{\pgfqpoint{2.759431in}{0.499444in}}%
\pgfpathlineto{\pgfqpoint{2.759431in}{0.505984in}}%
\pgfpathlineto{\pgfqpoint{2.704183in}{0.505984in}}%
\pgfpathlineto{\pgfqpoint{2.704183in}{0.499444in}}%
\pgfpathclose%
\pgfusepath{stroke}%
\end{pgfscope}%
\begin{pgfscope}%
\pgfpathrectangle{\pgfqpoint{0.515000in}{0.499444in}}{\pgfqpoint{3.487500in}{1.155000in}}%
\pgfusepath{clip}%
\pgfsetbuttcap%
\pgfsetmiterjoin%
\pgfsetlinewidth{1.003750pt}%
\definecolor{currentstroke}{rgb}{0.000000,0.000000,0.000000}%
\pgfsetstrokecolor{currentstroke}%
\pgfsetdash{}{0pt}%
\pgfpathmoveto{\pgfqpoint{2.842302in}{0.499444in}}%
\pgfpathlineto{\pgfqpoint{2.897550in}{0.499444in}}%
\pgfpathlineto{\pgfqpoint{2.897550in}{0.503739in}}%
\pgfpathlineto{\pgfqpoint{2.842302in}{0.503739in}}%
\pgfpathlineto{\pgfqpoint{2.842302in}{0.499444in}}%
\pgfpathclose%
\pgfusepath{stroke}%
\end{pgfscope}%
\begin{pgfscope}%
\pgfpathrectangle{\pgfqpoint{0.515000in}{0.499444in}}{\pgfqpoint{3.487500in}{1.155000in}}%
\pgfusepath{clip}%
\pgfsetbuttcap%
\pgfsetmiterjoin%
\pgfsetlinewidth{1.003750pt}%
\definecolor{currentstroke}{rgb}{0.000000,0.000000,0.000000}%
\pgfsetstrokecolor{currentstroke}%
\pgfsetdash{}{0pt}%
\pgfpathmoveto{\pgfqpoint{2.980421in}{0.499444in}}%
\pgfpathlineto{\pgfqpoint{3.035669in}{0.499444in}}%
\pgfpathlineto{\pgfqpoint{3.035669in}{0.502665in}}%
\pgfpathlineto{\pgfqpoint{2.980421in}{0.502665in}}%
\pgfpathlineto{\pgfqpoint{2.980421in}{0.499444in}}%
\pgfpathclose%
\pgfusepath{stroke}%
\end{pgfscope}%
\begin{pgfscope}%
\pgfpathrectangle{\pgfqpoint{0.515000in}{0.499444in}}{\pgfqpoint{3.487500in}{1.155000in}}%
\pgfusepath{clip}%
\pgfsetbuttcap%
\pgfsetmiterjoin%
\pgfsetlinewidth{1.003750pt}%
\definecolor{currentstroke}{rgb}{0.000000,0.000000,0.000000}%
\pgfsetstrokecolor{currentstroke}%
\pgfsetdash{}{0pt}%
\pgfpathmoveto{\pgfqpoint{3.118540in}{0.499444in}}%
\pgfpathlineto{\pgfqpoint{3.173787in}{0.499444in}}%
\pgfpathlineto{\pgfqpoint{3.173787in}{0.501494in}}%
\pgfpathlineto{\pgfqpoint{3.118540in}{0.501494in}}%
\pgfpathlineto{\pgfqpoint{3.118540in}{0.499444in}}%
\pgfpathclose%
\pgfusepath{stroke}%
\end{pgfscope}%
\begin{pgfscope}%
\pgfpathrectangle{\pgfqpoint{0.515000in}{0.499444in}}{\pgfqpoint{3.487500in}{1.155000in}}%
\pgfusepath{clip}%
\pgfsetbuttcap%
\pgfsetmiterjoin%
\pgfsetlinewidth{1.003750pt}%
\definecolor{currentstroke}{rgb}{0.000000,0.000000,0.000000}%
\pgfsetstrokecolor{currentstroke}%
\pgfsetdash{}{0pt}%
\pgfpathmoveto{\pgfqpoint{3.256659in}{0.499444in}}%
\pgfpathlineto{\pgfqpoint{3.311906in}{0.499444in}}%
\pgfpathlineto{\pgfqpoint{3.311906in}{0.500640in}}%
\pgfpathlineto{\pgfqpoint{3.256659in}{0.500640in}}%
\pgfpathlineto{\pgfqpoint{3.256659in}{0.499444in}}%
\pgfpathclose%
\pgfusepath{stroke}%
\end{pgfscope}%
\begin{pgfscope}%
\pgfpathrectangle{\pgfqpoint{0.515000in}{0.499444in}}{\pgfqpoint{3.487500in}{1.155000in}}%
\pgfusepath{clip}%
\pgfsetbuttcap%
\pgfsetmiterjoin%
\pgfsetlinewidth{1.003750pt}%
\definecolor{currentstroke}{rgb}{0.000000,0.000000,0.000000}%
\pgfsetstrokecolor{currentstroke}%
\pgfsetdash{}{0pt}%
\pgfpathmoveto{\pgfqpoint{3.394778in}{0.499444in}}%
\pgfpathlineto{\pgfqpoint{3.450025in}{0.499444in}}%
\pgfpathlineto{\pgfqpoint{3.450025in}{0.499981in}}%
\pgfpathlineto{\pgfqpoint{3.394778in}{0.499981in}}%
\pgfpathlineto{\pgfqpoint{3.394778in}{0.499444in}}%
\pgfpathclose%
\pgfusepath{stroke}%
\end{pgfscope}%
\begin{pgfscope}%
\pgfpathrectangle{\pgfqpoint{0.515000in}{0.499444in}}{\pgfqpoint{3.487500in}{1.155000in}}%
\pgfusepath{clip}%
\pgfsetbuttcap%
\pgfsetmiterjoin%
\pgfsetlinewidth{1.003750pt}%
\definecolor{currentstroke}{rgb}{0.000000,0.000000,0.000000}%
\pgfsetstrokecolor{currentstroke}%
\pgfsetdash{}{0pt}%
\pgfpathmoveto{\pgfqpoint{3.532896in}{0.499444in}}%
\pgfpathlineto{\pgfqpoint{3.588144in}{0.499444in}}%
\pgfpathlineto{\pgfqpoint{3.588144in}{0.499469in}}%
\pgfpathlineto{\pgfqpoint{3.532896in}{0.499469in}}%
\pgfpathlineto{\pgfqpoint{3.532896in}{0.499444in}}%
\pgfpathclose%
\pgfusepath{stroke}%
\end{pgfscope}%
\begin{pgfscope}%
\pgfpathrectangle{\pgfqpoint{0.515000in}{0.499444in}}{\pgfqpoint{3.487500in}{1.155000in}}%
\pgfusepath{clip}%
\pgfsetbuttcap%
\pgfsetmiterjoin%
\pgfsetlinewidth{1.003750pt}%
\definecolor{currentstroke}{rgb}{0.000000,0.000000,0.000000}%
\pgfsetstrokecolor{currentstroke}%
\pgfsetdash{}{0pt}%
\pgfpathmoveto{\pgfqpoint{3.671015in}{0.499444in}}%
\pgfpathlineto{\pgfqpoint{3.726263in}{0.499444in}}%
\pgfpathlineto{\pgfqpoint{3.726263in}{0.499444in}}%
\pgfpathlineto{\pgfqpoint{3.671015in}{0.499444in}}%
\pgfpathlineto{\pgfqpoint{3.671015in}{0.499444in}}%
\pgfpathclose%
\pgfusepath{stroke}%
\end{pgfscope}%
\begin{pgfscope}%
\pgfpathrectangle{\pgfqpoint{0.515000in}{0.499444in}}{\pgfqpoint{3.487500in}{1.155000in}}%
\pgfusepath{clip}%
\pgfsetbuttcap%
\pgfsetmiterjoin%
\pgfsetlinewidth{1.003750pt}%
\definecolor{currentstroke}{rgb}{0.000000,0.000000,0.000000}%
\pgfsetstrokecolor{currentstroke}%
\pgfsetdash{}{0pt}%
\pgfpathmoveto{\pgfqpoint{3.809134in}{0.499444in}}%
\pgfpathlineto{\pgfqpoint{3.864381in}{0.499444in}}%
\pgfpathlineto{\pgfqpoint{3.864381in}{0.499444in}}%
\pgfpathlineto{\pgfqpoint{3.809134in}{0.499444in}}%
\pgfpathlineto{\pgfqpoint{3.809134in}{0.499444in}}%
\pgfpathclose%
\pgfusepath{stroke}%
\end{pgfscope}%
\begin{pgfscope}%
\pgfpathrectangle{\pgfqpoint{0.515000in}{0.499444in}}{\pgfqpoint{3.487500in}{1.155000in}}%
\pgfusepath{clip}%
\pgfsetbuttcap%
\pgfsetmiterjoin%
\definecolor{currentfill}{rgb}{0.000000,0.000000,0.000000}%
\pgfsetfillcolor{currentfill}%
\pgfsetlinewidth{0.000000pt}%
\definecolor{currentstroke}{rgb}{0.000000,0.000000,0.000000}%
\pgfsetstrokecolor{currentstroke}%
\pgfsetstrokeopacity{0.000000}%
\pgfsetdash{}{0pt}%
\pgfpathmoveto{\pgfqpoint{0.549530in}{0.499444in}}%
\pgfpathlineto{\pgfqpoint{0.604778in}{0.499444in}}%
\pgfpathlineto{\pgfqpoint{0.604778in}{0.511377in}}%
\pgfpathlineto{\pgfqpoint{0.549530in}{0.511377in}}%
\pgfpathlineto{\pgfqpoint{0.549530in}{0.499444in}}%
\pgfpathclose%
\pgfusepath{fill}%
\end{pgfscope}%
\begin{pgfscope}%
\pgfpathrectangle{\pgfqpoint{0.515000in}{0.499444in}}{\pgfqpoint{3.487500in}{1.155000in}}%
\pgfusepath{clip}%
\pgfsetbuttcap%
\pgfsetmiterjoin%
\definecolor{currentfill}{rgb}{0.000000,0.000000,0.000000}%
\pgfsetfillcolor{currentfill}%
\pgfsetlinewidth{0.000000pt}%
\definecolor{currentstroke}{rgb}{0.000000,0.000000,0.000000}%
\pgfsetstrokecolor{currentstroke}%
\pgfsetstrokeopacity{0.000000}%
\pgfsetdash{}{0pt}%
\pgfpathmoveto{\pgfqpoint{0.687649in}{0.499444in}}%
\pgfpathlineto{\pgfqpoint{0.742896in}{0.499444in}}%
\pgfpathlineto{\pgfqpoint{0.742896in}{0.564352in}}%
\pgfpathlineto{\pgfqpoint{0.687649in}{0.564352in}}%
\pgfpathlineto{\pgfqpoint{0.687649in}{0.499444in}}%
\pgfpathclose%
\pgfusepath{fill}%
\end{pgfscope}%
\begin{pgfscope}%
\pgfpathrectangle{\pgfqpoint{0.515000in}{0.499444in}}{\pgfqpoint{3.487500in}{1.155000in}}%
\pgfusepath{clip}%
\pgfsetbuttcap%
\pgfsetmiterjoin%
\definecolor{currentfill}{rgb}{0.000000,0.000000,0.000000}%
\pgfsetfillcolor{currentfill}%
\pgfsetlinewidth{0.000000pt}%
\definecolor{currentstroke}{rgb}{0.000000,0.000000,0.000000}%
\pgfsetstrokecolor{currentstroke}%
\pgfsetstrokeopacity{0.000000}%
\pgfsetdash{}{0pt}%
\pgfpathmoveto{\pgfqpoint{0.825768in}{0.499444in}}%
\pgfpathlineto{\pgfqpoint{0.881015in}{0.499444in}}%
\pgfpathlineto{\pgfqpoint{0.881015in}{0.598637in}}%
\pgfpathlineto{\pgfqpoint{0.825768in}{0.598637in}}%
\pgfpathlineto{\pgfqpoint{0.825768in}{0.499444in}}%
\pgfpathclose%
\pgfusepath{fill}%
\end{pgfscope}%
\begin{pgfscope}%
\pgfpathrectangle{\pgfqpoint{0.515000in}{0.499444in}}{\pgfqpoint{3.487500in}{1.155000in}}%
\pgfusepath{clip}%
\pgfsetbuttcap%
\pgfsetmiterjoin%
\definecolor{currentfill}{rgb}{0.000000,0.000000,0.000000}%
\pgfsetfillcolor{currentfill}%
\pgfsetlinewidth{0.000000pt}%
\definecolor{currentstroke}{rgb}{0.000000,0.000000,0.000000}%
\pgfsetstrokecolor{currentstroke}%
\pgfsetstrokeopacity{0.000000}%
\pgfsetdash{}{0pt}%
\pgfpathmoveto{\pgfqpoint{0.963886in}{0.499444in}}%
\pgfpathlineto{\pgfqpoint{1.019134in}{0.499444in}}%
\pgfpathlineto{\pgfqpoint{1.019134in}{0.600857in}}%
\pgfpathlineto{\pgfqpoint{0.963886in}{0.600857in}}%
\pgfpathlineto{\pgfqpoint{0.963886in}{0.499444in}}%
\pgfpathclose%
\pgfusepath{fill}%
\end{pgfscope}%
\begin{pgfscope}%
\pgfpathrectangle{\pgfqpoint{0.515000in}{0.499444in}}{\pgfqpoint{3.487500in}{1.155000in}}%
\pgfusepath{clip}%
\pgfsetbuttcap%
\pgfsetmiterjoin%
\definecolor{currentfill}{rgb}{0.000000,0.000000,0.000000}%
\pgfsetfillcolor{currentfill}%
\pgfsetlinewidth{0.000000pt}%
\definecolor{currentstroke}{rgb}{0.000000,0.000000,0.000000}%
\pgfsetstrokecolor{currentstroke}%
\pgfsetstrokeopacity{0.000000}%
\pgfsetdash{}{0pt}%
\pgfpathmoveto{\pgfqpoint{1.102005in}{0.499444in}}%
\pgfpathlineto{\pgfqpoint{1.157253in}{0.499444in}}%
\pgfpathlineto{\pgfqpoint{1.157253in}{0.590999in}}%
\pgfpathlineto{\pgfqpoint{1.102005in}{0.590999in}}%
\pgfpathlineto{\pgfqpoint{1.102005in}{0.499444in}}%
\pgfpathclose%
\pgfusepath{fill}%
\end{pgfscope}%
\begin{pgfscope}%
\pgfpathrectangle{\pgfqpoint{0.515000in}{0.499444in}}{\pgfqpoint{3.487500in}{1.155000in}}%
\pgfusepath{clip}%
\pgfsetbuttcap%
\pgfsetmiterjoin%
\definecolor{currentfill}{rgb}{0.000000,0.000000,0.000000}%
\pgfsetfillcolor{currentfill}%
\pgfsetlinewidth{0.000000pt}%
\definecolor{currentstroke}{rgb}{0.000000,0.000000,0.000000}%
\pgfsetstrokecolor{currentstroke}%
\pgfsetstrokeopacity{0.000000}%
\pgfsetdash{}{0pt}%
\pgfpathmoveto{\pgfqpoint{1.240124in}{0.499444in}}%
\pgfpathlineto{\pgfqpoint{1.295372in}{0.499444in}}%
\pgfpathlineto{\pgfqpoint{1.295372in}{0.576968in}}%
\pgfpathlineto{\pgfqpoint{1.240124in}{0.576968in}}%
\pgfpathlineto{\pgfqpoint{1.240124in}{0.499444in}}%
\pgfpathclose%
\pgfusepath{fill}%
\end{pgfscope}%
\begin{pgfscope}%
\pgfpathrectangle{\pgfqpoint{0.515000in}{0.499444in}}{\pgfqpoint{3.487500in}{1.155000in}}%
\pgfusepath{clip}%
\pgfsetbuttcap%
\pgfsetmiterjoin%
\definecolor{currentfill}{rgb}{0.000000,0.000000,0.000000}%
\pgfsetfillcolor{currentfill}%
\pgfsetlinewidth{0.000000pt}%
\definecolor{currentstroke}{rgb}{0.000000,0.000000,0.000000}%
\pgfsetstrokecolor{currentstroke}%
\pgfsetstrokeopacity{0.000000}%
\pgfsetdash{}{0pt}%
\pgfpathmoveto{\pgfqpoint{1.378243in}{0.499444in}}%
\pgfpathlineto{\pgfqpoint{1.433490in}{0.499444in}}%
\pgfpathlineto{\pgfqpoint{1.433490in}{0.565060in}}%
\pgfpathlineto{\pgfqpoint{1.378243in}{0.565060in}}%
\pgfpathlineto{\pgfqpoint{1.378243in}{0.499444in}}%
\pgfpathclose%
\pgfusepath{fill}%
\end{pgfscope}%
\begin{pgfscope}%
\pgfpathrectangle{\pgfqpoint{0.515000in}{0.499444in}}{\pgfqpoint{3.487500in}{1.155000in}}%
\pgfusepath{clip}%
\pgfsetbuttcap%
\pgfsetmiterjoin%
\definecolor{currentfill}{rgb}{0.000000,0.000000,0.000000}%
\pgfsetfillcolor{currentfill}%
\pgfsetlinewidth{0.000000pt}%
\definecolor{currentstroke}{rgb}{0.000000,0.000000,0.000000}%
\pgfsetstrokecolor{currentstroke}%
\pgfsetstrokeopacity{0.000000}%
\pgfsetdash{}{0pt}%
\pgfpathmoveto{\pgfqpoint{1.516362in}{0.499444in}}%
\pgfpathlineto{\pgfqpoint{1.571609in}{0.499444in}}%
\pgfpathlineto{\pgfqpoint{1.571609in}{0.556983in}}%
\pgfpathlineto{\pgfqpoint{1.516362in}{0.556983in}}%
\pgfpathlineto{\pgfqpoint{1.516362in}{0.499444in}}%
\pgfpathclose%
\pgfusepath{fill}%
\end{pgfscope}%
\begin{pgfscope}%
\pgfpathrectangle{\pgfqpoint{0.515000in}{0.499444in}}{\pgfqpoint{3.487500in}{1.155000in}}%
\pgfusepath{clip}%
\pgfsetbuttcap%
\pgfsetmiterjoin%
\definecolor{currentfill}{rgb}{0.000000,0.000000,0.000000}%
\pgfsetfillcolor{currentfill}%
\pgfsetlinewidth{0.000000pt}%
\definecolor{currentstroke}{rgb}{0.000000,0.000000,0.000000}%
\pgfsetstrokecolor{currentstroke}%
\pgfsetstrokeopacity{0.000000}%
\pgfsetdash{}{0pt}%
\pgfpathmoveto{\pgfqpoint{1.654480in}{0.499444in}}%
\pgfpathlineto{\pgfqpoint{1.709728in}{0.499444in}}%
\pgfpathlineto{\pgfqpoint{1.709728in}{0.548150in}}%
\pgfpathlineto{\pgfqpoint{1.654480in}{0.548150in}}%
\pgfpathlineto{\pgfqpoint{1.654480in}{0.499444in}}%
\pgfpathclose%
\pgfusepath{fill}%
\end{pgfscope}%
\begin{pgfscope}%
\pgfpathrectangle{\pgfqpoint{0.515000in}{0.499444in}}{\pgfqpoint{3.487500in}{1.155000in}}%
\pgfusepath{clip}%
\pgfsetbuttcap%
\pgfsetmiterjoin%
\definecolor{currentfill}{rgb}{0.000000,0.000000,0.000000}%
\pgfsetfillcolor{currentfill}%
\pgfsetlinewidth{0.000000pt}%
\definecolor{currentstroke}{rgb}{0.000000,0.000000,0.000000}%
\pgfsetstrokecolor{currentstroke}%
\pgfsetstrokeopacity{0.000000}%
\pgfsetdash{}{0pt}%
\pgfpathmoveto{\pgfqpoint{1.792599in}{0.499444in}}%
\pgfpathlineto{\pgfqpoint{1.847847in}{0.499444in}}%
\pgfpathlineto{\pgfqpoint{1.847847in}{0.541269in}}%
\pgfpathlineto{\pgfqpoint{1.792599in}{0.541269in}}%
\pgfpathlineto{\pgfqpoint{1.792599in}{0.499444in}}%
\pgfpathclose%
\pgfusepath{fill}%
\end{pgfscope}%
\begin{pgfscope}%
\pgfpathrectangle{\pgfqpoint{0.515000in}{0.499444in}}{\pgfqpoint{3.487500in}{1.155000in}}%
\pgfusepath{clip}%
\pgfsetbuttcap%
\pgfsetmiterjoin%
\definecolor{currentfill}{rgb}{0.000000,0.000000,0.000000}%
\pgfsetfillcolor{currentfill}%
\pgfsetlinewidth{0.000000pt}%
\definecolor{currentstroke}{rgb}{0.000000,0.000000,0.000000}%
\pgfsetstrokecolor{currentstroke}%
\pgfsetstrokeopacity{0.000000}%
\pgfsetdash{}{0pt}%
\pgfpathmoveto{\pgfqpoint{1.930718in}{0.499444in}}%
\pgfpathlineto{\pgfqpoint{1.985966in}{0.499444in}}%
\pgfpathlineto{\pgfqpoint{1.985966in}{0.533606in}}%
\pgfpathlineto{\pgfqpoint{1.930718in}{0.533606in}}%
\pgfpathlineto{\pgfqpoint{1.930718in}{0.499444in}}%
\pgfpathclose%
\pgfusepath{fill}%
\end{pgfscope}%
\begin{pgfscope}%
\pgfpathrectangle{\pgfqpoint{0.515000in}{0.499444in}}{\pgfqpoint{3.487500in}{1.155000in}}%
\pgfusepath{clip}%
\pgfsetbuttcap%
\pgfsetmiterjoin%
\definecolor{currentfill}{rgb}{0.000000,0.000000,0.000000}%
\pgfsetfillcolor{currentfill}%
\pgfsetlinewidth{0.000000pt}%
\definecolor{currentstroke}{rgb}{0.000000,0.000000,0.000000}%
\pgfsetstrokecolor{currentstroke}%
\pgfsetstrokeopacity{0.000000}%
\pgfsetdash{}{0pt}%
\pgfpathmoveto{\pgfqpoint{2.068837in}{0.499444in}}%
\pgfpathlineto{\pgfqpoint{2.124084in}{0.499444in}}%
\pgfpathlineto{\pgfqpoint{2.124084in}{0.528433in}}%
\pgfpathlineto{\pgfqpoint{2.068837in}{0.528433in}}%
\pgfpathlineto{\pgfqpoint{2.068837in}{0.499444in}}%
\pgfpathclose%
\pgfusepath{fill}%
\end{pgfscope}%
\begin{pgfscope}%
\pgfpathrectangle{\pgfqpoint{0.515000in}{0.499444in}}{\pgfqpoint{3.487500in}{1.155000in}}%
\pgfusepath{clip}%
\pgfsetbuttcap%
\pgfsetmiterjoin%
\definecolor{currentfill}{rgb}{0.000000,0.000000,0.000000}%
\pgfsetfillcolor{currentfill}%
\pgfsetlinewidth{0.000000pt}%
\definecolor{currentstroke}{rgb}{0.000000,0.000000,0.000000}%
\pgfsetstrokecolor{currentstroke}%
\pgfsetstrokeopacity{0.000000}%
\pgfsetdash{}{0pt}%
\pgfpathmoveto{\pgfqpoint{2.206956in}{0.499444in}}%
\pgfpathlineto{\pgfqpoint{2.262203in}{0.499444in}}%
\pgfpathlineto{\pgfqpoint{2.262203in}{0.524627in}}%
\pgfpathlineto{\pgfqpoint{2.206956in}{0.524627in}}%
\pgfpathlineto{\pgfqpoint{2.206956in}{0.499444in}}%
\pgfpathclose%
\pgfusepath{fill}%
\end{pgfscope}%
\begin{pgfscope}%
\pgfpathrectangle{\pgfqpoint{0.515000in}{0.499444in}}{\pgfqpoint{3.487500in}{1.155000in}}%
\pgfusepath{clip}%
\pgfsetbuttcap%
\pgfsetmiterjoin%
\definecolor{currentfill}{rgb}{0.000000,0.000000,0.000000}%
\pgfsetfillcolor{currentfill}%
\pgfsetlinewidth{0.000000pt}%
\definecolor{currentstroke}{rgb}{0.000000,0.000000,0.000000}%
\pgfsetstrokecolor{currentstroke}%
\pgfsetstrokeopacity{0.000000}%
\pgfsetdash{}{0pt}%
\pgfpathmoveto{\pgfqpoint{2.345075in}{0.499444in}}%
\pgfpathlineto{\pgfqpoint{2.400322in}{0.499444in}}%
\pgfpathlineto{\pgfqpoint{2.400322in}{0.518502in}}%
\pgfpathlineto{\pgfqpoint{2.345075in}{0.518502in}}%
\pgfpathlineto{\pgfqpoint{2.345075in}{0.499444in}}%
\pgfpathclose%
\pgfusepath{fill}%
\end{pgfscope}%
\begin{pgfscope}%
\pgfpathrectangle{\pgfqpoint{0.515000in}{0.499444in}}{\pgfqpoint{3.487500in}{1.155000in}}%
\pgfusepath{clip}%
\pgfsetbuttcap%
\pgfsetmiterjoin%
\definecolor{currentfill}{rgb}{0.000000,0.000000,0.000000}%
\pgfsetfillcolor{currentfill}%
\pgfsetlinewidth{0.000000pt}%
\definecolor{currentstroke}{rgb}{0.000000,0.000000,0.000000}%
\pgfsetstrokecolor{currentstroke}%
\pgfsetstrokeopacity{0.000000}%
\pgfsetdash{}{0pt}%
\pgfpathmoveto{\pgfqpoint{2.483193in}{0.499444in}}%
\pgfpathlineto{\pgfqpoint{2.538441in}{0.499444in}}%
\pgfpathlineto{\pgfqpoint{2.538441in}{0.514402in}}%
\pgfpathlineto{\pgfqpoint{2.483193in}{0.514402in}}%
\pgfpathlineto{\pgfqpoint{2.483193in}{0.499444in}}%
\pgfpathclose%
\pgfusepath{fill}%
\end{pgfscope}%
\begin{pgfscope}%
\pgfpathrectangle{\pgfqpoint{0.515000in}{0.499444in}}{\pgfqpoint{3.487500in}{1.155000in}}%
\pgfusepath{clip}%
\pgfsetbuttcap%
\pgfsetmiterjoin%
\definecolor{currentfill}{rgb}{0.000000,0.000000,0.000000}%
\pgfsetfillcolor{currentfill}%
\pgfsetlinewidth{0.000000pt}%
\definecolor{currentstroke}{rgb}{0.000000,0.000000,0.000000}%
\pgfsetstrokecolor{currentstroke}%
\pgfsetstrokeopacity{0.000000}%
\pgfsetdash{}{0pt}%
\pgfpathmoveto{\pgfqpoint{2.621312in}{0.499444in}}%
\pgfpathlineto{\pgfqpoint{2.676560in}{0.499444in}}%
\pgfpathlineto{\pgfqpoint{2.676560in}{0.510547in}}%
\pgfpathlineto{\pgfqpoint{2.621312in}{0.510547in}}%
\pgfpathlineto{\pgfqpoint{2.621312in}{0.499444in}}%
\pgfpathclose%
\pgfusepath{fill}%
\end{pgfscope}%
\begin{pgfscope}%
\pgfpathrectangle{\pgfqpoint{0.515000in}{0.499444in}}{\pgfqpoint{3.487500in}{1.155000in}}%
\pgfusepath{clip}%
\pgfsetbuttcap%
\pgfsetmiterjoin%
\definecolor{currentfill}{rgb}{0.000000,0.000000,0.000000}%
\pgfsetfillcolor{currentfill}%
\pgfsetlinewidth{0.000000pt}%
\definecolor{currentstroke}{rgb}{0.000000,0.000000,0.000000}%
\pgfsetstrokecolor{currentstroke}%
\pgfsetstrokeopacity{0.000000}%
\pgfsetdash{}{0pt}%
\pgfpathmoveto{\pgfqpoint{2.759431in}{0.499444in}}%
\pgfpathlineto{\pgfqpoint{2.814678in}{0.499444in}}%
\pgfpathlineto{\pgfqpoint{2.814678in}{0.508473in}}%
\pgfpathlineto{\pgfqpoint{2.759431in}{0.508473in}}%
\pgfpathlineto{\pgfqpoint{2.759431in}{0.499444in}}%
\pgfpathclose%
\pgfusepath{fill}%
\end{pgfscope}%
\begin{pgfscope}%
\pgfpathrectangle{\pgfqpoint{0.515000in}{0.499444in}}{\pgfqpoint{3.487500in}{1.155000in}}%
\pgfusepath{clip}%
\pgfsetbuttcap%
\pgfsetmiterjoin%
\definecolor{currentfill}{rgb}{0.000000,0.000000,0.000000}%
\pgfsetfillcolor{currentfill}%
\pgfsetlinewidth{0.000000pt}%
\definecolor{currentstroke}{rgb}{0.000000,0.000000,0.000000}%
\pgfsetstrokecolor{currentstroke}%
\pgfsetstrokeopacity{0.000000}%
\pgfsetdash{}{0pt}%
\pgfpathmoveto{\pgfqpoint{2.897550in}{0.499444in}}%
\pgfpathlineto{\pgfqpoint{2.952797in}{0.499444in}}%
\pgfpathlineto{\pgfqpoint{2.952797in}{0.506594in}}%
\pgfpathlineto{\pgfqpoint{2.897550in}{0.506594in}}%
\pgfpathlineto{\pgfqpoint{2.897550in}{0.499444in}}%
\pgfpathclose%
\pgfusepath{fill}%
\end{pgfscope}%
\begin{pgfscope}%
\pgfpathrectangle{\pgfqpoint{0.515000in}{0.499444in}}{\pgfqpoint{3.487500in}{1.155000in}}%
\pgfusepath{clip}%
\pgfsetbuttcap%
\pgfsetmiterjoin%
\definecolor{currentfill}{rgb}{0.000000,0.000000,0.000000}%
\pgfsetfillcolor{currentfill}%
\pgfsetlinewidth{0.000000pt}%
\definecolor{currentstroke}{rgb}{0.000000,0.000000,0.000000}%
\pgfsetstrokecolor{currentstroke}%
\pgfsetstrokeopacity{0.000000}%
\pgfsetdash{}{0pt}%
\pgfpathmoveto{\pgfqpoint{3.035669in}{0.499444in}}%
\pgfpathlineto{\pgfqpoint{3.090916in}{0.499444in}}%
\pgfpathlineto{\pgfqpoint{3.090916in}{0.504837in}}%
\pgfpathlineto{\pgfqpoint{3.035669in}{0.504837in}}%
\pgfpathlineto{\pgfqpoint{3.035669in}{0.499444in}}%
\pgfpathclose%
\pgfusepath{fill}%
\end{pgfscope}%
\begin{pgfscope}%
\pgfpathrectangle{\pgfqpoint{0.515000in}{0.499444in}}{\pgfqpoint{3.487500in}{1.155000in}}%
\pgfusepath{clip}%
\pgfsetbuttcap%
\pgfsetmiterjoin%
\definecolor{currentfill}{rgb}{0.000000,0.000000,0.000000}%
\pgfsetfillcolor{currentfill}%
\pgfsetlinewidth{0.000000pt}%
\definecolor{currentstroke}{rgb}{0.000000,0.000000,0.000000}%
\pgfsetstrokecolor{currentstroke}%
\pgfsetstrokeopacity{0.000000}%
\pgfsetdash{}{0pt}%
\pgfpathmoveto{\pgfqpoint{3.173787in}{0.499444in}}%
\pgfpathlineto{\pgfqpoint{3.229035in}{0.499444in}}%
\pgfpathlineto{\pgfqpoint{3.229035in}{0.504520in}}%
\pgfpathlineto{\pgfqpoint{3.173787in}{0.504520in}}%
\pgfpathlineto{\pgfqpoint{3.173787in}{0.499444in}}%
\pgfpathclose%
\pgfusepath{fill}%
\end{pgfscope}%
\begin{pgfscope}%
\pgfpathrectangle{\pgfqpoint{0.515000in}{0.499444in}}{\pgfqpoint{3.487500in}{1.155000in}}%
\pgfusepath{clip}%
\pgfsetbuttcap%
\pgfsetmiterjoin%
\definecolor{currentfill}{rgb}{0.000000,0.000000,0.000000}%
\pgfsetfillcolor{currentfill}%
\pgfsetlinewidth{0.000000pt}%
\definecolor{currentstroke}{rgb}{0.000000,0.000000,0.000000}%
\pgfsetstrokecolor{currentstroke}%
\pgfsetstrokeopacity{0.000000}%
\pgfsetdash{}{0pt}%
\pgfpathmoveto{\pgfqpoint{3.311906in}{0.499444in}}%
\pgfpathlineto{\pgfqpoint{3.367154in}{0.499444in}}%
\pgfpathlineto{\pgfqpoint{3.367154in}{0.502543in}}%
\pgfpathlineto{\pgfqpoint{3.311906in}{0.502543in}}%
\pgfpathlineto{\pgfqpoint{3.311906in}{0.499444in}}%
\pgfpathclose%
\pgfusepath{fill}%
\end{pgfscope}%
\begin{pgfscope}%
\pgfpathrectangle{\pgfqpoint{0.515000in}{0.499444in}}{\pgfqpoint{3.487500in}{1.155000in}}%
\pgfusepath{clip}%
\pgfsetbuttcap%
\pgfsetmiterjoin%
\definecolor{currentfill}{rgb}{0.000000,0.000000,0.000000}%
\pgfsetfillcolor{currentfill}%
\pgfsetlinewidth{0.000000pt}%
\definecolor{currentstroke}{rgb}{0.000000,0.000000,0.000000}%
\pgfsetstrokecolor{currentstroke}%
\pgfsetstrokeopacity{0.000000}%
\pgfsetdash{}{0pt}%
\pgfpathmoveto{\pgfqpoint{3.450025in}{0.499444in}}%
\pgfpathlineto{\pgfqpoint{3.505273in}{0.499444in}}%
\pgfpathlineto{\pgfqpoint{3.505273in}{0.501103in}}%
\pgfpathlineto{\pgfqpoint{3.450025in}{0.501103in}}%
\pgfpathlineto{\pgfqpoint{3.450025in}{0.499444in}}%
\pgfpathclose%
\pgfusepath{fill}%
\end{pgfscope}%
\begin{pgfscope}%
\pgfpathrectangle{\pgfqpoint{0.515000in}{0.499444in}}{\pgfqpoint{3.487500in}{1.155000in}}%
\pgfusepath{clip}%
\pgfsetbuttcap%
\pgfsetmiterjoin%
\definecolor{currentfill}{rgb}{0.000000,0.000000,0.000000}%
\pgfsetfillcolor{currentfill}%
\pgfsetlinewidth{0.000000pt}%
\definecolor{currentstroke}{rgb}{0.000000,0.000000,0.000000}%
\pgfsetstrokecolor{currentstroke}%
\pgfsetstrokeopacity{0.000000}%
\pgfsetdash{}{0pt}%
\pgfpathmoveto{\pgfqpoint{3.588144in}{0.499444in}}%
\pgfpathlineto{\pgfqpoint{3.643391in}{0.499444in}}%
\pgfpathlineto{\pgfqpoint{3.643391in}{0.499761in}}%
\pgfpathlineto{\pgfqpoint{3.588144in}{0.499761in}}%
\pgfpathlineto{\pgfqpoint{3.588144in}{0.499444in}}%
\pgfpathclose%
\pgfusepath{fill}%
\end{pgfscope}%
\begin{pgfscope}%
\pgfpathrectangle{\pgfqpoint{0.515000in}{0.499444in}}{\pgfqpoint{3.487500in}{1.155000in}}%
\pgfusepath{clip}%
\pgfsetbuttcap%
\pgfsetmiterjoin%
\definecolor{currentfill}{rgb}{0.000000,0.000000,0.000000}%
\pgfsetfillcolor{currentfill}%
\pgfsetlinewidth{0.000000pt}%
\definecolor{currentstroke}{rgb}{0.000000,0.000000,0.000000}%
\pgfsetstrokecolor{currentstroke}%
\pgfsetstrokeopacity{0.000000}%
\pgfsetdash{}{0pt}%
\pgfpathmoveto{\pgfqpoint{3.726263in}{0.499444in}}%
\pgfpathlineto{\pgfqpoint{3.781510in}{0.499444in}}%
\pgfpathlineto{\pgfqpoint{3.781510in}{0.499444in}}%
\pgfpathlineto{\pgfqpoint{3.726263in}{0.499444in}}%
\pgfpathlineto{\pgfqpoint{3.726263in}{0.499444in}}%
\pgfpathclose%
\pgfusepath{fill}%
\end{pgfscope}%
\begin{pgfscope}%
\pgfpathrectangle{\pgfqpoint{0.515000in}{0.499444in}}{\pgfqpoint{3.487500in}{1.155000in}}%
\pgfusepath{clip}%
\pgfsetbuttcap%
\pgfsetmiterjoin%
\definecolor{currentfill}{rgb}{0.000000,0.000000,0.000000}%
\pgfsetfillcolor{currentfill}%
\pgfsetlinewidth{0.000000pt}%
\definecolor{currentstroke}{rgb}{0.000000,0.000000,0.000000}%
\pgfsetstrokecolor{currentstroke}%
\pgfsetstrokeopacity{0.000000}%
\pgfsetdash{}{0pt}%
\pgfpathmoveto{\pgfqpoint{3.864381in}{0.499444in}}%
\pgfpathlineto{\pgfqpoint{3.919629in}{0.499444in}}%
\pgfpathlineto{\pgfqpoint{3.919629in}{0.499444in}}%
\pgfpathlineto{\pgfqpoint{3.864381in}{0.499444in}}%
\pgfpathlineto{\pgfqpoint{3.864381in}{0.499444in}}%
\pgfpathclose%
\pgfusepath{fill}%
\end{pgfscope}%
\begin{pgfscope}%
\pgfsetbuttcap%
\pgfsetroundjoin%
\definecolor{currentfill}{rgb}{0.000000,0.000000,0.000000}%
\pgfsetfillcolor{currentfill}%
\pgfsetlinewidth{0.803000pt}%
\definecolor{currentstroke}{rgb}{0.000000,0.000000,0.000000}%
\pgfsetstrokecolor{currentstroke}%
\pgfsetdash{}{0pt}%
\pgfsys@defobject{currentmarker}{\pgfqpoint{0.000000in}{-0.048611in}}{\pgfqpoint{0.000000in}{0.000000in}}{%
\pgfpathmoveto{\pgfqpoint{0.000000in}{0.000000in}}%
\pgfpathlineto{\pgfqpoint{0.000000in}{-0.048611in}}%
\pgfusepath{stroke,fill}%
}%
\begin{pgfscope}%
\pgfsys@transformshift{0.549530in}{0.499444in}%
\pgfsys@useobject{currentmarker}{}%
\end{pgfscope}%
\end{pgfscope}%
\begin{pgfscope}%
\definecolor{textcolor}{rgb}{0.000000,0.000000,0.000000}%
\pgfsetstrokecolor{textcolor}%
\pgfsetfillcolor{textcolor}%
\pgftext[x=0.549530in,y=0.402222in,,top]{\color{textcolor}\rmfamily\fontsize{10.000000}{12.000000}\selectfont 0.0}%
\end{pgfscope}%
\begin{pgfscope}%
\pgfsetbuttcap%
\pgfsetroundjoin%
\definecolor{currentfill}{rgb}{0.000000,0.000000,0.000000}%
\pgfsetfillcolor{currentfill}%
\pgfsetlinewidth{0.803000pt}%
\definecolor{currentstroke}{rgb}{0.000000,0.000000,0.000000}%
\pgfsetstrokecolor{currentstroke}%
\pgfsetdash{}{0pt}%
\pgfsys@defobject{currentmarker}{\pgfqpoint{0.000000in}{-0.048611in}}{\pgfqpoint{0.000000in}{0.000000in}}{%
\pgfpathmoveto{\pgfqpoint{0.000000in}{0.000000in}}%
\pgfpathlineto{\pgfqpoint{0.000000in}{-0.048611in}}%
\pgfusepath{stroke,fill}%
}%
\begin{pgfscope}%
\pgfsys@transformshift{0.894827in}{0.499444in}%
\pgfsys@useobject{currentmarker}{}%
\end{pgfscope}%
\end{pgfscope}%
\begin{pgfscope}%
\definecolor{textcolor}{rgb}{0.000000,0.000000,0.000000}%
\pgfsetstrokecolor{textcolor}%
\pgfsetfillcolor{textcolor}%
\pgftext[x=0.894827in,y=0.402222in,,top]{\color{textcolor}\rmfamily\fontsize{10.000000}{12.000000}\selectfont 0.1}%
\end{pgfscope}%
\begin{pgfscope}%
\pgfsetbuttcap%
\pgfsetroundjoin%
\definecolor{currentfill}{rgb}{0.000000,0.000000,0.000000}%
\pgfsetfillcolor{currentfill}%
\pgfsetlinewidth{0.803000pt}%
\definecolor{currentstroke}{rgb}{0.000000,0.000000,0.000000}%
\pgfsetstrokecolor{currentstroke}%
\pgfsetdash{}{0pt}%
\pgfsys@defobject{currentmarker}{\pgfqpoint{0.000000in}{-0.048611in}}{\pgfqpoint{0.000000in}{0.000000in}}{%
\pgfpathmoveto{\pgfqpoint{0.000000in}{0.000000in}}%
\pgfpathlineto{\pgfqpoint{0.000000in}{-0.048611in}}%
\pgfusepath{stroke,fill}%
}%
\begin{pgfscope}%
\pgfsys@transformshift{1.240124in}{0.499444in}%
\pgfsys@useobject{currentmarker}{}%
\end{pgfscope}%
\end{pgfscope}%
\begin{pgfscope}%
\definecolor{textcolor}{rgb}{0.000000,0.000000,0.000000}%
\pgfsetstrokecolor{textcolor}%
\pgfsetfillcolor{textcolor}%
\pgftext[x=1.240124in,y=0.402222in,,top]{\color{textcolor}\rmfamily\fontsize{10.000000}{12.000000}\selectfont 0.2}%
\end{pgfscope}%
\begin{pgfscope}%
\pgfsetbuttcap%
\pgfsetroundjoin%
\definecolor{currentfill}{rgb}{0.000000,0.000000,0.000000}%
\pgfsetfillcolor{currentfill}%
\pgfsetlinewidth{0.803000pt}%
\definecolor{currentstroke}{rgb}{0.000000,0.000000,0.000000}%
\pgfsetstrokecolor{currentstroke}%
\pgfsetdash{}{0pt}%
\pgfsys@defobject{currentmarker}{\pgfqpoint{0.000000in}{-0.048611in}}{\pgfqpoint{0.000000in}{0.000000in}}{%
\pgfpathmoveto{\pgfqpoint{0.000000in}{0.000000in}}%
\pgfpathlineto{\pgfqpoint{0.000000in}{-0.048611in}}%
\pgfusepath{stroke,fill}%
}%
\begin{pgfscope}%
\pgfsys@transformshift{1.585421in}{0.499444in}%
\pgfsys@useobject{currentmarker}{}%
\end{pgfscope}%
\end{pgfscope}%
\begin{pgfscope}%
\definecolor{textcolor}{rgb}{0.000000,0.000000,0.000000}%
\pgfsetstrokecolor{textcolor}%
\pgfsetfillcolor{textcolor}%
\pgftext[x=1.585421in,y=0.402222in,,top]{\color{textcolor}\rmfamily\fontsize{10.000000}{12.000000}\selectfont 0.3}%
\end{pgfscope}%
\begin{pgfscope}%
\pgfsetbuttcap%
\pgfsetroundjoin%
\definecolor{currentfill}{rgb}{0.000000,0.000000,0.000000}%
\pgfsetfillcolor{currentfill}%
\pgfsetlinewidth{0.803000pt}%
\definecolor{currentstroke}{rgb}{0.000000,0.000000,0.000000}%
\pgfsetstrokecolor{currentstroke}%
\pgfsetdash{}{0pt}%
\pgfsys@defobject{currentmarker}{\pgfqpoint{0.000000in}{-0.048611in}}{\pgfqpoint{0.000000in}{0.000000in}}{%
\pgfpathmoveto{\pgfqpoint{0.000000in}{0.000000in}}%
\pgfpathlineto{\pgfqpoint{0.000000in}{-0.048611in}}%
\pgfusepath{stroke,fill}%
}%
\begin{pgfscope}%
\pgfsys@transformshift{1.930718in}{0.499444in}%
\pgfsys@useobject{currentmarker}{}%
\end{pgfscope}%
\end{pgfscope}%
\begin{pgfscope}%
\definecolor{textcolor}{rgb}{0.000000,0.000000,0.000000}%
\pgfsetstrokecolor{textcolor}%
\pgfsetfillcolor{textcolor}%
\pgftext[x=1.930718in,y=0.402222in,,top]{\color{textcolor}\rmfamily\fontsize{10.000000}{12.000000}\selectfont 0.4}%
\end{pgfscope}%
\begin{pgfscope}%
\pgfsetbuttcap%
\pgfsetroundjoin%
\definecolor{currentfill}{rgb}{0.000000,0.000000,0.000000}%
\pgfsetfillcolor{currentfill}%
\pgfsetlinewidth{0.803000pt}%
\definecolor{currentstroke}{rgb}{0.000000,0.000000,0.000000}%
\pgfsetstrokecolor{currentstroke}%
\pgfsetdash{}{0pt}%
\pgfsys@defobject{currentmarker}{\pgfqpoint{0.000000in}{-0.048611in}}{\pgfqpoint{0.000000in}{0.000000in}}{%
\pgfpathmoveto{\pgfqpoint{0.000000in}{0.000000in}}%
\pgfpathlineto{\pgfqpoint{0.000000in}{-0.048611in}}%
\pgfusepath{stroke,fill}%
}%
\begin{pgfscope}%
\pgfsys@transformshift{2.276015in}{0.499444in}%
\pgfsys@useobject{currentmarker}{}%
\end{pgfscope}%
\end{pgfscope}%
\begin{pgfscope}%
\definecolor{textcolor}{rgb}{0.000000,0.000000,0.000000}%
\pgfsetstrokecolor{textcolor}%
\pgfsetfillcolor{textcolor}%
\pgftext[x=2.276015in,y=0.402222in,,top]{\color{textcolor}\rmfamily\fontsize{10.000000}{12.000000}\selectfont 0.5}%
\end{pgfscope}%
\begin{pgfscope}%
\pgfsetbuttcap%
\pgfsetroundjoin%
\definecolor{currentfill}{rgb}{0.000000,0.000000,0.000000}%
\pgfsetfillcolor{currentfill}%
\pgfsetlinewidth{0.803000pt}%
\definecolor{currentstroke}{rgb}{0.000000,0.000000,0.000000}%
\pgfsetstrokecolor{currentstroke}%
\pgfsetdash{}{0pt}%
\pgfsys@defobject{currentmarker}{\pgfqpoint{0.000000in}{-0.048611in}}{\pgfqpoint{0.000000in}{0.000000in}}{%
\pgfpathmoveto{\pgfqpoint{0.000000in}{0.000000in}}%
\pgfpathlineto{\pgfqpoint{0.000000in}{-0.048611in}}%
\pgfusepath{stroke,fill}%
}%
\begin{pgfscope}%
\pgfsys@transformshift{2.621312in}{0.499444in}%
\pgfsys@useobject{currentmarker}{}%
\end{pgfscope}%
\end{pgfscope}%
\begin{pgfscope}%
\definecolor{textcolor}{rgb}{0.000000,0.000000,0.000000}%
\pgfsetstrokecolor{textcolor}%
\pgfsetfillcolor{textcolor}%
\pgftext[x=2.621312in,y=0.402222in,,top]{\color{textcolor}\rmfamily\fontsize{10.000000}{12.000000}\selectfont 0.6}%
\end{pgfscope}%
\begin{pgfscope}%
\pgfsetbuttcap%
\pgfsetroundjoin%
\definecolor{currentfill}{rgb}{0.000000,0.000000,0.000000}%
\pgfsetfillcolor{currentfill}%
\pgfsetlinewidth{0.803000pt}%
\definecolor{currentstroke}{rgb}{0.000000,0.000000,0.000000}%
\pgfsetstrokecolor{currentstroke}%
\pgfsetdash{}{0pt}%
\pgfsys@defobject{currentmarker}{\pgfqpoint{0.000000in}{-0.048611in}}{\pgfqpoint{0.000000in}{0.000000in}}{%
\pgfpathmoveto{\pgfqpoint{0.000000in}{0.000000in}}%
\pgfpathlineto{\pgfqpoint{0.000000in}{-0.048611in}}%
\pgfusepath{stroke,fill}%
}%
\begin{pgfscope}%
\pgfsys@transformshift{2.966609in}{0.499444in}%
\pgfsys@useobject{currentmarker}{}%
\end{pgfscope}%
\end{pgfscope}%
\begin{pgfscope}%
\definecolor{textcolor}{rgb}{0.000000,0.000000,0.000000}%
\pgfsetstrokecolor{textcolor}%
\pgfsetfillcolor{textcolor}%
\pgftext[x=2.966609in,y=0.402222in,,top]{\color{textcolor}\rmfamily\fontsize{10.000000}{12.000000}\selectfont 0.7}%
\end{pgfscope}%
\begin{pgfscope}%
\pgfsetbuttcap%
\pgfsetroundjoin%
\definecolor{currentfill}{rgb}{0.000000,0.000000,0.000000}%
\pgfsetfillcolor{currentfill}%
\pgfsetlinewidth{0.803000pt}%
\definecolor{currentstroke}{rgb}{0.000000,0.000000,0.000000}%
\pgfsetstrokecolor{currentstroke}%
\pgfsetdash{}{0pt}%
\pgfsys@defobject{currentmarker}{\pgfqpoint{0.000000in}{-0.048611in}}{\pgfqpoint{0.000000in}{0.000000in}}{%
\pgfpathmoveto{\pgfqpoint{0.000000in}{0.000000in}}%
\pgfpathlineto{\pgfqpoint{0.000000in}{-0.048611in}}%
\pgfusepath{stroke,fill}%
}%
\begin{pgfscope}%
\pgfsys@transformshift{3.311906in}{0.499444in}%
\pgfsys@useobject{currentmarker}{}%
\end{pgfscope}%
\end{pgfscope}%
\begin{pgfscope}%
\definecolor{textcolor}{rgb}{0.000000,0.000000,0.000000}%
\pgfsetstrokecolor{textcolor}%
\pgfsetfillcolor{textcolor}%
\pgftext[x=3.311906in,y=0.402222in,,top]{\color{textcolor}\rmfamily\fontsize{10.000000}{12.000000}\selectfont 0.8}%
\end{pgfscope}%
\begin{pgfscope}%
\pgfsetbuttcap%
\pgfsetroundjoin%
\definecolor{currentfill}{rgb}{0.000000,0.000000,0.000000}%
\pgfsetfillcolor{currentfill}%
\pgfsetlinewidth{0.803000pt}%
\definecolor{currentstroke}{rgb}{0.000000,0.000000,0.000000}%
\pgfsetstrokecolor{currentstroke}%
\pgfsetdash{}{0pt}%
\pgfsys@defobject{currentmarker}{\pgfqpoint{0.000000in}{-0.048611in}}{\pgfqpoint{0.000000in}{0.000000in}}{%
\pgfpathmoveto{\pgfqpoint{0.000000in}{0.000000in}}%
\pgfpathlineto{\pgfqpoint{0.000000in}{-0.048611in}}%
\pgfusepath{stroke,fill}%
}%
\begin{pgfscope}%
\pgfsys@transformshift{3.657203in}{0.499444in}%
\pgfsys@useobject{currentmarker}{}%
\end{pgfscope}%
\end{pgfscope}%
\begin{pgfscope}%
\definecolor{textcolor}{rgb}{0.000000,0.000000,0.000000}%
\pgfsetstrokecolor{textcolor}%
\pgfsetfillcolor{textcolor}%
\pgftext[x=3.657203in,y=0.402222in,,top]{\color{textcolor}\rmfamily\fontsize{10.000000}{12.000000}\selectfont 0.9}%
\end{pgfscope}%
\begin{pgfscope}%
\pgfsetbuttcap%
\pgfsetroundjoin%
\definecolor{currentfill}{rgb}{0.000000,0.000000,0.000000}%
\pgfsetfillcolor{currentfill}%
\pgfsetlinewidth{0.803000pt}%
\definecolor{currentstroke}{rgb}{0.000000,0.000000,0.000000}%
\pgfsetstrokecolor{currentstroke}%
\pgfsetdash{}{0pt}%
\pgfsys@defobject{currentmarker}{\pgfqpoint{0.000000in}{-0.048611in}}{\pgfqpoint{0.000000in}{0.000000in}}{%
\pgfpathmoveto{\pgfqpoint{0.000000in}{0.000000in}}%
\pgfpathlineto{\pgfqpoint{0.000000in}{-0.048611in}}%
\pgfusepath{stroke,fill}%
}%
\begin{pgfscope}%
\pgfsys@transformshift{4.002500in}{0.499444in}%
\pgfsys@useobject{currentmarker}{}%
\end{pgfscope}%
\end{pgfscope}%
\begin{pgfscope}%
\definecolor{textcolor}{rgb}{0.000000,0.000000,0.000000}%
\pgfsetstrokecolor{textcolor}%
\pgfsetfillcolor{textcolor}%
\pgftext[x=4.002500in,y=0.402222in,,top]{\color{textcolor}\rmfamily\fontsize{10.000000}{12.000000}\selectfont 1.0}%
\end{pgfscope}%
\begin{pgfscope}%
\definecolor{textcolor}{rgb}{0.000000,0.000000,0.000000}%
\pgfsetstrokecolor{textcolor}%
\pgfsetfillcolor{textcolor}%
\pgftext[x=2.258750in,y=0.223333in,,top]{\color{textcolor}\rmfamily\fontsize{10.000000}{12.000000}\selectfont \(\displaystyle p\)}%
\end{pgfscope}%
\begin{pgfscope}%
\pgfsetbuttcap%
\pgfsetroundjoin%
\definecolor{currentfill}{rgb}{0.000000,0.000000,0.000000}%
\pgfsetfillcolor{currentfill}%
\pgfsetlinewidth{0.803000pt}%
\definecolor{currentstroke}{rgb}{0.000000,0.000000,0.000000}%
\pgfsetstrokecolor{currentstroke}%
\pgfsetdash{}{0pt}%
\pgfsys@defobject{currentmarker}{\pgfqpoint{-0.048611in}{0.000000in}}{\pgfqpoint{-0.000000in}{0.000000in}}{%
\pgfpathmoveto{\pgfqpoint{-0.000000in}{0.000000in}}%
\pgfpathlineto{\pgfqpoint{-0.048611in}{0.000000in}}%
\pgfusepath{stroke,fill}%
}%
\begin{pgfscope}%
\pgfsys@transformshift{0.515000in}{0.499444in}%
\pgfsys@useobject{currentmarker}{}%
\end{pgfscope}%
\end{pgfscope}%
\begin{pgfscope}%
\definecolor{textcolor}{rgb}{0.000000,0.000000,0.000000}%
\pgfsetstrokecolor{textcolor}%
\pgfsetfillcolor{textcolor}%
\pgftext[x=0.348333in, y=0.451250in, left, base]{\color{textcolor}\rmfamily\fontsize{10.000000}{12.000000}\selectfont \(\displaystyle {0}\)}%
\end{pgfscope}%
\begin{pgfscope}%
\pgfsetbuttcap%
\pgfsetroundjoin%
\definecolor{currentfill}{rgb}{0.000000,0.000000,0.000000}%
\pgfsetfillcolor{currentfill}%
\pgfsetlinewidth{0.803000pt}%
\definecolor{currentstroke}{rgb}{0.000000,0.000000,0.000000}%
\pgfsetstrokecolor{currentstroke}%
\pgfsetdash{}{0pt}%
\pgfsys@defobject{currentmarker}{\pgfqpoint{-0.048611in}{0.000000in}}{\pgfqpoint{-0.000000in}{0.000000in}}{%
\pgfpathmoveto{\pgfqpoint{-0.000000in}{0.000000in}}%
\pgfpathlineto{\pgfqpoint{-0.048611in}{0.000000in}}%
\pgfusepath{stroke,fill}%
}%
\begin{pgfscope}%
\pgfsys@transformshift{0.515000in}{1.021809in}%
\pgfsys@useobject{currentmarker}{}%
\end{pgfscope}%
\end{pgfscope}%
\begin{pgfscope}%
\definecolor{textcolor}{rgb}{0.000000,0.000000,0.000000}%
\pgfsetstrokecolor{textcolor}%
\pgfsetfillcolor{textcolor}%
\pgftext[x=0.278889in, y=0.973615in, left, base]{\color{textcolor}\rmfamily\fontsize{10.000000}{12.000000}\selectfont \(\displaystyle {10}\)}%
\end{pgfscope}%
\begin{pgfscope}%
\pgfsetbuttcap%
\pgfsetroundjoin%
\definecolor{currentfill}{rgb}{0.000000,0.000000,0.000000}%
\pgfsetfillcolor{currentfill}%
\pgfsetlinewidth{0.803000pt}%
\definecolor{currentstroke}{rgb}{0.000000,0.000000,0.000000}%
\pgfsetstrokecolor{currentstroke}%
\pgfsetdash{}{0pt}%
\pgfsys@defobject{currentmarker}{\pgfqpoint{-0.048611in}{0.000000in}}{\pgfqpoint{-0.000000in}{0.000000in}}{%
\pgfpathmoveto{\pgfqpoint{-0.000000in}{0.000000in}}%
\pgfpathlineto{\pgfqpoint{-0.048611in}{0.000000in}}%
\pgfusepath{stroke,fill}%
}%
\begin{pgfscope}%
\pgfsys@transformshift{0.515000in}{1.544175in}%
\pgfsys@useobject{currentmarker}{}%
\end{pgfscope}%
\end{pgfscope}%
\begin{pgfscope}%
\definecolor{textcolor}{rgb}{0.000000,0.000000,0.000000}%
\pgfsetstrokecolor{textcolor}%
\pgfsetfillcolor{textcolor}%
\pgftext[x=0.278889in, y=1.495980in, left, base]{\color{textcolor}\rmfamily\fontsize{10.000000}{12.000000}\selectfont \(\displaystyle {20}\)}%
\end{pgfscope}%
\begin{pgfscope}%
\definecolor{textcolor}{rgb}{0.000000,0.000000,0.000000}%
\pgfsetstrokecolor{textcolor}%
\pgfsetfillcolor{textcolor}%
\pgftext[x=0.223333in,y=1.076944in,,bottom,rotate=90.000000]{\color{textcolor}\rmfamily\fontsize{10.000000}{12.000000}\selectfont Percent of Data Set}%
\end{pgfscope}%
\begin{pgfscope}%
\pgfsetrectcap%
\pgfsetmiterjoin%
\pgfsetlinewidth{0.803000pt}%
\definecolor{currentstroke}{rgb}{0.000000,0.000000,0.000000}%
\pgfsetstrokecolor{currentstroke}%
\pgfsetdash{}{0pt}%
\pgfpathmoveto{\pgfqpoint{0.515000in}{0.499444in}}%
\pgfpathlineto{\pgfqpoint{0.515000in}{1.654444in}}%
\pgfusepath{stroke}%
\end{pgfscope}%
\begin{pgfscope}%
\pgfsetrectcap%
\pgfsetmiterjoin%
\pgfsetlinewidth{0.803000pt}%
\definecolor{currentstroke}{rgb}{0.000000,0.000000,0.000000}%
\pgfsetstrokecolor{currentstroke}%
\pgfsetdash{}{0pt}%
\pgfpathmoveto{\pgfqpoint{4.002500in}{0.499444in}}%
\pgfpathlineto{\pgfqpoint{4.002500in}{1.654444in}}%
\pgfusepath{stroke}%
\end{pgfscope}%
\begin{pgfscope}%
\pgfsetrectcap%
\pgfsetmiterjoin%
\pgfsetlinewidth{0.803000pt}%
\definecolor{currentstroke}{rgb}{0.000000,0.000000,0.000000}%
\pgfsetstrokecolor{currentstroke}%
\pgfsetdash{}{0pt}%
\pgfpathmoveto{\pgfqpoint{0.515000in}{0.499444in}}%
\pgfpathlineto{\pgfqpoint{4.002500in}{0.499444in}}%
\pgfusepath{stroke}%
\end{pgfscope}%
\begin{pgfscope}%
\pgfsetrectcap%
\pgfsetmiterjoin%
\pgfsetlinewidth{0.803000pt}%
\definecolor{currentstroke}{rgb}{0.000000,0.000000,0.000000}%
\pgfsetstrokecolor{currentstroke}%
\pgfsetdash{}{0pt}%
\pgfpathmoveto{\pgfqpoint{0.515000in}{1.654444in}}%
\pgfpathlineto{\pgfqpoint{4.002500in}{1.654444in}}%
\pgfusepath{stroke}%
\end{pgfscope}%
\begin{pgfscope}%
\pgfsetbuttcap%
\pgfsetmiterjoin%
\definecolor{currentfill}{rgb}{1.000000,1.000000,1.000000}%
\pgfsetfillcolor{currentfill}%
\pgfsetfillopacity{0.800000}%
\pgfsetlinewidth{1.003750pt}%
\definecolor{currentstroke}{rgb}{0.800000,0.800000,0.800000}%
\pgfsetstrokecolor{currentstroke}%
\pgfsetstrokeopacity{0.800000}%
\pgfsetdash{}{0pt}%
\pgfpathmoveto{\pgfqpoint{3.225556in}{1.154445in}}%
\pgfpathlineto{\pgfqpoint{3.905278in}{1.154445in}}%
\pgfpathquadraticcurveto{\pgfqpoint{3.933056in}{1.154445in}}{\pgfqpoint{3.933056in}{1.182222in}}%
\pgfpathlineto{\pgfqpoint{3.933056in}{1.557222in}}%
\pgfpathquadraticcurveto{\pgfqpoint{3.933056in}{1.585000in}}{\pgfqpoint{3.905278in}{1.585000in}}%
\pgfpathlineto{\pgfqpoint{3.225556in}{1.585000in}}%
\pgfpathquadraticcurveto{\pgfqpoint{3.197778in}{1.585000in}}{\pgfqpoint{3.197778in}{1.557222in}}%
\pgfpathlineto{\pgfqpoint{3.197778in}{1.182222in}}%
\pgfpathquadraticcurveto{\pgfqpoint{3.197778in}{1.154445in}}{\pgfqpoint{3.225556in}{1.154445in}}%
\pgfpathlineto{\pgfqpoint{3.225556in}{1.154445in}}%
\pgfpathclose%
\pgfusepath{stroke,fill}%
\end{pgfscope}%
\begin{pgfscope}%
\pgfsetbuttcap%
\pgfsetmiterjoin%
\pgfsetlinewidth{1.003750pt}%
\definecolor{currentstroke}{rgb}{0.000000,0.000000,0.000000}%
\pgfsetstrokecolor{currentstroke}%
\pgfsetdash{}{0pt}%
\pgfpathmoveto{\pgfqpoint{3.253334in}{1.432222in}}%
\pgfpathlineto{\pgfqpoint{3.531111in}{1.432222in}}%
\pgfpathlineto{\pgfqpoint{3.531111in}{1.529444in}}%
\pgfpathlineto{\pgfqpoint{3.253334in}{1.529444in}}%
\pgfpathlineto{\pgfqpoint{3.253334in}{1.432222in}}%
\pgfpathclose%
\pgfusepath{stroke}%
\end{pgfscope}%
\begin{pgfscope}%
\definecolor{textcolor}{rgb}{0.000000,0.000000,0.000000}%
\pgfsetstrokecolor{textcolor}%
\pgfsetfillcolor{textcolor}%
\pgftext[x=3.642223in,y=1.432222in,left,base]{\color{textcolor}\rmfamily\fontsize{10.000000}{12.000000}\selectfont Neg}%
\end{pgfscope}%
\begin{pgfscope}%
\pgfsetbuttcap%
\pgfsetmiterjoin%
\definecolor{currentfill}{rgb}{0.000000,0.000000,0.000000}%
\pgfsetfillcolor{currentfill}%
\pgfsetlinewidth{0.000000pt}%
\definecolor{currentstroke}{rgb}{0.000000,0.000000,0.000000}%
\pgfsetstrokecolor{currentstroke}%
\pgfsetstrokeopacity{0.000000}%
\pgfsetdash{}{0pt}%
\pgfpathmoveto{\pgfqpoint{3.253334in}{1.236944in}}%
\pgfpathlineto{\pgfqpoint{3.531111in}{1.236944in}}%
\pgfpathlineto{\pgfqpoint{3.531111in}{1.334167in}}%
\pgfpathlineto{\pgfqpoint{3.253334in}{1.334167in}}%
\pgfpathlineto{\pgfqpoint{3.253334in}{1.236944in}}%
\pgfpathclose%
\pgfusepath{fill}%
\end{pgfscope}%
\begin{pgfscope}%
\definecolor{textcolor}{rgb}{0.000000,0.000000,0.000000}%
\pgfsetstrokecolor{textcolor}%
\pgfsetfillcolor{textcolor}%
\pgftext[x=3.642223in,y=1.236944in,left,base]{\color{textcolor}\rmfamily\fontsize{10.000000}{12.000000}\selectfont Pos}%
\end{pgfscope}%
\end{pgfpicture}%
\makeatother%
\endgroup%
	
&
	\vskip 0pt
	\hfil ROC Curve
	
	%% Creator: Matplotlib, PGF backend
%%
%% To include the figure in your LaTeX document, write
%%   \input{<filename>.pgf}
%%
%% Make sure the required packages are loaded in your preamble
%%   \usepackage{pgf}
%%
%% Also ensure that all the required font packages are loaded; for instance,
%% the lmodern package is sometimes necessary when using math font.
%%   \usepackage{lmodern}
%%
%% Figures using additional raster images can only be included by \input if
%% they are in the same directory as the main LaTeX file. For loading figures
%% from other directories you can use the `import` package
%%   \usepackage{import}
%%
%% and then include the figures with
%%   \import{<path to file>}{<filename>.pgf}
%%
%% Matplotlib used the following preamble
%%   
%%   \usepackage{fontspec}
%%   \makeatletter\@ifpackageloaded{underscore}{}{\usepackage[strings]{underscore}}\makeatother
%%
\begingroup%
\makeatletter%
\begin{pgfpicture}%
\pgfpathrectangle{\pgfpointorigin}{\pgfqpoint{2.221861in}{1.754444in}}%
\pgfusepath{use as bounding box, clip}%
\begin{pgfscope}%
\pgfsetbuttcap%
\pgfsetmiterjoin%
\definecolor{currentfill}{rgb}{1.000000,1.000000,1.000000}%
\pgfsetfillcolor{currentfill}%
\pgfsetlinewidth{0.000000pt}%
\definecolor{currentstroke}{rgb}{1.000000,1.000000,1.000000}%
\pgfsetstrokecolor{currentstroke}%
\pgfsetdash{}{0pt}%
\pgfpathmoveto{\pgfqpoint{0.000000in}{0.000000in}}%
\pgfpathlineto{\pgfqpoint{2.221861in}{0.000000in}}%
\pgfpathlineto{\pgfqpoint{2.221861in}{1.754444in}}%
\pgfpathlineto{\pgfqpoint{0.000000in}{1.754444in}}%
\pgfpathlineto{\pgfqpoint{0.000000in}{0.000000in}}%
\pgfpathclose%
\pgfusepath{fill}%
\end{pgfscope}%
\begin{pgfscope}%
\pgfsetbuttcap%
\pgfsetmiterjoin%
\definecolor{currentfill}{rgb}{1.000000,1.000000,1.000000}%
\pgfsetfillcolor{currentfill}%
\pgfsetlinewidth{0.000000pt}%
\definecolor{currentstroke}{rgb}{0.000000,0.000000,0.000000}%
\pgfsetstrokecolor{currentstroke}%
\pgfsetstrokeopacity{0.000000}%
\pgfsetdash{}{0pt}%
\pgfpathmoveto{\pgfqpoint{0.553581in}{0.499444in}}%
\pgfpathlineto{\pgfqpoint{2.103581in}{0.499444in}}%
\pgfpathlineto{\pgfqpoint{2.103581in}{1.654444in}}%
\pgfpathlineto{\pgfqpoint{0.553581in}{1.654444in}}%
\pgfpathlineto{\pgfqpoint{0.553581in}{0.499444in}}%
\pgfpathclose%
\pgfusepath{fill}%
\end{pgfscope}%
\begin{pgfscope}%
\pgfsetbuttcap%
\pgfsetroundjoin%
\definecolor{currentfill}{rgb}{0.000000,0.000000,0.000000}%
\pgfsetfillcolor{currentfill}%
\pgfsetlinewidth{0.803000pt}%
\definecolor{currentstroke}{rgb}{0.000000,0.000000,0.000000}%
\pgfsetstrokecolor{currentstroke}%
\pgfsetdash{}{0pt}%
\pgfsys@defobject{currentmarker}{\pgfqpoint{0.000000in}{-0.048611in}}{\pgfqpoint{0.000000in}{0.000000in}}{%
\pgfpathmoveto{\pgfqpoint{0.000000in}{0.000000in}}%
\pgfpathlineto{\pgfqpoint{0.000000in}{-0.048611in}}%
\pgfusepath{stroke,fill}%
}%
\begin{pgfscope}%
\pgfsys@transformshift{0.624035in}{0.499444in}%
\pgfsys@useobject{currentmarker}{}%
\end{pgfscope}%
\end{pgfscope}%
\begin{pgfscope}%
\definecolor{textcolor}{rgb}{0.000000,0.000000,0.000000}%
\pgfsetstrokecolor{textcolor}%
\pgfsetfillcolor{textcolor}%
\pgftext[x=0.624035in,y=0.402222in,,top]{\color{textcolor}\rmfamily\fontsize{10.000000}{12.000000}\selectfont \(\displaystyle {0.0}\)}%
\end{pgfscope}%
\begin{pgfscope}%
\pgfsetbuttcap%
\pgfsetroundjoin%
\definecolor{currentfill}{rgb}{0.000000,0.000000,0.000000}%
\pgfsetfillcolor{currentfill}%
\pgfsetlinewidth{0.803000pt}%
\definecolor{currentstroke}{rgb}{0.000000,0.000000,0.000000}%
\pgfsetstrokecolor{currentstroke}%
\pgfsetdash{}{0pt}%
\pgfsys@defobject{currentmarker}{\pgfqpoint{0.000000in}{-0.048611in}}{\pgfqpoint{0.000000in}{0.000000in}}{%
\pgfpathmoveto{\pgfqpoint{0.000000in}{0.000000in}}%
\pgfpathlineto{\pgfqpoint{0.000000in}{-0.048611in}}%
\pgfusepath{stroke,fill}%
}%
\begin{pgfscope}%
\pgfsys@transformshift{1.328581in}{0.499444in}%
\pgfsys@useobject{currentmarker}{}%
\end{pgfscope}%
\end{pgfscope}%
\begin{pgfscope}%
\definecolor{textcolor}{rgb}{0.000000,0.000000,0.000000}%
\pgfsetstrokecolor{textcolor}%
\pgfsetfillcolor{textcolor}%
\pgftext[x=1.328581in,y=0.402222in,,top]{\color{textcolor}\rmfamily\fontsize{10.000000}{12.000000}\selectfont \(\displaystyle {0.5}\)}%
\end{pgfscope}%
\begin{pgfscope}%
\pgfsetbuttcap%
\pgfsetroundjoin%
\definecolor{currentfill}{rgb}{0.000000,0.000000,0.000000}%
\pgfsetfillcolor{currentfill}%
\pgfsetlinewidth{0.803000pt}%
\definecolor{currentstroke}{rgb}{0.000000,0.000000,0.000000}%
\pgfsetstrokecolor{currentstroke}%
\pgfsetdash{}{0pt}%
\pgfsys@defobject{currentmarker}{\pgfqpoint{0.000000in}{-0.048611in}}{\pgfqpoint{0.000000in}{0.000000in}}{%
\pgfpathmoveto{\pgfqpoint{0.000000in}{0.000000in}}%
\pgfpathlineto{\pgfqpoint{0.000000in}{-0.048611in}}%
\pgfusepath{stroke,fill}%
}%
\begin{pgfscope}%
\pgfsys@transformshift{2.033126in}{0.499444in}%
\pgfsys@useobject{currentmarker}{}%
\end{pgfscope}%
\end{pgfscope}%
\begin{pgfscope}%
\definecolor{textcolor}{rgb}{0.000000,0.000000,0.000000}%
\pgfsetstrokecolor{textcolor}%
\pgfsetfillcolor{textcolor}%
\pgftext[x=2.033126in,y=0.402222in,,top]{\color{textcolor}\rmfamily\fontsize{10.000000}{12.000000}\selectfont \(\displaystyle {1.0}\)}%
\end{pgfscope}%
\begin{pgfscope}%
\definecolor{textcolor}{rgb}{0.000000,0.000000,0.000000}%
\pgfsetstrokecolor{textcolor}%
\pgfsetfillcolor{textcolor}%
\pgftext[x=1.328581in,y=0.223333in,,top]{\color{textcolor}\rmfamily\fontsize{10.000000}{12.000000}\selectfont False positive rate}%
\end{pgfscope}%
\begin{pgfscope}%
\pgfsetbuttcap%
\pgfsetroundjoin%
\definecolor{currentfill}{rgb}{0.000000,0.000000,0.000000}%
\pgfsetfillcolor{currentfill}%
\pgfsetlinewidth{0.803000pt}%
\definecolor{currentstroke}{rgb}{0.000000,0.000000,0.000000}%
\pgfsetstrokecolor{currentstroke}%
\pgfsetdash{}{0pt}%
\pgfsys@defobject{currentmarker}{\pgfqpoint{-0.048611in}{0.000000in}}{\pgfqpoint{-0.000000in}{0.000000in}}{%
\pgfpathmoveto{\pgfqpoint{-0.000000in}{0.000000in}}%
\pgfpathlineto{\pgfqpoint{-0.048611in}{0.000000in}}%
\pgfusepath{stroke,fill}%
}%
\begin{pgfscope}%
\pgfsys@transformshift{0.553581in}{0.551944in}%
\pgfsys@useobject{currentmarker}{}%
\end{pgfscope}%
\end{pgfscope}%
\begin{pgfscope}%
\definecolor{textcolor}{rgb}{0.000000,0.000000,0.000000}%
\pgfsetstrokecolor{textcolor}%
\pgfsetfillcolor{textcolor}%
\pgftext[x=0.278889in, y=0.503750in, left, base]{\color{textcolor}\rmfamily\fontsize{10.000000}{12.000000}\selectfont \(\displaystyle {0.0}\)}%
\end{pgfscope}%
\begin{pgfscope}%
\pgfsetbuttcap%
\pgfsetroundjoin%
\definecolor{currentfill}{rgb}{0.000000,0.000000,0.000000}%
\pgfsetfillcolor{currentfill}%
\pgfsetlinewidth{0.803000pt}%
\definecolor{currentstroke}{rgb}{0.000000,0.000000,0.000000}%
\pgfsetstrokecolor{currentstroke}%
\pgfsetdash{}{0pt}%
\pgfsys@defobject{currentmarker}{\pgfqpoint{-0.048611in}{0.000000in}}{\pgfqpoint{-0.000000in}{0.000000in}}{%
\pgfpathmoveto{\pgfqpoint{-0.000000in}{0.000000in}}%
\pgfpathlineto{\pgfqpoint{-0.048611in}{0.000000in}}%
\pgfusepath{stroke,fill}%
}%
\begin{pgfscope}%
\pgfsys@transformshift{0.553581in}{1.076944in}%
\pgfsys@useobject{currentmarker}{}%
\end{pgfscope}%
\end{pgfscope}%
\begin{pgfscope}%
\definecolor{textcolor}{rgb}{0.000000,0.000000,0.000000}%
\pgfsetstrokecolor{textcolor}%
\pgfsetfillcolor{textcolor}%
\pgftext[x=0.278889in, y=1.028750in, left, base]{\color{textcolor}\rmfamily\fontsize{10.000000}{12.000000}\selectfont \(\displaystyle {0.5}\)}%
\end{pgfscope}%
\begin{pgfscope}%
\pgfsetbuttcap%
\pgfsetroundjoin%
\definecolor{currentfill}{rgb}{0.000000,0.000000,0.000000}%
\pgfsetfillcolor{currentfill}%
\pgfsetlinewidth{0.803000pt}%
\definecolor{currentstroke}{rgb}{0.000000,0.000000,0.000000}%
\pgfsetstrokecolor{currentstroke}%
\pgfsetdash{}{0pt}%
\pgfsys@defobject{currentmarker}{\pgfqpoint{-0.048611in}{0.000000in}}{\pgfqpoint{-0.000000in}{0.000000in}}{%
\pgfpathmoveto{\pgfqpoint{-0.000000in}{0.000000in}}%
\pgfpathlineto{\pgfqpoint{-0.048611in}{0.000000in}}%
\pgfusepath{stroke,fill}%
}%
\begin{pgfscope}%
\pgfsys@transformshift{0.553581in}{1.601944in}%
\pgfsys@useobject{currentmarker}{}%
\end{pgfscope}%
\end{pgfscope}%
\begin{pgfscope}%
\definecolor{textcolor}{rgb}{0.000000,0.000000,0.000000}%
\pgfsetstrokecolor{textcolor}%
\pgfsetfillcolor{textcolor}%
\pgftext[x=0.278889in, y=1.553750in, left, base]{\color{textcolor}\rmfamily\fontsize{10.000000}{12.000000}\selectfont \(\displaystyle {1.0}\)}%
\end{pgfscope}%
\begin{pgfscope}%
\definecolor{textcolor}{rgb}{0.000000,0.000000,0.000000}%
\pgfsetstrokecolor{textcolor}%
\pgfsetfillcolor{textcolor}%
\pgftext[x=0.223333in,y=1.076944in,,bottom,rotate=90.000000]{\color{textcolor}\rmfamily\fontsize{10.000000}{12.000000}\selectfont True positive rate}%
\end{pgfscope}%
\begin{pgfscope}%
\pgfpathrectangle{\pgfqpoint{0.553581in}{0.499444in}}{\pgfqpoint{1.550000in}{1.155000in}}%
\pgfusepath{clip}%
\pgfsetbuttcap%
\pgfsetroundjoin%
\pgfsetlinewidth{1.505625pt}%
\definecolor{currentstroke}{rgb}{0.000000,0.000000,0.000000}%
\pgfsetstrokecolor{currentstroke}%
\pgfsetdash{{5.550000pt}{2.400000pt}}{0.000000pt}%
\pgfpathmoveto{\pgfqpoint{0.624035in}{0.551944in}}%
\pgfpathlineto{\pgfqpoint{2.033126in}{1.601944in}}%
\pgfusepath{stroke}%
\end{pgfscope}%
\begin{pgfscope}%
\pgfpathrectangle{\pgfqpoint{0.553581in}{0.499444in}}{\pgfqpoint{1.550000in}{1.155000in}}%
\pgfusepath{clip}%
\pgfsetrectcap%
\pgfsetroundjoin%
\pgfsetlinewidth{1.505625pt}%
\definecolor{currentstroke}{rgb}{0.000000,0.000000,0.000000}%
\pgfsetstrokecolor{currentstroke}%
\pgfsetdash{}{0pt}%
\pgfpathmoveto{\pgfqpoint{0.624035in}{0.551944in}}%
\pgfpathlineto{\pgfqpoint{0.625145in}{0.563895in}}%
\pgfpathlineto{\pgfqpoint{0.625270in}{0.564951in}}%
\pgfpathlineto{\pgfqpoint{0.626381in}{0.572525in}}%
\pgfpathlineto{\pgfqpoint{0.626584in}{0.573456in}}%
\pgfpathlineto{\pgfqpoint{0.627694in}{0.581279in}}%
\pgfpathlineto{\pgfqpoint{0.627921in}{0.582334in}}%
\pgfpathlineto{\pgfqpoint{0.629031in}{0.588729in}}%
\pgfpathlineto{\pgfqpoint{0.629328in}{0.589753in}}%
\pgfpathlineto{\pgfqpoint{0.630438in}{0.595279in}}%
\pgfpathlineto{\pgfqpoint{0.630688in}{0.596365in}}%
\pgfpathlineto{\pgfqpoint{0.631790in}{0.600929in}}%
\pgfpathlineto{\pgfqpoint{0.632033in}{0.602015in}}%
\pgfpathlineto{\pgfqpoint{0.633127in}{0.607323in}}%
\pgfpathlineto{\pgfqpoint{0.633307in}{0.608348in}}%
\pgfpathlineto{\pgfqpoint{0.633307in}{0.608410in}}%
\pgfpathlineto{\pgfqpoint{0.634386in}{0.612973in}}%
\pgfpathlineto{\pgfqpoint{0.634675in}{0.614059in}}%
\pgfpathlineto{\pgfqpoint{0.635785in}{0.619399in}}%
\pgfpathlineto{\pgfqpoint{0.636074in}{0.620485in}}%
\pgfpathlineto{\pgfqpoint{0.637185in}{0.625328in}}%
\pgfpathlineto{\pgfqpoint{0.637403in}{0.626321in}}%
\pgfpathlineto{\pgfqpoint{0.638498in}{0.631319in}}%
\pgfpathlineto{\pgfqpoint{0.638834in}{0.632405in}}%
\pgfpathlineto{\pgfqpoint{0.639944in}{0.637093in}}%
\pgfpathlineto{\pgfqpoint{0.640163in}{0.638179in}}%
\pgfpathlineto{\pgfqpoint{0.641273in}{0.643239in}}%
\pgfpathlineto{\pgfqpoint{0.641562in}{0.644326in}}%
\pgfpathlineto{\pgfqpoint{0.642649in}{0.648361in}}%
\pgfpathlineto{\pgfqpoint{0.642657in}{0.648361in}}%
\pgfpathlineto{\pgfqpoint{0.642860in}{0.649168in}}%
\pgfpathlineto{\pgfqpoint{0.643970in}{0.654290in}}%
\pgfpathlineto{\pgfqpoint{0.644134in}{0.655377in}}%
\pgfpathlineto{\pgfqpoint{0.645245in}{0.660219in}}%
\pgfpathlineto{\pgfqpoint{0.645502in}{0.661212in}}%
\pgfpathlineto{\pgfqpoint{0.646597in}{0.665496in}}%
\pgfpathlineto{\pgfqpoint{0.647035in}{0.666583in}}%
\pgfpathlineto{\pgfqpoint{0.648145in}{0.672326in}}%
\pgfpathlineto{\pgfqpoint{0.648395in}{0.673412in}}%
\pgfpathlineto{\pgfqpoint{0.649497in}{0.677603in}}%
\pgfpathlineto{\pgfqpoint{0.649833in}{0.678689in}}%
\pgfpathlineto{\pgfqpoint{0.650936in}{0.682942in}}%
\pgfpathlineto{\pgfqpoint{0.651272in}{0.684028in}}%
\pgfpathlineto{\pgfqpoint{0.652374in}{0.687971in}}%
\pgfpathlineto{\pgfqpoint{0.652687in}{0.689026in}}%
\pgfpathlineto{\pgfqpoint{0.653797in}{0.693186in}}%
\pgfpathlineto{\pgfqpoint{0.654149in}{0.694272in}}%
\pgfpathlineto{\pgfqpoint{0.655259in}{0.698587in}}%
\pgfpathlineto{\pgfqpoint{0.655525in}{0.699674in}}%
\pgfpathlineto{\pgfqpoint{0.656564in}{0.703057in}}%
\pgfpathlineto{\pgfqpoint{0.657104in}{0.704113in}}%
\pgfpathlineto{\pgfqpoint{0.658183in}{0.706969in}}%
\pgfpathlineto{\pgfqpoint{0.658605in}{0.708055in}}%
\pgfpathlineto{\pgfqpoint{0.659715in}{0.712028in}}%
\pgfpathlineto{\pgfqpoint{0.659973in}{0.713053in}}%
\pgfpathlineto{\pgfqpoint{0.661083in}{0.717119in}}%
\pgfpathlineto{\pgfqpoint{0.661482in}{0.718206in}}%
\pgfpathlineto{\pgfqpoint{0.662592in}{0.722459in}}%
\pgfpathlineto{\pgfqpoint{0.662967in}{0.723514in}}%
\pgfpathlineto{\pgfqpoint{0.664077in}{0.726339in}}%
\pgfpathlineto{\pgfqpoint{0.664523in}{0.727332in}}%
\pgfpathlineto{\pgfqpoint{0.665625in}{0.731057in}}%
\pgfpathlineto{\pgfqpoint{0.666141in}{0.732082in}}%
\pgfpathlineto{\pgfqpoint{0.667236in}{0.735279in}}%
\pgfpathlineto{\pgfqpoint{0.667751in}{0.736334in}}%
\pgfpathlineto{\pgfqpoint{0.668838in}{0.739408in}}%
\pgfpathlineto{\pgfqpoint{0.668862in}{0.739408in}}%
\pgfpathlineto{\pgfqpoint{0.672544in}{0.749589in}}%
\pgfpathlineto{\pgfqpoint{0.672989in}{0.750676in}}%
\pgfpathlineto{\pgfqpoint{0.674092in}{0.753687in}}%
\pgfpathlineto{\pgfqpoint{0.674428in}{0.754711in}}%
\pgfpathlineto{\pgfqpoint{0.675538in}{0.758436in}}%
\pgfpathlineto{\pgfqpoint{0.675929in}{0.759523in}}%
\pgfpathlineto{\pgfqpoint{0.677039in}{0.763062in}}%
\pgfpathlineto{\pgfqpoint{0.677563in}{0.764148in}}%
\pgfpathlineto{\pgfqpoint{0.678673in}{0.767345in}}%
\pgfpathlineto{\pgfqpoint{0.679165in}{0.768432in}}%
\pgfpathlineto{\pgfqpoint{0.680268in}{0.771691in}}%
\pgfpathlineto{\pgfqpoint{0.680510in}{0.772747in}}%
\pgfpathlineto{\pgfqpoint{0.681620in}{0.775696in}}%
\pgfpathlineto{\pgfqpoint{0.681995in}{0.776782in}}%
\pgfpathlineto{\pgfqpoint{0.683066in}{0.779824in}}%
\pgfpathlineto{\pgfqpoint{0.683426in}{0.780911in}}%
\pgfpathlineto{\pgfqpoint{0.684489in}{0.783643in}}%
\pgfpathlineto{\pgfqpoint{0.684989in}{0.784698in}}%
\pgfpathlineto{\pgfqpoint{0.686092in}{0.786995in}}%
\pgfpathlineto{\pgfqpoint{0.686420in}{0.788082in}}%
\pgfpathlineto{\pgfqpoint{0.687530in}{0.791062in}}%
\pgfpathlineto{\pgfqpoint{0.688046in}{0.792117in}}%
\pgfpathlineto{\pgfqpoint{0.689156in}{0.794507in}}%
\pgfpathlineto{\pgfqpoint{0.689594in}{0.795594in}}%
\pgfpathlineto{\pgfqpoint{0.690704in}{0.799474in}}%
\pgfpathlineto{\pgfqpoint{0.691048in}{0.800561in}}%
\pgfpathlineto{\pgfqpoint{0.692158in}{0.803913in}}%
\pgfpathlineto{\pgfqpoint{0.692698in}{0.804938in}}%
\pgfpathlineto{\pgfqpoint{0.693808in}{0.807855in}}%
\pgfpathlineto{\pgfqpoint{0.694206in}{0.808911in}}%
\pgfpathlineto{\pgfqpoint{0.695309in}{0.811674in}}%
\pgfpathlineto{\pgfqpoint{0.696044in}{0.812760in}}%
\pgfpathlineto{\pgfqpoint{0.697154in}{0.814964in}}%
\pgfpathlineto{\pgfqpoint{0.697701in}{0.816051in}}%
\pgfpathlineto{\pgfqpoint{0.698811in}{0.818255in}}%
\pgfpathlineto{\pgfqpoint{0.699382in}{0.819341in}}%
\pgfpathlineto{\pgfqpoint{0.700453in}{0.821980in}}%
\pgfpathlineto{\pgfqpoint{0.701031in}{0.823004in}}%
\pgfpathlineto{\pgfqpoint{0.702141in}{0.826077in}}%
\pgfpathlineto{\pgfqpoint{0.702923in}{0.827164in}}%
\pgfpathlineto{\pgfqpoint{0.704025in}{0.829150in}}%
\pgfpathlineto{\pgfqpoint{0.704455in}{0.830237in}}%
\pgfpathlineto{\pgfqpoint{0.705565in}{0.832969in}}%
\pgfpathlineto{\pgfqpoint{0.705995in}{0.834055in}}%
\pgfpathlineto{\pgfqpoint{0.707106in}{0.836787in}}%
\pgfpathlineto{\pgfqpoint{0.707676in}{0.837842in}}%
\pgfpathlineto{\pgfqpoint{0.708771in}{0.839984in}}%
\pgfpathlineto{\pgfqpoint{0.709232in}{0.841040in}}%
\pgfpathlineto{\pgfqpoint{0.710342in}{0.843399in}}%
\pgfpathlineto{\pgfqpoint{0.710717in}{0.844330in}}%
\pgfpathlineto{\pgfqpoint{0.710717in}{0.844392in}}%
\pgfpathlineto{\pgfqpoint{0.711827in}{0.846782in}}%
\pgfpathlineto{\pgfqpoint{0.712289in}{0.847869in}}%
\pgfpathlineto{\pgfqpoint{0.713383in}{0.850476in}}%
\pgfpathlineto{\pgfqpoint{0.713993in}{0.851563in}}%
\pgfpathlineto{\pgfqpoint{0.715095in}{0.853425in}}%
\pgfpathlineto{\pgfqpoint{0.715564in}{0.854481in}}%
\pgfpathlineto{\pgfqpoint{0.716666in}{0.856964in}}%
\pgfpathlineto{\pgfqpoint{0.717143in}{0.858051in}}%
\pgfpathlineto{\pgfqpoint{0.718230in}{0.860596in}}%
\pgfpathlineto{\pgfqpoint{0.718840in}{0.861683in}}%
\pgfpathlineto{\pgfqpoint{0.719942in}{0.864569in}}%
\pgfpathlineto{\pgfqpoint{0.720474in}{0.865625in}}%
\pgfpathlineto{\pgfqpoint{0.721584in}{0.868015in}}%
\pgfpathlineto{\pgfqpoint{0.722123in}{0.869071in}}%
\pgfpathlineto{\pgfqpoint{0.723225in}{0.871895in}}%
\pgfpathlineto{\pgfqpoint{0.723820in}{0.872951in}}%
\pgfpathlineto{\pgfqpoint{0.724930in}{0.875000in}}%
\pgfpathlineto{\pgfqpoint{0.725485in}{0.876086in}}%
\pgfpathlineto{\pgfqpoint{0.726556in}{0.878476in}}%
\pgfpathlineto{\pgfqpoint{0.727197in}{0.879563in}}%
\pgfpathlineto{\pgfqpoint{0.728268in}{0.881736in}}%
\pgfpathlineto{\pgfqpoint{0.728299in}{0.881736in}}%
\pgfpathlineto{\pgfqpoint{0.729042in}{0.882791in}}%
\pgfpathlineto{\pgfqpoint{0.730152in}{0.885430in}}%
\pgfpathlineto{\pgfqpoint{0.730941in}{0.886516in}}%
\pgfpathlineto{\pgfqpoint{0.732052in}{0.888224in}}%
\pgfpathlineto{\pgfqpoint{0.732646in}{0.889310in}}%
\pgfpathlineto{\pgfqpoint{0.733756in}{0.891390in}}%
\pgfpathlineto{\pgfqpoint{0.734295in}{0.892476in}}%
\pgfpathlineto{\pgfqpoint{0.735382in}{0.895053in}}%
\pgfpathlineto{\pgfqpoint{0.736156in}{0.896108in}}%
\pgfpathlineto{\pgfqpoint{0.737227in}{0.898219in}}%
\pgfpathlineto{\pgfqpoint{0.737649in}{0.899306in}}%
\pgfpathlineto{\pgfqpoint{0.738759in}{0.901634in}}%
\pgfpathlineto{\pgfqpoint{0.739416in}{0.902720in}}%
\pgfpathlineto{\pgfqpoint{0.740510in}{0.905017in}}%
\pgfpathlineto{\pgfqpoint{0.741206in}{0.906042in}}%
\pgfpathlineto{\pgfqpoint{0.742316in}{0.908059in}}%
\pgfpathlineto{\pgfqpoint{0.742856in}{0.909146in}}%
\pgfpathlineto{\pgfqpoint{0.743934in}{0.911164in}}%
\pgfpathlineto{\pgfqpoint{0.744583in}{0.912250in}}%
\pgfpathlineto{\pgfqpoint{0.745693in}{0.914702in}}%
\pgfpathlineto{\pgfqpoint{0.746327in}{0.915789in}}%
\pgfpathlineto{\pgfqpoint{0.747437in}{0.917745in}}%
\pgfpathlineto{\pgfqpoint{0.748140in}{0.918831in}}%
\pgfpathlineto{\pgfqpoint{0.749211in}{0.920880in}}%
\pgfpathlineto{\pgfqpoint{0.750032in}{0.921966in}}%
\pgfpathlineto{\pgfqpoint{0.751142in}{0.923829in}}%
\pgfpathlineto{\pgfqpoint{0.751752in}{0.924915in}}%
\pgfpathlineto{\pgfqpoint{0.752854in}{0.926405in}}%
\pgfpathlineto{\pgfqpoint{0.753589in}{0.927492in}}%
\pgfpathlineto{\pgfqpoint{0.754691in}{0.929913in}}%
\pgfpathlineto{\pgfqpoint{0.755473in}{0.931000in}}%
\pgfpathlineto{\pgfqpoint{0.756583in}{0.933110in}}%
\pgfpathlineto{\pgfqpoint{0.757326in}{0.934135in}}%
\pgfpathlineto{\pgfqpoint{0.758405in}{0.936494in}}%
\pgfpathlineto{\pgfqpoint{0.759108in}{0.937581in}}%
\pgfpathlineto{\pgfqpoint{0.761180in}{0.941337in}}%
\pgfpathlineto{\pgfqpoint{0.761821in}{0.942423in}}%
\pgfpathlineto{\pgfqpoint{0.762884in}{0.944068in}}%
\pgfpathlineto{\pgfqpoint{0.762931in}{0.944068in}}%
\pgfpathlineto{\pgfqpoint{0.763533in}{0.945155in}}%
\pgfpathlineto{\pgfqpoint{0.764628in}{0.946769in}}%
\pgfpathlineto{\pgfqpoint{0.765269in}{0.947855in}}%
\pgfpathlineto{\pgfqpoint{0.766379in}{0.949873in}}%
\pgfpathlineto{\pgfqpoint{0.767247in}{0.950960in}}%
\pgfpathlineto{\pgfqpoint{0.768294in}{0.952512in}}%
\pgfpathlineto{\pgfqpoint{0.769045in}{0.953598in}}%
\pgfpathlineto{\pgfqpoint{0.770147in}{0.955368in}}%
\pgfpathlineto{\pgfqpoint{0.770921in}{0.956361in}}%
\pgfpathlineto{\pgfqpoint{0.772015in}{0.957696in}}%
\pgfpathlineto{\pgfqpoint{0.772766in}{0.958782in}}%
\pgfpathlineto{\pgfqpoint{0.773876in}{0.960459in}}%
\pgfpathlineto{\pgfqpoint{0.774900in}{0.961545in}}%
\pgfpathlineto{\pgfqpoint{0.776010in}{0.963345in}}%
\pgfpathlineto{\pgfqpoint{0.776815in}{0.964432in}}%
\pgfpathlineto{\pgfqpoint{0.777918in}{0.966046in}}%
\pgfpathlineto{\pgfqpoint{0.778449in}{0.967133in}}%
\pgfpathlineto{\pgfqpoint{0.779536in}{0.968747in}}%
\pgfpathlineto{\pgfqpoint{0.780114in}{0.969802in}}%
\pgfpathlineto{\pgfqpoint{0.781217in}{0.971541in}}%
\pgfpathlineto{\pgfqpoint{0.782045in}{0.972627in}}%
\pgfpathlineto{\pgfqpoint{0.783148in}{0.974179in}}%
\pgfpathlineto{\pgfqpoint{0.783867in}{0.975266in}}%
\pgfpathlineto{\pgfqpoint{0.784961in}{0.977314in}}%
\pgfpathlineto{\pgfqpoint{0.785860in}{0.978370in}}%
\pgfpathlineto{\pgfqpoint{0.786947in}{0.980605in}}%
\pgfpathlineto{\pgfqpoint{0.787479in}{0.981629in}}%
\pgfpathlineto{\pgfqpoint{0.788534in}{0.982840in}}%
\pgfpathlineto{\pgfqpoint{0.789214in}{0.983926in}}%
\pgfpathlineto{\pgfqpoint{0.790309in}{0.986099in}}%
\pgfpathlineto{\pgfqpoint{0.791216in}{0.987186in}}%
\pgfpathlineto{\pgfqpoint{0.792279in}{0.989235in}}%
\pgfpathlineto{\pgfqpoint{0.793209in}{0.990290in}}%
\pgfpathlineto{\pgfqpoint{0.794319in}{0.992246in}}%
\pgfpathlineto{\pgfqpoint{0.794921in}{0.993332in}}%
\pgfpathlineto{\pgfqpoint{0.796031in}{0.995288in}}%
\pgfpathlineto{\pgfqpoint{0.796852in}{0.996374in}}%
\pgfpathlineto{\pgfqpoint{0.797954in}{0.997989in}}%
\pgfpathlineto{\pgfqpoint{0.798627in}{0.999044in}}%
\pgfpathlineto{\pgfqpoint{0.799682in}{1.000161in}}%
\pgfpathlineto{\pgfqpoint{0.800722in}{1.001248in}}%
\pgfpathlineto{\pgfqpoint{0.801816in}{1.002862in}}%
\pgfpathlineto{\pgfqpoint{0.802840in}{1.003918in}}%
\pgfpathlineto{\pgfqpoint{0.803935in}{1.005190in}}%
\pgfpathlineto{\pgfqpoint{0.805076in}{1.006277in}}%
\pgfpathlineto{\pgfqpoint{0.806186in}{1.007767in}}%
\pgfpathlineto{\pgfqpoint{0.806984in}{1.008822in}}%
\pgfpathlineto{\pgfqpoint{0.808094in}{1.010281in}}%
\pgfpathlineto{\pgfqpoint{0.808852in}{1.011368in}}%
\pgfpathlineto{\pgfqpoint{0.809947in}{1.012827in}}%
\pgfpathlineto{\pgfqpoint{0.810760in}{1.013882in}}%
\pgfpathlineto{\pgfqpoint{0.811870in}{1.015403in}}%
\pgfpathlineto{\pgfqpoint{0.812597in}{1.016490in}}%
\pgfpathlineto{\pgfqpoint{0.813676in}{1.018290in}}%
\pgfpathlineto{\pgfqpoint{0.814622in}{1.019377in}}%
\pgfpathlineto{\pgfqpoint{0.815700in}{1.021146in}}%
\pgfpathlineto{\pgfqpoint{0.816623in}{1.022232in}}%
\pgfpathlineto{\pgfqpoint{0.817733in}{1.023753in}}%
\pgfpathlineto{\pgfqpoint{0.818538in}{1.024840in}}%
\pgfpathlineto{\pgfqpoint{0.819594in}{1.026392in}}%
\pgfpathlineto{\pgfqpoint{0.820555in}{1.027479in}}%
\pgfpathlineto{\pgfqpoint{0.821642in}{1.028720in}}%
\pgfpathlineto{\pgfqpoint{0.822556in}{1.029807in}}%
\pgfpathlineto{\pgfqpoint{0.823651in}{1.031390in}}%
\pgfpathlineto{\pgfqpoint{0.824855in}{1.032445in}}%
\pgfpathlineto{\pgfqpoint{0.825957in}{1.034091in}}%
\pgfpathlineto{\pgfqpoint{0.826731in}{1.035177in}}%
\pgfpathlineto{\pgfqpoint{0.827841in}{1.036419in}}%
\pgfpathlineto{\pgfqpoint{0.828850in}{1.037505in}}%
\pgfpathlineto{\pgfqpoint{0.829944in}{1.039275in}}%
\pgfpathlineto{\pgfqpoint{0.830734in}{1.040299in}}%
\pgfpathlineto{\pgfqpoint{0.831828in}{1.042130in}}%
\pgfpathlineto{\pgfqpoint{0.832813in}{1.043186in}}%
\pgfpathlineto{\pgfqpoint{0.833915in}{1.044738in}}%
\pgfpathlineto{\pgfqpoint{0.834963in}{1.045824in}}%
\pgfpathlineto{\pgfqpoint{0.836058in}{1.047221in}}%
\pgfpathlineto{\pgfqpoint{0.837066in}{1.048308in}}%
\pgfpathlineto{\pgfqpoint{0.838137in}{1.050077in}}%
\pgfpathlineto{\pgfqpoint{0.838950in}{1.051164in}}%
\pgfpathlineto{\pgfqpoint{0.840013in}{1.052623in}}%
\pgfpathlineto{\pgfqpoint{0.841053in}{1.053709in}}%
\pgfpathlineto{\pgfqpoint{0.842147in}{1.054827in}}%
\pgfpathlineto{\pgfqpoint{0.843422in}{1.055913in}}%
\pgfpathlineto{\pgfqpoint{0.844532in}{1.057279in}}%
\pgfpathlineto{\pgfqpoint{0.845321in}{1.058365in}}%
\pgfpathlineto{\pgfqpoint{0.846377in}{1.059793in}}%
\pgfpathlineto{\pgfqpoint{0.847237in}{1.060880in}}%
\pgfpathlineto{\pgfqpoint{0.848183in}{1.062122in}}%
\pgfpathlineto{\pgfqpoint{0.849183in}{1.063208in}}%
\pgfpathlineto{\pgfqpoint{0.850286in}{1.064698in}}%
\pgfpathlineto{\pgfqpoint{0.851583in}{1.065785in}}%
\pgfpathlineto{\pgfqpoint{0.852686in}{1.067057in}}%
\pgfpathlineto{\pgfqpoint{0.853874in}{1.068144in}}%
\pgfpathlineto{\pgfqpoint{0.854921in}{1.069106in}}%
\pgfpathlineto{\pgfqpoint{0.854961in}{1.069106in}}%
\pgfpathlineto{\pgfqpoint{0.855805in}{1.070193in}}%
\pgfpathlineto{\pgfqpoint{0.856884in}{1.071310in}}%
\pgfpathlineto{\pgfqpoint{0.857837in}{1.072396in}}%
\pgfpathlineto{\pgfqpoint{0.858932in}{1.073514in}}%
\pgfpathlineto{\pgfqpoint{0.859714in}{1.074600in}}%
\pgfpathlineto{\pgfqpoint{0.860816in}{1.075873in}}%
\pgfpathlineto{\pgfqpoint{0.861715in}{1.076960in}}%
\pgfpathlineto{\pgfqpoint{0.862747in}{1.078543in}}%
\pgfpathlineto{\pgfqpoint{0.863654in}{1.079629in}}%
\pgfpathlineto{\pgfqpoint{0.864709in}{1.081026in}}%
\pgfpathlineto{\pgfqpoint{0.865983in}{1.082113in}}%
\pgfpathlineto{\pgfqpoint{0.867078in}{1.083541in}}%
\pgfpathlineto{\pgfqpoint{0.868321in}{1.084627in}}%
\pgfpathlineto{\pgfqpoint{0.869408in}{1.085993in}}%
\pgfpathlineto{\pgfqpoint{0.870393in}{1.087079in}}%
\pgfpathlineto{\pgfqpoint{0.871432in}{1.088445in}}%
\pgfpathlineto{\pgfqpoint{0.871503in}{1.088445in}}%
\pgfpathlineto{\pgfqpoint{0.872652in}{1.089532in}}%
\pgfpathlineto{\pgfqpoint{0.873762in}{1.090960in}}%
\pgfpathlineto{\pgfqpoint{0.874770in}{1.092015in}}%
\pgfpathlineto{\pgfqpoint{0.875865in}{1.093102in}}%
\pgfpathlineto{\pgfqpoint{0.875873in}{1.093102in}}%
\pgfpathlineto{\pgfqpoint{0.877178in}{1.094188in}}%
\pgfpathlineto{\pgfqpoint{0.878288in}{1.095957in}}%
\pgfpathlineto{\pgfqpoint{0.879352in}{1.097013in}}%
\pgfpathlineto{\pgfqpoint{0.880438in}{1.098255in}}%
\pgfpathlineto{\pgfqpoint{0.881845in}{1.099310in}}%
\pgfpathlineto{\pgfqpoint{0.882893in}{1.100614in}}%
\pgfpathlineto{\pgfqpoint{0.883808in}{1.101700in}}%
\pgfpathlineto{\pgfqpoint{0.884910in}{1.103004in}}%
\pgfpathlineto{\pgfqpoint{0.886208in}{1.104091in}}%
\pgfpathlineto{\pgfqpoint{0.887318in}{1.105674in}}%
\pgfpathlineto{\pgfqpoint{0.888326in}{1.106760in}}%
\pgfpathlineto{\pgfqpoint{0.889413in}{1.108033in}}%
\pgfpathlineto{\pgfqpoint{0.890625in}{1.109119in}}%
\pgfpathlineto{\pgfqpoint{0.891735in}{1.110516in}}%
\pgfpathlineto{\pgfqpoint{0.892728in}{1.111603in}}%
\pgfpathlineto{\pgfqpoint{0.893806in}{1.112813in}}%
\pgfpathlineto{\pgfqpoint{0.895049in}{1.113869in}}%
\pgfpathlineto{\pgfqpoint{0.896136in}{1.115079in}}%
\pgfpathlineto{\pgfqpoint{0.897035in}{1.116135in}}%
\pgfpathlineto{\pgfqpoint{0.898145in}{1.117501in}}%
\pgfpathlineto{\pgfqpoint{0.899208in}{1.118587in}}%
\pgfpathlineto{\pgfqpoint{0.900256in}{1.119549in}}%
\pgfpathlineto{\pgfqpoint{0.901092in}{1.120605in}}%
\pgfpathlineto{\pgfqpoint{0.902140in}{1.121878in}}%
\pgfpathlineto{\pgfqpoint{0.903297in}{1.122964in}}%
\pgfpathlineto{\pgfqpoint{0.904274in}{1.123989in}}%
\pgfpathlineto{\pgfqpoint{0.905713in}{1.125075in}}%
\pgfpathlineto{\pgfqpoint{0.906760in}{1.126379in}}%
\pgfpathlineto{\pgfqpoint{0.908113in}{1.127465in}}%
\pgfpathlineto{\pgfqpoint{0.909199in}{1.128583in}}%
\pgfpathlineto{\pgfqpoint{0.910177in}{1.129669in}}%
\pgfpathlineto{\pgfqpoint{0.911224in}{1.130880in}}%
\pgfpathlineto{\pgfqpoint{0.912280in}{1.131966in}}%
\pgfpathlineto{\pgfqpoint{0.913374in}{1.132929in}}%
\pgfpathlineto{\pgfqpoint{0.914508in}{1.134015in}}%
\pgfpathlineto{\pgfqpoint{0.915547in}{1.135660in}}%
\pgfpathlineto{\pgfqpoint{0.915571in}{1.135660in}}%
\pgfpathlineto{\pgfqpoint{0.916267in}{1.136747in}}%
\pgfpathlineto{\pgfqpoint{0.917330in}{1.138082in}}%
\pgfpathlineto{\pgfqpoint{0.918612in}{1.139168in}}%
\pgfpathlineto{\pgfqpoint{0.919620in}{1.140068in}}%
\pgfpathlineto{\pgfqpoint{0.920488in}{1.141124in}}%
\pgfpathlineto{\pgfqpoint{0.921583in}{1.142614in}}%
\pgfpathlineto{\pgfqpoint{0.922442in}{1.143638in}}%
\pgfpathlineto{\pgfqpoint{0.922442in}{1.143669in}}%
\pgfpathlineto{\pgfqpoint{0.923545in}{1.144818in}}%
\pgfpathlineto{\pgfqpoint{0.924569in}{1.145904in}}%
\pgfpathlineto{\pgfqpoint{0.925538in}{1.146742in}}%
\pgfpathlineto{\pgfqpoint{0.925601in}{1.146742in}}%
\pgfpathlineto{\pgfqpoint{0.926899in}{1.147829in}}%
\pgfpathlineto{\pgfqpoint{0.927993in}{1.149133in}}%
\pgfpathlineto{\pgfqpoint{0.929017in}{1.150188in}}%
\pgfpathlineto{\pgfqpoint{0.929017in}{1.150219in}}%
\pgfpathlineto{\pgfqpoint{0.930432in}{1.151771in}}%
\pgfpathlineto{\pgfqpoint{0.931730in}{1.152858in}}%
\pgfpathlineto{\pgfqpoint{0.932840in}{1.153789in}}%
\pgfpathlineto{\pgfqpoint{0.933903in}{1.154875in}}%
\pgfpathlineto{\pgfqpoint{0.934943in}{1.155776in}}%
\pgfpathlineto{\pgfqpoint{0.936194in}{1.156862in}}%
\pgfpathlineto{\pgfqpoint{0.937210in}{1.157762in}}%
\pgfpathlineto{\pgfqpoint{0.938484in}{1.158787in}}%
\pgfpathlineto{\pgfqpoint{0.938484in}{1.158849in}}%
\pgfpathlineto{\pgfqpoint{0.939587in}{1.159842in}}%
\pgfpathlineto{\pgfqpoint{0.940642in}{1.160929in}}%
\pgfpathlineto{\pgfqpoint{0.941729in}{1.161798in}}%
\pgfpathlineto{\pgfqpoint{0.943370in}{1.162884in}}%
\pgfpathlineto{\pgfqpoint{0.944480in}{1.164126in}}%
\pgfpathlineto{\pgfqpoint{0.945630in}{1.165181in}}%
\pgfpathlineto{\pgfqpoint{0.946685in}{1.166361in}}%
\pgfpathlineto{\pgfqpoint{0.946716in}{1.166361in}}%
\pgfpathlineto{\pgfqpoint{0.948186in}{1.167447in}}%
\pgfpathlineto{\pgfqpoint{0.949288in}{1.168534in}}%
\pgfpathlineto{\pgfqpoint{0.950602in}{1.169620in}}%
\pgfpathlineto{\pgfqpoint{0.951680in}{1.170490in}}%
\pgfpathlineto{\pgfqpoint{0.952705in}{1.171576in}}%
\pgfpathlineto{\pgfqpoint{0.953776in}{1.172787in}}%
\pgfpathlineto{\pgfqpoint{0.955159in}{1.173873in}}%
\pgfpathlineto{\pgfqpoint{0.956262in}{1.174742in}}%
\pgfpathlineto{\pgfqpoint{0.957677in}{1.175829in}}%
\pgfpathlineto{\pgfqpoint{0.958787in}{1.177040in}}%
\pgfpathlineto{\pgfqpoint{0.959897in}{1.178033in}}%
\pgfpathlineto{\pgfqpoint{0.960913in}{1.179244in}}%
\pgfpathlineto{\pgfqpoint{0.960944in}{1.179244in}}%
\pgfpathlineto{\pgfqpoint{0.961875in}{1.180330in}}%
\pgfpathlineto{\pgfqpoint{0.962985in}{1.181354in}}%
\pgfpathlineto{\pgfqpoint{0.964361in}{1.182441in}}%
\pgfpathlineto{\pgfqpoint{0.965455in}{1.183838in}}%
\pgfpathlineto{\pgfqpoint{0.966542in}{1.184924in}}%
\pgfpathlineto{\pgfqpoint{0.967652in}{1.186228in}}%
\pgfpathlineto{\pgfqpoint{0.969434in}{1.187314in}}%
\pgfpathlineto{\pgfqpoint{0.970537in}{1.188122in}}%
\pgfpathlineto{\pgfqpoint{0.971483in}{1.189208in}}%
\pgfpathlineto{\pgfqpoint{0.972507in}{1.190232in}}%
\pgfpathlineto{\pgfqpoint{0.974328in}{1.191319in}}%
\pgfpathlineto{\pgfqpoint{0.975438in}{1.192343in}}%
\pgfpathlineto{\pgfqpoint{0.976462in}{1.193430in}}%
\pgfpathlineto{\pgfqpoint{0.977541in}{1.194237in}}%
\pgfpathlineto{\pgfqpoint{0.978597in}{1.195323in}}%
\pgfpathlineto{\pgfqpoint{0.979644in}{1.196193in}}%
\pgfpathlineto{\pgfqpoint{0.980950in}{1.197279in}}%
\pgfpathlineto{\pgfqpoint{0.982013in}{1.198396in}}%
\pgfpathlineto{\pgfqpoint{0.983131in}{1.199483in}}%
\pgfpathlineto{\pgfqpoint{0.984139in}{1.200663in}}%
\pgfpathlineto{\pgfqpoint{0.985718in}{1.201749in}}%
\pgfpathlineto{\pgfqpoint{0.986805in}{1.202773in}}%
\pgfpathlineto{\pgfqpoint{0.988025in}{1.203860in}}%
\pgfpathlineto{\pgfqpoint{0.989135in}{1.205008in}}%
\pgfpathlineto{\pgfqpoint{0.990151in}{1.206095in}}%
\pgfpathlineto{\pgfqpoint{0.991246in}{1.207181in}}%
\pgfpathlineto{\pgfqpoint{0.992254in}{1.208268in}}%
\pgfpathlineto{\pgfqpoint{0.993333in}{1.209168in}}%
\pgfpathlineto{\pgfqpoint{0.995123in}{1.210255in}}%
\pgfpathlineto{\pgfqpoint{0.996218in}{1.211465in}}%
\pgfpathlineto{\pgfqpoint{0.997242in}{1.212552in}}%
\pgfpathlineto{\pgfqpoint{0.998344in}{1.213545in}}%
\pgfpathlineto{\pgfqpoint{0.999704in}{1.214632in}}%
\pgfpathlineto{\pgfqpoint{1.000752in}{1.215656in}}%
\pgfpathlineto{\pgfqpoint{1.002292in}{1.216742in}}%
\pgfpathlineto{\pgfqpoint{1.003402in}{1.217612in}}%
\pgfpathlineto{\pgfqpoint{1.004856in}{1.218698in}}%
\pgfpathlineto{\pgfqpoint{1.005833in}{1.219971in}}%
\pgfpathlineto{\pgfqpoint{1.007022in}{1.221057in}}%
\pgfpathlineto{\pgfqpoint{1.008061in}{1.222268in}}%
\pgfpathlineto{\pgfqpoint{1.009398in}{1.223354in}}%
\pgfpathlineto{\pgfqpoint{1.010469in}{1.224255in}}%
\pgfpathlineto{\pgfqpoint{1.011806in}{1.225341in}}%
\pgfpathlineto{\pgfqpoint{1.012799in}{1.226055in}}%
\pgfpathlineto{\pgfqpoint{1.014401in}{1.227142in}}%
\pgfpathlineto{\pgfqpoint{1.015480in}{1.227855in}}%
\pgfpathlineto{\pgfqpoint{1.016973in}{1.228942in}}%
\pgfpathlineto{\pgfqpoint{1.018052in}{1.229780in}}%
\pgfpathlineto{\pgfqpoint{1.019577in}{1.230867in}}%
\pgfpathlineto{\pgfqpoint{1.020679in}{1.231953in}}%
\pgfpathlineto{\pgfqpoint{1.021687in}{1.233008in}}%
\pgfpathlineto{\pgfqpoint{1.022774in}{1.234219in}}%
\pgfpathlineto{\pgfqpoint{1.023923in}{1.235306in}}%
\pgfpathlineto{\pgfqpoint{1.024971in}{1.235895in}}%
\pgfpathlineto{\pgfqpoint{1.025026in}{1.235895in}}%
\pgfpathlineto{\pgfqpoint{1.026620in}{1.236982in}}%
\pgfpathlineto{\pgfqpoint{1.027691in}{1.237820in}}%
\pgfpathlineto{\pgfqpoint{1.029255in}{1.238906in}}%
\pgfpathlineto{\pgfqpoint{1.030326in}{1.239838in}}%
\pgfpathlineto{\pgfqpoint{1.030349in}{1.239838in}}%
\pgfpathlineto{\pgfqpoint{1.031342in}{1.240924in}}%
\pgfpathlineto{\pgfqpoint{1.032405in}{1.241918in}}%
\pgfpathlineto{\pgfqpoint{1.033641in}{1.243004in}}%
\pgfpathlineto{\pgfqpoint{1.034735in}{1.243811in}}%
\pgfpathlineto{\pgfqpoint{1.036306in}{1.244898in}}%
\pgfpathlineto{\pgfqpoint{1.037377in}{1.245674in}}%
\pgfpathlineto{\pgfqpoint{1.038871in}{1.246760in}}%
\pgfpathlineto{\pgfqpoint{1.039832in}{1.247629in}}%
\pgfpathlineto{\pgfqpoint{1.039848in}{1.247629in}}%
\pgfpathlineto{\pgfqpoint{1.041443in}{1.248716in}}%
\pgfpathlineto{\pgfqpoint{1.042514in}{1.249337in}}%
\pgfpathlineto{\pgfqpoint{1.043694in}{1.250423in}}%
\pgfpathlineto{\pgfqpoint{1.044804in}{1.251013in}}%
\pgfpathlineto{\pgfqpoint{1.045883in}{1.252037in}}%
\pgfpathlineto{\pgfqpoint{1.046774in}{1.252689in}}%
\pgfpathlineto{\pgfqpoint{1.048776in}{1.253776in}}%
\pgfpathlineto{\pgfqpoint{1.049847in}{1.254676in}}%
\pgfpathlineto{\pgfqpoint{1.051504in}{1.255762in}}%
\pgfpathlineto{\pgfqpoint{1.052551in}{1.256445in}}%
\pgfpathlineto{\pgfqpoint{1.053974in}{1.257532in}}%
\pgfpathlineto{\pgfqpoint{1.055022in}{1.258494in}}%
\pgfpathlineto{\pgfqpoint{1.056132in}{1.259518in}}%
\pgfpathlineto{\pgfqpoint{1.057234in}{1.260419in}}%
\pgfpathlineto{\pgfqpoint{1.058360in}{1.261505in}}%
\pgfpathlineto{\pgfqpoint{1.059400in}{1.262654in}}%
\pgfpathlineto{\pgfqpoint{1.061894in}{1.263740in}}%
\pgfpathlineto{\pgfqpoint{1.062965in}{1.264827in}}%
\pgfpathlineto{\pgfqpoint{1.064075in}{1.265913in}}%
\pgfpathlineto{\pgfqpoint{1.065177in}{1.266658in}}%
\pgfpathlineto{\pgfqpoint{1.066084in}{1.267714in}}%
\pgfpathlineto{\pgfqpoint{1.067108in}{1.268645in}}%
\pgfpathlineto{\pgfqpoint{1.068554in}{1.269700in}}%
\pgfpathlineto{\pgfqpoint{1.069664in}{1.270787in}}%
\pgfpathlineto{\pgfqpoint{1.070954in}{1.271873in}}%
\pgfpathlineto{\pgfqpoint{1.072049in}{1.272618in}}%
\pgfpathlineto{\pgfqpoint{1.073620in}{1.273643in}}%
\pgfpathlineto{\pgfqpoint{1.074675in}{1.274357in}}%
\pgfpathlineto{\pgfqpoint{1.076083in}{1.275443in}}%
\pgfpathlineto{\pgfqpoint{1.077091in}{1.276312in}}%
\pgfpathlineto{\pgfqpoint{1.078662in}{1.277399in}}%
\pgfpathlineto{\pgfqpoint{1.079773in}{1.278206in}}%
\pgfpathlineto{\pgfqpoint{1.081469in}{1.279292in}}%
\pgfpathlineto{\pgfqpoint{1.082516in}{1.280130in}}%
\pgfpathlineto{\pgfqpoint{1.084205in}{1.281217in}}%
\pgfpathlineto{\pgfqpoint{1.085284in}{1.281869in}}%
\pgfpathlineto{\pgfqpoint{1.086933in}{1.282955in}}%
\pgfpathlineto{\pgfqpoint{1.088044in}{1.283793in}}%
\pgfpathlineto{\pgfqpoint{1.090178in}{1.284880in}}%
\pgfpathlineto{\pgfqpoint{1.091210in}{1.285625in}}%
\pgfpathlineto{\pgfqpoint{1.092930in}{1.286711in}}%
\pgfpathlineto{\pgfqpoint{1.093969in}{1.287301in}}%
\pgfpathlineto{\pgfqpoint{1.096096in}{1.288388in}}%
\pgfpathlineto{\pgfqpoint{1.097135in}{1.289443in}}%
\pgfpathlineto{\pgfqpoint{1.098582in}{1.290530in}}%
\pgfpathlineto{\pgfqpoint{1.099637in}{1.291368in}}%
\pgfpathlineto{\pgfqpoint{1.101654in}{1.292454in}}%
\pgfpathlineto{\pgfqpoint{1.102717in}{1.293261in}}%
\pgfpathlineto{\pgfqpoint{1.104594in}{1.294317in}}%
\pgfpathlineto{\pgfqpoint{1.105649in}{1.295217in}}%
\pgfpathlineto{\pgfqpoint{1.107408in}{1.296272in}}%
\pgfpathlineto{\pgfqpoint{1.108510in}{1.297142in}}%
\pgfpathlineto{\pgfqpoint{1.109941in}{1.298228in}}%
\pgfpathlineto{\pgfqpoint{1.111043in}{1.298973in}}%
\pgfpathlineto{\pgfqpoint{1.112974in}{1.300059in}}%
\pgfpathlineto{\pgfqpoint{1.113889in}{1.300898in}}%
\pgfpathlineto{\pgfqpoint{1.115507in}{1.301953in}}%
\pgfpathlineto{\pgfqpoint{1.116547in}{1.302760in}}%
\pgfpathlineto{\pgfqpoint{1.118415in}{1.303847in}}%
\pgfpathlineto{\pgfqpoint{1.119439in}{1.304623in}}%
\pgfpathlineto{\pgfqpoint{1.121229in}{1.305709in}}%
\pgfpathlineto{\pgfqpoint{1.122340in}{1.306609in}}%
\pgfpathlineto{\pgfqpoint{1.124560in}{1.307696in}}%
\pgfpathlineto{\pgfqpoint{1.125662in}{1.308410in}}%
\pgfpathlineto{\pgfqpoint{1.127233in}{1.309496in}}%
\pgfpathlineto{\pgfqpoint{1.128281in}{1.309931in}}%
\pgfpathlineto{\pgfqpoint{1.129407in}{1.311017in}}%
\pgfpathlineto{\pgfqpoint{1.130486in}{1.311576in}}%
\pgfpathlineto{\pgfqpoint{1.132338in}{1.312663in}}%
\pgfpathlineto{\pgfqpoint{1.133394in}{1.313811in}}%
\pgfpathlineto{\pgfqpoint{1.135114in}{1.314898in}}%
\pgfpathlineto{\pgfqpoint{1.136138in}{1.315860in}}%
\pgfpathlineto{\pgfqpoint{1.137561in}{1.316946in}}%
\pgfpathlineto{\pgfqpoint{1.138671in}{1.317909in}}%
\pgfpathlineto{\pgfqpoint{1.140265in}{1.318995in}}%
\pgfpathlineto{\pgfqpoint{1.141368in}{1.320020in}}%
\pgfpathlineto{\pgfqpoint{1.142587in}{1.321075in}}%
\pgfpathlineto{\pgfqpoint{1.143697in}{1.322068in}}%
\pgfpathlineto{\pgfqpoint{1.145245in}{1.323155in}}%
\pgfpathlineto{\pgfqpoint{1.146355in}{1.323807in}}%
\pgfpathlineto{\pgfqpoint{1.148591in}{1.324893in}}%
\pgfpathlineto{\pgfqpoint{1.149537in}{1.325483in}}%
\pgfpathlineto{\pgfqpoint{1.149678in}{1.325483in}}%
\pgfpathlineto{\pgfqpoint{1.151359in}{1.326569in}}%
\pgfpathlineto{\pgfqpoint{1.152367in}{1.327439in}}%
\pgfpathlineto{\pgfqpoint{1.154189in}{1.328525in}}%
\pgfpathlineto{\pgfqpoint{1.155299in}{1.329394in}}%
\pgfpathlineto{\pgfqpoint{1.156839in}{1.330481in}}%
\pgfpathlineto{\pgfqpoint{1.157871in}{1.331164in}}%
\pgfpathlineto{\pgfqpoint{1.159864in}{1.332250in}}%
\pgfpathlineto{\pgfqpoint{1.160935in}{1.333275in}}%
\pgfpathlineto{\pgfqpoint{1.162131in}{1.334361in}}%
\pgfpathlineto{\pgfqpoint{1.163187in}{1.335199in}}%
\pgfpathlineto{\pgfqpoint{1.165157in}{1.336286in}}%
\pgfpathlineto{\pgfqpoint{1.166243in}{1.337217in}}%
\pgfpathlineto{\pgfqpoint{1.168378in}{1.338272in}}%
\pgfpathlineto{\pgfqpoint{1.169472in}{1.339079in}}%
\pgfpathlineto{\pgfqpoint{1.171536in}{1.340166in}}%
\pgfpathlineto{\pgfqpoint{1.172646in}{1.340818in}}%
\pgfpathlineto{\pgfqpoint{1.174280in}{1.341904in}}%
\pgfpathlineto{\pgfqpoint{1.175210in}{1.342618in}}%
\pgfpathlineto{\pgfqpoint{1.176617in}{1.343705in}}%
\pgfpathlineto{\pgfqpoint{1.177649in}{1.344512in}}%
\pgfpathlineto{\pgfqpoint{1.180393in}{1.345598in}}%
\pgfpathlineto{\pgfqpoint{1.181496in}{1.346250in}}%
\pgfpathlineto{\pgfqpoint{1.183231in}{1.347337in}}%
\pgfpathlineto{\pgfqpoint{1.184271in}{1.348082in}}%
\pgfpathlineto{\pgfqpoint{1.185936in}{1.349168in}}%
\pgfpathlineto{\pgfqpoint{1.187038in}{1.349634in}}%
\pgfpathlineto{\pgfqpoint{1.188289in}{1.350720in}}%
\pgfpathlineto{\pgfqpoint{1.189391in}{1.351465in}}%
\pgfpathlineto{\pgfqpoint{1.191393in}{1.352521in}}%
\pgfpathlineto{\pgfqpoint{1.192448in}{1.353483in}}%
\pgfpathlineto{\pgfqpoint{1.194692in}{1.354569in}}%
\pgfpathlineto{\pgfqpoint{1.195763in}{1.355190in}}%
\pgfpathlineto{\pgfqpoint{1.197334in}{1.356277in}}%
\pgfpathlineto{\pgfqpoint{1.198272in}{1.356898in}}%
\pgfpathlineto{\pgfqpoint{1.199891in}{1.357984in}}%
\pgfpathlineto{\pgfqpoint{1.200969in}{1.358853in}}%
\pgfpathlineto{\pgfqpoint{1.200993in}{1.358853in}}%
\pgfpathlineto{\pgfqpoint{1.202494in}{1.359940in}}%
\pgfpathlineto{\pgfqpoint{1.203526in}{1.360716in}}%
\pgfpathlineto{\pgfqpoint{1.204980in}{1.361802in}}%
\pgfpathlineto{\pgfqpoint{1.206090in}{1.362268in}}%
\pgfpathlineto{\pgfqpoint{1.208130in}{1.363354in}}%
\pgfpathlineto{\pgfqpoint{1.209240in}{1.364161in}}%
\pgfpathlineto{\pgfqpoint{1.211015in}{1.365248in}}%
\pgfpathlineto{\pgfqpoint{1.211992in}{1.366086in}}%
\pgfpathlineto{\pgfqpoint{1.213876in}{1.367173in}}%
\pgfpathlineto{\pgfqpoint{1.214940in}{1.367762in}}%
\pgfpathlineto{\pgfqpoint{1.216362in}{1.368849in}}%
\pgfpathlineto{\pgfqpoint{1.217433in}{1.369221in}}%
\pgfpathlineto{\pgfqpoint{1.219896in}{1.370308in}}%
\pgfpathlineto{\pgfqpoint{1.220959in}{1.370960in}}%
\pgfpathlineto{\pgfqpoint{1.222890in}{1.372046in}}%
\pgfpathlineto{\pgfqpoint{1.223985in}{1.372791in}}%
\pgfpathlineto{\pgfqpoint{1.225947in}{1.373878in}}%
\pgfpathlineto{\pgfqpoint{1.226932in}{1.374498in}}%
\pgfpathlineto{\pgfqpoint{1.229402in}{1.375585in}}%
\pgfpathlineto{\pgfqpoint{1.230403in}{1.376113in}}%
\pgfpathlineto{\pgfqpoint{1.232514in}{1.377199in}}%
\pgfpathlineto{\pgfqpoint{1.233616in}{1.378130in}}%
\pgfpathlineto{\pgfqpoint{1.235539in}{1.379217in}}%
\pgfpathlineto{\pgfqpoint{1.236579in}{1.380148in}}%
\pgfpathlineto{\pgfqpoint{1.238838in}{1.381235in}}%
\pgfpathlineto{\pgfqpoint{1.239933in}{1.382042in}}%
\pgfpathlineto{\pgfqpoint{1.241903in}{1.383128in}}%
\pgfpathlineto{\pgfqpoint{1.243005in}{1.383780in}}%
\pgfpathlineto{\pgfqpoint{1.244959in}{1.384867in}}%
\pgfpathlineto{\pgfqpoint{1.246069in}{1.385674in}}%
\pgfpathlineto{\pgfqpoint{1.247907in}{1.386760in}}%
\pgfpathlineto{\pgfqpoint{1.248931in}{1.387567in}}%
\pgfpathlineto{\pgfqpoint{1.251041in}{1.388654in}}%
\pgfpathlineto{\pgfqpoint{1.252112in}{1.389461in}}%
\pgfpathlineto{\pgfqpoint{1.254739in}{1.390547in}}%
\pgfpathlineto{\pgfqpoint{1.255771in}{1.391261in}}%
\pgfpathlineto{\pgfqpoint{1.257858in}{1.392348in}}%
\pgfpathlineto{\pgfqpoint{1.258851in}{1.392844in}}%
\pgfpathlineto{\pgfqpoint{1.260821in}{1.393900in}}%
\pgfpathlineto{\pgfqpoint{1.261861in}{1.394831in}}%
\pgfpathlineto{\pgfqpoint{1.264613in}{1.395918in}}%
\pgfpathlineto{\pgfqpoint{1.265707in}{1.396756in}}%
\pgfpathlineto{\pgfqpoint{1.267974in}{1.397842in}}%
\pgfpathlineto{\pgfqpoint{1.268952in}{1.398494in}}%
\pgfpathlineto{\pgfqpoint{1.271328in}{1.399581in}}%
\pgfpathlineto{\pgfqpoint{1.272423in}{1.400232in}}%
\pgfpathlineto{\pgfqpoint{1.274041in}{1.401319in}}%
\pgfpathlineto{\pgfqpoint{1.275143in}{1.401816in}}%
\pgfpathlineto{\pgfqpoint{1.277434in}{1.402902in}}%
\pgfpathlineto{\pgfqpoint{1.278247in}{1.403399in}}%
\pgfpathlineto{\pgfqpoint{1.281210in}{1.404485in}}%
\pgfpathlineto{\pgfqpoint{1.282234in}{1.405168in}}%
\pgfpathlineto{\pgfqpoint{1.282312in}{1.405168in}}%
\pgfpathlineto{\pgfqpoint{1.284704in}{1.406255in}}%
\pgfpathlineto{\pgfqpoint{1.285689in}{1.406938in}}%
\pgfpathlineto{\pgfqpoint{1.287933in}{1.408024in}}%
\pgfpathlineto{\pgfqpoint{1.289012in}{1.408521in}}%
\pgfpathlineto{\pgfqpoint{1.291005in}{1.409607in}}%
\pgfpathlineto{\pgfqpoint{1.292076in}{1.410383in}}%
\pgfpathlineto{\pgfqpoint{1.293874in}{1.411470in}}%
\pgfpathlineto{\pgfqpoint{1.294945in}{1.412059in}}%
\pgfpathlineto{\pgfqpoint{1.297463in}{1.413146in}}%
\pgfpathlineto{\pgfqpoint{1.298338in}{1.413860in}}%
\pgfpathlineto{\pgfqpoint{1.298440in}{1.413860in}}%
\pgfpathlineto{\pgfqpoint{1.301098in}{1.414946in}}%
\pgfpathlineto{\pgfqpoint{1.302294in}{1.415660in}}%
\pgfpathlineto{\pgfqpoint{1.304592in}{1.416747in}}%
\pgfpathlineto{\pgfqpoint{1.305663in}{1.417430in}}%
\pgfpathlineto{\pgfqpoint{1.307477in}{1.418516in}}%
\pgfpathlineto{\pgfqpoint{1.308556in}{1.419261in}}%
\pgfpathlineto{\pgfqpoint{1.311487in}{1.420348in}}%
\pgfpathlineto{\pgfqpoint{1.312519in}{1.420938in}}%
\pgfpathlineto{\pgfqpoint{1.315044in}{1.422024in}}%
\pgfpathlineto{\pgfqpoint{1.316147in}{1.422396in}}%
\pgfpathlineto{\pgfqpoint{1.318359in}{1.423483in}}%
\pgfpathlineto{\pgfqpoint{1.319461in}{1.424104in}}%
\pgfpathlineto{\pgfqpoint{1.321557in}{1.425190in}}%
\pgfpathlineto{\pgfqpoint{1.322651in}{1.426122in}}%
\pgfpathlineto{\pgfqpoint{1.324918in}{1.427208in}}%
\pgfpathlineto{\pgfqpoint{1.325903in}{1.427581in}}%
\pgfpathlineto{\pgfqpoint{1.328788in}{1.428667in}}%
\pgfpathlineto{\pgfqpoint{1.329867in}{1.429412in}}%
\pgfpathlineto{\pgfqpoint{1.332447in}{1.430498in}}%
\pgfpathlineto{\pgfqpoint{1.333502in}{1.430964in}}%
\pgfpathlineto{\pgfqpoint{1.336058in}{1.432051in}}%
\pgfpathlineto{\pgfqpoint{1.337168in}{1.432920in}}%
\pgfpathlineto{\pgfqpoint{1.338888in}{1.434006in}}%
\pgfpathlineto{\pgfqpoint{1.339998in}{1.434565in}}%
\pgfpathlineto{\pgfqpoint{1.341804in}{1.435651in}}%
\pgfpathlineto{\pgfqpoint{1.342891in}{1.436645in}}%
\pgfpathlineto{\pgfqpoint{1.344931in}{1.437731in}}%
\pgfpathlineto{\pgfqpoint{1.345987in}{1.438290in}}%
\pgfpathlineto{\pgfqpoint{1.348113in}{1.439377in}}%
\pgfpathlineto{\pgfqpoint{1.349184in}{1.440059in}}%
\pgfpathlineto{\pgfqpoint{1.352382in}{1.441146in}}%
\pgfpathlineto{\pgfqpoint{1.353335in}{1.441612in}}%
\pgfpathlineto{\pgfqpoint{1.356204in}{1.442698in}}%
\pgfpathlineto{\pgfqpoint{1.357307in}{1.443257in}}%
\pgfpathlineto{\pgfqpoint{1.359730in}{1.444343in}}%
\pgfpathlineto{\pgfqpoint{1.360723in}{1.444995in}}%
\pgfpathlineto{\pgfqpoint{1.363741in}{1.446082in}}%
\pgfpathlineto{\pgfqpoint{1.364827in}{1.446578in}}%
\pgfpathlineto{\pgfqpoint{1.367227in}{1.447665in}}%
\pgfpathlineto{\pgfqpoint{1.368329in}{1.448317in}}%
\pgfpathlineto{\pgfqpoint{1.371019in}{1.449403in}}%
\pgfpathlineto{\pgfqpoint{1.372082in}{1.450024in}}%
\pgfpathlineto{\pgfqpoint{1.374662in}{1.451110in}}%
\pgfpathlineto{\pgfqpoint{1.375741in}{1.451607in}}%
\pgfpathlineto{\pgfqpoint{1.375756in}{1.451607in}}%
\pgfpathlineto{\pgfqpoint{1.378203in}{1.452694in}}%
\pgfpathlineto{\pgfqpoint{1.379251in}{1.453314in}}%
\pgfpathlineto{\pgfqpoint{1.382073in}{1.454401in}}%
\pgfpathlineto{\pgfqpoint{1.383113in}{1.454991in}}%
\pgfpathlineto{\pgfqpoint{1.385591in}{1.456077in}}%
\pgfpathlineto{\pgfqpoint{1.386693in}{1.456512in}}%
\pgfpathlineto{\pgfqpoint{1.388757in}{1.457598in}}%
\pgfpathlineto{\pgfqpoint{1.389859in}{1.458064in}}%
\pgfpathlineto{\pgfqpoint{1.393080in}{1.459150in}}%
\pgfpathlineto{\pgfqpoint{1.394112in}{1.459461in}}%
\pgfpathlineto{\pgfqpoint{1.395793in}{1.460547in}}%
\pgfpathlineto{\pgfqpoint{1.396864in}{1.460796in}}%
\pgfpathlineto{\pgfqpoint{1.399209in}{1.461882in}}%
\pgfpathlineto{\pgfqpoint{1.400241in}{1.462565in}}%
\pgfpathlineto{\pgfqpoint{1.403392in}{1.463651in}}%
\pgfpathlineto{\pgfqpoint{1.404463in}{1.464179in}}%
\pgfpathlineto{\pgfqpoint{1.407699in}{1.465266in}}%
\pgfpathlineto{\pgfqpoint{1.408747in}{1.465638in}}%
\pgfpathlineto{\pgfqpoint{1.412108in}{1.466725in}}%
\pgfpathlineto{\pgfqpoint{1.413109in}{1.467252in}}%
\pgfpathlineto{\pgfqpoint{1.416635in}{1.468339in}}%
\pgfpathlineto{\pgfqpoint{1.417526in}{1.468804in}}%
\pgfpathlineto{\pgfqpoint{1.417643in}{1.468804in}}%
\pgfpathlineto{\pgfqpoint{1.421075in}{1.469891in}}%
\pgfpathlineto{\pgfqpoint{1.422123in}{1.470481in}}%
\pgfpathlineto{\pgfqpoint{1.426000in}{1.471567in}}%
\pgfpathlineto{\pgfqpoint{1.426970in}{1.471816in}}%
\pgfpathlineto{\pgfqpoint{1.427095in}{1.471816in}}%
\pgfpathlineto{\pgfqpoint{1.429768in}{1.472902in}}%
\pgfpathlineto{\pgfqpoint{1.430824in}{1.473275in}}%
\pgfpathlineto{\pgfqpoint{1.434256in}{1.474361in}}%
\pgfpathlineto{\pgfqpoint{1.435358in}{1.475013in}}%
\pgfpathlineto{\pgfqpoint{1.438626in}{1.476099in}}%
\pgfpathlineto{\pgfqpoint{1.439689in}{1.476596in}}%
\pgfpathlineto{\pgfqpoint{1.442581in}{1.477683in}}%
\pgfpathlineto{\pgfqpoint{1.443527in}{1.478117in}}%
\pgfpathlineto{\pgfqpoint{1.446217in}{1.479204in}}%
\pgfpathlineto{\pgfqpoint{1.447311in}{1.479545in}}%
\pgfpathlineto{\pgfqpoint{1.450000in}{1.480632in}}%
\pgfpathlineto{\pgfqpoint{1.451071in}{1.481097in}}%
\pgfpathlineto{\pgfqpoint{1.453370in}{1.482184in}}%
\pgfpathlineto{\pgfqpoint{1.454480in}{1.482929in}}%
\pgfpathlineto{\pgfqpoint{1.457443in}{1.484015in}}%
\pgfpathlineto{\pgfqpoint{1.458506in}{1.484263in}}%
\pgfpathlineto{\pgfqpoint{1.461352in}{1.485350in}}%
\pgfpathlineto{\pgfqpoint{1.462399in}{1.485878in}}%
\pgfpathlineto{\pgfqpoint{1.465862in}{1.486964in}}%
\pgfpathlineto{\pgfqpoint{1.466793in}{1.487244in}}%
\pgfpathlineto{\pgfqpoint{1.470608in}{1.488330in}}%
\pgfpathlineto{\pgfqpoint{1.471663in}{1.488702in}}%
\pgfpathlineto{\pgfqpoint{1.474563in}{1.489789in}}%
\pgfpathlineto{\pgfqpoint{1.475619in}{1.490193in}}%
\pgfpathlineto{\pgfqpoint{1.478777in}{1.491248in}}%
\pgfpathlineto{\pgfqpoint{1.479887in}{1.491745in}}%
\pgfpathlineto{\pgfqpoint{1.482397in}{1.492831in}}%
\pgfpathlineto{\pgfqpoint{1.483436in}{1.493235in}}%
\pgfpathlineto{\pgfqpoint{1.486399in}{1.494321in}}%
\pgfpathlineto{\pgfqpoint{1.487486in}{1.494818in}}%
\pgfpathlineto{\pgfqpoint{1.490566in}{1.495873in}}%
\pgfpathlineto{\pgfqpoint{1.491426in}{1.496153in}}%
\pgfpathlineto{\pgfqpoint{1.494663in}{1.497239in}}%
\pgfpathlineto{\pgfqpoint{1.495702in}{1.497612in}}%
\pgfpathlineto{\pgfqpoint{1.499814in}{1.498698in}}%
\pgfpathlineto{\pgfqpoint{1.500502in}{1.499164in}}%
\pgfpathlineto{\pgfqpoint{1.500768in}{1.499164in}}%
\pgfpathlineto{\pgfqpoint{1.504044in}{1.500250in}}%
\pgfpathlineto{\pgfqpoint{1.504912in}{1.500840in}}%
\pgfpathlineto{\pgfqpoint{1.509211in}{1.501926in}}%
\pgfpathlineto{\pgfqpoint{1.510243in}{1.502361in}}%
\pgfpathlineto{\pgfqpoint{1.513097in}{1.503447in}}%
\pgfpathlineto{\pgfqpoint{1.514019in}{1.503882in}}%
\pgfpathlineto{\pgfqpoint{1.517904in}{1.504969in}}%
\pgfpathlineto{\pgfqpoint{1.518803in}{1.505248in}}%
\pgfpathlineto{\pgfqpoint{1.523033in}{1.506334in}}%
\pgfpathlineto{\pgfqpoint{1.524143in}{1.506800in}}%
\pgfpathlineto{\pgfqpoint{1.527895in}{1.507887in}}%
\pgfpathlineto{\pgfqpoint{1.529005in}{1.508414in}}%
\pgfpathlineto{\pgfqpoint{1.531007in}{1.509470in}}%
\pgfpathlineto{\pgfqpoint{1.532101in}{1.509873in}}%
\pgfpathlineto{\pgfqpoint{1.536018in}{1.510960in}}%
\pgfpathlineto{\pgfqpoint{1.537058in}{1.511301in}}%
\pgfpathlineto{\pgfqpoint{1.540349in}{1.512388in}}%
\pgfpathlineto{\pgfqpoint{1.541443in}{1.513040in}}%
\pgfpathlineto{\pgfqpoint{1.544273in}{1.514126in}}%
\pgfpathlineto{\pgfqpoint{1.545235in}{1.514592in}}%
\pgfpathlineto{\pgfqpoint{1.548026in}{1.515678in}}%
\pgfpathlineto{\pgfqpoint{1.549066in}{1.516237in}}%
\pgfpathlineto{\pgfqpoint{1.549089in}{1.516237in}}%
\pgfpathlineto{\pgfqpoint{1.552474in}{1.517323in}}%
\pgfpathlineto{\pgfqpoint{1.553506in}{1.517572in}}%
\pgfpathlineto{\pgfqpoint{1.553529in}{1.517572in}}%
\pgfpathlineto{\pgfqpoint{1.556485in}{1.518658in}}%
\pgfpathlineto{\pgfqpoint{1.557407in}{1.518938in}}%
\pgfpathlineto{\pgfqpoint{1.561847in}{1.520024in}}%
\pgfpathlineto{\pgfqpoint{1.562778in}{1.520272in}}%
\pgfpathlineto{\pgfqpoint{1.566554in}{1.521359in}}%
\pgfpathlineto{\pgfqpoint{1.567562in}{1.521576in}}%
\pgfpathlineto{\pgfqpoint{1.571330in}{1.522663in}}%
\pgfpathlineto{\pgfqpoint{1.572425in}{1.523066in}}%
\pgfpathlineto{\pgfqpoint{1.575982in}{1.524153in}}%
\pgfpathlineto{\pgfqpoint{1.576967in}{1.524556in}}%
\pgfpathlineto{\pgfqpoint{1.580578in}{1.525643in}}%
\pgfpathlineto{\pgfqpoint{1.581634in}{1.526077in}}%
\pgfpathlineto{\pgfqpoint{1.587442in}{1.527164in}}%
\pgfpathlineto{\pgfqpoint{1.588506in}{1.527691in}}%
\pgfpathlineto{\pgfqpoint{1.593399in}{1.528778in}}%
\pgfpathlineto{\pgfqpoint{1.594447in}{1.529181in}}%
\pgfpathlineto{\pgfqpoint{1.597676in}{1.530268in}}%
\pgfpathlineto{\pgfqpoint{1.598762in}{1.530702in}}%
\pgfpathlineto{\pgfqpoint{1.602530in}{1.531789in}}%
\pgfpathlineto{\pgfqpoint{1.603476in}{1.532130in}}%
\pgfpathlineto{\pgfqpoint{1.603562in}{1.532130in}}%
\pgfpathlineto{\pgfqpoint{1.608128in}{1.533217in}}%
\pgfpathlineto{\pgfqpoint{1.609129in}{1.533683in}}%
\pgfpathlineto{\pgfqpoint{1.613514in}{1.534738in}}%
\pgfpathlineto{\pgfqpoint{1.614601in}{1.535297in}}%
\pgfpathlineto{\pgfqpoint{1.617486in}{1.536383in}}%
\pgfpathlineto{\pgfqpoint{1.618557in}{1.536818in}}%
\pgfpathlineto{\pgfqpoint{1.621801in}{1.537904in}}%
\pgfpathlineto{\pgfqpoint{1.622895in}{1.538339in}}%
\pgfpathlineto{\pgfqpoint{1.627328in}{1.539425in}}%
\pgfpathlineto{\pgfqpoint{1.628344in}{1.539829in}}%
\pgfpathlineto{\pgfqpoint{1.632409in}{1.540915in}}%
\pgfpathlineto{\pgfqpoint{1.633488in}{1.541257in}}%
\pgfpathlineto{\pgfqpoint{1.638421in}{1.542343in}}%
\pgfpathlineto{\pgfqpoint{1.639273in}{1.542654in}}%
\pgfpathlineto{\pgfqpoint{1.643456in}{1.543740in}}%
\pgfpathlineto{\pgfqpoint{1.644527in}{1.544144in}}%
\pgfpathlineto{\pgfqpoint{1.648506in}{1.545230in}}%
\pgfpathlineto{\pgfqpoint{1.649335in}{1.545416in}}%
\pgfpathlineto{\pgfqpoint{1.649358in}{1.545416in}}%
\pgfpathlineto{\pgfqpoint{1.653971in}{1.546503in}}%
\pgfpathlineto{\pgfqpoint{1.654885in}{1.546751in}}%
\pgfpathlineto{\pgfqpoint{1.660420in}{1.547838in}}%
\pgfpathlineto{\pgfqpoint{1.661460in}{1.548117in}}%
\pgfpathlineto{\pgfqpoint{1.665416in}{1.549204in}}%
\pgfpathlineto{\pgfqpoint{1.666447in}{1.549545in}}%
\pgfpathlineto{\pgfqpoint{1.670458in}{1.550632in}}%
\pgfpathlineto{\pgfqpoint{1.671388in}{1.550880in}}%
\pgfpathlineto{\pgfqpoint{1.675782in}{1.551966in}}%
\pgfpathlineto{\pgfqpoint{1.676603in}{1.552184in}}%
\pgfpathlineto{\pgfqpoint{1.681739in}{1.553270in}}%
\pgfpathlineto{\pgfqpoint{1.682771in}{1.553456in}}%
\pgfpathlineto{\pgfqpoint{1.687876in}{1.554543in}}%
\pgfpathlineto{\pgfqpoint{1.688954in}{1.554946in}}%
\pgfpathlineto{\pgfqpoint{1.688970in}{1.554946in}}%
\pgfpathlineto{\pgfqpoint{1.693512in}{1.556033in}}%
\pgfpathlineto{\pgfqpoint{1.694536in}{1.556343in}}%
\pgfpathlineto{\pgfqpoint{1.699938in}{1.557430in}}%
\pgfpathlineto{\pgfqpoint{1.700697in}{1.557771in}}%
\pgfpathlineto{\pgfqpoint{1.707779in}{1.558858in}}%
\pgfpathlineto{\pgfqpoint{1.708843in}{1.559292in}}%
\pgfpathlineto{\pgfqpoint{1.714002in}{1.560379in}}%
\pgfpathlineto{\pgfqpoint{1.714987in}{1.560627in}}%
\pgfpathlineto{\pgfqpoint{1.720741in}{1.561714in}}%
\pgfpathlineto{\pgfqpoint{1.721601in}{1.562117in}}%
\pgfpathlineto{\pgfqpoint{1.728527in}{1.563204in}}%
\pgfpathlineto{\pgfqpoint{1.729598in}{1.563421in}}%
\pgfpathlineto{\pgfqpoint{1.733773in}{1.564507in}}%
\pgfpathlineto{\pgfqpoint{1.734875in}{1.564880in}}%
\pgfpathlineto{\pgfqpoint{1.739808in}{1.565966in}}%
\pgfpathlineto{\pgfqpoint{1.740676in}{1.566246in}}%
\pgfpathlineto{\pgfqpoint{1.740715in}{1.566246in}}%
\pgfpathlineto{\pgfqpoint{1.747243in}{1.567332in}}%
\pgfpathlineto{\pgfqpoint{1.747993in}{1.567612in}}%
\pgfpathlineto{\pgfqpoint{1.752887in}{1.568698in}}%
\pgfpathlineto{\pgfqpoint{1.753645in}{1.568853in}}%
\pgfpathlineto{\pgfqpoint{1.753903in}{1.568853in}}%
\pgfpathlineto{\pgfqpoint{1.760986in}{1.569940in}}%
\pgfpathlineto{\pgfqpoint{1.762081in}{1.570436in}}%
\pgfpathlineto{\pgfqpoint{1.769570in}{1.571523in}}%
\pgfpathlineto{\pgfqpoint{1.770109in}{1.571771in}}%
\pgfpathlineto{\pgfqpoint{1.779678in}{1.572858in}}%
\pgfpathlineto{\pgfqpoint{1.780538in}{1.572982in}}%
\pgfpathlineto{\pgfqpoint{1.780694in}{1.572982in}}%
\pgfpathlineto{\pgfqpoint{1.787027in}{1.574068in}}%
\pgfpathlineto{\pgfqpoint{1.788074in}{1.574379in}}%
\pgfpathlineto{\pgfqpoint{1.796424in}{1.575465in}}%
\pgfpathlineto{\pgfqpoint{1.797424in}{1.575589in}}%
\pgfpathlineto{\pgfqpoint{1.803944in}{1.576676in}}%
\pgfpathlineto{\pgfqpoint{1.804921in}{1.576986in}}%
\pgfpathlineto{\pgfqpoint{1.811410in}{1.578073in}}%
\pgfpathlineto{\pgfqpoint{1.812473in}{1.578290in}}%
\pgfpathlineto{\pgfqpoint{1.820385in}{1.579377in}}%
\pgfpathlineto{\pgfqpoint{1.821198in}{1.579594in}}%
\pgfpathlineto{\pgfqpoint{1.828554in}{1.580680in}}%
\pgfpathlineto{\pgfqpoint{1.829664in}{1.580991in}}%
\pgfpathlineto{\pgfqpoint{1.836692in}{1.582077in}}%
\pgfpathlineto{\pgfqpoint{1.837795in}{1.582263in}}%
\pgfpathlineto{\pgfqpoint{1.842587in}{1.583350in}}%
\pgfpathlineto{\pgfqpoint{1.843595in}{1.583474in}}%
\pgfpathlineto{\pgfqpoint{1.851913in}{1.584561in}}%
\pgfpathlineto{\pgfqpoint{1.852500in}{1.584685in}}%
\pgfpathlineto{\pgfqpoint{1.860161in}{1.585771in}}%
\pgfpathlineto{\pgfqpoint{1.860896in}{1.586020in}}%
\pgfpathlineto{\pgfqpoint{1.869675in}{1.587106in}}%
\pgfpathlineto{\pgfqpoint{1.870605in}{1.587416in}}%
\pgfpathlineto{\pgfqpoint{1.880002in}{1.588503in}}%
\pgfpathlineto{\pgfqpoint{1.880479in}{1.588565in}}%
\pgfpathlineto{\pgfqpoint{1.880620in}{1.588565in}}%
\pgfpathlineto{\pgfqpoint{1.886733in}{1.589651in}}%
\pgfpathlineto{\pgfqpoint{1.887179in}{1.589776in}}%
\pgfpathlineto{\pgfqpoint{1.887280in}{1.589776in}}%
\pgfpathlineto{\pgfqpoint{1.896552in}{1.590862in}}%
\pgfpathlineto{\pgfqpoint{1.897553in}{1.590986in}}%
\pgfpathlineto{\pgfqpoint{1.897646in}{1.590986in}}%
\pgfpathlineto{\pgfqpoint{1.906082in}{1.592073in}}%
\pgfpathlineto{\pgfqpoint{1.907137in}{1.592259in}}%
\pgfpathlineto{\pgfqpoint{1.916260in}{1.593345in}}%
\pgfpathlineto{\pgfqpoint{1.917363in}{1.593439in}}%
\pgfpathlineto{\pgfqpoint{1.928636in}{1.594525in}}%
\pgfpathlineto{\pgfqpoint{1.929128in}{1.594711in}}%
\pgfpathlineto{\pgfqpoint{1.929253in}{1.594711in}}%
\pgfpathlineto{\pgfqpoint{1.937579in}{1.595798in}}%
\pgfpathlineto{\pgfqpoint{1.938439in}{1.595891in}}%
\pgfpathlineto{\pgfqpoint{1.938509in}{1.595891in}}%
\pgfpathlineto{\pgfqpoint{1.953261in}{1.596977in}}%
\pgfpathlineto{\pgfqpoint{1.954043in}{1.597071in}}%
\pgfpathlineto{\pgfqpoint{1.954270in}{1.597071in}}%
\pgfpathlineto{\pgfqpoint{1.969686in}{1.598157in}}%
\pgfpathlineto{\pgfqpoint{1.970350in}{1.598219in}}%
\pgfpathlineto{\pgfqpoint{1.970436in}{1.598219in}}%
\pgfpathlineto{\pgfqpoint{1.984829in}{1.599306in}}%
\pgfpathlineto{\pgfqpoint{1.985884in}{1.599368in}}%
\pgfpathlineto{\pgfqpoint{2.002082in}{1.600454in}}%
\pgfpathlineto{\pgfqpoint{2.003028in}{1.600640in}}%
\pgfpathlineto{\pgfqpoint{2.027990in}{1.601727in}}%
\pgfpathlineto{\pgfqpoint{2.027990in}{1.601758in}}%
\pgfpathlineto{\pgfqpoint{2.033126in}{1.601944in}}%
\pgfpathlineto{\pgfqpoint{2.033126in}{1.601944in}}%
\pgfusepath{stroke}%
\end{pgfscope}%
\begin{pgfscope}%
\pgfsetrectcap%
\pgfsetmiterjoin%
\pgfsetlinewidth{0.803000pt}%
\definecolor{currentstroke}{rgb}{0.000000,0.000000,0.000000}%
\pgfsetstrokecolor{currentstroke}%
\pgfsetdash{}{0pt}%
\pgfpathmoveto{\pgfqpoint{0.553581in}{0.499444in}}%
\pgfpathlineto{\pgfqpoint{0.553581in}{1.654444in}}%
\pgfusepath{stroke}%
\end{pgfscope}%
\begin{pgfscope}%
\pgfsetrectcap%
\pgfsetmiterjoin%
\pgfsetlinewidth{0.803000pt}%
\definecolor{currentstroke}{rgb}{0.000000,0.000000,0.000000}%
\pgfsetstrokecolor{currentstroke}%
\pgfsetdash{}{0pt}%
\pgfpathmoveto{\pgfqpoint{2.103581in}{0.499444in}}%
\pgfpathlineto{\pgfqpoint{2.103581in}{1.654444in}}%
\pgfusepath{stroke}%
\end{pgfscope}%
\begin{pgfscope}%
\pgfsetrectcap%
\pgfsetmiterjoin%
\pgfsetlinewidth{0.803000pt}%
\definecolor{currentstroke}{rgb}{0.000000,0.000000,0.000000}%
\pgfsetstrokecolor{currentstroke}%
\pgfsetdash{}{0pt}%
\pgfpathmoveto{\pgfqpoint{0.553581in}{0.499444in}}%
\pgfpathlineto{\pgfqpoint{2.103581in}{0.499444in}}%
\pgfusepath{stroke}%
\end{pgfscope}%
\begin{pgfscope}%
\pgfsetrectcap%
\pgfsetmiterjoin%
\pgfsetlinewidth{0.803000pt}%
\definecolor{currentstroke}{rgb}{0.000000,0.000000,0.000000}%
\pgfsetstrokecolor{currentstroke}%
\pgfsetdash{}{0pt}%
\pgfpathmoveto{\pgfqpoint{0.553581in}{1.654444in}}%
\pgfpathlineto{\pgfqpoint{2.103581in}{1.654444in}}%
\pgfusepath{stroke}%
\end{pgfscope}%
\begin{pgfscope}%
\pgfsetbuttcap%
\pgfsetmiterjoin%
\definecolor{currentfill}{rgb}{1.000000,1.000000,1.000000}%
\pgfsetfillcolor{currentfill}%
\pgfsetfillopacity{0.800000}%
\pgfsetlinewidth{1.003750pt}%
\definecolor{currentstroke}{rgb}{0.800000,0.800000,0.800000}%
\pgfsetstrokecolor{currentstroke}%
\pgfsetstrokeopacity{0.800000}%
\pgfsetdash{}{0pt}%
\pgfpathmoveto{\pgfqpoint{0.832747in}{0.568889in}}%
\pgfpathlineto{\pgfqpoint{2.006358in}{0.568889in}}%
\pgfpathquadraticcurveto{\pgfqpoint{2.034136in}{0.568889in}}{\pgfqpoint{2.034136in}{0.596666in}}%
\pgfpathlineto{\pgfqpoint{2.034136in}{0.776388in}}%
\pgfpathquadraticcurveto{\pgfqpoint{2.034136in}{0.804166in}}{\pgfqpoint{2.006358in}{0.804166in}}%
\pgfpathlineto{\pgfqpoint{0.832747in}{0.804166in}}%
\pgfpathquadraticcurveto{\pgfqpoint{0.804970in}{0.804166in}}{\pgfqpoint{0.804970in}{0.776388in}}%
\pgfpathlineto{\pgfqpoint{0.804970in}{0.596666in}}%
\pgfpathquadraticcurveto{\pgfqpoint{0.804970in}{0.568889in}}{\pgfqpoint{0.832747in}{0.568889in}}%
\pgfpathlineto{\pgfqpoint{0.832747in}{0.568889in}}%
\pgfpathclose%
\pgfusepath{stroke,fill}%
\end{pgfscope}%
\begin{pgfscope}%
\pgfsetrectcap%
\pgfsetroundjoin%
\pgfsetlinewidth{1.505625pt}%
\definecolor{currentstroke}{rgb}{0.000000,0.000000,0.000000}%
\pgfsetstrokecolor{currentstroke}%
\pgfsetdash{}{0pt}%
\pgfpathmoveto{\pgfqpoint{0.860525in}{0.700000in}}%
\pgfpathlineto{\pgfqpoint{0.999414in}{0.700000in}}%
\pgfpathlineto{\pgfqpoint{1.138303in}{0.700000in}}%
\pgfusepath{stroke}%
\end{pgfscope}%
\begin{pgfscope}%
\definecolor{textcolor}{rgb}{0.000000,0.000000,0.000000}%
\pgfsetstrokecolor{textcolor}%
\pgfsetfillcolor{textcolor}%
\pgftext[x=1.249414in,y=0.651388in,left,base]{\color{textcolor}\rmfamily\fontsize{10.000000}{12.000000}\selectfont AUC=0.753}%
\end{pgfscope}%
\end{pgfpicture}%
\makeatother%
\endgroup%

\end{tabular}

\verb|LRC_Hard_Tomek_0_alpha_balanced_v1_Test|

\noindent\begin{tabular}{@{\hspace{-6pt}}p{4.3in} @{\hspace{-6pt}}p{2.0in}}
	\vskip 0pt
	\hfil Raw Model Output
	
	%% Creator: Matplotlib, PGF backend
%%
%% To include the figure in your LaTeX document, write
%%   \input{<filename>.pgf}
%%
%% Make sure the required packages are loaded in your preamble
%%   \usepackage{pgf}
%%
%% Also ensure that all the required font packages are loaded; for instance,
%% the lmodern package is sometimes necessary when using math font.
%%   \usepackage{lmodern}
%%
%% Figures using additional raster images can only be included by \input if
%% they are in the same directory as the main LaTeX file. For loading figures
%% from other directories you can use the `import` package
%%   \usepackage{import}
%%
%% and then include the figures with
%%   \import{<path to file>}{<filename>.pgf}
%%
%% Matplotlib used the following preamble
%%   
%%   \usepackage{fontspec}
%%   \makeatletter\@ifpackageloaded{underscore}{}{\usepackage[strings]{underscore}}\makeatother
%%
\begingroup%
\makeatletter%
\begin{pgfpicture}%
\pgfpathrectangle{\pgfpointorigin}{\pgfqpoint{4.509306in}{1.754444in}}%
\pgfusepath{use as bounding box, clip}%
\begin{pgfscope}%
\pgfsetbuttcap%
\pgfsetmiterjoin%
\definecolor{currentfill}{rgb}{1.000000,1.000000,1.000000}%
\pgfsetfillcolor{currentfill}%
\pgfsetlinewidth{0.000000pt}%
\definecolor{currentstroke}{rgb}{1.000000,1.000000,1.000000}%
\pgfsetstrokecolor{currentstroke}%
\pgfsetdash{}{0pt}%
\pgfpathmoveto{\pgfqpoint{0.000000in}{0.000000in}}%
\pgfpathlineto{\pgfqpoint{4.509306in}{0.000000in}}%
\pgfpathlineto{\pgfqpoint{4.509306in}{1.754444in}}%
\pgfpathlineto{\pgfqpoint{0.000000in}{1.754444in}}%
\pgfpathlineto{\pgfqpoint{0.000000in}{0.000000in}}%
\pgfpathclose%
\pgfusepath{fill}%
\end{pgfscope}%
\begin{pgfscope}%
\pgfsetbuttcap%
\pgfsetmiterjoin%
\definecolor{currentfill}{rgb}{1.000000,1.000000,1.000000}%
\pgfsetfillcolor{currentfill}%
\pgfsetlinewidth{0.000000pt}%
\definecolor{currentstroke}{rgb}{0.000000,0.000000,0.000000}%
\pgfsetstrokecolor{currentstroke}%
\pgfsetstrokeopacity{0.000000}%
\pgfsetdash{}{0pt}%
\pgfpathmoveto{\pgfqpoint{0.445556in}{0.499444in}}%
\pgfpathlineto{\pgfqpoint{4.320556in}{0.499444in}}%
\pgfpathlineto{\pgfqpoint{4.320556in}{1.654444in}}%
\pgfpathlineto{\pgfqpoint{0.445556in}{1.654444in}}%
\pgfpathlineto{\pgfqpoint{0.445556in}{0.499444in}}%
\pgfpathclose%
\pgfusepath{fill}%
\end{pgfscope}%
\begin{pgfscope}%
\pgfpathrectangle{\pgfqpoint{0.445556in}{0.499444in}}{\pgfqpoint{3.875000in}{1.155000in}}%
\pgfusepath{clip}%
\pgfsetbuttcap%
\pgfsetmiterjoin%
\pgfsetlinewidth{1.003750pt}%
\definecolor{currentstroke}{rgb}{0.000000,0.000000,0.000000}%
\pgfsetstrokecolor{currentstroke}%
\pgfsetdash{}{0pt}%
\pgfpathmoveto{\pgfqpoint{0.435556in}{0.499444in}}%
\pgfpathlineto{\pgfqpoint{0.483922in}{0.499444in}}%
\pgfpathlineto{\pgfqpoint{0.483922in}{0.504457in}}%
\pgfpathlineto{\pgfqpoint{0.435556in}{0.504457in}}%
\pgfusepath{stroke}%
\end{pgfscope}%
\begin{pgfscope}%
\pgfpathrectangle{\pgfqpoint{0.445556in}{0.499444in}}{\pgfqpoint{3.875000in}{1.155000in}}%
\pgfusepath{clip}%
\pgfsetbuttcap%
\pgfsetmiterjoin%
\pgfsetlinewidth{1.003750pt}%
\definecolor{currentstroke}{rgb}{0.000000,0.000000,0.000000}%
\pgfsetstrokecolor{currentstroke}%
\pgfsetdash{}{0pt}%
\pgfpathmoveto{\pgfqpoint{0.576001in}{0.499444in}}%
\pgfpathlineto{\pgfqpoint{0.637387in}{0.499444in}}%
\pgfpathlineto{\pgfqpoint{0.637387in}{0.661891in}}%
\pgfpathlineto{\pgfqpoint{0.576001in}{0.661891in}}%
\pgfpathlineto{\pgfqpoint{0.576001in}{0.499444in}}%
\pgfpathclose%
\pgfusepath{stroke}%
\end{pgfscope}%
\begin{pgfscope}%
\pgfpathrectangle{\pgfqpoint{0.445556in}{0.499444in}}{\pgfqpoint{3.875000in}{1.155000in}}%
\pgfusepath{clip}%
\pgfsetbuttcap%
\pgfsetmiterjoin%
\pgfsetlinewidth{1.003750pt}%
\definecolor{currentstroke}{rgb}{0.000000,0.000000,0.000000}%
\pgfsetstrokecolor{currentstroke}%
\pgfsetdash{}{0pt}%
\pgfpathmoveto{\pgfqpoint{0.729467in}{0.499444in}}%
\pgfpathlineto{\pgfqpoint{0.790853in}{0.499444in}}%
\pgfpathlineto{\pgfqpoint{0.790853in}{0.959762in}}%
\pgfpathlineto{\pgfqpoint{0.729467in}{0.959762in}}%
\pgfpathlineto{\pgfqpoint{0.729467in}{0.499444in}}%
\pgfpathclose%
\pgfusepath{stroke}%
\end{pgfscope}%
\begin{pgfscope}%
\pgfpathrectangle{\pgfqpoint{0.445556in}{0.499444in}}{\pgfqpoint{3.875000in}{1.155000in}}%
\pgfusepath{clip}%
\pgfsetbuttcap%
\pgfsetmiterjoin%
\pgfsetlinewidth{1.003750pt}%
\definecolor{currentstroke}{rgb}{0.000000,0.000000,0.000000}%
\pgfsetstrokecolor{currentstroke}%
\pgfsetdash{}{0pt}%
\pgfpathmoveto{\pgfqpoint{0.882932in}{0.499444in}}%
\pgfpathlineto{\pgfqpoint{0.944318in}{0.499444in}}%
\pgfpathlineto{\pgfqpoint{0.944318in}{1.239540in}}%
\pgfpathlineto{\pgfqpoint{0.882932in}{1.239540in}}%
\pgfpathlineto{\pgfqpoint{0.882932in}{0.499444in}}%
\pgfpathclose%
\pgfusepath{stroke}%
\end{pgfscope}%
\begin{pgfscope}%
\pgfpathrectangle{\pgfqpoint{0.445556in}{0.499444in}}{\pgfqpoint{3.875000in}{1.155000in}}%
\pgfusepath{clip}%
\pgfsetbuttcap%
\pgfsetmiterjoin%
\pgfsetlinewidth{1.003750pt}%
\definecolor{currentstroke}{rgb}{0.000000,0.000000,0.000000}%
\pgfsetstrokecolor{currentstroke}%
\pgfsetdash{}{0pt}%
\pgfpathmoveto{\pgfqpoint{1.036397in}{0.499444in}}%
\pgfpathlineto{\pgfqpoint{1.097783in}{0.499444in}}%
\pgfpathlineto{\pgfqpoint{1.097783in}{1.414988in}}%
\pgfpathlineto{\pgfqpoint{1.036397in}{1.414988in}}%
\pgfpathlineto{\pgfqpoint{1.036397in}{0.499444in}}%
\pgfpathclose%
\pgfusepath{stroke}%
\end{pgfscope}%
\begin{pgfscope}%
\pgfpathrectangle{\pgfqpoint{0.445556in}{0.499444in}}{\pgfqpoint{3.875000in}{1.155000in}}%
\pgfusepath{clip}%
\pgfsetbuttcap%
\pgfsetmiterjoin%
\pgfsetlinewidth{1.003750pt}%
\definecolor{currentstroke}{rgb}{0.000000,0.000000,0.000000}%
\pgfsetstrokecolor{currentstroke}%
\pgfsetdash{}{0pt}%
\pgfpathmoveto{\pgfqpoint{1.189863in}{0.499444in}}%
\pgfpathlineto{\pgfqpoint{1.251249in}{0.499444in}}%
\pgfpathlineto{\pgfqpoint{1.251249in}{1.517203in}}%
\pgfpathlineto{\pgfqpoint{1.189863in}{1.517203in}}%
\pgfpathlineto{\pgfqpoint{1.189863in}{0.499444in}}%
\pgfpathclose%
\pgfusepath{stroke}%
\end{pgfscope}%
\begin{pgfscope}%
\pgfpathrectangle{\pgfqpoint{0.445556in}{0.499444in}}{\pgfqpoint{3.875000in}{1.155000in}}%
\pgfusepath{clip}%
\pgfsetbuttcap%
\pgfsetmiterjoin%
\pgfsetlinewidth{1.003750pt}%
\definecolor{currentstroke}{rgb}{0.000000,0.000000,0.000000}%
\pgfsetstrokecolor{currentstroke}%
\pgfsetdash{}{0pt}%
\pgfpathmoveto{\pgfqpoint{1.343328in}{0.499444in}}%
\pgfpathlineto{\pgfqpoint{1.404714in}{0.499444in}}%
\pgfpathlineto{\pgfqpoint{1.404714in}{1.575947in}}%
\pgfpathlineto{\pgfqpoint{1.343328in}{1.575947in}}%
\pgfpathlineto{\pgfqpoint{1.343328in}{0.499444in}}%
\pgfpathclose%
\pgfusepath{stroke}%
\end{pgfscope}%
\begin{pgfscope}%
\pgfpathrectangle{\pgfqpoint{0.445556in}{0.499444in}}{\pgfqpoint{3.875000in}{1.155000in}}%
\pgfusepath{clip}%
\pgfsetbuttcap%
\pgfsetmiterjoin%
\pgfsetlinewidth{1.003750pt}%
\definecolor{currentstroke}{rgb}{0.000000,0.000000,0.000000}%
\pgfsetstrokecolor{currentstroke}%
\pgfsetdash{}{0pt}%
\pgfpathmoveto{\pgfqpoint{1.496793in}{0.499444in}}%
\pgfpathlineto{\pgfqpoint{1.558179in}{0.499444in}}%
\pgfpathlineto{\pgfqpoint{1.558179in}{1.599444in}}%
\pgfpathlineto{\pgfqpoint{1.496793in}{1.599444in}}%
\pgfpathlineto{\pgfqpoint{1.496793in}{0.499444in}}%
\pgfpathclose%
\pgfusepath{stroke}%
\end{pgfscope}%
\begin{pgfscope}%
\pgfpathrectangle{\pgfqpoint{0.445556in}{0.499444in}}{\pgfqpoint{3.875000in}{1.155000in}}%
\pgfusepath{clip}%
\pgfsetbuttcap%
\pgfsetmiterjoin%
\pgfsetlinewidth{1.003750pt}%
\definecolor{currentstroke}{rgb}{0.000000,0.000000,0.000000}%
\pgfsetstrokecolor{currentstroke}%
\pgfsetdash{}{0pt}%
\pgfpathmoveto{\pgfqpoint{1.650259in}{0.499444in}}%
\pgfpathlineto{\pgfqpoint{1.711645in}{0.499444in}}%
\pgfpathlineto{\pgfqpoint{1.711645in}{1.586364in}}%
\pgfpathlineto{\pgfqpoint{1.650259in}{1.586364in}}%
\pgfpathlineto{\pgfqpoint{1.650259in}{0.499444in}}%
\pgfpathclose%
\pgfusepath{stroke}%
\end{pgfscope}%
\begin{pgfscope}%
\pgfpathrectangle{\pgfqpoint{0.445556in}{0.499444in}}{\pgfqpoint{3.875000in}{1.155000in}}%
\pgfusepath{clip}%
\pgfsetbuttcap%
\pgfsetmiterjoin%
\pgfsetlinewidth{1.003750pt}%
\definecolor{currentstroke}{rgb}{0.000000,0.000000,0.000000}%
\pgfsetstrokecolor{currentstroke}%
\pgfsetdash{}{0pt}%
\pgfpathmoveto{\pgfqpoint{1.803724in}{0.499444in}}%
\pgfpathlineto{\pgfqpoint{1.865110in}{0.499444in}}%
\pgfpathlineto{\pgfqpoint{1.865110in}{1.539995in}}%
\pgfpathlineto{\pgfqpoint{1.803724in}{1.539995in}}%
\pgfpathlineto{\pgfqpoint{1.803724in}{0.499444in}}%
\pgfpathclose%
\pgfusepath{stroke}%
\end{pgfscope}%
\begin{pgfscope}%
\pgfpathrectangle{\pgfqpoint{0.445556in}{0.499444in}}{\pgfqpoint{3.875000in}{1.155000in}}%
\pgfusepath{clip}%
\pgfsetbuttcap%
\pgfsetmiterjoin%
\pgfsetlinewidth{1.003750pt}%
\definecolor{currentstroke}{rgb}{0.000000,0.000000,0.000000}%
\pgfsetstrokecolor{currentstroke}%
\pgfsetdash{}{0pt}%
\pgfpathmoveto{\pgfqpoint{1.957189in}{0.499444in}}%
\pgfpathlineto{\pgfqpoint{2.018575in}{0.499444in}}%
\pgfpathlineto{\pgfqpoint{2.018575in}{1.459164in}}%
\pgfpathlineto{\pgfqpoint{1.957189in}{1.459164in}}%
\pgfpathlineto{\pgfqpoint{1.957189in}{0.499444in}}%
\pgfpathclose%
\pgfusepath{stroke}%
\end{pgfscope}%
\begin{pgfscope}%
\pgfpathrectangle{\pgfqpoint{0.445556in}{0.499444in}}{\pgfqpoint{3.875000in}{1.155000in}}%
\pgfusepath{clip}%
\pgfsetbuttcap%
\pgfsetmiterjoin%
\pgfsetlinewidth{1.003750pt}%
\definecolor{currentstroke}{rgb}{0.000000,0.000000,0.000000}%
\pgfsetstrokecolor{currentstroke}%
\pgfsetdash{}{0pt}%
\pgfpathmoveto{\pgfqpoint{2.110655in}{0.499444in}}%
\pgfpathlineto{\pgfqpoint{2.172041in}{0.499444in}}%
\pgfpathlineto{\pgfqpoint{2.172041in}{1.390002in}}%
\pgfpathlineto{\pgfqpoint{2.110655in}{1.390002in}}%
\pgfpathlineto{\pgfqpoint{2.110655in}{0.499444in}}%
\pgfpathclose%
\pgfusepath{stroke}%
\end{pgfscope}%
\begin{pgfscope}%
\pgfpathrectangle{\pgfqpoint{0.445556in}{0.499444in}}{\pgfqpoint{3.875000in}{1.155000in}}%
\pgfusepath{clip}%
\pgfsetbuttcap%
\pgfsetmiterjoin%
\pgfsetlinewidth{1.003750pt}%
\definecolor{currentstroke}{rgb}{0.000000,0.000000,0.000000}%
\pgfsetstrokecolor{currentstroke}%
\pgfsetdash{}{0pt}%
\pgfpathmoveto{\pgfqpoint{2.264120in}{0.499444in}}%
\pgfpathlineto{\pgfqpoint{2.325506in}{0.499444in}}%
\pgfpathlineto{\pgfqpoint{2.325506in}{1.316847in}}%
\pgfpathlineto{\pgfqpoint{2.264120in}{1.316847in}}%
\pgfpathlineto{\pgfqpoint{2.264120in}{0.499444in}}%
\pgfpathclose%
\pgfusepath{stroke}%
\end{pgfscope}%
\begin{pgfscope}%
\pgfpathrectangle{\pgfqpoint{0.445556in}{0.499444in}}{\pgfqpoint{3.875000in}{1.155000in}}%
\pgfusepath{clip}%
\pgfsetbuttcap%
\pgfsetmiterjoin%
\pgfsetlinewidth{1.003750pt}%
\definecolor{currentstroke}{rgb}{0.000000,0.000000,0.000000}%
\pgfsetstrokecolor{currentstroke}%
\pgfsetdash{}{0pt}%
\pgfpathmoveto{\pgfqpoint{2.417585in}{0.499444in}}%
\pgfpathlineto{\pgfqpoint{2.478972in}{0.499444in}}%
\pgfpathlineto{\pgfqpoint{2.478972in}{1.230454in}}%
\pgfpathlineto{\pgfqpoint{2.417585in}{1.230454in}}%
\pgfpathlineto{\pgfqpoint{2.417585in}{0.499444in}}%
\pgfpathclose%
\pgfusepath{stroke}%
\end{pgfscope}%
\begin{pgfscope}%
\pgfpathrectangle{\pgfqpoint{0.445556in}{0.499444in}}{\pgfqpoint{3.875000in}{1.155000in}}%
\pgfusepath{clip}%
\pgfsetbuttcap%
\pgfsetmiterjoin%
\pgfsetlinewidth{1.003750pt}%
\definecolor{currentstroke}{rgb}{0.000000,0.000000,0.000000}%
\pgfsetstrokecolor{currentstroke}%
\pgfsetdash{}{0pt}%
\pgfpathmoveto{\pgfqpoint{2.571051in}{0.499444in}}%
\pgfpathlineto{\pgfqpoint{2.632437in}{0.499444in}}%
\pgfpathlineto{\pgfqpoint{2.632437in}{1.140380in}}%
\pgfpathlineto{\pgfqpoint{2.571051in}{1.140380in}}%
\pgfpathlineto{\pgfqpoint{2.571051in}{0.499444in}}%
\pgfpathclose%
\pgfusepath{stroke}%
\end{pgfscope}%
\begin{pgfscope}%
\pgfpathrectangle{\pgfqpoint{0.445556in}{0.499444in}}{\pgfqpoint{3.875000in}{1.155000in}}%
\pgfusepath{clip}%
\pgfsetbuttcap%
\pgfsetmiterjoin%
\pgfsetlinewidth{1.003750pt}%
\definecolor{currentstroke}{rgb}{0.000000,0.000000,0.000000}%
\pgfsetstrokecolor{currentstroke}%
\pgfsetdash{}{0pt}%
\pgfpathmoveto{\pgfqpoint{2.724516in}{0.499444in}}%
\pgfpathlineto{\pgfqpoint{2.785902in}{0.499444in}}%
\pgfpathlineto{\pgfqpoint{2.785902in}{1.056337in}}%
\pgfpathlineto{\pgfqpoint{2.724516in}{1.056337in}}%
\pgfpathlineto{\pgfqpoint{2.724516in}{0.499444in}}%
\pgfpathclose%
\pgfusepath{stroke}%
\end{pgfscope}%
\begin{pgfscope}%
\pgfpathrectangle{\pgfqpoint{0.445556in}{0.499444in}}{\pgfqpoint{3.875000in}{1.155000in}}%
\pgfusepath{clip}%
\pgfsetbuttcap%
\pgfsetmiterjoin%
\pgfsetlinewidth{1.003750pt}%
\definecolor{currentstroke}{rgb}{0.000000,0.000000,0.000000}%
\pgfsetstrokecolor{currentstroke}%
\pgfsetdash{}{0pt}%
\pgfpathmoveto{\pgfqpoint{2.877981in}{0.499444in}}%
\pgfpathlineto{\pgfqpoint{2.939368in}{0.499444in}}%
\pgfpathlineto{\pgfqpoint{2.939368in}{0.973625in}}%
\pgfpathlineto{\pgfqpoint{2.877981in}{0.973625in}}%
\pgfpathlineto{\pgfqpoint{2.877981in}{0.499444in}}%
\pgfpathclose%
\pgfusepath{stroke}%
\end{pgfscope}%
\begin{pgfscope}%
\pgfpathrectangle{\pgfqpoint{0.445556in}{0.499444in}}{\pgfqpoint{3.875000in}{1.155000in}}%
\pgfusepath{clip}%
\pgfsetbuttcap%
\pgfsetmiterjoin%
\pgfsetlinewidth{1.003750pt}%
\definecolor{currentstroke}{rgb}{0.000000,0.000000,0.000000}%
\pgfsetstrokecolor{currentstroke}%
\pgfsetdash{}{0pt}%
\pgfpathmoveto{\pgfqpoint{3.031447in}{0.499444in}}%
\pgfpathlineto{\pgfqpoint{3.092833in}{0.499444in}}%
\pgfpathlineto{\pgfqpoint{3.092833in}{0.893342in}}%
\pgfpathlineto{\pgfqpoint{3.031447in}{0.893342in}}%
\pgfpathlineto{\pgfqpoint{3.031447in}{0.499444in}}%
\pgfpathclose%
\pgfusepath{stroke}%
\end{pgfscope}%
\begin{pgfscope}%
\pgfpathrectangle{\pgfqpoint{0.445556in}{0.499444in}}{\pgfqpoint{3.875000in}{1.155000in}}%
\pgfusepath{clip}%
\pgfsetbuttcap%
\pgfsetmiterjoin%
\pgfsetlinewidth{1.003750pt}%
\definecolor{currentstroke}{rgb}{0.000000,0.000000,0.000000}%
\pgfsetstrokecolor{currentstroke}%
\pgfsetdash{}{0pt}%
\pgfpathmoveto{\pgfqpoint{3.184912in}{0.499444in}}%
\pgfpathlineto{\pgfqpoint{3.246298in}{0.499444in}}%
\pgfpathlineto{\pgfqpoint{3.246298in}{0.818463in}}%
\pgfpathlineto{\pgfqpoint{3.184912in}{0.818463in}}%
\pgfpathlineto{\pgfqpoint{3.184912in}{0.499444in}}%
\pgfpathclose%
\pgfusepath{stroke}%
\end{pgfscope}%
\begin{pgfscope}%
\pgfpathrectangle{\pgfqpoint{0.445556in}{0.499444in}}{\pgfqpoint{3.875000in}{1.155000in}}%
\pgfusepath{clip}%
\pgfsetbuttcap%
\pgfsetmiterjoin%
\pgfsetlinewidth{1.003750pt}%
\definecolor{currentstroke}{rgb}{0.000000,0.000000,0.000000}%
\pgfsetstrokecolor{currentstroke}%
\pgfsetdash{}{0pt}%
\pgfpathmoveto{\pgfqpoint{3.338377in}{0.499444in}}%
\pgfpathlineto{\pgfqpoint{3.399764in}{0.499444in}}%
\pgfpathlineto{\pgfqpoint{3.399764in}{0.752591in}}%
\pgfpathlineto{\pgfqpoint{3.338377in}{0.752591in}}%
\pgfpathlineto{\pgfqpoint{3.338377in}{0.499444in}}%
\pgfpathclose%
\pgfusepath{stroke}%
\end{pgfscope}%
\begin{pgfscope}%
\pgfpathrectangle{\pgfqpoint{0.445556in}{0.499444in}}{\pgfqpoint{3.875000in}{1.155000in}}%
\pgfusepath{clip}%
\pgfsetbuttcap%
\pgfsetmiterjoin%
\pgfsetlinewidth{1.003750pt}%
\definecolor{currentstroke}{rgb}{0.000000,0.000000,0.000000}%
\pgfsetstrokecolor{currentstroke}%
\pgfsetdash{}{0pt}%
\pgfpathmoveto{\pgfqpoint{3.491843in}{0.499444in}}%
\pgfpathlineto{\pgfqpoint{3.553229in}{0.499444in}}%
\pgfpathlineto{\pgfqpoint{3.553229in}{0.696667in}}%
\pgfpathlineto{\pgfqpoint{3.491843in}{0.696667in}}%
\pgfpathlineto{\pgfqpoint{3.491843in}{0.499444in}}%
\pgfpathclose%
\pgfusepath{stroke}%
\end{pgfscope}%
\begin{pgfscope}%
\pgfpathrectangle{\pgfqpoint{0.445556in}{0.499444in}}{\pgfqpoint{3.875000in}{1.155000in}}%
\pgfusepath{clip}%
\pgfsetbuttcap%
\pgfsetmiterjoin%
\pgfsetlinewidth{1.003750pt}%
\definecolor{currentstroke}{rgb}{0.000000,0.000000,0.000000}%
\pgfsetstrokecolor{currentstroke}%
\pgfsetdash{}{0pt}%
\pgfpathmoveto{\pgfqpoint{3.645308in}{0.499444in}}%
\pgfpathlineto{\pgfqpoint{3.706694in}{0.499444in}}%
\pgfpathlineto{\pgfqpoint{3.706694in}{0.641056in}}%
\pgfpathlineto{\pgfqpoint{3.645308in}{0.641056in}}%
\pgfpathlineto{\pgfqpoint{3.645308in}{0.499444in}}%
\pgfpathclose%
\pgfusepath{stroke}%
\end{pgfscope}%
\begin{pgfscope}%
\pgfpathrectangle{\pgfqpoint{0.445556in}{0.499444in}}{\pgfqpoint{3.875000in}{1.155000in}}%
\pgfusepath{clip}%
\pgfsetbuttcap%
\pgfsetmiterjoin%
\pgfsetlinewidth{1.003750pt}%
\definecolor{currentstroke}{rgb}{0.000000,0.000000,0.000000}%
\pgfsetstrokecolor{currentstroke}%
\pgfsetdash{}{0pt}%
\pgfpathmoveto{\pgfqpoint{3.798774in}{0.499444in}}%
\pgfpathlineto{\pgfqpoint{3.860160in}{0.499444in}}%
\pgfpathlineto{\pgfqpoint{3.860160in}{0.588422in}}%
\pgfpathlineto{\pgfqpoint{3.798774in}{0.588422in}}%
\pgfpathlineto{\pgfqpoint{3.798774in}{0.499444in}}%
\pgfpathclose%
\pgfusepath{stroke}%
\end{pgfscope}%
\begin{pgfscope}%
\pgfpathrectangle{\pgfqpoint{0.445556in}{0.499444in}}{\pgfqpoint{3.875000in}{1.155000in}}%
\pgfusepath{clip}%
\pgfsetbuttcap%
\pgfsetmiterjoin%
\pgfsetlinewidth{1.003750pt}%
\definecolor{currentstroke}{rgb}{0.000000,0.000000,0.000000}%
\pgfsetstrokecolor{currentstroke}%
\pgfsetdash{}{0pt}%
\pgfpathmoveto{\pgfqpoint{3.952239in}{0.499444in}}%
\pgfpathlineto{\pgfqpoint{4.013625in}{0.499444in}}%
\pgfpathlineto{\pgfqpoint{4.013625in}{0.541035in}}%
\pgfpathlineto{\pgfqpoint{3.952239in}{0.541035in}}%
\pgfpathlineto{\pgfqpoint{3.952239in}{0.499444in}}%
\pgfpathclose%
\pgfusepath{stroke}%
\end{pgfscope}%
\begin{pgfscope}%
\pgfpathrectangle{\pgfqpoint{0.445556in}{0.499444in}}{\pgfqpoint{3.875000in}{1.155000in}}%
\pgfusepath{clip}%
\pgfsetbuttcap%
\pgfsetmiterjoin%
\pgfsetlinewidth{1.003750pt}%
\definecolor{currentstroke}{rgb}{0.000000,0.000000,0.000000}%
\pgfsetstrokecolor{currentstroke}%
\pgfsetdash{}{0pt}%
\pgfpathmoveto{\pgfqpoint{4.105704in}{0.499444in}}%
\pgfpathlineto{\pgfqpoint{4.167090in}{0.499444in}}%
\pgfpathlineto{\pgfqpoint{4.167090in}{0.505867in}}%
\pgfpathlineto{\pgfqpoint{4.105704in}{0.505867in}}%
\pgfpathlineto{\pgfqpoint{4.105704in}{0.499444in}}%
\pgfpathclose%
\pgfusepath{stroke}%
\end{pgfscope}%
\begin{pgfscope}%
\pgfpathrectangle{\pgfqpoint{0.445556in}{0.499444in}}{\pgfqpoint{3.875000in}{1.155000in}}%
\pgfusepath{clip}%
\pgfsetbuttcap%
\pgfsetmiterjoin%
\definecolor{currentfill}{rgb}{0.000000,0.000000,0.000000}%
\pgfsetfillcolor{currentfill}%
\pgfsetlinewidth{0.000000pt}%
\definecolor{currentstroke}{rgb}{0.000000,0.000000,0.000000}%
\pgfsetstrokecolor{currentstroke}%
\pgfsetstrokeopacity{0.000000}%
\pgfsetdash{}{0pt}%
\pgfpathmoveto{\pgfqpoint{0.483922in}{0.499444in}}%
\pgfpathlineto{\pgfqpoint{0.545308in}{0.499444in}}%
\pgfpathlineto{\pgfqpoint{0.545308in}{0.499601in}}%
\pgfpathlineto{\pgfqpoint{0.483922in}{0.499601in}}%
\pgfpathlineto{\pgfqpoint{0.483922in}{0.499444in}}%
\pgfpathclose%
\pgfusepath{fill}%
\end{pgfscope}%
\begin{pgfscope}%
\pgfpathrectangle{\pgfqpoint{0.445556in}{0.499444in}}{\pgfqpoint{3.875000in}{1.155000in}}%
\pgfusepath{clip}%
\pgfsetbuttcap%
\pgfsetmiterjoin%
\definecolor{currentfill}{rgb}{0.000000,0.000000,0.000000}%
\pgfsetfillcolor{currentfill}%
\pgfsetlinewidth{0.000000pt}%
\definecolor{currentstroke}{rgb}{0.000000,0.000000,0.000000}%
\pgfsetstrokecolor{currentstroke}%
\pgfsetstrokeopacity{0.000000}%
\pgfsetdash{}{0pt}%
\pgfpathmoveto{\pgfqpoint{0.637387in}{0.499444in}}%
\pgfpathlineto{\pgfqpoint{0.698774in}{0.499444in}}%
\pgfpathlineto{\pgfqpoint{0.698774in}{0.500854in}}%
\pgfpathlineto{\pgfqpoint{0.637387in}{0.500854in}}%
\pgfpathlineto{\pgfqpoint{0.637387in}{0.499444in}}%
\pgfpathclose%
\pgfusepath{fill}%
\end{pgfscope}%
\begin{pgfscope}%
\pgfpathrectangle{\pgfqpoint{0.445556in}{0.499444in}}{\pgfqpoint{3.875000in}{1.155000in}}%
\pgfusepath{clip}%
\pgfsetbuttcap%
\pgfsetmiterjoin%
\definecolor{currentfill}{rgb}{0.000000,0.000000,0.000000}%
\pgfsetfillcolor{currentfill}%
\pgfsetlinewidth{0.000000pt}%
\definecolor{currentstroke}{rgb}{0.000000,0.000000,0.000000}%
\pgfsetstrokecolor{currentstroke}%
\pgfsetstrokeopacity{0.000000}%
\pgfsetdash{}{0pt}%
\pgfpathmoveto{\pgfqpoint{0.790853in}{0.499444in}}%
\pgfpathlineto{\pgfqpoint{0.852239in}{0.499444in}}%
\pgfpathlineto{\pgfqpoint{0.852239in}{0.506650in}}%
\pgfpathlineto{\pgfqpoint{0.790853in}{0.506650in}}%
\pgfpathlineto{\pgfqpoint{0.790853in}{0.499444in}}%
\pgfpathclose%
\pgfusepath{fill}%
\end{pgfscope}%
\begin{pgfscope}%
\pgfpathrectangle{\pgfqpoint{0.445556in}{0.499444in}}{\pgfqpoint{3.875000in}{1.155000in}}%
\pgfusepath{clip}%
\pgfsetbuttcap%
\pgfsetmiterjoin%
\definecolor{currentfill}{rgb}{0.000000,0.000000,0.000000}%
\pgfsetfillcolor{currentfill}%
\pgfsetlinewidth{0.000000pt}%
\definecolor{currentstroke}{rgb}{0.000000,0.000000,0.000000}%
\pgfsetstrokecolor{currentstroke}%
\pgfsetstrokeopacity{0.000000}%
\pgfsetdash{}{0pt}%
\pgfpathmoveto{\pgfqpoint{0.944318in}{0.499444in}}%
\pgfpathlineto{\pgfqpoint{1.005704in}{0.499444in}}%
\pgfpathlineto{\pgfqpoint{1.005704in}{0.517459in}}%
\pgfpathlineto{\pgfqpoint{0.944318in}{0.517459in}}%
\pgfpathlineto{\pgfqpoint{0.944318in}{0.499444in}}%
\pgfpathclose%
\pgfusepath{fill}%
\end{pgfscope}%
\begin{pgfscope}%
\pgfpathrectangle{\pgfqpoint{0.445556in}{0.499444in}}{\pgfqpoint{3.875000in}{1.155000in}}%
\pgfusepath{clip}%
\pgfsetbuttcap%
\pgfsetmiterjoin%
\definecolor{currentfill}{rgb}{0.000000,0.000000,0.000000}%
\pgfsetfillcolor{currentfill}%
\pgfsetlinewidth{0.000000pt}%
\definecolor{currentstroke}{rgb}{0.000000,0.000000,0.000000}%
\pgfsetstrokecolor{currentstroke}%
\pgfsetstrokeopacity{0.000000}%
\pgfsetdash{}{0pt}%
\pgfpathmoveto{\pgfqpoint{1.097783in}{0.499444in}}%
\pgfpathlineto{\pgfqpoint{1.159170in}{0.499444in}}%
\pgfpathlineto{\pgfqpoint{1.159170in}{0.531793in}}%
\pgfpathlineto{\pgfqpoint{1.097783in}{0.531793in}}%
\pgfpathlineto{\pgfqpoint{1.097783in}{0.499444in}}%
\pgfpathclose%
\pgfusepath{fill}%
\end{pgfscope}%
\begin{pgfscope}%
\pgfpathrectangle{\pgfqpoint{0.445556in}{0.499444in}}{\pgfqpoint{3.875000in}{1.155000in}}%
\pgfusepath{clip}%
\pgfsetbuttcap%
\pgfsetmiterjoin%
\definecolor{currentfill}{rgb}{0.000000,0.000000,0.000000}%
\pgfsetfillcolor{currentfill}%
\pgfsetlinewidth{0.000000pt}%
\definecolor{currentstroke}{rgb}{0.000000,0.000000,0.000000}%
\pgfsetstrokecolor{currentstroke}%
\pgfsetstrokeopacity{0.000000}%
\pgfsetdash{}{0pt}%
\pgfpathmoveto{\pgfqpoint{1.251249in}{0.499444in}}%
\pgfpathlineto{\pgfqpoint{1.312635in}{0.499444in}}%
\pgfpathlineto{\pgfqpoint{1.312635in}{0.547301in}}%
\pgfpathlineto{\pgfqpoint{1.251249in}{0.547301in}}%
\pgfpathlineto{\pgfqpoint{1.251249in}{0.499444in}}%
\pgfpathclose%
\pgfusepath{fill}%
\end{pgfscope}%
\begin{pgfscope}%
\pgfpathrectangle{\pgfqpoint{0.445556in}{0.499444in}}{\pgfqpoint{3.875000in}{1.155000in}}%
\pgfusepath{clip}%
\pgfsetbuttcap%
\pgfsetmiterjoin%
\definecolor{currentfill}{rgb}{0.000000,0.000000,0.000000}%
\pgfsetfillcolor{currentfill}%
\pgfsetlinewidth{0.000000pt}%
\definecolor{currentstroke}{rgb}{0.000000,0.000000,0.000000}%
\pgfsetstrokecolor{currentstroke}%
\pgfsetstrokeopacity{0.000000}%
\pgfsetdash{}{0pt}%
\pgfpathmoveto{\pgfqpoint{1.404714in}{0.499444in}}%
\pgfpathlineto{\pgfqpoint{1.466100in}{0.499444in}}%
\pgfpathlineto{\pgfqpoint{1.466100in}{0.569780in}}%
\pgfpathlineto{\pgfqpoint{1.404714in}{0.569780in}}%
\pgfpathlineto{\pgfqpoint{1.404714in}{0.499444in}}%
\pgfpathclose%
\pgfusepath{fill}%
\end{pgfscope}%
\begin{pgfscope}%
\pgfpathrectangle{\pgfqpoint{0.445556in}{0.499444in}}{\pgfqpoint{3.875000in}{1.155000in}}%
\pgfusepath{clip}%
\pgfsetbuttcap%
\pgfsetmiterjoin%
\definecolor{currentfill}{rgb}{0.000000,0.000000,0.000000}%
\pgfsetfillcolor{currentfill}%
\pgfsetlinewidth{0.000000pt}%
\definecolor{currentstroke}{rgb}{0.000000,0.000000,0.000000}%
\pgfsetstrokecolor{currentstroke}%
\pgfsetstrokeopacity{0.000000}%
\pgfsetdash{}{0pt}%
\pgfpathmoveto{\pgfqpoint{1.558179in}{0.499444in}}%
\pgfpathlineto{\pgfqpoint{1.619566in}{0.499444in}}%
\pgfpathlineto{\pgfqpoint{1.619566in}{0.589518in}}%
\pgfpathlineto{\pgfqpoint{1.558179in}{0.589518in}}%
\pgfpathlineto{\pgfqpoint{1.558179in}{0.499444in}}%
\pgfpathclose%
\pgfusepath{fill}%
\end{pgfscope}%
\begin{pgfscope}%
\pgfpathrectangle{\pgfqpoint{0.445556in}{0.499444in}}{\pgfqpoint{3.875000in}{1.155000in}}%
\pgfusepath{clip}%
\pgfsetbuttcap%
\pgfsetmiterjoin%
\definecolor{currentfill}{rgb}{0.000000,0.000000,0.000000}%
\pgfsetfillcolor{currentfill}%
\pgfsetlinewidth{0.000000pt}%
\definecolor{currentstroke}{rgb}{0.000000,0.000000,0.000000}%
\pgfsetstrokecolor{currentstroke}%
\pgfsetstrokeopacity{0.000000}%
\pgfsetdash{}{0pt}%
\pgfpathmoveto{\pgfqpoint{1.711645in}{0.499444in}}%
\pgfpathlineto{\pgfqpoint{1.773031in}{0.499444in}}%
\pgfpathlineto{\pgfqpoint{1.773031in}{0.600797in}}%
\pgfpathlineto{\pgfqpoint{1.711645in}{0.600797in}}%
\pgfpathlineto{\pgfqpoint{1.711645in}{0.499444in}}%
\pgfpathclose%
\pgfusepath{fill}%
\end{pgfscope}%
\begin{pgfscope}%
\pgfpathrectangle{\pgfqpoint{0.445556in}{0.499444in}}{\pgfqpoint{3.875000in}{1.155000in}}%
\pgfusepath{clip}%
\pgfsetbuttcap%
\pgfsetmiterjoin%
\definecolor{currentfill}{rgb}{0.000000,0.000000,0.000000}%
\pgfsetfillcolor{currentfill}%
\pgfsetlinewidth{0.000000pt}%
\definecolor{currentstroke}{rgb}{0.000000,0.000000,0.000000}%
\pgfsetstrokecolor{currentstroke}%
\pgfsetstrokeopacity{0.000000}%
\pgfsetdash{}{0pt}%
\pgfpathmoveto{\pgfqpoint{1.865110in}{0.499444in}}%
\pgfpathlineto{\pgfqpoint{1.926496in}{0.499444in}}%
\pgfpathlineto{\pgfqpoint{1.926496in}{0.629777in}}%
\pgfpathlineto{\pgfqpoint{1.865110in}{0.629777in}}%
\pgfpathlineto{\pgfqpoint{1.865110in}{0.499444in}}%
\pgfpathclose%
\pgfusepath{fill}%
\end{pgfscope}%
\begin{pgfscope}%
\pgfpathrectangle{\pgfqpoint{0.445556in}{0.499444in}}{\pgfqpoint{3.875000in}{1.155000in}}%
\pgfusepath{clip}%
\pgfsetbuttcap%
\pgfsetmiterjoin%
\definecolor{currentfill}{rgb}{0.000000,0.000000,0.000000}%
\pgfsetfillcolor{currentfill}%
\pgfsetlinewidth{0.000000pt}%
\definecolor{currentstroke}{rgb}{0.000000,0.000000,0.000000}%
\pgfsetstrokecolor{currentstroke}%
\pgfsetstrokeopacity{0.000000}%
\pgfsetdash{}{0pt}%
\pgfpathmoveto{\pgfqpoint{2.018575in}{0.499444in}}%
\pgfpathlineto{\pgfqpoint{2.079962in}{0.499444in}}%
\pgfpathlineto{\pgfqpoint{2.079962in}{0.642466in}}%
\pgfpathlineto{\pgfqpoint{2.018575in}{0.642466in}}%
\pgfpathlineto{\pgfqpoint{2.018575in}{0.499444in}}%
\pgfpathclose%
\pgfusepath{fill}%
\end{pgfscope}%
\begin{pgfscope}%
\pgfpathrectangle{\pgfqpoint{0.445556in}{0.499444in}}{\pgfqpoint{3.875000in}{1.155000in}}%
\pgfusepath{clip}%
\pgfsetbuttcap%
\pgfsetmiterjoin%
\definecolor{currentfill}{rgb}{0.000000,0.000000,0.000000}%
\pgfsetfillcolor{currentfill}%
\pgfsetlinewidth{0.000000pt}%
\definecolor{currentstroke}{rgb}{0.000000,0.000000,0.000000}%
\pgfsetstrokecolor{currentstroke}%
\pgfsetstrokeopacity{0.000000}%
\pgfsetdash{}{0pt}%
\pgfpathmoveto{\pgfqpoint{2.172041in}{0.499444in}}%
\pgfpathlineto{\pgfqpoint{2.233427in}{0.499444in}}%
\pgfpathlineto{\pgfqpoint{2.233427in}{0.651943in}}%
\pgfpathlineto{\pgfqpoint{2.172041in}{0.651943in}}%
\pgfpathlineto{\pgfqpoint{2.172041in}{0.499444in}}%
\pgfpathclose%
\pgfusepath{fill}%
\end{pgfscope}%
\begin{pgfscope}%
\pgfpathrectangle{\pgfqpoint{0.445556in}{0.499444in}}{\pgfqpoint{3.875000in}{1.155000in}}%
\pgfusepath{clip}%
\pgfsetbuttcap%
\pgfsetmiterjoin%
\definecolor{currentfill}{rgb}{0.000000,0.000000,0.000000}%
\pgfsetfillcolor{currentfill}%
\pgfsetlinewidth{0.000000pt}%
\definecolor{currentstroke}{rgb}{0.000000,0.000000,0.000000}%
\pgfsetstrokecolor{currentstroke}%
\pgfsetstrokeopacity{0.000000}%
\pgfsetdash{}{0pt}%
\pgfpathmoveto{\pgfqpoint{2.325506in}{0.499444in}}%
\pgfpathlineto{\pgfqpoint{2.386892in}{0.499444in}}%
\pgfpathlineto{\pgfqpoint{2.386892in}{0.659384in}}%
\pgfpathlineto{\pgfqpoint{2.325506in}{0.659384in}}%
\pgfpathlineto{\pgfqpoint{2.325506in}{0.499444in}}%
\pgfpathclose%
\pgfusepath{fill}%
\end{pgfscope}%
\begin{pgfscope}%
\pgfpathrectangle{\pgfqpoint{0.445556in}{0.499444in}}{\pgfqpoint{3.875000in}{1.155000in}}%
\pgfusepath{clip}%
\pgfsetbuttcap%
\pgfsetmiterjoin%
\definecolor{currentfill}{rgb}{0.000000,0.000000,0.000000}%
\pgfsetfillcolor{currentfill}%
\pgfsetlinewidth{0.000000pt}%
\definecolor{currentstroke}{rgb}{0.000000,0.000000,0.000000}%
\pgfsetstrokecolor{currentstroke}%
\pgfsetstrokeopacity{0.000000}%
\pgfsetdash{}{0pt}%
\pgfpathmoveto{\pgfqpoint{2.478972in}{0.499444in}}%
\pgfpathlineto{\pgfqpoint{2.540358in}{0.499444in}}%
\pgfpathlineto{\pgfqpoint{2.540358in}{0.669802in}}%
\pgfpathlineto{\pgfqpoint{2.478972in}{0.669802in}}%
\pgfpathlineto{\pgfqpoint{2.478972in}{0.499444in}}%
\pgfpathclose%
\pgfusepath{fill}%
\end{pgfscope}%
\begin{pgfscope}%
\pgfpathrectangle{\pgfqpoint{0.445556in}{0.499444in}}{\pgfqpoint{3.875000in}{1.155000in}}%
\pgfusepath{clip}%
\pgfsetbuttcap%
\pgfsetmiterjoin%
\definecolor{currentfill}{rgb}{0.000000,0.000000,0.000000}%
\pgfsetfillcolor{currentfill}%
\pgfsetlinewidth{0.000000pt}%
\definecolor{currentstroke}{rgb}{0.000000,0.000000,0.000000}%
\pgfsetstrokecolor{currentstroke}%
\pgfsetstrokeopacity{0.000000}%
\pgfsetdash{}{0pt}%
\pgfpathmoveto{\pgfqpoint{2.632437in}{0.499444in}}%
\pgfpathlineto{\pgfqpoint{2.693823in}{0.499444in}}%
\pgfpathlineto{\pgfqpoint{2.693823in}{0.676538in}}%
\pgfpathlineto{\pgfqpoint{2.632437in}{0.676538in}}%
\pgfpathlineto{\pgfqpoint{2.632437in}{0.499444in}}%
\pgfpathclose%
\pgfusepath{fill}%
\end{pgfscope}%
\begin{pgfscope}%
\pgfpathrectangle{\pgfqpoint{0.445556in}{0.499444in}}{\pgfqpoint{3.875000in}{1.155000in}}%
\pgfusepath{clip}%
\pgfsetbuttcap%
\pgfsetmiterjoin%
\definecolor{currentfill}{rgb}{0.000000,0.000000,0.000000}%
\pgfsetfillcolor{currentfill}%
\pgfsetlinewidth{0.000000pt}%
\definecolor{currentstroke}{rgb}{0.000000,0.000000,0.000000}%
\pgfsetstrokecolor{currentstroke}%
\pgfsetstrokeopacity{0.000000}%
\pgfsetdash{}{0pt}%
\pgfpathmoveto{\pgfqpoint{2.785902in}{0.499444in}}%
\pgfpathlineto{\pgfqpoint{2.847288in}{0.499444in}}%
\pgfpathlineto{\pgfqpoint{2.847288in}{0.669097in}}%
\pgfpathlineto{\pgfqpoint{2.785902in}{0.669097in}}%
\pgfpathlineto{\pgfqpoint{2.785902in}{0.499444in}}%
\pgfpathclose%
\pgfusepath{fill}%
\end{pgfscope}%
\begin{pgfscope}%
\pgfpathrectangle{\pgfqpoint{0.445556in}{0.499444in}}{\pgfqpoint{3.875000in}{1.155000in}}%
\pgfusepath{clip}%
\pgfsetbuttcap%
\pgfsetmiterjoin%
\definecolor{currentfill}{rgb}{0.000000,0.000000,0.000000}%
\pgfsetfillcolor{currentfill}%
\pgfsetlinewidth{0.000000pt}%
\definecolor{currentstroke}{rgb}{0.000000,0.000000,0.000000}%
\pgfsetstrokecolor{currentstroke}%
\pgfsetstrokeopacity{0.000000}%
\pgfsetdash{}{0pt}%
\pgfpathmoveto{\pgfqpoint{2.939368in}{0.499444in}}%
\pgfpathlineto{\pgfqpoint{3.000754in}{0.499444in}}%
\pgfpathlineto{\pgfqpoint{3.000754in}{0.677243in}}%
\pgfpathlineto{\pgfqpoint{2.939368in}{0.677243in}}%
\pgfpathlineto{\pgfqpoint{2.939368in}{0.499444in}}%
\pgfpathclose%
\pgfusepath{fill}%
\end{pgfscope}%
\begin{pgfscope}%
\pgfpathrectangle{\pgfqpoint{0.445556in}{0.499444in}}{\pgfqpoint{3.875000in}{1.155000in}}%
\pgfusepath{clip}%
\pgfsetbuttcap%
\pgfsetmiterjoin%
\definecolor{currentfill}{rgb}{0.000000,0.000000,0.000000}%
\pgfsetfillcolor{currentfill}%
\pgfsetlinewidth{0.000000pt}%
\definecolor{currentstroke}{rgb}{0.000000,0.000000,0.000000}%
\pgfsetstrokecolor{currentstroke}%
\pgfsetstrokeopacity{0.000000}%
\pgfsetdash{}{0pt}%
\pgfpathmoveto{\pgfqpoint{3.092833in}{0.499444in}}%
\pgfpathlineto{\pgfqpoint{3.154219in}{0.499444in}}%
\pgfpathlineto{\pgfqpoint{3.154219in}{0.675676in}}%
\pgfpathlineto{\pgfqpoint{3.092833in}{0.675676in}}%
\pgfpathlineto{\pgfqpoint{3.092833in}{0.499444in}}%
\pgfpathclose%
\pgfusepath{fill}%
\end{pgfscope}%
\begin{pgfscope}%
\pgfpathrectangle{\pgfqpoint{0.445556in}{0.499444in}}{\pgfqpoint{3.875000in}{1.155000in}}%
\pgfusepath{clip}%
\pgfsetbuttcap%
\pgfsetmiterjoin%
\definecolor{currentfill}{rgb}{0.000000,0.000000,0.000000}%
\pgfsetfillcolor{currentfill}%
\pgfsetlinewidth{0.000000pt}%
\definecolor{currentstroke}{rgb}{0.000000,0.000000,0.000000}%
\pgfsetstrokecolor{currentstroke}%
\pgfsetstrokeopacity{0.000000}%
\pgfsetdash{}{0pt}%
\pgfpathmoveto{\pgfqpoint{3.246298in}{0.499444in}}%
\pgfpathlineto{\pgfqpoint{3.307684in}{0.499444in}}%
\pgfpathlineto{\pgfqpoint{3.307684in}{0.673796in}}%
\pgfpathlineto{\pgfqpoint{3.246298in}{0.673796in}}%
\pgfpathlineto{\pgfqpoint{3.246298in}{0.499444in}}%
\pgfpathclose%
\pgfusepath{fill}%
\end{pgfscope}%
\begin{pgfscope}%
\pgfpathrectangle{\pgfqpoint{0.445556in}{0.499444in}}{\pgfqpoint{3.875000in}{1.155000in}}%
\pgfusepath{clip}%
\pgfsetbuttcap%
\pgfsetmiterjoin%
\definecolor{currentfill}{rgb}{0.000000,0.000000,0.000000}%
\pgfsetfillcolor{currentfill}%
\pgfsetlinewidth{0.000000pt}%
\definecolor{currentstroke}{rgb}{0.000000,0.000000,0.000000}%
\pgfsetstrokecolor{currentstroke}%
\pgfsetstrokeopacity{0.000000}%
\pgfsetdash{}{0pt}%
\pgfpathmoveto{\pgfqpoint{3.399764in}{0.499444in}}%
\pgfpathlineto{\pgfqpoint{3.461150in}{0.499444in}}%
\pgfpathlineto{\pgfqpoint{3.461150in}{0.669488in}}%
\pgfpathlineto{\pgfqpoint{3.399764in}{0.669488in}}%
\pgfpathlineto{\pgfqpoint{3.399764in}{0.499444in}}%
\pgfpathclose%
\pgfusepath{fill}%
\end{pgfscope}%
\begin{pgfscope}%
\pgfpathrectangle{\pgfqpoint{0.445556in}{0.499444in}}{\pgfqpoint{3.875000in}{1.155000in}}%
\pgfusepath{clip}%
\pgfsetbuttcap%
\pgfsetmiterjoin%
\definecolor{currentfill}{rgb}{0.000000,0.000000,0.000000}%
\pgfsetfillcolor{currentfill}%
\pgfsetlinewidth{0.000000pt}%
\definecolor{currentstroke}{rgb}{0.000000,0.000000,0.000000}%
\pgfsetstrokecolor{currentstroke}%
\pgfsetstrokeopacity{0.000000}%
\pgfsetdash{}{0pt}%
\pgfpathmoveto{\pgfqpoint{3.553229in}{0.499444in}}%
\pgfpathlineto{\pgfqpoint{3.614615in}{0.499444in}}%
\pgfpathlineto{\pgfqpoint{3.614615in}{0.658914in}}%
\pgfpathlineto{\pgfqpoint{3.553229in}{0.658914in}}%
\pgfpathlineto{\pgfqpoint{3.553229in}{0.499444in}}%
\pgfpathclose%
\pgfusepath{fill}%
\end{pgfscope}%
\begin{pgfscope}%
\pgfpathrectangle{\pgfqpoint{0.445556in}{0.499444in}}{\pgfqpoint{3.875000in}{1.155000in}}%
\pgfusepath{clip}%
\pgfsetbuttcap%
\pgfsetmiterjoin%
\definecolor{currentfill}{rgb}{0.000000,0.000000,0.000000}%
\pgfsetfillcolor{currentfill}%
\pgfsetlinewidth{0.000000pt}%
\definecolor{currentstroke}{rgb}{0.000000,0.000000,0.000000}%
\pgfsetstrokecolor{currentstroke}%
\pgfsetstrokeopacity{0.000000}%
\pgfsetdash{}{0pt}%
\pgfpathmoveto{\pgfqpoint{3.706694in}{0.499444in}}%
\pgfpathlineto{\pgfqpoint{3.768080in}{0.499444in}}%
\pgfpathlineto{\pgfqpoint{3.768080in}{0.637845in}}%
\pgfpathlineto{\pgfqpoint{3.706694in}{0.637845in}}%
\pgfpathlineto{\pgfqpoint{3.706694in}{0.499444in}}%
\pgfpathclose%
\pgfusepath{fill}%
\end{pgfscope}%
\begin{pgfscope}%
\pgfpathrectangle{\pgfqpoint{0.445556in}{0.499444in}}{\pgfqpoint{3.875000in}{1.155000in}}%
\pgfusepath{clip}%
\pgfsetbuttcap%
\pgfsetmiterjoin%
\definecolor{currentfill}{rgb}{0.000000,0.000000,0.000000}%
\pgfsetfillcolor{currentfill}%
\pgfsetlinewidth{0.000000pt}%
\definecolor{currentstroke}{rgb}{0.000000,0.000000,0.000000}%
\pgfsetstrokecolor{currentstroke}%
\pgfsetstrokeopacity{0.000000}%
\pgfsetdash{}{0pt}%
\pgfpathmoveto{\pgfqpoint{3.860160in}{0.499444in}}%
\pgfpathlineto{\pgfqpoint{3.921546in}{0.499444in}}%
\pgfpathlineto{\pgfqpoint{3.921546in}{0.598839in}}%
\pgfpathlineto{\pgfqpoint{3.860160in}{0.598839in}}%
\pgfpathlineto{\pgfqpoint{3.860160in}{0.499444in}}%
\pgfpathclose%
\pgfusepath{fill}%
\end{pgfscope}%
\begin{pgfscope}%
\pgfpathrectangle{\pgfqpoint{0.445556in}{0.499444in}}{\pgfqpoint{3.875000in}{1.155000in}}%
\pgfusepath{clip}%
\pgfsetbuttcap%
\pgfsetmiterjoin%
\definecolor{currentfill}{rgb}{0.000000,0.000000,0.000000}%
\pgfsetfillcolor{currentfill}%
\pgfsetlinewidth{0.000000pt}%
\definecolor{currentstroke}{rgb}{0.000000,0.000000,0.000000}%
\pgfsetstrokecolor{currentstroke}%
\pgfsetstrokeopacity{0.000000}%
\pgfsetdash{}{0pt}%
\pgfpathmoveto{\pgfqpoint{4.013625in}{0.499444in}}%
\pgfpathlineto{\pgfqpoint{4.075011in}{0.499444in}}%
\pgfpathlineto{\pgfqpoint{4.075011in}{0.565864in}}%
\pgfpathlineto{\pgfqpoint{4.013625in}{0.565864in}}%
\pgfpathlineto{\pgfqpoint{4.013625in}{0.499444in}}%
\pgfpathclose%
\pgfusepath{fill}%
\end{pgfscope}%
\begin{pgfscope}%
\pgfpathrectangle{\pgfqpoint{0.445556in}{0.499444in}}{\pgfqpoint{3.875000in}{1.155000in}}%
\pgfusepath{clip}%
\pgfsetbuttcap%
\pgfsetmiterjoin%
\definecolor{currentfill}{rgb}{0.000000,0.000000,0.000000}%
\pgfsetfillcolor{currentfill}%
\pgfsetlinewidth{0.000000pt}%
\definecolor{currentstroke}{rgb}{0.000000,0.000000,0.000000}%
\pgfsetstrokecolor{currentstroke}%
\pgfsetstrokeopacity{0.000000}%
\pgfsetdash{}{0pt}%
\pgfpathmoveto{\pgfqpoint{4.167090in}{0.499444in}}%
\pgfpathlineto{\pgfqpoint{4.228476in}{0.499444in}}%
\pgfpathlineto{\pgfqpoint{4.228476in}{0.515031in}}%
\pgfpathlineto{\pgfqpoint{4.167090in}{0.515031in}}%
\pgfpathlineto{\pgfqpoint{4.167090in}{0.499444in}}%
\pgfpathclose%
\pgfusepath{fill}%
\end{pgfscope}%
\begin{pgfscope}%
\pgfsetbuttcap%
\pgfsetroundjoin%
\definecolor{currentfill}{rgb}{0.000000,0.000000,0.000000}%
\pgfsetfillcolor{currentfill}%
\pgfsetlinewidth{0.803000pt}%
\definecolor{currentstroke}{rgb}{0.000000,0.000000,0.000000}%
\pgfsetstrokecolor{currentstroke}%
\pgfsetdash{}{0pt}%
\pgfsys@defobject{currentmarker}{\pgfqpoint{0.000000in}{-0.048611in}}{\pgfqpoint{0.000000in}{0.000000in}}{%
\pgfpathmoveto{\pgfqpoint{0.000000in}{0.000000in}}%
\pgfpathlineto{\pgfqpoint{0.000000in}{-0.048611in}}%
\pgfusepath{stroke,fill}%
}%
\begin{pgfscope}%
\pgfsys@transformshift{0.483922in}{0.499444in}%
\pgfsys@useobject{currentmarker}{}%
\end{pgfscope}%
\end{pgfscope}%
\begin{pgfscope}%
\definecolor{textcolor}{rgb}{0.000000,0.000000,0.000000}%
\pgfsetstrokecolor{textcolor}%
\pgfsetfillcolor{textcolor}%
\pgftext[x=0.483922in,y=0.402222in,,top]{\color{textcolor}\rmfamily\fontsize{10.000000}{12.000000}\selectfont 0.0}%
\end{pgfscope}%
\begin{pgfscope}%
\pgfsetbuttcap%
\pgfsetroundjoin%
\definecolor{currentfill}{rgb}{0.000000,0.000000,0.000000}%
\pgfsetfillcolor{currentfill}%
\pgfsetlinewidth{0.803000pt}%
\definecolor{currentstroke}{rgb}{0.000000,0.000000,0.000000}%
\pgfsetstrokecolor{currentstroke}%
\pgfsetdash{}{0pt}%
\pgfsys@defobject{currentmarker}{\pgfqpoint{0.000000in}{-0.048611in}}{\pgfqpoint{0.000000in}{0.000000in}}{%
\pgfpathmoveto{\pgfqpoint{0.000000in}{0.000000in}}%
\pgfpathlineto{\pgfqpoint{0.000000in}{-0.048611in}}%
\pgfusepath{stroke,fill}%
}%
\begin{pgfscope}%
\pgfsys@transformshift{0.867585in}{0.499444in}%
\pgfsys@useobject{currentmarker}{}%
\end{pgfscope}%
\end{pgfscope}%
\begin{pgfscope}%
\definecolor{textcolor}{rgb}{0.000000,0.000000,0.000000}%
\pgfsetstrokecolor{textcolor}%
\pgfsetfillcolor{textcolor}%
\pgftext[x=0.867585in,y=0.402222in,,top]{\color{textcolor}\rmfamily\fontsize{10.000000}{12.000000}\selectfont 0.1}%
\end{pgfscope}%
\begin{pgfscope}%
\pgfsetbuttcap%
\pgfsetroundjoin%
\definecolor{currentfill}{rgb}{0.000000,0.000000,0.000000}%
\pgfsetfillcolor{currentfill}%
\pgfsetlinewidth{0.803000pt}%
\definecolor{currentstroke}{rgb}{0.000000,0.000000,0.000000}%
\pgfsetstrokecolor{currentstroke}%
\pgfsetdash{}{0pt}%
\pgfsys@defobject{currentmarker}{\pgfqpoint{0.000000in}{-0.048611in}}{\pgfqpoint{0.000000in}{0.000000in}}{%
\pgfpathmoveto{\pgfqpoint{0.000000in}{0.000000in}}%
\pgfpathlineto{\pgfqpoint{0.000000in}{-0.048611in}}%
\pgfusepath{stroke,fill}%
}%
\begin{pgfscope}%
\pgfsys@transformshift{1.251249in}{0.499444in}%
\pgfsys@useobject{currentmarker}{}%
\end{pgfscope}%
\end{pgfscope}%
\begin{pgfscope}%
\definecolor{textcolor}{rgb}{0.000000,0.000000,0.000000}%
\pgfsetstrokecolor{textcolor}%
\pgfsetfillcolor{textcolor}%
\pgftext[x=1.251249in,y=0.402222in,,top]{\color{textcolor}\rmfamily\fontsize{10.000000}{12.000000}\selectfont 0.2}%
\end{pgfscope}%
\begin{pgfscope}%
\pgfsetbuttcap%
\pgfsetroundjoin%
\definecolor{currentfill}{rgb}{0.000000,0.000000,0.000000}%
\pgfsetfillcolor{currentfill}%
\pgfsetlinewidth{0.803000pt}%
\definecolor{currentstroke}{rgb}{0.000000,0.000000,0.000000}%
\pgfsetstrokecolor{currentstroke}%
\pgfsetdash{}{0pt}%
\pgfsys@defobject{currentmarker}{\pgfqpoint{0.000000in}{-0.048611in}}{\pgfqpoint{0.000000in}{0.000000in}}{%
\pgfpathmoveto{\pgfqpoint{0.000000in}{0.000000in}}%
\pgfpathlineto{\pgfqpoint{0.000000in}{-0.048611in}}%
\pgfusepath{stroke,fill}%
}%
\begin{pgfscope}%
\pgfsys@transformshift{1.634912in}{0.499444in}%
\pgfsys@useobject{currentmarker}{}%
\end{pgfscope}%
\end{pgfscope}%
\begin{pgfscope}%
\definecolor{textcolor}{rgb}{0.000000,0.000000,0.000000}%
\pgfsetstrokecolor{textcolor}%
\pgfsetfillcolor{textcolor}%
\pgftext[x=1.634912in,y=0.402222in,,top]{\color{textcolor}\rmfamily\fontsize{10.000000}{12.000000}\selectfont 0.3}%
\end{pgfscope}%
\begin{pgfscope}%
\pgfsetbuttcap%
\pgfsetroundjoin%
\definecolor{currentfill}{rgb}{0.000000,0.000000,0.000000}%
\pgfsetfillcolor{currentfill}%
\pgfsetlinewidth{0.803000pt}%
\definecolor{currentstroke}{rgb}{0.000000,0.000000,0.000000}%
\pgfsetstrokecolor{currentstroke}%
\pgfsetdash{}{0pt}%
\pgfsys@defobject{currentmarker}{\pgfqpoint{0.000000in}{-0.048611in}}{\pgfqpoint{0.000000in}{0.000000in}}{%
\pgfpathmoveto{\pgfqpoint{0.000000in}{0.000000in}}%
\pgfpathlineto{\pgfqpoint{0.000000in}{-0.048611in}}%
\pgfusepath{stroke,fill}%
}%
\begin{pgfscope}%
\pgfsys@transformshift{2.018575in}{0.499444in}%
\pgfsys@useobject{currentmarker}{}%
\end{pgfscope}%
\end{pgfscope}%
\begin{pgfscope}%
\definecolor{textcolor}{rgb}{0.000000,0.000000,0.000000}%
\pgfsetstrokecolor{textcolor}%
\pgfsetfillcolor{textcolor}%
\pgftext[x=2.018575in,y=0.402222in,,top]{\color{textcolor}\rmfamily\fontsize{10.000000}{12.000000}\selectfont 0.4}%
\end{pgfscope}%
\begin{pgfscope}%
\pgfsetbuttcap%
\pgfsetroundjoin%
\definecolor{currentfill}{rgb}{0.000000,0.000000,0.000000}%
\pgfsetfillcolor{currentfill}%
\pgfsetlinewidth{0.803000pt}%
\definecolor{currentstroke}{rgb}{0.000000,0.000000,0.000000}%
\pgfsetstrokecolor{currentstroke}%
\pgfsetdash{}{0pt}%
\pgfsys@defobject{currentmarker}{\pgfqpoint{0.000000in}{-0.048611in}}{\pgfqpoint{0.000000in}{0.000000in}}{%
\pgfpathmoveto{\pgfqpoint{0.000000in}{0.000000in}}%
\pgfpathlineto{\pgfqpoint{0.000000in}{-0.048611in}}%
\pgfusepath{stroke,fill}%
}%
\begin{pgfscope}%
\pgfsys@transformshift{2.402239in}{0.499444in}%
\pgfsys@useobject{currentmarker}{}%
\end{pgfscope}%
\end{pgfscope}%
\begin{pgfscope}%
\definecolor{textcolor}{rgb}{0.000000,0.000000,0.000000}%
\pgfsetstrokecolor{textcolor}%
\pgfsetfillcolor{textcolor}%
\pgftext[x=2.402239in,y=0.402222in,,top]{\color{textcolor}\rmfamily\fontsize{10.000000}{12.000000}\selectfont 0.5}%
\end{pgfscope}%
\begin{pgfscope}%
\pgfsetbuttcap%
\pgfsetroundjoin%
\definecolor{currentfill}{rgb}{0.000000,0.000000,0.000000}%
\pgfsetfillcolor{currentfill}%
\pgfsetlinewidth{0.803000pt}%
\definecolor{currentstroke}{rgb}{0.000000,0.000000,0.000000}%
\pgfsetstrokecolor{currentstroke}%
\pgfsetdash{}{0pt}%
\pgfsys@defobject{currentmarker}{\pgfqpoint{0.000000in}{-0.048611in}}{\pgfqpoint{0.000000in}{0.000000in}}{%
\pgfpathmoveto{\pgfqpoint{0.000000in}{0.000000in}}%
\pgfpathlineto{\pgfqpoint{0.000000in}{-0.048611in}}%
\pgfusepath{stroke,fill}%
}%
\begin{pgfscope}%
\pgfsys@transformshift{2.785902in}{0.499444in}%
\pgfsys@useobject{currentmarker}{}%
\end{pgfscope}%
\end{pgfscope}%
\begin{pgfscope}%
\definecolor{textcolor}{rgb}{0.000000,0.000000,0.000000}%
\pgfsetstrokecolor{textcolor}%
\pgfsetfillcolor{textcolor}%
\pgftext[x=2.785902in,y=0.402222in,,top]{\color{textcolor}\rmfamily\fontsize{10.000000}{12.000000}\selectfont 0.6}%
\end{pgfscope}%
\begin{pgfscope}%
\pgfsetbuttcap%
\pgfsetroundjoin%
\definecolor{currentfill}{rgb}{0.000000,0.000000,0.000000}%
\pgfsetfillcolor{currentfill}%
\pgfsetlinewidth{0.803000pt}%
\definecolor{currentstroke}{rgb}{0.000000,0.000000,0.000000}%
\pgfsetstrokecolor{currentstroke}%
\pgfsetdash{}{0pt}%
\pgfsys@defobject{currentmarker}{\pgfqpoint{0.000000in}{-0.048611in}}{\pgfqpoint{0.000000in}{0.000000in}}{%
\pgfpathmoveto{\pgfqpoint{0.000000in}{0.000000in}}%
\pgfpathlineto{\pgfqpoint{0.000000in}{-0.048611in}}%
\pgfusepath{stroke,fill}%
}%
\begin{pgfscope}%
\pgfsys@transformshift{3.169566in}{0.499444in}%
\pgfsys@useobject{currentmarker}{}%
\end{pgfscope}%
\end{pgfscope}%
\begin{pgfscope}%
\definecolor{textcolor}{rgb}{0.000000,0.000000,0.000000}%
\pgfsetstrokecolor{textcolor}%
\pgfsetfillcolor{textcolor}%
\pgftext[x=3.169566in,y=0.402222in,,top]{\color{textcolor}\rmfamily\fontsize{10.000000}{12.000000}\selectfont 0.7}%
\end{pgfscope}%
\begin{pgfscope}%
\pgfsetbuttcap%
\pgfsetroundjoin%
\definecolor{currentfill}{rgb}{0.000000,0.000000,0.000000}%
\pgfsetfillcolor{currentfill}%
\pgfsetlinewidth{0.803000pt}%
\definecolor{currentstroke}{rgb}{0.000000,0.000000,0.000000}%
\pgfsetstrokecolor{currentstroke}%
\pgfsetdash{}{0pt}%
\pgfsys@defobject{currentmarker}{\pgfqpoint{0.000000in}{-0.048611in}}{\pgfqpoint{0.000000in}{0.000000in}}{%
\pgfpathmoveto{\pgfqpoint{0.000000in}{0.000000in}}%
\pgfpathlineto{\pgfqpoint{0.000000in}{-0.048611in}}%
\pgfusepath{stroke,fill}%
}%
\begin{pgfscope}%
\pgfsys@transformshift{3.553229in}{0.499444in}%
\pgfsys@useobject{currentmarker}{}%
\end{pgfscope}%
\end{pgfscope}%
\begin{pgfscope}%
\definecolor{textcolor}{rgb}{0.000000,0.000000,0.000000}%
\pgfsetstrokecolor{textcolor}%
\pgfsetfillcolor{textcolor}%
\pgftext[x=3.553229in,y=0.402222in,,top]{\color{textcolor}\rmfamily\fontsize{10.000000}{12.000000}\selectfont 0.8}%
\end{pgfscope}%
\begin{pgfscope}%
\pgfsetbuttcap%
\pgfsetroundjoin%
\definecolor{currentfill}{rgb}{0.000000,0.000000,0.000000}%
\pgfsetfillcolor{currentfill}%
\pgfsetlinewidth{0.803000pt}%
\definecolor{currentstroke}{rgb}{0.000000,0.000000,0.000000}%
\pgfsetstrokecolor{currentstroke}%
\pgfsetdash{}{0pt}%
\pgfsys@defobject{currentmarker}{\pgfqpoint{0.000000in}{-0.048611in}}{\pgfqpoint{0.000000in}{0.000000in}}{%
\pgfpathmoveto{\pgfqpoint{0.000000in}{0.000000in}}%
\pgfpathlineto{\pgfqpoint{0.000000in}{-0.048611in}}%
\pgfusepath{stroke,fill}%
}%
\begin{pgfscope}%
\pgfsys@transformshift{3.936892in}{0.499444in}%
\pgfsys@useobject{currentmarker}{}%
\end{pgfscope}%
\end{pgfscope}%
\begin{pgfscope}%
\definecolor{textcolor}{rgb}{0.000000,0.000000,0.000000}%
\pgfsetstrokecolor{textcolor}%
\pgfsetfillcolor{textcolor}%
\pgftext[x=3.936892in,y=0.402222in,,top]{\color{textcolor}\rmfamily\fontsize{10.000000}{12.000000}\selectfont 0.9}%
\end{pgfscope}%
\begin{pgfscope}%
\pgfsetbuttcap%
\pgfsetroundjoin%
\definecolor{currentfill}{rgb}{0.000000,0.000000,0.000000}%
\pgfsetfillcolor{currentfill}%
\pgfsetlinewidth{0.803000pt}%
\definecolor{currentstroke}{rgb}{0.000000,0.000000,0.000000}%
\pgfsetstrokecolor{currentstroke}%
\pgfsetdash{}{0pt}%
\pgfsys@defobject{currentmarker}{\pgfqpoint{0.000000in}{-0.048611in}}{\pgfqpoint{0.000000in}{0.000000in}}{%
\pgfpathmoveto{\pgfqpoint{0.000000in}{0.000000in}}%
\pgfpathlineto{\pgfqpoint{0.000000in}{-0.048611in}}%
\pgfusepath{stroke,fill}%
}%
\begin{pgfscope}%
\pgfsys@transformshift{4.320556in}{0.499444in}%
\pgfsys@useobject{currentmarker}{}%
\end{pgfscope}%
\end{pgfscope}%
\begin{pgfscope}%
\definecolor{textcolor}{rgb}{0.000000,0.000000,0.000000}%
\pgfsetstrokecolor{textcolor}%
\pgfsetfillcolor{textcolor}%
\pgftext[x=4.320556in,y=0.402222in,,top]{\color{textcolor}\rmfamily\fontsize{10.000000}{12.000000}\selectfont 1.0}%
\end{pgfscope}%
\begin{pgfscope}%
\definecolor{textcolor}{rgb}{0.000000,0.000000,0.000000}%
\pgfsetstrokecolor{textcolor}%
\pgfsetfillcolor{textcolor}%
\pgftext[x=2.383056in,y=0.223333in,,top]{\color{textcolor}\rmfamily\fontsize{10.000000}{12.000000}\selectfont \(\displaystyle p\)}%
\end{pgfscope}%
\begin{pgfscope}%
\pgfsetbuttcap%
\pgfsetroundjoin%
\definecolor{currentfill}{rgb}{0.000000,0.000000,0.000000}%
\pgfsetfillcolor{currentfill}%
\pgfsetlinewidth{0.803000pt}%
\definecolor{currentstroke}{rgb}{0.000000,0.000000,0.000000}%
\pgfsetstrokecolor{currentstroke}%
\pgfsetdash{}{0pt}%
\pgfsys@defobject{currentmarker}{\pgfqpoint{-0.048611in}{0.000000in}}{\pgfqpoint{-0.000000in}{0.000000in}}{%
\pgfpathmoveto{\pgfqpoint{-0.000000in}{0.000000in}}%
\pgfpathlineto{\pgfqpoint{-0.048611in}{0.000000in}}%
\pgfusepath{stroke,fill}%
}%
\begin{pgfscope}%
\pgfsys@transformshift{0.445556in}{0.499444in}%
\pgfsys@useobject{currentmarker}{}%
\end{pgfscope}%
\end{pgfscope}%
\begin{pgfscope}%
\definecolor{textcolor}{rgb}{0.000000,0.000000,0.000000}%
\pgfsetstrokecolor{textcolor}%
\pgfsetfillcolor{textcolor}%
\pgftext[x=0.278889in, y=0.451250in, left, base]{\color{textcolor}\rmfamily\fontsize{10.000000}{12.000000}\selectfont \(\displaystyle {0}\)}%
\end{pgfscope}%
\begin{pgfscope}%
\pgfsetbuttcap%
\pgfsetroundjoin%
\definecolor{currentfill}{rgb}{0.000000,0.000000,0.000000}%
\pgfsetfillcolor{currentfill}%
\pgfsetlinewidth{0.803000pt}%
\definecolor{currentstroke}{rgb}{0.000000,0.000000,0.000000}%
\pgfsetstrokecolor{currentstroke}%
\pgfsetdash{}{0pt}%
\pgfsys@defobject{currentmarker}{\pgfqpoint{-0.048611in}{0.000000in}}{\pgfqpoint{-0.000000in}{0.000000in}}{%
\pgfpathmoveto{\pgfqpoint{-0.000000in}{0.000000in}}%
\pgfpathlineto{\pgfqpoint{-0.048611in}{0.000000in}}%
\pgfusepath{stroke,fill}%
}%
\begin{pgfscope}%
\pgfsys@transformshift{0.445556in}{0.834786in}%
\pgfsys@useobject{currentmarker}{}%
\end{pgfscope}%
\end{pgfscope}%
\begin{pgfscope}%
\definecolor{textcolor}{rgb}{0.000000,0.000000,0.000000}%
\pgfsetstrokecolor{textcolor}%
\pgfsetfillcolor{textcolor}%
\pgftext[x=0.278889in, y=0.786592in, left, base]{\color{textcolor}\rmfamily\fontsize{10.000000}{12.000000}\selectfont \(\displaystyle {2}\)}%
\end{pgfscope}%
\begin{pgfscope}%
\pgfsetbuttcap%
\pgfsetroundjoin%
\definecolor{currentfill}{rgb}{0.000000,0.000000,0.000000}%
\pgfsetfillcolor{currentfill}%
\pgfsetlinewidth{0.803000pt}%
\definecolor{currentstroke}{rgb}{0.000000,0.000000,0.000000}%
\pgfsetstrokecolor{currentstroke}%
\pgfsetdash{}{0pt}%
\pgfsys@defobject{currentmarker}{\pgfqpoint{-0.048611in}{0.000000in}}{\pgfqpoint{-0.000000in}{0.000000in}}{%
\pgfpathmoveto{\pgfqpoint{-0.000000in}{0.000000in}}%
\pgfpathlineto{\pgfqpoint{-0.048611in}{0.000000in}}%
\pgfusepath{stroke,fill}%
}%
\begin{pgfscope}%
\pgfsys@transformshift{0.445556in}{1.170128in}%
\pgfsys@useobject{currentmarker}{}%
\end{pgfscope}%
\end{pgfscope}%
\begin{pgfscope}%
\definecolor{textcolor}{rgb}{0.000000,0.000000,0.000000}%
\pgfsetstrokecolor{textcolor}%
\pgfsetfillcolor{textcolor}%
\pgftext[x=0.278889in, y=1.121933in, left, base]{\color{textcolor}\rmfamily\fontsize{10.000000}{12.000000}\selectfont \(\displaystyle {4}\)}%
\end{pgfscope}%
\begin{pgfscope}%
\pgfsetbuttcap%
\pgfsetroundjoin%
\definecolor{currentfill}{rgb}{0.000000,0.000000,0.000000}%
\pgfsetfillcolor{currentfill}%
\pgfsetlinewidth{0.803000pt}%
\definecolor{currentstroke}{rgb}{0.000000,0.000000,0.000000}%
\pgfsetstrokecolor{currentstroke}%
\pgfsetdash{}{0pt}%
\pgfsys@defobject{currentmarker}{\pgfqpoint{-0.048611in}{0.000000in}}{\pgfqpoint{-0.000000in}{0.000000in}}{%
\pgfpathmoveto{\pgfqpoint{-0.000000in}{0.000000in}}%
\pgfpathlineto{\pgfqpoint{-0.048611in}{0.000000in}}%
\pgfusepath{stroke,fill}%
}%
\begin{pgfscope}%
\pgfsys@transformshift{0.445556in}{1.505470in}%
\pgfsys@useobject{currentmarker}{}%
\end{pgfscope}%
\end{pgfscope}%
\begin{pgfscope}%
\definecolor{textcolor}{rgb}{0.000000,0.000000,0.000000}%
\pgfsetstrokecolor{textcolor}%
\pgfsetfillcolor{textcolor}%
\pgftext[x=0.278889in, y=1.457275in, left, base]{\color{textcolor}\rmfamily\fontsize{10.000000}{12.000000}\selectfont \(\displaystyle {6}\)}%
\end{pgfscope}%
\begin{pgfscope}%
\definecolor{textcolor}{rgb}{0.000000,0.000000,0.000000}%
\pgfsetstrokecolor{textcolor}%
\pgfsetfillcolor{textcolor}%
\pgftext[x=0.223333in,y=1.076944in,,bottom,rotate=90.000000]{\color{textcolor}\rmfamily\fontsize{10.000000}{12.000000}\selectfont Percent of Data Set}%
\end{pgfscope}%
\begin{pgfscope}%
\pgfsetrectcap%
\pgfsetmiterjoin%
\pgfsetlinewidth{0.803000pt}%
\definecolor{currentstroke}{rgb}{0.000000,0.000000,0.000000}%
\pgfsetstrokecolor{currentstroke}%
\pgfsetdash{}{0pt}%
\pgfpathmoveto{\pgfqpoint{0.445556in}{0.499444in}}%
\pgfpathlineto{\pgfqpoint{0.445556in}{1.654444in}}%
\pgfusepath{stroke}%
\end{pgfscope}%
\begin{pgfscope}%
\pgfsetrectcap%
\pgfsetmiterjoin%
\pgfsetlinewidth{0.803000pt}%
\definecolor{currentstroke}{rgb}{0.000000,0.000000,0.000000}%
\pgfsetstrokecolor{currentstroke}%
\pgfsetdash{}{0pt}%
\pgfpathmoveto{\pgfqpoint{4.320556in}{0.499444in}}%
\pgfpathlineto{\pgfqpoint{4.320556in}{1.654444in}}%
\pgfusepath{stroke}%
\end{pgfscope}%
\begin{pgfscope}%
\pgfsetrectcap%
\pgfsetmiterjoin%
\pgfsetlinewidth{0.803000pt}%
\definecolor{currentstroke}{rgb}{0.000000,0.000000,0.000000}%
\pgfsetstrokecolor{currentstroke}%
\pgfsetdash{}{0pt}%
\pgfpathmoveto{\pgfqpoint{0.445556in}{0.499444in}}%
\pgfpathlineto{\pgfqpoint{4.320556in}{0.499444in}}%
\pgfusepath{stroke}%
\end{pgfscope}%
\begin{pgfscope}%
\pgfsetrectcap%
\pgfsetmiterjoin%
\pgfsetlinewidth{0.803000pt}%
\definecolor{currentstroke}{rgb}{0.000000,0.000000,0.000000}%
\pgfsetstrokecolor{currentstroke}%
\pgfsetdash{}{0pt}%
\pgfpathmoveto{\pgfqpoint{0.445556in}{1.654444in}}%
\pgfpathlineto{\pgfqpoint{4.320556in}{1.654444in}}%
\pgfusepath{stroke}%
\end{pgfscope}%
\begin{pgfscope}%
\pgfsetbuttcap%
\pgfsetmiterjoin%
\definecolor{currentfill}{rgb}{1.000000,1.000000,1.000000}%
\pgfsetfillcolor{currentfill}%
\pgfsetfillopacity{0.800000}%
\pgfsetlinewidth{1.003750pt}%
\definecolor{currentstroke}{rgb}{0.800000,0.800000,0.800000}%
\pgfsetstrokecolor{currentstroke}%
\pgfsetstrokeopacity{0.800000}%
\pgfsetdash{}{0pt}%
\pgfpathmoveto{\pgfqpoint{3.543611in}{1.154445in}}%
\pgfpathlineto{\pgfqpoint{4.223333in}{1.154445in}}%
\pgfpathquadraticcurveto{\pgfqpoint{4.251111in}{1.154445in}}{\pgfqpoint{4.251111in}{1.182222in}}%
\pgfpathlineto{\pgfqpoint{4.251111in}{1.557222in}}%
\pgfpathquadraticcurveto{\pgfqpoint{4.251111in}{1.585000in}}{\pgfqpoint{4.223333in}{1.585000in}}%
\pgfpathlineto{\pgfqpoint{3.543611in}{1.585000in}}%
\pgfpathquadraticcurveto{\pgfqpoint{3.515833in}{1.585000in}}{\pgfqpoint{3.515833in}{1.557222in}}%
\pgfpathlineto{\pgfqpoint{3.515833in}{1.182222in}}%
\pgfpathquadraticcurveto{\pgfqpoint{3.515833in}{1.154445in}}{\pgfqpoint{3.543611in}{1.154445in}}%
\pgfpathlineto{\pgfqpoint{3.543611in}{1.154445in}}%
\pgfpathclose%
\pgfusepath{stroke,fill}%
\end{pgfscope}%
\begin{pgfscope}%
\pgfsetbuttcap%
\pgfsetmiterjoin%
\pgfsetlinewidth{1.003750pt}%
\definecolor{currentstroke}{rgb}{0.000000,0.000000,0.000000}%
\pgfsetstrokecolor{currentstroke}%
\pgfsetdash{}{0pt}%
\pgfpathmoveto{\pgfqpoint{3.571389in}{1.432222in}}%
\pgfpathlineto{\pgfqpoint{3.849167in}{1.432222in}}%
\pgfpathlineto{\pgfqpoint{3.849167in}{1.529444in}}%
\pgfpathlineto{\pgfqpoint{3.571389in}{1.529444in}}%
\pgfpathlineto{\pgfqpoint{3.571389in}{1.432222in}}%
\pgfpathclose%
\pgfusepath{stroke}%
\end{pgfscope}%
\begin{pgfscope}%
\definecolor{textcolor}{rgb}{0.000000,0.000000,0.000000}%
\pgfsetstrokecolor{textcolor}%
\pgfsetfillcolor{textcolor}%
\pgftext[x=3.960278in,y=1.432222in,left,base]{\color{textcolor}\rmfamily\fontsize{10.000000}{12.000000}\selectfont Neg}%
\end{pgfscope}%
\begin{pgfscope}%
\pgfsetbuttcap%
\pgfsetmiterjoin%
\definecolor{currentfill}{rgb}{0.000000,0.000000,0.000000}%
\pgfsetfillcolor{currentfill}%
\pgfsetlinewidth{0.000000pt}%
\definecolor{currentstroke}{rgb}{0.000000,0.000000,0.000000}%
\pgfsetstrokecolor{currentstroke}%
\pgfsetstrokeopacity{0.000000}%
\pgfsetdash{}{0pt}%
\pgfpathmoveto{\pgfqpoint{3.571389in}{1.236944in}}%
\pgfpathlineto{\pgfqpoint{3.849167in}{1.236944in}}%
\pgfpathlineto{\pgfqpoint{3.849167in}{1.334167in}}%
\pgfpathlineto{\pgfqpoint{3.571389in}{1.334167in}}%
\pgfpathlineto{\pgfqpoint{3.571389in}{1.236944in}}%
\pgfpathclose%
\pgfusepath{fill}%
\end{pgfscope}%
\begin{pgfscope}%
\definecolor{textcolor}{rgb}{0.000000,0.000000,0.000000}%
\pgfsetstrokecolor{textcolor}%
\pgfsetfillcolor{textcolor}%
\pgftext[x=3.960278in,y=1.236944in,left,base]{\color{textcolor}\rmfamily\fontsize{10.000000}{12.000000}\selectfont Pos}%
\end{pgfscope}%
\end{pgfpicture}%
\makeatother%
\endgroup%
	
&
	\vskip 0pt
	\hfil ROC Curve
	
	%% Creator: Matplotlib, PGF backend
%%
%% To include the figure in your LaTeX document, write
%%   \input{<filename>.pgf}
%%
%% Make sure the required packages are loaded in your preamble
%%   \usepackage{pgf}
%%
%% Also ensure that all the required font packages are loaded; for instance,
%% the lmodern package is sometimes necessary when using math font.
%%   \usepackage{lmodern}
%%
%% Figures using additional raster images can only be included by \input if
%% they are in the same directory as the main LaTeX file. For loading figures
%% from other directories you can use the `import` package
%%   \usepackage{import}
%%
%% and then include the figures with
%%   \import{<path to file>}{<filename>.pgf}
%%
%% Matplotlib used the following preamble
%%   
%%   \usepackage{fontspec}
%%   \makeatletter\@ifpackageloaded{underscore}{}{\usepackage[strings]{underscore}}\makeatother
%%
\begingroup%
\makeatletter%
\begin{pgfpicture}%
\pgfpathrectangle{\pgfpointorigin}{\pgfqpoint{2.221861in}{1.754444in}}%
\pgfusepath{use as bounding box, clip}%
\begin{pgfscope}%
\pgfsetbuttcap%
\pgfsetmiterjoin%
\definecolor{currentfill}{rgb}{1.000000,1.000000,1.000000}%
\pgfsetfillcolor{currentfill}%
\pgfsetlinewidth{0.000000pt}%
\definecolor{currentstroke}{rgb}{1.000000,1.000000,1.000000}%
\pgfsetstrokecolor{currentstroke}%
\pgfsetdash{}{0pt}%
\pgfpathmoveto{\pgfqpoint{0.000000in}{0.000000in}}%
\pgfpathlineto{\pgfqpoint{2.221861in}{0.000000in}}%
\pgfpathlineto{\pgfqpoint{2.221861in}{1.754444in}}%
\pgfpathlineto{\pgfqpoint{0.000000in}{1.754444in}}%
\pgfpathlineto{\pgfqpoint{0.000000in}{0.000000in}}%
\pgfpathclose%
\pgfusepath{fill}%
\end{pgfscope}%
\begin{pgfscope}%
\pgfsetbuttcap%
\pgfsetmiterjoin%
\definecolor{currentfill}{rgb}{1.000000,1.000000,1.000000}%
\pgfsetfillcolor{currentfill}%
\pgfsetlinewidth{0.000000pt}%
\definecolor{currentstroke}{rgb}{0.000000,0.000000,0.000000}%
\pgfsetstrokecolor{currentstroke}%
\pgfsetstrokeopacity{0.000000}%
\pgfsetdash{}{0pt}%
\pgfpathmoveto{\pgfqpoint{0.553581in}{0.499444in}}%
\pgfpathlineto{\pgfqpoint{2.103581in}{0.499444in}}%
\pgfpathlineto{\pgfqpoint{2.103581in}{1.654444in}}%
\pgfpathlineto{\pgfqpoint{0.553581in}{1.654444in}}%
\pgfpathlineto{\pgfqpoint{0.553581in}{0.499444in}}%
\pgfpathclose%
\pgfusepath{fill}%
\end{pgfscope}%
\begin{pgfscope}%
\pgfsetbuttcap%
\pgfsetroundjoin%
\definecolor{currentfill}{rgb}{0.000000,0.000000,0.000000}%
\pgfsetfillcolor{currentfill}%
\pgfsetlinewidth{0.803000pt}%
\definecolor{currentstroke}{rgb}{0.000000,0.000000,0.000000}%
\pgfsetstrokecolor{currentstroke}%
\pgfsetdash{}{0pt}%
\pgfsys@defobject{currentmarker}{\pgfqpoint{0.000000in}{-0.048611in}}{\pgfqpoint{0.000000in}{0.000000in}}{%
\pgfpathmoveto{\pgfqpoint{0.000000in}{0.000000in}}%
\pgfpathlineto{\pgfqpoint{0.000000in}{-0.048611in}}%
\pgfusepath{stroke,fill}%
}%
\begin{pgfscope}%
\pgfsys@transformshift{0.624035in}{0.499444in}%
\pgfsys@useobject{currentmarker}{}%
\end{pgfscope}%
\end{pgfscope}%
\begin{pgfscope}%
\definecolor{textcolor}{rgb}{0.000000,0.000000,0.000000}%
\pgfsetstrokecolor{textcolor}%
\pgfsetfillcolor{textcolor}%
\pgftext[x=0.624035in,y=0.402222in,,top]{\color{textcolor}\rmfamily\fontsize{10.000000}{12.000000}\selectfont \(\displaystyle {0.0}\)}%
\end{pgfscope}%
\begin{pgfscope}%
\pgfsetbuttcap%
\pgfsetroundjoin%
\definecolor{currentfill}{rgb}{0.000000,0.000000,0.000000}%
\pgfsetfillcolor{currentfill}%
\pgfsetlinewidth{0.803000pt}%
\definecolor{currentstroke}{rgb}{0.000000,0.000000,0.000000}%
\pgfsetstrokecolor{currentstroke}%
\pgfsetdash{}{0pt}%
\pgfsys@defobject{currentmarker}{\pgfqpoint{0.000000in}{-0.048611in}}{\pgfqpoint{0.000000in}{0.000000in}}{%
\pgfpathmoveto{\pgfqpoint{0.000000in}{0.000000in}}%
\pgfpathlineto{\pgfqpoint{0.000000in}{-0.048611in}}%
\pgfusepath{stroke,fill}%
}%
\begin{pgfscope}%
\pgfsys@transformshift{1.328581in}{0.499444in}%
\pgfsys@useobject{currentmarker}{}%
\end{pgfscope}%
\end{pgfscope}%
\begin{pgfscope}%
\definecolor{textcolor}{rgb}{0.000000,0.000000,0.000000}%
\pgfsetstrokecolor{textcolor}%
\pgfsetfillcolor{textcolor}%
\pgftext[x=1.328581in,y=0.402222in,,top]{\color{textcolor}\rmfamily\fontsize{10.000000}{12.000000}\selectfont \(\displaystyle {0.5}\)}%
\end{pgfscope}%
\begin{pgfscope}%
\pgfsetbuttcap%
\pgfsetroundjoin%
\definecolor{currentfill}{rgb}{0.000000,0.000000,0.000000}%
\pgfsetfillcolor{currentfill}%
\pgfsetlinewidth{0.803000pt}%
\definecolor{currentstroke}{rgb}{0.000000,0.000000,0.000000}%
\pgfsetstrokecolor{currentstroke}%
\pgfsetdash{}{0pt}%
\pgfsys@defobject{currentmarker}{\pgfqpoint{0.000000in}{-0.048611in}}{\pgfqpoint{0.000000in}{0.000000in}}{%
\pgfpathmoveto{\pgfqpoint{0.000000in}{0.000000in}}%
\pgfpathlineto{\pgfqpoint{0.000000in}{-0.048611in}}%
\pgfusepath{stroke,fill}%
}%
\begin{pgfscope}%
\pgfsys@transformshift{2.033126in}{0.499444in}%
\pgfsys@useobject{currentmarker}{}%
\end{pgfscope}%
\end{pgfscope}%
\begin{pgfscope}%
\definecolor{textcolor}{rgb}{0.000000,0.000000,0.000000}%
\pgfsetstrokecolor{textcolor}%
\pgfsetfillcolor{textcolor}%
\pgftext[x=2.033126in,y=0.402222in,,top]{\color{textcolor}\rmfamily\fontsize{10.000000}{12.000000}\selectfont \(\displaystyle {1.0}\)}%
\end{pgfscope}%
\begin{pgfscope}%
\definecolor{textcolor}{rgb}{0.000000,0.000000,0.000000}%
\pgfsetstrokecolor{textcolor}%
\pgfsetfillcolor{textcolor}%
\pgftext[x=1.328581in,y=0.223333in,,top]{\color{textcolor}\rmfamily\fontsize{10.000000}{12.000000}\selectfont False positive rate}%
\end{pgfscope}%
\begin{pgfscope}%
\pgfsetbuttcap%
\pgfsetroundjoin%
\definecolor{currentfill}{rgb}{0.000000,0.000000,0.000000}%
\pgfsetfillcolor{currentfill}%
\pgfsetlinewidth{0.803000pt}%
\definecolor{currentstroke}{rgb}{0.000000,0.000000,0.000000}%
\pgfsetstrokecolor{currentstroke}%
\pgfsetdash{}{0pt}%
\pgfsys@defobject{currentmarker}{\pgfqpoint{-0.048611in}{0.000000in}}{\pgfqpoint{-0.000000in}{0.000000in}}{%
\pgfpathmoveto{\pgfqpoint{-0.000000in}{0.000000in}}%
\pgfpathlineto{\pgfqpoint{-0.048611in}{0.000000in}}%
\pgfusepath{stroke,fill}%
}%
\begin{pgfscope}%
\pgfsys@transformshift{0.553581in}{0.551944in}%
\pgfsys@useobject{currentmarker}{}%
\end{pgfscope}%
\end{pgfscope}%
\begin{pgfscope}%
\definecolor{textcolor}{rgb}{0.000000,0.000000,0.000000}%
\pgfsetstrokecolor{textcolor}%
\pgfsetfillcolor{textcolor}%
\pgftext[x=0.278889in, y=0.503750in, left, base]{\color{textcolor}\rmfamily\fontsize{10.000000}{12.000000}\selectfont \(\displaystyle {0.0}\)}%
\end{pgfscope}%
\begin{pgfscope}%
\pgfsetbuttcap%
\pgfsetroundjoin%
\definecolor{currentfill}{rgb}{0.000000,0.000000,0.000000}%
\pgfsetfillcolor{currentfill}%
\pgfsetlinewidth{0.803000pt}%
\definecolor{currentstroke}{rgb}{0.000000,0.000000,0.000000}%
\pgfsetstrokecolor{currentstroke}%
\pgfsetdash{}{0pt}%
\pgfsys@defobject{currentmarker}{\pgfqpoint{-0.048611in}{0.000000in}}{\pgfqpoint{-0.000000in}{0.000000in}}{%
\pgfpathmoveto{\pgfqpoint{-0.000000in}{0.000000in}}%
\pgfpathlineto{\pgfqpoint{-0.048611in}{0.000000in}}%
\pgfusepath{stroke,fill}%
}%
\begin{pgfscope}%
\pgfsys@transformshift{0.553581in}{1.076944in}%
\pgfsys@useobject{currentmarker}{}%
\end{pgfscope}%
\end{pgfscope}%
\begin{pgfscope}%
\definecolor{textcolor}{rgb}{0.000000,0.000000,0.000000}%
\pgfsetstrokecolor{textcolor}%
\pgfsetfillcolor{textcolor}%
\pgftext[x=0.278889in, y=1.028750in, left, base]{\color{textcolor}\rmfamily\fontsize{10.000000}{12.000000}\selectfont \(\displaystyle {0.5}\)}%
\end{pgfscope}%
\begin{pgfscope}%
\pgfsetbuttcap%
\pgfsetroundjoin%
\definecolor{currentfill}{rgb}{0.000000,0.000000,0.000000}%
\pgfsetfillcolor{currentfill}%
\pgfsetlinewidth{0.803000pt}%
\definecolor{currentstroke}{rgb}{0.000000,0.000000,0.000000}%
\pgfsetstrokecolor{currentstroke}%
\pgfsetdash{}{0pt}%
\pgfsys@defobject{currentmarker}{\pgfqpoint{-0.048611in}{0.000000in}}{\pgfqpoint{-0.000000in}{0.000000in}}{%
\pgfpathmoveto{\pgfqpoint{-0.000000in}{0.000000in}}%
\pgfpathlineto{\pgfqpoint{-0.048611in}{0.000000in}}%
\pgfusepath{stroke,fill}%
}%
\begin{pgfscope}%
\pgfsys@transformshift{0.553581in}{1.601944in}%
\pgfsys@useobject{currentmarker}{}%
\end{pgfscope}%
\end{pgfscope}%
\begin{pgfscope}%
\definecolor{textcolor}{rgb}{0.000000,0.000000,0.000000}%
\pgfsetstrokecolor{textcolor}%
\pgfsetfillcolor{textcolor}%
\pgftext[x=0.278889in, y=1.553750in, left, base]{\color{textcolor}\rmfamily\fontsize{10.000000}{12.000000}\selectfont \(\displaystyle {1.0}\)}%
\end{pgfscope}%
\begin{pgfscope}%
\definecolor{textcolor}{rgb}{0.000000,0.000000,0.000000}%
\pgfsetstrokecolor{textcolor}%
\pgfsetfillcolor{textcolor}%
\pgftext[x=0.223333in,y=1.076944in,,bottom,rotate=90.000000]{\color{textcolor}\rmfamily\fontsize{10.000000}{12.000000}\selectfont True positive rate}%
\end{pgfscope}%
\begin{pgfscope}%
\pgfpathrectangle{\pgfqpoint{0.553581in}{0.499444in}}{\pgfqpoint{1.550000in}{1.155000in}}%
\pgfusepath{clip}%
\pgfsetbuttcap%
\pgfsetroundjoin%
\pgfsetlinewidth{1.505625pt}%
\definecolor{currentstroke}{rgb}{0.000000,0.000000,0.000000}%
\pgfsetstrokecolor{currentstroke}%
\pgfsetdash{{5.550000pt}{2.400000pt}}{0.000000pt}%
\pgfpathmoveto{\pgfqpoint{0.624035in}{0.551944in}}%
\pgfpathlineto{\pgfqpoint{2.033126in}{1.601944in}}%
\pgfusepath{stroke}%
\end{pgfscope}%
\begin{pgfscope}%
\pgfpathrectangle{\pgfqpoint{0.553581in}{0.499444in}}{\pgfqpoint{1.550000in}{1.155000in}}%
\pgfusepath{clip}%
\pgfsetrectcap%
\pgfsetroundjoin%
\pgfsetlinewidth{1.505625pt}%
\definecolor{currentstroke}{rgb}{0.000000,0.000000,0.000000}%
\pgfsetstrokecolor{currentstroke}%
\pgfsetdash{}{0pt}%
\pgfpathmoveto{\pgfqpoint{0.624035in}{0.551944in}}%
\pgfpathlineto{\pgfqpoint{0.625138in}{0.562530in}}%
\pgfpathlineto{\pgfqpoint{0.625294in}{0.563616in}}%
\pgfpathlineto{\pgfqpoint{0.626396in}{0.571501in}}%
\pgfpathlineto{\pgfqpoint{0.626553in}{0.572556in}}%
\pgfpathlineto{\pgfqpoint{0.627663in}{0.578827in}}%
\pgfpathlineto{\pgfqpoint{0.627928in}{0.579851in}}%
\pgfpathlineto{\pgfqpoint{0.629039in}{0.585594in}}%
\pgfpathlineto{\pgfqpoint{0.629336in}{0.586649in}}%
\pgfpathlineto{\pgfqpoint{0.630446in}{0.592516in}}%
\pgfpathlineto{\pgfqpoint{0.630579in}{0.593603in}}%
\pgfpathlineto{\pgfqpoint{0.631681in}{0.598166in}}%
\pgfpathlineto{\pgfqpoint{0.631915in}{0.599190in}}%
\pgfpathlineto{\pgfqpoint{0.633010in}{0.604467in}}%
\pgfpathlineto{\pgfqpoint{0.633026in}{0.604467in}}%
\pgfpathlineto{\pgfqpoint{0.633393in}{0.605523in}}%
\pgfpathlineto{\pgfqpoint{0.634472in}{0.610862in}}%
\pgfpathlineto{\pgfqpoint{0.634745in}{0.611918in}}%
\pgfpathlineto{\pgfqpoint{0.635824in}{0.617226in}}%
\pgfpathlineto{\pgfqpoint{0.636160in}{0.618250in}}%
\pgfpathlineto{\pgfqpoint{0.637255in}{0.622099in}}%
\pgfpathlineto{\pgfqpoint{0.637544in}{0.623155in}}%
\pgfpathlineto{\pgfqpoint{0.638654in}{0.627687in}}%
\pgfpathlineto{\pgfqpoint{0.638904in}{0.628618in}}%
\pgfpathlineto{\pgfqpoint{0.640015in}{0.633740in}}%
\pgfpathlineto{\pgfqpoint{0.640179in}{0.634640in}}%
\pgfpathlineto{\pgfqpoint{0.641265in}{0.639328in}}%
\pgfpathlineto{\pgfqpoint{0.641570in}{0.640321in}}%
\pgfpathlineto{\pgfqpoint{0.642680in}{0.644388in}}%
\pgfpathlineto{\pgfqpoint{0.642977in}{0.645443in}}%
\pgfpathlineto{\pgfqpoint{0.644072in}{0.649696in}}%
\pgfpathlineto{\pgfqpoint{0.644408in}{0.650689in}}%
\pgfpathlineto{\pgfqpoint{0.645502in}{0.655004in}}%
\pgfpathlineto{\pgfqpoint{0.645800in}{0.656028in}}%
\pgfpathlineto{\pgfqpoint{0.646878in}{0.660281in}}%
\pgfpathlineto{\pgfqpoint{0.647222in}{0.661368in}}%
\pgfpathlineto{\pgfqpoint{0.648332in}{0.665155in}}%
\pgfpathlineto{\pgfqpoint{0.648630in}{0.666241in}}%
\pgfpathlineto{\pgfqpoint{0.649732in}{0.671643in}}%
\pgfpathlineto{\pgfqpoint{0.650013in}{0.672667in}}%
\pgfpathlineto{\pgfqpoint{0.651108in}{0.676392in}}%
\pgfpathlineto{\pgfqpoint{0.651616in}{0.677479in}}%
\pgfpathlineto{\pgfqpoint{0.652710in}{0.681762in}}%
\pgfpathlineto{\pgfqpoint{0.653054in}{0.682849in}}%
\pgfpathlineto{\pgfqpoint{0.654164in}{0.687691in}}%
\pgfpathlineto{\pgfqpoint{0.654555in}{0.688654in}}%
\pgfpathlineto{\pgfqpoint{0.655650in}{0.693558in}}%
\pgfpathlineto{\pgfqpoint{0.655931in}{0.694583in}}%
\pgfpathlineto{\pgfqpoint{0.657041in}{0.698277in}}%
\pgfpathlineto{\pgfqpoint{0.657385in}{0.699363in}}%
\pgfpathlineto{\pgfqpoint{0.658495in}{0.703150in}}%
\pgfpathlineto{\pgfqpoint{0.658761in}{0.704206in}}%
\pgfpathlineto{\pgfqpoint{0.659871in}{0.707714in}}%
\pgfpathlineto{\pgfqpoint{0.660254in}{0.708800in}}%
\pgfpathlineto{\pgfqpoint{0.661364in}{0.712277in}}%
\pgfpathlineto{\pgfqpoint{0.661529in}{0.713363in}}%
\pgfpathlineto{\pgfqpoint{0.662639in}{0.716498in}}%
\pgfpathlineto{\pgfqpoint{0.663045in}{0.717585in}}%
\pgfpathlineto{\pgfqpoint{0.664155in}{0.721434in}}%
\pgfpathlineto{\pgfqpoint{0.664585in}{0.722365in}}%
\pgfpathlineto{\pgfqpoint{0.665649in}{0.725904in}}%
\pgfpathlineto{\pgfqpoint{0.665961in}{0.726960in}}%
\pgfpathlineto{\pgfqpoint{0.667056in}{0.730281in}}%
\pgfpathlineto{\pgfqpoint{0.667564in}{0.731368in}}%
\pgfpathlineto{\pgfqpoint{0.668658in}{0.734441in}}%
\pgfpathlineto{\pgfqpoint{0.669198in}{0.735527in}}%
\pgfpathlineto{\pgfqpoint{0.670292in}{0.738756in}}%
\pgfpathlineto{\pgfqpoint{0.670793in}{0.739842in}}%
\pgfpathlineto{\pgfqpoint{0.671879in}{0.742915in}}%
\pgfpathlineto{\pgfqpoint{0.672309in}{0.743971in}}%
\pgfpathlineto{\pgfqpoint{0.673411in}{0.747106in}}%
\pgfpathlineto{\pgfqpoint{0.673841in}{0.748193in}}%
\pgfpathlineto{\pgfqpoint{0.674928in}{0.751483in}}%
\pgfpathlineto{\pgfqpoint{0.674952in}{0.751483in}}%
\pgfpathlineto{\pgfqpoint{0.675374in}{0.752569in}}%
\pgfpathlineto{\pgfqpoint{0.676476in}{0.756046in}}%
\pgfpathlineto{\pgfqpoint{0.677000in}{0.757133in}}%
\pgfpathlineto{\pgfqpoint{0.678079in}{0.760858in}}%
\pgfpathlineto{\pgfqpoint{0.678493in}{0.761913in}}%
\pgfpathlineto{\pgfqpoint{0.679603in}{0.765390in}}%
\pgfpathlineto{\pgfqpoint{0.679931in}{0.766321in}}%
\pgfpathlineto{\pgfqpoint{0.681034in}{0.769798in}}%
\pgfpathlineto{\pgfqpoint{0.681354in}{0.770884in}}%
\pgfpathlineto{\pgfqpoint{0.682433in}{0.773678in}}%
\pgfpathlineto{\pgfqpoint{0.682785in}{0.774765in}}%
\pgfpathlineto{\pgfqpoint{0.683840in}{0.777217in}}%
\pgfpathlineto{\pgfqpoint{0.684231in}{0.778303in}}%
\pgfpathlineto{\pgfqpoint{0.685341in}{0.781004in}}%
\pgfpathlineto{\pgfqpoint{0.685771in}{0.782059in}}%
\pgfpathlineto{\pgfqpoint{0.686850in}{0.784605in}}%
\pgfpathlineto{\pgfqpoint{0.687358in}{0.785629in}}%
\pgfpathlineto{\pgfqpoint{0.688437in}{0.788702in}}%
\pgfpathlineto{\pgfqpoint{0.688757in}{0.789789in}}%
\pgfpathlineto{\pgfqpoint{0.689860in}{0.792117in}}%
\pgfpathlineto{\pgfqpoint{0.690329in}{0.793173in}}%
\pgfpathlineto{\pgfqpoint{0.691431in}{0.795470in}}%
\pgfpathlineto{\pgfqpoint{0.691845in}{0.796525in}}%
\pgfpathlineto{\pgfqpoint{0.692956in}{0.799785in}}%
\pgfpathlineto{\pgfqpoint{0.693425in}{0.800871in}}%
\pgfpathlineto{\pgfqpoint{0.694511in}{0.803727in}}%
\pgfpathlineto{\pgfqpoint{0.694918in}{0.804813in}}%
\pgfpathlineto{\pgfqpoint{0.696020in}{0.807452in}}%
\pgfpathlineto{\pgfqpoint{0.696481in}{0.808538in}}%
\pgfpathlineto{\pgfqpoint{0.697584in}{0.811581in}}%
\pgfpathlineto{\pgfqpoint{0.698076in}{0.812667in}}%
\pgfpathlineto{\pgfqpoint{0.699171in}{0.815244in}}%
\pgfpathlineto{\pgfqpoint{0.699694in}{0.816299in}}%
\pgfpathlineto{\pgfqpoint{0.700765in}{0.819062in}}%
\pgfpathlineto{\pgfqpoint{0.701266in}{0.820086in}}%
\pgfpathlineto{\pgfqpoint{0.702376in}{0.822569in}}%
\pgfpathlineto{\pgfqpoint{0.703111in}{0.823656in}}%
\pgfpathlineto{\pgfqpoint{0.704205in}{0.825922in}}%
\pgfpathlineto{\pgfqpoint{0.704619in}{0.827008in}}%
\pgfpathlineto{\pgfqpoint{0.705730in}{0.829678in}}%
\pgfpathlineto{\pgfqpoint{0.706222in}{0.830734in}}%
\pgfpathlineto{\pgfqpoint{0.707317in}{0.833434in}}%
\pgfpathlineto{\pgfqpoint{0.707778in}{0.834521in}}%
\pgfpathlineto{\pgfqpoint{0.708888in}{0.837594in}}%
\pgfpathlineto{\pgfqpoint{0.709607in}{0.838649in}}%
\pgfpathlineto{\pgfqpoint{0.710709in}{0.840946in}}%
\pgfpathlineto{\pgfqpoint{0.711343in}{0.842033in}}%
\pgfpathlineto{\pgfqpoint{0.712453in}{0.844454in}}%
\pgfpathlineto{\pgfqpoint{0.713125in}{0.845541in}}%
\pgfpathlineto{\pgfqpoint{0.714235in}{0.847310in}}%
\pgfpathlineto{\pgfqpoint{0.714782in}{0.848334in}}%
\pgfpathlineto{\pgfqpoint{0.715877in}{0.850787in}}%
\pgfpathlineto{\pgfqpoint{0.716346in}{0.851873in}}%
\pgfpathlineto{\pgfqpoint{0.717456in}{0.854729in}}%
\pgfpathlineto{\pgfqpoint{0.717847in}{0.855816in}}%
\pgfpathlineto{\pgfqpoint{0.718918in}{0.857740in}}%
\pgfpathlineto{\pgfqpoint{0.719356in}{0.858827in}}%
\pgfpathlineto{\pgfqpoint{0.720466in}{0.860627in}}%
\pgfpathlineto{\pgfqpoint{0.720896in}{0.861683in}}%
\pgfpathlineto{\pgfqpoint{0.722006in}{0.863980in}}%
\pgfpathlineto{\pgfqpoint{0.722420in}{0.865066in}}%
\pgfpathlineto{\pgfqpoint{0.723523in}{0.867394in}}%
\pgfpathlineto{\pgfqpoint{0.724117in}{0.868450in}}%
\pgfpathlineto{\pgfqpoint{0.725211in}{0.871057in}}%
\pgfpathlineto{\pgfqpoint{0.725758in}{0.872113in}}%
\pgfpathlineto{\pgfqpoint{0.726869in}{0.874782in}}%
\pgfpathlineto{\pgfqpoint{0.727424in}{0.875869in}}%
\pgfpathlineto{\pgfqpoint{0.728526in}{0.878166in}}%
\pgfpathlineto{\pgfqpoint{0.729034in}{0.879221in}}%
\pgfpathlineto{\pgfqpoint{0.730144in}{0.881736in}}%
\pgfpathlineto{\pgfqpoint{0.730605in}{0.882791in}}%
\pgfpathlineto{\pgfqpoint{0.731708in}{0.884933in}}%
\pgfpathlineto{\pgfqpoint{0.732161in}{0.886020in}}%
\pgfpathlineto{\pgfqpoint{0.733271in}{0.887789in}}%
\pgfpathlineto{\pgfqpoint{0.733936in}{0.888875in}}%
\pgfpathlineto{\pgfqpoint{0.735038in}{0.891421in}}%
\pgfpathlineto{\pgfqpoint{0.735609in}{0.892476in}}%
\pgfpathlineto{\pgfqpoint{0.736703in}{0.894370in}}%
\pgfpathlineto{\pgfqpoint{0.737250in}{0.895425in}}%
\pgfpathlineto{\pgfqpoint{0.738360in}{0.897133in}}%
\pgfpathlineto{\pgfqpoint{0.738806in}{0.898219in}}%
\pgfpathlineto{\pgfqpoint{0.739908in}{0.900330in}}%
\pgfpathlineto{\pgfqpoint{0.740549in}{0.901416in}}%
\pgfpathlineto{\pgfqpoint{0.741620in}{0.903434in}}%
\pgfpathlineto{\pgfqpoint{0.741636in}{0.903434in}}%
\pgfpathlineto{\pgfqpoint{0.742136in}{0.904521in}}%
\pgfpathlineto{\pgfqpoint{0.743246in}{0.906600in}}%
\pgfpathlineto{\pgfqpoint{0.743934in}{0.907687in}}%
\pgfpathlineto{\pgfqpoint{0.744998in}{0.909736in}}%
\pgfpathlineto{\pgfqpoint{0.745529in}{0.910791in}}%
\pgfpathlineto{\pgfqpoint{0.746639in}{0.912902in}}%
\pgfpathlineto{\pgfqpoint{0.747382in}{0.913957in}}%
\pgfpathlineto{\pgfqpoint{0.748492in}{0.915851in}}%
\pgfpathlineto{\pgfqpoint{0.749274in}{0.916906in}}%
\pgfpathlineto{\pgfqpoint{0.750376in}{0.919017in}}%
\pgfpathlineto{\pgfqpoint{0.751127in}{0.920104in}}%
\pgfpathlineto{\pgfqpoint{0.752221in}{0.921718in}}%
\pgfpathlineto{\pgfqpoint{0.752956in}{0.922804in}}%
\pgfpathlineto{\pgfqpoint{0.754058in}{0.924698in}}%
\pgfpathlineto{\pgfqpoint{0.754738in}{0.925785in}}%
\pgfpathlineto{\pgfqpoint{0.755833in}{0.927833in}}%
\pgfpathlineto{\pgfqpoint{0.756161in}{0.928920in}}%
\pgfpathlineto{\pgfqpoint{0.757263in}{0.931124in}}%
\pgfpathlineto{\pgfqpoint{0.758069in}{0.932210in}}%
\pgfpathlineto{\pgfqpoint{0.759179in}{0.934166in}}%
\pgfpathlineto{\pgfqpoint{0.759757in}{0.935252in}}%
\pgfpathlineto{\pgfqpoint{0.760828in}{0.937301in}}%
\pgfpathlineto{\pgfqpoint{0.761469in}{0.938388in}}%
\pgfpathlineto{\pgfqpoint{0.762478in}{0.939847in}}%
\pgfpathlineto{\pgfqpoint{0.763393in}{0.940902in}}%
\pgfpathlineto{\pgfqpoint{0.764495in}{0.942827in}}%
\pgfpathlineto{\pgfqpoint{0.765323in}{0.943913in}}%
\pgfpathlineto{\pgfqpoint{0.766426in}{0.945838in}}%
\pgfpathlineto{\pgfqpoint{0.767137in}{0.946924in}}%
\pgfpathlineto{\pgfqpoint{0.768232in}{0.949128in}}%
\pgfpathlineto{\pgfqpoint{0.768896in}{0.950215in}}%
\pgfpathlineto{\pgfqpoint{0.769998in}{0.952512in}}%
\pgfpathlineto{\pgfqpoint{0.770733in}{0.953598in}}%
\pgfpathlineto{\pgfqpoint{0.771828in}{0.955430in}}%
\pgfpathlineto{\pgfqpoint{0.772359in}{0.956485in}}%
\pgfpathlineto{\pgfqpoint{0.773469in}{0.958006in}}%
\pgfpathlineto{\pgfqpoint{0.774384in}{0.959062in}}%
\pgfpathlineto{\pgfqpoint{0.775455in}{0.960707in}}%
\pgfpathlineto{\pgfqpoint{0.776339in}{0.961762in}}%
\pgfpathlineto{\pgfqpoint{0.777449in}{0.963594in}}%
\pgfpathlineto{\pgfqpoint{0.778309in}{0.964680in}}%
\pgfpathlineto{\pgfqpoint{0.779419in}{0.966201in}}%
\pgfpathlineto{\pgfqpoint{0.779919in}{0.967257in}}%
\pgfpathlineto{\pgfqpoint{0.781029in}{0.969337in}}%
\pgfpathlineto{\pgfqpoint{0.781694in}{0.970392in}}%
\pgfpathlineto{\pgfqpoint{0.782804in}{0.972410in}}%
\pgfpathlineto{\pgfqpoint{0.783492in}{0.973496in}}%
\pgfpathlineto{\pgfqpoint{0.784602in}{0.975173in}}%
\pgfpathlineto{\pgfqpoint{0.785399in}{0.976259in}}%
\pgfpathlineto{\pgfqpoint{0.786470in}{0.977873in}}%
\pgfpathlineto{\pgfqpoint{0.787432in}{0.978960in}}%
\pgfpathlineto{\pgfqpoint{0.788526in}{0.980667in}}%
\pgfpathlineto{\pgfqpoint{0.789292in}{0.981722in}}%
\pgfpathlineto{\pgfqpoint{0.790395in}{0.983647in}}%
\pgfpathlineto{\pgfqpoint{0.791137in}{0.984734in}}%
\pgfpathlineto{\pgfqpoint{0.792216in}{0.986627in}}%
\pgfpathlineto{\pgfqpoint{0.792247in}{0.986627in}}%
\pgfpathlineto{\pgfqpoint{0.792857in}{0.987714in}}%
\pgfpathlineto{\pgfqpoint{0.793944in}{0.989483in}}%
\pgfpathlineto{\pgfqpoint{0.794608in}{0.990569in}}%
\pgfpathlineto{\pgfqpoint{0.795711in}{0.992215in}}%
\pgfpathlineto{\pgfqpoint{0.796453in}{0.993301in}}%
\pgfpathlineto{\pgfqpoint{0.797556in}{0.995257in}}%
\pgfpathlineto{\pgfqpoint{0.798251in}{0.996343in}}%
\pgfpathlineto{\pgfqpoint{0.799362in}{0.997585in}}%
\pgfpathlineto{\pgfqpoint{0.800440in}{0.998671in}}%
\pgfpathlineto{\pgfqpoint{0.801519in}{1.000286in}}%
\pgfpathlineto{\pgfqpoint{0.802231in}{1.001341in}}%
\pgfpathlineto{\pgfqpoint{0.803317in}{1.002831in}}%
\pgfpathlineto{\pgfqpoint{0.804162in}{1.003918in}}%
\pgfpathlineto{\pgfqpoint{0.805256in}{1.005377in}}%
\pgfpathlineto{\pgfqpoint{0.806100in}{1.006432in}}%
\pgfpathlineto{\pgfqpoint{0.807210in}{1.008015in}}%
\pgfpathlineto{\pgfqpoint{0.808016in}{1.009102in}}%
\pgfpathlineto{\pgfqpoint{0.809094in}{1.010871in}}%
\pgfpathlineto{\pgfqpoint{0.810181in}{1.011957in}}%
\pgfpathlineto{\pgfqpoint{0.811221in}{1.013261in}}%
\pgfpathlineto{\pgfqpoint{0.811893in}{1.014348in}}%
\pgfpathlineto{\pgfqpoint{0.812988in}{1.015931in}}%
\pgfpathlineto{\pgfqpoint{0.813871in}{1.017017in}}%
\pgfpathlineto{\pgfqpoint{0.814903in}{1.018321in}}%
\pgfpathlineto{\pgfqpoint{0.815857in}{1.019408in}}%
\pgfpathlineto{\pgfqpoint{0.816967in}{1.020929in}}%
\pgfpathlineto{\pgfqpoint{0.818139in}{1.021984in}}%
\pgfpathlineto{\pgfqpoint{0.819234in}{1.023536in}}%
\pgfpathlineto{\pgfqpoint{0.820172in}{1.024623in}}%
\pgfpathlineto{\pgfqpoint{0.821267in}{1.026392in}}%
\pgfpathlineto{\pgfqpoint{0.822244in}{1.027479in}}%
\pgfpathlineto{\pgfqpoint{0.823354in}{1.029403in}}%
\pgfpathlineto{\pgfqpoint{0.824347in}{1.030490in}}%
\pgfpathlineto{\pgfqpoint{0.825441in}{1.031949in}}%
\pgfpathlineto{\pgfqpoint{0.826309in}{1.033035in}}%
\pgfpathlineto{\pgfqpoint{0.827411in}{1.034618in}}%
\pgfpathlineto{\pgfqpoint{0.827419in}{1.034618in}}%
\pgfpathlineto{\pgfqpoint{0.827943in}{1.035643in}}%
\pgfpathlineto{\pgfqpoint{0.829006in}{1.037381in}}%
\pgfpathlineto{\pgfqpoint{0.829053in}{1.037381in}}%
\pgfpathlineto{\pgfqpoint{0.830460in}{1.038467in}}%
\pgfpathlineto{\pgfqpoint{0.831570in}{1.039989in}}%
\pgfpathlineto{\pgfqpoint{0.832289in}{1.041075in}}%
\pgfpathlineto{\pgfqpoint{0.833368in}{1.042472in}}%
\pgfpathlineto{\pgfqpoint{0.834603in}{1.043558in}}%
\pgfpathlineto{\pgfqpoint{0.835604in}{1.044924in}}%
\pgfpathlineto{\pgfqpoint{0.835690in}{1.044924in}}%
\pgfpathlineto{\pgfqpoint{0.836738in}{1.045980in}}%
\pgfpathlineto{\pgfqpoint{0.837816in}{1.047656in}}%
\pgfpathlineto{\pgfqpoint{0.838575in}{1.048742in}}%
\pgfpathlineto{\pgfqpoint{0.839638in}{1.050232in}}%
\pgfpathlineto{\pgfqpoint{0.839685in}{1.050232in}}%
\pgfpathlineto{\pgfqpoint{0.840381in}{1.051319in}}%
\pgfpathlineto{\pgfqpoint{0.841467in}{1.052436in}}%
\pgfpathlineto{\pgfqpoint{0.842687in}{1.053523in}}%
\pgfpathlineto{\pgfqpoint{0.843789in}{1.054982in}}%
\pgfpathlineto{\pgfqpoint{0.844774in}{1.056068in}}%
\pgfpathlineto{\pgfqpoint{0.845869in}{1.057714in}}%
\pgfpathlineto{\pgfqpoint{0.846799in}{1.058769in}}%
\pgfpathlineto{\pgfqpoint{0.847807in}{1.060383in}}%
\pgfpathlineto{\pgfqpoint{0.848839in}{1.061439in}}%
\pgfpathlineto{\pgfqpoint{0.849879in}{1.062370in}}%
\pgfpathlineto{\pgfqpoint{0.849949in}{1.062370in}}%
\pgfpathlineto{\pgfqpoint{0.851091in}{1.063456in}}%
\pgfpathlineto{\pgfqpoint{0.852170in}{1.064574in}}%
\pgfpathlineto{\pgfqpoint{0.853311in}{1.065660in}}%
\pgfpathlineto{\pgfqpoint{0.854421in}{1.066933in}}%
\pgfpathlineto{\pgfqpoint{0.855516in}{1.068020in}}%
\pgfpathlineto{\pgfqpoint{0.856571in}{1.069013in}}%
\pgfpathlineto{\pgfqpoint{0.857282in}{1.070099in}}%
\pgfpathlineto{\pgfqpoint{0.858385in}{1.071993in}}%
\pgfpathlineto{\pgfqpoint{0.859534in}{1.073079in}}%
\pgfpathlineto{\pgfqpoint{0.860636in}{1.074135in}}%
\pgfpathlineto{\pgfqpoint{0.861738in}{1.075221in}}%
\pgfpathlineto{\pgfqpoint{0.862849in}{1.076494in}}%
\pgfpathlineto{\pgfqpoint{0.863881in}{1.077581in}}%
\pgfpathlineto{\pgfqpoint{0.864991in}{1.079381in}}%
\pgfpathlineto{\pgfqpoint{0.865952in}{1.080467in}}%
\pgfpathlineto{\pgfqpoint{0.867047in}{1.081461in}}%
\pgfpathlineto{\pgfqpoint{0.867774in}{1.082547in}}%
\pgfpathlineto{\pgfqpoint{0.868868in}{1.083665in}}%
\pgfpathlineto{\pgfqpoint{0.869798in}{1.084751in}}%
\pgfpathlineto{\pgfqpoint{0.870901in}{1.086055in}}%
\pgfpathlineto{\pgfqpoint{0.871589in}{1.087142in}}%
\pgfpathlineto{\pgfqpoint{0.872589in}{1.089035in}}%
\pgfpathlineto{\pgfqpoint{0.873676in}{1.090122in}}%
\pgfpathlineto{\pgfqpoint{0.874786in}{1.091518in}}%
\pgfpathlineto{\pgfqpoint{0.875693in}{1.092605in}}%
\pgfpathlineto{\pgfqpoint{0.876772in}{1.094095in}}%
\pgfpathlineto{\pgfqpoint{0.878343in}{1.095181in}}%
\pgfpathlineto{\pgfqpoint{0.879406in}{1.096702in}}%
\pgfpathlineto{\pgfqpoint{0.880219in}{1.097789in}}%
\pgfpathlineto{\pgfqpoint{0.881322in}{1.099186in}}%
\pgfpathlineto{\pgfqpoint{0.882338in}{1.100272in}}%
\pgfpathlineto{\pgfqpoint{0.883417in}{1.101390in}}%
\pgfpathlineto{\pgfqpoint{0.884308in}{1.102414in}}%
\pgfpathlineto{\pgfqpoint{0.885410in}{1.103532in}}%
\pgfpathlineto{\pgfqpoint{0.886341in}{1.104618in}}%
\pgfpathlineto{\pgfqpoint{0.887427in}{1.105891in}}%
\pgfpathlineto{\pgfqpoint{0.888334in}{1.106946in}}%
\pgfpathlineto{\pgfqpoint{0.889444in}{1.108374in}}%
\pgfpathlineto{\pgfqpoint{0.890640in}{1.109461in}}%
\pgfpathlineto{\pgfqpoint{0.891711in}{1.110671in}}%
\pgfpathlineto{\pgfqpoint{0.892868in}{1.111758in}}%
\pgfpathlineto{\pgfqpoint{0.893939in}{1.112969in}}%
\pgfpathlineto{\pgfqpoint{0.894854in}{1.114055in}}%
\pgfpathlineto{\pgfqpoint{0.895941in}{1.115514in}}%
\pgfpathlineto{\pgfqpoint{0.896949in}{1.116600in}}%
\pgfpathlineto{\pgfqpoint{0.898036in}{1.118246in}}%
\pgfpathlineto{\pgfqpoint{0.898966in}{1.119332in}}%
\pgfpathlineto{\pgfqpoint{0.900037in}{1.120326in}}%
\pgfpathlineto{\pgfqpoint{0.900881in}{1.121412in}}%
\pgfpathlineto{\pgfqpoint{0.901945in}{1.122405in}}%
\pgfpathlineto{\pgfqpoint{0.903438in}{1.123492in}}%
\pgfpathlineto{\pgfqpoint{0.904540in}{1.124423in}}%
\pgfpathlineto{\pgfqpoint{0.905369in}{1.125510in}}%
\pgfpathlineto{\pgfqpoint{0.906479in}{1.127093in}}%
\pgfpathlineto{\pgfqpoint{0.907542in}{1.128179in}}%
\pgfpathlineto{\pgfqpoint{0.908637in}{1.129545in}}%
\pgfpathlineto{\pgfqpoint{0.909645in}{1.130600in}}%
\pgfpathlineto{\pgfqpoint{0.910755in}{1.131904in}}%
\pgfpathlineto{\pgfqpoint{0.911732in}{1.132991in}}%
\pgfpathlineto{\pgfqpoint{0.912741in}{1.134015in}}%
\pgfpathlineto{\pgfqpoint{0.913609in}{1.135071in}}%
\pgfpathlineto{\pgfqpoint{0.914687in}{1.135878in}}%
\pgfpathlineto{\pgfqpoint{0.915782in}{1.136964in}}%
\pgfpathlineto{\pgfqpoint{0.916892in}{1.138020in}}%
\pgfpathlineto{\pgfqpoint{0.918299in}{1.139075in}}%
\pgfpathlineto{\pgfqpoint{0.919378in}{1.140658in}}%
\pgfpathlineto{\pgfqpoint{0.920371in}{1.141714in}}%
\pgfpathlineto{\pgfqpoint{0.921473in}{1.142831in}}%
\pgfpathlineto{\pgfqpoint{0.922763in}{1.143918in}}%
\pgfpathlineto{\pgfqpoint{0.923865in}{1.145345in}}%
\pgfpathlineto{\pgfqpoint{0.925421in}{1.146401in}}%
\pgfpathlineto{\pgfqpoint{0.926515in}{1.147612in}}%
\pgfpathlineto{\pgfqpoint{0.927743in}{1.148698in}}%
\pgfpathlineto{\pgfqpoint{0.928775in}{1.149598in}}%
\pgfpathlineto{\pgfqpoint{0.929846in}{1.150685in}}%
\pgfpathlineto{\pgfqpoint{0.930932in}{1.152175in}}%
\pgfpathlineto{\pgfqpoint{0.931957in}{1.153261in}}%
\pgfpathlineto{\pgfqpoint{0.933051in}{1.154317in}}%
\pgfpathlineto{\pgfqpoint{0.934200in}{1.155403in}}%
\pgfpathlineto{\pgfqpoint{0.935279in}{1.156800in}}%
\pgfpathlineto{\pgfqpoint{0.936209in}{1.157824in}}%
\pgfpathlineto{\pgfqpoint{0.937304in}{1.158632in}}%
\pgfpathlineto{\pgfqpoint{0.938664in}{1.159718in}}%
\pgfpathlineto{\pgfqpoint{0.939680in}{1.160618in}}%
\pgfpathlineto{\pgfqpoint{0.941158in}{1.161705in}}%
\pgfpathlineto{\pgfqpoint{0.942268in}{1.163008in}}%
\pgfpathlineto{\pgfqpoint{0.943792in}{1.164095in}}%
\pgfpathlineto{\pgfqpoint{0.944863in}{1.165306in}}%
\pgfpathlineto{\pgfqpoint{0.946075in}{1.166361in}}%
\pgfpathlineto{\pgfqpoint{0.947177in}{1.167541in}}%
\pgfpathlineto{\pgfqpoint{0.948248in}{1.168627in}}%
\pgfpathlineto{\pgfqpoint{0.949351in}{1.169434in}}%
\pgfpathlineto{\pgfqpoint{0.950789in}{1.170521in}}%
\pgfpathlineto{\pgfqpoint{0.951899in}{1.171762in}}%
\pgfpathlineto{\pgfqpoint{0.953009in}{1.172849in}}%
\pgfpathlineto{\pgfqpoint{0.954120in}{1.173749in}}%
\pgfpathlineto{\pgfqpoint{0.955081in}{1.174804in}}%
\pgfpathlineto{\pgfqpoint{0.956129in}{1.175829in}}%
\pgfpathlineto{\pgfqpoint{0.957364in}{1.176915in}}%
\pgfpathlineto{\pgfqpoint{0.958411in}{1.177753in}}%
\pgfpathlineto{\pgfqpoint{0.959772in}{1.178840in}}%
\pgfpathlineto{\pgfqpoint{0.960788in}{1.180020in}}%
\pgfpathlineto{\pgfqpoint{0.961875in}{1.181106in}}%
\pgfpathlineto{\pgfqpoint{0.962961in}{1.182472in}}%
\pgfpathlineto{\pgfqpoint{0.964454in}{1.183558in}}%
\pgfpathlineto{\pgfqpoint{0.965557in}{1.185017in}}%
\pgfpathlineto{\pgfqpoint{0.966901in}{1.186104in}}%
\pgfpathlineto{\pgfqpoint{0.967980in}{1.187314in}}%
\pgfpathlineto{\pgfqpoint{0.968871in}{1.188401in}}%
\pgfpathlineto{\pgfqpoint{0.969950in}{1.189301in}}%
\pgfpathlineto{\pgfqpoint{0.971060in}{1.190357in}}%
\pgfpathlineto{\pgfqpoint{0.972163in}{1.191785in}}%
\pgfpathlineto{\pgfqpoint{0.973398in}{1.192871in}}%
\pgfpathlineto{\pgfqpoint{0.974500in}{1.193802in}}%
\pgfpathlineto{\pgfqpoint{0.975814in}{1.194889in}}%
\pgfpathlineto{\pgfqpoint{0.976799in}{1.195975in}}%
\pgfpathlineto{\pgfqpoint{0.976908in}{1.195975in}}%
\pgfpathlineto{\pgfqpoint{0.977862in}{1.197062in}}%
\pgfpathlineto{\pgfqpoint{0.978972in}{1.198148in}}%
\pgfpathlineto{\pgfqpoint{0.980152in}{1.199235in}}%
\pgfpathlineto{\pgfqpoint{0.981262in}{1.200445in}}%
\pgfpathlineto{\pgfqpoint{0.982529in}{1.201532in}}%
\pgfpathlineto{\pgfqpoint{0.983639in}{1.202525in}}%
\pgfpathlineto{\pgfqpoint{0.984851in}{1.203612in}}%
\pgfpathlineto{\pgfqpoint{0.985937in}{1.204667in}}%
\pgfpathlineto{\pgfqpoint{0.987329in}{1.205722in}}%
\pgfpathlineto{\pgfqpoint{0.988330in}{1.206871in}}%
\pgfpathlineto{\pgfqpoint{0.989565in}{1.207957in}}%
\pgfpathlineto{\pgfqpoint{0.990628in}{1.209075in}}%
\pgfpathlineto{\pgfqpoint{0.992105in}{1.210099in}}%
\pgfpathlineto{\pgfqpoint{0.993200in}{1.211279in}}%
\pgfpathlineto{\pgfqpoint{0.994505in}{1.212365in}}%
\pgfpathlineto{\pgfqpoint{0.995514in}{1.213048in}}%
\pgfpathlineto{\pgfqpoint{0.996702in}{1.214135in}}%
\pgfpathlineto{\pgfqpoint{0.997797in}{1.215159in}}%
\pgfpathlineto{\pgfqpoint{0.998735in}{1.216246in}}%
\pgfpathlineto{\pgfqpoint{0.999790in}{1.217425in}}%
\pgfpathlineto{\pgfqpoint{1.001033in}{1.218481in}}%
\pgfpathlineto{\pgfqpoint{1.001917in}{1.219319in}}%
\pgfpathlineto{\pgfqpoint{1.002026in}{1.219319in}}%
\pgfpathlineto{\pgfqpoint{1.004082in}{1.220405in}}%
\pgfpathlineto{\pgfqpoint{1.005184in}{1.221212in}}%
\pgfpathlineto{\pgfqpoint{1.006271in}{1.222299in}}%
\pgfpathlineto{\pgfqpoint{1.007366in}{1.223354in}}%
\pgfpathlineto{\pgfqpoint{1.009265in}{1.224441in}}%
\pgfpathlineto{\pgfqpoint{1.010625in}{1.225527in}}%
\pgfpathlineto{\pgfqpoint{1.012072in}{1.226614in}}%
\pgfpathlineto{\pgfqpoint{1.013158in}{1.227669in}}%
\pgfpathlineto{\pgfqpoint{1.014894in}{1.228756in}}%
\pgfpathlineto{\pgfqpoint{1.015918in}{1.229687in}}%
\pgfpathlineto{\pgfqpoint{1.017075in}{1.230773in}}%
\pgfpathlineto{\pgfqpoint{1.018162in}{1.231829in}}%
\pgfpathlineto{\pgfqpoint{1.018185in}{1.231829in}}%
\pgfpathlineto{\pgfqpoint{1.019694in}{1.232915in}}%
\pgfpathlineto{\pgfqpoint{1.020773in}{1.233847in}}%
\pgfpathlineto{\pgfqpoint{1.022156in}{1.234933in}}%
\pgfpathlineto{\pgfqpoint{1.023251in}{1.235709in}}%
\pgfpathlineto{\pgfqpoint{1.023267in}{1.235709in}}%
\pgfpathlineto{\pgfqpoint{1.024705in}{1.236765in}}%
\pgfpathlineto{\pgfqpoint{1.025760in}{1.237541in}}%
\pgfpathlineto{\pgfqpoint{1.027293in}{1.238627in}}%
\pgfpathlineto{\pgfqpoint{1.028395in}{1.239434in}}%
\pgfpathlineto{\pgfqpoint{1.029364in}{1.240521in}}%
\pgfpathlineto{\pgfqpoint{1.030435in}{1.241731in}}%
\pgfpathlineto{\pgfqpoint{1.031600in}{1.242818in}}%
\pgfpathlineto{\pgfqpoint{1.032702in}{1.243873in}}%
\pgfpathlineto{\pgfqpoint{1.034047in}{1.244960in}}%
\pgfpathlineto{\pgfqpoint{1.035095in}{1.245549in}}%
\pgfpathlineto{\pgfqpoint{1.036494in}{1.246636in}}%
\pgfpathlineto{\pgfqpoint{1.037573in}{1.247816in}}%
\pgfpathlineto{\pgfqpoint{1.038613in}{1.248902in}}%
\pgfpathlineto{\pgfqpoint{1.039551in}{1.249616in}}%
\pgfpathlineto{\pgfqpoint{1.041599in}{1.250702in}}%
\pgfpathlineto{\pgfqpoint{1.042701in}{1.251572in}}%
\pgfpathlineto{\pgfqpoint{1.044335in}{1.252658in}}%
\pgfpathlineto{\pgfqpoint{1.045406in}{1.253434in}}%
\pgfpathlineto{\pgfqpoint{1.046688in}{1.254521in}}%
\pgfpathlineto{\pgfqpoint{1.047791in}{1.255824in}}%
\pgfpathlineto{\pgfqpoint{1.048979in}{1.256911in}}%
\pgfpathlineto{\pgfqpoint{1.049995in}{1.257656in}}%
\pgfpathlineto{\pgfqpoint{1.051723in}{1.258742in}}%
\pgfpathlineto{\pgfqpoint{1.052833in}{1.259705in}}%
\pgfpathlineto{\pgfqpoint{1.053959in}{1.260791in}}%
\pgfpathlineto{\pgfqpoint{1.054936in}{1.261567in}}%
\pgfpathlineto{\pgfqpoint{1.056538in}{1.262654in}}%
\pgfpathlineto{\pgfqpoint{1.057578in}{1.263802in}}%
\pgfpathlineto{\pgfqpoint{1.059243in}{1.264889in}}%
\pgfpathlineto{\pgfqpoint{1.060314in}{1.265820in}}%
\pgfpathlineto{\pgfqpoint{1.061925in}{1.266906in}}%
\pgfpathlineto{\pgfqpoint{1.063004in}{1.268148in}}%
\pgfpathlineto{\pgfqpoint{1.064380in}{1.269235in}}%
\pgfpathlineto{\pgfqpoint{1.065490in}{1.269918in}}%
\pgfpathlineto{\pgfqpoint{1.066928in}{1.271004in}}%
\pgfpathlineto{\pgfqpoint{1.068030in}{1.272215in}}%
\pgfpathlineto{\pgfqpoint{1.069610in}{1.273270in}}%
\pgfpathlineto{\pgfqpoint{1.070712in}{1.274357in}}%
\pgfpathlineto{\pgfqpoint{1.072111in}{1.275412in}}%
\pgfpathlineto{\pgfqpoint{1.073182in}{1.276126in}}%
\pgfpathlineto{\pgfqpoint{1.074628in}{1.277212in}}%
\pgfpathlineto{\pgfqpoint{1.075731in}{1.277989in}}%
\pgfpathlineto{\pgfqpoint{1.077591in}{1.279075in}}%
\pgfpathlineto{\pgfqpoint{1.078694in}{1.279758in}}%
\pgfpathlineto{\pgfqpoint{1.080937in}{1.280844in}}%
\pgfpathlineto{\pgfqpoint{1.082032in}{1.281714in}}%
\pgfpathlineto{\pgfqpoint{1.083048in}{1.282800in}}%
\pgfpathlineto{\pgfqpoint{1.084150in}{1.283638in}}%
\pgfpathlineto{\pgfqpoint{1.085612in}{1.284725in}}%
\pgfpathlineto{\pgfqpoint{1.086722in}{1.285345in}}%
\pgfpathlineto{\pgfqpoint{1.088497in}{1.286432in}}%
\pgfpathlineto{\pgfqpoint{1.089576in}{1.287270in}}%
\pgfpathlineto{\pgfqpoint{1.091538in}{1.288357in}}%
\pgfpathlineto{\pgfqpoint{1.092617in}{1.289164in}}%
\pgfpathlineto{\pgfqpoint{1.094329in}{1.290219in}}%
\pgfpathlineto{\pgfqpoint{1.095361in}{1.290964in}}%
\pgfpathlineto{\pgfqpoint{1.096885in}{1.292051in}}%
\pgfpathlineto{\pgfqpoint{1.097886in}{1.292827in}}%
\pgfpathlineto{\pgfqpoint{1.099629in}{1.293913in}}%
\pgfpathlineto{\pgfqpoint{1.100489in}{1.294286in}}%
\pgfpathlineto{\pgfqpoint{1.100732in}{1.294286in}}%
\pgfpathlineto{\pgfqpoint{1.102287in}{1.295372in}}%
\pgfpathlineto{\pgfqpoint{1.103397in}{1.296241in}}%
\pgfpathlineto{\pgfqpoint{1.104758in}{1.297297in}}%
\pgfpathlineto{\pgfqpoint{1.105852in}{1.297980in}}%
\pgfpathlineto{\pgfqpoint{1.107291in}{1.299066in}}%
\pgfpathlineto{\pgfqpoint{1.108385in}{1.299935in}}%
\pgfpathlineto{\pgfqpoint{1.109980in}{1.301022in}}%
\pgfpathlineto{\pgfqpoint{1.111059in}{1.302046in}}%
\pgfpathlineto{\pgfqpoint{1.112740in}{1.303133in}}%
\pgfpathlineto{\pgfqpoint{1.113826in}{1.303722in}}%
\pgfpathlineto{\pgfqpoint{1.114897in}{1.304809in}}%
\pgfpathlineto{\pgfqpoint{1.115992in}{1.305585in}}%
\pgfpathlineto{\pgfqpoint{1.118110in}{1.306671in}}%
\pgfpathlineto{\pgfqpoint{1.119189in}{1.307447in}}%
\pgfpathlineto{\pgfqpoint{1.120635in}{1.308534in}}%
\pgfpathlineto{\pgfqpoint{1.121706in}{1.309217in}}%
\pgfpathlineto{\pgfqpoint{1.122942in}{1.310303in}}%
\pgfpathlineto{\pgfqpoint{1.124028in}{1.310831in}}%
\pgfpathlineto{\pgfqpoint{1.124044in}{1.310831in}}%
\pgfpathlineto{\pgfqpoint{1.125459in}{1.311918in}}%
\pgfpathlineto{\pgfqpoint{1.126530in}{1.312818in}}%
\pgfpathlineto{\pgfqpoint{1.128891in}{1.313904in}}%
\pgfpathlineto{\pgfqpoint{1.129977in}{1.315084in}}%
\pgfpathlineto{\pgfqpoint{1.131650in}{1.316170in}}%
\pgfpathlineto{\pgfqpoint{1.132628in}{1.316822in}}%
\pgfpathlineto{\pgfqpoint{1.133980in}{1.317909in}}%
\pgfpathlineto{\pgfqpoint{1.134981in}{1.318561in}}%
\pgfpathlineto{\pgfqpoint{1.136841in}{1.319647in}}%
\pgfpathlineto{\pgfqpoint{1.137936in}{1.320330in}}%
\pgfpathlineto{\pgfqpoint{1.139601in}{1.321416in}}%
\pgfpathlineto{\pgfqpoint{1.140680in}{1.322037in}}%
\pgfpathlineto{\pgfqpoint{1.142298in}{1.323124in}}%
\pgfpathlineto{\pgfqpoint{1.143322in}{1.323807in}}%
\pgfpathlineto{\pgfqpoint{1.145284in}{1.324893in}}%
\pgfpathlineto{\pgfqpoint{1.146199in}{1.325638in}}%
\pgfpathlineto{\pgfqpoint{1.147825in}{1.326725in}}%
\pgfpathlineto{\pgfqpoint{1.148873in}{1.327780in}}%
\pgfpathlineto{\pgfqpoint{1.149944in}{1.328867in}}%
\pgfpathlineto{\pgfqpoint{1.151022in}{1.329549in}}%
\pgfpathlineto{\pgfqpoint{1.152609in}{1.330636in}}%
\pgfpathlineto{\pgfqpoint{1.153610in}{1.331536in}}%
\pgfpathlineto{\pgfqpoint{1.155369in}{1.332623in}}%
\pgfpathlineto{\pgfqpoint{1.156464in}{1.333430in}}%
\pgfpathlineto{\pgfqpoint{1.158011in}{1.334516in}}%
\pgfpathlineto{\pgfqpoint{1.159075in}{1.335292in}}%
\pgfpathlineto{\pgfqpoint{1.161209in}{1.336379in}}%
\pgfpathlineto{\pgfqpoint{1.162272in}{1.336969in}}%
\pgfpathlineto{\pgfqpoint{1.164062in}{1.338055in}}%
\pgfpathlineto{\pgfqpoint{1.165172in}{1.338769in}}%
\pgfpathlineto{\pgfqpoint{1.166978in}{1.339855in}}%
\pgfpathlineto{\pgfqpoint{1.168002in}{1.340569in}}%
\pgfpathlineto{\pgfqpoint{1.169636in}{1.341656in}}%
\pgfpathlineto{\pgfqpoint{1.170676in}{1.342649in}}%
\pgfpathlineto{\pgfqpoint{1.173053in}{1.343736in}}%
\pgfpathlineto{\pgfqpoint{1.174163in}{1.344636in}}%
\pgfpathlineto{\pgfqpoint{1.175898in}{1.345722in}}%
\pgfpathlineto{\pgfqpoint{1.177001in}{1.346561in}}%
\pgfpathlineto{\pgfqpoint{1.178556in}{1.347647in}}%
\pgfpathlineto{\pgfqpoint{1.179635in}{1.348330in}}%
\pgfpathlineto{\pgfqpoint{1.181261in}{1.349416in}}%
\pgfpathlineto{\pgfqpoint{1.182348in}{1.350286in}}%
\pgfpathlineto{\pgfqpoint{1.184216in}{1.351372in}}%
\pgfpathlineto{\pgfqpoint{1.185318in}{1.352086in}}%
\pgfpathlineto{\pgfqpoint{1.186663in}{1.353173in}}%
\pgfpathlineto{\pgfqpoint{1.187656in}{1.353762in}}%
\pgfpathlineto{\pgfqpoint{1.189563in}{1.354849in}}%
\pgfpathlineto{\pgfqpoint{1.190634in}{1.355780in}}%
\pgfpathlineto{\pgfqpoint{1.192620in}{1.356867in}}%
\pgfpathlineto{\pgfqpoint{1.193683in}{1.357674in}}%
\pgfpathlineto{\pgfqpoint{1.194934in}{1.358760in}}%
\pgfpathlineto{\pgfqpoint{1.195950in}{1.359536in}}%
\pgfpathlineto{\pgfqpoint{1.196029in}{1.359536in}}%
\pgfpathlineto{\pgfqpoint{1.197788in}{1.360623in}}%
\pgfpathlineto{\pgfqpoint{1.198898in}{1.361275in}}%
\pgfpathlineto{\pgfqpoint{1.201290in}{1.362361in}}%
\pgfpathlineto{\pgfqpoint{1.202400in}{1.363230in}}%
\pgfpathlineto{\pgfqpoint{1.204581in}{1.364286in}}%
\pgfpathlineto{\pgfqpoint{1.205691in}{1.365031in}}%
\pgfpathlineto{\pgfqpoint{1.207528in}{1.366117in}}%
\pgfpathlineto{\pgfqpoint{1.208631in}{1.366769in}}%
\pgfpathlineto{\pgfqpoint{1.210491in}{1.367824in}}%
\pgfpathlineto{\pgfqpoint{1.211539in}{1.368507in}}%
\pgfpathlineto{\pgfqpoint{1.213267in}{1.369594in}}%
\pgfpathlineto{\pgfqpoint{1.214330in}{1.370463in}}%
\pgfpathlineto{\pgfqpoint{1.216159in}{1.371549in}}%
\pgfpathlineto{\pgfqpoint{1.217246in}{1.372232in}}%
\pgfpathlineto{\pgfqpoint{1.219622in}{1.373319in}}%
\pgfpathlineto{\pgfqpoint{1.220623in}{1.374157in}}%
\pgfpathlineto{\pgfqpoint{1.222437in}{1.375244in}}%
\pgfpathlineto{\pgfqpoint{1.223547in}{1.376113in}}%
\pgfpathlineto{\pgfqpoint{1.225157in}{1.377199in}}%
\pgfpathlineto{\pgfqpoint{1.226259in}{1.378161in}}%
\pgfpathlineto{\pgfqpoint{1.228980in}{1.379248in}}%
\pgfpathlineto{\pgfqpoint{1.229942in}{1.379807in}}%
\pgfpathlineto{\pgfqpoint{1.230043in}{1.379807in}}%
\pgfpathlineto{\pgfqpoint{1.231763in}{1.380893in}}%
\pgfpathlineto{\pgfqpoint{1.232850in}{1.381762in}}%
\pgfpathlineto{\pgfqpoint{1.235359in}{1.382849in}}%
\pgfpathlineto{\pgfqpoint{1.236430in}{1.383656in}}%
\pgfpathlineto{\pgfqpoint{1.238494in}{1.384742in}}%
\pgfpathlineto{\pgfqpoint{1.239557in}{1.385053in}}%
\pgfpathlineto{\pgfqpoint{1.241371in}{1.386139in}}%
\pgfpathlineto{\pgfqpoint{1.242426in}{1.386822in}}%
\pgfpathlineto{\pgfqpoint{1.244248in}{1.387878in}}%
\pgfpathlineto{\pgfqpoint{1.245342in}{1.388871in}}%
\pgfpathlineto{\pgfqpoint{1.247930in}{1.389957in}}%
\pgfpathlineto{\pgfqpoint{1.248946in}{1.390578in}}%
\pgfpathlineto{\pgfqpoint{1.250658in}{1.391634in}}%
\pgfpathlineto{\pgfqpoint{1.251643in}{1.392161in}}%
\pgfpathlineto{\pgfqpoint{1.253965in}{1.393248in}}%
\pgfpathlineto{\pgfqpoint{1.255052in}{1.393931in}}%
\pgfpathlineto{\pgfqpoint{1.256920in}{1.395017in}}%
\pgfpathlineto{\pgfqpoint{1.258023in}{1.395390in}}%
\pgfpathlineto{\pgfqpoint{1.259438in}{1.396476in}}%
\pgfpathlineto{\pgfqpoint{1.260509in}{1.397128in}}%
\pgfpathlineto{\pgfqpoint{1.263120in}{1.398215in}}%
\pgfpathlineto{\pgfqpoint{1.264222in}{1.399084in}}%
\pgfpathlineto{\pgfqpoint{1.265731in}{1.400170in}}%
\pgfpathlineto{\pgfqpoint{1.266778in}{1.400729in}}%
\pgfpathlineto{\pgfqpoint{1.268701in}{1.401816in}}%
\pgfpathlineto{\pgfqpoint{1.269765in}{1.402343in}}%
\pgfpathlineto{\pgfqpoint{1.272595in}{1.403430in}}%
\pgfpathlineto{\pgfqpoint{1.273642in}{1.403926in}}%
\pgfpathlineto{\pgfqpoint{1.275792in}{1.405013in}}%
\pgfpathlineto{\pgfqpoint{1.276652in}{1.405603in}}%
\pgfpathlineto{\pgfqpoint{1.279169in}{1.406689in}}%
\pgfpathlineto{\pgfqpoint{1.280272in}{1.407217in}}%
\pgfpathlineto{\pgfqpoint{1.282312in}{1.408303in}}%
\pgfpathlineto{\pgfqpoint{1.283344in}{1.408924in}}%
\pgfpathlineto{\pgfqpoint{1.286018in}{1.410011in}}%
\pgfpathlineto{\pgfqpoint{1.287096in}{1.410476in}}%
\pgfpathlineto{\pgfqpoint{1.288621in}{1.411563in}}%
\pgfpathlineto{\pgfqpoint{1.289590in}{1.411966in}}%
\pgfpathlineto{\pgfqpoint{1.292193in}{1.413053in}}%
\pgfpathlineto{\pgfqpoint{1.293304in}{1.413612in}}%
\pgfpathlineto{\pgfqpoint{1.295485in}{1.414698in}}%
\pgfpathlineto{\pgfqpoint{1.296579in}{1.415412in}}%
\pgfpathlineto{\pgfqpoint{1.298127in}{1.416498in}}%
\pgfpathlineto{\pgfqpoint{1.298932in}{1.416778in}}%
\pgfpathlineto{\pgfqpoint{1.299003in}{1.416778in}}%
\pgfpathlineto{\pgfqpoint{1.301426in}{1.417864in}}%
\pgfpathlineto{\pgfqpoint{1.302513in}{1.418702in}}%
\pgfpathlineto{\pgfqpoint{1.304397in}{1.419789in}}%
\pgfpathlineto{\pgfqpoint{1.305499in}{1.420410in}}%
\pgfpathlineto{\pgfqpoint{1.307868in}{1.421496in}}%
\pgfpathlineto{\pgfqpoint{1.308900in}{1.422024in}}%
\pgfpathlineto{\pgfqpoint{1.311355in}{1.423110in}}%
\pgfpathlineto{\pgfqpoint{1.312371in}{1.423949in}}%
\pgfpathlineto{\pgfqpoint{1.315005in}{1.425035in}}%
\pgfpathlineto{\pgfqpoint{1.316115in}{1.425439in}}%
\pgfpathlineto{\pgfqpoint{1.318750in}{1.426525in}}%
\pgfpathlineto{\pgfqpoint{1.319626in}{1.427084in}}%
\pgfpathlineto{\pgfqpoint{1.321955in}{1.428170in}}%
\pgfpathlineto{\pgfqpoint{1.322948in}{1.428729in}}%
\pgfpathlineto{\pgfqpoint{1.324699in}{1.429816in}}%
\pgfpathlineto{\pgfqpoint{1.325614in}{1.430343in}}%
\pgfpathlineto{\pgfqpoint{1.325755in}{1.430343in}}%
\pgfpathlineto{\pgfqpoint{1.328483in}{1.431430in}}%
\pgfpathlineto{\pgfqpoint{1.329593in}{1.432113in}}%
\pgfpathlineto{\pgfqpoint{1.331305in}{1.433199in}}%
\pgfpathlineto{\pgfqpoint{1.332337in}{1.433758in}}%
\pgfpathlineto{\pgfqpoint{1.334518in}{1.434844in}}%
\pgfpathlineto{\pgfqpoint{1.335628in}{1.435279in}}%
\pgfpathlineto{\pgfqpoint{1.337849in}{1.436365in}}%
\pgfpathlineto{\pgfqpoint{1.338896in}{1.436893in}}%
\pgfpathlineto{\pgfqpoint{1.340819in}{1.437980in}}%
\pgfpathlineto{\pgfqpoint{1.341773in}{1.438632in}}%
\pgfpathlineto{\pgfqpoint{1.344220in}{1.439718in}}%
\pgfpathlineto{\pgfqpoint{1.345322in}{1.440184in}}%
\pgfpathlineto{\pgfqpoint{1.347777in}{1.441270in}}%
\pgfpathlineto{\pgfqpoint{1.348746in}{1.441767in}}%
\pgfpathlineto{\pgfqpoint{1.351350in}{1.442853in}}%
\pgfpathlineto{\pgfqpoint{1.352358in}{1.443443in}}%
\pgfpathlineto{\pgfqpoint{1.354273in}{1.444530in}}%
\pgfpathlineto{\pgfqpoint{1.355282in}{1.444964in}}%
\pgfpathlineto{\pgfqpoint{1.358198in}{1.446051in}}%
\pgfpathlineto{\pgfqpoint{1.359261in}{1.446671in}}%
\pgfpathlineto{\pgfqpoint{1.361301in}{1.447758in}}%
\pgfpathlineto{\pgfqpoint{1.362380in}{1.448379in}}%
\pgfpathlineto{\pgfqpoint{1.364944in}{1.449465in}}%
\pgfpathlineto{\pgfqpoint{1.366000in}{1.450148in}}%
\pgfpathlineto{\pgfqpoint{1.368509in}{1.451235in}}%
\pgfpathlineto{\pgfqpoint{1.369580in}{1.451980in}}%
\pgfpathlineto{\pgfqpoint{1.371761in}{1.453066in}}%
\pgfpathlineto{\pgfqpoint{1.372746in}{1.453594in}}%
\pgfpathlineto{\pgfqpoint{1.372848in}{1.453594in}}%
\pgfpathlineto{\pgfqpoint{1.375772in}{1.454680in}}%
\pgfpathlineto{\pgfqpoint{1.376780in}{1.455177in}}%
\pgfpathlineto{\pgfqpoint{1.379071in}{1.456263in}}%
\pgfpathlineto{\pgfqpoint{1.379954in}{1.456636in}}%
\pgfpathlineto{\pgfqpoint{1.380111in}{1.456636in}}%
\pgfpathlineto{\pgfqpoint{1.382839in}{1.457722in}}%
\pgfpathlineto{\pgfqpoint{1.383949in}{1.458126in}}%
\pgfpathlineto{\pgfqpoint{1.386763in}{1.459212in}}%
\pgfpathlineto{\pgfqpoint{1.387874in}{1.459647in}}%
\pgfpathlineto{\pgfqpoint{1.390532in}{1.460734in}}%
\pgfpathlineto{\pgfqpoint{1.391603in}{1.461137in}}%
\pgfpathlineto{\pgfqpoint{1.395339in}{1.462224in}}%
\pgfpathlineto{\pgfqpoint{1.396301in}{1.462813in}}%
\pgfpathlineto{\pgfqpoint{1.396418in}{1.462813in}}%
\pgfpathlineto{\pgfqpoint{1.399694in}{1.463869in}}%
\pgfpathlineto{\pgfqpoint{1.400804in}{1.464365in}}%
\pgfpathlineto{\pgfqpoint{1.402837in}{1.465452in}}%
\pgfpathlineto{\pgfqpoint{1.403931in}{1.465918in}}%
\pgfpathlineto{\pgfqpoint{1.406925in}{1.467004in}}%
\pgfpathlineto{\pgfqpoint{1.407902in}{1.467563in}}%
\pgfpathlineto{\pgfqpoint{1.410920in}{1.468649in}}%
\pgfpathlineto{\pgfqpoint{1.411780in}{1.469053in}}%
\pgfpathlineto{\pgfqpoint{1.414860in}{1.470108in}}%
\pgfpathlineto{\pgfqpoint{1.415908in}{1.470760in}}%
\pgfpathlineto{\pgfqpoint{1.419089in}{1.471847in}}%
\pgfpathlineto{\pgfqpoint{1.420145in}{1.472312in}}%
\pgfpathlineto{\pgfqpoint{1.424210in}{1.473399in}}%
\pgfpathlineto{\pgfqpoint{1.425297in}{1.474020in}}%
\pgfpathlineto{\pgfqpoint{1.429995in}{1.475106in}}%
\pgfpathlineto{\pgfqpoint{1.430941in}{1.475603in}}%
\pgfpathlineto{\pgfqpoint{1.433873in}{1.476689in}}%
\pgfpathlineto{\pgfqpoint{1.434920in}{1.477124in}}%
\pgfpathlineto{\pgfqpoint{1.438071in}{1.478210in}}%
\pgfpathlineto{\pgfqpoint{1.438978in}{1.478614in}}%
\pgfpathlineto{\pgfqpoint{1.441534in}{1.479700in}}%
\pgfpathlineto{\pgfqpoint{1.442605in}{1.480259in}}%
\pgfpathlineto{\pgfqpoint{1.444825in}{1.481345in}}%
\pgfpathlineto{\pgfqpoint{1.445935in}{1.481780in}}%
\pgfpathlineto{\pgfqpoint{1.449828in}{1.482867in}}%
\pgfpathlineto{\pgfqpoint{1.450696in}{1.483270in}}%
\pgfpathlineto{\pgfqpoint{1.450853in}{1.483270in}}%
\pgfpathlineto{\pgfqpoint{1.454707in}{1.484357in}}%
\pgfpathlineto{\pgfqpoint{1.455801in}{1.484853in}}%
\pgfpathlineto{\pgfqpoint{1.458623in}{1.485940in}}%
\pgfpathlineto{\pgfqpoint{1.459686in}{1.486530in}}%
\pgfpathlineto{\pgfqpoint{1.464189in}{1.487616in}}%
\pgfpathlineto{\pgfqpoint{1.465253in}{1.488144in}}%
\pgfpathlineto{\pgfqpoint{1.467559in}{1.489230in}}%
\pgfpathlineto{\pgfqpoint{1.468614in}{1.489572in}}%
\pgfpathlineto{\pgfqpoint{1.471053in}{1.490658in}}%
\pgfpathlineto{\pgfqpoint{1.472163in}{1.491062in}}%
\pgfpathlineto{\pgfqpoint{1.474743in}{1.492148in}}%
\pgfpathlineto{\pgfqpoint{1.475846in}{1.492552in}}%
\pgfpathlineto{\pgfqpoint{1.478926in}{1.493638in}}%
\pgfpathlineto{\pgfqpoint{1.479919in}{1.494042in}}%
\pgfpathlineto{\pgfqpoint{1.484429in}{1.495128in}}%
\pgfpathlineto{\pgfqpoint{1.485508in}{1.495563in}}%
\pgfpathlineto{\pgfqpoint{1.488150in}{1.496649in}}%
\pgfpathlineto{\pgfqpoint{1.489073in}{1.496991in}}%
\pgfpathlineto{\pgfqpoint{1.492661in}{1.498077in}}%
\pgfpathlineto{\pgfqpoint{1.493771in}{1.498450in}}%
\pgfpathlineto{\pgfqpoint{1.496461in}{1.499536in}}%
\pgfpathlineto{\pgfqpoint{1.497563in}{1.499878in}}%
\pgfpathlineto{\pgfqpoint{1.500182in}{1.500902in}}%
\pgfpathlineto{\pgfqpoint{1.501245in}{1.501430in}}%
\pgfpathlineto{\pgfqpoint{1.504341in}{1.502516in}}%
\pgfpathlineto{\pgfqpoint{1.505435in}{1.502982in}}%
\pgfpathlineto{\pgfqpoint{1.508375in}{1.504068in}}%
\pgfpathlineto{\pgfqpoint{1.509422in}{1.504379in}}%
\pgfpathlineto{\pgfqpoint{1.513120in}{1.505465in}}%
\pgfpathlineto{\pgfqpoint{1.514058in}{1.505807in}}%
\pgfpathlineto{\pgfqpoint{1.517068in}{1.506893in}}%
\pgfpathlineto{\pgfqpoint{1.517873in}{1.507297in}}%
\pgfpathlineto{\pgfqpoint{1.517889in}{1.507297in}}%
\pgfpathlineto{\pgfqpoint{1.522243in}{1.508383in}}%
\pgfpathlineto{\pgfqpoint{1.523275in}{1.508663in}}%
\pgfpathlineto{\pgfqpoint{1.525980in}{1.509749in}}%
\pgfpathlineto{\pgfqpoint{1.527020in}{1.510215in}}%
\pgfpathlineto{\pgfqpoint{1.530381in}{1.511301in}}%
\pgfpathlineto{\pgfqpoint{1.531484in}{1.511705in}}%
\pgfpathlineto{\pgfqpoint{1.534267in}{1.512791in}}%
\pgfpathlineto{\pgfqpoint{1.535283in}{1.513102in}}%
\pgfpathlineto{\pgfqpoint{1.539129in}{1.514188in}}%
\pgfpathlineto{\pgfqpoint{1.540224in}{1.514809in}}%
\pgfpathlineto{\pgfqpoint{1.543312in}{1.515895in}}%
\pgfpathlineto{\pgfqpoint{1.543851in}{1.516361in}}%
\pgfpathlineto{\pgfqpoint{1.547799in}{1.517447in}}%
\pgfpathlineto{\pgfqpoint{1.548909in}{1.517913in}}%
\pgfpathlineto{\pgfqpoint{1.552841in}{1.519000in}}%
\pgfpathlineto{\pgfqpoint{1.553920in}{1.519372in}}%
\pgfpathlineto{\pgfqpoint{1.557556in}{1.520459in}}%
\pgfpathlineto{\pgfqpoint{1.558666in}{1.521142in}}%
\pgfpathlineto{\pgfqpoint{1.562043in}{1.522228in}}%
\pgfpathlineto{\pgfqpoint{1.563137in}{1.522632in}}%
\pgfpathlineto{\pgfqpoint{1.567453in}{1.523718in}}%
\pgfpathlineto{\pgfqpoint{1.568508in}{1.524215in}}%
\pgfpathlineto{\pgfqpoint{1.572855in}{1.525301in}}%
\pgfpathlineto{\pgfqpoint{1.573824in}{1.525581in}}%
\pgfpathlineto{\pgfqpoint{1.577209in}{1.526667in}}%
\pgfpathlineto{\pgfqpoint{1.578178in}{1.527102in}}%
\pgfpathlineto{\pgfqpoint{1.585230in}{1.528188in}}%
\pgfpathlineto{\pgfqpoint{1.586278in}{1.528467in}}%
\pgfpathlineto{\pgfqpoint{1.589553in}{1.529554in}}%
\pgfpathlineto{\pgfqpoint{1.590616in}{1.529957in}}%
\pgfpathlineto{\pgfqpoint{1.593978in}{1.531044in}}%
\pgfpathlineto{\pgfqpoint{1.595057in}{1.531261in}}%
\pgfpathlineto{\pgfqpoint{1.597762in}{1.532348in}}%
\pgfpathlineto{\pgfqpoint{1.598856in}{1.532875in}}%
\pgfpathlineto{\pgfqpoint{1.603117in}{1.533962in}}%
\pgfpathlineto{\pgfqpoint{1.604149in}{1.534365in}}%
\pgfpathlineto{\pgfqpoint{1.606775in}{1.535452in}}%
\pgfpathlineto{\pgfqpoint{1.607604in}{1.535700in}}%
\pgfpathlineto{\pgfqpoint{1.607792in}{1.535700in}}%
\pgfpathlineto{\pgfqpoint{1.611544in}{1.536787in}}%
\pgfpathlineto{\pgfqpoint{1.612654in}{1.537004in}}%
\pgfpathlineto{\pgfqpoint{1.615789in}{1.538091in}}%
\pgfpathlineto{\pgfqpoint{1.616837in}{1.538184in}}%
\pgfpathlineto{\pgfqpoint{1.620886in}{1.539270in}}%
\pgfpathlineto{\pgfqpoint{1.621762in}{1.539612in}}%
\pgfpathlineto{\pgfqpoint{1.621903in}{1.539612in}}%
\pgfpathlineto{\pgfqpoint{1.626421in}{1.540698in}}%
\pgfpathlineto{\pgfqpoint{1.627516in}{1.541102in}}%
\pgfpathlineto{\pgfqpoint{1.632519in}{1.542157in}}%
\pgfpathlineto{\pgfqpoint{1.633598in}{1.542623in}}%
\pgfpathlineto{\pgfqpoint{1.638257in}{1.543709in}}%
\pgfpathlineto{\pgfqpoint{1.639352in}{1.543895in}}%
\pgfpathlineto{\pgfqpoint{1.643393in}{1.544982in}}%
\pgfpathlineto{\pgfqpoint{1.644355in}{1.545385in}}%
\pgfpathlineto{\pgfqpoint{1.647974in}{1.546472in}}%
\pgfpathlineto{\pgfqpoint{1.649069in}{1.546751in}}%
\pgfpathlineto{\pgfqpoint{1.654408in}{1.547838in}}%
\pgfpathlineto{\pgfqpoint{1.655378in}{1.548179in}}%
\pgfpathlineto{\pgfqpoint{1.660162in}{1.549266in}}%
\pgfpathlineto{\pgfqpoint{1.661272in}{1.549576in}}%
\pgfpathlineto{\pgfqpoint{1.665760in}{1.550663in}}%
\pgfpathlineto{\pgfqpoint{1.666784in}{1.550880in}}%
\pgfpathlineto{\pgfqpoint{1.671435in}{1.551966in}}%
\pgfpathlineto{\pgfqpoint{1.672475in}{1.552339in}}%
\pgfpathlineto{\pgfqpoint{1.677040in}{1.553425in}}%
\pgfpathlineto{\pgfqpoint{1.678064in}{1.553767in}}%
\pgfpathlineto{\pgfqpoint{1.682575in}{1.554853in}}%
\pgfpathlineto{\pgfqpoint{1.683443in}{1.555071in}}%
\pgfpathlineto{\pgfqpoint{1.689048in}{1.556157in}}%
\pgfpathlineto{\pgfqpoint{1.690151in}{1.556623in}}%
\pgfpathlineto{\pgfqpoint{1.695318in}{1.557709in}}%
\pgfpathlineto{\pgfqpoint{1.696233in}{1.558020in}}%
\pgfpathlineto{\pgfqpoint{1.696358in}{1.558020in}}%
\pgfpathlineto{\pgfqpoint{1.701775in}{1.559106in}}%
\pgfpathlineto{\pgfqpoint{1.702588in}{1.559385in}}%
\pgfpathlineto{\pgfqpoint{1.708733in}{1.560472in}}%
\pgfpathlineto{\pgfqpoint{1.709835in}{1.560751in}}%
\pgfpathlineto{\pgfqpoint{1.714909in}{1.561838in}}%
\pgfpathlineto{\pgfqpoint{1.715941in}{1.562024in}}%
\pgfpathlineto{\pgfqpoint{1.716003in}{1.562024in}}%
\pgfpathlineto{\pgfqpoint{1.721859in}{1.563110in}}%
\pgfpathlineto{\pgfqpoint{1.722836in}{1.563328in}}%
\pgfpathlineto{\pgfqpoint{1.728676in}{1.564414in}}%
\pgfpathlineto{\pgfqpoint{1.729708in}{1.564725in}}%
\pgfpathlineto{\pgfqpoint{1.734805in}{1.565811in}}%
\pgfpathlineto{\pgfqpoint{1.735899in}{1.566184in}}%
\pgfpathlineto{\pgfqpoint{1.739918in}{1.567270in}}%
\pgfpathlineto{\pgfqpoint{1.740989in}{1.567581in}}%
\pgfpathlineto{\pgfqpoint{1.747149in}{1.568667in}}%
\pgfpathlineto{\pgfqpoint{1.748251in}{1.568915in}}%
\pgfpathlineto{\pgfqpoint{1.754990in}{1.570002in}}%
\pgfpathlineto{\pgfqpoint{1.755959in}{1.570281in}}%
\pgfpathlineto{\pgfqpoint{1.756077in}{1.570281in}}%
\pgfpathlineto{\pgfqpoint{1.762292in}{1.571368in}}%
\pgfpathlineto{\pgfqpoint{1.763363in}{1.571616in}}%
\pgfpathlineto{\pgfqpoint{1.771907in}{1.572671in}}%
\pgfpathlineto{\pgfqpoint{1.772900in}{1.572951in}}%
\pgfpathlineto{\pgfqpoint{1.778928in}{1.574037in}}%
\pgfpathlineto{\pgfqpoint{1.779928in}{1.574224in}}%
\pgfpathlineto{\pgfqpoint{1.786636in}{1.575310in}}%
\pgfpathlineto{\pgfqpoint{1.787715in}{1.575496in}}%
\pgfpathlineto{\pgfqpoint{1.794891in}{1.576583in}}%
\pgfpathlineto{\pgfqpoint{1.795743in}{1.576831in}}%
\pgfpathlineto{\pgfqpoint{1.802772in}{1.577918in}}%
\pgfpathlineto{\pgfqpoint{1.803811in}{1.578228in}}%
\pgfpathlineto{\pgfqpoint{1.810558in}{1.579314in}}%
\pgfpathlineto{\pgfqpoint{1.811473in}{1.579532in}}%
\pgfpathlineto{\pgfqpoint{1.821354in}{1.580618in}}%
\pgfpathlineto{\pgfqpoint{1.822292in}{1.580773in}}%
\pgfpathlineto{\pgfqpoint{1.822370in}{1.580773in}}%
\pgfpathlineto{\pgfqpoint{1.829477in}{1.581860in}}%
\pgfpathlineto{\pgfqpoint{1.830587in}{1.582263in}}%
\pgfpathlineto{\pgfqpoint{1.838467in}{1.583350in}}%
\pgfpathlineto{\pgfqpoint{1.839530in}{1.583443in}}%
\pgfpathlineto{\pgfqpoint{1.844572in}{1.584530in}}%
\pgfpathlineto{\pgfqpoint{1.844572in}{1.584561in}}%
\pgfpathlineto{\pgfqpoint{1.854790in}{1.585647in}}%
\pgfpathlineto{\pgfqpoint{1.855900in}{1.585895in}}%
\pgfpathlineto{\pgfqpoint{1.863679in}{1.586982in}}%
\pgfpathlineto{\pgfqpoint{1.864703in}{1.587168in}}%
\pgfpathlineto{\pgfqpoint{1.864781in}{1.587168in}}%
\pgfpathlineto{\pgfqpoint{1.872239in}{1.588255in}}%
\pgfpathlineto{\pgfqpoint{1.873310in}{1.588379in}}%
\pgfpathlineto{\pgfqpoint{1.882457in}{1.589465in}}%
\pgfpathlineto{\pgfqpoint{1.882832in}{1.589558in}}%
\pgfpathlineto{\pgfqpoint{1.890517in}{1.590645in}}%
\pgfpathlineto{\pgfqpoint{1.891400in}{1.590800in}}%
\pgfpathlineto{\pgfqpoint{1.891486in}{1.590800in}}%
\pgfpathlineto{\pgfqpoint{1.901657in}{1.591887in}}%
\pgfpathlineto{\pgfqpoint{1.902525in}{1.592011in}}%
\pgfpathlineto{\pgfqpoint{1.911726in}{1.593097in}}%
\pgfpathlineto{\pgfqpoint{1.912492in}{1.593221in}}%
\pgfpathlineto{\pgfqpoint{1.918144in}{1.594308in}}%
\pgfpathlineto{\pgfqpoint{1.919207in}{1.594463in}}%
\pgfpathlineto{\pgfqpoint{1.931997in}{1.595549in}}%
\pgfpathlineto{\pgfqpoint{1.933045in}{1.595767in}}%
\pgfpathlineto{\pgfqpoint{1.946819in}{1.596853in}}%
\pgfpathlineto{\pgfqpoint{1.947898in}{1.596977in}}%
\pgfpathlineto{\pgfqpoint{1.961954in}{1.598064in}}%
\pgfpathlineto{\pgfqpoint{1.961954in}{1.598095in}}%
\pgfpathlineto{\pgfqpoint{1.962165in}{1.598095in}}%
\pgfpathlineto{\pgfqpoint{1.982030in}{1.599181in}}%
\pgfpathlineto{\pgfqpoint{1.982585in}{1.599337in}}%
\pgfpathlineto{\pgfqpoint{1.983007in}{1.599337in}}%
\pgfpathlineto{\pgfqpoint{1.997087in}{1.600423in}}%
\pgfpathlineto{\pgfqpoint{1.997822in}{1.600547in}}%
\pgfpathlineto{\pgfqpoint{1.997931in}{1.600547in}}%
\pgfpathlineto{\pgfqpoint{2.023432in}{1.601634in}}%
\pgfpathlineto{\pgfqpoint{2.023432in}{1.601665in}}%
\pgfpathlineto{\pgfqpoint{2.024019in}{1.601665in}}%
\pgfpathlineto{\pgfqpoint{2.033126in}{1.601944in}}%
\pgfpathlineto{\pgfqpoint{2.033126in}{1.601944in}}%
\pgfusepath{stroke}%
\end{pgfscope}%
\begin{pgfscope}%
\pgfsetrectcap%
\pgfsetmiterjoin%
\pgfsetlinewidth{0.803000pt}%
\definecolor{currentstroke}{rgb}{0.000000,0.000000,0.000000}%
\pgfsetstrokecolor{currentstroke}%
\pgfsetdash{}{0pt}%
\pgfpathmoveto{\pgfqpoint{0.553581in}{0.499444in}}%
\pgfpathlineto{\pgfqpoint{0.553581in}{1.654444in}}%
\pgfusepath{stroke}%
\end{pgfscope}%
\begin{pgfscope}%
\pgfsetrectcap%
\pgfsetmiterjoin%
\pgfsetlinewidth{0.803000pt}%
\definecolor{currentstroke}{rgb}{0.000000,0.000000,0.000000}%
\pgfsetstrokecolor{currentstroke}%
\pgfsetdash{}{0pt}%
\pgfpathmoveto{\pgfqpoint{2.103581in}{0.499444in}}%
\pgfpathlineto{\pgfqpoint{2.103581in}{1.654444in}}%
\pgfusepath{stroke}%
\end{pgfscope}%
\begin{pgfscope}%
\pgfsetrectcap%
\pgfsetmiterjoin%
\pgfsetlinewidth{0.803000pt}%
\definecolor{currentstroke}{rgb}{0.000000,0.000000,0.000000}%
\pgfsetstrokecolor{currentstroke}%
\pgfsetdash{}{0pt}%
\pgfpathmoveto{\pgfqpoint{0.553581in}{0.499444in}}%
\pgfpathlineto{\pgfqpoint{2.103581in}{0.499444in}}%
\pgfusepath{stroke}%
\end{pgfscope}%
\begin{pgfscope}%
\pgfsetrectcap%
\pgfsetmiterjoin%
\pgfsetlinewidth{0.803000pt}%
\definecolor{currentstroke}{rgb}{0.000000,0.000000,0.000000}%
\pgfsetstrokecolor{currentstroke}%
\pgfsetdash{}{0pt}%
\pgfpathmoveto{\pgfqpoint{0.553581in}{1.654444in}}%
\pgfpathlineto{\pgfqpoint{2.103581in}{1.654444in}}%
\pgfusepath{stroke}%
\end{pgfscope}%
\begin{pgfscope}%
\pgfsetbuttcap%
\pgfsetmiterjoin%
\definecolor{currentfill}{rgb}{1.000000,1.000000,1.000000}%
\pgfsetfillcolor{currentfill}%
\pgfsetfillopacity{0.800000}%
\pgfsetlinewidth{1.003750pt}%
\definecolor{currentstroke}{rgb}{0.800000,0.800000,0.800000}%
\pgfsetstrokecolor{currentstroke}%
\pgfsetstrokeopacity{0.800000}%
\pgfsetdash{}{0pt}%
\pgfpathmoveto{\pgfqpoint{0.832747in}{0.568889in}}%
\pgfpathlineto{\pgfqpoint{2.006358in}{0.568889in}}%
\pgfpathquadraticcurveto{\pgfqpoint{2.034136in}{0.568889in}}{\pgfqpoint{2.034136in}{0.596666in}}%
\pgfpathlineto{\pgfqpoint{2.034136in}{0.776388in}}%
\pgfpathquadraticcurveto{\pgfqpoint{2.034136in}{0.804166in}}{\pgfqpoint{2.006358in}{0.804166in}}%
\pgfpathlineto{\pgfqpoint{0.832747in}{0.804166in}}%
\pgfpathquadraticcurveto{\pgfqpoint{0.804970in}{0.804166in}}{\pgfqpoint{0.804970in}{0.776388in}}%
\pgfpathlineto{\pgfqpoint{0.804970in}{0.596666in}}%
\pgfpathquadraticcurveto{\pgfqpoint{0.804970in}{0.568889in}}{\pgfqpoint{0.832747in}{0.568889in}}%
\pgfpathlineto{\pgfqpoint{0.832747in}{0.568889in}}%
\pgfpathclose%
\pgfusepath{stroke,fill}%
\end{pgfscope}%
\begin{pgfscope}%
\pgfsetrectcap%
\pgfsetroundjoin%
\pgfsetlinewidth{1.505625pt}%
\definecolor{currentstroke}{rgb}{0.000000,0.000000,0.000000}%
\pgfsetstrokecolor{currentstroke}%
\pgfsetdash{}{0pt}%
\pgfpathmoveto{\pgfqpoint{0.860525in}{0.700000in}}%
\pgfpathlineto{\pgfqpoint{0.999414in}{0.700000in}}%
\pgfpathlineto{\pgfqpoint{1.138303in}{0.700000in}}%
\pgfusepath{stroke}%
\end{pgfscope}%
\begin{pgfscope}%
\definecolor{textcolor}{rgb}{0.000000,0.000000,0.000000}%
\pgfsetstrokecolor{textcolor}%
\pgfsetfillcolor{textcolor}%
\pgftext[x=1.249414in,y=0.651388in,left,base]{\color{textcolor}\rmfamily\fontsize{10.000000}{12.000000}\selectfont AUC=0.754}%
\end{pgfscope}%
\end{pgfpicture}%
\makeatother%
\endgroup%

\end{tabular}


\verb|OBFC_Hard_Tomek_0_alpha_balanced_gamma_0_0_5_gamma_1_2_0_v1|

\noindent\begin{tabular}{@{\hspace{-6pt}}p{4.3in} @{\hspace{-6pt}}p{2.0in}}
	\vskip 0pt
	\hfil Raw Model Output
	
	%% Creator: Matplotlib, PGF backend
%%
%% To include the figure in your LaTeX document, write
%%   \input{<filename>.pgf}
%%
%% Make sure the required packages are loaded in your preamble
%%   \usepackage{pgf}
%%
%% Also ensure that all the required font packages are loaded; for instance,
%% the lmodern package is sometimes necessary when using math font.
%%   \usepackage{lmodern}
%%
%% Figures using additional raster images can only be included by \input if
%% they are in the same directory as the main LaTeX file. For loading figures
%% from other directories you can use the `import` package
%%   \usepackage{import}
%%
%% and then include the figures with
%%   \import{<path to file>}{<filename>.pgf}
%%
%% Matplotlib used the following preamble
%%   
%%   \usepackage{fontspec}
%%   \makeatletter\@ifpackageloaded{underscore}{}{\usepackage[strings]{underscore}}\makeatother
%%
\begingroup%
\makeatletter%
\begin{pgfpicture}%
\pgfpathrectangle{\pgfpointorigin}{\pgfqpoint{4.229831in}{1.754444in}}%
\pgfusepath{use as bounding box, clip}%
\begin{pgfscope}%
\pgfsetbuttcap%
\pgfsetmiterjoin%
\definecolor{currentfill}{rgb}{1.000000,1.000000,1.000000}%
\pgfsetfillcolor{currentfill}%
\pgfsetlinewidth{0.000000pt}%
\definecolor{currentstroke}{rgb}{1.000000,1.000000,1.000000}%
\pgfsetstrokecolor{currentstroke}%
\pgfsetdash{}{0pt}%
\pgfpathmoveto{\pgfqpoint{0.000000in}{0.000000in}}%
\pgfpathlineto{\pgfqpoint{4.229831in}{0.000000in}}%
\pgfpathlineto{\pgfqpoint{4.229831in}{1.754444in}}%
\pgfpathlineto{\pgfqpoint{0.000000in}{1.754444in}}%
\pgfpathlineto{\pgfqpoint{0.000000in}{0.000000in}}%
\pgfpathclose%
\pgfusepath{fill}%
\end{pgfscope}%
\begin{pgfscope}%
\pgfsetbuttcap%
\pgfsetmiterjoin%
\definecolor{currentfill}{rgb}{1.000000,1.000000,1.000000}%
\pgfsetfillcolor{currentfill}%
\pgfsetlinewidth{0.000000pt}%
\definecolor{currentstroke}{rgb}{0.000000,0.000000,0.000000}%
\pgfsetstrokecolor{currentstroke}%
\pgfsetstrokeopacity{0.000000}%
\pgfsetdash{}{0pt}%
\pgfpathmoveto{\pgfqpoint{0.553581in}{0.499444in}}%
\pgfpathlineto{\pgfqpoint{4.041081in}{0.499444in}}%
\pgfpathlineto{\pgfqpoint{4.041081in}{1.654444in}}%
\pgfpathlineto{\pgfqpoint{0.553581in}{1.654444in}}%
\pgfpathlineto{\pgfqpoint{0.553581in}{0.499444in}}%
\pgfpathclose%
\pgfusepath{fill}%
\end{pgfscope}%
\begin{pgfscope}%
\pgfpathrectangle{\pgfqpoint{0.553581in}{0.499444in}}{\pgfqpoint{3.487500in}{1.155000in}}%
\pgfusepath{clip}%
\pgfsetbuttcap%
\pgfsetmiterjoin%
\pgfsetlinewidth{1.003750pt}%
\definecolor{currentstroke}{rgb}{0.000000,0.000000,0.000000}%
\pgfsetstrokecolor{currentstroke}%
\pgfsetdash{}{0pt}%
\pgfpathmoveto{\pgfqpoint{0.543581in}{0.499444in}}%
\pgfpathlineto{\pgfqpoint{0.588110in}{0.499444in}}%
\pgfpathlineto{\pgfqpoint{0.588110in}{0.577033in}}%
\pgfpathlineto{\pgfqpoint{0.543581in}{0.577033in}}%
\pgfusepath{stroke}%
\end{pgfscope}%
\begin{pgfscope}%
\pgfpathrectangle{\pgfqpoint{0.553581in}{0.499444in}}{\pgfqpoint{3.487500in}{1.155000in}}%
\pgfusepath{clip}%
\pgfsetbuttcap%
\pgfsetmiterjoin%
\pgfsetlinewidth{1.003750pt}%
\definecolor{currentstroke}{rgb}{0.000000,0.000000,0.000000}%
\pgfsetstrokecolor{currentstroke}%
\pgfsetdash{}{0pt}%
\pgfpathmoveto{\pgfqpoint{0.670982in}{0.499444in}}%
\pgfpathlineto{\pgfqpoint{0.726229in}{0.499444in}}%
\pgfpathlineto{\pgfqpoint{0.726229in}{0.724731in}}%
\pgfpathlineto{\pgfqpoint{0.670982in}{0.724731in}}%
\pgfpathlineto{\pgfqpoint{0.670982in}{0.499444in}}%
\pgfpathclose%
\pgfusepath{stroke}%
\end{pgfscope}%
\begin{pgfscope}%
\pgfpathrectangle{\pgfqpoint{0.553581in}{0.499444in}}{\pgfqpoint{3.487500in}{1.155000in}}%
\pgfusepath{clip}%
\pgfsetbuttcap%
\pgfsetmiterjoin%
\pgfsetlinewidth{1.003750pt}%
\definecolor{currentstroke}{rgb}{0.000000,0.000000,0.000000}%
\pgfsetstrokecolor{currentstroke}%
\pgfsetdash{}{0pt}%
\pgfpathmoveto{\pgfqpoint{0.809100in}{0.499444in}}%
\pgfpathlineto{\pgfqpoint{0.864348in}{0.499444in}}%
\pgfpathlineto{\pgfqpoint{0.864348in}{0.913125in}}%
\pgfpathlineto{\pgfqpoint{0.809100in}{0.913125in}}%
\pgfpathlineto{\pgfqpoint{0.809100in}{0.499444in}}%
\pgfpathclose%
\pgfusepath{stroke}%
\end{pgfscope}%
\begin{pgfscope}%
\pgfpathrectangle{\pgfqpoint{0.553581in}{0.499444in}}{\pgfqpoint{3.487500in}{1.155000in}}%
\pgfusepath{clip}%
\pgfsetbuttcap%
\pgfsetmiterjoin%
\pgfsetlinewidth{1.003750pt}%
\definecolor{currentstroke}{rgb}{0.000000,0.000000,0.000000}%
\pgfsetstrokecolor{currentstroke}%
\pgfsetdash{}{0pt}%
\pgfpathmoveto{\pgfqpoint{0.947219in}{0.499444in}}%
\pgfpathlineto{\pgfqpoint{1.002467in}{0.499444in}}%
\pgfpathlineto{\pgfqpoint{1.002467in}{1.088460in}}%
\pgfpathlineto{\pgfqpoint{0.947219in}{1.088460in}}%
\pgfpathlineto{\pgfqpoint{0.947219in}{0.499444in}}%
\pgfpathclose%
\pgfusepath{stroke}%
\end{pgfscope}%
\begin{pgfscope}%
\pgfpathrectangle{\pgfqpoint{0.553581in}{0.499444in}}{\pgfqpoint{3.487500in}{1.155000in}}%
\pgfusepath{clip}%
\pgfsetbuttcap%
\pgfsetmiterjoin%
\pgfsetlinewidth{1.003750pt}%
\definecolor{currentstroke}{rgb}{0.000000,0.000000,0.000000}%
\pgfsetstrokecolor{currentstroke}%
\pgfsetdash{}{0pt}%
\pgfpathmoveto{\pgfqpoint{1.085338in}{0.499444in}}%
\pgfpathlineto{\pgfqpoint{1.140586in}{0.499444in}}%
\pgfpathlineto{\pgfqpoint{1.140586in}{1.232608in}}%
\pgfpathlineto{\pgfqpoint{1.085338in}{1.232608in}}%
\pgfpathlineto{\pgfqpoint{1.085338in}{0.499444in}}%
\pgfpathclose%
\pgfusepath{stroke}%
\end{pgfscope}%
\begin{pgfscope}%
\pgfpathrectangle{\pgfqpoint{0.553581in}{0.499444in}}{\pgfqpoint{3.487500in}{1.155000in}}%
\pgfusepath{clip}%
\pgfsetbuttcap%
\pgfsetmiterjoin%
\pgfsetlinewidth{1.003750pt}%
\definecolor{currentstroke}{rgb}{0.000000,0.000000,0.000000}%
\pgfsetstrokecolor{currentstroke}%
\pgfsetdash{}{0pt}%
\pgfpathmoveto{\pgfqpoint{1.223457in}{0.499444in}}%
\pgfpathlineto{\pgfqpoint{1.278704in}{0.499444in}}%
\pgfpathlineto{\pgfqpoint{1.278704in}{1.354507in}}%
\pgfpathlineto{\pgfqpoint{1.223457in}{1.354507in}}%
\pgfpathlineto{\pgfqpoint{1.223457in}{0.499444in}}%
\pgfpathclose%
\pgfusepath{stroke}%
\end{pgfscope}%
\begin{pgfscope}%
\pgfpathrectangle{\pgfqpoint{0.553581in}{0.499444in}}{\pgfqpoint{3.487500in}{1.155000in}}%
\pgfusepath{clip}%
\pgfsetbuttcap%
\pgfsetmiterjoin%
\pgfsetlinewidth{1.003750pt}%
\definecolor{currentstroke}{rgb}{0.000000,0.000000,0.000000}%
\pgfsetstrokecolor{currentstroke}%
\pgfsetdash{}{0pt}%
\pgfpathmoveto{\pgfqpoint{1.361576in}{0.499444in}}%
\pgfpathlineto{\pgfqpoint{1.416823in}{0.499444in}}%
\pgfpathlineto{\pgfqpoint{1.416823in}{1.458022in}}%
\pgfpathlineto{\pgfqpoint{1.361576in}{1.458022in}}%
\pgfpathlineto{\pgfqpoint{1.361576in}{0.499444in}}%
\pgfpathclose%
\pgfusepath{stroke}%
\end{pgfscope}%
\begin{pgfscope}%
\pgfpathrectangle{\pgfqpoint{0.553581in}{0.499444in}}{\pgfqpoint{3.487500in}{1.155000in}}%
\pgfusepath{clip}%
\pgfsetbuttcap%
\pgfsetmiterjoin%
\pgfsetlinewidth{1.003750pt}%
\definecolor{currentstroke}{rgb}{0.000000,0.000000,0.000000}%
\pgfsetstrokecolor{currentstroke}%
\pgfsetdash{}{0pt}%
\pgfpathmoveto{\pgfqpoint{1.499695in}{0.499444in}}%
\pgfpathlineto{\pgfqpoint{1.554942in}{0.499444in}}%
\pgfpathlineto{\pgfqpoint{1.554942in}{1.549620in}}%
\pgfpathlineto{\pgfqpoint{1.499695in}{1.549620in}}%
\pgfpathlineto{\pgfqpoint{1.499695in}{0.499444in}}%
\pgfpathclose%
\pgfusepath{stroke}%
\end{pgfscope}%
\begin{pgfscope}%
\pgfpathrectangle{\pgfqpoint{0.553581in}{0.499444in}}{\pgfqpoint{3.487500in}{1.155000in}}%
\pgfusepath{clip}%
\pgfsetbuttcap%
\pgfsetmiterjoin%
\pgfsetlinewidth{1.003750pt}%
\definecolor{currentstroke}{rgb}{0.000000,0.000000,0.000000}%
\pgfsetstrokecolor{currentstroke}%
\pgfsetdash{}{0pt}%
\pgfpathmoveto{\pgfqpoint{1.637813in}{0.499444in}}%
\pgfpathlineto{\pgfqpoint{1.693061in}{0.499444in}}%
\pgfpathlineto{\pgfqpoint{1.693061in}{1.599444in}}%
\pgfpathlineto{\pgfqpoint{1.637813in}{1.599444in}}%
\pgfpathlineto{\pgfqpoint{1.637813in}{0.499444in}}%
\pgfpathclose%
\pgfusepath{stroke}%
\end{pgfscope}%
\begin{pgfscope}%
\pgfpathrectangle{\pgfqpoint{0.553581in}{0.499444in}}{\pgfqpoint{3.487500in}{1.155000in}}%
\pgfusepath{clip}%
\pgfsetbuttcap%
\pgfsetmiterjoin%
\pgfsetlinewidth{1.003750pt}%
\definecolor{currentstroke}{rgb}{0.000000,0.000000,0.000000}%
\pgfsetstrokecolor{currentstroke}%
\pgfsetdash{}{0pt}%
\pgfpathmoveto{\pgfqpoint{1.775932in}{0.499444in}}%
\pgfpathlineto{\pgfqpoint{1.831180in}{0.499444in}}%
\pgfpathlineto{\pgfqpoint{1.831180in}{1.564770in}}%
\pgfpathlineto{\pgfqpoint{1.775932in}{1.564770in}}%
\pgfpathlineto{\pgfqpoint{1.775932in}{0.499444in}}%
\pgfpathclose%
\pgfusepath{stroke}%
\end{pgfscope}%
\begin{pgfscope}%
\pgfpathrectangle{\pgfqpoint{0.553581in}{0.499444in}}{\pgfqpoint{3.487500in}{1.155000in}}%
\pgfusepath{clip}%
\pgfsetbuttcap%
\pgfsetmiterjoin%
\pgfsetlinewidth{1.003750pt}%
\definecolor{currentstroke}{rgb}{0.000000,0.000000,0.000000}%
\pgfsetstrokecolor{currentstroke}%
\pgfsetdash{}{0pt}%
\pgfpathmoveto{\pgfqpoint{1.914051in}{0.499444in}}%
\pgfpathlineto{\pgfqpoint{1.969299in}{0.499444in}}%
\pgfpathlineto{\pgfqpoint{1.969299in}{1.516340in}}%
\pgfpathlineto{\pgfqpoint{1.914051in}{1.516340in}}%
\pgfpathlineto{\pgfqpoint{1.914051in}{0.499444in}}%
\pgfpathclose%
\pgfusepath{stroke}%
\end{pgfscope}%
\begin{pgfscope}%
\pgfpathrectangle{\pgfqpoint{0.553581in}{0.499444in}}{\pgfqpoint{3.487500in}{1.155000in}}%
\pgfusepath{clip}%
\pgfsetbuttcap%
\pgfsetmiterjoin%
\pgfsetlinewidth{1.003750pt}%
\definecolor{currentstroke}{rgb}{0.000000,0.000000,0.000000}%
\pgfsetstrokecolor{currentstroke}%
\pgfsetdash{}{0pt}%
\pgfpathmoveto{\pgfqpoint{2.052170in}{0.499444in}}%
\pgfpathlineto{\pgfqpoint{2.107417in}{0.499444in}}%
\pgfpathlineto{\pgfqpoint{2.107417in}{1.410923in}}%
\pgfpathlineto{\pgfqpoint{2.052170in}{1.410923in}}%
\pgfpathlineto{\pgfqpoint{2.052170in}{0.499444in}}%
\pgfpathclose%
\pgfusepath{stroke}%
\end{pgfscope}%
\begin{pgfscope}%
\pgfpathrectangle{\pgfqpoint{0.553581in}{0.499444in}}{\pgfqpoint{3.487500in}{1.155000in}}%
\pgfusepath{clip}%
\pgfsetbuttcap%
\pgfsetmiterjoin%
\pgfsetlinewidth{1.003750pt}%
\definecolor{currentstroke}{rgb}{0.000000,0.000000,0.000000}%
\pgfsetstrokecolor{currentstroke}%
\pgfsetdash{}{0pt}%
\pgfpathmoveto{\pgfqpoint{2.190289in}{0.499444in}}%
\pgfpathlineto{\pgfqpoint{2.245536in}{0.499444in}}%
\pgfpathlineto{\pgfqpoint{2.245536in}{1.275713in}}%
\pgfpathlineto{\pgfqpoint{2.190289in}{1.275713in}}%
\pgfpathlineto{\pgfqpoint{2.190289in}{0.499444in}}%
\pgfpathclose%
\pgfusepath{stroke}%
\end{pgfscope}%
\begin{pgfscope}%
\pgfpathrectangle{\pgfqpoint{0.553581in}{0.499444in}}{\pgfqpoint{3.487500in}{1.155000in}}%
\pgfusepath{clip}%
\pgfsetbuttcap%
\pgfsetmiterjoin%
\pgfsetlinewidth{1.003750pt}%
\definecolor{currentstroke}{rgb}{0.000000,0.000000,0.000000}%
\pgfsetstrokecolor{currentstroke}%
\pgfsetdash{}{0pt}%
\pgfpathmoveto{\pgfqpoint{2.328407in}{0.499444in}}%
\pgfpathlineto{\pgfqpoint{2.383655in}{0.499444in}}%
\pgfpathlineto{\pgfqpoint{2.383655in}{1.112612in}}%
\pgfpathlineto{\pgfqpoint{2.328407in}{1.112612in}}%
\pgfpathlineto{\pgfqpoint{2.328407in}{0.499444in}}%
\pgfpathclose%
\pgfusepath{stroke}%
\end{pgfscope}%
\begin{pgfscope}%
\pgfpathrectangle{\pgfqpoint{0.553581in}{0.499444in}}{\pgfqpoint{3.487500in}{1.155000in}}%
\pgfusepath{clip}%
\pgfsetbuttcap%
\pgfsetmiterjoin%
\pgfsetlinewidth{1.003750pt}%
\definecolor{currentstroke}{rgb}{0.000000,0.000000,0.000000}%
\pgfsetstrokecolor{currentstroke}%
\pgfsetdash{}{0pt}%
\pgfpathmoveto{\pgfqpoint{2.466526in}{0.499444in}}%
\pgfpathlineto{\pgfqpoint{2.521774in}{0.499444in}}%
\pgfpathlineto{\pgfqpoint{2.521774in}{0.936516in}}%
\pgfpathlineto{\pgfqpoint{2.466526in}{0.936516in}}%
\pgfpathlineto{\pgfqpoint{2.466526in}{0.499444in}}%
\pgfpathclose%
\pgfusepath{stroke}%
\end{pgfscope}%
\begin{pgfscope}%
\pgfpathrectangle{\pgfqpoint{0.553581in}{0.499444in}}{\pgfqpoint{3.487500in}{1.155000in}}%
\pgfusepath{clip}%
\pgfsetbuttcap%
\pgfsetmiterjoin%
\pgfsetlinewidth{1.003750pt}%
\definecolor{currentstroke}{rgb}{0.000000,0.000000,0.000000}%
\pgfsetstrokecolor{currentstroke}%
\pgfsetdash{}{0pt}%
\pgfpathmoveto{\pgfqpoint{2.604645in}{0.499444in}}%
\pgfpathlineto{\pgfqpoint{2.659893in}{0.499444in}}%
\pgfpathlineto{\pgfqpoint{2.659893in}{0.786282in}}%
\pgfpathlineto{\pgfqpoint{2.604645in}{0.786282in}}%
\pgfpathlineto{\pgfqpoint{2.604645in}{0.499444in}}%
\pgfpathclose%
\pgfusepath{stroke}%
\end{pgfscope}%
\begin{pgfscope}%
\pgfpathrectangle{\pgfqpoint{0.553581in}{0.499444in}}{\pgfqpoint{3.487500in}{1.155000in}}%
\pgfusepath{clip}%
\pgfsetbuttcap%
\pgfsetmiterjoin%
\pgfsetlinewidth{1.003750pt}%
\definecolor{currentstroke}{rgb}{0.000000,0.000000,0.000000}%
\pgfsetstrokecolor{currentstroke}%
\pgfsetdash{}{0pt}%
\pgfpathmoveto{\pgfqpoint{2.742764in}{0.499444in}}%
\pgfpathlineto{\pgfqpoint{2.798011in}{0.499444in}}%
\pgfpathlineto{\pgfqpoint{2.798011in}{0.656587in}}%
\pgfpathlineto{\pgfqpoint{2.742764in}{0.656587in}}%
\pgfpathlineto{\pgfqpoint{2.742764in}{0.499444in}}%
\pgfpathclose%
\pgfusepath{stroke}%
\end{pgfscope}%
\begin{pgfscope}%
\pgfpathrectangle{\pgfqpoint{0.553581in}{0.499444in}}{\pgfqpoint{3.487500in}{1.155000in}}%
\pgfusepath{clip}%
\pgfsetbuttcap%
\pgfsetmiterjoin%
\pgfsetlinewidth{1.003750pt}%
\definecolor{currentstroke}{rgb}{0.000000,0.000000,0.000000}%
\pgfsetstrokecolor{currentstroke}%
\pgfsetdash{}{0pt}%
\pgfpathmoveto{\pgfqpoint{2.880883in}{0.499444in}}%
\pgfpathlineto{\pgfqpoint{2.936130in}{0.499444in}}%
\pgfpathlineto{\pgfqpoint{2.936130in}{0.582928in}}%
\pgfpathlineto{\pgfqpoint{2.880883in}{0.582928in}}%
\pgfpathlineto{\pgfqpoint{2.880883in}{0.499444in}}%
\pgfpathclose%
\pgfusepath{stroke}%
\end{pgfscope}%
\begin{pgfscope}%
\pgfpathrectangle{\pgfqpoint{0.553581in}{0.499444in}}{\pgfqpoint{3.487500in}{1.155000in}}%
\pgfusepath{clip}%
\pgfsetbuttcap%
\pgfsetmiterjoin%
\pgfsetlinewidth{1.003750pt}%
\definecolor{currentstroke}{rgb}{0.000000,0.000000,0.000000}%
\pgfsetstrokecolor{currentstroke}%
\pgfsetdash{}{0pt}%
\pgfpathmoveto{\pgfqpoint{3.019001in}{0.499444in}}%
\pgfpathlineto{\pgfqpoint{3.074249in}{0.499444in}}%
\pgfpathlineto{\pgfqpoint{3.074249in}{0.545085in}}%
\pgfpathlineto{\pgfqpoint{3.019001in}{0.545085in}}%
\pgfpathlineto{\pgfqpoint{3.019001in}{0.499444in}}%
\pgfpathclose%
\pgfusepath{stroke}%
\end{pgfscope}%
\begin{pgfscope}%
\pgfpathrectangle{\pgfqpoint{0.553581in}{0.499444in}}{\pgfqpoint{3.487500in}{1.155000in}}%
\pgfusepath{clip}%
\pgfsetbuttcap%
\pgfsetmiterjoin%
\pgfsetlinewidth{1.003750pt}%
\definecolor{currentstroke}{rgb}{0.000000,0.000000,0.000000}%
\pgfsetstrokecolor{currentstroke}%
\pgfsetdash{}{0pt}%
\pgfpathmoveto{\pgfqpoint{3.157120in}{0.499444in}}%
\pgfpathlineto{\pgfqpoint{3.212368in}{0.499444in}}%
\pgfpathlineto{\pgfqpoint{3.212368in}{0.522011in}}%
\pgfpathlineto{\pgfqpoint{3.157120in}{0.522011in}}%
\pgfpathlineto{\pgfqpoint{3.157120in}{0.499444in}}%
\pgfpathclose%
\pgfusepath{stroke}%
\end{pgfscope}%
\begin{pgfscope}%
\pgfpathrectangle{\pgfqpoint{0.553581in}{0.499444in}}{\pgfqpoint{3.487500in}{1.155000in}}%
\pgfusepath{clip}%
\pgfsetbuttcap%
\pgfsetmiterjoin%
\pgfsetlinewidth{1.003750pt}%
\definecolor{currentstroke}{rgb}{0.000000,0.000000,0.000000}%
\pgfsetstrokecolor{currentstroke}%
\pgfsetdash{}{0pt}%
\pgfpathmoveto{\pgfqpoint{3.295239in}{0.499444in}}%
\pgfpathlineto{\pgfqpoint{3.350487in}{0.499444in}}%
\pgfpathlineto{\pgfqpoint{3.350487in}{0.505783in}}%
\pgfpathlineto{\pgfqpoint{3.295239in}{0.505783in}}%
\pgfpathlineto{\pgfqpoint{3.295239in}{0.499444in}}%
\pgfpathclose%
\pgfusepath{stroke}%
\end{pgfscope}%
\begin{pgfscope}%
\pgfpathrectangle{\pgfqpoint{0.553581in}{0.499444in}}{\pgfqpoint{3.487500in}{1.155000in}}%
\pgfusepath{clip}%
\pgfsetbuttcap%
\pgfsetmiterjoin%
\pgfsetlinewidth{1.003750pt}%
\definecolor{currentstroke}{rgb}{0.000000,0.000000,0.000000}%
\pgfsetstrokecolor{currentstroke}%
\pgfsetdash{}{0pt}%
\pgfpathmoveto{\pgfqpoint{3.433358in}{0.499444in}}%
\pgfpathlineto{\pgfqpoint{3.488605in}{0.499444in}}%
\pgfpathlineto{\pgfqpoint{3.488605in}{0.500141in}}%
\pgfpathlineto{\pgfqpoint{3.433358in}{0.500141in}}%
\pgfpathlineto{\pgfqpoint{3.433358in}{0.499444in}}%
\pgfpathclose%
\pgfusepath{stroke}%
\end{pgfscope}%
\begin{pgfscope}%
\pgfpathrectangle{\pgfqpoint{0.553581in}{0.499444in}}{\pgfqpoint{3.487500in}{1.155000in}}%
\pgfusepath{clip}%
\pgfsetbuttcap%
\pgfsetmiterjoin%
\pgfsetlinewidth{1.003750pt}%
\definecolor{currentstroke}{rgb}{0.000000,0.000000,0.000000}%
\pgfsetstrokecolor{currentstroke}%
\pgfsetdash{}{0pt}%
\pgfpathmoveto{\pgfqpoint{3.571477in}{0.499444in}}%
\pgfpathlineto{\pgfqpoint{3.626724in}{0.499444in}}%
\pgfpathlineto{\pgfqpoint{3.626724in}{0.499571in}}%
\pgfpathlineto{\pgfqpoint{3.571477in}{0.499571in}}%
\pgfpathlineto{\pgfqpoint{3.571477in}{0.499444in}}%
\pgfpathclose%
\pgfusepath{stroke}%
\end{pgfscope}%
\begin{pgfscope}%
\pgfpathrectangle{\pgfqpoint{0.553581in}{0.499444in}}{\pgfqpoint{3.487500in}{1.155000in}}%
\pgfusepath{clip}%
\pgfsetbuttcap%
\pgfsetmiterjoin%
\pgfsetlinewidth{1.003750pt}%
\definecolor{currentstroke}{rgb}{0.000000,0.000000,0.000000}%
\pgfsetstrokecolor{currentstroke}%
\pgfsetdash{}{0pt}%
\pgfpathmoveto{\pgfqpoint{3.709596in}{0.499444in}}%
\pgfpathlineto{\pgfqpoint{3.764843in}{0.499444in}}%
\pgfpathlineto{\pgfqpoint{3.764843in}{0.499508in}}%
\pgfpathlineto{\pgfqpoint{3.709596in}{0.499508in}}%
\pgfpathlineto{\pgfqpoint{3.709596in}{0.499444in}}%
\pgfpathclose%
\pgfusepath{stroke}%
\end{pgfscope}%
\begin{pgfscope}%
\pgfpathrectangle{\pgfqpoint{0.553581in}{0.499444in}}{\pgfqpoint{3.487500in}{1.155000in}}%
\pgfusepath{clip}%
\pgfsetbuttcap%
\pgfsetmiterjoin%
\pgfsetlinewidth{1.003750pt}%
\definecolor{currentstroke}{rgb}{0.000000,0.000000,0.000000}%
\pgfsetstrokecolor{currentstroke}%
\pgfsetdash{}{0pt}%
\pgfpathmoveto{\pgfqpoint{3.847714in}{0.499444in}}%
\pgfpathlineto{\pgfqpoint{3.902962in}{0.499444in}}%
\pgfpathlineto{\pgfqpoint{3.902962in}{0.499444in}}%
\pgfpathlineto{\pgfqpoint{3.847714in}{0.499444in}}%
\pgfpathlineto{\pgfqpoint{3.847714in}{0.499444in}}%
\pgfpathclose%
\pgfusepath{stroke}%
\end{pgfscope}%
\begin{pgfscope}%
\pgfpathrectangle{\pgfqpoint{0.553581in}{0.499444in}}{\pgfqpoint{3.487500in}{1.155000in}}%
\pgfusepath{clip}%
\pgfsetbuttcap%
\pgfsetmiterjoin%
\definecolor{currentfill}{rgb}{0.000000,0.000000,0.000000}%
\pgfsetfillcolor{currentfill}%
\pgfsetlinewidth{0.000000pt}%
\definecolor{currentstroke}{rgb}{0.000000,0.000000,0.000000}%
\pgfsetstrokecolor{currentstroke}%
\pgfsetstrokeopacity{0.000000}%
\pgfsetdash{}{0pt}%
\pgfpathmoveto{\pgfqpoint{0.588110in}{0.499444in}}%
\pgfpathlineto{\pgfqpoint{0.643358in}{0.499444in}}%
\pgfpathlineto{\pgfqpoint{0.643358in}{0.499825in}}%
\pgfpathlineto{\pgfqpoint{0.588110in}{0.499825in}}%
\pgfpathlineto{\pgfqpoint{0.588110in}{0.499444in}}%
\pgfpathclose%
\pgfusepath{fill}%
\end{pgfscope}%
\begin{pgfscope}%
\pgfpathrectangle{\pgfqpoint{0.553581in}{0.499444in}}{\pgfqpoint{3.487500in}{1.155000in}}%
\pgfusepath{clip}%
\pgfsetbuttcap%
\pgfsetmiterjoin%
\definecolor{currentfill}{rgb}{0.000000,0.000000,0.000000}%
\pgfsetfillcolor{currentfill}%
\pgfsetlinewidth{0.000000pt}%
\definecolor{currentstroke}{rgb}{0.000000,0.000000,0.000000}%
\pgfsetstrokecolor{currentstroke}%
\pgfsetstrokeopacity{0.000000}%
\pgfsetdash{}{0pt}%
\pgfpathmoveto{\pgfqpoint{0.726229in}{0.499444in}}%
\pgfpathlineto{\pgfqpoint{0.781477in}{0.499444in}}%
\pgfpathlineto{\pgfqpoint{0.781477in}{0.501346in}}%
\pgfpathlineto{\pgfqpoint{0.726229in}{0.501346in}}%
\pgfpathlineto{\pgfqpoint{0.726229in}{0.499444in}}%
\pgfpathclose%
\pgfusepath{fill}%
\end{pgfscope}%
\begin{pgfscope}%
\pgfpathrectangle{\pgfqpoint{0.553581in}{0.499444in}}{\pgfqpoint{3.487500in}{1.155000in}}%
\pgfusepath{clip}%
\pgfsetbuttcap%
\pgfsetmiterjoin%
\definecolor{currentfill}{rgb}{0.000000,0.000000,0.000000}%
\pgfsetfillcolor{currentfill}%
\pgfsetlinewidth{0.000000pt}%
\definecolor{currentstroke}{rgb}{0.000000,0.000000,0.000000}%
\pgfsetstrokecolor{currentstroke}%
\pgfsetstrokeopacity{0.000000}%
\pgfsetdash{}{0pt}%
\pgfpathmoveto{\pgfqpoint{0.864348in}{0.499444in}}%
\pgfpathlineto{\pgfqpoint{0.919596in}{0.499444in}}%
\pgfpathlineto{\pgfqpoint{0.919596in}{0.504959in}}%
\pgfpathlineto{\pgfqpoint{0.864348in}{0.504959in}}%
\pgfpathlineto{\pgfqpoint{0.864348in}{0.499444in}}%
\pgfpathclose%
\pgfusepath{fill}%
\end{pgfscope}%
\begin{pgfscope}%
\pgfpathrectangle{\pgfqpoint{0.553581in}{0.499444in}}{\pgfqpoint{3.487500in}{1.155000in}}%
\pgfusepath{clip}%
\pgfsetbuttcap%
\pgfsetmiterjoin%
\definecolor{currentfill}{rgb}{0.000000,0.000000,0.000000}%
\pgfsetfillcolor{currentfill}%
\pgfsetlinewidth{0.000000pt}%
\definecolor{currentstroke}{rgb}{0.000000,0.000000,0.000000}%
\pgfsetstrokecolor{currentstroke}%
\pgfsetstrokeopacity{0.000000}%
\pgfsetdash{}{0pt}%
\pgfpathmoveto{\pgfqpoint{1.002467in}{0.499444in}}%
\pgfpathlineto{\pgfqpoint{1.057714in}{0.499444in}}%
\pgfpathlineto{\pgfqpoint{1.057714in}{0.510664in}}%
\pgfpathlineto{\pgfqpoint{1.002467in}{0.510664in}}%
\pgfpathlineto{\pgfqpoint{1.002467in}{0.499444in}}%
\pgfpathclose%
\pgfusepath{fill}%
\end{pgfscope}%
\begin{pgfscope}%
\pgfpathrectangle{\pgfqpoint{0.553581in}{0.499444in}}{\pgfqpoint{3.487500in}{1.155000in}}%
\pgfusepath{clip}%
\pgfsetbuttcap%
\pgfsetmiterjoin%
\definecolor{currentfill}{rgb}{0.000000,0.000000,0.000000}%
\pgfsetfillcolor{currentfill}%
\pgfsetlinewidth{0.000000pt}%
\definecolor{currentstroke}{rgb}{0.000000,0.000000,0.000000}%
\pgfsetstrokecolor{currentstroke}%
\pgfsetstrokeopacity{0.000000}%
\pgfsetdash{}{0pt}%
\pgfpathmoveto{\pgfqpoint{1.140586in}{0.499444in}}%
\pgfpathlineto{\pgfqpoint{1.195833in}{0.499444in}}%
\pgfpathlineto{\pgfqpoint{1.195833in}{0.523279in}}%
\pgfpathlineto{\pgfqpoint{1.140586in}{0.523279in}}%
\pgfpathlineto{\pgfqpoint{1.140586in}{0.499444in}}%
\pgfpathclose%
\pgfusepath{fill}%
\end{pgfscope}%
\begin{pgfscope}%
\pgfpathrectangle{\pgfqpoint{0.553581in}{0.499444in}}{\pgfqpoint{3.487500in}{1.155000in}}%
\pgfusepath{clip}%
\pgfsetbuttcap%
\pgfsetmiterjoin%
\definecolor{currentfill}{rgb}{0.000000,0.000000,0.000000}%
\pgfsetfillcolor{currentfill}%
\pgfsetlinewidth{0.000000pt}%
\definecolor{currentstroke}{rgb}{0.000000,0.000000,0.000000}%
\pgfsetstrokecolor{currentstroke}%
\pgfsetstrokeopacity{0.000000}%
\pgfsetdash{}{0pt}%
\pgfpathmoveto{\pgfqpoint{1.278704in}{0.499444in}}%
\pgfpathlineto{\pgfqpoint{1.333952in}{0.499444in}}%
\pgfpathlineto{\pgfqpoint{1.333952in}{0.534562in}}%
\pgfpathlineto{\pgfqpoint{1.278704in}{0.534562in}}%
\pgfpathlineto{\pgfqpoint{1.278704in}{0.499444in}}%
\pgfpathclose%
\pgfusepath{fill}%
\end{pgfscope}%
\begin{pgfscope}%
\pgfpathrectangle{\pgfqpoint{0.553581in}{0.499444in}}{\pgfqpoint{3.487500in}{1.155000in}}%
\pgfusepath{clip}%
\pgfsetbuttcap%
\pgfsetmiterjoin%
\definecolor{currentfill}{rgb}{0.000000,0.000000,0.000000}%
\pgfsetfillcolor{currentfill}%
\pgfsetlinewidth{0.000000pt}%
\definecolor{currentstroke}{rgb}{0.000000,0.000000,0.000000}%
\pgfsetstrokecolor{currentstroke}%
\pgfsetstrokeopacity{0.000000}%
\pgfsetdash{}{0pt}%
\pgfpathmoveto{\pgfqpoint{1.416823in}{0.499444in}}%
\pgfpathlineto{\pgfqpoint{1.472071in}{0.499444in}}%
\pgfpathlineto{\pgfqpoint{1.472071in}{0.558206in}}%
\pgfpathlineto{\pgfqpoint{1.416823in}{0.558206in}}%
\pgfpathlineto{\pgfqpoint{1.416823in}{0.499444in}}%
\pgfpathclose%
\pgfusepath{fill}%
\end{pgfscope}%
\begin{pgfscope}%
\pgfpathrectangle{\pgfqpoint{0.553581in}{0.499444in}}{\pgfqpoint{3.487500in}{1.155000in}}%
\pgfusepath{clip}%
\pgfsetbuttcap%
\pgfsetmiterjoin%
\definecolor{currentfill}{rgb}{0.000000,0.000000,0.000000}%
\pgfsetfillcolor{currentfill}%
\pgfsetlinewidth{0.000000pt}%
\definecolor{currentstroke}{rgb}{0.000000,0.000000,0.000000}%
\pgfsetstrokecolor{currentstroke}%
\pgfsetstrokeopacity{0.000000}%
\pgfsetdash{}{0pt}%
\pgfpathmoveto{\pgfqpoint{1.554942in}{0.499444in}}%
\pgfpathlineto{\pgfqpoint{1.610190in}{0.499444in}}%
\pgfpathlineto{\pgfqpoint{1.610190in}{0.583562in}}%
\pgfpathlineto{\pgfqpoint{1.554942in}{0.583562in}}%
\pgfpathlineto{\pgfqpoint{1.554942in}{0.499444in}}%
\pgfpathclose%
\pgfusepath{fill}%
\end{pgfscope}%
\begin{pgfscope}%
\pgfpathrectangle{\pgfqpoint{0.553581in}{0.499444in}}{\pgfqpoint{3.487500in}{1.155000in}}%
\pgfusepath{clip}%
\pgfsetbuttcap%
\pgfsetmiterjoin%
\definecolor{currentfill}{rgb}{0.000000,0.000000,0.000000}%
\pgfsetfillcolor{currentfill}%
\pgfsetlinewidth{0.000000pt}%
\definecolor{currentstroke}{rgb}{0.000000,0.000000,0.000000}%
\pgfsetstrokecolor{currentstroke}%
\pgfsetstrokeopacity{0.000000}%
\pgfsetdash{}{0pt}%
\pgfpathmoveto{\pgfqpoint{1.693061in}{0.499444in}}%
\pgfpathlineto{\pgfqpoint{1.748308in}{0.499444in}}%
\pgfpathlineto{\pgfqpoint{1.748308in}{0.610186in}}%
\pgfpathlineto{\pgfqpoint{1.693061in}{0.610186in}}%
\pgfpathlineto{\pgfqpoint{1.693061in}{0.499444in}}%
\pgfpathclose%
\pgfusepath{fill}%
\end{pgfscope}%
\begin{pgfscope}%
\pgfpathrectangle{\pgfqpoint{0.553581in}{0.499444in}}{\pgfqpoint{3.487500in}{1.155000in}}%
\pgfusepath{clip}%
\pgfsetbuttcap%
\pgfsetmiterjoin%
\definecolor{currentfill}{rgb}{0.000000,0.000000,0.000000}%
\pgfsetfillcolor{currentfill}%
\pgfsetlinewidth{0.000000pt}%
\definecolor{currentstroke}{rgb}{0.000000,0.000000,0.000000}%
\pgfsetstrokecolor{currentstroke}%
\pgfsetstrokeopacity{0.000000}%
\pgfsetdash{}{0pt}%
\pgfpathmoveto{\pgfqpoint{1.831180in}{0.499444in}}%
\pgfpathlineto{\pgfqpoint{1.886427in}{0.499444in}}%
\pgfpathlineto{\pgfqpoint{1.886427in}{0.639789in}}%
\pgfpathlineto{\pgfqpoint{1.831180in}{0.639789in}}%
\pgfpathlineto{\pgfqpoint{1.831180in}{0.499444in}}%
\pgfpathclose%
\pgfusepath{fill}%
\end{pgfscope}%
\begin{pgfscope}%
\pgfpathrectangle{\pgfqpoint{0.553581in}{0.499444in}}{\pgfqpoint{3.487500in}{1.155000in}}%
\pgfusepath{clip}%
\pgfsetbuttcap%
\pgfsetmiterjoin%
\definecolor{currentfill}{rgb}{0.000000,0.000000,0.000000}%
\pgfsetfillcolor{currentfill}%
\pgfsetlinewidth{0.000000pt}%
\definecolor{currentstroke}{rgb}{0.000000,0.000000,0.000000}%
\pgfsetstrokecolor{currentstroke}%
\pgfsetstrokeopacity{0.000000}%
\pgfsetdash{}{0pt}%
\pgfpathmoveto{\pgfqpoint{1.969299in}{0.499444in}}%
\pgfpathlineto{\pgfqpoint{2.024546in}{0.499444in}}%
\pgfpathlineto{\pgfqpoint{2.024546in}{0.671927in}}%
\pgfpathlineto{\pgfqpoint{1.969299in}{0.671927in}}%
\pgfpathlineto{\pgfqpoint{1.969299in}{0.499444in}}%
\pgfpathclose%
\pgfusepath{fill}%
\end{pgfscope}%
\begin{pgfscope}%
\pgfpathrectangle{\pgfqpoint{0.553581in}{0.499444in}}{\pgfqpoint{3.487500in}{1.155000in}}%
\pgfusepath{clip}%
\pgfsetbuttcap%
\pgfsetmiterjoin%
\definecolor{currentfill}{rgb}{0.000000,0.000000,0.000000}%
\pgfsetfillcolor{currentfill}%
\pgfsetlinewidth{0.000000pt}%
\definecolor{currentstroke}{rgb}{0.000000,0.000000,0.000000}%
\pgfsetstrokecolor{currentstroke}%
\pgfsetstrokeopacity{0.000000}%
\pgfsetdash{}{0pt}%
\pgfpathmoveto{\pgfqpoint{2.107417in}{0.499444in}}%
\pgfpathlineto{\pgfqpoint{2.162665in}{0.499444in}}%
\pgfpathlineto{\pgfqpoint{2.162665in}{0.700579in}}%
\pgfpathlineto{\pgfqpoint{2.107417in}{0.700579in}}%
\pgfpathlineto{\pgfqpoint{2.107417in}{0.499444in}}%
\pgfpathclose%
\pgfusepath{fill}%
\end{pgfscope}%
\begin{pgfscope}%
\pgfpathrectangle{\pgfqpoint{0.553581in}{0.499444in}}{\pgfqpoint{3.487500in}{1.155000in}}%
\pgfusepath{clip}%
\pgfsetbuttcap%
\pgfsetmiterjoin%
\definecolor{currentfill}{rgb}{0.000000,0.000000,0.000000}%
\pgfsetfillcolor{currentfill}%
\pgfsetlinewidth{0.000000pt}%
\definecolor{currentstroke}{rgb}{0.000000,0.000000,0.000000}%
\pgfsetstrokecolor{currentstroke}%
\pgfsetstrokeopacity{0.000000}%
\pgfsetdash{}{0pt}%
\pgfpathmoveto{\pgfqpoint{2.245536in}{0.499444in}}%
\pgfpathlineto{\pgfqpoint{2.300784in}{0.499444in}}%
\pgfpathlineto{\pgfqpoint{2.300784in}{0.726252in}}%
\pgfpathlineto{\pgfqpoint{2.245536in}{0.726252in}}%
\pgfpathlineto{\pgfqpoint{2.245536in}{0.499444in}}%
\pgfpathclose%
\pgfusepath{fill}%
\end{pgfscope}%
\begin{pgfscope}%
\pgfpathrectangle{\pgfqpoint{0.553581in}{0.499444in}}{\pgfqpoint{3.487500in}{1.155000in}}%
\pgfusepath{clip}%
\pgfsetbuttcap%
\pgfsetmiterjoin%
\definecolor{currentfill}{rgb}{0.000000,0.000000,0.000000}%
\pgfsetfillcolor{currentfill}%
\pgfsetlinewidth{0.000000pt}%
\definecolor{currentstroke}{rgb}{0.000000,0.000000,0.000000}%
\pgfsetstrokecolor{currentstroke}%
\pgfsetstrokeopacity{0.000000}%
\pgfsetdash{}{0pt}%
\pgfpathmoveto{\pgfqpoint{2.383655in}{0.499444in}}%
\pgfpathlineto{\pgfqpoint{2.438902in}{0.499444in}}%
\pgfpathlineto{\pgfqpoint{2.438902in}{0.723400in}}%
\pgfpathlineto{\pgfqpoint{2.383655in}{0.723400in}}%
\pgfpathlineto{\pgfqpoint{2.383655in}{0.499444in}}%
\pgfpathclose%
\pgfusepath{fill}%
\end{pgfscope}%
\begin{pgfscope}%
\pgfpathrectangle{\pgfqpoint{0.553581in}{0.499444in}}{\pgfqpoint{3.487500in}{1.155000in}}%
\pgfusepath{clip}%
\pgfsetbuttcap%
\pgfsetmiterjoin%
\definecolor{currentfill}{rgb}{0.000000,0.000000,0.000000}%
\pgfsetfillcolor{currentfill}%
\pgfsetlinewidth{0.000000pt}%
\definecolor{currentstroke}{rgb}{0.000000,0.000000,0.000000}%
\pgfsetstrokecolor{currentstroke}%
\pgfsetstrokeopacity{0.000000}%
\pgfsetdash{}{0pt}%
\pgfpathmoveto{\pgfqpoint{2.521774in}{0.499444in}}%
\pgfpathlineto{\pgfqpoint{2.577021in}{0.499444in}}%
\pgfpathlineto{\pgfqpoint{2.577021in}{0.718709in}}%
\pgfpathlineto{\pgfqpoint{2.521774in}{0.718709in}}%
\pgfpathlineto{\pgfqpoint{2.521774in}{0.499444in}}%
\pgfpathclose%
\pgfusepath{fill}%
\end{pgfscope}%
\begin{pgfscope}%
\pgfpathrectangle{\pgfqpoint{0.553581in}{0.499444in}}{\pgfqpoint{3.487500in}{1.155000in}}%
\pgfusepath{clip}%
\pgfsetbuttcap%
\pgfsetmiterjoin%
\definecolor{currentfill}{rgb}{0.000000,0.000000,0.000000}%
\pgfsetfillcolor{currentfill}%
\pgfsetlinewidth{0.000000pt}%
\definecolor{currentstroke}{rgb}{0.000000,0.000000,0.000000}%
\pgfsetstrokecolor{currentstroke}%
\pgfsetstrokeopacity{0.000000}%
\pgfsetdash{}{0pt}%
\pgfpathmoveto{\pgfqpoint{2.659893in}{0.499444in}}%
\pgfpathlineto{\pgfqpoint{2.715140in}{0.499444in}}%
\pgfpathlineto{\pgfqpoint{2.715140in}{0.691261in}}%
\pgfpathlineto{\pgfqpoint{2.659893in}{0.691261in}}%
\pgfpathlineto{\pgfqpoint{2.659893in}{0.499444in}}%
\pgfpathclose%
\pgfusepath{fill}%
\end{pgfscope}%
\begin{pgfscope}%
\pgfpathrectangle{\pgfqpoint{0.553581in}{0.499444in}}{\pgfqpoint{3.487500in}{1.155000in}}%
\pgfusepath{clip}%
\pgfsetbuttcap%
\pgfsetmiterjoin%
\definecolor{currentfill}{rgb}{0.000000,0.000000,0.000000}%
\pgfsetfillcolor{currentfill}%
\pgfsetlinewidth{0.000000pt}%
\definecolor{currentstroke}{rgb}{0.000000,0.000000,0.000000}%
\pgfsetstrokecolor{currentstroke}%
\pgfsetstrokeopacity{0.000000}%
\pgfsetdash{}{0pt}%
\pgfpathmoveto{\pgfqpoint{2.798011in}{0.499444in}}%
\pgfpathlineto{\pgfqpoint{2.853259in}{0.499444in}}%
\pgfpathlineto{\pgfqpoint{2.853259in}{0.651072in}}%
\pgfpathlineto{\pgfqpoint{2.798011in}{0.651072in}}%
\pgfpathlineto{\pgfqpoint{2.798011in}{0.499444in}}%
\pgfpathclose%
\pgfusepath{fill}%
\end{pgfscope}%
\begin{pgfscope}%
\pgfpathrectangle{\pgfqpoint{0.553581in}{0.499444in}}{\pgfqpoint{3.487500in}{1.155000in}}%
\pgfusepath{clip}%
\pgfsetbuttcap%
\pgfsetmiterjoin%
\definecolor{currentfill}{rgb}{0.000000,0.000000,0.000000}%
\pgfsetfillcolor{currentfill}%
\pgfsetlinewidth{0.000000pt}%
\definecolor{currentstroke}{rgb}{0.000000,0.000000,0.000000}%
\pgfsetstrokecolor{currentstroke}%
\pgfsetstrokeopacity{0.000000}%
\pgfsetdash{}{0pt}%
\pgfpathmoveto{\pgfqpoint{2.936130in}{0.499444in}}%
\pgfpathlineto{\pgfqpoint{2.991378in}{0.499444in}}%
\pgfpathlineto{\pgfqpoint{2.991378in}{0.610376in}}%
\pgfpathlineto{\pgfqpoint{2.936130in}{0.610376in}}%
\pgfpathlineto{\pgfqpoint{2.936130in}{0.499444in}}%
\pgfpathclose%
\pgfusepath{fill}%
\end{pgfscope}%
\begin{pgfscope}%
\pgfpathrectangle{\pgfqpoint{0.553581in}{0.499444in}}{\pgfqpoint{3.487500in}{1.155000in}}%
\pgfusepath{clip}%
\pgfsetbuttcap%
\pgfsetmiterjoin%
\definecolor{currentfill}{rgb}{0.000000,0.000000,0.000000}%
\pgfsetfillcolor{currentfill}%
\pgfsetlinewidth{0.000000pt}%
\definecolor{currentstroke}{rgb}{0.000000,0.000000,0.000000}%
\pgfsetstrokecolor{currentstroke}%
\pgfsetstrokeopacity{0.000000}%
\pgfsetdash{}{0pt}%
\pgfpathmoveto{\pgfqpoint{3.074249in}{0.499444in}}%
\pgfpathlineto{\pgfqpoint{3.129497in}{0.499444in}}%
\pgfpathlineto{\pgfqpoint{3.129497in}{0.582231in}}%
\pgfpathlineto{\pgfqpoint{3.074249in}{0.582231in}}%
\pgfpathlineto{\pgfqpoint{3.074249in}{0.499444in}}%
\pgfpathclose%
\pgfusepath{fill}%
\end{pgfscope}%
\begin{pgfscope}%
\pgfpathrectangle{\pgfqpoint{0.553581in}{0.499444in}}{\pgfqpoint{3.487500in}{1.155000in}}%
\pgfusepath{clip}%
\pgfsetbuttcap%
\pgfsetmiterjoin%
\definecolor{currentfill}{rgb}{0.000000,0.000000,0.000000}%
\pgfsetfillcolor{currentfill}%
\pgfsetlinewidth{0.000000pt}%
\definecolor{currentstroke}{rgb}{0.000000,0.000000,0.000000}%
\pgfsetstrokecolor{currentstroke}%
\pgfsetstrokeopacity{0.000000}%
\pgfsetdash{}{0pt}%
\pgfpathmoveto{\pgfqpoint{3.212368in}{0.499444in}}%
\pgfpathlineto{\pgfqpoint{3.267615in}{0.499444in}}%
\pgfpathlineto{\pgfqpoint{3.267615in}{0.562263in}}%
\pgfpathlineto{\pgfqpoint{3.212368in}{0.562263in}}%
\pgfpathlineto{\pgfqpoint{3.212368in}{0.499444in}}%
\pgfpathclose%
\pgfusepath{fill}%
\end{pgfscope}%
\begin{pgfscope}%
\pgfpathrectangle{\pgfqpoint{0.553581in}{0.499444in}}{\pgfqpoint{3.487500in}{1.155000in}}%
\pgfusepath{clip}%
\pgfsetbuttcap%
\pgfsetmiterjoin%
\definecolor{currentfill}{rgb}{0.000000,0.000000,0.000000}%
\pgfsetfillcolor{currentfill}%
\pgfsetlinewidth{0.000000pt}%
\definecolor{currentstroke}{rgb}{0.000000,0.000000,0.000000}%
\pgfsetstrokecolor{currentstroke}%
\pgfsetstrokeopacity{0.000000}%
\pgfsetdash{}{0pt}%
\pgfpathmoveto{\pgfqpoint{3.350487in}{0.499444in}}%
\pgfpathlineto{\pgfqpoint{3.405734in}{0.499444in}}%
\pgfpathlineto{\pgfqpoint{3.405734in}{0.524610in}}%
\pgfpathlineto{\pgfqpoint{3.350487in}{0.524610in}}%
\pgfpathlineto{\pgfqpoint{3.350487in}{0.499444in}}%
\pgfpathclose%
\pgfusepath{fill}%
\end{pgfscope}%
\begin{pgfscope}%
\pgfpathrectangle{\pgfqpoint{0.553581in}{0.499444in}}{\pgfqpoint{3.487500in}{1.155000in}}%
\pgfusepath{clip}%
\pgfsetbuttcap%
\pgfsetmiterjoin%
\definecolor{currentfill}{rgb}{0.000000,0.000000,0.000000}%
\pgfsetfillcolor{currentfill}%
\pgfsetlinewidth{0.000000pt}%
\definecolor{currentstroke}{rgb}{0.000000,0.000000,0.000000}%
\pgfsetstrokecolor{currentstroke}%
\pgfsetstrokeopacity{0.000000}%
\pgfsetdash{}{0pt}%
\pgfpathmoveto{\pgfqpoint{3.488605in}{0.499444in}}%
\pgfpathlineto{\pgfqpoint{3.543853in}{0.499444in}}%
\pgfpathlineto{\pgfqpoint{3.543853in}{0.502740in}}%
\pgfpathlineto{\pgfqpoint{3.488605in}{0.502740in}}%
\pgfpathlineto{\pgfqpoint{3.488605in}{0.499444in}}%
\pgfpathclose%
\pgfusepath{fill}%
\end{pgfscope}%
\begin{pgfscope}%
\pgfpathrectangle{\pgfqpoint{0.553581in}{0.499444in}}{\pgfqpoint{3.487500in}{1.155000in}}%
\pgfusepath{clip}%
\pgfsetbuttcap%
\pgfsetmiterjoin%
\definecolor{currentfill}{rgb}{0.000000,0.000000,0.000000}%
\pgfsetfillcolor{currentfill}%
\pgfsetlinewidth{0.000000pt}%
\definecolor{currentstroke}{rgb}{0.000000,0.000000,0.000000}%
\pgfsetstrokecolor{currentstroke}%
\pgfsetstrokeopacity{0.000000}%
\pgfsetdash{}{0pt}%
\pgfpathmoveto{\pgfqpoint{3.626724in}{0.499444in}}%
\pgfpathlineto{\pgfqpoint{3.681972in}{0.499444in}}%
\pgfpathlineto{\pgfqpoint{3.681972in}{0.499508in}}%
\pgfpathlineto{\pgfqpoint{3.626724in}{0.499508in}}%
\pgfpathlineto{\pgfqpoint{3.626724in}{0.499444in}}%
\pgfpathclose%
\pgfusepath{fill}%
\end{pgfscope}%
\begin{pgfscope}%
\pgfpathrectangle{\pgfqpoint{0.553581in}{0.499444in}}{\pgfqpoint{3.487500in}{1.155000in}}%
\pgfusepath{clip}%
\pgfsetbuttcap%
\pgfsetmiterjoin%
\definecolor{currentfill}{rgb}{0.000000,0.000000,0.000000}%
\pgfsetfillcolor{currentfill}%
\pgfsetlinewidth{0.000000pt}%
\definecolor{currentstroke}{rgb}{0.000000,0.000000,0.000000}%
\pgfsetstrokecolor{currentstroke}%
\pgfsetstrokeopacity{0.000000}%
\pgfsetdash{}{0pt}%
\pgfpathmoveto{\pgfqpoint{3.764843in}{0.499444in}}%
\pgfpathlineto{\pgfqpoint{3.820091in}{0.499444in}}%
\pgfpathlineto{\pgfqpoint{3.820091in}{0.499508in}}%
\pgfpathlineto{\pgfqpoint{3.764843in}{0.499508in}}%
\pgfpathlineto{\pgfqpoint{3.764843in}{0.499444in}}%
\pgfpathclose%
\pgfusepath{fill}%
\end{pgfscope}%
\begin{pgfscope}%
\pgfpathrectangle{\pgfqpoint{0.553581in}{0.499444in}}{\pgfqpoint{3.487500in}{1.155000in}}%
\pgfusepath{clip}%
\pgfsetbuttcap%
\pgfsetmiterjoin%
\definecolor{currentfill}{rgb}{0.000000,0.000000,0.000000}%
\pgfsetfillcolor{currentfill}%
\pgfsetlinewidth{0.000000pt}%
\definecolor{currentstroke}{rgb}{0.000000,0.000000,0.000000}%
\pgfsetstrokecolor{currentstroke}%
\pgfsetstrokeopacity{0.000000}%
\pgfsetdash{}{0pt}%
\pgfpathmoveto{\pgfqpoint{3.902962in}{0.499444in}}%
\pgfpathlineto{\pgfqpoint{3.958209in}{0.499444in}}%
\pgfpathlineto{\pgfqpoint{3.958209in}{0.499444in}}%
\pgfpathlineto{\pgfqpoint{3.902962in}{0.499444in}}%
\pgfpathlineto{\pgfqpoint{3.902962in}{0.499444in}}%
\pgfpathclose%
\pgfusepath{fill}%
\end{pgfscope}%
\begin{pgfscope}%
\pgfsetbuttcap%
\pgfsetroundjoin%
\definecolor{currentfill}{rgb}{0.000000,0.000000,0.000000}%
\pgfsetfillcolor{currentfill}%
\pgfsetlinewidth{0.803000pt}%
\definecolor{currentstroke}{rgb}{0.000000,0.000000,0.000000}%
\pgfsetstrokecolor{currentstroke}%
\pgfsetdash{}{0pt}%
\pgfsys@defobject{currentmarker}{\pgfqpoint{0.000000in}{-0.048611in}}{\pgfqpoint{0.000000in}{0.000000in}}{%
\pgfpathmoveto{\pgfqpoint{0.000000in}{0.000000in}}%
\pgfpathlineto{\pgfqpoint{0.000000in}{-0.048611in}}%
\pgfusepath{stroke,fill}%
}%
\begin{pgfscope}%
\pgfsys@transformshift{0.588110in}{0.499444in}%
\pgfsys@useobject{currentmarker}{}%
\end{pgfscope}%
\end{pgfscope}%
\begin{pgfscope}%
\definecolor{textcolor}{rgb}{0.000000,0.000000,0.000000}%
\pgfsetstrokecolor{textcolor}%
\pgfsetfillcolor{textcolor}%
\pgftext[x=0.588110in,y=0.402222in,,top]{\color{textcolor}\rmfamily\fontsize{10.000000}{12.000000}\selectfont 0.0}%
\end{pgfscope}%
\begin{pgfscope}%
\pgfsetbuttcap%
\pgfsetroundjoin%
\definecolor{currentfill}{rgb}{0.000000,0.000000,0.000000}%
\pgfsetfillcolor{currentfill}%
\pgfsetlinewidth{0.803000pt}%
\definecolor{currentstroke}{rgb}{0.000000,0.000000,0.000000}%
\pgfsetstrokecolor{currentstroke}%
\pgfsetdash{}{0pt}%
\pgfsys@defobject{currentmarker}{\pgfqpoint{0.000000in}{-0.048611in}}{\pgfqpoint{0.000000in}{0.000000in}}{%
\pgfpathmoveto{\pgfqpoint{0.000000in}{0.000000in}}%
\pgfpathlineto{\pgfqpoint{0.000000in}{-0.048611in}}%
\pgfusepath{stroke,fill}%
}%
\begin{pgfscope}%
\pgfsys@transformshift{0.933407in}{0.499444in}%
\pgfsys@useobject{currentmarker}{}%
\end{pgfscope}%
\end{pgfscope}%
\begin{pgfscope}%
\definecolor{textcolor}{rgb}{0.000000,0.000000,0.000000}%
\pgfsetstrokecolor{textcolor}%
\pgfsetfillcolor{textcolor}%
\pgftext[x=0.933407in,y=0.402222in,,top]{\color{textcolor}\rmfamily\fontsize{10.000000}{12.000000}\selectfont 0.1}%
\end{pgfscope}%
\begin{pgfscope}%
\pgfsetbuttcap%
\pgfsetroundjoin%
\definecolor{currentfill}{rgb}{0.000000,0.000000,0.000000}%
\pgfsetfillcolor{currentfill}%
\pgfsetlinewidth{0.803000pt}%
\definecolor{currentstroke}{rgb}{0.000000,0.000000,0.000000}%
\pgfsetstrokecolor{currentstroke}%
\pgfsetdash{}{0pt}%
\pgfsys@defobject{currentmarker}{\pgfqpoint{0.000000in}{-0.048611in}}{\pgfqpoint{0.000000in}{0.000000in}}{%
\pgfpathmoveto{\pgfqpoint{0.000000in}{0.000000in}}%
\pgfpathlineto{\pgfqpoint{0.000000in}{-0.048611in}}%
\pgfusepath{stroke,fill}%
}%
\begin{pgfscope}%
\pgfsys@transformshift{1.278704in}{0.499444in}%
\pgfsys@useobject{currentmarker}{}%
\end{pgfscope}%
\end{pgfscope}%
\begin{pgfscope}%
\definecolor{textcolor}{rgb}{0.000000,0.000000,0.000000}%
\pgfsetstrokecolor{textcolor}%
\pgfsetfillcolor{textcolor}%
\pgftext[x=1.278704in,y=0.402222in,,top]{\color{textcolor}\rmfamily\fontsize{10.000000}{12.000000}\selectfont 0.2}%
\end{pgfscope}%
\begin{pgfscope}%
\pgfsetbuttcap%
\pgfsetroundjoin%
\definecolor{currentfill}{rgb}{0.000000,0.000000,0.000000}%
\pgfsetfillcolor{currentfill}%
\pgfsetlinewidth{0.803000pt}%
\definecolor{currentstroke}{rgb}{0.000000,0.000000,0.000000}%
\pgfsetstrokecolor{currentstroke}%
\pgfsetdash{}{0pt}%
\pgfsys@defobject{currentmarker}{\pgfqpoint{0.000000in}{-0.048611in}}{\pgfqpoint{0.000000in}{0.000000in}}{%
\pgfpathmoveto{\pgfqpoint{0.000000in}{0.000000in}}%
\pgfpathlineto{\pgfqpoint{0.000000in}{-0.048611in}}%
\pgfusepath{stroke,fill}%
}%
\begin{pgfscope}%
\pgfsys@transformshift{1.624001in}{0.499444in}%
\pgfsys@useobject{currentmarker}{}%
\end{pgfscope}%
\end{pgfscope}%
\begin{pgfscope}%
\definecolor{textcolor}{rgb}{0.000000,0.000000,0.000000}%
\pgfsetstrokecolor{textcolor}%
\pgfsetfillcolor{textcolor}%
\pgftext[x=1.624001in,y=0.402222in,,top]{\color{textcolor}\rmfamily\fontsize{10.000000}{12.000000}\selectfont 0.3}%
\end{pgfscope}%
\begin{pgfscope}%
\pgfsetbuttcap%
\pgfsetroundjoin%
\definecolor{currentfill}{rgb}{0.000000,0.000000,0.000000}%
\pgfsetfillcolor{currentfill}%
\pgfsetlinewidth{0.803000pt}%
\definecolor{currentstroke}{rgb}{0.000000,0.000000,0.000000}%
\pgfsetstrokecolor{currentstroke}%
\pgfsetdash{}{0pt}%
\pgfsys@defobject{currentmarker}{\pgfqpoint{0.000000in}{-0.048611in}}{\pgfqpoint{0.000000in}{0.000000in}}{%
\pgfpathmoveto{\pgfqpoint{0.000000in}{0.000000in}}%
\pgfpathlineto{\pgfqpoint{0.000000in}{-0.048611in}}%
\pgfusepath{stroke,fill}%
}%
\begin{pgfscope}%
\pgfsys@transformshift{1.969299in}{0.499444in}%
\pgfsys@useobject{currentmarker}{}%
\end{pgfscope}%
\end{pgfscope}%
\begin{pgfscope}%
\definecolor{textcolor}{rgb}{0.000000,0.000000,0.000000}%
\pgfsetstrokecolor{textcolor}%
\pgfsetfillcolor{textcolor}%
\pgftext[x=1.969299in,y=0.402222in,,top]{\color{textcolor}\rmfamily\fontsize{10.000000}{12.000000}\selectfont 0.4}%
\end{pgfscope}%
\begin{pgfscope}%
\pgfsetbuttcap%
\pgfsetroundjoin%
\definecolor{currentfill}{rgb}{0.000000,0.000000,0.000000}%
\pgfsetfillcolor{currentfill}%
\pgfsetlinewidth{0.803000pt}%
\definecolor{currentstroke}{rgb}{0.000000,0.000000,0.000000}%
\pgfsetstrokecolor{currentstroke}%
\pgfsetdash{}{0pt}%
\pgfsys@defobject{currentmarker}{\pgfqpoint{0.000000in}{-0.048611in}}{\pgfqpoint{0.000000in}{0.000000in}}{%
\pgfpathmoveto{\pgfqpoint{0.000000in}{0.000000in}}%
\pgfpathlineto{\pgfqpoint{0.000000in}{-0.048611in}}%
\pgfusepath{stroke,fill}%
}%
\begin{pgfscope}%
\pgfsys@transformshift{2.314596in}{0.499444in}%
\pgfsys@useobject{currentmarker}{}%
\end{pgfscope}%
\end{pgfscope}%
\begin{pgfscope}%
\definecolor{textcolor}{rgb}{0.000000,0.000000,0.000000}%
\pgfsetstrokecolor{textcolor}%
\pgfsetfillcolor{textcolor}%
\pgftext[x=2.314596in,y=0.402222in,,top]{\color{textcolor}\rmfamily\fontsize{10.000000}{12.000000}\selectfont 0.5}%
\end{pgfscope}%
\begin{pgfscope}%
\pgfsetbuttcap%
\pgfsetroundjoin%
\definecolor{currentfill}{rgb}{0.000000,0.000000,0.000000}%
\pgfsetfillcolor{currentfill}%
\pgfsetlinewidth{0.803000pt}%
\definecolor{currentstroke}{rgb}{0.000000,0.000000,0.000000}%
\pgfsetstrokecolor{currentstroke}%
\pgfsetdash{}{0pt}%
\pgfsys@defobject{currentmarker}{\pgfqpoint{0.000000in}{-0.048611in}}{\pgfqpoint{0.000000in}{0.000000in}}{%
\pgfpathmoveto{\pgfqpoint{0.000000in}{0.000000in}}%
\pgfpathlineto{\pgfqpoint{0.000000in}{-0.048611in}}%
\pgfusepath{stroke,fill}%
}%
\begin{pgfscope}%
\pgfsys@transformshift{2.659893in}{0.499444in}%
\pgfsys@useobject{currentmarker}{}%
\end{pgfscope}%
\end{pgfscope}%
\begin{pgfscope}%
\definecolor{textcolor}{rgb}{0.000000,0.000000,0.000000}%
\pgfsetstrokecolor{textcolor}%
\pgfsetfillcolor{textcolor}%
\pgftext[x=2.659893in,y=0.402222in,,top]{\color{textcolor}\rmfamily\fontsize{10.000000}{12.000000}\selectfont 0.6}%
\end{pgfscope}%
\begin{pgfscope}%
\pgfsetbuttcap%
\pgfsetroundjoin%
\definecolor{currentfill}{rgb}{0.000000,0.000000,0.000000}%
\pgfsetfillcolor{currentfill}%
\pgfsetlinewidth{0.803000pt}%
\definecolor{currentstroke}{rgb}{0.000000,0.000000,0.000000}%
\pgfsetstrokecolor{currentstroke}%
\pgfsetdash{}{0pt}%
\pgfsys@defobject{currentmarker}{\pgfqpoint{0.000000in}{-0.048611in}}{\pgfqpoint{0.000000in}{0.000000in}}{%
\pgfpathmoveto{\pgfqpoint{0.000000in}{0.000000in}}%
\pgfpathlineto{\pgfqpoint{0.000000in}{-0.048611in}}%
\pgfusepath{stroke,fill}%
}%
\begin{pgfscope}%
\pgfsys@transformshift{3.005190in}{0.499444in}%
\pgfsys@useobject{currentmarker}{}%
\end{pgfscope}%
\end{pgfscope}%
\begin{pgfscope}%
\definecolor{textcolor}{rgb}{0.000000,0.000000,0.000000}%
\pgfsetstrokecolor{textcolor}%
\pgfsetfillcolor{textcolor}%
\pgftext[x=3.005190in,y=0.402222in,,top]{\color{textcolor}\rmfamily\fontsize{10.000000}{12.000000}\selectfont 0.7}%
\end{pgfscope}%
\begin{pgfscope}%
\pgfsetbuttcap%
\pgfsetroundjoin%
\definecolor{currentfill}{rgb}{0.000000,0.000000,0.000000}%
\pgfsetfillcolor{currentfill}%
\pgfsetlinewidth{0.803000pt}%
\definecolor{currentstroke}{rgb}{0.000000,0.000000,0.000000}%
\pgfsetstrokecolor{currentstroke}%
\pgfsetdash{}{0pt}%
\pgfsys@defobject{currentmarker}{\pgfqpoint{0.000000in}{-0.048611in}}{\pgfqpoint{0.000000in}{0.000000in}}{%
\pgfpathmoveto{\pgfqpoint{0.000000in}{0.000000in}}%
\pgfpathlineto{\pgfqpoint{0.000000in}{-0.048611in}}%
\pgfusepath{stroke,fill}%
}%
\begin{pgfscope}%
\pgfsys@transformshift{3.350487in}{0.499444in}%
\pgfsys@useobject{currentmarker}{}%
\end{pgfscope}%
\end{pgfscope}%
\begin{pgfscope}%
\definecolor{textcolor}{rgb}{0.000000,0.000000,0.000000}%
\pgfsetstrokecolor{textcolor}%
\pgfsetfillcolor{textcolor}%
\pgftext[x=3.350487in,y=0.402222in,,top]{\color{textcolor}\rmfamily\fontsize{10.000000}{12.000000}\selectfont 0.8}%
\end{pgfscope}%
\begin{pgfscope}%
\pgfsetbuttcap%
\pgfsetroundjoin%
\definecolor{currentfill}{rgb}{0.000000,0.000000,0.000000}%
\pgfsetfillcolor{currentfill}%
\pgfsetlinewidth{0.803000pt}%
\definecolor{currentstroke}{rgb}{0.000000,0.000000,0.000000}%
\pgfsetstrokecolor{currentstroke}%
\pgfsetdash{}{0pt}%
\pgfsys@defobject{currentmarker}{\pgfqpoint{0.000000in}{-0.048611in}}{\pgfqpoint{0.000000in}{0.000000in}}{%
\pgfpathmoveto{\pgfqpoint{0.000000in}{0.000000in}}%
\pgfpathlineto{\pgfqpoint{0.000000in}{-0.048611in}}%
\pgfusepath{stroke,fill}%
}%
\begin{pgfscope}%
\pgfsys@transformshift{3.695784in}{0.499444in}%
\pgfsys@useobject{currentmarker}{}%
\end{pgfscope}%
\end{pgfscope}%
\begin{pgfscope}%
\definecolor{textcolor}{rgb}{0.000000,0.000000,0.000000}%
\pgfsetstrokecolor{textcolor}%
\pgfsetfillcolor{textcolor}%
\pgftext[x=3.695784in,y=0.402222in,,top]{\color{textcolor}\rmfamily\fontsize{10.000000}{12.000000}\selectfont 0.9}%
\end{pgfscope}%
\begin{pgfscope}%
\pgfsetbuttcap%
\pgfsetroundjoin%
\definecolor{currentfill}{rgb}{0.000000,0.000000,0.000000}%
\pgfsetfillcolor{currentfill}%
\pgfsetlinewidth{0.803000pt}%
\definecolor{currentstroke}{rgb}{0.000000,0.000000,0.000000}%
\pgfsetstrokecolor{currentstroke}%
\pgfsetdash{}{0pt}%
\pgfsys@defobject{currentmarker}{\pgfqpoint{0.000000in}{-0.048611in}}{\pgfqpoint{0.000000in}{0.000000in}}{%
\pgfpathmoveto{\pgfqpoint{0.000000in}{0.000000in}}%
\pgfpathlineto{\pgfqpoint{0.000000in}{-0.048611in}}%
\pgfusepath{stroke,fill}%
}%
\begin{pgfscope}%
\pgfsys@transformshift{4.041081in}{0.499444in}%
\pgfsys@useobject{currentmarker}{}%
\end{pgfscope}%
\end{pgfscope}%
\begin{pgfscope}%
\definecolor{textcolor}{rgb}{0.000000,0.000000,0.000000}%
\pgfsetstrokecolor{textcolor}%
\pgfsetfillcolor{textcolor}%
\pgftext[x=4.041081in,y=0.402222in,,top]{\color{textcolor}\rmfamily\fontsize{10.000000}{12.000000}\selectfont 1.0}%
\end{pgfscope}%
\begin{pgfscope}%
\definecolor{textcolor}{rgb}{0.000000,0.000000,0.000000}%
\pgfsetstrokecolor{textcolor}%
\pgfsetfillcolor{textcolor}%
\pgftext[x=2.297331in,y=0.223333in,,top]{\color{textcolor}\rmfamily\fontsize{10.000000}{12.000000}\selectfont \(\displaystyle p\)}%
\end{pgfscope}%
\begin{pgfscope}%
\pgfsetbuttcap%
\pgfsetroundjoin%
\definecolor{currentfill}{rgb}{0.000000,0.000000,0.000000}%
\pgfsetfillcolor{currentfill}%
\pgfsetlinewidth{0.803000pt}%
\definecolor{currentstroke}{rgb}{0.000000,0.000000,0.000000}%
\pgfsetstrokecolor{currentstroke}%
\pgfsetdash{}{0pt}%
\pgfsys@defobject{currentmarker}{\pgfqpoint{-0.048611in}{0.000000in}}{\pgfqpoint{-0.000000in}{0.000000in}}{%
\pgfpathmoveto{\pgfqpoint{-0.000000in}{0.000000in}}%
\pgfpathlineto{\pgfqpoint{-0.048611in}{0.000000in}}%
\pgfusepath{stroke,fill}%
}%
\begin{pgfscope}%
\pgfsys@transformshift{0.553581in}{0.499444in}%
\pgfsys@useobject{currentmarker}{}%
\end{pgfscope}%
\end{pgfscope}%
\begin{pgfscope}%
\definecolor{textcolor}{rgb}{0.000000,0.000000,0.000000}%
\pgfsetstrokecolor{textcolor}%
\pgfsetfillcolor{textcolor}%
\pgftext[x=0.278889in, y=0.451250in, left, base]{\color{textcolor}\rmfamily\fontsize{10.000000}{12.000000}\selectfont \(\displaystyle {0.0}\)}%
\end{pgfscope}%
\begin{pgfscope}%
\pgfsetbuttcap%
\pgfsetroundjoin%
\definecolor{currentfill}{rgb}{0.000000,0.000000,0.000000}%
\pgfsetfillcolor{currentfill}%
\pgfsetlinewidth{0.803000pt}%
\definecolor{currentstroke}{rgb}{0.000000,0.000000,0.000000}%
\pgfsetstrokecolor{currentstroke}%
\pgfsetdash{}{0pt}%
\pgfsys@defobject{currentmarker}{\pgfqpoint{-0.048611in}{0.000000in}}{\pgfqpoint{-0.000000in}{0.000000in}}{%
\pgfpathmoveto{\pgfqpoint{-0.000000in}{0.000000in}}%
\pgfpathlineto{\pgfqpoint{-0.048611in}{0.000000in}}%
\pgfusepath{stroke,fill}%
}%
\begin{pgfscope}%
\pgfsys@transformshift{0.553581in}{0.838690in}%
\pgfsys@useobject{currentmarker}{}%
\end{pgfscope}%
\end{pgfscope}%
\begin{pgfscope}%
\definecolor{textcolor}{rgb}{0.000000,0.000000,0.000000}%
\pgfsetstrokecolor{textcolor}%
\pgfsetfillcolor{textcolor}%
\pgftext[x=0.278889in, y=0.790495in, left, base]{\color{textcolor}\rmfamily\fontsize{10.000000}{12.000000}\selectfont \(\displaystyle {2.5}\)}%
\end{pgfscope}%
\begin{pgfscope}%
\pgfsetbuttcap%
\pgfsetroundjoin%
\definecolor{currentfill}{rgb}{0.000000,0.000000,0.000000}%
\pgfsetfillcolor{currentfill}%
\pgfsetlinewidth{0.803000pt}%
\definecolor{currentstroke}{rgb}{0.000000,0.000000,0.000000}%
\pgfsetstrokecolor{currentstroke}%
\pgfsetdash{}{0pt}%
\pgfsys@defobject{currentmarker}{\pgfqpoint{-0.048611in}{0.000000in}}{\pgfqpoint{-0.000000in}{0.000000in}}{%
\pgfpathmoveto{\pgfqpoint{-0.000000in}{0.000000in}}%
\pgfpathlineto{\pgfqpoint{-0.048611in}{0.000000in}}%
\pgfusepath{stroke,fill}%
}%
\begin{pgfscope}%
\pgfsys@transformshift{0.553581in}{1.177935in}%
\pgfsys@useobject{currentmarker}{}%
\end{pgfscope}%
\end{pgfscope}%
\begin{pgfscope}%
\definecolor{textcolor}{rgb}{0.000000,0.000000,0.000000}%
\pgfsetstrokecolor{textcolor}%
\pgfsetfillcolor{textcolor}%
\pgftext[x=0.278889in, y=1.129740in, left, base]{\color{textcolor}\rmfamily\fontsize{10.000000}{12.000000}\selectfont \(\displaystyle {5.0}\)}%
\end{pgfscope}%
\begin{pgfscope}%
\pgfsetbuttcap%
\pgfsetroundjoin%
\definecolor{currentfill}{rgb}{0.000000,0.000000,0.000000}%
\pgfsetfillcolor{currentfill}%
\pgfsetlinewidth{0.803000pt}%
\definecolor{currentstroke}{rgb}{0.000000,0.000000,0.000000}%
\pgfsetstrokecolor{currentstroke}%
\pgfsetdash{}{0pt}%
\pgfsys@defobject{currentmarker}{\pgfqpoint{-0.048611in}{0.000000in}}{\pgfqpoint{-0.000000in}{0.000000in}}{%
\pgfpathmoveto{\pgfqpoint{-0.000000in}{0.000000in}}%
\pgfpathlineto{\pgfqpoint{-0.048611in}{0.000000in}}%
\pgfusepath{stroke,fill}%
}%
\begin{pgfscope}%
\pgfsys@transformshift{0.553581in}{1.517180in}%
\pgfsys@useobject{currentmarker}{}%
\end{pgfscope}%
\end{pgfscope}%
\begin{pgfscope}%
\definecolor{textcolor}{rgb}{0.000000,0.000000,0.000000}%
\pgfsetstrokecolor{textcolor}%
\pgfsetfillcolor{textcolor}%
\pgftext[x=0.278889in, y=1.468986in, left, base]{\color{textcolor}\rmfamily\fontsize{10.000000}{12.000000}\selectfont \(\displaystyle {7.5}\)}%
\end{pgfscope}%
\begin{pgfscope}%
\definecolor{textcolor}{rgb}{0.000000,0.000000,0.000000}%
\pgfsetstrokecolor{textcolor}%
\pgfsetfillcolor{textcolor}%
\pgftext[x=0.223333in,y=1.076944in,,bottom,rotate=90.000000]{\color{textcolor}\rmfamily\fontsize{10.000000}{12.000000}\selectfont Percent of Data Set}%
\end{pgfscope}%
\begin{pgfscope}%
\pgfsetrectcap%
\pgfsetmiterjoin%
\pgfsetlinewidth{0.803000pt}%
\definecolor{currentstroke}{rgb}{0.000000,0.000000,0.000000}%
\pgfsetstrokecolor{currentstroke}%
\pgfsetdash{}{0pt}%
\pgfpathmoveto{\pgfqpoint{0.553581in}{0.499444in}}%
\pgfpathlineto{\pgfqpoint{0.553581in}{1.654444in}}%
\pgfusepath{stroke}%
\end{pgfscope}%
\begin{pgfscope}%
\pgfsetrectcap%
\pgfsetmiterjoin%
\pgfsetlinewidth{0.803000pt}%
\definecolor{currentstroke}{rgb}{0.000000,0.000000,0.000000}%
\pgfsetstrokecolor{currentstroke}%
\pgfsetdash{}{0pt}%
\pgfpathmoveto{\pgfqpoint{4.041081in}{0.499444in}}%
\pgfpathlineto{\pgfqpoint{4.041081in}{1.654444in}}%
\pgfusepath{stroke}%
\end{pgfscope}%
\begin{pgfscope}%
\pgfsetrectcap%
\pgfsetmiterjoin%
\pgfsetlinewidth{0.803000pt}%
\definecolor{currentstroke}{rgb}{0.000000,0.000000,0.000000}%
\pgfsetstrokecolor{currentstroke}%
\pgfsetdash{}{0pt}%
\pgfpathmoveto{\pgfqpoint{0.553581in}{0.499444in}}%
\pgfpathlineto{\pgfqpoint{4.041081in}{0.499444in}}%
\pgfusepath{stroke}%
\end{pgfscope}%
\begin{pgfscope}%
\pgfsetrectcap%
\pgfsetmiterjoin%
\pgfsetlinewidth{0.803000pt}%
\definecolor{currentstroke}{rgb}{0.000000,0.000000,0.000000}%
\pgfsetstrokecolor{currentstroke}%
\pgfsetdash{}{0pt}%
\pgfpathmoveto{\pgfqpoint{0.553581in}{1.654444in}}%
\pgfpathlineto{\pgfqpoint{4.041081in}{1.654444in}}%
\pgfusepath{stroke}%
\end{pgfscope}%
\begin{pgfscope}%
\pgfsetbuttcap%
\pgfsetmiterjoin%
\definecolor{currentfill}{rgb}{1.000000,1.000000,1.000000}%
\pgfsetfillcolor{currentfill}%
\pgfsetfillopacity{0.800000}%
\pgfsetlinewidth{1.003750pt}%
\definecolor{currentstroke}{rgb}{0.800000,0.800000,0.800000}%
\pgfsetstrokecolor{currentstroke}%
\pgfsetstrokeopacity{0.800000}%
\pgfsetdash{}{0pt}%
\pgfpathmoveto{\pgfqpoint{3.264136in}{1.154445in}}%
\pgfpathlineto{\pgfqpoint{3.943858in}{1.154445in}}%
\pgfpathquadraticcurveto{\pgfqpoint{3.971636in}{1.154445in}}{\pgfqpoint{3.971636in}{1.182222in}}%
\pgfpathlineto{\pgfqpoint{3.971636in}{1.557222in}}%
\pgfpathquadraticcurveto{\pgfqpoint{3.971636in}{1.585000in}}{\pgfqpoint{3.943858in}{1.585000in}}%
\pgfpathlineto{\pgfqpoint{3.264136in}{1.585000in}}%
\pgfpathquadraticcurveto{\pgfqpoint{3.236358in}{1.585000in}}{\pgfqpoint{3.236358in}{1.557222in}}%
\pgfpathlineto{\pgfqpoint{3.236358in}{1.182222in}}%
\pgfpathquadraticcurveto{\pgfqpoint{3.236358in}{1.154445in}}{\pgfqpoint{3.264136in}{1.154445in}}%
\pgfpathlineto{\pgfqpoint{3.264136in}{1.154445in}}%
\pgfpathclose%
\pgfusepath{stroke,fill}%
\end{pgfscope}%
\begin{pgfscope}%
\pgfsetbuttcap%
\pgfsetmiterjoin%
\pgfsetlinewidth{1.003750pt}%
\definecolor{currentstroke}{rgb}{0.000000,0.000000,0.000000}%
\pgfsetstrokecolor{currentstroke}%
\pgfsetdash{}{0pt}%
\pgfpathmoveto{\pgfqpoint{3.291914in}{1.432222in}}%
\pgfpathlineto{\pgfqpoint{3.569692in}{1.432222in}}%
\pgfpathlineto{\pgfqpoint{3.569692in}{1.529444in}}%
\pgfpathlineto{\pgfqpoint{3.291914in}{1.529444in}}%
\pgfpathlineto{\pgfqpoint{3.291914in}{1.432222in}}%
\pgfpathclose%
\pgfusepath{stroke}%
\end{pgfscope}%
\begin{pgfscope}%
\definecolor{textcolor}{rgb}{0.000000,0.000000,0.000000}%
\pgfsetstrokecolor{textcolor}%
\pgfsetfillcolor{textcolor}%
\pgftext[x=3.680803in,y=1.432222in,left,base]{\color{textcolor}\rmfamily\fontsize{10.000000}{12.000000}\selectfont Neg}%
\end{pgfscope}%
\begin{pgfscope}%
\pgfsetbuttcap%
\pgfsetmiterjoin%
\definecolor{currentfill}{rgb}{0.000000,0.000000,0.000000}%
\pgfsetfillcolor{currentfill}%
\pgfsetlinewidth{0.000000pt}%
\definecolor{currentstroke}{rgb}{0.000000,0.000000,0.000000}%
\pgfsetstrokecolor{currentstroke}%
\pgfsetstrokeopacity{0.000000}%
\pgfsetdash{}{0pt}%
\pgfpathmoveto{\pgfqpoint{3.291914in}{1.236944in}}%
\pgfpathlineto{\pgfqpoint{3.569692in}{1.236944in}}%
\pgfpathlineto{\pgfqpoint{3.569692in}{1.334167in}}%
\pgfpathlineto{\pgfqpoint{3.291914in}{1.334167in}}%
\pgfpathlineto{\pgfqpoint{3.291914in}{1.236944in}}%
\pgfpathclose%
\pgfusepath{fill}%
\end{pgfscope}%
\begin{pgfscope}%
\definecolor{textcolor}{rgb}{0.000000,0.000000,0.000000}%
\pgfsetstrokecolor{textcolor}%
\pgfsetfillcolor{textcolor}%
\pgftext[x=3.680803in,y=1.236944in,left,base]{\color{textcolor}\rmfamily\fontsize{10.000000}{12.000000}\selectfont Pos}%
\end{pgfscope}%
\end{pgfpicture}%
\makeatother%
\endgroup%
	
&
	\vskip 0pt
	\hfil ROC Curve
	
	%% Creator: Matplotlib, PGF backend
%%
%% To include the figure in your LaTeX document, write
%%   \input{<filename>.pgf}
%%
%% Make sure the required packages are loaded in your preamble
%%   \usepackage{pgf}
%%
%% Also ensure that all the required font packages are loaded; for instance,
%% the lmodern package is sometimes necessary when using math font.
%%   \usepackage{lmodern}
%%
%% Figures using additional raster images can only be included by \input if
%% they are in the same directory as the main LaTeX file. For loading figures
%% from other directories you can use the `import` package
%%   \usepackage{import}
%%
%% and then include the figures with
%%   \import{<path to file>}{<filename>.pgf}
%%
%% Matplotlib used the following preamble
%%   
%%   \usepackage{fontspec}
%%   \makeatletter\@ifpackageloaded{underscore}{}{\usepackage[strings]{underscore}}\makeatother
%%
\begingroup%
\makeatletter%
\begin{pgfpicture}%
\pgfpathrectangle{\pgfpointorigin}{\pgfqpoint{2.221861in}{1.754444in}}%
\pgfusepath{use as bounding box, clip}%
\begin{pgfscope}%
\pgfsetbuttcap%
\pgfsetmiterjoin%
\definecolor{currentfill}{rgb}{1.000000,1.000000,1.000000}%
\pgfsetfillcolor{currentfill}%
\pgfsetlinewidth{0.000000pt}%
\definecolor{currentstroke}{rgb}{1.000000,1.000000,1.000000}%
\pgfsetstrokecolor{currentstroke}%
\pgfsetdash{}{0pt}%
\pgfpathmoveto{\pgfqpoint{0.000000in}{0.000000in}}%
\pgfpathlineto{\pgfqpoint{2.221861in}{0.000000in}}%
\pgfpathlineto{\pgfqpoint{2.221861in}{1.754444in}}%
\pgfpathlineto{\pgfqpoint{0.000000in}{1.754444in}}%
\pgfpathlineto{\pgfqpoint{0.000000in}{0.000000in}}%
\pgfpathclose%
\pgfusepath{fill}%
\end{pgfscope}%
\begin{pgfscope}%
\pgfsetbuttcap%
\pgfsetmiterjoin%
\definecolor{currentfill}{rgb}{1.000000,1.000000,1.000000}%
\pgfsetfillcolor{currentfill}%
\pgfsetlinewidth{0.000000pt}%
\definecolor{currentstroke}{rgb}{0.000000,0.000000,0.000000}%
\pgfsetstrokecolor{currentstroke}%
\pgfsetstrokeopacity{0.000000}%
\pgfsetdash{}{0pt}%
\pgfpathmoveto{\pgfqpoint{0.553581in}{0.499444in}}%
\pgfpathlineto{\pgfqpoint{2.103581in}{0.499444in}}%
\pgfpathlineto{\pgfqpoint{2.103581in}{1.654444in}}%
\pgfpathlineto{\pgfqpoint{0.553581in}{1.654444in}}%
\pgfpathlineto{\pgfqpoint{0.553581in}{0.499444in}}%
\pgfpathclose%
\pgfusepath{fill}%
\end{pgfscope}%
\begin{pgfscope}%
\pgfsetbuttcap%
\pgfsetroundjoin%
\definecolor{currentfill}{rgb}{0.000000,0.000000,0.000000}%
\pgfsetfillcolor{currentfill}%
\pgfsetlinewidth{0.803000pt}%
\definecolor{currentstroke}{rgb}{0.000000,0.000000,0.000000}%
\pgfsetstrokecolor{currentstroke}%
\pgfsetdash{}{0pt}%
\pgfsys@defobject{currentmarker}{\pgfqpoint{0.000000in}{-0.048611in}}{\pgfqpoint{0.000000in}{0.000000in}}{%
\pgfpathmoveto{\pgfqpoint{0.000000in}{0.000000in}}%
\pgfpathlineto{\pgfqpoint{0.000000in}{-0.048611in}}%
\pgfusepath{stroke,fill}%
}%
\begin{pgfscope}%
\pgfsys@transformshift{0.624035in}{0.499444in}%
\pgfsys@useobject{currentmarker}{}%
\end{pgfscope}%
\end{pgfscope}%
\begin{pgfscope}%
\definecolor{textcolor}{rgb}{0.000000,0.000000,0.000000}%
\pgfsetstrokecolor{textcolor}%
\pgfsetfillcolor{textcolor}%
\pgftext[x=0.624035in,y=0.402222in,,top]{\color{textcolor}\rmfamily\fontsize{10.000000}{12.000000}\selectfont \(\displaystyle {0.0}\)}%
\end{pgfscope}%
\begin{pgfscope}%
\pgfsetbuttcap%
\pgfsetroundjoin%
\definecolor{currentfill}{rgb}{0.000000,0.000000,0.000000}%
\pgfsetfillcolor{currentfill}%
\pgfsetlinewidth{0.803000pt}%
\definecolor{currentstroke}{rgb}{0.000000,0.000000,0.000000}%
\pgfsetstrokecolor{currentstroke}%
\pgfsetdash{}{0pt}%
\pgfsys@defobject{currentmarker}{\pgfqpoint{0.000000in}{-0.048611in}}{\pgfqpoint{0.000000in}{0.000000in}}{%
\pgfpathmoveto{\pgfqpoint{0.000000in}{0.000000in}}%
\pgfpathlineto{\pgfqpoint{0.000000in}{-0.048611in}}%
\pgfusepath{stroke,fill}%
}%
\begin{pgfscope}%
\pgfsys@transformshift{1.328581in}{0.499444in}%
\pgfsys@useobject{currentmarker}{}%
\end{pgfscope}%
\end{pgfscope}%
\begin{pgfscope}%
\definecolor{textcolor}{rgb}{0.000000,0.000000,0.000000}%
\pgfsetstrokecolor{textcolor}%
\pgfsetfillcolor{textcolor}%
\pgftext[x=1.328581in,y=0.402222in,,top]{\color{textcolor}\rmfamily\fontsize{10.000000}{12.000000}\selectfont \(\displaystyle {0.5}\)}%
\end{pgfscope}%
\begin{pgfscope}%
\pgfsetbuttcap%
\pgfsetroundjoin%
\definecolor{currentfill}{rgb}{0.000000,0.000000,0.000000}%
\pgfsetfillcolor{currentfill}%
\pgfsetlinewidth{0.803000pt}%
\definecolor{currentstroke}{rgb}{0.000000,0.000000,0.000000}%
\pgfsetstrokecolor{currentstroke}%
\pgfsetdash{}{0pt}%
\pgfsys@defobject{currentmarker}{\pgfqpoint{0.000000in}{-0.048611in}}{\pgfqpoint{0.000000in}{0.000000in}}{%
\pgfpathmoveto{\pgfqpoint{0.000000in}{0.000000in}}%
\pgfpathlineto{\pgfqpoint{0.000000in}{-0.048611in}}%
\pgfusepath{stroke,fill}%
}%
\begin{pgfscope}%
\pgfsys@transformshift{2.033126in}{0.499444in}%
\pgfsys@useobject{currentmarker}{}%
\end{pgfscope}%
\end{pgfscope}%
\begin{pgfscope}%
\definecolor{textcolor}{rgb}{0.000000,0.000000,0.000000}%
\pgfsetstrokecolor{textcolor}%
\pgfsetfillcolor{textcolor}%
\pgftext[x=2.033126in,y=0.402222in,,top]{\color{textcolor}\rmfamily\fontsize{10.000000}{12.000000}\selectfont \(\displaystyle {1.0}\)}%
\end{pgfscope}%
\begin{pgfscope}%
\definecolor{textcolor}{rgb}{0.000000,0.000000,0.000000}%
\pgfsetstrokecolor{textcolor}%
\pgfsetfillcolor{textcolor}%
\pgftext[x=1.328581in,y=0.223333in,,top]{\color{textcolor}\rmfamily\fontsize{10.000000}{12.000000}\selectfont False positive rate}%
\end{pgfscope}%
\begin{pgfscope}%
\pgfsetbuttcap%
\pgfsetroundjoin%
\definecolor{currentfill}{rgb}{0.000000,0.000000,0.000000}%
\pgfsetfillcolor{currentfill}%
\pgfsetlinewidth{0.803000pt}%
\definecolor{currentstroke}{rgb}{0.000000,0.000000,0.000000}%
\pgfsetstrokecolor{currentstroke}%
\pgfsetdash{}{0pt}%
\pgfsys@defobject{currentmarker}{\pgfqpoint{-0.048611in}{0.000000in}}{\pgfqpoint{-0.000000in}{0.000000in}}{%
\pgfpathmoveto{\pgfqpoint{-0.000000in}{0.000000in}}%
\pgfpathlineto{\pgfqpoint{-0.048611in}{0.000000in}}%
\pgfusepath{stroke,fill}%
}%
\begin{pgfscope}%
\pgfsys@transformshift{0.553581in}{0.551944in}%
\pgfsys@useobject{currentmarker}{}%
\end{pgfscope}%
\end{pgfscope}%
\begin{pgfscope}%
\definecolor{textcolor}{rgb}{0.000000,0.000000,0.000000}%
\pgfsetstrokecolor{textcolor}%
\pgfsetfillcolor{textcolor}%
\pgftext[x=0.278889in, y=0.503750in, left, base]{\color{textcolor}\rmfamily\fontsize{10.000000}{12.000000}\selectfont \(\displaystyle {0.0}\)}%
\end{pgfscope}%
\begin{pgfscope}%
\pgfsetbuttcap%
\pgfsetroundjoin%
\definecolor{currentfill}{rgb}{0.000000,0.000000,0.000000}%
\pgfsetfillcolor{currentfill}%
\pgfsetlinewidth{0.803000pt}%
\definecolor{currentstroke}{rgb}{0.000000,0.000000,0.000000}%
\pgfsetstrokecolor{currentstroke}%
\pgfsetdash{}{0pt}%
\pgfsys@defobject{currentmarker}{\pgfqpoint{-0.048611in}{0.000000in}}{\pgfqpoint{-0.000000in}{0.000000in}}{%
\pgfpathmoveto{\pgfqpoint{-0.000000in}{0.000000in}}%
\pgfpathlineto{\pgfqpoint{-0.048611in}{0.000000in}}%
\pgfusepath{stroke,fill}%
}%
\begin{pgfscope}%
\pgfsys@transformshift{0.553581in}{1.076944in}%
\pgfsys@useobject{currentmarker}{}%
\end{pgfscope}%
\end{pgfscope}%
\begin{pgfscope}%
\definecolor{textcolor}{rgb}{0.000000,0.000000,0.000000}%
\pgfsetstrokecolor{textcolor}%
\pgfsetfillcolor{textcolor}%
\pgftext[x=0.278889in, y=1.028750in, left, base]{\color{textcolor}\rmfamily\fontsize{10.000000}{12.000000}\selectfont \(\displaystyle {0.5}\)}%
\end{pgfscope}%
\begin{pgfscope}%
\pgfsetbuttcap%
\pgfsetroundjoin%
\definecolor{currentfill}{rgb}{0.000000,0.000000,0.000000}%
\pgfsetfillcolor{currentfill}%
\pgfsetlinewidth{0.803000pt}%
\definecolor{currentstroke}{rgb}{0.000000,0.000000,0.000000}%
\pgfsetstrokecolor{currentstroke}%
\pgfsetdash{}{0pt}%
\pgfsys@defobject{currentmarker}{\pgfqpoint{-0.048611in}{0.000000in}}{\pgfqpoint{-0.000000in}{0.000000in}}{%
\pgfpathmoveto{\pgfqpoint{-0.000000in}{0.000000in}}%
\pgfpathlineto{\pgfqpoint{-0.048611in}{0.000000in}}%
\pgfusepath{stroke,fill}%
}%
\begin{pgfscope}%
\pgfsys@transformshift{0.553581in}{1.601944in}%
\pgfsys@useobject{currentmarker}{}%
\end{pgfscope}%
\end{pgfscope}%
\begin{pgfscope}%
\definecolor{textcolor}{rgb}{0.000000,0.000000,0.000000}%
\pgfsetstrokecolor{textcolor}%
\pgfsetfillcolor{textcolor}%
\pgftext[x=0.278889in, y=1.553750in, left, base]{\color{textcolor}\rmfamily\fontsize{10.000000}{12.000000}\selectfont \(\displaystyle {1.0}\)}%
\end{pgfscope}%
\begin{pgfscope}%
\definecolor{textcolor}{rgb}{0.000000,0.000000,0.000000}%
\pgfsetstrokecolor{textcolor}%
\pgfsetfillcolor{textcolor}%
\pgftext[x=0.223333in,y=1.076944in,,bottom,rotate=90.000000]{\color{textcolor}\rmfamily\fontsize{10.000000}{12.000000}\selectfont True positive rate}%
\end{pgfscope}%
\begin{pgfscope}%
\pgfpathrectangle{\pgfqpoint{0.553581in}{0.499444in}}{\pgfqpoint{1.550000in}{1.155000in}}%
\pgfusepath{clip}%
\pgfsetbuttcap%
\pgfsetroundjoin%
\pgfsetlinewidth{1.505625pt}%
\definecolor{currentstroke}{rgb}{0.000000,0.000000,0.000000}%
\pgfsetstrokecolor{currentstroke}%
\pgfsetdash{{5.550000pt}{2.400000pt}}{0.000000pt}%
\pgfpathmoveto{\pgfqpoint{0.624035in}{0.551944in}}%
\pgfpathlineto{\pgfqpoint{2.033126in}{1.601944in}}%
\pgfusepath{stroke}%
\end{pgfscope}%
\begin{pgfscope}%
\pgfpathrectangle{\pgfqpoint{0.553581in}{0.499444in}}{\pgfqpoint{1.550000in}{1.155000in}}%
\pgfusepath{clip}%
\pgfsetrectcap%
\pgfsetroundjoin%
\pgfsetlinewidth{1.505625pt}%
\definecolor{currentstroke}{rgb}{0.000000,0.000000,0.000000}%
\pgfsetstrokecolor{currentstroke}%
\pgfsetdash{}{0pt}%
\pgfpathmoveto{\pgfqpoint{0.624035in}{0.551944in}}%
\pgfpathlineto{\pgfqpoint{0.625145in}{0.568645in}}%
\pgfpathlineto{\pgfqpoint{0.625278in}{0.569700in}}%
\pgfpathlineto{\pgfqpoint{0.626388in}{0.582738in}}%
\pgfpathlineto{\pgfqpoint{0.626435in}{0.583359in}}%
\pgfpathlineto{\pgfqpoint{0.627545in}{0.594565in}}%
\pgfpathlineto{\pgfqpoint{0.627647in}{0.595527in}}%
\pgfpathlineto{\pgfqpoint{0.628749in}{0.605864in}}%
\pgfpathlineto{\pgfqpoint{0.628960in}{0.606796in}}%
\pgfpathlineto{\pgfqpoint{0.630070in}{0.614898in}}%
\pgfpathlineto{\pgfqpoint{0.630188in}{0.615860in}}%
\pgfpathlineto{\pgfqpoint{0.631298in}{0.623776in}}%
\pgfpathlineto{\pgfqpoint{0.631493in}{0.624862in}}%
\pgfpathlineto{\pgfqpoint{0.632603in}{0.632498in}}%
\pgfpathlineto{\pgfqpoint{0.632799in}{0.633523in}}%
\pgfpathlineto{\pgfqpoint{0.633909in}{0.641377in}}%
\pgfpathlineto{\pgfqpoint{0.634183in}{0.642463in}}%
\pgfpathlineto{\pgfqpoint{0.635293in}{0.649416in}}%
\pgfpathlineto{\pgfqpoint{0.635472in}{0.650472in}}%
\pgfpathlineto{\pgfqpoint{0.636583in}{0.657115in}}%
\pgfpathlineto{\pgfqpoint{0.636700in}{0.658108in}}%
\pgfpathlineto{\pgfqpoint{0.637802in}{0.663385in}}%
\pgfpathlineto{\pgfqpoint{0.638044in}{0.664472in}}%
\pgfpathlineto{\pgfqpoint{0.639147in}{0.670153in}}%
\pgfpathlineto{\pgfqpoint{0.639334in}{0.671208in}}%
\pgfpathlineto{\pgfqpoint{0.640444in}{0.676206in}}%
\pgfpathlineto{\pgfqpoint{0.640624in}{0.677261in}}%
\pgfpathlineto{\pgfqpoint{0.641734in}{0.683159in}}%
\pgfpathlineto{\pgfqpoint{0.641985in}{0.684246in}}%
\pgfpathlineto{\pgfqpoint{0.643095in}{0.689585in}}%
\pgfpathlineto{\pgfqpoint{0.643360in}{0.690671in}}%
\pgfpathlineto{\pgfqpoint{0.644471in}{0.695669in}}%
\pgfpathlineto{\pgfqpoint{0.644775in}{0.696725in}}%
\pgfpathlineto{\pgfqpoint{0.645862in}{0.701381in}}%
\pgfpathlineto{\pgfqpoint{0.646144in}{0.702467in}}%
\pgfpathlineto{\pgfqpoint{0.647254in}{0.707807in}}%
\pgfpathlineto{\pgfqpoint{0.647449in}{0.708800in}}%
\pgfpathlineto{\pgfqpoint{0.648551in}{0.713332in}}%
\pgfpathlineto{\pgfqpoint{0.648856in}{0.714419in}}%
\pgfpathlineto{\pgfqpoint{0.649966in}{0.718206in}}%
\pgfpathlineto{\pgfqpoint{0.650263in}{0.719261in}}%
\pgfpathlineto{\pgfqpoint{0.651374in}{0.723669in}}%
\pgfpathlineto{\pgfqpoint{0.651749in}{0.724538in}}%
\pgfpathlineto{\pgfqpoint{0.652859in}{0.730126in}}%
\pgfpathlineto{\pgfqpoint{0.653117in}{0.731181in}}%
\pgfpathlineto{\pgfqpoint{0.654211in}{0.735248in}}%
\pgfpathlineto{\pgfqpoint{0.654665in}{0.736334in}}%
\pgfpathlineto{\pgfqpoint{0.655767in}{0.739811in}}%
\pgfpathlineto{\pgfqpoint{0.656103in}{0.740898in}}%
\pgfpathlineto{\pgfqpoint{0.657198in}{0.745337in}}%
\pgfpathlineto{\pgfqpoint{0.657675in}{0.746423in}}%
\pgfpathlineto{\pgfqpoint{0.658777in}{0.751017in}}%
\pgfpathlineto{\pgfqpoint{0.659027in}{0.752104in}}%
\pgfpathlineto{\pgfqpoint{0.660137in}{0.755798in}}%
\pgfpathlineto{\pgfqpoint{0.660458in}{0.756853in}}%
\pgfpathlineto{\pgfqpoint{0.661568in}{0.760423in}}%
\pgfpathlineto{\pgfqpoint{0.661912in}{0.761510in}}%
\pgfpathlineto{\pgfqpoint{0.663014in}{0.765824in}}%
\pgfpathlineto{\pgfqpoint{0.663405in}{0.766880in}}%
\pgfpathlineto{\pgfqpoint{0.664507in}{0.770357in}}%
\pgfpathlineto{\pgfqpoint{0.664757in}{0.771412in}}%
\pgfpathlineto{\pgfqpoint{0.665867in}{0.775479in}}%
\pgfpathlineto{\pgfqpoint{0.666251in}{0.776565in}}%
\pgfpathlineto{\pgfqpoint{0.667353in}{0.779576in}}%
\pgfpathlineto{\pgfqpoint{0.667681in}{0.780538in}}%
\pgfpathlineto{\pgfqpoint{0.668791in}{0.783363in}}%
\pgfpathlineto{\pgfqpoint{0.669213in}{0.784450in}}%
\pgfpathlineto{\pgfqpoint{0.670323in}{0.788175in}}%
\pgfpathlineto{\pgfqpoint{0.670761in}{0.789261in}}%
\pgfpathlineto{\pgfqpoint{0.671864in}{0.792241in}}%
\pgfpathlineto{\pgfqpoint{0.672129in}{0.793266in}}%
\pgfpathlineto{\pgfqpoint{0.673239in}{0.797456in}}%
\pgfpathlineto{\pgfqpoint{0.673497in}{0.798543in}}%
\pgfpathlineto{\pgfqpoint{0.674608in}{0.801430in}}%
\pgfpathlineto{\pgfqpoint{0.674936in}{0.802516in}}%
\pgfpathlineto{\pgfqpoint{0.676038in}{0.806117in}}%
\pgfpathlineto{\pgfqpoint{0.676585in}{0.807204in}}%
\pgfpathlineto{\pgfqpoint{0.677680in}{0.809873in}}%
\pgfpathlineto{\pgfqpoint{0.678000in}{0.810960in}}%
\pgfpathlineto{\pgfqpoint{0.679095in}{0.813381in}}%
\pgfpathlineto{\pgfqpoint{0.679759in}{0.814467in}}%
\pgfpathlineto{\pgfqpoint{0.680854in}{0.817758in}}%
\pgfpathlineto{\pgfqpoint{0.681362in}{0.818844in}}%
\pgfpathlineto{\pgfqpoint{0.682441in}{0.821452in}}%
\pgfpathlineto{\pgfqpoint{0.682824in}{0.822538in}}%
\pgfpathlineto{\pgfqpoint{0.683934in}{0.825736in}}%
\pgfpathlineto{\pgfqpoint{0.684356in}{0.826791in}}%
\pgfpathlineto{\pgfqpoint{0.685435in}{0.829585in}}%
\pgfpathlineto{\pgfqpoint{0.685967in}{0.830609in}}%
\pgfpathlineto{\pgfqpoint{0.687077in}{0.833589in}}%
\pgfpathlineto{\pgfqpoint{0.687577in}{0.834676in}}%
\pgfpathlineto{\pgfqpoint{0.688656in}{0.837128in}}%
\pgfpathlineto{\pgfqpoint{0.689117in}{0.838215in}}%
\pgfpathlineto{\pgfqpoint{0.690227in}{0.840698in}}%
\pgfpathlineto{\pgfqpoint{0.690860in}{0.841785in}}%
\pgfpathlineto{\pgfqpoint{0.691971in}{0.844889in}}%
\pgfpathlineto{\pgfqpoint{0.692533in}{0.845913in}}%
\pgfpathlineto{\pgfqpoint{0.693636in}{0.848583in}}%
\pgfpathlineto{\pgfqpoint{0.694011in}{0.849669in}}%
\pgfpathlineto{\pgfqpoint{0.695121in}{0.852401in}}%
\pgfpathlineto{\pgfqpoint{0.695512in}{0.853456in}}%
\pgfpathlineto{\pgfqpoint{0.696622in}{0.855847in}}%
\pgfpathlineto{\pgfqpoint{0.697138in}{0.856933in}}%
\pgfpathlineto{\pgfqpoint{0.698240in}{0.859323in}}%
\pgfpathlineto{\pgfqpoint{0.698702in}{0.860410in}}%
\pgfpathlineto{\pgfqpoint{0.699812in}{0.863638in}}%
\pgfpathlineto{\pgfqpoint{0.700249in}{0.864663in}}%
\pgfpathlineto{\pgfqpoint{0.701352in}{0.867301in}}%
\pgfpathlineto{\pgfqpoint{0.701985in}{0.868388in}}%
\pgfpathlineto{\pgfqpoint{0.703087in}{0.870716in}}%
\pgfpathlineto{\pgfqpoint{0.703502in}{0.871802in}}%
\pgfpathlineto{\pgfqpoint{0.704612in}{0.874224in}}%
\pgfpathlineto{\pgfqpoint{0.705120in}{0.875310in}}%
\pgfpathlineto{\pgfqpoint{0.706230in}{0.877793in}}%
\pgfpathlineto{\pgfqpoint{0.706738in}{0.878880in}}%
\pgfpathlineto{\pgfqpoint{0.707848in}{0.880991in}}%
\pgfpathlineto{\pgfqpoint{0.708317in}{0.882077in}}%
\pgfpathlineto{\pgfqpoint{0.709427in}{0.884343in}}%
\pgfpathlineto{\pgfqpoint{0.709849in}{0.885368in}}%
\pgfpathlineto{\pgfqpoint{0.709849in}{0.885399in}}%
\pgfpathlineto{\pgfqpoint{0.710936in}{0.887634in}}%
\pgfpathlineto{\pgfqpoint{0.711335in}{0.888720in}}%
\pgfpathlineto{\pgfqpoint{0.712445in}{0.891173in}}%
\pgfpathlineto{\pgfqpoint{0.712922in}{0.892259in}}%
\pgfpathlineto{\pgfqpoint{0.714016in}{0.894680in}}%
\pgfpathlineto{\pgfqpoint{0.714032in}{0.894680in}}%
\pgfpathlineto{\pgfqpoint{0.714947in}{0.895767in}}%
\pgfpathlineto{\pgfqpoint{0.716033in}{0.898126in}}%
\pgfpathlineto{\pgfqpoint{0.716526in}{0.899212in}}%
\pgfpathlineto{\pgfqpoint{0.717620in}{0.901541in}}%
\pgfpathlineto{\pgfqpoint{0.717628in}{0.901541in}}%
\pgfpathlineto{\pgfqpoint{0.718316in}{0.902627in}}%
\pgfpathlineto{\pgfqpoint{0.719403in}{0.904459in}}%
\pgfpathlineto{\pgfqpoint{0.719989in}{0.905483in}}%
\pgfpathlineto{\pgfqpoint{0.721091in}{0.907283in}}%
\pgfpathlineto{\pgfqpoint{0.721482in}{0.908246in}}%
\pgfpathlineto{\pgfqpoint{0.722577in}{0.910046in}}%
\pgfpathlineto{\pgfqpoint{0.723124in}{0.911133in}}%
\pgfpathlineto{\pgfqpoint{0.724234in}{0.913212in}}%
\pgfpathlineto{\pgfqpoint{0.724695in}{0.914268in}}%
\pgfpathlineto{\pgfqpoint{0.725782in}{0.916689in}}%
\pgfpathlineto{\pgfqpoint{0.726235in}{0.917776in}}%
\pgfpathlineto{\pgfqpoint{0.727338in}{0.919887in}}%
\pgfpathlineto{\pgfqpoint{0.727721in}{0.920973in}}%
\pgfpathlineto{\pgfqpoint{0.728753in}{0.923425in}}%
\pgfpathlineto{\pgfqpoint{0.729550in}{0.924512in}}%
\pgfpathlineto{\pgfqpoint{0.730652in}{0.926809in}}%
\pgfpathlineto{\pgfqpoint{0.731246in}{0.927895in}}%
\pgfpathlineto{\pgfqpoint{0.732349in}{0.930130in}}%
\pgfpathlineto{\pgfqpoint{0.732771in}{0.931155in}}%
\pgfpathlineto{\pgfqpoint{0.733865in}{0.933607in}}%
\pgfpathlineto{\pgfqpoint{0.734647in}{0.934694in}}%
\pgfpathlineto{\pgfqpoint{0.735749in}{0.936494in}}%
\pgfpathlineto{\pgfqpoint{0.736242in}{0.937581in}}%
\pgfpathlineto{\pgfqpoint{0.737336in}{0.939381in}}%
\pgfpathlineto{\pgfqpoint{0.737930in}{0.940467in}}%
\pgfpathlineto{\pgfqpoint{0.739041in}{0.942671in}}%
\pgfpathlineto{\pgfqpoint{0.739736in}{0.943758in}}%
\pgfpathlineto{\pgfqpoint{0.741871in}{0.947793in}}%
\pgfpathlineto{\pgfqpoint{0.742457in}{0.948880in}}%
\pgfpathlineto{\pgfqpoint{0.743536in}{0.950898in}}%
\pgfpathlineto{\pgfqpoint{0.744341in}{0.951984in}}%
\pgfpathlineto{\pgfqpoint{0.745420in}{0.954561in}}%
\pgfpathlineto{\pgfqpoint{0.746147in}{0.955647in}}%
\pgfpathlineto{\pgfqpoint{0.747233in}{0.957944in}}%
\pgfpathlineto{\pgfqpoint{0.747656in}{0.958906in}}%
\pgfpathlineto{\pgfqpoint{0.747656in}{0.959000in}}%
\pgfpathlineto{\pgfqpoint{0.748766in}{0.961266in}}%
\pgfpathlineto{\pgfqpoint{0.749188in}{0.962352in}}%
\pgfpathlineto{\pgfqpoint{0.750275in}{0.964122in}}%
\pgfpathlineto{\pgfqpoint{0.751056in}{0.965208in}}%
\pgfpathlineto{\pgfqpoint{0.752119in}{0.966977in}}%
\pgfpathlineto{\pgfqpoint{0.752776in}{0.968033in}}%
\pgfpathlineto{\pgfqpoint{0.753878in}{0.970144in}}%
\pgfpathlineto{\pgfqpoint{0.754582in}{0.971230in}}%
\pgfpathlineto{\pgfqpoint{0.755677in}{0.973279in}}%
\pgfpathlineto{\pgfqpoint{0.755692in}{0.973279in}}%
\pgfpathlineto{\pgfqpoint{0.756341in}{0.974365in}}%
\pgfpathlineto{\pgfqpoint{0.757451in}{0.976135in}}%
\pgfpathlineto{\pgfqpoint{0.758178in}{0.977190in}}%
\pgfpathlineto{\pgfqpoint{0.759288in}{0.979053in}}%
\pgfpathlineto{\pgfqpoint{0.760007in}{0.980139in}}%
\pgfpathlineto{\pgfqpoint{0.761079in}{0.981909in}}%
\pgfpathlineto{\pgfqpoint{0.762095in}{0.982995in}}%
\pgfpathlineto{\pgfqpoint{0.763174in}{0.984858in}}%
\pgfpathlineto{\pgfqpoint{0.764041in}{0.985944in}}%
\pgfpathlineto{\pgfqpoint{0.765105in}{0.987558in}}%
\pgfpathlineto{\pgfqpoint{0.766027in}{0.988645in}}%
\pgfpathlineto{\pgfqpoint{0.767129in}{0.990259in}}%
\pgfpathlineto{\pgfqpoint{0.767724in}{0.991314in}}%
\pgfpathlineto{\pgfqpoint{0.768834in}{0.993208in}}%
\pgfpathlineto{\pgfqpoint{0.769303in}{0.994294in}}%
\pgfpathlineto{\pgfqpoint{0.770389in}{0.996250in}}%
\pgfpathlineto{\pgfqpoint{0.771327in}{0.997306in}}%
\pgfpathlineto{\pgfqpoint{0.772406in}{0.999261in}}%
\pgfpathlineto{\pgfqpoint{0.773024in}{1.000286in}}%
\pgfpathlineto{\pgfqpoint{0.774134in}{1.002334in}}%
\pgfpathlineto{\pgfqpoint{0.774705in}{1.003421in}}%
\pgfpathlineto{\pgfqpoint{0.775799in}{1.005190in}}%
\pgfpathlineto{\pgfqpoint{0.776589in}{1.006215in}}%
\pgfpathlineto{\pgfqpoint{0.777691in}{1.008046in}}%
\pgfpathlineto{\pgfqpoint{0.778449in}{1.009133in}}%
\pgfpathlineto{\pgfqpoint{0.779552in}{1.010747in}}%
\pgfpathlineto{\pgfqpoint{0.780294in}{1.011802in}}%
\pgfpathlineto{\pgfqpoint{0.781397in}{1.013696in}}%
\pgfpathlineto{\pgfqpoint{0.782061in}{1.014782in}}%
\pgfpathlineto{\pgfqpoint{0.783171in}{1.016148in}}%
\pgfpathlineto{\pgfqpoint{0.783898in}{1.017235in}}%
\pgfpathlineto{\pgfqpoint{0.785008in}{1.018787in}}%
\pgfpathlineto{\pgfqpoint{0.785837in}{1.019842in}}%
\pgfpathlineto{\pgfqpoint{0.786931in}{1.021953in}}%
\pgfpathlineto{\pgfqpoint{0.787815in}{1.023008in}}%
\pgfpathlineto{\pgfqpoint{0.788909in}{1.024405in}}%
\pgfpathlineto{\pgfqpoint{0.789746in}{1.025430in}}%
\pgfpathlineto{\pgfqpoint{0.790856in}{1.026951in}}%
\pgfpathlineto{\pgfqpoint{0.791653in}{1.028006in}}%
\pgfpathlineto{\pgfqpoint{0.792756in}{1.029683in}}%
\pgfpathlineto{\pgfqpoint{0.793576in}{1.030769in}}%
\pgfpathlineto{\pgfqpoint{0.794687in}{1.032228in}}%
\pgfpathlineto{\pgfqpoint{0.795726in}{1.033314in}}%
\pgfpathlineto{\pgfqpoint{0.796821in}{1.034867in}}%
\pgfpathlineto{\pgfqpoint{0.797579in}{1.035953in}}%
\pgfpathlineto{\pgfqpoint{0.798650in}{1.037567in}}%
\pgfpathlineto{\pgfqpoint{0.799784in}{1.038654in}}%
\pgfpathlineto{\pgfqpoint{0.800862in}{1.040299in}}%
\pgfpathlineto{\pgfqpoint{0.801848in}{1.041385in}}%
\pgfpathlineto{\pgfqpoint{0.802887in}{1.042472in}}%
\pgfpathlineto{\pgfqpoint{0.802950in}{1.042472in}}%
\pgfpathlineto{\pgfqpoint{0.803716in}{1.043558in}}%
\pgfpathlineto{\pgfqpoint{0.804818in}{1.045514in}}%
\pgfpathlineto{\pgfqpoint{0.805741in}{1.046600in}}%
\pgfpathlineto{\pgfqpoint{0.806851in}{1.048122in}}%
\pgfpathlineto{\pgfqpoint{0.807625in}{1.049208in}}%
\pgfpathlineto{\pgfqpoint{0.808649in}{1.050667in}}%
\pgfpathlineto{\pgfqpoint{0.809540in}{1.051753in}}%
\pgfpathlineto{\pgfqpoint{0.810642in}{1.052871in}}%
\pgfpathlineto{\pgfqpoint{0.811276in}{1.053926in}}%
\pgfpathlineto{\pgfqpoint{0.812362in}{1.055665in}}%
\pgfpathlineto{\pgfqpoint{0.813175in}{1.056751in}}%
\pgfpathlineto{\pgfqpoint{0.814285in}{1.058459in}}%
\pgfpathlineto{\pgfqpoint{0.814856in}{1.059545in}}%
\pgfpathlineto{\pgfqpoint{0.815888in}{1.061190in}}%
\pgfpathlineto{\pgfqpoint{0.816732in}{1.062277in}}%
\pgfpathlineto{\pgfqpoint{0.817796in}{1.063487in}}%
\pgfpathlineto{\pgfqpoint{0.818632in}{1.064574in}}%
\pgfpathlineto{\pgfqpoint{0.819680in}{1.066064in}}%
\pgfpathlineto{\pgfqpoint{0.820649in}{1.067150in}}%
\pgfpathlineto{\pgfqpoint{0.821712in}{1.068920in}}%
\pgfpathlineto{\pgfqpoint{0.822424in}{1.069975in}}%
\pgfpathlineto{\pgfqpoint{0.823495in}{1.071124in}}%
\pgfpathlineto{\pgfqpoint{0.824448in}{1.072210in}}%
\pgfpathlineto{\pgfqpoint{0.825543in}{1.073731in}}%
\pgfpathlineto{\pgfqpoint{0.826723in}{1.074818in}}%
\pgfpathlineto{\pgfqpoint{0.827771in}{1.076587in}}%
\pgfpathlineto{\pgfqpoint{0.828678in}{1.077643in}}%
\pgfpathlineto{\pgfqpoint{0.829756in}{1.079350in}}%
\pgfpathlineto{\pgfqpoint{0.830734in}{1.080312in}}%
\pgfpathlineto{\pgfqpoint{0.831789in}{1.081802in}}%
\pgfpathlineto{\pgfqpoint{0.832704in}{1.082889in}}%
\pgfpathlineto{\pgfqpoint{0.833783in}{1.084161in}}%
\pgfpathlineto{\pgfqpoint{0.834369in}{1.085155in}}%
\pgfpathlineto{\pgfqpoint{0.835448in}{1.086396in}}%
\pgfpathlineto{\pgfqpoint{0.836323in}{1.087483in}}%
\pgfpathlineto{\pgfqpoint{0.837394in}{1.088973in}}%
\pgfpathlineto{\pgfqpoint{0.838426in}{1.090059in}}%
\pgfpathlineto{\pgfqpoint{0.839536in}{1.091549in}}%
\pgfpathlineto{\pgfqpoint{0.840412in}{1.092636in}}%
\pgfpathlineto{\pgfqpoint{0.841491in}{1.093722in}}%
\pgfpathlineto{\pgfqpoint{0.842421in}{1.094809in}}%
\pgfpathlineto{\pgfqpoint{0.843500in}{1.096206in}}%
\pgfpathlineto{\pgfqpoint{0.844579in}{1.097261in}}%
\pgfpathlineto{\pgfqpoint{0.845681in}{1.098720in}}%
\pgfpathlineto{\pgfqpoint{0.846705in}{1.099776in}}%
\pgfpathlineto{\pgfqpoint{0.847792in}{1.101824in}}%
\pgfpathlineto{\pgfqpoint{0.848660in}{1.102911in}}%
\pgfpathlineto{\pgfqpoint{0.850059in}{1.104463in}}%
\pgfpathlineto{\pgfqpoint{0.851130in}{1.105549in}}%
\pgfpathlineto{\pgfqpoint{0.852240in}{1.107474in}}%
\pgfpathlineto{\pgfqpoint{0.853131in}{1.108561in}}%
\pgfpathlineto{\pgfqpoint{0.854241in}{1.110516in}}%
\pgfpathlineto{\pgfqpoint{0.855258in}{1.111603in}}%
\pgfpathlineto{\pgfqpoint{0.856868in}{1.113776in}}%
\pgfpathlineto{\pgfqpoint{0.857697in}{1.114862in}}%
\pgfpathlineto{\pgfqpoint{0.858783in}{1.116352in}}%
\pgfpathlineto{\pgfqpoint{0.859761in}{1.117439in}}%
\pgfpathlineto{\pgfqpoint{0.860785in}{1.118432in}}%
\pgfpathlineto{\pgfqpoint{0.861871in}{1.119487in}}%
\pgfpathlineto{\pgfqpoint{0.862974in}{1.121133in}}%
\pgfpathlineto{\pgfqpoint{0.863920in}{1.122219in}}%
\pgfpathlineto{\pgfqpoint{0.864952in}{1.123678in}}%
\pgfpathlineto{\pgfqpoint{0.865929in}{1.124765in}}%
\pgfpathlineto{\pgfqpoint{0.867039in}{1.126099in}}%
\pgfpathlineto{\pgfqpoint{0.868008in}{1.127186in}}%
\pgfpathlineto{\pgfqpoint{0.869071in}{1.128272in}}%
\pgfpathlineto{\pgfqpoint{0.869916in}{1.129328in}}%
\pgfpathlineto{\pgfqpoint{0.871026in}{1.130818in}}%
\pgfpathlineto{\pgfqpoint{0.872128in}{1.131904in}}%
\pgfpathlineto{\pgfqpoint{0.873207in}{1.133208in}}%
\pgfpathlineto{\pgfqpoint{0.874309in}{1.134294in}}%
\pgfpathlineto{\pgfqpoint{0.875419in}{1.135629in}}%
\pgfpathlineto{\pgfqpoint{0.876483in}{1.136685in}}%
\pgfpathlineto{\pgfqpoint{0.877585in}{1.138206in}}%
\pgfpathlineto{\pgfqpoint{0.878367in}{1.139292in}}%
\pgfpathlineto{\pgfqpoint{0.879406in}{1.140193in}}%
\pgfpathlineto{\pgfqpoint{0.880188in}{1.141279in}}%
\pgfpathlineto{\pgfqpoint{0.881259in}{1.142831in}}%
\pgfpathlineto{\pgfqpoint{0.882307in}{1.143918in}}%
\pgfpathlineto{\pgfqpoint{0.883401in}{1.145345in}}%
\pgfpathlineto{\pgfqpoint{0.884433in}{1.146432in}}%
\pgfpathlineto{\pgfqpoint{0.885520in}{1.148015in}}%
\pgfpathlineto{\pgfqpoint{0.886458in}{1.149071in}}%
\pgfpathlineto{\pgfqpoint{0.887568in}{1.150126in}}%
\pgfpathlineto{\pgfqpoint{0.888475in}{1.151212in}}%
\pgfpathlineto{\pgfqpoint{0.889561in}{1.152454in}}%
\pgfpathlineto{\pgfqpoint{0.890500in}{1.153541in}}%
\pgfpathlineto{\pgfqpoint{0.891586in}{1.154751in}}%
\pgfpathlineto{\pgfqpoint{0.892579in}{1.155838in}}%
\pgfpathlineto{\pgfqpoint{0.894025in}{1.157235in}}%
\pgfpathlineto{\pgfqpoint{0.895456in}{1.158321in}}%
\pgfpathlineto{\pgfqpoint{0.896550in}{1.159283in}}%
\pgfpathlineto{\pgfqpoint{0.897833in}{1.160370in}}%
\pgfpathlineto{\pgfqpoint{0.898935in}{1.161394in}}%
\pgfpathlineto{\pgfqpoint{0.900178in}{1.162481in}}%
\pgfpathlineto{\pgfqpoint{0.901218in}{1.163474in}}%
\pgfpathlineto{\pgfqpoint{0.902203in}{1.164561in}}%
\pgfpathlineto{\pgfqpoint{0.903274in}{1.165678in}}%
\pgfpathlineto{\pgfqpoint{0.904392in}{1.166765in}}%
\pgfpathlineto{\pgfqpoint{0.905463in}{1.168068in}}%
\pgfpathlineto{\pgfqpoint{0.906455in}{1.169155in}}%
\pgfpathlineto{\pgfqpoint{0.907464in}{1.170365in}}%
\pgfpathlineto{\pgfqpoint{0.907558in}{1.170365in}}%
\pgfpathlineto{\pgfqpoint{0.908590in}{1.171452in}}%
\pgfpathlineto{\pgfqpoint{0.909692in}{1.172383in}}%
\pgfpathlineto{\pgfqpoint{0.910825in}{1.173439in}}%
\pgfpathlineto{\pgfqpoint{0.911928in}{1.174991in}}%
\pgfpathlineto{\pgfqpoint{0.912764in}{1.176046in}}%
\pgfpathlineto{\pgfqpoint{0.913859in}{1.177257in}}%
\pgfpathlineto{\pgfqpoint{0.914797in}{1.178343in}}%
\pgfpathlineto{\pgfqpoint{0.915821in}{1.179337in}}%
\pgfpathlineto{\pgfqpoint{0.917205in}{1.180423in}}%
\pgfpathlineto{\pgfqpoint{0.918315in}{1.181789in}}%
\pgfpathlineto{\pgfqpoint{0.919175in}{1.182875in}}%
\pgfpathlineto{\pgfqpoint{0.920191in}{1.184055in}}%
\pgfpathlineto{\pgfqpoint{0.921426in}{1.185142in}}%
\pgfpathlineto{\pgfqpoint{0.922536in}{1.186694in}}%
\pgfpathlineto{\pgfqpoint{0.923553in}{1.187718in}}%
\pgfpathlineto{\pgfqpoint{0.924616in}{1.188898in}}%
\pgfpathlineto{\pgfqpoint{0.925437in}{1.189984in}}%
\pgfpathlineto{\pgfqpoint{0.926547in}{1.190977in}}%
\pgfpathlineto{\pgfqpoint{0.927735in}{1.192064in}}%
\pgfpathlineto{\pgfqpoint{0.928845in}{1.193150in}}%
\pgfpathlineto{\pgfqpoint{0.930127in}{1.194237in}}%
\pgfpathlineto{\pgfqpoint{0.931237in}{1.195199in}}%
\pgfpathlineto{\pgfqpoint{0.932230in}{1.196286in}}%
\pgfpathlineto{\pgfqpoint{0.933309in}{1.197527in}}%
\pgfpathlineto{\pgfqpoint{0.934841in}{1.198552in}}%
\pgfpathlineto{\pgfqpoint{0.935951in}{1.199669in}}%
\pgfpathlineto{\pgfqpoint{0.937108in}{1.200756in}}%
\pgfpathlineto{\pgfqpoint{0.938179in}{1.201780in}}%
\pgfpathlineto{\pgfqpoint{0.938211in}{1.201780in}}%
\pgfpathlineto{\pgfqpoint{0.939430in}{1.202867in}}%
\pgfpathlineto{\pgfqpoint{0.940532in}{1.203798in}}%
\pgfpathlineto{\pgfqpoint{0.941588in}{1.204884in}}%
\pgfpathlineto{\pgfqpoint{0.942690in}{1.206343in}}%
\pgfpathlineto{\pgfqpoint{0.943902in}{1.207430in}}%
\pgfpathlineto{\pgfqpoint{0.944942in}{1.208640in}}%
\pgfpathlineto{\pgfqpoint{0.946114in}{1.209727in}}%
\pgfpathlineto{\pgfqpoint{0.947177in}{1.210813in}}%
\pgfpathlineto{\pgfqpoint{0.948295in}{1.211900in}}%
\pgfpathlineto{\pgfqpoint{0.949320in}{1.212614in}}%
\pgfpathlineto{\pgfqpoint{0.950774in}{1.213700in}}%
\pgfpathlineto{\pgfqpoint{0.951813in}{1.215066in}}%
\pgfpathlineto{\pgfqpoint{0.952923in}{1.216153in}}%
\pgfpathlineto{\pgfqpoint{0.954018in}{1.216898in}}%
\pgfpathlineto{\pgfqpoint{0.955347in}{1.217984in}}%
\pgfpathlineto{\pgfqpoint{0.956457in}{1.219164in}}%
\pgfpathlineto{\pgfqpoint{0.957708in}{1.220250in}}%
\pgfpathlineto{\pgfqpoint{0.958779in}{1.221492in}}%
\pgfpathlineto{\pgfqpoint{0.959537in}{1.222547in}}%
\pgfpathlineto{\pgfqpoint{0.960647in}{1.223510in}}%
\pgfpathlineto{\pgfqpoint{0.962140in}{1.224596in}}%
\pgfpathlineto{\pgfqpoint{0.963219in}{1.225869in}}%
\pgfpathlineto{\pgfqpoint{0.964697in}{1.226955in}}%
\pgfpathlineto{\pgfqpoint{0.965752in}{1.228228in}}%
\pgfpathlineto{\pgfqpoint{0.966667in}{1.229314in}}%
\pgfpathlineto{\pgfqpoint{0.967761in}{1.230246in}}%
\pgfpathlineto{\pgfqpoint{0.968997in}{1.231332in}}%
\pgfpathlineto{\pgfqpoint{0.970060in}{1.232294in}}%
\pgfpathlineto{\pgfqpoint{0.971842in}{1.233381in}}%
\pgfpathlineto{\pgfqpoint{0.972913in}{1.234219in}}%
\pgfpathlineto{\pgfqpoint{0.972944in}{1.234219in}}%
\pgfpathlineto{\pgfqpoint{0.974094in}{1.235275in}}%
\pgfpathlineto{\pgfqpoint{0.975133in}{1.236454in}}%
\pgfpathlineto{\pgfqpoint{0.976830in}{1.237479in}}%
\pgfpathlineto{\pgfqpoint{0.977901in}{1.238503in}}%
\pgfpathlineto{\pgfqpoint{0.979519in}{1.239589in}}%
\pgfpathlineto{\pgfqpoint{0.980629in}{1.240552in}}%
\pgfpathlineto{\pgfqpoint{0.981669in}{1.241638in}}%
\pgfpathlineto{\pgfqpoint{0.982756in}{1.242507in}}%
\pgfpathlineto{\pgfqpoint{0.984030in}{1.243594in}}%
\pgfpathlineto{\pgfqpoint{0.985101in}{1.244432in}}%
\pgfpathlineto{\pgfqpoint{0.986531in}{1.245487in}}%
\pgfpathlineto{\pgfqpoint{0.987610in}{1.246729in}}%
\pgfpathlineto{\pgfqpoint{0.988869in}{1.247816in}}%
\pgfpathlineto{\pgfqpoint{0.989924in}{1.248561in}}%
\pgfpathlineto{\pgfqpoint{0.990894in}{1.249647in}}%
\pgfpathlineto{\pgfqpoint{0.991973in}{1.250858in}}%
\pgfpathlineto{\pgfqpoint{0.993004in}{1.251944in}}%
\pgfpathlineto{\pgfqpoint{0.994107in}{1.253000in}}%
\pgfpathlineto{\pgfqpoint{0.995256in}{1.254055in}}%
\pgfpathlineto{\pgfqpoint{0.996343in}{1.255142in}}%
\pgfpathlineto{\pgfqpoint{0.997312in}{1.256228in}}%
\pgfpathlineto{\pgfqpoint{0.998367in}{1.257066in}}%
\pgfpathlineto{\pgfqpoint{0.999861in}{1.258153in}}%
\pgfpathlineto{\pgfqpoint{1.000932in}{1.259022in}}%
\pgfpathlineto{\pgfqpoint{1.002855in}{1.260108in}}%
\pgfpathlineto{\pgfqpoint{1.003949in}{1.260884in}}%
\pgfpathlineto{\pgfqpoint{1.004809in}{1.261971in}}%
\pgfpathlineto{\pgfqpoint{1.005880in}{1.262747in}}%
\pgfpathlineto{\pgfqpoint{1.007287in}{1.263833in}}%
\pgfpathlineto{\pgfqpoint{1.008366in}{1.265230in}}%
\pgfpathlineto{\pgfqpoint{1.009523in}{1.266317in}}%
\pgfpathlineto{\pgfqpoint{1.010602in}{1.267341in}}%
\pgfpathlineto{\pgfqpoint{1.012275in}{1.268428in}}%
\pgfpathlineto{\pgfqpoint{1.013385in}{1.269421in}}%
\pgfpathlineto{\pgfqpoint{1.014933in}{1.270507in}}%
\pgfpathlineto{\pgfqpoint{1.016043in}{1.271532in}}%
\pgfpathlineto{\pgfqpoint{1.017302in}{1.272618in}}%
\pgfpathlineto{\pgfqpoint{1.018404in}{1.273487in}}%
\pgfpathlineto{\pgfqpoint{1.019803in}{1.274574in}}%
\pgfpathlineto{\pgfqpoint{1.020913in}{1.275412in}}%
\pgfpathlineto{\pgfqpoint{1.022149in}{1.276467in}}%
\pgfpathlineto{\pgfqpoint{1.023165in}{1.277492in}}%
\pgfpathlineto{\pgfqpoint{1.024314in}{1.278578in}}%
\pgfpathlineto{\pgfqpoint{1.025354in}{1.279323in}}%
\pgfpathlineto{\pgfqpoint{1.026417in}{1.280410in}}%
\pgfpathlineto{\pgfqpoint{1.027519in}{1.281124in}}%
\pgfpathlineto{\pgfqpoint{1.029075in}{1.282210in}}%
\pgfpathlineto{\pgfqpoint{1.030170in}{1.282769in}}%
\pgfpathlineto{\pgfqpoint{1.032343in}{1.283855in}}%
\pgfpathlineto{\pgfqpoint{1.033453in}{1.284756in}}%
\pgfpathlineto{\pgfqpoint{1.035204in}{1.285811in}}%
\pgfpathlineto{\pgfqpoint{1.036314in}{1.286680in}}%
\pgfpathlineto{\pgfqpoint{1.037979in}{1.287767in}}%
\pgfpathlineto{\pgfqpoint{1.039058in}{1.288481in}}%
\pgfpathlineto{\pgfqpoint{1.040872in}{1.289505in}}%
\pgfpathlineto{\pgfqpoint{1.040872in}{1.289536in}}%
\pgfpathlineto{\pgfqpoint{1.041927in}{1.290157in}}%
\pgfpathlineto{\pgfqpoint{1.043225in}{1.291244in}}%
\pgfpathlineto{\pgfqpoint{1.044335in}{1.292020in}}%
\pgfpathlineto{\pgfqpoint{1.045899in}{1.293106in}}%
\pgfpathlineto{\pgfqpoint{1.046899in}{1.293727in}}%
\pgfpathlineto{\pgfqpoint{1.048455in}{1.294813in}}%
\pgfpathlineto{\pgfqpoint{1.049565in}{1.295993in}}%
\pgfpathlineto{\pgfqpoint{1.050847in}{1.297079in}}%
\pgfpathlineto{\pgfqpoint{1.051918in}{1.298104in}}%
\pgfpathlineto{\pgfqpoint{1.053560in}{1.299159in}}%
\pgfpathlineto{\pgfqpoint{1.054670in}{1.299966in}}%
\pgfpathlineto{\pgfqpoint{1.055944in}{1.301053in}}%
\pgfpathlineto{\pgfqpoint{1.057054in}{1.301643in}}%
\pgfpathlineto{\pgfqpoint{1.058430in}{1.302729in}}%
\pgfpathlineto{\pgfqpoint{1.059501in}{1.303567in}}%
\pgfpathlineto{\pgfqpoint{1.061432in}{1.304654in}}%
\pgfpathlineto{\pgfqpoint{1.062511in}{1.305368in}}%
\pgfpathlineto{\pgfqpoint{1.064724in}{1.306423in}}%
\pgfpathlineto{\pgfqpoint{1.065795in}{1.307416in}}%
\pgfpathlineto{\pgfqpoint{1.066959in}{1.308503in}}%
\pgfpathlineto{\pgfqpoint{1.068069in}{1.309279in}}%
\pgfpathlineto{\pgfqpoint{1.069555in}{1.310365in}}%
\pgfpathlineto{\pgfqpoint{1.070665in}{1.311297in}}%
\pgfpathlineto{\pgfqpoint{1.071885in}{1.312383in}}%
\pgfpathlineto{\pgfqpoint{1.072830in}{1.312880in}}%
\pgfpathlineto{\pgfqpoint{1.074144in}{1.313966in}}%
\pgfpathlineto{\pgfqpoint{1.075246in}{1.314742in}}%
\pgfpathlineto{\pgfqpoint{1.077122in}{1.315829in}}%
\pgfpathlineto{\pgfqpoint{1.078232in}{1.316388in}}%
\pgfpathlineto{\pgfqpoint{1.079577in}{1.317474in}}%
\pgfpathlineto{\pgfqpoint{1.080593in}{1.318374in}}%
\pgfpathlineto{\pgfqpoint{1.080648in}{1.318374in}}%
\pgfpathlineto{\pgfqpoint{1.082501in}{1.319461in}}%
\pgfpathlineto{\pgfqpoint{1.083588in}{1.320206in}}%
\pgfpathlineto{\pgfqpoint{1.085182in}{1.321292in}}%
\pgfpathlineto{\pgfqpoint{1.086222in}{1.322317in}}%
\pgfpathlineto{\pgfqpoint{1.087668in}{1.323372in}}%
\pgfpathlineto{\pgfqpoint{1.088763in}{1.324334in}}%
\pgfpathlineto{\pgfqpoint{1.090287in}{1.325421in}}%
\pgfpathlineto{\pgfqpoint{1.091366in}{1.326290in}}%
\pgfpathlineto{\pgfqpoint{1.093774in}{1.327345in}}%
\pgfpathlineto{\pgfqpoint{1.094884in}{1.328122in}}%
\pgfpathlineto{\pgfqpoint{1.096612in}{1.329208in}}%
\pgfpathlineto{\pgfqpoint{1.097675in}{1.329829in}}%
\pgfpathlineto{\pgfqpoint{1.099934in}{1.330915in}}%
\pgfpathlineto{\pgfqpoint{1.100958in}{1.331971in}}%
\pgfpathlineto{\pgfqpoint{1.100997in}{1.331971in}}%
\pgfpathlineto{\pgfqpoint{1.102561in}{1.333057in}}%
\pgfpathlineto{\pgfqpoint{1.103671in}{1.333957in}}%
\pgfpathlineto{\pgfqpoint{1.105422in}{1.335044in}}%
\pgfpathlineto{\pgfqpoint{1.106501in}{1.335758in}}%
\pgfpathlineto{\pgfqpoint{1.108205in}{1.336751in}}%
\pgfpathlineto{\pgfqpoint{1.109300in}{1.337807in}}%
\pgfpathlineto{\pgfqpoint{1.111113in}{1.338893in}}%
\pgfpathlineto{\pgfqpoint{1.112216in}{1.339483in}}%
\pgfpathlineto{\pgfqpoint{1.113936in}{1.340569in}}%
\pgfpathlineto{\pgfqpoint{1.114874in}{1.341314in}}%
\pgfpathlineto{\pgfqpoint{1.115038in}{1.341314in}}%
\pgfpathlineto{\pgfqpoint{1.116617in}{1.342401in}}%
\pgfpathlineto{\pgfqpoint{1.117626in}{1.343208in}}%
\pgfpathlineto{\pgfqpoint{1.119166in}{1.344294in}}%
\pgfpathlineto{\pgfqpoint{1.120229in}{1.344977in}}%
\pgfpathlineto{\pgfqpoint{1.122011in}{1.346064in}}%
\pgfpathlineto{\pgfqpoint{1.122988in}{1.346778in}}%
\pgfpathlineto{\pgfqpoint{1.123121in}{1.346778in}}%
\pgfpathlineto{\pgfqpoint{1.124474in}{1.347864in}}%
\pgfpathlineto{\pgfqpoint{1.125584in}{1.348827in}}%
\pgfpathlineto{\pgfqpoint{1.127648in}{1.349913in}}%
\pgfpathlineto{\pgfqpoint{1.128687in}{1.350596in}}%
\pgfpathlineto{\pgfqpoint{1.129946in}{1.351683in}}%
\pgfpathlineto{\pgfqpoint{1.131025in}{1.352303in}}%
\pgfpathlineto{\pgfqpoint{1.133347in}{1.353390in}}%
\pgfpathlineto{\pgfqpoint{1.134246in}{1.353824in}}%
\pgfpathlineto{\pgfqpoint{1.134332in}{1.353824in}}%
\pgfpathlineto{\pgfqpoint{1.136044in}{1.354911in}}%
\pgfpathlineto{\pgfqpoint{1.137013in}{1.355563in}}%
\pgfpathlineto{\pgfqpoint{1.137154in}{1.355563in}}%
\pgfpathlineto{\pgfqpoint{1.139437in}{1.356618in}}%
\pgfpathlineto{\pgfqpoint{1.140523in}{1.357208in}}%
\pgfpathlineto{\pgfqpoint{1.142314in}{1.358294in}}%
\pgfpathlineto{\pgfqpoint{1.143416in}{1.359071in}}%
\pgfpathlineto{\pgfqpoint{1.145237in}{1.360157in}}%
\pgfpathlineto{\pgfqpoint{1.146254in}{1.360871in}}%
\pgfpathlineto{\pgfqpoint{1.148364in}{1.361957in}}%
\pgfpathlineto{\pgfqpoint{1.149459in}{1.362734in}}%
\pgfpathlineto{\pgfqpoint{1.151116in}{1.363820in}}%
\pgfpathlineto{\pgfqpoint{1.152101in}{1.364472in}}%
\pgfpathlineto{\pgfqpoint{1.153774in}{1.365558in}}%
\pgfpathlineto{\pgfqpoint{1.154861in}{1.366334in}}%
\pgfpathlineto{\pgfqpoint{1.156518in}{1.367421in}}%
\pgfpathlineto{\pgfqpoint{1.157566in}{1.368259in}}%
\pgfpathlineto{\pgfqpoint{1.157621in}{1.368259in}}%
\pgfpathlineto{\pgfqpoint{1.159223in}{1.369345in}}%
\pgfpathlineto{\pgfqpoint{1.160239in}{1.369904in}}%
\pgfpathlineto{\pgfqpoint{1.161772in}{1.370991in}}%
\pgfpathlineto{\pgfqpoint{1.162843in}{1.371518in}}%
\pgfpathlineto{\pgfqpoint{1.164672in}{1.372605in}}%
\pgfpathlineto{\pgfqpoint{1.165782in}{1.373319in}}%
\pgfpathlineto{\pgfqpoint{1.168292in}{1.374405in}}%
\pgfpathlineto{\pgfqpoint{1.169363in}{1.375119in}}%
\pgfpathlineto{\pgfqpoint{1.171075in}{1.376206in}}%
\pgfpathlineto{\pgfqpoint{1.172130in}{1.376889in}}%
\pgfpathlineto{\pgfqpoint{1.172185in}{1.376889in}}%
\pgfpathlineto{\pgfqpoint{1.174022in}{1.377975in}}%
\pgfpathlineto{\pgfqpoint{1.175124in}{1.378720in}}%
\pgfpathlineto{\pgfqpoint{1.176688in}{1.379807in}}%
\pgfpathlineto{\pgfqpoint{1.177759in}{1.380676in}}%
\pgfpathlineto{\pgfqpoint{1.179158in}{1.381731in}}%
\pgfpathlineto{\pgfqpoint{1.180268in}{1.382663in}}%
\pgfpathlineto{\pgfqpoint{1.182559in}{1.383749in}}%
\pgfpathlineto{\pgfqpoint{1.183567in}{1.384277in}}%
\pgfpathlineto{\pgfqpoint{1.185631in}{1.385363in}}%
\pgfpathlineto{\pgfqpoint{1.186663in}{1.386015in}}%
\pgfpathlineto{\pgfqpoint{1.186733in}{1.386015in}}%
\pgfpathlineto{\pgfqpoint{1.188696in}{1.387102in}}%
\pgfpathlineto{\pgfqpoint{1.189665in}{1.387878in}}%
\pgfpathlineto{\pgfqpoint{1.189728in}{1.387878in}}%
\pgfpathlineto{\pgfqpoint{1.192331in}{1.388964in}}%
\pgfpathlineto{\pgfqpoint{1.193441in}{1.389430in}}%
\pgfpathlineto{\pgfqpoint{1.195161in}{1.390516in}}%
\pgfpathlineto{\pgfqpoint{1.196224in}{1.391106in}}%
\pgfpathlineto{\pgfqpoint{1.198218in}{1.392193in}}%
\pgfpathlineto{\pgfqpoint{1.199242in}{1.392906in}}%
\pgfpathlineto{\pgfqpoint{1.201399in}{1.393993in}}%
\pgfpathlineto{\pgfqpoint{1.202455in}{1.394800in}}%
\pgfpathlineto{\pgfqpoint{1.204644in}{1.395887in}}%
\pgfpathlineto{\pgfqpoint{1.205707in}{1.396663in}}%
\pgfpathlineto{\pgfqpoint{1.207646in}{1.397749in}}%
\pgfpathlineto{\pgfqpoint{1.208717in}{1.398246in}}%
\pgfpathlineto{\pgfqpoint{1.210022in}{1.399332in}}%
\pgfpathlineto{\pgfqpoint{1.211085in}{1.400201in}}%
\pgfpathlineto{\pgfqpoint{1.213055in}{1.401288in}}%
\pgfpathlineto{\pgfqpoint{1.214056in}{1.402095in}}%
\pgfpathlineto{\pgfqpoint{1.214158in}{1.402095in}}%
\pgfpathlineto{\pgfqpoint{1.216292in}{1.403181in}}%
\pgfpathlineto{\pgfqpoint{1.217340in}{1.403989in}}%
\pgfpathlineto{\pgfqpoint{1.219904in}{1.405044in}}%
\pgfpathlineto{\pgfqpoint{1.220928in}{1.405572in}}%
\pgfpathlineto{\pgfqpoint{1.220998in}{1.405572in}}%
\pgfpathlineto{\pgfqpoint{1.223320in}{1.406658in}}%
\pgfpathlineto{\pgfqpoint{1.224422in}{1.407434in}}%
\pgfpathlineto{\pgfqpoint{1.225923in}{1.408459in}}%
\pgfpathlineto{\pgfqpoint{1.226885in}{1.409110in}}%
\pgfpathlineto{\pgfqpoint{1.226908in}{1.409110in}}%
\pgfpathlineto{\pgfqpoint{1.229410in}{1.410197in}}%
\pgfpathlineto{\pgfqpoint{1.230802in}{1.411221in}}%
\pgfpathlineto{\pgfqpoint{1.233264in}{1.412308in}}%
\pgfpathlineto{\pgfqpoint{1.234327in}{1.412867in}}%
\pgfpathlineto{\pgfqpoint{1.236422in}{1.413953in}}%
\pgfpathlineto{\pgfqpoint{1.237493in}{1.414667in}}%
\pgfpathlineto{\pgfqpoint{1.240284in}{1.415753in}}%
\pgfpathlineto{\pgfqpoint{1.241340in}{1.416250in}}%
\pgfpathlineto{\pgfqpoint{1.243974in}{1.417337in}}%
\pgfpathlineto{\pgfqpoint{1.245037in}{1.417926in}}%
\pgfpathlineto{\pgfqpoint{1.247406in}{1.419013in}}%
\pgfpathlineto{\pgfqpoint{1.248383in}{1.419696in}}%
\pgfpathlineto{\pgfqpoint{1.248508in}{1.419696in}}%
\pgfpathlineto{\pgfqpoint{1.250862in}{1.420782in}}%
\pgfpathlineto{\pgfqpoint{1.252105in}{1.421372in}}%
\pgfpathlineto{\pgfqpoint{1.254583in}{1.422459in}}%
\pgfpathlineto{\pgfqpoint{1.255544in}{1.422707in}}%
\pgfpathlineto{\pgfqpoint{1.258163in}{1.423762in}}%
\pgfpathlineto{\pgfqpoint{1.259211in}{1.424321in}}%
\pgfpathlineto{\pgfqpoint{1.262041in}{1.425408in}}%
\pgfpathlineto{\pgfqpoint{1.263081in}{1.425873in}}%
\pgfpathlineto{\pgfqpoint{1.265785in}{1.426960in}}%
\pgfpathlineto{\pgfqpoint{1.266833in}{1.427549in}}%
\pgfpathlineto{\pgfqpoint{1.269171in}{1.428636in}}%
\pgfpathlineto{\pgfqpoint{1.270140in}{1.429443in}}%
\pgfpathlineto{\pgfqpoint{1.272290in}{1.430498in}}%
\pgfpathlineto{\pgfqpoint{1.273384in}{1.431399in}}%
\pgfpathlineto{\pgfqpoint{1.275597in}{1.432485in}}%
\pgfpathlineto{\pgfqpoint{1.276605in}{1.432982in}}%
\pgfpathlineto{\pgfqpoint{1.276683in}{1.432982in}}%
\pgfpathlineto{\pgfqpoint{1.279748in}{1.434068in}}%
\pgfpathlineto{\pgfqpoint{1.280702in}{1.434565in}}%
\pgfpathlineto{\pgfqpoint{1.283492in}{1.435651in}}%
\pgfpathlineto{\pgfqpoint{1.284571in}{1.436024in}}%
\pgfpathlineto{\pgfqpoint{1.284587in}{1.436024in}}%
\pgfpathlineto{\pgfqpoint{1.286932in}{1.437110in}}%
\pgfpathlineto{\pgfqpoint{1.288034in}{1.437514in}}%
\pgfpathlineto{\pgfqpoint{1.290388in}{1.438600in}}%
\pgfpathlineto{\pgfqpoint{1.291263in}{1.439128in}}%
\pgfpathlineto{\pgfqpoint{1.294633in}{1.440215in}}%
\pgfpathlineto{\pgfqpoint{1.295633in}{1.440556in}}%
\pgfpathlineto{\pgfqpoint{1.297681in}{1.441643in}}%
\pgfpathlineto{\pgfqpoint{1.298659in}{1.442108in}}%
\pgfpathlineto{\pgfqpoint{1.300558in}{1.443164in}}%
\pgfpathlineto{\pgfqpoint{1.301567in}{1.443753in}}%
\pgfpathlineto{\pgfqpoint{1.304373in}{1.444840in}}%
\pgfpathlineto{\pgfqpoint{1.305405in}{1.445492in}}%
\pgfpathlineto{\pgfqpoint{1.307485in}{1.446578in}}%
\pgfpathlineto{\pgfqpoint{1.308556in}{1.447013in}}%
\pgfpathlineto{\pgfqpoint{1.311229in}{1.448099in}}%
\pgfpathlineto{\pgfqpoint{1.312324in}{1.448844in}}%
\pgfpathlineto{\pgfqpoint{1.314411in}{1.449931in}}%
\pgfpathlineto{\pgfqpoint{1.315435in}{1.450893in}}%
\pgfpathlineto{\pgfqpoint{1.318195in}{1.451980in}}%
\pgfpathlineto{\pgfqpoint{1.319071in}{1.452321in}}%
\pgfpathlineto{\pgfqpoint{1.321541in}{1.453408in}}%
\pgfpathlineto{\pgfqpoint{1.322581in}{1.453687in}}%
\pgfpathlineto{\pgfqpoint{1.322643in}{1.453687in}}%
\pgfpathlineto{\pgfqpoint{1.325489in}{1.454773in}}%
\pgfpathlineto{\pgfqpoint{1.326544in}{1.455301in}}%
\pgfpathlineto{\pgfqpoint{1.329061in}{1.456388in}}%
\pgfpathlineto{\pgfqpoint{1.330164in}{1.456822in}}%
\pgfpathlineto{\pgfqpoint{1.332361in}{1.457909in}}%
\pgfpathlineto{\pgfqpoint{1.333455in}{1.458436in}}%
\pgfpathlineto{\pgfqpoint{1.336340in}{1.459523in}}%
\pgfpathlineto{\pgfqpoint{1.337450in}{1.459926in}}%
\pgfpathlineto{\pgfqpoint{1.339944in}{1.461013in}}%
\pgfpathlineto{\pgfqpoint{1.341007in}{1.461572in}}%
\pgfpathlineto{\pgfqpoint{1.344157in}{1.462658in}}%
\pgfpathlineto{\pgfqpoint{1.345174in}{1.463124in}}%
\pgfpathlineto{\pgfqpoint{1.348254in}{1.464210in}}%
\pgfpathlineto{\pgfqpoint{1.349333in}{1.464862in}}%
\pgfpathlineto{\pgfqpoint{1.352804in}{1.465949in}}%
\pgfpathlineto{\pgfqpoint{1.353828in}{1.466476in}}%
\pgfpathlineto{\pgfqpoint{1.356486in}{1.467563in}}%
\pgfpathlineto{\pgfqpoint{1.357549in}{1.468059in}}%
\pgfpathlineto{\pgfqpoint{1.360066in}{1.469146in}}%
\pgfpathlineto{\pgfqpoint{1.361114in}{1.469518in}}%
\pgfpathlineto{\pgfqpoint{1.363944in}{1.470605in}}%
\pgfpathlineto{\pgfqpoint{1.365015in}{1.471008in}}%
\pgfpathlineto{\pgfqpoint{1.365030in}{1.471008in}}%
\pgfpathlineto{\pgfqpoint{1.367665in}{1.472095in}}%
\pgfpathlineto{\pgfqpoint{1.368556in}{1.472592in}}%
\pgfpathlineto{\pgfqpoint{1.371879in}{1.473678in}}%
\pgfpathlineto{\pgfqpoint{1.372989in}{1.474051in}}%
\pgfpathlineto{\pgfqpoint{1.376225in}{1.475137in}}%
\pgfpathlineto{\pgfqpoint{1.377265in}{1.475541in}}%
\pgfpathlineto{\pgfqpoint{1.379415in}{1.476627in}}%
\pgfpathlineto{\pgfqpoint{1.380470in}{1.477031in}}%
\pgfpathlineto{\pgfqpoint{1.383378in}{1.478086in}}%
\pgfpathlineto{\pgfqpoint{1.384285in}{1.478365in}}%
\pgfpathlineto{\pgfqpoint{1.384317in}{1.478365in}}%
\pgfpathlineto{\pgfqpoint{1.387342in}{1.479421in}}%
\pgfpathlineto{\pgfqpoint{1.388436in}{1.479980in}}%
\pgfpathlineto{\pgfqpoint{1.388452in}{1.479980in}}%
\pgfpathlineto{\pgfqpoint{1.391829in}{1.481066in}}%
\pgfpathlineto{\pgfqpoint{1.392916in}{1.481439in}}%
\pgfpathlineto{\pgfqpoint{1.395418in}{1.482525in}}%
\pgfpathlineto{\pgfqpoint{1.396496in}{1.482991in}}%
\pgfpathlineto{\pgfqpoint{1.398513in}{1.484077in}}%
\pgfpathlineto{\pgfqpoint{1.399545in}{1.484760in}}%
\pgfpathlineto{\pgfqpoint{1.402579in}{1.485847in}}%
\pgfpathlineto{\pgfqpoint{1.403360in}{1.486157in}}%
\pgfpathlineto{\pgfqpoint{1.406972in}{1.487244in}}%
\pgfpathlineto{\pgfqpoint{1.408082in}{1.487461in}}%
\pgfpathlineto{\pgfqpoint{1.412335in}{1.488516in}}%
\pgfpathlineto{\pgfqpoint{1.413390in}{1.488951in}}%
\pgfpathlineto{\pgfqpoint{1.416822in}{1.490037in}}%
\pgfpathlineto{\pgfqpoint{1.417846in}{1.490534in}}%
\pgfpathlineto{\pgfqpoint{1.422154in}{1.491620in}}%
\pgfpathlineto{\pgfqpoint{1.423147in}{1.491993in}}%
\pgfpathlineto{\pgfqpoint{1.426946in}{1.493079in}}%
\pgfpathlineto{\pgfqpoint{1.427931in}{1.493700in}}%
\pgfpathlineto{\pgfqpoint{1.431129in}{1.494787in}}%
\pgfpathlineto{\pgfqpoint{1.432184in}{1.495221in}}%
\pgfpathlineto{\pgfqpoint{1.434967in}{1.496308in}}%
\pgfpathlineto{\pgfqpoint{1.435655in}{1.496711in}}%
\pgfpathlineto{\pgfqpoint{1.439501in}{1.497798in}}%
\pgfpathlineto{\pgfqpoint{1.440189in}{1.498108in}}%
\pgfpathlineto{\pgfqpoint{1.440330in}{1.498108in}}%
\pgfpathlineto{\pgfqpoint{1.443035in}{1.499195in}}%
\pgfpathlineto{\pgfqpoint{1.444106in}{1.499629in}}%
\pgfpathlineto{\pgfqpoint{1.447460in}{1.500716in}}%
\pgfpathlineto{\pgfqpoint{1.448570in}{1.500995in}}%
\pgfpathlineto{\pgfqpoint{1.451282in}{1.502051in}}%
\pgfpathlineto{\pgfqpoint{1.452314in}{1.502765in}}%
\pgfpathlineto{\pgfqpoint{1.454933in}{1.503851in}}%
\pgfpathlineto{\pgfqpoint{1.455801in}{1.504161in}}%
\pgfpathlineto{\pgfqpoint{1.458999in}{1.505217in}}%
\pgfpathlineto{\pgfqpoint{1.460085in}{1.505620in}}%
\pgfpathlineto{\pgfqpoint{1.464455in}{1.506707in}}%
\pgfpathlineto{\pgfqpoint{1.465558in}{1.507110in}}%
\pgfpathlineto{\pgfqpoint{1.469107in}{1.508197in}}%
\pgfpathlineto{\pgfqpoint{1.470193in}{1.508569in}}%
\pgfpathlineto{\pgfqpoint{1.473227in}{1.509656in}}%
\pgfpathlineto{\pgfqpoint{1.474290in}{1.510091in}}%
\pgfpathlineto{\pgfqpoint{1.476721in}{1.511115in}}%
\pgfpathlineto{\pgfqpoint{1.477800in}{1.511612in}}%
\pgfpathlineto{\pgfqpoint{1.477823in}{1.511612in}}%
\pgfpathlineto{\pgfqpoint{1.481466in}{1.512698in}}%
\pgfpathlineto{\pgfqpoint{1.482514in}{1.513164in}}%
\pgfpathlineto{\pgfqpoint{1.486431in}{1.514250in}}%
\pgfpathlineto{\pgfqpoint{1.487533in}{1.514436in}}%
\pgfpathlineto{\pgfqpoint{1.490199in}{1.515523in}}%
\pgfpathlineto{\pgfqpoint{1.491168in}{1.515989in}}%
\pgfpathlineto{\pgfqpoint{1.495374in}{1.517075in}}%
\pgfpathlineto{\pgfqpoint{1.496429in}{1.517479in}}%
\pgfpathlineto{\pgfqpoint{1.496484in}{1.517479in}}%
\pgfpathlineto{\pgfqpoint{1.500596in}{1.518565in}}%
\pgfpathlineto{\pgfqpoint{1.501534in}{1.518751in}}%
\pgfpathlineto{\pgfqpoint{1.501652in}{1.518751in}}%
\pgfpathlineto{\pgfqpoint{1.505279in}{1.519838in}}%
\pgfpathlineto{\pgfqpoint{1.506350in}{1.520055in}}%
\pgfpathlineto{\pgfqpoint{1.509188in}{1.521110in}}%
\pgfpathlineto{\pgfqpoint{1.510087in}{1.521483in}}%
\pgfpathlineto{\pgfqpoint{1.510110in}{1.521483in}}%
\pgfpathlineto{\pgfqpoint{1.513573in}{1.522569in}}%
\pgfpathlineto{\pgfqpoint{1.514644in}{1.523004in}}%
\pgfpathlineto{\pgfqpoint{1.518475in}{1.524091in}}%
\pgfpathlineto{\pgfqpoint{1.519460in}{1.524680in}}%
\pgfpathlineto{\pgfqpoint{1.519554in}{1.524680in}}%
\pgfpathlineto{\pgfqpoint{1.523228in}{1.525767in}}%
\pgfpathlineto{\pgfqpoint{1.524198in}{1.526077in}}%
\pgfpathlineto{\pgfqpoint{1.528818in}{1.527164in}}%
\pgfpathlineto{\pgfqpoint{1.529897in}{1.527567in}}%
\pgfpathlineto{\pgfqpoint{1.534259in}{1.528654in}}%
\pgfpathlineto{\pgfqpoint{1.535252in}{1.528933in}}%
\pgfpathlineto{\pgfqpoint{1.535353in}{1.528933in}}%
\pgfpathlineto{\pgfqpoint{1.540021in}{1.530020in}}%
\pgfpathlineto{\pgfqpoint{1.541006in}{1.530516in}}%
\pgfpathlineto{\pgfqpoint{1.545493in}{1.531603in}}%
\pgfpathlineto{\pgfqpoint{1.546603in}{1.532193in}}%
\pgfpathlineto{\pgfqpoint{1.549988in}{1.533279in}}%
\pgfpathlineto{\pgfqpoint{1.550957in}{1.533527in}}%
\pgfpathlineto{\pgfqpoint{1.551098in}{1.533527in}}%
\pgfpathlineto{\pgfqpoint{1.556438in}{1.534614in}}%
\pgfpathlineto{\pgfqpoint{1.557470in}{1.534893in}}%
\pgfpathlineto{\pgfqpoint{1.561793in}{1.535980in}}%
\pgfpathlineto{\pgfqpoint{1.562793in}{1.536197in}}%
\pgfpathlineto{\pgfqpoint{1.566741in}{1.537283in}}%
\pgfpathlineto{\pgfqpoint{1.567789in}{1.537749in}}%
\pgfpathlineto{\pgfqpoint{1.571174in}{1.538836in}}%
\pgfpathlineto{\pgfqpoint{1.572276in}{1.539146in}}%
\pgfpathlineto{\pgfqpoint{1.576701in}{1.540232in}}%
\pgfpathlineto{\pgfqpoint{1.577686in}{1.540481in}}%
\pgfpathlineto{\pgfqpoint{1.577811in}{1.540481in}}%
\pgfpathlineto{\pgfqpoint{1.581423in}{1.541567in}}%
\pgfpathlineto{\pgfqpoint{1.582517in}{1.541940in}}%
\pgfpathlineto{\pgfqpoint{1.586622in}{1.543026in}}%
\pgfpathlineto{\pgfqpoint{1.587630in}{1.543368in}}%
\pgfpathlineto{\pgfqpoint{1.587700in}{1.543368in}}%
\pgfpathlineto{\pgfqpoint{1.592008in}{1.544454in}}%
\pgfpathlineto{\pgfqpoint{1.592993in}{1.544827in}}%
\pgfpathlineto{\pgfqpoint{1.598160in}{1.545913in}}%
\pgfpathlineto{\pgfqpoint{1.599184in}{1.546193in}}%
\pgfpathlineto{\pgfqpoint{1.604196in}{1.547248in}}%
\pgfpathlineto{\pgfqpoint{1.605306in}{1.547620in}}%
\pgfpathlineto{\pgfqpoint{1.609394in}{1.548707in}}%
\pgfpathlineto{\pgfqpoint{1.610426in}{1.549017in}}%
\pgfpathlineto{\pgfqpoint{1.615625in}{1.550073in}}%
\pgfpathlineto{\pgfqpoint{1.616227in}{1.550352in}}%
\pgfpathlineto{\pgfqpoint{1.616602in}{1.550352in}}%
\pgfpathlineto{\pgfqpoint{1.620972in}{1.551439in}}%
\pgfpathlineto{\pgfqpoint{1.622192in}{1.551811in}}%
\pgfpathlineto{\pgfqpoint{1.627688in}{1.552898in}}%
\pgfpathlineto{\pgfqpoint{1.628665in}{1.553456in}}%
\pgfpathlineto{\pgfqpoint{1.634082in}{1.554512in}}%
\pgfpathlineto{\pgfqpoint{1.634942in}{1.554760in}}%
\pgfpathlineto{\pgfqpoint{1.640337in}{1.555847in}}%
\pgfpathlineto{\pgfqpoint{1.641361in}{1.556095in}}%
\pgfpathlineto{\pgfqpoint{1.641447in}{1.556095in}}%
\pgfpathlineto{\pgfqpoint{1.645473in}{1.557181in}}%
\pgfpathlineto{\pgfqpoint{1.646301in}{1.557399in}}%
\pgfpathlineto{\pgfqpoint{1.646372in}{1.557399in}}%
\pgfpathlineto{\pgfqpoint{1.651789in}{1.558485in}}%
\pgfpathlineto{\pgfqpoint{1.652548in}{1.558796in}}%
\pgfpathlineto{\pgfqpoint{1.652767in}{1.558796in}}%
\pgfpathlineto{\pgfqpoint{1.657989in}{1.559882in}}%
\pgfpathlineto{\pgfqpoint{1.659013in}{1.560099in}}%
\pgfpathlineto{\pgfqpoint{1.664743in}{1.561186in}}%
\pgfpathlineto{\pgfqpoint{1.665830in}{1.561651in}}%
\pgfpathlineto{\pgfqpoint{1.672467in}{1.562738in}}%
\pgfpathlineto{\pgfqpoint{1.673546in}{1.563017in}}%
\pgfpathlineto{\pgfqpoint{1.677564in}{1.564104in}}%
\pgfpathlineto{\pgfqpoint{1.678534in}{1.564383in}}%
\pgfpathlineto{\pgfqpoint{1.684147in}{1.565470in}}%
\pgfpathlineto{\pgfqpoint{1.685139in}{1.565780in}}%
\pgfpathlineto{\pgfqpoint{1.693637in}{1.566867in}}%
\pgfpathlineto{\pgfqpoint{1.694685in}{1.567146in}}%
\pgfpathlineto{\pgfqpoint{1.701345in}{1.568232in}}%
\pgfpathlineto{\pgfqpoint{1.702088in}{1.568481in}}%
\pgfpathlineto{\pgfqpoint{1.708006in}{1.569567in}}%
\pgfpathlineto{\pgfqpoint{1.709069in}{1.569722in}}%
\pgfpathlineto{\pgfqpoint{1.714182in}{1.570809in}}%
\pgfpathlineto{\pgfqpoint{1.715269in}{1.571026in}}%
\pgfpathlineto{\pgfqpoint{1.722203in}{1.572113in}}%
\pgfpathlineto{\pgfqpoint{1.723266in}{1.572268in}}%
\pgfpathlineto{\pgfqpoint{1.730294in}{1.573354in}}%
\pgfpathlineto{\pgfqpoint{1.731365in}{1.573634in}}%
\pgfpathlineto{\pgfqpoint{1.739378in}{1.574720in}}%
\pgfpathlineto{\pgfqpoint{1.740434in}{1.574906in}}%
\pgfpathlineto{\pgfqpoint{1.749752in}{1.575993in}}%
\pgfpathlineto{\pgfqpoint{1.750667in}{1.576303in}}%
\pgfpathlineto{\pgfqpoint{1.755983in}{1.577390in}}%
\pgfpathlineto{\pgfqpoint{1.756726in}{1.577576in}}%
\pgfpathlineto{\pgfqpoint{1.757093in}{1.577576in}}%
\pgfpathlineto{\pgfqpoint{1.764066in}{1.578663in}}%
\pgfpathlineto{\pgfqpoint{1.765090in}{1.578880in}}%
\pgfpathlineto{\pgfqpoint{1.774049in}{1.579966in}}%
\pgfpathlineto{\pgfqpoint{1.775136in}{1.580215in}}%
\pgfpathlineto{\pgfqpoint{1.783095in}{1.581301in}}%
\pgfpathlineto{\pgfqpoint{1.783908in}{1.581425in}}%
\pgfpathlineto{\pgfqpoint{1.783994in}{1.581425in}}%
\pgfpathlineto{\pgfqpoint{1.793203in}{1.582512in}}%
\pgfpathlineto{\pgfqpoint{1.794258in}{1.582636in}}%
\pgfpathlineto{\pgfqpoint{1.801794in}{1.583722in}}%
\pgfpathlineto{\pgfqpoint{1.802717in}{1.584002in}}%
\pgfpathlineto{\pgfqpoint{1.808713in}{1.585088in}}%
\pgfpathlineto{\pgfqpoint{1.809776in}{1.585244in}}%
\pgfpathlineto{\pgfqpoint{1.818790in}{1.586330in}}%
\pgfpathlineto{\pgfqpoint{1.819814in}{1.586640in}}%
\pgfpathlineto{\pgfqpoint{1.828108in}{1.587727in}}%
\pgfpathlineto{\pgfqpoint{1.829140in}{1.587913in}}%
\pgfpathlineto{\pgfqpoint{1.829211in}{1.587913in}}%
\pgfpathlineto{\pgfqpoint{1.838952in}{1.589000in}}%
\pgfpathlineto{\pgfqpoint{1.839890in}{1.589155in}}%
\pgfpathlineto{\pgfqpoint{1.851561in}{1.590241in}}%
\pgfpathlineto{\pgfqpoint{1.852406in}{1.590428in}}%
\pgfpathlineto{\pgfqpoint{1.863538in}{1.591514in}}%
\pgfpathlineto{\pgfqpoint{1.864601in}{1.591762in}}%
\pgfpathlineto{\pgfqpoint{1.876296in}{1.592849in}}%
\pgfpathlineto{\pgfqpoint{1.877117in}{1.593004in}}%
\pgfpathlineto{\pgfqpoint{1.877219in}{1.593004in}}%
\pgfpathlineto{\pgfqpoint{1.890783in}{1.594091in}}%
\pgfpathlineto{\pgfqpoint{1.891392in}{1.594153in}}%
\pgfpathlineto{\pgfqpoint{1.891455in}{1.594153in}}%
\pgfpathlineto{\pgfqpoint{1.907332in}{1.595239in}}%
\pgfpathlineto{\pgfqpoint{1.908138in}{1.595332in}}%
\pgfpathlineto{\pgfqpoint{1.918652in}{1.596419in}}%
\pgfpathlineto{\pgfqpoint{1.918777in}{1.596481in}}%
\pgfpathlineto{\pgfqpoint{1.919004in}{1.596481in}}%
\pgfpathlineto{\pgfqpoint{1.934882in}{1.597567in}}%
\pgfpathlineto{\pgfqpoint{1.935609in}{1.597629in}}%
\pgfpathlineto{\pgfqpoint{1.935773in}{1.597629in}}%
\pgfpathlineto{\pgfqpoint{1.956412in}{1.598716in}}%
\pgfpathlineto{\pgfqpoint{1.957311in}{1.598778in}}%
\pgfpathlineto{\pgfqpoint{1.976417in}{1.599864in}}%
\pgfpathlineto{\pgfqpoint{1.977293in}{1.599957in}}%
\pgfpathlineto{\pgfqpoint{1.977496in}{1.599957in}}%
\pgfpathlineto{\pgfqpoint{2.001387in}{1.601044in}}%
\pgfpathlineto{\pgfqpoint{2.002145in}{1.601137in}}%
\pgfpathlineto{\pgfqpoint{2.033126in}{1.601944in}}%
\pgfpathlineto{\pgfqpoint{2.033126in}{1.601944in}}%
\pgfusepath{stroke}%
\end{pgfscope}%
\begin{pgfscope}%
\pgfsetrectcap%
\pgfsetmiterjoin%
\pgfsetlinewidth{0.803000pt}%
\definecolor{currentstroke}{rgb}{0.000000,0.000000,0.000000}%
\pgfsetstrokecolor{currentstroke}%
\pgfsetdash{}{0pt}%
\pgfpathmoveto{\pgfqpoint{0.553581in}{0.499444in}}%
\pgfpathlineto{\pgfqpoint{0.553581in}{1.654444in}}%
\pgfusepath{stroke}%
\end{pgfscope}%
\begin{pgfscope}%
\pgfsetrectcap%
\pgfsetmiterjoin%
\pgfsetlinewidth{0.803000pt}%
\definecolor{currentstroke}{rgb}{0.000000,0.000000,0.000000}%
\pgfsetstrokecolor{currentstroke}%
\pgfsetdash{}{0pt}%
\pgfpathmoveto{\pgfqpoint{2.103581in}{0.499444in}}%
\pgfpathlineto{\pgfqpoint{2.103581in}{1.654444in}}%
\pgfusepath{stroke}%
\end{pgfscope}%
\begin{pgfscope}%
\pgfsetrectcap%
\pgfsetmiterjoin%
\pgfsetlinewidth{0.803000pt}%
\definecolor{currentstroke}{rgb}{0.000000,0.000000,0.000000}%
\pgfsetstrokecolor{currentstroke}%
\pgfsetdash{}{0pt}%
\pgfpathmoveto{\pgfqpoint{0.553581in}{0.499444in}}%
\pgfpathlineto{\pgfqpoint{2.103581in}{0.499444in}}%
\pgfusepath{stroke}%
\end{pgfscope}%
\begin{pgfscope}%
\pgfsetrectcap%
\pgfsetmiterjoin%
\pgfsetlinewidth{0.803000pt}%
\definecolor{currentstroke}{rgb}{0.000000,0.000000,0.000000}%
\pgfsetstrokecolor{currentstroke}%
\pgfsetdash{}{0pt}%
\pgfpathmoveto{\pgfqpoint{0.553581in}{1.654444in}}%
\pgfpathlineto{\pgfqpoint{2.103581in}{1.654444in}}%
\pgfusepath{stroke}%
\end{pgfscope}%
\begin{pgfscope}%
\pgfsetbuttcap%
\pgfsetmiterjoin%
\definecolor{currentfill}{rgb}{1.000000,1.000000,1.000000}%
\pgfsetfillcolor{currentfill}%
\pgfsetfillopacity{0.800000}%
\pgfsetlinewidth{1.003750pt}%
\definecolor{currentstroke}{rgb}{0.800000,0.800000,0.800000}%
\pgfsetstrokecolor{currentstroke}%
\pgfsetstrokeopacity{0.800000}%
\pgfsetdash{}{0pt}%
\pgfpathmoveto{\pgfqpoint{0.832747in}{0.568889in}}%
\pgfpathlineto{\pgfqpoint{2.006358in}{0.568889in}}%
\pgfpathquadraticcurveto{\pgfqpoint{2.034136in}{0.568889in}}{\pgfqpoint{2.034136in}{0.596666in}}%
\pgfpathlineto{\pgfqpoint{2.034136in}{0.776388in}}%
\pgfpathquadraticcurveto{\pgfqpoint{2.034136in}{0.804166in}}{\pgfqpoint{2.006358in}{0.804166in}}%
\pgfpathlineto{\pgfqpoint{0.832747in}{0.804166in}}%
\pgfpathquadraticcurveto{\pgfqpoint{0.804970in}{0.804166in}}{\pgfqpoint{0.804970in}{0.776388in}}%
\pgfpathlineto{\pgfqpoint{0.804970in}{0.596666in}}%
\pgfpathquadraticcurveto{\pgfqpoint{0.804970in}{0.568889in}}{\pgfqpoint{0.832747in}{0.568889in}}%
\pgfpathlineto{\pgfqpoint{0.832747in}{0.568889in}}%
\pgfpathclose%
\pgfusepath{stroke,fill}%
\end{pgfscope}%
\begin{pgfscope}%
\pgfsetrectcap%
\pgfsetroundjoin%
\pgfsetlinewidth{1.505625pt}%
\definecolor{currentstroke}{rgb}{0.000000,0.000000,0.000000}%
\pgfsetstrokecolor{currentstroke}%
\pgfsetdash{}{0pt}%
\pgfpathmoveto{\pgfqpoint{0.860525in}{0.700000in}}%
\pgfpathlineto{\pgfqpoint{0.999414in}{0.700000in}}%
\pgfpathlineto{\pgfqpoint{1.138303in}{0.700000in}}%
\pgfusepath{stroke}%
\end{pgfscope}%
\begin{pgfscope}%
\definecolor{textcolor}{rgb}{0.000000,0.000000,0.000000}%
\pgfsetstrokecolor{textcolor}%
\pgfsetfillcolor{textcolor}%
\pgftext[x=1.249414in,y=0.651388in,left,base]{\color{textcolor}\rmfamily\fontsize{10.000000}{12.000000}\selectfont AUC=0.777}%
\end{pgfscope}%
\end{pgfpicture}%
\makeatother%
\endgroup%

\end{tabular}



%%%
\begin{comment}
If we set the discrimination threshold about $0.7$, the model would classify almost all of the samples, both positive and negative class, correctly, with about the same number of false positives (sending an ambulance when one is not needed, negative class samples with $p > 0.7$) and false negatives (not sending an ambulance when one is needed, positive class samples with $p < 0.7$).  If we (as a society) were willing to tolerate more false positives, we could set the discrimination threshold lower, and if budgets were tighter we could increase the $p$ threshold.  

The table below gives the number of true negatives (TN), false positives (FP), false negatives (FN), and true positives (TP) for the 499,496 samples in the test set, along with the precision and recall values, for different discrimination thresholds $p$.  The precision is the proportion of ambulances we sent that were needed, and the recall is the proportion of ambulances needed that we sent.  

$$\text{Precision} = \frac{TP}{FP+TP}, \qquad \text{Recall} = \frac{TP}{FN + TP}$$

\begin{center}
\begin{tabular}{rrrrrrrrrrrrrr}
\toprule
$p$ &   TN &       FP &      FN &      TP &  Precision &   Recall \\
\midrule
0.50 &  346,776 &   73,794 &       1 &  78,925 &  0.52 &  1.00       \\
0.60 &  390,335 &   30,235 &      89 &  78,837 &  0.72 &  1.00  \\
0.70 & 411,040 &    9,530 &   2,838 &  76,088 &  0.89 &  0.96 \\
0.80 & 418,739 &    1,831 &  19,174 &  59,752 &  0.97 &  0.76  \\
0.90 & 420,496 &       74 &  53,736 &  25,190 &  1.00 &  0.32 & \\
\bottomrule
\end{tabular}
\end{center}

\end{comment}
%%%



