%%%%%
\section{CRSS Binning}

Model building is more efficient and effective if the number of categories in each feature is reasonably small, with ``reasonably'' being fuzzy, but ten is a good target.  If some of the categories are essentially the same, it is better to bin (merge) them together, especially if some of the categories are very small.  

In the \acrfull{crss} data set, all of the features we plan to use are categorical, and most have a small number of categories.  The features for Age, Vehicle make, Vehicle model, Model year, and Vehicle body type each have more than fifty categories.  Some of them (age, model year) are ordered, and the rest are not.  To identify ``similar'' categories, I looked at how each category correlated with the target variable, being taken to a hospital.  

First I binned the HOSPITAL feature into a binary feature.  A few steps later I will get to imputing unknown values, but at this stage I binned the ``Not Reported'' and ``Reported as Unknown'' in the vastly majority category, ``Not Transported.''

\

\hfil\begin{tabular}{llrl}
	Original & Bin & Number of & Meaning \cr
	Code & & Samples & \cr\hline
	0 & 0 & 522,801 & Not Transported \cr
	1 & 1 & 2,549 & EMS Air \cr
	2 & 1 & 605 & Law Enforcement \cr
	3 & 1 & 30,368 & EMS Unknown Mode \cr
	4 & 1 & 8,926 & Transported Unknown Source \cr
	5 & 1 & 61,162 & EMS Ground \cr
	6 & 1 & 4,341 & Other \cr
	8 & 0 & 12,447 & Not Reported \cr
	9 & 0 & 1,075 & Reported as Unknown \cr	
	\cr
	All & 0 & 536,323 & 83.24\% \cr
	& 1 & 107,951 & 16.76\% \cr
\end{tabular}

\

Then for each value in AGE, I found the percentage of samples with that value and the percentage of samples of that value who were transported to a hospital.  A part of the results is in the table below.  Note the big shifts between ages 14, 15, 16, and 17, perhaps suggesting that new drivers are prone to more fender benders but not serious crashes, and that we should make $[15,16]$ its own bin.  

\begin{longtable}{ll p{0pt}@{\hspace{-.5\arrayrulewidth}}ccc}
	&  \multicolumn{3}{r}{Value} & Percent of & Percent with \cr
	& & && Samples with & this Value \cr
	&&&& this Value & Hospitalized \cr \hline
	 & \verb|AGE_IM| & 10 & 10 & 0.53 & 16.58 \cr
	 & \verb|AGE_IM| & 11 & 11 & 0.51 & 15.51 \cr
	 & \verb|AGE_IM| & 12 & 12 & 0.52 & 16.54 \cr
	 & \verb|AGE_IM| & 13 & 13 & 0.54 & 16.72 \cr
	 & \verb|AGE_IM| & 14 & 14 & 0.63 & 17.56 \cr
	 & \verb|AGE_IM| & 15 & 15 & 0.88 & 15.21 \cr
	 & \verb|AGE_IM| & 16 & 16 & 1.65 & 13.46 \cr
	 & \verb|AGE_IM| & 17 & 17 & 2.18 & 14.25 \cr
	 & \verb|AGE_IM| & 18 & 18 & 2.65 & 14.41 \cr
	 & \verb|AGE_IM| & 19 & 19 & 2.67 & 15.47 \cr
	 & \verb|AGE_IM| & 20 & 20 & 2.58 & 14.91 \cr
\end{longtable}

\

A similar shift in the hospitalization rate occurs in the early 50's, so we made another split between 51 and 52.    

\begin{longtable}{ll p{0pt}@{\hspace{-.5\arrayrulewidth}}ccc}
	&  \multicolumn{3}{r}{Value} & Percent of & Percent with \cr
	& & && Samples with & this Value \cr
	&&&& this Value & Hospitalized \cr \hline
	 & \verb|AGE_IM| & 44 & 44 & 1.39 & 15.23 \cr
	 & \verb|AGE_IM| & 45 & 45 & 1.36 & 16.08 \cr
	 & \verb|AGE_IM| & 46 & 46 & 1.33 & 15.69 \cr
	 & \verb|AGE_IM| & 47 & 47 & 1.39 & 15.26 \cr
	 & \verb|AGE_IM| & 48 & 48 & 1.34 & 16.46 \cr
	 & \verb|AGE_IM| & 49 & 49 & 1.32 & 16.59 \cr
	 & \verb|AGE_IM| & 50 & 50 & 1.29 & 16.64 \cr
	 & \verb|AGE_IM| & 51 & 51 & 1.39 & 16.24 \cr
	 & \verb|AGE_IM| & 52 & 52 & 1.36 & 16.57 \cr
	 & \verb|AGE_IM| & 53 & 53 & 1.24 & 17.05 \cr
	 & \verb|AGE_IM| & 54 & 54 & 1.31 & 17.74 \cr
	 & \verb|AGE_IM| & 55 & 55 & 1.28 & 17.66 \cr
	 & \verb|AGE_IM| & 56 & 56 & 1.23 & 17.21 \cr
\end{longtable}

Using the correlation to hospitalization rates, we set the bins for AGE at $[0,14], [15,16], [17,51]$, and $[52,\infty)$.

The same technique was especially useful with BODY\_TYP.  The table below shows the major body types ($\ge 0.40\%$ of samples).  The horizontal lines show where we divided the CRSS categories into bins. 

\

\begin{longtable}{cclp{3in}cc}
	& \multicolumn{2}{r}{CRSS Value} & Description & Percent of & Percent with \cr
	& & && Samples with & this Value \cr
	&&&& this Value & Hospitalized \cr \hline
	 & \verb|BODY_TYP| & 80 & Two Wheel Motorcycle (excluding motor scooters) & 2.11 & 61.48 \cr
	 \hline
	 & \verb|BODY_TYP| & 2 & 2-door sedan,hardtop,coupe & 2.92 & 15.66 \cr
	 & \verb|BODY_TYP| & 3 & 3-door/2-door hatchback & 0.73 & 15.46 \cr
	 & \verb|BODY_TYP| & 1 & Convertible(excludes sun-roof,t-bar) & 0.62 & 15.29 \cr
	 & \verb|BODY_TYP| & 30 & Compact Pickup (Only used in 2016) & 0.40 & 15.26 \cr
	 & \verb|BODY_TYP| & 4 & 4-door sedan, hardtop & 33.87 & 15.24 \cr
	 \hline
	 & \verb|BODY_TYP| & 19 & Utility Vehicle, Unknown body type & 0.99 & 14.57 \cr
	 & \verb|BODY_TYP| & 5 & 5-door/4-door hatchback & 2.38 & 14.00 \cr
	 & \verb|BODY_TYP| & 49 & Unknown light vehicle type (automobile,utility vehicle, van, or light truck) & 2.08 & 13.95 \cr
	 & \verb|BODY_TYP| & 8 & Sedan/Hardtop, number of doors unknown & 0.71 & 13.93 \cr
	 & \verb|BODY_TYP| & 6 & Station Wagon (excluding van and truck based) & 4.87 & 13.42 \cr
	 & \verb|BODY_TYP| & 14 & Compact Utility (Utility Vehicle Categories "Small" and "Midsize") & 14.46 & 13.26 \cr
	 & \verb|BODY_TYP| & 9 & Other or Unknown automobile type & 2.92 & 12.79 \cr
	 & \verb|BODY_TYP| & 20 & Minivan (Chrysler Town and Country, Caravan, Grand Caravan, Voyager, Voyager, Honda-Odyssey, ...) & 3.90 & 12.48 \cr
	 \hline
	 & \verb|BODY_TYP| & 34 & Light Pickup & 8.66 & 11.30 \cr
	 & \verb|BODY_TYP| & 31 & Standard Pickup (Only used in 2016) & 1.57 & 10.88 \cr
	 & \verb|BODY_TYP| & 15 & Large utility (ANSI D16.1 Utility Vehicle Categories and "Full Size" and "Large") & 4.83 & 10.76 \cr
	 & \verb|BODY_TYP| & 39 & Unknown (pickup style) light conventional truck type & 0.50 & 9.95 \cr
	 & \verb|BODY_TYP| & 21 & Large Van-Includes van-based buses (B150-B350, Sportsman, Royal Maxiwagon, Ram, Tradesman,...) & 1.10 & 8.74 \cr
	 \hline
	 & \verb|BODY_TYP| & 61 & Single-unit straight truck or Cab-Chassis (GVWR range 10,001 to 19,500 lbs.) & 0.62 & 5.26 \cr
	 & \verb|BODY_TYP| & 67 & Medium/heavy Pickup (GVWR greater than 10,000 lbs.) & 0.42 & 4.55 \cr
	 & \verb|BODY_TYP| & 63 & Single-unit straight truck or Cab-Chassis (GVWR greater than 26,000 lbs.) & 0.46 & 4.25 \cr
	 & \verb|BODY_TYP| & 66 & Truck-tractor (Cab only, or with any number of trailing unit; any weight) & 1.62 & 3.50 \cr
\end{longtable}

\

Based on the hospitalization rates, we binned the seventy-three CRSS categories into five.  

\

\begin{tabular}{rl}
	New Bin & CRSS Value \cr\hline
	Motorcycle  & [86,87,82,89,81,83,80,84,88,85,90,95,11,97,96,58,45,12,32,91] \cr
 	Car &  [10,2,3,1,59,30,4] \cr
	SUV & [19,42,5,49,8,16,6,14,52,9, 20] \cr
	Light Truck & [22,40,34,31,15,29,92,39,55,93,17,21,50,48,28,7,65] \cr
	Heavy Truck &  [51,61,67,63,62,66,79,78,64,72,98,60,99,71,73,94,41,13]\cr
\end{tabular}

\

Most of the groupings we could have done by name, putting the cars together, the SUV's together...  But some names do not fit their hospitalization profile, like ``Compact Pickup,''

\begin{quote}
Compact Pickup (S-10, LUV, Ram 50, Rampage, Courier, Ranger, S-5, Pup, Mazda Pickup, Mitsubishi Truck, Datsun/Nissan Pickup, Arrow Pickup, Scamp, Toyota Pickup, VW Pickup, D50, Colt P/U, T-10, S-15, T-15, Ram 100, Dakota, Sonoma)
\end{quote}

\noindent which has the hospitalization profile of a small car, not a light truck.

\

We binned other features similarly.  


