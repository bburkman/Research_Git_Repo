%%%%%
\section{CRSS Overview}

The \acrfull{crss} \cite{CRSS} is from the \acrfull{nhtsb}, part of the US \acrfull{dot}.  Available data is from 2016-2020.  In 2016, \acrshort{crss} replaced the \acrfull{nass ges}, which goes back to the 1970's.  

\begin{quote}
The CRSS obtains its data from a nationally representative probability sample selected from the more than six million police-reported crashes that occur annually. To be eligible for the CRSS sample, a crash report must be completed by the police; it must involve at least one motor vehicle traveling on a trafficway; and the crash must result in property damage, injury, or death.

These crash reports are chosen from 60 selected sites across the United States that reflect the geography, population, miles driven, and crashes in the United States. CRSS data collectors review crash reports from hundreds of law enforcement agencies within the sites, systematically sampling tens of thousands of crash reports each year. The collectors obtain copies of the selected crash reports and send them to a central location for coding. No other data is collected beyond that in the selected crash reports.

Trained personnel interpret and code data directly from the crash reports into an electronic data file. Approximately 120 data elements are coded into a common format. After coding, quality checks are performed on the data to ensure validity and consistency. When these are completed, CRSS data files and coding documentation become publicly available. \cite{CRSS_Manual}
\end{quote}

The data comes with a helpful user's manual \cite{CRSS_Manual} and a guide to their imputation of missing values that includes a history going back to the 1980's. \cite{CRSS_Imputation}


