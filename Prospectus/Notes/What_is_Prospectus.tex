%%%
\section{What is a Prospectus?}

%%%
\subsection{CACS}

For the dissertation proposal, you must pass a prospectus exam. In preparation for this exam, you write a dissertation proposal (prospectus). The prospectus includes a literature review describing the current state of the art in your chosen research area and what the open questions are for that area. You also develop a research plan to address one of the open questions that you uncovered in your prospectus. You also provide pilot results showing the promise of your proposed research. Finally, you formally defend the prospectus in a colloquium consisting of the dissertation committee and any members of the university who are interested, usually other members of the Center.

\

The oral PhD prospectus exam must be passed in a maximum of two attempts. The Dissertation
Committee will consist of a minimum of 3 members; of these, at least half must
be regular CMIX faculty members. The prospectus exam is conducted by the student’s Dissertation
Committee, plus one additional examiner from the graduate faculty of the Center
chosen in consultation with the student and the Dissertation Committee chairperson. The
student must prepare a written prospectus describing the current state of the art and the proposed
research. The prospectus exam can be scheduled when the written prospectus has been
approved by the Dissertation Committee.

%%%
\subsection{Arizona State}

From Arizona State U {\it Ph.D. Computer Science Dissertation Prospectus Procedures}


Student submits an electronic copy of the Dissertation Prospectus/Proposal to the committee and copies the Graduate Advisor. This should be done at least two weeks prior to the Prospectus Defense. The dissertation prospectus/proposal is not to exceed 20 pages and must contain:

1. A statement of the proposed research and why it is important.

2. An overview of the relevant literature.

3. A description of the student’s competence to conduct the proposed research. Passing the
comprehensive examination indicates competence in the area of the examination. The student
is encouraged to provide evidence of initial results in the scope of the dissertation research.

4. A discussion of how the research will be approached (including specific criteria for the
completion of the research broken down by research tasks, and the order in which the tasks will
be completed).

5. A projected time-table and outline of the dissertation.

%%%
\subsection{PhDAssistance.com}

Proposal Introduction:

The introduction section must contain an overview of your proposed research projects, its key concepts and problems statement or issues. You must able to show the reader or reviewer where your research fits. In general, you have to justify your research work within the field of computer science and then narrow it down to a particular research area (choosing a particular domain) and concern it will focus. Make it clear what exact problem or query your research will address and explain it in a brief research thesis statement.

Literature Review:

The literature must contain the outline of the previous work and research work previously carried out in your research area (topic related to your proposed project work). The main reason for writing a research proposal is to show the reader that you are familiar with what has already been done in the area. And to identify there is a gap in the particular research area that your work will fill.

Research Methods:

It must offer a clear and elaborate detail about the research work you will carry out. Must contain a detailed description of the equipment, techniques, or any other methodology you plan to include in your project should be covered here. Your projected schedule and budget should be added.

Bibliography:

Write a bibliography that cites all resources that were used in your literature review area. Many citing references found but computer science-related research project prepares APA (American Psychological Association) style of citing references format. It should be added at the beginning of your research proposal.  And it is a good way to familiarize others with your research topic, which can help them see the work you have included (relevant literature work).

%%%
\subsection{Rochester Institute of Technology}

\begin{itemize}
	\item Submit three weeks before defense.  
	\item 20-40 double-spaced pages
	\item The Dissertation Proposal should especially focus on describing the value of the new work and the history of related research, along with the general methodological approach towards completing the project.
\end{itemize}

Suggested format.

1. Title.

2. Problem Introduction. To what general area of Computing \& Information Sciences does
this apply? What are the issues? Why is this area important? What contribution (theory,
practice, methodology) would this make to the field?

3. Literature Review. Although the work is original, it will “stand on the shoulders” of others.
What are the major existing areas that this work is based upon? What previous thinking exists? The student should show awareness of the broad body of research that exists, should be conversant in debates and unresolved issues.

4. Research Questions. What question(s) do you intend to answer?

5. Methodology. How will the student go about answering the research questions? If human
subjects are needed, how will they be selected? Are the needed tools available (computer networks, hardware, etc.)? What will be built or created? (Testing protocols, programs, treatments, etc.)

6. References. This should include all references in the proposal.

%%%
\subsection{Rutgers}

2-5 pages

