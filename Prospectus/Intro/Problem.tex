%%%%%
\section{Problem}

%%%
\subsection{Application}

\index{Google Pixel Phone}
\index{iPhone}
\index{GM OnStar}

New (starting in 2022) Google Pixel phones have a feature that will automatically alert the police when involved in an automobile crash.  Apple says the feature is coming to iPhones \index{iPhone} and Apple Watches soon; those products already have a feature that detects a person falling, calls the person, and if no response, calls a neighbor, a friend, or the police.  One of my friends with multiple sclerosis uses this app.

Such systems (like GM OnStar) \index{GM OnStar}, built into vehicles, have existed for years, but soon they will become ubiquitous.  When the police receive a notification, based on the information they have, should they automatically deploy an ambulance?  In an accident with severe (but not instantly fatal) injuries, a few minutes' delay may have serious consequences, but sending an ambulance is expensive, and their supply is limited.  Can we develop a model that will, from the limited information the police can hope to have, from the datasets we have chosen, build a model to make a good prediction of whether an ambulance is needed?  

I am using ``police'' as a shorthand for ``the decision makers at the emergency call center.''  

This new cell phone feature will not be perfect; it will give many false positives and may not detect crashes with small objects, like pedestrians, \index{Pedestrian} that do not cause severe deceleration but are most likely to have severe injury.  The automated reports may, however, give us additional information like the number of people (number of phones) involved, and speed at time of impact.  This new phone feature will keep the crash analysis community busy for many years.  

The ``make a good predicition'' part is complicated.  We are not going to get 100\% accuracy.   What would we mean by ``good,'' and what would we use as a basis of comparison?  The current system relies mostly on phone calls from eyewitnesses who can give more information than the police will have in an automated notification.  These are thorny questions that we must address.  

%%%
\subsection{Datasets}

I am looking at two datasets,  the US \acrfull{dot} \acrfull{nhtsb} \acrfull{crss} data 2016-2020 data ($\approx 250,000$ records), and a census of Louisiana crash records 2014-18 ($\approx 800,000$ records).

%%%
\subsection{Imbalanced Data}
\index{Imbalanced Data}

In the 2014-2018 Louisiana data, we have over eight hundred thousand crash records.  If we are just looking for fatal crashes, about 3500 were fatal, 0.42\%.  If we built a model to predict whether a crash is fatal, and the model predicted that all crashes were nonfatal, that model would have correctly classified 99.58\% of crashes, or have 99.58\% {\it accuracy}.  In most contexts, that level of accuracy would be amazing, but in this context, such a model would be useless. 

In the CRSS dataset, which over represents severe crashes, 81.15\% of people involved in a crash were not transported to the hospital, and 16.75\% went to the hospital (the remaining 2.10\% unknown).  This nearly 5:1 imbalance is not as severe as the example with fatalities above, but still will be a challenge for our usual model building algorithms to give us the insights we seek.

The problem of imbalanced data appears in many applications, including spam detection and credit card fraud detection, and over the past decades the community has built many tools for addressing the problem.  Applying those tools is as much art as science, and the best combination of methods depends on the dataset and desired outcome.  The desired outcome is a moral, ethical, and political question as well as a technical one.  

%The dataset classifies the crashes as ``Fatal,'' ``Severe,'' ``Moderate,'' ``Complaint,'' and ``No Injury,'' also called PDO, ``Property Damage Only.''  Note that ``Fatal'' does not mean that everyone involved died, so there may be people moderately or seriously injured who need urgent medical attention.  If we are looking to model whether a crash is Fatal or Severe, those together are 1.1\%, and Fatal/Severe/Moderate is 6.8\%.  

%%%
\subsection{Tradeoffs}

Balancing false positives and false negatives in this application is additionally problematic because they have different costs.  The cost of a false positive (sending an ambulance when one is not needed) is measured in dollars, but the cost of a false negative (not sending an ambulance when one is needed) is measured in lives.  It is likely that this study will only illustrate the choices to be made rather than find a ``best'' solution that will significantly increase the number of true positives without increasing the number of false positives.  


