%%%%% Abstract
% I found an abstract in TRpC June 2022 with 365 words.
Starting in 2022, each new iPhone has a feature that will automatically notify police if the phone detects the deceleration profile of a crash.  From the data available from such an automatic notification, can we build a machine-learning model that will recommend whether police should immediately, perhaps automatically, dispatch an ambulance?  If the injuries are serious, time to medical care is critical, but few crashes result in serious injuries, and ambulances are in limited supply and expensive.  Such a model will not be perfect, with some false negatives (not sending an ambulance when one is needed) and many false positives (sending an ambulance when one is not needed), but better than random.  How much better depends on several things that we will investigate.

Whether a model is ``good'' is partly a political question, weighing prompt medical care against its high cost, but given a parameter $p$ of how to weigh those costs, we can build that tradeoff into the model.

We will show that the quality of the model depends mostly on what information is available to inform the decision of whether to immediately dispatch an ambulance.  Some information (location, time of day, day of week, weather) either come with the report or are easily to incorporate.  Other information (age and sex of phone's primary user, vehicle likely to be driven by that person) may be very helpful in making the decision to send the ambulance, but getting that information would require instantaneous communication between private and public databases.  Being able to interpret the location (is that precise location inside an intersection with high speed limits?) in real time would require planning and preparation.  Implementing such a system would require budgets, cooperation, and probably legislation, but knowing which data is most useful can help set priorities.  

For our work we used 