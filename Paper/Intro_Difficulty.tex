%%%
\subsection{Difficulty in Solving Problem}

Such a model will not be perfect, with some false negatives (not sending an ambulance when one is needed) and many false positives (sending an ambulance when one is not needed).  We will show that the quality of the model depends largely on what information is available.  Some information (location, time of day, day of week, weather) either comes with the automated report or is easy to get.  Other information (age and sex of phone's primary user, vehicle likely to be driven by that person) may be very helpful in predicting injury, but getting that information would require instantaneous communication between private and public databases.  Being able to interpret the location, ({\it e.g.} Is that precise location inside an intersection of two roads with high speed limits?) in real time would require planning and preparation.  

The problem is both political and ethical as well as technical.  How many false positives will we tolerate to have one fewer false negative?  We will show that, given such a marginal tolerance $p$, we can incorporate that tradeoff into the model, but each locality will have to decide that for itself.  Implementing such a system would require budgets, cooperation, and possibly legislation, but knowing which data is most useful can help set priorities.  


