%%%
\subsection{Machine Learning Challenges}

We deal with several machine learning challenges in our study, and their solutions are often as much art as science.  

\begin{description}

	\item [Two Datasets]  
	
For this study we used two datasets, the US Department of Transportation's  National Highway Transportation Safety Administration (NHTSA)'s Crash Report Sampling System (CRSS) 2016-2020 and a census of crash reports in Louisiana 2014-2018. \citep{CRSS}  Both datasets record about five hundred thousand crashes, with details on the crash and on each vehicle and each person involved.  The two datasets are similarly structured.  

We will look at the data at the level of each person involved in a crash (``crash person''), as each automated crash notification corresponds to a phone, which corresponds to a person.  	
	
The major difference between the two datasets is that, while the Louisiana is a census (all police crash reports), the CRSS is sample that intentionally over-represents more serious crashes.   In the Louisiana data, 8.5\% of the crash persons were transported to a hospital, while in the CRSS sample, it's 16.1\%.  
Comparing the results will require interpretation and justification.
	
	\item [Feature Selection and Engineering]  We need to select the features most relevant to crash severity; too many less-relevant features will muddle the model building.  CRSS has both ``Make'' and ``Body Type.''  Do these two features give enough different information that we should use both?  If not, which is more useful?
	
Some features have many values that we can usefully combine into bins or bands.  For instance, CRSS has ninety-seven different categories of vehicle body type, and our model building would work better if we could condense those to fewer than ten categories.  Is a compact pickup truck more like a car or a standard pickup?  We look at the likelihood that a crash vehicle's occupants require transportation to a hospital when making such classification decisions, and classify the compact pickup truck with the cars.  

The \verb|AGE| feature has values 0-120.   A simple approach would be to put it in decade bands, but in most states in the US, the driving age is 16.  The crash severity profiles for new drivers are different than for experienced drivers, so a split at 15/16 makes sense.  In our analysis, the crash severity profiles for ages 52-70 are similar to each other but different from 71+, so we broke them into bands there.  
	
We can also merge features into useful new features.  In both data sets, we have ``Day of Week'' and ``Hour.''  We would like to take from each to make ``Rush Hour,'' if it has a different hospitalization profile.  When does it start and end?  Is morning rush hour different from evening?  Does it start earlier on Fridays?  
	
	\item [Missing Data]  As with all real data, many samples (records) have missing values.  The Louisiana data is raw, with many values listed as ``unknown.'' For many features (fields, columns) of the CRSS, the authors have created another feature with the missing values imputed.  Trusting that those authors understand the data better than we do and may have access to redacted data, we use the imputed features.  For the Louisiana data, we will have to impute, delete, or ignore missing data; again, as much art as science.  

	
\end{description}


